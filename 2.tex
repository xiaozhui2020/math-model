\documentclass[10pt]{article}
\usepackage[utf8]{inputenc}
\usepackage[T1]{fontenc}
\usepackage{CJKutf8}
\usepackage{amsmath}
\usepackage{amsfonts}
\usepackage{amssymb}
\usepackage{mhchem}
\usepackage{stmaryrd}
\usepackage{bbold}
\usepackage{mathrsfs}
\usepackage{graphicx}
\usepackage[export]{adjustbox}
\graphicspath{ {./images/} }

\begin{document}
\section{7. 大连理工大学 2017 年研究生入学考试试题高等代数 
 李扬 
 微信公众号: sxkyliyang}
\begin{enumerate}
  \item \begin{CJK}{UTF8}{mj}填空题\end{CJK}.
\end{enumerate}
(1) $|\lambda E-A|=1$, \begin{CJK}{UTF8}{mj}求\end{CJK} $A$ \begin{CJK}{UTF8}{mj}的行列式\end{CJK}.

(2) $\left(\begin{array}{ll}1 & x \\ x & 1\end{array}\right)$ \begin{CJK}{UTF8}{mj}可以化成\end{CJK} $a, b$ \begin{CJK}{UTF8}{mj}两个行列式乘积\end{CJK}, \begin{CJK}{UTF8}{mj}求\end{CJK} $a, b$.

(3)\begin{CJK}{UTF8}{mj}最小二乘法\end{CJK}

(4)\begin{CJK}{UTF8}{mj}对偶基\end{CJK}

(5) \begin{CJK}{UTF8}{mj}已知\end{CJK} $A, B$ \begin{CJK}{UTF8}{mj}都是\end{CJK} $n \times n$ \begin{CJK}{UTF8}{mj}矩阵且可逆\end{CJK}, $D=\left(\begin{array}{cccc}A & 0 & 0 & 0 \\ E & B & 0 & 0 \\ 0 & 0 & B & E \\ 0 & 0 & 0 & A\end{array}\right)$, \begin{CJK}{UTF8}{mj}求\end{CJK} $D^{-1}$.

(6) $a, b$ \begin{CJK}{UTF8}{mj}取何值时\end{CJK}, $x-1$ \begin{CJK}{UTF8}{mj}是多项式\end{CJK} $f(x)=\left(x^{2}+a x+3\right)\left(x^{2}-b\right)$ \begin{CJK}{UTF8}{mj}的重因式\end{CJK}.

\begin{enumerate}
  \setcounter{enumi}{2}
  \item \begin{CJK}{UTF8}{mj}判别\end{CJK} $a, b$ \begin{CJK}{UTF8}{mj}为何值时线性方程组无解\end{CJK}, \begin{CJK}{UTF8}{mj}有惟一解\end{CJK}.

  \item \begin{CJK}{UTF8}{mj}一次型化标准形\end{CJK}.

  \item \begin{CJK}{UTF8}{mj}求若尔当标准形\end{CJK}.

  \item \begin{CJK}{UTF8}{mj}化对称矩阵为对角阵\end{CJK} (\begin{CJK}{UTF8}{mj}第九章知识\end{CJK}).

  \item \begin{CJK}{UTF8}{mj}子空间证明\end{CJK},\begin{CJK}{UTF8}{mj}一共三问\end{CJK}, \begin{CJK}{UTF8}{mj}复杂点\end{CJK}.

  \item $n$ \begin{CJK}{UTF8}{mj}维欧式空间\end{CJK}, $\alpha_{1}, \alpha_{2}, \cdots, \alpha_{n}$ \begin{CJK}{UTF8}{mj}两两夹角为锐角\end{CJK}, \begin{CJK}{UTF8}{mj}存在\end{CJK} $\beta$ \begin{CJK}{UTF8}{mj}与\end{CJK} $\alpha_{i}$ \begin{CJK}{UTF8}{mj}均成钝角\end{CJK}, \begin{CJK}{UTF8}{mj}证明\end{CJK} $\alpha_{1}, \cdots, \alpha_{i}$ \begin{CJK}{UTF8}{mj}线性无关\end{CJK}.

  \item \begin{CJK}{UTF8}{mj}子空间直和分解定理\end{CJK} $f(x)=()^{q_{1}}()^{q_{2}} \cdots()^{q_{n}}$.

  \item \begin{CJK}{UTF8}{mj}双线性函数\end{CJK}.

\end{enumerate}
\section{8. 大连理工大学 2009 年研究生入学考试试题数学分析 
 李扬 
 微信公众号: sxkyliyang}
\begin{enumerate}
  \item \begin{CJK}{UTF8}{mj}解答下列各题\end{CJK}.
\end{enumerate}
(1) \begin{CJK}{UTF8}{mj}判断下列数列是否收敛\end{CJK} $1+\frac{1}{2^{2}}+\frac{1}{3^{2}}+\cdots+\frac{1}{n^{2}}$.

( 2 ) \begin{CJK}{UTF8}{mj}设\end{CJK} $\left\{a_{n}\right\}$ \begin{CJK}{UTF8}{mj}是正数列\end{CJK}, \begin{CJK}{UTF8}{mj}若\end{CJK} $\lim _{n \rightarrow \infty} \sqrt[n]{a_{n}}=1$, \begin{CJK}{UTF8}{mj}证明\end{CJK}: $\prod_{n \rightarrow \infty} \sqrt[n]{a_{1}+a_{2}+\cdots+a_{n}}=1$.

(3) \begin{CJK}{UTF8}{mj}判断下列函数是否一致收敛\end{CJK}. $f(x)=e^{n} \cos \frac{1}{x}, x \in(0,1]$.

(4) \begin{CJK}{UTF8}{mj}设\end{CJK} $u=f\left(x y, \frac{y}{x}\right)$, \begin{CJK}{UTF8}{mj}求\end{CJK}: $\frac{\partial^{2} u}{\partial x^{2}}, \frac{\partial^{2} u}{\partial x \partial y}$.

(5) \begin{CJK}{UTF8}{mj}已知\end{CJK}: $f(a)$ \begin{CJK}{UTF8}{mj}存在\end{CJK}, \begin{CJK}{UTF8}{mj}求\end{CJK} $\lim _{x \rightarrow a} \frac{x f(a)-a f(x)}{x-a}$.

$(6)$ \begin{CJK}{UTF8}{mj}设\end{CJK} $f(x)$ \begin{CJK}{UTF8}{mj}在\end{CJK} $[a, b]$ \begin{CJK}{UTF8}{mj}上可导\end{CJK}, \begin{CJK}{UTF8}{mj}且\end{CJK} $f(a)=f(b)$, \begin{CJK}{UTF8}{mj}证明\end{CJK}: \begin{CJK}{UTF8}{mj}存在\end{CJK} $\xi \in(a, b)$, \begin{CJK}{UTF8}{mj}使得\end{CJK} $f(a)-f(\xi)=\frac{\xi f^{2}(\xi)}{2}$.

(7) \begin{CJK}{UTF8}{mj}求极限\end{CJK} $\lim _{x \rightarrow \infty} \frac{x^{n}}{(\ln x)^{2}}$.

$(8)$ \begin{CJK}{UTF8}{mj}求下列函数的\end{CJK} Fourier \begin{CJK}{UTF8}{mj}级数展开\end{CJK} $f(x)=\left\{\begin{array}{ll}\pi+x, & \pi \leqslant x<0 ; \\ \pi+x, & 0 \leqslant x \leqslant \pi .\end{array}\right.$.

(9) \begin{CJK}{UTF8}{mj}求\end{CJK} $f(x)=\sqrt[3]{x^{2}}(x-1)$ \begin{CJK}{UTF8}{mj}在\end{CJK} $(-\infty,+\infty)$ \begin{CJK}{UTF8}{mj}上的极值\end{CJK}.

( 10$)$ \begin{CJK}{UTF8}{mj}设\end{CJK} $V$ \begin{CJK}{UTF8}{mj}是由平面\end{CJK} $x=0, y=0, z=0$ \begin{CJK}{UTF8}{mj}和\end{CJK} $x+y+z=1$ \begin{CJK}{UTF8}{mj}所围成的区域\end{CJK}, \begin{CJK}{UTF8}{mj}求\end{CJK} $\iiint_{V} z^{2} \mathrm{~d} x \mathrm{~d} y \mathrm{~d} z$.

\begin{enumerate}
  \setcounter{enumi}{2}
  \item \begin{CJK}{UTF8}{mj}设定义在\end{CJK} $(-\infty,+\infty)$ \begin{CJK}{UTF8}{mj}上的\end{CJK} $f(x)$ \begin{CJK}{UTF8}{mj}满足\end{CJK}:\begin{CJK}{UTF8}{mj}对任意的\end{CJK} $x_{0} \in(-\infty,+\infty)$, \begin{CJK}{UTF8}{mj}都存在\end{CJK} $\delta>0$, \begin{CJK}{UTF8}{mj}使得\end{CJK} $f\left(x_{0}\right) \geqslant f(x)$, $x \in\left(x_{0}-\delta, x_{0}+\delta\right)$, \begin{CJK}{UTF8}{mj}证明存在一个区域\end{CJK} $I$ \begin{CJK}{UTF8}{mj}使得\end{CJK} $f(x)$ \begin{CJK}{UTF8}{mj}在\end{CJK} $I$ \begin{CJK}{UTF8}{mj}上是常数\end{CJK}.

  \item \begin{CJK}{UTF8}{mj}设\end{CJK} $f(x)$ \begin{CJK}{UTF8}{mj}是\end{CJK} $[a, b]$ \begin{CJK}{UTF8}{mj}上具有连续的导数\end{CJK}, $(0<a<b), f(a)=f(b)=0, \int_{a}^{b} f^{2}(x) \mathrm{d} x=1$, \begin{CJK}{UTF8}{mj}证明\end{CJK}

\end{enumerate}
$$
\int_{a}^{b} x^{2}\left(f^{\prime}(x)\right)^{2} \mathrm{~d} x>\frac{1}{4}
$$

\begin{enumerate}
  \setcounter{enumi}{4}
  \item \begin{CJK}{UTF8}{mj}给定函数列\end{CJK} $f_{n}(x)=\frac{x(\ln x)^{\alpha}}{n^{x}}(n=2,3, \cdots)$ \begin{CJK}{UTF8}{mj}试问当\end{CJK} $\alpha$ \begin{CJK}{UTF8}{mj}取何值时\end{CJK}, $\left\{f_{n}(x)\right\}$ \begin{CJK}{UTF8}{mj}在\end{CJK} $[0,+\infty)$ \begin{CJK}{UTF8}{mj}上一致连续\end{CJK}.

  \item \begin{CJK}{UTF8}{mj}设\end{CJK} $f(x)$ \begin{CJK}{UTF8}{mj}在\end{CJK} $[-1,1]$ \begin{CJK}{UTF8}{mj}上连续\end{CJK}, \begin{CJK}{UTF8}{mj}若有\end{CJK} $\int_{-1}^{1} x^{2 n+1} f(x) \mathrm{d} x=0(n=0,1,2)$, \begin{CJK}{UTF8}{mj}求证\end{CJK} $f(x)$ \begin{CJK}{UTF8}{mj}是偶函数\end{CJK}.

  \item \begin{CJK}{UTF8}{mj}设\end{CJK} $a>b>1$, \begin{CJK}{UTF8}{mj}证明\end{CJK}: $a^{b^{a}}>b^{a^{b}}$.

  \item \begin{CJK}{UTF8}{mj}计算曲面积分\end{CJK}

\end{enumerate}
$$
I=\iint_{\sum}(x-y+z) \mathrm{d} y \mathrm{~d} z+(y-z+x) \mathrm{d} z \mathrm{~d} y+(z-x+y) \mathrm{d} x \mathrm{~d} y
$$
\begin{CJK}{UTF8}{mj}其中\end{CJK}, $\sum: x^{2}+y^{2}=z^{2}$ \begin{CJK}{UTF8}{mj}介于\end{CJK} $z=0$ \begin{CJK}{UTF8}{mj}与\end{CJK} $z=1$ \begin{CJK}{UTF8}{mj}之间部分的外侧\end{CJK}.

\begin{enumerate}
  \setcounter{enumi}{8}
  \item \begin{CJK}{UTF8}{mj}证明积分\end{CJK} $\int_{0}^{1} \frac{\sin \frac{1}{x}}{x^{p}} \mathrm{~d} x$, \begin{CJK}{UTF8}{mj}在\end{CJK} $0<p<2$ \begin{CJK}{UTF8}{mj}时非一致收敛\end{CJK}, \begin{CJK}{UTF8}{mj}在\end{CJK} $0<p<2-\delta$ \begin{CJK}{UTF8}{mj}一致收敛\end{CJK}, $\delta$ \begin{CJK}{UTF8}{mj}是正常数\end{CJK}.
\end{enumerate}
\section{9. 大连理工大学 2010 年研究生入学考试试题数学分析 
 李扬 
 微信公众号: sxkyliyang}
\begin{enumerate}
  \item \begin{CJK}{UTF8}{mj}计算题\end{CJK}(\begin{CJK}{UTF8}{mj}每小题\end{CJK} 10 \begin{CJK}{UTF8}{mj}分\end{CJK}, \begin{CJK}{UTF8}{mj}共\end{CJK} 50 \begin{CJK}{UTF8}{mj}分\end{CJK}).
\end{enumerate}
(1)\begin{CJK}{UTF8}{mj}求极限\end{CJK}
$$
\lim _{x \rightarrow 0} \frac{x^{2}-\sin ^{2} x}{x^{2}\left(\sqrt{4+x^{2}}-2\right)}
$$
(2) \begin{CJK}{UTF8}{mj}设\end{CJK}
$$
f(x)=\int_{x}^{2 x} \int_{t}^{x} e^{-y^{2}} \mathrm{~d} y \mathrm{~d} t
$$
\begin{CJK}{UTF8}{mj}求\end{CJK} $f^{\prime}(x), f^{\prime \prime}(x)$.

(3) \begin{CJK}{UTF8}{mj}计算\end{CJK}
$$
I(b)=\int_{0}^{1} \frac{x^{b}-x}{\ln x} \mathrm{~d} x, b>1
$$
(4) \begin{CJK}{UTF8}{mj}求级数\end{CJK} $\sum_{n=1}^{\infty} n x^{n}$ \begin{CJK}{UTF8}{mj}的和\end{CJK}.

( 5 ) \begin{CJK}{UTF8}{mj}设\end{CJK} $a>0, b>0$, \begin{CJK}{UTF8}{mj}且有\end{CJK}
$$
x_{1}=\frac{1}{2}\left(a+\frac{b}{a}\right), x_{n+1}=\frac{1}{2}\left(x_{n}+\frac{b}{x_{n}}\right)
$$
\begin{CJK}{UTF8}{mj}求极限\end{CJK} $\lim _{n \rightarrow \infty} x_{n}$.

\begin{enumerate}
  \setcounter{enumi}{2}
  \item \begin{CJK}{UTF8}{mj}证明题\end{CJK}(\begin{CJK}{UTF8}{mj}每小题\end{CJK} 10 \begin{CJK}{UTF8}{mj}分\end{CJK}, \begin{CJK}{UTF8}{mj}共\end{CJK} 20 \begin{CJK}{UTF8}{mj}分\end{CJK}).
\end{enumerate}
(1) \begin{CJK}{UTF8}{mj}设\end{CJK} $x>0$, \begin{CJK}{UTF8}{mj}证明\end{CJK}
$$
\ln (1+x)>\frac{\arctan x}{1+x}
$$
$(2)$ \begin{CJK}{UTF8}{mj}设\end{CJK} $f(x)$ \begin{CJK}{UTF8}{mj}在\end{CJK} $[a, b]$ \begin{CJK}{UTF8}{mj}上连续\end{CJK}, \begin{CJK}{UTF8}{mj}证明若\end{CJK} $\int_{a}^{b} f^{2}(x) \mathrm{d} x=0$, \begin{CJK}{UTF8}{mj}则\end{CJK} $f(x)=0, x \in[a, b]$.

\begin{enumerate}
  \setcounter{enumi}{3}
  \item (15 \begin{CJK}{UTF8}{mj}分\end{CJK}) \begin{CJK}{UTF8}{mj}讨论\end{CJK} $\int_{0}^{+\infty} \frac{\sin x y}{x} \mathrm{~d} x$ \begin{CJK}{UTF8}{mj}关于\end{CJK} $y$ \begin{CJK}{UTF8}{mj}在所定义区间上的一致收敛性\end{CJK}:
\end{enumerate}
(1) $[a, b](0<a<b<+\infty)$;

(2) $(0,+\infty)$.

\begin{enumerate}
  \setcounter{enumi}{4}
  \item (13 \begin{CJK}{UTF8}{mj}分\end{CJK}) \begin{CJK}{UTF8}{mj}讨论函数\end{CJK}
\end{enumerate}
$$
f(x, y)= \begin{cases}\frac{x y}{\sqrt{x^{2}+y^{2}}}, & x^{2}+y^{2} \neq 0 \\ 0 . & x^{2}+y^{2}=0\end{cases}
$$
\begin{CJK}{UTF8}{mj}在\end{CJK} $(0,0)$ \begin{CJK}{UTF8}{mj}处的连续性\end{CJK}、\begin{CJK}{UTF8}{mj}可偏导性和可微性\end{CJK}.

\begin{enumerate}
  \setcounter{enumi}{5}
  \item (13 \begin{CJK}{UTF8}{mj}分\end{CJK}) \begin{CJK}{UTF8}{mj}设有隐函数\end{CJK} $F\left(\frac{x}{z}, \frac{y}{z}\right)=0$, \begin{CJK}{UTF8}{mj}其中\end{CJK} $F$ \begin{CJK}{UTF8}{mj}的偏导数连续\end{CJK}, \begin{CJK}{UTF8}{mj}求\end{CJK} $\frac{\partial z}{\partial x}, \frac{\partial z}{\partial y}$.

  \item (13 \begin{CJK}{UTF8}{mj}分\end{CJK}) \begin{CJK}{UTF8}{mj}若\end{CJK} $n \geqslant 1, x \geqslant 0, y \geqslant 0$, \begin{CJK}{UTF8}{mj}证明不等式\end{CJK}

\end{enumerate}
$$
\frac{x^{n}+y^{n}}{2} \geqslant\left(\frac{x+y}{2}\right)^{n}
$$

\begin{enumerate}
  \setcounter{enumi}{7}
  \item (13 \begin{CJK}{UTF8}{mj}分\end{CJK}) \begin{CJK}{UTF8}{mj}求球面\end{CJK} $x^{2}+y^{2}+z^{2}=50$ \begin{CJK}{UTF8}{mj}与雉面\end{CJK} $x^{2}+y^{2}=z^{2}$ \begin{CJK}{UTF8}{mj}所截出的曲线在点\end{CJK} $(3,4,5)$ \begin{CJK}{UTF8}{mj}处的切线与法平面方程\end{CJK}. 8. ( 15 \begin{CJK}{UTF8}{mj}分\end{CJK}) \begin{CJK}{UTF8}{mj}计算\end{CJK}
\end{enumerate}
$$
\iint_{\sigma} x^{3} \mathrm{~d} y \mathrm{~d} z+y^{3} \mathrm{~d} z \mathrm{~d} x+z^{3} \mathrm{~d} x \mathrm{~d} y
$$
\begin{CJK}{UTF8}{mj}其中\end{CJK} $\sigma$ \begin{CJK}{UTF8}{mj}为上半椭球面\end{CJK} $\frac{x^{2}}{a^{2}}+\frac{y^{2}}{b^{2}}+\frac{z^{2}}{c^{2}}=1$, \begin{CJK}{UTF8}{mj}定向取上侧\end{CJK}.

\section{0. 大连理工大学 2011 年研究生入学考试试题数学分析 
 李扬 
 微信公众号: sxkyliyang}
\begin{enumerate}
  \item \begin{CJK}{UTF8}{mj}解答下列各题\end{CJK} (\begin{CJK}{UTF8}{mj}每题\end{CJK} 6 \begin{CJK}{UTF8}{mj}分\end{CJK}, \begin{CJK}{UTF8}{mj}共\end{CJK} 60 \begin{CJK}{UTF8}{mj}分\end{CJK})
\end{enumerate}
(1) \begin{CJK}{UTF8}{mj}求极限\end{CJK}: $\lim _{x \rightarrow 0^{+}}\left(\frac{1}{x}-\frac{1}{\ln (1+x)}\right)$.

(2) $a_{1}=1, a_{n+1}=a_{n}+\frac{1}{a_{n}}, n=1,2, \cdots$, \begin{CJK}{UTF8}{mj}证明\end{CJK}: $\lim _{n \rightarrow \infty} \frac{a_{n}}{\sqrt{2 n}}=1$.

(3) \begin{CJK}{UTF8}{mj}利用对参数的微分法求\end{CJK} $I(b)=\int_{0}^{+\infty} e^{-x^{2}} \cos b x \mathrm{~d} x$, \begin{CJK}{UTF8}{mj}其中\end{CJK} $I(0)=\frac{\sqrt{\pi}}{2}$.

(4) \begin{CJK}{UTF8}{mj}证明\end{CJK}: \begin{CJK}{UTF8}{mj}当\end{CJK} $x>0$ \begin{CJK}{UTF8}{mj}时\end{CJK}, \begin{CJK}{UTF8}{mj}有\end{CJK} $x-\frac{x^{3}}{6}<\sin x<x-\frac{x^{3}}{6}+\frac{x^{5}}{120}$.

(5) \begin{CJK}{UTF8}{mj}求\end{CJK} $f(x, y)=2 x^{4}+y^{4}-2 x^{2}-2 y^{2}$ \begin{CJK}{UTF8}{mj}的所有极值点\end{CJK}.

(6) \begin{CJK}{UTF8}{mj}设\end{CJK} $\left\{a_{n}\right\}$ \begin{CJK}{UTF8}{mj}是有界数列\end{CJK}, \begin{CJK}{UTF8}{mj}定义\end{CJK} $f(x)=\sum_{n=1}^{\infty} a_{n} e^{-n x}, x>0$, \begin{CJK}{UTF8}{mj}证明\end{CJK}: $f(x)$ \begin{CJK}{UTF8}{mj}在\end{CJK} $(0,+\infty)$ \begin{CJK}{UTF8}{mj}上连续可微\end{CJK}.

( 7 ) \begin{CJK}{UTF8}{mj}设\end{CJK} $f(x, y), g(x, y)$ \begin{CJK}{UTF8}{mj}满足\end{CJK} $\frac{\partial f}{\partial x}=\frac{\partial g}{\partial y}, \frac{\partial f}{\partial y}=-\frac{\partial g}{\partial x}$, \begin{CJK}{UTF8}{mj}求证\end{CJK}: $\frac{\partial^{2}(f g)}{\partial x^{2}}+\frac{\partial^{2}(f g)}{\partial y^{2}}=0$.

(8) $f(x)=|\sin x|(-\pi \leqslant x \leqslant \pi)$, \begin{CJK}{UTF8}{mj}求\end{CJK} $f(x)$ \begin{CJK}{UTF8}{mj}的\end{CJK} Fourier \begin{CJK}{UTF8}{mj}级数\end{CJK}.

(9) \begin{CJK}{UTF8}{mj}计算\end{CJK} $I=\int_{0}^{1} \mathrm{~d} y \int_{y}^{1} \frac{y}{\sqrt{1+x^{3}}} \mathrm{~d} x$.

( 10 ) \begin{CJK}{UTF8}{mj}证明\end{CJK}: $f(x)=\left|x^{3}\right|$ \begin{CJK}{UTF8}{mj}在\end{CJK} $x=0$ \begin{CJK}{UTF8}{mj}处三阶导数不存在\end{CJK}.

\begin{enumerate}
  \setcounter{enumi}{2}
  \item (10 \begin{CJK}{UTF8}{mj}分\end{CJK}) \begin{CJK}{UTF8}{mj}设\end{CJK} $f(x)>0$, \begin{CJK}{UTF8}{mj}且在\end{CJK} $[0,1]$ \begin{CJK}{UTF8}{mj}连续\end{CJK}, \begin{CJK}{UTF8}{mj}证明\end{CJK}: $\lim _{n \rightarrow \infty} \sqrt[n]{\sum_{i=1}^{n}\left(f\left(\frac{i}{n}\right)\right)^{n} \frac{i}{n}}=\max _{x \in[0,1]} f(x)$.

  \item (10 \begin{CJK}{UTF8}{mj}分\end{CJK}) \begin{CJK}{UTF8}{mj}数列\end{CJK} $\left\{a_{n}\right\}$ \begin{CJK}{UTF8}{mj}单调递减趋于\end{CJK} 0 , \begin{CJK}{UTF8}{mj}且存在常数\end{CJK} $M$, \begin{CJK}{UTF8}{mj}使得对任意的\end{CJK} $n$, \begin{CJK}{UTF8}{mj}有\end{CJK} $\sum_{k=1}^{n}\left(a_{k}-a_{n}\right) \leqslant M$, \begin{CJK}{UTF8}{mj}求证\end{CJK}: $\sum_{n=1}^{\infty} a_{n}$ \begin{CJK}{UTF8}{mj}收\end{CJK} \begin{CJK}{UTF8}{mj}敛\end{CJK}.

  \item ( 10 \begin{CJK}{UTF8}{mj}分\end{CJK}) \begin{CJK}{UTF8}{mj}设\end{CJK} $f(x)$ \begin{CJK}{UTF8}{mj}在\end{CJK} $[0,1]$ \begin{CJK}{UTF8}{mj}上有二阶导数\end{CJK}, \begin{CJK}{UTF8}{mj}当\end{CJK} $x \in[0,1]$ \begin{CJK}{UTF8}{mj}时\end{CJK}, $|f(x)| \leqslant 1,\left|f^{\prime \prime}(x)\right| \leqslant 2$, \begin{CJK}{UTF8}{mj}证明\end{CJK}:\begin{CJK}{UTF8}{mj}当\end{CJK} $x \in[0,1]$ \begin{CJK}{UTF8}{mj}时\end{CJK}, $\left|f^{\prime}(x)\right| \leqslant 3$.

  \item (10 \begin{CJK}{UTF8}{mj}分\end{CJK}) $f(x)$ \begin{CJK}{UTF8}{mj}在\end{CJK} $[0,1]$ \begin{CJK}{UTF8}{mj}上连续且单调递增\end{CJK}, \begin{CJK}{UTF8}{mj}函数\end{CJK} $g(x)$ \begin{CJK}{UTF8}{mj}使得广义积分\end{CJK} $\int_{0}^{+\infty} \frac{g(x)}{x} \mathrm{~d} x$ \begin{CJK}{UTF8}{mj}收敛\end{CJK}, \begin{CJK}{UTF8}{mj}求证\end{CJK}:

\end{enumerate}
$$
\lim _{p \rightarrow+\infty} \int_{0}^{1} f(x) \frac{g(p x)}{x} \mathrm{~d} x=f(0) \int_{0}^{+\infty} \frac{g(x)}{x} \mathrm{~d} x .
$$

\begin{enumerate}
  \setcounter{enumi}{6}
  \item ( 10 \begin{CJK}{UTF8}{mj}分\end{CJK}) $f(x)$ \begin{CJK}{UTF8}{mj}在\end{CJK} $[a, b]$ \begin{CJK}{UTF8}{mj}上连续\end{CJK}, \begin{CJK}{UTF8}{mj}且对于任何满足\end{CJK} $\int_{a}^{b} g(x) \mathrm{d} x=0$ \begin{CJK}{UTF8}{mj}的连续函数\end{CJK} $g(x)$, \begin{CJK}{UTF8}{mj}有\end{CJK} $\int_{a}^{b} f(x) g(x) \mathrm{d} x=0$, \begin{CJK}{UTF8}{mj}求\end{CJK} \begin{CJK}{UTF8}{mj}证\end{CJK}: $f(x)$ \begin{CJK}{UTF8}{mj}恒为常数\end{CJK}.

  \item ( 10 \begin{CJK}{UTF8}{mj}分\end{CJK}) \begin{CJK}{UTF8}{mj}求\end{CJK} $x+2 y=1$ \begin{CJK}{UTF8}{mj}与\end{CJK} $x^{2}+2 y^{2}+z^{2}=1$ \begin{CJK}{UTF8}{mj}的交线上离原点最近的点\end{CJK}.

  \item ( 10 \begin{CJK}{UTF8}{mj}分\end{CJK}) $z=z(x, y)$ \begin{CJK}{UTF8}{mj}满足\end{CJK} $\frac{\partial^{2} z}{\partial x^{2}} \frac{\partial^{2} z}{\partial y^{2}}=\left(\frac{\partial^{2} z}{\partial x \partial y}\right)^{2}$, \begin{CJK}{UTF8}{mj}证明\end{CJK}: \begin{CJK}{UTF8}{mj}若把\end{CJK} $z=z(x, y)$ \begin{CJK}{UTF8}{mj}中的\end{CJK} $y$ \begin{CJK}{UTF8}{mj}看成\end{CJK} $x, z$ \begin{CJK}{UTF8}{mj}的函数\end{CJK}, \begin{CJK}{UTF8}{mj}有\end{CJK} $\frac{\partial^{2} y}{\partial x^{2}} \frac{\partial^{2} y}{\partial z^{2}}=\left(\frac{\partial^{2} y}{\partial x \partial z}\right)^{2}$

  \item (10 \begin{CJK}{UTF8}{mj}分\end{CJK}) \begin{CJK}{UTF8}{mj}求积分\end{CJK}

\end{enumerate}
$$
I=\int_{L}(y-z) \mathrm{d} x+(z-x) \mathrm{d} y+(x-y) \mathrm{d} z
$$
\begin{CJK}{UTF8}{mj}其中\end{CJK} $L$ \begin{CJK}{UTF8}{mj}为\end{CJK} $x^{2}+y^{2}=R^{2}$ \begin{CJK}{UTF8}{mj}与\end{CJK} $\frac{x}{a}+\frac{z}{h}=1(a>0, h>0)$ \begin{CJK}{UTF8}{mj}的交线\end{CJK}, \begin{CJK}{UTF8}{mj}从\end{CJK} $x$ \begin{CJK}{UTF8}{mj}轴正向看去为逆时针方向\end{CJK}. 10. ( 10 \begin{CJK}{UTF8}{mj}分\end{CJK}) \begin{CJK}{UTF8}{mj}设\end{CJK} $f(x)$ \begin{CJK}{UTF8}{mj}在\end{CJK} $[a, b]$ \begin{CJK}{UTF8}{mj}上无界\end{CJK}, \begin{CJK}{UTF8}{mj}证明\end{CJK}: \begin{CJK}{UTF8}{mj}在\end{CJK} $[a, b]$ \begin{CJK}{UTF8}{mj}上至少存在一点\end{CJK} $x$, \begin{CJK}{UTF8}{mj}使得\end{CJK} $f(x)$ \begin{CJK}{UTF8}{mj}在\end{CJK} $x$ \begin{CJK}{UTF8}{mj}的一个邻域内无界\end{CJK}.

\section{1. 大连理工大学 2012 年研究生入学考试试题数学分析 
 李扬 
 微信公众号: sxkyliyang}
\begin{enumerate}
  \item \begin{CJK}{UTF8}{mj}从以下题中选取\end{CJK} 6 \begin{CJK}{UTF8}{mj}题解答\end{CJK}, \begin{CJK}{UTF8}{mj}每题\end{CJK} 10 \begin{CJK}{UTF8}{mj}分\end{CJK}.
\end{enumerate}
(1) \begin{CJK}{UTF8}{mj}证明\end{CJK}: $f(x)=\frac{1}{x}$ \begin{CJK}{UTF8}{mj}于区间\end{CJK} $\left(\delta_{0}, 1\right.$ ) (\begin{CJK}{UTF8}{mj}其中\end{CJK} $0<\delta_{0}<1$ ) \begin{CJK}{UTF8}{mj}一致连续\end{CJK}, \begin{CJK}{UTF8}{mj}但是于\end{CJK} $(0,1)$ \begin{CJK}{UTF8}{mj}内不一致连续\end{CJK}.

( 2 ) \begin{CJK}{UTF8}{mj}证明\end{CJK}: \begin{CJK}{UTF8}{mj}若\end{CJK} $f(x)$ \begin{CJK}{UTF8}{mj}于\end{CJK} $[a, b]$ \begin{CJK}{UTF8}{mj}单调\end{CJK}, \begin{CJK}{UTF8}{mj}则\end{CJK} $f(x)$ \begin{CJK}{UTF8}{mj}于\end{CJK} $[a, b]$ \begin{CJK}{UTF8}{mj}内黎曼可积\end{CJK}.

(3) \begin{CJK}{UTF8}{mj}证明\end{CJK}: \begin{CJK}{UTF8}{mj}狄利克雷函数\end{CJK}: $f(x)=\left\{\begin{array}{ll}0, & x \text { 为无理数; } \\ \frac{1}{q}, & x=\frac{p}{q} \text { (有理数). }\end{array}\right.$, \begin{CJK}{UTF8}{mj}在所有无理点连续\end{CJK}, \begin{CJK}{UTF8}{mj}在有理点间断\end{CJK}.

(4) \begin{CJK}{UTF8}{mj}证明\end{CJK}: \begin{CJK}{UTF8}{mj}若\end{CJK} $f(x) \in C(a, b), \quad\left(\right.$ \begin{CJK}{UTF8}{mj}指\end{CJK} $(a, b)$ \begin{CJK}{UTF8}{mj}上的连续函数\end{CJK}), \begin{CJK}{UTF8}{mj}且任意\end{CJK} $(\alpha, \beta) \subset(a, b), \int_{\alpha}^{\beta} f(x) \mathrm{d} x=0$, \begin{CJK}{UTF8}{mj}那么\end{CJK} $f(x) \equiv 0, x \in(a, b) .$

(5) \begin{CJK}{UTF8}{mj}证明\end{CJK}: $\sum_{n=1}^{\infty} n e^{-n x}$ \begin{CJK}{UTF8}{mj}于\end{CJK} $(0,+\infty)$ \begin{CJK}{UTF8}{mj}不一致收敛\end{CJK}, \begin{CJK}{UTF8}{mj}但是对于\end{CJK} $\forall \delta>0$, \begin{CJK}{UTF8}{mj}于\end{CJK} $[\delta,+\infty)$ \begin{CJK}{UTF8}{mj}一致收敛\end{CJK}.

(6) \begin{CJK}{UTF8}{mj}证明\end{CJK}: $f(x)=\left\{\begin{array}{ll}x^{4} \sin \frac{1}{x}, & x \neq 0 ; \\ 0, & x=0 .\end{array}\right.$, \begin{CJK}{UTF8}{mj}在\end{CJK} $x=0$ \begin{CJK}{UTF8}{mj}处有连续的二阶导数\end{CJK}.

( 7 ) \begin{CJK}{UTF8}{mj}利用重积分计算三个半长轴分别为\end{CJK} $a, b, c$ \begin{CJK}{UTF8}{mj}的椭球体的体积\end{CJK}.

(8) \begin{CJK}{UTF8}{mj}计算第二类曲面积分\end{CJK}:
$$
\int_{\sum} x \mathrm{~d} y \mathrm{~d} z+y \mathrm{~d} z \mathrm{~d} x+z \mathrm{~d} x \mathrm{~d} y
$$
\begin{CJK}{UTF8}{mj}其中\end{CJK} $\sum$ \begin{CJK}{UTF8}{mj}是三角形\end{CJK} $(x, y, z>0, x+y+z=1)$, \begin{CJK}{UTF8}{mj}法方向按与\end{CJK} $x, y, z$ \begin{CJK}{UTF8}{mj}轴成锐角为正\end{CJK}.

\begin{enumerate}
  \setcounter{enumi}{2}
  \item \begin{CJK}{UTF8}{mj}从以下题中选取\end{CJK} 4 \begin{CJK}{UTF8}{mj}题解答\end{CJK}.
\end{enumerate}
(1) \begin{CJK}{UTF8}{mj}假设\end{CJK} $\lim _{n \rightarrow \infty} a_{n}=a$, \begin{CJK}{UTF8}{mj}证明\end{CJK}: $\lim _{n \rightarrow \infty} \frac{a_{1}+2 a_{2}+\cdots+n a_{n}}{n^{2}}=\frac{a}{2}$.

(2) \begin{CJK}{UTF8}{mj}计算积分\end{CJK}:
$$
I=\int_{\Gamma} \frac{x \mathrm{~d} y-y \mathrm{~d} x}{x^{2}+y^{2}}
$$
\begin{CJK}{UTF8}{mj}其中\end{CJK}, $\Gamma$ \begin{CJK}{UTF8}{mj}为包含原点的一条分段光滑闭曲线\end{CJK}, \begin{CJK}{UTF8}{mj}取正方向\end{CJK}.

(3) \begin{CJK}{UTF8}{mj}计算曲面积分\end{CJK}
$$
I=\iint_{S} x^{3} \mathrm{~d} y \mathrm{~d} z+y^{3} \mathrm{~d} z \mathrm{~d} x+z^{3} \mathrm{~d} x \mathrm{~d} y
$$
$S$ \begin{CJK}{UTF8}{mj}为椭球面\end{CJK} $\frac{x^{2}}{a^{2}}+\frac{y^{2}}{b^{2}}+\frac{z^{2}}{c^{2}}=1$ \begin{CJK}{UTF8}{mj}的外侧\end{CJK}.

(4) \begin{CJK}{UTF8}{mj}设\end{CJK} $\phi_{n}(x)>0, \phi_{n} \in C[-1,1], \int_{-1}^{1} \phi_{n}(x) \mathrm{d} x=1, n=1,2,3, \cdots$, \begin{CJK}{UTF8}{mj}对于任意的\end{CJK} $c>0, \phi_{n}(x)$ \begin{CJK}{UTF8}{mj}在\end{CJK} $[-1,-c][c, 1]$ \begin{CJK}{UTF8}{mj}一致收敛于\end{CJK} 0 . \begin{CJK}{UTF8}{mj}证明\end{CJK}: \begin{CJK}{UTF8}{mj}对于任意\end{CJK} $g(x) \in C[-1,1]: \lim _{n \rightarrow \infty} g(x) \phi_{n}(x)=g(0)$.

(5) \begin{CJK}{UTF8}{mj}证明\end{CJK}: \begin{CJK}{UTF8}{mj}一个严格递增函数的间断点只能是第一类间断点\end{CJK}.

(6) $f(x, y)$ \begin{CJK}{UTF8}{mj}于\end{CJK} $(-\infty, \infty) \times[a, b)$ \begin{CJK}{UTF8}{mj}连续\end{CJK}, $I(y)=\int_{-\infty}^{\infty} f(x, y) \mathrm{d} x$ \begin{CJK}{UTF8}{mj}于\end{CJK} $y \in[a, b)$ \begin{CJK}{UTF8}{mj}收敛\end{CJK}, \begin{CJK}{UTF8}{mj}但是\end{CJK} $\int_{-\infty}^{\infty} f(x, b) \mathrm{d} x$ \begin{CJK}{UTF8}{mj}发散\end{CJK}, \begin{CJK}{UTF8}{mj}证明\end{CJK}: $I(y)$ \begin{CJK}{UTF8}{mj}于\end{CJK} $y \in[a, b)$ \begin{CJK}{UTF8}{mj}非一致收敛\end{CJK}.

\section{2. 大连理工大学 2013 年研究生入学考试试题数学分析 
 李扬 
 微信公众号: sxkyliyang}
\begin{enumerate}
  \item \begin{CJK}{UTF8}{mj}解答下列各题\end{CJK}, \begin{CJK}{UTF8}{mj}每题\end{CJK} 6 \begin{CJK}{UTF8}{mj}分\end{CJK}.
\end{enumerate}
(1) \begin{CJK}{UTF8}{mj}设\end{CJK} $a_{1}=0.5, a_{n+1}=\frac{a_{n}^{2}+1}{2}, n=1,2, \cdots$. \begin{CJK}{UTF8}{mj}求\end{CJK} $\lim _{n \rightarrow \infty} a_{n}$.

(2) $f(x)$ \begin{CJK}{UTF8}{mj}在\end{CJK} $[a, b]$ \begin{CJK}{UTF8}{mj}上连续\end{CJK}, \begin{CJK}{UTF8}{mj}在\end{CJK} $(a, b)$ \begin{CJK}{UTF8}{mj}上可导\end{CJK}, \begin{CJK}{UTF8}{mj}且\end{CJK} $f(a)=f(b)=0$. \begin{CJK}{UTF8}{mj}求证\end{CJK}: \begin{CJK}{UTF8}{mj}对任意的\end{CJK} $\alpha \in \mathbb{R}$, \begin{CJK}{UTF8}{mj}存在\end{CJK} $\xi \in(a, b)$, \begin{CJK}{UTF8}{mj}使得\end{CJK} $f^{\prime}(\xi)=a f(\xi) .$

(3) \begin{CJK}{UTF8}{mj}设\end{CJK} $\alpha>0, a_{n}=\frac{\alpha(\alpha-1) \cdots(\alpha-n+1)}{n !}, n=1,2, \cdots$. \begin{CJK}{UTF8}{mj}判断级数\end{CJK} $\sum_{n=1}^{\infty}\left|a_{n}\right|$ \begin{CJK}{UTF8}{mj}的敛散性\end{CJK}.

(4) \begin{CJK}{UTF8}{mj}设\end{CJK} $D=\{(x, y): x \geqslant 0, y \geqslant 0,1 \leqslant x+y \leqslant 4\}$, \begin{CJK}{UTF8}{mj}计算二重积分\end{CJK}
$$
\iint_{D} e^{(x+y)^{2}} \mathrm{~d} x \mathrm{~d} y .
$$
(5) \begin{CJK}{UTF8}{mj}将函数\end{CJK} $f(x)=1,0<x<\pi$ \begin{CJK}{UTF8}{mj}展为正弦级数\end{CJK}.

(6) $f(x, y)=\cos (x y)$ \begin{CJK}{UTF8}{mj}在\end{CJK} $\mathbb{R}^{2}$ \begin{CJK}{UTF8}{mj}上是否一致连续\end{CJK}? \begin{CJK}{UTF8}{mj}证明你的结论\end{CJK}.

( 7 ) \begin{CJK}{UTF8}{mj}求幂级数\end{CJK} $\sum_{n=0}^{\infty} \frac{n}{n+1} x^{n}$ \begin{CJK}{UTF8}{mj}的和函数\end{CJK}.

$(8)$ \begin{CJK}{UTF8}{mj}设\end{CJK} $f(x, y)$ \begin{CJK}{UTF8}{mj}在\end{CJK} $\mathbb{R}^{2}$ \begin{CJK}{UTF8}{mj}上可微\end{CJK}, \begin{CJK}{UTF8}{mj}且\end{CJK} $\lim _{r \rightarrow+\infty}\left(x \frac{\partial f}{\partial x}+y \frac{\partial f}{\partial y}\right)=a>0$, \begin{CJK}{UTF8}{mj}其中\end{CJK} $r=\sqrt{x^{2}+y^{2}}$. \begin{CJK}{UTF8}{mj}证明\end{CJK}: $f(x, y)$ \begin{CJK}{UTF8}{mj}在\end{CJK} $\mathbb{R}^{2}$ \begin{CJK}{UTF8}{mj}上有\end{CJK} \begin{CJK}{UTF8}{mj}最小值\end{CJK}.

(9) \begin{CJK}{UTF8}{mj}试举出满足以下条件的函数\end{CJK}: $f(x)$ \begin{CJK}{UTF8}{mj}在\end{CJK} $x=0$ \begin{CJK}{UTF8}{mj}可导\end{CJK}, \begin{CJK}{UTF8}{mj}但\end{CJK} $f(x)$ \begin{CJK}{UTF8}{mj}在任何非零点都不连续\end{CJK}.

(10) $f(x)$ \begin{CJK}{UTF8}{mj}是\end{CJK} $[0,1]$ \begin{CJK}{UTF8}{mj}上的非负连续函数\end{CJK}, \begin{CJK}{UTF8}{mj}且\end{CJK} $f(0)=f(1)=0$. \begin{CJK}{UTF8}{mj}求证\end{CJK}: \begin{CJK}{UTF8}{mj}对任意的\end{CJK} $\alpha \in(0,1)$, \begin{CJK}{UTF8}{mj}存在\end{CJK} $\xi \in[0,1-\alpha]$, \begin{CJK}{UTF8}{mj}使\end{CJK} \begin{CJK}{UTF8}{mj}得\end{CJK} $f(\xi)=f(\xi+\alpha)$.

\begin{enumerate}
  \setcounter{enumi}{2}
  \item \begin{CJK}{UTF8}{mj}证明不等式\end{CJK}: $(1+x)^{1+\frac{1}{x}}<e^{1+\frac{x}{2}}, x>0$.

  \item \begin{CJK}{UTF8}{mj}设\end{CJK} $f(x)$ \begin{CJK}{UTF8}{mj}在\end{CJK} $[0,1]$ \begin{CJK}{UTF8}{mj}上二阶连续可微\end{CJK}, \begin{CJK}{UTF8}{mj}且\end{CJK} $f^{\prime \prime}(x) \geqslant 0, x \in[0,1]$ \begin{CJK}{UTF8}{mj}时\end{CJK}. \begin{CJK}{UTF8}{mj}若\end{CJK} $f(0)=0$, \begin{CJK}{UTF8}{mj}求证\end{CJK}: $\int_{0}^{1} x f(x) \mathrm{d} x \geqslant \frac{2}{3} \int_{0}^{1} f(x) \mathrm{d} x$.

  \item \begin{CJK}{UTF8}{mj}利用含参变量的积分求\end{CJK} $I(a)=\int_{0}^{\frac{\pi}{2}} \frac{\arctan (a \tan x)}{\tan x} \mathrm{~d} x$.

  \item \begin{CJK}{UTF8}{mj}设\end{CJK} $f(x)$ \begin{CJK}{UTF8}{mj}在\end{CJK} $[1,1]$ \begin{CJK}{UTF8}{mj}上三阶连续可微\end{CJK}, \begin{CJK}{UTF8}{mj}且\end{CJK} $f(-1)=0, f(1)=1, f^{\prime}(0)=0$. \begin{CJK}{UTF8}{mj}求证\end{CJK}: \begin{CJK}{UTF8}{mj}存在\end{CJK} $\xi \in(-1,1)$, \begin{CJK}{UTF8}{mj}使得\end{CJK} $f^{\prime \prime \prime}(\xi)=3$.

  \item \begin{CJK}{UTF8}{mj}设\end{CJK} $f(x, y)$ \begin{CJK}{UTF8}{mj}在\end{CJK} $\mathbb{R}^{2}$ \begin{CJK}{UTF8}{mj}上连续可微\end{CJK}, \begin{CJK}{UTF8}{mj}且\end{CJK} $\frac{\partial f}{\partial y}(0,0) \neq 0$. \begin{CJK}{UTF8}{mj}求证\end{CJK}: \begin{CJK}{UTF8}{mj}存在单射\end{CJK} $g:[-1,1] \rightarrow \mathbb{R}^{2}$, \begin{CJK}{UTF8}{mj}使得\end{CJK} $f \circ g$ \begin{CJK}{UTF8}{mj}是常函数\end{CJK}.

  \item \begin{CJK}{UTF8}{mj}计算\end{CJK}

\end{enumerate}
$$
\int_{L}\left(e^{x} \sin x+3 y-\cos y\right) \mathrm{d} x+\left(x \sin y-y^{4}\right) \mathrm{d} y
$$
\begin{CJK}{UTF8}{mj}其中\end{CJK} $L$ \begin{CJK}{UTF8}{mj}是\end{CJK} $y=\sin x, 0 \leqslant x \leqslant \pi$, \begin{CJK}{UTF8}{mj}方向是从\end{CJK} $A:(0,0)$ \begin{CJK}{UTF8}{mj}到\end{CJK} $B:(\pi, 0)$.

\begin{enumerate}
  \setcounter{enumi}{8}
  \item \begin{CJK}{UTF8}{mj}求\end{CJK} $f(x, y, z)=a^{2} x^{2}+b^{2} y^{2}+c^{2} z^{2}-\left(a x^{2}+b y^{2}+c z^{2}\right)^{2}$ \begin{CJK}{UTF8}{mj}在约束\end{CJK} $x^{2}+y^{2}+z^{2}=1$ \begin{CJK}{UTF8}{mj}下的最值\end{CJK}, \begin{CJK}{UTF8}{mj}其中\end{CJK} $a>b>c>0$.

  \item \begin{CJK}{UTF8}{mj}证明\end{CJK}: \begin{CJK}{UTF8}{mj}函数\end{CJK} $f(x)=\sum_{n=0}^{\infty} \frac{\sin \left(2^{n} x\right)}{n !}$ \begin{CJK}{UTF8}{mj}在\end{CJK} $(-\infty,+\infty)$ \begin{CJK}{UTF8}{mj}上二阶连续可微\end{CJK}, \begin{CJK}{UTF8}{mj}并计算\end{CJK} $f^{\prime}(0), f^{\prime \prime}(0)$.

  \item \begin{CJK}{UTF8}{mj}设无穷积分\end{CJK} $\int_{0}^{+\infty} f(x) \mathrm{d} x$ \begin{CJK}{UTF8}{mj}收敛\end{CJK}, \begin{CJK}{UTF8}{mj}证明\end{CJK}: $\lim _{x \rightarrow+\infty} \frac{\int_{0}^{x} t f(t) \mathrm{d} t}{x}=0$.

\end{enumerate}
\section{3. 大连理工大学 2014 年研究生入学考试试题数学分析 
 李扬 
 微信公众号: sxkyliyang}
\begin{enumerate}
  \item \begin{CJK}{UTF8}{mj}解答下列各题\end{CJK}, \begin{CJK}{UTF8}{mj}每题\end{CJK} 6 \begin{CJK}{UTF8}{mj}分\end{CJK}.
\end{enumerate}
(1) \begin{CJK}{UTF8}{mj}令\end{CJK} $b_{n}=a_{0}+a_{1} q+\cdots+a_{n} q^{n}, n=1,2, \cdots$, \begin{CJK}{UTF8}{mj}其中\end{CJK} $\left\{a_{n}\right\}_{n=0}^{\infty}$ \begin{CJK}{UTF8}{mj}是有界数列\end{CJK}, $0<q<1$. \begin{CJK}{UTF8}{mj}证明\end{CJK} $\left\{b_{n}\right\}$ \begin{CJK}{UTF8}{mj}收敛\end{CJK}.

$(2)$ \begin{CJK}{UTF8}{mj}设\end{CJK} $x \rightarrow x_{0}$ \begin{CJK}{UTF8}{mj}时\end{CJK} $\alpha=o(1)$, \begin{CJK}{UTF8}{mj}求证\end{CJK}: $\frac{1}{1+\alpha+\alpha^{2}}=1-\alpha+o\left(\alpha^{2}\right)$.

(3) \begin{CJK}{UTF8}{mj}计算\end{CJK} $\int_{0}^{\frac{\pi}{6}} \mathrm{~d} y \int_{y}^{\frac{\pi}{6}} \frac{\cos x}{x} \mathrm{~d} x$.

$(4)$ \begin{CJK}{UTF8}{mj}设\end{CJK} $f(x) \in C(\mathbb{R})$, \begin{CJK}{UTF8}{mj}且\end{CJK} $\lim _{x \rightarrow \infty} \frac{f(x)}{x^{2 n}}=0$. \begin{CJK}{UTF8}{mj}证明\end{CJK}: \begin{CJK}{UTF8}{mj}函数\end{CJK} $f(x)+x^{2 n}$ \begin{CJK}{UTF8}{mj}在\end{CJK} $\mathbb{R}$ \begin{CJK}{UTF8}{mj}上有最小值\end{CJK}.

(5) \begin{CJK}{UTF8}{mj}将函数\end{CJK} $f(x)=\ln \left(x+\sqrt{1+x^{2}}\right)$ \begin{CJK}{UTF8}{mj}在\end{CJK} $x=0$ \begin{CJK}{UTF8}{mj}处展为幂级数\end{CJK}.

(6) $f(x, y)=\left\{\begin{array}{ll}\frac{x y^{2}}{x^{2}+y^{4}}, & x^{2}+y^{2}>0 ; \\ 0 . & x=y=0 .\end{array}\right.$, \begin{CJK}{UTF8}{mj}证明\end{CJK} $f$ \begin{CJK}{UTF8}{mj}在原点处各个方向导数存在\end{CJK}, \begin{CJK}{UTF8}{mj}但\end{CJK} $f$ \begin{CJK}{UTF8}{mj}在原点不可微\end{CJK}.

$(7)$ \begin{CJK}{UTF8}{mj}设\end{CJK} $f(x)=\left\{\begin{array}{ll}0, & -\pi \leqslant x<0 ; \\ x . & 0 \leqslant x<\pi .\end{array}\right.$, \begin{CJK}{UTF8}{mj}且以\end{CJK} $2 \pi$ \begin{CJK}{UTF8}{mj}为周期\end{CJK}, \begin{CJK}{UTF8}{mj}求\end{CJK} $f(x)$ \begin{CJK}{UTF8}{mj}的\end{CJK} Fourier \begin{CJK}{UTF8}{mj}级数\end{CJK}.

(8) \begin{CJK}{UTF8}{mj}设\end{CJK} $f$ \begin{CJK}{UTF8}{mj}是非负递减函数\end{CJK}, \begin{CJK}{UTF8}{mj}且\end{CJK} $\int_{1}^{+\infty} f(x) \mathrm{d} x$ \begin{CJK}{UTF8}{mj}收敛\end{CJK}, \begin{CJK}{UTF8}{mj}证明\end{CJK}: \begin{CJK}{UTF8}{mj}当\end{CJK} $x \rightarrow+\infty$ \begin{CJK}{UTF8}{mj}时有\end{CJK} $f(x)=o\left(\frac{1}{x}\right)$.

(9) \begin{CJK}{UTF8}{mj}设曲线\end{CJK} $\Gamma: x=e^{t} \cos t, y=e^{t} \sin t, 0 \leqslant t \leqslant 2 \pi$, \begin{CJK}{UTF8}{mj}求曲线\end{CJK} $\Gamma$ \begin{CJK}{UTF8}{mj}的弧长\end{CJK}.

(10) \begin{CJK}{UTF8}{mj}计算\end{CJK} $\iint_{\Sigma} x^{3} \mathrm{~d} y \mathrm{~d} z+y^{3} \mathrm{~d} z \mathrm{~d} x+z^{3} \mathrm{~d} x \mathrm{~d} y, \Sigma$ \begin{CJK}{UTF8}{mj}为球面\end{CJK} $x^{2}+y^{2}+z^{2}=R^{2}$, \begin{CJK}{UTF8}{mj}方向朝外\end{CJK}.

\begin{enumerate}
  \setcounter{enumi}{2}
  \item \begin{CJK}{UTF8}{mj}设有常数\end{CJK} $L>0$ \begin{CJK}{UTF8}{mj}使得\end{CJK} $|f(x)-f(y)| \leqslant L|x-y|, x, y \in[a,+\infty)$, \begin{CJK}{UTF8}{mj}其中\end{CJK} $a>0$. \begin{CJK}{UTF8}{mj}求证\end{CJK}: $\frac{f(x)}{x}$ \begin{CJK}{UTF8}{mj}在\end{CJK} $[a,+\infty)$ \begin{CJK}{UTF8}{mj}上一致连\end{CJK} \begin{CJK}{UTF8}{mj}续\end{CJK}.

  \item \begin{CJK}{UTF8}{mj}设函数\end{CJK} $f(x)$ \begin{CJK}{UTF8}{mj}在\end{CJK} $[0,+\infty)$ \begin{CJK}{UTF8}{mj}上连续可微\end{CJK}, \begin{CJK}{UTF8}{mj}且有\end{CJK} $|f(x)| \leqslant\left|\int_{0}^{x} f(t) \mathrm{d} t\right|, x \geqslant 0$ \begin{CJK}{UTF8}{mj}时\end{CJK}. \begin{CJK}{UTF8}{mj}求证\end{CJK}: $f(x) \equiv 0$.

  \item \begin{CJK}{UTF8}{mj}设\end{CJK} $f(x)$ \begin{CJK}{UTF8}{mj}在\end{CJK} $x=0$ \begin{CJK}{UTF8}{mj}连续\end{CJK}, \begin{CJK}{UTF8}{mj}且\end{CJK} $\lim _{x \rightarrow 0} \frac{f(2 x)-f(x)}{x}=0$, \begin{CJK}{UTF8}{mj}证明\end{CJK}: $f^{\prime}(0)=0$.

  \item \begin{CJK}{UTF8}{mj}设\end{CJK} $f(x)$ \begin{CJK}{UTF8}{mj}在\end{CJK} $x_{0}$ \begin{CJK}{UTF8}{mj}点有\end{CJK} $n+1$ \begin{CJK}{UTF8}{mj}阶导数\end{CJK}, \begin{CJK}{UTF8}{mj}且\end{CJK} $f^{(n+1)}\left(x_{0}\right) \neq 0$. \begin{CJK}{UTF8}{mj}将\end{CJK} $f(x)$ \begin{CJK}{UTF8}{mj}在\end{CJK} $x_{0}$ \begin{CJK}{UTF8}{mj}点作泰勒展开有\end{CJK}

\end{enumerate}
$$
f\left(x_{0}+h\right)=f\left(x_{0}\right)+h f^{\prime}\left(x_{0}\right)+\cdots+\frac{h^{n}}{n !} f^{(n)}\left(x_{0}+\theta_{n} h\right)
$$
\begin{CJK}{UTF8}{mj}其中\end{CJK} $0<\theta_{n}<1$. \begin{CJK}{UTF8}{mj}证明\end{CJK}: $\lim _{h \rightarrow 0} \theta_{n}=\frac{1}{n+1}$.

\begin{enumerate}
  \setcounter{enumi}{6}
  \item \begin{CJK}{UTF8}{mj}设\end{CJK} $f(x)$ \begin{CJK}{UTF8}{mj}在\end{CJK} $[0,1]$ \begin{CJK}{UTF8}{mj}上可微\end{CJK}, \begin{CJK}{UTF8}{mj}且\end{CJK} $f^{\prime}(x) \neq 1, \forall x \in[0,1]$. \begin{CJK}{UTF8}{mj}再设\end{CJK} $f(0)=f(1), \int_{0}^{1} f(x)=0$. \begin{CJK}{UTF8}{mj}证明\end{CJK}: \begin{CJK}{UTF8}{mj}对任意的正整数\end{CJK} $n$, \begin{CJK}{UTF8}{mj}有\end{CJK} $\left|\sum_{k=0}^{n-1} f\left(\frac{k}{n}\right)\right|<\frac{1}{2}$.

  \item \begin{CJK}{UTF8}{mj}级数\end{CJK} $\sum_{n=1}^{\infty} a_{n}$ \begin{CJK}{UTF8}{mj}的部分和数列有界\end{CJK}, \begin{CJK}{UTF8}{mj}级数\end{CJK} $\sum_{n=1}^{\infty}\left(b_{n+1}-b_{n}\right)$ \begin{CJK}{UTF8}{mj}绝对收敛\end{CJK}, \begin{CJK}{UTF8}{mj}且\end{CJK} $\lim _{n \rightarrow \infty} b_{n}=0$. \begin{CJK}{UTF8}{mj}证明\end{CJK}: \begin{CJK}{UTF8}{mj}级数\end{CJK} $\sum_{n=1}^{\infty} a_{n} b_{n}$ \begin{CJK}{UTF8}{mj}收敛\end{CJK}.

  \item \begin{CJK}{UTF8}{mj}试求\end{CJK} $f(x, y)=x^{2}+2 y^{2}$ \begin{CJK}{UTF8}{mj}在闭区域\end{CJK} $D=\left\{(x, y): x^{2}+2 x y+3 y^{2} \leqslant 6\right\}$ \begin{CJK}{UTF8}{mj}上的最大值和最小值\end{CJK}.

  \item \begin{CJK}{UTF8}{mj}证明\end{CJK}: $\int_{0}^{+\infty} x^{-\lambda} e^{\sin x} \sin 2 x \mathrm{~d} x$ \begin{CJK}{UTF8}{mj}关于\end{CJK} $\lambda$ \begin{CJK}{UTF8}{mj}在\end{CJK} $(0,+\infty)$ \begin{CJK}{UTF8}{mj}上收敛\end{CJK}, \begin{CJK}{UTF8}{mj}但不是一致收敛\end{CJK}.

\end{enumerate}
10 . \begin{CJK}{UTF8}{mj}设\end{CJK} $f(x)=\sum_{n=1}^{\infty} \frac{(-1)^{n-1}}{n} e^{-n x}$, \begin{CJK}{UTF8}{mj}证明\end{CJK} $f(x)$ \begin{CJK}{UTF8}{mj}在\end{CJK} $x \geqslant 0$ \begin{CJK}{UTF8}{mj}上连续\end{CJK}, \begin{CJK}{UTF8}{mj}在\end{CJK} $x>0$ \begin{CJK}{UTF8}{mj}上连续可微\end{CJK}.

\section{4. 大连理工大学 2015 年研究生入学考试试题数学分析 
 李扬 
 微信公众号: sxkyliyang}
\begin{enumerate}
  \item \begin{CJK}{UTF8}{mj}解答下列各题\end{CJK}, \begin{CJK}{UTF8}{mj}每题\end{CJK} 6 \begin{CJK}{UTF8}{mj}分\end{CJK}, \begin{CJK}{UTF8}{mj}共\end{CJK} 60 \begin{CJK}{UTF8}{mj}分\end{CJK}.
\end{enumerate}
(1) \begin{CJK}{UTF8}{mj}求\end{CJK} $\lim _{n \rightarrow \infty} \frac{1}{n}\left(\sin \frac{\pi}{n}+\sin \frac{2 \pi}{n}+\cdots+\sin \frac{(n-1) \pi}{n}\right)$.

(2) \begin{CJK}{UTF8}{mj}计算极限\end{CJK}: $\lim _{x \rightarrow a} \frac{x^{a}-a^{x}}{x-a}(a>0)$.

(3) \begin{CJK}{UTF8}{mj}证明下列不等式\end{CJK}: $\int_{0}^{\frac{\pi}{2}} \frac{\sin x}{1+x^{2}} \mathrm{~d} x \leqslant \int_{0}^{\frac{\pi}{2}} \frac{\cos x}{1+x^{2}} \mathrm{~d} x$.

(4) $f(x)=x \cos \frac{1}{x}$ \begin{CJK}{UTF8}{mj}在\end{CJK} $(0,+\infty)$ \begin{CJK}{UTF8}{mj}上是否一致连续\end{CJK}? \begin{CJK}{UTF8}{mj}说明你的理由\end{CJK}.

(5) $f(x)$ \begin{CJK}{UTF8}{mj}在\end{CJK} $[0,1]$ \begin{CJK}{UTF8}{mj}上单调增加\end{CJK}, \begin{CJK}{UTF8}{mj}且\end{CJK} $f(0)>0, f(1)<1$, \begin{CJK}{UTF8}{mj}证明\end{CJK}: \begin{CJK}{UTF8}{mj}存在\end{CJK} $\xi \in(0,1)$ \begin{CJK}{UTF8}{mj}使得\end{CJK} $f(\xi)=\xi^{2}$.

$(6)$ \begin{CJK}{UTF8}{mj}设\end{CJK} $f(x, y)=\left\{\begin{array}{ll}\frac{x y}{\sqrt{x^{2}+y^{2}}}, & x^{2}+y^{2}>0 ; \\ 0 . & x^{2}+y^{2}=0 .\end{array}\right.$, \begin{CJK}{UTF8}{mj}求\end{CJK} $\frac{\partial f}{\partial x}(x, y), \frac{\partial f}{\partial y}(x, y)$.

( 7 ) \begin{CJK}{UTF8}{mj}设\end{CJK} $f(x)=\left(\frac{\pi-x}{2}\right)^{2}, x \in[0,2 \pi]$, \begin{CJK}{UTF8}{mj}求\end{CJK} $f(x)$ \begin{CJK}{UTF8}{mj}的\end{CJK} Fourier \begin{CJK}{UTF8}{mj}级数\end{CJK}, \begin{CJK}{UTF8}{mj}并求此级数\end{CJK} $\sum_{n=1}^{\infty} \frac{1}{n^{2}}$ \begin{CJK}{UTF8}{mj}的和\end{CJK}.

(8) \begin{CJK}{UTF8}{mj}求积分\end{CJK} $\iint_{D} \frac{x}{y} \mathrm{~d} x \mathrm{~d} y$, \begin{CJK}{UTF8}{mj}其中\end{CJK} $D$ \begin{CJK}{UTF8}{mj}由\end{CJK} $x+y=p, x+y=q(0<p<q), x=a y, x=b y(0<a<b)$ \begin{CJK}{UTF8}{mj}所围成\end{CJK}.

(9) \begin{CJK}{UTF8}{mj}求心形线\end{CJK} $r=a(1+\cos \theta)(a>0)$ \begin{CJK}{UTF8}{mj}所围图形的面积\end{CJK}.

$(10)$ \begin{CJK}{UTF8}{mj}若\end{CJK} $\sum u_{n}$ \begin{CJK}{UTF8}{mj}收敛\end{CJK}, $v_{n} \rightarrow 1, n \rightarrow \infty$, \begin{CJK}{UTF8}{mj}则\end{CJK} $\sum u_{n} v_{n}$ \begin{CJK}{UTF8}{mj}是否一定收敛\end{CJK}? \begin{CJK}{UTF8}{mj}说明你的理由\end{CJK}.

\begin{enumerate}
  \setcounter{enumi}{2}
  \item (10 \begin{CJK}{UTF8}{mj}分\end{CJK}) \begin{CJK}{UTF8}{mj}设函数\end{CJK} $f(x)$ \begin{CJK}{UTF8}{mj}在\end{CJK} $[0,1]$ \begin{CJK}{UTF8}{mj}上连续\end{CJK}, \begin{CJK}{UTF8}{mj}在\end{CJK} $(0,1)$ \begin{CJK}{UTF8}{mj}内可导\end{CJK}, $f(0)=0$, \begin{CJK}{UTF8}{mj}并满足\end{CJK} $\left|f^{\prime}(x)\right| \leqslant|f(x)|, x \in(0,1)$, \begin{CJK}{UTF8}{mj}证明\end{CJK}: $f(x) \equiv 0$, \begin{CJK}{UTF8}{mj}其中\end{CJK} $x \in(0,1)$.

  \item ( 10 \begin{CJK}{UTF8}{mj}分\end{CJK}) \begin{CJK}{UTF8}{mj}设\end{CJK} $f(x)$ \begin{CJK}{UTF8}{mj}在\end{CJK} $[2,+\infty)$ \begin{CJK}{UTF8}{mj}上连续可微\end{CJK}, $f^{\prime}(x)$ \begin{CJK}{UTF8}{mj}单调递增\end{CJK}, \begin{CJK}{UTF8}{mj}且\end{CJK} $\lim _{x \rightarrow \infty} f^{\prime}(x)=+\infty$. \begin{CJK}{UTF8}{mj}求证\end{CJK}: $\int_{2}^{+\infty} \cos (f(x)) \mathrm{d} x$ \begin{CJK}{UTF8}{mj}收\end{CJK} \begin{CJK}{UTF8}{mj}敛\end{CJK}.

  \item (10 \begin{CJK}{UTF8}{mj}分\end{CJK}) \begin{CJK}{UTF8}{mj}设\end{CJK} $f(x)$ \begin{CJK}{UTF8}{mj}在\end{CJK} $[0,2 \pi]$ \begin{CJK}{UTF8}{mj}上连续\end{CJK}, $\int_{0}^{2 \pi} f(x) \mathrm{d} x=\frac{\pi}{2}$, \begin{CJK}{UTF8}{mj}证明\end{CJK}: $\lim _{n \rightarrow \infty} \int_{0}^{2 \pi} f(x)|\sin n x| \mathrm{d} x=1$.

  \item ( 10 \begin{CJK}{UTF8}{mj}分\end{CJK}) \begin{CJK}{UTF8}{mj}设\end{CJK} $f_{n}(x)=\frac{x^{n}}{1+x^{2 n}}, n=1,2, \cdots$, \begin{CJK}{UTF8}{mj}分别讨论\end{CJK} $\left\{f_{n}(x)\right\}$ \begin{CJK}{UTF8}{mj}在\end{CJK} $[0,1-\lambda]$ \begin{CJK}{UTF8}{mj}和\end{CJK} $[1-\lambda, 1+\lambda]$ \begin{CJK}{UTF8}{mj}上的一致收敛性\end{CJK}, \begin{CJK}{UTF8}{mj}其中\end{CJK} $0<\lambda<1$ \begin{CJK}{UTF8}{mj}为常数\end{CJK}.

  \item ( 10 \begin{CJK}{UTF8}{mj}分\end{CJK}) \begin{CJK}{UTF8}{mj}设\end{CJK} $F(u)=\int_{0}^{+\infty} \frac{1-e^{-u t}}{t} \sin t \mathrm{~d} t$, \begin{CJK}{UTF8}{mj}证明\end{CJK}: $F(u)$ \begin{CJK}{UTF8}{mj}在\end{CJK} $[0,+\infty)$ \begin{CJK}{UTF8}{mj}上连续\end{CJK}, \begin{CJK}{UTF8}{mj}在\end{CJK} $(0,+\infty)$ \begin{CJK}{UTF8}{mj}上可微\end{CJK}, \begin{CJK}{UTF8}{mj}并求\end{CJK} $F(u)$ \begin{CJK}{UTF8}{mj}的\end{CJK} \begin{CJK}{UTF8}{mj}表达式\end{CJK}.

  \item ( 10 \begin{CJK}{UTF8}{mj}分\end{CJK}) \begin{CJK}{UTF8}{mj}求\end{CJK} $\int_{\Gamma}(y+z) \mathrm{d} x+(z+x) \mathrm{d} y+(x+y) \mathrm{d} z, \Gamma$ \begin{CJK}{UTF8}{mj}为椭圆\end{CJK} $x^{2}+y^{2}=2 y, y=z$ \begin{CJK}{UTF8}{mj}从点\end{CJK} $(0,1,0)$ \begin{CJK}{UTF8}{mj}向\end{CJK} $\Gamma$ \begin{CJK}{UTF8}{mj}看去\end{CJK}, $\Gamma$ \begin{CJK}{UTF8}{mj}为逆时针方向\end{CJK}.

  \item (10 \begin{CJK}{UTF8}{mj}分\end{CJK}) \begin{CJK}{UTF8}{mj}椭球面\end{CJK} $\frac{x^{2}}{a^{2}}+\frac{y^{2}}{b^{2}}+\frac{z^{2}}{c^{2}}=1$ \begin{CJK}{UTF8}{mj}在第一卦限中某点的切平面与坐标平面围成一个四面体\end{CJK}, \begin{CJK}{UTF8}{mj}求四面体体积的\end{CJK} \begin{CJK}{UTF8}{mj}最小值\end{CJK}.

  \item ( 10 \begin{CJK}{UTF8}{mj}分\end{CJK}) \begin{CJK}{UTF8}{mj}设\end{CJK} $\left\{a_{n}\right\},\left\{b_{n}\right\}, n=1,2,3, \cdots$ \begin{CJK}{UTF8}{mj}满足\end{CJK} $e^{a_{n}}=a_{n}+e^{b_{n}}, a_{n}>0$. \begin{CJK}{UTF8}{mj}证明\end{CJK}: \begin{CJK}{UTF8}{mj}若数项级数\end{CJK} $\sum_{n=1}^{\infty} a_{n}$ \begin{CJK}{UTF8}{mj}收敛\end{CJK}, \begin{CJK}{UTF8}{mj}则\end{CJK} $\sum \frac{b^{n}}{a^{n}}$ \begin{CJK}{UTF8}{mj}收敛\end{CJK}.

\end{enumerate}
\section{5. 大连理工大学 2016 年研究生入学考试试题数学分析 
 李扬 
 微信公众号: sxkyliyang}
\begin{enumerate}
  \item \begin{CJK}{UTF8}{mj}解答下列各题\end{CJK}, \begin{CJK}{UTF8}{mj}每题\end{CJK} 6 \begin{CJK}{UTF8}{mj}分\end{CJK}, \begin{CJK}{UTF8}{mj}共\end{CJK} 60 \begin{CJK}{UTF8}{mj}分\end{CJK}.
\end{enumerate}
(1) \begin{CJK}{UTF8}{mj}求极限\end{CJK}: $\lim _{n \rightarrow \infty} \frac{1}{n} \sqrt[n]{n(n+1)(n+2) \cdots(2 n-1)}$.

(2) \begin{CJK}{UTF8}{mj}计算\end{CJK}: $\lim _{x \rightarrow+\infty} \int_{0}^{x}|\sin t| \mathrm{d} t$.

(3) \begin{CJK}{UTF8}{mj}证明\end{CJK}: \begin{CJK}{UTF8}{mj}若\end{CJK} $f(x)$ \begin{CJK}{UTF8}{mj}为\end{CJK} $[a, b]$ \begin{CJK}{UTF8}{mj}上的凸函数\end{CJK}, \begin{CJK}{UTF8}{mj}则\end{CJK} $f\left(\frac{a+b}{2}\right) \leqslant \frac{1}{b-a} \int_{a}^{b} f(x) \mathrm{d} x \leqslant \frac{f(a)+f(b)}{2}$.

(4) \begin{CJK}{UTF8}{mj}确定\end{CJK} $a, b$ \begin{CJK}{UTF8}{mj}的值\end{CJK}, \begin{CJK}{UTF8}{mj}使得函数\end{CJK} $f(x)= \begin{cases}\frac{1}{x}(1-\cos a x), & x<0 \\ 0, & x=0 \text { 在 }(-\infty,+\infty) \text { 内处处可导, 并求它的导数. } \\ \frac{1}{x} \ln \left(b+x^{2}\right), & x<0\end{cases}$

( 5 ) \begin{CJK}{UTF8}{mj}设函数\end{CJK} $f(x)$ \begin{CJK}{UTF8}{mj}在闭区间\end{CJK} $[a, b]$ \begin{CJK}{UTF8}{mj}上连续\end{CJK}, \begin{CJK}{UTF8}{mj}证明\end{CJK}:
$$
\lim _{h \rightarrow 0^{+}} \frac{1}{h} \int_{a}^{x}[f(t+h)-f(t)] \mathrm{d} t=f(x)-f(a), \text { 其中 } a<x<b .
$$
(6) \begin{CJK}{UTF8}{mj}把函数\end{CJK} $f(x)=\left\{\begin{array}{ll}-\frac{\pi}{4}, & -\pi<x<0 \\ \frac{\pi}{4} . & 0<x<\pi .\end{array}\right.$ \begin{CJK}{UTF8}{mj}展开成\end{CJK} Fourier \begin{CJK}{UTF8}{mj}级数\end{CJK}, \begin{CJK}{UTF8}{mj}并由此推出\end{CJK}: $\frac{\pi}{4}=1-\frac{1}{3}+\frac{1}{5}-\frac{1}{7}+\cdots .$

$(7)$ \begin{CJK}{UTF8}{mj}证明\end{CJK}: \begin{CJK}{UTF8}{mj}若\end{CJK} $f(x)$ \begin{CJK}{UTF8}{mj}在\end{CJK} $[a,+\infty)$ \begin{CJK}{UTF8}{mj}上一致连续\end{CJK}, \begin{CJK}{UTF8}{mj}且\end{CJK} $\int_{a}^{+\infty} f(x) \mathrm{d} x$ \begin{CJK}{UTF8}{mj}收敛\end{CJK}, \begin{CJK}{UTF8}{mj}则\end{CJK} $\lim _{x \rightarrow+\infty} f(x)=0$.

(8) \begin{CJK}{UTF8}{mj}将函数\end{CJK} $f(x)=\sin ^{3} x$ \begin{CJK}{UTF8}{mj}在\end{CJK} $x=\frac{\pi}{6}$ \begin{CJK}{UTF8}{mj}处展开成幂级数\end{CJK}.

(9) \begin{CJK}{UTF8}{mj}求\end{CJK} $x^{2}+y^{2} \leqslant R^{2}$ \begin{CJK}{UTF8}{mj}与\end{CJK} $x^{2}+z^{2} \leqslant R^{2}$ \begin{CJK}{UTF8}{mj}公共部分的表面积与体积\end{CJK}.

( 10 ) \begin{CJK}{UTF8}{mj}若存在\end{CJK} $C$, \begin{CJK}{UTF8}{mj}使得\end{CJK} $\left|x_{2}-x_{1}\right|+\left|x_{3}-x_{2}\right|+\cdots+\left|x_{n}-x_{n-1}\right|<C$, \begin{CJK}{UTF8}{mj}称\end{CJK} $\left\{x_{n}\right\}$ \begin{CJK}{UTF8}{mj}为有界变差序列\end{CJK}, \begin{CJK}{UTF8}{mj}证明\end{CJK}: \begin{CJK}{UTF8}{mj}凡是有界\end{CJK} \begin{CJK}{UTF8}{mj}变差序列都是收敛的\end{CJK}, \begin{CJK}{UTF8}{mj}并举出一个收敛却无有界变差的例子\end{CJK}.

\begin{enumerate}
  \setcounter{enumi}{2}
  \item ( 10 \begin{CJK}{UTF8}{mj}分\end{CJK}) \begin{CJK}{UTF8}{mj}设函数\end{CJK} $f(x)$ \begin{CJK}{UTF8}{mj}在\end{CJK} $x=0$ \begin{CJK}{UTF8}{mj}的某邻域内有界\end{CJK}, \begin{CJK}{UTF8}{mj}且满足\end{CJK} $f(\alpha x)=\beta f(x), \alpha>1, \beta>1$, \begin{CJK}{UTF8}{mj}证明\end{CJK}: $f(x)$ \begin{CJK}{UTF8}{mj}在\end{CJK} $x=0$ \begin{CJK}{UTF8}{mj}处连续\end{CJK}.

  \item ( 10 \begin{CJK}{UTF8}{mj}分\end{CJK}) \begin{CJK}{UTF8}{mj}设函数\end{CJK} $f(x), g(x)$ \begin{CJK}{UTF8}{mj}在\end{CJK} $[a, b]$ \begin{CJK}{UTF8}{mj}上连续\end{CJK}, $f(x)$ \begin{CJK}{UTF8}{mj}不恒等于\end{CJK} $0, g(x)$ \begin{CJK}{UTF8}{mj}有正下界\end{CJK}, \begin{CJK}{UTF8}{mj}记\end{CJK} $d_{n}=\int_{a}^{b}|f(x)|^{n} g(x) \mathrm{d} x$, $n=1,2, \cdots$, \begin{CJK}{UTF8}{mj}证明\end{CJK}:

\end{enumerate}
$$
\lim _{n \rightarrow \infty} \frac{d_{n+1}}{d_{n}}=\max _{a \leqslant x \leqslant b}|f(x)|
$$

\begin{enumerate}
  \setcounter{enumi}{4}
  \item ( 10 \begin{CJK}{UTF8}{mj}分\end{CJK}) \begin{CJK}{UTF8}{mj}设函数\end{CJK} $f(x)$ \begin{CJK}{UTF8}{mj}在\end{CJK} $[0,1]$ \begin{CJK}{UTF8}{mj}上连续可微\end{CJK}, \begin{CJK}{UTF8}{mj}试证明\end{CJK}:
\end{enumerate}
$$
\lim _{n \rightarrow \infty} n \int_{0}^{1} x^{n} f(x) \mathrm{d} x=f(1)
$$

\begin{enumerate}
  \setcounter{enumi}{5}
  \item ( 10 \begin{CJK}{UTF8}{mj}分\end{CJK}) \begin{CJK}{UTF8}{mj}设函数\end{CJK} $f(x)$ \begin{CJK}{UTF8}{mj}在\end{CJK} $\Omega(t)=\left\{(x, y, z) \mid x^{2}+y^{2}+z^{2} \leqslant t^{2}\right\}$ \begin{CJK}{UTF8}{mj}上连续\end{CJK}, $S=\left\{(x, y, z) \mid x^{2}+y^{2}+z^{2}=t^{2}\right\}$, \begin{CJK}{UTF8}{mj}证明\end{CJK}:
\end{enumerate}
$$
\frac{\mathrm{d}}{\mathrm{d} t} \iiint_{\Omega(t)} f(x, y, z) \mathrm{d} V=\iint_{S} f(x, y, z) \mathrm{d} S .
$$

\begin{enumerate}
  \setcounter{enumi}{6}
  \item (10 \begin{CJK}{UTF8}{mj}分\end{CJK}) \begin{CJK}{UTF8}{mj}设\end{CJK} $f(x)$ \begin{CJK}{UTF8}{mj}为\end{CJK} $[a,+\infty)$ \begin{CJK}{UTF8}{mj}上的连续可微函数\end{CJK}, \begin{CJK}{UTF8}{mj}且当\end{CJK} $x \rightarrow+\infty$ \begin{CJK}{UTF8}{mj}时\end{CJK}, $f(x)$ \begin{CJK}{UTF8}{mj}递减趋于\end{CJK} 0 , \begin{CJK}{UTF8}{mj}则\end{CJK} $\int_{0}^{+\infty} f(x) \mathrm{d} x$ \begin{CJK}{UTF8}{mj}收敛等\end{CJK} \begin{CJK}{UTF8}{mj}价于\end{CJK} $\int_{0}^{+\infty} x f^{\prime}(x) \mathrm{d} x$ \begin{CJK}{UTF8}{mj}收敛\end{CJK}. 7. (10 \begin{CJK}{UTF8}{mj}分\end{CJK}) \begin{CJK}{UTF8}{mj}证明\end{CJK}: $\int_{0}^{+\infty} \frac{\sin x y}{x} \mathrm{~d} x$ \begin{CJK}{UTF8}{mj}关于\end{CJK} $y$ \begin{CJK}{UTF8}{mj}在\end{CJK} $[a,+\infty)(a>0)$ \begin{CJK}{UTF8}{mj}上一致收敛\end{CJK}, \begin{CJK}{UTF8}{mj}但是在\end{CJK} $(0,+\infty)$ \begin{CJK}{UTF8}{mj}上非一致收敛\end{CJK}.

  \item (10 \begin{CJK}{UTF8}{mj}分\end{CJK}) \begin{CJK}{UTF8}{mj}求积分\end{CJK}

\end{enumerate}
$$
\iint_{\Sigma} 4 x z \mathrm{~d} y \mathrm{~d} z-2 y z \mathrm{~d} z \mathrm{~d} x+\left(1-z^{2}\right) \mathrm{d} x \mathrm{~d} y,
$$
\begin{CJK}{UTF8}{mj}其中\end{CJK} $\Sigma$ \begin{CJK}{UTF8}{mj}为\end{CJK} $z=e^{y}, y \in[0, a]$ \begin{CJK}{UTF8}{mj}绕\end{CJK} $z$ \begin{CJK}{UTF8}{mj}轴旋转一周形成的曲面\end{CJK}, \begin{CJK}{UTF8}{mj}方向取下侧\end{CJK}.

\begin{enumerate}
  \setcounter{enumi}{9}
  \item (10 \begin{CJK}{UTF8}{mj}分\end{CJK}) \begin{CJK}{UTF8}{mj}试求\end{CJK} $f(x, y)=x^{2}+y^{2}-2 x$ \begin{CJK}{UTF8}{mj}在闭区域\end{CJK} $D=\left\{(x, y): x^{2}+y^{2} \leqslant 4\right\}$ \begin{CJK}{UTF8}{mj}上的最大值和最小值\end{CJK}.

  \item (10 \begin{CJK}{UTF8}{mj}分\end{CJK}) \begin{CJK}{UTF8}{mj}证明\end{CJK}: $\sum_{n=0}^{\infty}(-1)^{n}(1-x) x^{n}$ \begin{CJK}{UTF8}{mj}在\end{CJK} $[0,1]$ \begin{CJK}{UTF8}{mj}上绝对收敛且一致收敛\end{CJK}, \begin{CJK}{UTF8}{mj}但是并不绝对一致收敛\end{CJK}.

\end{enumerate}
\section{6. 大连理工大学 2017 年研究生入学考试试题数学分析 
 李扬 
 微信公众号: sxkyliyang}
\begin{enumerate}
  \item \begin{CJK}{UTF8}{mj}解答下列各题\end{CJK}, \begin{CJK}{UTF8}{mj}每题\end{CJK} 6 \begin{CJK}{UTF8}{mj}分\end{CJK}, \begin{CJK}{UTF8}{mj}共\end{CJK} 60 \begin{CJK}{UTF8}{mj}分\end{CJK}.
\end{enumerate}
(1) \begin{CJK}{UTF8}{mj}设\end{CJK} $\lim _{n \rightarrow \infty} \frac{a_{n}}{n}=0$, \begin{CJK}{UTF8}{mj}证明\end{CJK}: $\lim _{n \rightarrow \infty} \frac{\max \left\{a_{1}, a_{2}, \cdots, a_{n}\right\}}{n}=0$.

(2) $0 \leqslant b \leqslant a, p \geqslant 2$, \begin{CJK}{UTF8}{mj}证明\end{CJK}: $(a+b)^{p}+(a-b)^{p} \leqslant 2^{p-1}\left(a^{p}+b^{p}\right)$.

(3) $\int_{0}^{1} \mathrm{~d} x \int_{0}^{\sqrt{x}} e^{-\frac{y^{2}}{2}} \mathrm{~d} y$.

(4) $f(x)$ \begin{CJK}{UTF8}{mj}是周期函数\end{CJK}, \begin{CJK}{UTF8}{mj}且\end{CJK} $\lim _{x \rightarrow+\infty} f(x)=a$, \begin{CJK}{UTF8}{mj}证明\end{CJK}: $f(x)$ \begin{CJK}{UTF8}{mj}是常函数\end{CJK}.

$(5)$ \begin{CJK}{UTF8}{mj}设\end{CJK} $f(x)=\left\{\begin{array}{ll}x^{2} \sin (\ln |x|), & x \neq 0 \\ 0 . & x=0\end{array}\right.$ \begin{CJK}{UTF8}{mj}证明\end{CJK} $f(x)$ \begin{CJK}{UTF8}{mj}在\end{CJK} $x=0$ \begin{CJK}{UTF8}{mj}有一阶导数\end{CJK}, \begin{CJK}{UTF8}{mj}但无二阶导数\end{CJK}.

(6) $f(x, y)$ \begin{CJK}{UTF8}{mj}在\end{CJK} $\mathbb{R}^{2}$ \begin{CJK}{UTF8}{mj}上连续\end{CJK}, $g(x, y)=y f(x, y)$ \begin{CJK}{UTF8}{mj}求\end{CJK} $\frac{\partial g}{\partial x}(0,0), \frac{\partial g}{\partial y}(0,0)$.

$(7)$ \begin{CJK}{UTF8}{mj}令\end{CJK} $f(x)=\sum_{n=1}^{\infty} \frac{x^{n}}{n^{2}}, 0 \leqslant x \leqslant 1$, \begin{CJK}{UTF8}{mj}证明存在常数\end{CJK} $c$, \begin{CJK}{UTF8}{mj}使得\end{CJK} $f(x)+f(1-x)+\ln x \ln (1-x)=c$.

(8) \begin{CJK}{UTF8}{mj}设\end{CJK} $f(x)=\left\{\begin{array}{ll}1, & 0 \leqslant x \leqslant \pi \\ -1 . & -\pi<x<0\end{array}\right.$ \begin{CJK}{UTF8}{mj}是以\end{CJK} $2 \pi$ \begin{CJK}{UTF8}{mj}为周期的周期函数\end{CJK}, \begin{CJK}{UTF8}{mj}计算\end{CJK} $f(x)$ \begin{CJK}{UTF8}{mj}的傅里叶级数展开式\end{CJK}.

(9) \begin{CJK}{UTF8}{mj}设\end{CJK} $f(x, y)=\left\{\begin{array}{ll}\frac{x^{2} y}{x^{4}+y^{2}}, & x^{2}+y^{2}>0 \\ 0 . & x^{2}+y^{2}=0\end{array}\right.$, \begin{CJK}{UTF8}{mj}证明\end{CJK} $f(x, y)$ \begin{CJK}{UTF8}{mj}在点\end{CJK} $(0,0)$ \begin{CJK}{UTF8}{mj}沿任意方向的方向导数存在\end{CJK}, \begin{CJK}{UTF8}{mj}但\end{CJK} $f(x, y)$ \begin{CJK}{UTF8}{mj}在\end{CJK} $(0,0)$ \begin{CJK}{UTF8}{mj}不可微\end{CJK}.

(10) \begin{CJK}{UTF8}{mj}设级数\end{CJK} $\sum_{n=1}^{\infty} a_{n}, \sum_{n=1}^{\infty} b_{n}$ \begin{CJK}{UTF8}{mj}均收敛\end{CJK}, $\sum_{n=1}^{\infty} a_{n} b_{n}$ \begin{CJK}{UTF8}{mj}是否收敛\end{CJK}.

\begin{enumerate}
  \setcounter{enumi}{2}
  \item (10 \begin{CJK}{UTF8}{mj}分\end{CJK}) $f(x, y)$ \begin{CJK}{UTF8}{mj}在\end{CJK} $[a, b]$ \begin{CJK}{UTF8}{mj}上有定义\end{CJK}, \begin{CJK}{UTF8}{mj}若\end{CJK} $f(x)$ \begin{CJK}{UTF8}{mj}的每一个值都恰能取到三次\end{CJK}, \begin{CJK}{UTF8}{mj}证明\end{CJK} $f(x)$ \begin{CJK}{UTF8}{mj}在\end{CJK} $[a, b]$ \begin{CJK}{UTF8}{mj}上不连续\end{CJK}.

  \item ( 10 \begin{CJK}{UTF8}{mj}分\end{CJK}) \begin{CJK}{UTF8}{mj}设\end{CJK} $f(x)$ \begin{CJK}{UTF8}{mj}在\end{CJK} $[-1,1]$ \begin{CJK}{UTF8}{mj}上有三阶连续导数\end{CJK}, $f(-1)=0, f(1)=1, f^{\prime}(0)=0$, \begin{CJK}{UTF8}{mj}证明\end{CJK}: \begin{CJK}{UTF8}{mj}存在\end{CJK} $\xi \in(-1,1)$, \begin{CJK}{UTF8}{mj}使\end{CJK} $f^{\prime \prime}(\xi)=3$.

  \item ( 10 \begin{CJK}{UTF8}{mj}分\end{CJK}) \begin{CJK}{UTF8}{mj}设\end{CJK} $f_{n}(x)$ \begin{CJK}{UTF8}{mj}均是上的连续递增函数且函数列\end{CJK} $\left\{f_{n}(x)\right\}$ \begin{CJK}{UTF8}{mj}在\end{CJK} $[a, b]$ \begin{CJK}{UTF8}{mj}上逐点收敛于连续函数\end{CJK} $f(x)$, \begin{CJK}{UTF8}{mj}证明\end{CJK}: $\left\{f_{n}(x)\right\}$ \begin{CJK}{UTF8}{mj}在\end{CJK} $[a, b]$ \begin{CJK}{UTF8}{mj}上一致连续\end{CJK}.

  \item (10 \begin{CJK}{UTF8}{mj}分\end{CJK}) \begin{CJK}{UTF8}{mj}设\end{CJK} $f(x)$ \begin{CJK}{UTF8}{mj}在\end{CJK} $[0,+\infty)$ \begin{CJK}{UTF8}{mj}上有二阶连续导数\end{CJK}, \begin{CJK}{UTF8}{mj}且是有界函数\end{CJK}, $f(0)=0$, \begin{CJK}{UTF8}{mj}证明\end{CJK}:

\end{enumerate}
$$
\lim _{n \rightarrow \infty} \int_{0}^{+\infty} \frac{f(x)}{x^{2}} \sin n x \mathrm{~d} x=\frac{\pi}{2} f^{\prime}(0)
$$

\begin{enumerate}
  \setcounter{enumi}{6}
  \item (10 \begin{CJK}{UTF8}{mj}分\end{CJK}) \begin{CJK}{UTF8}{mj}设广义积分\end{CJK} $\int_{-\infty}^{+\infty} f(x) \mathrm{d} x$ \begin{CJK}{UTF8}{mj}绝对收敛\end{CJK}, $g(x)$ \begin{CJK}{UTF8}{mj}一致连续且有界\end{CJK}. \begin{CJK}{UTF8}{mj}证明\end{CJK}: $F(u)=\int_{-\infty}^{+\infty} f(x) g(u x) \mathrm{d} x$ \begin{CJK}{UTF8}{mj}一致\end{CJK} \begin{CJK}{UTF8}{mj}连续\end{CJK}.

  \item ( 10 \begin{CJK}{UTF8}{mj}分\end{CJK}) \begin{CJK}{UTF8}{mj}设\end{CJK} $f(x, y)$ \begin{CJK}{UTF8}{mj}在\end{CJK} $\mathbb{R}^{2}$ \begin{CJK}{UTF8}{mj}上有连续偏导数\end{CJK}, \begin{CJK}{UTF8}{mj}且\end{CJK} $f(x, y)=0$ \begin{CJK}{UTF8}{mj}当且仅当\end{CJK} $(x, y)=(0,0)$ \begin{CJK}{UTF8}{mj}证明\end{CJK}:

\end{enumerate}
$$
\frac{\partial f}{\partial x}(0,0)=\frac{\partial f}{\partial y}(0,0)=0
$$

\begin{enumerate}
  \setcounter{enumi}{8}
  \item (10 \begin{CJK}{UTF8}{mj}分\end{CJK}) \begin{CJK}{UTF8}{mj}求\end{CJK} $f(x, y, z)=5 x^{2}+5 y^{2}-8 x y$ \begin{CJK}{UTF8}{mj}在条件\end{CJK} $x^{2}+y^{2}-x y=73$ \begin{CJK}{UTF8}{mj}下的极值\end{CJK}. 9. (10 \begin{CJK}{UTF8}{mj}分\end{CJK}) \begin{CJK}{UTF8}{mj}设曲面\end{CJK} $P:|x-y+z|+|y-z+x|+|z-x+y|=1$ \begin{CJK}{UTF8}{mj}取外侧\end{CJK}, \begin{CJK}{UTF8}{mj}求\end{CJK}
\end{enumerate}
$$
\iint_{P}(x-y+z) \mathrm{d} y \mathrm{~d} z+(y-z+x) \mathrm{d} z \mathrm{~d} x+(z-x+y) \mathrm{d} x \mathrm{~d} y
$$

\begin{enumerate}
  \setcounter{enumi}{10}
  \item ( 10 \begin{CJK}{UTF8}{mj}分\end{CJK}) \begin{CJK}{UTF8}{mj}设\end{CJK} $\lim _{n \rightarrow \infty} a_{n}=a \neq 0$, \begin{CJK}{UTF8}{mj}且\end{CJK} $a_{n} \neq 0, n=1,2, \cdots$, \begin{CJK}{UTF8}{mj}问级数\end{CJK} $\sum_{n=1}^{\infty}\left|a_{n+1}-a_{n}\right|$ \begin{CJK}{UTF8}{mj}与\end{CJK} $\sum_{n=1}^{\infty}\left|\frac{1}{a_{n+1}}-\frac{1}{a_{n}}\right|$ \begin{CJK}{UTF8}{mj}是否同敛散\end{CJK}.
\end{enumerate}
\section{1. 电子科技大学 2009 年研究生入学考试试题数学分析}
\begin{CJK}{UTF8}{mj}李扬\end{CJK}

\begin{CJK}{UTF8}{mj}微信公众号\end{CJK}: sxkyliyang

\begin{CJK}{UTF8}{mj}一\end{CJK}、\begin{CJK}{UTF8}{mj}填空题\end{CJK}(\begin{CJK}{UTF8}{mj}每小题\end{CJK} 4 \begin{CJK}{UTF8}{mj}分\end{CJK}, \begin{CJK}{UTF8}{mj}共\end{CJK} 40 \begin{CJK}{UTF8}{mj}分\end{CJK})

\begin{enumerate}
  \item \begin{CJK}{UTF8}{mj}已知\end{CJK} $a_{1}=\sqrt{5}, a_{n}=\sqrt{5+a_{n-1}}(n=2,3, \cdots)$, \begin{CJK}{UTF8}{mj}则\end{CJK} $\lim _{n \rightarrow \infty} a_{n}=$

  \item $\lim _{x \rightarrow \infty}\left(\frac{x+1}{x-1}\right)^{\frac{2 x^{2}-1}{x+1}}=$

  \item \begin{CJK}{UTF8}{mj}曲线\end{CJK} $y=\ln \left(1-x^{2}\right), 0 \leqslant x \leqslant \frac{1}{2}$ \begin{CJK}{UTF8}{mj}的弧长为\end{CJK}

  \item $x^{y}=y^{x}$, \begin{CJK}{UTF8}{mj}则\end{CJK} $y^{\prime}(1)=$

  \item \begin{CJK}{UTF8}{mj}已知\end{CJK} $x^{2} y+x y^{2}=1, x$ \begin{CJK}{UTF8}{mj}为自变量\end{CJK}, \begin{CJK}{UTF8}{mj}则\end{CJK} $\mathrm{d}^{2} y=$

  \item \begin{CJK}{UTF8}{mj}当\end{CJK} $a=$ $b=$ \begin{CJK}{UTF8}{mj}时\end{CJK}, \begin{CJK}{UTF8}{mj}区间\end{CJK} $[a, b]$ \begin{CJK}{UTF8}{mj}上的积分\end{CJK} $\int_{a}^{b}\left(2+x-x^{2}\right) \mathrm{d} x$ \begin{CJK}{UTF8}{mj}的值最大\end{CJK};

  \item \begin{CJK}{UTF8}{mj}设\end{CJK} $b>a>0$, \begin{CJK}{UTF8}{mj}则含参变量积分\end{CJK} $\int_{0}^{1} \frac{x^{b}-x^{a}}{\ln x} \mathrm{~d} x$ \begin{CJK}{UTF8}{mj}的值为\end{CJK}

  \item \begin{CJK}{UTF8}{mj}设\end{CJK} $f(x)=\left\{\begin{array}{l}e^{3 x}, x \in[-\pi, 0) \\ \cos x, x \in[0, \pi)\end{array}\right.$ \begin{CJK}{UTF8}{mj}的\end{CJK} Fourier \begin{CJK}{UTF8}{mj}级数在\end{CJK} $x=0$ \begin{CJK}{UTF8}{mj}点的值为\end{CJK}

  \item \begin{CJK}{UTF8}{mj}若\end{CJK} $D$ \begin{CJK}{UTF8}{mj}为抛物线\end{CJK} $y^{2}=x$ \begin{CJK}{UTF8}{mj}与直线\end{CJK} $y=x-2$ \begin{CJK}{UTF8}{mj}所围成的闭区域\end{CJK}, \begin{CJK}{UTF8}{mj}则\end{CJK} $\iint_{D} x y \mathrm{~d} x \mathrm{~d} y=$

  \item \begin{CJK}{UTF8}{mj}设\end{CJK} $L$ \begin{CJK}{UTF8}{mj}为平面\end{CJK} $x+y+z=1$ \begin{CJK}{UTF8}{mj}被三个坐标面所截三角形\end{CJK} $\Sigma$ \begin{CJK}{UTF8}{mj}的边界\end{CJK}, \begin{CJK}{UTF8}{mj}若从\end{CJK} $x$ \begin{CJK}{UTF8}{mj}轴的正向去\end{CJK}, $L$ \begin{CJK}{UTF8}{mj}的定向为逆时针方向\end{CJK}, \begin{CJK}{UTF8}{mj}曲线积分\end{CJK}

\end{enumerate}
$$
\oint_{L}\left(y^{2}-z^{2}\right) \mathrm{d} x+\left(z^{2}-x^{2}\right) \mathrm{d} y+\left(x^{2}-y^{2}\right) \mathrm{d} z
$$
\begin{CJK}{UTF8}{mj}的值为\end{CJK}

\section{一、证明题}
\begin{enumerate}
  \item (12 \begin{CJK}{UTF8}{mj}分\end{CJK}) \begin{CJK}{UTF8}{mj}若数列\end{CJK} $\left\{x_{n}\right\}$ \begin{CJK}{UTF8}{mj}无界\end{CJK}, \begin{CJK}{UTF8}{mj}但非无穷大量\end{CJK}, \begin{CJK}{UTF8}{mj}证明\end{CJK}: \begin{CJK}{UTF8}{mj}存在子列\end{CJK} $\left\{x_{n_{k}}^{(1)}\right\} 与\left\{x_{n_{k}}^{(2)}\right\}$, \begin{CJK}{UTF8}{mj}其中\end{CJK} $\left\{x_{n_{k}}^{(1)}\right\}$ \begin{CJK}{UTF8}{mj}是无穷大量\end{CJK}, \begin{CJK}{UTF8}{mj}而\end{CJK} $\left\{x_{n_{k}}^{(2)}\right\}$ \begin{CJK}{UTF8}{mj}是收敛子列\end{CJK}.

  \item (12 \begin{CJK}{UTF8}{mj}分\end{CJK}) \begin{CJK}{UTF8}{mj}证明\end{CJK}: \begin{CJK}{UTF8}{mj}对于\end{CJK} $\forall r \in(0,1), f(x)=\frac{1}{x}$ \begin{CJK}{UTF8}{mj}在区间\end{CJK} $(r, 1)$ \begin{CJK}{UTF8}{mj}上一直连续\end{CJK}, \begin{CJK}{UTF8}{mj}而在区间\end{CJK} $(0,1)$ \begin{CJK}{UTF8}{mj}不一致连续\end{CJK}.

  \item (12 \begin{CJK}{UTF8}{mj}分\end{CJK}) Riemann \begin{CJK}{UTF8}{mj}函数\end{CJK}

\end{enumerate}
$$
R(x)=\left\{\begin{array}{l}
\frac{1}{q}, x=\frac{p}{q}\left(\frac{p}{q} \text { 为既约真分数 }\right) ; \\
1, x=0 ; \\
0, x \text { 是无理数 }
\end{array}\right.
$$
\begin{CJK}{UTF8}{mj}在无理点处处连续\end{CJK}, \begin{CJK}{UTF8}{mj}而在有理点处处不连续\end{CJK}.

\begin{enumerate}
  \setcounter{enumi}{4}
  \item (12 \begin{CJK}{UTF8}{mj}分\end{CJK}) \begin{CJK}{UTF8}{mj}证明\end{CJK}: \begin{CJK}{UTF8}{mj}极限\end{CJK} $\lim _{x \rightarrow a} f(x)$ \begin{CJK}{UTF8}{mj}存在当且仅当对于任何数列\end{CJK} $\left\{x_{n}\right\}: x_{n} \neq a(n=1,2, \cdots)$, \begin{CJK}{UTF8}{mj}若\end{CJK} $\lim _{n \rightarrow \infty} x_{n}=a$, \begin{CJK}{UTF8}{mj}则\end{CJK} $\lim _{n \rightarrow \infty} f\left(x_{n}\right)$ \begin{CJK}{UTF8}{mj}存在\end{CJK}.

  \item ( 12 \begin{CJK}{UTF8}{mj}分\end{CJK}) \begin{CJK}{UTF8}{mj}设级数\end{CJK} $\sum_{n=1}^{\infty} a_{n}$ \begin{CJK}{UTF8}{mj}收敛并且对于\end{CJK} $\forall n \in \mathbb{Z}, a_{n} \geqslant a_{n+1}>0$, \begin{CJK}{UTF8}{mj}试证明\end{CJK}: $\lim _{n \rightarrow \infty} n a_{n}=0$. 6. ( 12 \begin{CJK}{UTF8}{mj}分\end{CJK}) \begin{CJK}{UTF8}{mj}设\end{CJK} $f(x, y)$ \begin{CJK}{UTF8}{mj}是在\end{CJK} $D: x^{2}+y^{2} \leqslant a^{2}$ \begin{CJK}{UTF8}{mj}上具有连续的一阶偏导数的非负函数\end{CJK}, \begin{CJK}{UTF8}{mj}并且在\end{CJK} $D$ \begin{CJK}{UTF8}{mj}的边界上处处取值\end{CJK} \begin{CJK}{UTF8}{mj}为零\end{CJK}. \begin{CJK}{UTF8}{mj}证明\end{CJK}:

\end{enumerate}
$$
\left|\iint_{D} f(x, y) \mathrm{d} x \mathrm{~d} y\right| \leqslant \frac{1}{3} \pi a^{3} \max _{(x, y) \in D} \sqrt{f^{\prime 2}{ }_{x}+f^{\prime}{ }_{y}^{2}}
$$

\begin{enumerate}
  \setcounter{enumi}{7}
  \item (12 \begin{CJK}{UTF8}{mj}分\end{CJK}) \begin{CJK}{UTF8}{mj}证明\end{CJK}: \begin{CJK}{UTF8}{mj}含参变量的积分\end{CJK}
\end{enumerate}
$$
F(y)=\int_{0}^{+\infty} \frac{\cos (x y)}{\sqrt{x}} \mathrm{~d} x,
$$
\begin{CJK}{UTF8}{mj}关于\end{CJK} $y$ \begin{CJK}{UTF8}{mj}在\end{CJK} $\left[y_{0},+\infty\right)$ \begin{CJK}{UTF8}{mj}上一致收敛\end{CJK}, \begin{CJK}{UTF8}{mj}其中\end{CJK} $y_{0}>0$.

\begin{enumerate}
  \setcounter{enumi}{8}
  \item (12 \begin{CJK}{UTF8}{mj}分\end{CJK}) \begin{CJK}{UTF8}{mj}设函数\end{CJK} $P(x, y, z), Q(x, y, z), R(x, y, z)$ \begin{CJK}{UTF8}{mj}都在\end{CJK} $\mathbb{R}^{3}$ \begin{CJK}{UTF8}{mj}上具有连续的偏导数\end{CJK}, \begin{CJK}{UTF8}{mj}且对于任意光滑曲线\end{CJK} $\Sigma$ \begin{CJK}{UTF8}{mj}有\end{CJK}
\end{enumerate}
$$
\iint_{\Sigma} P \mathrm{~d} y \mathrm{~d} z+Q \mathrm{~d} z \mathrm{~d} x+R \mathrm{~d} x \mathrm{~d} y=0,
$$
\begin{CJK}{UTF8}{mj}证明\end{CJK}: \begin{CJK}{UTF8}{mj}在\end{CJK} $\mathbb{R}^{3}$ \begin{CJK}{UTF8}{mj}上\end{CJK}, \begin{CJK}{UTF8}{mj}恒有\end{CJK}
$$
\frac{\partial P}{\partial x}+\frac{\partial Q}{\partial y}+\frac{\partial R}{\partial z} \equiv 0 .
$$

\begin{enumerate}
  \setcounter{enumi}{9}
  \item ( 14 \begin{CJK}{UTF8}{mj}分\end{CJK}) \begin{CJK}{UTF8}{mj}设二元函数\end{CJK} $f(x, y)$ \begin{CJK}{UTF8}{mj}在正方形区域\end{CJK} $[0,1] \times[0,1]$ \begin{CJK}{UTF8}{mj}上连续\end{CJK}, \begin{CJK}{UTF8}{mj}记\end{CJK} $I=[0,1]$,
\end{enumerate}
(1) \begin{CJK}{UTF8}{mj}试比较\end{CJK} $\inf _{y \in I} \sup _{x \in I} f(x, y)$ \begin{CJK}{UTF8}{mj}与\end{CJK} $\sup _{y \in I} \inf _{x \in I} f(x, y)$ \begin{CJK}{UTF8}{mj}的大小\end{CJK}, \begin{CJK}{UTF8}{mj}并证明你的结论\end{CJK};

(2) \begin{CJK}{UTF8}{mj}证明\end{CJK}: \begin{CJK}{UTF8}{mj}如果\end{CJK} $f(x, y)$ \begin{CJK}{UTF8}{mj}关于二变量之一是一致单调递增\end{CJK} (\begin{CJK}{UTF8}{mj}或者一致单调递减\end{CJK})\begin{CJK}{UTF8}{mj}的\end{CJK}, \begin{CJK}{UTF8}{mj}则\end{CJK}
$$
\inf _{y \in I} \sup _{x \in I} f(x, y)=\sup _{y \in I} \inf _{x \in I} f(x, y) .
$$
\begin{CJK}{UTF8}{mj}注释\end{CJK}: \begin{CJK}{UTF8}{mj}称\end{CJK} $f(x, y)$ \begin{CJK}{UTF8}{mj}关于\end{CJK} $x$ \begin{CJK}{UTF8}{mj}是一致单调递增\end{CJK} (\begin{CJK}{UTF8}{mj}一致单调递减\end{CJK})\begin{CJK}{UTF8}{mj}的\end{CJK}, \begin{CJK}{UTF8}{mj}如果对于\end{CJK} $\forall y \in I, f(x, y)$ \begin{CJK}{UTF8}{mj}是关于\end{CJK} $x$ \begin{CJK}{UTF8}{mj}的单调递增\end{CJK} (\begin{CJK}{UTF8}{mj}单调递减\end{CJK}) \begin{CJK}{UTF8}{mj}函数\end{CJK}.

\section{2. 电子科技大学 2010 年研究生入学考试试题数学分析}
\begin{CJK}{UTF8}{mj}李扬\end{CJK}

\begin{CJK}{UTF8}{mj}微信公众号\end{CJK}: sxkyliyang

\begin{CJK}{UTF8}{mj}一\end{CJK}、\begin{CJK}{UTF8}{mj}填空题\end{CJK}(\begin{CJK}{UTF8}{mj}每小题\end{CJK} 4 \begin{CJK}{UTF8}{mj}分\end{CJK}, \begin{CJK}{UTF8}{mj}共\end{CJK} 20 \begin{CJK}{UTF8}{mj}分\end{CJK})

\begin{enumerate}
  \item $\lim _{x \rightarrow 0^{+}} \frac{\arctan x}{(x+2) \sin ^{2} \sqrt{x}}=$

  \item \begin{CJK}{UTF8}{mj}当\end{CJK} $a=$ $b=$ \begin{CJK}{UTF8}{mj}时\end{CJK}, \begin{CJK}{UTF8}{mj}定积分\end{CJK} $\int_{a}^{b}\left(2+x-x^{2}\right) \mathrm{d} x$ \begin{CJK}{UTF8}{mj}的值最大\end{CJK};

  \item \begin{CJK}{UTF8}{mj}设\end{CJK} $z=\sqrt[4]{\frac{x+y}{x-y}}$, \begin{CJK}{UTF8}{mj}则\end{CJK} $\mathrm{d} z=$

  \item \begin{CJK}{UTF8}{mj}幂级数\end{CJK} $\sum_{n=0}^{\infty} \frac{x^{n}}{n}$ \begin{CJK}{UTF8}{mj}的收敛区间为\end{CJK} , \begin{CJK}{UTF8}{mj}和函数\end{CJK} $S(x)=$

  \item \begin{CJK}{UTF8}{mj}符号函数\end{CJK} $\operatorname{sgn}(x)=\left\{\begin{aligned}-1, x<0 \\ 0, x &=0 \\ 1, x &>0 \end{aligned}\right.$ \begin{CJK}{UTF8}{mj}在\end{CJK} $[-\pi, \pi]$ \begin{CJK}{UTF8}{mj}上展开的\end{CJK} Fourier \begin{CJK}{UTF8}{mj}级数为\end{CJK}

\end{enumerate}
\section{二、解答题}
\begin{enumerate}
  \item (10 \begin{CJK}{UTF8}{mj}分\end{CJK}) \begin{CJK}{UTF8}{mj}证明\end{CJK}: $\lim _{x \rightarrow x_{0}} f(x)$ \begin{CJK}{UTF8}{mj}存在当且仅当\end{CJK} $\forall\left\{x_{n}\right\}: x_{n} \rightarrow x_{0}(n \rightarrow \infty), x_{n} \neq x_{0}(n=1,2, \cdots), \lim _{n \rightarrow \infty} f\left(x_{n}\right)$ \begin{CJK}{UTF8}{mj}存在\end{CJK}.

  \item (12 \begin{CJK}{UTF8}{mj}分\end{CJK}) \begin{CJK}{UTF8}{mj}设\end{CJK} $f(x)$ \begin{CJK}{UTF8}{mj}和\end{CJK} $g(x)$ \begin{CJK}{UTF8}{mj}都是区间\end{CJK} $I$ \begin{CJK}{UTF8}{mj}上的一致连续函数\end{CJK}, \begin{CJK}{UTF8}{mj}试问\end{CJK}:

\end{enumerate}
$$
f(x) \pm g(x), f(x) g(x), \frac{f(x)}{g(x)}(\forall x \in I, g(x) \neq 0)
$$
\begin{CJK}{UTF8}{mj}在区间\end{CJK} $I$ \begin{CJK}{UTF8}{mj}上是否一定一致连续\end{CJK}? \begin{CJK}{UTF8}{mj}请证明你的结论\end{CJK}.

\begin{enumerate}
  \setcounter{enumi}{3}
  \item (12 \begin{CJK}{UTF8}{mj}分\end{CJK}) \begin{CJK}{UTF8}{mj}设\end{CJK} $f(x)$ \begin{CJK}{UTF8}{mj}在\end{CJK} $[a, b]$ \begin{CJK}{UTF8}{mj}具有二阶导数\end{CJK}, $b(x)$ \begin{CJK}{UTF8}{mj}和\end{CJK} $c(x)$ \begin{CJK}{UTF8}{mj}在\end{CJK} $[a, b]$ \begin{CJK}{UTF8}{mj}连续并且\end{CJK} $\forall x \in[a, b]$, \begin{CJK}{UTF8}{mj}恒有\end{CJK} $c(x)<0$, \begin{CJK}{UTF8}{mj}试证明\end{CJK}:
\end{enumerate}
(1) \begin{CJK}{UTF8}{mj}如果\end{CJK}
$$
f^{\prime \prime}\left(x_{0}\right)+b\left(x_{0}\right) f^{\prime}\left(x_{0}\right)+c\left(x_{0}\right) f\left(x_{0}\right)=0,
$$
\begin{CJK}{UTF8}{mj}其中\end{CJK} $x_{0} \in[a, b]$, \begin{CJK}{UTF8}{mj}则\end{CJK} $f(x)$ \begin{CJK}{UTF8}{mj}在\end{CJK} $x_{0}$ \begin{CJK}{UTF8}{mj}点不可能取正的最大值\end{CJK}, \begin{CJK}{UTF8}{mj}也不可能取负的最小值\end{CJK};

(2) \begin{CJK}{UTF8}{mj}有人说\end{CJK}: \begin{CJK}{UTF8}{mj}如果等式\end{CJK}
$$
f^{\prime \prime}(x)+b(x) f^{\prime}(x)+c(x) f(x)=0,
$$
\begin{CJK}{UTF8}{mj}在区间\end{CJK} $[a, b]$ \begin{CJK}{UTF8}{mj}上恒成立\end{CJK}, \begin{CJK}{UTF8}{mj}并且\end{CJK} $f(a)=f(b)$, \begin{CJK}{UTF8}{mj}则\end{CJK} $f(x)$ \begin{CJK}{UTF8}{mj}在\end{CJK} $[a, b]$ \begin{CJK}{UTF8}{mj}恒等于一个数\end{CJK}. \begin{CJK}{UTF8}{mj}你认为这一结论成立么\end{CJK}? \begin{CJK}{UTF8}{mj}请证明你\end{CJK} \begin{CJK}{UTF8}{mj}的结果\end{CJK}.

\begin{enumerate}
  \setcounter{enumi}{4}
  \item (12 \begin{CJK}{UTF8}{mj}分\end{CJK}) \begin{CJK}{UTF8}{mj}设\end{CJK} $z=z(x, y)$ \begin{CJK}{UTF8}{mj}是由\end{CJK}
\end{enumerate}
$$
x^{2}-6 x y+10 y^{2}-2 y z-z^{2}+18=0,
$$
\begin{CJK}{UTF8}{mj}确定的函数\end{CJK}, \begin{CJK}{UTF8}{mj}求\end{CJK} $z=z(x, y)$ \begin{CJK}{UTF8}{mj}的极值点和极值\end{CJK}.

\begin{enumerate}
  \setcounter{enumi}{5}
  \item (12 \begin{CJK}{UTF8}{mj}分\end{CJK}) \begin{CJK}{UTF8}{mj}设\end{CJK} $\varphi(x)$ \begin{CJK}{UTF8}{mj}是\end{CJK} $(-\infty,+\infty)$ \begin{CJK}{UTF8}{mj}上周期为\end{CJK} 1 \begin{CJK}{UTF8}{mj}的周期函数\end{CJK}, \begin{CJK}{UTF8}{mj}且\end{CJK} $\int_{0}^{1} \varphi(x) \mathrm{d} x=0, f(x)$ \begin{CJK}{UTF8}{mj}在\end{CJK} $[0,1]$ \begin{CJK}{UTF8}{mj}上可微并且具有一\end{CJK} \begin{CJK}{UTF8}{mj}阶连续导数\end{CJK},
\end{enumerate}
$$
a_{n}=\int_{0}^{1} f(x) \varphi(n x) \mathrm{d} x, n=1,2, \cdots
$$
\begin{CJK}{UTF8}{mj}证明\end{CJK}: \begin{CJK}{UTF8}{mj}级数\end{CJK} $\sum_{n=1}^{\infty} a_{n}$ \begin{CJK}{UTF8}{mj}绝对收敛\end{CJK}. 6. (12 \begin{CJK}{UTF8}{mj}分\end{CJK}) \begin{CJK}{UTF8}{mj}求曲面\end{CJK} $z=e^{-x^{2}-y^{2}+2 y-1}$ \begin{CJK}{UTF8}{mj}与平面\end{CJK} $z=\frac{1}{e}$ \begin{CJK}{UTF8}{mj}所围成立体体积\end{CJK}.

\begin{enumerate}
  \setcounter{enumi}{7}
  \item (12 \begin{CJK}{UTF8}{mj}分\end{CJK}) \begin{CJK}{UTF8}{mj}设\end{CJK} $D=[0,1] \times[0,1]$, \begin{CJK}{UTF8}{mj}证明\end{CJK}:
\end{enumerate}
$$
1 \leqslant \iint_{D}\left[\sin \left(x^{2}\right)+\cos \left(x^{2}\right)\right] \mathrm{d} x \mathrm{~d} y \leqslant \sqrt{2}
$$

\begin{enumerate}
  \setcounter{enumi}{8}
  \item (12 \begin{CJK}{UTF8}{mj}分\end{CJK}) \begin{CJK}{UTF8}{mj}设函数\end{CJK} $P(x, y, z), Q(x, y, z), R(x, y, z)$ \begin{CJK}{UTF8}{mj}都在\end{CJK} $\mathbb{R}^{3}$ \begin{CJK}{UTF8}{mj}上具有连续的偏导数\end{CJK}, \begin{CJK}{UTF8}{mj}且对于任意光滑曲线\end{CJK} $\Sigma$ \begin{CJK}{UTF8}{mj}有\end{CJK}
\end{enumerate}
$$
\iint_{\Sigma} P \mathrm{~d} y \mathrm{~d} z+Q \mathrm{~d} z \mathrm{~d} x+R \mathrm{~d} x \mathrm{~d} y=0
$$
\begin{CJK}{UTF8}{mj}证明\end{CJK}: \begin{CJK}{UTF8}{mj}在\end{CJK} $\mathbb{R}^{3}$ \begin{CJK}{UTF8}{mj}上\end{CJK}, \begin{CJK}{UTF8}{mj}恒有\end{CJK}
$$
\frac{\partial P}{\partial x}+\frac{\partial Q}{\partial y}+\frac{\partial R}{\partial z} \equiv 0 .
$$

\begin{enumerate}
  \setcounter{enumi}{9}
  \item (12 \begin{CJK}{UTF8}{mj}分\end{CJK}) \begin{CJK}{UTF8}{mj}设\end{CJK} $f(x)$ \begin{CJK}{UTF8}{mj}在\end{CJK} $[0,+\infty)$ \begin{CJK}{UTF8}{mj}连续\end{CJK}, \begin{CJK}{UTF8}{mj}并且\end{CJK} $\lim _{x \rightarrow+\infty} f(x)=0$, \begin{CJK}{UTF8}{mj}证明\end{CJK}:
\end{enumerate}
$$
\int_{0}^{+\infty} \frac{f(a x)-f(b x)}{x} \mathrm{~d} x=f(0) \ln \frac{b}{a}
$$

\begin{enumerate}
  \setcounter{enumi}{10}
  \item ( 12 \begin{CJK}{UTF8}{mj}分\end{CJK}) \begin{CJK}{UTF8}{mj}设\end{CJK} $S(x)$ \begin{CJK}{UTF8}{mj}在\end{CJK} $[0,1]$ \begin{CJK}{UTF8}{mj}上连续\end{CJK}, \begin{CJK}{UTF8}{mj}并且\end{CJK} $S(1)=0$. \begin{CJK}{UTF8}{mj}证明\end{CJK}: $\left\{x^{n} S(x)\right\}$ \begin{CJK}{UTF8}{mj}在\end{CJK} $[0,1]$ \begin{CJK}{UTF8}{mj}上一致收敛\end{CJK}.

  \item (12 \begin{CJK}{UTF8}{mj}分\end{CJK}) \begin{CJK}{UTF8}{mj}计算积分\end{CJK}

\end{enumerate}
$$
g(\alpha, \beta)=\int_{0}^{+\infty} \frac{\arctan \alpha x \cdot \arctan \beta x}{x^{2}} \mathrm{~d} x
$$

\section{3. 电子科技大学 2011 年研究生入学考试试题数学分析 
 李扬 
 微信公众号: sxkyliyang}
\begin{CJK}{UTF8}{mj}一\end{CJK}、\begin{CJK}{UTF8}{mj}填空题\end{CJK}(\begin{CJK}{UTF8}{mj}每小题\end{CJK} 4 \begin{CJK}{UTF8}{mj}分\end{CJK}, \begin{CJK}{UTF8}{mj}共\end{CJK} 40 \begin{CJK}{UTF8}{mj}分\end{CJK})

\begin{enumerate}
  \item \begin{CJK}{UTF8}{mj}设\end{CJK} $x_{n+1}=\frac{1}{2}\left(x_{n}+\frac{2}{x_{n}}\right), n=1,2,3, \cdots$, \begin{CJK}{UTF8}{mj}则\end{CJK} $\lim _{n \rightarrow \infty} x_{n}=$

  \item $\lim _{x \rightarrow 0} \frac{(1+m x)^{n}-(1+n x)^{n}}{x}=$

  \item \begin{CJK}{UTF8}{mj}设\end{CJK} $f(x)=\left\{\begin{array}{l}x e^{-x^{2}}, x \geqslant 0 ; \\ \cos \frac{\pi}{2} x, x<0\end{array}\right.$, \begin{CJK}{UTF8}{mj}则\end{CJK} $\int_{0}^{2} f(x-1) \mathrm{d} x=$

  \item \begin{CJK}{UTF8}{mj}由方程\end{CJK} $e^{z}-x y z=0$ \begin{CJK}{UTF8}{mj}所确定的隐函数的二阶偏导数\end{CJK} $\frac{\partial^{2} z}{\partial x \partial y}=$

  \item \begin{CJK}{UTF8}{mj}曲线\end{CJK} $y=\frac{x^{2}}{x^{2}+1}$ \begin{CJK}{UTF8}{mj}在拐点处的切线方程为\end{CJK}

  \item \begin{CJK}{UTF8}{mj}数项级数\end{CJK}

\end{enumerate}
$$
\sum_{n=2}^{\infty}(-1)^{n} \frac{1}{2^{n}\left(n^{2}-1\right)},
$$
\begin{CJK}{UTF8}{mj}的和为\end{CJK}

\begin{enumerate}
  \setcounter{enumi}{7}
  \item \begin{CJK}{UTF8}{mj}旋转抛物面\end{CJK} $z=x^{2}+y^{2}$, \begin{CJK}{UTF8}{mj}平面\end{CJK} $x+y=1$ \begin{CJK}{UTF8}{mj}及三个坐标平面所围成的立体的体积为\end{CJK}

  \item \begin{CJK}{UTF8}{mj}若曲线\end{CJK} $L$ \begin{CJK}{UTF8}{mj}为星形线\end{CJK} $x^{\frac{2}{3}}+y^{\frac{2}{3}}=a^{\frac{2}{3}}$, \begin{CJK}{UTF8}{mj}则第一类曲线积分\end{CJK} $\oint_{L}|x|^{\frac{1}{3}} \mathrm{~d} s=$

  \item \begin{CJK}{UTF8}{mj}设\end{CJK} $\Sigma$ \begin{CJK}{UTF8}{mj}为闭曲面\end{CJK} $|x-y+z|+|y-z+x|+|z-x+y|=1$, \begin{CJK}{UTF8}{mj}方向取外侧\end{CJK}, \begin{CJK}{UTF8}{mj}则第二类曲线积分\end{CJK} $\iint_{\Sigma}(x-y+z) \mathrm{d} y \mathrm{~d} z+$ $(y-z+x) \mathrm{d} z \mathrm{~d} x+(z-x+y) \mathrm{d} x \mathrm{~d} y=$

  \item \begin{CJK}{UTF8}{mj}设函数\end{CJK} $f(x)=\left\{\begin{array}{l}1, x \in[-\pi, 0) \\ -1, x \in[0, \pi)\end{array}\right.$ \begin{CJK}{UTF8}{mj}在\end{CJK} $[-\pi, \pi]$ \begin{CJK}{UTF8}{mj}的\end{CJK} Fourier \begin{CJK}{UTF8}{mj}级数为\end{CJK}

\end{enumerate}
\section{一、证明题}
\begin{enumerate}
  \item (12 \begin{CJK}{UTF8}{mj}分\end{CJK}) \begin{CJK}{UTF8}{mj}当\end{CJK} $x \rightarrow+\infty$ \begin{CJK}{UTF8}{mj}时\end{CJK}, \begin{CJK}{UTF8}{mj}下列变量都是无穷大量\end{CJK}, \begin{CJK}{UTF8}{mj}试将它们从低阶到高阶进行排序\end{CJK}, \begin{CJK}{UTF8}{mj}并说明理由\end{CJK}.
\end{enumerate}
$$
a^{x}(a>1), x^{x}, x^{\alpha}(\alpha>0), \ln ^{k} x(k>0),[x] !
$$
\begin{CJK}{UTF8}{mj}其中\end{CJK}: $a, \alpha, k$ \begin{CJK}{UTF8}{mj}都是常数\end{CJK}.

\begin{enumerate}
  \setcounter{enumi}{2}
  \item (12 \begin{CJK}{UTF8}{mj}分\end{CJK}) \begin{CJK}{UTF8}{mj}求证明下列三个问题\end{CJK}:
\end{enumerate}
(1) \begin{CJK}{UTF8}{mj}若函数\end{CJK} $f(x)$ \begin{CJK}{UTF8}{mj}和\end{CJK} $g(x)$ \begin{CJK}{UTF8}{mj}连续\end{CJK}, \begin{CJK}{UTF8}{mj}证明\end{CJK} $\varphi(x)=\min \{f(x), g(x)\}$ \begin{CJK}{UTF8}{mj}与\end{CJK} $\psi(x)=\max \{f(x), g(x)\}$ \begin{CJK}{UTF8}{mj}也连续\end{CJK};

(2) \begin{CJK}{UTF8}{mj}设\end{CJK} $f_{1}(x), f_{2}(x), f_{3}(x)$ \begin{CJK}{UTF8}{mj}是同一个区间上的连续函数\end{CJK}, \begin{CJK}{UTF8}{mj}令\end{CJK} $f$ \begin{CJK}{UTF8}{mj}的值\end{CJK} $f(x)$ \begin{CJK}{UTF8}{mj}等于三值\end{CJK} $f_{1}(x), f_{2}(x), f_{3}(x)$ \begin{CJK}{UTF8}{mj}中介于其\end{CJK} \begin{CJK}{UTF8}{mj}它二值之间的那个值\end{CJK}, \begin{CJK}{UTF8}{mj}证明\end{CJK}: $f$ \begin{CJK}{UTF8}{mj}在\end{CJK} $[a, b]$ \begin{CJK}{UTF8}{mj}上连续\end{CJK};

(3) \begin{CJK}{UTF8}{mj}如果令\end{CJK}
$$
u_{n}(x)= \begin{cases}x, & |x| \leqslant n \\ n \cdot \operatorname{sgn}(x), & |x|>n\end{cases}
$$
\begin{CJK}{UTF8}{mj}并且\end{CJK} $f$ \begin{CJK}{UTF8}{mj}为实函数\end{CJK}, \begin{CJK}{UTF8}{mj}证明\end{CJK}: $f(x)$ \begin{CJK}{UTF8}{mj}连续的充要条件是对任何自然数\end{CJK} $n, g_{n}(x)=u_{n}[f(x)]$ \begin{CJK}{UTF8}{mj}都是连续函数\end{CJK}. 3. (12 \begin{CJK}{UTF8}{mj}分\end{CJK}) \begin{CJK}{UTF8}{mj}设\end{CJK} $f(0)=0$ \begin{CJK}{UTF8}{mj}并且导数\end{CJK} $f^{\prime}(x)$ \begin{CJK}{UTF8}{mj}在\end{CJK} $(0,+\infty)$ \begin{CJK}{UTF8}{mj}单调增加\end{CJK}, \begin{CJK}{UTF8}{mj}证明\end{CJK}: \begin{CJK}{UTF8}{mj}在\end{CJK} $(0,+\infty)$ \begin{CJK}{UTF8}{mj}内\end{CJK}, \begin{CJK}{UTF8}{mj}函数\end{CJK} $g(x)=\frac{f(x)}{x}$ \begin{CJK}{UTF8}{mj}是单调增\end{CJK} \begin{CJK}{UTF8}{mj}加的\end{CJK}.

\begin{enumerate}
  \setcounter{enumi}{4}
  \item (12 \begin{CJK}{UTF8}{mj}分\end{CJK}) \begin{CJK}{UTF8}{mj}设\end{CJK}
\end{enumerate}
$$
F(t)=\iint_{0 \leqslant x \leqslant t, 0 \leqslant y \leqslant t} e^{-\frac{t x}{y^{2}}} \mathrm{~d} x \mathrm{~d} y
$$
\begin{CJK}{UTF8}{mj}求\end{CJK} $F^{\prime}(0)$.

\begin{enumerate}
  \setcounter{enumi}{5}
  \item ( 12 \begin{CJK}{UTF8}{mj}分\end{CJK}) \begin{CJK}{UTF8}{mj}证明函数项级数\end{CJK}
\end{enumerate}
$$
\sum_{n=1}^{\infty} \frac{1}{n^{3}} \ln \left(1+n^{2} x^{2}\right)
$$
\begin{CJK}{UTF8}{mj}在\end{CJK} $[0,1]$ \begin{CJK}{UTF8}{mj}上一致收玫\end{CJK}, \begin{CJK}{UTF8}{mj}并讨论其和函数在\end{CJK} $[0,1]$ \begin{CJK}{UTF8}{mj}上的连续性\end{CJK}, \begin{CJK}{UTF8}{mj}可积性与可导性\end{CJK}.

\begin{enumerate}
  \setcounter{enumi}{6}
  \item (12 \begin{CJK}{UTF8}{mj}分\end{CJK}) \begin{CJK}{UTF8}{mj}当\end{CJK} $x>-1$ \begin{CJK}{UTF8}{mj}时\end{CJK}, $f(x)$ \begin{CJK}{UTF8}{mj}连续\end{CJK}, \begin{CJK}{UTF8}{mj}可微并且\end{CJK} $f(0)=\frac{6}{5}$, \begin{CJK}{UTF8}{mj}对半平面\end{CJK} $x>-1$ \begin{CJK}{UTF8}{mj}上的任一闭曲线\end{CJK}, \begin{CJK}{UTF8}{mj}有\end{CJK}
\end{enumerate}
$$
\oint_{L}\left[y-5 y e^{-2 x} f(x)\right] \mathrm{d} x+e^{-2 x} f(x) \mathrm{d} y=0
$$
, \begin{CJK}{UTF8}{mj}试求\end{CJK} $f(x)$ \begin{CJK}{UTF8}{mj}并且计算\end{CJK}
$$
\oint_{L}\left[y-5 y e^{-2 x} f(x)\right] \mathrm{d} x+e^{-2 x} f(x) \mathrm{d} y,
$$
\begin{CJK}{UTF8}{mj}其中\end{CJK}: $L$ \begin{CJK}{UTF8}{mj}为从点\end{CJK} $(1,0)$ \begin{CJK}{UTF8}{mj}到点\end{CJK} $(2,3)$ \begin{CJK}{UTF8}{mj}的弧段\end{CJK}.

\begin{enumerate}
  \setcounter{enumi}{7}
  \item (12 \begin{CJK}{UTF8}{mj}分\end{CJK}) \begin{CJK}{UTF8}{mj}证明\end{CJK}: \begin{CJK}{UTF8}{mj}含参变量的积分\end{CJK}
\end{enumerate}
$$
F(\alpha)=\int_{0}^{+\infty} x \sin x^{4} \cos \alpha x \mathrm{~d} x
$$
\begin{CJK}{UTF8}{mj}在任何有限区间\end{CJK} $[a, b]$ \begin{CJK}{UTF8}{mj}上都一致收敛\end{CJK}.

\begin{enumerate}
  \setcounter{enumi}{8}
  \item (12 \begin{CJK}{UTF8}{mj}分\end{CJK}) \begin{CJK}{UTF8}{mj}计算曲面积分\end{CJK}
\end{enumerate}
$$
\iint_{\Sigma} \frac{x \mathrm{~d} y \mathrm{~d} z+y \mathrm{~d} z \mathrm{~d} x+z \mathrm{~d} x \mathrm{~d} y}{\left(x^{2}+y^{2}+z^{2}\right)^{\frac{3}{2}}},
$$
\begin{CJK}{UTF8}{mj}其中\end{CJK}: $\Sigma$ \begin{CJK}{UTF8}{mj}是抛物面\end{CJK} $1-\frac{z}{5}=\frac{(x-2)^{2}}{16}+\frac{(y-1)^{2}}{9}(z \geqslant 0)$, \begin{CJK}{UTF8}{mj}方向取上侧\end{CJK}.

\begin{enumerate}
  \setcounter{enumi}{9}
  \item (14 \begin{CJK}{UTF8}{mj}分\end{CJK}) \begin{CJK}{UTF8}{mj}设函数\end{CJK} $\varphi(x)$ \begin{CJK}{UTF8}{mj}在\end{CJK} $[0,+\infty)$ \begin{CJK}{UTF8}{mj}上一致连续\end{CJK}. \begin{CJK}{UTF8}{mj}如果对于\end{CJK} $\forall x>0$, \begin{CJK}{UTF8}{mj}有\end{CJK} $\lim _{n \rightarrow \infty} \varphi(x+n)=C$, \begin{CJK}{UTF8}{mj}试证明\end{CJK}: $\lim _{x \rightarrow+\infty} \varphi(x)=C$, \begin{CJK}{UTF8}{mj}其中\end{CJK} $C$ \begin{CJK}{UTF8}{mj}为一个常数\end{CJK}.
\end{enumerate}
\section{4. 电子科技大学 2012 年研究生入学考试试题数学分析}
\begin{CJK}{UTF8}{mj}李扬\end{CJK}

\begin{CJK}{UTF8}{mj}微信公众号\end{CJK}: sxkyliyang

\begin{CJK}{UTF8}{mj}一\end{CJK}、\begin{CJK}{UTF8}{mj}填空题\end{CJK}(\begin{CJK}{UTF8}{mj}每小题\end{CJK} 5 \begin{CJK}{UTF8}{mj}分\end{CJK}, \begin{CJK}{UTF8}{mj}共\end{CJK} 40 \begin{CJK}{UTF8}{mj}分\end{CJK})

\begin{enumerate}
  \item \begin{CJK}{UTF8}{mj}设\end{CJK} $x_{1}>0, x_{n+1}=1+\frac{x_{n}}{1+x_{n}}$, \begin{CJK}{UTF8}{mj}则\end{CJK} $\lim _{n \rightarrow \infty} x_{n}=$

  \item $\lim _{x \rightarrow 0} \frac{x-\tan x}{\sin ^{3} x}=$

  \item \begin{CJK}{UTF8}{mj}设\end{CJK} $y=\left\{\begin{array}{l}x=\sin a t ; \\ y=\cos b t\end{array}\right.$, \begin{CJK}{UTF8}{mj}则\end{CJK} $\frac{\mathrm{d} y}{\mathrm{~d} x}=$ $\frac{\mathrm{d}^{y}}{\mathrm{~d} x^{2}}=$

  \item \begin{CJK}{UTF8}{mj}曲线\end{CJK} $x=\frac{1}{4} y^{2}-\frac{1}{2} \ln y, 1 \leqslant y \leqslant e$ \begin{CJK}{UTF8}{mj}的弧长为\end{CJK} $s=$

  \item \begin{CJK}{UTF8}{mj}已知\end{CJK} $y=x^{n} \cos 2 x$, \begin{CJK}{UTF8}{mj}则\end{CJK} $y$ \begin{CJK}{UTF8}{mj}的\end{CJK} $n$ \begin{CJK}{UTF8}{mj}阶微分\end{CJK} $\mathrm{d}^{n} y=$

  \item \begin{CJK}{UTF8}{mj}设\end{CJK} $F(y)=\int_{0}^{y}(x+y) \varphi(x) \mathrm{d} x$, \begin{CJK}{UTF8}{mj}其中\end{CJK} $\varphi(x)$ \begin{CJK}{UTF8}{mj}为可微函数\end{CJK}, \begin{CJK}{UTF8}{mj}则\end{CJK} $F^{\prime \prime}(y)=$

  \item \begin{CJK}{UTF8}{mj}设曲线\end{CJK} $L$ \begin{CJK}{UTF8}{mj}是从点\end{CJK} $A(0,0)$ \begin{CJK}{UTF8}{mj}到点\end{CJK} $B(2,0)$ \begin{CJK}{UTF8}{mj}的圆周\end{CJK} $x^{2}+y^{2}=2 x$ \begin{CJK}{UTF8}{mj}的上半部分\end{CJK}, \begin{CJK}{UTF8}{mj}则曲线积分\end{CJK} $\int_{L}\left(x^{2}-y\right) \mathrm{d} x-(x+$ $\left.\sin ^{2} y\right) \mathrm{d} y=$

  \item \begin{CJK}{UTF8}{mj}设\end{CJK} $f(x)=\frac{x^{2}}{2}-\pi^{2},|x| \leqslant \pi$, \begin{CJK}{UTF8}{mj}则\end{CJK} $f(x)$ \begin{CJK}{UTF8}{mj}的\end{CJK} Fourier \begin{CJK}{UTF8}{mj}级数为\end{CJK}

\end{enumerate}
\section{二、解答题}
\begin{enumerate}
  \item ( 12 \begin{CJK}{UTF8}{mj}分\end{CJK}) \begin{CJK}{UTF8}{mj}设\end{CJK} $f(x)=\sin \frac{1}{x}$, \begin{CJK}{UTF8}{mj}证明\end{CJK}: \begin{CJK}{UTF8}{mj}对于\end{CJK} $\forall \alpha \in(0,1), f(x)$ \begin{CJK}{UTF8}{mj}在区间\end{CJK} $(\alpha, 1)$ \begin{CJK}{UTF8}{mj}一致连续\end{CJK}, \begin{CJK}{UTF8}{mj}但在区间\end{CJK} $(0,1), f(x)$ \begin{CJK}{UTF8}{mj}非一致\end{CJK} \begin{CJK}{UTF8}{mj}连续\end{CJK}.

  \item (12 \begin{CJK}{UTF8}{mj}分\end{CJK}) \begin{CJK}{UTF8}{mj}设函数\end{CJK} $f(x)$ \begin{CJK}{UTF8}{mj}在\end{CJK} $[0,1]$ \begin{CJK}{UTF8}{mj}上具有二阶导数\end{CJK}, \begin{CJK}{UTF8}{mj}且满足\end{CJK} $\left|f^{\prime \prime}(x)\right| \leqslant 1, f(x)$ \begin{CJK}{UTF8}{mj}在区间\end{CJK} $(0,1)$ \begin{CJK}{UTF8}{mj}内取最大值\end{CJK} $\frac{1}{4}$, \begin{CJK}{UTF8}{mj}证明\end{CJK}:

\end{enumerate}
$$
|f(0)|+|f(1)| \leqslant 1
$$

\begin{enumerate}
  \setcounter{enumi}{3}
  \item (12 \begin{CJK}{UTF8}{mj}分\end{CJK}) \begin{CJK}{UTF8}{mj}设\end{CJK} $f(x)$ \begin{CJK}{UTF8}{mj}在\end{CJK} $[0,1]$ \begin{CJK}{UTF8}{mj}上连续\end{CJK}, \begin{CJK}{UTF8}{mj}且单调减少\end{CJK}, \begin{CJK}{UTF8}{mj}证明\end{CJK}: \begin{CJK}{UTF8}{mj}对于\end{CJK} $\forall \alpha \in[0,1]$, \begin{CJK}{UTF8}{mj}总有\end{CJK}
\end{enumerate}
$$
\int_{0}^{\alpha} f(x) \mathrm{d} x \leqslant \alpha \int_{0}^{1} f(x) \mathrm{d} x
$$

\begin{enumerate}
  \setcounter{enumi}{4}
  \item (12 \begin{CJK}{UTF8}{mj}分\end{CJK}) \begin{CJK}{UTF8}{mj}抛物面\end{CJK} $z=x^{2}+y^{2}$ \begin{CJK}{UTF8}{mj}被平面\end{CJK} $x+y+z=1$ \begin{CJK}{UTF8}{mj}截成一个椭圆\end{CJK}, \begin{CJK}{UTF8}{mj}求原点到这个椭圆的最长距离和最短距离\end{CJK}.

  \item ( 12 \begin{CJK}{UTF8}{mj}分\end{CJK}) \begin{CJK}{UTF8}{mj}设\end{CJK} $a, b>0$, \begin{CJK}{UTF8}{mj}证明\end{CJK}:\begin{CJK}{UTF8}{mj}存在\end{CJK} $\xi \in(a, b)$ \begin{CJK}{UTF8}{mj}使得\end{CJK}

\end{enumerate}
$$
a e^{b}-b e^{a}=(1-\xi) e^{\xi}(a-b) .
$$

\begin{enumerate}
  \setcounter{enumi}{6}
  \item (12 \begin{CJK}{UTF8}{mj}分\end{CJK}) \begin{CJK}{UTF8}{mj}求幂级数\end{CJK} $\sum_{n=1}^{\infty} \frac{(x-1)^{n}}{n 2^{n}}$ \begin{CJK}{UTF8}{mj}的和函数\end{CJK}, \begin{CJK}{UTF8}{mj}并由此求\end{CJK} $\sum_{n=1}^{\infty} \frac{1}{n 2^{n}}$.

  \item (12 \begin{CJK}{UTF8}{mj}分\end{CJK}) \begin{CJK}{UTF8}{mj}计算含参变量的积分\end{CJK}

\end{enumerate}
$$
F(x)=\int_{0}^{+\infty} e^{-t^{2}} \cos 2 x t \mathrm{~d} t
$$

\begin{enumerate}
  \setcounter{enumi}{8}
  \item ( 14 \begin{CJK}{UTF8}{mj}分\end{CJK}) \begin{CJK}{UTF8}{mj}计算\end{CJK}
\end{enumerate}
$$
I=\int_{\Gamma}\left(y^{2}+z^{2}\right) \mathrm{d} x+\left(z^{2}+x^{2}\right) \mathrm{d} y+\left(x^{2}+y^{2}\right) \mathrm{d} z,
$$
\begin{CJK}{UTF8}{mj}其中\end{CJK} $\Gamma$ \begin{CJK}{UTF8}{mj}是上半球面\end{CJK} $x^{2}+y^{2}+z^{2}=2 a z(z \geqslant 0)$ \begin{CJK}{UTF8}{mj}与圆柱面\end{CJK} $x^{2}+y^{2}=2 b x(a>b>0)$ \begin{CJK}{UTF8}{mj}的交线\end{CJK}, \begin{CJK}{UTF8}{mj}从\end{CJK} $z$ \begin{CJK}{UTF8}{mj}轴的正向看\end{CJK} \begin{CJK}{UTF8}{mj}去\end{CJK}, \begin{CJK}{UTF8}{mj}是逆时针方向\end{CJK}. 9. ( 12 \begin{CJK}{UTF8}{mj}分\end{CJK}) \begin{CJK}{UTF8}{mj}已知\end{CJK} Riemann \begin{CJK}{UTF8}{mj}函数\end{CJK}
$$
R(x)=\left\{\begin{array}{l}
\frac{1}{p}, x=\frac{q}{p}(p \in \mathbb{N}, q \in \mathbb{Z} \backslash\{0\}, p, q \text { 互质 }) ; \\
1, x=0 ; \\
0, x \text { 是无理数 }
\end{array}\right.
$$
\begin{CJK}{UTF8}{mj}试证明\end{CJK}:

(1) $R(x)$ \begin{CJK}{UTF8}{mj}在任意一点\end{CJK} $x_{0}$ \begin{CJK}{UTF8}{mj}都存在极限\end{CJK}, \begin{CJK}{UTF8}{mj}并且极限为\end{CJK} 0 ;

(2) $R(x)$ \begin{CJK}{UTF8}{mj}在无理点处取极小值\end{CJK}, \begin{CJK}{UTF8}{mj}在有理点处取极大值\end{CJK}.

\section{5. 电子科技大学 2013 年研究生入学考试试题数学分析}
\begin{CJK}{UTF8}{mj}李扬\end{CJK}

\begin{CJK}{UTF8}{mj}微信公众号\end{CJK}: sxkyliyang

\begin{CJK}{UTF8}{mj}一\end{CJK}、\begin{CJK}{UTF8}{mj}填空题\end{CJK}(\begin{CJK}{UTF8}{mj}每小题\end{CJK} 5 \begin{CJK}{UTF8}{mj}分\end{CJK}, \begin{CJK}{UTF8}{mj}共\end{CJK} 40 \begin{CJK}{UTF8}{mj}分\end{CJK})

\begin{enumerate}
  \item \begin{CJK}{UTF8}{mj}如果用\end{CJK} $[x]$ \begin{CJK}{UTF8}{mj}表示取整函数\end{CJK}, \begin{CJK}{UTF8}{mj}则极限\end{CJK} $\lim _{x \rightarrow \infty} \frac{x}{[x]}$ \begin{CJK}{UTF8}{mj}的值为\end{CJK}

  \item \begin{CJK}{UTF8}{mj}设\end{CJK} $\varphi(x)$ \begin{CJK}{UTF8}{mj}具有二阶连续导数且\end{CJK} $\varphi(0)=1$, \begin{CJK}{UTF8}{mj}如果\end{CJK}

\end{enumerate}
$$
f(x)= \begin{cases}\frac{\varphi(x)-\cos x}{x}, & x \neq 0 \\ a, & x=0\end{cases}
$$
\begin{CJK}{UTF8}{mj}在\end{CJK} $x=0$ \begin{CJK}{UTF8}{mj}点连续\end{CJK}, \begin{CJK}{UTF8}{mj}则\end{CJK}

\begin{enumerate}
  \setcounter{enumi}{3}
  \item \begin{CJK}{UTF8}{mj}如果\end{CJK}
\end{enumerate}
$$
f(x)=\frac{1}{1+x^{2}}+\sqrt{1-x^{2}} \int_{0}^{1} f(x) \mathrm{d} x
$$
\begin{CJK}{UTF8}{mj}则\end{CJK} $\int_{0}^{1} f(x) \mathrm{d} x=$

\begin{enumerate}
  \setcounter{enumi}{4}
  \item \begin{CJK}{UTF8}{mj}设\end{CJK} $z=\frac{y}{f\left(x^{2}-y^{2}\right)}$, \begin{CJK}{UTF8}{mj}其中\end{CJK} $f(t)$ \begin{CJK}{UTF8}{mj}连续可导且\end{CJK} $f(t) \neq 0$, \begin{CJK}{UTF8}{mj}则\end{CJK} $\frac{1}{x} \frac{\partial z}{\partial x}+\frac{1}{y} \frac{\partial z}{\partial y}=$

  \item \begin{CJK}{UTF8}{mj}将累次积分\end{CJK} $\int_{0}^{4} \mathrm{~d} x \int_{\sqrt{4 x-x^{2}}}^{\sqrt{4 x}} f(x, y) \mathrm{d} y$ \begin{CJK}{UTF8}{mj}更换积分次序后的积分为\end{CJK}

  \item \begin{CJK}{UTF8}{mj}曲面\end{CJK} $z=\frac{1}{2} x^{2}-y^{2}$ \begin{CJK}{UTF8}{mj}在点\end{CJK} $(2,-1,1)$ \begin{CJK}{UTF8}{mj}处的切平面方程为\end{CJK}

  \item \begin{CJK}{UTF8}{mj}设\end{CJK} $L$ \begin{CJK}{UTF8}{mj}为平面\end{CJK} $x+y+z=1$ \begin{CJK}{UTF8}{mj}被三个坐标面所截三角形\end{CJK} $\Sigma$ \begin{CJK}{UTF8}{mj}的边界\end{CJK}, \begin{CJK}{UTF8}{mj}若从\end{CJK} $x$ \begin{CJK}{UTF8}{mj}轴的正向去\end{CJK}, $L$ \begin{CJK}{UTF8}{mj}的定向为逆时针方向\end{CJK}, \begin{CJK}{UTF8}{mj}曲线积分\end{CJK}

\end{enumerate}
$$
\oint_{L}\left(y^{2}-z^{2}\right) \mathrm{d} x+\left(z^{2}-x^{2}\right) \mathrm{d} y+\left(x^{2}-y^{2}\right) \mathrm{d} z
$$
\begin{CJK}{UTF8}{mj}的值为\end{CJK}

\begin{enumerate}
  \setcounter{enumi}{8}
  \item \begin{CJK}{UTF8}{mj}设函数\end{CJK} $f(x)=\left\{\begin{array}{c}1, x \in[-\pi, 0) ; \\ 0, x \in[0, \pi) .\end{array}\right.$ \begin{CJK}{UTF8}{mj}则\end{CJK} $f(x)$ \begin{CJK}{UTF8}{mj}的\end{CJK} Fourier \begin{CJK}{UTF8}{mj}级数为\end{CJK}
\end{enumerate}
\section{二、解答题}
\begin{enumerate}
  \item (12 \begin{CJK}{UTF8}{mj}分\end{CJK}) \begin{CJK}{UTF8}{mj}设\end{CJK} $S$ \begin{CJK}{UTF8}{mj}是非空有上界的数集\end{CJK}, \begin{CJK}{UTF8}{mj}并且\end{CJK} $\sup S=\alpha \notin S$, \begin{CJK}{UTF8}{mj}证明\end{CJK}: \begin{CJK}{UTF8}{mj}存在严格单调递增的数列\end{CJK} $\left\{x_{n}\right\} \subset S$ \begin{CJK}{UTF8}{mj}使得\end{CJK} $\lim _{x \rightarrow \infty} x_{n}=\alpha .$

  \item ( 12 \begin{CJK}{UTF8}{mj}分\end{CJK}) \begin{CJK}{UTF8}{mj}若函数\end{CJK} $f(x)$ \begin{CJK}{UTF8}{mj}在区间\end{CJK} $[\alpha,+\infty)$ \begin{CJK}{UTF8}{mj}上连续且\end{CJK} $\lim _{x \rightarrow+\infty} f(x)$ \begin{CJK}{UTF8}{mj}存在\end{CJK}, \begin{CJK}{UTF8}{mj}证明\end{CJK}: $f(x)$ \begin{CJK}{UTF8}{mj}在区间\end{CJK} $[\alpha,+\infty)$ \begin{CJK}{UTF8}{mj}一致连续\end{CJK}.

  \item (12 \begin{CJK}{UTF8}{mj}分\end{CJK}) \begin{CJK}{UTF8}{mj}设函数\end{CJK} $f(x)$ \begin{CJK}{UTF8}{mj}在闭\end{CJK} $[a, b]$ \begin{CJK}{UTF8}{mj}连续\end{CJK}, \begin{CJK}{UTF8}{mj}在开区间\end{CJK} $(a, b)$ \begin{CJK}{UTF8}{mj}可导并且\end{CJK} $f^{\prime}(x) \neq 0,0<a<b$. \begin{CJK}{UTF8}{mj}试证明\end{CJK}: \begin{CJK}{UTF8}{mj}存在\end{CJK} $\xi, \eta \in(a, b)$ \begin{CJK}{UTF8}{mj}使得\end{CJK}

\end{enumerate}
$$
\frac{f^{\prime}(\xi)}{f^{\prime}(\eta)}=\frac{a^{2}+a b+b^{2}}{3 \eta^{2}}
$$

\begin{enumerate}
  \setcounter{enumi}{4}
  \item (12 \begin{CJK}{UTF8}{mj}分\end{CJK}) \begin{CJK}{UTF8}{mj}设\end{CJK} $f(x)$ \begin{CJK}{UTF8}{mj}在闭\end{CJK} $[0,2]$ \begin{CJK}{UTF8}{mj}上二阶可导\end{CJK}, \begin{CJK}{UTF8}{mj}并且满足\end{CJK} $|f(x)| \leqslant 1$ \begin{CJK}{UTF8}{mj}及\end{CJK} $\left|f^{\prime \prime}(x)\right| \leqslant 1$, \begin{CJK}{UTF8}{mj}证明\end{CJK}: \begin{CJK}{UTF8}{mj}对于\end{CJK} $\forall x \in[0,2]$, \begin{CJK}{UTF8}{mj}有\end{CJK} $\left|f^{\prime}(x)\right| \leqslant 2$. 5. (12 \begin{CJK}{UTF8}{mj}分\end{CJK}) \begin{CJK}{UTF8}{mj}设\end{CJK}
\end{enumerate}
$$
f(x)=\sum_{n=1}^{\infty} \frac{1}{2^{n}} \tan \frac{x}{2^{n}},
$$
\begin{CJK}{UTF8}{mj}证明\end{CJK}: $f(x)$ \begin{CJK}{UTF8}{mj}在\end{CJK} $\left[0, \frac{\pi}{2}\right]$ \begin{CJK}{UTF8}{mj}上连续\end{CJK}, \begin{CJK}{UTF8}{mj}并计算\end{CJK} $\int_{\frac{\pi}{4}}^{\frac{\pi}{2}} f(x) \mathrm{d} x$.

\begin{enumerate}
  \setcounter{enumi}{6}
  \item (12 \begin{CJK}{UTF8}{mj}分\end{CJK}) \begin{CJK}{UTF8}{mj}设\end{CJK} $f(x)$ \begin{CJK}{UTF8}{mj}在\end{CJK} $(0,+\infty)$ \begin{CJK}{UTF8}{mj}连续\end{CJK}, \begin{CJK}{UTF8}{mj}反常积分\end{CJK} $\int_{0}^{+\infty} x^{\alpha} f(x) \mathrm{d} x$ \begin{CJK}{UTF8}{mj}在\end{CJK} $\alpha=A$ \begin{CJK}{UTF8}{mj}与\end{CJK} $\alpha=B$ \begin{CJK}{UTF8}{mj}时都收敛\end{CJK} (\begin{CJK}{UTF8}{mj}其中\end{CJK} $\left.A<B\right)$, \begin{CJK}{UTF8}{mj}证\end{CJK} \begin{CJK}{UTF8}{mj}明\end{CJK}:
\end{enumerate}
$$
g(\alpha)=\int_{0}^{+\infty} x^{\alpha} f(x) \mathrm{d} x
$$
\begin{CJK}{UTF8}{mj}关于\end{CJK} $\alpha$ \begin{CJK}{UTF8}{mj}在\end{CJK} $[A, B]$ \begin{CJK}{UTF8}{mj}上一致收敛\end{CJK}.

\begin{enumerate}
  \setcounter{enumi}{7}
  \item ( 12 \begin{CJK}{UTF8}{mj}分\end{CJK}) \begin{CJK}{UTF8}{mj}计算由曲面\end{CJK} $\left(x^{2}+y^{2}\right)^{2}+z^{4}=y$ \begin{CJK}{UTF8}{mj}所围成的立体的体积\end{CJK}.

  \item (12 \begin{CJK}{UTF8}{mj}分\end{CJK}) \begin{CJK}{UTF8}{mj}计算曲面积分\end{CJK}

\end{enumerate}
$$
\int_{L} \frac{(x-y) \mathrm{d} x+(x+4 y) \mathrm{d} y}{x^{2}+4 y^{2}},
$$
\begin{CJK}{UTF8}{mj}其中\end{CJK} $L$ \begin{CJK}{UTF8}{mj}为单位圆\end{CJK} $x^{2}+y^{2}=1$, \begin{CJK}{UTF8}{mj}逆时针方向\end{CJK}.

\begin{enumerate}
  \setcounter{enumi}{9}
  \item ( 14 \begin{CJK}{UTF8}{mj}分\end{CJK}) \begin{CJK}{UTF8}{mj}求\end{CJK} $a, b$ \begin{CJK}{UTF8}{mj}的值\end{CJK}, \begin{CJK}{UTF8}{mj}使得椭圆\end{CJK} $\frac{x^{2}}{a^{2}}+\frac{y^{2}}{b^{2}}=1$ \begin{CJK}{UTF8}{mj}包含圆周\end{CJK} $(x-1)^{2}+y^{2}=1$, \begin{CJK}{UTF8}{mj}且面积最小\end{CJK}.
\end{enumerate}
\section{6. 电子科技大学 2014 年研究生入学考试试题数学分析 
 李扬 
 微信公众号: sxkyliyang}
\begin{CJK}{UTF8}{mj}一\end{CJK}、\begin{CJK}{UTF8}{mj}填空题\end{CJK}(\begin{CJK}{UTF8}{mj}每小题\end{CJK} 5 \begin{CJK}{UTF8}{mj}分\end{CJK}, \begin{CJK}{UTF8}{mj}共\end{CJK} 40 \begin{CJK}{UTF8}{mj}分\end{CJK})

\begin{enumerate}
  \item \begin{CJK}{UTF8}{mj}设\end{CJK} $x_{n}=\frac{\sqrt{7} n}{2 n^{2}+1}+\frac{\sqrt{7} n}{2 n^{2}+2}+\cdots+\frac{\sqrt{7} n}{2 n^{2}+n}$, \begin{CJK}{UTF8}{mj}则\end{CJK} $\lim _{n \rightarrow \infty} x_{n}=$

  \item $\lim _{x \rightarrow 0}\left(\frac{x-\sin ^{2} x}{x}\right)^{\frac{2}{x}}=$

  \item \begin{CJK}{UTF8}{mj}如果\end{CJK}

\end{enumerate}
$$
f(x)= \begin{cases}a x^{2}-b, & x \geqslant 1 \\ e^{x-1}, & x<0\end{cases}
$$
\begin{CJK}{UTF8}{mj}在\end{CJK} $x=1$ \begin{CJK}{UTF8}{mj}处可导\end{CJK}, \begin{CJK}{UTF8}{mj}则\end{CJK} $a=$ $b=$

\begin{enumerate}
  \setcounter{enumi}{4}
  \item \begin{CJK}{UTF8}{mj}已知\end{CJK} $y=x^{2} \ln x,(x>0), n>2$ \begin{CJK}{UTF8}{mj}为正整数\end{CJK}, \begin{CJK}{UTF8}{mj}则\end{CJK} $\mathrm{d}^{n} y=$

  \item $\frac{\mathrm{d}}{\mathrm{d} x} \int_{0}^{x^{2}} t^{2} e^{\sin t} \mathrm{~d} t=$

  \item \begin{CJK}{UTF8}{mj}设曲面方程为\end{CJK} $x^{2}+y^{3}+z^{2}=10$, \begin{CJK}{UTF8}{mj}则该曲面在点\end{CJK} $(1,2,-1)$ \begin{CJK}{UTF8}{mj}的切平面方程为\end{CJK} \begin{CJK}{UTF8}{mj}而法线方\end{CJK} \begin{CJK}{UTF8}{mj}程为\end{CJK}

  \item \begin{CJK}{UTF8}{mj}幂级数\end{CJK} $\sum_{n=1}^{\infty} \frac{(-1)^{n-1}}{n+2}(x-1)^{n}$, \begin{CJK}{UTF8}{mj}的收敛半径为\end{CJK} , \begin{CJK}{UTF8}{mj}收敛域为\end{CJK}

  \item \begin{CJK}{UTF8}{mj}设有向曲线\end{CJK} $L$ \begin{CJK}{UTF8}{mj}的方程\end{CJK} $x^{2}+y^{2}=2 x+1$, \begin{CJK}{UTF8}{mj}方向为顺时针方向\end{CJK}, \begin{CJK}{UTF8}{mj}则曲线积分\end{CJK}

\end{enumerate}
$$
\oint_{L}\left(e^{y} \cos x+x^{2}+3 y\right) \mathrm{d} x+\left(e^{y} \sin x+y^{2}+6 x\right) \mathrm{d} y=
$$

\section{二、解答题}
\begin{enumerate}
  \item ( 12 \begin{CJK}{UTF8}{mj}分\end{CJK}) \begin{CJK}{UTF8}{mj}设函数\end{CJK} $f(x)=x^{2}+2 x+1$, \begin{CJK}{UTF8}{mj}证明\end{CJK}: $f(x)$ \begin{CJK}{UTF8}{mj}在区间\end{CJK} $[0,+\infty)$ \begin{CJK}{UTF8}{mj}上非一致连续\end{CJK}, \begin{CJK}{UTF8}{mj}但对于任意实常数\end{CJK} $a>0$, $f(x)$ \begin{CJK}{UTF8}{mj}在区间\end{CJK} $[0, a]$ \begin{CJK}{UTF8}{mj}上一致连续\end{CJK}.

  \item ( 12 \begin{CJK}{UTF8}{mj}分\end{CJK}) \begin{CJK}{UTF8}{mj}设\end{CJK} $f(x)$ \begin{CJK}{UTF8}{mj}在区间\end{CJK} $[0,1]$ \begin{CJK}{UTF8}{mj}上连续\end{CJK}, \begin{CJK}{UTF8}{mj}在\end{CJK} $(0,1)$ \begin{CJK}{UTF8}{mj}内可导\end{CJK}, \begin{CJK}{UTF8}{mj}且\end{CJK} $\int_{0}^{1} f(x) \mathrm{d} x=0$. \begin{CJK}{UTF8}{mj}证明\end{CJK}: \begin{CJK}{UTF8}{mj}存在一点\end{CJK} $\xi \in(0,1)$, \begin{CJK}{UTF8}{mj}使得\end{CJK} $2 f(\xi)+\xi f^{\prime}(\xi)=0$.

  \item (12 \begin{CJK}{UTF8}{mj}分\end{CJK}) \begin{CJK}{UTF8}{mj}求函数\end{CJK} $y=x^{4}(12 \ln x-7)$ \begin{CJK}{UTF8}{mj}的凸性区间及拐点\end{CJK}.

  \item (12 \begin{CJK}{UTF8}{mj}分\end{CJK}) \begin{CJK}{UTF8}{mj}设\end{CJK} $f(x)$ \begin{CJK}{UTF8}{mj}在\end{CJK} $[0,+\infty)$ \begin{CJK}{UTF8}{mj}上可导\end{CJK}, $f(0)=0$, \begin{CJK}{UTF8}{mj}且其反函数为\end{CJK} $g(x)$, \begin{CJK}{UTF8}{mj}若\end{CJK} $\int_{0}^{f(x)} g(t) \mathrm{d} t=x^{2} e^{x}$, \begin{CJK}{UTF8}{mj}求\end{CJK} $f(x)$.

  \item ( 12 \begin{CJK}{UTF8}{mj}分\end{CJK}) \begin{CJK}{UTF8}{mj}设常数\end{CJK} $a>0$, \begin{CJK}{UTF8}{mj}证明\end{CJK}: \begin{CJK}{UTF8}{mj}函数项级数\end{CJK}

\end{enumerate}
$$
\sum_{n=0}^{\infty} 3^{n} \sin \frac{1}{8^{n} x}
$$
\begin{CJK}{UTF8}{mj}在区间\end{CJK} $[a,+\infty)$ \begin{CJK}{UTF8}{mj}上一致连续\end{CJK}.

\begin{enumerate}
  \setcounter{enumi}{6}
  \item (14 \begin{CJK}{UTF8}{mj}分\end{CJK}) \begin{CJK}{UTF8}{mj}求椭球面\end{CJK} $\frac{x^{2}}{a^{2}}+\frac{y^{2}}{b^{2}}+\frac{z^{2}}{c^{2}}=1$ \begin{CJK}{UTF8}{mj}在第一卦限部分上的切平面与三个坐标面所围成的四面体的最小体积\end{CJK}. 7. (12 \begin{CJK}{UTF8}{mj}分\end{CJK}) \begin{CJK}{UTF8}{mj}设函数\end{CJK} $P(x, y, z), Q(x, y, z), R(x, y, z)$ \begin{CJK}{UTF8}{mj}都在\end{CJK} $\mathbb{R}^{3}$ \begin{CJK}{UTF8}{mj}上具有连续的偏导数\end{CJK}, \begin{CJK}{UTF8}{mj}且对于任意光滑曲线\end{CJK} $\Sigma$ \begin{CJK}{UTF8}{mj}有\end{CJK}
\end{enumerate}
$$
\iint_{\Sigma} P \mathrm{~d} y \mathrm{~d} z+Q \mathrm{~d} z \mathrm{~d} x+R \mathrm{~d} x \mathrm{~d} y=0,
$$
\begin{CJK}{UTF8}{mj}证明\end{CJK}: \begin{CJK}{UTF8}{mj}在\end{CJK} $\mathbb{R}^{3}$ \begin{CJK}{UTF8}{mj}上\end{CJK}, \begin{CJK}{UTF8}{mj}恒有\end{CJK}
$$
\frac{\partial P}{\partial x}+\frac{\partial Q}{\partial y}+\frac{\partial R}{\partial z} \equiv 0 .
$$

\begin{enumerate}
  \setcounter{enumi}{8}
  \item (12 \begin{CJK}{UTF8}{mj}分\end{CJK}) \begin{CJK}{UTF8}{mj}求由平面\end{CJK} $y=0, y=k x(k>0), z=0$ \begin{CJK}{UTF8}{mj}以及球心在原点\end{CJK}, \begin{CJK}{UTF8}{mj}半径为\end{CJK} $R$ \begin{CJK}{UTF8}{mj}的上半球面所围的第一卦限的立\end{CJK} \begin{CJK}{UTF8}{mj}体体积\end{CJK}.

  \item ( 14 \begin{CJK}{UTF8}{mj}分\end{CJK}) \begin{CJK}{UTF8}{mj}设函数\end{CJK} $f(x)$ \begin{CJK}{UTF8}{mj}在区间\end{CJK} $[0,1]$ \begin{CJK}{UTF8}{mj}上二阶可导\end{CJK}, \begin{CJK}{UTF8}{mj}且有\end{CJK} $f(0)=f(1)=0, \min _{0 \leqslant x \leqslant 1} f(x)=-1$, \begin{CJK}{UTF8}{mj}证明\end{CJK}: \begin{CJK}{UTF8}{mj}存在\end{CJK} $\eta \in(0,1)$, \begin{CJK}{UTF8}{mj}使得\end{CJK} $f^{\prime \prime}(\eta) \geqslant 8$.

\end{enumerate}
\section{7. 电子科技大学 2015 年研究生入学考试试题数学分析}
\begin{CJK}{UTF8}{mj}李扬\end{CJK}

\begin{CJK}{UTF8}{mj}微信公众号\end{CJK}: sxkyliyang

\begin{CJK}{UTF8}{mj}一\end{CJK}、\begin{CJK}{UTF8}{mj}填空题\end{CJK}(\begin{CJK}{UTF8}{mj}每小题\end{CJK} 5 \begin{CJK}{UTF8}{mj}分\end{CJK}, \begin{CJK}{UTF8}{mj}共\end{CJK} 40 \begin{CJK}{UTF8}{mj}分\end{CJK})

\begin{enumerate}
  \item \begin{CJK}{UTF8}{mj}设\end{CJK} $a_{1}, a_{2}, a_{3}, a_{4}$ \begin{CJK}{UTF8}{mj}均为正实数\end{CJK}, \begin{CJK}{UTF8}{mj}则\end{CJK} $\lim _{x \rightarrow \infty}\left(\frac{a_{1}^{\frac{1}{x}}+a_{2}^{\frac{1}{x}}+a_{3}^{\frac{1}{x}}+a_{4}^{\frac{1}{x}}}{4}\right)^{4 x}=$

  \item \begin{CJK}{UTF8}{mj}如果\end{CJK}

\end{enumerate}
$$
f(x)=\left\{\begin{array}{l}
\sin x, \quad x<0 \\
\ln (a x+b), x \geqslant 0
\end{array}\right.
$$
\begin{CJK}{UTF8}{mj}在\end{CJK} $(-\infty,+\infty)$ \begin{CJK}{UTF8}{mj}上可导\end{CJK}, \begin{CJK}{UTF8}{mj}则\end{CJK} $a=$ ,$b=$

\begin{enumerate}
  \setcounter{enumi}{3}
  \item $\int_{-\frac{1}{2}}^{\frac{1}{2}}\left[x \sin ^{4} x+\frac{(\arcsin x)^{2}}{\sqrt{1-x^{2}}}\right] \mathrm{d} x=$

  \item \begin{CJK}{UTF8}{mj}交换累次积分的次序\end{CJK} $\int_{0}^{2 \pi} \mathrm{d} x \int_{0}^{\sin x} f(x, y) \mathrm{d} y=$

  \item \begin{CJK}{UTF8}{mj}设\end{CJK} $f(x, y, z)=\left(\frac{x}{y}\right)^{z}$, \begin{CJK}{UTF8}{mj}则其全微分\end{CJK} $\mathrm{d} f=$

\end{enumerate}
\begin{CJK}{UTF8}{mj}二\end{CJK}、(\begin{CJK}{UTF8}{mj}每小题\end{CJK} 7 \begin{CJK}{UTF8}{mj}题\end{CJK}, \begin{CJK}{UTF8}{mj}共\end{CJK} 14 \begin{CJK}{UTF8}{mj}分\end{CJK})

\begin{enumerate}
  \item ( 7 \begin{CJK}{UTF8}{mj}分\end{CJK}) \begin{CJK}{UTF8}{mj}已知\end{CJK} $\arctan \frac{x}{y}=\ln \sqrt{x^{2}+y^{2}}$, \begin{CJK}{UTF8}{mj}求\end{CJK} $\frac{\mathrm{d}^{2} y}{\mathrm{~d} x^{2}}$;

  \item ( 7 \begin{CJK}{UTF8}{mj}分\end{CJK}) \begin{CJK}{UTF8}{mj}求椭球面\end{CJK} $x^{2}+\frac{y^{2}}{4}+\frac{z^{2}}{4}=1$ \begin{CJK}{UTF8}{mj}上的点\end{CJK}, \begin{CJK}{UTF8}{mj}使其法线与三个坐标轴正方向成等角\end{CJK}.

\end{enumerate}
\begin{CJK}{UTF8}{mj}三\end{CJK}、(\begin{CJK}{UTF8}{mj}每小题\end{CJK} 8 \begin{CJK}{UTF8}{mj}题\end{CJK}, \begin{CJK}{UTF8}{mj}共\end{CJK} 16 \begin{CJK}{UTF8}{mj}分\end{CJK})

\begin{enumerate}
  \item (8 \begin{CJK}{UTF8}{mj}分\end{CJK}) \begin{CJK}{UTF8}{mj}求函数\end{CJK} $f(x)=\int_{0}^{x^{2}} \ln \left(1+t^{2}\right) \mathrm{d} t$ \begin{CJK}{UTF8}{mj}的极值点和极值\end{CJK}.

  \item (8 \begin{CJK}{UTF8}{mj}分\end{CJK}) \begin{CJK}{UTF8}{mj}计算\end{CJK}

\end{enumerate}
$$
I=\int_{\Sigma} \sin \sqrt{y^{2}+z^{2}} \mathrm{~d} y \mathrm{~d} z+\sin \sqrt{z^{2}+x^{2}} \mathrm{~d} z \mathrm{~d} x+\sin \sqrt{x^{2}+y^{2}} \mathrm{~d} x \mathrm{~d} y,
$$
\begin{CJK}{UTF8}{mj}其中\end{CJK} $\Sigma$ \begin{CJK}{UTF8}{mj}为雉面\end{CJK} $z=\sqrt{x^{2}+y^{2}}(0 \leqslant z \leqslant h)$ \begin{CJK}{UTF8}{mj}的上侧\end{CJK}.

\begin{CJK}{UTF8}{mj}四\end{CJK}、( 16 \begin{CJK}{UTF8}{mj}分\end{CJK}) \begin{CJK}{UTF8}{mj}证明\end{CJK}: $\sin \frac{1}{\sqrt{x}}$ \begin{CJK}{UTF8}{mj}在\end{CJK} $(\eta, 1)(0<\eta<1)$ \begin{CJK}{UTF8}{mj}上一致收敛\end{CJK}, \begin{CJK}{UTF8}{mj}但在\end{CJK} $(0,1)$ \begin{CJK}{UTF8}{mj}上不一致收敛\end{CJK}.

\begin{CJK}{UTF8}{mj}五\end{CJK}、(12 \begin{CJK}{UTF8}{mj}分\end{CJK}) \begin{CJK}{UTF8}{mj}设函数\end{CJK} $f(x)$ \begin{CJK}{UTF8}{mj}在闭区间\end{CJK} $[a, b]$ \begin{CJK}{UTF8}{mj}连续\end{CJK}, \begin{CJK}{UTF8}{mj}在开区间\end{CJK} $(a, b)$ \begin{CJK}{UTF8}{mj}存在二阶导数\end{CJK}, \begin{CJK}{UTF8}{mj}且\end{CJK} $f(a)=f(b)=0, f(c)>0$, \begin{CJK}{UTF8}{mj}其中\end{CJK} $a<c<b$, \begin{CJK}{UTF8}{mj}证明\end{CJK}: \begin{CJK}{UTF8}{mj}在\end{CJK} $(a, b)$ \begin{CJK}{UTF8}{mj}内至少存在一点\end{CJK} $\xi$, \begin{CJK}{UTF8}{mj}使\end{CJK} $f^{\prime \prime}(\xi)<0$.

\begin{CJK}{UTF8}{mj}六\end{CJK}、( 12 \begin{CJK}{UTF8}{mj}分\end{CJK}) \begin{CJK}{UTF8}{mj}设函数\end{CJK} $f(x)$ \begin{CJK}{UTF8}{mj}在区间\end{CJK} $[a, b]$ \begin{CJK}{UTF8}{mj}上可积\end{CJK}, \begin{CJK}{UTF8}{mj}且在\end{CJK} $[a, b]$ \begin{CJK}{UTF8}{mj}上满足\end{CJK} $|f(x)| \geqslant c>0$ ( $c$ \begin{CJK}{UTF8}{mj}为常数\end{CJK}), \begin{CJK}{UTF8}{mj}证明\end{CJK}: $\frac{1}{f(x)}$ \begin{CJK}{UTF8}{mj}在\end{CJK} $[a, b]$ \begin{CJK}{UTF8}{mj}上也可积\end{CJK}.

\begin{CJK}{UTF8}{mj}七\end{CJK}、( 12 \begin{CJK}{UTF8}{mj}分\end{CJK}) \begin{CJK}{UTF8}{mj}证明\end{CJK}: \begin{CJK}{UTF8}{mj}函数项级数\end{CJK} $\sum_{n=1}^{\infty} \frac{x^{n}}{(2 n-1) 2 n}$ \begin{CJK}{UTF8}{mj}在区间\end{CJK} $[0,1]$ \begin{CJK}{UTF8}{mj}上一致收敛\end{CJK}.

\begin{CJK}{UTF8}{mj}八\end{CJK}、(15 \begin{CJK}{UTF8}{mj}分\end{CJK}) \begin{CJK}{UTF8}{mj}求幂级数\end{CJK} $\sum_{n=1}^{\infty} \frac{x^{n}}{n(n+1)}$ \begin{CJK}{UTF8}{mj}的和函数\end{CJK}, \begin{CJK}{UTF8}{mj}并指出其定义域\end{CJK}.

\begin{CJK}{UTF8}{mj}九\end{CJK}、(12 \begin{CJK}{UTF8}{mj}分\end{CJK}) \begin{CJK}{UTF8}{mj}设一元函数\end{CJK} $f(u)$ \begin{CJK}{UTF8}{mj}在区间\end{CJK} $[-1,1]$ \begin{CJK}{UTF8}{mj}上连续\end{CJK}, \begin{CJK}{UTF8}{mj}证明\end{CJK}:
$$
\iiint_{\Omega} f(z) \mathrm{d} x \mathrm{~d} y \mathrm{~d} z=\pi \int_{-1}^{1} f(u)\left(1-u^{2}\right) \mathrm{d} u .
$$
\begin{CJK}{UTF8}{mj}其中\end{CJK} $\Omega$ \begin{CJK}{UTF8}{mj}为单位球体\end{CJK} $x^{2}+y^{2}+z^{2} \leqslant 1$.

\begin{CJK}{UTF8}{mj}十\end{CJK}、(16 \begin{CJK}{UTF8}{mj}分\end{CJK}) \begin{CJK}{UTF8}{mj}用闭区间套定理证明\end{CJK}: \begin{CJK}{UTF8}{mj}非空且有上界的数集必有上确界\end{CJK}.

\section{8. 电子科技大学 2016 年研究生入学考试试题数学分析}
\begin{CJK}{UTF8}{mj}李扬\end{CJK}

\begin{CJK}{UTF8}{mj}微信公众号\end{CJK}: sxkyliyang

\begin{CJK}{UTF8}{mj}一\end{CJK}、\begin{CJK}{UTF8}{mj}填空题\end{CJK}(\begin{CJK}{UTF8}{mj}每小题\end{CJK} 5 \begin{CJK}{UTF8}{mj}分\end{CJK}, \begin{CJK}{UTF8}{mj}共\end{CJK} 25 \begin{CJK}{UTF8}{mj}分\end{CJK})

\begin{enumerate}
  \item \begin{CJK}{UTF8}{mj}极限\end{CJK} $\lim _{x \rightarrow 1}(2-x)^{\tan \frac{\pi x}{2}}=$

  \item \begin{CJK}{UTF8}{mj}若直线\end{CJK} $y=x$ \begin{CJK}{UTF8}{mj}与曲线\end{CJK} $y=\log _{a} x$ \begin{CJK}{UTF8}{mj}相切\end{CJK}, \begin{CJK}{UTF8}{mj}则\end{CJK} $a=$ \begin{CJK}{UTF8}{mj}切点坐标为\end{CJK}

  \item \begin{CJK}{UTF8}{mj}抛物线\end{CJK} $y=x^{2}-4 x+6$ \begin{CJK}{UTF8}{mj}与直线\end{CJK} $y=x+2$ \begin{CJK}{UTF8}{mj}所围成的图形面积\end{CJK} $A=$

  \item \begin{CJK}{UTF8}{mj}设函数\end{CJK} $z=f(x, y)$ \begin{CJK}{UTF8}{mj}由方程\end{CJK} $x e^{x-y-z}=x-y+2 z$ \begin{CJK}{UTF8}{mj}所确定\end{CJK}, \begin{CJK}{UTF8}{mj}则\end{CJK} $\frac{\partial z}{\partial x}=$

  \item \begin{CJK}{UTF8}{mj}设区域\end{CJK} $D$ \begin{CJK}{UTF8}{mj}由直线\end{CJK} $y=x, x=2$ \begin{CJK}{UTF8}{mj}及曲线\end{CJK} $x y=2$ \begin{CJK}{UTF8}{mj}所围成\end{CJK}, \begin{CJK}{UTF8}{mj}则二重积分\end{CJK} $\iint_{D} f(x, y) \mathrm{d} x \mathrm{~d} y$ \begin{CJK}{UTF8}{mj}先对\end{CJK} $x$ \begin{CJK}{UTF8}{mj}后对\end{CJK} $y$ \begin{CJK}{UTF8}{mj}的累次积分\end{CJK} \begin{CJK}{UTF8}{mj}为\end{CJK}

\end{enumerate}
\begin{CJK}{UTF8}{mj}二\end{CJK}、\begin{CJK}{UTF8}{mj}计算题\end{CJK}(\begin{CJK}{UTF8}{mj}每小题\end{CJK} 7 \begin{CJK}{UTF8}{mj}题\end{CJK}, \begin{CJK}{UTF8}{mj}共\end{CJK} 14 \begin{CJK}{UTF8}{mj}分\end{CJK})

\begin{enumerate}
  \item ( 7 \begin{CJK}{UTF8}{mj}分\end{CJK}) \begin{CJK}{UTF8}{mj}设函数\end{CJK} $y=y(x)$ \begin{CJK}{UTF8}{mj}由参数方程\end{CJK} $f(x)=\left\{\begin{array}{l}x=a t \cos t ; \\ y=a t \sin t .\end{array}\right.$ \begin{CJK}{UTF8}{mj}所确定\end{CJK}, \begin{CJK}{UTF8}{mj}求\end{CJK} $\frac{\mathrm{d}^{2} y}{\mathrm{~d} x^{2}}$

  \item ( 7 \begin{CJK}{UTF8}{mj}分\end{CJK}) \begin{CJK}{UTF8}{mj}求幂级数\end{CJK} $\sum_{n=1}^{\infty} \frac{x^{2 n-1}}{2 n-1}$ \begin{CJK}{UTF8}{mj}的和函数及定义域\end{CJK}.

\end{enumerate}
\begin{CJK}{UTF8}{mj}三\end{CJK}、\begin{CJK}{UTF8}{mj}计算题\end{CJK}(\begin{CJK}{UTF8}{mj}每小题\end{CJK} 8 \begin{CJK}{UTF8}{mj}题\end{CJK}, \begin{CJK}{UTF8}{mj}共\end{CJK} 16 \begin{CJK}{UTF8}{mj}分\end{CJK})

\begin{enumerate}
  \item (8 \begin{CJK}{UTF8}{mj}分\end{CJK}) \begin{CJK}{UTF8}{mj}计算\end{CJK} $\int_{0}^{1} x^{7}|x-a| \mathrm{d} x$, \begin{CJK}{UTF8}{mj}其中\end{CJK} $a$ \begin{CJK}{UTF8}{mj}为常数\end{CJK};

  \item (8 \begin{CJK}{UTF8}{mj}分\end{CJK}) \begin{CJK}{UTF8}{mj}计算第二类曲线积分\end{CJK}

\end{enumerate}
$$
I=\int_{L}\left[e^{x} \sin y-b(x+y)\right] \mathrm{d} x+\left(e^{x} \cos y-a x\right) \mathrm{d} y
$$
\begin{CJK}{UTF8}{mj}其中\end{CJK} $a, b$ \begin{CJK}{UTF8}{mj}为正常数\end{CJK}, $L$ \begin{CJK}{UTF8}{mj}为曲线\end{CJK} $y=\sqrt{2 a x-x^{2}}$ \begin{CJK}{UTF8}{mj}上从\end{CJK} $(2 a, 0)$ \begin{CJK}{UTF8}{mj}到\end{CJK} $(0,0)$ \begin{CJK}{UTF8}{mj}的一段\end{CJK}.

\begin{CJK}{UTF8}{mj}四\end{CJK}、 (16 \begin{CJK}{UTF8}{mj}分\end{CJK}) \begin{CJK}{UTF8}{mj}证明\end{CJK}: $f(x)=\sqrt[3]{x}$ \begin{CJK}{UTF8}{mj}在\end{CJK} $[a,+\infty)(a>0)$ \begin{CJK}{UTF8}{mj}上一致连续\end{CJK}.

\begin{CJK}{UTF8}{mj}五\end{CJK}、(12 \begin{CJK}{UTF8}{mj}分\end{CJK}) \begin{CJK}{UTF8}{mj}设函数\end{CJK} $f(x), g(x)$ \begin{CJK}{UTF8}{mj}在区间\end{CJK} $[a, b]$ \begin{CJK}{UTF8}{mj}上连续\end{CJK}, \begin{CJK}{UTF8}{mj}且在\end{CJK} $(a, b)$ \begin{CJK}{UTF8}{mj}内可导\end{CJK}, \begin{CJK}{UTF8}{mj}证明\end{CJK}: \begin{CJK}{UTF8}{mj}存在\end{CJK} $\xi \in(a, b)$, \begin{CJK}{UTF8}{mj}使得\end{CJK}
$$
\left|\begin{array}{ll}
f(a) & f(b) \\
g(a) & g(b)
\end{array}\right|=(b-a)\left|\begin{array}{cc}
f(a) & f^{\prime}(\xi) \\
g(a) & g^{\prime}(\xi)
\end{array}\right| .
$$
\begin{CJK}{UTF8}{mj}六\end{CJK}、 ( 12 \begin{CJK}{UTF8}{mj}分\end{CJK}) \begin{CJK}{UTF8}{mj}证明\end{CJK}: \begin{CJK}{UTF8}{mj}函数项级数\end{CJK}
$$
\sum_{n=1}^{\infty} \frac{n^{2} x}{1+n^{8} x^{2}}
$$
\begin{CJK}{UTF8}{mj}在\end{CJK} $(-\infty,+\infty)$ \begin{CJK}{UTF8}{mj}上一致收敛\end{CJK}.

\begin{CJK}{UTF8}{mj}七\end{CJK}、 ( 14 \begin{CJK}{UTF8}{mj}分\end{CJK}) \begin{CJK}{UTF8}{mj}曲面\end{CJK} $\sqrt{x}+\sqrt{y}+\sqrt{z}=\sqrt{a}(a>0)$ \begin{CJK}{UTF8}{mj}上任意一点的切平面在各坐标轴上的截距之和等于\end{CJK} $a$.

\begin{CJK}{UTF8}{mj}八\end{CJK}、 (15 \begin{CJK}{UTF8}{mj}分\end{CJK}) \begin{CJK}{UTF8}{mj}计算三重积分\end{CJK}
$$
I=\int_{\Omega}\left(\sqrt{x^{2}+y^{2}+z^{2}}\right)^{5} \mathrm{~d} x \mathrm{~d} y \mathrm{~d} z
$$
\begin{CJK}{UTF8}{mj}其中\end{CJK} $\Omega$ \begin{CJK}{UTF8}{mj}为球体\end{CJK} $\left\{(x, y, z) \mid x^{2}+y^{2}+z^{2} \leqslant 2 z\right\}$.

\begin{CJK}{UTF8}{mj}九\end{CJK}、 ( 12 \begin{CJK}{UTF8}{mj}分\end{CJK}) \begin{CJK}{UTF8}{mj}设函数\end{CJK} $f(x)$ \begin{CJK}{UTF8}{mj}具有二阶导数\end{CJK}, $F(x)$ \begin{CJK}{UTF8}{mj}是可导的\end{CJK}, \begin{CJK}{UTF8}{mj}证明\end{CJK}: \begin{CJK}{UTF8}{mj}函数\end{CJK}
$$
u(x, t)=\frac{1}{2}[f(x-a t)+f(x+a t)]+\frac{1}{2 a} \int_{x-a t}^{x+a t} F(y) \mathrm{d} y .
$$
\begin{CJK}{UTF8}{mj}满足振动方程\end{CJK} $\frac{\partial u}{\partial t^{2}}=a^{2} \frac{\partial u}{\partial x^{2}}$, \begin{CJK}{UTF8}{mj}以及初始条件\end{CJK} $u(x, 0)=f(x), \frac{\partial u}{\partial t}(x, 0)=F(x)$.

\begin{CJK}{UTF8}{mj}十\end{CJK}、(16 \begin{CJK}{UTF8}{mj}分\end{CJK}) \begin{CJK}{UTF8}{mj}用确界存在定理证明零点存在定理\end{CJK}: \begin{CJK}{UTF8}{mj}若函数\end{CJK} $f(x)$ \begin{CJK}{UTF8}{mj}在闭区间\end{CJK} $[a, b]$ \begin{CJK}{UTF8}{mj}连续\end{CJK}, \begin{CJK}{UTF8}{mj}且\end{CJK} $f(a) \cdot f(b)<0$, \begin{CJK}{UTF8}{mj}则一定\end{CJK} \begin{CJK}{UTF8}{mj}存在\end{CJK} $\xi \in(a, b)$, \begin{CJK}{UTF8}{mj}使\end{CJK} $f(\xi)=0$.

\section{9. 电子科技大学 2009 年研究生入学考试试题线性代数 
 李扬 
 微信公众号: sxkyliyang}
\begin{enumerate}
  \item (12 \begin{CJK}{UTF8}{mj}分\end{CJK}) \begin{CJK}{UTF8}{mj}设\end{CJK} $A=\left(a_{i j}\right), a_{i j}=|i-j|, i, j=1,2, \cdots, n$, \begin{CJK}{UTF8}{mj}计算行列式\end{CJK} $|A|$.

  \item (10 \begin{CJK}{UTF8}{mj}分\end{CJK}) \begin{CJK}{UTF8}{mj}设\end{CJK} 4 \begin{CJK}{UTF8}{mj}阶矩阵\end{CJK} $B$ \begin{CJK}{UTF8}{mj}满足\end{CJK} $\left(\frac{1}{4} A^{*}\right)^{-1} B A^{-1}=2 A B+E, E$ \begin{CJK}{UTF8}{mj}表示单位矩阵\end{CJK},

\end{enumerate}
$$
A=\left(\begin{array}{llll}
2 & 0 & 0 & 0 \\
1 & 1 & 0 & 0 \\
0 & 0 & 2 & 1 \\
0 & 0 & 0 & 1
\end{array}\right)
$$
\begin{CJK}{UTF8}{mj}求\end{CJK} $B$.

\begin{enumerate}
  \setcounter{enumi}{3}
  \item ( 20 \begin{CJK}{UTF8}{mj}分\end{CJK}) \begin{CJK}{UTF8}{mj}设\end{CJK} $A$ \begin{CJK}{UTF8}{mj}是\end{CJK} $n$ \begin{CJK}{UTF8}{mj}阶矩阵\end{CJK} $(n \geqslant 2), A^{*}$ \begin{CJK}{UTF8}{mj}是\end{CJK} $A$ \begin{CJK}{UTF8}{mj}的伴随矩阵\end{CJK}, $r(A)$ \begin{CJK}{UTF8}{mj}表示\end{CJK} $A$ \begin{CJK}{UTF8}{mj}的秩\end{CJK}, \begin{CJK}{UTF8}{mj}证明\end{CJK}:
\end{enumerate}
$$
r\left(A^{*}\right)=\left\{\begin{array}{l}
n, \text { 当 } r(A)=n ; \\
1, \text { 当 } r(A)=n-1 ; \\
0, \text { 当 } r(A)<n-1 .
\end{array}\right.
$$

\begin{enumerate}
  \setcounter{enumi}{4}
  \item ( 15 \begin{CJK}{UTF8}{mj}分\end{CJK}) \begin{CJK}{UTF8}{mj}设有向量组\end{CJK} $(I): \alpha_{1}=(1,0,2), \alpha_{2}=(1,1,3), \alpha_{3}=(1,-1, a+2)$. \begin{CJK}{UTF8}{mj}向量组\end{CJK} $(I I): \beta_{1}=(1,2, a+$ $3), \beta_{2}=(2,1, a+6), \beta_{3}=(2,1, a+4)$. \begin{CJK}{UTF8}{mj}试问\end{CJK}:\begin{CJK}{UTF8}{mj}当\end{CJK} $a$ \begin{CJK}{UTF8}{mj}为何值时\end{CJK}, \begin{CJK}{UTF8}{mj}组\end{CJK} $(I)$ \begin{CJK}{UTF8}{mj}与组\end{CJK} $(I I)$ \begin{CJK}{UTF8}{mj}等价\end{CJK}? \begin{CJK}{UTF8}{mj}当\end{CJK} $a$ \begin{CJK}{UTF8}{mj}为何值时\end{CJK}, \begin{CJK}{UTF8}{mj}组\end{CJK} $(I)$ \begin{CJK}{UTF8}{mj}与组\end{CJK} $(I I)$ \begin{CJK}{UTF8}{mj}不等价\end{CJK}?

  \item ( 18 \begin{CJK}{UTF8}{mj}分\end{CJK}) \begin{CJK}{UTF8}{mj}设\end{CJK} $A$ \begin{CJK}{UTF8}{mj}是\end{CJK} $n$ \begin{CJK}{UTF8}{mj}阶矩阵\end{CJK}, $\alpha_{1}, \cdots, \alpha_{n}$ \begin{CJK}{UTF8}{mj}是\end{CJK} $n$ \begin{CJK}{UTF8}{mj}维列向量\end{CJK}, $\alpha_{n} \neq 0$, \begin{CJK}{UTF8}{mj}若\end{CJK} $A \alpha_{1}=\alpha_{2}, A \alpha_{2}=\alpha_{3}, \cdots, A \alpha_{n-1}=$ $\alpha_{n}, A \alpha_{n}=0 .$

\end{enumerate}
(1) \begin{CJK}{UTF8}{mj}证明\end{CJK}: $\alpha_{1}, \alpha_{2}, \cdots, \alpha_{n}$ \begin{CJK}{UTF8}{mj}线性无关\end{CJK};

(2) \begin{CJK}{UTF8}{mj}求\end{CJK} $A$ \begin{CJK}{UTF8}{mj}的特征值和特征向量\end{CJK}.

\begin{enumerate}
  \setcounter{enumi}{6}
  \item (16 \begin{CJK}{UTF8}{mj}分\end{CJK})
\end{enumerate}
(1) \begin{CJK}{UTF8}{mj}设\end{CJK} $n$ \begin{CJK}{UTF8}{mj}阶矩阵\end{CJK} $A$ \begin{CJK}{UTF8}{mj}的特征值为\end{CJK} $\lambda_{1}, \lambda_{2}, \cdots, \lambda_{n}$, \begin{CJK}{UTF8}{mj}属于\end{CJK} $\lambda_{i}$ \begin{CJK}{UTF8}{mj}的特征向量为\end{CJK} $\alpha_{i}(1 \leqslant i \leqslant n)$, \begin{CJK}{UTF8}{mj}求\end{CJK} $P^{-1} A P$ \begin{CJK}{UTF8}{mj}的特征值和\end{CJK} \begin{CJK}{UTF8}{mj}特征向量\end{CJK}.

(2) \begin{CJK}{UTF8}{mj}设\end{CJK} $f(x)$ \begin{CJK}{UTF8}{mj}是一个多项式\end{CJK}, $A$ \begin{CJK}{UTF8}{mj}是\end{CJK} $n$ \begin{CJK}{UTF8}{mj}阶可逆矩阵\end{CJK}, \begin{CJK}{UTF8}{mj}特征值为\end{CJK} $\lambda_{1}, \lambda_{2}, \cdots, \lambda_{n}$, \begin{CJK}{UTF8}{mj}求\end{CJK} $f\left(A^{-1}\right)$ \begin{CJK}{UTF8}{mj}的特征值\end{CJK}.

\begin{enumerate}
  \setcounter{enumi}{7}
  \item (10 \begin{CJK}{UTF8}{mj}分\end{CJK}) \begin{CJK}{UTF8}{mj}证明\end{CJK} $n$ \begin{CJK}{UTF8}{mj}维欧氏空间中的勾股定理\end{CJK}: \begin{CJK}{UTF8}{mj}向量\end{CJK} $\alpha$ \begin{CJK}{UTF8}{mj}与\end{CJK} $\beta$ \begin{CJK}{UTF8}{mj}正交的充分必要条件是\end{CJK} $\|\alpha\|^{2}+\|\beta\|^{2}=\|\alpha-\beta\|^{2}$.

  \item (12 \begin{CJK}{UTF8}{mj}分\end{CJK}) \begin{CJK}{UTF8}{mj}证明\end{CJK}: \begin{CJK}{UTF8}{mj}如果\end{CJK} $n$ \begin{CJK}{UTF8}{mj}维线性空间的两个线性子空间的维数之和大于\end{CJK} $n$, \begin{CJK}{UTF8}{mj}则这两个子空间有公共的非零向量\end{CJK}.

  \item ( 14 \begin{CJK}{UTF8}{mj}分\end{CJK}) \begin{CJK}{UTF8}{mj}设\end{CJK} $W$ \begin{CJK}{UTF8}{mj}为\end{CJK} $\alpha_{1}=(1,0,2,1), \alpha_{2}=(2,1,2,3), \alpha_{3}=(0,1,-2,1)$ \begin{CJK}{UTF8}{mj}张成的子空间\end{CJK}, \begin{CJK}{UTF8}{mj}求\end{CJK} $W$ \begin{CJK}{UTF8}{mj}的正交补\end{CJK} $W^{\perp}$ \begin{CJK}{UTF8}{mj}的\end{CJK} \begin{CJK}{UTF8}{mj}标准正交基\end{CJK}.

  \item ( 15 \begin{CJK}{UTF8}{mj}分\end{CJK}) \begin{CJK}{UTF8}{mj}设\end{CJK} $\varepsilon_{1}, \cdots, \varepsilon_{4}$ \begin{CJK}{UTF8}{mj}是四维线性空间\end{CJK} $V$ \begin{CJK}{UTF8}{mj}的一组基\end{CJK}, \begin{CJK}{UTF8}{mj}线性变换\end{CJK} $T$ \begin{CJK}{UTF8}{mj}在这组基下的矩阵为\end{CJK}

\end{enumerate}
$$
A=\left(\begin{array}{cccc}
1 & 0 & 2 & 1 \\
-1 & 2 & 1 & 3 \\
1 & 2 & 5 & 5 \\
2 & -2 & 1 & -2
\end{array}\right)
$$
(1) \begin{CJK}{UTF8}{mj}求\end{CJK} $T$ \begin{CJK}{UTF8}{mj}在基\end{CJK} $\eta_{1}=\varepsilon_{1}-2 \varepsilon_{2}+\varepsilon_{4}, \eta_{2}=3 \varepsilon_{2}-\varepsilon_{3}-\varepsilon_{4}, \eta_{3}=\varepsilon_{3}+\varepsilon_{4}, \eta_{4}=2 \varepsilon_{4}$ \begin{CJK}{UTF8}{mj}下的矩阵\end{CJK};

(2) \begin{CJK}{UTF8}{mj}求\end{CJK} $T$ \begin{CJK}{UTF8}{mj}的核\end{CJK};

(3) \begin{CJK}{UTF8}{mj}在\end{CJK} $T$ \begin{CJK}{UTF8}{mj}的核中选一组基\end{CJK}, \begin{CJK}{UTF8}{mj}把它扩充成\end{CJK} $V$ \begin{CJK}{UTF8}{mj}的一组基\end{CJK}, \begin{CJK}{UTF8}{mj}并求\end{CJK} $T$ \begin{CJK}{UTF8}{mj}在该基下的矩阵\end{CJK}.

\section{0. 电子科技大学 2010 年研究生入学考试试题线性代数 
 李扬 
 微信公众号: sxkyliyang}
\begin{enumerate}
  \item (13 \begin{CJK}{UTF8}{mj}分\end{CJK}) \begin{CJK}{UTF8}{mj}计算\end{CJK} 5 \begin{CJK}{UTF8}{mj}阶行列式\end{CJK}
\end{enumerate}
$$
D_{5}=\left|\begin{array}{ccccc}
1-a & a & 0 & 0 & 0 \\
-1 & 1-a & a & 0 & 0 \\
0 & -1 & 1-a & a & 0 \\
0 & 0 & -1 & 1-a & a \\
0 & 0 & 0 & -1 & 1-a
\end{array}\right|
$$

\begin{enumerate}
  \setcounter{enumi}{2}
  \item (13 \begin{CJK}{UTF8}{mj}分\end{CJK}) \begin{CJK}{UTF8}{mj}设\end{CJK} $n$ \begin{CJK}{UTF8}{mj}阶矩阵\end{CJK}
\end{enumerate}
$$
A=\left(\begin{array}{cccc}
1 & 1 & \cdots & 1 \\
& 1 & \cdots & 1 \\
& & \ddots & \vdots \\
& & & 1
\end{array}\right)
$$
\begin{CJK}{UTF8}{mj}求行列式\end{CJK} $|A|$ \begin{CJK}{UTF8}{mj}的所有元素的代数余子式之和\end{CJK}.

\begin{enumerate}
  \setcounter{enumi}{3}
  \item (13 \begin{CJK}{UTF8}{mj}分\end{CJK}) \begin{CJK}{UTF8}{mj}设秩\end{CJK} $\left(A_{m \times n}\right)=r>0$, \begin{CJK}{UTF8}{mj}证明\end{CJK}: \begin{CJK}{UTF8}{mj}存在秩为\end{CJK} $r$ \begin{CJK}{UTF8}{mj}的\end{CJK} $m \times r$ \begin{CJK}{UTF8}{mj}矩阵\end{CJK} $F$ \begin{CJK}{UTF8}{mj}与秩为\end{CJK} $r$ \begin{CJK}{UTF8}{mj}的\end{CJK} $r \times n$ \begin{CJK}{UTF8}{mj}矩阵\end{CJK} $G$, \begin{CJK}{UTF8}{mj}使\end{CJK} $A=F G$.

  \item ( 20 \begin{CJK}{UTF8}{mj}分\end{CJK}) \begin{CJK}{UTF8}{mj}已知齐次线性方程组\end{CJK}

\end{enumerate}
$$
(I)\left\{\begin{aligned}
x_{1}+x_{2}+\quad+x_{4} &=0 \\
a x_{1}+\quad+a^{2} x_{3}+&=0 \\
a x_{2}+\quad+a^{2} x_{4} &=0 .
\end{aligned}\right.
$$
\begin{CJK}{UTF8}{mj}的解都满足方程\end{CJK} $x_{1}+x_{2}+x_{3}=0$, \begin{CJK}{UTF8}{mj}求\end{CJK} $a$ \begin{CJK}{UTF8}{mj}和方程组\end{CJK} $(I)$ \begin{CJK}{UTF8}{mj}的通解\end{CJK}.

\begin{enumerate}
  \setcounter{enumi}{5}
  \item (14 \begin{CJK}{UTF8}{mj}分\end{CJK}) \begin{CJK}{UTF8}{mj}设\end{CJK} $A$ \begin{CJK}{UTF8}{mj}为\end{CJK} $r$ \begin{CJK}{UTF8}{mj}阶矩阵\end{CJK}, $B$ \begin{CJK}{UTF8}{mj}为\end{CJK} $r \times n$ \begin{CJK}{UTF8}{mj}矩阵\end{CJK}, $r(B)=r$, \begin{CJK}{UTF8}{mj}证明\end{CJK}:
\end{enumerate}
(1) \begin{CJK}{UTF8}{mj}若\end{CJK} $A B=0$, \begin{CJK}{UTF8}{mj}则\end{CJK} $A=0$;

(2) \begin{CJK}{UTF8}{mj}若\end{CJK} $A B=B$, \begin{CJK}{UTF8}{mj}则\end{CJK} $A=E$ (\begin{CJK}{UTF8}{mj}单位阵\end{CJK}).

\begin{enumerate}
  \setcounter{enumi}{6}
  \item (13 \begin{CJK}{UTF8}{mj}分\end{CJK}) \begin{CJK}{UTF8}{mj}设\end{CJK} $A$ \begin{CJK}{UTF8}{mj}是\end{CJK} $n$ \begin{CJK}{UTF8}{mj}阶实对称矩阵\end{CJK}, \begin{CJK}{UTF8}{mj}且\end{CJK} $A^{2}+2 A=0, r(A)=k$. \begin{CJK}{UTF8}{mj}求行列式\end{CJK} $|A+3 E|$.

  \item (15 \begin{CJK}{UTF8}{mj}分\end{CJK}) \begin{CJK}{UTF8}{mj}设二次型\end{CJK} $5 x_{1}^{2}+5 x_{2}^{2}+k x_{3}^{2}-2 x_{1} x_{2}+6 x_{1} x_{3}-6 x_{2} x_{3}$ \begin{CJK}{UTF8}{mj}的秩为\end{CJK} 2 .

\end{enumerate}
(1) \begin{CJK}{UTF8}{mj}求出\end{CJK} $k$, \begin{CJK}{UTF8}{mj}并写出\end{CJK} $f$ \begin{CJK}{UTF8}{mj}的标准形\end{CJK} (\begin{CJK}{UTF8}{mj}即平方和\end{CJK});

(2) \begin{CJK}{UTF8}{mj}问\end{CJK} $f\left(x_{1}, x_{2}, x_{3}\right)=1$ \begin{CJK}{UTF8}{mj}代表三维几何空间中何种几何曲面\end{CJK};

(3) \begin{CJK}{UTF8}{mj}求\end{CJK} $f$ \begin{CJK}{UTF8}{mj}在条件\end{CJK} $x_{1}^{2}+x_{2}^{2}+x_{3}^{2}=1$ \begin{CJK}{UTF8}{mj}下的最大值和最小值\end{CJK}.

\begin{enumerate}
  \setcounter{enumi}{8}
  \item ( 14 \begin{CJK}{UTF8}{mj}分\end{CJK}) \begin{CJK}{UTF8}{mj}设\end{CJK} $U$ \begin{CJK}{UTF8}{mj}是线性空间\end{CJK} $V$ \begin{CJK}{UTF8}{mj}的一个真子空间\end{CJK},\begin{CJK}{UTF8}{mj}问\end{CJK} $V$ \begin{CJK}{UTF8}{mj}中适合\end{CJK} $V=U+W$ \begin{CJK}{UTF8}{mj}的子空间\end{CJK} $W$ \begin{CJK}{UTF8}{mj}是否唯一\end{CJK}? \begin{CJK}{UTF8}{mj}说明理由\end{CJK}.

  \item ( 20 \begin{CJK}{UTF8}{mj}分\end{CJK}) \begin{CJK}{UTF8}{mj}设\end{CJK} $V$ \begin{CJK}{UTF8}{mj}是所有\end{CJK} 2 \begin{CJK}{UTF8}{mj}阶实方阵构成的实线性空间\end{CJK}, \begin{CJK}{UTF8}{mj}在定义内积\end{CJK} $(X, Y)=\operatorname{tr} X Y^{T}$ \begin{CJK}{UTF8}{mj}后\end{CJK}, $V$ \begin{CJK}{UTF8}{mj}成为一个欧氏空间\end{CJK}. \begin{CJK}{UTF8}{mj}现定义\end{CJK} $V$ \begin{CJK}{UTF8}{mj}上的变换\end{CJK} $\mathcal{A}: X \mapsto X+X^{T}$.

\end{enumerate}
(1) \begin{CJK}{UTF8}{mj}证明\end{CJK}: $\mathcal{A}$ \begin{CJK}{UTF8}{mj}是一个线性变换\end{CJK};

(2) \begin{CJK}{UTF8}{mj}求\end{CJK} $\mathcal{A}$ \begin{CJK}{UTF8}{mj}在基\end{CJK}
$$
\left\{\left(\begin{array}{ll}
1 & 0 \\
0 & 0
\end{array}\right),\left(\begin{array}{ll}
0 & 1 \\
0 & 0
\end{array}\right),\left(\begin{array}{ll}
0 & 0 \\
1 & 0
\end{array}\right),\left(\begin{array}{ll}
0 & 0 \\
0 & 1
\end{array}\right),\right\}
$$
\begin{CJK}{UTF8}{mj}下的表示矩阵\end{CJK};

(3) \begin{CJK}{UTF8}{mj}求\end{CJK} $\mathcal{A}$ \begin{CJK}{UTF8}{mj}的所有特征值和特征向量\end{CJK};

(4) \begin{CJK}{UTF8}{mj}求\end{CJK} $V$ \begin{CJK}{UTF8}{mj}的一组标准正交基\end{CJK}, \begin{CJK}{UTF8}{mj}使得\end{CJK} $\mathcal{A}$ \begin{CJK}{UTF8}{mj}在此基下的表示矩阵为对角阵\end{CJK}.

\begin{enumerate}
  \setcounter{enumi}{10}
  \item (10 \begin{CJK}{UTF8}{mj}分\end{CJK}) \begin{CJK}{UTF8}{mj}求齐次线性方程组\end{CJK}
\end{enumerate}
$$
\left\{\begin{array}{l}
x_{1}-2 x_{2}+3 x_{3}-4 x_{4}=0 \\
x_{1}+5 x_{2}+3 x_{3}+3 x_{4}=0
\end{array}\right.
$$
\begin{CJK}{UTF8}{mj}的解空间\end{CJK} $S$, \begin{CJK}{UTF8}{mj}并求\end{CJK} $S$ \begin{CJK}{UTF8}{mj}在\end{CJK} $\mathbb{R}^{4}$ \begin{CJK}{UTF8}{mj}中的正交补\end{CJK} $S^{\perp}$.

\section{1. 电子科技大学 2011 年研究生入学考试试题线性代数 
 李扬 
 微信公众号: sxkyliyang}
\begin{enumerate}
  \item (13 \begin{CJK}{UTF8}{mj}分\end{CJK}) \begin{CJK}{UTF8}{mj}计算行列式\end{CJK}
\end{enumerate}
$$
\left|\begin{array}{ccccc}
1 & 0 & \cdots & 0 & \beta_{1} \\
0 & 1 & \cdots & 0 & \beta_{2} \\
\vdots & \ddots & & \ddots & \vdots \\
0 & 0 & \cdots & 1 & \beta_{n} \\
\alpha_{1} & \alpha_{2} & \cdots & \alpha_{n} & 0
\end{array}\right|
$$

\begin{enumerate}
  \setcounter{enumi}{2}
  \item (15 \begin{CJK}{UTF8}{mj}分\end{CJK}) \begin{CJK}{UTF8}{mj}设\end{CJK}
\end{enumerate}
$$
A=\left(\begin{array}{lll}
1 & 0 & 0 \\
1 & 1 & 0 \\
1 & 1 & 1
\end{array}\right), B=\left(\begin{array}{lll}
0 & 1 & 1 \\
1 & 0 & 1 \\
1 & 1 & 0
\end{array}\right)
$$
\begin{CJK}{UTF8}{mj}且\end{CJK} $A X A+B X B=A X B+B X A+I$, \begin{CJK}{UTF8}{mj}其中\end{CJK} $I$ \begin{CJK}{UTF8}{mj}为单位矩阵\end{CJK}, \begin{CJK}{UTF8}{mj}求矩阵\end{CJK} $X$.

\begin{enumerate}
  \setcounter{enumi}{3}
  \item (11 \begin{CJK}{UTF8}{mj}分\end{CJK}) \begin{CJK}{UTF8}{mj}证明\end{CJK}: \begin{CJK}{UTF8}{mj}矩阵的非零子式所在的行向量组和列向量组都是线性无关的\end{CJK}.

  \item (12 \begin{CJK}{UTF8}{mj}分\end{CJK}) \begin{CJK}{UTF8}{mj}怎样的线性方程组给出\end{CJK} (\begin{CJK}{UTF8}{mj}产生\end{CJK})\begin{CJK}{UTF8}{mj}空间中三个没有公共点但两两相交的平面\end{CJK}? \begin{CJK}{UTF8}{mj}详细说明理由\end{CJK}.

  \item (15 \begin{CJK}{UTF8}{mj}分\end{CJK}) \begin{CJK}{UTF8}{mj}如果向量\end{CJK} $\beta=\left(b_{1}, b_{1}, \cdots, b_{n}\right)$ \begin{CJK}{UTF8}{mj}是线性方程组\end{CJK}

\end{enumerate}
$$
(I):\left\{\begin{array}{c}
a_{11} x_{1}+a_{12} x_{2}+\cdots+a_{1 n} x_{n}=0 ; \\
\cdots \\
a_{m 1} x_{1}+a_{m 2} x_{2}+\cdots+a_{m n} x_{n}=0 .
\end{array}\right.
$$
\begin{CJK}{UTF8}{mj}的系数矩阵的行向量组\end{CJK} $\alpha_{i}=\left(a_{i 1}, a_{i 2}, \cdots, a_{i n}\right)(i=1,2, \cdots, m)$ \begin{CJK}{UTF8}{mj}的线性组合\end{CJK}, \begin{CJK}{UTF8}{mj}则方程组\end{CJK} $(I)$ \begin{CJK}{UTF8}{mj}的解都是方程\end{CJK} $(I I): b_{1} x_{1}+b_{2}+x_{2}+\cdots+b_{n} x_{n}=0$ \begin{CJK}{UTF8}{mj}的解\end{CJK}.

\begin{enumerate}
  \setcounter{enumi}{6}
  \item ( 16 \begin{CJK}{UTF8}{mj}分\end{CJK}) \begin{CJK}{UTF8}{mj}已知\end{CJK} $\xi=\left(\begin{array}{c}1 \\ 1 \\ -1\end{array}\right)$ \begin{CJK}{UTF8}{mj}是矩阵\end{CJK} $A=\left(\begin{array}{ccc}2 & -1 & 2 \\ 5 & a & 3 \\ -1 & b & -2\end{array}\right)$, \begin{CJK}{UTF8}{mj}的一个特征向量\end{CJK}.
\end{enumerate}
(1) \begin{CJK}{UTF8}{mj}试确定参数\end{CJK} $a, b$ \begin{CJK}{UTF8}{mj}及特征向量\end{CJK} $\xi$ \begin{CJK}{UTF8}{mj}所对应的特征值\end{CJK};

(2) \begin{CJK}{UTF8}{mj}问\end{CJK} $A$ \begin{CJK}{UTF8}{mj}能否相似于对角矩阵\end{CJK}? \begin{CJK}{UTF8}{mj}说明理由\end{CJK}.

\begin{enumerate}
  \setcounter{enumi}{7}
  \item ( 15 \begin{CJK}{UTF8}{mj}分\end{CJK}) \begin{CJK}{UTF8}{mj}证明\end{CJK}: \begin{CJK}{UTF8}{mj}实二次型\end{CJK} $f\left(x_{1}, x_{2}, \cdots, x_{n}\right)=X^{T} A X$ \begin{CJK}{UTF8}{mj}在\end{CJK} $x_{1}^{2}+x_{2}^{2}+\cdots+x_{n}^{2}=1$ \begin{CJK}{UTF8}{mj}下的最大值恰为实对称矩阵\end{CJK} $A$ \begin{CJK}{UTF8}{mj}的最大特征值\end{CJK}.

  \item ( 20 \begin{CJK}{UTF8}{mj}分\end{CJK}) \begin{CJK}{UTF8}{mj}设\end{CJK} $P^{3}$ \begin{CJK}{UTF8}{mj}的线性变换\end{CJK} $\mathcal{A}(a, b, c)=(a+2 b-c, b+c, a+b-2 c)$.

\end{enumerate}
(1) \begin{CJK}{UTF8}{mj}求\end{CJK} $\mathcal{A}$ \begin{CJK}{UTF8}{mj}的值域\end{CJK} $\mathcal{A}\left(P^{3}\right)$ \begin{CJK}{UTF8}{mj}的维数和一组基\end{CJK};

(2) \begin{CJK}{UTF8}{mj}求\end{CJK} $\mathcal{A}$ \begin{CJK}{UTF8}{mj}的核\end{CJK} $\mathcal{A}^{-1}(0)$ \begin{CJK}{UTF8}{mj}的维数和一组基\end{CJK}.

\begin{enumerate}
  \setcounter{enumi}{9}
  \item ( 21 \begin{CJK}{UTF8}{mj}分\end{CJK}) \begin{CJK}{UTF8}{mj}已知欧氏空间\end{CJK} $R^{2 \times 2}$ \begin{CJK}{UTF8}{mj}的线性变换为\end{CJK}
\end{enumerate}
$$
\mathcal{A}\left(\begin{array}{ll}
x_{11} & x_{12} \\
x_{21} & x_{22}
\end{array}\right)=\left(\begin{array}{ll}
x_{12}+x_{21} & x_{11}+x_{22} \\
x_{11}+x_{22} & x_{12}+x_{21}
\end{array}\right), \forall\left(\begin{array}{ll}
x_{11} & x_{12} \\
x_{21} & x_{22}
\end{array}\right) \in R^{2 \times 2}
$$
(1) \begin{CJK}{UTF8}{mj}证明\end{CJK} $\mathcal{A}$ \begin{CJK}{UTF8}{mj}是对称变换\end{CJK};

(2) \begin{CJK}{UTF8}{mj}求\end{CJK} $R^{2 \times 2}$ \begin{CJK}{UTF8}{mj}的一组标准正交基\end{CJK}, \begin{CJK}{UTF8}{mj}使\end{CJK} $\mathcal{A}$ \begin{CJK}{UTF8}{mj}在该基下的矩阵为对角矩阵\end{CJK}.

\begin{enumerate}
  \setcounter{enumi}{10}
  \item (12 \begin{CJK}{UTF8}{mj}分\end{CJK}) \begin{CJK}{UTF8}{mj}证明\end{CJK}: $n$ \begin{CJK}{UTF8}{mj}为线性空间\end{CJK} $V$ \begin{CJK}{UTF8}{mj}的任一子空间\end{CJK} $W$ \begin{CJK}{UTF8}{mj}是某一线性变换\end{CJK} $\mathcal{A}$ \begin{CJK}{UTF8}{mj}的像集\end{CJK}.
\end{enumerate}
\section{2. 电子科技大学 2012 年研究生入学考试试题线性代数 
 李扬 
 微信公众号: sxkyliyang}
\begin{enumerate}
  \item (10 \begin{CJK}{UTF8}{mj}分\end{CJK}) \begin{CJK}{UTF8}{mj}计算四阶行列式\end{CJK}
\end{enumerate}
$$
D_{4}=\left|\begin{array}{llll}
0 & x & y & z \\
x & 0 & z & y \\
y & z & 0 & x \\
z & y & x & 0
\end{array}\right|
$$

\begin{enumerate}
  \setcounter{enumi}{2}
  \item ( 15 \begin{CJK}{UTF8}{mj}分\end{CJK}) \begin{CJK}{UTF8}{mj}设\end{CJK} 3 \begin{CJK}{UTF8}{mj}阶实矩阵\end{CJK} $A=\left(a_{i j}\right)_{3 \times 3}$ \begin{CJK}{UTF8}{mj}满足\end{CJK} $a_{i j}=A_{i j}(i, j=1,2,3)$, \begin{CJK}{UTF8}{mj}求行列式\end{CJK} $|A|$.

  \item ( 15 \begin{CJK}{UTF8}{mj}分\end{CJK}) \begin{CJK}{UTF8}{mj}设\end{CJK} $A$ \begin{CJK}{UTF8}{mj}是\end{CJK} $n$ \begin{CJK}{UTF8}{mj}阶实矩阵\end{CJK}, \begin{CJK}{UTF8}{mj}记\end{CJK} $A$ \begin{CJK}{UTF8}{mj}的秩为\end{CJK} $r(A)$. \begin{CJK}{UTF8}{mj}证明\end{CJK}: $r(A)=1$ \begin{CJK}{UTF8}{mj}的充要条件为存在非零向量\end{CJK} $\alpha=$ $\left(a_{1}, a_{2}, \cdots, a_{n}\right), \beta=\left(b_{1}, b_{2}, \cdots, b_{n}\right)$ \begin{CJK}{UTF8}{mj}使得\end{CJK} $A=\alpha^{T} \beta$.

  \item ( 15 \begin{CJK}{UTF8}{mj}分\end{CJK}) \begin{CJK}{UTF8}{mj}设\end{CJK} $n$ \begin{CJK}{UTF8}{mj}阶实矩阵\end{CJK} $A, B$ \begin{CJK}{UTF8}{mj}满足\end{CJK} $A+B+A B=O$.

\end{enumerate}
(1) \begin{CJK}{UTF8}{mj}证明\end{CJK}: $A+E$ \begin{CJK}{UTF8}{mj}可逆\end{CJK};

(2) \begin{CJK}{UTF8}{mj}证明\end{CJK}: $A B=B A$.

\begin{enumerate}
  \setcounter{enumi}{5}
  \item (15 \begin{CJK}{UTF8}{mj}分\end{CJK}) \begin{CJK}{UTF8}{mj}对于线性方程组\end{CJK}
\end{enumerate}
$$
\left\{\begin{array}{l}
\lambda x_{1}+x_{2}+x_{3}=\lambda-3 \\
x_{1}+\lambda x_{2}+x_{3}=-2 \\
x_{1}+x_{2}+\lambda x_{3}=-2
\end{array}\right.
$$
\begin{CJK}{UTF8}{mj}讨论\end{CJK} $\lambda$ \begin{CJK}{UTF8}{mj}取何值时\end{CJK}, \begin{CJK}{UTF8}{mj}方程组无解\end{CJK}, \begin{CJK}{UTF8}{mj}有唯一解和无穷多解\end{CJK}, \begin{CJK}{UTF8}{mj}在方程组有无穷多解时\end{CJK}, \begin{CJK}{UTF8}{mj}求出方程组的通解\end{CJK}.

\begin{enumerate}
  \setcounter{enumi}{6}
  \item (15 \begin{CJK}{UTF8}{mj}分\end{CJK}) \begin{CJK}{UTF8}{mj}设矩阵\end{CJK}
\end{enumerate}
$$
A=\left(\begin{array}{ccc}
1 & -1 & 1 \\
x & 4 & y \\
-3 & -3 & 5
\end{array}\right)
$$
\begin{CJK}{UTF8}{mj}已知\end{CJK} $A$ \begin{CJK}{UTF8}{mj}有三个线性无关的特征向量\end{CJK}, $\lambda=2$ \begin{CJK}{UTF8}{mj}是\end{CJK} $A$ \begin{CJK}{UTF8}{mj}的二重特征值\end{CJK}. \begin{CJK}{UTF8}{mj}试求可逆矩阵\end{CJK} $P$, \begin{CJK}{UTF8}{mj}使得\end{CJK} $P^{-1} A P$ \begin{CJK}{UTF8}{mj}为对角形矩\end{CJK} \begin{CJK}{UTF8}{mj}阵\end{CJK}.

\begin{enumerate}
  \setcounter{enumi}{7}
  \item ( 15 \begin{CJK}{UTF8}{mj}分\end{CJK}) \begin{CJK}{UTF8}{mj}设\end{CJK} $R^{n \times n}$ \begin{CJK}{UTF8}{mj}是实数域上所有\end{CJK} $n$ \begin{CJK}{UTF8}{mj}阶方阵关于方阵的加法和数乘所成线性空间\end{CJK}.
\end{enumerate}
$$
V=\left\{\left(a_{i j}\right)_{n \times n} \mid a_{i j}=a_{j i},(i, j=1,2, \cdots, n)\right\}
$$
\begin{CJK}{UTF8}{mj}是对称矩阵子空间\end{CJK},
$$
W=\left\{\left(a_{i j}\right)_{n \times n} \mid a_{i j}=0,(n \geqslant i>j \geqslant 1)\right\}
$$
\begin{CJK}{UTF8}{mj}是上三角矩阵子空间\end{CJK}. \begin{CJK}{UTF8}{mj}试分别写出\end{CJK} $V, V+W, V \cap W$ \begin{CJK}{UTF8}{mj}的一组基\end{CJK} (\begin{CJK}{UTF8}{mj}不用写求解过程\end{CJK}).

\begin{enumerate}
  \setcounter{enumi}{8}
  \item (15 \begin{CJK}{UTF8}{mj}分\end{CJK}) \begin{CJK}{UTF8}{mj}设\end{CJK} $V$ \begin{CJK}{UTF8}{mj}是\end{CJK} $n(\geqslant 2)$ \begin{CJK}{UTF8}{mj}维实线性空间\end{CJK}, $f_{1}, f_{2} \in V^{*}$ \begin{CJK}{UTF8}{mj}是\end{CJK} $V$ \begin{CJK}{UTF8}{mj}上的线性函数\end{CJK}, \begin{CJK}{UTF8}{mj}令\end{CJK} $V_{1}=\left\{\alpha \in V \mid f_{1}(\alpha)=0\right\}, V_{2}=$ $\left\{\alpha \in V \mid f_{2}(\alpha)=0\right\} .$
\end{enumerate}
(1) \begin{CJK}{UTF8}{mj}证明\end{CJK}: $V_{1}, V_{2}$ \begin{CJK}{UTF8}{mj}是\end{CJK} $V$ \begin{CJK}{UTF8}{mj}的子空间\end{CJK};

(2) \begin{CJK}{UTF8}{mj}证明\end{CJK}: \begin{CJK}{UTF8}{mj}存在\end{CJK} $\alpha \in V$, \begin{CJK}{UTF8}{mj}使得\end{CJK} $f_{1}(\alpha) f_{2}(\alpha) \neq 0$. 9. (25 \begin{CJK}{UTF8}{mj}分\end{CJK}) \begin{CJK}{UTF8}{mj}设\end{CJK} $A=\left(\begin{array}{ll}1 & 2 \\ 2 & 4\end{array}\right)$, \begin{CJK}{UTF8}{mj}规定\end{CJK} $\sigma_{A}: R^{2 \times 2} \rightarrow R^{2 \times 2}, B \mapsto A B, \forall B \in R^{2 \times 2}$.

(1) \begin{CJK}{UTF8}{mj}证明\end{CJK} $\sigma_{A}$ \begin{CJK}{UTF8}{mj}是线性变换\end{CJK};

(2) \begin{CJK}{UTF8}{mj}分别写出\end{CJK} $\sigma$ \begin{CJK}{UTF8}{mj}的核域\end{CJK} $\operatorname{ker} \sigma$ \begin{CJK}{UTF8}{mj}与值域\end{CJK} $\operatorname{Im} \sigma$ \begin{CJK}{UTF8}{mj}的一组基\end{CJK};

(3) \begin{CJK}{UTF8}{mj}写出\end{CJK} $\sigma_{A}$ \begin{CJK}{UTF8}{mj}在基\end{CJK}
$$
E_{11}=\left(\begin{array}{ll}
1 & 0 \\
0 & 0
\end{array}\right), E_{12}=\left(\begin{array}{ll}
0 & 1 \\
0 & 0
\end{array}\right), E_{21}=\left(\begin{array}{ll}
0 & 0 \\
1 & 0
\end{array}\right), E_{22}\left(\begin{array}{ll}
0 & 0 \\
0 & 1
\end{array}\right)
$$
\begin{CJK}{UTF8}{mj}下的矩阵\end{CJK};

(4) \begin{CJK}{UTF8}{mj}是否存在\end{CJK} $R^{2 \times 2}$ \begin{CJK}{UTF8}{mj}的一组基\end{CJK}, \begin{CJK}{UTF8}{mj}使得\end{CJK} $\sigma_{A}$ \begin{CJK}{UTF8}{mj}在该组基下的矩阵是对角矩阵\end{CJK}? \begin{CJK}{UTF8}{mj}为什么\end{CJK}?

\begin{enumerate}
  \setcounter{enumi}{10}
  \item ( 10 \begin{CJK}{UTF8}{mj}分\end{CJK}) \begin{CJK}{UTF8}{mj}设\end{CJK} $A$ \begin{CJK}{UTF8}{mj}为\end{CJK} $n$ \begin{CJK}{UTF8}{mj}阶对称正定矩阵\end{CJK}, \begin{CJK}{UTF8}{mj}证明\end{CJK}: \begin{CJK}{UTF8}{mj}存在对称正定矩阵\end{CJK} $S$ \begin{CJK}{UTF8}{mj}使得\end{CJK} $A=S^{2012}$.
\end{enumerate}
\section{3. 电子科技大学 2013 年研究生入学考试试题线性代数 
 李扬 
 微信公众号: sxkyliyang}
\begin{enumerate}
  \item (13 \begin{CJK}{UTF8}{mj}分\end{CJK}) \begin{CJK}{UTF8}{mj}设\end{CJK} $f(x)=c_{0}+c_{1} x_{1}+\cdots+c_{n} x^{n}$, \begin{CJK}{UTF8}{mj}证明\end{CJK}: \begin{CJK}{UTF8}{mj}若\end{CJK} $f(x)$ \begin{CJK}{UTF8}{mj}有\end{CJK} $n+1$ \begin{CJK}{UTF8}{mj}个不同的根\end{CJK}, \begin{CJK}{UTF8}{mj}则\end{CJK} $f(x)$ \begin{CJK}{UTF8}{mj}是零多项式\end{CJK}.

  \item (15 \begin{CJK}{UTF8}{mj}分\end{CJK}) \begin{CJK}{UTF8}{mj}设\end{CJK} $A, B$ \begin{CJK}{UTF8}{mj}是\end{CJK} $n$ \begin{CJK}{UTF8}{mj}阶矩阵\end{CJK}, \begin{CJK}{UTF8}{mj}证明行列式公式\end{CJK}:

\end{enumerate}
$$
\left|\begin{array}{ll}
A & B \\
B & A
\end{array}\right|=|A+B \| A-B|
$$

\begin{enumerate}
  \setcounter{enumi}{3}
  \item (14 \begin{CJK}{UTF8}{mj}分\end{CJK}) \begin{CJK}{UTF8}{mj}设\end{CJK} $A=\left(a_{i j}\right)_{3 \times 3}$ \begin{CJK}{UTF8}{mj}满足\end{CJK} $a_{i j}=A_{i j}, i, j=1,2,3 ; a_{33}=-1$, \begin{CJK}{UTF8}{mj}求行列式\end{CJK} $|A|$, \begin{CJK}{UTF8}{mj}并求方程组\end{CJK}
\end{enumerate}
$$
A\left(\begin{array}{l}
x_{1} \\
x_{2} \\
x_{3}
\end{array}\right)=\left(\begin{array}{l}
0 \\
0 \\
1
\end{array}\right)
$$
\begin{CJK}{UTF8}{mj}的解\end{CJK}, \begin{CJK}{UTF8}{mj}这里\end{CJK} $A_{i j}$ \begin{CJK}{UTF8}{mj}为代数余子式\end{CJK}.

\begin{enumerate}
  \setcounter{enumi}{4}
  \item (15 \begin{CJK}{UTF8}{mj}分\end{CJK}) \begin{CJK}{UTF8}{mj}如下说法是否正确\end{CJK}, \begin{CJK}{UTF8}{mj}请说明理由\end{CJK}.
\end{enumerate}
"\begin{CJK}{UTF8}{mj}设向量\end{CJK} $\beta$ \begin{CJK}{UTF8}{mj}可由向量组\end{CJK} $\alpha_{1}, \alpha_{2}, \cdots, \alpha_{n}$ \begin{CJK}{UTF8}{mj}线性表出\end{CJK}, \begin{CJK}{UTF8}{mj}则表示唯一是充分必要条件为\end{CJK} $\alpha_{1}, \alpha_{2}, \cdots, \alpha_{n}$ \begin{CJK}{UTF8}{mj}线性无关\end{CJK}."

\begin{enumerate}
  \setcounter{enumi}{5}
  \item (15 \begin{CJK}{UTF8}{mj}分\end{CJK}) \begin{CJK}{UTF8}{mj}已知下列方程组\end{CJK}:
\end{enumerate}
$$
\begin{aligned}
&\left\{\begin{array}{l}
x_{1}+x_{2}-2 x_{4}=-6 \\
4 x_{1}-x_{2}-x_{3}-x_{4}=1 \\
3 x_{1}-x_{2}-x_{3}=3
\end{array}\right. \\
&\left\{\begin{array}{l}
x_{1}+m x_{2}-x_{3}-x_{4}=-5 \\
n x_{2}-x_{3}-2 x_{4}=-11 \\
x_{3}-2 x_{4}=-t+1
\end{array}\right.
\end{aligned}
$$
(1) \begin{CJK}{UTF8}{mj}求解方程组\end{CJK} $(I)$ \begin{CJK}{UTF8}{mj}的通解\end{CJK};

(2) \begin{CJK}{UTF8}{mj}当\end{CJK} $m, n, t$ \begin{CJK}{UTF8}{mj}为何值时\end{CJK}, \begin{CJK}{UTF8}{mj}方程组\end{CJK} $(I)$ \begin{CJK}{UTF8}{mj}与\end{CJK} $(I I)$ \begin{CJK}{UTF8}{mj}同解\end{CJK}.

\begin{enumerate}
  \setcounter{enumi}{6}
  \item (17 \begin{CJK}{UTF8}{mj}分\end{CJK}) \begin{CJK}{UTF8}{mj}已知二次型\end{CJK} $f\left(x_{1}, x_{2}, x_{3}\right)=(1-a) x_{1}^{2}+(1-a) x_{2}^{2}+2 x_{3}^{2}+2(1+a) x_{1} x_{2}$ \begin{CJK}{UTF8}{mj}的秩为\end{CJK} 2 .
\end{enumerate}
(1) \begin{CJK}{UTF8}{mj}求参数\end{CJK} $a$;

(2) \begin{CJK}{UTF8}{mj}正交变换\end{CJK} $X=C Y$, \begin{CJK}{UTF8}{mj}把\end{CJK} $f\left(x_{1}, x_{2}, x_{3}\right)$ \begin{CJK}{UTF8}{mj}化为标准形\end{CJK};

(3) \begin{CJK}{UTF8}{mj}求方程\end{CJK} $f\left(x_{1}, x_{2}, x_{3}\right)=0$ \begin{CJK}{UTF8}{mj}的解\end{CJK}.

\begin{enumerate}
  \setcounter{enumi}{7}
  \item ( 16 \begin{CJK}{UTF8}{mj}分\end{CJK}) \begin{CJK}{UTF8}{mj}设\end{CJK} $A, B$ \begin{CJK}{UTF8}{mj}都是正定矩阵\end{CJK}, \begin{CJK}{UTF8}{mj}证明\end{CJK}:
\end{enumerate}
(1) \begin{CJK}{UTF8}{mj}方程\end{CJK} $|\lambda A-B|=0$ \begin{CJK}{UTF8}{mj}的根都大于零\end{CJK};

(2) \begin{CJK}{UTF8}{mj}方程\end{CJK} $|\lambda A-B|=0$ \begin{CJK}{UTF8}{mj}的根都等于\end{CJK} 1 \begin{CJK}{UTF8}{mj}当且仅当\end{CJK} $A=B$.

\begin{enumerate}
  \setcounter{enumi}{8}
  \item ( 16 \begin{CJK}{UTF8}{mj}分\end{CJK}) \begin{CJK}{UTF8}{mj}设\end{CJK} $\varepsilon_{1}, \cdots, \varepsilon_{4}$ \begin{CJK}{UTF8}{mj}是\end{CJK} 4 \begin{CJK}{UTF8}{mj}维线性空间\end{CJK} $V$ \begin{CJK}{UTF8}{mj}的一组基\end{CJK}, \begin{CJK}{UTF8}{mj}已知线性变换\end{CJK} $\mathcal{A}$ \begin{CJK}{UTF8}{mj}在这组基下的矩阵为\end{CJK}
\end{enumerate}
$$
\left(\begin{array}{cccc}
1 & 0 & 2 & 1 \\
-1 & 2 & 1 & 3 \\
1 & 2 & 5 & 5 \\
2 & -2 & 1 & -2
\end{array}\right)
$$
\begin{CJK}{UTF8}{mj}求\end{CJK} $\mathcal{A}$ \begin{CJK}{UTF8}{mj}的值域和核\end{CJK}.

\begin{enumerate}
  \setcounter{enumi}{9}
  \item ( 16 \begin{CJK}{UTF8}{mj}分\end{CJK}) \begin{CJK}{UTF8}{mj}设\end{CJK} $V$ \begin{CJK}{UTF8}{mj}是欧几里德空间\end{CJK}, $U$ \begin{CJK}{UTF8}{mj}是\end{CJK} $V$ \begin{CJK}{UTF8}{mj}的子空间\end{CJK}, $\beta \in U$. \begin{CJK}{UTF8}{mj}证明\end{CJK}: $\beta$ \begin{CJK}{UTF8}{mj}是\end{CJK} $\alpha \in V$ \begin{CJK}{UTF8}{mj}在\end{CJK} $U$ \begin{CJK}{UTF8}{mj}上的正交投影的充分必要\end{CJK} \begin{CJK}{UTF8}{mj}条件为\end{CJK}: $\forall \gamma \in U$, \begin{CJK}{UTF8}{mj}都有\end{CJK} $\|\alpha-\beta\| \leqslant\|\alpha-\gamma\|$.

  \item ( 16 \begin{CJK}{UTF8}{mj}分\end{CJK}) \begin{CJK}{UTF8}{mj}设\end{CJK} $V$ \begin{CJK}{UTF8}{mj}是\end{CJK} $n$ \begin{CJK}{UTF8}{mj}维线性空间\end{CJK}, $\mathcal{A}$ \begin{CJK}{UTF8}{mj}是\end{CJK} $V$ \begin{CJK}{UTF8}{mj}的一个线性变换\end{CJK}, \begin{CJK}{UTF8}{mj}满足\end{CJK} $A^{2}=A$, \begin{CJK}{UTF8}{mj}令\end{CJK}

\end{enumerate}
$$
V_{1}=\{\mathcal{A}(\alpha) \mid \alpha \in V\}, V_{2}=\{\alpha-\mathcal{A}(\alpha) \mid \alpha \in V\}
$$
\begin{CJK}{UTF8}{mj}证明\end{CJK}: $V=V_{1}+V_{2}$.

\section{4. 电子科技大学 2014 年研究生入学考试试题线性代数 
 李扬 
 微信公众号: sxkyliyang}
\begin{enumerate}
  \item (15 \begin{CJK}{UTF8}{mj}分\end{CJK}) \begin{CJK}{UTF8}{mj}计算五阶行列式\end{CJK}
\end{enumerate}
$$
D_{5}=\left|\begin{array}{lllll}
5 & 3 & 0 & 0 & 0 \\
2 & 5 & 3 & 0 & 0 \\
0 & 2 & 5 & 3 & 0 \\
0 & 0 & 2 & 5 & 3 \\
0 & 0 & 0 & 2 & 5
\end{array}\right| .
$$

\begin{enumerate}
  \setcounter{enumi}{2}
  \item (15 \begin{CJK}{UTF8}{mj}分\end{CJK}) \begin{CJK}{UTF8}{mj}设\end{CJK}
\end{enumerate}
$$
A=\left(\begin{array}{ccc}
2 & -1 & 3 \\
a & b & -3 \\
4 & -2 & c
\end{array}\right)
$$
\begin{CJK}{UTF8}{mj}如果\end{CJK} 3 \begin{CJK}{UTF8}{mj}阶矩阵\end{CJK} $B$ \begin{CJK}{UTF8}{mj}的伴随矩阵\end{CJK} $B^{*} \neq O$ \begin{CJK}{UTF8}{mj}且\end{CJK} $A B=O$, \begin{CJK}{UTF8}{mj}求\end{CJK} $A$.

\begin{enumerate}
  \setcounter{enumi}{3}
  \item (15 \begin{CJK}{UTF8}{mj}分\end{CJK}) \begin{CJK}{UTF8}{mj}已知\end{CJK} 4 \begin{CJK}{UTF8}{mj}阶方阵\end{CJK} $A=\left(\alpha_{1}, \alpha_{2}, \alpha_{3}, \alpha_{4}\right)$, \begin{CJK}{UTF8}{mj}线性方程组\end{CJK} $A X=B$ \begin{CJK}{UTF8}{mj}的通解为\end{CJK} $\left(\begin{array}{c}2 \\ 1 \\ 0 \\ 3\end{array}\right)+k\left(\begin{array}{c}1 \\ -1 \\ 2 \\ 0\end{array}\right)$. \begin{CJK}{UTF8}{mj}其中\end{CJK} $k$ \begin{CJK}{UTF8}{mj}为\end{CJK} \begin{CJK}{UTF8}{mj}任意常数\end{CJK}. \begin{CJK}{UTF8}{mj}试问\end{CJK}: $\alpha_{4}$ \begin{CJK}{UTF8}{mj}能否由\end{CJK} $\alpha_{1}, \alpha_{2}, \alpha_{3}$ \begin{CJK}{UTF8}{mj}线性表出\end{CJK}? \begin{CJK}{UTF8}{mj}为什么\end{CJK}?

  \item ( 15 \begin{CJK}{UTF8}{mj}分\end{CJK}) \begin{CJK}{UTF8}{mj}设\end{CJK} $A$ \begin{CJK}{UTF8}{mj}是\end{CJK} 3 \begin{CJK}{UTF8}{mj}阶方阵\end{CJK}, 3 \begin{CJK}{UTF8}{mj}维列向量组\end{CJK} $\alpha, A \alpha, A^{2} \alpha$ \begin{CJK}{UTF8}{mj}线性无关\end{CJK}, \begin{CJK}{UTF8}{mj}且\end{CJK} $A^{3} \alpha=3 A \alpha-2 A^{2} \alpha$, \begin{CJK}{UTF8}{mj}证明矩阵\end{CJK} $B=$ $\left(\alpha, A \alpha, A^{4} \alpha\right)$ \begin{CJK}{UTF8}{mj}可逆\end{CJK}.

  \item ( 20 \begin{CJK}{UTF8}{mj}分\end{CJK}) \begin{CJK}{UTF8}{mj}设\end{CJK} $\mathbb{R}^{3 \times 3}$ \begin{CJK}{UTF8}{mj}是实数域上所有\end{CJK} $n$ \begin{CJK}{UTF8}{mj}阶方阵关于方阵的加法和数乘所成线性空间\end{CJK},

\end{enumerate}
$$
V=\left\{\left(\begin{array}{lll}
a_{1} & a_{2} & a_{3} \\
a_{3} & a_{1} & a_{2} \\
a_{2} & a_{3} & a_{1}
\end{array}\right) \mid a_{1}, a_{2}, a_{3} \in \mathbb{R}\right\}, W=\left\{A \in \mathbb{R}^{3 \times 3} \mid A^{T}=-A\right\} .
$$
(1) \begin{CJK}{UTF8}{mj}分别写出\end{CJK} $V$ \begin{CJK}{UTF8}{mj}和\end{CJK} $W$ \begin{CJK}{UTF8}{mj}的维数和一组基\end{CJK} (\begin{CJK}{UTF8}{mj}不用写求解过程\end{CJK}).

(2) \begin{CJK}{UTF8}{mj}求\end{CJK} $V \cap W$ \begin{CJK}{UTF8}{mj}的一组基\end{CJK}.

\begin{enumerate}
  \setcounter{enumi}{6}
  \item ( 20 \begin{CJK}{UTF8}{mj}分\end{CJK}) \begin{CJK}{UTF8}{mj}设\end{CJK} $A=\left(\begin{array}{ccc}1 & -1 & -a \\ 2 & a & -2 \\ -a & -1 & 1\end{array}\right)$.
\end{enumerate}
(1) \begin{CJK}{UTF8}{mj}求\end{CJK} $A$ \begin{CJK}{UTF8}{mj}的特征值与特征向量\end{CJK};

(2) \begin{CJK}{UTF8}{mj}讨论\end{CJK} $A$ \begin{CJK}{UTF8}{mj}何时可以相似对角化\end{CJK}.

\begin{enumerate}
  \setcounter{enumi}{7}
  \item (15 \begin{CJK}{UTF8}{mj}分\end{CJK}) \begin{CJK}{UTF8}{mj}设\end{CJK} $V$ \begin{CJK}{UTF8}{mj}是数域\end{CJK} $F$ \begin{CJK}{UTF8}{mj}上的\end{CJK} $n$ \begin{CJK}{UTF8}{mj}维线性空间\end{CJK}, $\mathcal{A}$ \begin{CJK}{UTF8}{mj}是\end{CJK} $V$ \begin{CJK}{UTF8}{mj}上的线性变换\end{CJK}, \begin{CJK}{UTF8}{mj}证明\end{CJK}: $\mathcal{A}$ \begin{CJK}{UTF8}{mj}在\end{CJK} $V$ \begin{CJK}{UTF8}{mj}的两组不同基下的矩阵是\end{CJK} \begin{CJK}{UTF8}{mj}相似的\end{CJK}.

  \item ( 15 \begin{CJK}{UTF8}{mj}分\end{CJK}) \begin{CJK}{UTF8}{mj}设\end{CJK} $V=\operatorname{span}(\alpha, \beta)$ \begin{CJK}{UTF8}{mj}是欧氏空间\end{CJK} $\mathbb{R}^{4}$ (\begin{CJK}{UTF8}{mj}关于标准内积\end{CJK})\begin{CJK}{UTF8}{mj}的子空间\end{CJK}, \begin{CJK}{UTF8}{mj}其中\end{CJK}

\end{enumerate}
$$
\alpha=(1,2,1,1), \beta=(-2,0,0,1) .
$$
(1) \begin{CJK}{UTF8}{mj}求正交补\end{CJK} $V^{\perp}$ \begin{CJK}{UTF8}{mj}的一组标准正交基\end{CJK};

(2) \begin{CJK}{UTF8}{mj}证明\end{CJK}: \begin{CJK}{UTF8}{mj}不存在正交变换\end{CJK} $\mathcal{A}$ \begin{CJK}{UTF8}{mj}使得\end{CJK} $\mathcal{A} \alpha=\beta$. 9. ( 20 \begin{CJK}{UTF8}{mj}分\end{CJK}) \begin{CJK}{UTF8}{mj}设\end{CJK} 3 \begin{CJK}{UTF8}{mj}元实二次型\end{CJK} $f(X)=X^{T} A X$ \begin{CJK}{UTF8}{mj}的秩为\end{CJK} 2 , \begin{CJK}{UTF8}{mj}且满足条件\end{CJK} $A^{2}+2 A=O$.

(1) $f(X)+1=0$ \begin{CJK}{UTF8}{mj}表示什么二次曲面\end{CJK}? \begin{CJK}{UTF8}{mj}为什么\end{CJK}?

(2) \begin{CJK}{UTF8}{mj}当\end{CJK} $k$ \begin{CJK}{UTF8}{mj}满足什么条件时\end{CJK}, \begin{CJK}{UTF8}{mj}二次型\end{CJK} $f(X)+k\left(x_{1}^{2}+x_{2}^{2}+x_{3}^{2}\right)$ \begin{CJK}{UTF8}{mj}正定\end{CJK}? 15. \begin{CJK}{UTF8}{mj}电子科技大学\end{CJK} 2015 \begin{CJK}{UTF8}{mj}年研究生入学考试试题线性代数\end{CJK}

\begin{CJK}{UTF8}{mj}李扬\end{CJK}

\begin{CJK}{UTF8}{mj}微信公众号\end{CJK}: sxkyliyang

\begin{enumerate}
  \item ( 10 \begin{CJK}{UTF8}{mj}分\end{CJK}) \begin{CJK}{UTF8}{mj}求\end{CJK} $\left|\begin{array}{llll}x & 0 & 1 & 1 \\ 0 & x & 1 & 1 \\ 1 & 1 & x & 0 \\ 1 & 1 & 0 & x\end{array}\right|=0$. \begin{CJK}{UTF8}{mj}的根\end{CJK}.

  \item (20 \begin{CJK}{UTF8}{mj}分\end{CJK}) (\begin{CJK}{UTF8}{mj}不写计算过程\end{CJK}) \begin{CJK}{UTF8}{mj}试写出\end{CJK} 4 \begin{CJK}{UTF8}{mj}个实矩阵\end{CJK} $A, B, C, D$ \begin{CJK}{UTF8}{mj}使得\end{CJK}

\end{enumerate}
(1) $A^{2}=\left(\begin{array}{cc}-1 & 0 \\ 0 & -1\end{array}\right)$;

(2) $B^{2}-2 B+\left(\begin{array}{ll}5 & 0 \\ 0 & 5\end{array}\right)=O$;

(3) $C^{*}=\left(\begin{array}{ll}1 & 2 \\ 3 & 4\end{array}\right)$;

(4) $D\left(\begin{array}{ccc}a_{11} & a_{12} & a_{13} \\ a_{21} & a_{22} & a_{23} \\ a_{31} & a_{32} & a_{33}\end{array}\right)=\left(\begin{array}{ccc}a_{31} & a_{32} & a_{33} \\ a_{11} & a_{12} & a_{13} \\ a_{21}-a_{11} & a_{22}-a_{12} & a_{23}-a_{13}\end{array}\right)$.

\begin{enumerate}
  \setcounter{enumi}{3}
  \item (15 \begin{CJK}{UTF8}{mj}分\end{CJK}) \begin{CJK}{UTF8}{mj}设\end{CJK} $\mathbb{R}^{2 \times 2}$ \begin{CJK}{UTF8}{mj}是全体\end{CJK} 2 \begin{CJK}{UTF8}{mj}阶实矩阵所构成的线性空间\end{CJK}, \begin{CJK}{UTF8}{mj}问\end{CJK} $a$ \begin{CJK}{UTF8}{mj}满足什么条件时\end{CJK},
\end{enumerate}
$$
A_{1}=\left(\begin{array}{cc}
a & 1 \\
1 & 1
\end{array}\right), A_{2}=\left(\begin{array}{ll}
1 & a \\
1 & 1
\end{array}\right), A_{3}=\left(\begin{array}{ll}
1 & 1 \\
a & 1
\end{array}\right), A_{1}=\left(\begin{array}{ll}
1 & 1 \\
1 & a
\end{array}\right)
$$
\begin{CJK}{UTF8}{mj}是\end{CJK} $\mathbb{R}^{2 \times 2}$ \begin{CJK}{UTF8}{mj}的一组基\end{CJK}.

\begin{enumerate}
  \setcounter{enumi}{4}
  \item ( 15 \begin{CJK}{UTF8}{mj}分\end{CJK}) \begin{CJK}{UTF8}{mj}已知矩阵\end{CJK} $A=\left(\begin{array}{ccc}1 & 1 & 1 \\ 0 & 1 & -1 \\ 2 & 3 & a \\ 3 & 5 & 1\end{array}\right)$ \begin{CJK}{UTF8}{mj}与矩阵\end{CJK} $B=\left(\begin{array}{ccc}1 & 1 & 1 \\ 0 & 1 & 0 \\ 2 & a & 3 \\ a-1 & 5 & 1\end{array}\right)$ \begin{CJK}{UTF8}{mj}等价\end{CJK}, \begin{CJK}{UTF8}{mj}试问\end{CJK} $a$ \begin{CJK}{UTF8}{mj}的取值范围\end{CJK}.

  \item ( 20 \begin{CJK}{UTF8}{mj}分\end{CJK}) \begin{CJK}{UTF8}{mj}设\end{CJK} $A=\left(\begin{array}{lll}1 & 2 & 1 \\ 0 & 1 & a \\ 1 & a & 0\end{array}\right), B$ \begin{CJK}{UTF8}{mj}是\end{CJK} 3 \begin{CJK}{UTF8}{mj}阶非零矩阵且满足\end{CJK} $B A=O$. \begin{CJK}{UTF8}{mj}如果矩阵\end{CJK} $B$ \begin{CJK}{UTF8}{mj}的第\end{CJK} 1 \begin{CJK}{UTF8}{mj}列是\end{CJK} $(1,2,-3)^{T}$, \begin{CJK}{UTF8}{mj}求\end{CJK} \begin{CJK}{UTF8}{mj}矩阵\end{CJK} $B$.

  \item (15 \begin{CJK}{UTF8}{mj}分\end{CJK}) \begin{CJK}{UTF8}{mj}已知矩阵\end{CJK} $A=\left(\begin{array}{ccc}0 & 0 & 1 \\ 0 & 1 & 0 \\ 1 & 0 & 0\end{array}\right)$ \begin{CJK}{UTF8}{mj}和\end{CJK} $B=\left(\begin{array}{ccc}2 & 0 & 0 \\ 0 & 1 & 0 \\ 0 & 0 & -2\end{array}\right)$,

\end{enumerate}
(1) \begin{CJK}{UTF8}{mj}求可逆矩阵\end{CJK} $C$ \begin{CJK}{UTF8}{mj}使得\end{CJK} $C^{T} A C=B$.

(2) \begin{CJK}{UTF8}{mj}如果\end{CJK} $A+k I$ \begin{CJK}{UTF8}{mj}与\end{CJK} $B$ \begin{CJK}{UTF8}{mj}合同\end{CJK}, \begin{CJK}{UTF8}{mj}求\end{CJK} $k$ \begin{CJK}{UTF8}{mj}的取值范围\end{CJK}, \begin{CJK}{UTF8}{mj}这里\end{CJK} $I$ \begin{CJK}{UTF8}{mj}是\end{CJK} 3 \begin{CJK}{UTF8}{mj}阶单位矩阵\end{CJK}.

\begin{enumerate}
  \setcounter{enumi}{7}
  \item ( 20 \begin{CJK}{UTF8}{mj}分\end{CJK}) \begin{CJK}{UTF8}{mj}设\end{CJK} $n$ \begin{CJK}{UTF8}{mj}阶实矩阵\end{CJK} $A$ \begin{CJK}{UTF8}{mj}满足\end{CJK} $A^{2}-3 A+2 I=O$.
\end{enumerate}
(1) \begin{CJK}{UTF8}{mj}证明\end{CJK} $A$ \begin{CJK}{UTF8}{mj}的特征值均大于\end{CJK} 0 ;

(2) \begin{CJK}{UTF8}{mj}是否存在可逆实矩阵\end{CJK} $P$ \begin{CJK}{UTF8}{mj}使得\end{CJK} $P^{-1} A P$ \begin{CJK}{UTF8}{mj}为对角阵\end{CJK}? \begin{CJK}{UTF8}{mj}为什么\end{CJK}?

\begin{enumerate}
  \setcounter{enumi}{8}
  \item (15 \begin{CJK}{UTF8}{mj}分\end{CJK}) \begin{CJK}{UTF8}{mj}设\end{CJK} $A$ \begin{CJK}{UTF8}{mj}是欧氏空间\end{CJK} $\mathbb{R}^{n}$ \begin{CJK}{UTF8}{mj}上的正交变换\end{CJK}, $\alpha \in \mathbb{R}^{n}$ \begin{CJK}{UTF8}{mj}是\end{CJK} $A$ \begin{CJK}{UTF8}{mj}的某个特征值\end{CJK} $\lambda$ \begin{CJK}{UTF8}{mj}的特征向量\end{CJK}, \begin{CJK}{UTF8}{mj}证明\end{CJK}: $\lambda=1$ \begin{CJK}{UTF8}{mj}或\end{CJK} $-1$.

  \item (15 \begin{CJK}{UTF8}{mj}分\end{CJK}) \begin{CJK}{UTF8}{mj}设\end{CJK} $A, B$ \begin{CJK}{UTF8}{mj}都是\end{CJK} $n$ \begin{CJK}{UTF8}{mj}阶非零实方阵\end{CJK}. \begin{CJK}{UTF8}{mj}证明存在列向量\end{CJK} $\alpha \in \mathbb{R}^{n}$ \begin{CJK}{UTF8}{mj}使得\end{CJK} $A \alpha, B \alpha$ \begin{CJK}{UTF8}{mj}都不是零向量\end{CJK}.

\end{enumerate}
\section{6. 电子科技大学 2016 年研究生入学考试试题线性代数 
 李扬 
 微信公众号: sxkyliyang}
\begin{enumerate}
  \item ( 15 \begin{CJK}{UTF8}{mj}分\end{CJK}) \begin{CJK}{UTF8}{mj}已知\end{CJK} 3 \begin{CJK}{UTF8}{mj}阶矩阵\end{CJK} $A=\left(\alpha_{1}, \alpha_{2}, \beta_{1}\right), B=\left(\alpha_{2}, \alpha_{1}, \beta_{2}\right)$, \begin{CJK}{UTF8}{mj}其中\end{CJK} $\alpha_{1}, \alpha_{2}, \beta_{1}, \beta_{2}$ \begin{CJK}{UTF8}{mj}都是\end{CJK} 3 \begin{CJK}{UTF8}{mj}维列向量\end{CJK}. \begin{CJK}{UTF8}{mj}若\end{CJK} $|A|=$ $4,|B|=5$, \begin{CJK}{UTF8}{mj}求\end{CJK} $|3 A-2 B|$.

  \item ( 20 \begin{CJK}{UTF8}{mj}分\end{CJK}) \begin{CJK}{UTF8}{mj}是否存在满足如下条件的矩阵\end{CJK}? \begin{CJK}{UTF8}{mj}如果有\end{CJK}, \begin{CJK}{UTF8}{mj}请写出一个或一对这样的矩阵\end{CJK} (\begin{CJK}{UTF8}{mj}不必说明理由\end{CJK}). \begin{CJK}{UTF8}{mj}如果没有\end{CJK}, \begin{CJK}{UTF8}{mj}请说明理由\end{CJK}.

\end{enumerate}
(1) \begin{CJK}{UTF8}{mj}两个秩为\end{CJK} 2 \begin{CJK}{UTF8}{mj}的矩阵\end{CJK} $A_{4 \times 3}$ \begin{CJK}{UTF8}{mj}与\end{CJK} $B_{3 \times 4}$ \begin{CJK}{UTF8}{mj}使得\end{CJK} $A B=O$;

(2) 3 \begin{CJK}{UTF8}{mj}阶矩阵\end{CJK} $C$ \begin{CJK}{UTF8}{mj}使得\end{CJK} $C^{3} \neq O$, \begin{CJK}{UTF8}{mj}但是\end{CJK} $C^{4}=O$;

(3) 2 \begin{CJK}{UTF8}{mj}阶正交矩阵\end{CJK} $F$ \begin{CJK}{UTF8}{mj}和\end{CJK} $G$ \begin{CJK}{UTF8}{mj}使得\end{CJK} $F+G$ \begin{CJK}{UTF8}{mj}也是正交矩阵\end{CJK};

(4) 2 \begin{CJK}{UTF8}{mj}阶矩阵\end{CJK} $U, W$ \begin{CJK}{UTF8}{mj}使得\end{CJK} $U W-W U=I$.

\begin{enumerate}
  \setcounter{enumi}{3}
  \item ( 20 \begin{CJK}{UTF8}{mj}分\end{CJK}) \begin{CJK}{UTF8}{mj}设二阶矩阵\end{CJK} $A, B$ \begin{CJK}{UTF8}{mj}满足\end{CJK} $A B=3 A+2 B$.
\end{enumerate}
(1) \begin{CJK}{UTF8}{mj}证明\end{CJK}: $A B=B A$;

(2) \begin{CJK}{UTF8}{mj}设\end{CJK} $A^{*}=\left(\begin{array}{ll}1 & 2 \\ 3 & 4\end{array}\right)$, \begin{CJK}{UTF8}{mj}求\end{CJK} $B$.

\begin{enumerate}
  \setcounter{enumi}{4}
  \item (20 \begin{CJK}{UTF8}{mj}分\end{CJK}) \begin{CJK}{UTF8}{mj}设\end{CJK} $A=\left(\begin{array}{ll}1 & 2 \\ 3 & 4\end{array}\right)$, \begin{CJK}{UTF8}{mj}规定\end{CJK} 2 \begin{CJK}{UTF8}{mj}阶实矩阵线性空间\end{CJK} $\mathbb{R}^{2 \times 2}$ \begin{CJK}{UTF8}{mj}上的线性变换\end{CJK} $\sigma_{A}$ \begin{CJK}{UTF8}{mj}为\end{CJK}:
\end{enumerate}
$$
\sigma_{A}: \mathbb{R}^{2 \times 2} \rightarrow \mathbb{R}^{2 \times 2}, B \mapsto A B+B A, \forall B \in \mathbb{R}^{2 \times 2}
$$
(1) \begin{CJK}{UTF8}{mj}试计算线性变换\end{CJK} $\sigma_{A}$ \begin{CJK}{UTF8}{mj}在\end{CJK} $\mathbb{R}^{2 \times 2}$ \begin{CJK}{UTF8}{mj}的标准基\end{CJK} $\left(\begin{array}{ll}1 & 0 \\ 0 & 0\end{array}\right),\left(\begin{array}{ll}0 & 1 \\ 0 & 0\end{array}\right),\left(\begin{array}{ll}0 & 0 \\ 1 & 0\end{array}\right),\left(\begin{array}{ll}0 & 0 \\ 0 & 1\end{array}\right)$ \begin{CJK}{UTF8}{mj}下的矩阵\end{CJK}.

(2) \begin{CJK}{UTF8}{mj}写出线性变换\end{CJK} $\sigma_{A}$ \begin{CJK}{UTF8}{mj}的像空间\end{CJK} $\operatorname{Im} \sigma_{A}$ \begin{CJK}{UTF8}{mj}与空间\end{CJK} $\operatorname{ker} \sigma_{A}$.

\begin{enumerate}
  \setcounter{enumi}{5}
  \item (15 \begin{CJK}{UTF8}{mj}分\end{CJK}) \begin{CJK}{UTF8}{mj}已知非齐次线性方程组\end{CJK}
\end{enumerate}
$$
\left\{\begin{array}{l}
x_{1}+2 x_{2}+x_{3}=3 \\
2 x_{1}+(a+4) x_{2}-5 x_{3}=6 \\
-x_{1}-2 x_{2}+x_{3}=-3 .
\end{array}\right.
$$
\begin{CJK}{UTF8}{mj}有\end{CJK} 3 \begin{CJK}{UTF8}{mj}个线性无关的解\end{CJK}, \begin{CJK}{UTF8}{mj}求\end{CJK} $a$ \begin{CJK}{UTF8}{mj}的值以及原方程组的通解\end{CJK}.

\begin{enumerate}
  \setcounter{enumi}{6}
  \item ( 20 \begin{CJK}{UTF8}{mj}分\end{CJK}) \begin{CJK}{UTF8}{mj}设\end{CJK}
\end{enumerate}
$$
\left\{\begin{array}{l}
x_{n}=4 x_{n-1}-5 y_{n-1} \\
y_{n}=2 x_{n-1}-3_{n-1}
\end{array}\right.
$$
\begin{CJK}{UTF8}{mj}且\end{CJK} $x_{0}=2, y_{0}=1$, \begin{CJK}{UTF8}{mj}求\end{CJK} $x_{100}$.

\begin{enumerate}
  \setcounter{enumi}{7}
  \item ( 20 \begin{CJK}{UTF8}{mj}分\end{CJK}) (1) \begin{CJK}{UTF8}{mj}设\end{CJK} $\alpha=(1,3,4)^{T}, \beta=(5,0,-1)^{T} \in \mathbb{R}^{3}$, \begin{CJK}{UTF8}{mj}试求一个\end{CJK} 3 \begin{CJK}{UTF8}{mj}阶正交矩阵\end{CJK} $A$ \begin{CJK}{UTF8}{mj}使得\end{CJK} $A \alpha=\beta$ (\begin{CJK}{UTF8}{mj}不用写求解过\end{CJK} \begin{CJK}{UTF8}{mj}程\end{CJK}).
\end{enumerate}
(2) \begin{CJK}{UTF8}{mj}设非零向量\end{CJK} $\alpha, \beta \in \mathbb{R}^{n}$. \begin{CJK}{UTF8}{mj}证明\end{CJK}: \begin{CJK}{UTF8}{mj}存在正交矩阵\end{CJK} $A$ \begin{CJK}{UTF8}{mj}使得\end{CJK} $A \alpha=\beta$ \begin{CJK}{UTF8}{mj}当且仅当\end{CJK} $\alpha^{T} \alpha-\beta^{T} \beta=0$.

\begin{enumerate}
  \setcounter{enumi}{8}
  \item (20 \begin{CJK}{UTF8}{mj}分\end{CJK}) \begin{CJK}{UTF8}{mj}设\end{CJK} $A$ \begin{CJK}{UTF8}{mj}是\end{CJK} 3 \begin{CJK}{UTF8}{mj}阶实对称矩阵\end{CJK}, \begin{CJK}{UTF8}{mj}各行元素之和均为\end{CJK} 0 , \begin{CJK}{UTF8}{mj}且\end{CJK} $r(2 I-A)=2, A-3 I$ \begin{CJK}{UTF8}{mj}不可逆\end{CJK}.
\end{enumerate}
(1) $X^{T} A X=1$ \begin{CJK}{UTF8}{mj}表示什么样的二次曲面\end{CJK}? \begin{CJK}{UTF8}{mj}为什么\end{CJK}?

(2) \begin{CJK}{UTF8}{mj}求伴随矩阵\end{CJK} $A^{*}$.

\section{1. 东北师范大学 2010 年研究生入学考试试题高等代数与解析几何 
 李扬 
 微信公众号: sxkyliyang}
\begin{CJK}{UTF8}{mj}一\end{CJK}. \begin{CJK}{UTF8}{mj}高等代数\end{CJK} (\begin{CJK}{UTF8}{mj}共\end{CJK} 100 \begin{CJK}{UTF8}{mj}分\end{CJK})

\begin{enumerate}
  \item (15 \begin{CJK}{UTF8}{mj}分\end{CJK})
\end{enumerate}
(1) \begin{CJK}{UTF8}{mj}若\end{CJK}
$$
A=\left(\begin{array}{lll}
1 & 0 & 2 \\
0 & 1 & 3 \\
0 & 0 & 1
\end{array}\right)
$$
\begin{CJK}{UTF8}{mj}求\end{CJK} $A$ \begin{CJK}{UTF8}{mj}的逆矩阵\end{CJK} $A^{-1}$.

(2) \begin{CJK}{UTF8}{mj}若\end{CJK} $A^{2}=A, E$ \begin{CJK}{UTF8}{mj}为单位矩阵\end{CJK}, \begin{CJK}{UTF8}{mj}求\end{CJK} $(A+E)^{-1}$.

\begin{enumerate}
  \setcounter{enumi}{2}
  \item (10 \begin{CJK}{UTF8}{mj}分\end{CJK}) \begin{CJK}{UTF8}{mj}线性方程组\end{CJK}
\end{enumerate}
$$
\left\{\begin{array}{c}
a_{11} x_{1}+a_{12} x_{2}+\cdots+a_{1 n} x_{n}=b_{1} \\
a_{21} x_{1}+a_{22} x_{2}+\cdots+a_{2 n} x_{n}=b_{2} \\
\cdots \cdots \cdots \\
a_{m 1} x_{1}+a_{m 2} x_{2}+\cdots+a_{m n} x_{n}=b_{m} .
\end{array}\right.
$$
\begin{CJK}{UTF8}{mj}有解的充分必要条件为它的系数矩阵与增广矩阵有相同的秩\end{CJK}.

\begin{enumerate}
  \setcounter{enumi}{3}
  \item (10 \begin{CJK}{UTF8}{mj}分\end{CJK}) \begin{CJK}{UTF8}{mj}设\end{CJK} $V$ \begin{CJK}{UTF8}{mj}为复数域上的\end{CJK} $n$ \begin{CJK}{UTF8}{mj}维线性空间\end{CJK}. $\mathscr{A}, \mathscr{B} \in \operatorname{End} V$, \begin{CJK}{UTF8}{mj}且\end{CJK} $\mathscr{A} \mathscr{B}=\mathscr{B} \mathscr{A}$. \begin{CJK}{UTF8}{mj}证明\end{CJK}: $\mathscr{A}, \mathscr{B}$ \begin{CJK}{UTF8}{mj}至少有一个公共的\end{CJK} \begin{CJK}{UTF8}{mj}特征向量\end{CJK}.

  \item (10 \begin{CJK}{UTF8}{mj}分\end{CJK}) \begin{CJK}{UTF8}{mj}设\end{CJK}

\end{enumerate}
$$
f(x)=a_{0} x^{n}+a_{1} x^{n-1}+\cdots+a_{n}
$$
\begin{CJK}{UTF8}{mj}是一个\end{CJK} $n$ \begin{CJK}{UTF8}{mj}次多项式\end{CJK}, \begin{CJK}{UTF8}{mj}且\end{CJK} $a_{0}+a_{1}+\cdots+a_{n}=0$. \begin{CJK}{UTF8}{mj}求证\end{CJK}: $f\left(x^{k+1}\right)$ \begin{CJK}{UTF8}{mj}能被\end{CJK} $x^{k}+x^{k-1}+\cdots+x+1$ \begin{CJK}{UTF8}{mj}整除\end{CJK}, \begin{CJK}{UTF8}{mj}这里\end{CJK} $n, k$ \begin{CJK}{UTF8}{mj}是\end{CJK} \begin{CJK}{UTF8}{mj}正整数\end{CJK}.

\begin{enumerate}
  \setcounter{enumi}{5}
  \item (15 \begin{CJK}{UTF8}{mj}分\end{CJK}) \begin{CJK}{UTF8}{mj}把二次型\end{CJK}
\end{enumerate}
$$
f\left(x_{1}, x_{2}, x_{3}\right)=x_{1} x_{2}-x_{1} x_{3}+3 x_{2} x_{3}
$$
\begin{CJK}{UTF8}{mj}通过非退化线性替换化成平方和\end{CJK}.

\begin{enumerate}
  \setcounter{enumi}{6}
  \item (15 \begin{CJK}{UTF8}{mj}分\end{CJK}) \begin{CJK}{UTF8}{mj}求矩阵\end{CJK}
\end{enumerate}
$$
A=\left(\begin{array}{ccc}
4 & -1 & 2 \\
-9 & 4 & -6 \\
-9 & 3 & -5
\end{array}\right)
$$
\begin{CJK}{UTF8}{mj}的若尔当典范形\end{CJK}.

\begin{enumerate}
  \setcounter{enumi}{7}
  \item (15 \begin{CJK}{UTF8}{mj}分\end{CJK}) \begin{CJK}{UTF8}{mj}在标准欧几里得空间\end{CJK} $V=\mathbb{R}^{4}$ \begin{CJK}{UTF8}{mj}中有向量\end{CJK} $\alpha_{1}=(1,-1,-1,1), \alpha_{2}=(1,-1,0,1), \alpha_{3}=(1,-1,1,0)$, \begin{CJK}{UTF8}{mj}线\end{CJK} \begin{CJK}{UTF8}{mj}性子空间\end{CJK} $W=L\left(\alpha_{1}, \alpha_{2}, \alpha_{3}\right)$. \begin{CJK}{UTF8}{mj}求向量\end{CJK} $\beta=(2,4,1,2)$ \begin{CJK}{UTF8}{mj}在\end{CJK} $W$ \begin{CJK}{UTF8}{mj}上的正交投影\end{CJK}.

  \item (15 \begin{CJK}{UTF8}{mj}分\end{CJK}) \begin{CJK}{UTF8}{mj}设\end{CJK} $V$ \begin{CJK}{UTF8}{mj}为数域\end{CJK} $K$ \begin{CJK}{UTF8}{mj}上的\end{CJK} $n$ \begin{CJK}{UTF8}{mj}维向量空间\end{CJK}. \begin{CJK}{UTF8}{mj}证明\end{CJK}: \begin{CJK}{UTF8}{mj}对于任何大于\end{CJK} $n$ \begin{CJK}{UTF8}{mj}的自然数\end{CJK} $m$,\begin{CJK}{UTF8}{mj}一定存在由\end{CJK} $V$ \begin{CJK}{UTF8}{mj}的\end{CJK} $m$ \begin{CJK}{UTF8}{mj}个向量\end{CJK} \begin{CJK}{UTF8}{mj}组成的向量组\end{CJK}, \begin{CJK}{UTF8}{mj}使其中任何\end{CJK} $n$ \begin{CJK}{UTF8}{mj}个向量都线性无关\end{CJK}。

\end{enumerate}
\begin{CJK}{UTF8}{mj}二\end{CJK}. \begin{CJK}{UTF8}{mj}解析几何\end{CJK} (\begin{CJK}{UTF8}{mj}共\end{CJK} 50 \begin{CJK}{UTF8}{mj}分\end{CJK})

\begin{enumerate}
  \item ( 10 \begin{CJK}{UTF8}{mj}分\end{CJK}) \begin{CJK}{UTF8}{mj}确定二次曲线\end{CJK}
\end{enumerate}
$$
x^{2}-3 x y+y^{2}-2 x+8 y-3=0
$$
\begin{CJK}{UTF8}{mj}的类型和对称轴方程\end{CJK}. 2. (10 \begin{CJK}{UTF8}{mj}分\end{CJK}) \begin{CJK}{UTF8}{mj}求过直线\end{CJK}
$$
\left\{\begin{array}{l}
x-2 y+3 z-3=0 \\
2 x-y+z-2=0
\end{array}\right.
$$
\begin{CJK}{UTF8}{mj}且与平面\end{CJK}
$$
x-y+2 z-4=0
$$
\begin{CJK}{UTF8}{mj}垂直的平面的方程\end{CJK}.

\begin{enumerate}
  \setcounter{enumi}{3}
  \item (15 \begin{CJK}{UTF8}{mj}分\end{CJK}) \begin{CJK}{UTF8}{mj}求以原点为顶点\end{CJK}, \begin{CJK}{UTF8}{mj}以\end{CJK}
\end{enumerate}
$$
\left\{\begin{array}{l}
x^{2}-2 z+1=0 \\
y-z+1=0
\end{array}\right.
$$
\begin{CJK}{UTF8}{mj}为准线的锥面方程\end{CJK}.

\begin{enumerate}
  \setcounter{enumi}{4}
  \item (15 \begin{CJK}{UTF8}{mj}分\end{CJK}) \begin{CJK}{UTF8}{mj}已知两条直线\end{CJK}
\end{enumerate}
$$
L_{1}: \frac{x-1}{-1}=\frac{y}{2}=\frac{z+1}{1}
$$
\begin{CJK}{UTF8}{mj}和\end{CJK}
$$
L_{2}: \frac{x}{1}=\frac{y-1}{0}=\frac{z-2}{-1} .
$$
\begin{CJK}{UTF8}{mj}求\end{CJK} $L_{1}$ \begin{CJK}{UTF8}{mj}和\end{CJK} $L_{2}$ \begin{CJK}{UTF8}{mj}的距离以及它们的公垂线的方程\end{CJK}.

\section{2. 东北师范大学 2011 年研究生入学考试试题高等代数与解析几何 
 李扬 
 微信公众号: sxkyliyang}
\begin{CJK}{UTF8}{mj}一\end{CJK}. \begin{CJK}{UTF8}{mj}高等代数\end{CJK} (\begin{CJK}{UTF8}{mj}共\end{CJK} 100 \begin{CJK}{UTF8}{mj}分\end{CJK})

\begin{enumerate}
  \item (10 \begin{CJK}{UTF8}{mj}分\end{CJK}) \begin{CJK}{UTF8}{mj}设\end{CJK}
\end{enumerate}
$$
A=\left(\begin{array}{ccccc}
x_{1}-a_{1} & x_{2} & x_{3} & \cdots & x_{n} \\
x_{1} & x_{2}-a_{2} & x_{3} & \cdots & x_{n} \\
x_{1} & x_{2} & x_{3}-a_{3} & \cdots & x_{n} \\
\vdots & \vdots & \vdots & & \vdots \\
x_{1} & x_{2} & x_{3} & \cdots & x_{n}-a_{n}
\end{array}\right)
$$
\begin{CJK}{UTF8}{mj}为\end{CJK} $n$ \begin{CJK}{UTF8}{mj}阶方阵\end{CJK}, \begin{CJK}{UTF8}{mj}求矩阵\end{CJK} $A$ \begin{CJK}{UTF8}{mj}的行列式\end{CJK}.

\begin{enumerate}
  \setcounter{enumi}{2}
  \item (10 \begin{CJK}{UTF8}{mj}分\end{CJK}) \begin{CJK}{UTF8}{mj}求下述数域\end{CJK} $K$ \begin{CJK}{UTF8}{mj}上非齐次线性方程组的通解\end{CJK}:
\end{enumerate}
$$
\left\{\begin{array}{l}
x_{1}+x_{2}-2 x_{3}-2 x_{5}=-2 \\
-3 x_{1}-2 x_{2}+3 x_{3}-6 x_{4}+14 x_{5}=-1 \\
-2 x_{1}-x_{3}-11 x_{4}+18 x_{5}=-8 \\
4 x_{1}+2 x_{2}-4 x_{3}+10 x_{4}-20 x_{5}=2
\end{array}\right.
$$

\begin{enumerate}
  \setcounter{enumi}{3}
  \item ( 10 \begin{CJK}{UTF8}{mj}分\end{CJK}) \begin{CJK}{UTF8}{mj}在标准欧几里得空间\end{CJK} $V=\mathbb{R}^{4}$ \begin{CJK}{UTF8}{mj}中有向量组\end{CJK} $\alpha_{1}=(2,2,2,2), \alpha_{2}=(0,2,2,2), \alpha_{3}=(0,0,2,2)$. \begin{CJK}{UTF8}{mj}把它们\end{CJK} \begin{CJK}{UTF8}{mj}化成规范正交向量组\end{CJK}.

  \item (10 \begin{CJK}{UTF8}{mj}分\end{CJK}) \begin{CJK}{UTF8}{mj}证明\end{CJK}: \begin{CJK}{UTF8}{mj}双线性函数非退化的充分必要条件是它的度量矩阵非退化\end{CJK}.

  \item (15 \begin{CJK}{UTF8}{mj}分\end{CJK}) \begin{CJK}{UTF8}{mj}设向量组\end{CJK} $\alpha_{1}, \cdots, \alpha_{s}$ \begin{CJK}{UTF8}{mj}线性无关\end{CJK},

\end{enumerate}
$$
\beta=b_{1} \alpha_{1}+\cdots+b_{s} \alpha_{s}
$$
\begin{CJK}{UTF8}{mj}如果某个\end{CJK} $b_{i} \neq 0$ \begin{CJK}{UTF8}{mj}则用\end{CJK} $\beta$ \begin{CJK}{UTF8}{mj}替换\end{CJK} $\alpha_{i}$ \begin{CJK}{UTF8}{mj}得到的向量组\end{CJK} $\alpha_{1}, \cdots, \alpha_{i-1}, \beta, \alpha_{i+1}, \cdots, \alpha_{s}$ \begin{CJK}{UTF8}{mj}也线性无关\end{CJK}.

\begin{enumerate}
  \setcounter{enumi}{6}
  \item (15 \begin{CJK}{UTF8}{mj}分\end{CJK}) \begin{CJK}{UTF8}{mj}设\end{CJK} $A$ \begin{CJK}{UTF8}{mj}是\end{CJK} $s \times n$ \begin{CJK}{UTF8}{mj}矩阵\end{CJK}, $B$ \begin{CJK}{UTF8}{mj}是\end{CJK} $n \times s$ \begin{CJK}{UTF8}{mj}矩阵\end{CJK}. \begin{CJK}{UTF8}{mj}求证\end{CJK}:
\end{enumerate}
(1) \begin{CJK}{UTF8}{mj}当\end{CJK} $s=n$ \begin{CJK}{UTF8}{mj}时\end{CJK}, $|A B|=|A||B|$.

(2) \begin{CJK}{UTF8}{mj}当\end{CJK} $s>n$ \begin{CJK}{UTF8}{mj}时\end{CJK}, $|A B|=0$.

\begin{enumerate}
  \setcounter{enumi}{7}
  \item (15 \begin{CJK}{UTF8}{mj}分\end{CJK}) \begin{CJK}{UTF8}{mj}设多项式\end{CJK} $P(x), Q(x), R(x), S(x)$ \begin{CJK}{UTF8}{mj}满足\end{CJK}
\end{enumerate}
$$
P\left(x^{5}\right)+x Q\left(x^{5}\right)+x^{2} R\left(x^{5}\right)=\left(1+x+x^{2}+x^{3}+x^{4}\right) S(x)
$$
\begin{CJK}{UTF8}{mj}求证\end{CJK}: $S(1)=P(1)=Q(1)=R(1)=0$.

\begin{enumerate}
  \setcounter{enumi}{8}
  \item (15 \begin{CJK}{UTF8}{mj}分\end{CJK}) \begin{CJK}{UTF8}{mj}设矩阵\end{CJK}
\end{enumerate}
$$
A=\left(\begin{array}{ccc}
1 & 1 & -1 \\
1 & 0 & 0 \\
0 & 1 & -1
\end{array}\right)
$$
\begin{CJK}{UTF8}{mj}应用\end{CJK} $A$ \begin{CJK}{UTF8}{mj}的特征多项式或极小多项式求\end{CJK} $A^{100}$.

\section{二. 解析几何 (共 50 分)}
\begin{enumerate}
  \item (10 \begin{CJK}{UTF8}{mj}分\end{CJK}) \begin{CJK}{UTF8}{mj}用向量方法证明三角形的三条中线交于一点\end{CJK}. 2. (10 \begin{CJK}{UTF8}{mj}分\end{CJK}) \begin{CJK}{UTF8}{mj}求\end{CJK}
\end{enumerate}
$$
6 x^{2}-x y-y^{2}+3 x+y-1=0
$$
\begin{CJK}{UTF8}{mj}的渐近线方程\end{CJK}.

\begin{enumerate}
  \setcounter{enumi}{3}
  \item (15 \begin{CJK}{UTF8}{mj}分\end{CJK}) \begin{CJK}{UTF8}{mj}已知椭球面的主轴和坐标轴重合\end{CJK}, \begin{CJK}{UTF8}{mj}如果它又通过椭圆\end{CJK}
\end{enumerate}
$$
\left\{\begin{array}{l}
\frac{x^{2}}{9}+\frac{y^{2}}{16}=1 \\
z=0 .
\end{array}\right.
$$
\begin{CJK}{UTF8}{mj}和点\end{CJK} $(1,2, \sqrt{23})$, \begin{CJK}{UTF8}{mj}求这个椭球面的方程\end{CJK}.

\begin{enumerate}
  \setcounter{enumi}{4}
  \item (15 \begin{CJK}{UTF8}{mj}分\end{CJK}) \begin{CJK}{UTF8}{mj}已知直角坐标系中两条直线\end{CJK}
\end{enumerate}
$$
L_{1}:\left\{\begin{array}{l}
2 x-y+z+2=0 \\
x+2 y+4 z-4=0
\end{array}\right.
$$
\begin{CJK}{UTF8}{mj}和\end{CJK}
$$
L_{2}:\left\{\begin{array}{l}
x-2 y-2=0 \\
y-z+2=0
\end{array}\right.
$$
\begin{CJK}{UTF8}{mj}求\end{CJK} $L_{1}$ \begin{CJK}{UTF8}{mj}和\end{CJK} $L_{2}$ \begin{CJK}{UTF8}{mj}的距离以及它们的公垂线的方程\end{CJK}.

\section{3. 东北师范大学 2012 年研究生入学考试试题高等代数与解析几何 
 李扬 
 微信公众号: sxkyliyang}
\section{一. 高等代数 (共 100 分)}
\begin{enumerate}
  \item ( 20 \begin{CJK}{UTF8}{mj}分\end{CJK}) \begin{CJK}{UTF8}{mj}设向量组\end{CJK} $\alpha_{1}, \alpha_{2}, \cdots, \alpha_{r}$ \begin{CJK}{UTF8}{mj}线性无关\end{CJK}, \begin{CJK}{UTF8}{mj}试讨论向量组\end{CJK} $\alpha_{1}+\alpha_{2}, \alpha_{2}+\alpha_{3}, \cdots, \alpha_{r}+\alpha_{1}$ \begin{CJK}{UTF8}{mj}线性相关性\end{CJK}.

  \item ( 20 \begin{CJK}{UTF8}{mj}分\end{CJK}) \begin{CJK}{UTF8}{mj}设\end{CJK} $A, B$ \begin{CJK}{UTF8}{mj}都是\end{CJK} $n$ \begin{CJK}{UTF8}{mj}阶正定矩阵\end{CJK}, \begin{CJK}{UTF8}{mj}证明\end{CJK}:

\end{enumerate}
(1) $A B$ \begin{CJK}{UTF8}{mj}的特征值为实数\end{CJK}.

(2) \begin{CJK}{UTF8}{mj}如果\end{CJK} $A B=B A$, \begin{CJK}{UTF8}{mj}则\end{CJK} $A B$ \begin{CJK}{UTF8}{mj}是正定矩阵\end{CJK}.

\begin{enumerate}
  \setcounter{enumi}{3}
  \item ( 20 \begin{CJK}{UTF8}{mj}分\end{CJK}) \begin{CJK}{UTF8}{mj}设\end{CJK} $n$ \begin{CJK}{UTF8}{mj}阶方阵\end{CJK} $A$ \begin{CJK}{UTF8}{mj}满足\end{CJK} $A^{2}=E$. \begin{CJK}{UTF8}{mj}证明\end{CJK}:
\end{enumerate}
(1) $\operatorname{rank}(A+E)+\operatorname{rank}(A-E)=n$.

(2) $A$ \begin{CJK}{UTF8}{mj}可对角化\end{CJK}.

\begin{enumerate}
  \setcounter{enumi}{4}
  \item ( 20 \begin{CJK}{UTF8}{mj}分\end{CJK}) \begin{CJK}{UTF8}{mj}试确定所有整数\end{CJK} $m$, \begin{CJK}{UTF8}{mj}使得\end{CJK}
\end{enumerate}
$$
f(x)=x^{5}+m x-1
$$
\begin{CJK}{UTF8}{mj}在有理数域上不可约\end{CJK}.

\begin{enumerate}
  \setcounter{enumi}{5}
  \item ( 20 \begin{CJK}{UTF8}{mj}分\end{CJK}) \begin{CJK}{UTF8}{mj}设\end{CJK} $\mathscr{A}$ \begin{CJK}{UTF8}{mj}是线性空间\end{CJK} $V$ \begin{CJK}{UTF8}{mj}上的线性变换\end{CJK}, \begin{CJK}{UTF8}{mj}且\end{CJK} $\mathscr{A}^{2}=\mathscr{A}$. \begin{CJK}{UTF8}{mj}证明\end{CJK}:
\end{enumerate}
(1) $\operatorname{ker} \mathscr{A}=\{\alpha \in V \mid \mathscr{A}(\alpha)=0\}, \operatorname{Im} \mathscr{A}=\{\alpha \in V \mid \exists \beta \in V, \mathscr{A}(\beta)=\alpha\}$ \begin{CJK}{UTF8}{mj}都是\end{CJK} $V$ \begin{CJK}{UTF8}{mj}的子空间\end{CJK}.

(2) $V=\operatorname{ker} \mathscr{A} \oplus \operatorname{Im} \mathscr{A}$.

\section{二.解析几何 (共 50 分)}
\begin{enumerate}
  \item (15 \begin{CJK}{UTF8}{mj}分\end{CJK}) \begin{CJK}{UTF8}{mj}求过点\end{CJK} $P(1,0,-2)$, \begin{CJK}{UTF8}{mj}平行于平面\end{CJK}
\end{enumerate}
$$
\pi: 3 x-y+2 z-1=0
$$
\begin{CJK}{UTF8}{mj}且与直线\end{CJK}
$$
L: \frac{x-1}{4}=\frac{y-3}{-2}=\frac{z}{1}
$$
\begin{CJK}{UTF8}{mj}相交的直线方程\end{CJK}.

\begin{enumerate}
  \setcounter{enumi}{2}
  \item ( 15 \begin{CJK}{UTF8}{mj}分\end{CJK}) \begin{CJK}{UTF8}{mj}求证两圆\end{CJK}
\end{enumerate}
$$
S_{1}:\left\{\begin{array}{l}
x^{2}+y^{2}=4 \\
z=0
\end{array}\right.
$$
\begin{CJK}{UTF8}{mj}和\end{CJK}
$$
S_{2}:\left\{\begin{array}{l}
x^{2}+y^{2}+z^{2}-2 x-2 y-2=0 \\
x+y+z=1
\end{array}\right.
$$
\begin{CJK}{UTF8}{mj}在同一个球面上\end{CJK}.

\begin{enumerate}
  \setcounter{enumi}{3}
  \item (20 \begin{CJK}{UTF8}{mj}分\end{CJK}) \begin{CJK}{UTF8}{mj}证明\end{CJK}: \begin{CJK}{UTF8}{mj}方程\end{CJK}
\end{enumerate}
$$
x^{2}+y^{2}+z^{2}-(a x+b y+c z)^{2}=0(a, b, c \text { 不全为 } 0)
$$
\begin{CJK}{UTF8}{mj}表示一圆锥面\end{CJK}.

\section{4. 东北师范大学 2013 年研究生入学考试试题高等代数与解析几何 
 李扬 
 微信公众号: sxkyliyang}
\begin{CJK}{UTF8}{mj}一\end{CJK}. \begin{CJK}{UTF8}{mj}高等代数\end{CJK} (\begin{CJK}{UTF8}{mj}共\end{CJK} 100 \begin{CJK}{UTF8}{mj}分\end{CJK})

\begin{enumerate}
  \item ( 10 \begin{CJK}{UTF8}{mj}分\end{CJK}) \begin{CJK}{UTF8}{mj}设\end{CJK}
\end{enumerate}
$$
A=\left(\begin{array}{cccccc}
a+b & b & 0 & \cdots & 0 & 0 \\
a & a+b & b & \cdots & 0 & 0 \\
0 & a & a+b & \cdots & 0 & 0 \\
\vdots & \vdots & \vdots & & \vdots & \vdots \\
0 & 0 & 0 & \cdots & a+b & b \\
0 & 0 & 0 & \cdots & a & a+b
\end{array}\right)
$$
\begin{CJK}{UTF8}{mj}为\end{CJK} $n$ \begin{CJK}{UTF8}{mj}阶方阵\end{CJK}, \begin{CJK}{UTF8}{mj}若\end{CJK} $a \neq b$, \begin{CJK}{UTF8}{mj}求矩阵\end{CJK} $A$ \begin{CJK}{UTF8}{mj}的行列式\end{CJK}.

\begin{enumerate}
  \setcounter{enumi}{2}
  \item (10 \begin{CJK}{UTF8}{mj}分\end{CJK}) \begin{CJK}{UTF8}{mj}求下述数域\end{CJK} $K$ \begin{CJK}{UTF8}{mj}上非齐次线性方程组的解集\end{CJK}:
\end{enumerate}
$$
\left\{\begin{array}{l}
x_{1}-3 x_{2}+5 x_{3}-2 x_{4}=4 \\
-2 x_{1}+x_{2}-3 x_{3}+x_{4}=-7 \\
-x_{1}-7 x_{2}+9 x_{3}-4 x_{4}=-2
\end{array}\right.
$$

\begin{enumerate}
  \setcounter{enumi}{3}
  \item (10 \begin{CJK}{UTF8}{mj}分\end{CJK}) \begin{CJK}{UTF8}{mj}求矩阵\end{CJK}
\end{enumerate}
$$
\left(\begin{array}{ccc}
1 & 2 & -3 \\
1 & 1 & 2 \\
1 & -1 & 4
\end{array}\right)
$$
\begin{CJK}{UTF8}{mj}的极小多项式\end{CJK}.

\begin{enumerate}
  \setcounter{enumi}{4}
  \item ( 10 \begin{CJK}{UTF8}{mj}分\end{CJK}) \begin{CJK}{UTF8}{mj}设\end{CJK} $W$ \begin{CJK}{UTF8}{mj}为向量空间\end{CJK} $V$ \begin{CJK}{UTF8}{mj}的子空间\end{CJK}, $\eta_{1}, \eta_{2}, \cdots, \eta_{r}$ \begin{CJK}{UTF8}{mj}为\end{CJK} $W$ \begin{CJK}{UTF8}{mj}的一个基\end{CJK}, $\beta_{i}=\sum_{j=1}^{r} a_{i j} \eta_{j}, i=1,2, \cdots, r$. \begin{CJK}{UTF8}{mj}证明\end{CJK}: $\beta_{1}, \beta_{2}, \cdots, \beta_{r}$ \begin{CJK}{UTF8}{mj}也是\end{CJK} $W$ \begin{CJK}{UTF8}{mj}的一个基的充分必要条件是\end{CJK}
\end{enumerate}
$$
\left|\begin{array}{cccc}
a_{11} & a_{12} & \cdots & a_{1 r} \\
a_{21} & a_{22} & \cdots & a_{2 r} \\
\vdots & \vdots & & \vdots \\
a_{r 1} & a_{r 2} & \cdots & a_{r r}
\end{array}\right| \neq 0 .
$$

\begin{enumerate}
  \setcounter{enumi}{5}
  \item (15 \begin{CJK}{UTF8}{mj}分\end{CJK}) \begin{CJK}{UTF8}{mj}设\end{CJK}
\end{enumerate}
$$
A=\left(\begin{array}{ccc}
2 & -2 & 0 \\
-2 & 1 & -2 \\
0 & -2 & 0
\end{array}\right)
$$
\begin{CJK}{UTF8}{mj}求正交矩阵\end{CJK} $T$ \begin{CJK}{UTF8}{mj}使\end{CJK} $T^{-1} A T$ \begin{CJK}{UTF8}{mj}为对角矩阵\end{CJK}.

\begin{enumerate}
  \setcounter{enumi}{6}
  \item (15 \begin{CJK}{UTF8}{mj}分\end{CJK}) \begin{CJK}{UTF8}{mj}两个本原多项式的乘积还是本原多项式\end{CJK}.

  \item ( 15 \begin{CJK}{UTF8}{mj}分\end{CJK}) \begin{CJK}{UTF8}{mj}如果\end{CJK} $W_{1}$ \begin{CJK}{UTF8}{mj}与\end{CJK} $W_{2}$ \begin{CJK}{UTF8}{mj}是\end{CJK} $V$ \begin{CJK}{UTF8}{mj}的两个线性子空间\end{CJK}. \begin{CJK}{UTF8}{mj}求证\end{CJK}:

\end{enumerate}
$$
\operatorname{dim} W_{1}+\operatorname{dim} W_{2}=\operatorname{dim}\left(W_{1}+W_{2}\right)+\operatorname{dim}\left(W_{1} \cap W_{2}\right)
$$

\begin{enumerate}
  \setcounter{enumi}{8}
  \item ( 15 \begin{CJK}{UTF8}{mj}分\end{CJK}) \begin{CJK}{UTF8}{mj}设\end{CJK} $A$ \begin{CJK}{UTF8}{mj}是\end{CJK} $s \times n$ \begin{CJK}{UTF8}{mj}矩阵\end{CJK}, $B$ \begin{CJK}{UTF8}{mj}是\end{CJK} $n \times s$ \begin{CJK}{UTF8}{mj}矩阵\end{CJK}. \begin{CJK}{UTF8}{mj}求证\end{CJK}:
\end{enumerate}
(1) \begin{CJK}{UTF8}{mj}当\end{CJK} $s=n$ \begin{CJK}{UTF8}{mj}时\end{CJK}, $|A B|=|A||B|$.

(2) \begin{CJK}{UTF8}{mj}当\end{CJK} $s>n$ \begin{CJK}{UTF8}{mj}时\end{CJK}, $|A B|=0$.

\section{二. 解析几何 (共 50 分)}
\begin{enumerate}
  \item ( 10 \begin{CJK}{UTF8}{mj}分\end{CJK}) \begin{CJK}{UTF8}{mj}用向量方法证明三角形的三条角平分线交于一点\end{CJK}.

  \item (10 \begin{CJK}{UTF8}{mj}分\end{CJK}) \begin{CJK}{UTF8}{mj}求\end{CJK}

\end{enumerate}
$$
2 x y-4 x-2 y+3=0
$$
\begin{CJK}{UTF8}{mj}的渐近线方程\end{CJK}.

\begin{enumerate}
  \setcounter{enumi}{3}
  \item ( 15 \begin{CJK}{UTF8}{mj}分\end{CJK}) \begin{CJK}{UTF8}{mj}证明\end{CJK}
\end{enumerate}
$$
2 x^{2}+y^{2}-z^{2}+3 x y+x z-6 z=0
$$
\begin{CJK}{UTF8}{mj}是直纹面\end{CJK}, \begin{CJK}{UTF8}{mj}并求出其上过点\end{CJK} $(1,1,1)$ \begin{CJK}{UTF8}{mj}的直母线方程\end{CJK}.

\begin{enumerate}
  \setcounter{enumi}{4}
  \item ( 15 \begin{CJK}{UTF8}{mj}分\end{CJK}) \begin{CJK}{UTF8}{mj}已知直角坐标系中两条直线\end{CJK}
\end{enumerate}
$$
L_{1}:\left\{\begin{array}{l}
2 x-y+z+2=0 \\
x+2 y+4 z-4=0
\end{array}\right.
$$
\begin{CJK}{UTF8}{mj}和\end{CJK}
$$
L_{2}:\left\{\begin{array}{l}
x-2 y-2=0 \\
y-z+2=0
\end{array}\right.
$$
\begin{CJK}{UTF8}{mj}求\end{CJK} $L_{1}$ \begin{CJK}{UTF8}{mj}和\end{CJK} $L_{2}$ \begin{CJK}{UTF8}{mj}的距离以及它们的公垂线的方程\end{CJK}.

\section{5. 东北师范大学 2014 年研究生入学考试试题高等代数与解析几何 
 李扬 
 微信公众号: sxkyliyang}
\section{一. 高等代数 (共 100 分)}
\begin{enumerate}
  \item (15 \begin{CJK}{UTF8}{mj}分\end{CJK}) \begin{CJK}{UTF8}{mj}设矩阵\end{CJK}
\end{enumerate}
$$
A=\left[\begin{array}{ccccc}
x & y & y & \cdots & y \\
y & x & y & \cdots & y \\
y & y & x & \cdots & y \\
\vdots & \vdots & \vdots & & \vdots \\
y & y & y & \cdots & x
\end{array}\right]
$$
\begin{CJK}{UTF8}{mj}为\end{CJK} $n$ \begin{CJK}{UTF8}{mj}阶矩阵\end{CJK}.

(1) \begin{CJK}{UTF8}{mj}计算矩阵\end{CJK} $A$ \begin{CJK}{UTF8}{mj}的行列式\end{CJK}.

(2) \begin{CJK}{UTF8}{mj}设\end{CJK} $\alpha_{1}, \alpha_{2}, \cdots, \alpha_{n}$ \begin{CJK}{UTF8}{mj}分别是矩阵\end{CJK} $A$ \begin{CJK}{UTF8}{mj}的第\end{CJK} 1 \begin{CJK}{UTF8}{mj}列\end{CJK}, \begin{CJK}{UTF8}{mj}第\end{CJK} 2 \begin{CJK}{UTF8}{mj}列\end{CJK}, $\cdots$ \begin{CJK}{UTF8}{mj}第\end{CJK} $n$ \begin{CJK}{UTF8}{mj}列向量\end{CJK}. $x, y$ \begin{CJK}{UTF8}{mj}满足什么条件时\end{CJK}, \begin{CJK}{UTF8}{mj}向量组\end{CJK} $\left\{\alpha_{1}, \alpha_{2}, \cdots, \alpha_{n}\right\}$ \begin{CJK}{UTF8}{mj}是线性无关\end{CJK}.

\begin{enumerate}
  \setcounter{enumi}{2}
  \item (15 \begin{CJK}{UTF8}{mj}分\end{CJK}) \begin{CJK}{UTF8}{mj}当\end{CJK} $\lambda$ \begin{CJK}{UTF8}{mj}取何值时\end{CJK}, \begin{CJK}{UTF8}{mj}线性方程组\end{CJK}
\end{enumerate}
$$
\left\{\begin{array}{c}
\lambda x_{1}+9 x_{2}+3 x_{3}=2 \\
-x_{1}+(\lambda-1) x_{2}=\lambda \\
3 x_{1}-x_{2}+x_{3}=-4
\end{array}\right.
$$
\begin{CJK}{UTF8}{mj}无解\end{CJK}; \begin{CJK}{UTF8}{mj}有唯一解\end{CJK}; \begin{CJK}{UTF8}{mj}有无穷多解\end{CJK}. \begin{CJK}{UTF8}{mj}在有无穷多解时\end{CJK}, \begin{CJK}{UTF8}{mj}求出其通解\end{CJK}.

\begin{enumerate}
  \setcounter{enumi}{3}
  \item (15 \begin{CJK}{UTF8}{mj}分\end{CJK}) \begin{CJK}{UTF8}{mj}求矩阵\end{CJK}
\end{enumerate}
$$
\left(\begin{array}{ccc}
2 & 3 & 2 \\
1 & 8 & 2 \\
-2 & -14 & -3
\end{array}\right)
$$
\begin{CJK}{UTF8}{mj}的初等因子组\end{CJK}, \begin{CJK}{UTF8}{mj}最低多项式\end{CJK}, Jordan \begin{CJK}{UTF8}{mj}典范形\end{CJK}.

\begin{enumerate}
  \setcounter{enumi}{4}
  \item (15 \begin{CJK}{UTF8}{mj}分\end{CJK}) \begin{CJK}{UTF8}{mj}设\end{CJK} $A \in M_{n}(P)$ \begin{CJK}{UTF8}{mj}是幂等矩阵\end{CJK} (\begin{CJK}{UTF8}{mj}即\end{CJK} $\left.A^{2}=A\right)$, \begin{CJK}{UTF8}{mj}并且\end{CJK} $W_{1}=\left\{X \in P^{n} \mid A X=0\right\}, W_{2}=\left\{X \in P^{n} \mid A X=X\right\}$. \begin{CJK}{UTF8}{mj}证明\end{CJK}:
\end{enumerate}
$$
P^{n}=W_{1} \oplus W_{2} .
$$

\begin{enumerate}
  \setcounter{enumi}{5}
  \item ( 20 \begin{CJK}{UTF8}{mj}分\end{CJK}) \begin{CJK}{UTF8}{mj}设\end{CJK} $A$ \begin{CJK}{UTF8}{mj}是\end{CJK} $n$ \begin{CJK}{UTF8}{mj}级方阵\end{CJK}, \begin{CJK}{UTF8}{mj}且\end{CJK} $A=\left(\begin{array}{cc}B & C \\ D & E\end{array}\right)$. \begin{CJK}{UTF8}{mj}其中\end{CJK} $B$ \begin{CJK}{UTF8}{mj}是\end{CJK} $r$ \begin{CJK}{UTF8}{mj}级可逆矩阵\end{CJK}.
\end{enumerate}
$$
|A|=|B|\left|E-D B^{-1} C\right| .
$$
(2) \begin{CJK}{UTF8}{mj}若\end{CJK} $|A| \neq 0$, \begin{CJK}{UTF8}{mj}求证\end{CJK}:
$$
A^{-1}=\left(\begin{array}{cc}
I_{r} & -B^{-1} C \\
0 & I_{n-r}
\end{array}\right)\left(\begin{array}{cc}
B^{-1} & 0 \\
0 & \left(E-D B^{-1} C\right)^{-1}
\end{array}\right)\left(\begin{array}{cc}
I_{r} & 0 \\
-D B^{-1} & I_{n-r}
\end{array}\right)
$$

\begin{enumerate}
  \setcounter{enumi}{6}
  \item (20 \begin{CJK}{UTF8}{mj}分\end{CJK}) \begin{CJK}{UTF8}{mj}设\end{CJK} $f(x)$ \begin{CJK}{UTF8}{mj}是一个整系数多项式\end{CJK}, $a, b, c$ \begin{CJK}{UTF8}{mj}是三个互异的整数\end{CJK}. \begin{CJK}{UTF8}{mj}证明\end{CJK}: \begin{CJK}{UTF8}{mj}不可能有\end{CJK}
\end{enumerate}
$$
f(a)=b, f(b)=c, f(c)=a
$$

\begin{enumerate}
  \item (15 \begin{CJK}{UTF8}{mj}分\end{CJK}) \begin{CJK}{UTF8}{mj}求曲线\end{CJK}
\end{enumerate}
$$
x^{2}+4 x y+3 y^{2}-5 x-6 y+3=0
$$
\begin{CJK}{UTF8}{mj}的平行于\end{CJK} $x+4 y=0$ \begin{CJK}{UTF8}{mj}的切线方程\end{CJK}, \begin{CJK}{UTF8}{mj}并求出切点的坐标\end{CJK}.

\begin{enumerate}
  \setcounter{enumi}{2}
  \item ( 20 \begin{CJK}{UTF8}{mj}分\end{CJK}) \begin{CJK}{UTF8}{mj}求与直线\end{CJK}
\end{enumerate}
$$
\frac{x-2}{1}=\frac{y}{1}=\frac{z}{1}
$$
\begin{CJK}{UTF8}{mj}与\end{CJK}
$$
\frac{x}{2}=\frac{y-1}{1}=\frac{z}{-2}
$$
\begin{CJK}{UTF8}{mj}相交\end{CJK}, \begin{CJK}{UTF8}{mj}且与平面\end{CJK} $x+y+2=0$ \begin{CJK}{UTF8}{mj}平行的直线的轨迹\end{CJK}.

\begin{enumerate}
  \setcounter{enumi}{3}
  \item (15 \begin{CJK}{UTF8}{mj}分\end{CJK}) \begin{CJK}{UTF8}{mj}已知单叶双曲面的一个平面截线为\end{CJK}
\end{enumerate}
$$
\left\{\begin{array}{l}
\frac{x^{2}}{45}+\frac{y^{2}}{80}=1 \\
z=4
\end{array}\right.
$$
\begin{CJK}{UTF8}{mj}且过点\end{CJK} $(-3,4,-2)$, \begin{CJK}{UTF8}{mj}求它的标准方程\end{CJK}.

\section{6. 东北师范大学 2015 年研究生入学考试试题高等代数与解析几何 
 李扬 
 微信公众号: sxkyliyang}
\begin{CJK}{UTF8}{mj}一\end{CJK}. \begin{CJK}{UTF8}{mj}高等代数\end{CJK} (\begin{CJK}{UTF8}{mj}共\end{CJK} 100 \begin{CJK}{UTF8}{mj}分\end{CJK})

\begin{enumerate}
  \item (15 \begin{CJK}{UTF8}{mj}分\end{CJK}) \begin{CJK}{UTF8}{mj}计算\end{CJK} $n$ \begin{CJK}{UTF8}{mj}阶行列式\end{CJK}
\end{enumerate}
$$
D_{n}=\left|\begin{array}{ccccc}
a+1 & a & a & \cdots & a \\
a & a+2 & a & \cdots & a \\
a & a & a+3 & \cdots & a \\
\vdots & \vdots & \vdots & & \vdots \\
a & a & a & \cdots & a+n
\end{array}\right| .
$$

\begin{enumerate}
  \setcounter{enumi}{2}
  \item (15 \begin{CJK}{UTF8}{mj}分\end{CJK}) \begin{CJK}{UTF8}{mj}设\end{CJK}
\end{enumerate}
$$
f(x)=x^{n+1}+2 x^{n}-3(n \geqslant 1)
$$
\begin{CJK}{UTF8}{mj}把\end{CJK} $f(x)$ \begin{CJK}{UTF8}{mj}在有理数域上进行因式分解\end{CJK}.

\begin{enumerate}
  \setcounter{enumi}{3}
  \item (15 \begin{CJK}{UTF8}{mj}分\end{CJK}) \begin{CJK}{UTF8}{mj}求矩阵\end{CJK}
\end{enumerate}
$$
A=\left(\begin{array}{ccc}
-1 & -2 & 6 \\
-1 & 0 & 3 \\
-1 & -1 & 4
\end{array}\right)
$$
\begin{CJK}{UTF8}{mj}的初等因子及\end{CJK} Jordan \begin{CJK}{UTF8}{mj}标准形\end{CJK}.

\begin{enumerate}
  \setcounter{enumi}{4}
  \item (15 \begin{CJK}{UTF8}{mj}分\end{CJK}) \begin{CJK}{UTF8}{mj}设\end{CJK} $A, B$ \begin{CJK}{UTF8}{mj}分别是\end{CJK} $s \times n, s \times m$ \begin{CJK}{UTF8}{mj}矩阵\end{CJK}. \begin{CJK}{UTF8}{mj}证明\end{CJK}: \begin{CJK}{UTF8}{mj}矩阵方程\end{CJK} $A X=B$ \begin{CJK}{UTF8}{mj}有解的充分必要条件是\end{CJK}
\end{enumerate}
$$
\operatorname{rank}(A)=\operatorname{rank}(A \vdots B)
$$

\begin{enumerate}
  \setcounter{enumi}{5}
  \item ( 20 \begin{CJK}{UTF8}{mj}分\end{CJK}) \begin{CJK}{UTF8}{mj}设\end{CJK} $A$ \begin{CJK}{UTF8}{mj}是\end{CJK} $n$ \begin{CJK}{UTF8}{mj}阶实对称矩阵\end{CJK}, $\lambda_{1}, \lambda_{n}$ \begin{CJK}{UTF8}{mj}分别为\end{CJK} $A$ \begin{CJK}{UTF8}{mj}的最小和最大特征值\end{CJK}, \begin{CJK}{UTF8}{mj}证明\end{CJK}: \begin{CJK}{UTF8}{mj}对于实二次型\end{CJK} $f\left(x_{1}, x_{2}, \cdots, x_{n}\right)=$ $X^{T} A X$, \begin{CJK}{UTF8}{mj}恒有\end{CJK}
\end{enumerate}
$$
\lambda_{1} X^{T} X \leqslant X^{T} A X \leqslant \lambda_{n} X^{T} X
$$

\begin{enumerate}
  \setcounter{enumi}{6}
  \item ( 20 \begin{CJK}{UTF8}{mj}分\end{CJK}) \begin{CJK}{UTF8}{mj}设\end{CJK} $\mathscr{A}$ \begin{CJK}{UTF8}{mj}是\end{CJK} $n$ \begin{CJK}{UTF8}{mj}维线性空间\end{CJK} $V$ \begin{CJK}{UTF8}{mj}的可逆线性变换\end{CJK}.
\end{enumerate}
(1) \begin{CJK}{UTF8}{mj}试证\end{CJK} $\mathscr{A}$ \begin{CJK}{UTF8}{mj}的逆变换\end{CJK} $\mathscr{A}^{-1}$ \begin{CJK}{UTF8}{mj}可表成\end{CJK} $\mathscr{A}$ \begin{CJK}{UTF8}{mj}的多项式\end{CJK}.

(2) \begin{CJK}{UTF8}{mj}令\end{CJK} $f(\lambda)$ \begin{CJK}{UTF8}{mj}是\end{CJK} $\mathscr{A}$ \begin{CJK}{UTF8}{mj}的特征多项式\end{CJK}, \begin{CJK}{UTF8}{mj}试证当多项式\end{CJK} $g(\lambda)$ \begin{CJK}{UTF8}{mj}与\end{CJK} $f(\lambda)$ \begin{CJK}{UTF8}{mj}互素时\end{CJK}, $g(\mathscr{A})$ \begin{CJK}{UTF8}{mj}是可逆线性变换\end{CJK}.

\begin{CJK}{UTF8}{mj}二\end{CJK}. \begin{CJK}{UTF8}{mj}解析几何\end{CJK} (\begin{CJK}{UTF8}{mj}共\end{CJK} 50 \begin{CJK}{UTF8}{mj}分\end{CJK})

\begin{enumerate}
  \item ( 20 \begin{CJK}{UTF8}{mj}分\end{CJK}) \begin{CJK}{UTF8}{mj}在空间直角坐标系下\end{CJK}, \begin{CJK}{UTF8}{mj}已知两直线\end{CJK} $l_{1}$ \begin{CJK}{UTF8}{mj}与\end{CJK} $l_{2}$ \begin{CJK}{UTF8}{mj}的标准方程分别为\end{CJK}:
\end{enumerate}
$$
\begin{gathered}
l_{1}: \frac{x-2}{2}=\frac{y-1}{1}=\frac{z-3}{3}, \\
l_{2}: \frac{x}{2}=\frac{y-1}{-1}=\frac{z+3}{3} .
\end{gathered}
$$
(1) \begin{CJK}{UTF8}{mj}证明\end{CJK}: $l_{1}$ \begin{CJK}{UTF8}{mj}与\end{CJK} $l_{2}$ \begin{CJK}{UTF8}{mj}是异面直线\end{CJK}, \begin{CJK}{UTF8}{mj}并求\end{CJK} $l_{1}$ \begin{CJK}{UTF8}{mj}与\end{CJK} $l_{2}$ \begin{CJK}{UTF8}{mj}间的距离\end{CJK}.

(2) \begin{CJK}{UTF8}{mj}求与\end{CJK} $l_{1}, l_{2}$ \begin{CJK}{UTF8}{mj}都相交且与平面\end{CJK} $x+y+3=0$ \begin{CJK}{UTF8}{mj}平行的直线的轨迹方程\end{CJK}.

\begin{enumerate}
  \setcounter{enumi}{2}
  \item (15 \begin{CJK}{UTF8}{mj}分\end{CJK}) \begin{CJK}{UTF8}{mj}已知平面上二次曲线方程为\end{CJK}
\end{enumerate}
$$
3 x^{2}+8 x y-3 y^{2}-8 x+16 y+5=0 .
$$
\begin{CJK}{UTF8}{mj}试求它的渐近线方程\end{CJK}, \begin{CJK}{UTF8}{mj}并求该曲线在\end{CJK} $(1,0)$ \begin{CJK}{UTF8}{mj}点的切线方程\end{CJK}. 3. (15 \begin{CJK}{UTF8}{mj}分\end{CJK}) \begin{CJK}{UTF8}{mj}在空间直角坐标系下\end{CJK}, \begin{CJK}{UTF8}{mj}求过三直线\end{CJK}
$$
\begin{gathered}
l_{1}: \frac{x}{1}=\frac{y}{2}=\frac{z}{-1} \\
l_{2}: \frac{x-1}{1}=\frac{y}{2}=\frac{z-1}{-1} \\
l_{3}: \frac{x+1}{1}=\frac{y-2}{2}=\frac{z}{-1}
\end{gathered}
$$
\begin{CJK}{UTF8}{mj}的圆柱面方程\end{CJK}.

\section{7. 东北师范大学 2016 年研究生入学考试试题高等代数与解析几何 
 李扬 
 微信公众号: sxkyliyang}
\begin{CJK}{UTF8}{mj}一\end{CJK}. \begin{CJK}{UTF8}{mj}高等代数\end{CJK} (\begin{CJK}{UTF8}{mj}共\end{CJK} 100 \begin{CJK}{UTF8}{mj}分\end{CJK})

\begin{enumerate}
  \item ( 10 \begin{CJK}{UTF8}{mj}分\end{CJK}) \begin{CJK}{UTF8}{mj}计算\end{CJK} $n$ \begin{CJK}{UTF8}{mj}阶行列式\end{CJK}
\end{enumerate}
$$
D_{n}=\left|\begin{array}{cccccc}
a & b & b & \cdots & b & b \\
c & a & b & \cdots & b & b \\
c & c & a & \cdots & b & b \\
\vdots & \vdots & \vdots & & \vdots & \vdots \\
c & c & c & \cdots & a & b \\
c & c & c & \cdots & c & a
\end{array}\right|
$$

\begin{enumerate}
  \setcounter{enumi}{2}
  \item (15 \begin{CJK}{UTF8}{mj}分\end{CJK}) \begin{CJK}{UTF8}{mj}设\end{CJK}
\end{enumerate}
$$
f(x)=x^{n+1}+x^{n}-2(n \geqslant 1) .
$$
\begin{CJK}{UTF8}{mj}把\end{CJK} $f(x)$ \begin{CJK}{UTF8}{mj}在有理数域上进行因式分解\end{CJK}.

\begin{enumerate}
  \setcounter{enumi}{3}
  \item (15 \begin{CJK}{UTF8}{mj}分\end{CJK}) \begin{CJK}{UTF8}{mj}设\end{CJK}
\end{enumerate}
$$
A=\left(\begin{array}{ccccc}
a & 1 & 1 & \cdots & 1 \\
1 & a & 1 & \cdots & 1 \\
\vdots & \vdots & \vdots & & \vdots \\
1 & 1 & 1 & \cdots & a
\end{array}\right)
$$
\begin{CJK}{UTF8}{mj}求齐次线性方程组\end{CJK} $A X=0$ \begin{CJK}{UTF8}{mj}的解空间\end{CJK}.

\begin{enumerate}
  \setcounter{enumi}{4}
  \item ( 15 \begin{CJK}{UTF8}{mj}分\end{CJK}) \begin{CJK}{UTF8}{mj}设\end{CJK} $A$ \begin{CJK}{UTF8}{mj}是\end{CJK} $n$ \begin{CJK}{UTF8}{mj}阶实对称矩阵\end{CJK}, \begin{CJK}{UTF8}{mj}它的\end{CJK} $n$ \begin{CJK}{UTF8}{mj}个特征值的绝对值中最大者记为\end{CJK} $S_{r}(A)$. \begin{CJK}{UTF8}{mj}证明\end{CJK}: \begin{CJK}{UTF8}{mj}当\end{CJK} $t>S_{r}(A)$ \begin{CJK}{UTF8}{mj}时\end{CJK}, $t I+A$ \begin{CJK}{UTF8}{mj}是正定矩阵\end{CJK}.

  \item ( 15 \begin{CJK}{UTF8}{mj}分\end{CJK}) \begin{CJK}{UTF8}{mj}设\end{CJK} $A$ \begin{CJK}{UTF8}{mj}是\end{CJK} $n$ \begin{CJK}{UTF8}{mj}阶实对称矩阵\end{CJK}, $\lambda_{1}, \lambda_{2}, \cdots, \lambda_{n}$ \begin{CJK}{UTF8}{mj}是\end{CJK} $A$ \begin{CJK}{UTF8}{mj}的\end{CJK} $n$ \begin{CJK}{UTF8}{mj}个特征值\end{CJK}, \begin{CJK}{UTF8}{mj}相应的标准正交特征向量为\end{CJK} $\xi_{1}, \xi_{2}, \cdots, \xi_{n}$, \begin{CJK}{UTF8}{mj}证明\end{CJK}:

\end{enumerate}
$$
A=\lambda_{1} \xi_{1} \xi_{1}^{T}+\lambda_{2} \xi_{2} \xi_{2}^{T}+\cdots+\lambda_{n} \xi_{n} \xi_{n}^{T} .
$$
\begin{CJK}{UTF8}{mj}其中\end{CJK} $T$ \begin{CJK}{UTF8}{mj}表示转置\end{CJK}.

\begin{enumerate}
  \setcounter{enumi}{6}
  \item (15 \begin{CJK}{UTF8}{mj}分\end{CJK}) \begin{CJK}{UTF8}{mj}如果\end{CJK} $V_{1}, V_{2}$ \begin{CJK}{UTF8}{mj}都是域\end{CJK} $\mathbb{F}$ \begin{CJK}{UTF8}{mj}上线性空间\end{CJK} $V$ \begin{CJK}{UTF8}{mj}的有限维子空间\end{CJK}. \begin{CJK}{UTF8}{mj}证明\end{CJK}:
\end{enumerate}
$$
\operatorname{dim} V_{1}+\operatorname{dim} V_{2}=\operatorname{dim}\left(V_{1}+V_{2}\right)+\operatorname{dim}\left(V_{1} \cap V_{2}\right) .
$$

\begin{enumerate}
  \setcounter{enumi}{7}
  \item (15 \begin{CJK}{UTF8}{mj}分\end{CJK}) \begin{CJK}{UTF8}{mj}设\end{CJK} $\mathscr{A}$ \begin{CJK}{UTF8}{mj}是域\end{CJK} $\mathbb{F}$ \begin{CJK}{UTF8}{mj}上线性空间\end{CJK} $V$ \begin{CJK}{UTF8}{mj}上的一个线性变换\end{CJK}, $f(x), g(x) \in \mathbb{F}[x]$, \begin{CJK}{UTF8}{mj}且首项系数为\end{CJK} $1, d(x)=(f(x), g(x))$. \begin{CJK}{UTF8}{mj}证明\end{CJK}:
\end{enumerate}
$$
\operatorname{ker} d(\mathscr{A})=\operatorname{ker} f(\mathscr{A}) \cap \operatorname{ker} g(\mathscr{A})
$$

\section{二. 解析几何 (共 50 分)}
\begin{enumerate}
  \item ( 20 \begin{CJK}{UTF8}{mj}分\end{CJK}) \begin{CJK}{UTF8}{mj}试判断两直线\end{CJK}
\end{enumerate}
$$
\begin{aligned}
&l_{1}:\left\{\begin{array}{l}
x-2 y-3=0 \\
z-1=0
\end{array}\right. \\
&l_{2}:\left\{\begin{array}{l}
x-z+1=0 \\
y-2=0
\end{array}\right.
\end{aligned}
$$
\begin{CJK}{UTF8}{mj}的相关位置\end{CJK}. \begin{CJK}{UTF8}{mj}若共面求出其所在的平面\end{CJK}; \begin{CJK}{UTF8}{mj}若异面求出它们的距离及公垂线方程\end{CJK}. 2. (15 \begin{CJK}{UTF8}{mj}分\end{CJK}) \begin{CJK}{UTF8}{mj}设曲线方程为\end{CJK}
$$
\left\{\begin{array}{l}
x^{2}-2 z+1=0 \\
y-z+1=0
\end{array}\right.
$$
(1) \begin{CJK}{UTF8}{mj}求该曲线在\end{CJK} $x o y$ \begin{CJK}{UTF8}{mj}坐标面上的射影柱面方程\end{CJK}.

(2) \begin{CJK}{UTF8}{mj}求顶点在原点且以此曲线为准线的雉面方程\end{CJK}.

\begin{enumerate}
  \setcounter{enumi}{3}
  \item (15 \begin{CJK}{UTF8}{mj}分\end{CJK}) \begin{CJK}{UTF8}{mj}设三个平面的方程分别为\end{CJK}:
\end{enumerate}
$$
\begin{aligned}
&\pi_{1}: 2 x-3 y+z-1=0 \\
&\pi_{2}: x-5 y+2 z-3=0 \\
&\pi_{3}: 3 x-y+\lambda z+\mu=0
\end{aligned}
$$
\begin{CJK}{UTF8}{mj}试确定\end{CJK} $\lambda, \mu$ \begin{CJK}{UTF8}{mj}的值使得这三个平面交于同一条直线\end{CJK}.

\section{8. 东北师范大学 2017 年研究生入学考试试题高等代数与解析几何 
 李扬 
 微信公众号: sxkyliyang}
\section{一. 高等代数 (共 100 分)}
\begin{enumerate}
  \item (15 \begin{CJK}{UTF8}{mj}分\end{CJK}) \begin{CJK}{UTF8}{mj}设\end{CJK} $f(x), g(x) \in \mathbb{R}[x]$ \begin{CJK}{UTF8}{mj}为实数域\end{CJK} $\mathbb{R}$ \begin{CJK}{UTF8}{mj}上的非零多项式\end{CJK}. \begin{CJK}{UTF8}{mj}求证\end{CJK}: $f(x), g(x)$ \begin{CJK}{UTF8}{mj}互素的充分必要条件是存在多\end{CJK} \begin{CJK}{UTF8}{mj}项式\end{CJK} $u(x), v(x) \in \mathbb{R}[x]$ \begin{CJK}{UTF8}{mj}使得\end{CJK}
\end{enumerate}
$$
f(x) u(x)+g(x) v(x)=1
$$

\begin{enumerate}
  \setcounter{enumi}{2}
  \item (15 \begin{CJK}{UTF8}{mj}分\end{CJK}) \begin{CJK}{UTF8}{mj}若一个非零整系数多项式的全体系数的最大公因子为\end{CJK} 1 , \begin{CJK}{UTF8}{mj}则称它是本原多项式\end{CJK}. \begin{CJK}{UTF8}{mj}求证\end{CJK}: \begin{CJK}{UTF8}{mj}任意两个本原\end{CJK} \begin{CJK}{UTF8}{mj}多项式的乘积也是本原多项式\end{CJK}.

  \item (15 \begin{CJK}{UTF8}{mj}分\end{CJK}) \begin{CJK}{UTF8}{mj}计算以下行列式\end{CJK}

\end{enumerate}
$$
\operatorname{det}\left(\begin{array}{cccc}
1 & 2 & 3 & 7 \\
2 & 0 & 1 & 0 \\
0 & 3 & 0 & 2 \\
2 & 1 & 1 & 1
\end{array}\right)
$$

\begin{enumerate}
  \setcounter{enumi}{4}
  \item ( 15 \begin{CJK}{UTF8}{mj}分\end{CJK}) \begin{CJK}{UTF8}{mj}设\end{CJK} $M_{n}(\mathbb{R})$ \begin{CJK}{UTF8}{mj}表示实数域\end{CJK} $\mathbb{R}$ \begin{CJK}{UTF8}{mj}上全体\end{CJK} $n \times n$ \begin{CJK}{UTF8}{mj}矩阵的集合\end{CJK}.
\end{enumerate}
(1) \begin{CJK}{UTF8}{mj}求证\end{CJK}: \begin{CJK}{UTF8}{mj}矩阵的相合关系是\end{CJK} $M_{n}(\mathbb{R})$ \begin{CJK}{UTF8}{mj}上的一种等价关系\end{CJK}. (\begin{CJK}{UTF8}{mj}提示\end{CJK}: \begin{CJK}{UTF8}{mj}矩阵\end{CJK} $A, B$ \begin{CJK}{UTF8}{mj}相合意味着存在一个可逆矩阵\end{CJK} $P$ \begin{CJK}{UTF8}{mj}及它的转置\end{CJK} $P^{\prime}$ \begin{CJK}{UTF8}{mj}使得\end{CJK} $\left.B=P^{\prime} A P\right)$

(2) \begin{CJK}{UTF8}{mj}例举出\end{CJK} $M_{n}(\mathbb{R})$ \begin{CJK}{UTF8}{mj}上的一种关系\end{CJK}, \begin{CJK}{UTF8}{mj}使其并不是等价关系\end{CJK}.

\begin{enumerate}
  \setcounter{enumi}{5}
  \item (15 \begin{CJK}{UTF8}{mj}分\end{CJK}) \begin{CJK}{UTF8}{mj}计算下列方阵的特征多项式\end{CJK}, \begin{CJK}{UTF8}{mj}特征根及特征向量\end{CJK}:
\end{enumerate}
$$
\left(\begin{array}{ccc}
5 & 6 & -3 \\
-1 & 0 & 1 \\
1 & 2 & -1
\end{array}\right)
$$

\begin{enumerate}
  \setcounter{enumi}{6}
  \item (10 \begin{CJK}{UTF8}{mj}分\end{CJK}) \begin{CJK}{UTF8}{mj}令\end{CJK} $\mathscr{A}$ \begin{CJK}{UTF8}{mj}是\end{CJK} $n$ \begin{CJK}{UTF8}{mj}维欧式空间\end{CJK} $W$ \begin{CJK}{UTF8}{mj}上的线性变换\end{CJK}, \begin{CJK}{UTF8}{mj}在\end{CJK} $W$ \begin{CJK}{UTF8}{mj}的一组标准正交基下\end{CJK}, $\mathscr{A}$ \begin{CJK}{UTF8}{mj}的矩阵表示为\end{CJK} $A$. \begin{CJK}{UTF8}{mj}求证\end{CJK}: $\mathscr{A}$ \begin{CJK}{UTF8}{mj}是正交变换的充分必要条件是\end{CJK}
\end{enumerate}
$$
A^{\prime} A=E_{n}
$$
\begin{CJK}{UTF8}{mj}这里\end{CJK} $A^{\prime}$ \begin{CJK}{UTF8}{mj}表示矩阵的转置\end{CJK}, $E_{n}$ \begin{CJK}{UTF8}{mj}表示单位矩阵\end{CJK}.

\begin{enumerate}
  \setcounter{enumi}{7}
  \item ( 15 \begin{CJK}{UTF8}{mj}分\end{CJK}) \begin{CJK}{UTF8}{mj}计算以下复方阵的初等因式\end{CJK}:
\end{enumerate}
$$
A=\left(\begin{array}{ccc}
13 & 16 & 16 \\
-5 & -7 & -6 \\
-5 & -8 & -7
\end{array}\right)
$$

\section{二. 解析几何 (共 50 分)}
\begin{enumerate}
  \item (10 \begin{CJK}{UTF8}{mj}分\end{CJK}) \begin{CJK}{UTF8}{mj}已知直线\end{CJK}
\end{enumerate}
$$
\begin{aligned}
&l_{1}:\left\{\begin{array}{l}
x+y-2=0 \\
2 x+z+3=0
\end{array}\right. \\
&l_{2}:\left\{\begin{array}{l}
2 x+y-2=0 \\
3 y+4 z-4=0
\end{array}\right.
\end{aligned}
$$
(1) \begin{CJK}{UTF8}{mj}试判定\end{CJK} $l_{1}$ \begin{CJK}{UTF8}{mj}与\end{CJK} $l_{2}$ \begin{CJK}{UTF8}{mj}的位置关系\end{CJK}, \begin{CJK}{UTF8}{mj}并求出它们的夹角\end{CJK}.

(2) \begin{CJK}{UTF8}{mj}若\end{CJK} $l_{1}$ \begin{CJK}{UTF8}{mj}与\end{CJK} $l_{2}$ \begin{CJK}{UTF8}{mj}共面\end{CJK}, \begin{CJK}{UTF8}{mj}求它们所在平面的方程\end{CJK}; \begin{CJK}{UTF8}{mj}若\end{CJK} $l_{1}$ \begin{CJK}{UTF8}{mj}与\end{CJK} $l_{2}$ \begin{CJK}{UTF8}{mj}异面\end{CJK}, \begin{CJK}{UTF8}{mj}求它们的公垂线方程\end{CJK}. 2. ( 15 \begin{CJK}{UTF8}{mj}分\end{CJK}) \begin{CJK}{UTF8}{mj}证明对于单叶双曲面上的任意一条直母线\end{CJK}, \begin{CJK}{UTF8}{mj}在该单叶双曲面上一定存在另外一条与之平行的直母线\end{CJK}.

\begin{enumerate}
  \setcounter{enumi}{3}
  \item ( 20 \begin{CJK}{UTF8}{mj}分\end{CJK}) \begin{CJK}{UTF8}{mj}试求空间曲线\end{CJK}
\end{enumerate}
$$
\left\{\begin{array}{l}
x^{2}+z^{2}-y^{2}=0 \\
2 x-y^{2}+3=0
\end{array}\right.
$$
\begin{CJK}{UTF8}{mj}在坐标面\end{CJK} $x O z$ \begin{CJK}{UTF8}{mj}面与\end{CJK} $x o y$ \begin{CJK}{UTF8}{mj}面上的射影曲线方程\end{CJK}, \begin{CJK}{UTF8}{mj}并画出该空间曲线在第一\end{CJK}, \begin{CJK}{UTF8}{mj}第二卦限部分的图形\end{CJK}, \begin{CJK}{UTF8}{mj}写出作图步\end{CJK} \begin{CJK}{UTF8}{mj}骤\end{CJK}.

\section{9. 东北师范大学 2010 年研究生入学考试试题数学分析}
\begin{CJK}{UTF8}{mj}一\end{CJK}. \begin{CJK}{UTF8}{mj}计算题\end{CJK} (\begin{CJK}{UTF8}{mj}每题\end{CJK} 10 \begin{CJK}{UTF8}{mj}分\end{CJK}, \begin{CJK}{UTF8}{mj}共\end{CJK} 50 \begin{CJK}{UTF8}{mj}分\end{CJK})

$1 .$
$$
\lim _{n \rightarrow \infty}\left(\frac{1}{n^{2}+n+1}+\frac{1}{n^{2}+n+2}+\frac{1}{n^{2}+n+n}\right)
$$
$$
\lim _{x \rightarrow 0} \frac{\int_{0}^{x^{2}} e^{t^{2}} \mathrm{~d} t}{\int_{0}^{x}(2+x) e^{2 t^{2}} \mathrm{~d} t}
$$
$3 .$
$$
\int_{L}\left(2 x y^{3}-y^{2} \cos x+1\right) \mathrm{d} x+\left(3-2 y \sin x+3 x^{2} y^{2}\right) \mathrm{d} y .
$$
\begin{CJK}{UTF8}{mj}其中\end{CJK} $L$ \begin{CJK}{UTF8}{mj}为抛物线\end{CJK} $2 x=\pi y^{2}$ \begin{CJK}{UTF8}{mj}由\end{CJK} $(0,0)$ \begin{CJK}{UTF8}{mj}到\end{CJK} $\left(\frac{\pi}{2}, 1\right)$ \begin{CJK}{UTF8}{mj}的一段弧\end{CJK}.

\begin{enumerate}
  \setcounter{enumi}{4}
  \item \begin{CJK}{UTF8}{mj}求\end{CJK}
\end{enumerate}
$$
I=\oint_{L^{+}} \frac{x \mathrm{~d} y-y \mathrm{~d} x}{4 x^{2}+y^{2}}
$$
\begin{CJK}{UTF8}{mj}其中\end{CJK} $L$ \begin{CJK}{UTF8}{mj}为以\end{CJK} $(1,0)$ \begin{CJK}{UTF8}{mj}为圆心\end{CJK}, $R$ \begin{CJK}{UTF8}{mj}为半径的圆周\end{CJK} $(R \neq 1), L^{+}$\begin{CJK}{UTF8}{mj}表示逆时针方向\end{CJK}.

\begin{enumerate}
  \setcounter{enumi}{5}
  \item \begin{CJK}{UTF8}{mj}求\end{CJK}
\end{enumerate}
$$
\iiint_{x^{2}+y^{2}-z^{2} \leqslant 1}\left(x^{2}+y^{2}-z^{2}\right) \mathrm{d} x \mathrm{~d} y \mathrm{~d} z .
$$
\begin{CJK}{UTF8}{mj}二\end{CJK}. ( 20 \begin{CJK}{UTF8}{mj}分\end{CJK}) \begin{CJK}{UTF8}{mj}设\end{CJK} $y \geqslant a>0$, \begin{CJK}{UTF8}{mj}证明\end{CJK}: \begin{CJK}{UTF8}{mj}积分\end{CJK}
$$
\int_{0}^{+\infty} \frac{\cos x y}{x^{2}+y^{2}} \mathrm{~d} x
$$
\begin{CJK}{UTF8}{mj}一致收敛\end{CJK}.

\begin{CJK}{UTF8}{mj}三\end{CJK}. ( 20 \begin{CJK}{UTF8}{mj}分\end{CJK} $)$ \begin{CJK}{UTF8}{mj}已知\end{CJK}
$$
f(x, y)= \begin{cases}(x+y)^{2} \cos \frac{1}{x^{2}+y^{2}}, & x^{2}+y^{2} \neq 0 \\ 0, & x^{2}+y^{2}=0\end{cases}
$$
(1) \begin{CJK}{UTF8}{mj}求\end{CJK} $\frac{\partial f}{\partial x}, \frac{\partial f}{\partial y}$;

(2) $\frac{\partial f}{\partial x}, \frac{\partial f}{\partial y}$ \begin{CJK}{UTF8}{mj}是否在原点连续\end{CJK}?

(3) $f(x, y)$ \begin{CJK}{UTF8}{mj}在原点是否可微\end{CJK}, \begin{CJK}{UTF8}{mj}并说明理由\end{CJK}.

\begin{CJK}{UTF8}{mj}四\end{CJK}. (20\begin{CJK}{UTF8}{mj}分\end{CJK}) \begin{CJK}{UTF8}{mj}设\end{CJK} $f(x)$ \begin{CJK}{UTF8}{mj}在\end{CJK} $[0,1]$ \begin{CJK}{UTF8}{mj}上连续\end{CJK}, \begin{CJK}{UTF8}{mj}在\end{CJK} $(0,1)$ \begin{CJK}{UTF8}{mj}内二阶可导\end{CJK}, \begin{CJK}{UTF8}{mj}过点\end{CJK} $A(0, f(0))$ \begin{CJK}{UTF8}{mj}和\end{CJK} $B(1, f(1))$ \begin{CJK}{UTF8}{mj}的直线与曲线\end{CJK} $y=f(x)$ \begin{CJK}{UTF8}{mj}相交于点\end{CJK} $C(c, f(c))$, \begin{CJK}{UTF8}{mj}其中\end{CJK} $0<c<1$. \begin{CJK}{UTF8}{mj}证明\end{CJK}: \begin{CJK}{UTF8}{mj}在\end{CJK} $(0,1)$ \begin{CJK}{UTF8}{mj}内至少存在一点\end{CJK} $\ell$, \begin{CJK}{UTF8}{mj}使得\end{CJK}
$$
f^{\prime \prime}(\ell)=0
$$
\begin{CJK}{UTF8}{mj}五\end{CJK}. ( 20 \begin{CJK}{UTF8}{mj}分\end{CJK}) \begin{CJK}{UTF8}{mj}设\end{CJK} $f_{0}(x)$ \begin{CJK}{UTF8}{mj}在\end{CJK} $[a, b]$ \begin{CJK}{UTF8}{mj}上连续\end{CJK}, $g(x, y)$ \begin{CJK}{UTF8}{mj}在闭区域\end{CJK} $D=[a, b] \times[a, b]$ \begin{CJK}{UTF8}{mj}上连续\end{CJK}, \begin{CJK}{UTF8}{mj}且对任意的\end{CJK} $x \in[a, b]$, \begin{CJK}{UTF8}{mj}令\end{CJK}
$$
f_{n}(x)=\int_{a}^{x} g(x, t) f_{n-1}(t) \mathrm{d} t, n=1,2, \cdots,
$$
\begin{CJK}{UTF8}{mj}证明\end{CJK}: \begin{CJK}{UTF8}{mj}函数列\end{CJK} $\left\{f_{n}(x)\right\}$ \begin{CJK}{UTF8}{mj}在\end{CJK} $[a, b]$ \begin{CJK}{UTF8}{mj}上一致连续\end{CJK}.

\begin{CJK}{UTF8}{mj}六\end{CJK}. (20\begin{CJK}{UTF8}{mj}分\end{CJK}) \begin{CJK}{UTF8}{mj}是否存在\end{CJK} $\mathbb{R}$ \begin{CJK}{UTF8}{mj}上的连续函数\end{CJK} $f(x)$, \begin{CJK}{UTF8}{mj}使得\end{CJK} $f(f(x))=-x$ ? \begin{CJK}{UTF8}{mj}证明你的结论\end{CJK}.

\section{0. 东北师范大学 2011 年研究生入学考试试题数学分析 
 李扬 
 微信公众号: sxkyliyang}
\begin{CJK}{UTF8}{mj}一\end{CJK}. \begin{CJK}{UTF8}{mj}计算下列各题\end{CJK} (\begin{CJK}{UTF8}{mj}每题\end{CJK} 10 \begin{CJK}{UTF8}{mj}分\end{CJK}, \begin{CJK}{UTF8}{mj}共\end{CJK} 50 \begin{CJK}{UTF8}{mj}分\end{CJK})
$$
\lim _{x \rightarrow \infty} \frac{\left(\int_{0}^{x} e^{t^{2}} \mathrm{~d} t\right)^{2}}{\int_{0}^{x} t e^{2 t^{2}} \mathrm{~d} t}
$$
2 .
$$
\int_{0}^{\pi}(x \sin x)^{2} \mathrm{~d} x
$$
$3 .$
$$
\iiint_{x^{2}+y^{2}+z^{2} \leqslant 1}\left(x^{2}+y^{2}+z^{2}\right) \mathrm{d} x \mathrm{~d} y \mathrm{~d} z .
$$

\begin{enumerate}
  \setcounter{enumi}{4}
  \item \begin{CJK}{UTF8}{mj}设\end{CJK}
\end{enumerate}
$$
x_{n}=(1+a)\left(1+a^{2}\right) \cdots\left(1+a^{2^{n}}\right) .
$$
\begin{CJK}{UTF8}{mj}其中\end{CJK} $-1<a<1$, \begin{CJK}{UTF8}{mj}求\end{CJK} $\lim _{n \rightarrow \infty} x_{n}$.

\begin{enumerate}
  \setcounter{enumi}{5}
  \item \begin{CJK}{UTF8}{mj}求\end{CJK}
\end{enumerate}
$$
f(x, y, z)=x^{2}+y^{2}+z^{2}-x z-y z-3 x+y+4 z+7
$$
\begin{CJK}{UTF8}{mj}的极值点和极值\end{CJK}.

\begin{CJK}{UTF8}{mj}二\end{CJK}. ( 20 \begin{CJK}{UTF8}{mj}分\end{CJK}) \begin{CJK}{UTF8}{mj}证明\end{CJK}: \begin{CJK}{UTF8}{mj}若函数\end{CJK} $f(x)$ \begin{CJK}{UTF8}{mj}在\end{CJK} $[0,1]$ \begin{CJK}{UTF8}{mj}上二阶可导\end{CJK}, \begin{CJK}{UTF8}{mj}且\end{CJK} $\forall x \in[0,1]$, \begin{CJK}{UTF8}{mj}有\end{CJK} $\left|f^{\prime \prime}(x)\right| \leqslant 1$, \begin{CJK}{UTF8}{mj}又\end{CJK} $f(x)$ \begin{CJK}{UTF8}{mj}在\end{CJK} $(0,1)$ \begin{CJK}{UTF8}{mj}内取到最大\end{CJK} \begin{CJK}{UTF8}{mj}值\end{CJK}, \begin{CJK}{UTF8}{mj}则有\end{CJK}
$$
\left|f^{\prime}(0)\right|+\left|f^{\prime}(1)\right| \leqslant 1
$$
\begin{CJK}{UTF8}{mj}三\end{CJK}. ( 20 \begin{CJK}{UTF8}{mj}分\end{CJK}) \begin{CJK}{UTF8}{mj}设\end{CJK} $f(x)$ \begin{CJK}{UTF8}{mj}在\end{CJK} $[a, b]$ \begin{CJK}{UTF8}{mj}上连续可微\end{CJK}, \begin{CJK}{UTF8}{mj}证明\end{CJK}:
$$
\max _{a \leqslant x \leqslant b}|f(x)| \leqslant\left|\frac{1}{b-a} \int_{a}^{b} f(x) \mathrm{d} x\right|+\int_{a}^{b}\left|f^{\prime}(x)\right| \mathrm{d} x .
$$
\begin{CJK}{UTF8}{mj}四\end{CJK}. ( 15 \begin{CJK}{UTF8}{mj}分\end{CJK}) \begin{CJK}{UTF8}{mj}设函数\end{CJK} $f(x)$ \begin{CJK}{UTF8}{mj}连续\end{CJK},
$$
g(x)=\int_{0}^{1} f(t x) \mathrm{d} t
$$
H
$$
\lim _{x \rightarrow 0} \frac{f(x)}{x}=A .
$$
$A$ \begin{CJK}{UTF8}{mj}为常数\end{CJK}, \begin{CJK}{UTF8}{mj}求\end{CJK} $g^{\prime}(x)$ \begin{CJK}{UTF8}{mj}并讨论\end{CJK} $g^{\prime}(x)$ \begin{CJK}{UTF8}{mj}在\end{CJK} $x=0$ \begin{CJK}{UTF8}{mj}处的连续性\end{CJK}.

\begin{CJK}{UTF8}{mj}五\end{CJK}. ( 15 \begin{CJK}{UTF8}{mj}分\end{CJK}) \begin{CJK}{UTF8}{mj}设\end{CJK} $S$ \begin{CJK}{UTF8}{mj}是围成有界立体\end{CJK} $V$ \begin{CJK}{UTF8}{mj}的光滑闭曲面\end{CJK} (\begin{CJK}{UTF8}{mj}外法线方向为正向\end{CJK}), $n$ \begin{CJK}{UTF8}{mj}为曲面\end{CJK} $S$ \begin{CJK}{UTF8}{mj}的外法线方向\end{CJK}, $u, v$ \begin{CJK}{UTF8}{mj}在\end{CJK} $\bar{V}$ \begin{CJK}{UTF8}{mj}上\end{CJK} \begin{CJK}{UTF8}{mj}二阶连续可微\end{CJK}, \begin{CJK}{UTF8}{mj}证明\end{CJK}:
$$
\oiint_{S} v \frac{\partial u}{\partial n} \mathrm{~d} \sigma=\iiint_{V} v \Delta u \mathrm{~d} x \mathrm{~d} y \mathrm{~d} z+\iiint_{V} \nabla u \cdot \nabla v \mathrm{~d} x \mathrm{~d} y \mathrm{~d} z .
$$
\begin{CJK}{UTF8}{mj}六\end{CJK}. (15 \begin{CJK}{UTF8}{mj}分\end{CJK}) \begin{CJK}{UTF8}{mj}证明\end{CJK}: \begin{CJK}{UTF8}{mj}含参变量的无穷积分\end{CJK}
$$
I(y)=\int_{1}^{+\infty} \frac{\sin x}{1+x e^{y}} \mathrm{~d} x
$$
\begin{CJK}{UTF8}{mj}在\end{CJK} $[0,+\infty)$ \begin{CJK}{UTF8}{mj}上一致收敛\end{CJK}. \begin{CJK}{UTF8}{mj}七\end{CJK}. ( 15 \begin{CJK}{UTF8}{mj}分\end{CJK}) \begin{CJK}{UTF8}{mj}设函数\end{CJK} $\varphi(x)$ \begin{CJK}{UTF8}{mj}具有连续的导数\end{CJK}, \begin{CJK}{UTF8}{mj}在围绕原点的任意光滑的简单闭曲线\end{CJK} $C$ \begin{CJK}{UTF8}{mj}上\end{CJK}, \begin{CJK}{UTF8}{mj}曲线积分\end{CJK}
$$
\oint_{C} \frac{2 x y \mathrm{~d} x+\varphi(x) \mathrm{d} y}{x^{4}+y^{2}}
$$
\begin{CJK}{UTF8}{mj}的值为常数\end{CJK}.

(1) \begin{CJK}{UTF8}{mj}设\end{CJK} $L$ \begin{CJK}{UTF8}{mj}为正向闭曲线\end{CJK} $(x-2)^{2}+y^{2}=1$, \begin{CJK}{UTF8}{mj}证明\end{CJK}: $\oint_{L} \frac{2 x y \mathrm{~d} x+\varphi(x) \mathrm{d} y}{x^{4}+y^{2}}=0$.

(2) \begin{CJK}{UTF8}{mj}求函数\end{CJK} $\varphi(x)$.

(3) \begin{CJK}{UTF8}{mj}设\end{CJK} $C$ \begin{CJK}{UTF8}{mj}是围绕原点的光滑简单正向闭曲线\end{CJK}, \begin{CJK}{UTF8}{mj}求\end{CJK} $\oint_{C} \frac{2 x y \mathrm{~d} x+\varphi(x) \mathrm{d} y}{x^{4}+y^{2}}$.

\section{1. 东北师范大学 2012 年研究生入学考试试题数学分析}
\begin{CJK}{UTF8}{mj}李扬\end{CJK}

\begin{CJK}{UTF8}{mj}微信公众号\end{CJK}: sxkyliyang

\begin{CJK}{UTF8}{mj}一\end{CJK}. \begin{CJK}{UTF8}{mj}计算题\end{CJK} (\begin{CJK}{UTF8}{mj}每题\end{CJK} 10 \begin{CJK}{UTF8}{mj}分\end{CJK}, \begin{CJK}{UTF8}{mj}共\end{CJK} 40 \begin{CJK}{UTF8}{mj}分\end{CJK})
$$
\lim _{x \rightarrow \infty}\left(\frac{x-1}{x+1}\right)^{2 x}
$$
$$
\int \cos \sqrt{x} \mathrm{~d} x
$$

\begin{enumerate}
  \setcounter{enumi}{3}
  \item \begin{CJK}{UTF8}{mj}求函数\end{CJK}
\end{enumerate}
$$
u=e^{x y} \cdot \cos z
$$
\begin{CJK}{UTF8}{mj}的全微分\end{CJK}.

\begin{enumerate}
  \setcounter{enumi}{4}
  \item \begin{CJK}{UTF8}{mj}已知\end{CJK} $\lim _{n \rightarrow+\infty} x_{n}=a$, \begin{CJK}{UTF8}{mj}求\end{CJK}
\end{enumerate}
$$
\lim _{n \rightarrow+\infty} \frac{x_{1}+x_{2}+\cdots+x_{n}}{n} .
$$
\begin{CJK}{UTF8}{mj}二\end{CJK}. (10 \begin{CJK}{UTF8}{mj}分\end{CJK}) \begin{CJK}{UTF8}{mj}已知\end{CJK}
$$
\left\{\begin{array}{l}
x=a \cos t \\
y=b \sin t
\end{array}\right.
$$
\begin{CJK}{UTF8}{mj}求\end{CJK} $\frac{\mathrm{d} y}{\mathrm{~d} x}$ \begin{CJK}{UTF8}{mj}和\end{CJK} $\frac{\mathrm{d}^{2} y}{\mathrm{~d} x^{2}}$.

\begin{CJK}{UTF8}{mj}三\end{CJK}. (10 \begin{CJK}{UTF8}{mj}分\end{CJK}) \begin{CJK}{UTF8}{mj}用\end{CJK} $\varepsilon-E$ \begin{CJK}{UTF8}{mj}语言写出极限\end{CJK} $\lim _{x \rightarrow+\infty} f(x)$ \begin{CJK}{UTF8}{mj}存在的柯西收敛原理\end{CJK}, \begin{CJK}{UTF8}{mj}并利用该原理证明\end{CJK}
$$
\lim _{x \rightarrow+\infty} \frac{\cos x}{x}
$$
\begin{CJK}{UTF8}{mj}存在极限\end{CJK}.

\begin{CJK}{UTF8}{mj}四\end{CJK}. (10 \begin{CJK}{UTF8}{mj}分\end{CJK}) \begin{CJK}{UTF8}{mj}计算\end{CJK}:

(1) \begin{CJK}{UTF8}{mj}曲线\end{CJK} $x y=1$ \begin{CJK}{UTF8}{mj}与直线\end{CJK} $x=1, x=2$ \begin{CJK}{UTF8}{mj}及\end{CJK} $y=0$ \begin{CJK}{UTF8}{mj}所围成的图形的面积\end{CJK}.

(2) \begin{CJK}{UTF8}{mj}上述图形绕\end{CJK} $O x$ \begin{CJK}{UTF8}{mj}轴一周所生成旋转体的体积\end{CJK}.

\begin{CJK}{UTF8}{mj}五\end{CJK}. (10 \begin{CJK}{UTF8}{mj}分\end{CJK}) \begin{CJK}{UTF8}{mj}若\end{CJK} $\lim _{n \rightarrow \infty} \frac{x_{n}-1}{x_{n}+1}=0$, \begin{CJK}{UTF8}{mj}证明\end{CJK}
$$
\lim _{n \rightarrow \infty} x_{n}=1 .
$$
\begin{CJK}{UTF8}{mj}六\end{CJK}. (10 \begin{CJK}{UTF8}{mj}分\end{CJK}) \begin{CJK}{UTF8}{mj}若函数\end{CJK} $f(x)$ \begin{CJK}{UTF8}{mj}在\end{CJK} $[a, b]$ \begin{CJK}{UTF8}{mj}上非负连续\end{CJK}, $\int_{a}^{b} f(x) \mathrm{d} x=0$. \begin{CJK}{UTF8}{mj}证明\end{CJK}: \begin{CJK}{UTF8}{mj}对于任意的\end{CJK} $x \in[a, b]$, \begin{CJK}{UTF8}{mj}都有\end{CJK}
$$
f(x)=0 .
$$
\begin{CJK}{UTF8}{mj}七\end{CJK}. (10 \begin{CJK}{UTF8}{mj}分\end{CJK}) \begin{CJK}{UTF8}{mj}证明函数\end{CJK}
$$
f(x, y)= \begin{cases}\frac{x y}{x^{2}+y^{2}}, & (x, y) \neq(0,0) \\ 1, & (x, y)=(0,0)\end{cases}
$$
\begin{CJK}{UTF8}{mj}在\end{CJK} $(0,0)$ \begin{CJK}{UTF8}{mj}点不连续\end{CJK}. \begin{CJK}{UTF8}{mj}八\end{CJK}. ( 10 \begin{CJK}{UTF8}{mj}分\end{CJK}) \begin{CJK}{UTF8}{mj}设函数\end{CJK} $f(x)$ \begin{CJK}{UTF8}{mj}在\end{CJK} $[a, b]$ \begin{CJK}{UTF8}{mj}连续\end{CJK}, \begin{CJK}{UTF8}{mj}证明\end{CJK}: \begin{CJK}{UTF8}{mj}若对\end{CJK} $[a, b]$ \begin{CJK}{UTF8}{mj}上满足\end{CJK} $\int_{a}^{b} \varphi(x) \mathrm{d} x=0$ \begin{CJK}{UTF8}{mj}的任意连续函数\end{CJK} $\varphi(x)$, \begin{CJK}{UTF8}{mj}有\end{CJK}
$$
\int_{a}^{b} f(x) \varphi(x) \mathrm{d} x=0
$$
\begin{CJK}{UTF8}{mj}则\end{CJK} $f(x)$ \begin{CJK}{UTF8}{mj}是常值函数\end{CJK}.

\begin{CJK}{UTF8}{mj}九\end{CJK}. (10 \begin{CJK}{UTF8}{mj}分\end{CJK}) \begin{CJK}{UTF8}{mj}验证函数组\end{CJK}
$$
\left\{\begin{array}{l}
x+y+z=0 \\
x^{2}+y^{2}+z^{2}=1
\end{array}\right.
$$
\begin{CJK}{UTF8}{mj}在点\end{CJK} $\left(\frac{1}{\sqrt{2}}, \frac{-1}{\sqrt{2}}, 0\right)$ \begin{CJK}{UTF8}{mj}邻域内存在隐函数组\end{CJK} $x=x(z), y=y(z)$, \begin{CJK}{UTF8}{mj}并求\end{CJK} $\frac{\mathrm{d} x}{\mathrm{~d} z}, \frac{\mathrm{d} y}{\mathrm{~d} z}$

\begin{CJK}{UTF8}{mj}十\end{CJK}. (30 \begin{CJK}{UTF8}{mj}分\end{CJK}) \begin{CJK}{UTF8}{mj}讨论下列广义积分的敛散性\end{CJK}:

(1)
$$
\int_{0}^{1} \frac{x}{x^{3}+x^{2}+1} \mathrm{~d} x
$$
(2)
$$
\int_{0}^{1} \frac{x^{\alpha}}{\sqrt{1-x^{2}}} \mathrm{~d} x(-1<\alpha<0)
$$

\section{2. 东北师范大学 2013 年研究生入学考试试题数学分析}
\begin{CJK}{UTF8}{mj}李扬\end{CJK}

\begin{CJK}{UTF8}{mj}微信公众号\end{CJK}: sxkyliyang

\begin{CJK}{UTF8}{mj}一\end{CJK}. (15 \begin{CJK}{UTF8}{mj}分\end{CJK}) \begin{CJK}{UTF8}{mj}设\end{CJK} $a_{1}, a_{2}, a_{3}, a_{4}$ \begin{CJK}{UTF8}{mj}为已知大于零的数\end{CJK}, \begin{CJK}{UTF8}{mj}求函数\end{CJK}
$$
f\left(x_{1}, x_{2}, x_{3}, x_{4}\right)=a_{1} \ln x_{1}+a_{2} \ln x_{2}+a_{3} \ln x_{3}+a_{4} \ln x_{4}
$$
\begin{CJK}{UTF8}{mj}在限制条件\end{CJK} $x_{1}+x_{2}+x_{3}+x_{4}=M$ ( $M$ \begin{CJK}{UTF8}{mj}为大于零的实数\end{CJK})\begin{CJK}{UTF8}{mj}下的最大值点\end{CJK}.

\begin{CJK}{UTF8}{mj}二\end{CJK}. (15 \begin{CJK}{UTF8}{mj}分\end{CJK}) \begin{CJK}{UTF8}{mj}求积分\end{CJK}
$$
\int \frac{\sin x}{\sin x+\cos x} d x
$$
\begin{CJK}{UTF8}{mj}三\end{CJK}. ( 15 \begin{CJK}{UTF8}{mj}分\end{CJK}) \begin{CJK}{UTF8}{mj}求\end{CJK}
$$
I=\int_{L} \frac{x \mathrm{~d} y-y \mathrm{~d} x}{4 x^{2}+9 y^{2}}
$$
$L$ \begin{CJK}{UTF8}{mj}是取反时针方向的单位圆周\end{CJK}.

\begin{CJK}{UTF8}{mj}四\end{CJK}. (15 \begin{CJK}{UTF8}{mj}分\end{CJK}) \begin{CJK}{UTF8}{mj}证明广义积分\end{CJK}
$$
\int_{0}^{+\infty} \frac{\sin x y}{x} \mathrm{~d} x
$$
\begin{CJK}{UTF8}{mj}关于\end{CJK} $y$ \begin{CJK}{UTF8}{mj}在\end{CJK} $\left[y_{0},+\infty\right)\left(y_{0}>0\right)$ \begin{CJK}{UTF8}{mj}上一致收敛\end{CJK}.

\begin{CJK}{UTF8}{mj}五\end{CJK}. (15 \begin{CJK}{UTF8}{mj}分\end{CJK}) \begin{CJK}{UTF8}{mj}计算\end{CJK}
$$
I=\int_{L}\left(y^{2}-z^{2}\right) \mathrm{d} x+\left(z^{2}-x^{2}\right) \mathrm{d} y+\left(x^{2}-y^{2}\right) \mathrm{d} z
$$
\begin{CJK}{UTF8}{mj}其中\end{CJK} $L$ \begin{CJK}{UTF8}{mj}为平面\end{CJK} $x+y+z=1$ \begin{CJK}{UTF8}{mj}被三个坐标平面所截三角形的边界\end{CJK}, \begin{CJK}{UTF8}{mj}若从\end{CJK} $x$ \begin{CJK}{UTF8}{mj}轴的正向看去\end{CJK}, \begin{CJK}{UTF8}{mj}定向为逆时针方向\end{CJK}.

\begin{CJK}{UTF8}{mj}六\end{CJK}. (15 \begin{CJK}{UTF8}{mj}分\end{CJK}) \begin{CJK}{UTF8}{mj}计算由平面\end{CJK} $z=2$ \begin{CJK}{UTF8}{mj}与旋转抛物面\end{CJK} $x^{2}+y^{2}=2 z$ \begin{CJK}{UTF8}{mj}所围成立体的体积\end{CJK}.

\begin{CJK}{UTF8}{mj}七\end{CJK}. ( 15 \begin{CJK}{UTF8}{mj}分\end{CJK}) \begin{CJK}{UTF8}{mj}令\end{CJK}
$$
f(x, y)= \begin{cases}\frac{x^{2} y}{x^{2}+y^{2}}, & x^{2}+y^{2} \neq 0 \\ 0, & x^{2}+y^{2}=0\end{cases}
$$
(1) \begin{CJK}{UTF8}{mj}求证函数在\end{CJK} $(0,0)$ \begin{CJK}{UTF8}{mj}点连续\end{CJK};

(2) \begin{CJK}{UTF8}{mj}求\end{CJK} $f_{x}^{\prime}(0,0), f_{y}^{\prime}(0,0)$;

(3) \begin{CJK}{UTF8}{mj}判断函数在\end{CJK} $(0,0)$ \begin{CJK}{UTF8}{mj}点是否可微\end{CJK}, \begin{CJK}{UTF8}{mj}并说明理由\end{CJK}.

\begin{CJK}{UTF8}{mj}八\end{CJK}. (15\begin{CJK}{UTF8}{mj}分\end{CJK}) \begin{CJK}{UTF8}{mj}设\end{CJK} $f(x)$ \begin{CJK}{UTF8}{mj}在\end{CJK} $(0,+\infty)$ \begin{CJK}{UTF8}{mj}上存在一阶导数\end{CJK}, \begin{CJK}{UTF8}{mj}试证明\end{CJK}: \begin{CJK}{UTF8}{mj}若\end{CJK} $f^{\prime}(x)$ \begin{CJK}{UTF8}{mj}单调下降\end{CJK}, \begin{CJK}{UTF8}{mj}则\end{CJK} $f(x)-x f^{\prime}(x)$ \begin{CJK}{UTF8}{mj}单调上升\end{CJK}.

\begin{CJK}{UTF8}{mj}九\end{CJK}. (15\begin{CJK}{UTF8}{mj}分\end{CJK}) \begin{CJK}{UTF8}{mj}试证明\end{CJK}: \begin{CJK}{UTF8}{mj}函数\end{CJK} $f(x)=\sqrt{x} \ln (x)$ \begin{CJK}{UTF8}{mj}在\end{CJK} $(0,+\infty)$ \begin{CJK}{UTF8}{mj}上是一致连续的\end{CJK}.

\begin{CJK}{UTF8}{mj}十\end{CJK}. (15\begin{CJK}{UTF8}{mj}分\end{CJK}) \begin{CJK}{UTF8}{mj}证明不等式\end{CJK}:
$$
\frac{b-a}{b}<\ln \frac{b}{a}<\frac{b-a}{a}(0<a<b)
$$

\section{3. 东北师范大学 2014 年研究生入学考试试题数学分析}
\begin{CJK}{UTF8}{mj}李扬\end{CJK}

\begin{CJK}{UTF8}{mj}微信公众号\end{CJK}: sxkyliyang

\begin{CJK}{UTF8}{mj}一\end{CJK}. (15 \begin{CJK}{UTF8}{mj}分\end{CJK}) \begin{CJK}{UTF8}{mj}求极限\end{CJK}
$$
\lim _{x \rightarrow 0} \frac{\int_{0}^{\sin ^{2} x} \ln (1+t) \mathrm{d} t}{\sqrt{1+x^{4}}-1} .
$$
\begin{CJK}{UTF8}{mj}二\end{CJK}. ( 15 \begin{CJK}{UTF8}{mj}分\end{CJK}) \begin{CJK}{UTF8}{mj}求函数\end{CJK}
$$
f(x)= \begin{cases}x^{2} \sin \frac{1}{x}, & x \neq 0 \\ 0, & x=0\end{cases}
$$
\begin{CJK}{UTF8}{mj}的导数\end{CJK} $f^{\prime}(x)$, \begin{CJK}{UTF8}{mj}并讨论\end{CJK} $f^{\prime}(x)$ \begin{CJK}{UTF8}{mj}在\end{CJK} $(-\infty,+\infty)$ \begin{CJK}{UTF8}{mj}内的连续性\end{CJK}.

\begin{CJK}{UTF8}{mj}三\end{CJK}. ( 20 \begin{CJK}{UTF8}{mj}分\end{CJK} $)$ \begin{CJK}{UTF8}{mj}计算第二型曲面积分\end{CJK}
$$
\iint_{\Sigma} \frac{x \mathrm{~d} y \mathrm{~d} z+y \mathrm{~d} z \mathrm{~d} x+z \mathrm{~d} x \mathrm{~d} y}{\left(x^{2}+y^{2}+z^{2}\right)^{\frac{3}{2}}} .
$$
\begin{CJK}{UTF8}{mj}其中曲面\end{CJK} $\Sigma$ \begin{CJK}{UTF8}{mj}是球面\end{CJK} $x^{2}+y^{2}+z^{2}=R^{2}(R>0)$, \begin{CJK}{UTF8}{mj}取外侧\end{CJK}.

\begin{CJK}{UTF8}{mj}四\end{CJK}. ( 20 \begin{CJK}{UTF8}{mj}分\end{CJK}) \begin{CJK}{UTF8}{mj}设函数\end{CJK} $f(x)$ \begin{CJK}{UTF8}{mj}在\end{CJK} $[a, b]$ \begin{CJK}{UTF8}{mj}上连续\end{CJK}, \begin{CJK}{UTF8}{mj}求证\end{CJK}: $f(x)$ \begin{CJK}{UTF8}{mj}在\end{CJK} $[a, b]$ \begin{CJK}{UTF8}{mj}上能取到最大值和最小值\end{CJK}.

\begin{CJK}{UTF8}{mj}五\end{CJK}. (20 \begin{CJK}{UTF8}{mj}分\end{CJK}) \begin{CJK}{UTF8}{mj}设\end{CJK}
$$
A>0, x_{1}>0, x_{n+1}=\frac{1}{2}\left(x_{n}+\frac{A}{x_{n}}\right), n=1,2, \cdots
$$
\begin{CJK}{UTF8}{mj}求证\end{CJK}: \begin{CJK}{UTF8}{mj}数列\end{CJK} $\left\{x_{n}\right\}$ \begin{CJK}{UTF8}{mj}存在极限\end{CJK}, \begin{CJK}{UTF8}{mj}并求出该极限\end{CJK}.

\begin{CJK}{UTF8}{mj}六\end{CJK}. ( 20 \begin{CJK}{UTF8}{mj}分\end{CJK}) \begin{CJK}{UTF8}{mj}设函数\end{CJK} $F(x, y, z)$ \begin{CJK}{UTF8}{mj}有一阶连续偏导数\end{CJK}, \begin{CJK}{UTF8}{mj}函数\end{CJK} $u=u(x, y, z)$ \begin{CJK}{UTF8}{mj}由方程\end{CJK} $F\left(u^{2}-x^{2}, u^{2}-y^{2}, u^{2}-z^{2}\right)=0$ \begin{CJK}{UTF8}{mj}确\end{CJK} \begin{CJK}{UTF8}{mj}定\end{CJK}. \begin{CJK}{UTF8}{mj}试证\end{CJK}:
$$
\frac{1}{x} \frac{\partial u}{\partial x}+\frac{1}{y} \frac{\partial u}{\partial y}+\frac{1}{z} \frac{\partial u}{\partial z}=\frac{1}{u}
$$
\begin{CJK}{UTF8}{mj}七\end{CJK}. (20 \begin{CJK}{UTF8}{mj}分\end{CJK}) \begin{CJK}{UTF8}{mj}判定积分\end{CJK}
$$
I(\alpha)=\int_{0}^{+\infty} \sqrt{\alpha} e^{-\alpha x^{2}} d x
$$
\begin{CJK}{UTF8}{mj}对\end{CJK} $\alpha \geqslant 0$ \begin{CJK}{UTF8}{mj}的一致收敛性\end{CJK}.

\begin{CJK}{UTF8}{mj}八\end{CJK}. ( 20 \begin{CJK}{UTF8}{mj}分\end{CJK}) \begin{CJK}{UTF8}{mj}求证\end{CJK}:\begin{CJK}{UTF8}{mj}函数\end{CJK}
$$
\zeta(x)=\sum_{n=1}^{+\infty} \frac{1}{n^{x}}
$$
\begin{CJK}{UTF8}{mj}在\end{CJK} $(1,+\infty)$ \begin{CJK}{UTF8}{mj}内连续且无穷次可导\end{CJK}.

\section{4. 东北师范大学 2015 年研究生入学考试试题数学分析}
\begin{CJK}{UTF8}{mj}李扬\end{CJK}

\begin{CJK}{UTF8}{mj}微信公众号\end{CJK}: sxkyliyang

\begin{CJK}{UTF8}{mj}一\end{CJK}. (15 \begin{CJK}{UTF8}{mj}分\end{CJK}) \begin{CJK}{UTF8}{mj}求极限\end{CJK}
$$
\lim _{x \rightarrow 0} \frac{x-\int_{0}^{x} e^{t^{2}} \mathrm{~d} t}{x^{2} \sin (2 x)} .
$$
\begin{CJK}{UTF8}{mj}二\end{CJK}. (15 \begin{CJK}{UTF8}{mj}分\end{CJK}) \begin{CJK}{UTF8}{mj}试确定\end{CJK} $a$ \begin{CJK}{UTF8}{mj}和\end{CJK} $b$ \begin{CJK}{UTF8}{mj}的值\end{CJK}, \begin{CJK}{UTF8}{mj}使函数\end{CJK}
$$
f(x)= \begin{cases}e^{x} \cos x, & x<0 \\ a \sin x+b, & x \geqslant 0\end{cases}
$$
\begin{CJK}{UTF8}{mj}在\end{CJK} $(-\infty,+\infty)$ \begin{CJK}{UTF8}{mj}内连续可微\end{CJK}.

\begin{CJK}{UTF8}{mj}三\end{CJK}. ( 20 \begin{CJK}{UTF8}{mj}分\end{CJK}) \begin{CJK}{UTF8}{mj}计算第二型曲线积分\end{CJK}
$$
\oint_{C} \frac{x \mathrm{~d} y-y \mathrm{~d} x}{4 x^{2}+y^{2}} .
$$
\begin{CJK}{UTF8}{mj}其中\end{CJK} $C$ \begin{CJK}{UTF8}{mj}是平面上以\end{CJK} $(1,0)$ \begin{CJK}{UTF8}{mj}为圆心以\end{CJK} $R(R \neq 1)$ \begin{CJK}{UTF8}{mj}为半径的圆周\end{CJK}, \begin{CJK}{UTF8}{mj}取逆时针方向\end{CJK}.

\begin{CJK}{UTF8}{mj}四\end{CJK}. ( 20 \begin{CJK}{UTF8}{mj}分\end{CJK}) \begin{CJK}{UTF8}{mj}证明闭区间\end{CJK} $[a, b]$ \begin{CJK}{UTF8}{mj}上连续的函数\end{CJK} $f(x)$ \begin{CJK}{UTF8}{mj}在该区间上是一致连续的\end{CJK}.

\begin{CJK}{UTF8}{mj}五\end{CJK}. (20 \begin{CJK}{UTF8}{mj}分\end{CJK}) \begin{CJK}{UTF8}{mj}设\end{CJK}
$$
x_{1}>0, x_{n+1}=1+\frac{x_{n}}{1+x_{n}}, n=1,2, \cdots .
$$
\begin{CJK}{UTF8}{mj}求证\end{CJK}: \begin{CJK}{UTF8}{mj}数列\end{CJK} $\left\{x_{n}\right\}$ \begin{CJK}{UTF8}{mj}存在极限\end{CJK}, \begin{CJK}{UTF8}{mj}并求出该极限\end{CJK}.

\begin{CJK}{UTF8}{mj}六\end{CJK}. (20 \begin{CJK}{UTF8}{mj}分\end{CJK}) \begin{CJK}{UTF8}{mj}设\end{CJK} $f(u)$ \begin{CJK}{UTF8}{mj}连续可微\end{CJK}, \begin{CJK}{UTF8}{mj}函数\end{CJK} $z=z(x, y)$ \begin{CJK}{UTF8}{mj}由方程\end{CJK} $x^{2}+y^{2}+z^{2}=y f\left(\frac{z}{y}\right)$ \begin{CJK}{UTF8}{mj}确定\end{CJK}. \begin{CJK}{UTF8}{mj}求证\end{CJK}:
$$
\left(x^{2}-y^{2}-z^{2}\right) \frac{\partial z}{\partial x}+2 x y \frac{\partial z}{\partial y}=2 x z
$$
\begin{CJK}{UTF8}{mj}七\end{CJK}. (15 \begin{CJK}{UTF8}{mj}分\end{CJK}) \begin{CJK}{UTF8}{mj}判定积分\end{CJK}
$$
I(\alpha)=\int_{0}^{+\infty} \frac{x}{2+x^{\alpha}} \mathrm{d} x
$$
\begin{CJK}{UTF8}{mj}在\end{CJK} $\alpha \in(2,+\infty)$ \begin{CJK}{UTF8}{mj}的连续性\end{CJK}.

\begin{CJK}{UTF8}{mj}八\end{CJK}. \begin{CJK}{UTF8}{mj}设函数\end{CJK} $f(x)$ \begin{CJK}{UTF8}{mj}在\end{CJK} $x_{0}$ \begin{CJK}{UTF8}{mj}的邻域\end{CJK} $U\left(x_{0}\right)$ \begin{CJK}{UTF8}{mj}内能展成幂级数\end{CJK}, $\left\{x_{n}\right\} \subset U\left(x_{0}\right), x_{n} \neq x_{0}, n=1,2, \cdots$, \begin{CJK}{UTF8}{mj}使得\end{CJK} $\lim _{n \rightarrow \infty} x_{n}=x_{0}$ \begin{CJK}{UTF8}{mj}且\end{CJK} $f\left(x_{n}\right)=0$, \begin{CJK}{UTF8}{mj}求证\end{CJK}:
$$
f(x) \equiv 0, x \in U\left(x_{0}\right)
$$

\begin{enumerate}
  \setcounter{enumi}{15}
  \item \begin{CJK}{UTF8}{mj}东北师范大学\end{CJK} 2016 \begin{CJK}{UTF8}{mj}年研究生入学考试试题数学分析\end{CJK}
\end{enumerate}
\begin{CJK}{UTF8}{mj}李扬\end{CJK}

\begin{CJK}{UTF8}{mj}微信公众号\end{CJK}: sxkyliyang

\begin{CJK}{UTF8}{mj}一\end{CJK}. (15 \begin{CJK}{UTF8}{mj}分\end{CJK}) \begin{CJK}{UTF8}{mj}求极限\end{CJK}
$$
\lim _{n \rightarrow+\infty} \sum_{k=n^{2}}^{(n+1)^{2}} \frac{1}{\sqrt{k}}
$$
\begin{CJK}{UTF8}{mj}二\end{CJK}. (15 \begin{CJK}{UTF8}{mj}分\end{CJK}) \begin{CJK}{UTF8}{mj}确定\end{CJK} $a$ \begin{CJK}{UTF8}{mj}和\end{CJK} $b$, \begin{CJK}{UTF8}{mj}使得\end{CJK}
$$
f(x)= \begin{cases}e^{x}, & x<0 \\ a x+b, & x \geqslant 0\end{cases}
$$
\begin{CJK}{UTF8}{mj}在\end{CJK} $(-\infty,+\infty)$ \begin{CJK}{UTF8}{mj}内连续可微\end{CJK}.

\begin{CJK}{UTF8}{mj}三\end{CJK}. ( 20 \begin{CJK}{UTF8}{mj}分\end{CJK} $)$ \begin{CJK}{UTF8}{mj}计算第二型曲线积分\end{CJK}
$$
\oint_{C} \frac{x \mathrm{~d} y-y \mathrm{~d} x}{x^{2}+y^{2}} .
$$
\begin{CJK}{UTF8}{mj}其中\end{CJK} $C$ \begin{CJK}{UTF8}{mj}是平面上不经过原点的简单闭曲线\end{CJK}, \begin{CJK}{UTF8}{mj}取逆时针方向\end{CJK}.

\begin{CJK}{UTF8}{mj}四\end{CJK}. ( 20 \begin{CJK}{UTF8}{mj}分\end{CJK}) \begin{CJK}{UTF8}{mj}设函数\end{CJK} $f(x)$ \begin{CJK}{UTF8}{mj}在\end{CJK} $[a, b]$ \begin{CJK}{UTF8}{mj}上局部有界\end{CJK}, \begin{CJK}{UTF8}{mj}即对任意的\end{CJK} $x \in[a, b]$, \begin{CJK}{UTF8}{mj}存在\end{CJK} $\delta>0$, \begin{CJK}{UTF8}{mj}使\end{CJK} $f(x)$ \begin{CJK}{UTF8}{mj}在\end{CJK} $(x-\delta, x+\delta) \cap[a, b]$ \begin{CJK}{UTF8}{mj}上有界\end{CJK}. \begin{CJK}{UTF8}{mj}求证\end{CJK}: $f(x)$ \begin{CJK}{UTF8}{mj}在\end{CJK} $[a, b]$ \begin{CJK}{UTF8}{mj}上有界\end{CJK}.

\begin{CJK}{UTF8}{mj}五\end{CJK}. ( 20 \begin{CJK}{UTF8}{mj}分\end{CJK}) \begin{CJK}{UTF8}{mj}设函数\end{CJK} $f(x)$ \begin{CJK}{UTF8}{mj}在\end{CJK} $[a, b]$ \begin{CJK}{UTF8}{mj}上二阶可导\end{CJK}, \begin{CJK}{UTF8}{mj}且\end{CJK} $f(a)=f(b)=0, f_{+}^{\prime}(a) \cdot f_{-}^{\prime}(b)>0$. \begin{CJK}{UTF8}{mj}求证\end{CJK}: \begin{CJK}{UTF8}{mj}存在\end{CJK} $\xi \in(a, b)$, \begin{CJK}{UTF8}{mj}使\end{CJK}
$$
f^{\prime \prime}(\xi)=0
$$
\begin{CJK}{UTF8}{mj}六\end{CJK}. ( 20 \begin{CJK}{UTF8}{mj}分\end{CJK}) \begin{CJK}{UTF8}{mj}设\end{CJK} $F(u, v)$ \begin{CJK}{UTF8}{mj}连续可微\end{CJK}, \begin{CJK}{UTF8}{mj}函数\end{CJK} $z=z(x, y)$ \begin{CJK}{UTF8}{mj}由方程\end{CJK} $F\left(x+\frac{z}{y}, y+\frac{z}{x}\right)=0$ \begin{CJK}{UTF8}{mj}确定\end{CJK}. \begin{CJK}{UTF8}{mj}求证\end{CJK}:
$$
x \frac{\partial z}{\partial x}+y \frac{\partial z}{\partial y}=z-x y
$$
\begin{CJK}{UTF8}{mj}七\end{CJK}. (20 \begin{CJK}{UTF8}{mj}分\end{CJK}) \begin{CJK}{UTF8}{mj}判定积分\end{CJK}
$$
I(p)=\int_{1}^{+\infty} \frac{x \sin (p x)}{1+x^{2}} \mathrm{~d} x
$$
\begin{CJK}{UTF8}{mj}对\end{CJK} $p \geqslant p_{0}>0$ \begin{CJK}{UTF8}{mj}的一致收敛性\end{CJK}.

\begin{CJK}{UTF8}{mj}八\end{CJK}. ( 20 \begin{CJK}{UTF8}{mj}分\end{CJK}) \begin{CJK}{UTF8}{mj}设函数列\end{CJK} $\left\{f_{n}(x)\right\}$ \begin{CJK}{UTF8}{mj}在\end{CJK} $[a, b]$ \begin{CJK}{UTF8}{mj}上收敛到函数\end{CJK} $f(x) \in C[a, b]$. \begin{CJK}{UTF8}{mj}求证\end{CJK}: $\left\{f_{n}(x)\right\}$ \begin{CJK}{UTF8}{mj}在\end{CJK} $[a, b]$ \begin{CJK}{UTF8}{mj}上一致收敛到\end{CJK} $f(x)$ \begin{CJK}{UTF8}{mj}的充分必要条件是\end{CJK}, \begin{CJK}{UTF8}{mj}对\end{CJK} $[a, b]$ \begin{CJK}{UTF8}{mj}中任意收玫点列\end{CJK} $x_{n} \rightarrow x_{0}(n \rightarrow \infty)$, \begin{CJK}{UTF8}{mj}有\end{CJK}
$$
\lim _{n \rightarrow \infty} f_{n}\left(x_{n}\right)=f\left(x_{0}\right) .
$$

\section{6. 东北师范大学 2017 年研究生入学考试试题数学分析 
 李扬 
 微信公众号: sxkyliyang}
\begin{CJK}{UTF8}{mj}一\end{CJK}. \begin{CJK}{UTF8}{mj}计算题\end{CJK} (\begin{CJK}{UTF8}{mj}每题\end{CJK} 10 \begin{CJK}{UTF8}{mj}分\end{CJK}, \begin{CJK}{UTF8}{mj}共\end{CJK} 70 \begin{CJK}{UTF8}{mj}分\end{CJK})

$1 .$
$$
\lim _{n \rightarrow \infty}\left(\sqrt{\frac{1}{n(n+1)}}+\sqrt{\frac{1}{n(n+2)}}+\cdots+\sqrt{\frac{1}{2 n^{2}}}\right)
$$
$2 .$
$$
\int_{0}^{+\infty} e^{-x} \cos (2 x) \mathrm{d} x
$$
3 .
$$
\sum_{n=1}^{+\infty} \frac{n^{2}}{2^{n}} x^{n}(|x|<2)
$$
$4 .$
$$
\int_{-\infty}^{+\infty} e^{-x^{2}} \mathrm{~d} x
$$
$5 .$
$$
\lim _{x \rightarrow 0} x\left[x^{-1}\right]
$$
$$
\lim _{x \rightarrow 0}\left(\frac{1}{x}-\frac{1}{\sin x}\right)
$$

\begin{enumerate}
  \setcounter{enumi}{7}
  \item \begin{CJK}{UTF8}{mj}求\end{CJK}
\end{enumerate}
$$
A=\left\{(x, y, z, u) \mid x^{2}+y^{2}+z^{2}+u^{2} \leqslant 1\right\}
$$
\begin{CJK}{UTF8}{mj}的体积\end{CJK}.

\begin{CJK}{UTF8}{mj}二\end{CJK}. ( 20 \begin{CJK}{UTF8}{mj}分\end{CJK}) \begin{CJK}{UTF8}{mj}设\end{CJK} $f(x)$ \begin{CJK}{UTF8}{mj}在\end{CJK} $[2,+\infty)$ \begin{CJK}{UTF8}{mj}单调下降且\end{CJK} $\int_{2}^{+\infty} f(x) \mathrm{d} x$ \begin{CJK}{UTF8}{mj}收敛\end{CJK}. \begin{CJK}{UTF8}{mj}求证\end{CJK}:
$$
f(x) \geqslant 0(x \geqslant 2) .
$$
$$
\int_{2}^{+\infty} f(x) \sin (x) \mathrm{d} x
$$
\begin{CJK}{UTF8}{mj}收敛\end{CJK}.

\begin{CJK}{UTF8}{mj}三\end{CJK}. ( 20 \begin{CJK}{UTF8}{mj}分\end{CJK}) $f(x)$ \begin{CJK}{UTF8}{mj}是在\end{CJK} $(a, b)$ \begin{CJK}{UTF8}{mj}内有连续导数\end{CJK} $f^{\prime}(x)$, \begin{CJK}{UTF8}{mj}试证明在闭区间\end{CJK} $[\alpha, \beta](a<\alpha<\beta<b)$
$$
f_{n}(x)=n f\left(x+\frac{2}{n}\right)-n f(x)
$$
\begin{CJK}{UTF8}{mj}一致收敛于函数\end{CJK} $2 f^{\prime}(x)$.

\begin{CJK}{UTF8}{mj}四\end{CJK}. (20 \begin{CJK}{UTF8}{mj}分\end{CJK}) \begin{CJK}{UTF8}{mj}设序列\end{CJK} $\left\{x_{n}\right\}$ \begin{CJK}{UTF8}{mj}和\end{CJK} $\left\{y_{n}\right\}$ \begin{CJK}{UTF8}{mj}满足方程\end{CJK}:
$$
x_{n+1}=y_{n}+\theta x_{n}
$$
\begin{CJK}{UTF8}{mj}其中\end{CJK} $0<\theta<1$, \begin{CJK}{UTF8}{mj}试证明序列\end{CJK} $\left\{y_{n}\right\}$ \begin{CJK}{UTF8}{mj}收敛的充分必要条件是序列\end{CJK} $\left\{x_{n}\right\}$ \begin{CJK}{UTF8}{mj}收敛\end{CJK}. \begin{CJK}{UTF8}{mj}五\end{CJK}. (20 \begin{CJK}{UTF8}{mj}分\end{CJK}) $f(x, y, \theta)=\left(1-\theta^{2}\right)^{-\frac{1}{2}} e^{-\frac{x^{2}-2 \theta x y+y^{2}}{2\left(1-\theta^{2}\right)}}(|\theta|<1)$, \begin{CJK}{UTF8}{mj}试证明\end{CJK}:

(1)
$$
\begin{gathered}
\frac{\partial f(x, y, \theta)}{\partial \theta}=\frac{\partial^{2} f(x, y, \theta)}{\partial x \partial y} \\
a(\theta)=\int_{x \geqslant 0, y \geqslant 0} f(x, y, \theta) \mathrm{d} x \mathrm{~d} y=\frac{\pi}{2}+\sin ^{-1} \theta .
\end{gathered}
$$

\section{1. 哈尔滨工程大学 2009 年研究生入学考试试题高等代数}
\begin{CJK}{UTF8}{mj}李扬\end{CJK}

\begin{CJK}{UTF8}{mj}微信公众号\end{CJK}: sxkyliyang

\begin{CJK}{UTF8}{mj}一\end{CJK}. \begin{CJK}{UTF8}{mj}填空题\end{CJK}

\begin{enumerate}
  \item \begin{CJK}{UTF8}{mj}满足\end{CJK} $\mathbb{Q} \subseteq \mathbb{F} \subseteq \mathbb{Q}[\sqrt{2}]$ \begin{CJK}{UTF8}{mj}的数域\end{CJK} $\mathbb{F}$ \begin{CJK}{UTF8}{mj}有\end{CJK} ( ).

  \item \begin{CJK}{UTF8}{mj}有理数域上以\end{CJK} $\sqrt{2}+\sqrt{3}$ \begin{CJK}{UTF8}{mj}为根首项系数为\end{CJK} 1 \begin{CJK}{UTF8}{mj}的不可约多项式是\end{CJK} ( ).

  \item $n$ \begin{CJK}{UTF8}{mj}阶行列式\end{CJK} $\left|\begin{array}{cccc}1+x_{1}^{2} & x_{2} x_{1} & \cdots & x_{n} x_{1} \\ x_{1} x_{2} & 1+x_{2}^{2} & \cdots & x_{n} x_{2} \\ \vdots & \vdots & & \vdots \\ x_{1} x_{n} & x_{2} x_{n} & \cdots & 1+x_{n}^{2}\end{array}\right|$ \begin{CJK}{UTF8}{mj}的值为\end{CJK} $(\quad)$.

  \item \begin{CJK}{UTF8}{mj}若\end{CJK} $A$ \begin{CJK}{UTF8}{mj}为可逆转\end{CJK}, \begin{CJK}{UTF8}{mj}则\end{CJK} $\left(A^{-1}\right)^{*}=(\quad) A$.

  \item \begin{CJK}{UTF8}{mj}若\end{CJK} $V_{1}, V_{2}$ \begin{CJK}{UTF8}{mj}为\end{CJK} 3 \begin{CJK}{UTF8}{mj}维线性空间中两个不同的\end{CJK} 2 \begin{CJK}{UTF8}{mj}维子空间\end{CJK}, \begin{CJK}{UTF8}{mj}则\end{CJK} $\operatorname{dim}\left(V_{1}+V_{2}\right)=(\quad)$.

  \item \begin{CJK}{UTF8}{mj}令\end{CJK} $A \in \mathbb{R}^{n \times n}, A^{n-1} \neq 0, A^{n}=0$, \begin{CJK}{UTF8}{mj}则\end{CJK} $V=\{f(A) \mid f(x) \in \mathbb{R}[x]\}$ \begin{CJK}{UTF8}{mj}作为实数域上的线性空间其维数为\end{CJK}( ).

  \item $A \in \mathbb{R}^{m \times n}$, \begin{CJK}{UTF8}{mj}则线性方程组\end{CJK} $A X=b$ \begin{CJK}{UTF8}{mj}任何向量\end{CJK} $b \in \mathbb{R}^{m}$ \begin{CJK}{UTF8}{mj}都有解的充要条件为\end{CJK} ( ).

  \item \begin{CJK}{UTF8}{mj}若\end{CJK} $A$ \begin{CJK}{UTF8}{mj}为\end{CJK} 3 \begin{CJK}{UTF8}{mj}阶实对称阵\end{CJK}, \begin{CJK}{UTF8}{mj}其特征值为\end{CJK} $-3,1,4$, \begin{CJK}{UTF8}{mj}则当\end{CJK} $t$ \begin{CJK}{UTF8}{mj}满足\end{CJK} ( ) \begin{CJK}{UTF8}{mj}时\end{CJK}, $t E+A$ \begin{CJK}{UTF8}{mj}正定\end{CJK}.

  \item \begin{CJK}{UTF8}{mj}令\end{CJK} $A \in \mathbb{R}^{4 \times 4}$ \begin{CJK}{UTF8}{mj}的特征值为\end{CJK} $1,2,3,4$, \begin{CJK}{UTF8}{mj}则\end{CJK} $\operatorname{tr}\left(A^{2}\right)=(\quad)$.

  \item \begin{CJK}{UTF8}{mj}一切\end{CJK} $n$ \begin{CJK}{UTF8}{mj}阶幂等阵\end{CJK} $\left(A^{2}=A\right)$ \begin{CJK}{UTF8}{mj}在复数域内按相似可分为\end{CJK}( $)$ \begin{CJK}{UTF8}{mj}类\end{CJK}.

\end{enumerate}
\begin{CJK}{UTF8}{mj}二\end{CJK}. \begin{CJK}{UTF8}{mj}设\end{CJK} $V$ \begin{CJK}{UTF8}{mj}为实数域\end{CJK} $\mathbb{R}$ \begin{CJK}{UTF8}{mj}上的\end{CJK} 5 \begin{CJK}{UTF8}{mj}维线性空间\end{CJK}, $\mathscr{A}$ \begin{CJK}{UTF8}{mj}为其上的线性变换\end{CJK}, \begin{CJK}{UTF8}{mj}且\end{CJK} $\mathscr{A}$ \begin{CJK}{UTF8}{mj}在基\end{CJK} $\varepsilon_{1}, \varepsilon_{2}, \varepsilon_{3}, \varepsilon_{4}, \varepsilon_{5}$ \begin{CJK}{UTF8}{mj}之下的矩阵为\end{CJK}
$$
A=\left[\begin{array}{lllll} 
& & & 1 \\
& & & 1 & \\
& & 1 & & \\
& 1 & & & \\
1 & & & &
\end{array}\right]
$$
(1) \begin{CJK}{UTF8}{mj}求\end{CJK} $V$ \begin{CJK}{UTF8}{mj}的另一组基\end{CJK} $\alpha_{1}, \alpha_{2}, \alpha_{3}, \alpha_{4}, \alpha_{5}$, \begin{CJK}{UTF8}{mj}使\end{CJK} $\mathscr{A}$ \begin{CJK}{UTF8}{mj}在此基下的矩阵为对角阵\end{CJK}.

(2) \begin{CJK}{UTF8}{mj}求\end{CJK} $A^{n}$.

\begin{CJK}{UTF8}{mj}三\end{CJK}. \begin{CJK}{UTF8}{mj}设\end{CJK} $A \in \mathbb{R}^{n \times n}, R(A)=\left\{A x \mid x \in \mathbb{R}^{n}\right\}, N(A)=\left\{x \in \mathbb{R}^{n} \mid A x=0\right\}$. \begin{CJK}{UTF8}{mj}若\end{CJK} $A$ \begin{CJK}{UTF8}{mj}与\end{CJK} $A^{2}$ \begin{CJK}{UTF8}{mj}有相同的秩\end{CJK}. \begin{CJK}{UTF8}{mj}求证\end{CJK}:

(1) \begin{CJK}{UTF8}{mj}齐次线性方程组\end{CJK} $A x=0$ \begin{CJK}{UTF8}{mj}和\end{CJK} $A^{2} x=0$ \begin{CJK}{UTF8}{mj}同解\end{CJK}.

(2) $\mathbb{R}^{n}=R(A) \oplus N(A)$.

\begin{CJK}{UTF8}{mj}四\end{CJK}. \begin{CJK}{UTF8}{mj}设\end{CJK} $V$ \begin{CJK}{UTF8}{mj}为数域\end{CJK} $\mathbb{C}$ \begin{CJK}{UTF8}{mj}上的\end{CJK} $n$ \begin{CJK}{UTF8}{mj}维线性空间\end{CJK} $(n \geq 3), \mathscr{A}$ \begin{CJK}{UTF8}{mj}为其上的线性变换\end{CJK}, $\mathscr{A}^{n-2} \neq 0, \mathscr{A}^{n-1}=0$. \begin{CJK}{UTF8}{mj}求证\end{CJK}: $\mathscr{A}$ \begin{CJK}{UTF8}{mj}在\end{CJK} $V$ \begin{CJK}{UTF8}{mj}的某个基下的矩阵为\end{CJK}
$$
\left[\begin{array}{lllll}
0 & & & & \\
1 & \ddots & & & \\
& \ddots & \ddots & & \\
& & 1 & 0 & \\
& & & 0 & 0
\end{array}\right]
$$
\begin{CJK}{UTF8}{mj}五\end{CJK}. (1) \begin{CJK}{UTF8}{mj}求证任何一个正定矩阵\end{CJK} $A=B^{2}, B$ \begin{CJK}{UTF8}{mj}也为正定矩阵\end{CJK}.

(2) \begin{CJK}{UTF8}{mj}求证任何一个可逆实矩阵\end{CJK} $A=Q P, Q$ \begin{CJK}{UTF8}{mj}为正定矩阵\end{CJK}, $P$ \begin{CJK}{UTF8}{mj}为正交阵\end{CJK}.

\begin{CJK}{UTF8}{mj}六\end{CJK}. $\mathbb{F}$ \begin{CJK}{UTF8}{mj}为数域\end{CJK}, $A, B \in \mathbb{F}^{n \times n}(n \geq 1): A+B=E_{n}, A B=B A, A^{2}=A, B^{2}=B$. \begin{CJK}{UTF8}{mj}求证存在一个可逆矩阵\end{CJK} $P$ \begin{CJK}{UTF8}{mj}使得\end{CJK}
$$
P^{-1} A P=\left[\begin{array}{cc}
E_{s} & \\
& 0
\end{array}\right], P^{-1} B P=\left[\begin{array}{cc}
0 & \\
& E_{t}
\end{array}\right]
$$
\begin{CJK}{UTF8}{mj}这里\end{CJK} $s+t=n$.

\begin{CJK}{UTF8}{mj}七\end{CJK}. \begin{CJK}{UTF8}{mj}设\end{CJK} $A, B$ \begin{CJK}{UTF8}{mj}为\end{CJK} $n$ \begin{CJK}{UTF8}{mj}阶方阵\end{CJK}, $A$ \begin{CJK}{UTF8}{mj}为\end{CJK} $n$ \begin{CJK}{UTF8}{mj}阶幂零阵\end{CJK}, \begin{CJK}{UTF8}{mj}求证\end{CJK}
$$
|A+B|=|B|
$$
\begin{CJK}{UTF8}{mj}八\end{CJK}. \begin{CJK}{UTF8}{mj}设\end{CJK} $A \in \mathbb{C}^{n \times n}$ \begin{CJK}{UTF8}{mj}可逆\end{CJK}. \begin{CJK}{UTF8}{mj}求证矩阵方程\end{CJK} $A X A^{T}-X=0$ \begin{CJK}{UTF8}{mj}仅有零解的充要条件为\end{CJK} $A$ \begin{CJK}{UTF8}{mj}的任何两个特征值的乘积\end{CJK} \begin{CJK}{UTF8}{mj}不为\end{CJK} 1 .

\section{2. 哈尔滨工程大学 2010 年研究生入学考试试题高等代数}
\begin{CJK}{UTF8}{mj}李扬\end{CJK}

\begin{CJK}{UTF8}{mj}微信公众号\end{CJK}: sxkyliyang

\begin{CJK}{UTF8}{mj}一\end{CJK}. \begin{CJK}{UTF8}{mj}填空题\end{CJK}

\begin{enumerate}
  \item \begin{CJK}{UTF8}{mj}设\end{CJK} $f(x)=x^{5}-2 x^{4}+\frac{3}{2} x^{3}-3 x^{2}+\frac{1}{2} x-1$, \begin{CJK}{UTF8}{mj}则\end{CJK} $f(x)$ \begin{CJK}{UTF8}{mj}的有理根是\end{CJK} ( ).

  \item \begin{CJK}{UTF8}{mj}多项式\end{CJK} $f(x)=x^{3}+5 x-10$ \begin{CJK}{UTF8}{mj}在有理数上是\end{CJK} ( ) (\begin{CJK}{UTF8}{mj}可约或不可约\end{CJK}).

  \item \begin{CJK}{UTF8}{mj}设\end{CJK} $D=\left|\begin{array}{llll}1 & 2 & 3 & 4 \\ 3 & 2 & 4 & 1 \\ 0 & 2 & 3 & 1 \\ 0 & 2 & 4 & 3\end{array}\right| \cdot A_{i j}$ \begin{CJK}{UTF8}{mj}表示元素\end{CJK} $a_{i j}$ \begin{CJK}{UTF8}{mj}的代数余子式\end{CJK}, \begin{CJK}{UTF8}{mj}则\end{CJK} $2 A_{14}+A_{24}+A_{34}+A_{44}=($ ).

  \item \begin{CJK}{UTF8}{mj}对称多项式\end{CJK} $f\left(x_{1}, x_{2}, x_{3}\right)=x_{1}^{2}+x_{2}^{2}+x_{3}^{2}$ \begin{CJK}{UTF8}{mj}表示为初等对称多项式是\end{CJK} ( ).

  \item \begin{CJK}{UTF8}{mj}将矩阵\end{CJK} $\left(\begin{array}{ll}2 & 1 \\ 4 & 3\end{array}\right)$ \begin{CJK}{UTF8}{mj}写成初等矩阵之积\end{CJK} $(\quad)$.

  \item \begin{CJK}{UTF8}{mj}设\end{CJK} $\alpha_{1}=(1,1, k), \alpha_{2}=(1, k, 1), \alpha_{3}=(k, 1,1)$ \begin{CJK}{UTF8}{mj}是线性无关的\end{CJK}, \begin{CJK}{UTF8}{mj}则\end{CJK} $k$ \begin{CJK}{UTF8}{mj}的取值为\end{CJK} ( ) $)$.

  \item \begin{CJK}{UTF8}{mj}二次型\end{CJK} $f\left(x, x_{2}, x_{3}\right)=x_{1}^{2}+2 x_{2}^{2}+3 x_{3}^{2}+2 t x_{2} x_{3}$ \begin{CJK}{UTF8}{mj}是正定的\end{CJK}, \begin{CJK}{UTF8}{mj}则\end{CJK} $t$ \begin{CJK}{UTF8}{mj}的取值范围为\end{CJK} $(\quad)$.

  \item $\mathbb{R}^{3}$ \begin{CJK}{UTF8}{mj}中的向量\end{CJK} $\alpha=\left(a_{1}, a_{1}, a_{3}\right)$ \begin{CJK}{UTF8}{mj}在基\end{CJK} $\alpha_{1}=(1,1,1), \alpha_{2}=(0,1,1), \alpha_{3}=(0,0,1)$ \begin{CJK}{UTF8}{mj}下的坐标是\end{CJK} ( ).

  \item \begin{CJK}{UTF8}{mj}在\end{CJK} $\mathbb{R}^{3}$ \begin{CJK}{UTF8}{mj}中与向量\end{CJK} $(1,1,2)$ \begin{CJK}{UTF8}{mj}和\end{CJK} $(-1,1,0)$ \begin{CJK}{UTF8}{mj}都正交的单位向量是\end{CJK} $(\quad)$.

  \item \begin{CJK}{UTF8}{mj}令\end{CJK} $A \in \mathbb{R}^{4 \times 4}$ \begin{CJK}{UTF8}{mj}的特征值为\end{CJK} $1,2,3,4$, \begin{CJK}{UTF8}{mj}则\end{CJK} $\operatorname{tr}\left(A^{2}\right)=(\quad$.

\end{enumerate}
\begin{CJK}{UTF8}{mj}二\end{CJK}. \begin{CJK}{UTF8}{mj}在\end{CJK} $\mathbb{R}^{3}$ \begin{CJK}{UTF8}{mj}中\end{CJK}, \begin{CJK}{UTF8}{mj}线性变换\end{CJK} $\mathscr{A}$ \begin{CJK}{UTF8}{mj}定义为\end{CJK}
$$
\left\{\begin{array}{l}
\mathscr{A} \alpha_{1}=(1,0,0) \\
\mathscr{A} \alpha_{2}=(3,3,2) \\
\mathscr{A} \alpha_{3}=(3,3,1)
\end{array}\right.
$$
\begin{CJK}{UTF8}{mj}其中\end{CJK}
$$
\left\{\begin{array}{l}
\alpha_{1}=(1,0,0) \\
\alpha_{2}=(1,1,0) \\
\alpha_{3}=(1,1,1)
\end{array}\right.
$$
(1) \begin{CJK}{UTF8}{mj}求\end{CJK} $\mathscr{A}$ \begin{CJK}{UTF8}{mj}在基\end{CJK} $\alpha_{1}, \alpha_{2}, \alpha_{3}$ \begin{CJK}{UTF8}{mj}下的矩阵\end{CJK} $B$.

(2) \begin{CJK}{UTF8}{mj}求\end{CJK} $\mathscr{A}$ \begin{CJK}{UTF8}{mj}的特征值与特征向量\end{CJK}.

\begin{CJK}{UTF8}{mj}三\end{CJK}. \begin{CJK}{UTF8}{mj}设\end{CJK}
$$
V=\left\{A \in \mathbb{R}^{n \times n} \mid \operatorname{tr}(A)=0\right\}, W=\{a E \mid a \in \mathbb{R}\}
$$
(1) \begin{CJK}{UTF8}{mj}求证\end{CJK} $V$ \begin{CJK}{UTF8}{mj}为\end{CJK} $\mathbb{R}^{n \times n}$ \begin{CJK}{UTF8}{mj}的子空间\end{CJK}, \begin{CJK}{UTF8}{mj}并求\end{CJK} $\operatorname{dim} V$.

(2) \begin{CJK}{UTF8}{mj}求证\end{CJK} $\mathbb{R}^{n \times n}=V \oplus W$.

\begin{CJK}{UTF8}{mj}四\end{CJK}. \begin{CJK}{UTF8}{mj}设\end{CJK} 4 \begin{CJK}{UTF8}{mj}元二次型\end{CJK}
$$
f\left(x_{1}, x_{2}, x_{3}, x_{4}\right)=2 x_{1} x_{2}+2 x_{3} x_{4}
$$
(1) \begin{CJK}{UTF8}{mj}写出二次型\end{CJK} $f\left(x_{1}, x_{2}, x_{3}, x_{4}\right)$ \begin{CJK}{UTF8}{mj}的矩阵表达式\end{CJK} $f\left(x_{1}, x_{2}, x_{3}, x_{4}\right)=X^{T} A X$.

(2) \begin{CJK}{UTF8}{mj}求\end{CJK} $A$ \begin{CJK}{UTF8}{mj}的特征值和特征向量\end{CJK}.

(3) \begin{CJK}{UTF8}{mj}求正交矩阵\end{CJK} $P$, \begin{CJK}{UTF8}{mj}使得\end{CJK} $P^{-1} A P=\wedge$, \begin{CJK}{UTF8}{mj}其中\end{CJK} $\wedge$ \begin{CJK}{UTF8}{mj}是对角阵\end{CJK}.

(4) \begin{CJK}{UTF8}{mj}写出二次型\end{CJK} $f\left(x_{1}, x_{2}, x_{3}, x_{4}\right)$ \begin{CJK}{UTF8}{mj}的标准型\end{CJK}.

\begin{CJK}{UTF8}{mj}五\end{CJK}. \begin{CJK}{UTF8}{mj}设矩阵\end{CJK} $A$ \begin{CJK}{UTF8}{mj}与\end{CJK} $B$ \begin{CJK}{UTF8}{mj}没有公共的特征根\end{CJK}, $f_{A}(x)$ \begin{CJK}{UTF8}{mj}是\end{CJK} $A$ \begin{CJK}{UTF8}{mj}的特征多项式\end{CJK}, \begin{CJK}{UTF8}{mj}证明\end{CJK}:

(1) \begin{CJK}{UTF8}{mj}矩阵\end{CJK} $f_{A}(B)$ \begin{CJK}{UTF8}{mj}可逆\end{CJK}.

(2) \begin{CJK}{UTF8}{mj}矩阵方程\end{CJK} $A X=X B$ \begin{CJK}{UTF8}{mj}只有零解\end{CJK}.

\begin{CJK}{UTF8}{mj}六\end{CJK}. \begin{CJK}{UTF8}{mj}设\end{CJK} $A$ \begin{CJK}{UTF8}{mj}是\end{CJK} $n$ \begin{CJK}{UTF8}{mj}阶正定阵\end{CJK}, \begin{CJK}{UTF8}{mj}求证\end{CJK}: \begin{CJK}{UTF8}{mj}存在唯一的正定阵\end{CJK} $B$, \begin{CJK}{UTF8}{mj}使得\end{CJK}
$$
A=B^{2}
$$
\begin{CJK}{UTF8}{mj}七\end{CJK}. \begin{CJK}{UTF8}{mj}设\end{CJK} $\varepsilon_{1}, \varepsilon_{2}, \cdots, \varepsilon_{n}$ \begin{CJK}{UTF8}{mj}是\end{CJK} $n$ \begin{CJK}{UTF8}{mj}维线性空间\end{CJK} $V$ \begin{CJK}{UTF8}{mj}的一组基\end{CJK}, $\mathscr{A}$ \begin{CJK}{UTF8}{mj}是\end{CJK} $V$ \begin{CJK}{UTF8}{mj}上的线性变换\end{CJK}. \begin{CJK}{UTF8}{mj}证明\end{CJK}: $\mathscr{A}$ \begin{CJK}{UTF8}{mj}是可逆的\end{CJK}, \begin{CJK}{UTF8}{mj}当且仅当\end{CJK} $\mathscr{A}\left(\varepsilon_{1}\right), \mathscr{A}\left(\varepsilon_{2}\right), \cdots, \mathscr{A}\left(\varepsilon_{n}\right)$ \begin{CJK}{UTF8}{mj}也是\end{CJK} $V$ \begin{CJK}{UTF8}{mj}的基\end{CJK}.

\begin{CJK}{UTF8}{mj}八\end{CJK}. \begin{CJK}{UTF8}{mj}设\end{CJK} $\mathscr{A}$ \begin{CJK}{UTF8}{mj}与\end{CJK} $\mathscr{B}$ \begin{CJK}{UTF8}{mj}为\end{CJK} $n$ \begin{CJK}{UTF8}{mj}维欧式空间\end{CJK} $V$ \begin{CJK}{UTF8}{mj}上的两个线性变换\end{CJK}. \begin{CJK}{UTF8}{mj}若对任意的\end{CJK} $\alpha \in V$, \begin{CJK}{UTF8}{mj}有\end{CJK}
$$
(\mathscr{A} \alpha, \mathscr{B} \alpha)=(\mathscr{B} \alpha, \mathscr{B} \alpha)
$$
\begin{CJK}{UTF8}{mj}则\end{CJK} $\mathscr{A} V$ \begin{CJK}{UTF8}{mj}与\end{CJK} $\mathscr{B} V$ \begin{CJK}{UTF8}{mj}作为欧式空间是同构的\end{CJK}.

\section{3. 哈尔滨工程大学 2011 年研究生入学考试试题高等代数}
\begin{CJK}{UTF8}{mj}李扬\end{CJK}

\begin{CJK}{UTF8}{mj}微信公众号\end{CJK}: sxkyliyang

\begin{CJK}{UTF8}{mj}一\end{CJK}. \begin{CJK}{UTF8}{mj}填空题\end{CJK}

\begin{enumerate}
  \item \begin{CJK}{UTF8}{mj}多项式\end{CJK} $f(x)=2 x^{3}+4 x^{2}+6 x+1$ \begin{CJK}{UTF8}{mj}在有理数上是\end{CJK} ( ) (\begin{CJK}{UTF8}{mj}可约或不可约\end{CJK}).

  \item \begin{CJK}{UTF8}{mj}设\end{CJK} $A$ \begin{CJK}{UTF8}{mj}和\end{CJK} $B$ \begin{CJK}{UTF8}{mj}均为\end{CJK} $n$ \begin{CJK}{UTF8}{mj}阶方阵\end{CJK}, $A^{*}$ \begin{CJK}{UTF8}{mj}与\end{CJK} $B^{*}$ \begin{CJK}{UTF8}{mj}分别为它们的伴随矩阵\end{CJK}, $|A|=2,|B|=-3$, \begin{CJK}{UTF8}{mj}则\end{CJK} $\left|A^{-1} B^{*}-A^{*} B^{-1}\right|=$

  \item \begin{CJK}{UTF8}{mj}设\end{CJK} 3 \begin{CJK}{UTF8}{mj}阶方阵\end{CJK} $A$ \begin{CJK}{UTF8}{mj}按下列分块为\end{CJK} $A=\left(A_{1}, A_{2}, A_{3}\right)$, \begin{CJK}{UTF8}{mj}且\end{CJK} $|A|=5$, \begin{CJK}{UTF8}{mj}又设\end{CJK} $B=\left(A_{1}+2 A_{2}, 3 A_{1}+4 A_{3}\right)$, \begin{CJK}{UTF8}{mj}则\end{CJK} $|B|=(\quad) .$

  \item \begin{CJK}{UTF8}{mj}已知向量组\end{CJK} $\alpha_{1}=(1,2,-1,1), \alpha_{2}=(2,0, t, 0), \alpha_{3}=(0,-4,5,-2)$ \begin{CJK}{UTF8}{mj}的秩为\end{CJK} 2 , \begin{CJK}{UTF8}{mj}则\end{CJK} $t=(\quad)$.

  \item \begin{CJK}{UTF8}{mj}若实对称矩阵\end{CJK} $A$ \begin{CJK}{UTF8}{mj}与矩阵\end{CJK} $B=\left(\begin{array}{lll}1 & 0 & 0 \\ 0 & 0 & 2 \\ 0 & 2 & 0\end{array}\right)$ \begin{CJK}{UTF8}{mj}合同\end{CJK}, \begin{CJK}{UTF8}{mj}则二次型\end{CJK} $x^{T} A x$ \begin{CJK}{UTF8}{mj}的规范型为\end{CJK} ( ).

  \item $\mathbb{R}^{3}$ \begin{CJK}{UTF8}{mj}中的向量\end{CJK} $\alpha=\left(x_{1}, x_{1}, x_{3}\right)$ \begin{CJK}{UTF8}{mj}在基\end{CJK} $\alpha_{1}=(1,1,1), \alpha_{2}=(0,1,1), \alpha_{3}=(0,0,1)$ \begin{CJK}{UTF8}{mj}下的坐标是\end{CJK} ( ).

  \item \begin{CJK}{UTF8}{mj}设\end{CJK} 3 \begin{CJK}{UTF8}{mj}阶方阵\end{CJK} $A$ \begin{CJK}{UTF8}{mj}的三个特征值\end{CJK} $1,2,-2$, \begin{CJK}{UTF8}{mj}矩阵\end{CJK} $B$ \begin{CJK}{UTF8}{mj}与\end{CJK} $A$ \begin{CJK}{UTF8}{mj}相似\end{CJK},\begin{CJK}{UTF8}{mj}则\end{CJK} $B$ \begin{CJK}{UTF8}{mj}的伴随矩阵\end{CJK} $B^{*}$ \begin{CJK}{UTF8}{mj}的三个特征值为\end{CJK}( $)$.

  \item \begin{CJK}{UTF8}{mj}设矩阵\end{CJK} $A=\left(\begin{array}{lll}1 & 1 & 1 \\ 1 & 1 & 1 \\ 1 & 1 & 1\end{array}\right)$, \begin{CJK}{UTF8}{mj}则\end{CJK} $A$ \begin{CJK}{UTF8}{mj}的最小多项式为\end{CJK} $(\quad)$.

  \item $\mathbb{R}^{3}$ \begin{CJK}{UTF8}{mj}中的子空间\end{CJK} $V_{1}=L(\alpha)$, \begin{CJK}{UTF8}{mj}其中\end{CJK} $\alpha=(1,1,1)$, \begin{CJK}{UTF8}{mj}则\end{CJK} $V_{1}^{\perp}=(\quad)$.

  \item \begin{CJK}{UTF8}{mj}特征值为\end{CJK} $1,1,1,1$ \begin{CJK}{UTF8}{mj}的一切\end{CJK} $4 \times 4$ \begin{CJK}{UTF8}{mj}复数矩阵在复数域内按相似可分为\end{CJK} ( ) \begin{CJK}{UTF8}{mj}类\end{CJK}.

\end{enumerate}
\begin{CJK}{UTF8}{mj}一\end{CJK}. \begin{CJK}{UTF8}{mj}已知实矩阵\end{CJK}
$$
A=\left(\begin{array}{ll}
2 & 2 \\
2 & x
\end{array}\right), B=\left(\begin{array}{ll}
4 & y \\
3 & 1
\end{array}\right)
$$
\begin{CJK}{UTF8}{mj}问\end{CJK}:

(1) $x, y$ \begin{CJK}{UTF8}{mj}为何值时\end{CJK}, $A$ \begin{CJK}{UTF8}{mj}合同于\end{CJK} $B$ ?

(2) $x, y$ \begin{CJK}{UTF8}{mj}为何值时\end{CJK}, $A$ \begin{CJK}{UTF8}{mj}相似于\end{CJK} $B$ ?

\begin{CJK}{UTF8}{mj}三\end{CJK}. \begin{CJK}{UTF8}{mj}设\end{CJK}
$$
A=\left(\begin{array}{ccc}
1 & 2 & 1 \\
\alpha & 1 & \beta \\
1 & \beta & 1
\end{array}\right), B=\left(\begin{array}{lll}
0 & 0 & 0 \\
0 & 1 & 0 \\
0 & 0 & 2
\end{array}\right)
$$
\begin{CJK}{UTF8}{mj}且\end{CJK} $A$ \begin{CJK}{UTF8}{mj}与\end{CJK} $B$ \begin{CJK}{UTF8}{mj}相似\end{CJK}.

(1) \begin{CJK}{UTF8}{mj}求\end{CJK} $\alpha$ \begin{CJK}{UTF8}{mj}和\end{CJK} $\beta$ \begin{CJK}{UTF8}{mj}的值\end{CJK}.

(2) \begin{CJK}{UTF8}{mj}求可逆矩阵\end{CJK} $P$ \begin{CJK}{UTF8}{mj}使得\end{CJK} $P^{-1} A P=B$.

\begin{CJK}{UTF8}{mj}四\end{CJK}. \begin{CJK}{UTF8}{mj}在\end{CJK} $\mathbb{R}^{3}$ \begin{CJK}{UTF8}{mj}中定义线性变换\end{CJK} $\mathscr{A}$ \begin{CJK}{UTF8}{mj}为\end{CJK}
$$
\mathscr{A}\left(x_{1}, x_{2}, x_{3}\right)=\left(2 x_{1}-x_{2}, x_{2}+x_{3}, x_{1}\right)
$$
(1) \begin{CJK}{UTF8}{mj}求\end{CJK} $\mathscr{A}$ \begin{CJK}{UTF8}{mj}在基\end{CJK} $\varepsilon_{1}=(1,0,0), \varepsilon_{2}=(0,1,0), \varepsilon_{3}=(0,0,1)$ \begin{CJK}{UTF8}{mj}下的矩阵\end{CJK}.

(2) \begin{CJK}{UTF8}{mj}设\end{CJK} $\alpha=(1,0,-2)$, \begin{CJK}{UTF8}{mj}求\end{CJK} $\mathscr{A} \alpha$ \begin{CJK}{UTF8}{mj}在基\end{CJK} $\alpha_{1}=(2,0,1), \alpha_{2}=(0,-1,1), \alpha_{3}=(-1,0,2)$ \begin{CJK}{UTF8}{mj}下的坐标\end{CJK}.

(3) $\mathscr{A}$ \begin{CJK}{UTF8}{mj}是否可逆\end{CJK}? \begin{CJK}{UTF8}{mj}若可逆\end{CJK}, \begin{CJK}{UTF8}{mj}求\end{CJK} $\mathscr{A}^{-1}$, \begin{CJK}{UTF8}{mj}若不可逆\end{CJK}, \begin{CJK}{UTF8}{mj}说明原因\end{CJK}.

\begin{CJK}{UTF8}{mj}五\end{CJK}. \begin{CJK}{UTF8}{mj}分块矩阵\end{CJK}
$$
\left(\begin{array}{cc}
A & B \\
B^{T} & D
\end{array}\right)
$$
\begin{CJK}{UTF8}{mj}为正定矩阵\end{CJK}, \begin{CJK}{UTF8}{mj}其中\end{CJK} $B^{T}$ \begin{CJK}{UTF8}{mj}是\end{CJK} $B$ \begin{CJK}{UTF8}{mj}的转置\end{CJK}. \begin{CJK}{UTF8}{mj}证明\end{CJK}:

(1) $A$ \begin{CJK}{UTF8}{mj}可逆\end{CJK}.

(2) $D-B^{T} A^{-1} B$ \begin{CJK}{UTF8}{mj}也是正定\end{CJK}.

\begin{CJK}{UTF8}{mj}六\end{CJK}. \begin{CJK}{UTF8}{mj}给定数域\end{CJK} $P$ \begin{CJK}{UTF8}{mj}上的分块矩阵\end{CJK} $M=\left(\begin{array}{cc}A & C \\ 0 & B\end{array}\right)$, \begin{CJK}{UTF8}{mj}其中\end{CJK} $A$ \begin{CJK}{UTF8}{mj}为\end{CJK} $m \times n$ \begin{CJK}{UTF8}{mj}的矩阵\end{CJK}, $B$ \begin{CJK}{UTF8}{mj}为\end{CJK} $k \times l$ \begin{CJK}{UTF8}{mj}的矩阵\end{CJK}, \begin{CJK}{UTF8}{mj}证明\end{CJK}:
$$
\operatorname{rank}(A)+\operatorname{rank}(B) \leq \operatorname{rank}(M)
$$
\begin{CJK}{UTF8}{mj}注\end{CJK}: $\operatorname{rank}(A)$ \begin{CJK}{UTF8}{mj}表示矩阵\end{CJK} $A$ \begin{CJK}{UTF8}{mj}的秩\end{CJK}.

\begin{CJK}{UTF8}{mj}七\end{CJK}. \begin{CJK}{UTF8}{mj}设\end{CJK} $A$ \begin{CJK}{UTF8}{mj}是半正定矩阵\end{CJK}, \begin{CJK}{UTF8}{mj}证明存在唯一的半正定矩阵\end{CJK} $B$ \begin{CJK}{UTF8}{mj}使得\end{CJK}
$$
A=B^{2}
$$
\begin{CJK}{UTF8}{mj}八\end{CJK}. \begin{CJK}{UTF8}{mj}设\end{CJK} $V$ \begin{CJK}{UTF8}{mj}为\end{CJK} $n$ \begin{CJK}{UTF8}{mj}维欧式空间\end{CJK}, \begin{CJK}{UTF8}{mj}求证\end{CJK}:

(1) \begin{CJK}{UTF8}{mj}对\end{CJK} $V$ \begin{CJK}{UTF8}{mj}中每个线性变换\end{CJK} $\mathscr{A}$, \begin{CJK}{UTF8}{mj}都存在唯一的共轭变换\end{CJK} $\mathscr{A}^{*}$, \begin{CJK}{UTF8}{mj}即存在唯一的线性变换\end{CJK} $\mathscr{A}^{*}$, \begin{CJK}{UTF8}{mj}使得对\end{CJK} $\forall \alpha, \beta \in$ $V$, \begin{CJK}{UTF8}{mj}有\end{CJK}
$$
(\mathscr{A} \alpha, \beta)=\left(\alpha, \mathscr{A}^{*} \beta\right)
$$
(2) $\mathscr{A}$ \begin{CJK}{UTF8}{mj}为对称变换当且仅当\end{CJK} $\mathscr{A}^{*}=\mathscr{A}$.

(3) $\mathscr{A}$ \begin{CJK}{UTF8}{mj}为正交变换当且仅当\end{CJK}
$$
\mathscr{A} \mathscr{A}^{*}=\mathscr{A}^{*} \mathscr{A}=\mathscr{E} .
$$
\begin{CJK}{UTF8}{mj}其中\end{CJK} $\mathscr{E}$ \begin{CJK}{UTF8}{mj}是\end{CJK} $V$ \begin{CJK}{UTF8}{mj}上的恒等变换\end{CJK}.

\section{4. 哈尔滨工程大学 2012 年研究生入学考试试题高等代数}
\begin{CJK}{UTF8}{mj}李扬\end{CJK}

\begin{CJK}{UTF8}{mj}微信公众号\end{CJK}: sxkyliyang

\begin{CJK}{UTF8}{mj}一\end{CJK}. \begin{CJK}{UTF8}{mj}填空题\end{CJK}

\begin{enumerate}
  \item \begin{CJK}{UTF8}{mj}三阶行列式有两个元素为\end{CJK} 4 , \begin{CJK}{UTF8}{mj}其余为\end{CJK} $\pm 1$, \begin{CJK}{UTF8}{mj}则此行列式可能的最大值为\end{CJK} ( ) ).

  \item $\gamma_{1}, \gamma_{2}, \alpha, \beta$ \begin{CJK}{UTF8}{mj}皆为三维列向量\end{CJK}, $A=\left(\alpha, 2 \gamma_{1}, 3 \gamma_{2}\right), B=\left(\beta, \gamma_{1}, 2 \gamma_{2}\right)$ \begin{CJK}{UTF8}{mj}且\end{CJK} $|A|=18,|B|=4$, \begin{CJK}{UTF8}{mj}则\end{CJK} $|A-B|=$ $(\quad)$

  \item \begin{CJK}{UTF8}{mj}三阶方阵\end{CJK} $A$ \begin{CJK}{UTF8}{mj}的特征值为\end{CJK} $1,-1,2$, \begin{CJK}{UTF8}{mj}则\end{CJK} $A^{2}+4 A^{-1}$ \begin{CJK}{UTF8}{mj}的特征值\end{CJK}( $)$.

  \item \begin{CJK}{UTF8}{mj}若不可约多项式\end{CJK} $p(x)$ \begin{CJK}{UTF8}{mj}是\end{CJK} $f^{(k)}(x)$ \begin{CJK}{UTF8}{mj}的\end{CJK} $s$ \begin{CJK}{UTF8}{mj}重因子\end{CJK}, \begin{CJK}{UTF8}{mj}且\end{CJK} $p(x) \mid f(x)$, \begin{CJK}{UTF8}{mj}那么\end{CJK} $p(x)(\quad) f(x)$ \begin{CJK}{UTF8}{mj}的\end{CJK} $s+k$ \begin{CJK}{UTF8}{mj}重因子\end{CJK}.

  \item $A=\left(\begin{array}{ccc}0 & a & 9 \\ 0 & 6 & 0 \\ 4 & 2 b & 0\end{array}\right)$ \begin{CJK}{UTF8}{mj}相似于对角阵\end{CJK}, \begin{CJK}{UTF8}{mj}则\end{CJK} $a$ \begin{CJK}{UTF8}{mj}与\end{CJK} $b$ \begin{CJK}{UTF8}{mj}的关系式为\end{CJK} ( $)$.

  \item \begin{CJK}{UTF8}{mj}设\end{CJK} $\mathbb{R}^{2}$ \begin{CJK}{UTF8}{mj}中的内积为\end{CJK} $(\alpha, \beta)=\alpha^{\prime} A \beta, A=\left(\begin{array}{ll}2 & 1 \\ 1 & 2\end{array}\right)$, \begin{CJK}{UTF8}{mj}则\end{CJK} $\left(\begin{array}{l}1 \\ 0\end{array}\right),\left(\begin{array}{l}0 \\ 1\end{array}\right)$ \begin{CJK}{UTF8}{mj}在此内积之下的度量矩阵为\end{CJK}( ).

  \item \begin{CJK}{UTF8}{mj}令\end{CJK} $A \in \mathbb{R}^{4 \times 4}$ \begin{CJK}{UTF8}{mj}的特征值为\end{CJK} $1,2,3,4$, \begin{CJK}{UTF8}{mj}则\end{CJK} $\operatorname{tr}\left(A^{2}\right)=(\quad)$.

  \item \begin{CJK}{UTF8}{mj}设\end{CJK} $A, B \in \mathbb{R}^{m \times n}$, \begin{CJK}{UTF8}{mj}在矩阵方程\end{CJK} $A X=B$ \begin{CJK}{UTF8}{mj}有解的充要条件为\end{CJK} ( ).

\end{enumerate}
9 . \begin{CJK}{UTF8}{mj}设\end{CJK} $A$ \begin{CJK}{UTF8}{mj}为正交矩阵\end{CJK}, \begin{CJK}{UTF8}{mj}且\end{CJK} $|A|=-1$, \begin{CJK}{UTF8}{mj}则\end{CJK} $A$ \begin{CJK}{UTF8}{mj}必有特征值为\end{CJK} ( ).

\begin{enumerate}
  \setcounter{enumi}{10}
  \item \begin{CJK}{UTF8}{mj}向量组\end{CJK} $\alpha_{1}=(1,1, k), \alpha_{2}=(1, k, 1), \alpha_{3}=(k, 1,1)$ \begin{CJK}{UTF8}{mj}是线性无关的\end{CJK}, \begin{CJK}{UTF8}{mj}则\end{CJK} $k(\quad)$.
\end{enumerate}
\begin{CJK}{UTF8}{mj}一\end{CJK} \begin{CJK}{UTF8}{mj}设\end{CJK}
$$
A=\left(\begin{array}{ccc}
a & 1 & 2 \\
1 & b & 1 \\
1 & 3 b & 1
\end{array}\right)
$$
$B$ \begin{CJK}{UTF8}{mj}是三阶非零方阵\end{CJK}, \begin{CJK}{UTF8}{mj}且\end{CJK} $A B=O$, \begin{CJK}{UTF8}{mj}求\end{CJK} $a, b$ \begin{CJK}{UTF8}{mj}以及\end{CJK} $B$ \begin{CJK}{UTF8}{mj}的秩\end{CJK}.

\begin{CJK}{UTF8}{mj}三\end{CJK}. \begin{CJK}{UTF8}{mj}设\end{CJK} $A$ \begin{CJK}{UTF8}{mj}是\end{CJK} $n$ \begin{CJK}{UTF8}{mj}阶正定矩阵\end{CJK}, $B$ \begin{CJK}{UTF8}{mj}为\end{CJK} $n$ \begin{CJK}{UTF8}{mj}阶实方阵\end{CJK}, \begin{CJK}{UTF8}{mj}证明\end{CJK}:

(1) \begin{CJK}{UTF8}{mj}若\end{CJK} $B$ \begin{CJK}{UTF8}{mj}正定\end{CJK}, \begin{CJK}{UTF8}{mj}则\end{CJK} $A B$ \begin{CJK}{UTF8}{mj}的特征值皆大于\end{CJK} 0 .

(2) \begin{CJK}{UTF8}{mj}若\end{CJK} $B$ \begin{CJK}{UTF8}{mj}正定\end{CJK}, \begin{CJK}{UTF8}{mj}且\end{CJK} $A B=B A$, \begin{CJK}{UTF8}{mj}则\end{CJK} $A B$ \begin{CJK}{UTF8}{mj}正定\end{CJK}.

\begin{CJK}{UTF8}{mj}四\end{CJK}. $A$ \begin{CJK}{UTF8}{mj}为\end{CJK} $n$ \begin{CJK}{UTF8}{mj}阶方阵\end{CJK}, \begin{CJK}{UTF8}{mj}如果\end{CJK} $A^{2}=A$, \begin{CJK}{UTF8}{mj}其中\end{CJK} $E$ \begin{CJK}{UTF8}{mj}是\end{CJK} $n$ \begin{CJK}{UTF8}{mj}阶单位矩阵\end{CJK}.

\begin{CJK}{UTF8}{mj}車\end{CJK}. \begin{CJK}{UTF8}{mj}设\end{CJK}
$$
A=\left(\begin{array}{ccc}
1 & 4 & 2 \\
0 & -3 & 4 \\
0 & 4 & 3
\end{array}\right)
$$
\begin{CJK}{UTF8}{mj}试求\end{CJK} $A^{n}$.

\begin{CJK}{UTF8}{mj}六\end{CJK}. \begin{CJK}{UTF8}{mj}设\end{CJK} $A$ \begin{CJK}{UTF8}{mj}为\end{CJK} $n$ \begin{CJK}{UTF8}{mj}阶实方阵\end{CJK}, \begin{CJK}{UTF8}{mj}已知\end{CJK} $A$ \begin{CJK}{UTF8}{mj}的特征值全为实数\end{CJK}, \begin{CJK}{UTF8}{mj}且\end{CJK}
$$
A A^{T}=A^{T} A
$$
\begin{CJK}{UTF8}{mj}证明\end{CJK}: $A$ \begin{CJK}{UTF8}{mj}必为对称矩阵\end{CJK}. \begin{CJK}{UTF8}{mj}七\end{CJK}. \begin{CJK}{UTF8}{mj}设\end{CJK}
$$
A=\left(\begin{array}{ccc}
a & 1 & 2 \\
1 & b & 1 \\
1 & 3 b & 1
\end{array}\right) .
$$
$B$ \begin{CJK}{UTF8}{mj}是三阶非零方阵\end{CJK}, \begin{CJK}{UTF8}{mj}且\end{CJK} $A B=O$, \begin{CJK}{UTF8}{mj}求\end{CJK} $a, b$ \begin{CJK}{UTF8}{mj}以及\end{CJK} $B$ \begin{CJK}{UTF8}{mj}的秩\end{CJK}.

\begin{CJK}{UTF8}{mj}八\end{CJK}. \begin{CJK}{UTF8}{mj}设\end{CJK} $A, B$ \begin{CJK}{UTF8}{mj}为\end{CJK} $n$ \begin{CJK}{UTF8}{mj}元实对称矩阵\end{CJK}, \begin{CJK}{UTF8}{mj}且\end{CJK} $B$ \begin{CJK}{UTF8}{mj}正定\end{CJK}, \begin{CJK}{UTF8}{mj}求证\end{CJK}: \begin{CJK}{UTF8}{mj}存在一个实可逆阵\end{CJK} $P$ \begin{CJK}{UTF8}{mj}使得\end{CJK} $P^{T} A P$ \begin{CJK}{UTF8}{mj}和\end{CJK} $P^{T} B P$ \begin{CJK}{UTF8}{mj}同时为对角\end{CJK} \begin{CJK}{UTF8}{mj}阵\end{CJK}.

\section{5. 哈尔滨工程大学 2013 年研究生入学考试试题高等代数}
\begin{CJK}{UTF8}{mj}李扬\end{CJK}

\begin{CJK}{UTF8}{mj}微信公众号\end{CJK}: sxkyliyang

\begin{CJK}{UTF8}{mj}一\end{CJK}. \begin{CJK}{UTF8}{mj}填空题\end{CJK}

\begin{enumerate}
  \item \begin{CJK}{UTF8}{mj}设\end{CJK} $f(x)=x^{4}-10 x^{2}+1, g(x)=x^{4}-4 \sqrt{2} x^{3}+6 x^{2}+4 \sqrt{2} x+1$, \begin{CJK}{UTF8}{mj}则\end{CJK} $f(x)$ \begin{CJK}{UTF8}{mj}与\end{CJK} $g(x)$ \begin{CJK}{UTF8}{mj}的首\end{CJK} 1 \begin{CJK}{UTF8}{mj}最大公因式为\end{CJK} ( ).

  \item \begin{CJK}{UTF8}{mj}行列式\end{CJK} $\left|\begin{array}{lllll}5 & 3 & 3 & 3 & 3 \\ 3 & 5 & 3 & 3 & 3 \\ 3 & 3 & 5 & 3 & 3 \\ 3 & 3 & 3 & 5 & 3 \\ 3 & 3 & 3 & 3 & 5\end{array}\right|$ \begin{CJK}{UTF8}{mj}中\end{CJK},\begin{CJK}{UTF8}{mj}第一行元素代数余子式之和为\end{CJK}(

  \item $A=\left(\begin{array}{ccc}4 & 0 & 2 \\ 8 & 0 & 4 \\ -10 & 0 & -5\end{array}\right)$, \begin{CJK}{UTF8}{mj}则\end{CJK} $A^{2015}=(\quad)$.

  \item \begin{CJK}{UTF8}{mj}设\end{CJK} $A$ \begin{CJK}{UTF8}{mj}为\end{CJK} 3 \begin{CJK}{UTF8}{mj}阶方阵\end{CJK}, $|A|=2$, \begin{CJK}{UTF8}{mj}则\end{CJK} $A^{*}$ \begin{CJK}{UTF8}{mj}为\end{CJK} $A$ \begin{CJK}{UTF8}{mj}的伴随阵\end{CJK}, \begin{CJK}{UTF8}{mj}若\end{CJK} $M=\left(\begin{array}{cc}A^{2}+3 A^{*} & 2 A^{*} \\ A & 0\end{array}\right)$, \begin{CJK}{UTF8}{mj}则\end{CJK} $\left(M^{-1}\right)^{*}=\left(\begin{array}{c}\text { ( } \\ 0\end{array}\right)$.

  \item \begin{CJK}{UTF8}{mj}多项式空间\end{CJK} $\mathbb{R}[x]_{2}$ \begin{CJK}{UTF8}{mj}上定义内积\end{CJK} $(f(x), g(x))=\int_{0}^{1} f(x) g(x) \mathrm{d} x$, \begin{CJK}{UTF8}{mj}则\end{CJK} $\mathbb{R}[x]_{2}$ \begin{CJK}{UTF8}{mj}的一组标准正交基为\end{CJK} $f_{1}(x)=$ $1, f_{2}(x)=()$.

  \item \begin{CJK}{UTF8}{mj}线性空间\end{CJK} $\mathbb{R}^{2 \times 2}$ \begin{CJK}{UTF8}{mj}中\end{CJK},

\end{enumerate}
$$
\begin{aligned}
&\text { 基 (1): } A_{1}=\left(\begin{array}{ll}
1 & 0 \\
0 & 0
\end{array}\right), A_{2}=\left(\begin{array}{ll}
1 & 1 \\
0 & 0
\end{array}\right), A_{3}=\left(\begin{array}{ll}
1 & 1 \\
1 & 0
\end{array}\right), A_{4}=\left(\begin{array}{ll}
1 & 1 \\
1 & 1
\end{array}\right) \\
&\text { 基 }(2): B_{1}=\left(\begin{array}{ll}
1 & 0 \\
1 & 1
\end{array}\right), B_{2}=\left(\begin{array}{ll}
0 & 1 \\
1 & 1
\end{array}\right), B_{3}=\left(\begin{array}{ll}
1 & 1 \\
1 & 0
\end{array}\right), B_{4}=\left(\begin{array}{ll}
1 & 1 \\
0 & 1
\end{array}\right)
\end{aligned}
$$
\begin{CJK}{UTF8}{mj}则在基\end{CJK} $(1)$ \begin{CJK}{UTF8}{mj}与基\end{CJK} (2) \begin{CJK}{UTF8}{mj}下有相同坐标的矩阵的为\end{CJK} $k=(\quad)$ ( $k$ \begin{CJK}{UTF8}{mj}为任意常数\end{CJK}).

\begin{enumerate}
  \setcounter{enumi}{7}
  \item \begin{CJK}{UTF8}{mj}设\end{CJK} $A$ \begin{CJK}{UTF8}{mj}为\end{CJK} 3 \begin{CJK}{UTF8}{mj}阶半正定阵\end{CJK}, \begin{CJK}{UTF8}{mj}向量\end{CJK} $\alpha, \beta$ \begin{CJK}{UTF8}{mj}线性无关\end{CJK}, \begin{CJK}{UTF8}{mj}若\end{CJK} $\alpha^{T} A \alpha=\beta^{T} A \beta=0$, \begin{CJK}{UTF8}{mj}且\end{CJK} $\operatorname{tr}(A)=2$, \begin{CJK}{UTF8}{mj}则二次型\end{CJK} $f\left(x_{1}, x_{2}, x_{3}\right)=x^{T} A x$ \begin{CJK}{UTF8}{mj}经正交变换\end{CJK} $x=P y$ \begin{CJK}{UTF8}{mj}化成的标准型为\end{CJK} $(\quad)$.

  \item \begin{CJK}{UTF8}{mj}设\end{CJK} $A=\left(\begin{array}{ll}1 & 3 \\ 0 & 2\end{array}\right)$, \begin{CJK}{UTF8}{mj}则\end{CJK} $3 A^{8}-9 A^{7}+6 A^{6}+A^{5}-3 A^{4}+2 A^{3}+2 A-E=(\quad)$.

  \item \begin{CJK}{UTF8}{mj}设\end{CJK} 3 \begin{CJK}{UTF8}{mj}阶方阵\end{CJK} $A$ \begin{CJK}{UTF8}{mj}的特征值为\end{CJK} $-1,-2,-2$, \begin{CJK}{UTF8}{mj}则\end{CJK} $\left|\left(\frac{1}{2} A\right)^{*}\right|=(\quad)$.

\end{enumerate}
\section{6. 哈尔滨工程大学 2014 年研究生入学考试试题高等代数}
\begin{CJK}{UTF8}{mj}李扬\end{CJK}

\begin{CJK}{UTF8}{mj}微信公众号\end{CJK}: sxkyliyang

\begin{CJK}{UTF8}{mj}一\end{CJK}. \begin{CJK}{UTF8}{mj}填空题\end{CJK}

\begin{enumerate}
  \item \begin{CJK}{UTF8}{mj}当\end{CJK} $a, b$ \begin{CJK}{UTF8}{mj}满足\end{CJK} $(\quad)$ \begin{CJK}{UTF8}{mj}时\end{CJK}, \begin{CJK}{UTF8}{mj}多项式\end{CJK} $f(x)=x^{4}+4 a x=b$ \begin{CJK}{UTF8}{mj}有重根\end{CJK}.
\end{enumerate}
$$
\text { 阶行列式 }\left|\begin{array}{cccccc}
5 & 3 & 0 & \cdots & 0 & 0 \\
2 & 5 & 3 & \cdots & 0 & 0 \\
0 & 2 & 5 & \cdots & 0 & 0 \\
\vdots & \vdots & \vdots & & \vdots & \vdots \\
0 & 0 & 0 & \cdots & 5 & 3 \\
0 & 0 & 0 & \cdots & 2 & 5
\end{array}\right| \text { 的值为 }(
$$

\begin{enumerate}
  \setcounter{enumi}{3}
  \item \begin{CJK}{UTF8}{mj}设\end{CJK} $A$ \begin{CJK}{UTF8}{mj}为\end{CJK} $n$ \begin{CJK}{UTF8}{mj}阶反对称阵\end{CJK}, $\alpha$ \begin{CJK}{UTF8}{mj}是\end{CJK} $n$ \begin{CJK}{UTF8}{mj}维单位列向量\end{CJK}, \begin{CJK}{UTF8}{mj}则\end{CJK} $\alpha^{T}(A-E) \alpha=(\quad)$.

  \item \begin{CJK}{UTF8}{mj}已知向量组\end{CJK} $\alpha_{1}, \alpha_{2}, \alpha_{3}$ \begin{CJK}{UTF8}{mj}线性无关\end{CJK}, \begin{CJK}{UTF8}{mj}向量组\end{CJK} $\alpha_{1}, \alpha_{2}, \alpha_{3}, \alpha_{4}$ \begin{CJK}{UTF8}{mj}的秩为\end{CJK} 3 , \begin{CJK}{UTF8}{mj}向量组\end{CJK} $\alpha_{1}, \alpha_{2}, \alpha_{3}, \alpha_{5}$ \begin{CJK}{UTF8}{mj}的秩为\end{CJK} 4 , \begin{CJK}{UTF8}{mj}则\end{CJK} \begin{CJK}{UTF8}{mj}向量组\end{CJK} $\alpha_{1}, \alpha_{2}, \alpha_{3}, \alpha_{5}-\alpha_{4}$ \begin{CJK}{UTF8}{mj}的秩为\end{CJK} ( ).

  \item \begin{CJK}{UTF8}{mj}已知向量组\end{CJK} $\alpha_{2}, \alpha_{3}, \alpha_{4}$ \begin{CJK}{UTF8}{mj}线性无关\end{CJK}, $\alpha_{1}=2 \alpha_{2}-\alpha_{3}, \beta=\alpha_{1}+\alpha_{2}+\alpha_{3}+\alpha_{4}, A=\left(\alpha_{1}, \alpha_{2}, \alpha_{3}, \alpha_{4}\right)$, \begin{CJK}{UTF8}{mj}则方程\end{CJK} \begin{CJK}{UTF8}{mj}组\end{CJK} $A X=\beta$ \begin{CJK}{UTF8}{mj}的通解为\end{CJK} ( ).

  \item \begin{CJK}{UTF8}{mj}线性空间\end{CJK} $\mathbb{R}^{2 \times 2}$ \begin{CJK}{UTF8}{mj}中\end{CJK},

\end{enumerate}
$$
\text { 基 (1): } A_{1}=\left(\begin{array}{ll}
1 & 0 \\
0 & 0
\end{array}\right), A_{2}=\left(\begin{array}{cc}
1 & 1 \\
0 & 0
\end{array}\right), A_{3}=\left(\begin{array}{ll}
1 & 1 \\
1 & 0
\end{array}\right), A_{4}=\left(\begin{array}{ll}
1 & 1 \\
1 & 1
\end{array}\right)
$$
\begin{CJK}{UTF8}{mj}至\end{CJK}!
$$
\text { 基 }(2): B_{1}=\left(\begin{array}{ll}
1 & 0 \\
1 & 1
\end{array}\right), B_{2}=\left(\begin{array}{ll}
0 & 1 \\
1 & 1
\end{array}\right), B_{3}=\left(\begin{array}{cc}
1 & 1 \\
1 & 0
\end{array}\right), B_{4}=\left(\begin{array}{ll}
1 & 1 \\
0 & 1
\end{array}\right)
$$
\begin{CJK}{UTF8}{mj}的过渡矩阵为\end{CJK} ( ).

\begin{enumerate}
  \setcounter{enumi}{7}
  \item \begin{CJK}{UTF8}{mj}当\end{CJK} $a$ \begin{CJK}{UTF8}{mj}满足\end{CJK} $(\quad)$ \begin{CJK}{UTF8}{mj}时\end{CJK}, \begin{CJK}{UTF8}{mj}实二次型\end{CJK} $f\left(x_{1}, x_{2}, x_{3}\right)=x_{1}^{2}+x_{2}^{2}+x_{3}^{2}+2 a x_{1} x_{2}$ \begin{CJK}{UTF8}{mj}正定\end{CJK}.

  \item \begin{CJK}{UTF8}{mj}矩阵\end{CJK} $A=\left(\begin{array}{lll}1 & 2 & 3 \\ 0 & 4 & 5 \\ 0 & 0 & 4\end{array}\right)$ \begin{CJK}{UTF8}{mj}的约当标准型为\end{CJK} ( ).

  \item \begin{CJK}{UTF8}{mj}设\end{CJK} $A$ \begin{CJK}{UTF8}{mj}为\end{CJK} 3 \begin{CJK}{UTF8}{mj}阶奇异阵\end{CJK}, $A+E$ \begin{CJK}{UTF8}{mj}的行向量组线性相关\end{CJK}, \begin{CJK}{UTF8}{mj}秩\end{CJK} $(A+2 E)=2$, \begin{CJK}{UTF8}{mj}则\end{CJK} $|A+3 E|=(\quad)$.

  \item \begin{CJK}{UTF8}{mj}在向量空间\end{CJK} $\mathbb{R}^{2}$ \begin{CJK}{UTF8}{mj}中规定内积\end{CJK} (\begin{CJK}{UTF8}{mj}不一定是标准内积\end{CJK}) \begin{CJK}{UTF8}{mj}后得到欧式空间\end{CJK} $V$, \begin{CJK}{UTF8}{mj}且\end{CJK} $V$ \begin{CJK}{UTF8}{mj}的基\end{CJK} $\alpha_{1}=(2,1), \alpha_{2}=(3,2)$ \begin{CJK}{UTF8}{mj}的度量矩阵为\end{CJK} $A=\left[\begin{array}{cc}6 & 10 \\ 10 & 17\end{array}\right]$, \begin{CJK}{UTF8}{mj}则基\end{CJK} $e_{1}=(1,0), e_{2}=(0,1)$ \begin{CJK}{UTF8}{mj}的度量矩阵为\end{CJK} ( ).

\end{enumerate}
\begin{CJK}{UTF8}{mj}二\end{CJK}. \begin{CJK}{UTF8}{mj}设\end{CJK} $V$ \begin{CJK}{UTF8}{mj}为\end{CJK} $\mathbb{R}$ \begin{CJK}{UTF8}{mj}上的三维线性空间\end{CJK}, $\mathscr{A}$ \begin{CJK}{UTF8}{mj}为\end{CJK} $V$ \begin{CJK}{UTF8}{mj}的一个线性变换\end{CJK}, $\alpha_{1}, \alpha_{2}, \alpha_{3}$ \begin{CJK}{UTF8}{mj}是\end{CJK} $V$ \begin{CJK}{UTF8}{mj}的一组基\end{CJK},
$$
\begin{aligned}
&\mathscr{A}\left(\alpha_{1}\right)=2 \alpha_{1}+\alpha_{2}+\alpha_{3} \\
&\mathscr{A}\left(\alpha_{2}\right)=\alpha_{1}+2 \alpha_{2}+\alpha_{3}
\end{aligned}
$$
$$
\mathscr{A}\left(\alpha_{3}\right)=\alpha_{1}+\alpha_{2}+2 \alpha_{3}
$$
(1) \begin{CJK}{UTF8}{mj}求\end{CJK} $\mathscr{A}$ \begin{CJK}{UTF8}{mj}在基\end{CJK} $\alpha_{1}, \alpha_{2}, \alpha_{3}$ \begin{CJK}{UTF8}{mj}下的矩阵\end{CJK}.

(2) \begin{CJK}{UTF8}{mj}求\end{CJK} $\mathscr{A}$ \begin{CJK}{UTF8}{mj}的特征值\end{CJK}, \begin{CJK}{UTF8}{mj}特征向量\end{CJK}.

(3) \begin{CJK}{UTF8}{mj}求\end{CJK} $V$ \begin{CJK}{UTF8}{mj}的一组基\end{CJK}, \begin{CJK}{UTF8}{mj}使\end{CJK} $\mathscr{A}$ \begin{CJK}{UTF8}{mj}在该基下的矩阵为对角阵\end{CJK}.

\begin{CJK}{UTF8}{mj}三\end{CJK}. \begin{CJK}{UTF8}{mj}设\end{CJK} $A$ \begin{CJK}{UTF8}{mj}为\end{CJK} $n$ \begin{CJK}{UTF8}{mj}阶方阵\end{CJK} $(n>1)$, \begin{CJK}{UTF8}{mj}求证\end{CJK}:

(1) \begin{CJK}{UTF8}{mj}若\end{CJK} $\mathrm{r}(A)=1$, \begin{CJK}{UTF8}{mj}则存在\end{CJK} $n$ \begin{CJK}{UTF8}{mj}行\end{CJK} 1 \begin{CJK}{UTF8}{mj}列矩阵\end{CJK} $B$ \begin{CJK}{UTF8}{mj}和\end{CJK} 1 \begin{CJK}{UTF8}{mj}行\end{CJK} $n$ \begin{CJK}{UTF8}{mj}列矩阵\end{CJK} $C$ \begin{CJK}{UTF8}{mj}使\end{CJK} $A=B C$.

(2) \begin{CJK}{UTF8}{mj}若\end{CJK} $\mathrm{r}(A)=1$, \begin{CJK}{UTF8}{mj}且\end{CJK} $\operatorname{tr} A=1$ \begin{CJK}{UTF8}{mj}则\end{CJK} $A^{n}=A$.

\begin{CJK}{UTF8}{mj}四\end{CJK}. \begin{CJK}{UTF8}{mj}设\end{CJK}
$$
V=\left\{A \mid \operatorname{tr} A=0, A \in \mathbb{R}^{2 \times 2}\right\} .
$$
(1) \begin{CJK}{UTF8}{mj}求证\end{CJK}: $V$ \begin{CJK}{UTF8}{mj}按通常的矩阵加法和数乘构成实数域上的线性空间\end{CJK}.

(2) \begin{CJK}{UTF8}{mj}求\end{CJK} $\operatorname{dim} V$, \begin{CJK}{UTF8}{mj}找出\end{CJK} $V$ \begin{CJK}{UTF8}{mj}的一组基\end{CJK}, \begin{CJK}{UTF8}{mj}并用基的定义说明找出矩阵是\end{CJK} $V$ \begin{CJK}{UTF8}{mj}的基\end{CJK}.

\begin{CJK}{UTF8}{mj}五\end{CJK}. \begin{CJK}{UTF8}{mj}设有向量组\end{CJK} $\alpha_{1}=(1,1,1,2), \alpha_{2}=(3, a+4,2 a+5, a+7), \alpha_{3}=(4,6,8,10), \alpha_{4}=(2,3,2 a+3,5)$. \begin{CJK}{UTF8}{mj}当\end{CJK} $a, b$ \begin{CJK}{UTF8}{mj}如何取值时\end{CJK} $\beta=(0,1,3, b)$ \begin{CJK}{UTF8}{mj}不能由\end{CJK} $\alpha_{1}, \alpha_{2}, \alpha_{3}, \alpha_{4}$ \begin{CJK}{UTF8}{mj}线性表示\end{CJK}?

\begin{CJK}{UTF8}{mj}六\end{CJK}. \begin{CJK}{UTF8}{mj}设\end{CJK} $V$ \begin{CJK}{UTF8}{mj}为\end{CJK} $\mathbb{R}$ \begin{CJK}{UTF8}{mj}上的\end{CJK} $n$ \begin{CJK}{UTF8}{mj}维线性空间\end{CJK}, $V_{1}, V_{2}, V_{3}$ \begin{CJK}{UTF8}{mj}是\end{CJK} $V$ \begin{CJK}{UTF8}{mj}的子空间\end{CJK}.

(1) \begin{CJK}{UTF8}{mj}判断命题\end{CJK} “\begin{CJK}{UTF8}{mj}若\end{CJK} $V_{1} \cap V_{2}=\{0\}, V_{2} \cap V_{3}=\{0\}, V_{3} \cap V_{1}=\{0\}$ \begin{CJK}{UTF8}{mj}则\end{CJK} $V_{1}+V_{2}+V_{3}$ \begin{CJK}{UTF8}{mj}为直和\end{CJK}" \begin{CJK}{UTF8}{mj}是否正确\end{CJK}, \begin{CJK}{UTF8}{mj}若正\end{CJK} \begin{CJK}{UTF8}{mj}确给出证明\end{CJK}, \begin{CJK}{UTF8}{mj}若不正确举出反例\end{CJK}.

(2) \begin{CJK}{UTF8}{mj}判断命题\end{CJK} “\begin{CJK}{UTF8}{mj}若\end{CJK} $V_{1} \cap V_{2}=\{0\}, V_{3} \cap\left(V_{1}+V_{2}\right)=\{0\}$ \begin{CJK}{UTF8}{mj}则\end{CJK} $V_{1}+V_{2}+V_{3}$ \begin{CJK}{UTF8}{mj}为直和\end{CJK}” \begin{CJK}{UTF8}{mj}是否正确\end{CJK}, \begin{CJK}{UTF8}{mj}若正确给出\end{CJK} \begin{CJK}{UTF8}{mj}证明\end{CJK}, \begin{CJK}{UTF8}{mj}若不正确举出反例\end{CJK}.

\begin{CJK}{UTF8}{mj}七\end{CJK}. \begin{CJK}{UTF8}{mj}设\end{CJK} $n$ \begin{CJK}{UTF8}{mj}阶实对称阵\end{CJK} $A$ \begin{CJK}{UTF8}{mj}的特征值\end{CJK} $\lambda_{1}, \lambda_{2}, \cdots, \lambda_{n}$ \begin{CJK}{UTF8}{mj}满足\end{CJK} $1<\lambda_{1} \leq \lambda_{2} \leq \cdots \leq \lambda_{n}<2$, \begin{CJK}{UTF8}{mj}求证\end{CJK}: \begin{CJK}{UTF8}{mj}对任意零实向量\end{CJK} $X$ \begin{CJK}{UTF8}{mj}志有\end{CJK}
$$
X^{T} X<X^{T} A X<2 X^{T} X
$$
\begin{CJK}{UTF8}{mj}八\end{CJK}. \begin{CJK}{UTF8}{mj}求证\end{CJK}: \begin{CJK}{UTF8}{mj}在\end{CJK} $n$ \begin{CJK}{UTF8}{mj}维欧式空间中\end{CJK}, \begin{CJK}{UTF8}{mj}两两夹角成钝角的元素不多于\end{CJK} $n+1$ \begin{CJK}{UTF8}{mj}个\end{CJK}.

\section{7. 哈尔滨工程大学 2015 年研究生入学考试试题高等代数}
\begin{CJK}{UTF8}{mj}李扬\end{CJK}

\begin{CJK}{UTF8}{mj}微信公众号\end{CJK}: sxkyliyang

\begin{CJK}{UTF8}{mj}一\end{CJK}. \begin{CJK}{UTF8}{mj}填空题\end{CJK}

\begin{enumerate}
  \item \begin{CJK}{UTF8}{mj}若\end{CJK} $P$ \begin{CJK}{UTF8}{mj}为包含\end{CJK} $\mathbb{Q}$ \begin{CJK}{UTF8}{mj}和\end{CJK} $\sqrt{3}$ \begin{CJK}{UTF8}{mj}的最小数域\end{CJK}, \begin{CJK}{UTF8}{mj}则\end{CJK} $P$ \begin{CJK}{UTF8}{mj}视为\end{CJK} $\mathbb{Q}$ \begin{CJK}{UTF8}{mj}上的线性空间其维数是\end{CJK} ( ) ).

  \item \begin{CJK}{UTF8}{mj}若\end{CJK} $f(x)$ \begin{CJK}{UTF8}{mj}为数域\end{CJK} $P$ \begin{CJK}{UTF8}{mj}上的不可约多项式\end{CJK}, \begin{CJK}{UTF8}{mj}则\end{CJK} $f(x)$ \begin{CJK}{UTF8}{mj}与\end{CJK} $f^{\prime}(x)$ \begin{CJK}{UTF8}{mj}的关系是\end{CJK} $(\quad)$.

  \item \begin{CJK}{UTF8}{mj}若\end{CJK} $A$ \begin{CJK}{UTF8}{mj}为奇数阶反对称阵\end{CJK}, \begin{CJK}{UTF8}{mj}则\end{CJK} $|A|=(\quad)$.

  \item \begin{CJK}{UTF8}{mj}设\end{CJK} $A$ \begin{CJK}{UTF8}{mj}为方阵\end{CJK}, \begin{CJK}{UTF8}{mj}且\end{CJK} $A^{3}=0$, \begin{CJK}{UTF8}{mj}则\end{CJK} $(E-A)^{-1}=(\quad)$.

  \item \begin{CJK}{UTF8}{mj}向量组\end{CJK} $\alpha_{1}, \alpha_{2}, \alpha_{3}, \alpha_{4}, \alpha_{5} \in \mathbb{R}^{5}$ \begin{CJK}{UTF8}{mj}线性无关\end{CJK}, \begin{CJK}{UTF8}{mj}则向量组\end{CJK} $\alpha_{1}+\alpha_{2}, \alpha_{2}+\alpha_{3}, \alpha_{3}+\alpha_{4}, \alpha_{4}+\alpha_{5}, \alpha_{5}+\alpha_{1}$ \begin{CJK}{UTF8}{mj}的线\end{CJK} \begin{CJK}{UTF8}{mj}性相关性是\end{CJK} ( $)$.

  \item \begin{CJK}{UTF8}{mj}设\end{CJK} $A, B$ \begin{CJK}{UTF8}{mj}为\end{CJK} $n$ \begin{CJK}{UTF8}{mj}阶方阵\end{CJK}, \begin{CJK}{UTF8}{mj}且\end{CJK} $A B=0$, \begin{CJK}{UTF8}{mj}则\end{CJK} $\mathrm{r}(A)+\mathrm{r}(B) \leq(\quad)$.

  \item \begin{CJK}{UTF8}{mj}设\end{CJK} $\mathscr{A}$ \begin{CJK}{UTF8}{mj}为\end{CJK} $n$ \begin{CJK}{UTF8}{mj}维线性空间\end{CJK} $V$ \begin{CJK}{UTF8}{mj}的线性变换\end{CJK}, $\operatorname{ker} \mathscr{A}=0$, \begin{CJK}{UTF8}{mj}则\end{CJK} $\mathscr{A}$ \begin{CJK}{UTF8}{mj}为\end{CJK} $(\quad)$ \begin{CJK}{UTF8}{mj}线性变换\end{CJK}

  \item \begin{CJK}{UTF8}{mj}设\end{CJK} $A, B$ \begin{CJK}{UTF8}{mj}为\end{CJK} $n$ \begin{CJK}{UTF8}{mj}阶方阵\end{CJK}, \begin{CJK}{UTF8}{mj}且\end{CJK} $A$ \begin{CJK}{UTF8}{mj}可逆\end{CJK}, \begin{CJK}{UTF8}{mj}则\end{CJK} $A B$ \begin{CJK}{UTF8}{mj}与\end{CJK} $B A$ \begin{CJK}{UTF8}{mj}的关系是\end{CJK} ( ).

  \item \begin{CJK}{UTF8}{mj}若\end{CJK} $A, B$ \begin{CJK}{UTF8}{mj}为同阶正交阵\end{CJK}, \begin{CJK}{UTF8}{mj}且\end{CJK} $|A B|=-1$, \begin{CJK}{UTF8}{mj}则\end{CJK} $|A+B|=(\quad)$.

  \item \begin{CJK}{UTF8}{mj}设\end{CJK} $A$ \begin{CJK}{UTF8}{mj}为\end{CJK} $m \times n$ \begin{CJK}{UTF8}{mj}实矩阵\end{CJK}, $r(A)=n$, \begin{CJK}{UTF8}{mj}则\end{CJK} $n$ \begin{CJK}{UTF8}{mj}元二次型\end{CJK} $X^{T}\left(A^{T} A\right) X$ \begin{CJK}{UTF8}{mj}正定性为\end{CJK} ( )

\end{enumerate}
\begin{CJK}{UTF8}{mj}二\end{CJK}. \begin{CJK}{UTF8}{mj}设\end{CJK} $V$ \begin{CJK}{UTF8}{mj}为\end{CJK} $\mathbb{R}$ \begin{CJK}{UTF8}{mj}上的三维线性空间\end{CJK}, $\mathscr{A}$ \begin{CJK}{UTF8}{mj}为\end{CJK} $V$ \begin{CJK}{UTF8}{mj}的一个线性变换\end{CJK}, \begin{CJK}{UTF8}{mj}且\end{CJK} $\mathscr{A}$ \begin{CJK}{UTF8}{mj}在\end{CJK} $V$ \begin{CJK}{UTF8}{mj}的基\end{CJK} $\alpha_{1}, \alpha_{2}, \alpha_{3}$ \begin{CJK}{UTF8}{mj}下的矩阵为\end{CJK}
$$
A=\left[\begin{array}{ccc}
1 & 4 & 2 \\
0 & -3 & 4 \\
0 & 4 & 3
\end{array}\right] \text {. }
$$
(1) \begin{CJK}{UTF8}{mj}求\end{CJK} $V$ \begin{CJK}{UTF8}{mj}的另一个基\end{CJK} $\beta_{1}, \beta_{2}, \beta_{3}$ \begin{CJK}{UTF8}{mj}使\end{CJK} $\mathscr{A}$ \begin{CJK}{UTF8}{mj}在此基下的矩阵\end{CJK} $B$ \begin{CJK}{UTF8}{mj}为对角阵\end{CJK}.

(2) \begin{CJK}{UTF8}{mj}求\end{CJK} $A^{k}$.

\begin{CJK}{UTF8}{mj}三\end{CJK}. \begin{CJK}{UTF8}{mj}对齐次线性方程组\end{CJK}
$$
\left\{\begin{array}{l}
x_{1}+x_{2}+x_{3}+x_{4}+x_{5}=0 \\
3 x_{1}+2 x_{2}+x_{3}+x_{4}-3 x_{5}=0 \\
x_{2}+2 x_{3}+2 x_{4}+6 x_{5}=0 \\
5 x_{1}+4 x_{2}+3 x_{3}+3 x_{4}-x_{5}=0
\end{array}\right.
$$
(1) \begin{CJK}{UTF8}{mj}求其中一个基础解系\end{CJK}.

(2) \begin{CJK}{UTF8}{mj}求其向量形式的通解\end{CJK}.

\begin{CJK}{UTF8}{mj}四\end{CJK}. \begin{CJK}{UTF8}{mj}设\end{CJK} $\mathscr{A}, \mathscr{B}$ \begin{CJK}{UTF8}{mj}为\end{CJK} $n$ \begin{CJK}{UTF8}{mj}维线性空间\end{CJK} $V$ \begin{CJK}{UTF8}{mj}上的线性变换\end{CJK},
$$
(\mathscr{A}+\mathscr{B})^{2}=\mathscr{A}+\mathscr{B}, \mathscr{A}^{2}=\mathscr{A}, \mathscr{B}^{2}=\mathscr{B}
$$
\begin{CJK}{UTF8}{mj}求证\end{CJK}: $\mathscr{A} \mathscr{B}=0$.

\begin{CJK}{UTF8}{mj}五\end{CJK}. \begin{CJK}{UTF8}{mj}设\end{CJK} $\mathscr{A}$ \begin{CJK}{UTF8}{mj}为数域\end{CJK} $P$ \begin{CJK}{UTF8}{mj}上\end{CJK} $n$ \begin{CJK}{UTF8}{mj}维线性空间\end{CJK} $V$ \begin{CJK}{UTF8}{mj}上的线性变换\end{CJK}, $f_{1}(x), f_{2}(x)$ \begin{CJK}{UTF8}{mj}为\end{CJK} $P[x]$ \begin{CJK}{UTF8}{mj}中两个互素的多项式\end{CJK}, $f(x)=$ $f_{1}(x) f_{2}(x)$, \begin{CJK}{UTF8}{mj}求证\end{CJK}:
$$
\operatorname{ker} f(\mathscr{A})=\operatorname{ker} f_{1}(\mathscr{A}) \oplus \operatorname{ker} f_{2}(\mathscr{A})
$$
\begin{CJK}{UTF8}{mj}六\end{CJK}. \begin{CJK}{UTF8}{mj}设\end{CJK} $V$ \begin{CJK}{UTF8}{mj}为数域\end{CJK} $\mathbb{F}$ \begin{CJK}{UTF8}{mj}上的\end{CJK} $n$ \begin{CJK}{UTF8}{mj}维线性空间\end{CJK}, $\mathscr{A}$ \begin{CJK}{UTF8}{mj}为\end{CJK} $V$ \begin{CJK}{UTF8}{mj}的线性变换\end{CJK},
$$
\mathscr{A}^{2}=\mathscr{A}
$$
\begin{CJK}{UTF8}{mj}求证\end{CJK}:

(1) $V=\mathscr{A}(V) \oplus \operatorname{ker} \mathscr{A}$.

(2) \begin{CJK}{UTF8}{mj}存在\end{CJK} $V$ \begin{CJK}{UTF8}{mj}的一个基\end{CJK} $\varepsilon_{1}, \varepsilon_{2}, \cdots, \varepsilon_{n}$, \begin{CJK}{UTF8}{mj}在此基下\end{CJK} $\mathscr{A}$ \begin{CJK}{UTF8}{mj}的矩阵为\end{CJK} $A=\operatorname{diag}\{1, \cdots, 1,0, \cdots, 0\}$ (\begin{CJK}{UTF8}{mj}对角线为\end{CJK} $1, \cdots, 1,0, \cdots, 0$ \begin{CJK}{UTF8}{mj}的对角阵\end{CJK}).

\begin{CJK}{UTF8}{mj}七\end{CJK}. \begin{CJK}{UTF8}{mj}设\end{CJK} $A$ \begin{CJK}{UTF8}{mj}为\end{CJK} $n$ \begin{CJK}{UTF8}{mj}阶方阵\end{CJK}, $\lambda_{0}$ \begin{CJK}{UTF8}{mj}为\end{CJK} $A$ \begin{CJK}{UTF8}{mj}的特征值\end{CJK}. \begin{CJK}{UTF8}{mj}此时我们称\end{CJK} $n-r\left(\lambda_{0} E-A\right)$ \begin{CJK}{UTF8}{mj}为\end{CJK} $\lambda_{0}$ \begin{CJK}{UTF8}{mj}的几何重数\end{CJK}, $\lambda_{0}$ \begin{CJK}{UTF8}{mj}作为\end{CJK} $A$ \begin{CJK}{UTF8}{mj}的特征\end{CJK} \begin{CJK}{UTF8}{mj}多项式\end{CJK} $|\lambda E-A|$ \begin{CJK}{UTF8}{mj}之根的重数称为\end{CJK} $\lambda_{0}$ \begin{CJK}{UTF8}{mj}的代数重数\end{CJK}. \begin{CJK}{UTF8}{mj}求证\end{CJK}: $\lambda_{0}$ \begin{CJK}{UTF8}{mj}的几何重数不超过其代数重数\end{CJK}.

\begin{CJK}{UTF8}{mj}八\end{CJK}. \begin{CJK}{UTF8}{mj}设\end{CJK} $A, B$ \begin{CJK}{UTF8}{mj}为\end{CJK} $n$ \begin{CJK}{UTF8}{mj}阶实对称阵\end{CJK}, \begin{CJK}{UTF8}{mj}且\end{CJK} $A$ \begin{CJK}{UTF8}{mj}正定\end{CJK}, \begin{CJK}{UTF8}{mj}求证\end{CJK}: \begin{CJK}{UTF8}{mj}存在一个可逆矩阵\end{CJK} $P$ \begin{CJK}{UTF8}{mj}使得\end{CJK} $P^{T} A P$ \begin{CJK}{UTF8}{mj}和\end{CJK} $P^{T} B P$ \begin{CJK}{UTF8}{mj}同时为对角阵\end{CJK}.

\section{8. 哈尔滨工程大学 2016 年研究生入学考试试题高等代数}
\begin{CJK}{UTF8}{mj}李扬\end{CJK}

\begin{CJK}{UTF8}{mj}微信公众号\end{CJK}: sxkyliyang

\begin{CJK}{UTF8}{mj}一\end{CJK}. \begin{CJK}{UTF8}{mj}填空题\end{CJK} (\begin{CJK}{UTF8}{mj}每小题\end{CJK} 4 \begin{CJK}{UTF8}{mj}分\end{CJK}, \begin{CJK}{UTF8}{mj}共\end{CJK} 20 \begin{CJK}{UTF8}{mj}分\end{CJK})

\begin{enumerate}
  \item \begin{CJK}{UTF8}{mj}设\end{CJK} $(x-1)^{2} \mid A x^{4}+B x^{2}+1$, \begin{CJK}{UTF8}{mj}则\end{CJK} $A+B=(\quad)$.

  \item \begin{CJK}{UTF8}{mj}设\end{CJK} $f(x)=\left|\begin{array}{cccc}1 & 1 & 1 & 1 \\ 1 & x & x^{2} & x^{3} \\ 1 & -2 & 4 & -8 \\ 1 & 3 & 9 & 27\end{array}\right|=0$ \begin{CJK}{UTF8}{mj}的三个根为\end{CJK} $x_{1}, x_{2}, x_{3}$, \begin{CJK}{UTF8}{mj}则\end{CJK} $x_{1}+x_{2}+x_{3}=(\quad)$.

  \item \begin{CJK}{UTF8}{mj}若方程组\end{CJK} $\left\{\begin{array}{l}x_{1}-x_{2}=a_{1} ; \\ x_{2}-x_{3}=a_{2} ; \\ x_{3}-x_{4}=a_{3} ; \\ x_{4}-x_{5}=a_{4} \\ x_{5}-x_{1}=a_{5}\end{array}\right.$ \begin{CJK}{UTF8}{mj}有解\end{CJK}, \begin{CJK}{UTF8}{mj}则\end{CJK} $a_{1}+a_{2}+a_{3}+a_{4}+a_{5}=(\quad)$.

  \item \begin{CJK}{UTF8}{mj}若\end{CJK} 3 \begin{CJK}{UTF8}{mj}阶可逆阵\end{CJK} $A$ \begin{CJK}{UTF8}{mj}交换\end{CJK} 1,3 \begin{CJK}{UTF8}{mj}行得\end{CJK} $B$, \begin{CJK}{UTF8}{mj}则\end{CJK} $A B^{-1}=(\quad)$.

  \item \begin{CJK}{UTF8}{mj}设\end{CJK} $A=\left[\begin{array}{lll}1 & 2 & 3 \\ 0 & 4 & 5 \\ 0 & 0 & 6\end{array}\right], f(x)=(x-1)(x-2)^{2}(x-3)^{3}(x-4)^{4}(x-5)^{5}(x-6)^{6}+x+1$ \begin{CJK}{UTF8}{mj}则\end{CJK} $f(A)=(\quad)$. \begin{CJK}{UTF8}{mj}二\end{CJK}. ( 15 \begin{CJK}{UTF8}{mj}分\end{CJK}) \begin{CJK}{UTF8}{mj}设\end{CJK}

\end{enumerate}
$$
D_{n}=\left|\begin{array}{cccccc}
x & y & y & \cdots & y & y \\
z & x & y & \cdots & y & y \\
z & z & x & \cdots & y & y \\
\vdots & \vdots & \vdots & & \vdots & \vdots \\
z & z & z & \cdots & x & y \\
z & z & z & \cdots & z & x
\end{array}\right|
$$
(1) \begin{CJK}{UTF8}{mj}当\end{CJK} $y=z$ \begin{CJK}{UTF8}{mj}时\end{CJK}, \begin{CJK}{UTF8}{mj}计算\end{CJK} $D_{n}$.

(2) \begin{CJK}{UTF8}{mj}当\end{CJK} $y \neq z$ \begin{CJK}{UTF8}{mj}时\end{CJK}, \begin{CJK}{UTF8}{mj}计算\end{CJK} $D_{n}$.

\begin{CJK}{UTF8}{mj}三\end{CJK}. ( 15 \begin{CJK}{UTF8}{mj}分\end{CJK}) \begin{CJK}{UTF8}{mj}设\end{CJK} $\alpha_{1}, \alpha_{2}, \cdots, \alpha_{n}$ \begin{CJK}{UTF8}{mj}为一组\end{CJK} $n$ \begin{CJK}{UTF8}{mj}维向量\end{CJK}, \begin{CJK}{UTF8}{mj}求证\end{CJK} $\alpha_{1}, \alpha_{2}, \cdots, \alpha_{n}$ \begin{CJK}{UTF8}{mj}线性无关的充分必要条件为任意\end{CJK} $n$ \begin{CJK}{UTF8}{mj}维向量均可由\end{CJK} $\alpha_{1}, \alpha_{2}, \cdots, \alpha_{n}$ \begin{CJK}{UTF8}{mj}线性表示\end{CJK}.

\begin{CJK}{UTF8}{mj}四\end{CJK}. (15 \begin{CJK}{UTF8}{mj}分\end{CJK}) \begin{CJK}{UTF8}{mj}设\end{CJK} $A, B, C, D$ \begin{CJK}{UTF8}{mj}均为\end{CJK} $n$ \begin{CJK}{UTF8}{mj}阶方阵\end{CJK}, \begin{CJK}{UTF8}{mj}且\end{CJK} $|A| \neq 0, A C=C A$. \begin{CJK}{UTF8}{mj}求证\end{CJK}:
$$
\left|\begin{array}{ll}
A & B \\
C & D
\end{array}\right|=|A D-C B| .
$$
\begin{CJK}{UTF8}{mj}五\end{CJK}. ( 15 \begin{CJK}{UTF8}{mj}分\end{CJK}) \begin{CJK}{UTF8}{mj}设\end{CJK} $A=\left(\begin{array}{lll}2 & 1 & 0 \\ 0 & 2 & 1 \\ 0 & 0 & 2\end{array}\right)$, \begin{CJK}{UTF8}{mj}求证\end{CJK}:
$$
A^{n}=\left(\begin{array}{ccc}
2^{n} & 2^{n-1} n & 2^{n-3} n(n-1) \\
0 & 2^{n} & 2^{n-1} n \\
0 & 0 & 2^{n}
\end{array}\right)
$$
\begin{CJK}{UTF8}{mj}六\end{CJK}. (15 \begin{CJK}{UTF8}{mj}分\end{CJK}) \begin{CJK}{UTF8}{mj}设\end{CJK} $A$ \begin{CJK}{UTF8}{mj}为\end{CJK} $n$ \begin{CJK}{UTF8}{mj}阶正交阵\end{CJK}.

(1) \begin{CJK}{UTF8}{mj}求证\end{CJK}: \begin{CJK}{UTF8}{mj}对任意的\end{CJK} $n$ \begin{CJK}{UTF8}{mj}维列向量\end{CJK} $X$, \begin{CJK}{UTF8}{mj}有\end{CJK} $\|A X\|=\|X\|$.

(2) \begin{CJK}{UTF8}{mj}若\end{CJK} $\lambda$ \begin{CJK}{UTF8}{mj}为\end{CJK} $A$ \begin{CJK}{UTF8}{mj}的一个特征值\end{CJK}, \begin{CJK}{UTF8}{mj}求证\end{CJK}: $|\lambda|=1$.

\begin{CJK}{UTF8}{mj}七\end{CJK}. (15 \begin{CJK}{UTF8}{mj}分\end{CJK}) \begin{CJK}{UTF8}{mj}设\end{CJK} $V$ \begin{CJK}{UTF8}{mj}为\end{CJK} $\mathbb{R}$ \begin{CJK}{UTF8}{mj}上的\end{CJK} 3 \begin{CJK}{UTF8}{mj}维线性空间\end{CJK}, $\alpha_{1}, \alpha_{2}, \alpha_{3}$ \begin{CJK}{UTF8}{mj}是空间\end{CJK} $V$ \begin{CJK}{UTF8}{mj}的一组基\end{CJK}, $\beta_{1}=\alpha_{1}+\alpha_{2}, \beta_{2}=\alpha_{2}+\alpha_{3}, \beta_{3}=$ $\alpha_{3}+\alpha_{1}$.

(1)\begin{CJK}{UTF8}{mj}求证\end{CJK}: $\beta_{1}, \beta_{2}, \beta_{3}$ \begin{CJK}{UTF8}{mj}也是空间\end{CJK} $V$ \begin{CJK}{UTF8}{mj}的基\end{CJK}.

(2) \begin{CJK}{UTF8}{mj}求基\end{CJK} $\beta_{1}, \beta_{2}, \beta_{3}$ \begin{CJK}{UTF8}{mj}到\end{CJK} $\alpha_{1}, \alpha_{2}, \alpha_{3}$ \begin{CJK}{UTF8}{mj}的过渡矩阵\end{CJK}.

(3) \begin{CJK}{UTF8}{mj}求\end{CJK} $\gamma=3 \alpha_{1}+\alpha_{2}-4 \alpha_{3}$ \begin{CJK}{UTF8}{mj}在基\end{CJK} $\beta_{1}, \beta_{2}, \beta_{3}$ \begin{CJK}{UTF8}{mj}下的坐标\end{CJK}.

\begin{CJK}{UTF8}{mj}八\end{CJK}. (15 \begin{CJK}{UTF8}{mj}分\end{CJK}) \begin{CJK}{UTF8}{mj}用正交线性替换化二次型为标准型\end{CJK}, \begin{CJK}{UTF8}{mj}并判断\end{CJK}
$$
f\left(x_{1}, x_{2}, x_{3}\right)=1
$$
\begin{CJK}{UTF8}{mj}为何种几何曲面\end{CJK}? \begin{CJK}{UTF8}{mj}其中\end{CJK} $f\left(x_{1}, x_{2}, x_{3}\right)=5 x_{1}^{2}+5 x_{2}^{2}+3 x_{3}^{2}-2 x_{1} x_{2}+6 x_{1} x_{3}-6 x_{2} x_{3}$.

\begin{CJK}{UTF8}{mj}九\end{CJK}. ( 15 \begin{CJK}{UTF8}{mj}分\end{CJK}) \begin{CJK}{UTF8}{mj}设\end{CJK} $f(x)$ \begin{CJK}{UTF8}{mj}是一个整系数多项式\end{CJK}, $f(0)$ \begin{CJK}{UTF8}{mj}与\end{CJK} $f(1)$ \begin{CJK}{UTF8}{mj}均为奇数\end{CJK}, \begin{CJK}{UTF8}{mj}求证\end{CJK}: $f(x)$ \begin{CJK}{UTF8}{mj}不能有整数根\end{CJK}.

\begin{CJK}{UTF8}{mj}十\end{CJK}. ( 10 \begin{CJK}{UTF8}{mj}分\end{CJK}) \begin{CJK}{UTF8}{mj}设\end{CJK} $V$ \begin{CJK}{UTF8}{mj}为\end{CJK} $\mathbb{R}$ \begin{CJK}{UTF8}{mj}上的\end{CJK} $n$ \begin{CJK}{UTF8}{mj}维线性空间\end{CJK}, $\mathscr{A}$ \begin{CJK}{UTF8}{mj}为\end{CJK} $V$ \begin{CJK}{UTF8}{mj}上的一个线性变换\end{CJK}, $\operatorname{Im}(\mathscr{A})$ \begin{CJK}{UTF8}{mj}与\end{CJK} $\operatorname{ker} \mathscr{A}$ \begin{CJK}{UTF8}{mj}分别为线性变换\end{CJK} $\mathscr{A}$ \begin{CJK}{UTF8}{mj}的值域和核空间\end{CJK}, \begin{CJK}{UTF8}{mj}求证\end{CJK}:
$$
\operatorname{Im}(\mathscr{A}) \oplus \text { ker } \mathscr{A}=V
$$
\begin{CJK}{UTF8}{mj}的充分必要条件为\end{CJK}
$$
\operatorname{ker} \mathscr{A}=\operatorname{ker} \mathscr{A}^{2} .
$$

\section{9. 哈尔滨工程大学 2017 年研究生入学考试试题高等代数}
\begin{CJK}{UTF8}{mj}李扬\end{CJK}

\begin{CJK}{UTF8}{mj}微信公众号\end{CJK}: sxkyliyang

\begin{CJK}{UTF8}{mj}一\end{CJK}. \begin{CJK}{UTF8}{mj}填空题\end{CJK}

\begin{enumerate}
  \item \begin{CJK}{UTF8}{mj}求\end{CJK} $\left|\begin{array}{llll}x & 0 & x & 0 \\ 1 & 1 & 1 & 0 \\ 0 & x & 1 & 3 \\ 1 & 2 & 3 & 4\end{array}\right|=($ ).

  \item \begin{CJK}{UTF8}{mj}设\end{CJK} $f(x), g(x)$ \begin{CJK}{UTF8}{mj}的是大公因式为\end{CJK} $x+1$, \begin{CJK}{UTF8}{mj}最小公倍式为\end{CJK} $(x+1)^{2}(x+2)(x+3)$. \begin{CJK}{UTF8}{mj}则\end{CJK} $f(x) g(x)=(\quad)$.

  \item \begin{CJK}{UTF8}{mj}关于互素问题\end{CJK}, \begin{CJK}{UTF8}{mj}忘记了\end{CJK}.

  \item \begin{CJK}{UTF8}{mj}求逆矩阵的问题\end{CJK}, \begin{CJK}{UTF8}{mj}忘记了\end{CJK}.

  \item \begin{CJK}{UTF8}{mj}复数域上特征值全为\end{CJK} 1 \begin{CJK}{UTF8}{mj}的\end{CJK} 4 \begin{CJK}{UTF8}{mj}阶方阵\end{CJK}, \begin{CJK}{UTF8}{mj}按相似分为\end{CJK}( ) \begin{CJK}{UTF8}{mj}类\end{CJK}.

\end{enumerate}
\begin{CJK}{UTF8}{mj}二\end{CJK}.\begin{CJK}{UTF8}{mj}求证\end{CJK}
$$
\left|\begin{array}{ccccc}
\cos a & 1 & & & \\
1 & 2 \cos a & 1 & & \\
& \ddots & \ddots & \ddots & \\
& & 1 & 2 \cos a & 1 \\
& & & 1 & 2 \cos a
\end{array}\right|=\cos n a
$$
\begin{CJK}{UTF8}{mj}三\end{CJK}.
$$
A=\left(\begin{array}{cccc}
a_{1} & & & \\
& a_{2} & & \\
& & \ddots & \\
& & & a_{n}
\end{array}\right) .
$$
$a_{1}, a_{2}, \cdots, a_{n}$ \begin{CJK}{UTF8}{mj}互不相同\end{CJK}, $A B=B A$, \begin{CJK}{UTF8}{mj}证明\end{CJK}: $B$ \begin{CJK}{UTF8}{mj}为对角矩阵\end{CJK}.

\begin{CJK}{UTF8}{mj}四\end{CJK}. \begin{CJK}{UTF8}{mj}方阵\end{CJK} $A$ \begin{CJK}{UTF8}{mj}交换\end{CJK} 2,3 \begin{CJK}{UTF8}{mj}行得\end{CJK} $B, B$ \begin{CJK}{UTF8}{mj}交换\end{CJK} 2,3 \begin{CJK}{UTF8}{mj}列得\end{CJK} $C$. \begin{CJK}{UTF8}{mj}求证\end{CJK}: $A$ \begin{CJK}{UTF8}{mj}与\end{CJK} $C$ \begin{CJK}{UTF8}{mj}相似且合同\end{CJK}.

\begin{CJK}{UTF8}{mj}五\end{CJK}. \begin{CJK}{UTF8}{mj}求证\end{CJK}:
$$
f\left(x_{1}, x_{2}, x_{3}\right)=2 x_{1}^{2}+2 x_{2}^{2}+2 x_{3}^{2}+2 x_{1} x_{2}+2 x_{1} x_{3}+2 x_{2} x_{3}
$$
\begin{CJK}{UTF8}{mj}是正定的\end{CJK}.

\begin{CJK}{UTF8}{mj}六\end{CJK}. \begin{CJK}{UTF8}{mj}证明\end{CJK}: $B$ \begin{CJK}{UTF8}{mj}的列向量是方程组\end{CJK} $A X=0$ \begin{CJK}{UTF8}{mj}的解的充要条件是\end{CJK} $A B=0$, \begin{CJK}{UTF8}{mj}这里\end{CJK} $A, B$ \begin{CJK}{UTF8}{mj}分别为\end{CJK} $m \times n$ \begin{CJK}{UTF8}{mj}和\end{CJK} $n \times s$ \begin{CJK}{UTF8}{mj}矩阵\end{CJK}.

\begin{CJK}{UTF8}{mj}七\end{CJK}. $V=\{B$ \begin{CJK}{UTF8}{mj}为二阶矩阵\end{CJK} $\mid A B=B A\}$, \begin{CJK}{UTF8}{mj}这里\end{CJK}
$$
A=\left(\begin{array}{ll}
1 & 1 \\
0 & 1
\end{array}\right)
$$
\begin{CJK}{UTF8}{mj}求\end{CJK} $V$ \begin{CJK}{UTF8}{mj}的基并将\end{CJK} $A^{-1}$ \begin{CJK}{UTF8}{mj}用\end{CJK} $V$ \begin{CJK}{UTF8}{mj}的一组基线性表示\end{CJK}.

\begin{CJK}{UTF8}{mj}八\end{CJK}. \begin{CJK}{UTF8}{mj}纯计算题\end{CJK}, \begin{CJK}{UTF8}{mj}求向量组的秩\end{CJK}, \begin{CJK}{UTF8}{mj}极大线性无关组\end{CJK}, \begin{CJK}{UTF8}{mj}并将其余向量用极大线性无关组线性表示\end{CJK}.

\begin{CJK}{UTF8}{mj}九\end{CJK}. \begin{CJK}{UTF8}{mj}忘记了\end{CJK}.

\begin{CJK}{UTF8}{mj}十\end{CJK}. $n$ \begin{CJK}{UTF8}{mj}维线性空间中的线性变换\end{CJK} $\mathscr{A}$ \begin{CJK}{UTF8}{mj}的属性特征值\end{CJK} $\lambda_{0}$ \begin{CJK}{UTF8}{mj}有\end{CJK} $k$ \begin{CJK}{UTF8}{mj}个线性无关的特征向量\end{CJK}, \begin{CJK}{UTF8}{mj}求证\end{CJK}: $\mathscr{A}$ \begin{CJK}{UTF8}{mj}的特征值\end{CJK} $\lambda_{0}$ \begin{CJK}{UTF8}{mj}的\end{CJK} \begin{CJK}{UTF8}{mj}重数至少为\end{CJK} $k$.

\section{0. 哈尔滨工程大学 2009 年研究生入学考试试题数学分析}
\begin{CJK}{UTF8}{mj}李扬\end{CJK}

\begin{CJK}{UTF8}{mj}微信公众号\end{CJK}: sxkyliyang

\begin{enumerate}
  \item \begin{CJK}{UTF8}{mj}求极限\end{CJK}
\end{enumerate}
$$
\lim _{x \rightarrow 0} \frac{\mathrm{e}^{x}-\mathrm{e}^{\sin x}}{x-\sin x}
$$

\begin{enumerate}
  \setcounter{enumi}{2}
  \item \begin{CJK}{UTF8}{mj}设\end{CJK} $f(x)$ \begin{CJK}{UTF8}{mj}在\end{CJK} $[a, b]$ \begin{CJK}{UTF8}{mj}上可微\end{CJK}, \begin{CJK}{UTF8}{mj}证明\end{CJK}: \begin{CJK}{UTF8}{mj}存在\end{CJK} $\xi \in(a, b)$, \begin{CJK}{UTF8}{mj}使成立\end{CJK}
\end{enumerate}
$$
2 \xi(f(b)-f(a))=\left(b^{2}-a^{2}\right) f^{\prime}(\xi)
$$
3 . \begin{CJK}{UTF8}{mj}设\end{CJK}
$$
f(x)=\mathrm{e}^{x^{2}} \sin x
$$
\begin{CJK}{UTF8}{mj}求\end{CJK} $f^{(2012)}(0)$.

\begin{enumerate}
  \setcounter{enumi}{4}
  \item \begin{CJK}{UTF8}{mj}证明\end{CJK}: \begin{CJK}{UTF8}{mj}若函数\end{CJK} $f(x)$ \begin{CJK}{UTF8}{mj}在\end{CJK} $[a, b]$ \begin{CJK}{UTF8}{mj}上连续\end{CJK}, \begin{CJK}{UTF8}{mj}在\end{CJK} $(a, b)$ \begin{CJK}{UTF8}{mj}内存在二阶导数\end{CJK}, \begin{CJK}{UTF8}{mj}且\end{CJK} $f(a)=f(b)=0, f(c)>0$, \begin{CJK}{UTF8}{mj}其中\end{CJK} $a<c<b$, \begin{CJK}{UTF8}{mj}则在\end{CJK} $(a, b)$ \begin{CJK}{UTF8}{mj}至少存在一点\end{CJK} $\xi$ \begin{CJK}{UTF8}{mj}使\end{CJK}
\end{enumerate}
$$
f^{\prime \prime}(\xi)<0
$$

\begin{enumerate}
  \setcounter{enumi}{5}
  \item \begin{CJK}{UTF8}{mj}计算\end{CJK}
\end{enumerate}
$$
\int_{a}^{a+1}(a+x)^{2}(a-x)^{10} \mathrm{~d} x
$$

\begin{enumerate}
  \setcounter{enumi}{6}
  \item \begin{CJK}{UTF8}{mj}证明\end{CJK}: \begin{CJK}{UTF8}{mj}函数级数\end{CJK}
\end{enumerate}
$$
\sum_{n=1}^{x}\left(1-\cos \frac{x}{n}\right) .
$$

\begin{enumerate}
  \setcounter{enumi}{7}
  \item \begin{CJK}{UTF8}{mj}计算\end{CJK}
\end{enumerate}
$$
\iiint_{V} z^{2} \mathrm{~d} x \mathrm{~d} y \mathrm{~d} z .
$$
\begin{CJK}{UTF8}{mj}其中\end{CJK} $V$ \begin{CJK}{UTF8}{mj}是由\end{CJK} $x^{2}+y^{2}+z^{2} \leq 4, x^{2}+y^{2}+z^{2} \leq 4 z$ \begin{CJK}{UTF8}{mj}所确定的立体\end{CJK}.

\begin{enumerate}
  \setcounter{enumi}{8}
  \item \begin{CJK}{UTF8}{mj}设\end{CJK} $f(x)$ \begin{CJK}{UTF8}{mj}在闭区间\end{CJK} $[a, b]$ \begin{CJK}{UTF8}{mj}上二阶可导且\end{CJK} $f^{\prime \prime}(x)<0$, \begin{CJK}{UTF8}{mj}证明不等式\end{CJK}
\end{enumerate}
$$
\int_{a}^{b} f(x) \mathrm{d} x \leq f\left(\frac{a+b}{2}\right)(b-a) .
$$

\begin{enumerate}
  \setcounter{enumi}{9}
  \item \begin{CJK}{UTF8}{mj}设\end{CJK} $L$ \begin{CJK}{UTF8}{mj}为单位圆周\end{CJK} $x^{2}+y^{2}=1$, \begin{CJK}{UTF8}{mj}方向为逆时针\end{CJK}, \begin{CJK}{UTF8}{mj}求积分\end{CJK}
\end{enumerate}
$$
I=\oint_{L} \frac{(x-y) \mathrm{d} x+(x+4 y) \mathrm{d} y}{x^{2}+y^{2}} .
$$

\begin{enumerate}
  \setcounter{enumi}{10}
  \item \begin{CJK}{UTF8}{mj}设\end{CJK} $f(x)$ \begin{CJK}{UTF8}{mj}在\end{CJK} $\mathbb{R}$ \begin{CJK}{UTF8}{mj}上有连续一阶导数且\end{CJK}
\end{enumerate}
$$
\int_{-\infty}^{+\infty}\left(f^{2}(x)+\left(f^{\prime}(x)\right)^{2}\right) \mathrm{d} x=1
$$
\begin{CJK}{UTF8}{mj}证明\end{CJK}:

(1) $\lim _{x \rightarrow \infty} f(x)=0$.

(2) $\forall x \in \mathbb{R}, f(x)<\frac{\sqrt{2}}{2}$.

\section{1. 哈尔滨工程大学 2010 年研究生入学考试试题数学分析}
\begin{CJK}{UTF8}{mj}李扬\end{CJK}

\begin{CJK}{UTF8}{mj}微信公众号\end{CJK}: sxkyliyang

\begin{enumerate}
  \item \begin{CJK}{UTF8}{mj}求极限\end{CJK}
\end{enumerate}
$$
\lim _{n \rightarrow \infty} \frac{\mathrm{e}^{x^{3}}-1-x^{3}}{\sin ^{6} 2 x} .
$$

\begin{enumerate}
  \setcounter{enumi}{2}
  \item \begin{CJK}{UTF8}{mj}确定实数\end{CJK} $a$ \begin{CJK}{UTF8}{mj}使得函数\end{CJK}
\end{enumerate}
$$
f(x)= \begin{cases}\frac{(1+x)^{a}-1}{x}, & x>0 ; \\ 1, & x<0 .\end{cases}
$$
\begin{CJK}{UTF8}{mj}连续\end{CJK}.

\begin{enumerate}
  \setcounter{enumi}{3}
  \item \begin{CJK}{UTF8}{mj}设\end{CJK} $f(x)$ \begin{CJK}{UTF8}{mj}在\end{CJK} $[a, b]$ \begin{CJK}{UTF8}{mj}上连续\end{CJK}, $\int_{a}^{b} f(x) \mathrm{d} x=0$. \begin{CJK}{UTF8}{mj}证明存在\end{CJK} $\xi \in(a, b)$, \begin{CJK}{UTF8}{mj}使得\end{CJK}
\end{enumerate}
$$
\int_{a}^{\xi} f(x) \mathrm{d} x=f(\xi)
$$

\begin{enumerate}
  \setcounter{enumi}{4}
  \item \begin{CJK}{UTF8}{mj}设\end{CJK} $\sum_{n=1}^{\infty} a_{n}$ \begin{CJK}{UTF8}{mj}是正项发散级数\end{CJK}, \begin{CJK}{UTF8}{mj}试判断级数\end{CJK}
\end{enumerate}
$$
\sum_{n=1}^{\infty} \frac{a_{n}}{1+a_{n}}
$$
\begin{CJK}{UTF8}{mj}的敛散性\end{CJK}.

\begin{enumerate}
  \setcounter{enumi}{5}
  \item \begin{CJK}{UTF8}{mj}设\end{CJK} $f(x, y)$ \begin{CJK}{UTF8}{mj}是\end{CJK} $\mathbb{R}$ \begin{CJK}{UTF8}{mj}上的连续函数\end{CJK}, \begin{CJK}{UTF8}{mj}试将累次积分\end{CJK}
\end{enumerate}
$$
\int_{-1}^{1} \mathrm{~d} x \int_{x^{2}+x}^{x+1} f(x, y) \mathrm{d} y
$$
\begin{CJK}{UTF8}{mj}交换积分次序\end{CJK}.

\begin{enumerate}
  \setcounter{enumi}{6}
  \item \begin{CJK}{UTF8}{mj}计算\end{CJK}
\end{enumerate}
$$
I=\iint_{S} x^{2} \mathrm{~d} y \mathrm{~d} z+y^{2} \mathrm{~d} z \mathrm{~d} x+z^{2} \mathrm{~d} x \mathrm{~d} y .
$$
\begin{CJK}{UTF8}{mj}其中\end{CJK} $S$ \begin{CJK}{UTF8}{mj}为雉面\end{CJK} $z=\sqrt{x^{2}+y^{2}}$ \begin{CJK}{UTF8}{mj}在\end{CJK} $0 \leq z \leq h$ \begin{CJK}{UTF8}{mj}部分的下侧\end{CJK}.

\begin{enumerate}
  \setcounter{enumi}{7}
  \item \begin{CJK}{UTF8}{mj}设\end{CJK} $f(x)$ \begin{CJK}{UTF8}{mj}在\end{CJK} $[0,1]$ \begin{CJK}{UTF8}{mj}上可微\end{CJK}, \begin{CJK}{UTF8}{mj}且\end{CJK} $f(x)$ \begin{CJK}{UTF8}{mj}的每一个零点都是简单零点\end{CJK}, \begin{CJK}{UTF8}{mj}即若\end{CJK} $f\left(x_{0}\right)=0$ \begin{CJK}{UTF8}{mj}则\end{CJK} $f^{\prime}\left(x_{0}\right) \neq 0$. \begin{CJK}{UTF8}{mj}证明\end{CJK}: $f(x)$ \begin{CJK}{UTF8}{mj}在\end{CJK} $[0,1]$ \begin{CJK}{UTF8}{mj}上只有有限个零点\end{CJK}.

  \item \begin{CJK}{UTF8}{mj}设周期为\end{CJK} $2 \pi$ \begin{CJK}{UTF8}{mj}的函数\end{CJK} $f(x)$ \begin{CJK}{UTF8}{mj}可积且绝对可积满足条件\end{CJK} $f(x+\pi)=-f(x)$. \begin{CJK}{UTF8}{mj}试证明函数在\end{CJK} $(-\pi, \pi)$ \begin{CJK}{UTF8}{mj}内的傅里\end{CJK} \begin{CJK}{UTF8}{mj}叶级数的偶数项系数全为\end{CJK} 0 .

  \item \begin{CJK}{UTF8}{mj}求积分\end{CJK}

\end{enumerate}
$$
I=\iint_{D}\left|x-y^{2}\right| \mathrm{d} x \mathrm{~d} y
$$
\begin{CJK}{UTF8}{mj}其中\end{CJK} $D \in\{(x, y): 0 \leq x \leq 1,-1 \leq y \leq 1\}$.

\begin{enumerate}
  \setcounter{enumi}{10}
  \item \begin{CJK}{UTF8}{mj}证明\end{CJK}
\end{enumerate}
$$
\sum_{n=1}^{\infty} n \mathrm{e}^{-n x}
$$
\begin{CJK}{UTF8}{mj}的和函数在\end{CJK} $(0, \infty)$ \begin{CJK}{UTF8}{mj}上连续且具有任意阶连续导数\end{CJK}.

\section{2. 哈尔滨工程大学 2011 年研究生入学考试试题数学分析}
\begin{CJK}{UTF8}{mj}李扬\end{CJK}

\begin{CJK}{UTF8}{mj}微信公众号\end{CJK}: sxkyliyang

\begin{enumerate}
  \item \begin{CJK}{UTF8}{mj}计算极限\end{CJK}
\end{enumerate}
$$
\lim _{x \rightarrow 0}\left(\frac{3^{x}-x \ln 3}{2^{x}-x \ln 2}\right)^{\frac{1}{x^{2}}} .
$$

\begin{enumerate}
  \setcounter{enumi}{2}
  \item \begin{CJK}{UTF8}{mj}计算积分\end{CJK}
\end{enumerate}
$$
I=\int_{0}^{1} \frac{\ln (1+x)}{1+x^{2}} \mathrm{~d} x
$$

\begin{enumerate}
  \setcounter{enumi}{3}
  \item \begin{CJK}{UTF8}{mj}对于\end{CJK} $x>0$, \begin{CJK}{UTF8}{mj}证明不等式\end{CJK}
\end{enumerate}
$$
\int_{0}^{x}\left(t-t^{2}\right) \sin ^{2 n} t \mathrm{~d} t \leq \frac{1}{(2 n+2)(2 n+3)}
$$

\begin{enumerate}
  \setcounter{enumi}{4}
  \item \begin{CJK}{UTF8}{mj}求极限\end{CJK}
\end{enumerate}
$$
\lim _{n \rightarrow \infty} \frac{1}{n} \sqrt[n]{(n+1)(n+2)(n+3) \cdots(2 n)}
$$

\begin{enumerate}
  \setcounter{enumi}{5}
  \item \begin{CJK}{UTF8}{mj}计算积分\end{CJK}
\end{enumerate}
$$
\iint_{S}\left(\sin ^{4} x \mathrm{~d} y \mathrm{~d} z+\mathrm{e}^{|z|} \mathrm{d} z \mathrm{~d} x+z^{2} \mathrm{~d} x \mathrm{~d} y\right) .
$$
\begin{CJK}{UTF8}{mj}其中\end{CJK} $S$ \begin{CJK}{UTF8}{mj}为半球面\end{CJK} $x^{2}+y^{2}+z^{2}=1, z \geq 0$, \begin{CJK}{UTF8}{mj}定向为上侧\end{CJK}.

\begin{enumerate}
  \setcounter{enumi}{6}
  \item $f(x)$ \begin{CJK}{UTF8}{mj}在\end{CJK} $(0,1)$ \begin{CJK}{UTF8}{mj}上连续有界\end{CJK}, $\lim _{x \rightarrow 0^{+}} f(x)$ \begin{CJK}{UTF8}{mj}不存在\end{CJK}. \begin{CJK}{UTF8}{mj}证明存在数列\end{CJK} $x_{n} \rightarrow 0(n \rightarrow \infty)$, \begin{CJK}{UTF8}{mj}使得\end{CJK}
\end{enumerate}
$$
f^{\prime}\left(x_{n}\right)=0
$$

\begin{enumerate}
  \setcounter{enumi}{7}
  \item \begin{CJK}{UTF8}{mj}证明广义积分\end{CJK}
\end{enumerate}
$$
\int_{0}^{+\infty} \frac{\ln x}{1-x^{2}} \mathrm{~d} x
$$
\begin{CJK}{UTF8}{mj}收敛\end{CJK}.

\begin{enumerate}
  \setcounter{enumi}{8}
  \item \begin{CJK}{UTF8}{mj}计算\end{CJK}
\end{enumerate}
$$
\oint_{l} \frac{x}{x^{2}+y^{2}} \mathrm{~d} y-\frac{y}{x^{2}+y^{2}} \mathrm{~d} x .
$$
\begin{CJK}{UTF8}{mj}其中\end{CJK} $l$ \begin{CJK}{UTF8}{mj}是由\end{CJK} $y=x^{2}-1$ \begin{CJK}{UTF8}{mj}与\end{CJK} $y=x+1$ \begin{CJK}{UTF8}{mj}所围成区域的边界\end{CJK}, \begin{CJK}{UTF8}{mj}沿逆时针方向\end{CJK}.

\begin{enumerate}
  \setcounter{enumi}{9}
  \item \begin{CJK}{UTF8}{mj}设\end{CJK} $f(x)$ \begin{CJK}{UTF8}{mj}在\end{CJK} $[a, b]$ \begin{CJK}{UTF8}{mj}上连续\end{CJK}, \begin{CJK}{UTF8}{mj}在\end{CJK} $(a, b)$ \begin{CJK}{UTF8}{mj}内可导\end{CJK} $(b>a>0), f(a) \neq f(b), \cdots$

  \item \begin{CJK}{UTF8}{mj}设函数\end{CJK} $u(x)$ \begin{CJK}{UTF8}{mj}在\end{CJK} $(0,+\infty)$ \begin{CJK}{UTF8}{mj}上二阶连续可导\end{CJK}, \begin{CJK}{UTF8}{mj}且满足\end{CJK}

\end{enumerate}
$$
u^{\prime \prime}(x)=\frac{u^{2}(x)}{x^{2}}+1
$$
\begin{CJK}{UTF8}{mj}证明\end{CJK}:

(1) \begin{CJK}{UTF8}{mj}要么\end{CJK} $\lim _{x \rightarrow 0^{+}} u(x)$ \begin{CJK}{UTF8}{mj}存在有限\end{CJK}, \begin{CJK}{UTF8}{mj}要么\end{CJK} $\lim _{x \rightarrow 0^{+}} u(x)=+\infty$.

(2) \begin{CJK}{UTF8}{mj}若\end{CJK} $\lim _{x \rightarrow 0^{+}} x u^{\prime}(x)=0$, \begin{CJK}{UTF8}{mj}则\end{CJK} $\lim _{x \rightarrow 0^{+}} u(x)=0$.

\section{3. 哈尔滨工程大学 2012 年研究生入学考试试题数学分析}
\begin{CJK}{UTF8}{mj}李扬\end{CJK}

\begin{CJK}{UTF8}{mj}微信公众号\end{CJK}: sxkyliyang

\begin{enumerate}
  \item \begin{CJK}{UTF8}{mj}求极限\end{CJK}
\end{enumerate}
$$
\lim _{x \rightarrow 0} \frac{\int_{0}^{x}(1-\cos t) \mathrm{d} t}{\frac{1}{3} x^{3}} .
$$

\begin{enumerate}
  \setcounter{enumi}{2}
  \item \begin{CJK}{UTF8}{mj}设\end{CJK} $f(x)$ \begin{CJK}{UTF8}{mj}在\end{CJK} $[0,1]$ \begin{CJK}{UTF8}{mj}上连续\end{CJK}. \begin{CJK}{UTF8}{mj}证明\end{CJK}:
\end{enumerate}
$$
\lim _{t \rightarrow 0^{+}} \int_{0}^{1} \frac{t f(x)}{t^{2}+x^{2}} \mathrm{~d} x=\frac{\pi}{2} f(0) .
$$

\begin{enumerate}
  \setcounter{enumi}{3}
  \item \begin{CJK}{UTF8}{mj}计算\end{CJK}
\end{enumerate}
$$
\iint_{S} 4 z x \mathrm{~d} y \mathrm{~d} z-2 y z \mathrm{~d} z \mathrm{~d} x+\left(z-z^{2}\right) \mathrm{d} x \mathrm{~d} y .
$$
\begin{CJK}{UTF8}{mj}其中\end{CJK} $S$ \begin{CJK}{UTF8}{mj}是\end{CJK} yoz \begin{CJK}{UTF8}{mj}平面上的曲线\end{CJK} $z=\mathrm{e}^{y}(0 \leq y \leq 2)$ \begin{CJK}{UTF8}{mj}绕\end{CJK} $\mathrm{oz}$ \begin{CJK}{UTF8}{mj}轴旋转一周所成的曲面的下侧\end{CJK}.

\begin{enumerate}
  \setcounter{enumi}{4}
  \item \begin{CJK}{UTF8}{mj}若\end{CJK} $n \geq 1$ \begin{CJK}{UTF8}{mj}及\end{CJK} $x \geq 0, y \geq 0$. \begin{CJK}{UTF8}{mj}证明不等式\end{CJK}:
\end{enumerate}
$$
\frac{x^{n}+y^{n}}{2} \geq\left(\frac{x+y}{2}\right)^{n}
$$

\begin{enumerate}
  \setcounter{enumi}{5}
  \item \begin{CJK}{UTF8}{mj}计算积分\end{CJK}
\end{enumerate}
$$
\iint_{S} x z \mathrm{~d} y \mathrm{~d} z+\left(x^{2}-z\right) y \mathrm{~d} z \mathrm{~d} x-x^{2} z \mathrm{~d} x \mathrm{~d} y
$$
\begin{CJK}{UTF8}{mj}其中\end{CJK} $S$ \begin{CJK}{UTF8}{mj}是旋转抛物面\end{CJK} $x^{2}+y^{2}=a^{2} z(a>0)$ \begin{CJK}{UTF8}{mj}取\end{CJK} $0 \leq z \leq 1$ \begin{CJK}{UTF8}{mj}部分\end{CJK}, \begin{CJK}{UTF8}{mj}下侧为正\end{CJK}.

\begin{enumerate}
  \setcounter{enumi}{6}
  \item \begin{CJK}{UTF8}{mj}设\end{CJK} $y=f(x)$ \begin{CJK}{UTF8}{mj}在\end{CJK} $[0,+\infty)$ \begin{CJK}{UTF8}{mj}一致连续\end{CJK}, \begin{CJK}{UTF8}{mj}且对任意\end{CJK} $x \in[0,1], \lim _{n \rightarrow \infty} f(x+n)=0(n$ \begin{CJK}{UTF8}{mj}为正整数\end{CJK} $)$, \begin{CJK}{UTF8}{mj}证明\end{CJK}:
\end{enumerate}
$$
\lim _{x \rightarrow \infty} f(x)=0 .
$$

\begin{enumerate}
  \setcounter{enumi}{7}
  \item \begin{CJK}{UTF8}{mj}求幂级数\end{CJK}
\end{enumerate}
$$
\sum_{n=0}^{\infty} \frac{n}{n+1} x^{n}
$$
\begin{CJK}{UTF8}{mj}的收敛域及和函数\end{CJK}.

\begin{enumerate}
  \setcounter{enumi}{8}
  \item \begin{CJK}{UTF8}{mj}设\end{CJK} $f(x)$ \begin{CJK}{UTF8}{mj}在\end{CJK} $[0,1]$ \begin{CJK}{UTF8}{mj}上可微\end{CJK}, \begin{CJK}{UTF8}{mj}且\end{CJK} $f(x)$ \begin{CJK}{UTF8}{mj}的每一个零点都是简单零点\end{CJK}, \begin{CJK}{UTF8}{mj}即若\end{CJK} $f\left(x_{0}\right)=0$ \begin{CJK}{UTF8}{mj}则\end{CJK} $f^{\prime}\left(x_{0}\right) \neq 0$. \begin{CJK}{UTF8}{mj}证明\end{CJK}: $f(x)$ \begin{CJK}{UTF8}{mj}在\end{CJK} $[0,1]$ \begin{CJK}{UTF8}{mj}上只有有限个零点\end{CJK}.

  \item \begin{CJK}{UTF8}{mj}设\end{CJK} $f(x)$ \begin{CJK}{UTF8}{mj}在\end{CJK} $[0,2]$ \begin{CJK}{UTF8}{mj}上二次可微\end{CJK}, \begin{CJK}{UTF8}{mj}且\end{CJK} $|f(x)| \leq 1,\left|f^{\prime \prime}(x)\right| \leq 1$. \begin{CJK}{UTF8}{mj}证明\end{CJK}:

\end{enumerate}
$$
\left|f^{\prime}(x)\right| \leq 2
$$

\begin{enumerate}
  \setcounter{enumi}{10}
  \item \begin{CJK}{UTF8}{mj}设\end{CJK} $f(x)$ \begin{CJK}{UTF8}{mj}在\end{CJK} $[a, b]$ \begin{CJK}{UTF8}{mj}上非负且三阶可导\end{CJK}, \begin{CJK}{UTF8}{mj}方程\end{CJK} $f(x)=0$ \begin{CJK}{UTF8}{mj}在\end{CJK} $(a, b)$ \begin{CJK}{UTF8}{mj}内有两个不同的实根\end{CJK}. \begin{CJK}{UTF8}{mj}证明\end{CJK}: \begin{CJK}{UTF8}{mj}存在\end{CJK} $\xi \in(a, b)$, \begin{CJK}{UTF8}{mj}使得\end{CJK}
\end{enumerate}
$$
f^{(3)}(\xi)=0
$$

\section{4. 哈尔滨工程大学 2013 年研究生入学考试试题数学分析}
\begin{CJK}{UTF8}{mj}李扬\end{CJK}

\begin{CJK}{UTF8}{mj}微信公众号\end{CJK}: sxkyliyang

\begin{enumerate}
  \item \begin{CJK}{UTF8}{mj}求不定积分\end{CJK}
\end{enumerate}
$$
\int \frac{x^{3}}{\sqrt{1-x^{2}}} \mathrm{~d} x
$$

\begin{enumerate}
  \setcounter{enumi}{2}
  \item \begin{CJK}{UTF8}{mj}求极限\end{CJK}
\end{enumerate}
$$
\lim _{x \rightarrow 0} \frac{\mathrm{e}^{x} \sin x-x(1+x)}{x^{3}} .
$$

\begin{enumerate}
  \setcounter{enumi}{3}
  \item \begin{CJK}{UTF8}{mj}利用有上界的非空数集必有上确界证明\end{CJK}: \begin{CJK}{UTF8}{mj}单调增加的有界数列必有极限\end{CJK}.

  \item \begin{CJK}{UTF8}{mj}求\end{CJK}

\end{enumerate}
$$
I=\int \frac{}{\ln x} \mathrm{~d} x(b>a>0) .
$$

\begin{enumerate}
  \setcounter{enumi}{5}
  \item \begin{CJK}{UTF8}{mj}函数\end{CJK} $z=z(x, y)$ \begin{CJK}{UTF8}{mj}由方程\end{CJK} $\mathrm{e}^{z}-z+x y=3$ \begin{CJK}{UTF8}{mj}所确定\end{CJK}, \begin{CJK}{UTF8}{mj}求\end{CJK}
\end{enumerate}
$$
\frac{\partial^{2} z}{\partial x^{2}}
$$

\begin{enumerate}
  \setcounter{enumi}{6}
  \item \begin{CJK}{UTF8}{mj}求\end{CJK}
\end{enumerate}
$$
f(x)= \begin{cases}0, & -\pi \leq x<0 \\ x, & 0 \leq x \leq \pi\end{cases}
$$
\begin{CJK}{UTF8}{mj}的傅里叶级数\end{CJK}.

\begin{enumerate}
  \setcounter{enumi}{7}
  \item \begin{CJK}{UTF8}{mj}证明\end{CJK}: $\sin \frac{1}{x}$ \begin{CJK}{UTF8}{mj}在\end{CJK} $(0,1)$ \begin{CJK}{UTF8}{mj}上不一致连续\end{CJK}, \begin{CJK}{UTF8}{mj}但在\end{CJK} $(a, 1)(a>0)$ \begin{CJK}{UTF8}{mj}上一致连续\end{CJK}.

  \item \begin{CJK}{UTF8}{mj}已知函数\end{CJK} $f(x)$ \begin{CJK}{UTF8}{mj}在\end{CJK} $[0,3]$ \begin{CJK}{UTF8}{mj}上连续\end{CJK}, \begin{CJK}{UTF8}{mj}在\end{CJK} $(0,3)$ \begin{CJK}{UTF8}{mj}内可导\end{CJK}, \begin{CJK}{UTF8}{mj}且\end{CJK} $\int_{0}^{1} f(x) \mathrm{d} x=0, f(1)+f(3)=0$. \begin{CJK}{UTF8}{mj}证明\end{CJK}: \begin{CJK}{UTF8}{mj}存在一点\end{CJK} $\xi \in(0,3)$, \begin{CJK}{UTF8}{mj}使\end{CJK}

\end{enumerate}
$$
f^{\prime}(\xi)+2 f(\xi)=0
$$
\begin{CJK}{UTF8}{mj}成立\end{CJK}.

\begin{enumerate}
  \setcounter{enumi}{9}
  \item \begin{CJK}{UTF8}{mj}设有半球体\end{CJK} $\Omega: x^{2}+y^{2}+z^{2} \leq 4, z \geq 0, \Omega$ \begin{CJK}{UTF8}{mj}内任意一点的密度\end{CJK} $\rho=0$. \begin{CJK}{UTF8}{mj}求\end{CJK}:\\
(1) \begin{CJK}{UTF8}{mj}此半球体的质量\end{CJK}.\\
(2) \begin{CJK}{UTF8}{mj}此半球体的重心\end{CJK}.

  \item \begin{CJK}{UTF8}{mj}求幂级数\end{CJK}

\end{enumerate}
$$
\sum_{n=1}^{\infty} \frac{n}{n+1} x^{n}
$$
\begin{CJK}{UTF8}{mj}的收敛区间及和函数\end{CJK}.

\begin{enumerate}
  \setcounter{enumi}{11}
  \item \begin{CJK}{UTF8}{mj}设\end{CJK} $u=u(x, y)$ \begin{CJK}{UTF8}{mj}具有二阶连续偏导数\end{CJK}, \begin{CJK}{UTF8}{mj}试证明\end{CJK}:
\end{enumerate}
$$
\oint_{C} \frac{\partial u}{\partial n} \mathrm{~d} s=\iint_{D}\left(\frac{\partial^{2} u}{\partial x^{2}}+\frac{\partial^{2} u}{\partial y^{2}}\right) \mathrm{d} x \mathrm{~d} y
$$
\begin{CJK}{UTF8}{mj}其中\end{CJK} $C$ \begin{CJK}{UTF8}{mj}是\end{CJK} $D$ \begin{CJK}{UTF8}{mj}的边界\end{CJK}, $\vec{n}$ \begin{CJK}{UTF8}{mj}是\end{CJK} $C$ \begin{CJK}{UTF8}{mj}的外法向单位向量\end{CJK}. 12. \begin{CJK}{UTF8}{mj}设\end{CJK}
$$
l(y)=\int_{1}^{+\infty} y \mathrm{e}^{-y x} \mathrm{~d} x
$$
\begin{CJK}{UTF8}{mj}证明\end{CJK}:

(1) \begin{CJK}{UTF8}{mj}对任意的\end{CJK} $b>a>0, l(y)$ \begin{CJK}{UTF8}{mj}在\end{CJK} $[a, b]$ \begin{CJK}{UTF8}{mj}上一致收敛\end{CJK}.

(2) \begin{CJK}{UTF8}{mj}在任意区间\end{CJK} $[0, b]$ \begin{CJK}{UTF8}{mj}上\end{CJK} $l(y)$ \begin{CJK}{UTF8}{mj}不一致收敛\end{CJK}.

\section{5. 哈尔滨工程大学 2014 年研究生入学考试试题数学分析}
\begin{CJK}{UTF8}{mj}李扬\end{CJK}

\begin{CJK}{UTF8}{mj}微信公众号\end{CJK}: sxkyliyang

\begin{CJK}{UTF8}{mj}一\end{CJK}. \begin{CJK}{UTF8}{mj}填空题\end{CJK} (\begin{CJK}{UTF8}{mj}每小题\end{CJK} 5 \begin{CJK}{UTF8}{mj}分\end{CJK}, \begin{CJK}{UTF8}{mj}共\end{CJK} 50 \begin{CJK}{UTF8}{mj}分\end{CJK})

\begin{enumerate}
  \item $\lim _{x \rightarrow 0} \frac{x\left(\mathrm{e}^{x}-1\right)}{\cos x-1}=(\quad)$.

  \item $\left.\operatorname{grad}\left(x y^{2} z\right)\right|_{(x, y, z)=(1,1,1)}=(\quad)$.

  \item \begin{CJK}{UTF8}{mj}曲面\end{CJK} $x^{2}+2 y^{2}+z^{2}-10=0$, \begin{CJK}{UTF8}{mj}在点\end{CJK} $(1,2,1)$ \begin{CJK}{UTF8}{mj}处的法线方向为\end{CJK} ( $)$.

  \item \begin{CJK}{UTF8}{mj}函数\end{CJK} $f(x, y, z)=x y z^{2}$ \begin{CJK}{UTF8}{mj}在\end{CJK} $(1,1,1)$ \begin{CJK}{UTF8}{mj}点沿\end{CJK} $(1,1,1)$ \begin{CJK}{UTF8}{mj}方向的方向导数为\end{CJK} $(\quad)$.

  \item \begin{CJK}{UTF8}{mj}幂级数\end{CJK} $\sum_{n=0}^{\infty} \frac{(n !)^{3}}{(3 n) !}(x-1)^{3 n}$ \begin{CJK}{UTF8}{mj}的收敛半径为\end{CJK} ( ).

  \item \begin{CJK}{UTF8}{mj}设\end{CJK} $F^{\prime}(x)=f(x)$, \begin{CJK}{UTF8}{mj}则\end{CJK} $\mathrm{e}^{F(\sin x)}=(\quad)$.

  \item $\left(x \mathrm{e}^{x}\right)^{(100)}=(\quad) .$

  \item $\operatorname{div}(x, x y, x y z)=(\quad)$.

  \item $\sup (\sin x)=(\quad)$.

  \item $\sum_{n=1}^{\infty} \frac{1}{n(\ln n)^{p}(\ln \ln n)^{5}}$ \begin{CJK}{UTF8}{mj}收敛\end{CJK}, \begin{CJK}{UTF8}{mj}则\end{CJK} $p \in(\quad)$.

\end{enumerate}
\begin{CJK}{UTF8}{mj}二\end{CJK}. \begin{CJK}{UTF8}{mj}选择题\end{CJK} (\begin{CJK}{UTF8}{mj}每小题\end{CJK} 3 \begin{CJK}{UTF8}{mj}分\end{CJK}, \begin{CJK}{UTF8}{mj}共\end{CJK} 15 \begin{CJK}{UTF8}{mj}分\end{CJK})

\begin{enumerate}
  \item \begin{CJK}{UTF8}{mj}若\end{CJK} $f(x)=-x, x \in(0, \pi]$ \begin{CJK}{UTF8}{mj}展开为余弦傅里叶级数时\end{CJK}, \begin{CJK}{UTF8}{mj}其和在\end{CJK} $x=-\pi$ \begin{CJK}{UTF8}{mj}点的值为\end{CJK} $(\quad)$.\\
A. $-\pi$\\
B. $\pi$\\
C. 0\\
D. 1

  \item $\int_{0}^{x} t \mathrm{e}^{t} \mathrm{~d} t=(\quad)$.\\
A. $t \mathrm{e}^{t}$\\
B. $t \mathrm{e}^{t}-\mathrm{e}^{t}$\\
C. $\mathrm{e}^{t}$\\
D. $t$

  \item \begin{CJK}{UTF8}{mj}设\end{CJK} $f(x, y)$ \begin{CJK}{UTF8}{mj}在\end{CJK} $[0,1 ; 0,1]$ \begin{CJK}{UTF8}{mj}上连续\end{CJK}, \begin{CJK}{UTF8}{mj}则\end{CJK} $\int_{0}^{1} \mathrm{~d} y \int_{y}^{\sqrt{y}} f(x, y) \mathrm{d} x$ \begin{CJK}{UTF8}{mj}等于\end{CJK} $(\quad)$.\\
A. $\int_{y}^{\sqrt{y}} \mathrm{~d} x \int_{0}^{1} f(x, y) \mathrm{d} y$\\
B. $-\int_{0}^{1} \mathrm{~d} x \int_{x^{2}}^{x} f(x, y) \mathrm{d} y$\\
C. $\int_{0}^{1} \mathrm{~d} x \int_{x}^{\sqrt{x}} f(x, y) \mathrm{d} y$\\
D. $\int_{y}^{1} \mathrm{~d} x \int_{x^{2}}^{x} f(x, y) \mathrm{d} y$

  \item \begin{CJK}{UTF8}{mj}设函数\end{CJK} $f(x, y)=x^{6}+2 x^{3} y+x^{2}+2 x+y^{2}+3$, \begin{CJK}{UTF8}{mj}则\end{CJK} $\frac{\partial^{4} f}{\partial x^{3} \partial y}=(\quad)$.\\
A. $12-\pi$\\
B. $1 \pi$\\
C. $6 x^{5}$\\
D. $6 \cdot 5 \cdot 4 x^{3}-12 y+2 x^{3}+2 y$ 5. \begin{CJK}{UTF8}{mj}设\end{CJK} $f(x, y)=|x|-|y|$, \begin{CJK}{UTF8}{mj}则\end{CJK} $f(x, y)$ \begin{CJK}{UTF8}{mj}在点\end{CJK} $(0,0)$ \begin{CJK}{UTF8}{mj}处\end{CJK} $(\quad)$.\\
A. \begin{CJK}{UTF8}{mj}偏导数存在\end{CJK}, \begin{CJK}{UTF8}{mj}不连续\end{CJK}\\
B. \begin{CJK}{UTF8}{mj}连续\end{CJK}, \begin{CJK}{UTF8}{mj}偏导数不存在\end{CJK}\\
C. \begin{CJK}{UTF8}{mj}偏导数存在\end{CJK}, \begin{CJK}{UTF8}{mj}但不可微\end{CJK}\\
D. \begin{CJK}{UTF8}{mj}可微\end{CJK}

\end{enumerate}
\begin{CJK}{UTF8}{mj}三\end{CJK}. \begin{CJK}{UTF8}{mj}求极限\end{CJK}
$$
\lim _{n \rightarrow+\infty}\left(\frac{1^{3}}{n^{4}}+\frac{2^{3}}{n^{4}}+\cdots+\frac{n^{3}}{n^{4}}\right)
$$
\begin{CJK}{UTF8}{mj}四\end{CJK}. \begin{CJK}{UTF8}{mj}求极限\end{CJK}
$$
\lim _{x \rightarrow 0} \frac{\int_{0}^{x} \sin t^{4} \mathrm{~d} t}{x^{5}}
$$
\begin{CJK}{UTF8}{mj}五\end{CJK}. \begin{CJK}{UTF8}{mj}隐函数由\end{CJK}
$$
\left\{\begin{array}{l}
x+\mathrm{e}^{y}-\mathrm{e}^{z}=0 \\
x^{2}+y+z=1
\end{array}\right.
$$
\begin{CJK}{UTF8}{mj}确定\end{CJK}, \begin{CJK}{UTF8}{mj}求\end{CJK} $\frac{\mathrm{d}^{2} y}{\mathrm{~d} x^{2}}$ \begin{CJK}{UTF8}{mj}在\end{CJK} $x=0$ \begin{CJK}{UTF8}{mj}处的值\end{CJK}.

\begin{CJK}{UTF8}{mj}只\end{CJK}. \begin{CJK}{UTF8}{mj}将\end{CJK}
$$
x^{2} \sin (x-1)
$$
\begin{CJK}{UTF8}{mj}展开为\end{CJK} $x$ \begin{CJK}{UTF8}{mj}的幂级数\end{CJK}.

\begin{CJK}{UTF8}{mj}七\end{CJK}. \begin{CJK}{UTF8}{mj}计算\end{CJK}
$$
\iint_{S} x y \mathrm{~d} y \mathrm{~d} z+y z \mathrm{~d} x \mathrm{~d} z+x z \mathrm{~d} x \mathrm{~d} y .
$$
\begin{CJK}{UTF8}{mj}其中\end{CJK} $S$ \begin{CJK}{UTF8}{mj}为如下正方体\end{CJK} $V$ \begin{CJK}{UTF8}{mj}的边界\end{CJK} (\begin{CJK}{UTF8}{mj}方向向外\end{CJK}) $V=\{(x, y, z) \mid 0 \leq x \leq 1,0 \leq y \leq 1,0 \leq z \leq 1\}$.

\begin{CJK}{UTF8}{mj}八\end{CJK}. \begin{CJK}{UTF8}{mj}求函数\end{CJK}
$$
f(x, y, z)=x-2 y+2 z
$$
\begin{CJK}{UTF8}{mj}在限制条件\end{CJK} $x^{2}+y^{2}+z^{2}=1$ \begin{CJK}{UTF8}{mj}下的条件极值\end{CJK}.

\begin{CJK}{UTF8}{mj}九\end{CJK}. \begin{CJK}{UTF8}{mj}设\end{CJK} $\sum_{n=1}^{\infty} w_{n}$ \begin{CJK}{UTF8}{mj}收敛\end{CJK}, \begin{CJK}{UTF8}{mj}其中\end{CJK} $w_{n}=\sum_{k=j_{n-1}+1}^{j_{n}} u_{k}, n=1,2, \cdots, j_{0}=0, j_{0}<j_{1}<\cdots$. \begin{CJK}{UTF8}{mj}若\end{CJK}
$$
\lim _{n \rightarrow \infty} \max _{1 \leq p \leq j_{n}-j_{n-1}}\left|\sum_{k=j_{n}+1}^{j_{n-1}+p} u_{k}\right|=0
$$
\begin{CJK}{UTF8}{mj}证明\end{CJK}: \begin{CJK}{UTF8}{mj}级数\end{CJK} $\sum_{n=1}^{\infty} u_{n}$ \begin{CJK}{UTF8}{mj}收敛\end{CJK}.

\section{6. 哈尔滨工程大学 2015 年研究生入学考试试题数学分析}
\begin{CJK}{UTF8}{mj}李扬\end{CJK}

\begin{CJK}{UTF8}{mj}微信公众号\end{CJK}: sxkyliyang

\begin{enumerate}
  \item \begin{CJK}{UTF8}{mj}求定积分\end{CJK}
\end{enumerate}
$$
\int_{1}^{2} x \ln x \mathrm{~d} x
$$

\begin{enumerate}
  \setcounter{enumi}{2}
  \item \begin{CJK}{UTF8}{mj}求函数\end{CJK}
\end{enumerate}
$$
y=x^{3} \mathrm{e}^{x}
$$
\begin{CJK}{UTF8}{mj}的\end{CJK} 100 \begin{CJK}{UTF8}{mj}阶导数\end{CJK}.

\begin{enumerate}
  \setcounter{enumi}{3}
  \item \begin{CJK}{UTF8}{mj}求极限\end{CJK}
\end{enumerate}
$$
\lim _{n \rightarrow+\infty}\left(\frac{1}{n}+\frac{1}{n^{2}}+\cdots+\frac{n-2}{n^{2}}+\frac{n-1}{n^{2}}\right) .
$$

\begin{enumerate}
  \setcounter{enumi}{4}
  \item \begin{CJK}{UTF8}{mj}证明在闭区间\end{CJK} $[a, b](a<b)$ \begin{CJK}{UTF8}{mj}上连续的函数\end{CJK}, \begin{CJK}{UTF8}{mj}在闭区间\end{CJK} $[a, b]$ \begin{CJK}{UTF8}{mj}上有界\end{CJK}.

  \item \begin{CJK}{UTF8}{mj}讨论\end{CJK}

\end{enumerate}
$$
f(x)=(1+x)^{\frac{1}{x}}(x>-1)
$$
\begin{CJK}{UTF8}{mj}的单调性\end{CJK}, \begin{CJK}{UTF8}{mj}并求\end{CJK} $\lim _{x \rightarrow 0} f(x), \lim _{x \rightarrow-1^{+}} f(x), \lim _{x \rightarrow+\infty} f(x)$.

\begin{enumerate}
  \setcounter{enumi}{6}
  \item \begin{CJK}{UTF8}{mj}计算二重积分\end{CJK}
\end{enumerate}
$$
\iint_{D} \cos \sqrt{\left(x^{2}+y^{2}\right)} \mathrm{d} x \mathrm{~d} y
$$
\begin{CJK}{UTF8}{mj}其中\end{CJK} $D$ \begin{CJK}{UTF8}{mj}为\end{CJK} $1 \leq x^{2}+y^{2} \leq 9$ \begin{CJK}{UTF8}{mj}在\end{CJK} $1,2,3$ \begin{CJK}{UTF8}{mj}象限部分\end{CJK}.

\begin{enumerate}
  \setcounter{enumi}{7}
  \item \begin{CJK}{UTF8}{mj}将\end{CJK}
\end{enumerate}
$$
f(x)= \begin{cases}x+\pi, & x \in[-\pi, 0) \\ x, & x \in[0, \pi]\end{cases}
$$
\begin{CJK}{UTF8}{mj}在\end{CJK} $[-\pi, \pi]$ \begin{CJK}{UTF8}{mj}上展开为傅里叶级数\end{CJK}, \begin{CJK}{UTF8}{mj}并求该傅里叶级数在\end{CJK} $x=\pi$ \begin{CJK}{UTF8}{mj}点处的和\end{CJK}.

\begin{enumerate}
  \setcounter{enumi}{8}
  \item \begin{CJK}{UTF8}{mj}求\end{CJK}
\end{enumerate}
$$
\sum_{n=1}^{\infty} \frac{(-1)^{n} n}{(2 n+1) !} x^{n-1}
$$
\begin{CJK}{UTF8}{mj}的收敛范围\end{CJK}, \begin{CJK}{UTF8}{mj}并利用其求级数\end{CJK}
$$
\sum_{n=1}^{\infty} \frac{(-1)^{n} n}{(2 n+1) !}
$$
\begin{CJK}{UTF8}{mj}的和\end{CJK}.

\begin{enumerate}
  \setcounter{enumi}{9}
  \item \begin{CJK}{UTF8}{mj}讨论\end{CJK} $p, q$ \begin{CJK}{UTF8}{mj}为何值时\end{CJK}, \begin{CJK}{UTF8}{mj}级数\end{CJK}
\end{enumerate}
$$
\sum_{n=3}^{\infty} \frac{1}{n^{q}(\ln n)(\ln \ln n)^{p}}
$$
\begin{CJK}{UTF8}{mj}收敛\end{CJK}, \begin{CJK}{UTF8}{mj}并简单说明理由\end{CJK}.

\begin{enumerate}
  \setcounter{enumi}{10}
  \item \begin{CJK}{UTF8}{mj}说明当二元函数\end{CJK} $F(u, v)$ \begin{CJK}{UTF8}{mj}满足怎样条件时\end{CJK}, \begin{CJK}{UTF8}{mj}由\end{CJK}
\end{enumerate}
$$
F(x y, z-2 x)=0
$$
\begin{CJK}{UTF8}{mj}可唯一确定可微隐函数\end{CJK} $z=z(x, y)$, \begin{CJK}{UTF8}{mj}并求该隐函数的全微分\end{CJK}. 11. \begin{CJK}{UTF8}{mj}改变二次积分\end{CJK}
$$
\int_{0}^{1} \mathrm{~d} y \int_{1-y}^{1} f(x, y) \mathrm{d} x+\int_{1}^{2} \mathrm{~d} y \int_{y-1}^{1} f(x, y) \mathrm{d} x
$$
\begin{CJK}{UTF8}{mj}的积分次序\end{CJK}.

\begin{enumerate}
  \setcounter{enumi}{12}
  \item \begin{CJK}{UTF8}{mj}讨论分片函数\end{CJK}
\end{enumerate}
$$
f(x, y)= \begin{cases}\frac{\left(x^{2}+y^{2}\right)}{\sin \left(\frac{1}{x^{2}+y^{2}}\right)}, & x^{2}+y^{2} \neq 0 \\ 0, & x^{2}+y^{2}=0\end{cases}
$$
\begin{CJK}{UTF8}{mj}的可微性\end{CJK}.

\begin{enumerate}
  \setcounter{enumi}{13}
  \item \begin{CJK}{UTF8}{mj}设\end{CJK} $l$ \begin{CJK}{UTF8}{mj}是单连通区域\end{CJK} $\sigma$ \begin{CJK}{UTF8}{mj}的边界\end{CJK}, $\mathbf{n}$ \begin{CJK}{UTF8}{mj}是\end{CJK} $l$ \begin{CJK}{UTF8}{mj}的外法线方向\end{CJK}, $\mathbf{r}$ \begin{CJK}{UTF8}{mj}是\end{CJK} $l$ \begin{CJK}{UTF8}{mj}上动点\end{CJK} $(x, y)$ \begin{CJK}{UTF8}{mj}到一定点\end{CJK} $\left(x_{0}, y_{0}\right)$ \begin{CJK}{UTF8}{mj}的向量\end{CJK}, $r$ \begin{CJK}{UTF8}{mj}是从\end{CJK} $l$ \begin{CJK}{UTF8}{mj}上动点\end{CJK} $(x, y)$ \begin{CJK}{UTF8}{mj}到一定点\end{CJK} $\left(x_{0}, y_{0}\right)$ \begin{CJK}{UTF8}{mj}的距离\end{CJK}. \begin{CJK}{UTF8}{mj}证明\end{CJK}: \begin{CJK}{UTF8}{mj}当\end{CJK} $\left(x_{0}, y_{0}\right)$ \begin{CJK}{UTF8}{mj}在\end{CJK} $\sigma$ \begin{CJK}{UTF8}{mj}外时\end{CJK},
\end{enumerate}
$$
\oint_{l} \frac{\cos (\mathbf{r}, \mathbf{n})}{r} \mathrm{~d} s=0 .
$$
\begin{CJK}{UTF8}{mj}当\end{CJK} $\left(x_{0}, y_{0}\right)$ \begin{CJK}{UTF8}{mj}在\end{CJK} $\sigma$ \begin{CJK}{UTF8}{mj}内时\end{CJK},
$$
\oint_{l} \frac{\cos (\mathbf{r}, \mathbf{n})}{r} \mathrm{~d} s=2 \pi
$$

\begin{enumerate}
  \setcounter{enumi}{14}
  \item \begin{CJK}{UTF8}{mj}将\end{CJK} $x \int_{0}^{x} \sin t^{3} \mathrm{~d} t$ \begin{CJK}{UTF8}{mj}展开为\end{CJK} $x$ \begin{CJK}{UTF8}{mj}的幂级数\end{CJK}, \begin{CJK}{UTF8}{mj}并利用此结果求极限\end{CJK}
\end{enumerate}
$$
\lim _{x \rightarrow 0} \frac{x \int_{0}^{x} \sin t^{3} \mathrm{~d} t}{x^{5}}
$$

\begin{enumerate}
  \setcounter{enumi}{15}
  \item \begin{CJK}{UTF8}{mj}求雉面\end{CJK} $z^{2}=x^{2}+y^{2}$ \begin{CJK}{UTF8}{mj}与平面\end{CJK} $x+y+z=2$ \begin{CJK}{UTF8}{mj}的交线到原点的最短距离\end{CJK}.
\end{enumerate}
\section{7. 哈尔滨工程大学 2016 年研究生入学考试试题数学分析}
\begin{CJK}{UTF8}{mj}李扬\end{CJK}

\begin{CJK}{UTF8}{mj}微信公众号\end{CJK}: sxkyliyang

\begin{enumerate}
  \item \begin{CJK}{UTF8}{mj}求\end{CJK}
\end{enumerate}
$$
\lim _{x \rightarrow 0}\left(\frac{\sin x}{x}\right)^{\frac{1}{6 x^{2}+1}}
$$
\begin{CJK}{UTF8}{mj}的值\end{CJK}.

\begin{enumerate}
  \setcounter{enumi}{2}
  \item \begin{CJK}{UTF8}{mj}利用单调有界数列必有极限\end{CJK}, \begin{CJK}{UTF8}{mj}证明\end{CJK}:
\end{enumerate}
$$
y_{n}=\lim _{n \rightarrow \infty}\left(1+\frac{1}{n}\right)^{n}
$$
\begin{CJK}{UTF8}{mj}存在\end{CJK}.

\begin{enumerate}
  \setcounter{enumi}{3}
  \item \begin{CJK}{UTF8}{mj}求不定积分\end{CJK}
\end{enumerate}
$$
I=\int \sqrt{x^{2}+a^{2}} \mathrm{~d} x
$$
\begin{CJK}{UTF8}{mj}的值\end{CJK}.

\begin{enumerate}
  \setcounter{enumi}{4}
  \item \begin{CJK}{UTF8}{mj}已知\end{CJK}
\end{enumerate}
$$
f(x)= \begin{cases}\frac{\sin x}{x}, & x \neq 0 \\ 1, & x=0\end{cases}
$$
\begin{CJK}{UTF8}{mj}求\end{CJK} $f^{\prime}(0), f^{\prime \prime}(0), f^{\prime \prime \prime}(0)$ \begin{CJK}{UTF8}{mj}的值\end{CJK}.

\begin{enumerate}
  \setcounter{enumi}{5}
  \item \begin{CJK}{UTF8}{mj}利用格林公式计算曲线积分\end{CJK}
\end{enumerate}
$$
\int_{l} \mathrm{e}^{x}[(1-\cos y) \mathrm{d} x-(y-\sin y) \mathrm{d} y]
$$
\begin{CJK}{UTF8}{mj}其中\end{CJK} $l$ \begin{CJK}{UTF8}{mj}为区域\end{CJK} $0<x<\pi, 0<y<\sin x$ \begin{CJK}{UTF8}{mj}的边界\end{CJK}.

\begin{enumerate}
  \setcounter{enumi}{6}
  \item (1) \begin{CJK}{UTF8}{mj}叙述数列的柯西收敛原理\end{CJK}.
\end{enumerate}
(2) \begin{CJK}{UTF8}{mj}利用该原理证明数列\end{CJK}
$$
x_{n}=\sum_{k=1}^{n} \frac{\sin k}{2^{k}}(n=1,2, \cdots)
$$
\begin{CJK}{UTF8}{mj}为收敛列\end{CJK}.

\begin{enumerate}
  \setcounter{enumi}{7}
  \item \begin{CJK}{UTF8}{mj}设函数\end{CJK} $f(x)$ \begin{CJK}{UTF8}{mj}可导\end{CJK}, \begin{CJK}{UTF8}{mj}且有\end{CJK}
\end{enumerate}
$$
g(x)=\int_{0}^{x} y f(x-y) \mathrm{d} y
$$
\begin{CJK}{UTF8}{mj}求\end{CJK} $g^{\prime \prime}(x)$. \begin{CJK}{UTF8}{mj}的值\end{CJK}.

\begin{enumerate}
  \setcounter{enumi}{8}
  \item \begin{CJK}{UTF8}{mj}设\end{CJK}
\end{enumerate}
$$
\frac{\partial u}{\partial x} u=f\left(x y^{2}, \frac{x}{y}\right) .
$$
\begin{CJK}{UTF8}{mj}其中\end{CJK} $f$ \begin{CJK}{UTF8}{mj}具有连续的二阶偏导数\end{CJK}. \begin{CJK}{UTF8}{mj}求\end{CJK} $\frac{\partial u}{\partial x}, \frac{\partial u}{\partial y}, \frac{\partial^{2} u}{\partial x^{2}}, \frac{\partial^{2} u}{\partial x \partial y}$ \begin{CJK}{UTF8}{mj}的值\end{CJK}.

\begin{enumerate}
  \setcounter{enumi}{9}
  \item \begin{CJK}{UTF8}{mj}证明函数\end{CJK}
\end{enumerate}
$$
f(x, y)= \begin{cases}\frac{x y}{\sqrt{x^{2}+y^{2}}}, & x^{2}+y^{2} \neq 0 \\ 0, & x^{2}+y^{2}=0 .\end{cases}
$$
\begin{CJK}{UTF8}{mj}在点\end{CJK} $(0,0)$ \begin{CJK}{UTF8}{mj}点的邻域中连续\end{CJK}, $f_{x}(x, y), f_{y}(x, y)$ \begin{CJK}{UTF8}{mj}有界\end{CJK}, \begin{CJK}{UTF8}{mj}但在\end{CJK} $(0,0)$ \begin{CJK}{UTF8}{mj}点处不可微\end{CJK}. 10. \begin{CJK}{UTF8}{mj}求出由抛物线\end{CJK} $y^{2}=p x, y^{2}=q x(0<p<q)$ \begin{CJK}{UTF8}{mj}以及双曲线\end{CJK} $x y=a, x y=b(0<a<b)$ \begin{CJK}{UTF8}{mj}所围成区域的面积\end{CJK}.

\begin{enumerate}
  \setcounter{enumi}{11}
  \item \begin{CJK}{UTF8}{mj}将以下函数展开为正弦级数\end{CJK}:
\end{enumerate}
$$
f(x)= \begin{cases}\sin \frac{\pi x}{l}, & 0<x<\frac{l}{2} \\ 0, & \frac{l}{2}<x<l\end{cases}
$$

\begin{enumerate}
  \setcounter{enumi}{12}
  \item \begin{CJK}{UTF8}{mj}设\end{CJK} $f(x)$ \begin{CJK}{UTF8}{mj}在\end{CJK} $[a, b]$ \begin{CJK}{UTF8}{mj}上连续\end{CJK}, \begin{CJK}{UTF8}{mj}在\end{CJK} $(a, b)$ \begin{CJK}{UTF8}{mj}内可导\end{CJK}, $b>a>0, f(a) \neq f(b)$. \begin{CJK}{UTF8}{mj}证明\end{CJK}: \begin{CJK}{UTF8}{mj}存在\end{CJK} $\xi, \eta \in(a, b)$, \begin{CJK}{UTF8}{mj}使得\end{CJK}
\end{enumerate}
$$
f^{\prime}(\xi)=\frac{a+b}{2 \eta} f^{\prime}(\eta)
$$

\section{8. 哈尔滨工程大学 2017 年研究生入学考试试题数学分析}
\begin{CJK}{UTF8}{mj}李扬\end{CJK}

\begin{CJK}{UTF8}{mj}微信公众号\end{CJK}: sxkyliyang

1 . \begin{CJK}{UTF8}{mj}求\end{CJK}
$$
\lim _{x \rightarrow \infty} \frac{\sin m x}{\sin n x}(m, n \neq 0) .
$$

\begin{enumerate}
  \setcounter{enumi}{2}
  \item \begin{CJK}{UTF8}{mj}利用单调有界准则证明数列\end{CJK} $\left\{y_{n}\right\}$ \begin{CJK}{UTF8}{mj}的极限存在\end{CJK}, \begin{CJK}{UTF8}{mj}其中\end{CJK} $a>0$,
\end{enumerate}
$$
y_{1}=\sqrt{a}, y_{2}=\sqrt{a+\sqrt{a}}, \cdots, y_{n}=\sqrt{a+\sqrt{a+\cdots+\sqrt{a+\cdots+\sqrt{a}}}} .
$$

\begin{enumerate}
  \setcounter{enumi}{3}
  \item \begin{CJK}{UTF8}{mj}求\end{CJK}
\end{enumerate}
$$
I=\int \sqrt{x^{2}+a^{2}} \mathrm{~d} x(a \neq 0)
$$

\begin{enumerate}
  \setcounter{enumi}{4}
  \item \begin{CJK}{UTF8}{mj}利用格林公式计算积分\end{CJK}
\end{enumerate}
$$
\int_{L}(x+y)^{2}-(x+y)^{2} \mathrm{~d} y .
$$
\begin{CJK}{UTF8}{mj}其中曲线\end{CJK} $L$ \begin{CJK}{UTF8}{mj}是沿以\end{CJK} $A(1,1), B(3,2), C(2,5)$ \begin{CJK}{UTF8}{mj}为顶点的三角形的正向边界\end{CJK}.

\begin{enumerate}
  \setcounter{enumi}{5}
  \item \begin{CJK}{UTF8}{mj}叙述级数收敛的柯西收敛准则\end{CJK}, \begin{CJK}{UTF8}{mj}并证明级数\end{CJK}
\end{enumerate}
$$
\sum_{n=0}^{\infty}(-1)^{n} \frac{1}{n+1}
$$
\begin{CJK}{UTF8}{mj}收敛\end{CJK}.

\begin{enumerate}
  \setcounter{enumi}{6}
  \item \begin{CJK}{UTF8}{mj}证明\end{CJK}:
\end{enumerate}
$$
y=\sin \frac{1}{x}
$$
\begin{CJK}{UTF8}{mj}在\end{CJK} $(c, 1)$ \begin{CJK}{UTF8}{mj}上一致收敛\end{CJK}, \begin{CJK}{UTF8}{mj}但在\end{CJK} $(0,1)$ \begin{CJK}{UTF8}{mj}上不一致收敛\end{CJK}, \begin{CJK}{UTF8}{mj}其中\end{CJK} $0<c<1$.

\begin{enumerate}
  \setcounter{enumi}{7}
  \item \begin{CJK}{UTF8}{mj}设函数\end{CJK} $f(x)$ \begin{CJK}{UTF8}{mj}可导\end{CJK}, \begin{CJK}{UTF8}{mj}且有\end{CJK}
\end{enumerate}
$$
g(x)=\int_{0}^{x} y f(x-y) \mathrm{d} y
$$
\begin{CJK}{UTF8}{mj}求\end{CJK} $g^{\prime \prime}(x)$ \begin{CJK}{UTF8}{mj}的值\end{CJK}.

\begin{enumerate}
  \setcounter{enumi}{8}
  \item \begin{CJK}{UTF8}{mj}设\end{CJK} $u=f(r), r=\sqrt{x^{2}+y^{2}}$, \begin{CJK}{UTF8}{mj}求证\end{CJK}:
\end{enumerate}
$$
\frac{\partial^{2} u}{\partial x^{2}}+\frac{\partial^{2} u}{\partial y^{2}}=\frac{\partial^{2} u}{\partial r^{2}}+\frac{1}{r} \frac{\partial u}{\partial r}
$$

\begin{enumerate}
  \setcounter{enumi}{9}
  \item \begin{CJK}{UTF8}{mj}证明\end{CJK}:
\end{enumerate}
$$
f(x, y)= \begin{cases}\frac{x y}{\sqrt{x^{2}+y^{2}}}, & (x, y) \neq(0,0) \\ 0, & (x, y)=(0,0) .\end{cases}
$$
\begin{CJK}{UTF8}{mj}在\end{CJK} $(0,0)$ \begin{CJK}{UTF8}{mj}连续\end{CJK}, $f_{x}, f_{y}$ \begin{CJK}{UTF8}{mj}在区域\end{CJK} $\mathbb{R}^{2}$ \begin{CJK}{UTF8}{mj}内有界\end{CJK}, \begin{CJK}{UTF8}{mj}但在\end{CJK} $(0,0)$ \begin{CJK}{UTF8}{mj}不可微\end{CJK}.

$10 .$
$$
F(y)=\int_{y}^{y^{2}} \frac{\sin x y}{x} \mathrm{~d} x
$$
\begin{CJK}{UTF8}{mj}求\end{CJK} $F^{\prime}(y)$. 11. \begin{CJK}{UTF8}{mj}这题不记得了\end{CJK}, \begin{CJK}{UTF8}{mj}遗㨔\end{CJK}.

\begin{enumerate}
  \setcounter{enumi}{12}
  \item $f(x)$ \begin{CJK}{UTF8}{mj}在\end{CJK} $[a, b]$ \begin{CJK}{UTF8}{mj}上连续\end{CJK}, $f(x)>0$,
\end{enumerate}
$$
F(x)=\int_{a}^{x} f(t) \mathrm{d} t+\int_{b}^{x} \frac{\mathrm{d} t}{f(t)} .
$$
\begin{CJK}{UTF8}{mj}求证\end{CJK}: $F(x)=0$ \begin{CJK}{UTF8}{mj}在\end{CJK} $(a, b)$ \begin{CJK}{UTF8}{mj}上仅有一个实数根\end{CJK}

\section{1. 湖南大学 2007 年研究生入学考试试题数学分析}
\begin{CJK}{UTF8}{mj}李扬\end{CJK}

\begin{CJK}{UTF8}{mj}微信公众号\end{CJK}: sxkyliyang

\begin{enumerate}
  \item ( 18 \begin{CJK}{UTF8}{mj}分\end{CJK}) \begin{CJK}{UTF8}{mj}计算\end{CJK}
\end{enumerate}
(1) $\lim _{n \rightarrow \infty} \sum_{k=1}^{n} \frac{k^{2}}{n^{3}+2 n+k}$;

(2) $\lim _{n \rightarrow \infty} \ln \sqrt[n]{2\left(2+\frac{2}{n}\right)\left(2+\frac{4}{n}\right) \cdots\left(2+\frac{2(n-1)}{n}\right)}$.

\begin{enumerate}
  \setcounter{enumi}{2}
  \item (16 \begin{CJK}{UTF8}{mj}分\end{CJK}) \begin{CJK}{UTF8}{mj}设\end{CJK} $x_{1}=1,2 x_{n+1}=x_{n}+\sqrt{x_{n}^{2}+\frac{1}{n^{p}}}(p>1), n=1,2, \cdots$, \begin{CJK}{UTF8}{mj}证明\end{CJK}:\begin{CJK}{UTF8}{mj}数列\end{CJK} $\left\{x_{n}\right\}$ \begin{CJK}{UTF8}{mj}收敛\end{CJK}.

  \item (16 \begin{CJK}{UTF8}{mj}分\end{CJK}) \begin{CJK}{UTF8}{mj}设\end{CJK} $f(x)$ \begin{CJK}{UTF8}{mj}在\end{CJK} $[a, b]$ \begin{CJK}{UTF8}{mj}上连续\end{CJK}, \begin{CJK}{UTF8}{mj}在\end{CJK} $(a, b)$ \begin{CJK}{UTF8}{mj}内可导\end{CJK}, \begin{CJK}{UTF8}{mj}且\end{CJK} $b>a>0$, \begin{CJK}{UTF8}{mj}证明\end{CJK}: \begin{CJK}{UTF8}{mj}存在\end{CJK} $\xi, \eta \in(a, b)$ \begin{CJK}{UTF8}{mj}使得\end{CJK}

\end{enumerate}
$$
f^{\prime}(\xi)=\frac{a^{2}+a b+b^{2}}{3 \eta} f^{\prime}(\eta)
$$

\begin{enumerate}
  \setcounter{enumi}{4}
  \item (16 \begin{CJK}{UTF8}{mj}分\end{CJK}) \begin{CJK}{UTF8}{mj}确定下面函数的连续区间\end{CJK}
\end{enumerate}
$$
g(y)=\int_{0}^{+\infty} \frac{\ln \left(1+x^{2}\right)}{x^{y}} \mathrm{~d} x
$$

\begin{enumerate}
  \setcounter{enumi}{5}
  \item ( 16 \begin{CJK}{UTF8}{mj}分\end{CJK}) \begin{CJK}{UTF8}{mj}设\end{CJK} $f_{n}(x)$ \begin{CJK}{UTF8}{mj}在\end{CJK} $[a, b]$ \begin{CJK}{UTF8}{mj}上连续\end{CJK} $(n=1,2, \cdots)$, \begin{CJK}{UTF8}{mj}且\end{CJK} $\left\{f_{n}(x)\right\}$ \begin{CJK}{UTF8}{mj}在开区间\end{CJK} $(a, b)$ \begin{CJK}{UTF8}{mj}内一致收敛于\end{CJK} $f(x)$. \begin{CJK}{UTF8}{mj}证明\end{CJK} $\left\{f_{n}(x)\right\}$ \begin{CJK}{UTF8}{mj}在闭区间\end{CJK} $[a, b]$ \begin{CJK}{UTF8}{mj}上一致收敛\end{CJK}.

  \item ( 18 \begin{CJK}{UTF8}{mj}分\end{CJK}) \begin{CJK}{UTF8}{mj}设\end{CJK} $f(t)$ \begin{CJK}{UTF8}{mj}是\end{CJK} $[0,1]$ \begin{CJK}{UTF8}{mj}上的连续函数\end{CJK}, \begin{CJK}{UTF8}{mj}令\end{CJK}

\end{enumerate}
$$
F(x, y)=\int_{0}^{1} f(t)|x+y-1| \mathrm{d} t .
$$
\begin{CJK}{UTF8}{mj}其中\end{CJK} $x, y$ \begin{CJK}{UTF8}{mj}满足\end{CJK} $x^{2}+y^{2} \leqslant 1$, \begin{CJK}{UTF8}{mj}求二阶偏导数\end{CJK} $F_{x x}$ \begin{CJK}{UTF8}{mj}和\end{CJK} $F_{y y}$.

\begin{enumerate}
  \setcounter{enumi}{7}
  \item (16 \begin{CJK}{UTF8}{mj}分\end{CJK}) \begin{CJK}{UTF8}{mj}求函数\end{CJK}
\end{enumerate}
$$
f(x)=\arctan \frac{2 x}{2-x^{2}}+\frac{1}{4} \ln \left|x^{2}-2 x+2\right|-\frac{1}{4} \ln \left|x^{2}+2 x+2\right|-\frac{1}{2} \arctan (x-1)-\frac{1}{2} \arctan (x+1),
$$
\begin{CJK}{UTF8}{mj}关于\end{CJK} $x$ \begin{CJK}{UTF8}{mj}的幂级数展开式和收敛半径\end{CJK}.

\begin{enumerate}
  \setcounter{enumi}{8}
  \item ( 16 \begin{CJK}{UTF8}{mj}分\end{CJK}) \begin{CJK}{UTF8}{mj}计算积分\end{CJK}
\end{enumerate}
$$
I=\iint_{D} \frac{(x+y)(\ln (x+y)-\ln y)}{\sqrt{2-x-y}} \mathrm{~d} x \mathrm{~d} y,
$$
\begin{CJK}{UTF8}{mj}其中区域\end{CJK} $D$ \begin{CJK}{UTF8}{mj}为\end{CJK} $x=0, x+y=1, y=x$ \begin{CJK}{UTF8}{mj}所围成的三角形区域\end{CJK}.

\begin{enumerate}
  \setcounter{enumi}{9}
  \item ( 16 \begin{CJK}{UTF8}{mj}分\end{CJK}) \begin{CJK}{UTF8}{mj}设\end{CJK} $f(x, y)$ \begin{CJK}{UTF8}{mj}在区域\end{CJK} $C:|x-1| \leqslant 2,|y-1| \leqslant 2$ \begin{CJK}{UTF8}{mj}上具有二阶连续偏导数\end{CJK}, $f(1,1)=0$, \begin{CJK}{UTF8}{mj}且在点\end{CJK} $(1,1)$ \begin{CJK}{UTF8}{mj}达到极值\end{CJK}, \begin{CJK}{UTF8}{mj}又设\end{CJK}
\end{enumerate}
$$
\left|\frac{\partial^{2} f(x, y)}{\partial x^{l} \partial y^{2-l}}\right| \leqslant M,(x, y) \in G,
$$
\begin{CJK}{UTF8}{mj}其中\end{CJK} $0 \leqslant l \leqslant 2$, \begin{CJK}{UTF8}{mj}取区域\end{CJK} $D: 0 \leqslant x \leqslant 1,0 \leqslant y \leqslant 1$, \begin{CJK}{UTF8}{mj}试证\end{CJK}:
$$
I=\int_{D} f(x, y) \mathrm{d} x \mathrm{~d} y \leqslant \frac{7}{12} M
$$

\section{2. 湖南大学 2008 年研究生入学考试试题数学分析}
\begin{CJK}{UTF8}{mj}李扬\end{CJK}

\begin{CJK}{UTF8}{mj}微信公众号\end{CJK}: sxkyliyang

\begin{enumerate}
  \item ( 16 \begin{CJK}{UTF8}{mj}分\end{CJK}) \begin{CJK}{UTF8}{mj}设实数列\end{CJK} $\left\{x_{n}\right\}$ \begin{CJK}{UTF8}{mj}满足\end{CJK} $x_{n}-x_{n-2} \rightarrow 0(n \rightarrow \infty)$. \begin{CJK}{UTF8}{mj}证明\end{CJK}:
\end{enumerate}
$$
\lim _{n \rightarrow \infty} \frac{x_{n}-x_{n-1}}{n}=0 .
$$

\begin{enumerate}
  \setcounter{enumi}{2}
  \item ( 16 \begin{CJK}{UTF8}{mj}分\end{CJK}) \begin{CJK}{UTF8}{mj}设函数\end{CJK} $f(x)$ \begin{CJK}{UTF8}{mj}在\end{CJK} $(0,1)$ \begin{CJK}{UTF8}{mj}内有定义\end{CJK}, \begin{CJK}{UTF8}{mj}且有\end{CJK} $e^{x} f(x)$ \begin{CJK}{UTF8}{mj}和\end{CJK} $e^{-f(x)}$ \begin{CJK}{UTF8}{mj}为\end{CJK} $(0,1)$ \begin{CJK}{UTF8}{mj}内的单调递增函数\end{CJK}. \begin{CJK}{UTF8}{mj}证明\end{CJK} $f(x)$ \begin{CJK}{UTF8}{mj}在\end{CJK} $(0,1)$ \begin{CJK}{UTF8}{mj}内连续\end{CJK}.

  \item ( 16 \begin{CJK}{UTF8}{mj}分\end{CJK}) \begin{CJK}{UTF8}{mj}设函数\end{CJK} $f(x)$ \begin{CJK}{UTF8}{mj}在\end{CJK} $[0,1]$ \begin{CJK}{UTF8}{mj}上可微\end{CJK}, \begin{CJK}{UTF8}{mj}且令\end{CJK}

\end{enumerate}
$$
\sup _{0<x<1}\left|f^{\prime}(x)\right|=C<\infty
$$
\begin{CJK}{UTF8}{mj}证明\end{CJK}, \begin{CJK}{UTF8}{mj}对任何正整数\end{CJK} $n$, \begin{CJK}{UTF8}{mj}有\end{CJK}
$$
\left|\sum_{j=0}^{n-1} \frac{f\left(\frac{j}{n}\right)}{n}-\int_{0}^{1} f(x) \mathrm{d} x\right| \leqslant \frac{C}{2 n}
$$

\begin{enumerate}
  \setcounter{enumi}{4}
  \item ( 16 \begin{CJK}{UTF8}{mj}分\end{CJK}) \begin{CJK}{UTF8}{mj}计算积分\end{CJK}
\end{enumerate}
$$
I=\iint_{D} \frac{\sin y \cos y}{y} \mathrm{~d} x \mathrm{~d} y,
$$
\begin{CJK}{UTF8}{mj}其中\end{CJK} $D$ \begin{CJK}{UTF8}{mj}是由直线\end{CJK} $y=x$ \begin{CJK}{UTF8}{mj}与抛物线\end{CJK} $x=y^{2}$ \begin{CJK}{UTF8}{mj}所围成的区域\end{CJK}.

\begin{enumerate}
  \setcounter{enumi}{5}
  \item (16 \begin{CJK}{UTF8}{mj}分\end{CJK}) \begin{CJK}{UTF8}{mj}证明\end{CJK}
\end{enumerate}
$$
\iint_{S} f(a x+b y+c z) \mathrm{d} x \mathrm{~d} y=2 \int_{-1}^{1} \sqrt{1-u^{2}} f\left(u \sqrt{a^{2}+b^{2}}+c\right) \mathrm{d} u .
$$
\begin{CJK}{UTF8}{mj}其中\end{CJK} $S: x^{2}+y^{2} \leqslant 1, a^{2}+b^{2} \neq 0$.

\begin{enumerate}
  \setcounter{enumi}{6}
  \item (16 \begin{CJK}{UTF8}{mj}分\end{CJK}) \begin{CJK}{UTF8}{mj}求\end{CJK} $g^{\prime}(\alpha)$, \begin{CJK}{UTF8}{mj}设\end{CJK}
\end{enumerate}
$$
g(\alpha)=\int_{1}^{+\infty} \frac{\arctan \alpha x}{x^{2} \sqrt{x^{2}-1}} \mathrm{~d} x .
$$

\begin{enumerate}
  \setcounter{enumi}{7}
  \item ( 22 \begin{CJK}{UTF8}{mj}分\end{CJK}) \begin{CJK}{UTF8}{mj}设函数列\end{CJK} $f_{n}(x)=n^{\alpha} x e^{-n x}$, \begin{CJK}{UTF8}{mj}当参数\end{CJK} $\alpha$ \begin{CJK}{UTF8}{mj}取什么值时\end{CJK}, \begin{CJK}{UTF8}{mj}有\end{CJK}\\
(1)\begin{CJK}{UTF8}{mj}函数列在闭区间\end{CJK} $[0,1]$ \begin{CJK}{UTF8}{mj}上一致收敛\end{CJK};\\
(2) $\lim _{n \rightarrow \infty} \int_{0}^{1} f_{n}(x) \mathrm{d} x$ \begin{CJK}{UTF8}{mj}可以积分号下去极限\end{CJK}.

  \item (16 \begin{CJK}{UTF8}{mj}分\end{CJK}) \begin{CJK}{UTF8}{mj}证明恒等式\end{CJK}

\end{enumerate}
$$
\int_{0}^{1} \frac{\mathrm{d} x}{x^{x}}=\sum_{n=1}^{\infty} \frac{1}{n^{n}}
$$

\begin{enumerate}
  \setcounter{enumi}{9}
  \item (16 \begin{CJK}{UTF8}{mj}分\end{CJK}) \begin{CJK}{UTF8}{mj}设\end{CJK} $p(x)$ \begin{CJK}{UTF8}{mj}为实系数多项式\end{CJK}, \begin{CJK}{UTF8}{mj}证明\end{CJK}
\end{enumerate}
$$
\lim _{n \rightarrow \infty}(n+1) \int_{0}^{1} x^{n} p(x) \mathrm{d} x=p(1),
$$
\begin{CJK}{UTF8}{mj}如果\end{CJK} $f(x)$ \begin{CJK}{UTF8}{mj}为区间\end{CJK} $[0,1]$ \begin{CJK}{UTF8}{mj}上的连续函数\end{CJK}, \begin{CJK}{UTF8}{mj}关于下式\end{CJK}
$$
\lim _{n \rightarrow \infty}(n+1) \int_{0}^{1} x^{n} f(x) \mathrm{d} x,
$$
\begin{CJK}{UTF8}{mj}你能得到一个什么结论\end{CJK}, \begin{CJK}{UTF8}{mj}并证明你的结论\end{CJK}.

\section{3. 湖南大学 2009 年研究生入学考试试题数学分析}
\begin{CJK}{UTF8}{mj}李扬\end{CJK}

\begin{CJK}{UTF8}{mj}微信公众号\end{CJK}: sxkyliyang

\begin{enumerate}
  \item (16 \begin{CJK}{UTF8}{mj}分\end{CJK}) \begin{CJK}{UTF8}{mj}求极限\end{CJK}
\end{enumerate}
$$
\lim _{n \rightarrow \infty} \prod_{k=2}^{n} \frac{k^{3}-1}{k^{3}+1} .
$$

\begin{enumerate}
  \setcounter{enumi}{2}
  \item (16 \begin{CJK}{UTF8}{mj}分\end{CJK}) \begin{CJK}{UTF8}{mj}设\end{CJK} $f$ \begin{CJK}{UTF8}{mj}在\end{CJK} $[a, b]$ \begin{CJK}{UTF8}{mj}上连续\end{CJK}, \begin{CJK}{UTF8}{mj}若对开区间\end{CJK} $(a, b)$ \begin{CJK}{UTF8}{mj}中的任一点均非\end{CJK} $f$ \begin{CJK}{UTF8}{mj}的极值点\end{CJK}, \begin{CJK}{UTF8}{mj}则\end{CJK} $f$ \begin{CJK}{UTF8}{mj}在\end{CJK} $[a, b]$ \begin{CJK}{UTF8}{mj}上单调\end{CJK}.

  \item ( 16 \begin{CJK}{UTF8}{mj}分\end{CJK}) \begin{CJK}{UTF8}{mj}已知\end{CJK} $f(x)$ \begin{CJK}{UTF8}{mj}在\end{CJK} $[0,1]$ \begin{CJK}{UTF8}{mj}上连续\end{CJK}, \begin{CJK}{UTF8}{mj}并且有\end{CJK}

\end{enumerate}
$$
\int_{0}^{1} f(x) \mathrm{d} x=0, \int_{0}^{1} x f(x) \mathrm{d} x=1
$$
\begin{CJK}{UTF8}{mj}证明\end{CJK}: \begin{CJK}{UTF8}{mj}存在\end{CJK} $\xi \in[0,1]$, \begin{CJK}{UTF8}{mj}使得\end{CJK} $|f(\xi)|>4$.

\begin{enumerate}
  \setcounter{enumi}{4}
  \item ( 16 \begin{CJK}{UTF8}{mj}分\end{CJK}) \begin{CJK}{UTF8}{mj}设函数\end{CJK} $f(x)$ \begin{CJK}{UTF8}{mj}在\end{CJK} $(-\infty,+\infty)$ \begin{CJK}{UTF8}{mj}上无限次可微\end{CJK}, \begin{CJK}{UTF8}{mj}且满足\end{CJK}:
\end{enumerate}
\begin{enumerate}
  \item \begin{CJK}{UTF8}{mj}存在\end{CJK} $M>0$, \begin{CJK}{UTF8}{mj}使得\end{CJK} $\left|f^{(k)}(x)\right| \leqslant M, x \in(-\infty,+\infty), k=1,2, \cdots$;

  \item $f\left(\frac{1}{2^{k}}\right)=0, n=1,2, \cdots .$

\end{enumerate}
\begin{CJK}{UTF8}{mj}证明\end{CJK}: $f(x)$ \begin{CJK}{UTF8}{mj}在\end{CJK} $(-\infty,+\infty)$ \begin{CJK}{UTF8}{mj}上恒为零\end{CJK}.

\begin{enumerate}
  \setcounter{enumi}{5}
  \item (16 \begin{CJK}{UTF8}{mj}分\end{CJK}) \begin{CJK}{UTF8}{mj}计算积分\end{CJK}
\end{enumerate}
$$
\int_{0}^{+\infty} \frac{1}{x^{4}+1} \mathrm{~d} x
$$

\begin{enumerate}
  \setcounter{enumi}{6}
  \item (16 \begin{CJK}{UTF8}{mj}分\end{CJK}) \begin{CJK}{UTF8}{mj}积分\end{CJK} $\int_{1}^{+\infty} f(x) \mathrm{d} x$ \begin{CJK}{UTF8}{mj}收敛\end{CJK}, \begin{CJK}{UTF8}{mj}且\end{CJK} $f(x)$ \begin{CJK}{UTF8}{mj}在\end{CJK} $[1,+\infty)$ \begin{CJK}{UTF8}{mj}上单调递减\end{CJK}, \begin{CJK}{UTF8}{mj}试证\end{CJK}:
\end{enumerate}
$$
\lim _{x \rightarrow+\infty} x f(x)=0 .
$$

\begin{enumerate}
  \setcounter{enumi}{7}
  \item ( 22 \begin{CJK}{UTF8}{mj}分\end{CJK}) \begin{CJK}{UTF8}{mj}设二元函数\end{CJK}
\end{enumerate}
$$
f(x, y)=\left\{\begin{array}{lr}
\left(x^{2}+y^{2}\right) \cos \frac{1}{\sqrt{x^{2}+y^{2}}}, & x^{2}+y^{2} \neq 0 \\
0, & x^{2}+y^{2}=0
\end{array}\right.
$$

\begin{enumerate}
  \item \begin{CJK}{UTF8}{mj}求\end{CJK} $f_{x}^{\prime}(0,0), f_{y}^{\prime}(0,0)$;

  \item \begin{CJK}{UTF8}{mj}证明\end{CJK}: $f_{x}^{\prime}(0,0), f_{y}^{\prime}(0,0)$ \begin{CJK}{UTF8}{mj}在\end{CJK} $(0,0)$ \begin{CJK}{UTF8}{mj}点不连续\end{CJK};

  \item \begin{CJK}{UTF8}{mj}证明\end{CJK}: $f(x, y)$ \begin{CJK}{UTF8}{mj}在\end{CJK} $(0,0)$ \begin{CJK}{UTF8}{mj}点可微\end{CJK}.

  \item (16 \begin{CJK}{UTF8}{mj}分\end{CJK}) \begin{CJK}{UTF8}{mj}求积分\end{CJK}

\end{enumerate}
$$
\iint_{\Sigma} y^{2} z \mathrm{~d} x \mathrm{~d} y+x z \mathrm{~d} y \mathrm{~d} z+x^{2} y \mathrm{~d} x \mathrm{~d} z .
$$
\begin{CJK}{UTF8}{mj}其中\end{CJK} $\Sigma$ \begin{CJK}{UTF8}{mj}是\end{CJK} $z=x^{2}+y^{2}, x^{2}+y^{2}=1$ \begin{CJK}{UTF8}{mj}和坐标面在第一卦限所围成曲面的外侧\end{CJK}.

\begin{enumerate}
  \setcounter{enumi}{9}
  \item ( 16 \begin{CJK}{UTF8}{mj}分\end{CJK}) \begin{CJK}{UTF8}{mj}记空间区域\end{CJK} $V_{t}=\{(x, y, z) \mid 0 \leqslant x \leqslant t, 0 \leqslant y \leqslant t, 0 \leqslant z \leqslant t\}$, \begin{CJK}{UTF8}{mj}设\end{CJK}
\end{enumerate}
$$
F(t)=\iiint_{V_{t}} f(x y z) \mathrm{d} x \mathrm{~d} y \mathrm{~d} z
$$
\begin{CJK}{UTF8}{mj}其中\end{CJK} $f(u)$ \begin{CJK}{UTF8}{mj}有一阶连续导数\end{CJK}, \begin{CJK}{UTF8}{mj}求\end{CJK} $F^{\prime}(t)$.

\section{4. 湖南大学 2010 年研究生入学考试试题数学分析 
 李扬 
 微信公众号: sxkyliyang}
\begin{enumerate}
  \item (16 \begin{CJK}{UTF8}{mj}分\end{CJK}) \begin{CJK}{UTF8}{mj}设正项级数\end{CJK} $\sum_{n=1}^{\infty} a_{n}$ \begin{CJK}{UTF8}{mj}收玫\end{CJK}, \begin{CJK}{UTF8}{mj}数列\end{CJK} $\left\{y_{n}\right\}: y_{1}=1,2 y_{n+1}=y_{n}+\sqrt{y_{n}^{2}+a_{n}},(n=1,2, \cdots)$. \begin{CJK}{UTF8}{mj}证明\end{CJK}: $\left\{y_{n}\right\}$ \begin{CJK}{UTF8}{mj}是递增的收玫数列\end{CJK}.

  \item (22 \begin{CJK}{UTF8}{mj}分\end{CJK}) \begin{CJK}{UTF8}{mj}假设函数\end{CJK} $f(x):[0,1] \rightarrow R$ \begin{CJK}{UTF8}{mj}有连续导数\end{CJK}, \begin{CJK}{UTF8}{mj}并且\end{CJK} $\int_{0}^{1} f(x) \mathrm{d} x=0$, \begin{CJK}{UTF8}{mj}证明\end{CJK}: \begin{CJK}{UTF8}{mj}对于\end{CJK} $\forall \alpha \in(0,1)$, \begin{CJK}{UTF8}{mj}有\end{CJK}

\end{enumerate}
$$
\left|\int_{0}^{\alpha} f(x) \mathrm{d} x\right| \leqslant \frac{1}{8} \max _{0 \leqslant x \leqslant 1}\left|f^{\prime}(x)\right|
$$

\begin{enumerate}
  \setcounter{enumi}{3}
  \item (16 \begin{CJK}{UTF8}{mj}分\end{CJK}) \begin{CJK}{UTF8}{mj}计算积分\end{CJK}
\end{enumerate}
$$
\int_{0}^{\frac{\pi}{2}} \cos 2 n x \ln \cos x d x
$$

\begin{enumerate}
  \setcounter{enumi}{4}
  \item (16 \begin{CJK}{UTF8}{mj}分\end{CJK}) \begin{CJK}{UTF8}{mj}计算\end{CJK}
\end{enumerate}
$$
f(y)=\int_{0}^{+\infty} e^{-x^{2}} \cos (2 x y) \mathrm{d} x,-\infty<y<+\infty
$$

\begin{enumerate}
  \setcounter{enumi}{5}
  \item ( 16 \begin{CJK}{UTF8}{mj}分\end{CJK}) \begin{CJK}{UTF8}{mj}设\end{CJK} $u(x, y)$ \begin{CJK}{UTF8}{mj}的所有二阶偏导数都连续\end{CJK}, \begin{CJK}{UTF8}{mj}并且\end{CJK}
\end{enumerate}
$$
\frac{\partial^{2} u}{\partial x^{2}}-\frac{\partial^{2} u}{\partial y^{2}}=0
$$
\begin{CJK}{UTF8}{mj}现若已知\end{CJK}
$$
u(x, 2 x)=x, u_{x}^{\prime}(x, 2 x)=x^{2}
$$
\begin{CJK}{UTF8}{mj}试求\end{CJK} $u_{x x}(x, 2 x), u_{y y}(x, 2 x)$.

\begin{enumerate}
  \setcounter{enumi}{6}
  \item ( 16 \begin{CJK}{UTF8}{mj}分\end{CJK}) \begin{CJK}{UTF8}{mj}计算线积分\end{CJK}
\end{enumerate}
$$
\oint_{C}\left[(x+1)^{2}+(y-2)^{2}\right] \mathrm{d} S
$$
\begin{CJK}{UTF8}{mj}其中\end{CJK} $C$ \begin{CJK}{UTF8}{mj}表示曲面\end{CJK} $x^{2}+y^{2}+z^{2}=1$ \begin{CJK}{UTF8}{mj}与\end{CJK} $x+y+z=0$ \begin{CJK}{UTF8}{mj}的交线\end{CJK}.

\begin{enumerate}
  \setcounter{enumi}{7}
  \item (16 \begin{CJK}{UTF8}{mj}分\end{CJK}) \begin{CJK}{UTF8}{mj}设\end{CJK} $f(x)$ \begin{CJK}{UTF8}{mj}为\end{CJK} $[1,2]$ \begin{CJK}{UTF8}{mj}上的连续正值函数\end{CJK}, \begin{CJK}{UTF8}{mj}令\end{CJK} $M_{n}=\int_{1}^{2} x^{n} f(x) \mathrm{d} x, n=1,2, \cdots$, \begin{CJK}{UTF8}{mj}证明\end{CJK}: \begin{CJK}{UTF8}{mj}幂级数\end{CJK} $\sum_{n=1}^{\infty} \frac{t^{n}}{M_{n}}$ \begin{CJK}{UTF8}{mj}的收敛半径\end{CJK} $r$ \begin{CJK}{UTF8}{mj}满足\end{CJK} $\frac{1}{2} \leqslant \frac{1}{r} \leqslant 1$.

  \item ( 16 \begin{CJK}{UTF8}{mj}分\end{CJK}) \begin{CJK}{UTF8}{mj}设\end{CJK} $f(x)=(\arctan x)^{2}$, \begin{CJK}{UTF8}{mj}求\end{CJK} $f^{(n)}(0)$.

  \item ( 16 \begin{CJK}{UTF8}{mj}分\end{CJK}) \begin{CJK}{UTF8}{mj}计算三重积分\end{CJK}

\end{enumerate}
$$
\iiint_{x^{2}+y^{2}+z^{2} \leqslant 1, x^{2}+y^{2}-z^{2} \geqslant \frac{1}{2}} z \mathrm{~d} x \mathrm{~d} y \mathrm{~d} z .
$$

\section{5. 湖南大学 2011 年研究生入学考试试题数学分析 
 李扬 
 微信公众号: sxkyliyang}
\begin{enumerate}
  \item (16 \begin{CJK}{UTF8}{mj}分\end{CJK}) $x_{n} \in(0,1), x_{0}=p, x_{n+1}=p+\varepsilon \sin x_{n},(n=0,1,2, \cdots)$, \begin{CJK}{UTF8}{mj}证明\end{CJK}: $\eta=\lim _{n \rightarrow \infty} x_{n}$ \begin{CJK}{UTF8}{mj}存在\end{CJK}, \begin{CJK}{UTF8}{mj}且\end{CJK} $\eta$ \begin{CJK}{UTF8}{mj}为方程\end{CJK} $x \sin x=p$ \begin{CJK}{UTF8}{mj}的唯一根\end{CJK}.

  \item ( 22 \begin{CJK}{UTF8}{mj}分\end{CJK}) $f(x)$ \begin{CJK}{UTF8}{mj}在\end{CJK} $[0,1]$ \begin{CJK}{UTF8}{mj}上连续\end{CJK}, $f(1)=0$, \begin{CJK}{UTF8}{mj}证明\end{CJK}:

\end{enumerate}
(1) $\left\{x^{n}\right\}$ \begin{CJK}{UTF8}{mj}在\end{CJK} $[0,1]$ \begin{CJK}{UTF8}{mj}上不一致收敛\end{CJK};

(2) $\left\{f(x) x^{n}\right\}$ \begin{CJK}{UTF8}{mj}在\end{CJK} $[0,1]$ \begin{CJK}{UTF8}{mj}上一致收敛\end{CJK}.

\begin{enumerate}
  \setcounter{enumi}{3}
  \item ( 16 \begin{CJK}{UTF8}{mj}分\end{CJK}) \begin{CJK}{UTF8}{mj}已知\end{CJK} $\sum_{n=1}^{\infty} \frac{1}{n^{2}}=\frac{\pi^{2}}{6}$, \begin{CJK}{UTF8}{mj}求\end{CJK} $\int_{0}^{+\infty} \ln \left(1+e^{-x}\right) \mathrm{d} x$.

  \item (16 \begin{CJK}{UTF8}{mj}分\end{CJK}) \begin{CJK}{UTF8}{mj}函数\end{CJK} $f(x), g(x)$ \begin{CJK}{UTF8}{mj}在\end{CJK} $[a, b]$ \begin{CJK}{UTF8}{mj}上黎曼可积\end{CJK}, $\int_{a}^{b} g(x) \mathrm{d} x=1, g(x) \geqslant 0$, \begin{CJK}{UTF8}{mj}且\end{CJK} $\varphi^{\prime \prime}(x) \geqslant 0$, \begin{CJK}{UTF8}{mj}证明\end{CJK}:

\end{enumerate}
$$
\varphi\left(\int_{a}^{b} g(x) f(x) \mathrm{d} x\right) \leqslant \int_{a}^{b} g(x) \varphi(f(x)) \mathrm{d} x
$$

\begin{enumerate}
  \setcounter{enumi}{5}
  \item (16 \begin{CJK}{UTF8}{mj}分\end{CJK}) \begin{CJK}{UTF8}{mj}求\end{CJK}
\end{enumerate}
$$
f(y)=\int_{0}^{+\infty} \frac{1-e^{-x y}}{x e^{2 x}} \mathrm{~d} x, y>-2 .
$$

\begin{enumerate}
  \setcounter{enumi}{6}
  \item (16 \begin{CJK}{UTF8}{mj}分\end{CJK}) \begin{CJK}{UTF8}{mj}函数\end{CJK} $f(\xi, \eta)$ \begin{CJK}{UTF8}{mj}的所有二阶偏导数都连续\end{CJK}, \begin{CJK}{UTF8}{mj}并且满足拉普拉斯方程\end{CJK}
\end{enumerate}
$$
\frac{\partial^{2} f}{\partial \xi^{2}}+\frac{\partial^{2} f}{\partial \eta^{2}}=0
$$
\begin{CJK}{UTF8}{mj}证明函数\end{CJK} $z=f\left(x^{2}-y^{2}, 2 x y\right)$ \begin{CJK}{UTF8}{mj}也满足拉普拉斯方程\end{CJK}
$$
\frac{\partial^{2} z}{\partial x^{2}}+\frac{\partial^{2} z}{\partial y^{2}}=0
$$

\begin{enumerate}
  \setcounter{enumi}{7}
  \item ( 16 \begin{CJK}{UTF8}{mj}分\end{CJK}) \begin{CJK}{UTF8}{mj}计算曲面积分\end{CJK} $\iint_{S}\left(6 x^{2}+4 y x^{2}+z\right) \mathrm{d} s, S$ \begin{CJK}{UTF8}{mj}为单位球面\end{CJK} $x^{2}+y^{2}+z^{2}=1$.

  \item ( 16 \begin{CJK}{UTF8}{mj}分\end{CJK}) \begin{CJK}{UTF8}{mj}设\end{CJK} $f(x)$ \begin{CJK}{UTF8}{mj}在\end{CJK} $[0,1]$ \begin{CJK}{UTF8}{mj}上黎曼可积\end{CJK}, \begin{CJK}{UTF8}{mj}在\end{CJK} $x=1$ \begin{CJK}{UTF8}{mj}可导\end{CJK}, $f(1)=0, f^{\prime}(1)=a$, \begin{CJK}{UTF8}{mj}证明\end{CJK}:

\end{enumerate}
$$
\lim _{n \rightarrow \infty} n^{2} \int_{0}^{1} x^{n} f(x) \mathrm{d} x=-a .
$$

\begin{enumerate}
  \setcounter{enumi}{9}
  \item ( 16 \begin{CJK}{UTF8}{mj}分\end{CJK}) \begin{CJK}{UTF8}{mj}已知\end{CJK} $a \leqslant b \leqslant c$, \begin{CJK}{UTF8}{mj}且\end{CJK} $x \in[0, a], y \in[0, b], z \in[0, c]$, \begin{CJK}{UTF8}{mj}又设\end{CJK} $f(x, y, z)=\min (x, y, z)$, \begin{CJK}{UTF8}{mj}计算\end{CJK}
\end{enumerate}
$$
\int_{0}^{a} \int_{0}^{b} \int_{0}^{c} f(x, y, z) \mathrm{d} z \mathrm{~d} y \mathrm{~d} x
$$

\section{6. 湖南大学 2012 年研究生入学考试试题数学分析 
 李扬 
 微信公众号: sxkyliyang}
\begin{enumerate}
  \item (16 \begin{CJK}{UTF8}{mj}分\end{CJK}) \begin{CJK}{UTF8}{mj}求下列极限\end{CJK}:\\
(1) $\lim _{n \rightarrow \infty}(n !)^{\frac{1}{n^{2}}}$;\\
(2) $\lim _{x \rightarrow 0} \frac{1}{x^{4}}\left[\ln \left(1+\sin ^{2} x\right)-6(\sqrt[3]{2-\cos x}-1)\right]$.

  \item ( 22 \begin{CJK}{UTF8}{mj}分\end{CJK}) \begin{CJK}{UTF8}{mj}设\end{CJK} $f(x)$ \begin{CJK}{UTF8}{mj}在\end{CJK} $[a, b]$ \begin{CJK}{UTF8}{mj}上连续\end{CJK}, \begin{CJK}{UTF8}{mj}对于\end{CJK} $\forall x \in[a, b]$, \begin{CJK}{UTF8}{mj}存在\end{CJK} $y \in[a, b]$, \begin{CJK}{UTF8}{mj}使得\end{CJK}

\end{enumerate}
$$
|f(y)| \leqslant L|f(x)|, 0<L<1
$$
\begin{CJK}{UTF8}{mj}证明\end{CJK}: \begin{CJK}{UTF8}{mj}至少存在一点\end{CJK} $\xi \in[a, b]$, \begin{CJK}{UTF8}{mj}使得\end{CJK} $f(\xi)=0$.

\begin{enumerate}
  \setcounter{enumi}{3}
  \item (18 \begin{CJK}{UTF8}{mj}分\end{CJK}) \begin{CJK}{UTF8}{mj}设\end{CJK} $f(x)$ \begin{CJK}{UTF8}{mj}在每个有限区间\end{CJK} $[a, b]$ \begin{CJK}{UTF8}{mj}上可积\end{CJK}, \begin{CJK}{UTF8}{mj}且\end{CJK} $\lim _{x \rightarrow+\infty} f(x)=L, \lim _{x \rightarrow-\infty} f(x)=M$ \begin{CJK}{UTF8}{mj}存在\end{CJK}, \begin{CJK}{UTF8}{mj}证明\end{CJK}: \begin{CJK}{UTF8}{mj}对任何\end{CJK} \begin{CJK}{UTF8}{mj}一个实数\end{CJK} $r>0$, \begin{CJK}{UTF8}{mj}反常积分\end{CJK}
\end{enumerate}
$$
\int_{-\infty}^{+\infty}[f(x+r)-f(x)] \mathrm{d} x
$$
\begin{CJK}{UTF8}{mj}存在\end{CJK}, \begin{CJK}{UTF8}{mj}并求其值\end{CJK}.

\begin{enumerate}
  \setcounter{enumi}{4}
  \item ( 16 \begin{CJK}{UTF8}{mj}分\end{CJK}) \begin{CJK}{UTF8}{mj}研究数列的敛散性\end{CJK}:
\end{enumerate}
$$
x_{n}=\sum_{k=1}^{n} k^{-\frac{2}{3}}-3 n^{\frac{1}{3}}
$$

\begin{enumerate}
  \setcounter{enumi}{5}
  \item (16 \begin{CJK}{UTF8}{mj}分\end{CJK}) \begin{CJK}{UTF8}{mj}设\end{CJK} $f$ \begin{CJK}{UTF8}{mj}为可微函数\end{CJK}, $u=f\left(x^{2}+y^{2}+z^{2}\right)$, \begin{CJK}{UTF8}{mj}且\end{CJK} $x, y, z$ \begin{CJK}{UTF8}{mj}满足方程\end{CJK} $3 x+2 y^{2}+z^{3}=6 x y z(*)$, \begin{CJK}{UTF8}{mj}试对于以\end{CJK} \begin{CJK}{UTF8}{mj}下两种情况\end{CJK}, \begin{CJK}{UTF8}{mj}分别求出\end{CJK} $\frac{\partial u}{\partial x}$ \begin{CJK}{UTF8}{mj}在点\end{CJK} $P(1,1,1)$ \begin{CJK}{UTF8}{mj}处的值\end{CJK}.
\end{enumerate}
(1) \begin{CJK}{UTF8}{mj}由方程\end{CJK} $(*)$ \begin{CJK}{UTF8}{mj}确定的隐函数\end{CJK} $z=z(x, y)$;

(2) \begin{CJK}{UTF8}{mj}由方程\end{CJK} $(*)$ \begin{CJK}{UTF8}{mj}确定的隐函数\end{CJK} $y=y(x, z)$.

\begin{enumerate}
  \setcounter{enumi}{6}
  \item (18 \begin{CJK}{UTF8}{mj}分\end{CJK}) \begin{CJK}{UTF8}{mj}设\end{CJK} $f(x, y)=\operatorname{sgn}(x-y)$, \begin{CJK}{UTF8}{mj}证明\end{CJK}: \begin{CJK}{UTF8}{mj}含参量积分\end{CJK} $F(y)=\int_{0}^{1} f(x, y) \mathrm{d} x$ \begin{CJK}{UTF8}{mj}在\end{CJK} $(-\infty,+\infty)$ \begin{CJK}{UTF8}{mj}上连续\end{CJK}.

  \item ( 16 \begin{CJK}{UTF8}{mj}分\end{CJK}) \begin{CJK}{UTF8}{mj}设区域\end{CJK} $D$ \begin{CJK}{UTF8}{mj}为\end{CJK} $x^{2}+y^{2} \leqslant 1$, \begin{CJK}{UTF8}{mj}证明\end{CJK}:

\end{enumerate}
$$
\frac{59}{231} \pi \leqslant \iint_{D} \sqrt{\left(x^{2}+y^{2}\right)^{3}} \sin \left(x^{2}+y^{2}\right) \mathrm{d} x \mathrm{~d} y \leqslant \frac{2}{7} \pi
$$

\begin{enumerate}
  \setcounter{enumi}{8}
  \item (16 \begin{CJK}{UTF8}{mj}分\end{CJK}) \begin{CJK}{UTF8}{mj}计算曲面积分\end{CJK} $\iint_{D}|x y z| \mathrm{d} s$, \begin{CJK}{UTF8}{mj}其中\end{CJK} $S$ \begin{CJK}{UTF8}{mj}为曲面\end{CJK} $z=x^{2}+y^{2}$ \begin{CJK}{UTF8}{mj}被平面\end{CJK} $z=1$ \begin{CJK}{UTF8}{mj}所割下的部分\end{CJK}.

  \item ( 16 \begin{CJK}{UTF8}{mj}分\end{CJK}) \begin{CJK}{UTF8}{mj}对于三角形\end{CJK} $\triangle A B C$, \begin{CJK}{UTF8}{mj}求\end{CJK} $18 \sin A+4 \sin B+3 \sin C$ \begin{CJK}{UTF8}{mj}的最大值\end{CJK}.

\end{enumerate}
\section{7. 湖南大学 2013 年研究生入学考试试题数学分析 
 李扬 
 微信公众号: sxkyliyang}
\begin{enumerate}
  \item (16 \begin{CJK}{UTF8}{mj}分\end{CJK}) \begin{CJK}{UTF8}{mj}设\end{CJK} $f(1)=2 \int_{0}^{\frac{1}{2}} e^{1-x^{2}} f(x) \mathrm{d} x$, \begin{CJK}{UTF8}{mj}证明\end{CJK}: \begin{CJK}{UTF8}{mj}存在\end{CJK} $\xi \in(0,1)$, \begin{CJK}{UTF8}{mj}使得\end{CJK} $f^{\prime}(\xi)=2 \xi f(\xi)$.

  \item ( 16 \begin{CJK}{UTF8}{mj}分\end{CJK}) \begin{CJK}{UTF8}{mj}函数\end{CJK} $f(x)$ \begin{CJK}{UTF8}{mj}满足\end{CJK}

\end{enumerate}
$$
\left|f^{\prime}(x)\right| \leqslant r<1,-\infty<x<+\infty .
$$
\begin{CJK}{UTF8}{mj}设\end{CJK} $x_{n+1}=f\left(x_{n}\right)$, \begin{CJK}{UTF8}{mj}证明\end{CJK}: \begin{CJK}{UTF8}{mj}极限\end{CJK} $\lim _{n \rightarrow \infty} x_{n}$ \begin{CJK}{UTF8}{mj}存在\end{CJK}.

\begin{enumerate}
  \setcounter{enumi}{3}
  \item (16 \begin{CJK}{UTF8}{mj}分\end{CJK}) \begin{CJK}{UTF8}{mj}展开下面函数为\end{CJK} $x$ \begin{CJK}{UTF8}{mj}的幂级数\end{CJK}, \begin{CJK}{UTF8}{mj}并确定收敛域\end{CJK}:
\end{enumerate}
$$
f(x)=\frac{-1}{(1-x)^{2}}+x \ln \left(\sqrt{1+x^{2}}-x\right) .
$$

\begin{enumerate}
  \setcounter{enumi}{4}
  \item ( 20 \begin{CJK}{UTF8}{mj}分\end{CJK}) \begin{CJK}{UTF8}{mj}证明\end{CJK}:
\end{enumerate}
(1) $\int_{1}^{k \pi} \frac{|\sin x|}{x} \mathrm{~d} x>\frac{2}{\pi} \ln \frac{k+1}{2}$;

(2) $\int_{0}^{+\infty} \frac{\sin x}{x} \mathrm{~d} x$ \begin{CJK}{UTF8}{mj}收敛但非绝对收敛\end{CJK}.

\begin{enumerate}
  \setcounter{enumi}{5}
  \item ( 16 \begin{CJK}{UTF8}{mj}分\end{CJK}) \begin{CJK}{UTF8}{mj}设\end{CJK} $S(x)=\sum_{n=1}^{\infty} a_{n} x^{n}$, \begin{CJK}{UTF8}{mj}其中\end{CJK} $a_{1}=a_{2}=1, a_{n}=a_{n-1}+a_{n-2},(n>2)$, \begin{CJK}{UTF8}{mj}求和函数\end{CJK} $S(x)$ \begin{CJK}{UTF8}{mj}及其收敛半\end{CJK} \begin{CJK}{UTF8}{mj}径\end{CJK}.

  \item (18 \begin{CJK}{UTF8}{mj}分\end{CJK}) \begin{CJK}{UTF8}{mj}已知\end{CJK}

\end{enumerate}
$$
u(x, t)=\frac{1}{2} \int_{0}^{1} \mathrm{~d} \eta \int_{x-1+\eta}^{x+1+\eta} f(\xi, \eta) \mathrm{d} \xi
$$
\begin{CJK}{UTF8}{mj}且有\end{CJK} $f(\xi, \eta), f_{\xi}(\xi, \eta)$ \begin{CJK}{UTF8}{mj}连续\end{CJK}, \begin{CJK}{UTF8}{mj}试求\end{CJK}
$$
\frac{\partial^{2} u}{\partial t^{2}}-\frac{\partial^{2} u}{\partial x^{2}}
$$

\begin{enumerate}
  \setcounter{enumi}{7}
  \item ( 16 \begin{CJK}{UTF8}{mj}分\end{CJK}) \begin{CJK}{UTF8}{mj}设函数\end{CJK} $f(x, y), f_{y}(x, y)$ \begin{CJK}{UTF8}{mj}在\end{CJK} $\left(x_{0}, y_{0}\right)$ \begin{CJK}{UTF8}{mj}的邻域内连续\end{CJK}, \begin{CJK}{UTF8}{mj}证明\end{CJK}: \begin{CJK}{UTF8}{mj}在\end{CJK} $x=x_{0}$ \begin{CJK}{UTF8}{mj}的某邻域内\end{CJK}, \begin{CJK}{UTF8}{mj}由方程\end{CJK} $y=y_{0}+\int_{x_{0}}^{x} f(\xi, y) \mathrm{d} \xi$ \begin{CJK}{UTF8}{mj}可以确定某个可导函数\end{CJK} $y=y(x)$, \begin{CJK}{UTF8}{mj}并求\end{CJK} $y^{\prime}(x)$.

  \item (16 \begin{CJK}{UTF8}{mj}分\end{CJK}) \begin{CJK}{UTF8}{mj}证明不等式\end{CJK}

\end{enumerate}
$$
e^{y}+x \ln x-x-x y \geqslant 0,(x \geqslant 1, y \geqslant 0)
$$

\begin{enumerate}
  \setcounter{enumi}{9}
  \item ( 16 \begin{CJK}{UTF8}{mj}分\end{CJK}) \begin{CJK}{UTF8}{mj}计算\end{CJK} $I=\int_{L} y^{2} \mathrm{~d} x+z^{2} \mathrm{~d} y+x^{2} \mathrm{~d} z$, \begin{CJK}{UTF8}{mj}其中\end{CJK}
\end{enumerate}
$$
L:\left\{\begin{array}{l}
x^{2}+y^{2}+z^{2}=R^{2} ; \\
x^{2}+y^{2}=R x
\end{array} \quad R>0, z \geqslant 0\right.
$$
\begin{CJK}{UTF8}{mj}取逆时针方向为正向\end{CJK}.

\section{8. 湖南大学 2014 年研究生入学考试试题数学分析}
\begin{CJK}{UTF8}{mj}李扬\end{CJK}

\begin{CJK}{UTF8}{mj}微信公众号\end{CJK}: sxkyliyang

\begin{enumerate}
  \item ( 15 \begin{CJK}{UTF8}{mj}分\end{CJK}) \begin{CJK}{UTF8}{mj}用极限定义证明若\end{CJK} $\lim _{n \rightarrow \infty} a_{n}=a$, \begin{CJK}{UTF8}{mj}则\end{CJK}
\end{enumerate}
$$
\lim _{n \rightarrow \infty} \frac{a_{1}+a_{2}+\cdots+a_{n}}{n}=a .
$$

\begin{enumerate}
  \setcounter{enumi}{2}
  \item (20 \begin{CJK}{UTF8}{mj}分\end{CJK})
\end{enumerate}
(1) \begin{CJK}{UTF8}{mj}设\end{CJK}
$$
\lim _{x \rightarrow 0} \frac{\ln \left(1+\frac{f(x)}{\sin 3 x}\right)}{2^{x}-1}=2,
$$
\begin{CJK}{UTF8}{mj}求\end{CJK} $\lim _{x \rightarrow 0} \frac{f(x)}{x^{2}}$.

(2) \begin{CJK}{UTF8}{mj}设\end{CJK} $f(x)$ \begin{CJK}{UTF8}{mj}有一阶连续导数\end{CJK}, \begin{CJK}{UTF8}{mj}且\end{CJK} $f(0)=0, f^{\prime}(0)=2$, \begin{CJK}{UTF8}{mj}求\end{CJK}
$$
\lim _{x \rightarrow 0}(1+f(x))^{\frac{1}{\ln (1+x)}} .
$$

\begin{enumerate}
  \setcounter{enumi}{3}
  \item ( 15 \begin{CJK}{UTF8}{mj}分\end{CJK}) \begin{CJK}{UTF8}{mj}已知\end{CJK} $f(x)$ \begin{CJK}{UTF8}{mj}有三阶导数\end{CJK}, \begin{CJK}{UTF8}{mj}且\end{CJK} $g(x)=|x-1|^{3} f(x)$, \begin{CJK}{UTF8}{mj}试证\end{CJK}:\begin{CJK}{UTF8}{mj}当\end{CJK} $f(1)=0$ \begin{CJK}{UTF8}{mj}时\end{CJK}, $g(x)$ \begin{CJK}{UTF8}{mj}在\end{CJK} $x=1$ \begin{CJK}{UTF8}{mj}处有三阶\end{CJK} \begin{CJK}{UTF8}{mj}导数\end{CJK}, \begin{CJK}{UTF8}{mj}但当\end{CJK} $f(x) \neq 0$ \begin{CJK}{UTF8}{mj}时\end{CJK}, $g(x)$ \begin{CJK}{UTF8}{mj}在\end{CJK} $x=1$ \begin{CJK}{UTF8}{mj}处无三阶导数\end{CJK}.

  \item ( 20 \begin{CJK}{UTF8}{mj}分\end{CJK}) \begin{CJK}{UTF8}{mj}设\end{CJK} $f^{\prime}(x)$ \begin{CJK}{UTF8}{mj}在\end{CJK} $[0,1]$ \begin{CJK}{UTF8}{mj}上有界可积\end{CJK}, \begin{CJK}{UTF8}{mj}且\end{CJK} $h=\frac{1}{n}$, \begin{CJK}{UTF8}{mj}证明\end{CJK}:

\end{enumerate}
$$
\int_{0}^{1} f(x) \mathrm{d} x=\sum_{k=1}^{n} f(k h) \cdot h-\frac{h}{2}[f(1)-f(0)]+o\left(\frac{1}{n}\right)
$$

\begin{enumerate}
  \setcounter{enumi}{5}
  \item (16 \begin{CJK}{UTF8}{mj}分\end{CJK}) \begin{CJK}{UTF8}{mj}设\end{CJK} $f(x)$ \begin{CJK}{UTF8}{mj}在\end{CJK} $[0,1]$ \begin{CJK}{UTF8}{mj}上二阶导数连续\end{CJK}, $f(0)=f(1)=0$, \begin{CJK}{UTF8}{mj}并且\end{CJK} $x \in(0,1)$ \begin{CJK}{UTF8}{mj}时\end{CJK}, $\left|f^{\prime \prime}(x)\right| \leqslant A$, \begin{CJK}{UTF8}{mj}求证\end{CJK}
\end{enumerate}
$$
\left|f^{\prime}(x)\right| \leqslant \frac{A}{2}, \forall x \in(0,1] \text {. }
$$

\begin{enumerate}
  \setcounter{enumi}{6}
  \item ( 20 \begin{CJK}{UTF8}{mj}分\end{CJK}) \begin{CJK}{UTF8}{mj}计算第一型曲面积分\end{CJK}
\end{enumerate}
$$
\iint_{S} \frac{1}{x^{2}+y^{2}+z^{2}} \mathrm{~d} S
$$
\begin{CJK}{UTF8}{mj}其中\end{CJK} $S=\left\{(x, y, z) \mid x^{2}+y^{2}=1,0 \leqslant z \leqslant 1\right\}$.

\begin{enumerate}
  \setcounter{enumi}{7}
  \item ( 20 \begin{CJK}{UTF8}{mj}分\end{CJK}) \begin{CJK}{UTF8}{mj}设\end{CJK} $f_{0}(x)$ \begin{CJK}{UTF8}{mj}在\end{CJK} $[a, b]$ \begin{CJK}{UTF8}{mj}上连续\end{CJK}, $g(x, y)$ \begin{CJK}{UTF8}{mj}在闭区间\end{CJK} $D=\{a \leqslant x \leqslant b, a \leqslant y \leqslant b\}$ \begin{CJK}{UTF8}{mj}上连续\end{CJK}, \begin{CJK}{UTF8}{mj}对任何\end{CJK} $x \in[a, b]$, \begin{CJK}{UTF8}{mj}令\end{CJK} $f_{n}(x)=\int_{a}^{x} g(x, y) f_{n-1}(y) \mathrm{d} y, n=1,2,3, \cdots$, \begin{CJK}{UTF8}{mj}证明\end{CJK}: \begin{CJK}{UTF8}{mj}函数列\end{CJK} $\left\{f_{n}(x)\right\}$ \begin{CJK}{UTF8}{mj}在\end{CJK} $[a, b]$ \begin{CJK}{UTF8}{mj}上一致收敛\end{CJK} \begin{CJK}{UTF8}{mj}于零\end{CJK}.

  \item (20 \begin{CJK}{UTF8}{mj}分\end{CJK}) \begin{CJK}{UTF8}{mj}求椭球\end{CJK}

\end{enumerate}
$$
\frac{x^{2}}{a^{2}}+\frac{y^{2}}{b^{2}}+\frac{z^{2}}{c^{2}}=1
$$
\begin{CJK}{UTF8}{mj}在第一卦限中的切平面与三个坐标平面所围成四面体的最小体积\end{CJK} $V$.

\section{9. 湖南大学 2015 年研究生入学考试试题数学分析}
\begin{CJK}{UTF8}{mj}李扬\end{CJK}

\begin{CJK}{UTF8}{mj}微信公众号\end{CJK}: sxkyliyang

\begin{enumerate}
  \item (16 \begin{CJK}{UTF8}{mj}分\end{CJK}) \begin{CJK}{UTF8}{mj}已知函数\end{CJK} $f(x)=|\sin (x-1)|^{3}, x \in(0,2)$.
\end{enumerate}
(1) \begin{CJK}{UTF8}{mj}证明\end{CJK} $f^{\prime \prime}(0)$ \begin{CJK}{UTF8}{mj}不存在\end{CJK};

(2) \begin{CJK}{UTF8}{mj}说明点\end{CJK} $x=0$ \begin{CJK}{UTF8}{mj}是不是导数\end{CJK} $f^{\prime \prime}(x)$ \begin{CJK}{UTF8}{mj}的可去间断点\end{CJK}.

\begin{enumerate}
  \setcounter{enumi}{2}
  \item ( 15 \begin{CJK}{UTF8}{mj}分\end{CJK})\begin{CJK}{UTF8}{mj}设\end{CJK} $f(x)$ \begin{CJK}{UTF8}{mj}在区间\end{CJK} $[0,1]$ \begin{CJK}{UTF8}{mj}上可微\end{CJK}, \begin{CJK}{UTF8}{mj}且\end{CJK} $f(0)=0, f(1)=1, m_{1}, m_{2}, m_{3}$ \begin{CJK}{UTF8}{mj}均大于零\end{CJK}, \begin{CJK}{UTF8}{mj}证明\end{CJK}: \begin{CJK}{UTF8}{mj}在区间\end{CJK} $[0,1]$ \begin{CJK}{UTF8}{mj}内存在三个互不相等的数\end{CJK} $x_{1}, x_{2}, x_{3}$, \begin{CJK}{UTF8}{mj}使得\end{CJK}
\end{enumerate}
$$
\frac{m_{1}}{f^{\prime}\left(x_{1}\right)}+\frac{m_{2}}{f^{\prime}\left(x_{2}\right)}+\frac{m_{3}}{f^{\prime}\left(x_{3}\right)}=m_{1}+m_{2}+m_{3} .
$$

\begin{enumerate}
  \setcounter{enumi}{3}
  \item ( 16 \begin{CJK}{UTF8}{mj}分\end{CJK}) \begin{CJK}{UTF8}{mj}设\end{CJK} $S_{n}=\frac{1}{n+1}+\frac{1}{n+2}+\cdots+\frac{1}{n+n}$. \begin{CJK}{UTF8}{mj}证明\end{CJK}:
\end{enumerate}
$$
\lim _{n \rightarrow \infty} n\left(\int_{0}^{1} \frac{1}{1+x} \mathrm{~d} x-S_{n}\right)=\frac{1}{4} .
$$

\begin{enumerate}
  \setcounter{enumi}{4}
  \item ( 15 \begin{CJK}{UTF8}{mj}分\end{CJK}) \begin{CJK}{UTF8}{mj}要求通过线性变换\end{CJK}
\end{enumerate}
$$
\left\{\begin{array}{l}
\xi=x+\lambda y \\
\eta=x+\mu y
\end{array}\right.
$$
\begin{CJK}{UTF8}{mj}将方程\end{CJK}
$$
A \frac{\partial^{2} u}{\partial x^{2}}+2 B \frac{\partial^{2} u}{\partial x \partial y}+C \frac{\partial^{2} u}{\partial y^{2}}=0
$$
\begin{CJK}{UTF8}{mj}化简成\end{CJK} $\frac{\partial^{2} u}{\partial \xi \partial \eta}=0$, \begin{CJK}{UTF8}{mj}求\end{CJK} $\lambda, \mu$ \begin{CJK}{UTF8}{mj}的值\end{CJK}. (\begin{CJK}{UTF8}{mj}其中\end{CJK} $A, B, C$ \begin{CJK}{UTF8}{mj}为常数\end{CJK}, \begin{CJK}{UTF8}{mj}且\end{CJK} $A C-B^{2}<0$ ).

\begin{enumerate}
  \setcounter{enumi}{5}
  \item (20 \begin{CJK}{UTF8}{mj}分\end{CJK}) \begin{CJK}{UTF8}{mj}证明方程组\end{CJK}
\end{enumerate}
$$
\left\{\begin{array}{l}
u^{2}+v^{2}-x=0 \\
u+v-y=0
\end{array}\right.
$$
\begin{CJK}{UTF8}{mj}在点\end{CJK} $\left(x_{0}, y_{0}, u_{0}, v_{0}\right)=(25,7,3,4)$ \begin{CJK}{UTF8}{mj}附近具有唯一的一组连续可微解\end{CJK} $u=f(x, y), v=g(x, y)$, \begin{CJK}{UTF8}{mj}适合\end{CJK} $u_{0}=$ $f\left(x_{0}, y_{0}\right), v_{0}=g\left(x_{0}, y_{0}\right)$, \begin{CJK}{UTF8}{mj}并具体求出此组解\end{CJK}.

\begin{enumerate}
  \setcounter{enumi}{6}
  \item (20 \begin{CJK}{UTF8}{mj}分\end{CJK}) \begin{CJK}{UTF8}{mj}求函数\end{CJK}
\end{enumerate}
$$
f(x)=\ln \left|2 \cos \frac{x}{2}\right|, x \neq(2 m+1) \pi,(m=0, \pm 1, \pm 2, \cdots),
$$
\begin{CJK}{UTF8}{mj}的\end{CJK} Fourier \begin{CJK}{UTF8}{mj}级数展开式\end{CJK}. $\left(\frac{\sin \left(n+\frac{1}{2}\right) x}{2 \sin \frac{x}{2}}=\frac{1}{2}+\sum_{k=1}^{n} \cos k x ; \frac{\sin \left(n-\frac{1}{2}\right) x}{2 \sin \frac{x}{2}}=\frac{1}{2}+\sum_{k=1}^{n-1} \cos k x\right)$.

\begin{enumerate}
  \setcounter{enumi}{7}
  \item ( 17 \begin{CJK}{UTF8}{mj}分\end{CJK}) \begin{CJK}{UTF8}{mj}计算\end{CJK} $I(x)=\int_{0}^{+\infty} e^{-t^{2}} \cos 2 x t \mathrm{~d} t$.

  \item ( 15 \begin{CJK}{UTF8}{mj}分\end{CJK}) \begin{CJK}{UTF8}{mj}求幂级数\end{CJK} $S(x)=\sum_{n=0}^{\infty}(-1)^{n} \frac{x^{n}}{2 n+1}$ \begin{CJK}{UTF8}{mj}在其收敛域上的和\end{CJK}.

  \item (16 \begin{CJK}{UTF8}{mj}分\end{CJK}) \begin{CJK}{UTF8}{mj}曲面\end{CJK} $S: x^{2}+y^{2}+a z=4 a^{2}$ \begin{CJK}{UTF8}{mj}将区域\end{CJK} $\Omega: x^{2}+y^{2}+z^{2} \leqslant 4 a z$ \begin{CJK}{UTF8}{mj}分成两部分\end{CJK}, \begin{CJK}{UTF8}{mj}试求出这两部分的面\end{CJK} \begin{CJK}{UTF8}{mj}积比\end{CJK}. (\begin{CJK}{UTF8}{mj}其中\end{CJK} $a>0$ \begin{CJK}{UTF8}{mj}为常数\end{CJK}).

\end{enumerate}
\section{0. 湖南大学 2016 年研究生入学考试试题数学分析}
\begin{CJK}{UTF8}{mj}李扬\end{CJK}

\begin{CJK}{UTF8}{mj}微信公众号\end{CJK}: sxkyliyang

\begin{enumerate}
  \item (16 \begin{CJK}{UTF8}{mj}分\end{CJK}) \begin{CJK}{UTF8}{mj}设数列\end{CJK} $\left\{b_{n}\right\}$ \begin{CJK}{UTF8}{mj}有界\end{CJK}, \begin{CJK}{UTF8}{mj}令\end{CJK}
\end{enumerate}
$$
a_{n}=\frac{b_{1}}{1 \cdot 2}+\frac{b_{2}}{2 \cdot 3}+\cdots+\frac{b_{n}}{n \cdot(n+1)}, n \in \mathbb{N}^{+},
$$
\begin{CJK}{UTF8}{mj}证明\end{CJK}:\begin{CJK}{UTF8}{mj}数列\end{CJK} $\left\{a_{n}\right\}$ \begin{CJK}{UTF8}{mj}收敛\end{CJK}.

\begin{enumerate}
  \setcounter{enumi}{2}
  \item $(15$ \begin{CJK}{UTF8}{mj}分\end{CJK})\begin{CJK}{UTF8}{mj}设\end{CJK}
\end{enumerate}
$$
f(x)=\left\{\begin{array}{l}
|x|, x \neq 0 \\
1, x=0 .
\end{array}\right.
$$
\begin{CJK}{UTF8}{mj}证明\end{CJK}: \begin{CJK}{UTF8}{mj}不存在一个函数以函数以\end{CJK} $f(x)$ \begin{CJK}{UTF8}{mj}为其导函数\end{CJK}.

\begin{enumerate}
  \setcounter{enumi}{3}
  \item (16 \begin{CJK}{UTF8}{mj}分\end{CJK}) \begin{CJK}{UTF8}{mj}若函数\end{CJK} $f(x)$ \begin{CJK}{UTF8}{mj}在\end{CJK} $[a, b]$ \begin{CJK}{UTF8}{mj}和\end{CJK} $[b, c]$ \begin{CJK}{UTF8}{mj}上分别一致连续\end{CJK}, \begin{CJK}{UTF8}{mj}证明\end{CJK}: $f(x)$ \begin{CJK}{UTF8}{mj}在\end{CJK} $[a, c]$ \begin{CJK}{UTF8}{mj}上一致连续\end{CJK}.

  \item (15 \begin{CJK}{UTF8}{mj}分\end{CJK}) \begin{CJK}{UTF8}{mj}讨论反常积分\end{CJK}

\end{enumerate}
$$
\int_{0}^{+\infty} \frac{e^{\sin x} \sin 2 x}{x^{p}} \mathrm{~d} x, p>0
$$
\begin{CJK}{UTF8}{mj}的敛散性\end{CJK}.

\begin{enumerate}
  \setcounter{enumi}{5}
  \item ( 20 \begin{CJK}{UTF8}{mj}分\end{CJK}) \begin{CJK}{UTF8}{mj}设\end{CJK}
\end{enumerate}
$$
\left\{\begin{array}{l}
F(u, v, x, y)=u y+v x+x^{2} \\
G(u, v, x, y)=x+y+u v-1
\end{array}\right.
$$
\begin{CJK}{UTF8}{mj}以及点\end{CJK} $P_{0}\left(x_{0}, y_{0}, u_{0}, v_{0}\right)=(-1,0,2,1)$.

(1) \begin{CJK}{UTF8}{mj}证明\end{CJK}: \begin{CJK}{UTF8}{mj}在\end{CJK} $P_{0}$ \begin{CJK}{UTF8}{mj}某一领域附近\end{CJK} $F=0, G=0$ \begin{CJK}{UTF8}{mj}定义唯一的一组连续可微解\end{CJK} $x=f(u, v), y=g(u, v)$;

(2) \begin{CJK}{UTF8}{mj}求\end{CJK} Jacobi \begin{CJK}{UTF8}{mj}行列式\end{CJK} $\left.\frac{\partial(x, y)}{\partial(u, v)}\right|_{P_{0}}$.

\begin{enumerate}
  \setcounter{enumi}{6}
  \item ( 15 \begin{CJK}{UTF8}{mj}分\end{CJK}) \begin{CJK}{UTF8}{mj}设\end{CJK} $[0,2 \pi]$ \begin{CJK}{UTF8}{mj}上的连续函数列\end{CJK} $\left\{\varphi_{n}(x)\right\}$ \begin{CJK}{UTF8}{mj}满足\end{CJK}
\end{enumerate}
$$
\int_{0}^{2 \pi} \varphi_{n}(x) \varphi_{m}(x) \mathrm{d} x=\left\{\begin{array}{l}
0, n \neq m \\
1, n=m
\end{array}\right.
$$
\begin{CJK}{UTF8}{mj}对于\end{CJK} $[0,2 \pi]$ \begin{CJK}{UTF8}{mj}上的可积函数\end{CJK} $f(x)$, \begin{CJK}{UTF8}{mj}令\end{CJK} $\left.a_{n}=\int_{0}^{2 \pi} \varphi_{(} x\right) \varphi_{m}(x) \mathrm{d} x$. \begin{CJK}{UTF8}{mj}证明\end{CJK} $\sum_{n=1}^{\infty} a_{n}^{2}$ \begin{CJK}{UTF8}{mj}收敛且有不等式\end{CJK}
$$
\sum_{n=1}^{\infty} a_{n}^{2} \leqslant \int_{0}^{2 \pi} f(x) \varphi_{n}(x) \mathrm{d} x
$$

\begin{enumerate}
  \setcounter{enumi}{7}
  \item ( 17 \begin{CJK}{UTF8}{mj}分\end{CJK}) \begin{CJK}{UTF8}{mj}设函数\end{CJK} $f(x)$ \begin{CJK}{UTF8}{mj}在\end{CJK} $(-\infty,+\infty)$ \begin{CJK}{UTF8}{mj}上连续\end{CJK}, \begin{CJK}{UTF8}{mj}若\end{CJK} $\lim _{x \rightarrow+\infty} f(x)=+\infty, \lim _{x \rightarrow-\infty} f(x)=+\infty$, \begin{CJK}{UTF8}{mj}且\end{CJK} $f(x)$ \begin{CJK}{UTF8}{mj}在\end{CJK} $x=a$ \begin{CJK}{UTF8}{mj}处达到最小值\end{CJK}, \begin{CJK}{UTF8}{mj}若\end{CJK} $f(a)<a$, \begin{CJK}{UTF8}{mj}证明\end{CJK}: $F(x)=f(f(x))$ \begin{CJK}{UTF8}{mj}至少在两点达到最小值\end{CJK}.

  \item (15 \begin{CJK}{UTF8}{mj}分\end{CJK}) \begin{CJK}{UTF8}{mj}讨论函数\end{CJK}

\end{enumerate}
$$
f(x, y)= \begin{cases}\frac{x^{2}-y^{2}}{x^{2}+y^{2}}, & (x, y) \neq(0,0) \\ 0, & (x, y)=(0,0)\end{cases}
$$
\begin{CJK}{UTF8}{mj}在\end{CJK} $(0,0)$ \begin{CJK}{UTF8}{mj}处的连续性和可微性\end{CJK}. 9. (16 \begin{CJK}{UTF8}{mj}分\end{CJK}) \begin{CJK}{UTF8}{mj}求三重积分\end{CJK}
$$
I=\iiint_{\Omega} y^{2} \mathrm{~d} x \mathrm{~d} y \mathrm{~d} z
$$
\begin{CJK}{UTF8}{mj}其中\end{CJK} $\Omega:\left(x-\frac{1}{2}\right)^{2}+\left(y-\frac{1}{2}\right)^{2}+\left(z-\frac{1}{2}\right)^{2} \leqslant \frac{3}{4}$.

\section{1. 湖南大学 2017 年研究生入学考试试题数学分析}
\begin{CJK}{UTF8}{mj}李扬\end{CJK}

\begin{CJK}{UTF8}{mj}微信公众号\end{CJK}: sxkyliyang

\begin{enumerate}
  \item \begin{CJK}{UTF8}{mj}计算题\end{CJK}(\begin{CJK}{UTF8}{mj}每小题\end{CJK} 7 \begin{CJK}{UTF8}{mj}分\end{CJK}, \begin{CJK}{UTF8}{mj}共\end{CJK} 28 \begin{CJK}{UTF8}{mj}分\end{CJK})
\end{enumerate}
(1) \begin{CJK}{UTF8}{mj}求\end{CJK} $\lim _{n \rightarrow \infty} \cos \frac{x}{2} \cos \frac{x}{2^{2}} \cdots \cos \frac{x}{2^{n}}(x \neq 0)$;

(2) \begin{CJK}{UTF8}{mj}计算\end{CJK} $\int x \arcsin x \mathrm{~d} x$;

(3) \begin{CJK}{UTF8}{mj}已知\end{CJK} $\int_{1}^{2} f(x) \mathrm{d} x=4$, \begin{CJK}{UTF8}{mj}求\end{CJK} $\int_{0}^{2} x f\left(2 x^{2}+1\right) \mathrm{d} x$;

(4) \begin{CJK}{UTF8}{mj}设\end{CJK} $F(x)$ \begin{CJK}{UTF8}{mj}是\end{CJK} $\cos x$ \begin{CJK}{UTF8}{mj}的一个原函数\end{CJK}, \begin{CJK}{UTF8}{mj}求\end{CJK} $\frac{\mathrm{d} F(\sin x)}{\mathrm{d} x}$.

\begin{enumerate}
  \setcounter{enumi}{2}
  \item ( 12 \begin{CJK}{UTF8}{mj}分\end{CJK}) \begin{CJK}{UTF8}{mj}设\end{CJK} $f(x)$ \begin{CJK}{UTF8}{mj}在\end{CJK} $[0,1]$ \begin{CJK}{UTF8}{mj}上有一阶连续导数\end{CJK}, \begin{CJK}{UTF8}{mj}且\end{CJK} $f(0)=f(1)=0$, \begin{CJK}{UTF8}{mj}记\end{CJK} $M=\max _{0 \leqslant x \leqslant 1}\left|f^{\prime}(x)\right|$, \begin{CJK}{UTF8}{mj}求证\end{CJK}
\end{enumerate}
$$
\left|\int_{0}^{1} f(x) \mathrm{d} x\right| \leqslant \frac{M}{4}
$$

\begin{enumerate}
  \setcounter{enumi}{3}
  \item (12 \begin{CJK}{UTF8}{mj}分\end{CJK}) \begin{CJK}{UTF8}{mj}讨论函数\end{CJK}
\end{enumerate}
$$
f(x, y)= \begin{cases}\frac{x y^{3}}{x^{2}+y^{4}}, & x^{2}+y^{2} \neq 0 \\ 0, & x^{2}+y^{2}=0 .\end{cases}
$$
\begin{CJK}{UTF8}{mj}在原点\end{CJK} $(0,0)$ \begin{CJK}{UTF8}{mj}处的连续性\end{CJK}, \begin{CJK}{UTF8}{mj}可偏导性和可微性\end{CJK}.

\begin{enumerate}
  \setcounter{enumi}{4}
  \item (15 \begin{CJK}{UTF8}{mj}分\end{CJK})
\end{enumerate}
(1)\begin{CJK}{UTF8}{mj}叙述函数\end{CJK} $f(x)$ \begin{CJK}{UTF8}{mj}在区间\end{CJK} $I$ \begin{CJK}{UTF8}{mj}上的一致连续的定义\end{CJK};

(2) \begin{CJK}{UTF8}{mj}设\end{CJK} $f(x), g(x)$ \begin{CJK}{UTF8}{mj}是\end{CJK} $I$ \begin{CJK}{UTF8}{mj}上的有界一致连续函数\end{CJK}, \begin{CJK}{UTF8}{mj}证明\end{CJK} $f(x) g(x)$ \begin{CJK}{UTF8}{mj}在\end{CJK} $I$ \begin{CJK}{UTF8}{mj}上一致连续\end{CJK}.

\begin{enumerate}
  \setcounter{enumi}{5}
  \item ( 12 \begin{CJK}{UTF8}{mj}分\end{CJK}) \begin{CJK}{UTF8}{mj}计算曲面积分\end{CJK} $\iint_{S} x y z \mathrm{~d} x \mathrm{~d} y$, \begin{CJK}{UTF8}{mj}其中\end{CJK} $S$ \begin{CJK}{UTF8}{mj}为\end{CJK} $x^{2}+y^{2}+z^{2}=1$ \begin{CJK}{UTF8}{mj}外侧在\end{CJK} $x \geqslant 0, y \geqslant 0$ \begin{CJK}{UTF8}{mj}的部分\end{CJK}

  \item (12 \begin{CJK}{UTF8}{mj}分\end{CJK}) \begin{CJK}{UTF8}{mj}设含参变量函数\end{CJK} $f(x)=\int_{x}^{x^{2}} \frac{\sin u x}{u} \mathrm{~d} u,(x>0)$, \begin{CJK}{UTF8}{mj}求\end{CJK} $f^{\prime}(x)$.

  \item (14 \begin{CJK}{UTF8}{mj}分\end{CJK}) \begin{CJK}{UTF8}{mj}设\end{CJK} $f(x)=\sum_{n=1}^{\infty}(-1)^{n+1} \frac{e^{-n x}}{n}$, \begin{CJK}{UTF8}{mj}求\end{CJK}:

\end{enumerate}
(1) $f(x)$ \begin{CJK}{UTF8}{mj}连续的取值范围\end{CJK};

(2) $f(x)$ \begin{CJK}{UTF8}{mj}可导的取值范围\end{CJK}.

\begin{enumerate}
  \setcounter{enumi}{8}
  \item ( 12 \begin{CJK}{UTF8}{mj}分\end{CJK}) \begin{CJK}{UTF8}{mj}设\end{CJK} $f(x)$ \begin{CJK}{UTF8}{mj}设函数\end{CJK} $u(x, y)$ \begin{CJK}{UTF8}{mj}在由封闭的光滑曲线\end{CJK} $L$ \begin{CJK}{UTF8}{mj}所围成的区域上具有二阶连续偏导数\end{CJK}, \begin{CJK}{UTF8}{mj}证明\end{CJK}
\end{enumerate}
$$
\iint_{D}\left(\frac{\partial^{2} u}{\partial x^{2}}+\frac{\partial^{2} u}{\partial y^{2}}\right) \mathrm{d} \sigma=\oint_{L} \frac{\partial u}{\partial n} \mathrm{~d} s
$$
\begin{CJK}{UTF8}{mj}其中\end{CJK} $\frac{\partial u}{\partial n}$ \begin{CJK}{UTF8}{mj}是\end{CJK} $u(x, y)$ \begin{CJK}{UTF8}{mj}沿\end{CJK} $L$ \begin{CJK}{UTF8}{mj}外法线方向\end{CJK} $n$ \begin{CJK}{UTF8}{mj}的方向导数\end{CJK}.

\begin{enumerate}
  \setcounter{enumi}{9}
  \item (15 \begin{CJK}{UTF8}{mj}分\end{CJK}) \begin{CJK}{UTF8}{mj}设\end{CJK} $f(x, y)$ \begin{CJK}{UTF8}{mj}具有二阶连续偏导数\end{CJK}, $g(x, y)=f\left(e^{x y}, x^{2}+y^{2}\right)$, \begin{CJK}{UTF8}{mj}且\end{CJK}
\end{enumerate}
$$
f(x, y)=1-x-y+o\left(\sqrt{(x-1)^{2}+y^{2}}\right)
$$
\begin{CJK}{UTF8}{mj}证明\end{CJK}: $g(x, y)$ \begin{CJK}{UTF8}{mj}在\end{CJK} $(0,0)$ \begin{CJK}{UTF8}{mj}取得极值\end{CJK}, \begin{CJK}{UTF8}{mj}判断此极值是极大值还是极小值\end{CJK}? \begin{CJK}{UTF8}{mj}并求出此极值\end{CJK}. 10. ( 18 \begin{CJK}{UTF8}{mj}分\end{CJK}) \begin{CJK}{UTF8}{mj}设函数\end{CJK} $g(x)$ \begin{CJK}{UTF8}{mj}非负连续\end{CJK}, \begin{CJK}{UTF8}{mj}且存在正数\end{CJK} $a$ \begin{CJK}{UTF8}{mj}使得\end{CJK}
$$
g(x)+a \int_{x-1}^{x} g(t) \mathrm{d} t=A
$$
(1) \begin{CJK}{UTF8}{mj}求证\end{CJK}: $g(x)$ \begin{CJK}{UTF8}{mj}可导\end{CJK};

(2) \begin{CJK}{UTF8}{mj}令\end{CJK} $F(x)=e^{a x} g(x)$, \begin{CJK}{UTF8}{mj}证明\end{CJK} $F(x)$ \begin{CJK}{UTF8}{mj}为增函数\end{CJK};

(3) \begin{CJK}{UTF8}{mj}求证\end{CJK}: $g(x)>A$;

(4) \begin{CJK}{UTF8}{mj}给出\end{CJK} $g(x)$ \begin{CJK}{UTF8}{mj}恒为常数的一个充分条件\end{CJK}.

\section{2. 湖南大学 2007 年研究生入学考试试题高等代数 
 李扬 
 微信公众号: sxkyliyang}
\begin{enumerate}
  \item (20 \begin{CJK}{UTF8}{mj}分\end{CJK}) \begin{CJK}{UTF8}{mj}已知多项式\end{CJK} $f(x)=f_{1}(x) f_{2}(x), g(x)=m(x) d(x)$, \begin{CJK}{UTF8}{mj}其中符号\end{CJK} $\partial()$ \begin{CJK}{UTF8}{mj}表示该多项式的次数\end{CJK}, \begin{CJK}{UTF8}{mj}证明\end{CJK}: \begin{CJK}{UTF8}{mj}存\end{CJK} \begin{CJK}{UTF8}{mj}在多项式\end{CJK} $u_{1}(x)$ \begin{CJK}{UTF8}{mj}和\end{CJK} $u_{2}(x)$, \begin{CJK}{UTF8}{mj}使得\end{CJK}
\end{enumerate}
$$
g(x)=u_{1}(x) f_{2}(x)+u_{2}(x) f_{1}(x)
$$
\begin{CJK}{UTF8}{mj}且\end{CJK} $\partial\left(u_{k}(x)\right)<\partial\left(f_{k}(x)\right), k=1,2$.

\begin{enumerate}
  \setcounter{enumi}{2}
  \item ( 20 \begin{CJK}{UTF8}{mj}分\end{CJK}) \begin{CJK}{UTF8}{mj}设\end{CJK} $A=\left(a_{i j}\right)$ \begin{CJK}{UTF8}{mj}是数域\end{CJK} $K$ \begin{CJK}{UTF8}{mj}上的一个\end{CJK} $n$ \begin{CJK}{UTF8}{mj}阶矩阵\end{CJK}, \begin{CJK}{UTF8}{mj}且\end{CJK} $a_{i j}=a_{i}-b_{j}$.
\end{enumerate}
(1) \begin{CJK}{UTF8}{mj}求\end{CJK} $|A|$;

(2) \begin{CJK}{UTF8}{mj}当\end{CJK} $n \geqslant 2$, \begin{CJK}{UTF8}{mj}且\end{CJK} $a_{1} \neq a_{2}, b_{1} \neq b_{2}$ \begin{CJK}{UTF8}{mj}时\end{CJK}, \begin{CJK}{UTF8}{mj}求齐次线性方程组\end{CJK} $A X=0$ \begin{CJK}{UTF8}{mj}的解空间的维数和一组基\end{CJK}.

\begin{enumerate}
  \setcounter{enumi}{3}
  \item ( 20 \begin{CJK}{UTF8}{mj}分\end{CJK}) \begin{CJK}{UTF8}{mj}试求解线性方程组\end{CJK}
\end{enumerate}
$$
\left\{\begin{array}{l}
x_{1}+x_{2}+\cdots+x_{n}=1 \\
x_{2}+x_{3}+\cdots+x_{n+1}=2 \\
\cdots \\
x_{n+1}+x_{n+2}+\cdots+x_{2 n}=n+1
\end{array}\right.
$$

\begin{enumerate}
  \setcounter{enumi}{4}
  \item ( 20 \begin{CJK}{UTF8}{mj}分\end{CJK}) \begin{CJK}{UTF8}{mj}设\end{CJK} $A$ \begin{CJK}{UTF8}{mj}为\end{CJK} $n$ \begin{CJK}{UTF8}{mj}阶实对称矩阵\end{CJK}, $\beta$ \begin{CJK}{UTF8}{mj}为常向量\end{CJK}, \begin{CJK}{UTF8}{mj}函数\end{CJK}
\end{enumerate}
$$
f(x)=x A x^{T}+2 \beta x^{T},
$$
\begin{CJK}{UTF8}{mj}试分别在下列条件下\end{CJK}, \begin{CJK}{UTF8}{mj}讨论\end{CJK} $f(x)$ \begin{CJK}{UTF8}{mj}的最大值和最小值\end{CJK}.\\
(1) $A$ \begin{CJK}{UTF8}{mj}是正定矩阵\end{CJK};\\
(2) $A$ \begin{CJK}{UTF8}{mj}是负定矩阵\end{CJK};\\
(3) $A$ \begin{CJK}{UTF8}{mj}既不是正定矩阵\end{CJK}, \begin{CJK}{UTF8}{mj}也不是负定矩阵\end{CJK}.

\begin{enumerate}
  \setcounter{enumi}{5}
  \item ( 20 \begin{CJK}{UTF8}{mj}分\end{CJK}) \begin{CJK}{UTF8}{mj}设\end{CJK} $\eta$ \begin{CJK}{UTF8}{mj}是\end{CJK} $n$ \begin{CJK}{UTF8}{mj}维欧氏空间\end{CJK} $V$ \begin{CJK}{UTF8}{mj}中的一个单位向量\end{CJK}, \begin{CJK}{UTF8}{mj}定义线性变换\end{CJK}
\end{enumerate}
$$
A \alpha=2(\eta, \alpha) \eta-\alpha, \forall \alpha \in V
$$
\begin{CJK}{UTF8}{mj}证明\end{CJK}:

(1) $A$ \begin{CJK}{UTF8}{mj}是正交变换\end{CJK};

(2) \begin{CJK}{UTF8}{mj}若\end{CJK} $n$ \begin{CJK}{UTF8}{mj}为奇数\end{CJK}, \begin{CJK}{UTF8}{mj}则\end{CJK} $A$ \begin{CJK}{UTF8}{mj}是第一类的\end{CJK}; \begin{CJK}{UTF8}{mj}若\end{CJK} $A$ \begin{CJK}{UTF8}{mj}为偶数\end{CJK}, \begin{CJK}{UTF8}{mj}则\end{CJK} $A$ \begin{CJK}{UTF8}{mj}是第二类的\end{CJK};

(3) $W=\{\beta \in V \mid A \beta=\beta\}$ \begin{CJK}{UTF8}{mj}是\end{CJK} $V$ \begin{CJK}{UTF8}{mj}的一维子空间\end{CJK}.

\begin{enumerate}
  \setcounter{enumi}{6}
  \item (20 \begin{CJK}{UTF8}{mj}分\end{CJK}) \begin{CJK}{UTF8}{mj}设\end{CJK} $A, B$ \begin{CJK}{UTF8}{mj}为同价方阵\end{CJK},
\end{enumerate}
(1) \begin{CJK}{UTF8}{mj}若\end{CJK} $A, B$ \begin{CJK}{UTF8}{mj}相似\end{CJK}, \begin{CJK}{UTF8}{mj}证明\end{CJK} $A, B$ \begin{CJK}{UTF8}{mj}的特征多项式相同\end{CJK};

(2) \begin{CJK}{UTF8}{mj}举例说明\end{CJK} (1) \begin{CJK}{UTF8}{mj}的逆命题不成立\end{CJK};

(3) \begin{CJK}{UTF8}{mj}证明当\end{CJK} $A$ \begin{CJK}{UTF8}{mj}和\end{CJK} $B$ \begin{CJK}{UTF8}{mj}均为实对称矩阵时\end{CJK}, (1) \begin{CJK}{UTF8}{mj}的逆命题成立\end{CJK}.

\begin{enumerate}
  \setcounter{enumi}{7}
  \item ( 20 \begin{CJK}{UTF8}{mj}分\end{CJK}) \begin{CJK}{UTF8}{mj}设\end{CJK} $A$ \begin{CJK}{UTF8}{mj}是实数域\end{CJK} $R$ \begin{CJK}{UTF8}{mj}上\end{CJK} $n$ \begin{CJK}{UTF8}{mj}维线性空间\end{CJK} $V$ \begin{CJK}{UTF8}{mj}的一个线性变换\end{CJK}, $\eta_{1}, \eta_{2}, \cdots, \eta_{r}$ \begin{CJK}{UTF8}{mj}是值域\end{CJK} $A(V)$ \begin{CJK}{UTF8}{mj}的一组基\end{CJK}.
\end{enumerate}
(1) \begin{CJK}{UTF8}{mj}若\end{CJK} $A\left(\varepsilon_{i}\right)=\eta_{i}, i=2, \cdots, r$, \begin{CJK}{UTF8}{mj}证明\end{CJK} $V$ \begin{CJK}{UTF8}{mj}是\end{CJK} $L\left(\varepsilon_{1}, \varepsilon_{2}, \cdots, \varepsilon_{r}\right)$ \begin{CJK}{UTF8}{mj}与\end{CJK} $A^{-1}(0)$ \begin{CJK}{UTF8}{mj}的直和\end{CJK}, \begin{CJK}{UTF8}{mj}其中\end{CJK} $L\left(\varepsilon_{1}, \varepsilon_{2}, \cdots, \varepsilon_{r}\right)$ \begin{CJK}{UTF8}{mj}是\end{CJK} $V$ \begin{CJK}{UTF8}{mj}中由\end{CJK} $\varepsilon_{1}, \varepsilon_{2}, \cdots, \varepsilon_{r}$ \begin{CJK}{UTF8}{mj}生成的线性子空间\end{CJK}, $A^{-1}(0)$ \begin{CJK}{UTF8}{mj}为\end{CJK} $A$ \begin{CJK}{UTF8}{mj}的核\end{CJK}; (2) \begin{CJK}{UTF8}{mj}若\end{CJK} $A^{2}=0$, \begin{CJK}{UTF8}{mj}且\end{CJK} $A$ \begin{CJK}{UTF8}{mj}的秩为\end{CJK} $r$, \begin{CJK}{UTF8}{mj}证明可以找到\end{CJK} $V$ \begin{CJK}{UTF8}{mj}的一组基\end{CJK}, \begin{CJK}{UTF8}{mj}使得\end{CJK} $A$ \begin{CJK}{UTF8}{mj}在这组基下的矩阵为\end{CJK} $A=\left(\begin{array}{cc}0 & N \\ 0 & 0\end{array}\right)$, \begin{CJK}{UTF8}{mj}其中\end{CJK} $N$ \begin{CJK}{UTF8}{mj}是一个\end{CJK} $(n-r) \times r$ \begin{CJK}{UTF8}{mj}的矩阵\end{CJK}, \begin{CJK}{UTF8}{mj}且其秩等于\end{CJK} $r$.

\begin{enumerate}
  \setcounter{enumi}{8}
  \item ( 10 \begin{CJK}{UTF8}{mj}分\end{CJK}) \begin{CJK}{UTF8}{mj}设实数域\end{CJK} $R$ \begin{CJK}{UTF8}{mj}上的\end{CJK} $n$ \begin{CJK}{UTF8}{mj}维线性空间\end{CJK} $V$ \begin{CJK}{UTF8}{mj}的线性变换\end{CJK} $A$ \begin{CJK}{UTF8}{mj}有\end{CJK} $n$ \begin{CJK}{UTF8}{mj}个互异的特征值\end{CJK}, \begin{CJK}{UTF8}{mj}证明线性变换\end{CJK} $B$ \begin{CJK}{UTF8}{mj}与\end{CJK} $A$ \begin{CJK}{UTF8}{mj}可交换的充分必要条件是存在不全为零的常数\end{CJK} $k_{0}, k_{1}, k_{2}, \cdots, k_{n-1}$, \begin{CJK}{UTF8}{mj}使得\end{CJK}
\end{enumerate}
$$
B=k_{0} E+k_{1} A+k_{2} A^{2}+\cdots+k_{n-1} A^{n-1}
$$
\begin{CJK}{UTF8}{mj}其中\end{CJK} $E$ \begin{CJK}{UTF8}{mj}是恒等变换\end{CJK}.

\section{3. 湖南大学 2008 年研究生入学考试试题高等代数 
 李扬 
 微信公众号: sxkyliyang}
\begin{enumerate}
  \item (20 \begin{CJK}{UTF8}{mj}分\end{CJK}) \begin{CJK}{UTF8}{mj}设\end{CJK} $f(x)$ \begin{CJK}{UTF8}{mj}是一个不可约多项式\end{CJK}, $a$ \begin{CJK}{UTF8}{mj}和\end{CJK} $\frac{1}{a}$ \begin{CJK}{UTF8}{mj}都是\end{CJK} $f(x)$ \begin{CJK}{UTF8}{mj}的根\end{CJK}, \begin{CJK}{UTF8}{mj}又若\end{CJK} $b$ \begin{CJK}{UTF8}{mj}也是\end{CJK} $f(x)$ \begin{CJK}{UTF8}{mj}的根\end{CJK}, \begin{CJK}{UTF8}{mj}试证\end{CJK} $\frac{1}{b}$ \begin{CJK}{UTF8}{mj}也是\end{CJK} $f(x)$ \begin{CJK}{UTF8}{mj}的根\end{CJK}.

  \item ( 20 \begin{CJK}{UTF8}{mj}分\end{CJK}) \begin{CJK}{UTF8}{mj}设\end{CJK} $A=\left(a_{i j}\right)_{n \times n}, b=\left(b_{1}, b_{2}, \cdots, b_{n}\right)^{T}$, \begin{CJK}{UTF8}{mj}试证\end{CJK}:

\end{enumerate}
(1) \begin{CJK}{UTF8}{mj}若方程组\end{CJK} $A y=b$ \begin{CJK}{UTF8}{mj}有解\end{CJK}, \begin{CJK}{UTF8}{mj}则方程组\end{CJK} $A^{T} X=0$ \begin{CJK}{UTF8}{mj}的任一解\end{CJK} $x_{1}, x_{2}, \cdots, x_{n}$, \begin{CJK}{UTF8}{mj}必满足方程\end{CJK} $b_{1} x_{1}+{ }_{2} x_{2}+\cdots+$ $b_{n} x_{n}=0$

(2) \begin{CJK}{UTF8}{mj}方程组\end{CJK} $A y=b$ \begin{CJK}{UTF8}{mj}有解的充分必要条件是\end{CJK} $\left[\begin{array}{c}A^{T} \\ b^{T}\end{array}\right] x=\left[\begin{array}{l}0 \\ 1\end{array}\right]$ \begin{CJK}{UTF8}{mj}无解\end{CJK}.

\begin{enumerate}
  \setcounter{enumi}{3}
  \item ( 15 \begin{CJK}{UTF8}{mj}分\end{CJK}) \begin{CJK}{UTF8}{mj}设\end{CJK} 5 \begin{CJK}{UTF8}{mj}阶行列式\end{CJK}
\end{enumerate}
$$
D_{5}=\left|\begin{array}{lllll}
1 & 2 & 3 & 4 & 5 \\
2 & 2 & 2 & 1 & 1 \\
3 & 1 & 2 & 4 & 5 \\
1 & 1 & 1 & 2 & 2 \\
4 & 3 & 1 & 5 & 0
\end{array}\right|=27
$$
\begin{CJK}{UTF8}{mj}试计算\end{CJK} $A_{41}+A_{42}+A_{43}+A_{44}+A_{45}$ \begin{CJK}{UTF8}{mj}及\end{CJK} $A_{41}$, \begin{CJK}{UTF8}{mj}其中\end{CJK} $A_{4 j}, j=1,2, \cdots, 5$ \begin{CJK}{UTF8}{mj}是\end{CJK} $D_{5}$ \begin{CJK}{UTF8}{mj}中第\end{CJK} 4 \begin{CJK}{UTF8}{mj}行各元素的代数余\end{CJK} \begin{CJK}{UTF8}{mj}子式\end{CJK}.

\begin{enumerate}
  \setcounter{enumi}{4}
  \item ( 15 \begin{CJK}{UTF8}{mj}分\end{CJK}) \begin{CJK}{UTF8}{mj}已知三元二次型\end{CJK} $X^{T} A X$ \begin{CJK}{UTF8}{mj}经正交变换化为\end{CJK} $2 y_{1}^{2}-y_{2}^{2}-y_{3}^{2}$, \begin{CJK}{UTF8}{mj}又知\end{CJK} $A^{*} \alpha=\alpha$, \begin{CJK}{UTF8}{mj}其中\end{CJK} $\alpha=(1,1,-1)^{T}$, $A^{*}$ \begin{CJK}{UTF8}{mj}为\end{CJK} $A$ \begin{CJK}{UTF8}{mj}的伴随矩阵\end{CJK}, \begin{CJK}{UTF8}{mj}求次此二次型的表达式\end{CJK}.

  \item ( 20 \begin{CJK}{UTF8}{mj}分\end{CJK}) \begin{CJK}{UTF8}{mj}设\end{CJK} $A$ \begin{CJK}{UTF8}{mj}为\end{CJK} $n$ \begin{CJK}{UTF8}{mj}阶方阵\end{CJK}, \begin{CJK}{UTF8}{mj}而\end{CJK}

\end{enumerate}
$$
W_{1}=\left\{x \in R^{n} \mid A x=0\right\}, W_{2}=\left\{x \in R^{n} \mid(A-E) x=0\right\}
$$
\begin{CJK}{UTF8}{mj}证明\end{CJK}: $A$ \begin{CJK}{UTF8}{mj}为幂等矩阵当且仅当\end{CJK} $R^{n}=W_{1} \oplus W_{2}$.

\begin{enumerate}
  \setcounter{enumi}{6}
  \item ( 20 \begin{CJK}{UTF8}{mj}分\end{CJK}) \begin{CJK}{UTF8}{mj}设\end{CJK} $A$ \begin{CJK}{UTF8}{mj}是数域\end{CJK} $P$ \begin{CJK}{UTF8}{mj}上\end{CJK} $n$ \begin{CJK}{UTF8}{mj}维线性空间\end{CJK} $V$ \begin{CJK}{UTF8}{mj}的一个线性变换\end{CJK}, $\alpha_{1}, \alpha_{2}, \cdots, \alpha_{n}$ \begin{CJK}{UTF8}{mj}为\end{CJK} $V$ \begin{CJK}{UTF8}{mj}中\end{CJK} $n$ \begin{CJK}{UTF8}{mj}个非零元素\end{CJK}, $\lambda \in P$. \begin{CJK}{UTF8}{mj}若\end{CJK}
\end{enumerate}
$$
(A-\lambda E) \alpha_{1}=0,(A-\lambda E) \alpha_{i+1}=\alpha_{i}, i=1,2, \cdots, n-1
$$
\begin{CJK}{UTF8}{mj}其中\end{CJK} $E$ \begin{CJK}{UTF8}{mj}为\end{CJK} $V$ \begin{CJK}{UTF8}{mj}上的线性变换\end{CJK}.

(1) \begin{CJK}{UTF8}{mj}证明\end{CJK} $\alpha_{1}, \alpha_{2}, \cdots, \alpha_{n}$ \begin{CJK}{UTF8}{mj}为\end{CJK} $V$ \begin{CJK}{UTF8}{mj}的一组基\end{CJK};

(2) \begin{CJK}{UTF8}{mj}求\end{CJK} $A$ \begin{CJK}{UTF8}{mj}在基\end{CJK} $\alpha_{1}, \alpha_{2}, \cdots, \alpha_{n}$ \begin{CJK}{UTF8}{mj}下的矩阵\end{CJK}.

\begin{enumerate}
  \setcounter{enumi}{7}
  \item ( 20 \begin{CJK}{UTF8}{mj}分\end{CJK}) \begin{CJK}{UTF8}{mj}欧氏空间\end{CJK} $V$ \begin{CJK}{UTF8}{mj}中的线性变换\end{CJK} $A$ \begin{CJK}{UTF8}{mj}称为反对称的\end{CJK}, \begin{CJK}{UTF8}{mj}若对\end{CJK} $V$ \begin{CJK}{UTF8}{mj}中任意向量\end{CJK} $\alpha, \beta$ \begin{CJK}{UTF8}{mj}都有\end{CJK} $(A \alpha, \beta)=-(\alpha, A \beta)$. \begin{CJK}{UTF8}{mj}试证明\end{CJK}:
\end{enumerate}
(1) \begin{CJK}{UTF8}{mj}对有限维欧氏空间\end{CJK} $V$ \begin{CJK}{UTF8}{mj}来说\end{CJK}, \begin{CJK}{UTF8}{mj}线性变换\end{CJK} $A$ \begin{CJK}{UTF8}{mj}为反对称的充分且必要的条件是\end{CJK}, $A$ \begin{CJK}{UTF8}{mj}在标准正交基下的矩\end{CJK} \begin{CJK}{UTF8}{mj}阵为反对称矩阵\end{CJK};

(2) \begin{CJK}{UTF8}{mj}如果\end{CJK} $V_{1} \subset V$ \begin{CJK}{UTF8}{mj}是反对称变换的不变子空间\end{CJK}, \begin{CJK}{UTF8}{mj}则\end{CJK} $V_{1}^{\perp}$ \begin{CJK}{UTF8}{mj}也是\end{CJK}. 8. ( 10 \begin{CJK}{UTF8}{mj}分\end{CJK}) \begin{CJK}{UTF8}{mj}设\end{CJK} $A$ \begin{CJK}{UTF8}{mj}为\end{CJK} $n$ \begin{CJK}{UTF8}{mj}阶幂零矩阵\end{CJK}, \begin{CJK}{UTF8}{mj}即存在正整数\end{CJK} $k$ \begin{CJK}{UTF8}{mj}使\end{CJK} $A^{k}=0$.\\
(1) \begin{CJK}{UTF8}{mj}求\end{CJK} $A$ \begin{CJK}{UTF8}{mj}的全部特征值\end{CJK};\\
(2) \begin{CJK}{UTF8}{mj}若\end{CJK} $A$ \begin{CJK}{UTF8}{mj}的秩为\end{CJK} $r$, \begin{CJK}{UTF8}{mj}证明\end{CJK} $A^{r+1}$;\\
(3) \begin{CJK}{UTF8}{mj}求行列式\end{CJK} $\operatorname{det}(E+A)$, \begin{CJK}{UTF8}{mj}其中\end{CJK} $E_{n}$ \begin{CJK}{UTF8}{mj}为\end{CJK} $n$ \begin{CJK}{UTF8}{mj}阶单位阵\end{CJK}.

\section{4. 湖南大学 2009 年研究生入学考试试题高等代数 
 李扬 
 微信公众号: sxkyliyang}
\begin{enumerate}
  \item (15 \begin{CJK}{UTF8}{mj}分\end{CJK}) \begin{CJK}{UTF8}{mj}计算\end{CJK} $n$ \begin{CJK}{UTF8}{mj}阶行列式\end{CJK}
\end{enumerate}
$$
D_{n}=\left|\begin{array}{ccccc}
a_{1}+b_{1} & b_{1} & b_{1} & \cdots & b_{1} \\
b_{2} & a_{2}+b_{2} & b_{2} & \cdots & b_{2} \\
\vdots & \vdots & \vdots & \cdots & \vdots \\
b_{n} & b_{n} & b_{n} & \cdots & b_{n}
\end{array}\right| \text {, 其中 } a_{i} \neq 0,(i=1,2, \cdots, n) .
$$

\begin{enumerate}
  \setcounter{enumi}{2}
  \item (15 \begin{CJK}{UTF8}{mj}分\end{CJK}) \begin{CJK}{UTF8}{mj}设\end{CJK} $d(x)$ \begin{CJK}{UTF8}{mj}和\end{CJK} $f(x)$ \begin{CJK}{UTF8}{mj}与\end{CJK} $g(x)$ \begin{CJK}{UTF8}{mj}的公因式\end{CJK}, \begin{CJK}{UTF8}{mj}证明\end{CJK}:
\end{enumerate}
(1) $d(x)$ \begin{CJK}{UTF8}{mj}是\end{CJK} $f(x)$ \begin{CJK}{UTF8}{mj}与\end{CJK} $g(x)$ \begin{CJK}{UTF8}{mj}的一个最大公因式的充要条件是\end{CJK}
$$
d(x)=u(x) f(x)+v(x) g(x)
$$
(2) \begin{CJK}{UTF8}{mj}若\end{CJK} $h(x)$ \begin{CJK}{UTF8}{mj}是任一首项为\end{CJK} 1 \begin{CJK}{UTF8}{mj}的多项式\end{CJK}, \begin{CJK}{UTF8}{mj}则\end{CJK}
$$
(f(x) h(x), g(x) h(x))=(f(x), g(x)) h(x)
$$

\begin{enumerate}
  \setcounter{enumi}{3}
  \item ( 20 \begin{CJK}{UTF8}{mj}分\end{CJK}) \begin{CJK}{UTF8}{mj}设\end{CJK} $A, B$ \begin{CJK}{UTF8}{mj}均为\end{CJK} $n$ \begin{CJK}{UTF8}{mj}阶方阵\end{CJK}, \begin{CJK}{UTF8}{mj}证明\end{CJK}:
\end{enumerate}
(1) \begin{CJK}{UTF8}{mj}矩阵\end{CJK} $A B$ \begin{CJK}{UTF8}{mj}的秩等于矩阵\end{CJK} $B$ \begin{CJK}{UTF8}{mj}的秩的充要条件是方程组\end{CJK} $A B x=0$ \begin{CJK}{UTF8}{mj}和\end{CJK} $B x=0$ \begin{CJK}{UTF8}{mj}同解\end{CJK};

(2) $r\left(A^{n}\right)=r\left(A^{n+1}\right)$.

\begin{enumerate}
  \setcounter{enumi}{4}
  \item (15 \begin{CJK}{UTF8}{mj}分\end{CJK}) \begin{CJK}{UTF8}{mj}设\end{CJK} $\alpha_{1}, \alpha_{2}, \alpha_{3}$ \begin{CJK}{UTF8}{mj}是一组三维向量\end{CJK}, \begin{CJK}{UTF8}{mj}证明\end{CJK}: $\alpha_{1}, \alpha_{2}, \alpha_{3}$ \begin{CJK}{UTF8}{mj}线性无关的充要条件是任意一个三维向量都能\end{CJK} \begin{CJK}{UTF8}{mj}被它们线性表出\end{CJK}, \begin{CJK}{UTF8}{mj}并作出几何解释\end{CJK}.

  \item ( 15 \begin{CJK}{UTF8}{mj}分\end{CJK}) \begin{CJK}{UTF8}{mj}设\end{CJK} $x_{1}^{2}+x_{2}^{2}+\cdots+x_{n}^{2}=1$, \begin{CJK}{UTF8}{mj}证明二次型\end{CJK}

\end{enumerate}
$$
f\left(x_{1}, x_{2}, \cdots, x_{n}\right)=x^{\prime} A x
$$
\begin{CJK}{UTF8}{mj}的最小值为矩阵\end{CJK} $A$ \begin{CJK}{UTF8}{mj}的最小特征值\end{CJK}.

\begin{enumerate}
  \setcounter{enumi}{6}
  \item ( 20 \begin{CJK}{UTF8}{mj}分\end{CJK}) \begin{CJK}{UTF8}{mj}设\end{CJK} $V$ \begin{CJK}{UTF8}{mj}是数域\end{CJK} $P$ \begin{CJK}{UTF8}{mj}上的一个\end{CJK} $n$ \begin{CJK}{UTF8}{mj}维线性空间\end{CJK}, \begin{CJK}{UTF8}{mj}设\end{CJK} $\alpha_{1}, \alpha_{2}, \cdots, \alpha_{n}$ \begin{CJK}{UTF8}{mj}是\end{CJK} $V$ \begin{CJK}{UTF8}{mj}的一组基\end{CJK}. \begin{CJK}{UTF8}{mj}用\end{CJK} $V_{1}$ \begin{CJK}{UTF8}{mj}表示由向量\end{CJK} $\alpha_{1}, \alpha_{2}, \cdots, \alpha_{n}$ \begin{CJK}{UTF8}{mj}生成的子空间\end{CJK}, \begin{CJK}{UTF8}{mj}令\end{CJK}
\end{enumerate}
$$
V_{1}=\left\{k \alpha_{1}+k \alpha_{2}+\cdots+k \alpha_{n}, k \in P\right\}, V_{2}=\left\{k_{1} \alpha_{1}+k_{2} \alpha_{2}+\cdots+k_{n} \alpha_{n} \mid \sum_{i=1}^{n} k_{i}=0, k_{i} \in P\right\}
$$
\begin{CJK}{UTF8}{mj}证明\end{CJK}:

(1) $V_{1}, V_{2}$ \begin{CJK}{UTF8}{mj}是\end{CJK} $V$ \begin{CJK}{UTF8}{mj}的子空间\end{CJK};

(2) $V=V_{1} \oplus V_{2}$.

\begin{enumerate}
  \setcounter{enumi}{7}
  \item ( 15 \begin{CJK}{UTF8}{mj}分\end{CJK}) \begin{CJK}{UTF8}{mj}设\end{CJK} $\sigma$ \begin{CJK}{UTF8}{mj}是\end{CJK} $n$ \begin{CJK}{UTF8}{mj}维向量空间\end{CJK} $V$ \begin{CJK}{UTF8}{mj}上的一个线性幂等变换\end{CJK}, \begin{CJK}{UTF8}{mj}即\end{CJK} $\sigma^{2}=\sigma$, \begin{CJK}{UTF8}{mj}试证\end{CJK}:
\end{enumerate}
(1) $\sigma$ \begin{CJK}{UTF8}{mj}的特征值只能是\end{CJK} 0 \begin{CJK}{UTF8}{mj}和\end{CJK} 1 ;

(2) $\sigma+\varepsilon$ \begin{CJK}{UTF8}{mj}为\end{CJK} $V$ \begin{CJK}{UTF8}{mj}上的可逆线性变换\end{CJK}, \begin{CJK}{UTF8}{mj}其中\end{CJK} $\varepsilon$ \begin{CJK}{UTF8}{mj}是\end{CJK} $V$ \begin{CJK}{UTF8}{mj}上的单位变换\end{CJK}.

\begin{enumerate}
  \setcounter{enumi}{8}
  \item ( 15 \begin{CJK}{UTF8}{mj}分\end{CJK}) \begin{CJK}{UTF8}{mj}设\end{CJK} $\alpha_{0}, \alpha_{1}, \cdots, \alpha_{n-1}$ \begin{CJK}{UTF8}{mj}是\end{CJK} $n$ \begin{CJK}{UTF8}{mj}个实数\end{CJK}, $C$ \begin{CJK}{UTF8}{mj}是如下的\end{CJK} $n$ \begin{CJK}{UTF8}{mj}阶方阵\end{CJK}
\end{enumerate}
$$
C=\left|\begin{array}{cccccc}
0 & 1 & 0 & \cdots & 0 & 0 \\
0 & 0 & 1 & \cdots & 0 & 0 \\
\vdots & \vdots & \vdots & \cdots & \vdots & \vdots \\
0 & 0 & 0 & \cdots & 0 & 1 \\
-a_{0} & -a_{1} & -a_{2} & \cdots & -a_{n-2} & -a_{n-1}
\end{array}\right|
$$
(1) \begin{CJK}{UTF8}{mj}若\end{CJK} $\lambda$ \begin{CJK}{UTF8}{mj}是\end{CJK} $C$ \begin{CJK}{UTF8}{mj}的特征值\end{CJK}, \begin{CJK}{UTF8}{mj}试证\end{CJK}: $\left(1, \lambda, \lambda^{2}, \cdots, \lambda^{n-1}\right)^{\prime}$ \begin{CJK}{UTF8}{mj}是特征值\end{CJK} $\lambda$ \begin{CJK}{UTF8}{mj}的特征向量\end{CJK};

(2) \begin{CJK}{UTF8}{mj}若\end{CJK} $\lambda_{1}, \lambda_{2}, \cdots, \lambda_{n}$ \begin{CJK}{UTF8}{mj}是矩阵\end{CJK} $C$ \begin{CJK}{UTF8}{mj}的两两互异的特征值\end{CJK}, \begin{CJK}{UTF8}{mj}试求可逆矩阵\end{CJK} $P$, \begin{CJK}{UTF8}{mj}使得\end{CJK} $P^{-1} C P$ \begin{CJK}{UTF8}{mj}为对角矩阵\end{CJK}.

\begin{enumerate}
  \setcounter{enumi}{9}
  \item ( 20 \begin{CJK}{UTF8}{mj}分\end{CJK}) \begin{CJK}{UTF8}{mj}设\end{CJK} $\sigma, \tau$ \begin{CJK}{UTF8}{mj}是\end{CJK} $n$ \begin{CJK}{UTF8}{mj}维欧氏空间\end{CJK} $V$ \begin{CJK}{UTF8}{mj}上的线性变换\end{CJK}, $\sigma$ \begin{CJK}{UTF8}{mj}在\end{CJK} $V$ \begin{CJK}{UTF8}{mj}的一组标准正交基下的矩阵为\end{CJK} $A$. \begin{CJK}{UTF8}{mj}且对\end{CJK} $\forall \alpha, \beta \in V$ \begin{CJK}{UTF8}{mj}有\end{CJK} $(\sigma(\alpha), \beta)=(\alpha, \tau(\beta))$. \begin{CJK}{UTF8}{mj}证明\end{CJK}:
\end{enumerate}
(1) $\tau$ \begin{CJK}{UTF8}{mj}在标准正交基\end{CJK} $\left\{e_{1}, e_{2}, \cdots, e_{n}\right\}$ \begin{CJK}{UTF8}{mj}下的矩阵为\end{CJK} $A$ \begin{CJK}{UTF8}{mj}的转置矩阵\end{CJK} $A^{\prime}$;

(2) \begin{CJK}{UTF8}{mj}若\end{CJK} $U$ \begin{CJK}{UTF8}{mj}为\end{CJK} $\sigma$ \begin{CJK}{UTF8}{mj}的不变子空间\end{CJK}, \begin{CJK}{UTF8}{mj}则\end{CJK} $U^{\perp}$ \begin{CJK}{UTF8}{mj}为\end{CJK} $\tau$ \begin{CJK}{UTF8}{mj}的不变子空间\end{CJK};

(3) \begin{CJK}{UTF8}{mj}若\end{CJK} $\sigma$ \begin{CJK}{UTF8}{mj}是可逆的线性变换\end{CJK}, \begin{CJK}{UTF8}{mj}则\end{CJK} $\tau$ \begin{CJK}{UTF8}{mj}也是可逆的\end{CJK}.

\section{5. 湖南大学 2011 年研究生入学考试试题高等代数 
 李扬 
 微信公众号: sxkyliyang}
\begin{enumerate}
  \item (15 \begin{CJK}{UTF8}{mj}分\end{CJK}) \begin{CJK}{UTF8}{mj}计算\end{CJK} $n$ \begin{CJK}{UTF8}{mj}阶行列式\end{CJK}
\end{enumerate}
$$
D_{n}=\left|\begin{array}{cccccc}
1+x & z & & & & \\
y & 1+x & z & & & \\
& y & 1+x & \ddots & & \\
& & \ddots & \ddots & z & \\
& & y & 1+x & z \\
& & & y & 1+x
\end{array}\right|,
$$
\begin{CJK}{UTF8}{mj}其中\end{CJK}: $x=y z$.

\begin{enumerate}
  \setcounter{enumi}{2}
  \item ( 15 \begin{CJK}{UTF8}{mj}分\end{CJK}) \begin{CJK}{UTF8}{mj}已知多项式\end{CJK} $f(x)$ \begin{CJK}{UTF8}{mj}满足\end{CJK} $f(3)=0, f(4)=1$. \begin{CJK}{UTF8}{mj}求\end{CJK} $f(x)$ \begin{CJK}{UTF8}{mj}除以\end{CJK} $(x-3)(x-4)$ \begin{CJK}{UTF8}{mj}的余式\end{CJK}.

  \item (20 \begin{CJK}{UTF8}{mj}分\end{CJK}) \begin{CJK}{UTF8}{mj}设\end{CJK}

\end{enumerate}
$$
f(x)=1+\frac{1}{2 !} x^{2}+\frac{1}{4 !} x^{4}+\cdots+\frac{1}{(2 k) !} x^{2 k}(k \geqslant 1)
$$
\begin{CJK}{UTF8}{mj}证明\end{CJK} $f(x)$ \begin{CJK}{UTF8}{mj}不存在三重根\end{CJK}.

\begin{enumerate}
  \setcounter{enumi}{4}
  \item ( 15 \begin{CJK}{UTF8}{mj}分\end{CJK}) \begin{CJK}{UTF8}{mj}设矩阵\end{CJK} $A, B$ \begin{CJK}{UTF8}{mj}分别为\end{CJK} $m \times n$ \begin{CJK}{UTF8}{mj}和\end{CJK} $n \times m$ \begin{CJK}{UTF8}{mj}阶矩阵\end{CJK}, $C$ \begin{CJK}{UTF8}{mj}为\end{CJK} $n$ \begin{CJK}{UTF8}{mj}阶可逆矩阵\end{CJK}, \begin{CJK}{UTF8}{mj}且矩阵\end{CJK} $A$ \begin{CJK}{UTF8}{mj}的秩为\end{CJK} $r(r<n)$. \begin{CJK}{UTF8}{mj}并且\end{CJK} $A(C+B A)=0$, \begin{CJK}{UTF8}{mj}证明\end{CJK}:
\end{enumerate}
(1) \begin{CJK}{UTF8}{mj}矩阵\end{CJK} $C+B A$ \begin{CJK}{UTF8}{mj}的秩为\end{CJK} $n-r$;

(2) \begin{CJK}{UTF8}{mj}线性方程组\end{CJK} $A X=0$ \begin{CJK}{UTF8}{mj}的通解为\end{CJK} $x=(C+B A) z$, \begin{CJK}{UTF8}{mj}其中\end{CJK} $z$ \begin{CJK}{UTF8}{mj}为任意的\end{CJK} $n$ \begin{CJK}{UTF8}{mj}维列向量\end{CJK}.

\begin{enumerate}
  \setcounter{enumi}{5}
  \item ( 15 \begin{CJK}{UTF8}{mj}分\end{CJK}) \begin{CJK}{UTF8}{mj}设\end{CJK} $A=\left(a_{i j}\right)^{n \times n}$, \begin{CJK}{UTF8}{mj}且\end{CJK} $|A|=0$, \begin{CJK}{UTF8}{mj}证明\end{CJK}: $A$ \begin{CJK}{UTF8}{mj}的伴随矩阵\end{CJK} $A^{*}$ \begin{CJK}{UTF8}{mj}的\end{CJK} $n$ \begin{CJK}{UTF8}{mj}个特征值中至少有\end{CJK} $n-1$ \begin{CJK}{UTF8}{mj}个为\end{CJK} 0 , \begin{CJK}{UTF8}{mj}且\end{CJK} \begin{CJK}{UTF8}{mj}一个非零特征值\end{CJK} (\begin{CJK}{UTF8}{mj}如果存在\end{CJK})\begin{CJK}{UTF8}{mj}等于\end{CJK} $A_{11}+A_{22}+\cdots+A_{n n}$, \begin{CJK}{UTF8}{mj}其中\end{CJK} $A_{i j}$ \begin{CJK}{UTF8}{mj}为矩阵\end{CJK} $A$ \begin{CJK}{UTF8}{mj}的关于元素\end{CJK} $a_{i j}$ \begin{CJK}{UTF8}{mj}的代数余子\end{CJK} \begin{CJK}{UTF8}{mj}式\end{CJK}.

  \item ( 20 \begin{CJK}{UTF8}{mj}分\end{CJK}) \begin{CJK}{UTF8}{mj}设\end{CJK} $A$ \begin{CJK}{UTF8}{mj}为\end{CJK} $n$ \begin{CJK}{UTF8}{mj}阶实对称矩阵\end{CJK}, $b=\left(b_{1}, b_{2}, \cdots, b_{n}\right)^{\prime}$ \begin{CJK}{UTF8}{mj}为\end{CJK} $n$ \begin{CJK}{UTF8}{mj}维实的列向量\end{CJK}, \begin{CJK}{UTF8}{mj}证明\end{CJK}:

\end{enumerate}
(1) \begin{CJK}{UTF8}{mj}若\end{CJK} $A>0$, \begin{CJK}{UTF8}{mj}则\end{CJK} $A^{-1}>0$, \begin{CJK}{UTF8}{mj}这里\end{CJK} $A>0$ \begin{CJK}{UTF8}{mj}表示\end{CJK} $A$ \begin{CJK}{UTF8}{mj}为正定矩阵\end{CJK};

(2) \begin{CJK}{UTF8}{mj}若\end{CJK} $A-b b^{\prime}>0$, \begin{CJK}{UTF8}{mj}则\end{CJK} $A>0$ \begin{CJK}{UTF8}{mj}且\end{CJK} $b^{\prime} A^{-1} b<1$.

\begin{enumerate}
  \setcounter{enumi}{7}
  \item (15 \begin{CJK}{UTF8}{mj}分\end{CJK}) \begin{CJK}{UTF8}{mj}设\end{CJK} $A \in P^{n \times n}$, \begin{CJK}{UTF8}{mj}且\end{CJK} $A^{2}=E_{n}$, \begin{CJK}{UTF8}{mj}其中\end{CJK} $E_{n}$ \begin{CJK}{UTF8}{mj}为\end{CJK} $n$ \begin{CJK}{UTF8}{mj}阶单位矩阵\end{CJK}, \begin{CJK}{UTF8}{mj}令\end{CJK}
\end{enumerate}
$$
V_{1}=\left\{x \in P^{n} \mid A x=x\right\}, V_{2}=\left\{x \in P^{n} \mid A x=-x\right\}
$$
\begin{CJK}{UTF8}{mj}证明\end{CJK}:

(1) $V_{1}$ \begin{CJK}{UTF8}{mj}和\end{CJK} $V_{2}$ \begin{CJK}{UTF8}{mj}均为\end{CJK} $P^{n}$ \begin{CJK}{UTF8}{mj}的子空间\end{CJK};

(2) $P^{n}=V_{1} \oplus V_{2}$, \begin{CJK}{UTF8}{mj}其中\end{CJK} $\oplus$ \begin{CJK}{UTF8}{mj}表示子空间的直和\end{CJK}.

\begin{enumerate}
  \setcounter{enumi}{8}
  \item ( 15 \begin{CJK}{UTF8}{mj}分\end{CJK}) \begin{CJK}{UTF8}{mj}设\end{CJK} $V$ \begin{CJK}{UTF8}{mj}是复数域\end{CJK} $C$ \begin{CJK}{UTF8}{mj}上的线性空间\end{CJK}, $\sigma$ \begin{CJK}{UTF8}{mj}和\end{CJK} $\tau$ \begin{CJK}{UTF8}{mj}均为\end{CJK} $V$ \begin{CJK}{UTF8}{mj}上的线性变换\end{CJK}, \begin{CJK}{UTF8}{mj}且满足\end{CJK} $\sigma \tau=\tau \sigma$, \begin{CJK}{UTF8}{mj}又设\end{CJK} $\lambda_{0}$ \begin{CJK}{UTF8}{mj}为\end{CJK} $\sigma$ \begin{CJK}{UTF8}{mj}的一个特征值\end{CJK}, \begin{CJK}{UTF8}{mj}证明\end{CJK}:
\end{enumerate}
(1) $V_{\lambda_{0}}$ \begin{CJK}{UTF8}{mj}为\end{CJK} $\tau$ \begin{CJK}{UTF8}{mj}的不变子空间\end{CJK}, \begin{CJK}{UTF8}{mj}其中\end{CJK} $V_{\lambda_{0}}=\left\{\alpha \in V \mid \sigma \alpha=\lambda_{0} \alpha\right\}$ \begin{CJK}{UTF8}{mj}为\end{CJK} $\sigma$ \begin{CJK}{UTF8}{mj}的特征子空间\end{CJK};

(2) $\sigma$ \begin{CJK}{UTF8}{mj}与\end{CJK} $\tau$ \begin{CJK}{UTF8}{mj}至少有一个公共的特征向量\end{CJK}. 9. ( 15 \begin{CJK}{UTF8}{mj}分\end{CJK}) \begin{CJK}{UTF8}{mj}试确定正交矩阵\end{CJK} $T$, \begin{CJK}{UTF8}{mj}使得\end{CJK} $T^{\prime} A T$ \begin{CJK}{UTF8}{mj}为对角矩阵\end{CJK}, \begin{CJK}{UTF8}{mj}其中\end{CJK}
$$
A=\left(\begin{array}{ccc}
1 & -1 & -1 \\
-1 & 1 & -1 \\
-1 & -1 & 1
\end{array}\right)
$$

\section{6. 湖南大学 2012 年研究生入学考试试题高等代数 
 李扬 
 微信公众号: sxkyliyang}
\begin{enumerate}
  \item (15 \begin{CJK}{UTF8}{mj}分\end{CJK}) \begin{CJK}{UTF8}{mj}计算\end{CJK} $n$ \begin{CJK}{UTF8}{mj}阶行列式\end{CJK}
\end{enumerate}
$$
D_{n}=\left|\begin{array}{cccccc}
3 & 2 & 0 & \cdots & 0 & 0 \\
1 & 3 & 2 & \cdots & 0 & 0 \\
& \vdots & & & \vdots & \\
0 & 0 & 0 & \cdots & 3 & 2 \\
0 & 0 & 0 & \cdots & 1 & 3
\end{array}\right| .
$$

\begin{enumerate}
  \setcounter{enumi}{2}
  \item (15 \begin{CJK}{UTF8}{mj}分\end{CJK}) \begin{CJK}{UTF8}{mj}设\end{CJK} $f(x), g(x), h(x), k(x)$ \begin{CJK}{UTF8}{mj}是数域\end{CJK} $P$ \begin{CJK}{UTF8}{mj}上的多项式\end{CJK}, \begin{CJK}{UTF8}{mj}且有\end{CJK}
\end{enumerate}
$$
\begin{aligned}
&\left(x^{2}+1\right) h(x)+(x+1) f(x)+(x+2) g(x)=0 \\
&\left(x^{2}+1\right) k(x)+(x-1) f(x)+(x-2) g(x)=0
\end{aligned}
$$
\begin{CJK}{UTF8}{mj}证明\end{CJK}: $\left(x^{2}+1\right)$ \begin{CJK}{UTF8}{mj}是\end{CJK} $f(x)$ \begin{CJK}{UTF8}{mj}和\end{CJK} $g(x)$ \begin{CJK}{UTF8}{mj}的公因式\end{CJK}.

\begin{enumerate}
  \setcounter{enumi}{3}
  \item ( 15 \begin{CJK}{UTF8}{mj}分\end{CJK}) \begin{CJK}{UTF8}{mj}设\end{CJK} $A$ \begin{CJK}{UTF8}{mj}是\end{CJK} $m \times n$ \begin{CJK}{UTF8}{mj}阶实矩阵\end{CJK}, $r(A)$ \begin{CJK}{UTF8}{mj}为矩阵\end{CJK} $A$ \begin{CJK}{UTF8}{mj}的秩\end{CJK}, \begin{CJK}{UTF8}{mj}证明\end{CJK}:
\end{enumerate}
(1) $r(A)=r\left(A^{\prime} A\right)$;

(2) \begin{CJK}{UTF8}{mj}对任意\end{CJK} $m$ \begin{CJK}{UTF8}{mj}维实向量\end{CJK} $b$, \begin{CJK}{UTF8}{mj}线性方程组\end{CJK} $A^{\prime} A x=A^{\prime} b$ \begin{CJK}{UTF8}{mj}必定有解\end{CJK}.

\begin{enumerate}
  \setcounter{enumi}{4}
  \item ( 20 \begin{CJK}{UTF8}{mj}分\end{CJK}) \begin{CJK}{UTF8}{mj}设\end{CJK} $A$ \begin{CJK}{UTF8}{mj}为\end{CJK} $n$ \begin{CJK}{UTF8}{mj}阶方阵\end{CJK}, \begin{CJK}{UTF8}{mj}且秩\end{CJK} $r(A)=r$, \begin{CJK}{UTF8}{mj}证明\end{CJK}:
\end{enumerate}
(1) \begin{CJK}{UTF8}{mj}存在\end{CJK} $n$ \begin{CJK}{UTF8}{mj}阶可逆矩阵\end{CJK} $P$, \begin{CJK}{UTF8}{mj}使得\end{CJK} $P A P^{-1}=\left(\begin{array}{c}B \\ 0\end{array}\right)$, \begin{CJK}{UTF8}{mj}其中\end{CJK} $B$ \begin{CJK}{UTF8}{mj}为\end{CJK} $r \times n$ \begin{CJK}{UTF8}{mj}阶行满秩矩阵\end{CJK};

(2) \begin{CJK}{UTF8}{mj}若\end{CJK} $A$ \begin{CJK}{UTF8}{mj}为幂等矩阵\end{CJK}, \begin{CJK}{UTF8}{mj}即\end{CJK} $A^{2}=A$, \begin{CJK}{UTF8}{mj}则\end{CJK} $\operatorname{tr}(A)=r$, \begin{CJK}{UTF8}{mj}其中\end{CJK} $\operatorname{tr}(A)$ \begin{CJK}{UTF8}{mj}为矩阵\end{CJK} $A$ \begin{CJK}{UTF8}{mj}的迹\end{CJK}.

\begin{enumerate}
  \setcounter{enumi}{5}
  \item ( 20 \begin{CJK}{UTF8}{mj}分\end{CJK}) \begin{CJK}{UTF8}{mj}设\end{CJK} $A$ \begin{CJK}{UTF8}{mj}和\end{CJK} $B$ \begin{CJK}{UTF8}{mj}分别为\end{CJK} $n \times m$ \begin{CJK}{UTF8}{mj}阶和\end{CJK} $m \times n$ \begin{CJK}{UTF8}{mj}阶矩阵\end{CJK},
\end{enumerate}
(1) \begin{CJK}{UTF8}{mj}证明\end{CJK}: $\lambda^{m}\left|\lambda E_{n}-A B\right|=\lambda^{n}\left|\lambda E_{m}-B A\right|$, \begin{CJK}{UTF8}{mj}其中\end{CJK} $E_{n}$ \begin{CJK}{UTF8}{mj}和\end{CJK} $E_{m}$ \begin{CJK}{UTF8}{mj}分别为\end{CJK} $n$ \begin{CJK}{UTF8}{mj}阶和\end{CJK} $m$ \begin{CJK}{UTF8}{mj}阶单位矩阵\end{CJK};

(2) \begin{CJK}{UTF8}{mj}若\end{CJK}
$$
A=\left(\begin{array}{cccc}
a_{1} b_{1} & a_{1} b_{2} & \cdots & a_{1} b_{n} \\
a_{2} b_{1} & a_{2} b_{2} & \cdots & a_{2} b_{n} \\
\vdots & \vdots & & \vdots \\
a_{n} b_{1} & a_{n} b_{2} & \cdots & a_{n} b_{n}
\end{array}\right)
$$
\begin{CJK}{UTF8}{mj}求矩阵\end{CJK} $E_{n}+A$ \begin{CJK}{UTF8}{mj}的全部特征值\end{CJK}.

\begin{enumerate}
  \setcounter{enumi}{6}
  \item ( 20 \begin{CJK}{UTF8}{mj}分\end{CJK}) \begin{CJK}{UTF8}{mj}设\end{CJK} $A$ \begin{CJK}{UTF8}{mj}为\end{CJK} $n$ \begin{CJK}{UTF8}{mj}实对称矩阵\end{CJK}, \begin{CJK}{UTF8}{mj}其特征值满足\end{CJK} $\lambda_{1} \leqslant \lambda_{2} \leqslant \cdots \leqslant \lambda_{n}$, \begin{CJK}{UTF8}{mj}证明\end{CJK}:
\end{enumerate}
(1) \begin{CJK}{UTF8}{mj}对任意非零实向量\end{CJK} $x \in \mathbb{R}^{n}$, \begin{CJK}{UTF8}{mj}均有\end{CJK} $\lambda \leqslant \frac{x^{\prime} A x}{x^{\prime} x} \leqslant \lambda_{n}$;

(2) \begin{CJK}{UTF8}{mj}设\end{CJK} $B$ \begin{CJK}{UTF8}{mj}为\end{CJK} $n$ \begin{CJK}{UTF8}{mj}阶实对称正定矩阵\end{CJK}, \begin{CJK}{UTF8}{mj}则\end{CJK} $\lambda=\min _{\lambda \in \mathbb{R}^{n}} \frac{x^{\prime} A x}{x^{\prime} B x}$, \begin{CJK}{UTF8}{mj}其中\end{CJK} $\lambda$ \begin{CJK}{UTF8}{mj}为矩阵\end{CJK} $A B^{-1}$ \begin{CJK}{UTF8}{mj}的最小特征值\end{CJK}; 7. ( 20 \begin{CJK}{UTF8}{mj}分\end{CJK}) \begin{CJK}{UTF8}{mj}设\end{CJK} $\mathcal{A}$ \begin{CJK}{UTF8}{mj}为\end{CJK} $n$ \begin{CJK}{UTF8}{mj}维线性空间\end{CJK} $V$ \begin{CJK}{UTF8}{mj}上的线性变换\end{CJK}, \begin{CJK}{UTF8}{mj}且\end{CJK} $\mathcal{A}^{2}=\mathcal{E}$, \begin{CJK}{UTF8}{mj}其中\end{CJK} $\mathcal{E}$ \begin{CJK}{UTF8}{mj}为\end{CJK} $V$ \begin{CJK}{UTF8}{mj}上的单位变换\end{CJK}.\\
(1) $\mathcal{A}$ \begin{CJK}{UTF8}{mj}的特征值只能是\end{CJK} $\pm 1$;\\
(2) $V=V_{1} \oplus V_{-1}$, \begin{CJK}{UTF8}{mj}其中\end{CJK} $V_{1}$ \begin{CJK}{UTF8}{mj}和\end{CJK} $V_{-1}$ \begin{CJK}{UTF8}{mj}分别为线性变换\end{CJK} $\mathcal{A}$ \begin{CJK}{UTF8}{mj}关于特征值\end{CJK} 1 \begin{CJK}{UTF8}{mj}和\end{CJK} $-1$ \begin{CJK}{UTF8}{mj}的特征子空间\end{CJK};\\
(3) \begin{CJK}{UTF8}{mj}存在\end{CJK} $V$ \begin{CJK}{UTF8}{mj}上的线性变换\end{CJK} $\mathcal{A}_{1}$ \begin{CJK}{UTF8}{mj}和\end{CJK} $\mathcal{A}_{2}$, \begin{CJK}{UTF8}{mj}使得\end{CJK} $\mathcal{A}_{1}+\mathcal{A}_{2}=\mathcal{E}$, \begin{CJK}{UTF8}{mj}且\end{CJK} $\mathcal{A}=\mathcal{A}_{1}-\mathcal{A}_{2}$.

\begin{enumerate}
  \setcounter{enumi}{8}
  \item ( 10 \begin{CJK}{UTF8}{mj}分\end{CJK}) \begin{CJK}{UTF8}{mj}设矩阵\end{CJK}
\end{enumerate}
$$
A=\left(\begin{array}{ccc}
2 & 0 & 0 \\
a & 2 & 0 \\
b & c & -1
\end{array}\right)
$$
\begin{CJK}{UTF8}{mj}试问矩阵\end{CJK} $A$ \begin{CJK}{UTF8}{mj}可能有什么样的若尔当标准形\end{CJK}? \begin{CJK}{UTF8}{mj}试给出该矩阵可对角化的充分必要条件\end{CJK}.

\begin{enumerate}
  \setcounter{enumi}{9}
  \item ( 15 \begin{CJK}{UTF8}{mj}分\end{CJK}) \begin{CJK}{UTF8}{mj}设\end{CJK}
\end{enumerate}
$$
A=\left(\begin{array}{ccc}
1 & -1 & 0 \\
-1 & 2 & 0 \\
0 & 0 & 1
\end{array}\right), B=\left(\begin{array}{ccc}
1 & 0 & 1 \\
0 & 0 & 0 \\
1 & 0 & 1
\end{array}\right)
$$
\begin{CJK}{UTF8}{mj}求可逆矩阵\end{CJK} $T$ \begin{CJK}{UTF8}{mj}使得\end{CJK} $T^{\prime} A T$ \begin{CJK}{UTF8}{mj}和\end{CJK} $T^{\prime} B T$ \begin{CJK}{UTF8}{mj}同时为对角矩阵\end{CJK}.

\section{7. 湖南大学 2013 年研究生入学考试试题高等代数}
\begin{CJK}{UTF8}{mj}李扬\end{CJK}

\begin{CJK}{UTF8}{mj}微信公众号\end{CJK}: sxkyliyang

\begin{enumerate}
  \item ( 15 \begin{CJK}{UTF8}{mj}分\end{CJK}) \begin{CJK}{UTF8}{mj}计算\end{CJK} $n$ \begin{CJK}{UTF8}{mj}阶行列式\end{CJK}
\end{enumerate}
$$
D_{n}=\left|\begin{array}{cccccc}
x & -1 & -1 & \cdots & -1 & -1 \\
1 & x & -1 & \cdots & -1 & -1 \\
1 & 1 & x & \cdots & -1 & -1 \\
\vdots & \vdots & \vdots & & \vdots & \vdots \\
1 & 1 & 1 & \cdots & x & -1 \\
1 & 1 & 1 & \cdots & 1 & x
\end{array}\right|,
$$

\begin{enumerate}
  \setcounter{enumi}{2}
  \item (15 \begin{CJK}{UTF8}{mj}分\end{CJK}) \begin{CJK}{UTF8}{mj}设\end{CJK} $A, B$ \begin{CJK}{UTF8}{mj}均为实数域上的\end{CJK} $n$ \begin{CJK}{UTF8}{mj}阶方阵\end{CJK}, $r(A)$ \begin{CJK}{UTF8}{mj}和\end{CJK} $r(B)$ \begin{CJK}{UTF8}{mj}分别表示矩阵\end{CJK} $A$ \begin{CJK}{UTF8}{mj}和矩阵\end{CJK} $B$ \begin{CJK}{UTF8}{mj}的秩\end{CJK}, \begin{CJK}{UTF8}{mj}证明\end{CJK}:
\end{enumerate}
(1) \begin{CJK}{UTF8}{mj}若\end{CJK} $A B=0$, \begin{CJK}{UTF8}{mj}则\end{CJK} $r(A)+r(B) \leqslant n$;

(2) \begin{CJK}{UTF8}{mj}若\end{CJK} $A x=0$ \begin{CJK}{UTF8}{mj}与\end{CJK} $B x=0$ \begin{CJK}{UTF8}{mj}同解当且仅当存在\end{CJK} $n$ \begin{CJK}{UTF8}{mj}阶方阵\end{CJK} $P$ \begin{CJK}{UTF8}{mj}和\end{CJK} $Q$, \begin{CJK}{UTF8}{mj}使得\end{CJK} $A=P B$ \begin{CJK}{UTF8}{mj}和\end{CJK} $B=Q A$.

\begin{enumerate}
  \setcounter{enumi}{3}
  \item ( 15 \begin{CJK}{UTF8}{mj}分\end{CJK}) \begin{CJK}{UTF8}{mj}设\end{CJK} $A$ \begin{CJK}{UTF8}{mj}为\end{CJK} $n$ \begin{CJK}{UTF8}{mj}阶实对称正定矩阵\end{CJK}, $B$ \begin{CJK}{UTF8}{mj}为\end{CJK} $n \times m$ \begin{CJK}{UTF8}{mj}阶实矩阵\end{CJK}, \begin{CJK}{UTF8}{mj}令\end{CJK} $M=\left(\begin{array}{c}A \\ B \\ 0\end{array}\right)$, \begin{CJK}{UTF8}{mj}证明\end{CJK}: \begin{CJK}{UTF8}{mj}若\end{CJK} $B$ \begin{CJK}{UTF8}{mj}为列\end{CJK} \begin{CJK}{UTF8}{mj}满秩矩阵\end{CJK}, \begin{CJK}{UTF8}{mj}则二次型\end{CJK} $f(X)=X^{\prime} M X$ \begin{CJK}{UTF8}{mj}的正负惯性指数分别为\end{CJK} $n$ \begin{CJK}{UTF8}{mj}和\end{CJK} $m$.

  \item (15 \begin{CJK}{UTF8}{mj}分\end{CJK}) \begin{CJK}{UTF8}{mj}设\end{CJK} $A, B$ \begin{CJK}{UTF8}{mj}均为\end{CJK} $n$ \begin{CJK}{UTF8}{mj}阶实对称矩阵\end{CJK}, \begin{CJK}{UTF8}{mj}且\end{CJK} $A$ \begin{CJK}{UTF8}{mj}是正定矩阵\end{CJK}, $B$ \begin{CJK}{UTF8}{mj}是半正定矩阵\end{CJK}, \begin{CJK}{UTF8}{mj}证明\end{CJK}:

\end{enumerate}
(1) \begin{CJK}{UTF8}{mj}存在\end{CJK} $n$ \begin{CJK}{UTF8}{mj}阶可逆矩阵\end{CJK} $T$, \begin{CJK}{UTF8}{mj}使得\end{CJK} $T^{\prime} A T$ \begin{CJK}{UTF8}{mj}和\end{CJK} $T^{\prime} B T$ \begin{CJK}{UTF8}{mj}同为对角矩阵\end{CJK};

(2) $A+B$ \begin{CJK}{UTF8}{mj}为正定矩阵的充分必要条件是\end{CJK} $B A^{-1}$ \begin{CJK}{UTF8}{mj}的所有特征值都大于\end{CJK} $-1$.

\begin{enumerate}
  \setcounter{enumi}{5}
  \item ( 25 \begin{CJK}{UTF8}{mj}分\end{CJK}) \begin{CJK}{UTF8}{mj}设\end{CJK} $V$ \begin{CJK}{UTF8}{mj}是数域\end{CJK} $P$ \begin{CJK}{UTF8}{mj}上的\end{CJK} $n$ \begin{CJK}{UTF8}{mj}维线性空间\end{CJK}, $\mathscr{A}$ \begin{CJK}{UTF8}{mj}是\end{CJK} $V$ \begin{CJK}{UTF8}{mj}的线性变换\end{CJK}, \begin{CJK}{UTF8}{mj}如果存在\end{CJK} $V$ \begin{CJK}{UTF8}{mj}中的一组基\end{CJK}, \begin{CJK}{UTF8}{mj}使得\end{CJK} $\mathscr{A}$ \begin{CJK}{UTF8}{mj}在\end{CJK} \begin{CJK}{UTF8}{mj}这组基下的矩阵为若当块矩阵\end{CJK}
\end{enumerate}
$$
\left(\begin{array}{ccccc}
0 & 0 & \cdots & 0 & 0 \\
1 & 0 & \cdots & 0 & 0 \\
\vdots & \vdots & & \vdots & \vdots \\
0 & 0 & \cdots & 1 & 0
\end{array}\right)
$$
\begin{CJK}{UTF8}{mj}证明\end{CJK}:

(1) \begin{CJK}{UTF8}{mj}存在向量\end{CJK} $\varepsilon \in V$, \begin{CJK}{UTF8}{mj}使得\end{CJK} $\mathscr{A}^{n-1} \varepsilon \neq 0$, \begin{CJK}{UTF8}{mj}但\end{CJK} $\mathscr{A}^{n} \varepsilon=0$;

(2) $V$ \begin{CJK}{UTF8}{mj}中包含向量\end{CJK} $\varepsilon$ \begin{CJK}{UTF8}{mj}的\end{CJK} $\mathscr{A}-$ \begin{CJK}{UTF8}{mj}子空间\end{CJK} (\begin{CJK}{UTF8}{mj}即\end{CJK} $\mathscr{A}$ \begin{CJK}{UTF8}{mj}的不变子空间\end{CJK}) \begin{CJK}{UTF8}{mj}只有线性空间\end{CJK} $V$ \begin{CJK}{UTF8}{mj}自身\end{CJK};

(3) $V$ \begin{CJK}{UTF8}{mj}中的任一非零\end{CJK} $\mathscr{A}$ - \begin{CJK}{UTF8}{mj}子空间都包含\end{CJK} $\mathscr{A}^{n-1} \varepsilon$;

(4) $V$ \begin{CJK}{UTF8}{mj}不能分解为两个非平凡\end{CJK} $\mathscr{A}$ - \begin{CJK}{UTF8}{mj}子空间的直和\end{CJK}.

\begin{enumerate}
  \setcounter{enumi}{6}
  \item ( 15 \begin{CJK}{UTF8}{mj}分\end{CJK}) \begin{CJK}{UTF8}{mj}设\end{CJK} $f(x)=a_{n} x^{n}+a_{n-1} x^{n-1}+\cdots+a_{1} x+a_{0}$ \begin{CJK}{UTF8}{mj}是一个整系数多项式\end{CJK}. \begin{CJK}{UTF8}{mj}如果存在一个素数\end{CJK} $p$, \begin{CJK}{UTF8}{mj}使得\end{CJK}:
\end{enumerate}
(1) $p$ \begin{CJK}{UTF8}{mj}不整除\end{CJK} $a_{n}$;

(2) $p$ \begin{CJK}{UTF8}{mj}整除\end{CJK} $a_{n-1}, \cdots, a_{1}, a_{0}$;

(3) $p^{2}$ \begin{CJK}{UTF8}{mj}不整除\end{CJK} $a_{0}$.

\begin{CJK}{UTF8}{mj}证明\end{CJK}: \begin{CJK}{UTF8}{mj}多项式\end{CJK} $f(x)$ \begin{CJK}{UTF8}{mj}在有理数域上不可约\end{CJK}. 7. ( 18 \begin{CJK}{UTF8}{mj}分\end{CJK}) \begin{CJK}{UTF8}{mj}设\end{CJK} 4 \begin{CJK}{UTF8}{mj}元齐次线性方程组\end{CJK} $(I)$ \begin{CJK}{UTF8}{mj}为\end{CJK}
$$
\left\{\begin{array}{l}
2 x_{1}+3 x_{2}-x_{3}=0 \\
x_{1}+2 x_{2}+x_{3}-x_{4}=0
\end{array}\right.
$$
\begin{CJK}{UTF8}{mj}已知另一个\end{CJK} 4 \begin{CJK}{UTF8}{mj}元齐次线性方程组\end{CJK} $(I I)$ \begin{CJK}{UTF8}{mj}的一个基础解系为\end{CJK} $\alpha_{1}=(2,-1, a+2,1)^{\prime}, \alpha_{2}=(-1,2,4, a+8)^{\prime}$.

(1) \begin{CJK}{UTF8}{mj}求齐次线性方程组\end{CJK} $(I)$ \begin{CJK}{UTF8}{mj}的一个基础解系\end{CJK};

(2) \begin{CJK}{UTF8}{mj}当\end{CJK} $a$ \begin{CJK}{UTF8}{mj}为何值时\end{CJK}, \begin{CJK}{UTF8}{mj}齐次线性方程组\end{CJK} $(I)$ \begin{CJK}{UTF8}{mj}和\end{CJK} $(I I)$ \begin{CJK}{UTF8}{mj}有非零公共解\end{CJK}? \begin{CJK}{UTF8}{mj}并求出全部的非零公共解\end{CJK}. (\begin{CJK}{UTF8}{mj}请给出必\end{CJK} \begin{CJK}{UTF8}{mj}要的计算步骤\end{CJK})

\begin{enumerate}
  \setcounter{enumi}{8}
  \item ( 20 \begin{CJK}{UTF8}{mj}分\end{CJK}) \begin{CJK}{UTF8}{mj}用正交线性替换化二次型\end{CJK}
\end{enumerate}
$$
f(x)=-x_{1}^{2}+2 x_{2}^{2}+2 x_{3}^{2}+4 x_{1} x_{2}-4 x_{1} x_{3}-8 x_{2} x_{3}
$$
\begin{CJK}{UTF8}{mj}为标准型\end{CJK}, \begin{CJK}{UTF8}{mj}求出相应的正交变换矩阵\end{CJK} $T$. (\begin{CJK}{UTF8}{mj}请给出必要的计算步骤\end{CJK})

\begin{enumerate}
  \setcounter{enumi}{9}
  \item ( 12 \begin{CJK}{UTF8}{mj}分\end{CJK}) \begin{CJK}{UTF8}{mj}设向量\end{CJK} $\alpha_{1}=(-1,-2,-1,0), \alpha_{2}=(1,-1,-1,-1), \beta_{1}=(-2,1,0,-1), \beta_{2}=(-1,1,-3,-7)$, \begin{CJK}{UTF8}{mj}记\end{CJK} $\alpha_{1}, \alpha_{2}$ \begin{CJK}{UTF8}{mj}的生成子空间为\end{CJK} $L\left(\alpha_{1}, \alpha_{2}\right), \beta_{1}, \beta_{2}$ \begin{CJK}{UTF8}{mj}的生成子空间为\end{CJK} $L\left(\beta_{1}, \beta_{2}\right)$, \begin{CJK}{UTF8}{mj}求这两个生成子空间的交\end{CJK} $L\left(\alpha_{1}, \alpha_{2}\right) \cap L\left(\beta_{1}, \beta_{2}\right)$ \begin{CJK}{UTF8}{mj}的维数和一组基\end{CJK}. (\begin{CJK}{UTF8}{mj}请给出必要的计算步骤\end{CJK})
\end{enumerate}
\section{8. 湖南大学 2014 年研究生入学考试试题高等代数 
 李扬 
 微信公众号: sxkyliyang}
\begin{enumerate}
  \item ( 16 \begin{CJK}{UTF8}{mj}分\end{CJK}) \begin{CJK}{UTF8}{mj}证明\end{CJK}:
\end{enumerate}
$$
1-\frac{1}{2 !} x^{2}+\frac{1}{4 !} x^{4}-\frac{1}{6 !} x^{6}+\cdots+(-1)^{m} \frac{1}{(2 m) !} x^{2 m}
$$
\begin{CJK}{UTF8}{mj}没有二重根\end{CJK};

\begin{enumerate}
  \setcounter{enumi}{2}
  \item ( 16 \begin{CJK}{UTF8}{mj}分\end{CJK}) \begin{CJK}{UTF8}{mj}证明\end{CJK}: $C_{n}=D_{n}$, \begin{CJK}{UTF8}{mj}其中\end{CJK}
\end{enumerate}
$$
C_{n}=\left|\begin{array}{cccccc}
2 & 1 & 0 & \cdots & 0 & 0 \\
1 & 2 & 1 & \cdots & 0 & 0 \\
0 & 1 & 2 & \cdots & 0 & 0 \\
\vdots & \vdots & \vdots & \cdots & \vdots & \vdots \\
0 & 0 & 0 & \cdots & 2 & 1 \\
0 & 0 & 0 & \cdots & 1 & 2
\end{array}\right|, D_{n}=\left|\begin{array}{cccccc}
2 & -1 & 0 & \cdots & 0 & 0 \\
-1 & 2 & -1 & \cdots & 0 & 0 \\
0 & -1 & 2 & \cdots & 0 & 0 \\
\vdots & \vdots & \vdots & \cdots & \vdots & \vdots \\
0 & 0 & 0 & \cdots & 2 & -1 \\
0 & 0 & 0 & \cdots & -1 & 2
\end{array}\right|
$$
\begin{CJK}{UTF8}{mj}并求出\end{CJK} $C_{n}$ \begin{CJK}{UTF8}{mj}的值\end{CJK}.

\begin{enumerate}
  \setcounter{enumi}{3}
  \item (18 \begin{CJK}{UTF8}{mj}分\end{CJK}) $\lambda$ \begin{CJK}{UTF8}{mj}取何值时\end{CJK}, \begin{CJK}{UTF8}{mj}线性方程组\end{CJK}
\end{enumerate}
$$
\left\{\begin{array}{l}
(2 \lambda+1) x_{1}-\lambda x_{2}+(\lambda+1) x_{3}=\lambda-1 \\
(\lambda-2) x_{1}+(\lambda-1) x_{2}+(\lambda-2) x_{3}=\lambda \\
(2 \lambda-1) x_{1}+(\lambda-1) x_{2}+(2 \lambda-1) x_{3}=\lambda
\end{array}\right.
$$
\begin{CJK}{UTF8}{mj}有唯一解\end{CJK}, \begin{CJK}{UTF8}{mj}无解\end{CJK}, \begin{CJK}{UTF8}{mj}无穷多解\end{CJK}? \begin{CJK}{UTF8}{mj}在有无穷多解时\end{CJK}, \begin{CJK}{UTF8}{mj}求通解\end{CJK}.

\begin{enumerate}
  \setcounter{enumi}{4}
  \item ( 18 \begin{CJK}{UTF8}{mj}分\end{CJK}) \begin{CJK}{UTF8}{mj}已知\end{CJK} $A^{3}=2 E, B=A^{2}-2 A+2 E$, \begin{CJK}{UTF8}{mj}证明\end{CJK} $B$ \begin{CJK}{UTF8}{mj}可逆\end{CJK}, \begin{CJK}{UTF8}{mj}并求出其逆\end{CJK}.

  \item (18 \begin{CJK}{UTF8}{mj}分\end{CJK}) \begin{CJK}{UTF8}{mj}设\end{CJK} $A$ \begin{CJK}{UTF8}{mj}为\end{CJK} $m$ \begin{CJK}{UTF8}{mj}阶实对称正定矩阵\end{CJK}, $B$ \begin{CJK}{UTF8}{mj}为\end{CJK} $m \times n$ \begin{CJK}{UTF8}{mj}实矩阵\end{CJK}, \begin{CJK}{UTF8}{mj}试证\end{CJK} $B^{T} A B$ \begin{CJK}{UTF8}{mj}为正定矩阵的充分必要条件是\end{CJK} $r(B)=n$.

  \item (18 \begin{CJK}{UTF8}{mj}分\end{CJK}) \begin{CJK}{UTF8}{mj}设\end{CJK} $\alpha_{1}, \alpha_{2}, \cdots, \alpha_{n}$ \begin{CJK}{UTF8}{mj}是\end{CJK} $n$ \begin{CJK}{UTF8}{mj}维线性空间\end{CJK} $V$ \begin{CJK}{UTF8}{mj}的一组基\end{CJK}, $\beta_{1}=\alpha_{1}, \beta_{2}=\alpha_{1}+\alpha_{2}, \cdots, \beta_{n}=\alpha_{1}+\alpha_{2}+$ $\cdots+\alpha_{n}$ \begin{CJK}{UTF8}{mj}是\end{CJK} $V$ \begin{CJK}{UTF8}{mj}的一组基\end{CJK}, \begin{CJK}{UTF8}{mj}又若\end{CJK} $\alpha \in V$ \begin{CJK}{UTF8}{mj}在前一组基下的坐标为\end{CJK} $(n, n-1, \cdots, 2,1)^{T}$, \begin{CJK}{UTF8}{mj}求\end{CJK} $\alpha$ \begin{CJK}{UTF8}{mj}在后一组基下的\end{CJK} \begin{CJK}{UTF8}{mj}坐标\end{CJK}.

  \item (18 \begin{CJK}{UTF8}{mj}分\end{CJK}) \begin{CJK}{UTF8}{mj}设\end{CJK} $A$ \begin{CJK}{UTF8}{mj}为\end{CJK} $n \times n$ \begin{CJK}{UTF8}{mj}阶实矩阵\end{CJK}, $A^{2}=A$, \begin{CJK}{UTF8}{mj}证明\end{CJK} $A$ \begin{CJK}{UTF8}{mj}相似于对角矩阵\end{CJK} $\Lambda$, \begin{CJK}{UTF8}{mj}这里\end{CJK}

\end{enumerate}
$$
\Lambda=\left(\begin{array}{cccccc}
1 & & & & & \\
& \ddots & & & & \\
& & 1 & & & \\
& & & 0 & & \\
& & & & \ddots & \\
& & & & 0
\end{array}\right)
$$

\begin{enumerate}
  \setcounter{enumi}{8}
  \item ( 18 \begin{CJK}{UTF8}{mj}分\end{CJK}) \begin{CJK}{UTF8}{mj}设\end{CJK} $A, B$ \begin{CJK}{UTF8}{mj}是两个实对称矩阵\end{CJK}, \begin{CJK}{UTF8}{mj}且\end{CJK} $B$ \begin{CJK}{UTF8}{mj}是正定矩阵\end{CJK}, \begin{CJK}{UTF8}{mj}证明存在\end{CJK} $n$ \begin{CJK}{UTF8}{mj}阶实可逆矩阵\end{CJK} $X$, \begin{CJK}{UTF8}{mj}使\end{CJK} $X^{T} A X$ \begin{CJK}{UTF8}{mj}与\end{CJK} $X^{T} B X$ \begin{CJK}{UTF8}{mj}同时为对角矩阵\end{CJK}.

  \item ( 10 \begin{CJK}{UTF8}{mj}分\end{CJK}) \begin{CJK}{UTF8}{mj}设\end{CJK} $E$ \begin{CJK}{UTF8}{mj}为\end{CJK} $n$ \begin{CJK}{UTF8}{mj}阶单位阵\end{CJK}, $\omega \in \mathbb{R}^{n}$ \begin{CJK}{UTF8}{mj}为\end{CJK} $n$ \begin{CJK}{UTF8}{mj}维单位列向量\end{CJK}, \begin{CJK}{UTF8}{mj}记\end{CJK} $H=E-2 \omega \omega^{T}$, \begin{CJK}{UTF8}{mj}试求\end{CJK} $H$ \begin{CJK}{UTF8}{mj}的特征值和行列\end{CJK} \begin{CJK}{UTF8}{mj}式\end{CJK}.

\end{enumerate}
\section{9. 湖南大学 2015 年研究生入学考试试题高等代数}
\begin{CJK}{UTF8}{mj}李扬\end{CJK}

\begin{CJK}{UTF8}{mj}微信公众号\end{CJK}: sxkyliyang

\begin{enumerate}
  \item (15 \begin{CJK}{UTF8}{mj}分\end{CJK}) \begin{CJK}{UTF8}{mj}设\end{CJK} $f(x)=x^{6}-2 x^{5}-5 x^{2}+11 x-2$.
\end{enumerate}
(1) \begin{CJK}{UTF8}{mj}试求\end{CJK} $f(x)=0$ \begin{CJK}{UTF8}{mj}的有理根\end{CJK};

(2) \begin{CJK}{UTF8}{mj}在有理数域\end{CJK} $\mathbb{Q}$ \begin{CJK}{UTF8}{mj}上把\end{CJK} $f(x)$ \begin{CJK}{UTF8}{mj}分解成不可约多项式的乘积\end{CJK}.

\begin{enumerate}
  \setcounter{enumi}{2}
  \item (16 \begin{CJK}{UTF8}{mj}分\end{CJK}) \begin{CJK}{UTF8}{mj}计算\end{CJK} $n$ \begin{CJK}{UTF8}{mj}阶行列式\end{CJK}:
\end{enumerate}
$$
\left|\begin{array}{ccccc}
x & y & \cdots & y & y \\
-y & x & \cdots & y & y \\
\vdots & \vdots & & \vdots & \vdots \\
-y & -y & \cdots & x & y \\
-y & -y & \cdots & -y & x
\end{array}\right| \text {; }
$$
(1)

\begin{enumerate}
  \setcounter{enumi}{3}
  \item ( 15 \begin{CJK}{UTF8}{mj}分\end{CJK}) \begin{CJK}{UTF8}{mj}在数域\end{CJK} $P$ \begin{CJK}{UTF8}{mj}上\end{CJK}, \begin{CJK}{UTF8}{mj}设方程组\end{CJK} $x_{1}+2 x_{2}+\cdots+n x_{n}=0$ \begin{CJK}{UTF8}{mj}的解空间为\end{CJK} $M$, \begin{CJK}{UTF8}{mj}方程组\end{CJK} $x_{1}=\frac{1}{2} x_{2}=\cdots=\frac{1}{n} x_{n}$ \begin{CJK}{UTF8}{mj}的解空间为\end{CJK} $N$, \begin{CJK}{UTF8}{mj}证明\end{CJK}: $P^{n}=M \oplus N$.

  \item (15 \begin{CJK}{UTF8}{mj}分\end{CJK}) \begin{CJK}{UTF8}{mj}设\end{CJK} $A$ \begin{CJK}{UTF8}{mj}是\end{CJK} $n$ \begin{CJK}{UTF8}{mj}阶方阵\end{CJK}, \begin{CJK}{UTF8}{mj}证明\end{CJK}:

\end{enumerate}
(1) $A$ \begin{CJK}{UTF8}{mj}的秩为\end{CJK} 1 \begin{CJK}{UTF8}{mj}的充分必要条件是存在两个\end{CJK} $n$ \begin{CJK}{UTF8}{mj}阶非零向量\end{CJK} $\alpha, \beta \in \mathbb{R}^{n}$, \begin{CJK}{UTF8}{mj}使得\end{CJK} $A=\alpha \beta^{T}$;

(2) $A=\alpha \beta^{T}$ \begin{CJK}{UTF8}{mj}时\end{CJK}, $A$ \begin{CJK}{UTF8}{mj}的特征值是\end{CJK} $0,0, \cdots, 0, \beta^{T} \alpha$.

\begin{enumerate}
  \setcounter{enumi}{5}
  \item ( 15 \begin{CJK}{UTF8}{mj}分\end{CJK}) \begin{CJK}{UTF8}{mj}求正交矩阵\end{CJK} $Q$, \begin{CJK}{UTF8}{mj}使\end{CJK} $Q^{T} A Q$ \begin{CJK}{UTF8}{mj}为对角矩阵\end{CJK}, \begin{CJK}{UTF8}{mj}并写出此对角矩阵\end{CJK}, \begin{CJK}{UTF8}{mj}其中\end{CJK}
\end{enumerate}
$$
A=\left(\begin{array}{llll}
2 & 1 & 0 & 0 \\
1 & 2 & 0 & 0 \\
0 & 0 & 0 & 1 \\
0 & 0 & 1 & 0
\end{array}\right)
$$

\begin{enumerate}
  \setcounter{enumi}{6}
  \item (15 \begin{CJK}{UTF8}{mj}分\end{CJK}) \begin{CJK}{UTF8}{mj}设\end{CJK}
\end{enumerate}
$$
\begin{aligned}
&\alpha_{1}=(1,2,1,0), \alpha_{2}=(-1,1,1,1) \\
&\beta_{1}=(2,-1,0,1), \beta_{2}=(1,-1,3,7),
\end{aligned}
$$
\begin{CJK}{UTF8}{mj}求向量组\end{CJK} $\alpha_{1}, \alpha_{2}$ \begin{CJK}{UTF8}{mj}生成的子空间\end{CJK} $V$ \begin{CJK}{UTF8}{mj}与向量组\end{CJK} $\beta_{1}, \beta_{2}$ \begin{CJK}{UTF8}{mj}生成的子空间\end{CJK} $W$ \begin{CJK}{UTF8}{mj}的交\end{CJK} $V \cap W$ \begin{CJK}{UTF8}{mj}的基与维数\end{CJK}.

\begin{enumerate}
  \setcounter{enumi}{7}
  \item ( 20 \begin{CJK}{UTF8}{mj}分\end{CJK}) \begin{CJK}{UTF8}{mj}设\end{CJK} $\varepsilon_{1}, \varepsilon_{2}, \varepsilon_{3}, \varepsilon_{4}$ \begin{CJK}{UTF8}{mj}是线性空间\end{CJK} $V$ \begin{CJK}{UTF8}{mj}的一组基\end{CJK}, \begin{CJK}{UTF8}{mj}已知线性变换\end{CJK} $\mathcal{T}$ \begin{CJK}{UTF8}{mj}在这组基下的矩阵为\end{CJK}
\end{enumerate}
$$
A=\left(\begin{array}{cccc}
1 & 0 & 2 & 1 \\
-1 & 2 & 1 & 3 \\
1 & 2 & 5 & 5 \\
2 & -2 & 1 & -2
\end{array}\right)
$$
(1)\begin{CJK}{UTF8}{mj}求\end{CJK} $\mathcal{T}$ \begin{CJK}{UTF8}{mj}在基\end{CJK} $\eta_{1}=\varepsilon_{1}-2 \varepsilon_{2}+\varepsilon_{4}, \eta_{2}=3 \varepsilon_{2}-\varepsilon_{3}-\varepsilon_{4}, \eta_{3}=\varepsilon_{3}+\varepsilon_{4}, \eta_{4}=2 \varepsilon_{4}$ \begin{CJK}{UTF8}{mj}下的矩阵\end{CJK} $B$;

(2) \begin{CJK}{UTF8}{mj}求\end{CJK} $\mathcal{T}$ \begin{CJK}{UTF8}{mj}的值域求\end{CJK} $\mathcal{T} V$ \begin{CJK}{UTF8}{mj}与核求\end{CJK} $\mathcal{T}^{-1}(0)$;

(3) \begin{CJK}{UTF8}{mj}在\end{CJK} $\mathcal{T}^{-1}(0)$ \begin{CJK}{UTF8}{mj}中选一组基\end{CJK}, \begin{CJK}{UTF8}{mj}将它扩充成\end{CJK} $V$ \begin{CJK}{UTF8}{mj}的一组基\end{CJK}.

\begin{enumerate}
  \setcounter{enumi}{8}
  \item (15 \begin{CJK}{UTF8}{mj}分\end{CJK}) \begin{CJK}{UTF8}{mj}设\end{CJK}
\end{enumerate}
$$
A=\left(\begin{array}{ccc}
3 & 7 & -3 \\
-2 & -5 & 2 \\
-4 & -10 & 3
\end{array}\right)
$$
\begin{CJK}{UTF8}{mj}求\end{CJK} $A$ \begin{CJK}{UTF8}{mj}的\end{CJK} Jordan \begin{CJK}{UTF8}{mj}标准形\end{CJK} $J$.

\begin{enumerate}
  \setcounter{enumi}{9}
  \item ( 15 \begin{CJK}{UTF8}{mj}分\end{CJK}) \begin{CJK}{UTF8}{mj}设\end{CJK} $A$ \begin{CJK}{UTF8}{mj}为\end{CJK} $n$ \begin{CJK}{UTF8}{mj}阶实矩阵\end{CJK}, \begin{CJK}{UTF8}{mj}且\end{CJK} $|A| \neq 0$, \begin{CJK}{UTF8}{mj}证明\end{CJK} $A$ \begin{CJK}{UTF8}{mj}可分解成一正交矩阵\end{CJK} $Q$ \begin{CJK}{UTF8}{mj}与一上三角矩阵\end{CJK} $R$ \begin{CJK}{UTF8}{mj}的乘积\end{CJK}, \begin{CJK}{UTF8}{mj}即\end{CJK} $A=Q R$.

  \item ( 10 \begin{CJK}{UTF8}{mj}分\end{CJK}) \begin{CJK}{UTF8}{mj}设\end{CJK} $x=\left(x_{1}, x_{2}, x_{3}\right)^{T} \in \mathbb{R}^{3}$, \begin{CJK}{UTF8}{mj}记\end{CJK} $\|x\|_{1}=|x|_{1}+|x|_{2}+|x|_{3}$, \begin{CJK}{UTF8}{mj}试问一向量\end{CJK} $\widehat{x}$, \begin{CJK}{UTF8}{mj}使\end{CJK} $A \widehat{x}=b$, \begin{CJK}{UTF8}{mj}且\end{CJK} $\|\widehat{x}\|_{1}=\min _{A x=b}\|x\|_{1}$, \begin{CJK}{UTF8}{mj}其中\end{CJK}:

\end{enumerate}
$$
A=\left(\begin{array}{ccc}
1 & -1 & 0 \\
4 & 0 & 1
\end{array}\right), b=\left(\begin{array}{l}
0 \\
2
\end{array}\right)
$$

\section{0. 湖南大学 2017 年研究生入学考试试题高等代数 
 李扬 
 微信公众号: sxkyliyang}
\begin{enumerate}
  \item (15 \begin{CJK}{UTF8}{mj}分\end{CJK})
\end{enumerate}
(1) \begin{CJK}{UTF8}{mj}设\end{CJK} $a, b, c, d$ \begin{CJK}{UTF8}{mj}均为有理数\end{CJK}, $\sqrt{d}$ \begin{CJK}{UTF8}{mj}是无理数\end{CJK}, \begin{CJK}{UTF8}{mj}且\end{CJK} $b \neq 0$. \begin{CJK}{UTF8}{mj}若\end{CJK} $a+b \sqrt{d}$ \begin{CJK}{UTF8}{mj}是有理数系多项式\end{CJK} $f(x)$ \begin{CJK}{UTF8}{mj}的根\end{CJK}, \begin{CJK}{UTF8}{mj}证明\end{CJK}: $a-b \sqrt{d}$ \begin{CJK}{UTF8}{mj}也是\end{CJK} $f(x)$ \begin{CJK}{UTF8}{mj}的根\end{CJK}.

(2) \begin{CJK}{UTF8}{mj}构造一个次数最低的首项系数为\end{CJK} 1 \begin{CJK}{UTF8}{mj}的有理系数多项式\end{CJK}, \begin{CJK}{UTF8}{mj}使得\end{CJK} $1+\sqrt{2}, 3-i$ \begin{CJK}{UTF8}{mj}都是它的根\end{CJK}.

\begin{enumerate}
  \setcounter{enumi}{2}
  \item ( 10 \begin{CJK}{UTF8}{mj}分\end{CJK}) \begin{CJK}{UTF8}{mj}设\end{CJK} $A$ \begin{CJK}{UTF8}{mj}是一个\end{CJK} $n$ \begin{CJK}{UTF8}{mj}级方阵\end{CJK}, \begin{CJK}{UTF8}{mj}证明\end{CJK}: \begin{CJK}{UTF8}{mj}存在正整数\end{CJK} $m$, \begin{CJK}{UTF8}{mj}使得对任意\end{CJK} $s \geqslant m$ \begin{CJK}{UTF8}{mj}有\end{CJK}
\end{enumerate}
$$
r\left(A^{s}\right)=r\left(A^{m}\right)
$$
\begin{CJK}{UTF8}{mj}这里\end{CJK} $r(A)$ \begin{CJK}{UTF8}{mj}表示矩阵\end{CJK} $A$ \begin{CJK}{UTF8}{mj}的秩\end{CJK}.

\begin{enumerate}
  \setcounter{enumi}{3}
  \item (15 \begin{CJK}{UTF8}{mj}分\end{CJK}) \begin{CJK}{UTF8}{mj}设矩阵\end{CJK}
\end{enumerate}
$$
A=\left(\begin{array}{cccc}
a_{1} & a_{2} & \cdots & a_{n} \\
a_{n} & a_{1} & \cdots & a_{n-1} \\
\vdots & \vdots & & \vdots \\
a_{2} & a_{3} & \cdots & a_{1}
\end{array}\right), P=\left(\begin{array}{ccccc}
0 & 1 & 0 & \cdots & 0 \\
0 & 0 & 1 & \cdots & 0 \\
\vdots & \vdots & \vdots & & \vdots \\
0 & 0 & 0 & \cdots & 1 \\
1 & 0 & 0 & \cdots & 0
\end{array}\right)
$$
\begin{CJK}{UTF8}{mj}其中\end{CJK} $P$ \begin{CJK}{UTF8}{mj}为\end{CJK} $n$ \begin{CJK}{UTF8}{mj}阶置换矩阵\end{CJK}, \begin{CJK}{UTF8}{mj}它的每一行每一列均只有一个元素为\end{CJK} 1 , \begin{CJK}{UTF8}{mj}其余元素均为\end{CJK} 0 .

(1) \begin{CJK}{UTF8}{mj}证明存在\end{CJK} $n-1$ \begin{CJK}{UTF8}{mj}次多项式\end{CJK} $f(\lambda)$, \begin{CJK}{UTF8}{mj}使得\end{CJK} $A=f(P)$;

(2) \begin{CJK}{UTF8}{mj}求矩阵\end{CJK} $A$ \begin{CJK}{UTF8}{mj}的所有特征值与特征向量\end{CJK};

(3) \begin{CJK}{UTF8}{mj}计算行列式\end{CJK} $|A|$.

\begin{enumerate}
  \setcounter{enumi}{4}
  \item (15 \begin{CJK}{UTF8}{mj}分\end{CJK}) \begin{CJK}{UTF8}{mj}设\end{CJK} $A \in \mathbb{R}^{m \times n}$ \begin{CJK}{UTF8}{mj}是\end{CJK} $m \times n$ \begin{CJK}{UTF8}{mj}阶实矩阵\end{CJK}, $b \in \mathbb{R}^{m}$ \begin{CJK}{UTF8}{mj}是\end{CJK} $m$ \begin{CJK}{UTF8}{mj}维实向量\end{CJK}, $A^{T}$ \begin{CJK}{UTF8}{mj}表示矩阵\end{CJK} $A$ \begin{CJK}{UTF8}{mj}的转置\end{CJK}. \begin{CJK}{UTF8}{mj}证明\end{CJK}:
\end{enumerate}
(1) \begin{CJK}{UTF8}{mj}线性方程组\end{CJK} $A x=b$ \begin{CJK}{UTF8}{mj}有解的充要条件是\end{CJK} $b$ \begin{CJK}{UTF8}{mj}与齐次线性方程组\end{CJK} $A x=0$ \begin{CJK}{UTF8}{mj}的解空间正交\end{CJK};

(2) \begin{CJK}{UTF8}{mj}若线性方程组\end{CJK} $A x=b$ \begin{CJK}{UTF8}{mj}无解\end{CJK}, \begin{CJK}{UTF8}{mj}则存在\end{CJK} $\widehat{x} \in \mathbb{R}^{n}$, \begin{CJK}{UTF8}{mj}使得对\end{CJK} $\forall x \in \mathbb{R}^{n}$, \begin{CJK}{UTF8}{mj}有\end{CJK}
$$
\|A \widehat{x}-b\| \leqslant\|A x-b\|,
$$
\begin{CJK}{UTF8}{mj}其中\end{CJK} $\|x\|=\sqrt{\langle x, x\rangle},\langle,$,$rangle 为 \mathbb{R}^{n}$ \begin{CJK}{UTF8}{mj}中是内积\end{CJK}.

\begin{enumerate}
  \setcounter{enumi}{5}
  \item ( 15 \begin{CJK}{UTF8}{mj}分\end{CJK}) \begin{CJK}{UTF8}{mj}设\end{CJK} $n$ \begin{CJK}{UTF8}{mj}阶实方阵\end{CJK} $A$ \begin{CJK}{UTF8}{mj}的秩\end{CJK} $r(A)=r$, \begin{CJK}{UTF8}{mj}证明\end{CJK}:
\end{enumerate}
(1) \begin{CJK}{UTF8}{mj}存在列满秩\end{CJK} $n \times r$ \begin{CJK}{UTF8}{mj}实矩阵\end{CJK} $B$ \begin{CJK}{UTF8}{mj}和行满秩\end{CJK} $r \times n$ \begin{CJK}{UTF8}{mj}实矩阵\end{CJK} $C$, \begin{CJK}{UTF8}{mj}使得\end{CJK} $A=B C$;

(2) \begin{CJK}{UTF8}{mj}若\end{CJK}
$$
\mathbb{R}^{n}=\left\{y \in \mathbb{R}^{n}: y=B x, x \in \mathbb{R}^{r}\right\} \oplus\left\{x \in \mathbb{R}^{n}: C x=0\right\},
$$
\begin{CJK}{UTF8}{mj}则矩阵\end{CJK} $C B$ \begin{CJK}{UTF8}{mj}可逆\end{CJK}, \begin{CJK}{UTF8}{mj}其中\end{CJK} $\oplus$ \begin{CJK}{UTF8}{mj}表示两个子空间的直和\end{CJK}.

\begin{enumerate}
  \setcounter{enumi}{6}
  \item ( 15 \begin{CJK}{UTF8}{mj}分\end{CJK}) \begin{CJK}{UTF8}{mj}设\end{CJK} $A$ \begin{CJK}{UTF8}{mj}为实对称正定矩阵\end{CJK}, $B$ \begin{CJK}{UTF8}{mj}为实对称矩阵\end{CJK}, $P^{T}$ \begin{CJK}{UTF8}{mj}表示矩阵的转置\end{CJK}, \begin{CJK}{UTF8}{mj}证明\end{CJK}:
\end{enumerate}
(1) \begin{CJK}{UTF8}{mj}存在可逆矩阵\end{CJK} $P$, \begin{CJK}{UTF8}{mj}使得\end{CJK} $P^{T}(A+B) P$ \begin{CJK}{UTF8}{mj}为对角矩阵\end{CJK};

(2) \begin{CJK}{UTF8}{mj}若\end{CJK} $B$ \begin{CJK}{UTF8}{mj}也是正定的\end{CJK}, \begin{CJK}{UTF8}{mj}则\end{CJK} $\frac{1}{2}(A+B)-2\left(A^{-1}+B^{-1}\right)^{-1}$ \begin{CJK}{UTF8}{mj}为半正定矩阵\end{CJK}. 7. (10 \begin{CJK}{UTF8}{mj}分\end{CJK}) \begin{CJK}{UTF8}{mj}设\end{CJK} $m_{A}(x), m_{B}(x)$ \begin{CJK}{UTF8}{mj}分别为\end{CJK} $n$ \begin{CJK}{UTF8}{mj}阶方阵\end{CJK} $A$ \begin{CJK}{UTF8}{mj}和\end{CJK} $B$ \begin{CJK}{UTF8}{mj}的最小多项式\end{CJK}, $\lambda_{A}(x)$ \begin{CJK}{UTF8}{mj}为矩阵\end{CJK} $A$ \begin{CJK}{UTF8}{mj}的特征多项式\end{CJK}, \begin{CJK}{UTF8}{mj}试\end{CJK} \begin{CJK}{UTF8}{mj}问当\end{CJK} $m_{A}(x), m_{B}(x)$ \begin{CJK}{UTF8}{mj}互素时\end{CJK}, $\lambda_{A}(B)$ \begin{CJK}{UTF8}{mj}是否可逆\end{CJK}? \begin{CJK}{UTF8}{mj}证明你的结论\end{CJK}.

\begin{enumerate}
  \setcounter{enumi}{8}
  \item ( 15 \begin{CJK}{UTF8}{mj}分\end{CJK}) \begin{CJK}{UTF8}{mj}设\end{CJK} $A$ \begin{CJK}{UTF8}{mj}为\end{CJK} $n$ \begin{CJK}{UTF8}{mj}阶实对称矩阵\end{CJK},
\end{enumerate}
$$
S_{A}=\left\{x \in \mathbb{R}^{n}: x^{T} A x=0\right\}
$$
\begin{CJK}{UTF8}{mj}这里\end{CJK} $x^{T}$ \begin{CJK}{UTF8}{mj}表示\end{CJK} $x$ \begin{CJK}{UTF8}{mj}的转置向量\end{CJK}, $B^{T}$ \begin{CJK}{UTF8}{mj}表示矩阵的转置\end{CJK}, \begin{CJK}{UTF8}{mj}证明\end{CJK}:

(1) \begin{CJK}{UTF8}{mj}若\end{CJK} $A$ \begin{CJK}{UTF8}{mj}为半正定矩阵\end{CJK}, \begin{CJK}{UTF8}{mj}则存在\end{CJK} $n$ \begin{CJK}{UTF8}{mj}阶实方阵\end{CJK} $B$, \begin{CJK}{UTF8}{mj}使得\end{CJK} $A=B^{T} B$;

(2) $S_{A}$ \begin{CJK}{UTF8}{mj}为\end{CJK} $\mathbb{R}^{n}$ \begin{CJK}{UTF8}{mj}的线性子空间的充要条件是\end{CJK} $A$ \begin{CJK}{UTF8}{mj}为半正定或半负正定矩阵矩阵\end{CJK}.

\begin{enumerate}
  \setcounter{enumi}{9}
  \item ( 20 \begin{CJK}{UTF8}{mj}分\end{CJK}) \begin{CJK}{UTF8}{mj}设\end{CJK} $\langle,$,$rangle 为 n$ \begin{CJK}{UTF8}{mj}维欧氏空间\end{CJK} $V$ \begin{CJK}{UTF8}{mj}中的内积\end{CJK}, \begin{CJK}{UTF8}{mj}对\end{CJK} $V$ \begin{CJK}{UTF8}{mj}中任意一个单位向量\end{CJK} $\eta$, \begin{CJK}{UTF8}{mj}定义\end{CJK} $V$ \begin{CJK}{UTF8}{mj}上的线性变换\end{CJK}
\end{enumerate}
$$
\mathcal{A}_{\eta}(\alpha)=\alpha-2\langle\eta, \alpha\rangle \eta
$$
\begin{CJK}{UTF8}{mj}证明\end{CJK}:

(1) $\mathcal{A}_{\eta}$ \begin{CJK}{UTF8}{mj}是正交变换\end{CJK};

(2) $\mathcal{A}_{\eta}^{2}=i d_{V}$, \begin{CJK}{UTF8}{mj}这里\end{CJK} $i d_{V}$ \begin{CJK}{UTF8}{mj}表示\end{CJK} $V$ \begin{CJK}{UTF8}{mj}上的单位变换\end{CJK};

(3) \begin{CJK}{UTF8}{mj}设\end{CJK} $\alpha, \beta$ \begin{CJK}{UTF8}{mj}是\end{CJK} $V$ \begin{CJK}{UTF8}{mj}中两个不同的单位向量\end{CJK}, \begin{CJK}{UTF8}{mj}这存在\end{CJK} $V$ \begin{CJK}{UTF8}{mj}中一个由单位向量\end{CJK} $\eta$ \begin{CJK}{UTF8}{mj}决定的正交变换\end{CJK} $\mathcal{A}_{\eta}$, \begin{CJK}{UTF8}{mj}使得\end{CJK}
$$
\mathcal{A}_{\eta}(\alpha)=\beta
$$

\begin{enumerate}
  \setcounter{enumi}{10}
  \item ( 20 \begin{CJK}{UTF8}{mj}分\end{CJK}) \begin{CJK}{UTF8}{mj}设\end{CJK} $V=\mathbb{R}^{n}$ \begin{CJK}{UTF8}{mj}是\end{CJK} $n$ \begin{CJK}{UTF8}{mj}为向量空间\end{CJK}, $A$ \begin{CJK}{UTF8}{mj}是\end{CJK} $n$ \begin{CJK}{UTF8}{mj}阶反对称方阵\end{CJK}, \begin{CJK}{UTF8}{mj}定义\end{CJK} $f: V \times V \rightarrow \mathbb{R}^{n}$ \begin{CJK}{UTF8}{mj}如下\end{CJK}:
\end{enumerate}
$$
f(\alpha, \beta)=\alpha^{T} A \beta,
$$
$\alpha^{T}$ \begin{CJK}{UTF8}{mj}表示\end{CJK} $\alpha$ \begin{CJK}{UTF8}{mj}的转置向量\end{CJK}, \begin{CJK}{UTF8}{mj}记\end{CJK} $N=\{\alpha \in V$ : \begin{CJK}{UTF8}{mj}对所有\end{CJK} $\beta \in V$ \begin{CJK}{UTF8}{mj}有\end{CJK} $f(\alpha, \beta)=0\}$, \begin{CJK}{UTF8}{mj}证明\end{CJK}:

(1) $N$ \begin{CJK}{UTF8}{mj}是\end{CJK} $V$ \begin{CJK}{UTF8}{mj}的线性子空间\end{CJK};

(2) \begin{CJK}{UTF8}{mj}矩阵\end{CJK} $A$ \begin{CJK}{UTF8}{mj}的秩\end{CJK} $r(A)=n-\operatorname{dim} N$, \begin{CJK}{UTF8}{mj}其中\end{CJK} $\operatorname{dim} N$ \begin{CJK}{UTF8}{mj}表示\end{CJK} $N$ \begin{CJK}{UTF8}{mj}的维数\end{CJK};

(3) $r(A)$ \begin{CJK}{UTF8}{mj}是偶数\end{CJK}.

\section{1. 湖南师范大学 2009 年研究生入学考试高等代数试题}
\begin{CJK}{UTF8}{mj}李扬\end{CJK}

\begin{CJK}{UTF8}{mj}微信公众号\end{CJK}: sxkyliyang

\begin{CJK}{UTF8}{mj}一\end{CJK}. (20 \begin{CJK}{UTF8}{mj}分\end{CJK})

\begin{enumerate}
  \item \begin{CJK}{UTF8}{mj}证明\end{CJK}:
\end{enumerate}
$$
\left(x^{m},(1+x)^{n}\right)=1
$$
\begin{CJK}{UTF8}{mj}其中\end{CJK} $m, n$ \begin{CJK}{UTF8}{mj}为任意正整数\end{CJK};

\begin{enumerate}
  \setcounter{enumi}{2}
  \item \begin{CJK}{UTF8}{mj}在实数域上分解因式\end{CJK}: $x^{6}+27$;

  \item \begin{CJK}{UTF8}{mj}设\end{CJK} $f(x)$ \begin{CJK}{UTF8}{mj}为一整系数多项式\end{CJK}, \begin{CJK}{UTF8}{mj}若\end{CJK} $f(x)-1$ \begin{CJK}{UTF8}{mj}有\end{CJK} 5 \begin{CJK}{UTF8}{mj}个不同整数根\end{CJK}, \begin{CJK}{UTF8}{mj}证明\end{CJK}: $f(x)-12$ \begin{CJK}{UTF8}{mj}无整数根\end{CJK}.

\end{enumerate}
\begin{CJK}{UTF8}{mj}二\end{CJK}. (15 \begin{CJK}{UTF8}{mj}分\end{CJK})

\begin{enumerate}
  \item \begin{CJK}{UTF8}{mj}计算行列式\end{CJK}:
\end{enumerate}
$$
\left|\begin{array}{llll}
1 & 2 & 3 & 4 \\
2 & 3 & 4 & 1 \\
3 & 4 & 1 & 2 \\
4 & 3 & 2 & 1
\end{array}\right|
$$

\begin{enumerate}
  \setcounter{enumi}{2}
  \item \begin{CJK}{UTF8}{mj}设\end{CJK} $A, B$ \begin{CJK}{UTF8}{mj}均为\end{CJK} $n$ \begin{CJK}{UTF8}{mj}阶方阵\end{CJK}, $E$ \begin{CJK}{UTF8}{mj}为\end{CJK} $n$ \begin{CJK}{UTF8}{mj}阶单位矩阵\end{CJK}, \begin{CJK}{UTF8}{mj}证明\end{CJK}:
\end{enumerate}
$$
|x E-A B|=|x E-B A| .
$$
\begin{CJK}{UTF8}{mj}三\end{CJK}. ( 20 \begin{CJK}{UTF8}{mj}分\end{CJK})

\begin{enumerate}
  \item \begin{CJK}{UTF8}{mj}证明\end{CJK}: \begin{CJK}{UTF8}{mj}线性方程组\end{CJK}:
\end{enumerate}
$$
A X=\beta
$$
( $A$ \begin{CJK}{UTF8}{mj}为\end{CJK} $n$ \begin{CJK}{UTF8}{mj}阶方阵\end{CJK}) \begin{CJK}{UTF8}{mj}对任意\end{CJK} $n$ \begin{CJK}{UTF8}{mj}维列向量\end{CJK} $\beta$ \begin{CJK}{UTF8}{mj}都有解的充要条件是\end{CJK} $A$ \begin{CJK}{UTF8}{mj}可逆\end{CJK};

\begin{enumerate}
  \setcounter{enumi}{2}
  \item \begin{CJK}{UTF8}{mj}若方程\end{CJK}
\end{enumerate}
$$
\left\{\begin{array}{l}
\lambda x_{1}+x_{2}+x_{3}=b_{1} \\
x_{1}+\lambda x_{2}+x_{3}=b_{2} \\
x_{1}+x_{2}+\lambda x_{3}=b_{3}
\end{array}\right.
$$
\begin{CJK}{UTF8}{mj}对任意的数\end{CJK} $b_{1}, b_{2}, b_{3}$ \begin{CJK}{UTF8}{mj}都有解\end{CJK}, \begin{CJK}{UTF8}{mj}求\end{CJK} $\lambda$ \begin{CJK}{UTF8}{mj}的值\end{CJK}.

\begin{CJK}{UTF8}{mj}四\end{CJK}. ( 20 \begin{CJK}{UTF8}{mj}分\end{CJK}) \begin{CJK}{UTF8}{mj}设\end{CJK} $A$ \begin{CJK}{UTF8}{mj}为实数域上的\end{CJK} $n$ \begin{CJK}{UTF8}{mj}阶方阵\end{CJK}, \begin{CJK}{UTF8}{mj}证明\end{CJK}:

\begin{enumerate}
  \item $\operatorname{rank}(A)=\operatorname{rank}\left(A^{\prime} A\right)$, \begin{CJK}{UTF8}{mj}其中\end{CJK} $A^{\prime}$ \begin{CJK}{UTF8}{mj}为\end{CJK} $A$ \begin{CJK}{UTF8}{mj}的转置\end{CJK};

  \item \begin{CJK}{UTF8}{mj}若\end{CJK} $A$ \begin{CJK}{UTF8}{mj}是秩为\end{CJK} $r$ \begin{CJK}{UTF8}{mj}的对称矩阵\end{CJK}, \begin{CJK}{UTF8}{mj}则存在一个可逆矩阵\end{CJK} $P$, \begin{CJK}{UTF8}{mj}使得\end{CJK} $A$ \begin{CJK}{UTF8}{mj}的\end{CJK} $r$ \begin{CJK}{UTF8}{mj}阶主子阵\end{CJK} $M$ \begin{CJK}{UTF8}{mj}满足\end{CJK}:

\end{enumerate}
$$
\left(P^{\prime} A P\right)=\left(\begin{array}{c}
E_{r} \\
R
\end{array}\right) \mathrm{M}\left(\begin{array}{ll}
E_{r} & R^{\prime}
\end{array}\right)
$$
\begin{CJK}{UTF8}{mj}其中\end{CJK} $E_{r}$ \begin{CJK}{UTF8}{mj}是\end{CJK} $r$ \begin{CJK}{UTF8}{mj}阶单位矩阵\end{CJK}, $R$ \begin{CJK}{UTF8}{mj}是\end{CJK} $(n-r) \times r$ \begin{CJK}{UTF8}{mj}矩阵\end{CJK}.

\begin{CJK}{UTF8}{mj}五\end{CJK}. (15 \begin{CJK}{UTF8}{mj}分\end{CJK})

\begin{enumerate}
  \item \begin{CJK}{UTF8}{mj}设二次型\end{CJK} $f\left(x_{1}, x_{2}\right)=a x_{1}^{2}+2 b x_{1} x_{2}+c x_{2}^{2}$, \begin{CJK}{UTF8}{mj}求二次型\end{CJK}
\end{enumerate}
$$
g\left(x_{1}, x_{2}\right)=\left|\begin{array}{ccc}
0 & x_{1} & x_{2} \\
-x_{1} & a & b \\
-x_{2} & b & c
\end{array}\right|
$$
\begin{CJK}{UTF8}{mj}的矩阵\end{CJK}, \begin{CJK}{UTF8}{mj}并证明\end{CJK} $f\left(x_{1}, x_{2}\right)$ \begin{CJK}{UTF8}{mj}是正定的当且仅当\end{CJK} $g\left(x_{1}, x_{2}\right)$ \begin{CJK}{UTF8}{mj}是正定的\end{CJK}.

\begin{enumerate}
  \setcounter{enumi}{2}
  \item \begin{CJK}{UTF8}{mj}设\end{CJK} $A=\left(a_{i j}\right)_{n \times n}$ \begin{CJK}{UTF8}{mj}是实对称矩阵\end{CJK}, \begin{CJK}{UTF8}{mj}证明\end{CJK}: \begin{CJK}{UTF8}{mj}二次型\end{CJK}
\end{enumerate}
$$
f\left(x_{1}, x_{2}, \cdots, x_{n}\right)=\left|\begin{array}{ccccc}
0 & x_{1} & x_{2} & \cdots & x_{n} \\
x_{1} & a_{11} & a_{12} & \cdots & a_{1 n} \\
x_{2} & a_{21} & a_{22} & \cdots & a_{2 n} \\
\vdots & \vdots & \vdots & & \vdots \\
x_{n} & a_{n 1} & a_{n 2} & \cdots & a_{n n}
\end{array}\right|
$$
\begin{CJK}{UTF8}{mj}的矩阵是\end{CJK} $A$ \begin{CJK}{UTF8}{mj}的伴随矩阵\end{CJK} $A^{*}$.

\begin{CJK}{UTF8}{mj}六\end{CJK}. (20 \begin{CJK}{UTF8}{mj}分\end{CJK}) \begin{CJK}{UTF8}{mj}设\end{CJK}
$$
\alpha_{1}=\left(\begin{array}{ll}
1 & 0 \\
0 & 0
\end{array}\right), \alpha_{2}=\left(\begin{array}{ll}
1 & 1 \\
0 & 0
\end{array}\right), \alpha_{3}=\left(\begin{array}{ll}
1 & 1 \\
1 & 0
\end{array}\right), \alpha_{4}=\left(\begin{array}{ll}
1 & 1 \\
1 & 1
\end{array}\right)
$$
\begin{CJK}{UTF8}{mj}是数域\end{CJK} $P$ \begin{CJK}{UTF8}{mj}上的线性空间\end{CJK} $V=P^{2 \times 2}$ \begin{CJK}{UTF8}{mj}的一个基\end{CJK}.

\begin{enumerate}
  \item \begin{CJK}{UTF8}{mj}求由\end{CJK} $V$ \begin{CJK}{UTF8}{mj}的基\end{CJK}
\end{enumerate}
$$
\varepsilon_{1}=\left(\begin{array}{ll}
1 & 0 \\
0 & 0
\end{array}\right), \varepsilon_{2}=\left(\begin{array}{ll}
0 & 1 \\
0 & 0
\end{array}\right), \varepsilon_{3}=\left(\begin{array}{ll}
0 & 0 \\
1 & 0
\end{array}\right), \varepsilon_{4}=\left(\begin{array}{ll}
0 & 0 \\
0 & 1
\end{array}\right)
$$
\begin{CJK}{UTF8}{mj}到基\end{CJK} $\alpha_{1}, \alpha_{2}, \alpha_{3}, \alpha_{4}$ \begin{CJK}{UTF8}{mj}的过渡矩阵\end{CJK};

\begin{enumerate}
  \setcounter{enumi}{2}
  \item \begin{CJK}{UTF8}{mj}求\end{CJK}
\end{enumerate}
$$
\beta=\left(\begin{array}{ll}
1 & 2 \\
3 & 4
\end{array}\right)
$$
\begin{CJK}{UTF8}{mj}在基\end{CJK} $\alpha_{1}, \alpha_{2}, \alpha_{3}, \alpha_{4}$ \begin{CJK}{UTF8}{mj}下的坐标\end{CJK}.

\begin{CJK}{UTF8}{mj}七\end{CJK}. ( 20 \begin{CJK}{UTF8}{mj}分\end{CJK}) \begin{CJK}{UTF8}{mj}设\end{CJK} $V$ \begin{CJK}{UTF8}{mj}是数域\end{CJK} $P$ \begin{CJK}{UTF8}{mj}上的\end{CJK} $n$ \begin{CJK}{UTF8}{mj}维线性空间\end{CJK}, $\alpha_{1}, \alpha_{2}, \cdots, \alpha_{n}$ \begin{CJK}{UTF8}{mj}是\end{CJK} $V$ \begin{CJK}{UTF8}{mj}的一个基\end{CJK}, $V_{1}$ \begin{CJK}{UTF8}{mj}是由\end{CJK} $\alpha_{1}+\alpha_{2}+\cdots+\alpha_{n}$ \begin{CJK}{UTF8}{mj}生成\end{CJK} \begin{CJK}{UTF8}{mj}的子空间\end{CJK},
$$
V_{2}=\left\{\sum_{i=1}^{n} k_{i} a_{i}=\mid \sum_{i=1}^{n} k_{i}=0, k_{i} \in P\right\}
$$

\begin{enumerate}
  \item \begin{CJK}{UTF8}{mj}证明\end{CJK}: $V_{2}$ \begin{CJK}{UTF8}{mj}是\end{CJK} $V$ \begin{CJK}{UTF8}{mj}的子空间\end{CJK};

  \item \begin{CJK}{UTF8}{mj}证明\end{CJK}: $V=V_{1} \oplus V_{2}$;

  \item \begin{CJK}{UTF8}{mj}若\end{CJK} $V$ \begin{CJK}{UTF8}{mj}上线性变换\end{CJK} $\mathscr{A}$ \begin{CJK}{UTF8}{mj}在基\end{CJK} $\alpha_{1}, \alpha_{2}, \cdots, \alpha_{n}$ \begin{CJK}{UTF8}{mj}下的矩阵是\end{CJK}:

\end{enumerate}
$$
A=\left(\begin{array}{ccccc}
a_{1} & a_{2} & a_{3} & \cdots & a_{n} \\
a_{n} & a_{1} & a_{2} & \cdots & a_{n-1} \\
a_{n-1} & a_{n} & a_{1} & \cdots & a_{n-2} \\
\vdots & \vdots & \vdots & & \vdots \\
a_{2} & a_{3} & a_{4} & \cdots & a_{1}
\end{array}\right)
$$
\begin{CJK}{UTF8}{mj}证明\end{CJK}: $V_{1}$ \begin{CJK}{UTF8}{mj}与\end{CJK} $V_{2}$ \begin{CJK}{UTF8}{mj}都是\end{CJK} $\mathscr{A}$ \begin{CJK}{UTF8}{mj}的不变子空间\end{CJK}.

\begin{CJK}{UTF8}{mj}八\end{CJK}. ( 20 \begin{CJK}{UTF8}{mj}分\end{CJK}) \begin{CJK}{UTF8}{mj}设\end{CJK} $V$ \begin{CJK}{UTF8}{mj}是\end{CJK} $n \geqslant 3$ \begin{CJK}{UTF8}{mj}维欧氏空间\end{CJK}, \begin{CJK}{UTF8}{mj}对于\end{CJK} $V$ \begin{CJK}{UTF8}{mj}中每个非零向量\end{CJK} $\alpha$, \begin{CJK}{UTF8}{mj}定义\end{CJK}:
$$
\begin{gathered}
\mathscr{A}_{\alpha}: V \rightarrow V, \\
\forall \xi \in V, \mathscr{A}_{\alpha}(\xi)=2 \frac{(\xi, \alpha)}{(\alpha, \alpha)} \alpha-\xi
\end{gathered}
$$
\begin{CJK}{UTF8}{mj}证明\end{CJK}:

\begin{enumerate}
  \item $\mathscr{A}_{\alpha}$ \begin{CJK}{UTF8}{mj}是正交变换\end{CJK};

  \item $\mathscr{A}_{\alpha}$ \begin{CJK}{UTF8}{mj}的特征值是\end{CJK} $-1(n-1$ \begin{CJK}{UTF8}{mj}重\end{CJK}) \begin{CJK}{UTF8}{mj}与\end{CJK} 1 ;

  \item \begin{CJK}{UTF8}{mj}若\end{CJK} $\alpha_{1}, \alpha_{2}, \cdots, \alpha_{n}$ \begin{CJK}{UTF8}{mj}是\end{CJK} $V$ \begin{CJK}{UTF8}{mj}的正交基\end{CJK}, \begin{CJK}{UTF8}{mj}则\end{CJK} $\mathscr{A}_{\alpha_{1}}+\mathscr{A}_{\alpha_{2}}+\cdots+\mathscr{A}_{\alpha_{n}}$ \begin{CJK}{UTF8}{mj}是\end{CJK} $V$ \begin{CJK}{UTF8}{mj}上的数乘变换\end{CJK}.

\end{enumerate}
\section{2. 湖南师范大学 2010 年研究生入学考试高等代数试题}
\begin{CJK}{UTF8}{mj}李扬\end{CJK}

\begin{CJK}{UTF8}{mj}微信公众号\end{CJK}: sxkyliyang

\begin{CJK}{UTF8}{mj}一\end{CJK}. \begin{CJK}{UTF8}{mj}简单说明题\end{CJK} (\begin{CJK}{UTF8}{mj}注\end{CJK}: \begin{CJK}{UTF8}{mj}不必对每题做大细致地推导\end{CJK}, \begin{CJK}{UTF8}{mj}只要写出大概步骤\end{CJK}. \begin{CJK}{UTF8}{mj}本大题共\end{CJK} 5 \begin{CJK}{UTF8}{mj}小题\end{CJK}, \begin{CJK}{UTF8}{mj}每题\end{CJK} 6 \begin{CJK}{UTF8}{mj}分\end{CJK}, \begin{CJK}{UTF8}{mj}共\end{CJK} 30 \begin{CJK}{UTF8}{mj}分\end{CJK})

\begin{enumerate}
  \item \begin{CJK}{UTF8}{mj}设\end{CJK} $n$ \begin{CJK}{UTF8}{mj}阶实方阵\end{CJK} $A, B$ \begin{CJK}{UTF8}{mj}满足\end{CJK}: $A B=0$, \begin{CJK}{UTF8}{mj}简略地说明一定有\end{CJK}:
\end{enumerate}
$$
\operatorname{rank}(A)+\operatorname{rank}(B) \leqslant n .
$$

\begin{enumerate}
  \setcounter{enumi}{2}
  \item \begin{CJK}{UTF8}{mj}设\end{CJK} $A, B$ \begin{CJK}{UTF8}{mj}为\end{CJK} $n$ \begin{CJK}{UTF8}{mj}阶正定矩阵\end{CJK}, \begin{CJK}{UTF8}{mj}请给出乘积\end{CJK} $A B$ \begin{CJK}{UTF8}{mj}仍是正定矩阵的充要条件\end{CJK}, \begin{CJK}{UTF8}{mj}并简略说明\end{CJK}.

  \item \begin{CJK}{UTF8}{mj}设正整数\end{CJK} $m$ \begin{CJK}{UTF8}{mj}与\end{CJK} $n$ \begin{CJK}{UTF8}{mj}为一奇一偶\end{CJK}, \begin{CJK}{UTF8}{mj}请简略地说明此时有\end{CJK}:

\end{enumerate}
$$
\left(x^{m}+1, x^{n}+1\right)=1 .
$$

\begin{enumerate}
  \setcounter{enumi}{4}
  \item \begin{CJK}{UTF8}{mj}设\end{CJK} $V$ \begin{CJK}{UTF8}{mj}是\end{CJK} $n$ \begin{CJK}{UTF8}{mj}维线性空间\end{CJK}, $W$ \begin{CJK}{UTF8}{mj}是\end{CJK} $V$ \begin{CJK}{UTF8}{mj}的非平凡子空间\end{CJK}, \begin{CJK}{UTF8}{mj}简略说明一定存在两个互不相同的非平凡子空间\end{CJK} $U_{1}, U_{2}$, \begin{CJK}{UTF8}{mj}使得\end{CJK}
\end{enumerate}
$$
V=W \oplus U_{1}=W \oplus U_{2} .
$$

\begin{enumerate}
  \setcounter{enumi}{5}
  \item \begin{CJK}{UTF8}{mj}设\end{CJK} $\mathscr{A}$ \begin{CJK}{UTF8}{mj}是\end{CJK} $n$ \begin{CJK}{UTF8}{mj}维线性空间\end{CJK} $V$ \begin{CJK}{UTF8}{mj}的线性变换\end{CJK}, \begin{CJK}{UTF8}{mj}请给出简略说明一定存在正整数\end{CJK} $m$, \begin{CJK}{UTF8}{mj}使得\end{CJK}:
\end{enumerate}
$$
\mathscr{A}^{2 m} V=\mathscr{A}^{m} V .
$$
\begin{CJK}{UTF8}{mj}二\end{CJK}. \begin{CJK}{UTF8}{mj}计算题\end{CJK} (\begin{CJK}{UTF8}{mj}本大题共\end{CJK} 5 \begin{CJK}{UTF8}{mj}小题\end{CJK}, \begin{CJK}{UTF8}{mj}共\end{CJK} 60 \begin{CJK}{UTF8}{mj}分\end{CJK})

\begin{enumerate}
  \item (15 \begin{CJK}{UTF8}{mj}分\end{CJK}) \begin{CJK}{UTF8}{mj}设矩阵\end{CJK} $A=\left(\alpha_{1}, \alpha_{2}, \alpha_{3}, \alpha_{4}\right)$, \begin{CJK}{UTF8}{mj}其中\end{CJK} $\alpha_{1}, \alpha_{2}, \alpha_{3}, \alpha_{4}$ \begin{CJK}{UTF8}{mj}均为\end{CJK} $n$ \begin{CJK}{UTF8}{mj}维列向量\end{CJK}, \begin{CJK}{UTF8}{mj}且\end{CJK} $\alpha_{1}, \alpha_{2}, \alpha_{4}$ \begin{CJK}{UTF8}{mj}线性无关\end{CJK}, $\alpha_{1}-3 \alpha_{2}+2 \alpha_{3}=\alpha_{4}$. \begin{CJK}{UTF8}{mj}如果\end{CJK} $\beta=\alpha_{1}+2 \alpha_{2}+3 \alpha_{3}+4 \alpha_{4}$, \begin{CJK}{UTF8}{mj}试求线性方程组\end{CJK}
\end{enumerate}
$$
A X=\beta
$$
\begin{CJK}{UTF8}{mj}的通解\end{CJK}.

\begin{enumerate}
  \setcounter{enumi}{2}
  \item ( 15 \begin{CJK}{UTF8}{mj}分\end{CJK}) \begin{CJK}{UTF8}{mj}在多项式线性空间\end{CJK} $V=P[x]_{n}$ \begin{CJK}{UTF8}{mj}中\end{CJK}, \begin{CJK}{UTF8}{mj}规定线性变换\end{CJK} $\mathscr{A}$ \begin{CJK}{UTF8}{mj}为\end{CJK}
\end{enumerate}
$$
\mathscr{A}(f(x))=x f^{\prime}(x)-f(x), \forall f(x) \in V .
$$
\begin{CJK}{UTF8}{mj}试求出\end{CJK} $\mathscr{A}$ \begin{CJK}{UTF8}{mj}的值域\end{CJK} $\mathscr{A} V$ \begin{CJK}{UTF8}{mj}以及\end{CJK} $\mathscr{A} V$ \begin{CJK}{UTF8}{mj}的一个基\end{CJK}.

\begin{enumerate}
  \setcounter{enumi}{3}
  \item (10 \begin{CJK}{UTF8}{mj}分\end{CJK}) \begin{CJK}{UTF8}{mj}计算行列式\end{CJK}:
\end{enumerate}
$$
D=\left|\begin{array}{cccc}
a+x_{1} & a+x_{1}^{2} & \cdots & a+x_{1}^{n} \\
a+x_{2} & a+x_{2}^{2} & \cdots & a+x_{2}^{n} \\
\vdots & \vdots & & \vdots \\
a+x_{n} & a+x_{n}^{2} & \cdots & a+x_{n}^{n}
\end{array}\right|
$$

\begin{enumerate}
  \setcounter{enumi}{4}
  \item ( 10 \begin{CJK}{UTF8}{mj}分\end{CJK}) \begin{CJK}{UTF8}{mj}设\end{CJK} $V$ \begin{CJK}{UTF8}{mj}是一个\end{CJK} $n$ \begin{CJK}{UTF8}{mj}维欧氏空间\end{CJK}, $\varepsilon_{1}, \varepsilon_{2}, \cdots, \varepsilon_{n}$ \begin{CJK}{UTF8}{mj}是\end{CJK} $V$ \begin{CJK}{UTF8}{mj}的自然基\end{CJK}, \begin{CJK}{UTF8}{mj}即\end{CJK} $\varepsilon_{1}=(0, \cdots, 0,1,0, \cdots, 0)^{\prime}($ \begin{CJK}{UTF8}{mj}第\end{CJK} $i$ \begin{CJK}{UTF8}{mj}个分量为\end{CJK} 1 , \begin{CJK}{UTF8}{mj}其余分量为\end{CJK} 0$)$, \begin{CJK}{UTF8}{mj}向量\end{CJK} $\xi=(1,2,3, \cdots, n)^{\prime}$ \begin{CJK}{UTF8}{mj}和\end{CJK} $\eta=(1,1, \cdots, 1)^{\prime}$, \begin{CJK}{UTF8}{mj}且\end{CJK} $\mathscr{A}$ \begin{CJK}{UTF8}{mj}是一个由\end{CJK} $\eta$ \begin{CJK}{UTF8}{mj}决定的\end{CJK} \begin{CJK}{UTF8}{mj}镜面反射\end{CJK}, \begin{CJK}{UTF8}{mj}即\end{CJK}
\end{enumerate}
$$
\mathscr{A} \alpha=\alpha-\frac{2(\alpha, \eta)}{(\eta, \eta)} \eta, \forall \alpha \in V .
$$
\begin{CJK}{UTF8}{mj}写出\end{CJK} $\mathscr{A}$ \begin{CJK}{UTF8}{mj}在此基\end{CJK} $\varepsilon_{1}, \varepsilon_{2}, \cdots, \varepsilon_{n}$ \begin{CJK}{UTF8}{mj}下的矩阵\end{CJK}, \begin{CJK}{UTF8}{mj}并求出向量\end{CJK} $\xi$ \begin{CJK}{UTF8}{mj}在\end{CJK} $\mathscr{A}$ \begin{CJK}{UTF8}{mj}下的像\end{CJK} $\mathscr{A} \xi$.

\begin{CJK}{UTF8}{mj}三\end{CJK}. \begin{CJK}{UTF8}{mj}证明题\end{CJK} (\begin{CJK}{UTF8}{mj}本大题共\end{CJK} 5 \begin{CJK}{UTF8}{mj}小题\end{CJK}, \begin{CJK}{UTF8}{mj}共\end{CJK} 60 \begin{CJK}{UTF8}{mj}分\end{CJK})

\begin{enumerate}
  \item ( 10 \begin{CJK}{UTF8}{mj}分\end{CJK}) \begin{CJK}{UTF8}{mj}设\end{CJK} $A$ \begin{CJK}{UTF8}{mj}在\end{CJK} $n \times n$ \begin{CJK}{UTF8}{mj}矩阵\end{CJK}, \begin{CJK}{UTF8}{mj}满足\end{CJK}: $A^{2}=E$. \begin{CJK}{UTF8}{mj}证明\end{CJK}:
\end{enumerate}
$$
\operatorname{rank}(A+E)+\operatorname{rank}(A-E)=n
$$

\begin{enumerate}
  \setcounter{enumi}{2}
  \item ( 10 \begin{CJK}{UTF8}{mj}分\end{CJK}) \begin{CJK}{UTF8}{mj}设\end{CJK} $V_{1}$ \begin{CJK}{UTF8}{mj}是有限维欧氏空间\end{CJK} $V$ \begin{CJK}{UTF8}{mj}的子空间\end{CJK}, $V_{1}^{\perp}$ \begin{CJK}{UTF8}{mj}是\end{CJK} $V_{1}$ \begin{CJK}{UTF8}{mj}的正交补\end{CJK}, \begin{CJK}{UTF8}{mj}即\end{CJK} $V=V_{1} \oplus V_{1}^{\perp}$, \begin{CJK}{UTF8}{mj}定义\end{CJK} $V$ \begin{CJK}{UTF8}{mj}到\end{CJK} $V_{1}$ \begin{CJK}{UTF8}{mj}的投\end{CJK} \begin{CJK}{UTF8}{mj}影变换\end{CJK} $\mathscr{A}$ \begin{CJK}{UTF8}{mj}如下\end{CJK}:
\end{enumerate}
$$
\forall x=x_{1}+x_{2} \in V, x_{1} \in V_{1}, x_{2} \in V_{1}^{\perp}, \mathscr{A}(x)=x
$$
\begin{CJK}{UTF8}{mj}证明\end{CJK}:

(1) $\mathscr{A}$ \begin{CJK}{UTF8}{mj}是\end{CJK} $V$ \begin{CJK}{UTF8}{mj}上的线性变换\end{CJK};

(2) $\mathscr{A}$ \begin{CJK}{UTF8}{mj}是对称变换且\end{CJK} $\mathscr{A}^{2}=\mathscr{A}$.

\begin{enumerate}
  \setcounter{enumi}{3}
  \item (10 \begin{CJK}{UTF8}{mj}分\end{CJK}) \begin{CJK}{UTF8}{mj}设\end{CJK} $\mathscr{T}$ \begin{CJK}{UTF8}{mj}是\end{CJK} $n$ \begin{CJK}{UTF8}{mj}维线性空间\end{CJK} $V$ \begin{CJK}{UTF8}{mj}的线性变换且\end{CJK} $\mathscr{T}^{n-1} \neq 0, \mathscr{T}^{n}=0$, \begin{CJK}{UTF8}{mj}证明\end{CJK}: \begin{CJK}{UTF8}{mj}存在\end{CJK} $V$ \begin{CJK}{UTF8}{mj}的基\end{CJK} $\alpha_{1}, \alpha_{2}, \cdots, \alpha_{n}$ , \begin{CJK}{UTF8}{mj}使得\end{CJK} $\mathscr{T}$ \begin{CJK}{UTF8}{mj}在此基下的矩阵为\end{CJK}
\end{enumerate}
$$
A=\left(\begin{array}{cc}
O & O \\
E_{n-1} & O
\end{array}\right)
$$

\begin{enumerate}
  \setcounter{enumi}{4}
  \item ( 15 \begin{CJK}{UTF8}{mj}分\end{CJK}) \begin{CJK}{UTF8}{mj}设\end{CJK} $\mathbb{F}[x]$ \begin{CJK}{UTF8}{mj}表示数域\end{CJK} $\mathbb{F}$ \begin{CJK}{UTF8}{mj}上的全体多项式集合\end{CJK}, $c$ \begin{CJK}{UTF8}{mj}是\end{CJK} $\mathbb{F}[x]$ \begin{CJK}{UTF8}{mj}中的某一非零多项式的根\end{CJK}, \begin{CJK}{UTF8}{mj}令\end{CJK}
\end{enumerate}
$$
I=\{f(x) \in \mathbb{F}[x] \mid f(c)=0\} .
$$
\begin{CJK}{UTF8}{mj}证明\end{CJK}:

(1) \begin{CJK}{UTF8}{mj}在\end{CJK} $I$ \begin{CJK}{UTF8}{mj}中存在这样的多项式\end{CJK} $p(x)$, \begin{CJK}{UTF8}{mj}使得\end{CJK} $I$ \begin{CJK}{UTF8}{mj}中每个多项式\end{CJK} $f(x)$ \begin{CJK}{UTF8}{mj}都有\end{CJK} $p(x) \mid f(x)$;

(2) $p(x)$ \begin{CJK}{UTF8}{mj}是\end{CJK} $\mathbb{F}[x]$ \begin{CJK}{UTF8}{mj}中不可约多项式\end{CJK}.

\begin{enumerate}
  \setcounter{enumi}{5}
  \item (15 \begin{CJK}{UTF8}{mj}分\end{CJK}) \begin{CJK}{UTF8}{mj}设\end{CJK} $\beta_{0}$ \begin{CJK}{UTF8}{mj}是线性方程组\end{CJK} $A X=b$ \begin{CJK}{UTF8}{mj}的一个解向量\end{CJK}, \begin{CJK}{UTF8}{mj}其中\end{CJK} $b \neq 0$, \begin{CJK}{UTF8}{mj}而\end{CJK} $\alpha_{1}, \alpha_{2}, \cdots, \alpha_{n-r}$ \begin{CJK}{UTF8}{mj}是\end{CJK} $A X=0$ \begin{CJK}{UTF8}{mj}的一\end{CJK} \begin{CJK}{UTF8}{mj}个基础解系\end{CJK}, \begin{CJK}{UTF8}{mj}证明\end{CJK}:
\end{enumerate}
(1)
$$
\beta_{0}, \beta_{1}=\beta_{0}+\alpha_{1}, \beta_{2}=\beta_{0}+\alpha_{2}, \cdots, \beta_{n-r}=\beta_{0}+\alpha_{n-r}
$$
\begin{CJK}{UTF8}{mj}是方程组\end{CJK} $A X=b$ \begin{CJK}{UTF8}{mj}的\end{CJK} $n-r+1$ \begin{CJK}{UTF8}{mj}个线性无关的解向量\end{CJK};

(2) $A X=b$ \begin{CJK}{UTF8}{mj}的任意解向量\end{CJK} $\beta$ \begin{CJK}{UTF8}{mj}可表为\end{CJK}:
$$
\beta=k_{0} \beta_{0}+k_{1} \beta_{1}+\cdots+k_{n-r} \beta_{n-r}
$$
\begin{CJK}{UTF8}{mj}其中\end{CJK} $\sum_{i=0}^{n-r} k_{i}=1$.

\section{3. 湖南师范大学 2011 年研究生入学考试高等代数试题}
\begin{CJK}{UTF8}{mj}李扬\end{CJK}

\begin{CJK}{UTF8}{mj}微信公众号\end{CJK}: sxkyliyang

\begin{CJK}{UTF8}{mj}一\end{CJK}. \begin{CJK}{UTF8}{mj}简答题\end{CJK} (\begin{CJK}{UTF8}{mj}要求考生对本大题中的每个小题给出答案\end{CJK}, \begin{CJK}{UTF8}{mj}并做简略的说明\end{CJK}, \begin{CJK}{UTF8}{mj}本大题共\end{CJK} 5 \begin{CJK}{UTF8}{mj}小题\end{CJK}, \begin{CJK}{UTF8}{mj}每题\end{CJK} 6 \begin{CJK}{UTF8}{mj}分\end{CJK}, \begin{CJK}{UTF8}{mj}共\end{CJK} 30 \begin{CJK}{UTF8}{mj}分\end{CJK})

\begin{enumerate}
  \item \begin{CJK}{UTF8}{mj}若\end{CJK} $f(x)$ \begin{CJK}{UTF8}{mj}与\end{CJK} $g(x)$ \begin{CJK}{UTF8}{mj}的最大公因式为\end{CJK} $d(x)$, \begin{CJK}{UTF8}{mj}则\end{CJK} $f\left(x^{2}\right)$ \begin{CJK}{UTF8}{mj}与\end{CJK} $g\left(x^{2}\right)$ \begin{CJK}{UTF8}{mj}的最大公因式是否必为\end{CJK} $d\left(x^{2}\right)$ ?

  \item \begin{CJK}{UTF8}{mj}若\end{CJK} $A$ \begin{CJK}{UTF8}{mj}和\end{CJK} $B$ \begin{CJK}{UTF8}{mj}都是\end{CJK} $n$ \begin{CJK}{UTF8}{mj}阶正定矩阵\end{CJK}, \begin{CJK}{UTF8}{mj}则\end{CJK} $A+B, A B, A^{-1}$ \begin{CJK}{UTF8}{mj}中哪些必是正定矩阵\end{CJK}?

  \item \begin{CJK}{UTF8}{mj}一个向量组的任何一个线性无关组是否必可扩充为它的一个极大线性无关组\end{CJK}?

  \item \begin{CJK}{UTF8}{mj}下列\end{CJK} $\mathbb{R}^{2}$ \begin{CJK}{UTF8}{mj}上的线性变换\end{CJK} $\mathscr{A}$ \begin{CJK}{UTF8}{mj}是否可以对角化\end{CJK}?

\end{enumerate}
(1) $\mathscr{A}(x, y)=(x-2 y, x-y)$;

(2) $\mathscr{A}(x, y)=(x+y, 2 y)$.

\begin{enumerate}
  \setcounter{enumi}{5}
  \item \begin{CJK}{UTF8}{mj}设\end{CJK} $\mathscr{A}$ \begin{CJK}{UTF8}{mj}为数域\end{CJK} $\mathbb{F}$ \begin{CJK}{UTF8}{mj}上的\end{CJK} $n$ \begin{CJK}{UTF8}{mj}维线性空间\end{CJK} $V$ \begin{CJK}{UTF8}{mj}上的线性变换\end{CJK}, $V_{1}$ \begin{CJK}{UTF8}{mj}和\end{CJK} $V_{2}$ \begin{CJK}{UTF8}{mj}为\end{CJK} $V$ \begin{CJK}{UTF8}{mj}的任意两个子空间\end{CJK}, \begin{CJK}{UTF8}{mj}问\end{CJK}
\end{enumerate}
$$
\mathscr{A}\left(V_{1} \cap V_{2}\right)=\mathscr{A}\left(V_{1}\right) \cap \mathscr{A}\left(V_{2}\right)
$$
\begin{CJK}{UTF8}{mj}是否成立\end{CJK}?

\begin{CJK}{UTF8}{mj}二\end{CJK}. \begin{CJK}{UTF8}{mj}解答题\end{CJK} (\begin{CJK}{UTF8}{mj}本大题共\end{CJK} 8 \begin{CJK}{UTF8}{mj}小题\end{CJK}, \begin{CJK}{UTF8}{mj}每题\end{CJK} 15 \begin{CJK}{UTF8}{mj}分\end{CJK}, \begin{CJK}{UTF8}{mj}共\end{CJK} 120 \begin{CJK}{UTF8}{mj}分\end{CJK})

\begin{enumerate}
  \item \begin{CJK}{UTF8}{mj}计算\end{CJK} $n(n \geqslant 2)$ \begin{CJK}{UTF8}{mj}阶行列式\end{CJK}
\end{enumerate}
$$
D_{n}=\left|\begin{array}{ccccccc}
\cos \alpha & 1 & 0 & 0 & \cdots & 0 & 0 \\
1 & 2 \cos \alpha & 1 & 0 & \cdots & 0 & 0 \\
0 & 1 & 2 \cos \alpha & 1 & \cdots & 0 & 0 \\
0 & 0 & 1 & 2 \cos \alpha & \cdots & 0 & 0 \\
\vdots & \vdots & \vdots & \vdots & & \vdots & \vdots \\
0 & 0 & 0 & 0 & \cdots & 2 \cos \alpha & 1 \\
0 & 0 & 0 & 0 & \cdots & 1 & 2 \cos \alpha
\end{array}\right| .
$$

\begin{enumerate}
  \setcounter{enumi}{2}
  \item \begin{CJK}{UTF8}{mj}已知数域\end{CJK} $P$ \begin{CJK}{UTF8}{mj}上的矩阵\end{CJK}
\end{enumerate}
$$
A=\left(\begin{array}{rr}
1 & -1 \\
1 & -1 \\
1 & -1
\end{array}\right)
$$
\begin{CJK}{UTF8}{mj}令\end{CJK} $S(A)=\left\{B \in P^{2 \times 3} \mid A B=0\right\}$. \begin{CJK}{UTF8}{mj}证明\end{CJK}: $S(A)$ \begin{CJK}{UTF8}{mj}是矩阵空间\end{CJK} $P^{2 \times 3}$ \begin{CJK}{UTF8}{mj}的一个子空间\end{CJK}, \begin{CJK}{UTF8}{mj}并求\end{CJK} $S(A)$ \begin{CJK}{UTF8}{mj}的维数和一组\end{CJK} \begin{CJK}{UTF8}{mj}基\end{CJK}.

\begin{enumerate}
  \setcounter{enumi}{3}
  \item \begin{CJK}{UTF8}{mj}用正交线性变换化下列二次型为标准型\end{CJK}:
\end{enumerate}
$$
f\left(x_{1}, x_{2}, x_{3}\right)=2 x_{1}^{2}+5 x_{2}^{2}+5 x_{3}^{2}+4 x_{1} x_{2}-4 x_{1} x_{3}-8 x_{2} x_{3} .
$$

\begin{enumerate}
  \setcounter{enumi}{4}
  \item \begin{CJK}{UTF8}{mj}已知实数\end{CJK} $a_{1}, a_{2}, \cdots, a_{n}$ \begin{CJK}{UTF8}{mj}满足\end{CJK} $\sum_{i=1}^{n} a_{i}=0$, \begin{CJK}{UTF8}{mj}令\end{CJK}
\end{enumerate}
$$
A=\left(\begin{array}{cccc}
a_{1}^{2}+1 & a_{1} a_{2}+1 & \cdots & a_{1} a_{n}+1 \\
a_{2} a_{1}+1 & a_{2}^{2}+1 & \cdots & a_{2} a_{n}+1 \\
\vdots & \vdots & & \vdots \\
a_{n} a_{1}+1 & a_{n} a_{2}+1 & \cdots & a_{n}^{2}+1
\end{array}\right) .
$$
(1) \begin{CJK}{UTF8}{mj}证明\end{CJK}: \begin{CJK}{UTF8}{mj}存在一个\end{CJK} $n \times 2$ \begin{CJK}{UTF8}{mj}矩阵\end{CJK} $B$, \begin{CJK}{UTF8}{mj}使得\end{CJK} $A=B B^{\prime}$;

(2) \begin{CJK}{UTF8}{mj}求\end{CJK} $n$ \begin{CJK}{UTF8}{mj}阶实矩阵\end{CJK} $A$ \begin{CJK}{UTF8}{mj}的特征值\end{CJK}. 5. \begin{CJK}{UTF8}{mj}设\end{CJK} $V$ \begin{CJK}{UTF8}{mj}是一个\end{CJK} $n$ \begin{CJK}{UTF8}{mj}维欧氏空间\end{CJK}, $V_{1}, V_{2}$ \begin{CJK}{UTF8}{mj}都是\end{CJK} $V$ \begin{CJK}{UTF8}{mj}的子空间\end{CJK}, \begin{CJK}{UTF8}{mj}而且\end{CJK}
$$
\operatorname{dim}\left(V_{1}\right)<\operatorname{dim}\left(V_{2}\right)
$$
\begin{CJK}{UTF8}{mj}证明\end{CJK}: $V_{2}$ \begin{CJK}{UTF8}{mj}中存在非零向量\end{CJK} $\eta$, \begin{CJK}{UTF8}{mj}使得\end{CJK} $\forall \alpha \in V_{1}$ \begin{CJK}{UTF8}{mj}有\end{CJK} $(\alpha, \eta)=0$.

\section{4. 湖南师范大学 2012 年研究生入学考试高等代数试题}
\begin{CJK}{UTF8}{mj}李扬\end{CJK}

\begin{CJK}{UTF8}{mj}微信公众号\end{CJK}: sxkyliyang

\begin{CJK}{UTF8}{mj}一\end{CJK}. \begin{CJK}{UTF8}{mj}填空题\end{CJK} (\begin{CJK}{UTF8}{mj}本大题共\end{CJK} 5 \begin{CJK}{UTF8}{mj}小题\end{CJK}, \begin{CJK}{UTF8}{mj}每题\end{CJK} 6\begin{CJK}{UTF8}{mj}分\end{CJK}, \begin{CJK}{UTF8}{mj}共\end{CJK} 30 \begin{CJK}{UTF8}{mj}分\end{CJK})

\begin{enumerate}
  \item \begin{CJK}{UTF8}{mj}若\end{CJK} $x-1$ \begin{CJK}{UTF8}{mj}除多项式\end{CJK} $f(x)$ \begin{CJK}{UTF8}{mj}的余式是\end{CJK} $3, x-2$ \begin{CJK}{UTF8}{mj}除\end{CJK} $f(x)$ \begin{CJK}{UTF8}{mj}的余式是\end{CJK} 4 , \begin{CJK}{UTF8}{mj}则\end{CJK} $x^{2}-3 x+2$ \begin{CJK}{UTF8}{mj}除\end{CJK} $f(x)$ \begin{CJK}{UTF8}{mj}的余式是\end{CJK}

  \item \begin{CJK}{UTF8}{mj}设\end{CJK} $A_{i j}$ \begin{CJK}{UTF8}{mj}是行列式\end{CJK} $D=\left|\begin{array}{llll}1 & 0 & x & 1 \\ 0 & 3 & y & 0 \\ 2 & 0 & z & 0 \\ 2 & 0 & 1 & 2\end{array}\right|$ \begin{CJK}{UTF8}{mj}的第\end{CJK} $i$ \begin{CJK}{UTF8}{mj}行第\end{CJK} $j$ \begin{CJK}{UTF8}{mj}列元素的代数余子式\end{CJK}, $D=670$, \begin{CJK}{UTF8}{mj}则\end{CJK} $6 A_{41}+2 A_{43}+6 A_{44}=$

  \item \begin{CJK}{UTF8}{mj}设向量组\end{CJK} $\alpha_{1}, \alpha_{2}, \alpha_{3}$ \begin{CJK}{UTF8}{mj}线性无关\end{CJK}, \begin{CJK}{UTF8}{mj}当\end{CJK} $t$ \begin{CJK}{UTF8}{mj}满足\end{CJK} \begin{CJK}{UTF8}{mj}时\end{CJK}, $\alpha_{1}+t \alpha_{2}, \alpha_{2}+t \alpha_{3}, \alpha_{3}+t \alpha_{1}$ \begin{CJK}{UTF8}{mj}也线性无关\end{CJK}.

  \item \begin{CJK}{UTF8}{mj}实二次型\end{CJK} $f\left(x_{1}, x_{2}, x_{3}\right)=x_{1} x_{2}+x_{2} x_{3}+x_{3} x_{1}$ \begin{CJK}{UTF8}{mj}的秩是\end{CJK} , \begin{CJK}{UTF8}{mj}符号差是\end{CJK}

  \item \begin{CJK}{UTF8}{mj}若矩阵\end{CJK} $A=\left(\begin{array}{ccc}1 & y & -1 \\ y & x & 0 \\ -1 & 0 & 1\end{array}\right)$ \begin{CJK}{UTF8}{mj}和\end{CJK} $B=\left(\begin{array}{lll}0 & 0 & 0 \\ 0 & 1 & 0 \\ 0 & 0 & 2\end{array}\right)$ \begin{CJK}{UTF8}{mj}相似\end{CJK}, \begin{CJK}{UTF8}{mj}则\end{CJK} $x=$ $y=$

\end{enumerate}
\begin{CJK}{UTF8}{mj}二\end{CJK}. \begin{CJK}{UTF8}{mj}计算题\end{CJK} (\begin{CJK}{UTF8}{mj}本大题共\end{CJK} 5 \begin{CJK}{UTF8}{mj}小题\end{CJK}, \begin{CJK}{UTF8}{mj}每题\end{CJK} 15 \begin{CJK}{UTF8}{mj}分\end{CJK}, \begin{CJK}{UTF8}{mj}共\end{CJK} 75 \begin{CJK}{UTF8}{mj}分\end{CJK})

\begin{enumerate}
  \item \begin{CJK}{UTF8}{mj}当\end{CJK} $a, b$ \begin{CJK}{UTF8}{mj}满足什么条件时\end{CJK}, \begin{CJK}{UTF8}{mj}多项式\end{CJK}
\end{enumerate}
$$
f(x)=x^{3}+3 a x+2 b
$$
\begin{CJK}{UTF8}{mj}有重根\end{CJK}?

\begin{enumerate}
  \setcounter{enumi}{2}
  \item \begin{CJK}{UTF8}{mj}计算\end{CJK} $n(n \geqslant 2)$ \begin{CJK}{UTF8}{mj}行列式\end{CJK}:
\end{enumerate}
$$
D=\left|\begin{array}{ccccc}
1 & 2 & 2 & \cdots & 2 \\
2 & 2 & 2 & \cdots & 2 \\
2 & 2 & 3 & \cdots & 2 \\
\vdots & \vdots & \vdots & & \vdots \\
2 & 2 & 2 & \cdots & n
\end{array}\right|
$$

\begin{enumerate}
  \setcounter{enumi}{3}
  \item \begin{CJK}{UTF8}{mj}设\end{CJK}
\end{enumerate}
$$
J=\left(\begin{array}{cccc}
1 & 1 & \cdots & 1 \\
1 & 1 & \cdots & 1 \\
\vdots & \vdots & & \vdots \\
1 & 1 & \cdots & 1
\end{array}\right)
$$
\begin{CJK}{UTF8}{mj}是\end{CJK} $n$ \begin{CJK}{UTF8}{mj}阶全\end{CJK} 1 \begin{CJK}{UTF8}{mj}矩阵\end{CJK}, \begin{CJK}{UTF8}{mj}其中\end{CJK} $k$ \begin{CJK}{UTF8}{mj}为一个数\end{CJK}.

(1) \begin{CJK}{UTF8}{mj}当\end{CJK} $k$ \begin{CJK}{UTF8}{mj}为何值时\end{CJK}, \begin{CJK}{UTF8}{mj}有\end{CJK} $A^{2}=k A$ \begin{CJK}{UTF8}{mj}成立\end{CJK};

(2) \begin{CJK}{UTF8}{mj}对\end{CJK} (1) \begin{CJK}{UTF8}{mj}中所求得的数\end{CJK} $k$, \begin{CJK}{UTF8}{mj}计算\end{CJK} $A$ \begin{CJK}{UTF8}{mj}的特征值\end{CJK}.

\begin{enumerate}
  \setcounter{enumi}{4}
  \item \begin{CJK}{UTF8}{mj}已知线性空间\end{CJK} $P^{3}$ \begin{CJK}{UTF8}{mj}上的线性变换\end{CJK}
\end{enumerate}
$$
\mathscr{A}(x, y, z)=(x+2 y-z, y+z, x+y-2 z) .
$$
\begin{CJK}{UTF8}{mj}求\end{CJK} $\mathscr{A}$ \begin{CJK}{UTF8}{mj}的值域\end{CJK} $\mathscr{A} P^{3}$ \begin{CJK}{UTF8}{mj}与核\end{CJK} $\mathscr{A}^{-1}(0)$ \begin{CJK}{UTF8}{mj}的基与维数\end{CJK}.

\begin{enumerate}
  \setcounter{enumi}{5}
  \item \begin{CJK}{UTF8}{mj}设\end{CJK} $\alpha_{1}, \alpha_{2}, \alpha_{3}$ \begin{CJK}{UTF8}{mj}是\end{CJK} 3 \begin{CJK}{UTF8}{mj}维欧氏空间\end{CJK} $V$ \begin{CJK}{UTF8}{mj}的一组基\end{CJK}, \begin{CJK}{UTF8}{mj}其度量矩阵为\end{CJK}
\end{enumerate}
$$
A=\left(\begin{array}{ccc}
1 & -1 & 2 \\
-1 & 2 & -1 \\
2 & -1 & 6
\end{array}\right)
$$
(1) \begin{CJK}{UTF8}{mj}求\end{CJK} $\beta=\alpha_{1}+\alpha_{2}$ \begin{CJK}{UTF8}{mj}的长度\end{CJK};

(2) \begin{CJK}{UTF8}{mj}求参数\end{CJK} $\lambda$ \begin{CJK}{UTF8}{mj}的值\end{CJK}, \begin{CJK}{UTF8}{mj}使得\end{CJK} $\gamma=\alpha_{1}+\alpha_{2}+\lambda \alpha_{3}$ \begin{CJK}{UTF8}{mj}与\end{CJK} $\beta$ \begin{CJK}{UTF8}{mj}正交\end{CJK}.

\begin{CJK}{UTF8}{mj}三\end{CJK}. \begin{CJK}{UTF8}{mj}证明题\end{CJK} (\begin{CJK}{UTF8}{mj}本大题共\end{CJK} 3 \begin{CJK}{UTF8}{mj}小题\end{CJK}, \begin{CJK}{UTF8}{mj}每题\end{CJK} 15 \begin{CJK}{UTF8}{mj}分\end{CJK}, \begin{CJK}{UTF8}{mj}共\end{CJK} 45 \begin{CJK}{UTF8}{mj}分\end{CJK})

\begin{enumerate}
  \item \begin{CJK}{UTF8}{mj}设\end{CJK} $m, n$ \begin{CJK}{UTF8}{mj}是正整数\end{CJK}. \begin{CJK}{UTF8}{mj}证明\end{CJK}: $\left(x^{m}-1, x^{n}-1\right)=x-1$ \begin{CJK}{UTF8}{mj}当且仅当\end{CJK}
\end{enumerate}
$$
(m, n)=1
$$

\begin{enumerate}
  \setcounter{enumi}{2}
  \item \begin{CJK}{UTF8}{mj}设\end{CJK} $V_{1}$ \begin{CJK}{UTF8}{mj}是\end{CJK} $n$ \begin{CJK}{UTF8}{mj}维欧氏空间\end{CJK} $V$ \begin{CJK}{UTF8}{mj}上的一个子空间\end{CJK}, $V_{1}^{\perp}$ \begin{CJK}{UTF8}{mj}是\end{CJK} $V_{1}$ \begin{CJK}{UTF8}{mj}的正交补\end{CJK}.
\end{enumerate}
(1) \begin{CJK}{UTF8}{mj}证明\end{CJK}:
$$
V=V_{1} \oplus V_{1}^{\perp}
$$
(2) \begin{CJK}{UTF8}{mj}设\end{CJK} $\mathscr{A}$ \begin{CJK}{UTF8}{mj}是\end{CJK} $V$ \begin{CJK}{UTF8}{mj}到\end{CJK} $V_{1}$ \begin{CJK}{UTF8}{mj}的投影变换\end{CJK},
$$
\forall \alpha=\alpha_{1}+\alpha_{2} \in V, \alpha_{1} \in V_{1}, \mathscr{A}(\alpha)=\alpha_{1}
$$
\begin{CJK}{UTF8}{mj}证明\end{CJK}: $\mathscr{A}$ \begin{CJK}{UTF8}{mj}是\end{CJK} $V$ \begin{CJK}{UTF8}{mj}上的对称变换且\end{CJK} $\mathscr{A}^{2}=\mathscr{A}$.

\begin{enumerate}
  \setcounter{enumi}{3}
  \item \begin{CJK}{UTF8}{mj}设\end{CJK} $A$ \begin{CJK}{UTF8}{mj}为\end{CJK} $n$ \begin{CJK}{UTF8}{mj}阶实方阵\end{CJK}, $A^{\prime}$ \begin{CJK}{UTF8}{mj}是\end{CJK} $A$ \begin{CJK}{UTF8}{mj}的转置矩阵\end{CJK}, \begin{CJK}{UTF8}{mj}且\end{CJK} $A+A^{\prime}=J-E$, \begin{CJK}{UTF8}{mj}其中\end{CJK} $J$ \begin{CJK}{UTF8}{mj}和\end{CJK} $E$ \begin{CJK}{UTF8}{mj}分别是元素全为\end{CJK} 1 \begin{CJK}{UTF8}{mj}的\end{CJK} $n$ \begin{CJK}{UTF8}{mj}阶矩\end{CJK} \begin{CJK}{UTF8}{mj}阵和单位矩阵\end{CJK}.
\end{enumerate}
(1) \begin{CJK}{UTF8}{mj}对任一复向量\end{CJK} $\alpha$, \begin{CJK}{UTF8}{mj}证明\end{CJK}:
$$
0 \leqslant \bar{\alpha}^{\prime} J \alpha \leqslant n \bar{\alpha}^{\prime} \alpha,
$$
\begin{CJK}{UTF8}{mj}其中\end{CJK} $\bar{\alpha}^{\prime}$ \begin{CJK}{UTF8}{mj}是\end{CJK} $\alpha$ \begin{CJK}{UTF8}{mj}的共轭转置\end{CJK}.

(2) \begin{CJK}{UTF8}{mj}证明\end{CJK}: $A$ \begin{CJK}{UTF8}{mj}的任一特征值的实部在\end{CJK} $-\frac{1}{2}$ \begin{CJK}{UTF8}{mj}和\end{CJK} $\frac{n-1}{2}$ \begin{CJK}{UTF8}{mj}之间\end{CJK}.

\section{5. 湖南师范大学 2013 年研究生入学考试高等代数试题}
\begin{CJK}{UTF8}{mj}李扬\end{CJK}

\begin{CJK}{UTF8}{mj}微信公众号\end{CJK}: sxkyliyang

\begin{CJK}{UTF8}{mj}一\end{CJK}. \begin{CJK}{UTF8}{mj}填空题\end{CJK} (\begin{CJK}{UTF8}{mj}本大题共\end{CJK} 5 \begin{CJK}{UTF8}{mj}小题\end{CJK}, \begin{CJK}{UTF8}{mj}每空\end{CJK} 6 \begin{CJK}{UTF8}{mj}分\end{CJK}, \begin{CJK}{UTF8}{mj}共\end{CJK} 30 \begin{CJK}{UTF8}{mj}分\end{CJK})

\begin{enumerate}
  \item \begin{CJK}{UTF8}{mj}若\end{CJK} $(x-1)^{3}$ \begin{CJK}{UTF8}{mj}除多项式\end{CJK} $f(x)$ \begin{CJK}{UTF8}{mj}的余式为\end{CJK} $x^{2}-3 x+4$, \begin{CJK}{UTF8}{mj}则\end{CJK} $x-1$ \begin{CJK}{UTF8}{mj}除多项式\end{CJK} $f(x)$ \begin{CJK}{UTF8}{mj}的余式是\end{CJK}

  \item \begin{CJK}{UTF8}{mj}设\end{CJK} $A_{i j}$ \begin{CJK}{UTF8}{mj}是行列式\end{CJK} $D=\left|\begin{array}{cccc}1 & 2 & 2 & 9 \\ 2 & 3 & 3 & -2 \\ a & b & 1 & c \\ -3 & -4 & -2 & 4\end{array}\right|=2013$ \begin{CJK}{UTF8}{mj}的第\end{CJK} $i$ \begin{CJK}{UTF8}{mj}行\end{CJK}, \begin{CJK}{UTF8}{mj}第\end{CJK} $j$ \begin{CJK}{UTF8}{mj}列的代数余子式\end{CJK}, \begin{CJK}{UTF8}{mj}则\end{CJK} $2 A_{14}+3 A_{24}+$ $3 A_{34}-2 A_{44}=$

  \item \begin{CJK}{UTF8}{mj}将秩为\end{CJK} $r$ \begin{CJK}{UTF8}{mj}的\end{CJK} $n$ \begin{CJK}{UTF8}{mj}元实二次型按合同分类\end{CJK}, \begin{CJK}{UTF8}{mj}一共可以分为\end{CJK} \begin{CJK}{UTF8}{mj}类\end{CJK}.

  \item \begin{CJK}{UTF8}{mj}设\end{CJK} $A=\left(\begin{array}{ccc}-2 & 2 & 3 \\ 0 & x & 1 \\ 0 & 2 & 1\end{array}\right)$ \begin{CJK}{UTF8}{mj}与\end{CJK} $B=\left(\begin{array}{ccc}2 & 0 & 0 \\ 0 & y & 0 \\ 0 & 0 & -1\end{array}\right)$ \begin{CJK}{UTF8}{mj}相似\end{CJK}, \begin{CJK}{UTF8}{mj}则\end{CJK} $x+y=$

  \item \begin{CJK}{UTF8}{mj}设\end{CJK} $\alpha_{1}, \alpha_{2}, \alpha_{3}$ \begin{CJK}{UTF8}{mj}是三维欧氏空间\end{CJK} $V$ \begin{CJK}{UTF8}{mj}的一组基\end{CJK}, \begin{CJK}{UTF8}{mj}其度量矩阵是\end{CJK} $A=\left(\begin{array}{ccc}3 & 0 & 0 \\ 0 & 1 & -1 \\ 0 & -1 & 2\end{array}\right)$, \begin{CJK}{UTF8}{mj}向量\end{CJK} $\beta=\alpha_{1}-\alpha_{2}$, \begin{CJK}{UTF8}{mj}则\end{CJK} $|\beta|=$

\end{enumerate}
\begin{CJK}{UTF8}{mj}二\end{CJK}. \begin{CJK}{UTF8}{mj}计算题\end{CJK} (\begin{CJK}{UTF8}{mj}本大题共\end{CJK} 4 \begin{CJK}{UTF8}{mj}小题\end{CJK}, \begin{CJK}{UTF8}{mj}每小题\end{CJK} 15 \begin{CJK}{UTF8}{mj}分\end{CJK}, \begin{CJK}{UTF8}{mj}共\end{CJK} 60 \begin{CJK}{UTF8}{mj}分\end{CJK})

\begin{enumerate}
  \item \begin{CJK}{UTF8}{mj}设\end{CJK}
\end{enumerate}
$$
\begin{gathered}
f(x)=x^{3}+(a+1) x^{2}+4 x+2 b, \\
g(x)=x^{3}+a x^{2}+2 b .
\end{gathered}
$$
\begin{CJK}{UTF8}{mj}且\end{CJK} $(f(x), g(x))$ \begin{CJK}{UTF8}{mj}是一个二次多项式\end{CJK}, \begin{CJK}{UTF8}{mj}求\end{CJK} $a, b$ \begin{CJK}{UTF8}{mj}的值\end{CJK}.

\begin{enumerate}
  \setcounter{enumi}{2}
  \item \begin{CJK}{UTF8}{mj}计算\end{CJK} $n$ \begin{CJK}{UTF8}{mj}阶行列式\end{CJK}
\end{enumerate}
$$
D_{n}=\left|\begin{array}{ccccc}
1+x & 2 & 3 & \cdots & n \\
x & 1 & 2 & \cdots & n-1 \\
x & x & 1 & \cdots & n-2 \\
\vdots & \vdots & \vdots & & \vdots \\
x & x & x & \cdots & 1
\end{array}\right|
$$
3 . \begin{CJK}{UTF8}{mj}设\end{CJK} $n(n \geqslant 3)$ \begin{CJK}{UTF8}{mj}元实二次型\end{CJK}
$$
f\left(x_{1}, x_{2}, \cdots, x_{n}\right)=x_{1} x_{2}+x_{1} x_{3}+\cdots+x_{1} x_{n}+x_{2} x_{3}
$$
(1) \begin{CJK}{UTF8}{mj}当\end{CJK} $n=3$ \begin{CJK}{UTF8}{mj}是\end{CJK}, \begin{CJK}{UTF8}{mj}用非退化线性替换化\end{CJK} $f\left(x_{1}, x_{2}, \cdots, x_{n}\right)$ \begin{CJK}{UTF8}{mj}为规范形\end{CJK};

(2) \begin{CJK}{UTF8}{mj}当\end{CJK} $n>3$ \begin{CJK}{UTF8}{mj}是\end{CJK}, \begin{CJK}{UTF8}{mj}用非退化线性替换化\end{CJK} $f\left(x_{1}, x_{2}, \cdots, x_{n}\right)$ \begin{CJK}{UTF8}{mj}为规范形\end{CJK}.

\begin{enumerate}
  \setcounter{enumi}{4}
  \item \begin{CJK}{UTF8}{mj}设\end{CJK}
\end{enumerate}
$$
A=\left(\begin{array}{lll}
4 & 2 & 2 \\
2 & 4 & 2 \\
2 & 2 & 4
\end{array}\right)
$$
\begin{CJK}{UTF8}{mj}求正交矩阵\end{CJK} $T$, \begin{CJK}{UTF8}{mj}使\end{CJK} $T^{\prime} A T$ \begin{CJK}{UTF8}{mj}为对角形\end{CJK}.

\begin{CJK}{UTF8}{mj}三\end{CJK}. (\begin{CJK}{UTF8}{mj}本大题共\end{CJK} 4 \begin{CJK}{UTF8}{mj}小题\end{CJK}, \begin{CJK}{UTF8}{mj}每空\end{CJK} 15 \begin{CJK}{UTF8}{mj}分\end{CJK}, \begin{CJK}{UTF8}{mj}共\end{CJK} 60 \begin{CJK}{UTF8}{mj}分\end{CJK}) 1. \begin{CJK}{UTF8}{mj}设\end{CJK} $a \neq 0$, \begin{CJK}{UTF8}{mj}且\end{CJK} $f(x)$ \begin{CJK}{UTF8}{mj}满足\end{CJK} $(x-a) \mid f\left(x^{n}\right)$, \begin{CJK}{UTF8}{mj}证明\end{CJK}:
$$
\left(x^{n}-a^{n}\right) \mid f\left(x^{n}\right)
$$

\begin{enumerate}
  \setcounter{enumi}{2}
  \item \begin{CJK}{UTF8}{mj}设\end{CJK} $A, B$ \begin{CJK}{UTF8}{mj}为\end{CJK} $n$ \begin{CJK}{UTF8}{mj}阶方阵\end{CJK}, $f_{A}(x)$ \begin{CJK}{UTF8}{mj}为\end{CJK} $A$ \begin{CJK}{UTF8}{mj}的特征多项式\end{CJK}, \begin{CJK}{UTF8}{mj}证明\end{CJK}: $f_{A}(B)$ \begin{CJK}{UTF8}{mj}可逆的充要条件是\end{CJK} $A, B$ \begin{CJK}{UTF8}{mj}没有公共的特征\end{CJK} \begin{CJK}{UTF8}{mj}值\end{CJK}.

  \item \begin{CJK}{UTF8}{mj}设\end{CJK} $s$ \begin{CJK}{UTF8}{mj}是一个实数\end{CJK}, \begin{CJK}{UTF8}{mj}在实数域\end{CJK} $\mathbb{R}$ \begin{CJK}{UTF8}{mj}上的多项式空间\end{CJK} $\mathbb{R}[x]$ \begin{CJK}{UTF8}{mj}中\end{CJK}, \begin{CJK}{UTF8}{mj}令\end{CJK}

\end{enumerate}
$$
W=\{f(x) \in \mathbb{R}[x] \mid f(s)=0, \partial(f(x) \leqslant n \text { 或 } f(x)=0)\} .
$$
\begin{CJK}{UTF8}{mj}证明\end{CJK}:

(1) $W$ \begin{CJK}{UTF8}{mj}是\end{CJK} $\mathbb{R}[X]$ \begin{CJK}{UTF8}{mj}的一个子空间\end{CJK};

(2) $g_{i}(x)=x^{i}-s^{i}(i=1,2, \cdots), n$ \begin{CJK}{UTF8}{mj}是\end{CJK} $W$ \begin{CJK}{UTF8}{mj}的一组基\end{CJK}.

\begin{enumerate}
  \setcounter{enumi}{4}
  \item \begin{CJK}{UTF8}{mj}设\end{CJK} $A$ \begin{CJK}{UTF8}{mj}是实数域\end{CJK} $\mathbb{R}$ \begin{CJK}{UTF8}{mj}上的\end{CJK} $n$ \begin{CJK}{UTF8}{mj}阶方阵\end{CJK}, \begin{CJK}{UTF8}{mj}向量\end{CJK} $\alpha \in \mathbb{R}^{n}$ (\begin{CJK}{UTF8}{mj}实数域\end{CJK} $\mathbb{R}$ \begin{CJK}{UTF8}{mj}上的\end{CJK} $n$ \begin{CJK}{UTF8}{mj}维列向量\end{CJK}), \begin{CJK}{UTF8}{mj}使得\end{CJK}
\end{enumerate}
$$
\alpha, A \alpha, A^{2} \alpha, \cdots, A^{n-1} \alpha
$$
\begin{CJK}{UTF8}{mj}是\end{CJK} $\mathbb{R}^{n}$ \begin{CJK}{UTF8}{mj}的一个基\end{CJK}, \begin{CJK}{UTF8}{mj}如果\end{CJK} $\mathbb{R}$ \begin{CJK}{UTF8}{mj}上的\end{CJK} $n$ \begin{CJK}{UTF8}{mj}阶方阵\end{CJK} $B$ \begin{CJK}{UTF8}{mj}满足条件\end{CJK} $A B=B A$, \begin{CJK}{UTF8}{mj}证明\end{CJK}:

(1) \begin{CJK}{UTF8}{mj}存在实数域\end{CJK} $\mathbb{R}$ \begin{CJK}{UTF8}{mj}上的一个次数不超过\end{CJK} $n-1$ \begin{CJK}{UTF8}{mj}的多项式\end{CJK} $f(x)$ \begin{CJK}{UTF8}{mj}使得\end{CJK} $B \alpha=f(A) \alpha$;

(2) \begin{CJK}{UTF8}{mj}对于\end{CJK} (1) \begin{CJK}{UTF8}{mj}中找到的多项式\end{CJK} $f(x)$, \begin{CJK}{UTF8}{mj}必有\end{CJK} $B=f(A)$.

\section{6. 湖南师范大学 2014 年研究生入学考试高等代数试题}
\begin{CJK}{UTF8}{mj}李扬\end{CJK}

\begin{CJK}{UTF8}{mj}微信公众号\end{CJK}: sxkyliyang

\begin{CJK}{UTF8}{mj}一\end{CJK}. \begin{CJK}{UTF8}{mj}填空题\end{CJK} (\begin{CJK}{UTF8}{mj}本大题共\end{CJK} 5 \begin{CJK}{UTF8}{mj}小题\end{CJK}, \begin{CJK}{UTF8}{mj}每空\end{CJK} 6 \begin{CJK}{UTF8}{mj}分\end{CJK}, \begin{CJK}{UTF8}{mj}共\end{CJK} 30 \begin{CJK}{UTF8}{mj}分\end{CJK})

\begin{enumerate}
  \item \begin{CJK}{UTF8}{mj}若\end{CJK} $x=r$ \begin{CJK}{UTF8}{mj}是\end{CJK} $f(x)$ \begin{CJK}{UTF8}{mj}的\end{CJK} 5 \begin{CJK}{UTF8}{mj}重根\end{CJK}, \begin{CJK}{UTF8}{mj}那么\end{CJK} $x=r$ \begin{CJK}{UTF8}{mj}是\end{CJK} $\left[f^{\prime}(x)\right]^{2}+\left[f^{\prime \prime}(x)\right]^{3}$ \begin{CJK}{UTF8}{mj}的\end{CJK} \begin{CJK}{UTF8}{mj}重根\end{CJK}.
\end{enumerate}
2 . \begin{CJK}{UTF8}{mj}若实矩阵\end{CJK} $A=\left(\begin{array}{ll}1 & a \\ 3 & b\end{array}\right)$ \begin{CJK}{UTF8}{mj}与\end{CJK} $B=\left(\begin{array}{cc}1 & -1 \\ -1 & 1\end{array}\right)$ \begin{CJK}{UTF8}{mj}合同\end{CJK}, \begin{CJK}{UTF8}{mj}则\end{CJK} $a+b=$

\begin{enumerate}
  \setcounter{enumi}{3}
  \item \begin{CJK}{UTF8}{mj}已知\end{CJK} 4 \begin{CJK}{UTF8}{mj}阶不可逆矩阵\end{CJK} $A$ \begin{CJK}{UTF8}{mj}的三个特征值是\end{CJK} $\frac{1}{3}, \frac{1}{4}, \frac{1}{5}$, \begin{CJK}{UTF8}{mj}那么行列式\end{CJK} $|A+2 E|=$

  \item \begin{CJK}{UTF8}{mj}设\end{CJK} $P[x]_{4}$ \begin{CJK}{UTF8}{mj}是数域\end{CJK} $P$ \begin{CJK}{UTF8}{mj}上的所有次数不大于\end{CJK} 3 \begin{CJK}{UTF8}{mj}的多项式以及零多项式所组成的线性空间\end{CJK}. \begin{CJK}{UTF8}{mj}已知\end{CJK} $1,1+x, 1+$ $x+x^{2}, 1+x+x^{2}+x^{3}$ \begin{CJK}{UTF8}{mj}是一个基\end{CJK}, \begin{CJK}{UTF8}{mj}则\end{CJK} $P[x]_{4}$ \begin{CJK}{UTF8}{mj}中元素\end{CJK} $2+x+x^{3}$ \begin{CJK}{UTF8}{mj}关乎该基的坐标是\end{CJK}

  \item \begin{CJK}{UTF8}{mj}设\end{CJK} $n$ \begin{CJK}{UTF8}{mj}维线性空间\end{CJK} $V$ \begin{CJK}{UTF8}{mj}的线性变换\end{CJK} $\mathscr{A}$ \begin{CJK}{UTF8}{mj}在\end{CJK} $V$ \begin{CJK}{UTF8}{mj}的一个基下的矩阵是\end{CJK} $A$, \begin{CJK}{UTF8}{mj}且已知齐次线性方程组\end{CJK} $A X=0$ \begin{CJK}{UTF8}{mj}的解\end{CJK} \begin{CJK}{UTF8}{mj}空间的维数是\end{CJK} $s$, \begin{CJK}{UTF8}{mj}则\end{CJK} $\operatorname{dim} \mathscr{A} V=$

\end{enumerate}
\begin{CJK}{UTF8}{mj}二\end{CJK}. \begin{CJK}{UTF8}{mj}计算题\end{CJK} (\begin{CJK}{UTF8}{mj}本大题共\end{CJK} 4 \begin{CJK}{UTF8}{mj}小题\end{CJK}, \begin{CJK}{UTF8}{mj}每小题\end{CJK} 15 \begin{CJK}{UTF8}{mj}分\end{CJK}, \begin{CJK}{UTF8}{mj}共\end{CJK} 60 \begin{CJK}{UTF8}{mj}分\end{CJK})

\begin{enumerate}
  \item \begin{CJK}{UTF8}{mj}设多项式\end{CJK}
\end{enumerate}
$$
f(x)=6 x^{4}-13 x^{3}+13 x^{2}-2
$$
(1) \begin{CJK}{UTF8}{mj}求出\end{CJK} $f(x)$ \begin{CJK}{UTF8}{mj}的全体有理根\end{CJK};

(2) \begin{CJK}{UTF8}{mj}在复数域上将\end{CJK} $f(x)$ \begin{CJK}{UTF8}{mj}分解为不可约多项式的乘积\end{CJK}.

\begin{enumerate}
  \setcounter{enumi}{2}
  \item \begin{CJK}{UTF8}{mj}计算\end{CJK} $n$ \begin{CJK}{UTF8}{mj}阶多项式\end{CJK}
\end{enumerate}
$$
D_{n}=\left|\begin{array}{ccccc}
2 & 4 & 4 & \cdots & 4 \\
3 & 2 & 4 & \cdots & 4 \\
3 & 3 & 2 & \cdots & 4 \\
\vdots & \vdots & \vdots & & \vdots \\
3 & 3 & 3 & \cdots & 2
\end{array}\right| .
$$

\begin{enumerate}
  \setcounter{enumi}{3}
  \item \begin{CJK}{UTF8}{mj}设\end{CJK} $\mathscr{T}$ \begin{CJK}{UTF8}{mj}为实线性空间\end{CJK} $\mathbb{R}^{3} \rightarrow \mathbb{R}^{3}$ \begin{CJK}{UTF8}{mj}的线性变换\end{CJK}, \begin{CJK}{UTF8}{mj}已知\end{CJK}
\end{enumerate}
$$
\mathscr{T}(1,0,0)=(1,0,1), \mathscr{T}(0,1,0)=(2,1,1), \mathscr{T}(0,0,1)=(-1,1,-2)
$$
(1) \begin{CJK}{UTF8}{mj}试用矩阵\end{CJK} $A$ \begin{CJK}{UTF8}{mj}表示此变换\end{CJK} $\mathscr{T}\left(x_{1}, x_{2}, x_{3}\right)=\left(x_{1}, x_{2}, x_{3}\right) A$;

(2) \begin{CJK}{UTF8}{mj}求\end{CJK} $\mathscr{T}$ \begin{CJK}{UTF8}{mj}的值域\end{CJK} $\mathscr{T}\left(\mathbb{R}^{3}\right)$ \begin{CJK}{UTF8}{mj}的一个基\end{CJK};

(3) \begin{CJK}{UTF8}{mj}求\end{CJK} $\mathscr{T}$ \begin{CJK}{UTF8}{mj}的核\end{CJK} $\mathscr{T}^{-1}(0)$ \begin{CJK}{UTF8}{mj}的一个基\end{CJK}.

\begin{enumerate}
  \setcounter{enumi}{4}
  \item \begin{CJK}{UTF8}{mj}设\end{CJK} 3 \begin{CJK}{UTF8}{mj}阶实对称矩阵\end{CJK} $A$ \begin{CJK}{UTF8}{mj}的特征值为\end{CJK} $\lambda_{1}=-1, \lambda_{2}=\lambda_{3}=1$, \begin{CJK}{UTF8}{mj}且\end{CJK} $\lambda_{1}$ \begin{CJK}{UTF8}{mj}对应的特征向量为\end{CJK} $\alpha_{1}=(0,1,1)^{\prime}$, \begin{CJK}{UTF8}{mj}试计\end{CJK} \begin{CJK}{UTF8}{mj}算\end{CJK}:
\end{enumerate}
(1) \begin{CJK}{UTF8}{mj}求矩阵\end{CJK} $A$ \begin{CJK}{UTF8}{mj}的对应于的特征值\end{CJK} 1 \begin{CJK}{UTF8}{mj}的特征向量\end{CJK};

(2) \begin{CJK}{UTF8}{mj}求矩阵\end{CJK} $A$;

(3) \begin{CJK}{UTF8}{mj}求正交矩阵\end{CJK} $T$, \begin{CJK}{UTF8}{mj}使得\end{CJK} $T^{\prime} A T$ \begin{CJK}{UTF8}{mj}为对角矩阵\end{CJK}.

\begin{CJK}{UTF8}{mj}三\end{CJK}. (\begin{CJK}{UTF8}{mj}本大题共\end{CJK} 4 \begin{CJK}{UTF8}{mj}小题\end{CJK}, \begin{CJK}{UTF8}{mj}每小题\end{CJK} 15 \begin{CJK}{UTF8}{mj}分\end{CJK}, \begin{CJK}{UTF8}{mj}共\end{CJK} 60 \begin{CJK}{UTF8}{mj}分\end{CJK})

\begin{enumerate}
  \item \begin{CJK}{UTF8}{mj}若\end{CJK} $f(x)$ \begin{CJK}{UTF8}{mj}有重因式\end{CJK}, \begin{CJK}{UTF8}{mj}且\end{CJK}
\end{enumerate}
$$
\left(f^{\prime}(x), f^{\prime \prime}(x)\right)=1
$$
\begin{CJK}{UTF8}{mj}证明\end{CJK}: $f(x)$ \begin{CJK}{UTF8}{mj}的重因式都是\end{CJK} 2 \begin{CJK}{UTF8}{mj}的重因式\end{CJK}. 2. \begin{CJK}{UTF8}{mj}设\end{CJK} $n$ \begin{CJK}{UTF8}{mj}阶方阵\end{CJK} $A$ \begin{CJK}{UTF8}{mj}是幂等矩阵\end{CJK}, \begin{CJK}{UTF8}{mj}证明\end{CJK}: $\operatorname{tr}(A)=A$ \begin{CJK}{UTF8}{mj}的秩\end{CJK}.

\begin{enumerate}
  \setcounter{enumi}{3}
  \item \begin{CJK}{UTF8}{mj}设\end{CJK} $V$ \begin{CJK}{UTF8}{mj}是数域\end{CJK} $P$ \begin{CJK}{UTF8}{mj}上的\end{CJK} $n$ \begin{CJK}{UTF8}{mj}维线性空间\end{CJK}, $\mathscr{A}$ \begin{CJK}{UTF8}{mj}是\end{CJK} $V$ \begin{CJK}{UTF8}{mj}的线性变换且\end{CJK} $\mathscr{A}^{n-1} \neq 0, \mathscr{A}^{n}=0$. \begin{CJK}{UTF8}{mj}证明\end{CJK}:
\end{enumerate}
(1) \begin{CJK}{UTF8}{mj}存在\end{CJK} $V$ \begin{CJK}{UTF8}{mj}的一个向量\end{CJK} $\alpha$ \begin{CJK}{UTF8}{mj}使得\end{CJK}
$$
\alpha, \mathscr{A} \alpha, \mathscr{A}^{2} \alpha, \cdots, \mathscr{A}^{n-1} \alpha
$$
\begin{CJK}{UTF8}{mj}是\end{CJK} $V$ \begin{CJK}{UTF8}{mj}的一个基\end{CJK};

(2) \begin{CJK}{UTF8}{mj}对于整数\end{CJK} $1 \leqslant k \leqslant n$, \begin{CJK}{UTF8}{mj}有\end{CJK}
$$
\operatorname{dim} \mathscr{A}^{k} V=n-k
$$

\begin{enumerate}
  \setcounter{enumi}{4}
  \item \begin{CJK}{UTF8}{mj}设\end{CJK} $\mathscr{A}, \mathscr{B}$ \begin{CJK}{UTF8}{mj}是线性空间\end{CJK} $V$ \begin{CJK}{UTF8}{mj}的两个线性变换\end{CJK}, \begin{CJK}{UTF8}{mj}且\end{CJK} $\mathscr{A}^{2}=\mathscr{A}$.
\end{enumerate}
(1) \begin{CJK}{UTF8}{mj}证明\end{CJK}:
$$
\mathscr{A}^{-1}(0)=\{\alpha-\mathscr{A} \alpha \mid \alpha \in V\}
$$
(2)
$$
\mathscr{B} \mathscr{A}=\mathscr{A} \mathscr{B}
$$
\begin{CJK}{UTF8}{mj}当且仅当\end{CJK} $\mathscr{A}(V), \mathscr{A}^{-1}(0)$ \begin{CJK}{UTF8}{mj}都是\end{CJK} $\mathscr{B}$ \begin{CJK}{UTF8}{mj}的不变子空间\end{CJK}.

\section{7. 湖南师范大学 2015 年研究生入学考试高等代数试题}
\begin{CJK}{UTF8}{mj}李扬\end{CJK}

\begin{CJK}{UTF8}{mj}微信公众号\end{CJK}: sxkyliyang

\begin{CJK}{UTF8}{mj}一\end{CJK}. \begin{CJK}{UTF8}{mj}填空题\end{CJK} (\begin{CJK}{UTF8}{mj}本大题共\end{CJK} 5 \begin{CJK}{UTF8}{mj}小题\end{CJK}, \begin{CJK}{UTF8}{mj}每空\end{CJK} 6 \begin{CJK}{UTF8}{mj}分\end{CJK}, \begin{CJK}{UTF8}{mj}共\end{CJK} 30 \begin{CJK}{UTF8}{mj}分\end{CJK})

\begin{enumerate}
  \item \begin{CJK}{UTF8}{mj}在实数域上分解因式\end{CJK} $x^{6}+27=$

  \item \begin{CJK}{UTF8}{mj}矩阵方程\end{CJK} $X\left(\begin{array}{ll}1 & 2 \\ 2 & 6\end{array}\right)=\left(\begin{array}{ll}2 & 0 \\ 1 & 5\end{array}\right)$ \begin{CJK}{UTF8}{mj}的解是\end{CJK} $X=$

  \item \begin{CJK}{UTF8}{mj}设线性空间\end{CJK} $\mathbb{R}^{3}$ \begin{CJK}{UTF8}{mj}的线性变换\end{CJK} $\mathscr{A}$ \begin{CJK}{UTF8}{mj}在基\end{CJK} $\alpha_{1}, \alpha_{2}, \alpha_{3}$ \begin{CJK}{UTF8}{mj}下的矩阵为\end{CJK} $A=\left(\begin{array}{lll}0 & 0 & 1 \\ 0 & 1 & 0 \\ 1 & 0 & 0\end{array}\right), \alpha \in \mathbb{R}^{3}$ \begin{CJK}{UTF8}{mj}的象\end{CJK} $\mathscr{A} \alpha$ \begin{CJK}{UTF8}{mj}在\end{CJK} $\alpha_{1}, \alpha_{2}, \alpha_{3}$ \begin{CJK}{UTF8}{mj}下的坐标为\end{CJK} $(-1,2,3)^{\prime}$, \begin{CJK}{UTF8}{mj}则\end{CJK} $\alpha=$

  \item \begin{CJK}{UTF8}{mj}三元实二次型\end{CJK} $2 x_{1}^{2}+x_{1} x_{2}-5 x_{1} x_{3}$ \begin{CJK}{UTF8}{mj}的规范形是\end{CJK}

  \item \begin{CJK}{UTF8}{mj}设\end{CJK} $W=L\left(\alpha_{1}, \alpha_{2}\right)$ \begin{CJK}{UTF8}{mj}是欧氏空间\end{CJK} $\mathbb{R}^{4}$ \begin{CJK}{UTF8}{mj}中由向量\end{CJK} $\alpha_{1}=(1,1,0,0), \alpha_{2}=(0,1,1,0)$ \begin{CJK}{UTF8}{mj}生成的子空间\end{CJK}, \begin{CJK}{UTF8}{mj}则\end{CJK} $W$ \begin{CJK}{UTF8}{mj}的正\end{CJK} \begin{CJK}{UTF8}{mj}交补空间\end{CJK} $W^{\perp}$ \begin{CJK}{UTF8}{mj}的一个标准正交基为\end{CJK}

\end{enumerate}
\begin{CJK}{UTF8}{mj}二\end{CJK}. \begin{CJK}{UTF8}{mj}计算题\end{CJK} (\begin{CJK}{UTF8}{mj}本大题共\end{CJK} 4 \begin{CJK}{UTF8}{mj}小题\end{CJK}, \begin{CJK}{UTF8}{mj}每小题\end{CJK} 15 \begin{CJK}{UTF8}{mj}分\end{CJK}, \begin{CJK}{UTF8}{mj}共\end{CJK} 60 \begin{CJK}{UTF8}{mj}分\end{CJK})

\begin{enumerate}
  \item \begin{CJK}{UTF8}{mj}若线性方程组\end{CJK}
\end{enumerate}
$$
\left\{\begin{array}{l}
x_{1}+\lambda x_{2}+\lambda x_{3}=b_{1} \\
\lambda x_{1}+x_{2}+\lambda x_{3}=b_{2} \\
\lambda x_{1}+\lambda x_{2}+x_{3}=b_{3}
\end{array}\right.
$$
\begin{CJK}{UTF8}{mj}对于任意的\end{CJK} $b_{1}, b_{2}, b_{3}$ \begin{CJK}{UTF8}{mj}都有解\end{CJK}, \begin{CJK}{UTF8}{mj}求\end{CJK} $\lambda$ \begin{CJK}{UTF8}{mj}的值\end{CJK}.

\begin{enumerate}
  \setcounter{enumi}{2}
  \item \begin{CJK}{UTF8}{mj}设\end{CJK} $V_{1}, V_{2}$ \begin{CJK}{UTF8}{mj}分别为齐次线性方程组\end{CJK}
\end{enumerate}
$$
x_{1}+x_{2}-x_{3}-x_{4}=0
$$
$$
\left\{\begin{array}{l}
x_{1}-x_{2}+x_{3}=0 \\
x_{1}+x_{2}+x_{3}-x_{4}=0
\end{array}\right.
$$
\begin{CJK}{UTF8}{mj}的解空间\end{CJK} (\begin{CJK}{UTF8}{mj}作为\end{CJK} $\mathbb{R}^{4}$ \begin{CJK}{UTF8}{mj}的子空间\end{CJK})

(1) \begin{CJK}{UTF8}{mj}分别求出\end{CJK} $V_{1}$ \begin{CJK}{UTF8}{mj}与\end{CJK} $V_{2}$ \begin{CJK}{UTF8}{mj}的一组基\end{CJK};

(2) \begin{CJK}{UTF8}{mj}求出\end{CJK} $V_{1} \cap V_{2}$ \begin{CJK}{UTF8}{mj}的一组基\end{CJK};

(3) \begin{CJK}{UTF8}{mj}求出\end{CJK} $V_{1}+V_{2}$ \begin{CJK}{UTF8}{mj}的维数\end{CJK}.

\begin{enumerate}
  \setcounter{enumi}{3}
  \item \begin{CJK}{UTF8}{mj}求矩阵\end{CJK} $(E+A)^{-1}-E$, \begin{CJK}{UTF8}{mj}其中矩阵\end{CJK}
\end{enumerate}
$$
A=\left(\begin{array}{llll}
0 & 0 & 0 & 0 \\
2 & 0 & 0 & 0 \\
3 & 2 & 0 & 0 \\
4 & 3 & 2 & 0
\end{array}\right)
$$

\begin{enumerate}
  \setcounter{enumi}{4}
  \item \begin{CJK}{UTF8}{mj}设矩阵\end{CJK}
\end{enumerate}
$$
A=\left(\begin{array}{ccc}
a & -1 & c \\
5 & b & c \\
1-c & 0 & -a
\end{array}\right) .
$$
\begin{CJK}{UTF8}{mj}已知\end{CJK} $A$ \begin{CJK}{UTF8}{mj}的行列式\end{CJK} $|A|=-1, A$ \begin{CJK}{UTF8}{mj}的伴随矩阵\end{CJK} $A^{*}$ \begin{CJK}{UTF8}{mj}有一个特征值\end{CJK} $\lambda_{0}$, \begin{CJK}{UTF8}{mj}且\end{CJK} $A^{*}$ \begin{CJK}{UTF8}{mj}属于\end{CJK} $\lambda_{0}$ \begin{CJK}{UTF8}{mj}的一个特征向量是\end{CJK} $\alpha=(-1,-1,1)^{\prime}$, \begin{CJK}{UTF8}{mj}试求\end{CJK} $a, b, c$ \begin{CJK}{UTF8}{mj}和\end{CJK} $\lambda_{0}$ \begin{CJK}{UTF8}{mj}的值\end{CJK}. \begin{CJK}{UTF8}{mj}三\end{CJK}. \begin{CJK}{UTF8}{mj}证明题\end{CJK} (\begin{CJK}{UTF8}{mj}本大题共\end{CJK} 4 \begin{CJK}{UTF8}{mj}小题\end{CJK}, \begin{CJK}{UTF8}{mj}每小题\end{CJK} 15 \begin{CJK}{UTF8}{mj}分\end{CJK}, \begin{CJK}{UTF8}{mj}共\end{CJK} 60 \begin{CJK}{UTF8}{mj}分\end{CJK})

\begin{enumerate}
  \item \begin{CJK}{UTF8}{mj}设正整数\end{CJK} $m, n, a \neq 0$, \begin{CJK}{UTF8}{mj}且\end{CJK} $m$ \begin{CJK}{UTF8}{mj}与\end{CJK} $n$ \begin{CJK}{UTF8}{mj}互素\end{CJK}, $n$ \begin{CJK}{UTF8}{mj}是偶数\end{CJK}, \begin{CJK}{UTF8}{mj}求证\end{CJK}:
\end{enumerate}
$$
\left(x^{m}-a^{m}, x^{n}+a^{n}\right)=1 .
$$

\begin{enumerate}
  \setcounter{enumi}{2}
  \item \begin{CJK}{UTF8}{mj}用\end{CJK} $\operatorname{rank}_{A}$ \begin{CJK}{UTF8}{mj}表示矩阵\end{CJK} $A$ \begin{CJK}{UTF8}{mj}的秩\end{CJK}, \begin{CJK}{UTF8}{mj}证明\end{CJK}: \begin{CJK}{UTF8}{mj}如果\end{CJK} $r a n k_{B}=\operatorname{rank}_{A B}$, \begin{CJK}{UTF8}{mj}则对任意的矩阵\end{CJK} $C$ \begin{CJK}{UTF8}{mj}有\end{CJK}
\end{enumerate}
$$
\operatorname{rank}_{A B C}=\operatorname{rank}_{B C} .
$$

\begin{enumerate}
  \setcounter{enumi}{3}
  \item \begin{CJK}{UTF8}{mj}设\end{CJK} $A, B, C, D$ \begin{CJK}{UTF8}{mj}都是数域\end{CJK} $P$ \begin{CJK}{UTF8}{mj}上的\end{CJK} $n$ \begin{CJK}{UTF8}{mj}方阵\end{CJK}, \begin{CJK}{UTF8}{mj}它们关于矩阵乘法两两可换\end{CJK}, $E$ \begin{CJK}{UTF8}{mj}是\end{CJK} $n$ \begin{CJK}{UTF8}{mj}阶单位矩阵\end{CJK}, \begin{CJK}{UTF8}{mj}且\end{CJK} $A C+B D=E$. \begin{CJK}{UTF8}{mj}令\end{CJK} $V, V_{2}, V_{2}$ \begin{CJK}{UTF8}{mj}分别是线性方程组\end{CJK} $(A B) X=0, A X=0$ \begin{CJK}{UTF8}{mj}和\end{CJK} $B X=0$ \begin{CJK}{UTF8}{mj}的解空间\end{CJK}, \begin{CJK}{UTF8}{mj}证明\end{CJK}:
\end{enumerate}
$$
V=V_{1} \oplus V_{2} .
$$

\begin{enumerate}
  \setcounter{enumi}{4}
  \item \begin{CJK}{UTF8}{mj}设\end{CJK} $\mathscr{A}$ \begin{CJK}{UTF8}{mj}是\end{CJK} $n$ \begin{CJK}{UTF8}{mj}维线性空间的线性变换\end{CJK}, \begin{CJK}{UTF8}{mj}则下列条件等价\end{CJK};
\end{enumerate}
(1) $\mathscr{A}^{2}=\mathscr{I}$ (\begin{CJK}{UTF8}{mj}恒等变换\end{CJK});

(2) $\mathscr{A}$ \begin{CJK}{UTF8}{mj}的特征值为\end{CJK} $\pm 1$, \begin{CJK}{UTF8}{mj}且\end{CJK} $V=V_{1} \oplus V_{-1}$, \begin{CJK}{UTF8}{mj}其中\end{CJK} $V_{1}, V_{-1}$ \begin{CJK}{UTF8}{mj}分别为特征值\end{CJK} $1,-1$ \begin{CJK}{UTF8}{mj}的特征子空间\end{CJK};

(3) $V$ \begin{CJK}{UTF8}{mj}中有一组基\end{CJK}, \begin{CJK}{UTF8}{mj}使\end{CJK} $\mathscr{A}$ \begin{CJK}{UTF8}{mj}在该基下的矩阵为\end{CJK} $\Lambda=\left(\begin{array}{cc}I_{r} & 0 \\ 0 & -I_{n-r}\end{array}\right)$.

\section{8. 湖南师范大学 2016 年研究生入学考试高等代数试题}
\begin{CJK}{UTF8}{mj}李扬\end{CJK}

\begin{CJK}{UTF8}{mj}微信公众号\end{CJK}: sxkyliyang

\begin{CJK}{UTF8}{mj}一\end{CJK}. \begin{CJK}{UTF8}{mj}填空题\end{CJK} (\begin{CJK}{UTF8}{mj}本大题共\end{CJK} 5 \begin{CJK}{UTF8}{mj}小题\end{CJK}, \begin{CJK}{UTF8}{mj}每空\end{CJK} 6 \begin{CJK}{UTF8}{mj}分\end{CJK}, \begin{CJK}{UTF8}{mj}共\end{CJK} 30 \begin{CJK}{UTF8}{mj}分\end{CJK})

\begin{enumerate}
  \item \begin{CJK}{UTF8}{mj}多项式\end{CJK} $30 x^{3}-31 x^{2}+10 x-1$ \begin{CJK}{UTF8}{mj}的全部有理根是\end{CJK}

  \item \begin{CJK}{UTF8}{mj}矩阵方程\end{CJK} $\left(\begin{array}{cc}1 & -2 \\ 0 & 3\end{array}\right) X=\left(\begin{array}{cc}1 & -1 \\ 0 & 1\end{array}\right)$ \begin{CJK}{UTF8}{mj}的解是\end{CJK}

  \item \begin{CJK}{UTF8}{mj}三元复二次型\end{CJK} $2 x_{1}^{2}+x_{1} x_{2}-3 x_{1} x_{3}$ \begin{CJK}{UTF8}{mj}的规范形是\end{CJK}

  \item \begin{CJK}{UTF8}{mj}设\end{CJK} $A=\left(\begin{array}{lll}2 & 0 & 0 \\ 0 & 2 & 0 \\ 0 & 0 & 3\end{array}\right), B=\left(\begin{array}{lll}2 & 2 & 0 \\ 0 & 2 & 0 \\ 0 & 0 & 3\end{array}\right)$, \begin{CJK}{UTF8}{mj}问\end{CJK} $A$ \begin{CJK}{UTF8}{mj}与\end{CJK} $B$ \begin{CJK}{UTF8}{mj}是否相似\end{CJK}.

  \item \begin{CJK}{UTF8}{mj}已知\end{CJK} 2016 \begin{CJK}{UTF8}{mj}阶方阵\end{CJK} $A$ \begin{CJK}{UTF8}{mj}的全部特征值\end{CJK} $\lambda_{1}=0, \lambda_{2}=1, \lambda_{3}=2, \cdots, \lambda_{2015}=2014, \lambda_{2016}=2015$, \begin{CJK}{UTF8}{mj}且\end{CJK} $P$ \begin{CJK}{UTF8}{mj}是\end{CJK} 2016 \begin{CJK}{UTF8}{mj}阶可逆矩阵\end{CJK}, \begin{CJK}{UTF8}{mj}则\end{CJK} $\left|E+P^{-1} A P\right|=$

\end{enumerate}
\begin{CJK}{UTF8}{mj}二\end{CJK}. \begin{CJK}{UTF8}{mj}计算题\end{CJK} (\begin{CJK}{UTF8}{mj}本大题共\end{CJK} 6 \begin{CJK}{UTF8}{mj}小题\end{CJK}, \begin{CJK}{UTF8}{mj}每空\end{CJK} 15 \begin{CJK}{UTF8}{mj}分\end{CJK}, \begin{CJK}{UTF8}{mj}共\end{CJK} 60 \begin{CJK}{UTF8}{mj}分\end{CJK})

\begin{enumerate}
  \item \begin{CJK}{UTF8}{mj}计算\end{CJK} $n$ \begin{CJK}{UTF8}{mj}阶行列式\end{CJK}.
\end{enumerate}
$$
D_{n}=\left|\begin{array}{ccccc}
1 & 2 & 3 & \cdots & n \\
n & 1 & 2 & \cdots & n-1 \\
n-1 & n & 1 & \cdots & n-2 \\
\vdots & \vdots & \vdots & & \vdots \\
2 & 3 & 4 & \cdots & 1
\end{array}\right| .
$$

\begin{enumerate}
  \setcounter{enumi}{2}
  \item \begin{CJK}{UTF8}{mj}已知\end{CJK} $a^{2}+b^{2}+c^{2}+d^{2}=\sqrt{-1}$, \begin{CJK}{UTF8}{mj}设矩阵\end{CJK}
\end{enumerate}
$$
A=\left(\begin{array}{cccc}
a & b & c & d \\
-b & a & d & -c \\
-c & -d & a & b \\
-d & c & -b & a
\end{array}\right)
$$
\begin{CJK}{UTF8}{mj}求逆矩阵\end{CJK} $A^{-1}$.

\begin{enumerate}
  \setcounter{enumi}{3}
  \item \begin{CJK}{UTF8}{mj}在数域\end{CJK} $P$ \begin{CJK}{UTF8}{mj}上的\end{CJK} $2 \times 2$ \begin{CJK}{UTF8}{mj}矩阵空间\end{CJK} $P^{2 \times 2}$ \begin{CJK}{UTF8}{mj}中\end{CJK}, $B=\left(\begin{array}{ll}1 & 2 \\ 0 & 2\end{array}\right)$, \begin{CJK}{UTF8}{mj}定义线性变换\end{CJK} $\mathscr{A}$ \begin{CJK}{UTF8}{mj}如下\end{CJK}:
\end{enumerate}
$$
\mathscr{A}(X)=X B-B X, \forall X \in P^{2 \times 2}
$$
\begin{CJK}{UTF8}{mj}试求\end{CJK}:

(1) $\mathscr{A}$ \begin{CJK}{UTF8}{mj}的特征多项式\end{CJK};

(2) $\mathscr{A}$ \begin{CJK}{UTF8}{mj}的属于特征值\end{CJK} 0 \begin{CJK}{UTF8}{mj}的特征向量\end{CJK}.

\begin{enumerate}
  \setcounter{enumi}{4}
  \item \begin{CJK}{UTF8}{mj}已知实二次型\end{CJK}
\end{enumerate}
$$
f\left(x_{1}, x_{2}, x_{3}\right)=(1+t) x_{1}^{2}+2 x_{2}^{2}+(1+t) x_{3}^{2}+2(1-t) x_{1} x_{3}
$$
\begin{CJK}{UTF8}{mj}的秩为\end{CJK} 2 .

(1) \begin{CJK}{UTF8}{mj}求\end{CJK} $t$ \begin{CJK}{UTF8}{mj}的值\end{CJK};

(2) \begin{CJK}{UTF8}{mj}求正交线性替换\end{CJK}, \begin{CJK}{UTF8}{mj}使二次型\end{CJK} $f\left(x_{1}, x_{2}, x_{3}\right)$ \begin{CJK}{UTF8}{mj}为标准型\end{CJK}.

\begin{CJK}{UTF8}{mj}三\end{CJK}. \begin{CJK}{UTF8}{mj}证明题\end{CJK} (\begin{CJK}{UTF8}{mj}本大题共\end{CJK} 4 \begin{CJK}{UTF8}{mj}小题\end{CJK}, \begin{CJK}{UTF8}{mj}每小题\end{CJK} 15 \begin{CJK}{UTF8}{mj}分\end{CJK}, \begin{CJK}{UTF8}{mj}共\end{CJK} 60 \begin{CJK}{UTF8}{mj}分\end{CJK}) 1. \begin{CJK}{UTF8}{mj}若\end{CJK} $n$ \begin{CJK}{UTF8}{mj}阶方阵\end{CJK} $A$ \begin{CJK}{UTF8}{mj}的每一行元素之和是一个定值\end{CJK}, \begin{CJK}{UTF8}{mj}这样的矩阵我们称之为\end{CJK} $n$ \begin{CJK}{UTF8}{mj}阶行等和矩阵\end{CJK}. \begin{CJK}{UTF8}{mj}试证明\end{CJK}: \begin{CJK}{UTF8}{mj}如果\end{CJK} $A, B$ \begin{CJK}{UTF8}{mj}都是\end{CJK} $n$ \begin{CJK}{UTF8}{mj}阶行等和矩阵\end{CJK}, $k$ \begin{CJK}{UTF8}{mj}是任意的数\end{CJK}, \begin{CJK}{UTF8}{mj}那么\end{CJK} $A+B, k A, k B$ \begin{CJK}{UTF8}{mj}都是\end{CJK} $n$ \begin{CJK}{UTF8}{mj}阶行等和矩阵\end{CJK}.

\begin{enumerate}
  \setcounter{enumi}{2}
  \item \begin{CJK}{UTF8}{mj}用\end{CJK} $\operatorname{rank}_{A}$ \begin{CJK}{UTF8}{mj}表示\end{CJK} $A$ \begin{CJK}{UTF8}{mj}的秩\end{CJK}, \begin{CJK}{UTF8}{mj}证明\end{CJK}: \begin{CJK}{UTF8}{mj}如果\end{CJK} $\operatorname{rank}_{A}=\operatorname{rank}_{A^{2}}$, \begin{CJK}{UTF8}{mj}则有\end{CJK}
\end{enumerate}
$$
\operatorname{rank}_{A^{2}}=\operatorname{rank}_{A^{3}} .
$$

\begin{enumerate}
  \setcounter{enumi}{3}
  \item \begin{CJK}{UTF8}{mj}设\end{CJK} $V$ \begin{CJK}{UTF8}{mj}是\end{CJK} $n$ \begin{CJK}{UTF8}{mj}维线性空间\end{CJK}, $\mathscr{A}$ \begin{CJK}{UTF8}{mj}是\end{CJK} $V$ \begin{CJK}{UTF8}{mj}上的线性变换\end{CJK}, \begin{CJK}{UTF8}{mj}证明下列条件等价\end{CJK}:
\end{enumerate}
(1) \begin{CJK}{UTF8}{mj}存在自然数\end{CJK} $m$, \begin{CJK}{UTF8}{mj}使得\end{CJK}
$$
\mathscr{A}^{2 m}(V)=\mathscr{A}^{m}(V),\left(\mathscr{A}^{2 m}\right)^{-1}(0)=\left(\mathscr{A}^{m}\right)^{-1}(0)
$$
(2)
$$
V=\mathscr{A}^{m}(V) \oplus\left(\mathscr{A}^{m}\right)^{-1}(0)
$$

\begin{enumerate}
  \setcounter{enumi}{4}
  \item \begin{CJK}{UTF8}{mj}设\end{CJK} $V$ \begin{CJK}{UTF8}{mj}是\end{CJK} $n$ \begin{CJK}{UTF8}{mj}维欧氏空间\end{CJK} $(n \geqslant 3), \alpha$ \begin{CJK}{UTF8}{mj}是\end{CJK} $V$ \begin{CJK}{UTF8}{mj}中的一个非零向量\end{CJK}, $\mathscr{A}_{\alpha}$ \begin{CJK}{UTF8}{mj}是一个由\end{CJK} $\alpha$ \begin{CJK}{UTF8}{mj}决定的镜面反射\end{CJK}, \begin{CJK}{UTF8}{mj}即\end{CJK}
\end{enumerate}
$$
\mathscr{A}_{\alpha}: V \rightarrow V, \mathscr{A}_{\alpha}(\eta)=\eta-\frac{2(\eta, \alpha)}{(\alpha, \alpha)} \alpha
$$
\begin{CJK}{UTF8}{mj}证明\end{CJK}:

(1) $\mathscr{A}_{\alpha}$ \begin{CJK}{UTF8}{mj}是正交变换\end{CJK};

(2) $\mathscr{A}_{\alpha}$ \begin{CJK}{UTF8}{mj}的特征值是\end{CJK} $-1$ \begin{CJK}{UTF8}{mj}与\end{CJK} $1(n-1$ \begin{CJK}{UTF8}{mj}重\end{CJK});

(3) \begin{CJK}{UTF8}{mj}若\end{CJK} $\alpha_{1}, \alpha_{2}, \cdots, \alpha_{n}$ \begin{CJK}{UTF8}{mj}是\end{CJK} $V$ \begin{CJK}{UTF8}{mj}的标准正交基\end{CJK}, \begin{CJK}{UTF8}{mj}则\end{CJK} $\mathscr{A}_{\alpha_{1}}+\mathscr{A}_{\alpha_{2}}+\cdots+\mathscr{A}_{\alpha_{n}}$ \begin{CJK}{UTF8}{mj}是\end{CJK} $V$ \begin{CJK}{UTF8}{mj}上的数乘变换\end{CJK}.

\section{9. 湖南师范大学 2009 年研究生入学考试试题数学分析}
\begin{CJK}{UTF8}{mj}李扬\end{CJK}

\begin{CJK}{UTF8}{mj}微信公众号\end{CJK}: sxkyliyang

\begin{CJK}{UTF8}{mj}一\end{CJK}、\begin{CJK}{UTF8}{mj}基本填空题\end{CJK}(\begin{CJK}{UTF8}{mj}每小题\end{CJK} 6 \begin{CJK}{UTF8}{mj}分\end{CJK}, \begin{CJK}{UTF8}{mj}共\end{CJK} 72 \begin{CJK}{UTF8}{mj}分\end{CJK})

\begin{enumerate}
  \item \begin{CJK}{UTF8}{mj}设\end{CJK} $z^{3}+z=x \ln (x+y)$ \begin{CJK}{UTF8}{mj}能决定隐函数\end{CJK} $z=z(x, y)$, \begin{CJK}{UTF8}{mj}则\end{CJK} $\frac{\partial^{2} z}{\partial x \partial y}(1,0)=$

  \item $\max _{0 \leqslant x \leqslant 1} x^{n}(1-x)^{m}=$

  \item $\lim _{n \rightarrow+\infty} n \sin \pi\left(\sqrt{1+n^{2}}-n\right)=$

  \item \begin{CJK}{UTF8}{mj}若\end{CJK} $\lim _{n \rightarrow-\infty}\left\{\sqrt{x^{2}+4 x-1}-a x-b\right\}=0$, \begin{CJK}{UTF8}{mj}则\end{CJK} $b=$

  \item \begin{CJK}{UTF8}{mj}设\end{CJK} $f(x, y)=\left\{\begin{array}{ll}|x y|^{p} \sin \frac{1}{\sqrt{x^{2}+y^{2}}}, & x^{2}+y^{2} \neq 0 \\ 0, & x^{2}+y^{2}=0\end{array}\right.$ \begin{CJK}{UTF8}{mj}在\end{CJK} $(0,0)$ \begin{CJK}{UTF8}{mj}点可微\end{CJK}, \begin{CJK}{UTF8}{mj}则\end{CJK} $p$ \begin{CJK}{UTF8}{mj}的取值范围是\end{CJK}

  \item \begin{CJK}{UTF8}{mj}曲线积分\end{CJK} $\int_{C}\left(x^{2}+y^{2}+z^{2}-2 x-2 y-2 z\right) \mathrm{d} s=$ , \begin{CJK}{UTF8}{mj}其中\end{CJK} $C$ \begin{CJK}{UTF8}{mj}为球面\end{CJK} $x^{2}+y^{2}+z^{2}=1$ \begin{CJK}{UTF8}{mj}和平面\end{CJK} $x+y+z=0$ \begin{CJK}{UTF8}{mj}的交线\end{CJK}.

  \item \begin{CJK}{UTF8}{mj}设\end{CJK} $y=x \mathrm{e}^{-x}$, \begin{CJK}{UTF8}{mj}则\end{CJK} $y^{n}(0)=$

  \item \begin{CJK}{UTF8}{mj}不定积分\end{CJK} $\int \arctan 2 x \mathrm{~d} x=$

  \item \begin{CJK}{UTF8}{mj}定积分\end{CJK} $\int_{0}^{\pi} \frac{x \sin x}{1+\cos ^{2} x} \mathrm{~d} x=$

  \item \begin{CJK}{UTF8}{mj}幂级数\end{CJK} $\sum_{n=1}^{\infty}\left(1+\frac{1}{n}\right)^{-n^{2}} x^{n}$ \begin{CJK}{UTF8}{mj}的收敛半径为\end{CJK}

  \item \begin{CJK}{UTF8}{mj}曲线\end{CJK} $\left\{\begin{array}{l}x=\theta-\sin \theta \\ y=1-\cos \theta\end{array}\right.$ \begin{CJK}{UTF8}{mj}在\end{CJK} $(\pi, 2)$ \begin{CJK}{UTF8}{mj}点的切线方程为\end{CJK}

  \item \begin{CJK}{UTF8}{mj}第二型曲线积分\end{CJK} $\int_{x^{2}+y^{2}=1}\left(x-\frac{\sin y}{1+y^{2}}\right) \mathrm{d} y-y \mathrm{~d} x=$ \begin{CJK}{UTF8}{mj}其中曲线取正向\end{CJK}.

\end{enumerate}
\section{一、证明题}
\begin{enumerate}
  \item (\begin{CJK}{UTF8}{mj}本题共\end{CJK} 18 \begin{CJK}{UTF8}{mj}分\end{CJK})
\end{enumerate}
\begin{CJK}{UTF8}{mj}设\end{CJK} $f(x)$ \begin{CJK}{UTF8}{mj}在\end{CJK} $(-\infty,+\infty)$ \begin{CJK}{UTF8}{mj}上可导\end{CJK}, \begin{CJK}{UTF8}{mj}且\end{CJK}
$$
k=\sup _{x \in(-\infty,+\infty)}\left|f^{\prime}(x)\right|<+\infty
$$
(1) \begin{CJK}{UTF8}{mj}求证\end{CJK}: $f(x)$ \begin{CJK}{UTF8}{mj}在\end{CJK} $(-\infty,+\infty)$ \begin{CJK}{UTF8}{mj}上一致连续\end{CJK};

(2) \begin{CJK}{UTF8}{mj}若\end{CJK} $0<k<1$, \begin{CJK}{UTF8}{mj}任取\end{CJK} $x_{0} \in(-\infty,+\infty)$, \begin{CJK}{UTF8}{mj}令\end{CJK} $x_{n+1}=f\left(x_{n}\right)(n=1,2, \cdots)$, \begin{CJK}{UTF8}{mj}求证\end{CJK}: $\left\{x_{n}\right\}$ \begin{CJK}{UTF8}{mj}收敛\end{CJK};

(3) \begin{CJK}{UTF8}{mj}若\end{CJK} $0<k<1$, \begin{CJK}{UTF8}{mj}证明\end{CJK}: $f(x)$ \begin{CJK}{UTF8}{mj}在\end{CJK} $(-\infty,+\infty)$ \begin{CJK}{UTF8}{mj}上有唯一不动点\end{CJK}.

\begin{enumerate}
  \setcounter{enumi}{2}
  \item (\begin{CJK}{UTF8}{mj}本题共\end{CJK} 16 \begin{CJK}{UTF8}{mj}分\end{CJK})
\end{enumerate}
(1) \begin{CJK}{UTF8}{mj}对任意\end{CJK} $R>0$, \begin{CJK}{UTF8}{mj}求\end{CJK}
$$
I_{R}=\iint_{x^{2}+y^{2} \leqslant R^{2}} \mathrm{e}^{-x^{2}-y^{2}} \mathrm{~d} x \mathrm{~d} y
$$
(2) \begin{CJK}{UTF8}{mj}求证\end{CJK}:
$$
\int_{0}^{+\infty} \mathrm{e}^{-x^{2}} \mathrm{~d} x=\frac{\sqrt{\pi}}{2}
$$

\begin{enumerate}
  \setcounter{enumi}{3}
  \item (\begin{CJK}{UTF8}{mj}本题共\end{CJK} 16 \begin{CJK}{UTF8}{mj}分\end{CJK})
\end{enumerate}
\begin{CJK}{UTF8}{mj}求第二类曲面积分\end{CJK}
$$
I=\iint_{S} \frac{x^{3} \mathrm{~d} y \mathrm{~d} z+y^{3} \mathrm{~d} z \mathrm{~d} x+z^{3} \mathrm{~d} x \mathrm{~d} y}{\left(x^{2}+y^{2}+z^{2}\right)^{\frac{3}{2}}}
$$
\begin{CJK}{UTF8}{mj}其中\end{CJK} $S$ \begin{CJK}{UTF8}{mj}为球面\end{CJK} $x^{2}+y^{2}+z^{2}=a^{2}(a>0)$, \begin{CJK}{UTF8}{mj}且\end{CJK} $S$ \begin{CJK}{UTF8}{mj}取外侧\end{CJK}.

\begin{enumerate}
  \setcounter{enumi}{4}
  \item (\begin{CJK}{UTF8}{mj}本题共\end{CJK} 12 \begin{CJK}{UTF8}{mj}分\end{CJK})
\end{enumerate}
\begin{CJK}{UTF8}{mj}设常数\end{CJK} $0<c<1, f(x)$ \begin{CJK}{UTF8}{mj}在\end{CJK} $x=0$ \begin{CJK}{UTF8}{mj}点连续\end{CJK}, \begin{CJK}{UTF8}{mj}且\end{CJK} $\lim _{x \rightarrow 0} \frac{f(x)-f(c x)}{x}=A$ \begin{CJK}{UTF8}{mj}存在有限\end{CJK}, \begin{CJK}{UTF8}{mj}求证\end{CJK}: $f(x)$ \begin{CJK}{UTF8}{mj}在\end{CJK} $x=0$ \begin{CJK}{UTF8}{mj}点可导\end{CJK}, \begin{CJK}{UTF8}{mj}并\end{CJK} \begin{CJK}{UTF8}{mj}证明\end{CJK}:
$$
f^{\prime}(0)=\frac{A}{1-c}
$$

\begin{enumerate}
  \setcounter{enumi}{5}
  \item (\begin{CJK}{UTF8}{mj}本题共\end{CJK} 10 \begin{CJK}{UTF8}{mj}分\end{CJK})
\end{enumerate}
(1) \begin{CJK}{UTF8}{mj}求证\end{CJK}:
$$
\int_{0}^{+\infty} \sqrt{\alpha} \mathrm{e}^{-\alpha x^{2}} \mathrm{~d} x
$$
\begin{CJK}{UTF8}{mj}关于\end{CJK} $\alpha$ \begin{CJK}{UTF8}{mj}在\end{CJK} $(0,+\infty)$ \begin{CJK}{UTF8}{mj}上不一致连续\end{CJK};

(2) \begin{CJK}{UTF8}{mj}对任意\end{CJK} $\sigma>0$, \begin{CJK}{UTF8}{mj}求证\end{CJK}:
$$
\int_{0}^{+\infty} \sqrt{\alpha} \mathrm{e}^{-\alpha x^{2}} \mathrm{~d} x
$$
\begin{CJK}{UTF8}{mj}关于\end{CJK} $\alpha$ \begin{CJK}{UTF8}{mj}在\end{CJK} $[\sigma,+\infty)$ \begin{CJK}{UTF8}{mj}上一致收敛\end{CJK}.

\begin{enumerate}
  \setcounter{enumi}{6}
  \item (\begin{CJK}{UTF8}{mj}本题共\end{CJK} 6 \begin{CJK}{UTF8}{mj}分\end{CJK}) \begin{CJK}{UTF8}{mj}求证\end{CJK}:
\end{enumerate}
(1) \begin{CJK}{UTF8}{mj}对任一收敛正项级数\end{CJK} $\sum_{n=1}^{\infty} a_{n}$, \begin{CJK}{UTF8}{mj}必存在收敛正项级数\end{CJK} $\sum_{n=1}^{\infty} b_{n}$, \begin{CJK}{UTF8}{mj}满足\end{CJK}:
$$
\lim _{n \rightarrow+\infty} \frac{a_{n}}{b_{n}}=0
$$
(2) \begin{CJK}{UTF8}{mj}对任一通项为正的发散级数\end{CJK} $\sum_{n=1}^{\infty} a_{n}$, \begin{CJK}{UTF8}{mj}必存在发散正项级数\end{CJK} $\sum_{n=1}^{\infty} b_{n}$, \begin{CJK}{UTF8}{mj}满足\end{CJK}:
$$
\lim _{n \rightarrow+\infty} \frac{b_{n}}{a_{n}}=0
$$

\section{0. 湖南师范大学 2010 年研究生入学考试试题数学分析}
\begin{CJK}{UTF8}{mj}李扬\end{CJK}

\begin{CJK}{UTF8}{mj}微信公众号\end{CJK}: sxkyliyang

\begin{CJK}{UTF8}{mj}一\end{CJK}、\begin{CJK}{UTF8}{mj}基本填空题\end{CJK}(\begin{CJK}{UTF8}{mj}每小题\end{CJK} 5 \begin{CJK}{UTF8}{mj}分\end{CJK}, \begin{CJK}{UTF8}{mj}共\end{CJK} 85 \begin{CJK}{UTF8}{mj}分\end{CJK})

\begin{enumerate}
  \item $\lim _{n \rightarrow \infty} \frac{1+2^{2009}+3^{2009}+\cdots+n^{2009}}{n^{2010}}=$

  \item \begin{CJK}{UTF8}{mj}设\end{CJK} $z=z(x, y)$ \begin{CJK}{UTF8}{mj}是\end{CJK} $z^{2}+z=\sin (x+y)$ \begin{CJK}{UTF8}{mj}确定的隐函数\end{CJK}, \begin{CJK}{UTF8}{mj}则\end{CJK} $\mathrm{d}^{2} z=$

  \item \begin{CJK}{UTF8}{mj}若\end{CJK} $\lim _{x \rightarrow 0} \frac{\sin x-x+b x^{5}+a x^{3}}{x^{5}}=0$, \begin{CJK}{UTF8}{mj}则\end{CJK} $a=$ $b=$

  \item \begin{CJK}{UTF8}{mj}不定积分\end{CJK} $\int \mathrm{e}^{\alpha x} \sin x \mathrm{~d} x=$

  \item \begin{CJK}{UTF8}{mj}曲线积分\end{CJK} $\int_{C} x y \mathrm{~d} x+\frac{1}{2} x^{2} \mathrm{~d} y=$ , \begin{CJK}{UTF8}{mj}其中\end{CJK} $C$ \begin{CJK}{UTF8}{mj}是位于上半平面中以\end{CJK} $(0,0)$ \begin{CJK}{UTF8}{mj}为起点\end{CJK}, $(1,1)$ \begin{CJK}{UTF8}{mj}为终点的有向光滑\end{CJK} \begin{CJK}{UTF8}{mj}曲线\end{CJK}.

  \item \begin{CJK}{UTF8}{mj}幂级数\end{CJK} $\sum_{n=1}^{\infty}\left(n \sin \frac{1}{n}\right)^{n^{3}} x^{n}$ \begin{CJK}{UTF8}{mj}的收敛半径是\end{CJK}

  \item \begin{CJK}{UTF8}{mj}曲线\end{CJK} $\left\{\begin{array}{l}x=t^{2}+1 ; \\ y=t^{4} .\end{array}\right.$ \begin{CJK}{UTF8}{mj}在\end{CJK} $(1,0)$ \begin{CJK}{UTF8}{mj}点处的曲率半径为\end{CJK}

  \item \begin{CJK}{UTF8}{mj}函数\end{CJK} $y=\frac{2 x}{1+x^{2}}$ \begin{CJK}{UTF8}{mj}的拐点有\end{CJK} \begin{CJK}{UTF8}{mj}渐近线有\end{CJK}

  \item $\int_{0}^{\pi} \sum_{n=1}^{\infty} \frac{(-1)^{n}}{(2 n+1) !} x^{2(n+1)} \mathrm{d} x=$

  \item \begin{CJK}{UTF8}{mj}设\end{CJK} $y=(1+x)^{x}(x>0)$, \begin{CJK}{UTF8}{mj}则\end{CJK} $\mathrm{d} y=$

  \item \begin{CJK}{UTF8}{mj}若\end{CJK} $u(x, y)$ \begin{CJK}{UTF8}{mj}二阶可导\end{CJK}, \begin{CJK}{UTF8}{mj}则\end{CJK} $\frac{\partial^{2} u}{\partial x \partial y}=\frac{\partial^{2} u}{\partial y \partial x}$ \begin{CJK}{UTF8}{mj}充分条件之一是\end{CJK}

  \item \begin{CJK}{UTF8}{mj}若\end{CJK} $b>a>0$, \begin{CJK}{UTF8}{mj}则\end{CJK} $\int_{0}^{1} \frac{x^{b}-x^{a}}{\ln x} \mathrm{~d} x=$

  \item \begin{CJK}{UTF8}{mj}若级数\end{CJK} $\sum_{n=1}^{\infty} \ln \left(1+\frac{(-1)^{p}}{n^{p}}\right)$ \begin{CJK}{UTF8}{mj}收敛\end{CJK}, \begin{CJK}{UTF8}{mj}则\end{CJK} $p$ \begin{CJK}{UTF8}{mj}的取值范围是\end{CJK}

  \item \begin{CJK}{UTF8}{mj}方程\end{CJK} $\mathrm{e}^{x}-a x^{2}-b x-c=0$ \begin{CJK}{UTF8}{mj}最多有多少个实根\end{CJK}

  \item \begin{CJK}{UTF8}{mj}已知\end{CJK} $\int_{0}^{1} \mathrm{~d} x \int_{0}^{x} \mathrm{~d} y \int_{0}^{x y} f(x, y, z) \mathrm{d} z=\int_{0}^{1} \mathrm{~d} x \int_{0}^{x^{2}} \mathrm{~d} z \int_{a(x, z)}^{b(x, z)} f(x, y, z) \mathrm{d} y$, \begin{CJK}{UTF8}{mj}则\end{CJK} $a(x, z)=$ $b(x, z)=$

  \item $\lim _{\substack{x \rightarrow 0 \\ y \rightarrow 0}}\left(\frac{2010}{|x|+|y|}+\left(x^{2}+y^{2}\right)^{x^{2}+y^{2}-1}\right)\left(x^{2}+y^{2}\right)=$

  \item \begin{CJK}{UTF8}{mj}若\end{CJK} $\gamma$ \begin{CJK}{UTF8}{mj}表示圆周\end{CJK} $x^{2}+y^{2}=a^{2}$ \begin{CJK}{UTF8}{mj}并取逆时针方向\end{CJK}, \begin{CJK}{UTF8}{mj}则\end{CJK} $\int_{\gamma} \frac{-x \mathrm{~d} x+y \mathrm{~d} y}{x^{2}+y^{2}}=$

\end{enumerate}
\begin{CJK}{UTF8}{mj}二\end{CJK}、\begin{CJK}{UTF8}{mj}计算题\end{CJK}(\begin{CJK}{UTF8}{mj}每题\end{CJK} 10 \begin{CJK}{UTF8}{mj}分\end{CJK}, \begin{CJK}{UTF8}{mj}共\end{CJK} 50 \begin{CJK}{UTF8}{mj}分\end{CJK})

\begin{enumerate}
  \item \begin{CJK}{UTF8}{mj}求函数\end{CJK}
\end{enumerate}
$$
z=x^{2}+y^{2}
$$
\begin{CJK}{UTF8}{mj}在闭区域\end{CJK} $(x-\sqrt{2})^{2}+(y-\sqrt{2})^{2} \leqslant 9$ \begin{CJK}{UTF8}{mj}上的最大值和最小值\end{CJK}. 2. \begin{CJK}{UTF8}{mj}将\end{CJK}
$$
f(x)= \begin{cases}0, & x \in[-1,0) \\ x^{2}, & x \in[0,1)\end{cases}
$$
\begin{CJK}{UTF8}{mj}展开成傅里叶级数\end{CJK}.

\begin{enumerate}
  \setcounter{enumi}{3}
  \item \begin{CJK}{UTF8}{mj}求极限\end{CJK}
\end{enumerate}
$$
\lim _{x \rightarrow 0^{+}}(\sin x)^{\alpha} \int_{x}^{1} \frac{f(t)}{t^{\alpha+1}} \mathrm{~d} t
$$
\begin{CJK}{UTF8}{mj}其中\end{CJK} $\alpha>0, f(x)$ \begin{CJK}{UTF8}{mj}为\end{CJK} $[0,1]$ \begin{CJK}{UTF8}{mj}上的连续函数\end{CJK}.

\begin{enumerate}
  \setcounter{enumi}{4}
  \item \begin{CJK}{UTF8}{mj}计算第二类曲面积分\end{CJK}
\end{enumerate}
$$
I=\iint_{z=\sqrt{R^{2}-x^{2}-y^{2}}} \frac{x \mathrm{~d} y \mathrm{~d} z+y \mathrm{~d} z \mathrm{~d} x+z \mathrm{~d} x \mathrm{~d} y}{\left(x^{2}+y^{2}+z^{2}\right)^{\frac{n}{2}}},
$$
\begin{CJK}{UTF8}{mj}其中\end{CJK} $n$ \begin{CJK}{UTF8}{mj}为整数\end{CJK}, \begin{CJK}{UTF8}{mj}曲面的正向为上侧\end{CJK}.

\begin{enumerate}
  \setcounter{enumi}{5}
  \item \begin{CJK}{UTF8}{mj}令\end{CJK}
\end{enumerate}
$$
f(x)= \begin{cases}\frac{x^{k} y}{x^{2}+y^{2}}, & x^{2}+y^{2} \neq 0 \\ 0, & x^{2}+y^{2}=0 .\end{cases}
$$
\begin{CJK}{UTF8}{mj}求\end{CJK} $k$ \begin{CJK}{UTF8}{mj}的最大取值区间使得\end{CJK} $f(x, y)$ \begin{CJK}{UTF8}{mj}在\end{CJK} $(0,0)$ \begin{CJK}{UTF8}{mj}点处可微\end{CJK}.

\begin{CJK}{UTF8}{mj}三\end{CJK}、\begin{CJK}{UTF8}{mj}证明题\end{CJK}(\begin{CJK}{UTF8}{mj}共\end{CJK} 15 \begin{CJK}{UTF8}{mj}分\end{CJK})

\begin{enumerate}
  \item (10 \begin{CJK}{UTF8}{mj}分\end{CJK}) \begin{CJK}{UTF8}{mj}用有限覆盖定理证明维尔斯特拉斯聚点定理\end{CJK}.

  \item (5 \begin{CJK}{UTF8}{mj}分\end{CJK}) \begin{CJK}{UTF8}{mj}设\end{CJK} $f(x)$ \begin{CJK}{UTF8}{mj}在\end{CJK} $(-\infty,+\infty)$ \begin{CJK}{UTF8}{mj}上连续可导且\end{CJK} $\lim _{x \rightarrow+\infty} f(x)=\lim _{x \rightarrow-\infty}=A$. \begin{CJK}{UTF8}{mj}证明\end{CJK}: \begin{CJK}{UTF8}{mj}存在一点\end{CJK} $\xi \in(-\infty,+\infty)$ \begin{CJK}{UTF8}{mj}使得\end{CJK}

\end{enumerate}
$$
f^{\prime}(\xi)=0 .
$$

\section{1. 湖南师范大学 2011 年研究生入学考试试题数学分析}
\begin{CJK}{UTF8}{mj}李扬\end{CJK}

\begin{CJK}{UTF8}{mj}微信公众号\end{CJK}: sxkyliyang

\begin{CJK}{UTF8}{mj}一\end{CJK}、\begin{CJK}{UTF8}{mj}基本填空题\end{CJK}(\begin{CJK}{UTF8}{mj}每小题\end{CJK} 10 \begin{CJK}{UTF8}{mj}分\end{CJK}, \begin{CJK}{UTF8}{mj}共\end{CJK} 100 \begin{CJK}{UTF8}{mj}分\end{CJK})

\begin{enumerate}
  \item \begin{CJK}{UTF8}{mj}求广义积分\end{CJK}
\end{enumerate}
$$
\int_{0}^{+\infty} \mathrm{e}^{-x} \sin 2 x \mathrm{~d} x
$$

\begin{enumerate}
  \setcounter{enumi}{2}
  \item \begin{CJK}{UTF8}{mj}求极限\end{CJK}
\end{enumerate}
$$
\lim _{x \rightarrow 0} \frac{(1+x)^{x}-\cos \frac{x}{2}}{\ln \left(1-x^{2}\right)}
$$

\begin{enumerate}
  \setcounter{enumi}{3}
  \item \begin{CJK}{UTF8}{mj}求积分\end{CJK}
\end{enumerate}
$$
\int \max \{2,|x|\} \mathrm{d} x .
$$

\begin{enumerate}
  \setcounter{enumi}{4}
  \item \begin{CJK}{UTF8}{mj}设\end{CJK}
\end{enumerate}
$$
f(x, y)= \begin{cases}\frac{x y\left(x^{2}-y^{2}\right)}{x^{2}+y^{2}}, & (x, y) \neq(0,0) \\ 0, & (x, y)=(0,0) .\end{cases}
$$
\begin{CJK}{UTF8}{mj}求\end{CJK} $\lim _{y \rightarrow 0} \frac{f_{x}^{\prime}(0, y)-f_{x}^{\prime}(0,0)}{y}$.

\begin{enumerate}
  \setcounter{enumi}{5}
  \item \begin{CJK}{UTF8}{mj}求幂级数\end{CJK}
\end{enumerate}
$$
\sum_{n=1}^{\infty} \frac{(-1)^{n} x^{2 n}}{(2 n+1)(n+1)}
$$
\begin{CJK}{UTF8}{mj}的和函数\end{CJK}.

\begin{enumerate}
  \setcounter{enumi}{6}
  \item \begin{CJK}{UTF8}{mj}求三重积分\end{CJK}
\end{enumerate}
$$
\iiint_{V}\left(x+|y|+z^{2}\right)
$$
\begin{CJK}{UTF8}{mj}其中\end{CJK} $V: x^{2}+y^{2}+z^{2} \leqslant 1$.

\begin{enumerate}
  \setcounter{enumi}{7}
  \item \begin{CJK}{UTF8}{mj}求极限\end{CJK}
\end{enumerate}
$$
\lim _{n \rightarrow \infty} \sum_{j=1}^{2 n} \sum_{i=1}^{n} \frac{2}{n^{2}} \cdot \frac{2 i+j}{n}
$$

\begin{enumerate}
  \setcounter{enumi}{8}
  \item \begin{CJK}{UTF8}{mj}假设\end{CJK}
\end{enumerate}
$$
\mathrm{d} u(x, y)=\frac{y \mathrm{~d} x-x \mathrm{~d} y}{3 x^{2}-2 x y+3 y^{2}}
$$
\begin{CJK}{UTF8}{mj}求\end{CJK} $u(x, y)$.

\begin{enumerate}
  \setcounter{enumi}{9}
  \item \begin{CJK}{UTF8}{mj}求第二类曲线积分\end{CJK}
\end{enumerate}
$$
\int_{x^{2}+y^{2}=1} \frac{x \mathrm{~d} y-y \mathrm{~d} x}{4 x^{2}+9 y^{2}}
$$
\begin{CJK}{UTF8}{mj}其中积分曲线取正向\end{CJK}.

10 . \begin{CJK}{UTF8}{mj}假设\end{CJK} $f(x)$ \begin{CJK}{UTF8}{mj}具有一阶连续导数\end{CJK}, \begin{CJK}{UTF8}{mj}求\end{CJK}
$$
\oiint_{\Sigma} \frac{1}{y} f\left(\frac{x}{y}\right) \mathrm{d} y \mathrm{~d} z+\frac{1}{x} f\left(\frac{x}{y}\right) \mathrm{d} z \mathrm{~d} x+z \mathrm{~d} x \mathrm{~d} y
$$
\begin{CJK}{UTF8}{mj}其中\end{CJK} $\Sigma$ \begin{CJK}{UTF8}{mj}表示由曲面\end{CJK} $y=x^{2}+z^{2}+1$ \begin{CJK}{UTF8}{mj}和\end{CJK} $y=9-x^{2}-z^{2}$ \begin{CJK}{UTF8}{mj}所围成立体的表面\end{CJK}, \begin{CJK}{UTF8}{mj}方向取外侧\end{CJK}. \begin{CJK}{UTF8}{mj}二\end{CJK}、\begin{CJK}{UTF8}{mj}基本证明\end{CJK}(\begin{CJK}{UTF8}{mj}每题\end{CJK} 15 \begin{CJK}{UTF8}{mj}分\end{CJK}, \begin{CJK}{UTF8}{mj}共\end{CJK} 45 \begin{CJK}{UTF8}{mj}分\end{CJK})

\begin{enumerate}
  \item \begin{CJK}{UTF8}{mj}设\end{CJK} $f(x)$ \begin{CJK}{UTF8}{mj}在\end{CJK} $[0,+\infty)$ \begin{CJK}{UTF8}{mj}上连续\end{CJK}, $\lim _{x \rightarrow+\infty} f(x)=A, f(0)<A<+\infty$. \begin{CJK}{UTF8}{mj}证明\end{CJK}: $f(x)$ \begin{CJK}{UTF8}{mj}在\end{CJK} $[0,+\infty)$ \begin{CJK}{UTF8}{mj}上存在最小值\end{CJK}.

  \item \begin{CJK}{UTF8}{mj}设非负函数\end{CJK} $f(x)$ \begin{CJK}{UTF8}{mj}在\end{CJK} $[0,1]$ \begin{CJK}{UTF8}{mj}上连续且不为\end{CJK} 0 , \begin{CJK}{UTF8}{mj}证明\end{CJK}:

\end{enumerate}
$$
\int_{0}^{1} f(x) \mathrm{d} x>0
$$

\begin{enumerate}
  \setcounter{enumi}{3}
  \item \begin{CJK}{UTF8}{mj}设\end{CJK} $f(x)$ \begin{CJK}{UTF8}{mj}在\end{CJK} $[1,2]$ \begin{CJK}{UTF8}{mj}上连续\end{CJK}, $D$ \begin{CJK}{UTF8}{mj}由\end{CJK} $x y=1, x y=2, y=x$ \begin{CJK}{UTF8}{mj}和\end{CJK} $y=4 x$ \begin{CJK}{UTF8}{mj}所围闭域在第一象限部分\end{CJK}, \begin{CJK}{UTF8}{mj}证明\end{CJK}:
\end{enumerate}
$$
\iint_{D} f(x y) \mathrm{d} x \mathrm{~d} y=\ln 2 \int_{1}^{2} f(u) \mathrm{d} u .
$$
\begin{CJK}{UTF8}{mj}三\end{CJK}、(5 \begin{CJK}{UTF8}{mj}分\end{CJK}) \begin{CJK}{UTF8}{mj}假设平面域\end{CJK} $D$ \begin{CJK}{UTF8}{mj}的边界\end{CJK} $C$ \begin{CJK}{UTF8}{mj}是一条光滑的\end{CJK} Jordan \begin{CJK}{UTF8}{mj}曲线\end{CJK}, \begin{CJK}{UTF8}{mj}函数\end{CJK} $f(x, y)$ \begin{CJK}{UTF8}{mj}在\end{CJK} $D$ \begin{CJK}{UTF8}{mj}内满足\end{CJK}: $f_{x^{2}}(x, y)+f_{y^{2}}(x, y) \equiv 0$. \begin{CJK}{UTF8}{mj}证明\end{CJK}:
$$
\oint_{C} f(x, y) \frac{\partial f(x, y)}{\partial \vec{n}} \mathrm{~d} s=\iint_{D}\left[\left(\frac{\partial f(x, y)}{\partial x}\right)^{2}+\left(\frac{\partial f(x, y)}{\partial y}\right)^{2}\right] \mathrm{d} x \mathrm{~d} y
$$
\begin{CJK}{UTF8}{mj}其中\end{CJK} $\vec{n}$ \begin{CJK}{UTF8}{mj}表示\end{CJK} $C$ \begin{CJK}{UTF8}{mj}的外法向量\end{CJK}.

\section{2. 湖南师范大学 2012 年研究生入学考试试题数学分析}
\begin{CJK}{UTF8}{mj}李扬\end{CJK}

\begin{CJK}{UTF8}{mj}微信公众号\end{CJK}: sxkyliyang

\begin{CJK}{UTF8}{mj}一\end{CJK}、\begin{CJK}{UTF8}{mj}基本填空题\end{CJK}(\begin{CJK}{UTF8}{mj}每小题\end{CJK} 6 \begin{CJK}{UTF8}{mj}分\end{CJK}, \begin{CJK}{UTF8}{mj}共\end{CJK} 72 \begin{CJK}{UTF8}{mj}分\end{CJK})

\begin{enumerate}
  \item \begin{CJK}{UTF8}{mj}极限\end{CJK} $\lim _{x \rightarrow 0}\left(\frac{\sin x}{x}\right)^{\frac{1}{x^{2}}}=$

  \item \begin{CJK}{UTF8}{mj}当参数\end{CJK} $p$ \begin{CJK}{UTF8}{mj}满足条件\end{CJK} \begin{CJK}{UTF8}{mj}时\end{CJK}, \begin{CJK}{UTF8}{mj}级数\end{CJK} $\sum_{n=3}^{\infty} \frac{1}{n \ln n(\ln \ln n)^{1-2 p}}$ \begin{CJK}{UTF8}{mj}收敛\end{CJK}.

  \item \begin{CJK}{UTF8}{mj}设\end{CJK} $\alpha$ \begin{CJK}{UTF8}{mj}为有限数\end{CJK}, \begin{CJK}{UTF8}{mj}则\end{CJK} $\alpha$ \begin{CJK}{UTF8}{mj}为非空实数集合\end{CJK} $E$ \begin{CJK}{UTF8}{mj}之下确界的定义是\end{CJK}

  \item \begin{CJK}{UTF8}{mj}设方程\end{CJK} $\ln \sqrt{x^{2}+y^{2}}=\arctan \frac{y}{x}$ \begin{CJK}{UTF8}{mj}能决定可导的函数\end{CJK} $y=y(x)$, \begin{CJK}{UTF8}{mj}则\end{CJK} $\frac{\mathrm{d} y}{\mathrm{~d} x}=$

\end{enumerate}
5 . \begin{CJK}{UTF8}{mj}幂级数\end{CJK} $\sum_{n=1}^{\infty} \frac{(x-1)^{3 n}}{n \cdot 8^{n}}$ \begin{CJK}{UTF8}{mj}收敛区间是\end{CJK}

\begin{enumerate}
  \setcounter{enumi}{6}
  \item \begin{CJK}{UTF8}{mj}设\end{CJK} $f(u)$ \begin{CJK}{UTF8}{mj}二阶可导\end{CJK}, $z=f\left(\frac{y}{x}\right)$, \begin{CJK}{UTF8}{mj}则\end{CJK} $\frac{\partial^{2} z}{\partial x \partial y}=$

  \item \begin{CJK}{UTF8}{mj}不定积分\end{CJK} $\int \frac{x}{\sin ^{2} x} \mathrm{~d} x=$

  \item \begin{CJK}{UTF8}{mj}第一型曲线积分\end{CJK} $\int_{C} x \mathrm{~d} s=$ , \begin{CJK}{UTF8}{mj}其中\end{CJK} $C$ \begin{CJK}{UTF8}{mj}为球面\end{CJK} $x^{2}+y^{2}+z^{2}=1$ \begin{CJK}{UTF8}{mj}和平面\end{CJK} $x+y+z=1$ \begin{CJK}{UTF8}{mj}的交线\end{CJK}.

  \item \begin{CJK}{UTF8}{mj}定积分\end{CJK} $\int_{-1}^{1}\left(\frac{\cos x}{1+x^{4}}+\sin ^{2} \pi x\right) \mathrm{d} x=$

  \item \begin{CJK}{UTF8}{mj}函数\end{CJK} $f(x, y, z)=|x|+y^{2}+z$ \begin{CJK}{UTF8}{mj}在\end{CJK} $(0,0,0)$ \begin{CJK}{UTF8}{mj}点沿方向\end{CJK} $\vec{l}=(-1,-1,1)$ \begin{CJK}{UTF8}{mj}的方向导数\end{CJK} $\frac{\partial f}{\partial l}(0,0,0)=$

  \item \begin{CJK}{UTF8}{mj}设\end{CJK} $f(t)$ \begin{CJK}{UTF8}{mj}在\end{CJK} $(-\infty,+\infty)$ \begin{CJK}{UTF8}{mj}上连续\end{CJK}, $F(x)=\int_{-1}^{1} \int_{-1}^{1} f(x+u+v) \mathrm{d} u \mathrm{~d} v$, \begin{CJK}{UTF8}{mj}则\end{CJK} $F^{\prime \prime}(x)=$

  \item \begin{CJK}{UTF8}{mj}当常数\end{CJK} $\alpha=$ \begin{CJK}{UTF8}{mj}时\end{CJK}, \begin{CJK}{UTF8}{mj}积分\end{CJK} $\int_{L}\left(\frac{\cos x}{1+x^{2}}+a x y\right) \mathrm{d} x+x^{2} \mathrm{~d} y$ \begin{CJK}{UTF8}{mj}在全平面上与光滑路径\end{CJK} $L$ \begin{CJK}{UTF8}{mj}无关\end{CJK}.

\end{enumerate}
\section{二、解答题}
\begin{enumerate}
  \item ( 18 \begin{CJK}{UTF8}{mj}分\end{CJK}) \begin{CJK}{UTF8}{mj}设\end{CJK}
\end{enumerate}
$$
x_{1}>0, \frac{x_{n+1}}{n+1}=\ln \left(1+\frac{x_{n}}{n}\right) \quad(n=1,2, \cdots) .
$$
\begin{CJK}{UTF8}{mj}若记\end{CJK} $y_{n}=\frac{x_{n}}{n}$,

(1) \begin{CJK}{UTF8}{mj}求证\end{CJK}: $\left\{y_{n}\right\}$ \begin{CJK}{UTF8}{mj}收敛\end{CJK}, \begin{CJK}{UTF8}{mj}且收敛于\end{CJK} 0 ;

(2) \begin{CJK}{UTF8}{mj}求\end{CJK} $\lim _{n \rightarrow \infty}\left(\frac{1}{y_{n+1}}-\frac{1}{y_{n}}\right)$;

(3) \begin{CJK}{UTF8}{mj}求\end{CJK} $\lim _{n \rightarrow \infty} x_{n}$.

\begin{enumerate}
  \setcounter{enumi}{2}
  \item ( 14 \begin{CJK}{UTF8}{mj}分\end{CJK}) \begin{CJK}{UTF8}{mj}设\end{CJK} $f(x)$ \begin{CJK}{UTF8}{mj}和\end{CJK} $g(x)$ \begin{CJK}{UTF8}{mj}在\end{CJK} $[a, b]$ \begin{CJK}{UTF8}{mj}上连续\end{CJK}, \begin{CJK}{UTF8}{mj}在\end{CJK} $(a, b)$ \begin{CJK}{UTF8}{mj}上可导\end{CJK}, \begin{CJK}{UTF8}{mj}求证\end{CJK}: \begin{CJK}{UTF8}{mj}存在\end{CJK} $c \in(a, b)$ \begin{CJK}{UTF8}{mj}使得\end{CJK}
\end{enumerate}
$$
\{f(a)-f(c)\} g^{\prime}(c)=\{g(c)-g(b)\} f^{\prime}(c)
$$

\begin{enumerate}
  \setcounter{enumi}{3}
  \item (14 \begin{CJK}{UTF8}{mj}分\end{CJK}) \begin{CJK}{UTF8}{mj}求第二类曲面积分\end{CJK}
\end{enumerate}
$$
I=\iint_{S} \frac{x \mathrm{~d} y \mathrm{~d} z-y \mathrm{~d} z \mathrm{~d} x+\left(z^{3}+1\right) \mathrm{d} x \mathrm{~d} y}{\left(2 x^{2}+3 y^{2}+z^{2}\right)^{3}},
$$
\begin{CJK}{UTF8}{mj}其中\end{CJK} $S$ \begin{CJK}{UTF8}{mj}为球面\end{CJK}: $z=\sqrt{1-2 x^{2}-3 y^{2}}$, \begin{CJK}{UTF8}{mj}且\end{CJK} $S$ \begin{CJK}{UTF8}{mj}取上侧\end{CJK}.

\begin{enumerate}
  \setcounter{enumi}{4}
  \item ( 14 \begin{CJK}{UTF8}{mj}分\end{CJK}) \begin{CJK}{UTF8}{mj}设\end{CJK} $f(x)$ \begin{CJK}{UTF8}{mj}在\end{CJK} $[0,+\infty)$ \begin{CJK}{UTF8}{mj}上可导\end{CJK}, $f(0)=0$, \begin{CJK}{UTF8}{mj}当\end{CJK} $x \geqslant 0$ \begin{CJK}{UTF8}{mj}时\end{CJK} $\left|f^{\prime}(x)\right| \leqslant|f(x)|$, \begin{CJK}{UTF8}{mj}求证\end{CJK}: \begin{CJK}{UTF8}{mj}在\end{CJK} $[0,+\infty)$ \begin{CJK}{UTF8}{mj}上\end{CJK}
\end{enumerate}
$$
f(x) \equiv 0 .
$$

\begin{enumerate}
  \setcounter{enumi}{5}
  \item (10 \begin{CJK}{UTF8}{mj}分\end{CJK}) \begin{CJK}{UTF8}{mj}设\end{CJK} $a>0$, \begin{CJK}{UTF8}{mj}已知\end{CJK} $\int_{0}^{+\infty} \mathrm{e}^{-x^{2}} \mathrm{~d} x=\frac{\sqrt{\pi}}{2}$, \begin{CJK}{UTF8}{mj}求\end{CJK}
\end{enumerate}
$$
I(y)=\int_{0}^{+\infty} \mathrm{e}^{-a x^{2}} \cos 2 y x \mathrm{~d} x
$$

\begin{enumerate}
  \setcounter{enumi}{6}
  \item ( 8 \begin{CJK}{UTF8}{mj}分\end{CJK}) \begin{CJK}{UTF8}{mj}证明\end{CJK}: \begin{CJK}{UTF8}{mj}数列\end{CJK} $\left\{\left(1+\frac{1}{n}\right)^{n+\frac{1}{2}}\right\}$ \begin{CJK}{UTF8}{mj}严格单调递减\end{CJK}.
\end{enumerate}
\section{3. 湖南师范大学 2013 年研究生入学考试试题数学分析}
\begin{CJK}{UTF8}{mj}李扬\end{CJK}

\begin{CJK}{UTF8}{mj}微信公众号\end{CJK}: sxkyliyang

\begin{CJK}{UTF8}{mj}一\end{CJK}、\begin{CJK}{UTF8}{mj}基本填空题\end{CJK}(\begin{CJK}{UTF8}{mj}每小题\end{CJK} 7 \begin{CJK}{UTF8}{mj}分\end{CJK}, \begin{CJK}{UTF8}{mj}共\end{CJK} 70 \begin{CJK}{UTF8}{mj}分\end{CJK})

\begin{enumerate}
  \item \begin{CJK}{UTF8}{mj}若函数\end{CJK} $f(x)=\left\{\begin{array}{ll}x^{2}, & x \leqslant x_{0} ; \\ a x+b, & x>x_{0} .\end{array}\right.$ \begin{CJK}{UTF8}{mj}在\end{CJK} $x_{0}$ \begin{CJK}{UTF8}{mj}点可导\end{CJK}, \begin{CJK}{UTF8}{mj}则\end{CJK} $a=$

  \item \begin{CJK}{UTF8}{mj}定积分\end{CJK} $\int_{-3}^{2} \min \left\{2, x^{2}\right\} \mathrm{d} x=$

  \item \begin{CJK}{UTF8}{mj}设方程组\end{CJK} $\left\{\begin{array}{l}x=\theta-\sin \theta ; \\ y=1-\cos \theta .\end{array}\right.$ \begin{CJK}{UTF8}{mj}能决定可导的隐函数\end{CJK} $y=y(x)$, \begin{CJK}{UTF8}{mj}则\end{CJK} $\frac{\mathrm{d} y}{\mathrm{~d} x}=$

  \item \begin{CJK}{UTF8}{mj}幂级数\end{CJK} $\sum_{n=1}^{\infty} \frac{(x-1)^{n}}{n \cdot 3^{n}}$ \begin{CJK}{UTF8}{mj}的收敛半径\end{CJK} $R=$

  \item \begin{CJK}{UTF8}{mj}设\end{CJK} $f(u, v)$ \begin{CJK}{UTF8}{mj}的两个一阶偏导数都连续\end{CJK}, $z=f\left(x-y, \frac{x}{y}\right)$, \begin{CJK}{UTF8}{mj}则\end{CJK} $\frac{\partial z}{\partial y}=$

  \item \begin{CJK}{UTF8}{mj}能使广义积分\end{CJK} $\int_{0}^{+\infty} \frac{1}{x^{2}+x^{p}} \mathrm{~d} x$ \begin{CJK}{UTF8}{mj}收敛的参数\end{CJK} $p$ \begin{CJK}{UTF8}{mj}的点集为\end{CJK}

  \item \begin{CJK}{UTF8}{mj}极限\end{CJK} $\lim _{x \rightarrow 0} x\left[\frac{1}{x}\right]=$ \begin{CJK}{UTF8}{mj}其中\end{CJK} [.] \begin{CJK}{UTF8}{mj}表示取整\end{CJK}.

  \item \begin{CJK}{UTF8}{mj}曲线\end{CJK} $S$ \begin{CJK}{UTF8}{mj}的面积\end{CJK} $\iint_{S} \mathrm{~d} S=$ , \begin{CJK}{UTF8}{mj}其中曲面\end{CJK} $S$ \begin{CJK}{UTF8}{mj}为\end{CJK} $z=\frac{x^{2}+y^{2}}{2}\left(x^{2}+y^{2} \leqslant 1\right)$.

  \item \begin{CJK}{UTF8}{mj}曲面积分\end{CJK} $\int_{C}\left(\mathrm{e}^{x} \sin y-m y\right) \mathrm{d} x+\left(\mathrm{e}^{x} \cos y-m\right) \mathrm{d} y=$ , \begin{CJK}{UTF8}{mj}其中\end{CJK} $C$ \begin{CJK}{UTF8}{mj}为\end{CJK} $x^{2}+y^{2}=a x$ \begin{CJK}{UTF8}{mj}上半圆周从\end{CJK} $A(a, 0)$ \begin{CJK}{UTF8}{mj}到\end{CJK} $O(0,0)$ \begin{CJK}{UTF8}{mj}那段\end{CJK} $(a>0)$.

  \item \begin{CJK}{UTF8}{mj}拉格朗日微分中值定理的表述是\end{CJK}

\end{enumerate}
\begin{CJK}{UTF8}{mj}二\end{CJK}、\begin{CJK}{UTF8}{mj}讨论题\end{CJK} (\begin{CJK}{UTF8}{mj}给出结论\end{CJK}, \begin{CJK}{UTF8}{mj}并论证你的判断或给出反例并验证\end{CJK}, \begin{CJK}{UTF8}{mj}每题\end{CJK} 10 \begin{CJK}{UTF8}{mj}分\end{CJK}, \begin{CJK}{UTF8}{mj}共\end{CJK} 30 \begin{CJK}{UTF8}{mj}分\end{CJK})

\begin{enumerate}
  \item \begin{CJK}{UTF8}{mj}若\end{CJK} $f(x)$ \begin{CJK}{UTF8}{mj}可导\end{CJK}, \begin{CJK}{UTF8}{mj}且\end{CJK} $f^{\prime}\left(x_{0}\right)>0$, \begin{CJK}{UTF8}{mj}是否一定存在点\end{CJK} $x_{0}$ \begin{CJK}{UTF8}{mj}某邻域使得在此邻域内单调递增\end{CJK}?

  \item \begin{CJK}{UTF8}{mj}给定两个级数\end{CJK} $\sum_{n=1}^{\infty} a_{n}$ \begin{CJK}{UTF8}{mj}和\end{CJK} $\sum_{n=1}^{\infty} b_{n}$, \begin{CJK}{UTF8}{mj}若\end{CJK}

\end{enumerate}
$$
\lim _{n \rightarrow \infty} \frac{b_{n}}{a_{n}}=1
$$
\begin{CJK}{UTF8}{mj}这两个级数是否一定同时收敛或同时发散\end{CJK}?

\begin{enumerate}
  \setcounter{enumi}{3}
  \item \begin{CJK}{UTF8}{mj}设\end{CJK} $f_{0}(x)$ \begin{CJK}{UTF8}{mj}在\end{CJK} $[0, a]$ \begin{CJK}{UTF8}{mj}上连续\end{CJK},
\end{enumerate}
$$
f_{n}(x)=\int_{0}^{x} f_{n-1}(x) \mathrm{d} t
$$
\begin{CJK}{UTF8}{mj}是否一定有\end{CJK} $\left\{f_{n}(x)\right\}$ \begin{CJK}{UTF8}{mj}在\end{CJK} $[0, a]$ \begin{CJK}{UTF8}{mj}上一致收敛\end{CJK}?

\begin{CJK}{UTF8}{mj}三\end{CJK}、\begin{CJK}{UTF8}{mj}解答题\end{CJK} 1. ( 14 \begin{CJK}{UTF8}{mj}分\end{CJK}) \begin{CJK}{UTF8}{mj}设\end{CJK}
$$
f(x, y)= \begin{cases}\frac{x y}{\sqrt{x^{2}+y^{2}}}, & (x, y) \neq(0,0) \\ 0, & (x, y)=(0,0)\end{cases}
$$
\begin{CJK}{UTF8}{mj}求证\end{CJK}:\\
(1) $f(x, y)$ \begin{CJK}{UTF8}{mj}在\end{CJK} $(0,0)$ \begin{CJK}{UTF8}{mj}点连续\end{CJK};\\
(2) $f_{x}^{\prime}(x, y)$ \begin{CJK}{UTF8}{mj}和\end{CJK} $f_{y}^{\prime}(x, y)$ \begin{CJK}{UTF8}{mj}有界\end{CJK};\\
(3) $f(x, y)$ \begin{CJK}{UTF8}{mj}在\end{CJK} $(0,0)$ \begin{CJK}{UTF8}{mj}点不可微\end{CJK}.

\begin{enumerate}
  \setcounter{enumi}{2}
  \item (12 \begin{CJK}{UTF8}{mj}分\end{CJK}) \begin{CJK}{UTF8}{mj}设\end{CJK} $f(x)$ \begin{CJK}{UTF8}{mj}在\end{CJK} $[0,+\infty)$ \begin{CJK}{UTF8}{mj}上\end{CJK} $f^{\prime}(x) \geqslant 0, f(0)>0$, \begin{CJK}{UTF8}{mj}如果积分\end{CJK}
\end{enumerate}
$$
\int_{0}^{+\infty} \frac{d x}{f(x)+f^{\prime}(x)}
$$
\begin{CJK}{UTF8}{mj}收玫\end{CJK}, \begin{CJK}{UTF8}{mj}求证\end{CJK}: $\int_{0}^{+\infty} \frac{\mathrm{d} x}{f(x)}$ \begin{CJK}{UTF8}{mj}也收敛\end{CJK}.

\begin{enumerate}
  \setcounter{enumi}{3}
  \item (12 \begin{CJK}{UTF8}{mj}分\end{CJK}) \begin{CJK}{UTF8}{mj}求曲面\end{CJK}
\end{enumerate}
$$
z=x^{2}+y^{2}, z=2\left(x^{2}+y^{2}\right), x y=a^{2}, x y=2 a^{2}, x=2 y, 2 x=y
$$
\begin{CJK}{UTF8}{mj}所围区域的体积\end{CJK} $(x>0, y>0)$.

\begin{enumerate}
  \setcounter{enumi}{4}
  \item ( 12 \begin{CJK}{UTF8}{mj}分\end{CJK}) \begin{CJK}{UTF8}{mj}设\end{CJK} $f(x)$ \begin{CJK}{UTF8}{mj}在\end{CJK} $[0,1]$ \begin{CJK}{UTF8}{mj}上可导\end{CJK}, $f(0)=f(1), \int_{0}^{1} f(x) \mathrm{d}=0$, \begin{CJK}{UTF8}{mj}且对一切\end{CJK} $x \in[0,1]$ \begin{CJK}{UTF8}{mj}都有\end{CJK} $f^{\prime}(x) \neq 1$, \begin{CJK}{UTF8}{mj}记\end{CJK} $g(x)=f(x)-x, n \geqslant 2$ \begin{CJK}{UTF8}{mj}为正整数\end{CJK}, \begin{CJK}{UTF8}{mj}求证\end{CJK}:
\end{enumerate}
(1) $g(x)$ \begin{CJK}{UTF8}{mj}在\end{CJK} $[0,1]$ \begin{CJK}{UTF8}{mj}上严格单调递减\end{CJK};

(2)
$$
-\frac{n}{2}<\sum_{k=0}^{n-1} g\left(\frac{k}{n}\right)<-\frac{n}{2}+1
$$
$(3)$
$$
\left|\sum_{k=0}^{n-1} g\left(\frac{k}{n}\right)\right|<\frac{1}{2}
$$

\section{4. 湖南师范大学 2014 年研究生入学考试试题数学分析}
\begin{CJK}{UTF8}{mj}李扬\end{CJK}

\begin{CJK}{UTF8}{mj}微信公众号\end{CJK}: sxkyliyang

\begin{CJK}{UTF8}{mj}一\end{CJK}、\begin{CJK}{UTF8}{mj}基本填空题\end{CJK}(\begin{CJK}{UTF8}{mj}每小题\end{CJK} 7 \begin{CJK}{UTF8}{mj}分\end{CJK}, \begin{CJK}{UTF8}{mj}共\end{CJK} 70 \begin{CJK}{UTF8}{mj}分\end{CJK})

\begin{enumerate}
  \item \begin{CJK}{UTF8}{mj}若函数\end{CJK} $f(x)=\left\{\begin{array}{l}|x|^{m} \sin \frac{1}{x}, x \neq 0 ; \\ 0, \quad x=0 .\end{array}\right.$ \begin{CJK}{UTF8}{mj}的导数连续\end{CJK}, \begin{CJK}{UTF8}{mj}则\end{CJK} $m$ \begin{CJK}{UTF8}{mj}的取值范围是\end{CJK}

  \item \begin{CJK}{UTF8}{mj}极限\end{CJK} $\lim _{n \rightarrow \infty}\left(\frac{1}{n+\frac{1^{2}}{n}}+\frac{1}{n+\frac{2^{2}}{n}}+\cdots+\frac{1}{n+\frac{n^{2}}{n}}\right)=$

  \item \begin{CJK}{UTF8}{mj}已知\end{CJK} $y=x^{x}(x>0)$, \begin{CJK}{UTF8}{mj}则\end{CJK} $\mathrm{d}^{2} y=$

  \item \begin{CJK}{UTF8}{mj}极限\end{CJK} $\lim _{x \rightarrow 1}(1-x) \tan \frac{\pi x}{2}=$

  \item \begin{CJK}{UTF8}{mj}不定积分\end{CJK} $\int \ln \left(x+\sqrt{x^{2}+1}\right) \mathrm{d} x=$

  \item \begin{CJK}{UTF8}{mj}曲面\end{CJK} $z=x^{2}+y^{2}-1$ \begin{CJK}{UTF8}{mj}在点\end{CJK} $(2,1,4)$ \begin{CJK}{UTF8}{mj}的切平面的方程为\end{CJK}

  \item \begin{CJK}{UTF8}{mj}设\end{CJK} $f(u, v)$ \begin{CJK}{UTF8}{mj}的两个一阶偏导数都连续\end{CJK}, $z=f\left(2 x-y, \frac{y}{x}\right)$, \begin{CJK}{UTF8}{mj}则\end{CJK} $\frac{\partial z}{\partial x}=$

  \item \begin{CJK}{UTF8}{mj}曲线积分\end{CJK} $\oint_{C} \frac{\mathrm{e}^{x^{2}}-x y}{x^{2}+y^{2}} \mathrm{~d} x+\frac{x y^{2}-\sin y^{2}}{x^{2}+y^{2}} \mathrm{~d} y=$ , \begin{CJK}{UTF8}{mj}其中\end{CJK} $C$ \begin{CJK}{UTF8}{mj}为曲线\end{CJK} $x^{2}+y^{2}=a^{2}$, \begin{CJK}{UTF8}{mj}方向取逆时针方向\end{CJK} $(a>0)$

  \item \begin{CJK}{UTF8}{mj}幂级数\end{CJK} $\sum_{n=1}^{\infty} \frac{x^{2 n-1}}{3^{n} n}$ \begin{CJK}{UTF8}{mj}的收敛区间为\end{CJK}

  \item \begin{CJK}{UTF8}{mj}若广义积分\end{CJK} $\int_{0}^{+\infty} \frac{\ln (1+x)}{x^{\lambda}} \mathrm{d} x$ \begin{CJK}{UTF8}{mj}收敛\end{CJK}, \begin{CJK}{UTF8}{mj}则\end{CJK} $\lambda$ \begin{CJK}{UTF8}{mj}的取值范围是\end{CJK}

\end{enumerate}
\begin{CJK}{UTF8}{mj}二\end{CJK}、\begin{CJK}{UTF8}{mj}讨论题\end{CJK} (\begin{CJK}{UTF8}{mj}给出结论\end{CJK}, \begin{CJK}{UTF8}{mj}并论证你的判断或给出反例并验证\end{CJK}, \begin{CJK}{UTF8}{mj}每题\end{CJK} 10 \begin{CJK}{UTF8}{mj}分\end{CJK}, \begin{CJK}{UTF8}{mj}共\end{CJK} 30 \begin{CJK}{UTF8}{mj}分\end{CJK})

\begin{enumerate}
  \item \begin{CJK}{UTF8}{mj}若数项级数\end{CJK}
\end{enumerate}
$$
\sum_{n=1}^{\infty}(-1)^{n-1} n^{2} a_{n}
$$
\begin{CJK}{UTF8}{mj}收敛\end{CJK}, \begin{CJK}{UTF8}{mj}判断级数\end{CJK} $\sum_{n=1}^{\infty} a_{n}$ \begin{CJK}{UTF8}{mj}的敛散性\end{CJK}, \begin{CJK}{UTF8}{mj}若收敛要指出是条件收敛还是绝对收敛\end{CJK}.

\begin{enumerate}
  \setcounter{enumi}{2}
  \item \begin{CJK}{UTF8}{mj}函数\end{CJK} $f(x)$ \begin{CJK}{UTF8}{mj}在点可导\end{CJK}, \begin{CJK}{UTF8}{mj}是否一定存在点\end{CJK} $x_{0}$ \begin{CJK}{UTF8}{mj}的某邻域使得\end{CJK} $f(x)$ \begin{CJK}{UTF8}{mj}在此邻域内的每一点都可导\end{CJK}?

  \item \begin{CJK}{UTF8}{mj}若\end{CJK} $f(x)$ \begin{CJK}{UTF8}{mj}在\end{CJK} $[1,+\infty)$ \begin{CJK}{UTF8}{mj}上连续\end{CJK}, \begin{CJK}{UTF8}{mj}且\end{CJK} $\int_{0}^{+\infty} f(x) \mathrm{d} x$ \begin{CJK}{UTF8}{mj}收敛\end{CJK}, \begin{CJK}{UTF8}{mj}是否一定有\end{CJK}

\end{enumerate}
$$
\lim _{x \rightarrow+\infty} f(x)=0 ?
$$

\section{三、解答题}
\begin{enumerate}
  \item (20 \begin{CJK}{UTF8}{mj}分\end{CJK}) \begin{CJK}{UTF8}{mj}设\end{CJK} $f(x)$ \begin{CJK}{UTF8}{mj}在\end{CJK} $[0, a]$ \begin{CJK}{UTF8}{mj}上有二阶连续的导数\end{CJK}, $f^{\prime}(0)=1, f^{\prime \prime}(0) \neq 0$, \begin{CJK}{UTF8}{mj}且当\end{CJK} $x \in(0, a)$ \begin{CJK}{UTF8}{mj}时有\end{CJK} $a<f(x)<x$, \begin{CJK}{UTF8}{mj}任取\end{CJK} $x_{1} \in(0, a)$, \begin{CJK}{UTF8}{mj}令\end{CJK}
\end{enumerate}
$$
x_{n+1}=f\left(x_{n}\right)(n=1,2, \cdots)
$$
(1) \begin{CJK}{UTF8}{mj}求证\end{CJK}: $\left\{x_{n}\right\}$ \begin{CJK}{UTF8}{mj}收敛并求\end{CJK} $\lim _{n \rightarrow \infty} x_{n}$;

(2) \begin{CJK}{UTF8}{mj}数列\end{CJK} $\left\{n x_{n}\right\}$ \begin{CJK}{UTF8}{mj}是否收敛\end{CJK}? \begin{CJK}{UTF8}{mj}若收敛求\end{CJK} $\lim _{n \rightarrow \infty} n x_{n}$. 2. ( 10 \begin{CJK}{UTF8}{mj}分\end{CJK}) \begin{CJK}{UTF8}{mj}设\end{CJK} $f(x)$ \begin{CJK}{UTF8}{mj}为区间\end{CJK} $I$ \begin{CJK}{UTF8}{mj}上的凸函数\end{CJK}, \begin{CJK}{UTF8}{mj}即对任意\end{CJK} $x_{1}, x_{2} \in I$ \begin{CJK}{UTF8}{mj}以及任意\end{CJK} $\lambda \in[0,1]$, \begin{CJK}{UTF8}{mj}满足不等式\end{CJK}
$$
f\left[\lambda x_{1}+(1-\lambda) x_{2}\right] \leqslant \lambda f\left(x_{1}\right)+(1-\lambda) f\left(x_{2}\right)
$$
\begin{CJK}{UTF8}{mj}求证\end{CJK}: \begin{CJK}{UTF8}{mj}对\end{CJK} $I$ \begin{CJK}{UTF8}{mj}内的任意三点\end{CJK} $a, b, c$, \begin{CJK}{UTF8}{mj}若\end{CJK} $a<b<c$, \begin{CJK}{UTF8}{mj}则有\end{CJK}
$$
\frac{f(b)-f(a)}{b-a}<\frac{f(c)-f(a)}{c-a}<\frac{f(c)-f(b)}{c-b}
$$

\begin{enumerate}
  \setcounter{enumi}{3}
  \item (10 \begin{CJK}{UTF8}{mj}分\end{CJK}) \begin{CJK}{UTF8}{mj}求证\end{CJK}:
\end{enumerate}
$$
f(x)= \begin{cases}\frac{x^{3}}{x^{2}+y^{2}}, & (x, y) \neq(0,0) \\ 0, & (x, y)=(0,0)\end{cases}
$$
\begin{CJK}{UTF8}{mj}在\end{CJK} $(0,0)$ \begin{CJK}{UTF8}{mj}点不可微\end{CJK}.

\begin{enumerate}
  \setcounter{enumi}{4}
  \item (10 \begin{CJK}{UTF8}{mj}分\end{CJK}) \begin{CJK}{UTF8}{mj}求曲面积分\end{CJK}
\end{enumerate}
$$
\iint_{S}\left(x^{2}+y^{2}\right) \mathrm{d} S
$$
\begin{CJK}{UTF8}{mj}其中\end{CJK} $S$ \begin{CJK}{UTF8}{mj}为立体\end{CJK} $\sqrt{x^{2}+y^{2}} \leqslant z \leqslant 1$ \begin{CJK}{UTF8}{mj}的边界曲面\end{CJK}.

\section{5. 湖南师范大学 2015 年研究生入学考试试题数学分析}
\begin{CJK}{UTF8}{mj}李扬\end{CJK}

\begin{CJK}{UTF8}{mj}微信公众号\end{CJK}: sxkyliyang

\begin{CJK}{UTF8}{mj}一\end{CJK}、\begin{CJK}{UTF8}{mj}基本填空题\end{CJK}(\begin{CJK}{UTF8}{mj}每小题\end{CJK} 7 \begin{CJK}{UTF8}{mj}分\end{CJK}, \begin{CJK}{UTF8}{mj}共\end{CJK} 70 \begin{CJK}{UTF8}{mj}分\end{CJK})

\begin{enumerate}
  \item \begin{CJK}{UTF8}{mj}数列极限\end{CJK} $\lim _{n \rightarrow \infty} n^{2}\left(\frac{1}{n}-\ln \frac{n+1}{n}\right)=$

  \item \begin{CJK}{UTF8}{mj}二重积分\end{CJK} $\iint_{x^{2}+y^{2} \leqslant 1} \mathrm{e}^{-x^{2}-y^{2}} \mathrm{~d} x \mathrm{~d} y=$

  \item \begin{CJK}{UTF8}{mj}不定积分\end{CJK} $\int \frac{x}{\sin ^{2} x} \mathrm{~d} x=$

  \item \begin{CJK}{UTF8}{mj}幂级数\end{CJK} $\sum_{n=1}^{\infty} \frac{(3 x+1)^{2 n}}{4^{n}}$ \begin{CJK}{UTF8}{mj}的收敛半径为\end{CJK}

  \item \begin{CJK}{UTF8}{mj}曲线\end{CJK} $\left\{\begin{array}{l}x=t-\sin t ; \\ y=1-\cos t .\end{array} \quad(0 \leqslant t \leqslant 2 \pi)\right.$ \begin{CJK}{UTF8}{mj}和\end{CJK} $x$ \begin{CJK}{UTF8}{mj}轴所围成的区域的面积\end{CJK}

  \item \begin{CJK}{UTF8}{mj}若函数\end{CJK} $\left\{\begin{array}{l}\arctan x, 0 \leqslant x \leqslant 1 ; \\ a x,-1<x<0 .\end{array}\right.$ \begin{CJK}{UTF8}{mj}在\end{CJK} $(-1,1)$ \begin{CJK}{UTF8}{mj}可导\end{CJK}, \begin{CJK}{UTF8}{mj}则\end{CJK} $a$ \begin{CJK}{UTF8}{mj}的值为\end{CJK}

  \item \begin{CJK}{UTF8}{mj}曲面\end{CJK} $z=x^{3}+y^{3}+z$ \begin{CJK}{UTF8}{mj}在下侧点\end{CJK} $(0,1,1)$ \begin{CJK}{UTF8}{mj}的法向量\end{CJK} $\vec{n}$ \begin{CJK}{UTF8}{mj}为\end{CJK}

  \item \begin{CJK}{UTF8}{mj}设\end{CJK} $f(x)$ \begin{CJK}{UTF8}{mj}二阶可导\end{CJK}, $u=y f\left(\frac{y}{x}\right)$, \begin{CJK}{UTF8}{mj}则\end{CJK} $\frac{\partial^{2} u}{\partial x \partial y}=$

  \item \begin{CJK}{UTF8}{mj}第二型曲线积分\end{CJK} $\int_{x^{2}+y^{2}=1}\left(\frac{\sin y}{1+y^{2}}-x\right) \mathrm{d} y+y \mathrm{~d} x=$

  \item \begin{CJK}{UTF8}{mj}若广义积分\end{CJK} $\int_{0}^{+\infty} \frac{\ln (1+x)}{x^{\lambda}} \mathrm{d} x$ \begin{CJK}{UTF8}{mj}收敛\end{CJK}, \begin{CJK}{UTF8}{mj}则\end{CJK} $\lambda$ \begin{CJK}{UTF8}{mj}的取值范围是\end{CJK}

\end{enumerate}
\section{一、解答题}
\begin{enumerate}
  \item \begin{CJK}{UTF8}{mj}计算\end{CJK}:
\end{enumerate}
(1) \begin{CJK}{UTF8}{mj}级数\end{CJK} $\sum_{n=1}^{\infty} a_{n}$ \begin{CJK}{UTF8}{mj}和\end{CJK} $\sum_{n=1}^{\infty} b_{n}$ \begin{CJK}{UTF8}{mj}收敛\end{CJK}, \begin{CJK}{UTF8}{mj}且\end{CJK} $a_{n} \leqslant u_{n} \leqslant b_{n}(n=1,2,3, \cdots)$, \begin{CJK}{UTF8}{mj}问\end{CJK}
$$
\sum_{n=1}^{\infty} u_{n}
$$
\begin{CJK}{UTF8}{mj}是否一定收敛\end{CJK}?

(2) \begin{CJK}{UTF8}{mj}函数\end{CJK}
$$
\begin{cases}x^{2} \cos \frac{1}{x}, & x \neq 0 \\ x, & x=0\end{cases}
$$
\begin{CJK}{UTF8}{mj}在\end{CJK} $x=0$ \begin{CJK}{UTF8}{mj}的导数是否连续\end{CJK}?

(3) \begin{CJK}{UTF8}{mj}闭区间\end{CJK} $[a, b]$ \begin{CJK}{UTF8}{mj}上的有界函数是否一定可积\end{CJK}? 2. \begin{CJK}{UTF8}{mj}设\end{CJK}
$$
x_{n}=\sum_{k=1}^{n} \frac{1}{k}-\ln n(n=1,2,3, \cdots)
$$
\begin{CJK}{UTF8}{mj}求证\end{CJK}: \begin{CJK}{UTF8}{mj}对一切自然数\end{CJK} $n$ \begin{CJK}{UTF8}{mj}都有\end{CJK}:\\
(1) $\frac{1}{n}<\ln \left(1+\frac{1}{n}\right)<\frac{1}{n}$;\\
(2) \begin{CJK}{UTF8}{mj}数列\end{CJK} $\left\{x_{n}\right\}$ \begin{CJK}{UTF8}{mj}收敛\end{CJK}.

\begin{enumerate}
  \setcounter{enumi}{3}
  \item \begin{CJK}{UTF8}{mj}设\end{CJK} $\int_{0}^{+\infty} \mathrm{e}^{-x^{2}} \mathrm{~d} x=\frac{\sqrt{\pi}}{2}$, \begin{CJK}{UTF8}{mj}求积分\end{CJK}
\end{enumerate}
$$
I=\int_{0}^{+\infty} \frac{\mathrm{e}^{-a x^{2}}-\mathrm{e}^{-b x^{2}}}{x^{2}} \mathrm{~d} x
$$
\begin{CJK}{UTF8}{mj}其中\end{CJK} $0<a<b$.

\begin{enumerate}
  \setcounter{enumi}{4}
  \item \begin{CJK}{UTF8}{mj}设\end{CJK} $a, b, p, q$ \begin{CJK}{UTF8}{mj}都是正数\end{CJK}, \begin{CJK}{UTF8}{mj}求\end{CJK}:\\
(1) \begin{CJK}{UTF8}{mj}求\end{CJK}\\
$f(x)=x^{p}(1-x)^{q}$
\end{enumerate}
\begin{CJK}{UTF8}{mj}在\end{CJK} $[0,1]$ \begin{CJK}{UTF8}{mj}上的最大值\end{CJK};

(2) \begin{CJK}{UTF8}{mj}求证\end{CJK}:
$$
\left(\frac{a}{p}\right)^{p}\left(\frac{b}{q}\right)^{q} \leqslant\left(\frac{a+b}{p+q}\right)^{p+q}
$$

\begin{enumerate}
  \setcounter{enumi}{5}
  \item \begin{CJK}{UTF8}{mj}求第二型曲面积分\end{CJK}:
\end{enumerate}
$$
I=\iint_{S} \frac{x^{3} \mathrm{~d} y \mathrm{~d} z+y^{3} \mathrm{~d} z \mathrm{~d} x+z^{3} \mathrm{~d} x \mathrm{~d} y}{\left(x^{2}+y^{2}+z^{2}\right)^{\frac{3}{2}}},
$$
\begin{CJK}{UTF8}{mj}其中\end{CJK} $S$ \begin{CJK}{UTF8}{mj}为球面\end{CJK} $x^{2}+y^{2}+z^{2}=a^{2}$, \begin{CJK}{UTF8}{mj}且取下侧\end{CJK}.

\section{6. 湖南师范大学 2016 年研究生入学考试试题数学分析}
\begin{CJK}{UTF8}{mj}李扬\end{CJK}

\begin{CJK}{UTF8}{mj}微信公众号\end{CJK}: sxkyliyang

\begin{CJK}{UTF8}{mj}一\end{CJK}、\begin{CJK}{UTF8}{mj}基本填空题\end{CJK}(\begin{CJK}{UTF8}{mj}每小题\end{CJK} 7 \begin{CJK}{UTF8}{mj}分\end{CJK}, \begin{CJK}{UTF8}{mj}共\end{CJK} 70 \begin{CJK}{UTF8}{mj}分\end{CJK})

\begin{enumerate}
  \item \begin{CJK}{UTF8}{mj}数列极限\end{CJK} $\lim _{n \rightarrow \infty} n^{2}\left(\frac{1}{n}-\ln \frac{n+1}{n}\right)=$

  \item \begin{CJK}{UTF8}{mj}设\end{CJK} $F(x)=\int_{a}^{b} f(y)|x-y| \mathrm{d} y$, \begin{CJK}{UTF8}{mj}其中\end{CJK} $a<x<b, f(y)$ \begin{CJK}{UTF8}{mj}为可微函数\end{CJK}, \begin{CJK}{UTF8}{mj}则\end{CJK} $F^{\prime \prime}(x)=$

  \item \begin{CJK}{UTF8}{mj}曲线积分\end{CJK} $\oint_{C} \mathrm{e}^{x}[(1-\cos y) \mathrm{d} x-(y-\sin y) \mathrm{d} y]=$ , \begin{CJK}{UTF8}{mj}其中\end{CJK} $C$ \begin{CJK}{UTF8}{mj}为区域\end{CJK} $0<x<\pi, 0<y<\sin x$ \begin{CJK}{UTF8}{mj}的边界\end{CJK}, \begin{CJK}{UTF8}{mj}取逆时针方向\end{CJK}.

  \item \begin{CJK}{UTF8}{mj}幂级数\end{CJK} $\sum_{n=1}^{\infty} \frac{(3 x+1)^{2 n}}{4^{n}}$ \begin{CJK}{UTF8}{mj}的收敛半径为\end{CJK}

  \item \begin{CJK}{UTF8}{mj}曲线\end{CJK} $\left\{\begin{array}{l}x=t-\sin t ; \\ y=1-\cos t .\end{array} \quad(0 \leqslant t \leqslant 2 \pi)\right.$ \begin{CJK}{UTF8}{mj}和\end{CJK} $x$ \begin{CJK}{UTF8}{mj}轴所围成的区域的面积\end{CJK} $S=$

  \item \begin{CJK}{UTF8}{mj}若函数\end{CJK} $f(x)=\left\{\begin{array}{l}\arctan x, 0 \leqslant x \leqslant 1 ; \\ a x, \quad-1<x<0 .\end{array}\right.$ \begin{CJK}{UTF8}{mj}在\end{CJK} $(-1,1)$ \begin{CJK}{UTF8}{mj}可导\end{CJK}, \begin{CJK}{UTF8}{mj}则\end{CJK} $a$ \begin{CJK}{UTF8}{mj}的值为\end{CJK}

  \item \begin{CJK}{UTF8}{mj}曲面\end{CJK} $z=x^{3}+y^{2}+x$ \begin{CJK}{UTF8}{mj}的上侧在点\end{CJK} $(0,1,1)$ \begin{CJK}{UTF8}{mj}的法向量\end{CJK} $\vec{n}$ \begin{CJK}{UTF8}{mj}为\end{CJK}

  \item \begin{CJK}{UTF8}{mj}设\end{CJK} $f$ \begin{CJK}{UTF8}{mj}可微\end{CJK}, $u=x^{3} f\left(x y, \frac{y}{x}\right)$, \begin{CJK}{UTF8}{mj}则\end{CJK} $\frac{\partial u}{\partial x}=$

  \item \begin{CJK}{UTF8}{mj}函数\end{CJK} $f(x)=\pi-|x|(-\pi \leqslant x \leqslant \pi)$ \begin{CJK}{UTF8}{mj}的\end{CJK} Fourier \begin{CJK}{UTF8}{mj}级数为\end{CJK}

  \item \begin{CJK}{UTF8}{mj}若广义积分\end{CJK} $\int_{0}^{2} \frac{\mathrm{d} x}{|\ln x|^{p}}$ \begin{CJK}{UTF8}{mj}收敛\end{CJK}, \begin{CJK}{UTF8}{mj}则\end{CJK} $p$ \begin{CJK}{UTF8}{mj}的最大取值范围是\end{CJK}

\end{enumerate}
\begin{CJK}{UTF8}{mj}二\end{CJK}、\begin{CJK}{UTF8}{mj}讨论题\end{CJK} (\begin{CJK}{UTF8}{mj}给出结论\end{CJK}, \begin{CJK}{UTF8}{mj}并论证你的判断或给出反例并验证\end{CJK}, \begin{CJK}{UTF8}{mj}每题\end{CJK} 10 \begin{CJK}{UTF8}{mj}分\end{CJK}, \begin{CJK}{UTF8}{mj}共\end{CJK} 30 \begin{CJK}{UTF8}{mj}分\end{CJK})

\begin{enumerate}
  \item \begin{CJK}{UTF8}{mj}函数\end{CJK}
\end{enumerate}
$$
f(x)= \begin{cases}x^{n} \sin \frac{1}{x}, & x \neq 0 \\ 0, & x=0\end{cases}
$$
\begin{CJK}{UTF8}{mj}其中\end{CJK} $n$ \begin{CJK}{UTF8}{mj}为整数\end{CJK}, \begin{CJK}{UTF8}{mj}则当\end{CJK} $n \in\{2,3, \cdots\}$ \begin{CJK}{UTF8}{mj}时\end{CJK}, $f(x)$ \begin{CJK}{UTF8}{mj}的导函数\end{CJK} $f^{\prime}(x)$ \begin{CJK}{UTF8}{mj}在\end{CJK} $x=0$ \begin{CJK}{UTF8}{mj}连续\end{CJK};

\begin{enumerate}
  \setcounter{enumi}{2}
  \item \begin{CJK}{UTF8}{mj}设常数\end{CJK} $a>0$ \begin{CJK}{UTF8}{mj}和\end{CJK} $A>0, f(x)$ \begin{CJK}{UTF8}{mj}为\end{CJK} $(-\infty,+\infty)$ \begin{CJK}{UTF8}{mj}上的非负连续函数\end{CJK}, \begin{CJK}{UTF8}{mj}并且满足下列等式\end{CJK}:
\end{enumerate}
$$
f(x)+\int_{x-1}^{x} f(t) \mathrm{d} t \equiv A,
$$
\begin{CJK}{UTF8}{mj}则\end{CJK} $\mathrm{e}^{a x} f(x)$ \begin{CJK}{UTF8}{mj}必为\end{CJK} $(-\infty,+\infty)$ \begin{CJK}{UTF8}{mj}上的递增函数\end{CJK}.

\begin{enumerate}
  \setcounter{enumi}{3}
  \item \begin{CJK}{UTF8}{mj}设\end{CJK}
\end{enumerate}
$$
x_{n}=\sum_{k=1}^{n} \frac{1}{k}-\ln n(n=1,2,3, \cdots)
$$
\begin{CJK}{UTF8}{mj}求证\end{CJK}: \begin{CJK}{UTF8}{mj}对一切自然数\end{CJK} $n$ \begin{CJK}{UTF8}{mj}都有\end{CJK}:

(1) $\frac{1}{n+1}<\ln \left(1+\frac{1}{n}\right)<\frac{1}{n}$;

(2) \begin{CJK}{UTF8}{mj}数列\end{CJK} $\left\{x_{n}\right\}$ \begin{CJK}{UTF8}{mj}收敛\end{CJK}. 4. \begin{CJK}{UTF8}{mj}设区域\end{CJK} $D$ \begin{CJK}{UTF8}{mj}内的函数\end{CJK} $f(x, y)$ \begin{CJK}{UTF8}{mj}关于变量\end{CJK} $x$ \begin{CJK}{UTF8}{mj}连续\end{CJK}, \begin{CJK}{UTF8}{mj}关于变量\end{CJK} $y$ \begin{CJK}{UTF8}{mj}满足\end{CJK} Lipschitz \begin{CJK}{UTF8}{mj}条件\end{CJK}: \begin{CJK}{UTF8}{mj}存在常数\end{CJK} $L>0$, \begin{CJK}{UTF8}{mj}对任意的\end{CJK} $\left(x, y_{1}\right),\left(x, y_{2}\right) \in D$ \begin{CJK}{UTF8}{mj}有\end{CJK}
$$
\left|f\left(x, y_{1}\right)-f\left(x, y_{2}\right)\right| \leqslant L\left|y_{1}-y_{2}\right| .
$$
\begin{CJK}{UTF8}{mj}证明\end{CJK}: $f(x, y)$ \begin{CJK}{UTF8}{mj}是\end{CJK} $D$ \begin{CJK}{UTF8}{mj}内的连续函数\end{CJK}.

\begin{enumerate}
  \setcounter{enumi}{5}
  \item \begin{CJK}{UTF8}{mj}计算\end{CJK}
\end{enumerate}
$$
I=\iint_{S} x z^{2} \mathrm{~d} y \mathrm{~d} z+\left(x^{2} y-z^{3}\right) \mathrm{d} z \mathrm{~d} x+\left(2 x y+y^{2} z\right) \mathrm{d} x \mathrm{~d} y
$$
\begin{CJK}{UTF8}{mj}其中\end{CJK} $S$ \begin{CJK}{UTF8}{mj}是曲面\end{CJK} $z=\sqrt{a^{2}-x^{2}-y^{2}}$, \begin{CJK}{UTF8}{mj}方向是下侧\end{CJK}.

\begin{enumerate}
  \setcounter{enumi}{6}
  \item \begin{CJK}{UTF8}{mj}分析函数序列\end{CJK} $\left\{\frac{x^{n}}{1+x^{n}}\right\}$ \begin{CJK}{UTF8}{mj}在下列区间上的一致收敛性\end{CJK}:
\end{enumerate}
(1) $1-a \leqslant x \leqslant 1+a$;

(2) $1+a \leqslant x<+\infty$, \begin{CJK}{UTF8}{mj}其中\end{CJK} $0<a<1$.

\section{1. 华东师范大学 2009 年研究生入学考试试题高等代数 
 李扬 
 微信公众号: sxkyliyang}
\begin{enumerate}
  \item (10 \begin{CJK}{UTF8}{mj}分\end{CJK})\begin{CJK}{UTF8}{mj}解下列线性方程组\end{CJK}
\end{enumerate}
$$
\left\{\begin{array}{l}
4 x_{1}+3 x_{2}+2 x_{3}+x_{4}=17 \\
3 x_{1}+2 x_{2}+x_{3}+4 x_{4}=17 \\
2 x_{1}+x_{2}+4 x_{3}+3 x_{4}=17 \\
x_{1}+4 x_{2}+3 x_{3}+2 x_{4}=17
\end{array}\right.
$$

\begin{enumerate}
  \setcounter{enumi}{2}
  \item (10 \begin{CJK}{UTF8}{mj}分\end{CJK}) \begin{CJK}{UTF8}{mj}设\end{CJK} $\mathcal{A}$ \begin{CJK}{UTF8}{mj}是\end{CJK} $n$ \begin{CJK}{UTF8}{mj}维线性空间\end{CJK} $V$ \begin{CJK}{UTF8}{mj}上的秩为\end{CJK} 1 \begin{CJK}{UTF8}{mj}的线性变换\end{CJK}, $\mathcal{B}=\mathcal{A}-\mathcal{E}$ (\begin{CJK}{UTF8}{mj}其中\end{CJK} $\mathcal{E}$ \begin{CJK}{UTF8}{mj}为恒等变换\end{CJK}), \begin{CJK}{UTF8}{mj}试求线性变换\end{CJK} $\mathcal{B}$ \begin{CJK}{UTF8}{mj}的最小多项式\end{CJK}.

  \item ( 10 \begin{CJK}{UTF8}{mj}分\end{CJK}) \begin{CJK}{UTF8}{mj}若分块矩阵\end{CJK} $A=\left(A_{i j}\right)_{s \times t}, \operatorname{rank}\left(A_{i j}\right)=r_{i j}, i=1, \cdots, s, j=1, \cdots t$. \begin{CJK}{UTF8}{mj}证明\end{CJK}:

\end{enumerate}
$$
\operatorname{rank}(A) \leq \sum_{i=1}^{s} \sum_{j=1}^{t} r_{i j}
$$

\begin{enumerate}
  \setcounter{enumi}{4}
  \item (15 \begin{CJK}{UTF8}{mj}分\end{CJK}) \begin{CJK}{UTF8}{mj}设\end{CJK} $x_{1}, x_{2}, x_{3}$ \begin{CJK}{UTF8}{mj}是多项式\end{CJK} $f(x)=x^{3}+5 x^{2}-2 x-7$ \begin{CJK}{UTF8}{mj}的根\end{CJK}, \begin{CJK}{UTF8}{mj}令\end{CJK}
\end{enumerate}
$$
s_{k}=x_{1}^{k}+x_{2}^{k}+x_{3}^{k},(k=1,2,3,4),
$$
\begin{CJK}{UTF8}{mj}试求\end{CJK} $s_{1}, s_{2}, s_{3}, s_{4}$.

\begin{enumerate}
  \setcounter{enumi}{5}
  \item (15 \begin{CJK}{UTF8}{mj}分\end{CJK}) \begin{CJK}{UTF8}{mj}我们知道\end{CJK}, \begin{CJK}{UTF8}{mj}对任意的实函数\end{CJK} $f(x)$, \begin{CJK}{UTF8}{mj}均存在唯一的偶函数\end{CJK} $g(x)$ \begin{CJK}{UTF8}{mj}和奇函数\end{CJK} $h(x)$, \begin{CJK}{UTF8}{mj}使得\end{CJK} $f(x)=g(x)+h(x)$.
\end{enumerate}
(1) \begin{CJK}{UTF8}{mj}证明\end{CJK}: \begin{CJK}{UTF8}{mj}对复数域上的任一方阵\end{CJK} $A$, \begin{CJK}{UTF8}{mj}均存在唯一的对称矩阵\end{CJK} $B$ \begin{CJK}{UTF8}{mj}和反对称矩阵\end{CJK} $C$, \begin{CJK}{UTF8}{mj}使得\end{CJK} $A=B+C$.

(2) \begin{CJK}{UTF8}{mj}根据上述两个结论\end{CJK}, \begin{CJK}{UTF8}{mj}试给出有关线性空间上的线性变换的结论\end{CJK}, \begin{CJK}{UTF8}{mj}使上述两个结论是其特例\end{CJK}.

\begin{enumerate}
  \setcounter{enumi}{6}
  \item (15 \begin{CJK}{UTF8}{mj}分\end{CJK}) \begin{CJK}{UTF8}{mj}设多项式\end{CJK} $f(x)=\left(x-a_{1}\right)\left(x-a_{2}\right)\left(x-a_{3}\right)\left(x-a_{4}\right)+1$, \begin{CJK}{UTF8}{mj}其中\end{CJK} $a_{1}<a_{2}<a_{3}<a_{4}$ \begin{CJK}{UTF8}{mj}都是整数\end{CJK}. \begin{CJK}{UTF8}{mj}证明\end{CJK}: $f(x)$ \begin{CJK}{UTF8}{mj}在有理数域上可约\end{CJK} $\Longleftrightarrow a_{4}-a_{1}=3$.

  \item (15 \begin{CJK}{UTF8}{mj}分\end{CJK}) \begin{CJK}{UTF8}{mj}设\end{CJK} $S$ \begin{CJK}{UTF8}{mj}是数域\end{CJK} $K$ \begin{CJK}{UTF8}{mj}上某\end{CJK} $n$ \begin{CJK}{UTF8}{mj}元的非齐次线性方程组\end{CJK} $(*)$ \begin{CJK}{UTF8}{mj}的非空解集\end{CJK}, \begin{CJK}{UTF8}{mj}且\end{CJK} (*) \begin{CJK}{UTF8}{mj}的增广矩阵的秩为\end{CJK} $r$. \begin{CJK}{UTF8}{mj}证明\end{CJK}:

\end{enumerate}
(1) \begin{CJK}{UTF8}{mj}如果\end{CJK} $r_{0}, r_{1}, \cdots, r_{s}$ \begin{CJK}{UTF8}{mj}是\end{CJK} $S$ \begin{CJK}{UTF8}{mj}的一组线性无关的向量\end{CJK}, \begin{CJK}{UTF8}{mj}则\end{CJK} $s \leq n-r$.

(2) $S$ \begin{CJK}{UTF8}{mj}中存在线性无关的向量组\end{CJK} $r_{0}, r_{1}, \cdots, r_{n-r}$.

(3) \begin{CJK}{UTF8}{mj}假设\end{CJK} $r_{0}, r_{1}, \cdots, r_{n-r}$ \begin{CJK}{UTF8}{mj}是\end{CJK} $S$ \begin{CJK}{UTF8}{mj}中任意一组线性无关的向量\end{CJK}, \begin{CJK}{UTF8}{mj}则\end{CJK} $\forall \gamma \in S$, \begin{CJK}{UTF8}{mj}存在\end{CJK} $k_{0}, k_{1}, \cdots, k_{n-r}$, \begin{CJK}{UTF8}{mj}且\end{CJK} $\sum_{i=0}^{n-r} k_{i}=1$, \begin{CJK}{UTF8}{mj}使\end{CJK} $\gamma=\sum_{i=0}^{n-r} k_{i} \gamma_{i}$.

\begin{enumerate}
  \setcounter{enumi}{8}
  \item ( 20 \begin{CJK}{UTF8}{mj}分\end{CJK}) \begin{CJK}{UTF8}{mj}设\end{CJK} 4 \begin{CJK}{UTF8}{mj}元二次型\end{CJK}
\end{enumerate}
$$
f\left(x_{1}, x_{2}, x_{3}, x_{4}\right)=2 \sum_{i=1}^{4} x_{i}^{2}-2\left(x_{1} x_{2}+x_{2} x_{3}+x_{3} x_{4}+x_{4} x_{1}\right) .
$$
\begin{CJK}{UTF8}{mj}试用正交线性替换化二次型为标准型\end{CJK}, \begin{CJK}{UTF8}{mj}并求它的正\end{CJK}、\begin{CJK}{UTF8}{mj}负惯性指数以及符号差\end{CJK}.

\begin{enumerate}
  \setcounter{enumi}{9}
  \item ( 20 \begin{CJK}{UTF8}{mj}分\end{CJK}) \begin{CJK}{UTF8}{mj}设\end{CJK} $A$ \begin{CJK}{UTF8}{mj}是一实方阵\end{CJK}, \begin{CJK}{UTF8}{mj}证明\end{CJK}: \begin{CJK}{UTF8}{mj}存在正交矩阵\end{CJK} $S, T$ \begin{CJK}{UTF8}{mj}以及上三角矩阵\end{CJK} $P, Q$, \begin{CJK}{UTF8}{mj}使得\end{CJK}
\end{enumerate}
$$
A=S P=Q T
$$

\begin{enumerate}
  \setcounter{enumi}{10}
  \item ( 20 \begin{CJK}{UTF8}{mj}分\end{CJK}) \begin{CJK}{UTF8}{mj}已知\end{CJK} $W=\mathbb{R}^{3}$ \begin{CJK}{UTF8}{mj}是\end{CJK} 3 \begin{CJK}{UTF8}{mj}维标准的欧式空间\end{CJK}.
\end{enumerate}
(1) \begin{CJK}{UTF8}{mj}设\end{CJK} $V$ \begin{CJK}{UTF8}{mj}是由\end{CJK} $W$ \begin{CJK}{UTF8}{mj}中的向量\end{CJK} $\alpha, \beta, \gamma$ \begin{CJK}{UTF8}{mj}所张成的平行六面体的体积\end{CJK}. \begin{CJK}{UTF8}{mj}证明\end{CJK}: $V=\sqrt{|G(\alpha, \beta, \gamma)|}$, \begin{CJK}{UTF8}{mj}其中矩阵\end{CJK}
$$
G(\alpha, \beta, \gamma)=\left(\begin{array}{lll}
(\alpha, \alpha) & (\alpha, \beta) & (\alpha, \gamma) \\
(\beta, \alpha) & (\beta, \beta) & (\beta, \gamma) \\
(\gamma, \alpha) & (\gamma, \beta) & (\gamma, \gamma)
\end{array}\right)
$$
(2) \begin{CJK}{UTF8}{mj}设\end{CJK} $A B C D$ \begin{CJK}{UTF8}{mj}是一个对棱长度均相等的四面体\end{CJK}. \begin{CJK}{UTF8}{mj}假设四面体的三对对棱长度分别为\end{CJK} $4,5,6$, \begin{CJK}{UTF8}{mj}试求该四面体的\end{CJK} \begin{CJK}{UTF8}{mj}体积\end{CJK}.

\section{2. 华东师范大学 2010 年研究生入学考试试题高等代数 
 李扬 
 微信公众号: sxkyliyang}
\begin{enumerate}
  \item ( 18 \begin{CJK}{UTF8}{mj}分\end{CJK}) \begin{CJK}{UTF8}{mj}设\end{CJK} $A$ \begin{CJK}{UTF8}{mj}是一个秩为\end{CJK} $r$ \begin{CJK}{UTF8}{mj}的\end{CJK} $n$ \begin{CJK}{UTF8}{mj}阶方阵\end{CJK} $(0<r<n)$. \begin{CJK}{UTF8}{mj}证明\end{CJK}:
\end{enumerate}
(1) \begin{CJK}{UTF8}{mj}存在秩为\end{CJK} 1 \begin{CJK}{UTF8}{mj}的方阵\end{CJK} $B_{1}, B_{2}, \cdots, B_{r}$, \begin{CJK}{UTF8}{mj}使得\end{CJK} $A=B_{1}+B_{2}+\cdots+B_{r}$.

(2) \begin{CJK}{UTF8}{mj}存在秩为\end{CJK} $n-1$ \begin{CJK}{UTF8}{mj}的方阵\end{CJK} $C_{1}, C_{2}, \cdots C_{n-r}$, \begin{CJK}{UTF8}{mj}使得\end{CJK} $A=C_{1} C_{2} \cdots C_{n-r}$.

(3) \begin{CJK}{UTF8}{mj}若\end{CJK} $A$ \begin{CJK}{UTF8}{mj}为对称矩阵\end{CJK}, \begin{CJK}{UTF8}{mj}则存在秩为\end{CJK} 1 \begin{CJK}{UTF8}{mj}的对称矩阵\end{CJK} $D_{1}, D_{2}, \cdots, D_{r}$, \begin{CJK}{UTF8}{mj}使得\end{CJK} $A=D_{1}+D_{2}+\cdots+D_{r}$.

\begin{enumerate}
  \setcounter{enumi}{2}
  \item ( 12 \begin{CJK}{UTF8}{mj}分\end{CJK}) \begin{CJK}{UTF8}{mj}设矩阵\end{CJK}
\end{enumerate}
$$
A=\left(\begin{array}{ccc}
\frac{1}{2} & 1 & 1 \\
0 & \frac{1}{3} & 1 \\
0 & 0 & \frac{1}{5}
\end{array}\right)
$$
\begin{CJK}{UTF8}{mj}求\end{CJK} $A^{n} \quad(n=1,2, \cdots)$.

\begin{enumerate}
  \setcounter{enumi}{3}
  \item ( 12 \begin{CJK}{UTF8}{mj}分\end{CJK}) \begin{CJK}{UTF8}{mj}设\end{CJK} $A$ \begin{CJK}{UTF8}{mj}是一个实系数方阵\end{CJK}.
\end{enumerate}
(1) \begin{CJK}{UTF8}{mj}举例说明\end{CJK}: \begin{CJK}{UTF8}{mj}如果\end{CJK} $A$ \begin{CJK}{UTF8}{mj}的行向量组两两正交\end{CJK}, \begin{CJK}{UTF8}{mj}它的列向量组末必两两正交\end{CJK}.

(2) \begin{CJK}{UTF8}{mj}证明\end{CJK}: \begin{CJK}{UTF8}{mj}如果\end{CJK} $A$ \begin{CJK}{UTF8}{mj}的行向量组是长度相等的正交组\end{CJK}, \begin{CJK}{UTF8}{mj}则它的列向量组也是长度相等的正交组\end{CJK}.

\begin{enumerate}
  \setcounter{enumi}{4}
  \item ( 20 \begin{CJK}{UTF8}{mj}分\end{CJK}) \begin{CJK}{UTF8}{mj}设\end{CJK} $V=\mathbb{R}^{n}$ \begin{CJK}{UTF8}{mj}是带有标准内积\end{CJK} $(\alpha, \beta)$ \begin{CJK}{UTF8}{mj}的\end{CJK} $n$ \begin{CJK}{UTF8}{mj}维欧式空间\end{CJK}, $a$ \begin{CJK}{UTF8}{mj}是\end{CJK} $V$ \begin{CJK}{UTF8}{mj}中任意给定的非零向量\end{CJK}. \begin{CJK}{UTF8}{mj}定义\end{CJK} $V$ \begin{CJK}{UTF8}{mj}的反\end{CJK} \begin{CJK}{UTF8}{mj}射变换\end{CJK} $\sigma_{\alpha}$ \begin{CJK}{UTF8}{mj}如下\end{CJK}:
\end{enumerate}
$$
\sigma_{\alpha}(\beta)=\beta-2 \frac{(\beta, \alpha)}{(\alpha, \alpha)} \alpha, \beta \in V
$$
(1) \begin{CJK}{UTF8}{mj}证明\end{CJK}: \begin{CJK}{UTF8}{mj}此反射变换是正交变换\end{CJK}.

(2) \begin{CJK}{UTF8}{mj}证明\end{CJK}: \begin{CJK}{UTF8}{mj}此反射变换可对角化\end{CJK}.

(3) \begin{CJK}{UTF8}{mj}假设\end{CJK} $n=2, \alpha=(a, b)$, \begin{CJK}{UTF8}{mj}求\end{CJK} $\sigma_{\alpha}$ \begin{CJK}{UTF8}{mj}在自然基下\end{CJK} $e_{1}=(1,0), e_{2}=(0,1)$ \begin{CJK}{UTF8}{mj}的矩阵\end{CJK}.

(4)\begin{CJK}{UTF8}{mj}假设\end{CJK} $n=2$, \begin{CJK}{UTF8}{mj}证明\end{CJK}: $V$ \begin{CJK}{UTF8}{mj}的每个正交变换均可写成不超过两个反射变换的乘积\end{CJK}.

\begin{enumerate}
  \setcounter{enumi}{5}
  \item ( 14 \begin{CJK}{UTF8}{mj}分\end{CJK}) \begin{CJK}{UTF8}{mj}设\end{CJK}
\end{enumerate}
$$
J_{n}(c)=\left(\begin{array}{ccccc}
c & 1 & 0 & \cdots & 0 \\
0 & c & 1 & \cdots & 0 \\
0 & 0 & c & \cdots & 0 \\
\vdots & \vdots & \vdots & & \vdots \\
0 & 0 & 0 & \cdots & c
\end{array}\right)
$$
\begin{CJK}{UTF8}{mj}其中\end{CJK} $c \in \mathbb{C}$. \begin{CJK}{UTF8}{mj}求\end{CJK} $J_{n}(c)$ \begin{CJK}{UTF8}{mj}的伴随矩阵\end{CJK} $J_{n}(c)^{\star}$ \begin{CJK}{UTF8}{mj}的\end{CJK} Jordan \begin{CJK}{UTF8}{mj}标准型\end{CJK}.

\begin{enumerate}
  \setcounter{enumi}{6}
  \item (16 \begin{CJK}{UTF8}{mj}分\end{CJK}) \begin{CJK}{UTF8}{mj}设\end{CJK} $\mathbb{R}^{4}$ \begin{CJK}{UTF8}{mj}中的向量组\end{CJK}
\end{enumerate}
$$
\alpha_{1}=(1,0,0,-1), \alpha_{2}=(0,1,1,2), \alpha_{3}=(1,2,2,3),
$$
\begin{CJK}{UTF8}{mj}它们生成的子空间为\end{CJK} $V_{1}$, \begin{CJK}{UTF8}{mj}向量组\end{CJK}
$$
\beta_{1}=(1,-1,-1,-3), \beta_{2}=(-1,1,1,1), \beta_{3}=(3,-3,-3,-7),
$$
\begin{CJK}{UTF8}{mj}它们生成的子空间为\end{CJK} $V_{2}$. \begin{CJK}{UTF8}{mj}求子空间\end{CJK} $V_{1}+V_{2}$ \begin{CJK}{UTF8}{mj}和\end{CJK} $V_{1} \cap V_{2}$ \begin{CJK}{UTF8}{mj}的基和维数\end{CJK}.

\begin{enumerate}
  \setcounter{enumi}{7}
  \item (12 \begin{CJK}{UTF8}{mj}分\end{CJK})\begin{CJK}{UTF8}{mj}设\end{CJK} $n$ \begin{CJK}{UTF8}{mj}是正整数\end{CJK}. \begin{CJK}{UTF8}{mj}证明\end{CJK}: $x^{4}+n$ \begin{CJK}{UTF8}{mj}在有理数域上可约的充要条件是存在整数\end{CJK} $m$, \begin{CJK}{UTF8}{mj}使得\end{CJK} $n=4 m^{4}$. 8. ( 14 \begin{CJK}{UTF8}{mj}分\end{CJK}) \begin{CJK}{UTF8}{mj}设有\end{CJK} $n$ \begin{CJK}{UTF8}{mj}阶矩阵\end{CJK} $A=\left(\begin{array}{ccccc}0 & 1 & \cdots & 1 & 1 \\ 1 & 0 & \cdots & 1 & 1 \\ \vdots & \vdots & \ddots & \vdots & \vdots \\ 1 & 1 & \cdots & 0 & 1 \\ 1 & 1 & \cdots & 1 & 0\end{array}\right)$, \begin{CJK}{UTF8}{mj}其中\end{CJK} $n>1$.
\end{enumerate}
(1) \begin{CJK}{UTF8}{mj}求\end{CJK} $A^{-1}$.

(2) \begin{CJK}{UTF8}{mj}求对角阵\end{CJK} $D$ \begin{CJK}{UTF8}{mj}以及可逆矩阵\end{CJK} $X$, \begin{CJK}{UTF8}{mj}使得\end{CJK} $X^{-1} A X=D$.

\begin{enumerate}
  \setcounter{enumi}{9}
  \item ( 12 \begin{CJK}{UTF8}{mj}分\end{CJK}) \begin{CJK}{UTF8}{mj}求一个\end{CJK} 4 \begin{CJK}{UTF8}{mj}阶方阵\end{CJK} $A$, \begin{CJK}{UTF8}{mj}使\end{CJK} $A$ \begin{CJK}{UTF8}{mj}的第一行为\end{CJK} $(4,-2,9,7)$, \begin{CJK}{UTF8}{mj}且\end{CJK} $|A|=1$. \begin{CJK}{UTF8}{mj}这样的矩阵共有多少个\end{CJK}?

  \item ( 20 \begin{CJK}{UTF8}{mj}分\end{CJK}) \begin{CJK}{UTF8}{mj}设有实二次函数\end{CJK}

\end{enumerate}
$$
\begin{gathered}
f\left(x_{1}, x_{2}, \cdots x_{n}\right)=\sum_{i, j=1}^{n} a_{i j} x_{i} x_{j}+\sum_{i=1}^{n} 2 b_{i} x_{i}+c, a_{i j}=a_{j i} \\
A=\left(a_{i j}\right)_{n \times n}, D=\left(\begin{array}{cc}
A & B^{T} \\
B & c
\end{array}\right) \text { 其中 } B=\left(b_{1}, b_{2}, \cdots, b_{n}\right) .
\end{gathered}
$$
(1) \begin{CJK}{UTF8}{mj}证明\end{CJK}: $A$ \begin{CJK}{UTF8}{mj}负定时\end{CJK}, $f$ \begin{CJK}{UTF8}{mj}有最大值\end{CJK}, \begin{CJK}{UTF8}{mj}且\end{CJK} $f_{\max }=\frac{|D|}{|A|}$.

(2) \begin{CJK}{UTF8}{mj}设\end{CJK} $A$ \begin{CJK}{UTF8}{mj}负定\end{CJK}, \begin{CJK}{UTF8}{mj}试确定当\end{CJK} $x_{1}, x_{2}, \cdots, x_{n}$ \begin{CJK}{UTF8}{mj}为何值时\end{CJK}, $f$ \begin{CJK}{UTF8}{mj}取得最大值\end{CJK}, \begin{CJK}{UTF8}{mj}并说明理由\end{CJK}.

\section{3. 华东师范大学 2011 年研究生入学考试试题高等代数 
 李扬 
 微信公众号: sxkyliyang}
\begin{enumerate}
  \item (10 \begin{CJK}{UTF8}{mj}分\end{CJK})\begin{CJK}{UTF8}{mj}求\end{CJK} $n$ \begin{CJK}{UTF8}{mj}阶矩阵\end{CJK} $M_{n}$ \begin{CJK}{UTF8}{mj}的逆矩阵\end{CJK} $(n \geq 3)$, \begin{CJK}{UTF8}{mj}其中\end{CJK}
\end{enumerate}
$$
M_{n}=\left(\begin{array}{cccccc}
a & 0 & 1 & 0 & \cdots & 0 \\
0 & a & 0 & 1 & \ddots & \vdots \\
\vdots & 0 & \ddots & \ddots & \ddots & 0 \\
\vdots & \vdots & \ddots & \ddots & \ddots & 1 \\
\vdots & \vdots & & \ddots & a & 0 \\
0 & 0 & \cdots & \cdots & 0 & a
\end{array}\right)(a \neq 0)
$$

\begin{enumerate}
  \setcounter{enumi}{2}
  \item ( 15 \begin{CJK}{UTF8}{mj}分\end{CJK}) \begin{CJK}{UTF8}{mj}设\end{CJK} $K$ \begin{CJK}{UTF8}{mj}是一数域\end{CJK}, \begin{CJK}{UTF8}{mj}方阵\end{CJK} $A \in M_{m}(K), C \in M_{n}(K)$, \begin{CJK}{UTF8}{mj}如果对任意矩阵\end{CJK} $B \in M_{m \times n}(K)$ \begin{CJK}{UTF8}{mj}总有\end{CJK}
\end{enumerate}
$$
\operatorname{rank}\left(\begin{array}{cc}
A & B \\
0 & C
\end{array}\right)=\operatorname{rank}(A)+\operatorname{rank}(C)
$$
\begin{CJK}{UTF8}{mj}证明\end{CJK}: $A$ \begin{CJK}{UTF8}{mj}或\end{CJK} $C$ \begin{CJK}{UTF8}{mj}可逆\end{CJK}.

\begin{enumerate}
  \setcounter{enumi}{3}
  \item (10 \begin{CJK}{UTF8}{mj}分\end{CJK}) \begin{CJK}{UTF8}{mj}设有方阵\end{CJK} $A \in M_{m+n}(K)$, \begin{CJK}{UTF8}{mj}且\end{CJK} $a_{i j}=0,1 \leq i, j \leq m$ \begin{CJK}{UTF8}{mj}或\end{CJK} $m+1<i, j \leq m+n$. \begin{CJK}{UTF8}{mj}证明\end{CJK}: \begin{CJK}{UTF8}{mj}若\end{CJK} $\lambda_{0}$ \begin{CJK}{UTF8}{mj}为\end{CJK} $A$ \begin{CJK}{UTF8}{mj}的特\end{CJK} \begin{CJK}{UTF8}{mj}征值\end{CJK}, \begin{CJK}{UTF8}{mj}则\end{CJK} $-\lambda_{0}$ \begin{CJK}{UTF8}{mj}也为\end{CJK} $A$ \begin{CJK}{UTF8}{mj}的特征值\end{CJK}.

  \item (30 \begin{CJK}{UTF8}{mj}分\end{CJK}) \begin{CJK}{UTF8}{mj}设\end{CJK} $\mathcal{A}$ \begin{CJK}{UTF8}{mj}是数域\end{CJK} $K$ \begin{CJK}{UTF8}{mj}上有限维线性空间\end{CJK} $V$ \begin{CJK}{UTF8}{mj}上的线性变换\end{CJK}, $W$ \begin{CJK}{UTF8}{mj}是\end{CJK} $V$ \begin{CJK}{UTF8}{mj}的\end{CJK} $\mathcal{A}$-\begin{CJK}{UTF8}{mj}子空间\end{CJK}.

\end{enumerate}
(1) \begin{CJK}{UTF8}{mj}在\end{CJK} $V$ \begin{CJK}{UTF8}{mj}上定义一个二元关系\end{CJK} $\sim: u \sim v \Leftrightarrow u-v \in W$. \begin{CJK}{UTF8}{mj}证明\end{CJK}: $\sim$ \begin{CJK}{UTF8}{mj}是一个等价关系\end{CJK}.

(2) \begin{CJK}{UTF8}{mj}设\end{CJK} $V / W=\{[u] \mid u \in W\}$ \begin{CJK}{UTF8}{mj}是由\end{CJK} (1) \begin{CJK}{UTF8}{mj}中的等价关系所确定的所有等价类组成的集合\end{CJK}. \begin{CJK}{UTF8}{mj}在此集合上定义加法及\end{CJK} \begin{CJK}{UTF8}{mj}标量乘法运算如下\end{CJK}:
$$
[u]+[v]:=[u+v], k[u]:=[k u], \quad k \in K, u, v \in V
$$
\begin{CJK}{UTF8}{mj}证明\end{CJK}: $V / W$ \begin{CJK}{UTF8}{mj}按照这样定义的运算构成数域\end{CJK} $K$ \begin{CJK}{UTF8}{mj}上线性空间\end{CJK}(\begin{CJK}{UTF8}{mj}称为由\end{CJK} $W$ \begin{CJK}{UTF8}{mj}确定的\end{CJK} $V$ \begin{CJK}{UTF8}{mj}的商空间\end{CJK}).

(3) \begin{CJK}{UTF8}{mj}证明\end{CJK}:
$$
\operatorname{dim}(V / W)=\operatorname{dim}(V)-\operatorname{dim}(W)
$$
(4) \begin{CJK}{UTF8}{mj}定义\end{CJK} $V / W$ \begin{CJK}{UTF8}{mj}上的变换\end{CJK} $\mathcal{B}: \mathcal{B}([u])=[\mathcal{A}(u)], u \in V$. \begin{CJK}{UTF8}{mj}证明\end{CJK}: $\mathcal{B}$ \begin{CJK}{UTF8}{mj}是商空间\end{CJK} $V / W$ \begin{CJK}{UTF8}{mj}上的线性变换\end{CJK}.

(5) \begin{CJK}{UTF8}{mj}证明\end{CJK}:
$$
f_{\mathcal{A}}(\lambda)=f_{\mathcal{A} \mid W}(\lambda) f_{\mathcal{B}}(\lambda),
$$
\begin{CJK}{UTF8}{mj}其中\end{CJK} $f_{\mathcal{A}}(\lambda)$ \begin{CJK}{UTF8}{mj}表示线性变换\end{CJK} $\mathcal{A}$ \begin{CJK}{UTF8}{mj}的特征多项式\end{CJK}, $\left.\mathcal{A}\right|_{W}$ \begin{CJK}{UTF8}{mj}表示\end{CJK} $\mathcal{A}$ \begin{CJK}{UTF8}{mj}在\end{CJK} $W$ \begin{CJK}{UTF8}{mj}上的限制\end{CJK}.

\begin{enumerate}
  \setcounter{enumi}{5}
  \item ( 15 \begin{CJK}{UTF8}{mj}分\end{CJK}) \begin{CJK}{UTF8}{mj}求出所有满足条件\end{CJK} $(x-1) f(x+1)=(x+2) f(x)$ \begin{CJK}{UTF8}{mj}的非零实系数多项式\end{CJK} $f(x)$.

  \item ( 25 \begin{CJK}{UTF8}{mj}分\end{CJK}) \begin{CJK}{UTF8}{mj}设\end{CJK} $a_{1}=(1,2,-1,0,4), a_{2}=(-1,3,2,4,1), a_{3}=(2,9,-1,4,13), W=L\left(a_{1}, a_{2}, a_{3}\right)$ \begin{CJK}{UTF8}{mj}是由这三个向\end{CJK} \begin{CJK}{UTF8}{mj}量生成的数域\end{CJK} $K$ \begin{CJK}{UTF8}{mj}上的线性空间\end{CJK} $K^{5}$ \begin{CJK}{UTF8}{mj}的子空间\end{CJK}.

\end{enumerate}
(1) \begin{CJK}{UTF8}{mj}求以\end{CJK} $W$ \begin{CJK}{UTF8}{mj}作为其解空间的齐次线性方程组\end{CJK};

(2) \begin{CJK}{UTF8}{mj}求以\end{CJK} $W^{\prime}=\{\eta+\alpha \mid \alpha \in W\}$ \begin{CJK}{UTF8}{mj}为其解集的非齐次线性方程组\end{CJK}, \begin{CJK}{UTF8}{mj}其中\end{CJK} $\eta=(1,2,1,2,1)$.

\begin{enumerate}
  \setcounter{enumi}{7}
  \item (15 \begin{CJK}{UTF8}{mj}分\end{CJK}) \begin{CJK}{UTF8}{mj}证明\end{CJK}: \begin{CJK}{UTF8}{mj}三次方程\end{CJK} $x^{3}-a_{1} x^{2}+a_{2} x-a_{3}=0$ \begin{CJK}{UTF8}{mj}的三个根成等差数列的充要条件是\end{CJK}
\end{enumerate}
$$
2 a_{1}^{3}-9 a_{1} a_{2}+27 a_{3}=0 .
$$

\begin{enumerate}
  \setcounter{enumi}{8}
  \item (15 \begin{CJK}{UTF8}{mj}分\end{CJK}) \begin{CJK}{UTF8}{mj}求矩阵\end{CJK}
\end{enumerate}
$$
A=\left(\begin{array}{cccc}
2 & 1 & -1 & 0 \\
20 & 3 & -1 & -20 \\
20 & 3 & 1 & -20 \\
5 & 1 & -1 & -3
\end{array}\right)
$$
\begin{CJK}{UTF8}{mj}的特征值\end{CJK}、\begin{CJK}{UTF8}{mj}极小多项式以及\end{CJK}.Jordan \begin{CJK}{UTF8}{mj}标准型\end{CJK}.

\begin{enumerate}
  \setcounter{enumi}{9}
  \item (15 \begin{CJK}{UTF8}{mj}分\end{CJK}) \begin{CJK}{UTF8}{mj}设\end{CJK}
\end{enumerate}
$$
f\left(x_{1}, x_{2}, \cdots, x_{n}\right)=\sum_{i=1}^{n} \sum_{j=1}^{n} a_{i j} x_{i} x_{j}
$$
\begin{CJK}{UTF8}{mj}是一个实二次型\end{CJK}, \begin{CJK}{UTF8}{mj}其中\end{CJK} $A=\left(a_{i j}\right)$ \begin{CJK}{UTF8}{mj}是实对称矩阵\end{CJK}. \begin{CJK}{UTF8}{mj}将二次型看作\end{CJK} $n$ \begin{CJK}{UTF8}{mj}元实函数\end{CJK}, \begin{CJK}{UTF8}{mj}用代数方法确定它在\end{CJK}
$$
S=\left\{X=\left(x_{1}, x_{2}, \cdots x_{n}\right)^{T} \in \mathbb{R}^{n} \mid x_{1}^{2}+x_{2}^{2}+\cdots+x_{n}^{2}=1\right\}
$$
\begin{CJK}{UTF8}{mj}上的取值范围\end{CJK}.

\section{4. 华东师范大学 2012 年研究生入学考试试题高等代数 
 李扬 
 微信公众号: sxkyliyang}
\begin{enumerate}
  \item (10 \begin{CJK}{UTF8}{mj}分\end{CJK}) \begin{CJK}{UTF8}{mj}计算行列式\end{CJK}:
\end{enumerate}
$$
D=\left|\begin{array}{lllll}
1 & 4 & 6 & 4 & 1 \\
1 & 1 & 4 & 6 & 4 \\
4 & 1 & 1 & 4 & 6 \\
6 & 4 & 1 & 1 & 4 \\
4 & 6 & 4 & 1 & 1
\end{array}\right|
$$

\begin{enumerate}
  \setcounter{enumi}{2}
  \item (15 \begin{CJK}{UTF8}{mj}分\end{CJK}) \begin{CJK}{UTF8}{mj}试给出三次方程\end{CJK} $x^{3}-a_{1} x^{2}+a_{2} x-a_{3}=0$ \begin{CJK}{UTF8}{mj}的三个根成等比数列的充分必要条件\end{CJK} $\left(\right.$ \begin{CJK}{UTF8}{mj}用\end{CJK} $a_{1}, a_{2}, a_{3}$ \begin{CJK}{UTF8}{mj}的形式\end{CJK} \begin{CJK}{UTF8}{mj}表示\end{CJK}, \begin{CJK}{UTF8}{mj}且尽可能简单\end{CJK}), \begin{CJK}{UTF8}{mj}并证明你的结论\end{CJK}.

  \item ( 20 \begin{CJK}{UTF8}{mj}分\end{CJK}) \begin{CJK}{UTF8}{mj}设\end{CJK} $K$ \begin{CJK}{UTF8}{mj}是数域\end{CJK}, $W \in K^{n}$ \begin{CJK}{UTF8}{mj}是\end{CJK} $K$ \begin{CJK}{UTF8}{mj}上的线性方程组\end{CJK} $A X=B$ \begin{CJK}{UTF8}{mj}的非空解集\end{CJK}, \begin{CJK}{UTF8}{mj}其中\end{CJK} $A \in M_{m \times n}(K)$, $X=\left(x_{1}, x_{2}, \cdots, x_{n}\right)^{T}, B \in M_{m \times 1}(K)$. \begin{CJK}{UTF8}{mj}证明\end{CJK}:

\end{enumerate}
(1) \begin{CJK}{UTF8}{mj}存在该方程组的特解\end{CJK} $\gamma_{0}$, \begin{CJK}{UTF8}{mj}及\end{CJK} $K^{n}$ \begin{CJK}{UTF8}{mj}的子空间\end{CJK} $V$, \begin{CJK}{UTF8}{mj}使\end{CJK} $W=\gamma_{0}+V=\left\{\gamma_{0}+\eta \mid \eta \in V\right\}$;

(2) \begin{CJK}{UTF8}{mj}若取\end{CJK} $\gamma_{0}=(2,0,1,2)^{T}, V$ \begin{CJK}{UTF8}{mj}是由\end{CJK} $(2,1,0,0)^{T},(4,0,-1,0)^{T},(1,0,0,1)^{T},(3,0,-1,-1)^{T}$ \begin{CJK}{UTF8}{mj}生成的\end{CJK} $K^{n}$ \begin{CJK}{UTF8}{mj}的子\end{CJK} \begin{CJK}{UTF8}{mj}空间\end{CJK}. \begin{CJK}{UTF8}{mj}试求一线性方程组\end{CJK}, \begin{CJK}{UTF8}{mj}使其解集等于\end{CJK} $\gamma_{0}+V$.

\begin{enumerate}
  \setcounter{enumi}{4}
  \item ( 20 \begin{CJK}{UTF8}{mj}分\end{CJK}) \begin{CJK}{UTF8}{mj}设\end{CJK} $A$ \begin{CJK}{UTF8}{mj}是\end{CJK} $n$ \begin{CJK}{UTF8}{mj}阶实对称矩阵\end{CJK}. \begin{CJK}{UTF8}{mj}证明\end{CJK}:
\end{enumerate}
(1) \begin{CJK}{UTF8}{mj}存在正定矩阵\end{CJK} $B$ \begin{CJK}{UTF8}{mj}和负定矩阵\end{CJK} $C$, \begin{CJK}{UTF8}{mj}使得\end{CJK} $A=B+C$. \begin{CJK}{UTF8}{mj}这样的分解唯一吗\end{CJK}? \begin{CJK}{UTF8}{mj}说明理由\end{CJK}.

(2) \begin{CJK}{UTF8}{mj}如果\end{CJK} $A$ \begin{CJK}{UTF8}{mj}正定\end{CJK}, \begin{CJK}{UTF8}{mj}且\end{CJK} $n>1$, \begin{CJK}{UTF8}{mj}则存在不定矩阵\end{CJK} $D$, \begin{CJK}{UTF8}{mj}使得\end{CJK} $A=D^{2}$.

\begin{enumerate}
  \setcounter{enumi}{5}
  \item (14 \begin{CJK}{UTF8}{mj}分\end{CJK}) \begin{CJK}{UTF8}{mj}统计学中将各元素为非负数\end{CJK}, \begin{CJK}{UTF8}{mj}且每行元素之和为\end{CJK} 1 \begin{CJK}{UTF8}{mj}的方阵\end{CJK} $A$ \begin{CJK}{UTF8}{mj}称为\end{CJK}"\begin{CJK}{UTF8}{mj}转移概率矩阵\end{CJK}". \begin{CJK}{UTF8}{mj}证明\end{CJK}:
\end{enumerate}
(1) \begin{CJK}{UTF8}{mj}转移概率矩阵\end{CJK} $A$ \begin{CJK}{UTF8}{mj}必以\end{CJK} 1 \begin{CJK}{UTF8}{mj}为其一个特征值\end{CJK};

(2) \begin{CJK}{UTF8}{mj}转移概率矩阵\end{CJK} $A$ \begin{CJK}{UTF8}{mj}与\end{CJK} $B$ \begin{CJK}{UTF8}{mj}的乘积仍是转移概率矩阵\end{CJK}.

\begin{enumerate}
  \setcounter{enumi}{6}
  \item (20 \begin{CJK}{UTF8}{mj}分\end{CJK}) \begin{CJK}{UTF8}{mj}设\end{CJK} $K_{n}$ \begin{CJK}{UTF8}{mj}是数域\end{CJK} $K$ \begin{CJK}{UTF8}{mj}上的线性空间\end{CJK}.
\end{enumerate}
(1) \begin{CJK}{UTF8}{mj}若\end{CJK} $K_{n}=V_{1} \oplus V_{2}$, \begin{CJK}{UTF8}{mj}其中\end{CJK} $V_{1}, V_{2}$ \begin{CJK}{UTF8}{mj}是\end{CJK} $K_{n}$ \begin{CJK}{UTF8}{mj}的两个非平凡子空间\end{CJK}. \begin{CJK}{UTF8}{mj}证明\end{CJK}: \begin{CJK}{UTF8}{mj}存在唯一幂等矩阵\end{CJK} $A \in M_{n}(K)$, \begin{CJK}{UTF8}{mj}使\end{CJK}
$$
V_{1}=\left\{X \in K^{n} \mid A X=0\right\}, V_{2}=\left\{X \in K^{n} \mid A X=X\right\}
$$
(2)\begin{CJK}{UTF8}{mj}若取子空间\end{CJK}
$$
V_{1}=\left\{X=\left(a_{1}, a_{2}, \cdots, a_{n}\right)^{T} \in K^{n} \mid a_{1}+a_{2}+\cdots+a_{n}=0\right\}, \quad V_{2}=\left\{X \in K^{n} \mid X=(a, a, \cdots, a)^{T}\right\} .
$$
\begin{CJK}{UTF8}{mj}证明\end{CJK}: $K^{n}=V_{1} \oplus V_{2}$, \begin{CJK}{UTF8}{mj}并求\end{CJK} $(1)$ \begin{CJK}{UTF8}{mj}小题中对应的幂等矩阵\end{CJK} $A$.

\begin{enumerate}
  \setcounter{enumi}{7}
  \item (16 \begin{CJK}{UTF8}{mj}分\end{CJK}) \begin{CJK}{UTF8}{mj}求矩阵\end{CJK}
\end{enumerate}
$$
A=\left(\begin{array}{cccc}
5 & 0 & -4 & -4 \\
6 & 8 & 1 & 8 \\
14 & 7 & -6 & 0 \\
-6 & -7 & -1 & -7
\end{array}\right)
$$
\begin{CJK}{UTF8}{mj}的特征多项式\end{CJK}、\begin{CJK}{UTF8}{mj}初等因子组\end{CJK}、\begin{CJK}{UTF8}{mj}极小多项式以及\end{CJK} Jordan \begin{CJK}{UTF8}{mj}标准型\end{CJK}.

\begin{enumerate}
  \setcounter{enumi}{8}
  \item ( 20 \begin{CJK}{UTF8}{mj}分\end{CJK}) \begin{CJK}{UTF8}{mj}证明\end{CJK}: \begin{CJK}{UTF8}{mj}对每个正整数\end{CJK} $n$, \begin{CJK}{UTF8}{mj}均存在\end{CJK} $A \in M_{n}(\mathbb{R})$, \begin{CJK}{UTF8}{mj}使得\end{CJK} $A^{3}=A+2 E_{n}$. \begin{CJK}{UTF8}{mj}对任何满足条件\end{CJK} $A^{3}=A+2 E_{n}$ \begin{CJK}{UTF8}{mj}的\end{CJK} \begin{CJK}{UTF8}{mj}矩阵\end{CJK} $A$, \begin{CJK}{UTF8}{mj}均有\end{CJK} $|A|>0$.

  \item ( 15 \begin{CJK}{UTF8}{mj}分\end{CJK}) \begin{CJK}{UTF8}{mj}求所有\end{CJK} 3 \begin{CJK}{UTF8}{mj}阶复矩阵\end{CJK} $A$, \begin{CJK}{UTF8}{mj}使得\end{CJK} $A$ \begin{CJK}{UTF8}{mj}与\end{CJK} $A^{2}$ \begin{CJK}{UTF8}{mj}相似\end{CJK}.

\end{enumerate}
\section{5. 华东师范大学 2013 年研究生入学考试试题高等代数 
 李扬 
 微信公众号: sxkyliyang}
\begin{enumerate}
  \item ( 15 \begin{CJK}{UTF8}{mj}分\end{CJK}) \begin{CJK}{UTF8}{mj}计算下列行列式的值\end{CJK}:
\end{enumerate}
(1)
$$
\left|\begin{array}{cccc}
2 & -1 & 1 & 2 \\
-1 & 2 & -1 & 0 \\
0 & -1 & 2 & -1 \\
-1 & 2 & -2 & 2
\end{array}\right|
$$
(2)
$$
\left|\begin{array}{cccccc}
1 & 2 & 2 & 2 & \cdots & 2 \\
2 & 2 & 2 & 2 & \cdots & 2 \\
2 & 2 & 3 & 2 & \cdots & 2 \\
2 & 2 & 2 & 4 & \cdots & 2 \\
\vdots & \vdots & \vdots & \vdots & & \vdots \\
2 & 2 & 2 & 2 & \cdots & n
\end{array}\right|
$$

\begin{enumerate}
  \setcounter{enumi}{2}
  \item (12 \begin{CJK}{UTF8}{mj}分\end{CJK}) \begin{CJK}{UTF8}{mj}设\end{CJK} $A$ \begin{CJK}{UTF8}{mj}为\end{CJK} $4 \times 2$ \begin{CJK}{UTF8}{mj}矩阵\end{CJK}, $B$ \begin{CJK}{UTF8}{mj}为\end{CJK} $2 \times 4$ \begin{CJK}{UTF8}{mj}矩阵\end{CJK}, \begin{CJK}{UTF8}{mj}且满足\end{CJK}
\end{enumerate}
$$
A B=\left(\begin{array}{cccc}
1 & 0 & -1 & 0 \\
0 & 1 & 0 & -1 \\
-1 & 0 & 1 & 0 \\
0 & -1 & 0 & 1
\end{array}\right)
$$
\begin{CJK}{UTF8}{mj}求\end{CJK} $B A$.

\begin{enumerate}
  \setcounter{enumi}{3}
  \item ( 20 \begin{CJK}{UTF8}{mj}分\end{CJK}) \begin{CJK}{UTF8}{mj}设\end{CJK}
\end{enumerate}
$$
A=\left(\begin{array}{ccccc}
-3 & 1 & 0 & 0 & 0 \\
0 & -3 & 0 & 0 & 0 \\
0 & 0 & 0 & 0 & 0 \\
0 & 0 & 1 & 0 & 0 \\
0 & 0 & 0 & 1 & 0
\end{array}\right)
$$
\begin{CJK}{UTF8}{mj}求\end{CJK} $A^{2}$ \begin{CJK}{UTF8}{mj}的不变因子组\end{CJK}, \begin{CJK}{UTF8}{mj}初等因子组\end{CJK}, \begin{CJK}{UTF8}{mj}极小多项式\end{CJK}, \begin{CJK}{UTF8}{mj}以及\end{CJK} Jordan \begin{CJK}{UTF8}{mj}标准型\end{CJK}.

\begin{enumerate}
  \setcounter{enumi}{4}
  \item (15 \begin{CJK}{UTF8}{mj}分\end{CJK}) \begin{CJK}{UTF8}{mj}设多项式\end{CJK} $f(x)=x^{4}-x^{3}+2 x^{2}-x+1, g(x)=x^{3}-2 x^{2}+2 x-1$. \begin{CJK}{UTF8}{mj}求\end{CJK} $f(x), g(x)$ \begin{CJK}{UTF8}{mj}的首一最大公因式\end{CJK} $(f(x), g(x))$ \begin{CJK}{UTF8}{mj}以及多项式\end{CJK} $u(x), v(x)$, \begin{CJK}{UTF8}{mj}使得\end{CJK}
\end{enumerate}
$$
(f(x), g(x))=u(x) f(x)+v(x) g(x) .
$$

\begin{enumerate}
  \setcounter{enumi}{5}
  \item ( 15 \begin{CJK}{UTF8}{mj}分\end{CJK}) \begin{CJK}{UTF8}{mj}求次数最低的多项式\end{CJK} $f(x)$, \begin{CJK}{UTF8}{mj}使得\end{CJK} $f(1)=1, f(-1)=-1, f(2)=2, f(-2)=-8$.

  \item (20 \begin{CJK}{UTF8}{mj}分\end{CJK}) \begin{CJK}{UTF8}{mj}用正交线性替换化下列二次型为标准型\end{CJK}

\end{enumerate}
$$
2 x_{1}^{2}+4 x_{1} x_{2}-4 x_{1} x_{3}+5 x_{2}^{2}-8 x_{2} x_{3}+5 x_{3}^{2}
$$

\begin{enumerate}
  \setcounter{enumi}{7}
  \item (18 \begin{CJK}{UTF8}{mj}分\end{CJK}) \begin{CJK}{UTF8}{mj}证明\end{CJK}:
\end{enumerate}
(1) \begin{CJK}{UTF8}{mj}矩阵\end{CJK} $A$ \begin{CJK}{UTF8}{mj}幂零的充要条件为\end{CJK} 0 \begin{CJK}{UTF8}{mj}是\end{CJK} $A$ \begin{CJK}{UTF8}{mj}的唯一特征值\end{CJK}.

(2) \begin{CJK}{UTF8}{mj}对于任何矩阵\end{CJK} $A \in M_{n}(C)$, \begin{CJK}{UTF8}{mj}存在可对角化矩阵\end{CJK} $B$ \begin{CJK}{UTF8}{mj}以及幂零阵\end{CJK} $C$, \begin{CJK}{UTF8}{mj}使得\end{CJK} $B C=C B$, \begin{CJK}{UTF8}{mj}且\end{CJK} $A$ \begin{CJK}{UTF8}{mj}有分解\end{CJK} $A=B+C$, \begin{CJK}{UTF8}{mj}并且这样的分解是唯一的\end{CJK}. 8. ( 15 \begin{CJK}{UTF8}{mj}分\end{CJK}) \begin{CJK}{UTF8}{mj}设\end{CJK} $(a, b)$ \begin{CJK}{UTF8}{mj}是欧几里得空间\end{CJK} $V$ \begin{CJK}{UTF8}{mj}的内积函数\end{CJK}. \begin{CJK}{UTF8}{mj}对于任何给定的\end{CJK} $\gamma \in V$, \begin{CJK}{UTF8}{mj}定义\end{CJK} $V$ \begin{CJK}{UTF8}{mj}的函数\end{CJK} $f_{\gamma}: \alpha \mapsto(\alpha, \gamma)$, \begin{CJK}{UTF8}{mj}即\end{CJK} $f_{\gamma}(\alpha)=(\alpha, \gamma)$. \begin{CJK}{UTF8}{mj}证明\end{CJK}:

(1) $f_{\gamma}$ \begin{CJK}{UTF8}{mj}是\end{CJK} $V$ \begin{CJK}{UTF8}{mj}的线性函数\end{CJK};

(2) $V$ \begin{CJK}{UTF8}{mj}的线性函数都具有\end{CJK} $f_{\gamma}$ \begin{CJK}{UTF8}{mj}的形式\end{CJK}.

\begin{enumerate}
  \setcounter{enumi}{9}
  \item ( 20 \begin{CJK}{UTF8}{mj}分\end{CJK}) \begin{CJK}{UTF8}{mj}证明\end{CJK}: \begin{CJK}{UTF8}{mj}对任何实系数可逆矩阵\end{CJK} $A$, \begin{CJK}{UTF8}{mj}存在正交矩阵\end{CJK} $Q$ \begin{CJK}{UTF8}{mj}以及上三角矩阵\end{CJK} $R$, \begin{CJK}{UTF8}{mj}使得\end{CJK} $A=Q R$, \begin{CJK}{UTF8}{mj}且如果要求\end{CJK} $T$ \begin{CJK}{UTF8}{mj}的主对角线上的元素均大于零\end{CJK}, \begin{CJK}{UTF8}{mj}则此分解是唯一的\end{CJK}. \begin{CJK}{UTF8}{mj}对\end{CJK}
\end{enumerate}
$$
A=\left(\begin{array}{ccc}
1 & 1 & 0 \\
2 & -1 & 5 \\
-2 & 4 & 2
\end{array}\right)
$$
\begin{CJK}{UTF8}{mj}求这样的分解\end{CJK}.

\section{6. 华东师范大学 2014 年研究生入学考试试题高等代数 
 李扬 
 微信公众号: sxkyliyang}
\begin{enumerate}
  \item ( 10 \begin{CJK}{UTF8}{mj}分\end{CJK}) \begin{CJK}{UTF8}{mj}计算行列式\end{CJK}
\end{enumerate}
$$
D=\left|\begin{array}{cccc}
a^{2}+m & b a & c a & d a \\
a b & b^{2}+m & c b & d b \\
a c & b c & c^{2}+m & d c \\
a d & b d & c d & d^{2}+m
\end{array}\right|
$$

\begin{enumerate}
  \setcounter{enumi}{2}
  \item (15 \begin{CJK}{UTF8}{mj}分\end{CJK}) \begin{CJK}{UTF8}{mj}设矩阵\end{CJK} $A=\left(a_{i j}\right) \in M_{m \times n}(\mathbb{R}), B=\left(b_{1}, \cdots, b_{m}\right)^{T} \in M_{m \times 1}(\mathbb{R})$. \begin{CJK}{UTF8}{mj}证明\end{CJK}: \begin{CJK}{UTF8}{mj}线性方程组\end{CJK} $A^{T} A X=A^{T} B$ \begin{CJK}{UTF8}{mj}一定有解\end{CJK}.

  \item (15 \begin{CJK}{UTF8}{mj}分\end{CJK}) \begin{CJK}{UTF8}{mj}设矩阵\end{CJK} $A \in M_{n}(\mathbb{C})$ \begin{CJK}{UTF8}{mj}的特征值互不相同\end{CJK}. \begin{CJK}{UTF8}{mj}定义\end{CJK} $C(A)=\left\{B \in M_{n}(\mathbb{C}) \mid A B=B A\right\}$.

\end{enumerate}
(1) \begin{CJK}{UTF8}{mj}验证\end{CJK}: $C(A)$ \begin{CJK}{UTF8}{mj}是复线性空间\end{CJK} $M_{n}(\mathbb{C})$ \begin{CJK}{UTF8}{mj}的线性子空间\end{CJK}.

(2) \begin{CJK}{UTF8}{mj}证明\end{CJK}: \begin{CJK}{UTF8}{mj}对于任意\end{CJK} $B, C \in C(A)$, \begin{CJK}{UTF8}{mj}有\end{CJK} $B C=C B$.

\begin{enumerate}
  \setcounter{enumi}{4}
  \item ( 20 \begin{CJK}{UTF8}{mj}分\end{CJK}) \begin{CJK}{UTF8}{mj}设\end{CJK} $V$ \begin{CJK}{UTF8}{mj}是数域\end{CJK} $K$ \begin{CJK}{UTF8}{mj}上的\end{CJK} 4 \begin{CJK}{UTF8}{mj}维线性空间\end{CJK}, $a_{1}, a_{2}, a_{3}, a_{4}$ \begin{CJK}{UTF8}{mj}是\end{CJK} $V$ \begin{CJK}{UTF8}{mj}的一组基\end{CJK}. \begin{CJK}{UTF8}{mj}若\end{CJK} $\mathcal{A}$ \begin{CJK}{UTF8}{mj}是\end{CJK} $V$ \begin{CJK}{UTF8}{mj}上的线性变换\end{CJK}, \begin{CJK}{UTF8}{mj}且在基\end{CJK} $a_{1}, a_{2}, a_{3}, a_{4}$ \begin{CJK}{UTF8}{mj}下的矩阵为准对角阵\end{CJK} $\left(\begin{array}{cccc}1 & 1 & 0 & 0 \\ 0 & 1 & 0 & 0 \\ 0 & 0 & 3 & 0 \\ 0 & 0 & 0 & 3\end{array}\right)$, \begin{CJK}{UTF8}{mj}试求所有\end{CJK} $\mathcal{A}$-\begin{CJK}{UTF8}{mj}不变子空间\end{CJK}.

  \item ( 15 \begin{CJK}{UTF8}{mj}分\end{CJK}) \begin{CJK}{UTF8}{mj}设\end{CJK} $A \in M_{n}(\mathbb{R})$ \begin{CJK}{UTF8}{mj}是半正定矩阵\end{CJK}, \begin{CJK}{UTF8}{mj}且存在整数\end{CJK} $m>1$, \begin{CJK}{UTF8}{mj}使得\end{CJK} $A^{m}=E_{n}$, \begin{CJK}{UTF8}{mj}求\end{CJK} $A$. \begin{CJK}{UTF8}{mj}若将上述\end{CJK} “\begin{CJK}{UTF8}{mj}半正定\end{CJK}” \begin{CJK}{UTF8}{mj}的条件改为\end{CJK} “\begin{CJK}{UTF8}{mj}半负定\end{CJK}”, \begin{CJK}{UTF8}{mj}你能得出什么结论\end{CJK}?

  \item ( 20 \begin{CJK}{UTF8}{mj}分\end{CJK}) \begin{CJK}{UTF8}{mj}设\end{CJK} $V$ \begin{CJK}{UTF8}{mj}是实数域上的\end{CJK} $n$ \begin{CJK}{UTF8}{mj}维欧式空间\end{CJK}, $e_{1}, \cdots, e_{n}$ \begin{CJK}{UTF8}{mj}是一组基\end{CJK}, \begin{CJK}{UTF8}{mj}满足内积\end{CJK} $\left(e_{i}, e_{j}\right) \leq 0(i \neq j)$.

\end{enumerate}
(1) \begin{CJK}{UTF8}{mj}证明\end{CJK}: \begin{CJK}{UTF8}{mj}存在一个非零向量\end{CJK} $v \in V$, \begin{CJK}{UTF8}{mj}满足\end{CJK} $\left(e_{i}, v\right) \geq 0,(i=1,2, \cdots, n)$.

(2) \begin{CJK}{UTF8}{mj}假设\end{CJK} $v=a_{1} e_{1}+\cdots+a_{n} e_{n} \in V$ \begin{CJK}{UTF8}{mj}是任何满足\end{CJK} (1) \begin{CJK}{UTF8}{mj}的向量\end{CJK}, \begin{CJK}{UTF8}{mj}证明\end{CJK}: $a_{i} \geq 0, i=1,2, \cdots, n$.

(3) \begin{CJK}{UTF8}{mj}设\end{CJK} $u=b_{1} e_{1}+\cdots+b_{n} e_{n} \in V$ \begin{CJK}{UTF8}{mj}是另一个满足\end{CJK} $(1)$ \begin{CJK}{UTF8}{mj}的向量\end{CJK}, \begin{CJK}{UTF8}{mj}并定义\end{CJK} $w=c_{1} e_{1}+\cdots+c_{n} e_{n} \in V$, \begin{CJK}{UTF8}{mj}其中\end{CJK} $c_{i}=\min \left\{a_{i}, b_{i}\right\}, i=1,2, \cdots, n$, \begin{CJK}{UTF8}{mj}证明\end{CJK}: \begin{CJK}{UTF8}{mj}向量\end{CJK} $w$ \begin{CJK}{UTF8}{mj}也满足\end{CJK} (1).

\begin{enumerate}
  \setcounter{enumi}{7}
  \item (25 \begin{CJK}{UTF8}{mj}分\end{CJK}) \begin{CJK}{UTF8}{mj}设\end{CJK} $n$ \begin{CJK}{UTF8}{mj}阶矩阵\end{CJK}
\end{enumerate}
$$
A_{n}=\left(\begin{array}{cccccc}
-2 & 1 & & & & \\
1 & -2 & 1 & & & \\
& 1 & -2 & 1 & & \\
& & 1 & \ddots & \ddots & \\
& & & \ddots & \ddots & 1 \\
& & & & 1 & -2
\end{array}\right)
$$
\begin{CJK}{UTF8}{mj}其特征多项式记为\end{CJK} $f_{n}(\lambda)$.

(1) \begin{CJK}{UTF8}{mj}证明\end{CJK}: $f_{n}(\lambda)=(\lambda+2) f_{n-1}(\lambda)-f n-2(\lambda)$.

(2) \begin{CJK}{UTF8}{mj}求\end{CJK} $f_{1}(\lambda), f_{2}(\lambda), f_{3}(\lambda)$, \begin{CJK}{UTF8}{mj}并求相应的特征值及特征向量\end{CJK}.

(3) \begin{CJK}{UTF8}{mj}试写出\end{CJK} $A_{3}$ \begin{CJK}{UTF8}{mj}的若尔当典范型\end{CJK}.

\begin{enumerate}
  \setcounter{enumi}{8}
  \item ( 15 \begin{CJK}{UTF8}{mj}分\end{CJK}) \begin{CJK}{UTF8}{mj}设\end{CJK} $A \in M_{n}(\mathbb{C})$ \begin{CJK}{UTF8}{mj}是一个幂零矩阵\end{CJK}(\begin{CJK}{UTF8}{mj}即存在正整数\end{CJK} $m$, \begin{CJK}{UTF8}{mj}使得\end{CJK} $\left.A^{m}=0\right)$, \begin{CJK}{UTF8}{mj}定义矩阵\end{CJK} $\exp (A)=\sum_{k=0}^{\infty} \frac{A^{k}}{k !}$. \begin{CJK}{UTF8}{mj}证明\end{CJK}: $\exp (A)$ \begin{CJK}{UTF8}{mj}是可逆矩阵\end{CJK}, \begin{CJK}{UTF8}{mj}且\end{CJK} $\exp (A)^{-1}=\exp (-A)$.

  \item ( 15 \begin{CJK}{UTF8}{mj}分\end{CJK}) \begin{CJK}{UTF8}{mj}设\end{CJK} $A_{1}, A_{2}, \cdots, A_{n}$ \begin{CJK}{UTF8}{mj}都是数域\end{CJK} $K$ \begin{CJK}{UTF8}{mj}上的\end{CJK} $n$ \begin{CJK}{UTF8}{mj}阶非零矩阵\end{CJK}, $A_{i}^{2}=A_{i}(i=1,2, \cdots, n), A_{i} A_{j}=0(i \neq j ; i, j=$ $1,2, \cdots, n)$. (1) \begin{CJK}{UTF8}{mj}证明\end{CJK}: $A_{i}(i=1,2, \cdots, n)$ \begin{CJK}{UTF8}{mj}都可以对角化\end{CJK};

\end{enumerate}
(2) \begin{CJK}{UTF8}{mj}求数域\end{CJK} $K$ \begin{CJK}{UTF8}{mj}上的\end{CJK} $n$ \begin{CJK}{UTF8}{mj}阶可逆矩阵\end{CJK} $P$, \begin{CJK}{UTF8}{mj}使得\end{CJK} $P^{-1} A_{1} P, P^{-1} A_{2} P, \cdots, P^{-1} A_{n} P$ \begin{CJK}{UTF8}{mj}为对角矩阵\end{CJK}.

\section{7. 华东师范大学 2015 年研究生入学考试试题高等代数 
 李扬 
 微信公众号: sxkyliyang}
\begin{enumerate}
  \item ( 20 \begin{CJK}{UTF8}{mj}分\end{CJK}) \begin{CJK}{UTF8}{mj}求一个\end{CJK} 3 \begin{CJK}{UTF8}{mj}阶实对称矩阵\end{CJK} $A$, \begin{CJK}{UTF8}{mj}满足\end{CJK}: \begin{CJK}{UTF8}{mj}特征值为\end{CJK} $6,3,3$, \begin{CJK}{UTF8}{mj}且\end{CJK} 6 \begin{CJK}{UTF8}{mj}对应的特征向量为\end{CJK} $a_{1}=(1,1,1)^{T}$.
\end{enumerate}
2 . ( 20 \begin{CJK}{UTF8}{mj}分\end{CJK}) \begin{CJK}{UTF8}{mj}设矩阵\end{CJK} $A=\left(\begin{array}{cccc}1 & -2 & -2 & -2 \\ -2 & 1 & -2 & -2 \\ -2 & -2 & 1 & -2 \\ -2 & -2 & -2 & 1\end{array}\right)$ \begin{CJK}{UTF8}{mj}求一个正交矩阵\end{CJK} $T$, \begin{CJK}{UTF8}{mj}使\end{CJK} $T^{-1} A T$ \begin{CJK}{UTF8}{mj}为对角阵\end{CJK}.

\begin{enumerate}
  \setcounter{enumi}{3}
  \item (15 \begin{CJK}{UTF8}{mj}分\end{CJK}) \begin{CJK}{UTF8}{mj}求解下面的方程组\end{CJK}
\end{enumerate}
\includegraphics[max width=\textwidth]{2022_04_18_c36afefb0fdf29df1be6g-172}

\begin{enumerate}
  \setcounter{enumi}{5}
  \item (15 \begin{CJK}{UTF8}{mj}分\end{CJK}) \begin{CJK}{UTF8}{mj}证明\end{CJK}: \begin{CJK}{UTF8}{mj}复数域上的方程组\end{CJK}
\end{enumerate}
$$
\left\{\begin{array}{l}
x_{1}+x_{2}+\cdots+x_{n}=0 \\
x_{1}^{2}+x_{2}^{2}+\cdots+x_{n}^{2}=0 \\
\cdots \cdots \\
x_{1}^{n}+x_{2}^{n}+\cdots+x_{n}^{n}=0
\end{array}\right.
$$
\begin{CJK}{UTF8}{mj}只有零解\end{CJK}.

\begin{enumerate}
  \setcounter{enumi}{6}
  \item (15 \begin{CJK}{UTF8}{mj}分\end{CJK}) \begin{CJK}{UTF8}{mj}设\end{CJK} $A, B$ \begin{CJK}{UTF8}{mj}为复数域的\end{CJK} $n$ \begin{CJK}{UTF8}{mj}阶矩阵\end{CJK}, \begin{CJK}{UTF8}{mj}且\end{CJK} $A^{2}=A, B^{2}=B, \operatorname{rank}(A)=\operatorname{rank}(B)$, \begin{CJK}{UTF8}{mj}证明\end{CJK}: $A$ \begin{CJK}{UTF8}{mj}与\end{CJK} $B$ \begin{CJK}{UTF8}{mj}相似\end{CJK}.

  \item ( 20 \begin{CJK}{UTF8}{mj}分\end{CJK}) \begin{CJK}{UTF8}{mj}设\end{CJK} $A, B \in \mathbb{R}^{2 \times 2}$, \begin{CJK}{UTF8}{mj}且\end{CJK}

\end{enumerate}
$$
A^{2}=B^{2}=E, A B+B A=0
$$
\begin{CJK}{UTF8}{mj}证明\end{CJK}: \begin{CJK}{UTF8}{mj}存在可逆矩阵\end{CJK} $P$ \begin{CJK}{UTF8}{mj}使得\end{CJK}
$$
P^{-1} A P=\left(\begin{array}{cc}
1 & 0 \\
0 & -1
\end{array}\right), P^{-1} B P=\left(\begin{array}{ll}
0 & 1 \\
1 & 0
\end{array}\right)
$$

\begin{enumerate}
  \setcounter{enumi}{8}
  \item ( 20 \begin{CJK}{UTF8}{mj}分\end{CJK}) \begin{CJK}{UTF8}{mj}域\end{CJK} $\mathbb{C}$ \begin{CJK}{UTF8}{mj}上全体\end{CJK} $n$ \begin{CJK}{UTF8}{mj}阶矩阵组成一个\end{CJK} $n^{2}$ \begin{CJK}{UTF8}{mj}维线性空间\end{CJK} $M_{n}(\mathbb{C}), A \in M_{n}(\mathbb{C}), A$ \begin{CJK}{UTF8}{mj}可对角化\end{CJK}, \begin{CJK}{UTF8}{mj}特征值为\end{CJK} $\lambda_{1}, \lambda_{2}, \cdots \lambda_{n}$ (\begin{CJK}{UTF8}{mj}不一定不相等\end{CJK}), \begin{CJK}{UTF8}{mj}设\end{CJK} $\mathcal{A}$ \begin{CJK}{UTF8}{mj}是域\end{CJK} $\mathbb{C}$ \begin{CJK}{UTF8}{mj}的变换\end{CJK}, $\mathcal{A}(B)=A B-B A$,
\end{enumerate}
(1) \begin{CJK}{UTF8}{mj}证明\end{CJK}: $\mathcal{A}$ \begin{CJK}{UTF8}{mj}为域\end{CJK} $\mathbb{C}$ \begin{CJK}{UTF8}{mj}上的线性变换\end{CJK},

(2) \begin{CJK}{UTF8}{mj}求\end{CJK} $A$ \begin{CJK}{UTF8}{mj}的所有\end{CJK} $n^{2}$ \begin{CJK}{UTF8}{mj}个特征值\end{CJK}.

\section{8. 华东师范大学 2016 年研究生入学考试试题高等代数 
 李扬 
 微信公众号: sxkyliyang}
\begin{enumerate}
  \item ( 15 \begin{CJK}{UTF8}{mj}分\end{CJK}) \begin{CJK}{UTF8}{mj}设\end{CJK} $M$ \begin{CJK}{UTF8}{mj}是二阶矩阵\end{CJK}, \begin{CJK}{UTF8}{mj}求证\end{CJK}:
\end{enumerate}
$$
M\left(\begin{array}{cc}
0 & 1 \\
-1 & 0
\end{array}\right) M^{T}=\left(\begin{array}{cc}
0 & 1 \\
-1 & 0
\end{array}\right) \Leftrightarrow|M|=1
$$

\begin{enumerate}
  \setcounter{enumi}{2}
  \item (15 \begin{CJK}{UTF8}{mj}分\end{CJK}) \begin{CJK}{UTF8}{mj}在矩阵\end{CJK}
\end{enumerate}
$$
A=\left(\begin{array}{cccc}
1 & 2 & \cdots & n \\
n+1 & n+2 & \cdots & 2 n \\
\vdots & \vdots & & \vdots \\
(n-1) n+1 & (n-1) n+2 & \cdots & n^{2}
\end{array}\right)
$$
\begin{CJK}{UTF8}{mj}中取\end{CJK} $n$ \begin{CJK}{UTF8}{mj}个数\end{CJK}, \begin{CJK}{UTF8}{mj}使得每行每列都恰好只被取到一个数\end{CJK}. \begin{CJK}{UTF8}{mj}问\end{CJK}:\begin{CJK}{UTF8}{mj}这些取出的数相加之和会有哪些可能的值\end{CJK}?

\begin{enumerate}
  \setcounter{enumi}{3}
  \item ( 30 \begin{CJK}{UTF8}{mj}分\end{CJK}) \begin{CJK}{UTF8}{mj}已知矩阵\end{CJK}
\end{enumerate}
$$
A=\left(\begin{array}{cccc}
3 & 1 & 0 & -1 \\
1 & 3 & -1 & 0 \\
0 & -1 & 3 & 1 \\
-1 & 0 & 1 & 3
\end{array}\right)
$$
\begin{CJK}{UTF8}{mj}求正交矩阵\end{CJK} $Q$, \begin{CJK}{UTF8}{mj}使得\end{CJK} $Q^{-1} A Q$ \begin{CJK}{UTF8}{mj}为对角矩阵\end{CJK}, \begin{CJK}{UTF8}{mj}并写出得到的对角矩阵\end{CJK}.

\begin{enumerate}
  \setcounter{enumi}{4}
  \item ( 15 \begin{CJK}{UTF8}{mj}分\end{CJK}) \begin{CJK}{UTF8}{mj}设\end{CJK} $\varphi$ \begin{CJK}{UTF8}{mj}是\end{CJK} $n$ \begin{CJK}{UTF8}{mj}维线性空间\end{CJK} $V$ \begin{CJK}{UTF8}{mj}上的线性变换\end{CJK}, $\alpha$ \begin{CJK}{UTF8}{mj}是\end{CJK} $V$ \begin{CJK}{UTF8}{mj}中的向量\end{CJK}. \begin{CJK}{UTF8}{mj}已知整数\end{CJK} $m$ \begin{CJK}{UTF8}{mj}满足\end{CJK} $\varphi^{m}(\alpha) \neq 0$, \begin{CJK}{UTF8}{mj}但\end{CJK} $\varphi^{m+1}(\alpha)=0$. \begin{CJK}{UTF8}{mj}求证\end{CJK} $\alpha, \varphi(\alpha), \cdots, \varphi^{m}(\alpha)$ \begin{CJK}{UTF8}{mj}线性无关\end{CJK}.

  \item ( 20 \begin{CJK}{UTF8}{mj}分\end{CJK}) \begin{CJK}{UTF8}{mj}设\end{CJK} $V$ \begin{CJK}{UTF8}{mj}是数域\end{CJK} $K$ \begin{CJK}{UTF8}{mj}上的线性空间\end{CJK}, $X$ \begin{CJK}{UTF8}{mj}是一个集合\end{CJK}. \begin{CJK}{UTF8}{mj}已知存在一双射\end{CJK} $\varphi: X \rightarrow V$. \begin{CJK}{UTF8}{mj}先在\end{CJK} $X$ \begin{CJK}{UTF8}{mj}上定义加法和\end{CJK} \begin{CJK}{UTF8}{mj}数乘运算如下\end{CJK}:

\end{enumerate}
$$
\begin{aligned}
&x \oplus y=\varphi^{-1}(\varphi(x)+\varphi(y)), x, y \in X, \\
&x \circ y=\varphi^{-1}(\lambda \varphi(x)), \quad \lambda \in K, x \in X
\end{aligned}
$$
\begin{CJK}{UTF8}{mj}验证\end{CJK} $X$ \begin{CJK}{UTF8}{mj}关于上述定义的加法与数乘构成\end{CJK} $K$ \begin{CJK}{UTF8}{mj}上的一个线性空间\end{CJK}, \begin{CJK}{UTF8}{mj}并且\end{CJK} $\varphi$ \begin{CJK}{UTF8}{mj}是线性空间之间的一个同构\end{CJK}.

\begin{enumerate}
  \setcounter{enumi}{6}
  \item ( 20 \begin{CJK}{UTF8}{mj}分\end{CJK}) \begin{CJK}{UTF8}{mj}设\end{CJK} $V$ \begin{CJK}{UTF8}{mj}是全体\end{CJK} $n$ \begin{CJK}{UTF8}{mj}阶实系数矩阵构成的线性空间\end{CJK}, \begin{CJK}{UTF8}{mj}定义运算\end{CJK}
\end{enumerate}
$$
(A, B)=\operatorname{Tr}\left(A^{T} B\right), \quad A, B \in V
$$
(1) \begin{CJK}{UTF8}{mj}证明\end{CJK}: $(,,$, \begin{CJK}{UTF8}{mj}是内积\end{CJK}, $V$ \begin{CJK}{UTF8}{mj}是\end{CJK} $n^{2}$ \begin{CJK}{UTF8}{mj}维欧式空间\end{CJK}.

(2) \begin{CJK}{UTF8}{mj}设\end{CJK} $T \in V$ \begin{CJK}{UTF8}{mj}是给定矩阵\end{CJK}, \begin{CJK}{UTF8}{mj}定义映射\end{CJK}
$$
\phi(A)=T A, A \in V
$$
\begin{CJK}{UTF8}{mj}证明\end{CJK}: $\phi$ \begin{CJK}{UTF8}{mj}是\end{CJK} $V$ \begin{CJK}{UTF8}{mj}上的线性映射\end{CJK}.

(3) \begin{CJK}{UTF8}{mj}求\end{CJK} $\phi$ \begin{CJK}{UTF8}{mj}的伴随算子\end{CJK}.

\begin{enumerate}
  \setcounter{enumi}{7}
  \item ( 15 \begin{CJK}{UTF8}{mj}分\end{CJK}) \begin{CJK}{UTF8}{mj}证明\end{CJK}: \begin{CJK}{UTF8}{mj}下列二次型\end{CJK}
\end{enumerate}
$$
f\left(x_{1}, x_{2}, \cdots, x_{n}\right)=n \sum_{i=1}^{n} x_{i}^{2}-\left(\sum_{i=1}^{n} x_{i}\right)^{2}
$$
\begin{CJK}{UTF8}{mj}是半正定型\end{CJK}. 8. ( 20 \begin{CJK}{UTF8}{mj}分\end{CJK}) \begin{CJK}{UTF8}{mj}已知实矩阵\end{CJK}
$$
A=\left(\begin{array}{ccccc}
a_{1} & b_{1} & & & \\
c_{1} & a_{2} & b_{2} & & \\
& \ddots & \ddots & \ddots & \\
& & \ddots & \ddots & b_{n-1} \\
& & & c_{n-1} & a_{n}
\end{array}\right)
$$
\begin{CJK}{UTF8}{mj}满足\end{CJK} $b_{i} c_{i}>0,(i=1,2, \cdots, n-1)$. \begin{CJK}{UTF8}{mj}求证\end{CJK}: $A$ \begin{CJK}{UTF8}{mj}有\end{CJK} $n$ \begin{CJK}{UTF8}{mj}个两两不同的实特征值\end{CJK}.

(\begin{CJK}{UTF8}{mj}提示\end{CJK}: \begin{CJK}{UTF8}{mj}先考虑\end{CJK} $b_{i}=c_{i}(i=1,2, \cdots, n-1$ ) \begin{CJK}{UTF8}{mj}的特殊情况\end{CJK}; \begin{CJK}{UTF8}{mj}对一般情形\end{CJK}, \begin{CJK}{UTF8}{mj}试找出一个实对角可逆矩阵\end{CJK} $D$ \begin{CJK}{UTF8}{mj}使得\end{CJK} $D^{-1} A D$ \begin{CJK}{UTF8}{mj}符合该特殊情形\end{CJK}.)

\section{9. 华东师范大学 2017 年研究生入学考试试题高等代数 
 李扬 
 微信公众号: sxkyliyang}
\begin{enumerate}
  \item ( 20 \begin{CJK}{UTF8}{mj}分\end{CJK}) \begin{CJK}{UTF8}{mj}当实数\end{CJK} $\lambda$ \begin{CJK}{UTF8}{mj}为何值时\end{CJK}, \begin{CJK}{UTF8}{mj}方程组\end{CJK}
\end{enumerate}
$$
\left\{\begin{array}{l}
(\lambda-2) x_{1}-x_{2}-x_{3}=-2 \\
4 x_{1}+(\lambda-1) x_{2}+4 x_{3}=7 \\
x_{1}+x_{2}+x_{3}=2
\end{array}\right.
$$
\begin{CJK}{UTF8}{mj}有唯一解\end{CJK}, \begin{CJK}{UTF8}{mj}无解\end{CJK}, \begin{CJK}{UTF8}{mj}有无穷多个解\end{CJK}; \begin{CJK}{UTF8}{mj}有解时\end{CJK}, \begin{CJK}{UTF8}{mj}请求出求全部解\end{CJK}.

\begin{enumerate}
  \setcounter{enumi}{2}
  \item ( 12 \begin{CJK}{UTF8}{mj}分\end{CJK}) \begin{CJK}{UTF8}{mj}已知二次型\end{CJK}
\end{enumerate}
$$
f\left(x_{1}, x_{2}, x_{3}\right)=2 x_{1}^{2}+x_{2}^{2}+3 x_{3}^{2}+2 \lambda x_{1} x_{2}+2 x_{1} x_{3}
$$
\begin{CJK}{UTF8}{mj}正定\end{CJK}, \begin{CJK}{UTF8}{mj}求\end{CJK} $\lambda$ \begin{CJK}{UTF8}{mj}的取值范围\end{CJK}.

\begin{enumerate}
  \setcounter{enumi}{3}
  \item ( 20 \begin{CJK}{UTF8}{mj}分\end{CJK}) \begin{CJK}{UTF8}{mj}已知实对称矩阵\end{CJK}
\end{enumerate}
$$
A=\left(\begin{array}{lll}
4 & 1 & 1 \\
1 & 4 & 1 \\
1 & 1 & 4
\end{array}\right)
$$
\begin{CJK}{UTF8}{mj}求正交矩阵\end{CJK} $T$, \begin{CJK}{UTF8}{mj}使得\end{CJK} $T^{-1} A T$ \begin{CJK}{UTF8}{mj}为对角矩阵\end{CJK}.

\begin{enumerate}
  \setcounter{enumi}{4}
  \item ( 20 \begin{CJK}{UTF8}{mj}分\end{CJK}) \begin{CJK}{UTF8}{mj}设\end{CJK} $K$ \begin{CJK}{UTF8}{mj}是数域\end{CJK},
\end{enumerate}
(1) \begin{CJK}{UTF8}{mj}证明\end{CJK}: \begin{CJK}{UTF8}{mj}一元多项式\end{CJK} $x^{2}+x^{3}$ \begin{CJK}{UTF8}{mj}不能写成另一多项式的平方\end{CJK};

(2) \begin{CJK}{UTF8}{mj}证明\end{CJK}: \begin{CJK}{UTF8}{mj}二元多项式\end{CJK} $y^{2}-x^{2}-x^{3}$ \begin{CJK}{UTF8}{mj}是二元多项式环\end{CJK} $K[x, y]$ \begin{CJK}{UTF8}{mj}中的不可约多项式\end{CJK}, \begin{CJK}{UTF8}{mj}也就是说它不能写成两个非\end{CJK} \begin{CJK}{UTF8}{mj}常数多项式的乘积\end{CJK}.

\begin{enumerate}
  \setcounter{enumi}{5}
  \item ( 13 \begin{CJK}{UTF8}{mj}分\end{CJK}) \begin{CJK}{UTF8}{mj}设\end{CJK} $A, B$ \begin{CJK}{UTF8}{mj}是同阶方阵\end{CJK}, \begin{CJK}{UTF8}{mj}若\end{CJK} $A$ \begin{CJK}{UTF8}{mj}可逆\end{CJK}, \begin{CJK}{UTF8}{mj}证明\end{CJK} $A B$ \begin{CJK}{UTF8}{mj}与\end{CJK} $B A$ \begin{CJK}{UTF8}{mj}相似\end{CJK}. \begin{CJK}{UTF8}{mj}问当\end{CJK} $A$ \begin{CJK}{UTF8}{mj}不可逆时\end{CJK},\begin{CJK}{UTF8}{mj}结论是否成立\end{CJK}?

  \item (10 \begin{CJK}{UTF8}{mj}分\end{CJK}) \begin{CJK}{UTF8}{mj}给定\end{CJK} $m+n$ \begin{CJK}{UTF8}{mj}阶分块方阵\end{CJK}

\end{enumerate}
$$
A=\left(\begin{array}{cc}
0_{m} & B_{m \times n} \\
C_{n \times m} & 0_{n}
\end{array}\right)
$$
\begin{CJK}{UTF8}{mj}证明\end{CJK}: \begin{CJK}{UTF8}{mj}若\end{CJK} $\lambda$ \begin{CJK}{UTF8}{mj}为\end{CJK} $A$ \begin{CJK}{UTF8}{mj}的特征值\end{CJK}, \begin{CJK}{UTF8}{mj}则\end{CJK} $-\lambda$ \begin{CJK}{UTF8}{mj}也为\end{CJK} $A$ \begin{CJK}{UTF8}{mj}的特征值\end{CJK}.

\begin{enumerate}
  \setcounter{enumi}{7}
  \item ( 20 \begin{CJK}{UTF8}{mj}分\end{CJK}) (1) \begin{CJK}{UTF8}{mj}求证\end{CJK}: 3 \begin{CJK}{UTF8}{mj}阶复矩阵\end{CJK} $A$ \begin{CJK}{UTF8}{mj}与\end{CJK} $B$ \begin{CJK}{UTF8}{mj}相似的充要条件是它们有相同的特征多项式和极小多项式\end{CJK};
\end{enumerate}
(2) \begin{CJK}{UTF8}{mj}举例说明\end{CJK} 4 \begin{CJK}{UTF8}{mj}阶复矩阵即使有相同的特征多项式和极小多项式也不一定相似\end{CJK}.

\begin{enumerate}
  \setcounter{enumi}{8}
  \item ( 15 \begin{CJK}{UTF8}{mj}分\end{CJK}) \begin{CJK}{UTF8}{mj}设\end{CJK} $f: U \rightarrow V, g: V \rightarrow W$ \begin{CJK}{UTF8}{mj}是数域\end{CJK} $K$ \begin{CJK}{UTF8}{mj}上有限维的线性映射\end{CJK}, \begin{CJK}{UTF8}{mj}证明\end{CJK}:
\end{enumerate}
$$
\operatorname{dim}(\operatorname{Ker} f)+\operatorname{dim}(\operatorname{Im} f \cap \operatorname{Ker} g)=\operatorname{dim}(\operatorname{Ker}(g f))
$$

\begin{enumerate}
  \setcounter{enumi}{9}
  \item ( 20 \begin{CJK}{UTF8}{mj}分\end{CJK}) \begin{CJK}{UTF8}{mj}设\end{CJK} $f(x), g(x)$ \begin{CJK}{UTF8}{mj}是数域\end{CJK} $K$ \begin{CJK}{UTF8}{mj}上的非零多项式\end{CJK}, $A$ \begin{CJK}{UTF8}{mj}是\end{CJK} $n(n \geq 2)$ \begin{CJK}{UTF8}{mj}阶方阵\end{CJK}.
\end{enumerate}
(1) \begin{CJK}{UTF8}{mj}证明\end{CJK}: \begin{CJK}{UTF8}{mj}若\end{CJK} $g(A)$ \begin{CJK}{UTF8}{mj}可逆\end{CJK}, \begin{CJK}{UTF8}{mj}则\end{CJK} $f(A) g(A)^{*}=g(A)^{*} f(A)$, \begin{CJK}{UTF8}{mj}其中\end{CJK} $g(A)^{*}$ \begin{CJK}{UTF8}{mj}为\end{CJK} $g(A)$ \begin{CJK}{UTF8}{mj}的伴随矩阵\end{CJK}.

(2) $g(A)$ \begin{CJK}{UTF8}{mj}不可逆时\end{CJK}, \begin{CJK}{UTF8}{mj}结论是否成立\end{CJK}?

\section{0. 华东师范大学 2018 年研究生入学考试试题高等代数 
 李扬 
 微信公众号: sxkyliyang}
\begin{enumerate}
  \item (15 \begin{CJK}{UTF8}{mj}分\end{CJK}) \begin{CJK}{UTF8}{mj}当实数\end{CJK} $a, d$ \begin{CJK}{UTF8}{mj}取何值时\end{CJK}, \begin{CJK}{UTF8}{mj}下列方程无解\end{CJK}、\begin{CJK}{UTF8}{mj}有唯一解\end{CJK}、\begin{CJK}{UTF8}{mj}有无穷多个解\end{CJK}?\begin{CJK}{UTF8}{mj}有解时\end{CJK}, \begin{CJK}{UTF8}{mj}求出所有解\end{CJK},
\end{enumerate}
$$
\left\{\begin{array}{l}
-x_{2}-2 x_{3}-2 x_{4}-6 x_{5}=a-3 \\
x_{1}-x_{3}-x_{4}+(d-5) x_{5}=-4 \\
2 x_{1}+2 x_{2}+2 x_{3}+2 x_{4}+(d+2) x_{5}=-a, \\
2 x_{2}+4 x_{3}+4 x_{4}+12 x_{5}=-a+6
\end{array}\right.
$$

\begin{enumerate}
  \setcounter{enumi}{2}
  \item ( 10 \begin{CJK}{UTF8}{mj}分\end{CJK}) \begin{CJK}{UTF8}{mj}已知\end{CJK} 5 \begin{CJK}{UTF8}{mj}阶复方阵\end{CJK} $A$ \begin{CJK}{UTF8}{mj}的特征多项式为\end{CJK} $f_{A}(\lambda)$ \begin{CJK}{UTF8}{mj}与极小多项式\end{CJK} $m_{A}(\lambda)$ \begin{CJK}{UTF8}{mj}分别为\end{CJK} $f_{A}(\lambda)=(\lambda-1)^{3}(\lambda+2)^{2}$, $m_{A}(\lambda)=(\lambda-1)^{2}(\lambda+2)$. \begin{CJK}{UTF8}{mj}求\end{CJK} $A$ \begin{CJK}{UTF8}{mj}的\end{CJK} Jordan \begin{CJK}{UTF8}{mj}典范型\end{CJK}.

  \item (10 \begin{CJK}{UTF8}{mj}分\end{CJK}) \begin{CJK}{UTF8}{mj}已知实二次型\end{CJK} $Q$ \begin{CJK}{UTF8}{mj}满足\end{CJK} $Q(\alpha)=0 \Longleftrightarrow \alpha=0$. \begin{CJK}{UTF8}{mj}求证\end{CJK}: $Q$ \begin{CJK}{UTF8}{mj}或者正定或者负定\end{CJK}.

  \item ( 15 \begin{CJK}{UTF8}{mj}分\end{CJK}) \begin{CJK}{UTF8}{mj}设\end{CJK} $A=\frac{1}{3}\left(\begin{array}{ccc}2 & -1 & 2 \\ -1 & 2 & 2 \\ 2 & 2 & 1\end{array}\right)$. \begin{CJK}{UTF8}{mj}求一个正交矩阵\end{CJK} $P$ \begin{CJK}{UTF8}{mj}使得\end{CJK} $P^{T} A P$ \begin{CJK}{UTF8}{mj}为对角阵\end{CJK}, \begin{CJK}{UTF8}{mj}并写出该对角阵\end{CJK}.

  \item (20 \begin{CJK}{UTF8}{mj}分\end{CJK}) (1) \begin{CJK}{UTF8}{mj}利用初等变换将下列矩阵化成简化的行阶梯形矩阵\end{CJK}.

\end{enumerate}
$$
\left(\begin{array}{ccccccc}
1 & 2 & -1 & 0 & 2 & 1 & 5 \\
-1 & -2 & 0 & 0 & 1 & -2 & -3 \\
1 & 2 & -3 & 0 & 5 & 1 & 6
\end{array}\right)
$$
(2) \begin{CJK}{UTF8}{mj}设\end{CJK} $V$ \begin{CJK}{UTF8}{mj}数域\end{CJK} $K$ \begin{CJK}{UTF8}{mj}上的有限维线性空间\end{CJK}, \begin{CJK}{UTF8}{mj}给定他的一组基\end{CJK} $e_{1}, e_{2}, \cdots, e_{n}$. \begin{CJK}{UTF8}{mj}对于\end{CJK} $V$ \begin{CJK}{UTF8}{mj}中的一个非零向量\end{CJK} $\alpha=$ $\sum_{i=1}^{n} \lambda_{i} e_{i}$, \begin{CJK}{UTF8}{mj}若\end{CJK} $i$ \begin{CJK}{UTF8}{mj}是最小正整数使得\end{CJK} $\lambda_{i}$ \begin{CJK}{UTF8}{mj}不为\end{CJK} 0 , \begin{CJK}{UTF8}{mj}则称\end{CJK} $e_{i}$ \begin{CJK}{UTF8}{mj}为它的\end{CJK} $\mathrm{Tip}$, \begin{CJK}{UTF8}{mj}记为\end{CJK} $e_{i}=\operatorname{Tip}(\alpha)$, \begin{CJK}{UTF8}{mj}对于\end{CJK} $V$ \begin{CJK}{UTF8}{mj}的一个子空间\end{CJK} $W$, \begin{CJK}{UTF8}{mj}定义\end{CJK}
$$
\operatorname{Tip}(W)=\{\operatorname{tip}(\alpha): \alpha \in W, \alpha \neq 0\}
$$
$\operatorname{NonTip}(W)=\left\{e_{1}, e_{2}, \cdots, e_{n}\right\}-\operatorname{Tip}(W) .$

\begin{CJK}{UTF8}{mj}现设\end{CJK} $V=K^{7}$ \begin{CJK}{UTF8}{mj}是\end{CJK} 7 \begin{CJK}{UTF8}{mj}维行向量组成的空间\end{CJK}, \begin{CJK}{UTF8}{mj}取它的标准基\end{CJK} $e_{1}, e_{2}, \cdots, e_{7}$. \begin{CJK}{UTF8}{mj}令\end{CJK} $W$ \begin{CJK}{UTF8}{mj}为\end{CJK} (1) \begin{CJK}{UTF8}{mj}中矩阵的行向量张成的子\end{CJK} \begin{CJK}{UTF8}{mj}空间\end{CJK}. \begin{CJK}{UTF8}{mj}求\end{CJK} $\operatorname{Tip}(W)$ \begin{CJK}{UTF8}{mj}和\end{CJK} $\operatorname{NonTip}(W)$.

(3)\begin{CJK}{UTF8}{mj}设\end{CJK} $V$ \begin{CJK}{UTF8}{mj}是数域\end{CJK} $K$ \begin{CJK}{UTF8}{mj}上的有限维线性空间\end{CJK}, \begin{CJK}{UTF8}{mj}给定它的一组基\end{CJK} $e_{1}, e_{2}, \cdots, e_{n}$, \begin{CJK}{UTF8}{mj}设\end{CJK} $W$ \begin{CJK}{UTF8}{mj}是\end{CJK} $V$ \begin{CJK}{UTF8}{mj}的一个子空间\end{CJK}. \begin{CJK}{UTF8}{mj}证\end{CJK}:
$$
V=W \oplus \operatorname{Span}_{k}(\operatorname{NonTip}(W)),
$$
\begin{CJK}{UTF8}{mj}这里\end{CJK} $\operatorname{Span}_{k}(\operatorname{NonTip}(W))$ \begin{CJK}{UTF8}{mj}是\end{CJK} $\operatorname{NonTip}(W)$ \begin{CJK}{UTF8}{mj}张成的子空间\end{CJK}.

\begin{enumerate}
  \setcounter{enumi}{6}
  \item (15 \begin{CJK}{UTF8}{mj}分\end{CJK}) \begin{CJK}{UTF8}{mj}设\end{CJK} $f$ \begin{CJK}{UTF8}{mj}与\end{CJK} $g$ \begin{CJK}{UTF8}{mj}是从有限维线性空间\end{CJK} $U$ \begin{CJK}{UTF8}{mj}到有限维线性空间\end{CJK} $W$ \begin{CJK}{UTF8}{mj}的两个线性映射\end{CJK}. \begin{CJK}{UTF8}{mj}若\end{CJK} $\operatorname{Im} f=\operatorname{Im} g$, \begin{CJK}{UTF8}{mj}这里\end{CJK} $\operatorname{Im} f$ \begin{CJK}{UTF8}{mj}是\end{CJK} $f$ \begin{CJK}{UTF8}{mj}的像\end{CJK}, \begin{CJK}{UTF8}{mj}证明\end{CJK}: \begin{CJK}{UTF8}{mj}存在\end{CJK} $U$ \begin{CJK}{UTF8}{mj}上的可逆线性变换\end{CJK} $h$, \begin{CJK}{UTF8}{mj}使得\end{CJK} $g=f \circ h$.

  \item ( 25 \begin{CJK}{UTF8}{mj}分\end{CJK}) \begin{CJK}{UTF8}{mj}设\end{CJK} $K$ \begin{CJK}{UTF8}{mj}是一个数域\end{CJK}, $m, n$ \begin{CJK}{UTF8}{mj}为自然数\end{CJK}, $M_{m, n}(K), M_{m}(K)$ \begin{CJK}{UTF8}{mj}分别是数域\end{CJK} $K$ \begin{CJK}{UTF8}{mj}上\end{CJK} $m \times n$ \begin{CJK}{UTF8}{mj}阶与\end{CJK} $m$ \begin{CJK}{UTF8}{mj}阶矩阵生成\end{CJK} \begin{CJK}{UTF8}{mj}的空间\end{CJK}, $A$ \begin{CJK}{UTF8}{mj}是秩为\end{CJK} $r$ \begin{CJK}{UTF8}{mj}的\end{CJK} $m \times n$ \begin{CJK}{UTF8}{mj}阶矩阵\end{CJK}. \begin{CJK}{UTF8}{mj}定义\end{CJK}

\end{enumerate}
$$
f: M_{m}(K) \rightarrow M_{m, n}(K), f(X)=X A
$$
(1) \begin{CJK}{UTF8}{mj}证明\end{CJK}: $f$ \begin{CJK}{UTF8}{mj}是一个线性映射\end{CJK}; (3) \begin{CJK}{UTF8}{mj}对于任意的\end{CJK} $m, n, r$, \begin{CJK}{UTF8}{mj}求\end{CJK} $f$ \begin{CJK}{UTF8}{mj}的秩\end{CJK};

(4) \begin{CJK}{UTF8}{mj}对于任意的\end{CJK} $m, n, r$, \begin{CJK}{UTF8}{mj}求\end{CJK} $f$ \begin{CJK}{UTF8}{mj}的核\end{CJK} $\operatorname{ker} f$ \begin{CJK}{UTF8}{mj}的维数\end{CJK}.

\begin{enumerate}
  \setcounter{enumi}{8}
  \item ( 20 \begin{CJK}{UTF8}{mj}分\end{CJK}) \begin{CJK}{UTF8}{mj}设\end{CJK} $M_{k, n}$ \begin{CJK}{UTF8}{mj}是所有\end{CJK} $k \times n$ \begin{CJK}{UTF8}{mj}阶复矩阵的集合\end{CJK}, $N_{k}^{+}$\begin{CJK}{UTF8}{mj}是所有\end{CJK} $k$ \begin{CJK}{UTF8}{mj}阶下三角幂么方阵的集合\end{CJK}, $N_{k}^{+}$\begin{CJK}{UTF8}{mj}是所有\end{CJK} $n$ \begin{CJK}{UTF8}{mj}阶上\end{CJK} \begin{CJK}{UTF8}{mj}三角幂么方阵的集合\end{CJK}. \begin{CJK}{UTF8}{mj}这里的幂么矩阵是指对角线上全为\end{CJK} 1 \begin{CJK}{UTF8}{mj}的上三角或下三角矩阵\end{CJK}. \begin{CJK}{UTF8}{mj}在\end{CJK} $M_{k, n}$ \begin{CJK}{UTF8}{mj}中定义如下关系\end{CJK}
\end{enumerate}
$$
A \sim B \Leftrightarrow \exists P \in N_{k}^{-}, Q \in N_{k}^{+} \text {使得 } A=P B Q .
$$
(1) \begin{CJK}{UTF8}{mj}求证\end{CJK} $\sim$ \begin{CJK}{UTF8}{mj}是\end{CJK} $M_{k, n}$ \begin{CJK}{UTF8}{mj}上的等价关系\end{CJK}.

(2) \begin{CJK}{UTF8}{mj}设\end{CJK} $r=\min \{k, n\}$, \begin{CJK}{UTF8}{mj}求证\end{CJK} $\Delta_{1}, \cdots, \Delta_{r}$ \begin{CJK}{UTF8}{mj}是上述等价关系的不变量\end{CJK}, \begin{CJK}{UTF8}{mj}也就是说两个满足该等价关系的矩阵具有\end{CJK} \begin{CJK}{UTF8}{mj}相同的\end{CJK} $\Delta_{1}, \cdots, \Delta_{r}$ \begin{CJK}{UTF8}{mj}值\end{CJK}, \begin{CJK}{UTF8}{mj}这里\end{CJK} $\Delta_{i}(i=1, \cdots, r)$ \begin{CJK}{UTF8}{mj}是矩阵的第\end{CJK} $i$ \begin{CJK}{UTF8}{mj}个顺序主子式\end{CJK}.

\begin{enumerate}
  \setcounter{enumi}{9}
  \item ( 20 \begin{CJK}{UTF8}{mj}分\end{CJK}) \begin{CJK}{UTF8}{mj}设\end{CJK} $\lambda_{1}, \cdots, \lambda_{n}$ \begin{CJK}{UTF8}{mj}是数域\end{CJK} $K$ \begin{CJK}{UTF8}{mj}上的\end{CJK} $n$ \begin{CJK}{UTF8}{mj}个两两不同的数\end{CJK}, $V$ \begin{CJK}{UTF8}{mj}是\end{CJK} $K$ \begin{CJK}{UTF8}{mj}上的线性空间\end{CJK}, $\varphi$ \begin{CJK}{UTF8}{mj}是\end{CJK} $V$ \begin{CJK}{UTF8}{mj}上的线性变换\end{CJK}, \begin{CJK}{UTF8}{mj}且\end{CJK} \begin{CJK}{UTF8}{mj}它在基\end{CJK} $\xi_{1}, \cdots, \xi_{n}$ \begin{CJK}{UTF8}{mj}下的矩阵为对角矩阵\end{CJK} $A=\operatorname{diag}\left\{\lambda_{1}, \cdots, \lambda_{n}\right\}$.
\end{enumerate}
(1) \begin{CJK}{UTF8}{mj}设\end{CJK} $W$ \begin{CJK}{UTF8}{mj}是\end{CJK} $\varphi$ \begin{CJK}{UTF8}{mj}的不变子空间\end{CJK}, $x_{1} \xi_{1}+\cdots+x_{n} \xi_{n} \in W$, \begin{CJK}{UTF8}{mj}其中\end{CJK} $x_{1}, \cdots, x_{n} \in K$, \begin{CJK}{UTF8}{mj}证明\end{CJK}: \begin{CJK}{UTF8}{mj}若某个\end{CJK} $x_{i}$ \begin{CJK}{UTF8}{mj}不为\end{CJK} 0 , \begin{CJK}{UTF8}{mj}则\end{CJK} $\xi_{i} \in W$.

(2) \begin{CJK}{UTF8}{mj}求\end{CJK} $\varphi$ \begin{CJK}{UTF8}{mj}的不变子空间个数\end{CJK}.

\section{1. 华东师范大学 2009 年研究生入学考试试题数学分析 
 李扬 
 微信公众号: sxkyliyang}
\begin{enumerate}
  \item \begin{CJK}{UTF8}{mj}判断下列命题是否正确\end{CJK}, \begin{CJK}{UTF8}{mj}若正确给出证明\end{CJK}, \begin{CJK}{UTF8}{mj}若错误举出反例\end{CJK} (\begin{CJK}{UTF8}{mj}每小题\end{CJK} 9 \begin{CJK}{UTF8}{mj}分\end{CJK}, \begin{CJK}{UTF8}{mj}共\end{CJK} 54 \begin{CJK}{UTF8}{mj}分\end{CJK})
\end{enumerate}
(1) \begin{CJK}{UTF8}{mj}设\end{CJK} $\lim _{x \rightarrow a} g(x)=A, \lim _{y \rightarrow A} f(y)=B$, \begin{CJK}{UTF8}{mj}此处\end{CJK} $a, A, B$ \begin{CJK}{UTF8}{mj}均为实数\end{CJK}. \begin{CJK}{UTF8}{mj}则\end{CJK}
$$
\lim _{x \rightarrow a} f(g(x))=B .
$$
(2) \begin{CJK}{UTF8}{mj}设\end{CJK} $f(x)$ \begin{CJK}{UTF8}{mj}为闭区间\end{CJK} $[a, b]$ \begin{CJK}{UTF8}{mj}上不恒为\end{CJK} 0 \begin{CJK}{UTF8}{mj}的连续函数\end{CJK}, $D(x)$ \begin{CJK}{UTF8}{mj}为\end{CJK} Dirichlet \begin{CJK}{UTF8}{mj}函数\end{CJK}, \begin{CJK}{UTF8}{mj}则\end{CJK} $f(x) D(x)$ \begin{CJK}{UTF8}{mj}在\end{CJK} $[a, b]$ \begin{CJK}{UTF8}{mj}上不可积\end{CJK}.

(3) \begin{CJK}{UTF8}{mj}存在实数\end{CJK} $a_{0}, a_{n}, b_{n},(n=1,2, \cdots)$ \begin{CJK}{UTF8}{mj}使得\end{CJK}
$$
\frac{a_{0}}{2}+\sum_{n=1}^{\infty}\left(a_{n} \cos n x+b_{n} \sin n x\right)= \begin{cases}1, & x \in[1,2] \\ 0, & x \in[4,5]\end{cases}
$$
(4) \begin{CJK}{UTF8}{mj}已知\end{CJK} $f(x)$ \begin{CJK}{UTF8}{mj}在\end{CJK} $x=2$ \begin{CJK}{UTF8}{mj}处连续\end{CJK}, \begin{CJK}{UTF8}{mj}且\end{CJK} $\lim _{x \rightarrow 2} \frac{f(x)}{x-2}=1$, \begin{CJK}{UTF8}{mj}则\end{CJK} $f(x)$ \begin{CJK}{UTF8}{mj}在\end{CJK} $x=2$ \begin{CJK}{UTF8}{mj}处可导\end{CJK}.

(5) \begin{CJK}{UTF8}{mj}如果\end{CJK} $f(x)$ \begin{CJK}{UTF8}{mj}在\end{CJK} $x_{0}$ \begin{CJK}{UTF8}{mj}处可导\end{CJK}, \begin{CJK}{UTF8}{mj}则\end{CJK} $f(x)$ \begin{CJK}{UTF8}{mj}在\end{CJK} $x_{0}$ \begin{CJK}{UTF8}{mj}的一个邻域上连续\end{CJK}.

(6) \begin{CJK}{UTF8}{mj}若多项式函数列\end{CJK} $\left\{f_{n}(x)\right\}$ \begin{CJK}{UTF8}{mj}在\end{CJK} $(-\infty,+\infty)$ \begin{CJK}{UTF8}{mj}上一致收敛于函数\end{CJK} $f(x)$, \begin{CJK}{UTF8}{mj}则\end{CJK} $f(x)$ \begin{CJK}{UTF8}{mj}必是多项式函数\end{CJK}.

\begin{enumerate}
  \setcounter{enumi}{2}
  \item \begin{CJK}{UTF8}{mj}求解下列各题\end{CJK} (\begin{CJK}{UTF8}{mj}每小题\end{CJK} 12 \begin{CJK}{UTF8}{mj}分\end{CJK}, \begin{CJK}{UTF8}{mj}共\end{CJK} 36 \begin{CJK}{UTF8}{mj}分\end{CJK})
\end{enumerate}
(1) \begin{CJK}{UTF8}{mj}设\end{CJK} $a>0, a \neq 1$. \begin{CJK}{UTF8}{mj}求极限\end{CJK}
$$
\lim _{x \rightarrow+\infty}\left(\frac{a^{x}-1}{(a-1) x}\right)^{\frac{1}{x}} .
$$
(2) \begin{CJK}{UTF8}{mj}设圆盘\end{CJK} $(x-a)^{2}+(y-b)^{2} \leq R^{2}$ \begin{CJK}{UTF8}{mj}上各点的密度等于该点到圆心的距离\end{CJK}, \begin{CJK}{UTF8}{mj}求此圆盘的质量\end{CJK}.

(3) \begin{CJK}{UTF8}{mj}设\end{CJK} $S$ \begin{CJK}{UTF8}{mj}为\end{CJK} $R^{3}$ \begin{CJK}{UTF8}{mj}中封闭光滑曲面\end{CJK}, $\vec{l}$ \begin{CJK}{UTF8}{mj}为任意固定方向\end{CJK}, $\vec{n}$ \begin{CJK}{UTF8}{mj}为曲面\end{CJK} $S$ \begin{CJK}{UTF8}{mj}的外法线方向\end{CJK}. \begin{CJK}{UTF8}{mj}求\end{CJK}
$$
\iint_{S} \cos (\vec{n}, \vec{l}) \mathrm{d} S
$$

\begin{enumerate}
  \setcounter{enumi}{3}
  \item \begin{CJK}{UTF8}{mj}证明下列各题\end{CJK} (\begin{CJK}{UTF8}{mj}每小题\end{CJK} 10 \begin{CJK}{UTF8}{mj}分\end{CJK}, \begin{CJK}{UTF8}{mj}共\end{CJK} 60 \begin{CJK}{UTF8}{mj}分\end{CJK})
\end{enumerate}
(1) \begin{CJK}{UTF8}{mj}设\end{CJK} $P_{0}$ \begin{CJK}{UTF8}{mj}是曲面\end{CJK} $S: \frac{x^{2}}{a^{2}}+\frac{y^{2}}{b^{2}}+\frac{z^{2}}{c^{2}}=1$ \begin{CJK}{UTF8}{mj}外的一点\end{CJK}, $P_{1} \in S$. \begin{CJK}{UTF8}{mj}若\end{CJK} $\left|\overline{P_{0} P_{1}}\right|=\max _{P \in S}\left|\overline{P_{0} P}\right|$, \begin{CJK}{UTF8}{mj}求证直线\end{CJK} $P_{0} P$ \begin{CJK}{UTF8}{mj}是\end{CJK} $S$ \begin{CJK}{UTF8}{mj}在\end{CJK} $P_{1}$ \begin{CJK}{UTF8}{mj}处的法线\end{CJK}.

(2) \begin{CJK}{UTF8}{mj}设函数\end{CJK}
$$
f(x, y)= \begin{cases}\frac{y^{3} \sin \frac{y}{x}}{x^{2}+y^{2}}, & x \neq 0 \\ 0, & x=0 .\end{cases}
$$
\begin{CJK}{UTF8}{mj}求证\end{CJK}: \begin{CJK}{UTF8}{mj}在原点处\end{CJK} $f(x, y)$ \begin{CJK}{UTF8}{mj}连续\end{CJK}、\begin{CJK}{UTF8}{mj}沿任何方向的方向导数存在\end{CJK}、\begin{CJK}{UTF8}{mj}但不可微\end{CJK}.

(3) \begin{CJK}{UTF8}{mj}设\end{CJK} $a<b, c<d$ \begin{CJK}{UTF8}{mj}均为实数\end{CJK}. \begin{CJK}{UTF8}{mj}已知\end{CJK} $f(x)$ \begin{CJK}{UTF8}{mj}在\end{CJK} $(a, b)$ \begin{CJK}{UTF8}{mj}上单调\end{CJK}, \begin{CJK}{UTF8}{mj}其值域为\end{CJK} $c<d$, \begin{CJK}{UTF8}{mj}求证\end{CJK} $f(x)$ \begin{CJK}{UTF8}{mj}在\end{CJK} $(a, b)$ \begin{CJK}{UTF8}{mj}上一致连续\end{CJK}.

(4) \begin{CJK}{UTF8}{mj}设数列满足条件\end{CJK} $\forall n \in \mathbb{N}, a_{n}>0$ \begin{CJK}{UTF8}{mj}且\end{CJK}
$$
\lim _{n \rightarrow \infty} \frac{a_{n}}{a_{n+2}+a_{n+4}}=0
$$
\begin{CJK}{UTF8}{mj}求证\end{CJK} $\left\{a_{n}\right\}$ \begin{CJK}{UTF8}{mj}为无界数列\end{CJK}.

(5) \begin{CJK}{UTF8}{mj}设\end{CJK} $f(x)$ \begin{CJK}{UTF8}{mj}在\end{CJK} $[0,+\infty)$ \begin{CJK}{UTF8}{mj}上连续且有界\end{CJK}. \begin{CJK}{UTF8}{mj}证明对于任意正数\end{CJK} $T$, \begin{CJK}{UTF8}{mj}存在\end{CJK} $x_{n} \rightarrow \infty$, \begin{CJK}{UTF8}{mj}使得\end{CJK}
$$
\lim _{n \rightarrow \infty}\left(f\left(x_{n}+T\right)-f\left(x_{n}\right)\right)=0 .
$$
(6) \begin{CJK}{UTF8}{mj}设函数\end{CJK} $f(x)$ \begin{CJK}{UTF8}{mj}在闭区间\end{CJK} $[a, b](a<b)$ \begin{CJK}{UTF8}{mj}上可积\end{CJK}, $\int_{a}^{b} f(x) \mathrm{d} x=0$. \begin{CJK}{UTF8}{mj}证明\end{CJK}: \begin{CJK}{UTF8}{mj}若对任意的\end{CJK} $x \in[a, b]$ \begin{CJK}{UTF8}{mj}有\end{CJK} $f(x) \neq 0$, \begin{CJK}{UTF8}{mj}则\end{CJK} \begin{CJK}{UTF8}{mj}存在\end{CJK} $[c, d] \subseteq[a, b]$, \begin{CJK}{UTF8}{mj}使得对任意的\end{CJK} $x \in[c, d]$ \begin{CJK}{UTF8}{mj}有\end{CJK} $f(x)>0$.

\section{2. 华东师范大学 2010 年研究生入学考试试题数学分析 
 李扬 
 微信公众号: sxkyliyang}
\begin{enumerate}
  \item \begin{CJK}{UTF8}{mj}求解下列各题\end{CJK} (\begin{CJK}{UTF8}{mj}每小题\end{CJK} 12 \begin{CJK}{UTF8}{mj}分\end{CJK}, \begin{CJK}{UTF8}{mj}共\end{CJK} 60 \begin{CJK}{UTF8}{mj}分\end{CJK})
\end{enumerate}
(1) \begin{CJK}{UTF8}{mj}设曲线\end{CJK} $\Gamma: x=x(t)=t^{2}, y=y(t)=e^{t}+2, z=z(t)=t+\cos t, t \in \mathbb{R}$. \begin{CJK}{UTF8}{mj}试求\end{CJK} $\Gamma$ \begin{CJK}{UTF8}{mj}在点\end{CJK} $(x(0), y(0), z(0))$ \begin{CJK}{UTF8}{mj}处的法线方程与切平面方程\end{CJK}.

(2) \begin{CJK}{UTF8}{mj}求由方程\end{CJK} $x^{2}+2 x y+2 y^{2}=1$ \begin{CJK}{UTF8}{mj}所确定的隐函数\end{CJK} $y=f(x)$ \begin{CJK}{UTF8}{mj}的极值\end{CJK}.

(3) \begin{CJK}{UTF8}{mj}计算\end{CJK}
$$
\iint_{S}(x-a)^{3} \mathrm{~d} y \mathrm{~d} z+(y-b)^{3} \mathrm{~d} z \mathrm{~d} x+(z-c)^{3} \mathrm{~d} x \mathrm{~d} y
$$
\begin{CJK}{UTF8}{mj}其中\end{CJK} $S$ \begin{CJK}{UTF8}{mj}是球面\end{CJK} $(x-a)^{2}+(y-b)^{2}+(z-c)^{2}=1$ \begin{CJK}{UTF8}{mj}的外侧\end{CJK}.

(4) \begin{CJK}{UTF8}{mj}求函数\end{CJK} $f(x)=\frac{e^{x}}{2-2 x}$ \begin{CJK}{UTF8}{mj}在\end{CJK} $x=0$ \begin{CJK}{UTF8}{mj}处的泰勒展开式\end{CJK}, \begin{CJK}{UTF8}{mj}并求\end{CJK} $f^{(n)}(0)$.

(5) \begin{CJK}{UTF8}{mj}设\end{CJK}
$$
g(x, y)=\int_{0}^{+\infty} \frac{\arctan (t x) \arctan (t y)}{t^{2}} \mathrm{~d} t,(x, y) \in(0,+\infty) \times(0,+\infty)
$$
\begin{CJK}{UTF8}{mj}试求\end{CJK} $g_{x y}(x, y)$.

\begin{enumerate}
  \setcounter{enumi}{2}
  \item \begin{CJK}{UTF8}{mj}证明下列各题\end{CJK} (\begin{CJK}{UTF8}{mj}每小题\end{CJK} 12 \begin{CJK}{UTF8}{mj}分\end{CJK}, \begin{CJK}{UTF8}{mj}共\end{CJK} 60 \begin{CJK}{UTF8}{mj}分\end{CJK})
\end{enumerate}
(1) \begin{CJK}{UTF8}{mj}已知\end{CJK} $f(x, y)$ \begin{CJK}{UTF8}{mj}在点\end{CJK} $\left(x_{0}, y_{0}\right)$ \begin{CJK}{UTF8}{mj}可微且\end{CJK} $f\left(x_{0}, y_{0}\right)=0, g(x, y)$ \begin{CJK}{UTF8}{mj}在点\end{CJK} $\left(x_{0}, y_{0}\right)$ \begin{CJK}{UTF8}{mj}连续\end{CJK}. \begin{CJK}{UTF8}{mj}试证\end{CJK} $f(x, y) g(x, y)$ \begin{CJK}{UTF8}{mj}在点\end{CJK} $\left(x_{0}, y_{0}\right)$ \begin{CJK}{UTF8}{mj}可微\end{CJK}, \begin{CJK}{UTF8}{mj}且\end{CJK} $d(f g)\left(x_{0}, y_{0}\right)=g\left(x_{0}, y_{0}\right) \mathrm{d} f\left(x_{0}, y_{0}\right)$.

(2) \begin{CJK}{UTF8}{mj}设\end{CJK} $f(x)$ \begin{CJK}{UTF8}{mj}为定义在\end{CJK} $[a,+\infty)$ \begin{CJK}{UTF8}{mj}上的正值连续函数\end{CJK}. \begin{CJK}{UTF8}{mj}证明\end{CJK}: \begin{CJK}{UTF8}{mj}若\end{CJK} $\lim _{x \rightarrow+\infty} \frac{f(x+1)}{f(x)}=q<1$, \begin{CJK}{UTF8}{mj}则反常积分\end{CJK} $\int_{a}^{+\infty} f(x) \mathrm{d} x$ \begin{CJK}{UTF8}{mj}收敛\end{CJK}.

(3) \begin{CJK}{UTF8}{mj}证明\end{CJK}:

(i) $n \in \mathbb{N}_{+}$, \begin{CJK}{UTF8}{mj}关于\end{CJK} $x$ \begin{CJK}{UTF8}{mj}的方程\end{CJK} $\sum_{k=1}^{n} e^{k x}=n+1$ \begin{CJK}{UTF8}{mj}在\end{CJK} $[0,1]$ \begin{CJK}{UTF8}{mj}上必定存在唯一实根\end{CJK}(\begin{CJK}{UTF8}{mj}记为\end{CJK} $\left.a_{n}\right)$;

(ii) \begin{CJK}{UTF8}{mj}数列\end{CJK} $\left\{a_{n}\right\}$ \begin{CJK}{UTF8}{mj}必定收敛\end{CJK}, \begin{CJK}{UTF8}{mj}并求其极限\end{CJK}.

(4) \begin{CJK}{UTF8}{mj}设\end{CJK} $f(x, y)$ \begin{CJK}{UTF8}{mj}在\end{CJK} $[a, b] \times[c, d](a<b, c<d)$ \begin{CJK}{UTF8}{mj}上连续\end{CJK}, \begin{CJK}{UTF8}{mj}令\end{CJK} $M(x)=\max _{y \in[c, d]} f(x, y), x \in[a, b]$, \begin{CJK}{UTF8}{mj}证明\end{CJK}: $M(x)$ \begin{CJK}{UTF8}{mj}在\end{CJK} $[a, b]$ \begin{CJK}{UTF8}{mj}上连续\end{CJK}.

(5) \begin{CJK}{UTF8}{mj}设\end{CJK} $f(x)$ \begin{CJK}{UTF8}{mj}在\end{CJK} $[a, b]$ \begin{CJK}{UTF8}{mj}上可导\end{CJK} $(a<b)$. \begin{CJK}{UTF8}{mj}证明\end{CJK}: $f(x)$ \begin{CJK}{UTF8}{mj}在\end{CJK} $[a, b]$ \begin{CJK}{UTF8}{mj}上一致可导的充要条件是\end{CJK} $f^{\prime}(x)$ \begin{CJK}{UTF8}{mj}在\end{CJK} $[a, b]$ \begin{CJK}{UTF8}{mj}上连续\end{CJK}.

\begin{CJK}{UTF8}{mj}说明\end{CJK} $f(x)$ \begin{CJK}{UTF8}{mj}在\end{CJK} $[a, b]$ \begin{CJK}{UTF8}{mj}上一致可导是指\end{CJK}: $\forall \varepsilon>0, \exists \delta>0$, \begin{CJK}{UTF8}{mj}使得\end{CJK} $\forall x, y \in[a, b]$, \begin{CJK}{UTF8}{mj}只要\end{CJK} $0<|x-y|<\delta$, \begin{CJK}{UTF8}{mj}就有\end{CJK} $\left|\frac{f(x)-f(y)}{x-y}-f^{\prime}(x)\right|<\varepsilon$ \begin{CJK}{UTF8}{mj}成立\end{CJK}.

\begin{enumerate}
  \setcounter{enumi}{3}
  \item (30 \begin{CJK}{UTF8}{mj}分\end{CJK}) ) \begin{CJK}{UTF8}{mj}设可积函数列\end{CJK} $\left\{f_{n}(x)\right\}$ \begin{CJK}{UTF8}{mj}在\end{CJK} $[a, b]$ \begin{CJK}{UTF8}{mj}上一致收敛于\end{CJK} $f(x)$. \begin{CJK}{UTF8}{mj}证明\end{CJK}:
\end{enumerate}
(1) $f(x)$ \begin{CJK}{UTF8}{mj}在\end{CJK} $[a, b]$ \begin{CJK}{UTF8}{mj}上可积\end{CJK}, \begin{CJK}{UTF8}{mj}且\end{CJK} $\lim _{n \rightarrow \infty} \int_{a}^{b} f_{n}(x) \mathrm{d} x=\int_{a}^{b} f(x) \mathrm{d} x$;

(2) $\left\{f_{n}(x)\right\}$ \begin{CJK}{UTF8}{mj}在\end{CJK} $[a, b]$ \begin{CJK}{UTF8}{mj}上一致可积\end{CJK} $\left(\right.$ \begin{CJK}{UTF8}{mj}一致可积是指\end{CJK}: $\forall \varepsilon>0, \exists \delta>0$, \begin{CJK}{UTF8}{mj}使得对任意分割\end{CJK} $T: a=x_{0}<x_{1}<\cdots<$ $x_{k}=b$, \begin{CJK}{UTF8}{mj}只要\end{CJK} $\max _{1 \leq i \leq k} \Delta x_{i}<\delta$, \begin{CJK}{UTF8}{mj}就有\end{CJK}
$$
\left|\int_{a}^{b} f_{n}(x) \mathrm{d} x-\sum_{i=1}^{k} f_{n}\left(\xi_{i}\right) \Delta x_{i}\right|<\varepsilon .
$$
\begin{CJK}{UTF8}{mj}对任意\end{CJK} $\xi_{i} \in\left[x_{i-1}, x_{i}\right], 1 \leq i \leq k$, \begin{CJK}{UTF8}{mj}及任意\end{CJK} $n \in \mathbb{N}_{+}$\begin{CJK}{UTF8}{mj}都成立\end{CJK});

(3) \begin{CJK}{UTF8}{mj}举例说明\end{CJK} (2) \begin{CJK}{UTF8}{mj}的逆不真\end{CJK}.

\section{3. 华东师范大学 2011 年研究生入学考试试题数学分析 
 李扬 
 微信公众号: sxkyliyang}
\begin{enumerate}
  \item \begin{CJK}{UTF8}{mj}求解下列各题\end{CJK} (\begin{CJK}{UTF8}{mj}每小题\end{CJK} 12 \begin{CJK}{UTF8}{mj}分\end{CJK}, \begin{CJK}{UTF8}{mj}共\end{CJK} 48 \begin{CJK}{UTF8}{mj}分\end{CJK})
\end{enumerate}
(1) \begin{CJK}{UTF8}{mj}设\end{CJK} $f(x, y)$ \begin{CJK}{UTF8}{mj}在\end{CJK} $D=[0,1] \times[0,1]$ \begin{CJK}{UTF8}{mj}上连续\end{CJK}, \begin{CJK}{UTF8}{mj}求\end{CJK}
$$
\lim _{n \rightarrow \infty}\left(\iint_{D}|f(x, y)|^{n} \mathrm{~d} x \mathrm{~d} y\right)^{\frac{1}{n}}
$$
(2) \begin{CJK}{UTF8}{mj}求幂级数\end{CJK} $\sum_{n=1}^{\infty} n x^{2 n+1}$ \begin{CJK}{UTF8}{mj}的收敛域及和函数\end{CJK}.

(3) \begin{CJK}{UTF8}{mj}求椭球面\end{CJK} $\frac{x^{2}}{3}+(y-1)^{2}+\frac{z^{2}}{2}=1$ \begin{CJK}{UTF8}{mj}被平面\end{CJK} $2 x+y+z=1$ \begin{CJK}{UTF8}{mj}截成的椭圆的面积\end{CJK}.

(4) \begin{CJK}{UTF8}{mj}设\end{CJK} $f(x)$ \begin{CJK}{UTF8}{mj}在\end{CJK} $\mathbb{R}$ \begin{CJK}{UTF8}{mj}连续可微\end{CJK}, $L$ \begin{CJK}{UTF8}{mj}为逐段光滑闭曲线\end{CJK}, \begin{CJK}{UTF8}{mj}求\end{CJK}
$$
\oint_{L} f(\sin x+\sin y)(\cos x \mathrm{~d} x+\cos y \mathrm{~d} y)
$$

\begin{enumerate}
  \setcounter{enumi}{2}
  \item \begin{CJK}{UTF8}{mj}证明下列各题\end{CJK} (\begin{CJK}{UTF8}{mj}每小题\end{CJK} 12 \begin{CJK}{UTF8}{mj}分\end{CJK}, \begin{CJK}{UTF8}{mj}共\end{CJK} 72 \begin{CJK}{UTF8}{mj}分\end{CJK})
\end{enumerate}
(1) \begin{CJK}{UTF8}{mj}用\end{CJK} $\varepsilon-\delta$ \begin{CJK}{UTF8}{mj}语言证明\end{CJK}
$$
\lim _{x \rightarrow 2^{-}}\left(\sqrt{\frac{1}{2-x}+2}-\sqrt{\frac{1}{2-x}-2}\right)=0
$$
(2) \begin{CJK}{UTF8}{mj}证明\end{CJK}: $F(\alpha)=\int_{0}^{+\infty} \frac{x^{2}}{1+2 x^{\alpha}} \mathrm{d} x$ \begin{CJK}{UTF8}{mj}在\end{CJK} $(3,+\infty)$ \begin{CJK}{UTF8}{mj}上连续\end{CJK}.

(3) \begin{CJK}{UTF8}{mj}设\end{CJK} $f(x)$ \begin{CJK}{UTF8}{mj}在\end{CJK} $[a, b](a<b$ \begin{CJK}{UTF8}{mj}为实数\end{CJK} $)$ \begin{CJK}{UTF8}{mj}上连续\end{CJK}, \begin{CJK}{UTF8}{mj}且\end{CJK} $f(a) f(b)<0$, \begin{CJK}{UTF8}{mj}试用有限覆盖定理证明存在\end{CJK} $\xi \in(a, b)$, \begin{CJK}{UTF8}{mj}使得\end{CJK} $f(\xi)=0 .$

(4) \begin{CJK}{UTF8}{mj}设\end{CJK} $f(x) \in C[a,+\infty)$ (\begin{CJK}{UTF8}{mj}即\end{CJK} $f(x)$ \begin{CJK}{UTF8}{mj}在\end{CJK} $[a,+\infty)$ \begin{CJK}{UTF8}{mj}上连续\end{CJK}) \begin{CJK}{UTF8}{mj}且\end{CJK} $\int_{a}^{+\infty}|f(x)| \mathrm{d} x$ \begin{CJK}{UTF8}{mj}收敛\end{CJK}, \begin{CJK}{UTF8}{mj}则存在数列\end{CJK} $x_{n} \subset[a,+\infty)$ \begin{CJK}{UTF8}{mj}满足\end{CJK} \begin{CJK}{UTF8}{mj}条件\end{CJK}: $\lim _{n \rightarrow \infty} x_{n}=+\infty$ \begin{CJK}{UTF8}{mj}及\end{CJK} $\lim _{n \rightarrow \infty} x_{n} f\left(x_{n}\right)=0$.

(5) \begin{CJK}{UTF8}{mj}设\end{CJK} $f(x, y, z)$ \begin{CJK}{UTF8}{mj}在\end{CJK} $[a,+\infty) \times[b,+\infty) \times[c,+\infty)$ \begin{CJK}{UTF8}{mj}上连续且无下界\end{CJK}, \begin{CJK}{UTF8}{mj}对于任意的\end{CJK} $s \in \mathbb{R}, f(x, y, z)=s$ \begin{CJK}{UTF8}{mj}的解集为\end{CJK} \begin{CJK}{UTF8}{mj}有界集\end{CJK}, \begin{CJK}{UTF8}{mj}证明\end{CJK}:
$$
\lim _{(x, y, z) \rightarrow(+\infty,+\infty,+\infty)} f(x, y, z)=-\infty .
$$
(6) \begin{CJK}{UTF8}{mj}设\end{CJK} $\left\{I_{n}\right\}$ \begin{CJK}{UTF8}{mj}为\end{CJK} $\mathbb{R}^{2}$ \begin{CJK}{UTF8}{mj}中一列闭集并且满足\end{CJK}: \begin{CJK}{UTF8}{mj}对任意的\end{CJK} $n \in \mathbb{N}, I_{n} \subseteq I_{n+1}, \bigcup_{n=1}^{\infty} I_{n}=(0,1) \times(0,1)$. \begin{CJK}{UTF8}{mj}证明存在\end{CJK} $P_{0} \in(0,1) \times(0,1), r>0$ \begin{CJK}{UTF8}{mj}及\end{CJK} $k \in \mathbb{N}$, \begin{CJK}{UTF8}{mj}使得闭球\end{CJK} $B\left(P_{0} ; r\right) \subseteq I_{k}$.

\begin{enumerate}
  \setcounter{enumi}{3}
  \item \begin{CJK}{UTF8}{mj}证明下列结论\end{CJK} (\begin{CJK}{UTF8}{mj}每小题\end{CJK} 10 \begin{CJK}{UTF8}{mj}分\end{CJK}, \begin{CJK}{UTF8}{mj}共\end{CJK} 20 \begin{CJK}{UTF8}{mj}分\end{CJK})
\end{enumerate}
(1) \begin{CJK}{UTF8}{mj}设\end{CJK} $f(x)$ \begin{CJK}{UTF8}{mj}为\end{CJK} $\mathbb{R}$ \begin{CJK}{UTF8}{mj}上的实值函数\end{CJK}, \begin{CJK}{UTF8}{mj}证明\end{CJK}: $f(x)$ \begin{CJK}{UTF8}{mj}的极值所组成的集合是至多可数的\end{CJK}.

(2) \begin{CJK}{UTF8}{mj}设\end{CJK} $f(x)$ \begin{CJK}{UTF8}{mj}为\end{CJK} $\mathbb{R}$ \begin{CJK}{UTF8}{mj}上的连续函数且\end{CJK} $\mathbb{R}$ \begin{CJK}{UTF8}{mj}上每一点均为\end{CJK} $f(x)$ \begin{CJK}{UTF8}{mj}的极值点\end{CJK}, \begin{CJK}{UTF8}{mj}求证\end{CJK}: $f(x)$ \begin{CJK}{UTF8}{mj}为常值函数\end{CJK}.

\begin{enumerate}
  \setcounter{enumi}{4}
  \item (10 \begin{CJK}{UTF8}{mj}分\end{CJK}) \begin{CJK}{UTF8}{mj}设\end{CJK} $f: \mathbb{R} \rightarrow \mathbb{R}$ \begin{CJK}{UTF8}{mj}是连续函数\end{CJK}, \begin{CJK}{UTF8}{mj}记\end{CJK} $f^{n}=f \circ f \circ \cdots \circ f$ \begin{CJK}{UTF8}{mj}为\end{CJK} $f(x)$ \begin{CJK}{UTF8}{mj}的\end{CJK} $n$ \begin{CJK}{UTF8}{mj}次复合\end{CJK}. \begin{CJK}{UTF8}{mj}又设\end{CJK} $f(x)$ \begin{CJK}{UTF8}{mj}没有不动点\end{CJK}, \begin{CJK}{UTF8}{mj}即对\end{CJK} \begin{CJK}{UTF8}{mj}任意\end{CJK} $x_{0} \in \mathbb{R}$ \begin{CJK}{UTF8}{mj}均有\end{CJK} $f\left(x_{0}\right) \neq x_{0}$. \begin{CJK}{UTF8}{mj}证明\end{CJK}: \begin{CJK}{UTF8}{mj}对任意\end{CJK} $x_{0} \in \mathbb{R}$, \begin{CJK}{UTF8}{mj}数列\end{CJK}
\end{enumerate}
$$
f^{n}\left(x_{0}\right), n=1,2, \cdots
$$
\begin{CJK}{UTF8}{mj}是无界的\end{CJK}.

\section{4. 华东师范大学 2012 年研究生入学考试试题数学分析 
 李扬 
 微信公众号: sxkyliyang}
\begin{enumerate}
  \item \begin{CJK}{UTF8}{mj}判断下列命题是否正确\end{CJK}, \begin{CJK}{UTF8}{mj}若正确给出证明\end{CJK}, \begin{CJK}{UTF8}{mj}若错误举出反例\end{CJK}(\begin{CJK}{UTF8}{mj}每小题\end{CJK} 8 \begin{CJK}{UTF8}{mj}分\end{CJK}, \begin{CJK}{UTF8}{mj}共\end{CJK} 32 \begin{CJK}{UTF8}{mj}分\end{CJK})
\end{enumerate}
(1) \begin{CJK}{UTF8}{mj}如果函数\end{CJK} $f(x)$ \begin{CJK}{UTF8}{mj}在区间\end{CJK} $[a, b]$ \begin{CJK}{UTF8}{mj}上可积\end{CJK}, \begin{CJK}{UTF8}{mj}则函数\end{CJK} $f(x)$ \begin{CJK}{UTF8}{mj}在\end{CJK} $[a, b]$ \begin{CJK}{UTF8}{mj}上存在原函数\end{CJK}.

(2) \begin{CJK}{UTF8}{mj}如果函数\end{CJK} $f(x, y, z)$ \begin{CJK}{UTF8}{mj}是有向光滑曲面\end{CJK} $S$ \begin{CJK}{UTF8}{mj}上的非负连续函数\end{CJK}, \begin{CJK}{UTF8}{mj}则\end{CJK} $\iint_{S} f(x, y, z) \mathrm{d} y \mathrm{~d} z \geq 0$.

(3) \begin{CJK}{UTF8}{mj}设函数\end{CJK} $f(x)$ \begin{CJK}{UTF8}{mj}在点\end{CJK} $x_{0}$ \begin{CJK}{UTF8}{mj}的空心邻域\end{CJK} $U_{+}^{0}\left(x_{0}\right)$ \begin{CJK}{UTF8}{mj}内有定义\end{CJK}. \begin{CJK}{UTF8}{mj}如果对于任何满足条件\end{CJK}
$$
\left\{x_{n}\right\} \subset U_{+}^{0}\left(x_{0}\right), \lim _{n \rightarrow+\infty} x_{n}=x_{0}, x_{n+1}<x_{n}\left(n \in \mathbb{N}_{+}\right)
$$
\begin{CJK}{UTF8}{mj}的数列\end{CJK} $\left\{x_{n}\right\}$ \begin{CJK}{UTF8}{mj}都有\end{CJK} $\lim _{n \rightarrow+\infty} f\left(x_{n}\right)$ \begin{CJK}{UTF8}{mj}存在\end{CJK}, \begin{CJK}{UTF8}{mj}则\end{CJK} $\lim _{x \rightarrow x_{0}^{+}} f(x)$ \begin{CJK}{UTF8}{mj}存在\end{CJK}.

(4) \begin{CJK}{UTF8}{mj}如果级数\end{CJK} $\sum_{n=1}^{\infty} a_{n}^{2}$ \begin{CJK}{UTF8}{mj}收玫\end{CJK}, \begin{CJK}{UTF8}{mj}则级数\end{CJK} $\sum_{n=1}^{\infty} \frac{a_{n}}{n}$ \begin{CJK}{UTF8}{mj}收玫\end{CJK}.

\begin{enumerate}
  \setcounter{enumi}{2}
  \item \begin{CJK}{UTF8}{mj}求解下列各题\end{CJK} (\begin{CJK}{UTF8}{mj}每小题\end{CJK} 8 \begin{CJK}{UTF8}{mj}分\end{CJK}, \begin{CJK}{UTF8}{mj}共\end{CJK} 40 \begin{CJK}{UTF8}{mj}分\end{CJK})
\end{enumerate}
(1) \begin{CJK}{UTF8}{mj}求极限\end{CJK}
$$
\lim _{x \rightarrow 0} \frac{e^{x} \sin x-x(1+x)}{x^{3}} .
$$
(2) \begin{CJK}{UTF8}{mj}设\end{CJK} $p>0,0<a<1$. \begin{CJK}{UTF8}{mj}讨论级数\end{CJK} $\sum_{n=1}^{\infty}\left(\left(1+\frac{p}{n}\right)^{n}-1-\frac{1}{n}\right)$ \begin{CJK}{UTF8}{mj}的敛散性\end{CJK}.

(3) \begin{CJK}{UTF8}{mj}计算积分\end{CJK}
$$
\iint_{S} y z \mathrm{~d} x \mathrm{~d} y+z x \mathrm{~d} y \mathrm{~d} z+x y \mathrm{~d} z \mathrm{~d} x
$$
\begin{CJK}{UTF8}{mj}其中\end{CJK} $S$ \begin{CJK}{UTF8}{mj}是由\end{CJK} $z=1, x^{2}+y^{2}=1$ \begin{CJK}{UTF8}{mj}以及三个坐标面所围成的第一卦限部分的外侧\end{CJK}.

(4) \begin{CJK}{UTF8}{mj}求函数\end{CJK}
$$
f(x, y)=x+x y^{2}+y^{2}
$$
\begin{CJK}{UTF8}{mj}在\end{CJK} $x^{2}+y^{2} \leq 1$ \begin{CJK}{UTF8}{mj}上的最大值和最小值\end{CJK}.

(5) \begin{CJK}{UTF8}{mj}讨论无穷积分\end{CJK}
$$
\int_{1}^{+\infty} \frac{\left(e^{\frac{1}{x^{2}}}-1\right)^{a}}{\ln ^{b}\left(1+\frac{1}{x}\right)} d x
$$
\begin{CJK}{UTF8}{mj}的敛散性\end{CJK}, \begin{CJK}{UTF8}{mj}其中\end{CJK} $a, b$ \begin{CJK}{UTF8}{mj}为常数\end{CJK}.

\begin{enumerate}
  \setcounter{enumi}{3}
  \item \begin{CJK}{UTF8}{mj}证明下列各题\end{CJK} (\begin{CJK}{UTF8}{mj}每小题\end{CJK} 13 \begin{CJK}{UTF8}{mj}分\end{CJK}, \begin{CJK}{UTF8}{mj}共\end{CJK} 78 \begin{CJK}{UTF8}{mj}分\end{CJK})
\end{enumerate}
(1) \begin{CJK}{UTF8}{mj}设函数\end{CJK} $f(x, y, z)$ \begin{CJK}{UTF8}{mj}和\end{CJK} $g(x, y, z)$ \begin{CJK}{UTF8}{mj}都在可求长的连续曲线\end{CJK} $L \subset \mathbb{R}^{3}$ \begin{CJK}{UTF8}{mj}上连续\end{CJK}, \begin{CJK}{UTF8}{mj}且\end{CJK} $g(x, y, z)$ \begin{CJK}{UTF8}{mj}是非负的\end{CJK}. \begin{CJK}{UTF8}{mj}证明\end{CJK}: \begin{CJK}{UTF8}{mj}存在\end{CJK} $(\xi, \eta, \zeta) \in L$, \begin{CJK}{UTF8}{mj}使得\end{CJK}
$$
\int_{L} f(x, y, z) g(x, y, z) \mathrm{d} s=f(\xi, \eta, \zeta) \int_{L} g(x, y, z) \mathrm{d} s .
$$
(2) \begin{CJK}{UTF8}{mj}设\end{CJK} $f(x)$ \begin{CJK}{UTF8}{mj}为\end{CJK} $[0,+\infty)$ \begin{CJK}{UTF8}{mj}上的连续可微的凹函数\end{CJK}, \begin{CJK}{UTF8}{mj}且\end{CJK} $\lim _{x \rightarrow+\infty} f(x)$ \begin{CJK}{UTF8}{mj}存在\end{CJK}, \begin{CJK}{UTF8}{mj}证明\end{CJK}: $\lim _{x \rightarrow+\infty} f^{\prime}(x)=0$.

(3) \begin{CJK}{UTF8}{mj}设\end{CJK} $f(x)$ \begin{CJK}{UTF8}{mj}在\end{CJK} $[0,1]$ \begin{CJK}{UTF8}{mj}上连续可微\end{CJK}, \begin{CJK}{UTF8}{mj}且\end{CJK}
$$
f(0)=0,0<f^{\prime}(x)<1 .
$$
\begin{CJK}{UTF8}{mj}证明\end{CJK}:
$$
\int_{0}^{1}(f(x))^{3} \mathrm{~d} x \leq\left(\int_{0}^{1} f(x) \mathrm{d} x\right)^{2}, t \in[0,1]
$$
(4) \begin{CJK}{UTF8}{mj}设函数\end{CJK}
$$
f(x)=\sum_{n=1}^{\infty} \frac{\left|x-\frac{1}{n}\right|}{2^{n}}, x \in(0,+\infty)
$$
\begin{CJK}{UTF8}{mj}证明\end{CJK}: (i) \begin{CJK}{UTF8}{mj}函数\end{CJK} $f(x)$ \begin{CJK}{UTF8}{mj}在点\end{CJK} $x \neq \frac{1}{n}(n=1,2, \cdots)$ \begin{CJK}{UTF8}{mj}处可导\end{CJK}.

(ii) \begin{CJK}{UTF8}{mj}函数\end{CJK} $f(x)$ \begin{CJK}{UTF8}{mj}在点\end{CJK} $x=\frac{1}{n}(n=1,2, \cdots)$ \begin{CJK}{UTF8}{mj}处不可导\end{CJK}.

(5) \begin{CJK}{UTF8}{mj}设\end{CJK} $D$ \begin{CJK}{UTF8}{mj}为有界闭区域\end{CJK}, $f(x, y)$ \begin{CJK}{UTF8}{mj}在\end{CJK} $D$ \begin{CJK}{UTF8}{mj}上连续且存在一阶偏导数\end{CJK}, \begin{CJK}{UTF8}{mj}已知\end{CJK}
$$
\frac{\partial f}{\partial x}+\frac{\partial f}{\partial y}=f,\left.f\right|_{\partial D}=0 .
$$
\begin{CJK}{UTF8}{mj}证明\end{CJK}: $f(x, y) \equiv 0$.

(6) \begin{CJK}{UTF8}{mj}设函数\end{CJK} $f(x)$ \begin{CJK}{UTF8}{mj}在\end{CJK} $[0,1]$ \begin{CJK}{UTF8}{mj}上连续可微\end{CJK}, \begin{CJK}{UTF8}{mj}且满足\end{CJK}
$$
f^{\prime \prime}(x) \leq 0,0 \leq f(x) \leq 1, f(0)=f(1)=0
$$
\begin{CJK}{UTF8}{mj}证明\end{CJK}: \begin{CJK}{UTF8}{mj}曲线\end{CJK} $C: y=f(x), x \in[0,1]$ \begin{CJK}{UTF8}{mj}的弧长\end{CJK} $s \leq 3$.

\section{5. 华东师范大学 2013 年研究生入学考试试题数学分析 
 李扬 
 微信公众号: sxkyliyang}
\begin{enumerate}
  \item \begin{CJK}{UTF8}{mj}判断下列命题是否正确\end{CJK}, \begin{CJK}{UTF8}{mj}若正确给出证明\end{CJK}, \begin{CJK}{UTF8}{mj}若错误举出反例\end{CJK}(\begin{CJK}{UTF8}{mj}每小题\end{CJK} 6 \begin{CJK}{UTF8}{mj}分\end{CJK}, \begin{CJK}{UTF8}{mj}共\end{CJK} 36 \begin{CJK}{UTF8}{mj}分\end{CJK})
\end{enumerate}
(1) \begin{CJK}{UTF8}{mj}已知数列\end{CJK} $a_{n}>0$, \begin{CJK}{UTF8}{mj}且\end{CJK} $\lim _{n \rightarrow \infty} a_{n}=0$, \begin{CJK}{UTF8}{mj}则级数\end{CJK} $\sum_{n=1}^{\infty}(-1)^{n} a_{n}$ \begin{CJK}{UTF8}{mj}收敛\end{CJK}.

(2) \begin{CJK}{UTF8}{mj}如果函数\end{CJK} $f(u)$ \begin{CJK}{UTF8}{mj}在点\end{CJK} $u_{0}$ \begin{CJK}{UTF8}{mj}连续\end{CJK}, \begin{CJK}{UTF8}{mj}且\end{CJK} $\lim _{x \rightarrow x_{0}} g(x)=u_{0}$, \begin{CJK}{UTF8}{mj}则\end{CJK} $\lim _{x \rightarrow x_{0}} f(g(x))=f\left(u_{0}\right)$.

(3) \begin{CJK}{UTF8}{mj}如果对任何\end{CJK} $n \in \mathbb{N}_{+}, u_{n}(x)$ \begin{CJK}{UTF8}{mj}在\end{CJK} $[a, b]$ \begin{CJK}{UTF8}{mj}上连续\end{CJK}, \begin{CJK}{UTF8}{mj}级数\end{CJK} $\sum_{n=1}^{\infty} u_{n}(x)$ \begin{CJK}{UTF8}{mj}在\end{CJK} $[a, b]$ \begin{CJK}{UTF8}{mj}上收敛\end{CJK}, \begin{CJK}{UTF8}{mj}且和函数在\end{CJK} $[a, b]$ \begin{CJK}{UTF8}{mj}上连续\end{CJK}, \begin{CJK}{UTF8}{mj}则\end{CJK} \begin{CJK}{UTF8}{mj}级数\end{CJK} $\sum_{n=1}^{\infty} u_{n}(x)$ \begin{CJK}{UTF8}{mj}在\end{CJK} $[a, b]$ \begin{CJK}{UTF8}{mj}上一致收敛\end{CJK}.

(4) \begin{CJK}{UTF8}{mj}如果函数\end{CJK} $f(x)$ \begin{CJK}{UTF8}{mj}在有限区间\end{CJK} $(a, b)$ \begin{CJK}{UTF8}{mj}上连续且有界\end{CJK}, \begin{CJK}{UTF8}{mj}则\end{CJK} $f(x)$ \begin{CJK}{UTF8}{mj}在\end{CJK} $(a, b)$ \begin{CJK}{UTF8}{mj}上一致连续\end{CJK}.

(5) \begin{CJK}{UTF8}{mj}如果函数\end{CJK} $f(x, y)$ \begin{CJK}{UTF8}{mj}在点\end{CJK} $P_{0}\left(x_{0}, y_{0}\right)$ \begin{CJK}{UTF8}{mj}处的两个二阶偏导数\end{CJK} $f_{x y}\left(P_{0}\right)$ \begin{CJK}{UTF8}{mj}与\end{CJK} $f_{y x}\left(P_{0}\right)$ \begin{CJK}{UTF8}{mj}都存在\end{CJK}, \begin{CJK}{UTF8}{mj}则\end{CJK} $f_{x y}\left(P_{0}\right)=f_{y x}\left(P_{0}\right)$.

(6) \begin{CJK}{UTF8}{mj}如果\end{CJK} $f(x)$ \begin{CJK}{UTF8}{mj}在\end{CJK} $[a,+\infty)$ \begin{CJK}{UTF8}{mj}上可导\end{CJK}, \begin{CJK}{UTF8}{mj}且\end{CJK} $\lim _{x \rightarrow+\infty} f(x)=A \in \mathbb{R}$, \begin{CJK}{UTF8}{mj}则\end{CJK} $\lim _{x \rightarrow+\infty} f^{\prime}(x)=0$.

\begin{enumerate}
  \setcounter{enumi}{2}
  \item \begin{CJK}{UTF8}{mj}求解下列各题\end{CJK} (\begin{CJK}{UTF8}{mj}每小题\end{CJK} 9 \begin{CJK}{UTF8}{mj}分\end{CJK}, \begin{CJK}{UTF8}{mj}共\end{CJK} 36 \begin{CJK}{UTF8}{mj}分\end{CJK})
\end{enumerate}
(1) \begin{CJK}{UTF8}{mj}求和\end{CJK}
$$
\sum_{n=2}^{\infty} \frac{(-1)^{n}}{n^{2}+n-2}
$$
(2) \begin{CJK}{UTF8}{mj}计算积分\end{CJK} $\int_{L} x y \mathrm{~d} s$, \begin{CJK}{UTF8}{mj}其中\end{CJK} $L$ \begin{CJK}{UTF8}{mj}为曲线\end{CJK} $x^{2}+y^{2}+z^{2}=1$ \begin{CJK}{UTF8}{mj}与平面\end{CJK} $x+y+z=0$ \begin{CJK}{UTF8}{mj}的交线\end{CJK}.

(3) \begin{CJK}{UTF8}{mj}计算积分\end{CJK}
$$
\int_{0}^{1} \frac{1}{\left[\frac{1}{x}\right]} d x
$$
(4)\begin{CJK}{UTF8}{mj}求极限\end{CJK}
$$
\lim _{x \rightarrow 0^{+}} \frac{\int_{1}^{+\infty} \frac{e^{-x y}-1}{y^{3}} \mathrm{~d} y}{\ln (1+x)}
$$

\begin{enumerate}
  \setcounter{enumi}{3}
  \item \begin{CJK}{UTF8}{mj}证明下列各题\end{CJK} (\begin{CJK}{UTF8}{mj}每小题\end{CJK} 13 \begin{CJK}{UTF8}{mj}分\end{CJK}, \begin{CJK}{UTF8}{mj}共\end{CJK} 78 \begin{CJK}{UTF8}{mj}分\end{CJK})
\end{enumerate}
(1) \begin{CJK}{UTF8}{mj}证明\end{CJK}: \begin{CJK}{UTF8}{mj}对于任何自然数\end{CJK} $n$, \begin{CJK}{UTF8}{mj}有\end{CJK}
$$
0<e-\left(1+\frac{1}{n}\right)^{n}<\frac{3}{n}
$$
(2) \begin{CJK}{UTF8}{mj}设\end{CJK} $g(x)$ \begin{CJK}{UTF8}{mj}在\end{CJK} $[a,+\infty)$ \begin{CJK}{UTF8}{mj}上不变号\end{CJK}, $\int_{a}^{+\infty} g(x) \mathrm{d} x$ \begin{CJK}{UTF8}{mj}收敛\end{CJK}, \begin{CJK}{UTF8}{mj}且\end{CJK} $f(x)$ \begin{CJK}{UTF8}{mj}在\end{CJK} $[a,+\infty)$ \begin{CJK}{UTF8}{mj}上连续有界\end{CJK}. \begin{CJK}{UTF8}{mj}证明\end{CJK}: \begin{CJK}{UTF8}{mj}存在\end{CJK} $\xi \in[a,+\infty)$, \begin{CJK}{UTF8}{mj}使得\end{CJK}
$$
\int_{a}^{+\infty} f(x) g(x) \mathrm{d} x=f(\xi) \int_{a}^{+\infty} g(x) \mathrm{d} x
$$
(3) \begin{CJK}{UTF8}{mj}设\end{CJK} $f(x)$ \begin{CJK}{UTF8}{mj}在\end{CJK} $[a,+\infty)$ \begin{CJK}{UTF8}{mj}上可微\end{CJK}, $0 \leq p<1$, \begin{CJK}{UTF8}{mj}且\end{CJK} $\lim _{x \rightarrow+\infty} x^{p} f^{\prime}(x)=A \in \mathbb{R}$. \begin{CJK}{UTF8}{mj}证明\end{CJK}: $f(x)$ \begin{CJK}{UTF8}{mj}在\end{CJK} $[a,+\infty)$ \begin{CJK}{UTF8}{mj}上一致连续\end{CJK}.

(4) \begin{CJK}{UTF8}{mj}设函数\end{CJK} $f(x)$ \begin{CJK}{UTF8}{mj}在\end{CJK} $(-\infty,+\infty)$ \begin{CJK}{UTF8}{mj}上连续\end{CJK}, $a$ \begin{CJK}{UTF8}{mj}和\end{CJK} $b$ \begin{CJK}{UTF8}{mj}是常数\end{CJK}. \begin{CJK}{UTF8}{mj}证明\end{CJK}:
$$
\iint_{x^{2}+y^{2} \leq 1} f(a x+b y) \mathrm{d} x \mathrm{~d} y=\iint_{u^{2}+v^{2} \leq 1} f\left(u \sqrt{a^{2}+b^{2}}\right) \mathrm{d} u \mathrm{~d} v .
$$
(5) \begin{CJK}{UTF8}{mj}设函数\end{CJK} $f(x, y)$ \begin{CJK}{UTF8}{mj}和\end{CJK} $g(x, y)$ \begin{CJK}{UTF8}{mj}在区域\end{CJK} $[0,1] \times[0,1]$ \begin{CJK}{UTF8}{mj}上连续\end{CJK}, \begin{CJK}{UTF8}{mj}且存在\end{CJK} $[0,1] \times[0,1]$ \begin{CJK}{UTF8}{mj}中的点列\end{CJK} $\left\{P_{n}\left(x_{n}, y_{n}\right)\right\}$ \begin{CJK}{UTF8}{mj}使得\end{CJK}
$$
f\left(x_{n}, y_{n}\right)=g\left(x_{n+1}, y_{n+1}\right), n \in \mathbb{N}_{+} .
$$
\begin{CJK}{UTF8}{mj}证明\end{CJK}: \begin{CJK}{UTF8}{mj}存在\end{CJK} $\left(x_{0}, y_{0}\right) \in[0,1] \times[0,1]$, \begin{CJK}{UTF8}{mj}使得\end{CJK} $f\left(x_{0}, y_{0}\right)=g\left(x_{0}, y_{0}\right)$.

(6) \begin{CJK}{UTF8}{mj}设\end{CJK} $f(x)$ \begin{CJK}{UTF8}{mj}在\end{CJK} $(-\infty,+\infty)$ \begin{CJK}{UTF8}{mj}上连续可微\end{CJK}, $\lim _{x \rightarrow+\infty} f^{\prime}(x)=A \in \mathbb{R}$, \begin{CJK}{UTF8}{mj}且对于任何\end{CJK} $x \in(-\infty,+\infty)$ \begin{CJK}{UTF8}{mj}有\end{CJK}
$$
f(x+1)-f(x)=f^{\prime}(x) .
$$
\begin{CJK}{UTF8}{mj}证明\end{CJK}: $f(x)$ \begin{CJK}{UTF8}{mj}是线性函数\end{CJK}, \begin{CJK}{UTF8}{mj}即存在常数\end{CJK} $a$ \begin{CJK}{UTF8}{mj}和\end{CJK} $b$ \begin{CJK}{UTF8}{mj}使得\end{CJK} $f(x)=a x+b, x \in(-\infty,+\infty)$.

\section{6. 华东师范大学 2014 年研究生入学考试试题数学分析 
 李扬 
 微信公众号: sxkyliyang}
\begin{enumerate}
  \item \begin{CJK}{UTF8}{mj}判断下列命题是否正确\end{CJK}, \begin{CJK}{UTF8}{mj}若正确给出证明\end{CJK}, \begin{CJK}{UTF8}{mj}若错误举出反例\end{CJK}(\begin{CJK}{UTF8}{mj}每小题\end{CJK} 6 \begin{CJK}{UTF8}{mj}分\end{CJK}, \begin{CJK}{UTF8}{mj}共\end{CJK} 36 \begin{CJK}{UTF8}{mj}分\end{CJK})
\end{enumerate}
(1) \begin{CJK}{UTF8}{mj}如果\end{CJK} $\forall p \in \mathbb{N}_{+}$, \begin{CJK}{UTF8}{mj}有\end{CJK} $\lim _{n \rightarrow \infty}\left(a_{n+p}-a_{n}\right)=0$, \begin{CJK}{UTF8}{mj}则数列\end{CJK} $\left\{a_{n}\right\}$ \begin{CJK}{UTF8}{mj}收敛\end{CJK}.

(2) \begin{CJK}{UTF8}{mj}如果偏导数\end{CJK} $f_{x}\left(x_{0}, y_{0}\right)$ \begin{CJK}{UTF8}{mj}和\end{CJK} $f_{y}\left(x_{0}, y_{0}\right)$ \begin{CJK}{UTF8}{mj}都存在\end{CJK}, \begin{CJK}{UTF8}{mj}则\end{CJK} $f(x, y)$ \begin{CJK}{UTF8}{mj}在\end{CJK} $P_{0}\left(x_{0}, y_{0}\right)$ \begin{CJK}{UTF8}{mj}连续\end{CJK}.

$3)$ \begin{CJK}{UTF8}{mj}如果级数\end{CJK} $\sum_{n=1}^{\infty} u_{n}(x)$ \begin{CJK}{UTF8}{mj}在区间\end{CJK} $[a, b]$ \begin{CJK}{UTF8}{mj}上一致收敛\end{CJK}, \begin{CJK}{UTF8}{mj}则对于任何\end{CJK} $x \in[a, b]$, \begin{CJK}{UTF8}{mj}级数\end{CJK} $\sum_{n=1}^{\infty} u_{n}(x)$ \begin{CJK}{UTF8}{mj}是绝对收敛的\end{CJK}.

(4) \begin{CJK}{UTF8}{mj}如果函数\end{CJK} $f(x)$ \begin{CJK}{UTF8}{mj}在\end{CJK} $[a, b]$ \begin{CJK}{UTF8}{mj}上存在原函数\end{CJK}, \begin{CJK}{UTF8}{mj}则\end{CJK} $f(x)$ \begin{CJK}{UTF8}{mj}在\end{CJK} $[a, b]$ \begin{CJK}{UTF8}{mj}上可积\end{CJK}.

(5) \begin{CJK}{UTF8}{mj}如果函数\end{CJK} $f(x)$ \begin{CJK}{UTF8}{mj}在有限开区间\end{CJK} $(a, b)$ \begin{CJK}{UTF8}{mj}上连续\end{CJK}, $f^{2}(x)$ \begin{CJK}{UTF8}{mj}在\end{CJK} $(a, b)$ \begin{CJK}{UTF8}{mj}上一致连续\end{CJK}, \begin{CJK}{UTF8}{mj}则\end{CJK} $f(x)$ \begin{CJK}{UTF8}{mj}在\end{CJK} $(a, b)$ \begin{CJK}{UTF8}{mj}上一致连续\end{CJK}.

(6) \begin{CJK}{UTF8}{mj}如果\end{CJK} $\int_{a}^{+\infty} f(x) \mathrm{d} x$ \begin{CJK}{UTF8}{mj}收敛\end{CJK}, \begin{CJK}{UTF8}{mj}且\end{CJK} $\varphi(x)$ \begin{CJK}{UTF8}{mj}在\end{CJK} $[a,+\infty)$ \begin{CJK}{UTF8}{mj}上有界\end{CJK}, \begin{CJK}{UTF8}{mj}则\end{CJK} $\int_{a}^{+\infty} f(x) \varphi(x) \mathrm{d} x$ \begin{CJK}{UTF8}{mj}收敛\end{CJK}.

\begin{enumerate}
  \setcounter{enumi}{2}
  \item \begin{CJK}{UTF8}{mj}求解下列各题\end{CJK} (\begin{CJK}{UTF8}{mj}每小题\end{CJK} 9 \begin{CJK}{UTF8}{mj}分\end{CJK}, \begin{CJK}{UTF8}{mj}共\end{CJK} 36 \begin{CJK}{UTF8}{mj}分\end{CJK})
\end{enumerate}
(1) \begin{CJK}{UTF8}{mj}求极限\end{CJK}
$$
\lim _{x \rightarrow 0}\left[\frac{e^{x}+e^{2 x}+\cdots+e^{n x}}{n}\right]^{\frac{1}{x}}
$$
(2) \begin{CJK}{UTF8}{mj}设函数\end{CJK} $f(x)$ \begin{CJK}{UTF8}{mj}在\end{CJK} $(-\infty,+\infty)$ \begin{CJK}{UTF8}{mj}上连续可导\end{CJK}, \begin{CJK}{UTF8}{mj}计算积分\end{CJK}
$$
\iint_{S} x^{3} \mathrm{~d} y \mathrm{~d} z+\left(\frac{1}{z} f\left(\frac{y}{z}\right)+y^{3}\right) \mathrm{d} z \mathrm{~d} x+\left(\frac{1}{y} f\left(\frac{y}{z}\right)+z^{3}\right) \mathrm{d} x \mathrm{~d} y
$$
\begin{CJK}{UTF8}{mj}其中\end{CJK} $S$ \begin{CJK}{UTF8}{mj}为雉面\end{CJK} $y^{2}+z^{2}=x^{2}(x>0)$ \begin{CJK}{UTF8}{mj}与球面\end{CJK} $x^{2}+y^{2}+z^{2}=1$ \begin{CJK}{UTF8}{mj}及\end{CJK} $x^{2}+y^{2}+z^{2}=4$ \begin{CJK}{UTF8}{mj}所围成立体的表面\end{CJK}, \begin{CJK}{UTF8}{mj}取外侧\end{CJK} \begin{CJK}{UTF8}{mj}为正向\end{CJK}.

(3) \begin{CJK}{UTF8}{mj}将函数\end{CJK} $f(x)=\arctan \frac{4+x^{2}}{4-x^{2}}$ \begin{CJK}{UTF8}{mj}展成\end{CJK} $x$ \begin{CJK}{UTF8}{mj}的幂级数\end{CJK}, \begin{CJK}{UTF8}{mj}并求级数\end{CJK} $\sum_{n=0}^{\infty} \frac{(-1)^{n}}{(2 n+1) 2^{2 n+1}}$ \begin{CJK}{UTF8}{mj}的和\end{CJK}.

(4) \begin{CJK}{UTF8}{mj}设函数\end{CJK} $f(x, y)$ \begin{CJK}{UTF8}{mj}在\end{CJK} $D=\left\{(x, y) \mid x^{2}+y^{2} \leq 1\right\}$ \begin{CJK}{UTF8}{mj}上具有二阶偏导数\end{CJK}, \begin{CJK}{UTF8}{mj}且满足\end{CJK} $\frac{\partial^{2} f}{\partial x^{2}}+\frac{\partial^{2} f}{\partial y^{2}}=e^{-\left(x^{2}+y^{2}\right)}$. \begin{CJK}{UTF8}{mj}求积分\end{CJK}
$$
\iint_{D}\left(x \frac{\partial f}{\partial x}+y \frac{\partial f}{\partial y}\right) \mathrm{d} x \mathrm{~d} y
$$

\begin{enumerate}
  \setcounter{enumi}{3}
  \item \begin{CJK}{UTF8}{mj}证明下列各题\end{CJK} (\begin{CJK}{UTF8}{mj}每小题\end{CJK} 13 \begin{CJK}{UTF8}{mj}分\end{CJK}, \begin{CJK}{UTF8}{mj}共\end{CJK} 78 \begin{CJK}{UTF8}{mj}分\end{CJK})
\end{enumerate}
(1) \begin{CJK}{UTF8}{mj}设函数\end{CJK} $f(x)$ \begin{CJK}{UTF8}{mj}在\end{CJK} $[0,1]$ \begin{CJK}{UTF8}{mj}上二阶可导\end{CJK}, \begin{CJK}{UTF8}{mj}且\end{CJK} $f^{\prime \prime}(x)>0, x \in[0,1]$. \begin{CJK}{UTF8}{mj}证明对任意的正整数\end{CJK} $n$ \begin{CJK}{UTF8}{mj}都有\end{CJK}
$$
\int_{0}^{1} f\left(x^{n}\right) \mathrm{d} x \geq f\left(\frac{1}{n+1}\right)
$$
(2) \begin{CJK}{UTF8}{mj}设数列\end{CJK} $a_{n}$ \begin{CJK}{UTF8}{mj}满足\end{CJK}
$$
a_{1}=1, a_{2}=4, a_{n+1}=\frac{a_{n-1}+a_{n}}{2}, n=2,3,4 \cdots
$$
\begin{CJK}{UTF8}{mj}证明\end{CJK}: \begin{CJK}{UTF8}{mj}数列\end{CJK} $\left\{a_{n}\right\}$ \begin{CJK}{UTF8}{mj}收敛\end{CJK}.

(3) \begin{CJK}{UTF8}{mj}设函数\end{CJK} $f(x)$ \begin{CJK}{UTF8}{mj}定义在\end{CJK} $[1,+\infty)$ \begin{CJK}{UTF8}{mj}上\end{CJK}, $f(1)=1$, \begin{CJK}{UTF8}{mj}且当\end{CJK} $x \geq 1$ \begin{CJK}{UTF8}{mj}时\end{CJK}, \begin{CJK}{UTF8}{mj}有\end{CJK}
$$
f^{\prime}(x)=\frac{1}{x^{2}+f^{2}(x)} \text {. }
$$
\begin{CJK}{UTF8}{mj}证明\end{CJK}: $\lim _{x \rightarrow+\infty} f(x)$ \begin{CJK}{UTF8}{mj}有限且其极限小于\end{CJK} $1+\frac{\pi}{4}$.

(4) \begin{CJK}{UTF8}{mj}设函数\end{CJK} $f(x)$ \begin{CJK}{UTF8}{mj}在\end{CJK} $[0,+\infty)$ \begin{CJK}{UTF8}{mj}上一致连续\end{CJK}, \begin{CJK}{UTF8}{mj}且\end{CJK} $\forall x \in[0,1]$, \begin{CJK}{UTF8}{mj}有\end{CJK} $\lim _{n \rightarrow \infty} f(x+n)=0$, \begin{CJK}{UTF8}{mj}证明\end{CJK}:
$$
\lim _{x \rightarrow+\infty} f(x)=0 .
$$
(6) \begin{CJK}{UTF8}{mj}设\end{CJK} $\left\{f_{n}(x)\right\}$ \begin{CJK}{UTF8}{mj}为定义在\end{CJK} $[a, b]$ \begin{CJK}{UTF8}{mj}上的函数列\end{CJK}, \begin{CJK}{UTF8}{mj}且存在常数\end{CJK} $L \geq 0$, \begin{CJK}{UTF8}{mj}使得\end{CJK} $\forall x, y \in[a, b]$ \begin{CJK}{UTF8}{mj}都有\end{CJK}
$$
\left|f_{n}(x)-f_{n}(y)\right| \leq L|x-y|, n \in \mathbb{N}_{+} .
$$
\begin{CJK}{UTF8}{mj}证明\end{CJK}: (i) \begin{CJK}{UTF8}{mj}若\end{CJK} $\left\{f_{n}(x)\right\}$ \begin{CJK}{UTF8}{mj}在\end{CJK} $[a, b]$ \begin{CJK}{UTF8}{mj}上收敛于\end{CJK} $f(x)$, \begin{CJK}{UTF8}{mj}则\end{CJK} $\left\{f_{n}(x)\right\}$ \begin{CJK}{UTF8}{mj}在\end{CJK} $[a, b]$ \begin{CJK}{UTF8}{mj}上一致收玫于\end{CJK} $f(x)$.

(ii) \begin{CJK}{UTF8}{mj}若对于任何\end{CJK} $x \in[a, b]$, \begin{CJK}{UTF8}{mj}数列\end{CJK} $\left\{f_{n}(x)\right\}$ \begin{CJK}{UTF8}{mj}都有界\end{CJK}, \begin{CJK}{UTF8}{mj}则\end{CJK} $\left\{f_{n}(x)\right\}$ \begin{CJK}{UTF8}{mj}存在\end{CJK} $[a, b]$ \begin{CJK}{UTF8}{mj}上一致收玫的子列\end{CJK}.

\section{7. 华东师范大学 2015 年研究生入学考试试题数学分析 
 李扬 
 微信公众号: sxkyliyang}
\begin{enumerate}
  \item \begin{CJK}{UTF8}{mj}判断下列命题是否正确\end{CJK}, \begin{CJK}{UTF8}{mj}若正确给出证明\end{CJK}, \begin{CJK}{UTF8}{mj}若错误举出反例\end{CJK}(\begin{CJK}{UTF8}{mj}每小题\end{CJK} 6 \begin{CJK}{UTF8}{mj}分\end{CJK}, \begin{CJK}{UTF8}{mj}共\end{CJK} 36 \begin{CJK}{UTF8}{mj}分\end{CJK})
\end{enumerate}
(1) \begin{CJK}{UTF8}{mj}如果\end{CJK} $\forall \varepsilon>0, \exists N \in \mathbb{N}_{+}$, \begin{CJK}{UTF8}{mj}当\end{CJK} $n>N$ \begin{CJK}{UTF8}{mj}时\end{CJK}, \begin{CJK}{UTF8}{mj}有\end{CJK} $\left|a_{n}-a_{N}\right|<\varepsilon$, \begin{CJK}{UTF8}{mj}则数列\end{CJK} $\left\{a_{n}\right\}$ \begin{CJK}{UTF8}{mj}收敛\end{CJK}.

(2) \begin{CJK}{UTF8}{mj}如果函数列\end{CJK} $\left\{f_{n}(x)\right\}$ \begin{CJK}{UTF8}{mj}在\end{CJK} $[a, b]$ \begin{CJK}{UTF8}{mj}上一致收敛于连续函数\end{CJK} $f(x)$, \begin{CJK}{UTF8}{mj}则\end{CJK} $\forall n \in \mathbb{N}_{+}$, \begin{CJK}{UTF8}{mj}均有\end{CJK} $\left\{f_{n}(x)\right\}$ \begin{CJK}{UTF8}{mj}在\end{CJK} $[a, b]$ \begin{CJK}{UTF8}{mj}上连续\end{CJK}.

(3) \begin{CJK}{UTF8}{mj}如果函数\end{CJK} $f(x)$ \begin{CJK}{UTF8}{mj}在\end{CJK} $x_{0}$ \begin{CJK}{UTF8}{mj}点连续\end{CJK}, \begin{CJK}{UTF8}{mj}且\end{CJK}
$$
\lim _{n \rightarrow \infty} \frac{f\left(x_{0}+\frac{1}{n}\right)-f\left(x_{0}\right)}{\frac{1}{n}}
$$
\begin{CJK}{UTF8}{mj}存在\end{CJK}, \begin{CJK}{UTF8}{mj}则\end{CJK} $f(x)$ \begin{CJK}{UTF8}{mj}在\end{CJK} $x_{0}$ \begin{CJK}{UTF8}{mj}点的右导数存在\end{CJK}.

(4) \begin{CJK}{UTF8}{mj}如果函数\end{CJK} $f(x), g(x)$ \begin{CJK}{UTF8}{mj}在\end{CJK} $[a, b]$ \begin{CJK}{UTF8}{mj}上连续\end{CJK}, \begin{CJK}{UTF8}{mj}则\end{CJK} $\exists \xi \in[a, b]$, \begin{CJK}{UTF8}{mj}使得\end{CJK}
$$
\int_{a}^{b} f(x) g(x) \mathrm{d} x=f(\xi) \int_{a}^{b} g(x) \mathrm{d} x .
$$
(5) \begin{CJK}{UTF8}{mj}如果函数\end{CJK} $f(x, y)$ \begin{CJK}{UTF8}{mj}的偏导数在点\end{CJK} $P_{0}\left(x_{0}, y_{0}\right)$ \begin{CJK}{UTF8}{mj}的某邻域内存在且有界\end{CJK}, \begin{CJK}{UTF8}{mj}则\end{CJK} $f(x, y)$ \begin{CJK}{UTF8}{mj}在点\end{CJK} $P_{0}\left(x_{0}, y_{0}\right)$ \begin{CJK}{UTF8}{mj}连续\end{CJK}.

(6) \begin{CJK}{UTF8}{mj}如果函数\end{CJK} $f(x)$ \begin{CJK}{UTF8}{mj}在\end{CJK} $[a,+\infty)$ \begin{CJK}{UTF8}{mj}上的非负连续\end{CJK}, \begin{CJK}{UTF8}{mj}且\end{CJK} $\int_{a}^{+\infty} f(x) \mathrm{d} x$ \begin{CJK}{UTF8}{mj}收敛\end{CJK}, \begin{CJK}{UTF8}{mj}则\end{CJK} $\lim _{x \rightarrow+\infty} f(x)=0$.

\begin{enumerate}
  \setcounter{enumi}{2}
  \item \begin{CJK}{UTF8}{mj}求解下列各题\end{CJK} (\begin{CJK}{UTF8}{mj}每小题\end{CJK} 9 \begin{CJK}{UTF8}{mj}分\end{CJK}, \begin{CJK}{UTF8}{mj}共\end{CJK} 36 \begin{CJK}{UTF8}{mj}分\end{CJK})
\end{enumerate}
(1) \begin{CJK}{UTF8}{mj}求极限\end{CJK}
$$
\lim _{n \rightarrow \infty} \frac{2^{n+1} n !}{n^{n}}
$$
(2) \begin{CJK}{UTF8}{mj}计算积分\end{CJK}
$$
\iint_{S}\left(x^{2}+y^{2}-z^{3}\right) \mathrm{d} s
$$
\begin{CJK}{UTF8}{mj}其中\end{CJK} $S$ \begin{CJK}{UTF8}{mj}为\end{CJK} $[-1,1] \times[-1,1] \times[-1,1]$ \begin{CJK}{UTF8}{mj}的表面\end{CJK}.

(3) \begin{CJK}{UTF8}{mj}计算积分\end{CJK}
$$
\int_{-1}^{1}\left|x-x^{2}\right| \mathrm{d} x
$$
(4) \begin{CJK}{UTF8}{mj}求\end{CJK}
$$
\sum_{n=1}^{\infty} \frac{(-1)^{n-1}}{n(n+2)} x^{n-1}
$$
\begin{CJK}{UTF8}{mj}的和函数\end{CJK}.

\begin{enumerate}
  \setcounter{enumi}{3}
  \item \begin{CJK}{UTF8}{mj}证明下列各题\end{CJK} (\begin{CJK}{UTF8}{mj}每小题\end{CJK} 13 \begin{CJK}{UTF8}{mj}分\end{CJK}, \begin{CJK}{UTF8}{mj}共\end{CJK} 78 \begin{CJK}{UTF8}{mj}分\end{CJK})
\end{enumerate}
(1) \begin{CJK}{UTF8}{mj}设函数\end{CJK} $f(x)$ \begin{CJK}{UTF8}{mj}在\end{CJK} $[a, b]$ \begin{CJK}{UTF8}{mj}内每一点的左右极限都存在\end{CJK}, \begin{CJK}{UTF8}{mj}证明\end{CJK}: $f(x)$ \begin{CJK}{UTF8}{mj}在\end{CJK} $[a, b]$ \begin{CJK}{UTF8}{mj}上有界\end{CJK}.

(2) \begin{CJK}{UTF8}{mj}设函数\end{CJK} $g(x)$ \begin{CJK}{UTF8}{mj}在\end{CJK} $[a,+\infty)$ \begin{CJK}{UTF8}{mj}上一致连续\end{CJK}, $f(x)$ \begin{CJK}{UTF8}{mj}在\end{CJK} $[a,+\infty)$ \begin{CJK}{UTF8}{mj}上连续\end{CJK}, \begin{CJK}{UTF8}{mj}且\end{CJK}
$$
\lim _{x \rightarrow+\infty}[f(x)-g(x)]=0,
$$
\begin{CJK}{UTF8}{mj}证明\end{CJK}: $f(x)$ \begin{CJK}{UTF8}{mj}在\end{CJK} $[a,+\infty)$ \begin{CJK}{UTF8}{mj}上一致连续\end{CJK}.

(3) \begin{CJK}{UTF8}{mj}设正项级数\end{CJK} $\sum_{n=1}^{\infty} a_{n}$ \begin{CJK}{UTF8}{mj}收敛\end{CJK}, \begin{CJK}{UTF8}{mj}且对任意的正整数\end{CJK} $n$ \begin{CJK}{UTF8}{mj}都有\end{CJK}
$$
a_{k} \leq a_{n}, \forall n<k \leq 2 n
$$
\begin{CJK}{UTF8}{mj}证明\end{CJK}: $\lim _{n \rightarrow \infty} n a_{n}=0$.

(4) \begin{CJK}{UTF8}{mj}设函数\end{CJK} $f(x)$ \begin{CJK}{UTF8}{mj}在\end{CJK} $(-\infty,+\infty)$ \begin{CJK}{UTF8}{mj}上三阶可导\end{CJK}, \begin{CJK}{UTF8}{mj}且\end{CJK} $f(x), f^{\prime \prime \prime}(x)$ \begin{CJK}{UTF8}{mj}在\end{CJK} $(-\infty,+\infty)$ \begin{CJK}{UTF8}{mj}上都有界\end{CJK}, \begin{CJK}{UTF8}{mj}证明\end{CJK}: $f^{\prime}(x), f^{\prime \prime}(x)$ \begin{CJK}{UTF8}{mj}在\end{CJK} $(-\infty,+\infty)$ \begin{CJK}{UTF8}{mj}上有界\end{CJK}.

(5) \begin{CJK}{UTF8}{mj}已知函数\end{CJK} $f(x)$ \begin{CJK}{UTF8}{mj}在\end{CJK} $[0, b]$ \begin{CJK}{UTF8}{mj}上可积\end{CJK}, \begin{CJK}{UTF8}{mj}且\end{CJK} $\lim _{x \rightarrow+\infty} f(x)=\alpha$, \begin{CJK}{UTF8}{mj}令\end{CJK}
$$
F(t)=t \int_{0}^{+\infty} e^{-t x} f(x) \mathrm{d} x
$$
\begin{CJK}{UTF8}{mj}证明\end{CJK}: (i) $f(x)$ \begin{CJK}{UTF8}{mj}在\end{CJK} $[0,1]$ \begin{CJK}{UTF8}{mj}内有且仅有两个零点\end{CJK}.

(ii) \begin{CJK}{UTF8}{mj}存在\end{CJK} $\xi \in(0,1)$, \begin{CJK}{UTF8}{mj}使得\end{CJK} $f^{\prime}(\xi)=\int_{0}^{\xi} f(t) \mathrm{d} t$.

\section{8. 华东师范大学 2016 年研究生入学考试试题数学分析 
 李扬 
 微信公众号: sxkyliyang}
\begin{enumerate}
  \item \begin{CJK}{UTF8}{mj}判断下列命题是否正确\end{CJK}, \begin{CJK}{UTF8}{mj}若正确给出证明\end{CJK}, \begin{CJK}{UTF8}{mj}若错误举出反例\end{CJK} (\begin{CJK}{UTF8}{mj}每小题\end{CJK} 6 \begin{CJK}{UTF8}{mj}分\end{CJK}, \begin{CJK}{UTF8}{mj}共\end{CJK} 36 \begin{CJK}{UTF8}{mj}分\end{CJK})
\end{enumerate}
(1) \begin{CJK}{UTF8}{mj}对数列\end{CJK} $\left\{a_{n}\right\}$ \begin{CJK}{UTF8}{mj}的任意两个子列\end{CJK} $\left\{a_{n_{k}}\right\}$ \begin{CJK}{UTF8}{mj}与\end{CJK} $\left\{a_{m_{k}}\right\}$ \begin{CJK}{UTF8}{mj}均有\end{CJK} $\lim _{k \rightarrow \infty}\left(a_{n_{k}}-a_{m_{k}}\right)=0$, \begin{CJK}{UTF8}{mj}则\end{CJK} $\left\{a_{n}\right\}$ \begin{CJK}{UTF8}{mj}收玫\end{CJK}.

(2) \begin{CJK}{UTF8}{mj}如果\end{CJK} $f(x, y)$ \begin{CJK}{UTF8}{mj}在\end{CJK} $\left(x_{0}, y_{0}\right)$ \begin{CJK}{UTF8}{mj}沿任意方向的方向导数都存在\end{CJK}, \begin{CJK}{UTF8}{mj}则偏导数\end{CJK} $f_{x}\left(x_{0}, y_{0}\right)$ \begin{CJK}{UTF8}{mj}与\end{CJK} $f_{y}\left(x_{0}, y_{0}\right)$ \begin{CJK}{UTF8}{mj}均存在\end{CJK}.

(3) \begin{CJK}{UTF8}{mj}设函数\end{CJK} $f(x)$ \begin{CJK}{UTF8}{mj}在\end{CJK} $[a,+\infty)$ \begin{CJK}{UTF8}{mj}上连续\end{CJK}, \begin{CJK}{UTF8}{mj}当\end{CJK} $x \rightarrow+\infty$ \begin{CJK}{UTF8}{mj}时\end{CJK}, $f(x)$ \begin{CJK}{UTF8}{mj}以\end{CJK} $y=c x+d$ \begin{CJK}{UTF8}{mj}为渐近线\end{CJK},\begin{CJK}{UTF8}{mj}则\end{CJK} $f(x)$ \begin{CJK}{UTF8}{mj}在\end{CJK} $[a,+\infty)$ \begin{CJK}{UTF8}{mj}上一\end{CJK} \begin{CJK}{UTF8}{mj}致连续\end{CJK}.

(4) \begin{CJK}{UTF8}{mj}如果级数\end{CJK} $\sum_{n=1}^{\infty} a_{n}$ \begin{CJK}{UTF8}{mj}收敛\end{CJK}, \begin{CJK}{UTF8}{mj}数列\end{CJK} $\left\{b_{n}\right\}$ \begin{CJK}{UTF8}{mj}满足\end{CJK} $\lim _{n \rightarrow \infty} b_{n}=1$, \begin{CJK}{UTF8}{mj}则\end{CJK} $\sum_{n=1}^{\infty} a_{n} b_{n}$ \begin{CJK}{UTF8}{mj}收敛\end{CJK}.

(5) \begin{CJK}{UTF8}{mj}如果函数\end{CJK} $f(x)$ \begin{CJK}{UTF8}{mj}在区间\end{CJK} $I$ \begin{CJK}{UTF8}{mj}上有原函数\end{CJK}, \begin{CJK}{UTF8}{mj}则\end{CJK} $f(x)$ \begin{CJK}{UTF8}{mj}在\end{CJK} $I$ \begin{CJK}{UTF8}{mj}上无第二类间断点\end{CJK}.

(6) \begin{CJK}{UTF8}{mj}如果函数\end{CJK} $f(x)$ \begin{CJK}{UTF8}{mj}在\end{CJK} $[a,+\infty)$ \begin{CJK}{UTF8}{mj}的任意子区间上可积\end{CJK}, \begin{CJK}{UTF8}{mj}且对任意的正数\end{CJK} $\varepsilon, B$, \begin{CJK}{UTF8}{mj}都存在\end{CJK} $A(\geq a)$, \begin{CJK}{UTF8}{mj}使得\end{CJK} $\left|\int_{A}^{A+B} f(x) \mathrm{d} x\right|<\varepsilon$, \begin{CJK}{UTF8}{mj}则\end{CJK} $\int_{a}^{+\infty} f(x) \mathrm{d} x$ \begin{CJK}{UTF8}{mj}收敛\end{CJK}.

\begin{enumerate}
  \setcounter{enumi}{2}
  \item \begin{CJK}{UTF8}{mj}求解下列各题\end{CJK} (\begin{CJK}{UTF8}{mj}每小题\end{CJK} 8 \begin{CJK}{UTF8}{mj}分\end{CJK}, \begin{CJK}{UTF8}{mj}共\end{CJK} 40 \begin{CJK}{UTF8}{mj}分\end{CJK})
\end{enumerate}
(7) \begin{CJK}{UTF8}{mj}求极限\end{CJK}
$$
\lim _{x \rightarrow+\infty}\left(\sin \frac{1}{x}+\cos \frac{1}{x}\right)^{x}
$$
(8) \begin{CJK}{UTF8}{mj}求定积分\end{CJK}
$$
\int_{0}^{2 \pi} \sqrt{1+\cos x} d x
$$
(9) \begin{CJK}{UTF8}{mj}求曲面积分\end{CJK}
$$
\iint_{S} 4 x z \mathrm{~d} y \mathrm{~d} z-2 y z \mathrm{~d} y \mathrm{~d} x+\left(1-z^{2}\right) \mathrm{d} x \mathrm{~d} y .
$$
\begin{CJK}{UTF8}{mj}其中\end{CJK} $S$ \begin{CJK}{UTF8}{mj}为\end{CJK} $z=e^{y}, y \in[0, a]$ \begin{CJK}{UTF8}{mj}绕\end{CJK} $z$ \begin{CJK}{UTF8}{mj}轴旋转一周形成的曲面\end{CJK}, \begin{CJK}{UTF8}{mj}方向取下侧\end{CJK}.

(10) \begin{CJK}{UTF8}{mj}求级数\end{CJK}
$$
\sum_{n=0}^{\infty}(-1)^{n} \frac{1}{3 n+1}
$$
\begin{CJK}{UTF8}{mj}的和\end{CJK}.

(11) \begin{CJK}{UTF8}{mj}设\end{CJK} $f(r)$ \begin{CJK}{UTF8}{mj}为\end{CJK} $(0,+\infty)$ \begin{CJK}{UTF8}{mj}上的二阶连续可导函数\end{CJK}, $f(1)=f^{\prime}(1)=1, u(x, y)=f\left(\sqrt{x^{2}+y^{2}}\right)$ \begin{CJK}{UTF8}{mj}满足\end{CJK} Laplace \begin{CJK}{UTF8}{mj}条\end{CJK} \begin{CJK}{UTF8}{mj}件\end{CJK}
$$
\frac{\partial^{2} u}{\partial x^{2}}+\frac{\partial^{2} u}{\partial y^{2}}=0
$$
\begin{CJK}{UTF8}{mj}试确定\end{CJK} $f(r)$ \begin{CJK}{UTF8}{mj}所满足的微分方程\end{CJK}, \begin{CJK}{UTF8}{mj}并求出\end{CJK} $f(r)$ \begin{CJK}{UTF8}{mj}的解析式\end{CJK}.

\begin{enumerate}
  \setcounter{enumi}{3}
  \item \begin{CJK}{UTF8}{mj}证明下列各题\end{CJK}(\begin{CJK}{UTF8}{mj}第\end{CJK} $(12)-(15)$ \begin{CJK}{UTF8}{mj}题\end{CJK}, \begin{CJK}{UTF8}{mj}每小题\end{CJK} 13 \begin{CJK}{UTF8}{mj}分\end{CJK}, \begin{CJK}{UTF8}{mj}第\end{CJK} $(16)-(17)$ \begin{CJK}{UTF8}{mj}题\end{CJK}, \begin{CJK}{UTF8}{mj}每小题\end{CJK} 11 \begin{CJK}{UTF8}{mj}分\end{CJK}, \begin{CJK}{UTF8}{mj}共\end{CJK} 74 \begin{CJK}{UTF8}{mj}分\end{CJK})
\end{enumerate}
(12) \begin{CJK}{UTF8}{mj}设函数\end{CJK} $f(x)$ \begin{CJK}{UTF8}{mj}在\end{CJK} $[a, b]$ \begin{CJK}{UTF8}{mj}上连续\end{CJK}, $\forall x \in[a, b], \exists y \in[a, b]$, \begin{CJK}{UTF8}{mj}使得\end{CJK} $|f(y)| \leq \frac{1}{2}|f(x)|$. \begin{CJK}{UTF8}{mj}证明\end{CJK}: \begin{CJK}{UTF8}{mj}存在\end{CJK} $\xi \in[a, b]$, \begin{CJK}{UTF8}{mj}使得\end{CJK} $f(x)=0$.

(13) \begin{CJK}{UTF8}{mj}设函数\end{CJK} $f(x)$ \begin{CJK}{UTF8}{mj}在\end{CJK} $[a,+\infty)$ \begin{CJK}{UTF8}{mj}上有定义\end{CJK}, \begin{CJK}{UTF8}{mj}当\end{CJK} $\lim _{x \rightarrow+\infty} f(x)$ \begin{CJK}{UTF8}{mj}存在\end{CJK}, \begin{CJK}{UTF8}{mj}又\end{CJK} $f^{\prime}(x)$ \begin{CJK}{UTF8}{mj}在\end{CJK} $(a,+\infty)$ \begin{CJK}{UTF8}{mj}上存在且一致连续\end{CJK}, \begin{CJK}{UTF8}{mj}试证明\end{CJK}: $\lim _{x \rightarrow+\infty} f^{\prime}(x)=0 .$

\includegraphics[max width=\textwidth]{2022_04_18_c36afefb0fdf29df1be6g-188}
$$
f(x) \geq \frac{2}{b-a} \int_{a}^{b} f(t) \mathrm{d} t, \quad f x \in[a, b] .
$$
(15) \begin{CJK}{UTF8}{mj}设函数\end{CJK} $f(x)$ \begin{CJK}{UTF8}{mj}在\end{CJK} $\Omega(t)=\left\{(x, y, z) \mid x^{2}+y^{2}+z^{2} \leq t^{2}\right\}$ \begin{CJK}{UTF8}{mj}上连续\end{CJK}, $S=\left\{(x, y, z) \mid x^{2}+y^{2}+z^{2}=t^{2}\right\}$, \begin{CJK}{UTF8}{mj}证明\end{CJK}:
$$
\frac{\mathrm{d}}{\mathrm{d} t} \iiint_{\Omega(t)} f(x, y, z) \mathrm{d} V=\iint_{S} f(x, y, z) \mathrm{d} S .
$$
(16) \begin{CJK}{UTF8}{mj}设函数\end{CJK} $f(x)$ \begin{CJK}{UTF8}{mj}在\end{CJK} $[0,1]$ \begin{CJK}{UTF8}{mj}上连续\end{CJK}, \begin{CJK}{UTF8}{mj}试证明\end{CJK}:
$$
\lim _{n \rightarrow \infty} \int_{0}^{1} \frac{n}{1+n^{2} x^{2}} f(x) \mathrm{d} x=\frac{\pi}{2} f(0) .
$$
(17) \begin{CJK}{UTF8}{mj}设\end{CJK} $\left\{a_{n}\right\}$ \begin{CJK}{UTF8}{mj}是单调递减的正数列\end{CJK}, \begin{CJK}{UTF8}{mj}证明\end{CJK}:\begin{CJK}{UTF8}{mj}函数项级数\end{CJK} $\sum_{n=0}^{\infty} a_{n} \sin n x$ \begin{CJK}{UTF8}{mj}在\end{CJK} $[0, \pi]$ \begin{CJK}{UTF8}{mj}上一致收敛的充要条件为\end{CJK} $\lim _{n \rightarrow \infty} n a_{n}=0 .$

\section{9. 华东师范大学 2017 年研究生入学考试试题数学分析 
 李扬 
 微信公众号: sxkyliyang}
\begin{enumerate}
  \item \begin{CJK}{UTF8}{mj}判断下列命题是否正确\end{CJK}, \begin{CJK}{UTF8}{mj}若正确给出证明\end{CJK}, \begin{CJK}{UTF8}{mj}若错误举出反例\end{CJK}(\begin{CJK}{UTF8}{mj}每小题\end{CJK} 6 \begin{CJK}{UTF8}{mj}分\end{CJK}, \begin{CJK}{UTF8}{mj}共\end{CJK} 36 \begin{CJK}{UTF8}{mj}分\end{CJK})
\end{enumerate}
(1) \begin{CJK}{UTF8}{mj}如果函数\end{CJK} $f(x)$ \begin{CJK}{UTF8}{mj}在\end{CJK} $[a, b]$ \begin{CJK}{UTF8}{mj}上可积\end{CJK}, \begin{CJK}{UTF8}{mj}则\end{CJK} $f(x)$ \begin{CJK}{UTF8}{mj}在\end{CJK} $[a, b]$ \begin{CJK}{UTF8}{mj}上至多有有限个不连续点\end{CJK}.

(2) \begin{CJK}{UTF8}{mj}如果数列\end{CJK} $\left\{a_{n}\right\}$ \begin{CJK}{UTF8}{mj}单调\end{CJK}, \begin{CJK}{UTF8}{mj}且\end{CJK} $\lim _{n \rightarrow \infty}\left(a_{n+1}-a_{n}\right)=0$, \begin{CJK}{UTF8}{mj}则\end{CJK} $\left\{a_{n}\right\}$ \begin{CJK}{UTF8}{mj}收敛\end{CJK}.

(3) \begin{CJK}{UTF8}{mj}如果函数\end{CJK} $f(x, y)$ \begin{CJK}{UTF8}{mj}在\end{CJK} $\left(x_{0}, y_{0}\right)$ \begin{CJK}{UTF8}{mj}处连续\end{CJK}, \begin{CJK}{UTF8}{mj}偏导数\end{CJK} $f_{x}\left(x_{0}, y_{0}\right)$ \begin{CJK}{UTF8}{mj}与\end{CJK} $f_{y}\left(x_{0}, y_{0}\right)$ \begin{CJK}{UTF8}{mj}均存在\end{CJK}, \begin{CJK}{UTF8}{mj}则\end{CJK} $f(x, y)$ \begin{CJK}{UTF8}{mj}在\end{CJK} $\left(x_{0}, y_{0}\right)$ \begin{CJK}{UTF8}{mj}处可微\end{CJK}.

(4) \begin{CJK}{UTF8}{mj}设函数\end{CJK} $f(x)$ \begin{CJK}{UTF8}{mj}在\end{CJK} $(0,1)$ \begin{CJK}{UTF8}{mj}上一致连续\end{CJK}, \begin{CJK}{UTF8}{mj}则\end{CJK} $f(x)$ \begin{CJK}{UTF8}{mj}在\end{CJK} $(0,1)$ \begin{CJK}{UTF8}{mj}上有界\end{CJK}.

$(5)$ \begin{CJK}{UTF8}{mj}如果\end{CJK} $u_{n}(x)(n=1,2, \cdots)$ \begin{CJK}{UTF8}{mj}在\end{CJK} $[a, b]$ \begin{CJK}{UTF8}{mj}上连续\end{CJK}, \begin{CJK}{UTF8}{mj}且函数项级数\end{CJK} $\sum_{n=1}^{\infty} u_{n}(x)$ \begin{CJK}{UTF8}{mj}在\end{CJK} $(a, b)$ \begin{CJK}{UTF8}{mj}上一致收敛\end{CJK}, \begin{CJK}{UTF8}{mj}则\end{CJK} $\sum_{n=1}^{\infty} u_{n}(x)$ \begin{CJK}{UTF8}{mj}在\end{CJK} $[a, b]$ \begin{CJK}{UTF8}{mj}上一致收敛\end{CJK}.

(6) \begin{CJK}{UTF8}{mj}如果含参变量积分\end{CJK} $\int_{a}^{+\infty} f(x, y) \mathrm{d} x$ \begin{CJK}{UTF8}{mj}在\end{CJK} $[a, b]$ \begin{CJK}{UTF8}{mj}一致收敛\end{CJK}, \begin{CJK}{UTF8}{mj}则在\end{CJK} $[a, b]$ \begin{CJK}{UTF8}{mj}上处处绝对收敛\end{CJK}.

\begin{enumerate}
  \setcounter{enumi}{2}
  \item \begin{CJK}{UTF8}{mj}求解下列各题\end{CJK}(\begin{CJK}{UTF8}{mj}每小题\end{CJK} 8 \begin{CJK}{UTF8}{mj}分\end{CJK}, \begin{CJK}{UTF8}{mj}共\end{CJK} 40 \begin{CJK}{UTF8}{mj}分\end{CJK})
\end{enumerate}
(7) \begin{CJK}{UTF8}{mj}求下面积分的值\end{CJK}
$$
\iint_{D} \sqrt{y^{2}-x y} \mathrm{~d} x \mathrm{~d} y .
$$
\begin{CJK}{UTF8}{mj}其中\end{CJK} $D$ \begin{CJK}{UTF8}{mj}是由\end{CJK} $y=x, y=1, x=0$ \begin{CJK}{UTF8}{mj}围成的封闭图形\end{CJK}.

(8) \begin{CJK}{UTF8}{mj}求极限\end{CJK}
$$
\lim _{x \rightarrow 0}\left[\frac{e^{x}+e^{2 x}+\cdots+e^{n x}}{n}\right]^{\frac{1}{x}} .
$$
(9) \begin{CJK}{UTF8}{mj}设函数\end{CJK} $y=y(x), z=z(x)$ \begin{CJK}{UTF8}{mj}由方程\end{CJK} $z=x f(x+y)$ \begin{CJK}{UTF8}{mj}和\end{CJK} $F(x, y, z)=0$ \begin{CJK}{UTF8}{mj}所确定\end{CJK}, $f$ \begin{CJK}{UTF8}{mj}一阶连续可导\end{CJK}, $F$ \begin{CJK}{UTF8}{mj}具有一\end{CJK} \begin{CJK}{UTF8}{mj}阶连续偏导数\end{CJK}, \begin{CJK}{UTF8}{mj}求\end{CJK} $\frac{\mathrm{d} z}{\mathrm{~d} x}$.

(10) \begin{CJK}{UTF8}{mj}求曲面积分\end{CJK}
$$
\iint_{S}\left(x^{2}-y z\right) \mathrm{d} y \mathrm{~d} z+\left(y^{2}-z x\right) \mathrm{d} z \mathrm{~d} x+\left(z^{2}-x y\right) \mathrm{d} x \mathrm{~d} y,
$$
\begin{CJK}{UTF8}{mj}其中\end{CJK} $S$ \begin{CJK}{UTF8}{mj}为曲面\end{CJK} $z=1-\sqrt{x^{2}+y^{2}}$ \begin{CJK}{UTF8}{mj}被平面\end{CJK} $z=0, z=1$ \begin{CJK}{UTF8}{mj}所截的部分\end{CJK}, \begin{CJK}{UTF8}{mj}方向取下侧\end{CJK}.

(11) \begin{CJK}{UTF8}{mj}求级数\end{CJK}
$$
\sum_{n=1}^{\infty} \frac{1}{3^{n}+(-2)^{n}} \cdot \frac{x^{n}}{n}
$$
\begin{CJK}{UTF8}{mj}的收敛域\end{CJK}.

\begin{enumerate}
  \setcounter{enumi}{3}
  \item \begin{CJK}{UTF8}{mj}证明下列各题\end{CJK}(\begin{CJK}{UTF8}{mj}第\end{CJK} $(12)-(15)$ \begin{CJK}{UTF8}{mj}题\end{CJK}, \begin{CJK}{UTF8}{mj}每小题\end{CJK} 13 \begin{CJK}{UTF8}{mj}分\end{CJK}, \begin{CJK}{UTF8}{mj}第\end{CJK} $(16)-(17)$ \begin{CJK}{UTF8}{mj}题\end{CJK}, \begin{CJK}{UTF8}{mj}每小题\end{CJK} 11 \begin{CJK}{UTF8}{mj}分\end{CJK}, \begin{CJK}{UTF8}{mj}共\end{CJK} 74 \begin{CJK}{UTF8}{mj}分\end{CJK})
\end{enumerate}
(12) \begin{CJK}{UTF8}{mj}设函数\end{CJK} $f(x)$ \begin{CJK}{UTF8}{mj}在\end{CJK} $[a,+\infty)$ \begin{CJK}{UTF8}{mj}上一致连续\end{CJK}, $\int_{a}^{+\infty} f(x) \mathrm{d} x$ \begin{CJK}{UTF8}{mj}收敛\end{CJK}, \begin{CJK}{UTF8}{mj}证明\end{CJK}: $\lim _{x \rightarrow+\infty} f(x)=0$.

(13) \begin{CJK}{UTF8}{mj}设函数\end{CJK} $f(x)$ \begin{CJK}{UTF8}{mj}在\end{CJK} $[0,+\infty)$ \begin{CJK}{UTF8}{mj}上一致连续\end{CJK}, \begin{CJK}{UTF8}{mj}对任意的\end{CJK} $x>0$, \begin{CJK}{UTF8}{mj}均有\end{CJK} $\lim _{n \rightarrow \infty} f(x+n)=0$, \begin{CJK}{UTF8}{mj}证明\end{CJK}: \begin{CJK}{UTF8}{mj}函数列\end{CJK} $f(x+n)$ \begin{CJK}{UTF8}{mj}在\end{CJK} $[0,1]$ \begin{CJK}{UTF8}{mj}上一致收敛于\end{CJK} $0 .$

(14) \begin{CJK}{UTF8}{mj}设函数\end{CJK} $f(x)$ \begin{CJK}{UTF8}{mj}在\end{CJK} $[a, b]$ \begin{CJK}{UTF8}{mj}上二阶连续可导\end{CJK}, \begin{CJK}{UTF8}{mj}且\end{CJK} $f\left(\frac{a+b}{2}\right)=0, M=\max _{x \in[a, b]}\left\{\left|f^{\prime \prime}(x)\right|\right\}$. \begin{CJK}{UTF8}{mj}证明\end{CJK}:
$$
\left|\int_{a}^{b} f(x) \mathrm{d} x\right| \leq \frac{M}{24}(b-a)^{3}
$$
(15) (i) \begin{CJK}{UTF8}{mj}证明\end{CJK}: \begin{CJK}{UTF8}{mj}级数\end{CJK} $\sum_{n=1}^{\infty} n e^{-n x}$ \begin{CJK}{UTF8}{mj}在\end{CJK} $(0,+\infty)$ \begin{CJK}{UTF8}{mj}上处处收敛\end{CJK}, \begin{CJK}{UTF8}{mj}但非一致收敛\end{CJK};

(ii) \begin{CJK}{UTF8}{mj}证明\end{CJK}: (i) \begin{CJK}{UTF8}{mj}中级数的和函数为\end{CJK} $S(x)=\frac{e^{x}}{\left(e^{x}-1\right)^{2}}$.

(16) \begin{CJK}{UTF8}{mj}平面区域\end{CJK} $D$ \begin{CJK}{UTF8}{mj}由光滑封闭曲线\end{CJK} $L$ \begin{CJK}{UTF8}{mj}围成\end{CJK}, $\vec{n}$ \begin{CJK}{UTF8}{mj}为\end{CJK} $L$ \begin{CJK}{UTF8}{mj}上任意一点处的外法向向量\end{CJK}, $u(x, y) \in C^{2}(D)$. \begin{CJK}{UTF8}{mj}证明\end{CJK}:
$$
\iint_{D}\left(u_{x}^{2}+u_{y}^{2}\right) \mathrm{d} x \mathrm{~d} y=\oint_{L} u \frac{\partial u}{\partial \vec{n}} \mathrm{~d} s-\iint_{D} u\left(u_{x x}+u_{y y}\right) \mathrm{d} x \mathrm{~d} y .
$$
\begin{CJK}{UTF8}{mj}其中\end{CJK} $\frac{\partial u}{\partial \vec{n}}$ \begin{CJK}{UTF8}{mj}表示\end{CJK} $u$ \begin{CJK}{UTF8}{mj}沿\end{CJK} $\vec{n}$ \begin{CJK}{UTF8}{mj}的方向导数\end{CJK}.

$(17)$ \begin{CJK}{UTF8}{mj}设函数\end{CJK} $f(x)$ \begin{CJK}{UTF8}{mj}在\end{CJK} $(0,+\infty)$ \begin{CJK}{UTF8}{mj}上可导\end{CJK}, \begin{CJK}{UTF8}{mj}且\end{CJK} $\lim _{x \rightarrow+\infty}[f(x+1)-f(x)]=0$, \begin{CJK}{UTF8}{mj}证明\end{CJK}:

(i) $\lim _{x \rightarrow+\infty} \frac{f(x)}{x}=0$.

(ii) \begin{CJK}{UTF8}{mj}在\end{CJK} $(0,+\infty)$ \begin{CJK}{UTF8}{mj}上存在单调递增趋于\end{CJK} $+\infty$ \begin{CJK}{UTF8}{mj}的数列\end{CJK} $\left\{x_{n}\right\}$, \begin{CJK}{UTF8}{mj}使得\end{CJK} $\lim _{n \rightarrow+\infty} f^{\prime}\left(x_{n}\right)=0$.

\section{0. 华东师范大学 2018 年研究生入学考试试题数学分析 
 李扬 
 微信公众号: sxkyliyang}
\begin{enumerate}
  \item \begin{CJK}{UTF8}{mj}判断下列命题是否正确\end{CJK}, \begin{CJK}{UTF8}{mj}若正确给出证明\end{CJK}, \begin{CJK}{UTF8}{mj}若错误举出反例\end{CJK} (\begin{CJK}{UTF8}{mj}每小题\end{CJK} 5 \begin{CJK}{UTF8}{mj}分\end{CJK}, \begin{CJK}{UTF8}{mj}共\end{CJK} 30 \begin{CJK}{UTF8}{mj}分\end{CJK})
\end{enumerate}
(1) \begin{CJK}{UTF8}{mj}若对任意的\end{CJK} $N$, \begin{CJK}{UTF8}{mj}总存在\end{CJK} $\varepsilon>0$, \begin{CJK}{UTF8}{mj}当\end{CJK} $n>N$ \begin{CJK}{UTF8}{mj}时\end{CJK}, \begin{CJK}{UTF8}{mj}有\end{CJK} $\left|a_{n}-a\right|<\varepsilon$ \begin{CJK}{UTF8}{mj}成立\end{CJK}, \begin{CJK}{UTF8}{mj}则\end{CJK} $\lim _{n \rightarrow \infty} a_{n}=a$.

(2) \begin{CJK}{UTF8}{mj}若函数\end{CJK} $f(x)$ \begin{CJK}{UTF8}{mj}在区间\end{CJK} $I$ \begin{CJK}{UTF8}{mj}上单调且存在原函数\end{CJK}, \begin{CJK}{UTF8}{mj}则\end{CJK} $f(x)$ \begin{CJK}{UTF8}{mj}在\end{CJK} $I$ \begin{CJK}{UTF8}{mj}上连续\end{CJK}.

(3) \begin{CJK}{UTF8}{mj}若函数\end{CJK} $f(x, y)$ \begin{CJK}{UTF8}{mj}在\end{CJK} $\left(x_{0}, y_{0}\right)$ \begin{CJK}{UTF8}{mj}处沿任意方向的方向导数均存在\end{CJK}, \begin{CJK}{UTF8}{mj}则\end{CJK} $f(x, y)$ \begin{CJK}{UTF8}{mj}在该点处连续\end{CJK}.

(4) \begin{CJK}{UTF8}{mj}若函数\end{CJK} $f(x)$ \begin{CJK}{UTF8}{mj}和\end{CJK} $g(x)$ \begin{CJK}{UTF8}{mj}在\end{CJK} $(0,1)$ \begin{CJK}{UTF8}{mj}内均一致连续\end{CJK}, \begin{CJK}{UTF8}{mj}则\end{CJK} $f(x) g(x)$ \begin{CJK}{UTF8}{mj}在\end{CJK} $(0,1)$ \begin{CJK}{UTF8}{mj}内也一致连续\end{CJK}.

(5) \begin{CJK}{UTF8}{mj}若\end{CJK} $a_{n} \leq b_{n} \leq c_{n}(n=1,2, \cdots)$ \begin{CJK}{UTF8}{mj}且级数\end{CJK} $\sum_{n=1}^{\infty} a_{n}$ \begin{CJK}{UTF8}{mj}与\end{CJK} $\sum_{n=1}^{\infty} c_{n}$ \begin{CJK}{UTF8}{mj}都收玫\end{CJK}, \begin{CJK}{UTF8}{mj}则级数\end{CJK} $\sum_{n=1}^{\infty} b_{n}$ \begin{CJK}{UTF8}{mj}也收敛\end{CJK}.

(6) \begin{CJK}{UTF8}{mj}若广义积分\end{CJK} $\int_{a}^{+\infty} f(x) \mathrm{d} x$ \begin{CJK}{UTF8}{mj}与\end{CJK} $\int_{a}^{+\infty} g(x) \mathrm{d} x$ \begin{CJK}{UTF8}{mj}皆绝对收敛\end{CJK}, \begin{CJK}{UTF8}{mj}则\end{CJK} $\int_{a}^{+\infty} f(x) g(x) \mathrm{d} x$ \begin{CJK}{UTF8}{mj}收敛\end{CJK}.

\begin{enumerate}
  \setcounter{enumi}{2}
  \item \begin{CJK}{UTF8}{mj}求解下列各题\end{CJK} (\begin{CJK}{UTF8}{mj}每小题\end{CJK} 9 \begin{CJK}{UTF8}{mj}分\end{CJK}, \begin{CJK}{UTF8}{mj}共\end{CJK} 45 \begin{CJK}{UTF8}{mj}分\end{CJK})
\end{enumerate}
(7) \begin{CJK}{UTF8}{mj}计算极限\end{CJK}
$$
\lim _{n \rightarrow \infty} \frac{1}{n}\left(\sqrt{1+\cos \frac{\pi}{n}}+\sqrt{1+\cos \frac{2 \pi}{n}}+\cdots+\sqrt{1+\cos \frac{n \pi}{n}}\right) .
$$
(8) \begin{CJK}{UTF8}{mj}设函数\end{CJK} $z=f\left(x y, \frac{x}{y}\right)+g\left(\frac{y}{x}\right)$, \begin{CJK}{UTF8}{mj}其中\end{CJK} $f$ \begin{CJK}{UTF8}{mj}具有二阶连续偏导数\end{CJK}, $g$ \begin{CJK}{UTF8}{mj}二阶连续可导\end{CJK}, \begin{CJK}{UTF8}{mj}求\end{CJK} $\frac{\partial^{2} z}{\partial x \partial y}$.

(9) \begin{CJK}{UTF8}{mj}设函数\end{CJK}
$$
f(x)=\lim _{a \rightarrow x}\left(\frac{\sin a}{\sin x}\right)^{\frac{x}{\sin a-\sin x}}
$$
\begin{CJK}{UTF8}{mj}求\end{CJK} $f(x)$ \begin{CJK}{UTF8}{mj}的所有间断点并判断其类型\end{CJK}.

(10) \begin{CJK}{UTF8}{mj}求级数\end{CJK}
$$
\sum_{n=1}^{\infty} \sum_{k=1}^{n} \frac{k}{2^{n-1}}
$$
\begin{CJK}{UTF8}{mj}的和\end{CJK}.

(11) \begin{CJK}{UTF8}{mj}计算曲面积分\end{CJK}
$$
\iint_{S} \frac{a x \mathrm{~d} y \mathrm{~d} z+(z+a)^{2} \mathrm{~d} x \mathrm{~d} y}{\sqrt{x^{2}+y^{2}+z^{2}}} .
$$
\begin{CJK}{UTF8}{mj}其中\end{CJK} $S$ \begin{CJK}{UTF8}{mj}为下半球面\end{CJK} $z=-\sqrt{a^{2}-x^{2}-y^{2}}$ \begin{CJK}{UTF8}{mj}的上侧\end{CJK}, $a>0$ \begin{CJK}{UTF8}{mj}为常数\end{CJK}.

\begin{enumerate}
  \setcounter{enumi}{3}
  \item \begin{CJK}{UTF8}{mj}证明下列各题\end{CJK}(\begin{CJK}{UTF8}{mj}第\end{CJK} $(12)-(16)$ \begin{CJK}{UTF8}{mj}题\end{CJK}, \begin{CJK}{UTF8}{mj}每小题\end{CJK} 13 \begin{CJK}{UTF8}{mj}分\end{CJK}, \begin{CJK}{UTF8}{mj}第\end{CJK} (17) \begin{CJK}{UTF8}{mj}题\end{CJK} 10 \begin{CJK}{UTF8}{mj}分\end{CJK}, \begin{CJK}{UTF8}{mj}共\end{CJK} 75 \begin{CJK}{UTF8}{mj}分\end{CJK})
\end{enumerate}
(12) \begin{CJK}{UTF8}{mj}已知函数列\end{CJK} $\left\{f_{n}(x)\right\}$ \begin{CJK}{UTF8}{mj}在\end{CJK} $[a, b]$ \begin{CJK}{UTF8}{mj}上收敛于连续函数\end{CJK} $f(x)$. \begin{CJK}{UTF8}{mj}证明\end{CJK}: $\left\{f_{n}(x)\right\}$ \begin{CJK}{UTF8}{mj}在\end{CJK} $[a, b]$ \begin{CJK}{UTF8}{mj}上一致收敛于\end{CJK} $f(x)$ \begin{CJK}{UTF8}{mj}的充\end{CJK} \begin{CJK}{UTF8}{mj}要条件为\end{CJK}: \begin{CJK}{UTF8}{mj}对任意数列\end{CJK} $\left\{x_{n}\right\} \subset[a, b]$, \begin{CJK}{UTF8}{mj}若\end{CJK} $\lim _{n \rightarrow+\infty} x_{n}=x_{0}$, \begin{CJK}{UTF8}{mj}则有\end{CJK}
$$
\lim _{n \rightarrow+\infty} f_{n}\left(x_{n}\right)=f\left(x_{0}\right)
$$
(13) \begin{CJK}{UTF8}{mj}设函数\end{CJK} $f(x)$ \begin{CJK}{UTF8}{mj}在\end{CJK} $[0,1]$ \begin{CJK}{UTF8}{mj}上可导\end{CJK}, \begin{CJK}{UTF8}{mj}且\end{CJK} $f(0)=0,0 \leq f^{\prime}(x) \leq 1$, \begin{CJK}{UTF8}{mj}证明不等式\end{CJK}
$$
\left(\int_{0}^{1} f(x) \mathrm{d} x\right)^{2} \geq \int_{0}^{1} f^{3}(x) \mathrm{d} x
$$
(14) \begin{CJK}{UTF8}{mj}已知函数\end{CJK} $f(x)$ \begin{CJK}{UTF8}{mj}在\end{CJK} $[0,+\infty)$ \begin{CJK}{UTF8}{mj}上连续\end{CJK}, \begin{CJK}{UTF8}{mj}且\end{CJK} $\lim _{x \rightarrow+\infty} f(x)=A \in \mathbb{R}$, \begin{CJK}{UTF8}{mj}证明\end{CJK}:

(i) $f(x)$ \begin{CJK}{UTF8}{mj}在\end{CJK} $[0,+\infty)$ \begin{CJK}{UTF8}{mj}上有界且一致连续\end{CJK};

(ii) $\lim _{x \rightarrow+\infty} \frac{1}{x} \int_{0}^{x} f(t) \mathrm{d} t=A$.

(15) \begin{CJK}{UTF8}{mj}设函数\end{CJK}
$$
F(x)=\int_{0}^{+\infty} \frac{\left(1-e^{-x t}\right) \cos t}{t} \mathrm{~d} t
$$
\begin{CJK}{UTF8}{mj}试证\end{CJK}: $F(x)$ \begin{CJK}{UTF8}{mj}在\end{CJK} $[0,+\infty)$ \begin{CJK}{UTF8}{mj}上连续\end{CJK}, \begin{CJK}{UTF8}{mj}在\end{CJK} $(0,+\infty)$ \begin{CJK}{UTF8}{mj}内可导\end{CJK}.

(16) \begin{CJK}{UTF8}{mj}设级数\end{CJK} $\sum_{n=1}^{\infty} u_{n}(x)$ \begin{CJK}{UTF8}{mj}在\end{CJK} $[a, b]$ \begin{CJK}{UTF8}{mj}上处处收敛\end{CJK}, \begin{CJK}{UTF8}{mj}每一个\end{CJK} $u_{n}(x)$ \begin{CJK}{UTF8}{mj}均在\end{CJK} $[a, b]$ \begin{CJK}{UTF8}{mj}上连续可导\end{CJK}, \begin{CJK}{UTF8}{mj}且存在常数\end{CJK} $M>0$, \begin{CJK}{UTF8}{mj}使得\end{CJK} $\left|\sum_{k=1}^{n} u_{k}^{\prime}(x)\right| \leq M$ \begin{CJK}{UTF8}{mj}对任意\end{CJK} $x \in[a, b]$ \begin{CJK}{UTF8}{mj}及\end{CJK} $n \in \mathbb{N}_{+}$\begin{CJK}{UTF8}{mj}成立\end{CJK}, \begin{CJK}{UTF8}{mj}证明\end{CJK}: $\sum_{n=1}^{\infty} u_{n}(x)$ \begin{CJK}{UTF8}{mj}在\end{CJK} $[a, b]$ \begin{CJK}{UTF8}{mj}上一致收敛\end{CJK}.

(17) \begin{CJK}{UTF8}{mj}设函数\end{CJK} $f(x)$ \begin{CJK}{UTF8}{mj}在\end{CJK} $[0,1]$ \begin{CJK}{UTF8}{mj}上可导\end{CJK}, \begin{CJK}{UTF8}{mj}且\end{CJK} $f(0)=0, f(1)=1$, \begin{CJK}{UTF8}{mj}又\end{CJK} $p_{1}, p_{2}, \cdots, p_{n}$ \begin{CJK}{UTF8}{mj}为\end{CJK} $n$ \begin{CJK}{UTF8}{mj}个正数\end{CJK}, \begin{CJK}{UTF8}{mj}证明\end{CJK}: \begin{CJK}{UTF8}{mj}在\end{CJK} $(0,1)$ \begin{CJK}{UTF8}{mj}内存\end{CJK} \begin{CJK}{UTF8}{mj}在一组互异的数\end{CJK} $x_{1}, x_{2}, \cdots, x_{n}$, \begin{CJK}{UTF8}{mj}使得\end{CJK}
$$
\sum_{i=1}^{n} \frac{p_{i}}{f^{\prime}\left(x_{i}\right)}=\sum_{i=1}^{n} p_{i}
$$

\section{1. 华南理工大学 2009 年研究生入学考试试题高等代数 
 李扬 
 微信公众号: sxkyliyang}
\begin{enumerate}
  \item ( 15 \begin{CJK}{UTF8}{mj}分\end{CJK}) \begin{CJK}{UTF8}{mj}设\end{CJK} $f(x), g(x)$ \begin{CJK}{UTF8}{mj}是\end{CJK} $P[x]$ \begin{CJK}{UTF8}{mj}中的非零多项式\end{CJK}, \begin{CJK}{UTF8}{mj}且\end{CJK} $g(x)=s^{m}(x) g_{1}(x)$, \begin{CJK}{UTF8}{mj}这里\end{CJK} $m \geq 1,\left(s(x), g_{1}(x)\right)=$ $1, s(x) \mid f(x)$. \begin{CJK}{UTF8}{mj}证明\end{CJK}: \begin{CJK}{UTF8}{mj}不存在\end{CJK} $f_{1}(x), r(x) \in P[x]$, \begin{CJK}{UTF8}{mj}且\end{CJK} $r(x) \neq 0, \partial(r(x))<\partial(s(x))$ \begin{CJK}{UTF8}{mj}使得\end{CJK}
\end{enumerate}
$$
\frac{f(x)}{g(x)}=\frac{r(x)}{s^{m}(x)}+\frac{f_{1}(x)}{s^{m-1}(x) g_{1}(x)} .
$$

\begin{enumerate}
  \setcounter{enumi}{2}
  \item ( 15 \begin{CJK}{UTF8}{mj}分\end{CJK}) \begin{CJK}{UTF8}{mj}设\end{CJK} $P[X]_{n}$ \begin{CJK}{UTF8}{mj}表示数域\end{CJK} $P$ \begin{CJK}{UTF8}{mj}上所有次数\end{CJK} $<n$ \begin{CJK}{UTF8}{mj}的多项式及零多项式构成的线性空间\end{CJK}, \begin{CJK}{UTF8}{mj}令多项式\end{CJK}
\end{enumerate}
$$
f_{i}(x)=\left(x-a_{1}\right) \cdots\left(x-a_{i-1}\right)\left(x-a_{i+1}\right) \cdots\left(x-a_{n}\right)
$$
\begin{CJK}{UTF8}{mj}其中\end{CJK} $i=1,2, \cdots, n$, \begin{CJK}{UTF8}{mj}且\end{CJK} $a_{1}, a_{2}, \cdots, a_{n}$ \begin{CJK}{UTF8}{mj}是数域\end{CJK} $P$ \begin{CJK}{UTF8}{mj}中\end{CJK} $n$ \begin{CJK}{UTF8}{mj}个互不相同的数\end{CJK}.

(1) \begin{CJK}{UTF8}{mj}证明\end{CJK}: $f_{1}(x), f_{2}(x), \cdots, f_{n}(x)$ \begin{CJK}{UTF8}{mj}是\end{CJK} $P[X]_{n}$ \begin{CJK}{UTF8}{mj}的一组基\end{CJK};

(2) \begin{CJK}{UTF8}{mj}在\end{CJK} (1) \begin{CJK}{UTF8}{mj}中\end{CJK}, \begin{CJK}{UTF8}{mj}取\end{CJK} $a_{1}, a_{2}, \cdots, a_{n}$ \begin{CJK}{UTF8}{mj}为全体\end{CJK} $n$ \begin{CJK}{UTF8}{mj}次单位根\end{CJK}, \begin{CJK}{UTF8}{mj}求由基\end{CJK} $1, x, \cdots, x^{n-1}$ \begin{CJK}{UTF8}{mj}到基\end{CJK} $f_{1}(x), f_{2}(x), \cdots, f_{n}(x)$ \begin{CJK}{UTF8}{mj}的过渡矩阵\end{CJK} $T$.

\begin{enumerate}
  \setcounter{enumi}{3}
  \item ( 20 \begin{CJK}{UTF8}{mj}分\end{CJK}) \begin{CJK}{UTF8}{mj}设\end{CJK} $n$ \begin{CJK}{UTF8}{mj}阶方阵\end{CJK} $A$ \begin{CJK}{UTF8}{mj}满足\end{CJK} $A^{2}=A$, \begin{CJK}{UTF8}{mj}且\end{CJK} $A$ \begin{CJK}{UTF8}{mj}的秩\end{CJK} $\mathrm{r}(A)=r$.
\end{enumerate}
(1) \begin{CJK}{UTF8}{mj}证明\end{CJK}: $\operatorname{tr}(A)=r$, \begin{CJK}{UTF8}{mj}这里\end{CJK} $A$ \begin{CJK}{UTF8}{mj}的迹\end{CJK} $\operatorname{tr}(A)$ \begin{CJK}{UTF8}{mj}定义为\end{CJK} $A$ \begin{CJK}{UTF8}{mj}的主对角线上的元素之和\end{CJK};

(2) \begin{CJK}{UTF8}{mj}求\end{CJK} $|A+E|$ \begin{CJK}{UTF8}{mj}的值\end{CJK}.

\begin{enumerate}
  \setcounter{enumi}{4}
  \item ( 20 \begin{CJK}{UTF8}{mj}分\end{CJK}) \begin{CJK}{UTF8}{mj}设\end{CJK} $\varepsilon_{1}, \varepsilon_{2}, \varepsilon_{3}$ \begin{CJK}{UTF8}{mj}是欧氏空间\end{CJK} $V$ \begin{CJK}{UTF8}{mj}的一组标准正交基\end{CJK}, \begin{CJK}{UTF8}{mj}设\end{CJK}
\end{enumerate}
$$
\alpha_{1}=\varepsilon_{1}+\varepsilon_{2}-\varepsilon_{3}, \alpha_{2}=\varepsilon_{1}-\varepsilon_{2}-\varepsilon_{3}, W=L\left(\alpha_{1}, \alpha_{2}\right)
$$
(1)\begin{CJK}{UTF8}{mj}求\end{CJK} $W$ \begin{CJK}{UTF8}{mj}的一组标准正交基\end{CJK};

(2) \begin{CJK}{UTF8}{mj}求\end{CJK} $W^{\perp}$ \begin{CJK}{UTF8}{mj}的一组标准正交基\end{CJK};

(3) \begin{CJK}{UTF8}{mj}求\end{CJK} $\alpha=\varepsilon_{2}+2 \varepsilon_{3}$ \begin{CJK}{UTF8}{mj}在\end{CJK} $W$ \begin{CJK}{UTF8}{mj}中的内射影\end{CJK} (\begin{CJK}{UTF8}{mj}即求\end{CJK} $\beta \in W$, \begin{CJK}{UTF8}{mj}使\end{CJK} $\alpha=\beta+\gamma, \gamma \in W^{\perp}$ ), \begin{CJK}{UTF8}{mj}并求\end{CJK} $\alpha$ \begin{CJK}{UTF8}{mj}到\end{CJK} $W$ \begin{CJK}{UTF8}{mj}距离\end{CJK}.

\begin{enumerate}
  \setcounter{enumi}{5}
  \item ( 20 \begin{CJK}{UTF8}{mj}分\end{CJK}) \begin{CJK}{UTF8}{mj}设\end{CJK} $\mathscr{A}$ \begin{CJK}{UTF8}{mj}是数域\end{CJK} $P$ \begin{CJK}{UTF8}{mj}上的\end{CJK} $n$ \begin{CJK}{UTF8}{mj}维线性空间\end{CJK} $V$ \begin{CJK}{UTF8}{mj}的线性变换\end{CJK}, $f(x), g(x) \in P[x]$. \begin{CJK}{UTF8}{mj}证明\end{CJK}:
\end{enumerate}
(1)
$$
f(\mathscr{A})^{-1}(0)+g(\mathscr{A})^{-1}(0) \subseteq(f(\mathscr{A}) g(\mathscr{A}))^{-1}(0)
$$
(2) \begin{CJK}{UTF8}{mj}当\end{CJK} $f(x)$ \begin{CJK}{UTF8}{mj}与\end{CJK} $g(x)$ \begin{CJK}{UTF8}{mj}互素时\end{CJK}, \begin{CJK}{UTF8}{mj}有\end{CJK}
$$
f(\mathscr{A})^{-1}(0) \oplus g(\mathscr{A})^{-1}(0)=(f(\mathscr{A}) g(\mathscr{A}))^{-1}(0)
$$

\begin{enumerate}
  \setcounter{enumi}{6}
  \item ( 20 \begin{CJK}{UTF8}{mj}分\end{CJK}) \begin{CJK}{UTF8}{mj}设\end{CJK} $f\left(x_{1}, x_{2}, \cdots, x_{n}\right)=X^{\prime} A X$ \begin{CJK}{UTF8}{mj}为\end{CJK} $n$ \begin{CJK}{UTF8}{mj}元实二次型\end{CJK}, \begin{CJK}{UTF8}{mj}若矩阵\end{CJK} $A$ \begin{CJK}{UTF8}{mj}的顺序主子式\end{CJK} $\Delta_{k}(k=1,2, \cdots, n)$ \begin{CJK}{UTF8}{mj}都不为\end{CJK} \begin{CJK}{UTF8}{mj}零\end{CJK}, \begin{CJK}{UTF8}{mj}证明\end{CJK} $f\left(x_{1}, x_{2}, \cdots, x_{n}\right)$ \begin{CJK}{UTF8}{mj}可以经过非退化的线性替换化为下述标准型\end{CJK}
\end{enumerate}
$$
\lambda_{1} y_{1}^{2}+\lambda_{2} y_{2}^{2}+\cdots+\lambda_{n} y_{n}^{2}
$$
\begin{CJK}{UTF8}{mj}这里\end{CJK} $\lambda_{i}=\frac{\Delta_{i}}{\Delta_{i-1}}, i=1,2, \cdots, n$, \begin{CJK}{UTF8}{mj}并且\end{CJK} $\Delta_{0}=1$.

\begin{enumerate}
  \setcounter{enumi}{7}
  \item ( 20 \begin{CJK}{UTF8}{mj}分\end{CJK}) \begin{CJK}{UTF8}{mj}设\end{CJK} $A, B$ \begin{CJK}{UTF8}{mj}分别为数域\end{CJK} $P$ \begin{CJK}{UTF8}{mj}上的\end{CJK} $m \times n$ \begin{CJK}{UTF8}{mj}和\end{CJK} $n \times s$ \begin{CJK}{UTF8}{mj}矩阵\end{CJK}, \begin{CJK}{UTF8}{mj}又\end{CJK} $W=\left\{B \alpha \mid A B \alpha=0, \alpha\right.$ \begin{CJK}{UTF8}{mj}为\end{CJK} $P$ \begin{CJK}{UTF8}{mj}上的\end{CJK} $s$ \begin{CJK}{UTF8}{mj}维列向量\end{CJK}, \begin{CJK}{UTF8}{mj}即\end{CJK} $\left.\alpha \in P^{s \times 1}\right\}$ \begin{CJK}{UTF8}{mj}是\end{CJK} $n$ \begin{CJK}{UTF8}{mj}维列向量空间\end{CJK} $P^{n \times 1}$ \begin{CJK}{UTF8}{mj}的子空间\end{CJK}, \begin{CJK}{UTF8}{mj}证明\end{CJK}:
\end{enumerate}
$$
\operatorname{dim}(W)=\mathrm{r}(B)-\mathrm{r}(A B)
$$

\begin{enumerate}
  \setcounter{enumi}{8}
  \item ( 20 \begin{CJK}{UTF8}{mj}分\end{CJK}) \begin{CJK}{UTF8}{mj}设\end{CJK} $f(X, Y)$ \begin{CJK}{UTF8}{mj}为定义在数域\end{CJK} $P$ \begin{CJK}{UTF8}{mj}上的\end{CJK} $n$ \begin{CJK}{UTF8}{mj}维线性空间\end{CJK} $V$ \begin{CJK}{UTF8}{mj}上的一个双线性函数\end{CJK}, \begin{CJK}{UTF8}{mj}证明\end{CJK}:
\end{enumerate}
$$
f(X, Y)=X^{\prime} A Y=\sum_{i=1}^{n} \sum_{j=1}^{n} a_{i j} x_{i} y_{j}
$$
\begin{CJK}{UTF8}{mj}可以表示为两个线性函数\end{CJK}
$$
f_{1}(X)=\sum_{i=1}^{n} b_{i} x_{i}, f_{2}(Y)=\sum_{i=1}^{n} c_{i} y_{i}
$$
\begin{CJK}{UTF8}{mj}之积的充分必要条件是\end{CJK} $f(X, Y)$ \begin{CJK}{UTF8}{mj}的度量矩阵\end{CJK} $A$ \begin{CJK}{UTF8}{mj}的秩\end{CJK} $r(A) \leq 1$.

\section{2. 华南理工大学 2010 年研究生入学考试试题高等代数 
 李扬 
 微信公众号: sxkyliyang}
\begin{enumerate}
  \item (15 \begin{CJK}{UTF8}{mj}分\end{CJK}) \begin{CJK}{UTF8}{mj}设\end{CJK} $m, n$ \begin{CJK}{UTF8}{mj}为自然数\end{CJK}, \begin{CJK}{UTF8}{mj}证明\end{CJK}
\end{enumerate}
$$
\left(x^{m}-1, x^{n}-1\right)=x^{(m, n)}-1 .
$$

\begin{enumerate}
  \setcounter{enumi}{2}
  \item ( 15 \begin{CJK}{UTF8}{mj}分\end{CJK}) \begin{CJK}{UTF8}{mj}当\end{CJK} $a, b$ \begin{CJK}{UTF8}{mj}为何值时\end{CJK}, \begin{CJK}{UTF8}{mj}下列线性方程组无解\end{CJK}? \begin{CJK}{UTF8}{mj}有唯一解\end{CJK}? \begin{CJK}{UTF8}{mj}有无穷多解\end{CJK}? \begin{CJK}{UTF8}{mj}当方程组有解时\end{CJK}, \begin{CJK}{UTF8}{mj}写出其全部解\end{CJK}.
\end{enumerate}
$$
\left\{\begin{array}{l}
x+y-z=0 \\
2 x+(a+3) y-3 z=3 \\
-2 x+(a-1) y+b z=-1
\end{array}\right.
$$

\begin{enumerate}
  \setcounter{enumi}{3}
  \item ( 20 \begin{CJK}{UTF8}{mj}分\end{CJK}) \begin{CJK}{UTF8}{mj}设\end{CJK} $V$ \begin{CJK}{UTF8}{mj}是\end{CJK} $n$ \begin{CJK}{UTF8}{mj}维线性空间\end{CJK} $(n \geq 3), X$ \begin{CJK}{UTF8}{mj}和\end{CJK} $Y$ \begin{CJK}{UTF8}{mj}为\end{CJK} $V$ \begin{CJK}{UTF8}{mj}的两个空间\end{CJK}, \begin{CJK}{UTF8}{mj}并且\end{CJK}
\end{enumerate}
$$
\operatorname{dim}(X)=n-1, \operatorname{dim}(Y)=n-2 .
$$
(1) \begin{CJK}{UTF8}{mj}证明\end{CJK}: $\operatorname{dim}(X \cap Y)=n-2$ \begin{CJK}{UTF8}{mj}或\end{CJK} $n-3$.

(2) \begin{CJK}{UTF8}{mj}证明\end{CJK}: $\operatorname{dim}(X \cap Y)=n-2$ \begin{CJK}{UTF8}{mj}当且仅当\end{CJK} $Y$ \begin{CJK}{UTF8}{mj}是\end{CJK} $X$ \begin{CJK}{UTF8}{mj}的子空间\end{CJK}.

(3) \begin{CJK}{UTF8}{mj}举例说明\end{CJK}: \begin{CJK}{UTF8}{mj}存在满足题设条件的线性空间\end{CJK} $V$ \begin{CJK}{UTF8}{mj}及其子空间\end{CJK} $X$ \begin{CJK}{UTF8}{mj}和\end{CJK} $Y$ \begin{CJK}{UTF8}{mj}使得\end{CJK} $\operatorname{dim}(X \cap Y)=n-2$.

\begin{enumerate}
  \setcounter{enumi}{4}
  \item (15 \begin{CJK}{UTF8}{mj}分\end{CJK}) \begin{CJK}{UTF8}{mj}设\end{CJK} $A$ \begin{CJK}{UTF8}{mj}是\end{CJK} $n$ \begin{CJK}{UTF8}{mj}阶实对称矩阵\end{CJK}, \begin{CJK}{UTF8}{mj}若\end{CJK} $A$ \begin{CJK}{UTF8}{mj}的前\end{CJK} $n-1$ \begin{CJK}{UTF8}{mj}个顺序主子式均大于零\end{CJK}, \begin{CJK}{UTF8}{mj}而\end{CJK} $|A|=0$. \begin{CJK}{UTF8}{mj}证明\end{CJK}: $n$ \begin{CJK}{UTF8}{mj}元二次型\end{CJK}
\end{enumerate}
$$
f\left(x_{1}, x_{2}, \cdots, x_{n}\right)=X^{\prime} A X
$$
\begin{CJK}{UTF8}{mj}是半正定的\end{CJK}, \begin{CJK}{UTF8}{mj}其中\end{CJK} $X=\left(x_{1}, x_{2}, \cdots, x_{n}\right)^{\prime}$.

\begin{enumerate}
  \setcounter{enumi}{5}
  \item (15 \begin{CJK}{UTF8}{mj}分\end{CJK}) \begin{CJK}{UTF8}{mj}设\end{CJK} $\mathscr{A}$ \begin{CJK}{UTF8}{mj}是数域\end{CJK} $\mathbb{R}$ \begin{CJK}{UTF8}{mj}上的\end{CJK} $n$ \begin{CJK}{UTF8}{mj}维线性空间\end{CJK} $V$ \begin{CJK}{UTF8}{mj}上的线性变换\end{CJK}, $\mathscr{A}^{2}=\varepsilon$ (\begin{CJK}{UTF8}{mj}恒等变换\end{CJK}). \begin{CJK}{UTF8}{mj}令\end{CJK} $V^{+}=\{x \in V \mid \mathscr{A} x=x\}, V^{-}=$ $\{x \in V \mid \mathscr{A} x=-x\}$. \begin{CJK}{UTF8}{mj}证明\end{CJK}:
\end{enumerate}
$$
V=V^{+} \oplus V^{-}
$$

\begin{enumerate}
  \setcounter{enumi}{6}
  \item ( 15 \begin{CJK}{UTF8}{mj}分\end{CJK}) \begin{CJK}{UTF8}{mj}设\end{CJK} $A=\left(a_{1}, a_{2}, \cdots, a_{n}\right)$ \begin{CJK}{UTF8}{mj}为非零实\end{CJK} $1 \times n$ \begin{CJK}{UTF8}{mj}矩阵\end{CJK}, \begin{CJK}{UTF8}{mj}求\end{CJK}:\\
(1) $\mathrm{r}\left(A^{\prime} A\right)$;\\
(2) $A^{\prime} A$ \begin{CJK}{UTF8}{mj}的特征值和特征向量\end{CJK}.

  \item (15 \begin{CJK}{UTF8}{mj}分\end{CJK}) \begin{CJK}{UTF8}{mj}设\end{CJK} $\alpha$ \begin{CJK}{UTF8}{mj}为欧式空间\end{CJK} $V$ \begin{CJK}{UTF8}{mj}的非零向量\end{CJK}, \begin{CJK}{UTF8}{mj}对\end{CJK} $\xi \in V$ \begin{CJK}{UTF8}{mj}定义\end{CJK}

\end{enumerate}
$$
\mathscr{A} \xi=\xi-\frac{2(\xi, \alpha)}{(\alpha, \alpha)} \alpha .
$$
(1) \begin{CJK}{UTF8}{mj}证明\end{CJK}: $\mathscr{A}$ \begin{CJK}{UTF8}{mj}为\end{CJK} $V$ \begin{CJK}{UTF8}{mj}的正交变换\end{CJK}.

(2) \begin{CJK}{UTF8}{mj}记\end{CJK} $W=L(\alpha)^{\perp}$, \begin{CJK}{UTF8}{mj}则\end{CJK} $W$ \begin{CJK}{UTF8}{mj}是\end{CJK} $n-1$ \begin{CJK}{UTF8}{mj}维子空间\end{CJK}, \begin{CJK}{UTF8}{mj}并且\end{CJK}
$$
\mathscr{A} \xi= \begin{cases}\xi, & \xi \in W \\ -\xi, & \xi=\alpha\end{cases}
$$
(3) \begin{CJK}{UTF8}{mj}设\end{CJK} $V$ \begin{CJK}{UTF8}{mj}的维数是\end{CJK} 4 , \begin{CJK}{UTF8}{mj}令\end{CJK} $\varepsilon_{1}, \varepsilon_{2}, \varepsilon_{3}, \varepsilon_{4}$ \begin{CJK}{UTF8}{mj}为\end{CJK} $V$ \begin{CJK}{UTF8}{mj}的标准正交基\end{CJK}, \begin{CJK}{UTF8}{mj}并设\end{CJK} $\alpha=-\frac{1}{2} \varepsilon_{1}-\frac{1}{2} \varepsilon_{2}-\frac{1}{2} \varepsilon_{3}+\frac{1}{2} \varepsilon_{4}$, \begin{CJK}{UTF8}{mj}求\end{CJK} $\mathscr{A}$ \begin{CJK}{UTF8}{mj}在\end{CJK} $\varepsilon_{1}, \varepsilon_{2}, \varepsilon_{3}, \varepsilon_{4}$ \begin{CJK}{UTF8}{mj}下的矩阵\end{CJK}.

\begin{enumerate}
  \setcounter{enumi}{8}
  \item ( 20 \begin{CJK}{UTF8}{mj}分\end{CJK}) \begin{CJK}{UTF8}{mj}在欧式空间中有三组向量\end{CJK}: $\alpha_{1}, \alpha_{2}, \cdots, \alpha_{s} ; \beta_{1}, \beta_{2}, \cdots, \beta_{s}$ \begin{CJK}{UTF8}{mj}和\end{CJK} $\gamma_{1}, \gamma_{2}, \cdots, \gamma_{s}$. \begin{CJK}{UTF8}{mj}如果\end{CJK} $\alpha_{1}, \alpha_{2}, \cdots, \alpha_{s}$ \begin{CJK}{UTF8}{mj}是线性无关的\end{CJK}, $\beta_{1}, \beta_{2}, \cdots, \beta_{s}$ \begin{CJK}{UTF8}{mj}和\end{CJK} $\gamma_{1}, \gamma_{2}, \cdots, \gamma_{s}$ \begin{CJK}{UTF8}{mj}都是两两正交的单位向量\end{CJK}, \begin{CJK}{UTF8}{mj}并且对一切\end{CJK} $i, 1 \leq i \leq s$, \begin{CJK}{UTF8}{mj}均有\end{CJK}
\end{enumerate}
$$
L\left(\alpha_{1}, \cdots, \alpha_{i}\right)=L\left(\beta_{1}, \cdots, \beta_{i}\right)=L\left(\gamma_{1}, \cdots, \gamma_{i}\right) .
$$
\begin{CJK}{UTF8}{mj}证明\end{CJK}: \begin{CJK}{UTF8}{mj}对每个\end{CJK} $i$, \begin{CJK}{UTF8}{mj}有\end{CJK} $\beta_{i}=\pm \gamma_{i}$.

\begin{enumerate}
  \setcounter{enumi}{9}
  \item ( 20 \begin{CJK}{UTF8}{mj}分\end{CJK}) \begin{CJK}{UTF8}{mj}设\end{CJK} $A, B$ \begin{CJK}{UTF8}{mj}都是实对称矩阵\end{CJK}, \begin{CJK}{UTF8}{mj}证明\end{CJK}: \begin{CJK}{UTF8}{mj}当且仅当\end{CJK} $A B=B A$ \begin{CJK}{UTF8}{mj}时\end{CJK}, \begin{CJK}{UTF8}{mj}有正交矩阵\end{CJK} $Q$, \begin{CJK}{UTF8}{mj}使\end{CJK} $Q^{-1} A Q$ \begin{CJK}{UTF8}{mj}与\end{CJK} $Q^{-1} B Q$ \begin{CJK}{UTF8}{mj}同时\end{CJK} \begin{CJK}{UTF8}{mj}为对角矩阵\end{CJK}.
\end{enumerate}
\section{3. 华南理工大学 2011 年研究生入学考试试题高等代数 
 李扬 
 微信公众号: sxkyliyang}
\begin{enumerate}
  \item ( 20 \begin{CJK}{UTF8}{mj}分\end{CJK}) \begin{CJK}{UTF8}{mj}设\end{CJK} $x_{0}$ \begin{CJK}{UTF8}{mj}是数域\end{CJK} $P$ \begin{CJK}{UTF8}{mj}上的多项式\end{CJK} $u(x)=f(x) g^{\prime}(x)-f^{\prime}(x) g(x)$ \begin{CJK}{UTF8}{mj}的\end{CJK} $k$ \begin{CJK}{UTF8}{mj}重根\end{CJK}, \begin{CJK}{UTF8}{mj}记\end{CJK}
\end{enumerate}
$$
v(x)=f\left(x_{0}\right) g(x)-f(x) g\left(x_{0}\right)
$$
\begin{CJK}{UTF8}{mj}为非零多项式\end{CJK}. \begin{CJK}{UTF8}{mj}试证\end{CJK}: $x_{0}$ \begin{CJK}{UTF8}{mj}为数域\end{CJK} $P$ \begin{CJK}{UTF8}{mj}上多项式\end{CJK} $v(x)$ \begin{CJK}{UTF8}{mj}的\end{CJK} $k+1$ \begin{CJK}{UTF8}{mj}重根\end{CJK}. \begin{CJK}{UTF8}{mj}反之亦然\end{CJK}.

\begin{enumerate}
  \setcounter{enumi}{2}
  \item ( 20 \begin{CJK}{UTF8}{mj}分\end{CJK}) \begin{CJK}{UTF8}{mj}设\end{CJK} $A, B$ \begin{CJK}{UTF8}{mj}是数域\end{CJK} $P$ \begin{CJK}{UTF8}{mj}上的\end{CJK} $n$ \begin{CJK}{UTF8}{mj}阶方阵\end{CJK}, $X=\left(x_{1}, x_{2}, \cdots, x_{n}\right)^{\prime}$. \begin{CJK}{UTF8}{mj}已知齐次线性方程组\end{CJK} $A X=0$ \begin{CJK}{UTF8}{mj}和\end{CJK} $B X=0$ \begin{CJK}{UTF8}{mj}分别有\end{CJK} $l, m$ \begin{CJK}{UTF8}{mj}个线性无关的解向量\end{CJK}, \begin{CJK}{UTF8}{mj}这里\end{CJK} $l \geq 0, m \geq 0$. \begin{CJK}{UTF8}{mj}证明\end{CJK}:
\end{enumerate}
(1) \begin{CJK}{UTF8}{mj}方程组\end{CJK} $(A B) X=0$ \begin{CJK}{UTF8}{mj}至少有\end{CJK} $\max \{l, m\}$ \begin{CJK}{UTF8}{mj}个线性无关的解向量\end{CJK};

(2) \begin{CJK}{UTF8}{mj}若\end{CJK} $l+m>n$, \begin{CJK}{UTF8}{mj}则\end{CJK} $(A+B) X=0$ \begin{CJK}{UTF8}{mj}必有非零解\end{CJK};

(3) \begin{CJK}{UTF8}{mj}如果\end{CJK} $A X=0$ \begin{CJK}{UTF8}{mj}和\end{CJK} $B X=0$ \begin{CJK}{UTF8}{mj}无公共的非零解向量\end{CJK}, \begin{CJK}{UTF8}{mj}且\end{CJK} $l+m=n$, \begin{CJK}{UTF8}{mj}则\end{CJK} $P^{n}$ \begin{CJK}{UTF8}{mj}中任一向量\end{CJK} $\alpha$ \begin{CJK}{UTF8}{mj}都可唯一地表示成\end{CJK} $\alpha=\beta+\gamma$, \begin{CJK}{UTF8}{mj}这里\end{CJK} $\beta, \gamma$ \begin{CJK}{UTF8}{mj}分别是\end{CJK} $A X=0$ \begin{CJK}{UTF8}{mj}和\end{CJK} $B X=0$ \begin{CJK}{UTF8}{mj}的解向量\end{CJK}.

\begin{enumerate}
  \setcounter{enumi}{3}
  \item ( 15 \begin{CJK}{UTF8}{mj}分\end{CJK}) \begin{CJK}{UTF8}{mj}设\end{CJK} $A, B, C, D$ \begin{CJK}{UTF8}{mj}为\end{CJK} $n$ \begin{CJK}{UTF8}{mj}阶方阵\end{CJK}. \begin{CJK}{UTF8}{mj}若\end{CJK} $\left(\begin{array}{cc}A & B \\ C & D\end{array}\right)$ \begin{CJK}{UTF8}{mj}的秩是\end{CJK} $n$, \begin{CJK}{UTF8}{mj}证明\end{CJK}:
\end{enumerate}
$$
\left|\begin{array}{cc}
|A| & |B| \\
|C| & |D|
\end{array}\right|=0
$$
\begin{CJK}{UTF8}{mj}而且\end{CJK}, \begin{CJK}{UTF8}{mj}若\end{CJK} $A$ \begin{CJK}{UTF8}{mj}是可逆的\end{CJK}, \begin{CJK}{UTF8}{mj}则\end{CJK}
$$
D=C A^{-1} B
$$

\begin{enumerate}
  \setcounter{enumi}{4}
  \item ( 15 \begin{CJK}{UTF8}{mj}分\end{CJK}) \begin{CJK}{UTF8}{mj}设\end{CJK} $\mathscr{A}$ \begin{CJK}{UTF8}{mj}是\end{CJK} $n$ \begin{CJK}{UTF8}{mj}维线性空间\end{CJK} $V$ \begin{CJK}{UTF8}{mj}的一个线性变换\end{CJK}, $W$ \begin{CJK}{UTF8}{mj}是\end{CJK} $V$ \begin{CJK}{UTF8}{mj}的子空间\end{CJK}, $\mathscr{A} W$ \begin{CJK}{UTF8}{mj}表示\end{CJK} $W$ \begin{CJK}{UTF8}{mj}中向量的像组成的子空\end{CJK} \begin{CJK}{UTF8}{mj}间\end{CJK}, \begin{CJK}{UTF8}{mj}证明\end{CJK}:
\end{enumerate}
$$
\operatorname{dim}(\mathscr{A} W) \geq \mathrm{r}(\mathscr{A})+\operatorname{dim} W-n
$$

\begin{enumerate}
  \setcounter{enumi}{5}
  \item ( 20 \begin{CJK}{UTF8}{mj}分\end{CJK}) \begin{CJK}{UTF8}{mj}设\end{CJK} $\alpha_{1}, \alpha_{2}, \cdots, \alpha_{n}$ \begin{CJK}{UTF8}{mj}为数域\end{CJK} $P$ \begin{CJK}{UTF8}{mj}上的\end{CJK} $n$ \begin{CJK}{UTF8}{mj}维线性空间上的一组基\end{CJK}, $A$ \begin{CJK}{UTF8}{mj}为\end{CJK} $P$ \begin{CJK}{UTF8}{mj}上的一个\end{CJK} $n \times s$ \begin{CJK}{UTF8}{mj}矩阵\end{CJK}. \begin{CJK}{UTF8}{mj}若\end{CJK}
\end{enumerate}
$$
\left(\beta_{1}, \beta_{2}, \cdots, \beta_{s}\right)=\left(\alpha_{1}, \alpha_{2}, \cdots, \alpha_{n}\right) A
$$
\begin{CJK}{UTF8}{mj}则\end{CJK} $L\left(\beta_{1}, \beta_{2}, \cdots, \beta_{s}\right)$ \begin{CJK}{UTF8}{mj}的维数\end{CJK} $=\mathrm{r}(A)$.

\begin{enumerate}
  \setcounter{enumi}{6}
  \item (20 \begin{CJK}{UTF8}{mj}分\end{CJK}) \begin{CJK}{UTF8}{mj}设\end{CJK} $A=\left(a_{i j}\right)_{n \times n}, B=\left(b_{k l}\right)_{n \times n}$ \begin{CJK}{UTF8}{mj}为两个半正定的实对称矩阵\end{CJK}. \begin{CJK}{UTF8}{mj}证明\end{CJK}: $n$ \begin{CJK}{UTF8}{mj}阶实方阵\end{CJK}
\end{enumerate}
$$
C=\left(\begin{array}{cccc}
a_{11} b_{11} & a_{12} b_{12} & \cdots & a_{1 n} b_{1 n} \\
\vdots & \vdots & & \vdots \\
a_{n 1} b_{n 1} & a_{n 2} b_{n 2} & \cdots & a_{n n} b_{n n}
\end{array}\right)
$$
\begin{CJK}{UTF8}{mj}也是半正定的\end{CJK}.

\begin{enumerate}
  \setcounter{enumi}{7}
  \item ( 25 \begin{CJK}{UTF8}{mj}分\end{CJK}) \begin{CJK}{UTF8}{mj}用\end{CJK} $J$ \begin{CJK}{UTF8}{mj}表示元素全为\end{CJK} 1 \begin{CJK}{UTF8}{mj}的\end{CJK} $n$ \begin{CJK}{UTF8}{mj}阶矩阵\end{CJK}, $n \geq 2$, \begin{CJK}{UTF8}{mj}设\end{CJK}
\end{enumerate}
$$
f(x)=a+b x
$$
\begin{CJK}{UTF8}{mj}是有理数域\end{CJK} $\mathbb{Q}$ \begin{CJK}{UTF8}{mj}上的一元多项式\end{CJK}, \begin{CJK}{UTF8}{mj}令\end{CJK} $A=f(J)$,

(1) \begin{CJK}{UTF8}{mj}求\end{CJK} $J$ \begin{CJK}{UTF8}{mj}的全部特征值和全部特征向量\end{CJK};

(2) \begin{CJK}{UTF8}{mj}求\end{CJK} $A$ \begin{CJK}{UTF8}{mj}的所有特征子空间\end{CJK};

(3) $A$ \begin{CJK}{UTF8}{mj}是否可以对角化\end{CJK}? \begin{CJK}{UTF8}{mj}如果可对角化\end{CJK}, \begin{CJK}{UTF8}{mj}求出\end{CJK} $\mathbb{Q}$ \begin{CJK}{UTF8}{mj}上的一个可逆矩阵\end{CJK} $P$, \begin{CJK}{UTF8}{mj}使得\end{CJK} $P^{-1} A P$ \begin{CJK}{UTF8}{mj}为对角矩阵\end{CJK}, \begin{CJK}{UTF8}{mj}并写出这个\end{CJK} \begin{CJK}{UTF8}{mj}对角矩阵\end{CJK}. 8. ( 15 \begin{CJK}{UTF8}{mj}分\end{CJK}) \begin{CJK}{UTF8}{mj}设\end{CJK} $A$ \begin{CJK}{UTF8}{mj}是\end{CJK} $n$ \begin{CJK}{UTF8}{mj}阶复矩阵\end{CJK}, \begin{CJK}{UTF8}{mj}证明\end{CJK}: $A^{n}=0$ \begin{CJK}{UTF8}{mj}当且仅当\end{CJK}
$$
\operatorname{tr}\left(A^{k}\right)=0, k=1,2, \cdots, n .
$$
\begin{CJK}{UTF8}{mj}这里\end{CJK} $\operatorname{tr}(X)$ \begin{CJK}{UTF8}{mj}表示方阵\end{CJK} $X$ \begin{CJK}{UTF8}{mj}的迹\end{CJK}.

\section{4. 华南理工大学 2012 年研究生入学考试试题高等代数 
 李扬 
 微信公众号: sxkyliyang}
\begin{enumerate}
  \item (15 \begin{CJK}{UTF8}{mj}分\end{CJK}) \begin{CJK}{UTF8}{mj}设\end{CJK} $f(x), g(x) \in P[x], f(x) \neq 0$, \begin{CJK}{UTF8}{mj}证明下列条件等价\end{CJK}:
\end{enumerate}
(1) $f(x) \mid g(x)$;

(2) $\forall k \in N$ \begin{CJK}{UTF8}{mj}使得\end{CJK} $f^{k}(x) \mid g^{k}(x)$;

(3) \begin{CJK}{UTF8}{mj}存在自然数\end{CJK} $m$ \begin{CJK}{UTF8}{mj}使得\end{CJK} $f^{m}(x) \mid g^{m}(x)$.

\begin{enumerate}
  \setcounter{enumi}{2}
  \item ( 20 \begin{CJK}{UTF8}{mj}分\end{CJK}) (1) \begin{CJK}{UTF8}{mj}求\end{CJK} $n \times n$ \begin{CJK}{UTF8}{mj}可逆矩阵\end{CJK}
\end{enumerate}
$$
T=\left(\begin{array}{cccc} 
& & & t_{1} \\
& & t_{2} & \\
& . & & \\
t_{n} & & &
\end{array}\right)
$$
\begin{CJK}{UTF8}{mj}的逆\end{CJK}.

(2) \begin{CJK}{UTF8}{mj}已知\end{CJK} $A=\left(a_{i j}\right)$ \begin{CJK}{UTF8}{mj}为一个\end{CJK} $n \times n$ \begin{CJK}{UTF8}{mj}矩阵\end{CJK}, \begin{CJK}{UTF8}{mj}而\end{CJK} $T$ \begin{CJK}{UTF8}{mj}如上所示\end{CJK}, \begin{CJK}{UTF8}{mj}但\end{CJK} $t_{1}=t_{2}=\cdots=t_{n}=1$. \begin{CJK}{UTF8}{mj}试求\end{CJK}
$$
T^{-1} A T \text {. }
$$

\begin{enumerate}
  \setcounter{enumi}{3}
  \item ( 20 \begin{CJK}{UTF8}{mj}分\end{CJK}) \begin{CJK}{UTF8}{mj}对\end{CJK} $\lambda$ \begin{CJK}{UTF8}{mj}的不同的值判断下列方程组是否有解\end{CJK}, \begin{CJK}{UTF8}{mj}有解时求出其全部解\end{CJK}:
\end{enumerate}
$$
\left\{\begin{array}{l}
\lambda x_{1}+x_{2}+x_{3}=\lambda+1 \\
x_{1}+\lambda x_{2}+x_{3}=2 \\
x_{1}+x_{2}+\lambda x_{3}=2
\end{array}\right.
$$

\begin{enumerate}
  \setcounter{enumi}{4}
  \item ( 20 \begin{CJK}{UTF8}{mj}分\end{CJK}) \begin{CJK}{UTF8}{mj}设\end{CJK} $\mathscr{A}$ \begin{CJK}{UTF8}{mj}是\end{CJK} $n$ \begin{CJK}{UTF8}{mj}维线性空间\end{CJK} $V$ \begin{CJK}{UTF8}{mj}上的一个线性变换\end{CJK}, \begin{CJK}{UTF8}{mj}证明\end{CJK}:
\end{enumerate}
$$
\operatorname{dim}\left(\mathscr{A}^{-1}(0)\right)+\operatorname{dim}(\mathscr{A} V)=n .
$$
\begin{CJK}{UTF8}{mj}这里符号\end{CJK} $\mathscr{A}^{-1}(0)$ \begin{CJK}{UTF8}{mj}表示\end{CJK} $\mathscr{A}$ \begin{CJK}{UTF8}{mj}的核\end{CJK}, $\mathscr{A} V$ \begin{CJK}{UTF8}{mj}表示\end{CJK} $\mathscr{A}$ \begin{CJK}{UTF8}{mj}的值域\end{CJK}.

\begin{enumerate}
  \setcounter{enumi}{5}
  \item (15 \begin{CJK}{UTF8}{mj}分\end{CJK}) \begin{CJK}{UTF8}{mj}求矩阵\end{CJK}
\end{enumerate}
$$
\left(\begin{array}{ccc}
-1 & -1 & 0 \\
-4 & 3 & 0 \\
1 & 0 & 2
\end{array}\right)
$$
\begin{CJK}{UTF8}{mj}的若当标准型\end{CJK}

\begin{enumerate}
  \setcounter{enumi}{6}
  \item ( 20 \begin{CJK}{UTF8}{mj}分\end{CJK}) \begin{CJK}{UTF8}{mj}设\end{CJK} $V=\mathbb{C}^{n \times n}$ \begin{CJK}{UTF8}{mj}表示复数域\end{CJK} $\mathbb{C}$ \begin{CJK}{UTF8}{mj}上的\end{CJK} $n$ \begin{CJK}{UTF8}{mj}阶方阵关于矩阵的加法和数与矩阵的数量乘法构成的线性空间\end{CJK}, $A \in \mathbb{C}^{n \times n}$. \begin{CJK}{UTF8}{mj}定义\end{CJK} $V$ \begin{CJK}{UTF8}{mj}上的变换\end{CJK} $\mathscr{A}$ \begin{CJK}{UTF8}{mj}如下\end{CJK}:
\end{enumerate}
$$
\mathscr{A}(X)=A X-X A, \forall X \in \mathbb{C}^{n \times n}
$$
\begin{CJK}{UTF8}{mj}证明\end{CJK}:

(1) $\mathscr{A}$ \begin{CJK}{UTF8}{mj}是线性变换\end{CJK};

(2) $\mathscr{A}(X Y)=X \mathscr{A}(Y)+\mathscr{A}(X) Y$;

(3) 0 \begin{CJK}{UTF8}{mj}是\end{CJK} $\mathscr{A}$ \begin{CJK}{UTF8}{mj}的一个特征值\end{CJK};

(4) \begin{CJK}{UTF8}{mj}若\end{CJK} $A^{k}=0$, \begin{CJK}{UTF8}{mj}则\end{CJK} $\mathscr{A}^{2 k}=0$. 7. ( 25 \begin{CJK}{UTF8}{mj}分\end{CJK}) \begin{CJK}{UTF8}{mj}设\end{CJK} $\mathscr{A}$ \begin{CJK}{UTF8}{mj}是欧氏空间\end{CJK} $V$ \begin{CJK}{UTF8}{mj}的一个线性变换\end{CJK}, \begin{CJK}{UTF8}{mj}且\end{CJK} $\mathscr{A}$ \begin{CJK}{UTF8}{mj}在标准正交基\end{CJK} $\varepsilon_{1}, \varepsilon_{2}, \varepsilon_{3}, \varepsilon_{4}$ \begin{CJK}{UTF8}{mj}下的矩阵为\end{CJK}:
$$
A=\left(\begin{array}{cccc}
2 & 1 & 1 & -1 \\
1 & 2 & -1 & 1 \\
1 & -1 & 2 & 1 \\
-1 & 1 & 1 & 2
\end{array}\right)
$$
(1) \begin{CJK}{UTF8}{mj}证明\end{CJK}: $\mathscr{I}$ (\begin{CJK}{UTF8}{mj}恒等变换\end{CJK}), $\mathscr{A}, \mathscr{A}^{2}$ \begin{CJK}{UTF8}{mj}线性相关\end{CJK};

(2) \begin{CJK}{UTF8}{mj}求\end{CJK} $V$ \begin{CJK}{UTF8}{mj}的一组标准正交基\end{CJK}, \begin{CJK}{UTF8}{mj}使\end{CJK} $\mathscr{A}$ \begin{CJK}{UTF8}{mj}在该基下的矩阵为对角矩阵\end{CJK}.

\begin{enumerate}
  \setcounter{enumi}{8}
  \item (15 \begin{CJK}{UTF8}{mj}分\end{CJK}) \begin{CJK}{UTF8}{mj}设\end{CJK} $A, B$ \begin{CJK}{UTF8}{mj}为\end{CJK} $n$ \begin{CJK}{UTF8}{mj}阶实方阵\end{CJK}, \begin{CJK}{UTF8}{mj}且\end{CJK} $A$ \begin{CJK}{UTF8}{mj}为非零半正定矩阵\end{CJK}, $B$ \begin{CJK}{UTF8}{mj}为正定矩阵\end{CJK}, \begin{CJK}{UTF8}{mj}证明\end{CJK}:
\end{enumerate}
$$
|A+B|>|B|
$$

\section{5. 华南理工大学 2013 年研究生入学考试试题高等代数 
 李扬 
 微信公众号: sxkyliyang}
\begin{enumerate}
  \item (15 \begin{CJK}{UTF8}{mj}分\end{CJK}) \begin{CJK}{UTF8}{mj}设\end{CJK} $P$ \begin{CJK}{UTF8}{mj}是一个数域\end{CJK}, $f(x), g(x) \in P[x]$, \begin{CJK}{UTF8}{mj}且\end{CJK} $\partial(g(x)) \geq 1$. \begin{CJK}{UTF8}{mj}证明\end{CJK}: \begin{CJK}{UTF8}{mj}存在唯一的多项式序列\end{CJK} $f_{0}(x), f_{1}(x), \cdots, f_{r}(x)$, \begin{CJK}{UTF8}{mj}使得对\end{CJK} $0 \leq i \leq r$ \begin{CJK}{UTF8}{mj}有\end{CJK}
\end{enumerate}
$$
\partial\left(f_{i}(x)\right)<\partial(g(x))
$$
\begin{CJK}{UTF8}{mj}或者\end{CJK} $f_{i}(x)=0$, \begin{CJK}{UTF8}{mj}且\end{CJK}
$$
f(x)=f_{0}(x)+f_{1}(x) g(x)+f_{2}(x) g(x)+f_{2}(x) g^{2}(x)+\cdots+f_{r}(x) g^{r}(x) .
$$

\begin{enumerate}
  \setcounter{enumi}{2}
  \item ( 15 \begin{CJK}{UTF8}{mj}分\end{CJK}) \begin{CJK}{UTF8}{mj}计算下列\end{CJK} $n$ \begin{CJK}{UTF8}{mj}阶行列式\end{CJK}:
\end{enumerate}
$$
\left|\begin{array}{cccccc}
1+a_{1} & 1 & 1 & \cdots & 1 & 1 \\
1 & 1+a_{2} & 1 & \cdots & 1 & 1 \\
1 & 1 & 1+a_{3} & \cdots & 1 & 1 \\
\vdots & \vdots & \vdots & & \vdots & \vdots \\
1 & 1 & 1 & \cdots & 1 & 1+a_{n}
\end{array}\right|
$$

\begin{enumerate}
  \setcounter{enumi}{3}
  \item ( 15 \begin{CJK}{UTF8}{mj}分\end{CJK}) \begin{CJK}{UTF8}{mj}设\end{CJK} $n$ \begin{CJK}{UTF8}{mj}阶实对称矩阵\end{CJK} $A=\left(a_{i j}\right)$ \begin{CJK}{UTF8}{mj}是正定的\end{CJK}, $b_{1}, b_{2}, \cdots, b_{n}$ \begin{CJK}{UTF8}{mj}是任意\end{CJK} $n$ \begin{CJK}{UTF8}{mj}个非零的实数\end{CJK}, \begin{CJK}{UTF8}{mj}证明\end{CJK}: \begin{CJK}{UTF8}{mj}矩阵\end{CJK} $B=\left(a_{i j} b_{i} b_{j}\right)$ \begin{CJK}{UTF8}{mj}也是正定的\end{CJK}.

  \item ( 25 \begin{CJK}{UTF8}{mj}分\end{CJK}) \begin{CJK}{UTF8}{mj}讨论参数\end{CJK} $a, b$ \begin{CJK}{UTF8}{mj}取何值时\end{CJK}, \begin{CJK}{UTF8}{mj}线性方程组\end{CJK}

\end{enumerate}
$$
\left\{\begin{array}{l}
x_{1}+x_{2}-2 x_{3}+3 x_{4}=0 \\
2 x_{1}+x_{2}-6 x_{3}+4 x_{4}=-1 \\
3 x_{1}+2 x_{2}+a x_{3}+7 x_{4}=-1 \\
x_{1}-x_{2}-6 x_{3}-x_{4}=b
\end{array}\right.
$$
\begin{CJK}{UTF8}{mj}有解\end{CJK}? \begin{CJK}{UTF8}{mj}无解\end{CJK}? \begin{CJK}{UTF8}{mj}当有解的情形\end{CJK}, \begin{CJK}{UTF8}{mj}求出一般解\end{CJK}.

\begin{enumerate}
  \setcounter{enumi}{5}
  \item ( 20 \begin{CJK}{UTF8}{mj}分\end{CJK}) \begin{CJK}{UTF8}{mj}设\end{CJK} $\mathscr{A}$ \begin{CJK}{UTF8}{mj}是\end{CJK} $n$ \begin{CJK}{UTF8}{mj}维线性空间\end{CJK} $V$ \begin{CJK}{UTF8}{mj}的一个线性变换\end{CJK}.
\end{enumerate}
(1) \begin{CJK}{UTF8}{mj}证明\end{CJK}: $V$ \begin{CJK}{UTF8}{mj}的包含值域\end{CJK} $\mathscr{A} V$ \begin{CJK}{UTF8}{mj}的任何子空间\end{CJK} $W$ \begin{CJK}{UTF8}{mj}都是\end{CJK} $\mathscr{A}$-\begin{CJK}{UTF8}{mj}子空间\end{CJK};

(2) \begin{CJK}{UTF8}{mj}在核\end{CJK} $\mathscr{A}^{-1}(0)$ \begin{CJK}{UTF8}{mj}中取一组基\end{CJK} $\alpha_{1}, \cdots, \alpha_{m}$, \begin{CJK}{UTF8}{mj}并将它扩充为\end{CJK} $V$ \begin{CJK}{UTF8}{mj}的一组基\end{CJK} $\alpha_{1}, \cdots, \alpha_{m}, \alpha_{m+1}, \cdots, \alpha_{n}$, \begin{CJK}{UTF8}{mj}问\end{CJK}:\begin{CJK}{UTF8}{mj}在\end{CJK} \begin{CJK}{UTF8}{mj}这组基下\end{CJK} $\mathscr{A}$ \begin{CJK}{UTF8}{mj}的矩阵\end{CJK} $A$ \begin{CJK}{UTF8}{mj}有什么样的形状\end{CJK}?

\begin{enumerate}
  \setcounter{enumi}{6}
  \item ( 25 \begin{CJK}{UTF8}{mj}分\end{CJK}) \begin{CJK}{UTF8}{mj}设\end{CJK} $V$ \begin{CJK}{UTF8}{mj}为\end{CJK} 4 \begin{CJK}{UTF8}{mj}维欧式空间\end{CJK}, $\varepsilon_{1}, \varepsilon_{2}, \varepsilon_{3}, \varepsilon_{4}$ \begin{CJK}{UTF8}{mj}为\end{CJK} $V$ \begin{CJK}{UTF8}{mj}的一组标准正交基\end{CJK}, \begin{CJK}{UTF8}{mj}令\end{CJK}
\end{enumerate}
$$
\begin{aligned}
&\alpha_{1}=\varepsilon_{2}+\varepsilon_{3}+\varepsilon_{4} \\
&\alpha_{2}=\varepsilon_{1}+\varepsilon_{3}+\varepsilon_{4} \\
&\alpha_{3}=\varepsilon_{1}+\varepsilon_{2}+\varepsilon_{4} \\
&\alpha_{4}=\varepsilon_{1}+\varepsilon_{2}+\varepsilon_{3}
\end{aligned}
$$
(1) \begin{CJK}{UTF8}{mj}将\end{CJK} $\alpha_{1}, \alpha_{2}, \alpha_{3}, \alpha_{4}$ \begin{CJK}{UTF8}{mj}化为单位正交的向量组\end{CJK} $\beta_{1}, \beta_{2}, \beta_{3}, \beta_{4}$;

(2) \begin{CJK}{UTF8}{mj}求由基\end{CJK} $\varepsilon_{1}, \varepsilon_{2}, \varepsilon_{3}, \varepsilon_{4}$ \begin{CJK}{UTF8}{mj}到基\end{CJK} $\beta_{1}, \beta_{2}, \beta_{3}, \beta_{4}$ \begin{CJK}{UTF8}{mj}的过渡矩阵\end{CJK};

(3) \begin{CJK}{UTF8}{mj}令\end{CJK} $W_{1}=L\left(\alpha_{1}, \alpha_{2}\right), U_{1}=W_{1}^{\perp} ; W_{2}=L\left(\alpha_{2}, \alpha_{4}\right), U_{2}=W_{2}^{\perp}$. \begin{CJK}{UTF8}{mj}试用基向量\end{CJK} $\varepsilon_{1}, \varepsilon_{2}, \varepsilon_{3}, \varepsilon_{4}$ \begin{CJK}{UTF8}{mj}表示子空间\end{CJK} $U_{1}+U_{2}$, \begin{CJK}{UTF8}{mj}并确定其维数\end{CJK}. 7. ( 20 \begin{CJK}{UTF8}{mj}分\end{CJK}) \begin{CJK}{UTF8}{mj}设\end{CJK} $n$ \begin{CJK}{UTF8}{mj}阶方阵\end{CJK} $A$ \begin{CJK}{UTF8}{mj}满足\end{CJK} $A^{2}=2 A$, \begin{CJK}{UTF8}{mj}且\end{CJK} $A$ \begin{CJK}{UTF8}{mj}的秩\end{CJK} $r(A)=r$.\\
(1) \begin{CJK}{UTF8}{mj}证明\end{CJK}: $\mathrm{r}(A-2 E)=n-r$;\\
(2) \begin{CJK}{UTF8}{mj}证明\end{CJK}: $A$ \begin{CJK}{UTF8}{mj}相似于对角阵\end{CJK};\\
(3) \begin{CJK}{UTF8}{mj}求行列式\end{CJK} $|A-E|$ \begin{CJK}{UTF8}{mj}的值\end{CJK}.

\begin{enumerate}
  \setcounter{enumi}{8}
  \item ( 15 \begin{CJK}{UTF8}{mj}分\end{CJK}) \begin{CJK}{UTF8}{mj}设\end{CJK} $A$ \begin{CJK}{UTF8}{mj}是数域\end{CJK} $P$ \begin{CJK}{UTF8}{mj}上的\end{CJK} $n$ \begin{CJK}{UTF8}{mj}阶方阵\end{CJK}, $f(x), g(x)$ \begin{CJK}{UTF8}{mj}为数域\end{CJK} $P$ \begin{CJK}{UTF8}{mj}上的多项式\end{CJK}, \begin{CJK}{UTF8}{mj}且\end{CJK} $(f(x), g(x))=1$. \begin{CJK}{UTF8}{mj}令\end{CJK} $h(x)=$ $f(x) g(x)$, \begin{CJK}{UTF8}{mj}用\end{CJK} $V_{1}, V_{2}, V$ \begin{CJK}{UTF8}{mj}分别表示\end{CJK} $n$ \begin{CJK}{UTF8}{mj}元齐次线性方程组\end{CJK} $f(A) X=0, g(A) X=0, h(A) X=0$ \begin{CJK}{UTF8}{mj}的解空间\end{CJK}, \begin{CJK}{UTF8}{mj}这里\end{CJK} $X=\left(x_{1}, x_{2}, \cdots, x_{n}\right)^{\prime}$. \begin{CJK}{UTF8}{mj}证明\end{CJK}:
\end{enumerate}
$$
V=V_{1} \oplus V_{2}
$$

\section{6. 华南理工大学 2014 年研究生入学考试试题高等代数 
 李扬 
 微信公众号: sxkyliyang}
\begin{enumerate}
  \item ( 15 \begin{CJK}{UTF8}{mj}分\end{CJK}) \begin{CJK}{UTF8}{mj}设\end{CJK} $P$ \begin{CJK}{UTF8}{mj}是一个数域\end{CJK}, $f(x), g(x) \in P[x]$, \begin{CJK}{UTF8}{mj}证明\end{CJK}: $f(x)$ \begin{CJK}{UTF8}{mj}与\end{CJK} $g(x)$ \begin{CJK}{UTF8}{mj}互素的充分必要条件是\end{CJK} $f\left(x^{n}\right)$ \begin{CJK}{UTF8}{mj}与\end{CJK} $g\left(x^{n}\right)$ \begin{CJK}{UTF8}{mj}互\end{CJK} \begin{CJK}{UTF8}{mj}素\end{CJK}, \begin{CJK}{UTF8}{mj}这里\end{CJK} $n$ \begin{CJK}{UTF8}{mj}是任意给定的自然数\end{CJK}.

  \item (20 \begin{CJK}{UTF8}{mj}分\end{CJK}) \begin{CJK}{UTF8}{mj}计算行列式\end{CJK}:

\end{enumerate}
$$
D=\left|\begin{array}{cccccc}
x_{1} y_{1} & x_{1} y_{2} & x_{1} y_{3} & \cdots & x_{1} y_{n-1} & x_{1} y_{n} \\
x_{1} y_{2} & x_{2} y_{2} & x_{2} y_{3} & \cdots & x_{2} y_{n-1} & x_{2} y_{n} \\
x_{1} y_{3} & x_{2} y_{3} & x_{3} y_{3} & \cdots & x_{3} y_{n-1} & x_{3} y_{n} \\
\vdots & \vdots & \vdots & & \vdots & \vdots \\
x_{1} y_{n-1} & x_{2} y_{n-1} & x_{3} y_{n-1} & \cdots & x_{n-1} y_{n-1} & x_{n-1} y_{n} \\
x_{1} y_{n} & x_{2} y_{n} & x_{3} y_{n} & \cdots & x_{n-1} y_{n} & x_{n} y_{n}
\end{array}\right|
$$

\begin{enumerate}
  \setcounter{enumi}{3}
  \item (20 \begin{CJK}{UTF8}{mj}分\end{CJK}) \begin{CJK}{UTF8}{mj}证明\end{CJK}: (1) \begin{CJK}{UTF8}{mj}若\end{CJK} $\lambda \neq 0$ \begin{CJK}{UTF8}{mj}是矩阵\end{CJK} $A$ \begin{CJK}{UTF8}{mj}的一个特征值\end{CJK}, \begin{CJK}{UTF8}{mj}则\end{CJK} $\frac{1}{\lambda}|A|$ \begin{CJK}{UTF8}{mj}是\end{CJK} $A$ \begin{CJK}{UTF8}{mj}的伴随矩阵\end{CJK} $A^{*}$ \begin{CJK}{UTF8}{mj}的一个特征值\end{CJK};
\end{enumerate}
(2) \begin{CJK}{UTF8}{mj}若\end{CJK} $\alpha$ \begin{CJK}{UTF8}{mj}是矩阵\end{CJK} $A$ \begin{CJK}{UTF8}{mj}的一个特征向量\end{CJK}, \begin{CJK}{UTF8}{mj}则\end{CJK} $\alpha$ \begin{CJK}{UTF8}{mj}也是\end{CJK} $A^{*}$ \begin{CJK}{UTF8}{mj}的一个特征向量\end{CJK}.

\begin{enumerate}
  \setcounter{enumi}{4}
  \item ( 20 \begin{CJK}{UTF8}{mj}分\end{CJK}) \begin{CJK}{UTF8}{mj}设\end{CJK}
\end{enumerate}
$$
W=\left\{f(x) \mid f(1)=0, f(x) \in \mathbb{R}[x]_{n}\right\},
$$
\begin{CJK}{UTF8}{mj}这里\end{CJK} $\mathbb{R}[x]_{n}$ \begin{CJK}{UTF8}{mj}表示实数域\end{CJK} $\mathbb{R}$ \begin{CJK}{UTF8}{mj}上的次数小于\end{CJK} $n$ \begin{CJK}{UTF8}{mj}的多项式添上零多项式构成的线性空间\end{CJK}.

(1) \begin{CJK}{UTF8}{mj}证明\end{CJK} $W$ \begin{CJK}{UTF8}{mj}是\end{CJK} $\mathbb{R}[x]_{n}$ \begin{CJK}{UTF8}{mj}的线性子空间\end{CJK};

(2) \begin{CJK}{UTF8}{mj}求\end{CJK} $W$ \begin{CJK}{UTF8}{mj}的维数与一组基\end{CJK}.

\begin{enumerate}
  \setcounter{enumi}{5}
  \item ( 20 \begin{CJK}{UTF8}{mj}分\end{CJK}) \begin{CJK}{UTF8}{mj}已知齐次线性方程组\end{CJK}
\end{enumerate}
$$
\left\{\begin{array}{c}
\left(a_{1}+b\right) x_{1}+a_{2} x_{2}+\cdots+a_{n} x_{n}=0 \\
a_{1} x_{1}+\left(a_{2}+b\right) x_{2}+\cdots+a_{n} x_{n}=0 \\
\vdots \\
a_{1} x_{1}+a_{2} x_{2}+\cdots+\left(a_{n}+b\right) x_{n}=0
\end{array}\right.
$$
\begin{CJK}{UTF8}{mj}其中\end{CJK} $\sum_{i=1}^{n} a_{i} \neq 0$. \begin{CJK}{UTF8}{mj}试讨论\end{CJK} $a_{1}, a_{2}, \cdots, a_{n}, b$ \begin{CJK}{UTF8}{mj}满足什么关系时\end{CJK},

(1) \begin{CJK}{UTF8}{mj}方程组仅有零解\end{CJK}?

(2) \begin{CJK}{UTF8}{mj}方程组有非零解\end{CJK}? \begin{CJK}{UTF8}{mj}在有非零解时\end{CJK}, \begin{CJK}{UTF8}{mj}求此方程组的一个基础解系\end{CJK}.

\begin{enumerate}
  \setcounter{enumi}{6}
  \item ( 20 \begin{CJK}{UTF8}{mj}分\end{CJK}) \begin{CJK}{UTF8}{mj}设二次型\end{CJK}
\end{enumerate}
$$
f\left(x_{1}, x_{2}, x_{3}\right)=x_{1}^{2}+x_{2}^{2}+x_{3}^{2}+2 a x_{1} x_{2}+2 b x_{2} x_{3}+2 x_{1} x_{3}
$$
\begin{CJK}{UTF8}{mj}经正交变换\end{CJK} $X=P Y$ \begin{CJK}{UTF8}{mj}化为标准形\end{CJK} $f=y_{2}^{2}+2 y_{3}^{2}$, \begin{CJK}{UTF8}{mj}其中\end{CJK} $X=\left(x_{1}, x_{2}, x_{3}\right)^{\prime}, Y=\left(y_{1}, y_{2}, y_{3}\right)^{\prime}$ \begin{CJK}{UTF8}{mj}是\end{CJK} 3 \begin{CJK}{UTF8}{mj}维列向量\end{CJK}, $P$ \begin{CJK}{UTF8}{mj}是\end{CJK} 3 \begin{CJK}{UTF8}{mj}阶正交矩阵\end{CJK},

(1) \begin{CJK}{UTF8}{mj}求常数\end{CJK} $a, b$ \begin{CJK}{UTF8}{mj}的值\end{CJK};

(2) \begin{CJK}{UTF8}{mj}求正交矩阵\end{CJK} $P$.

\begin{enumerate}
  \setcounter{enumi}{7}
  \item ( 20 \begin{CJK}{UTF8}{mj}分\end{CJK}) \begin{CJK}{UTF8}{mj}设\end{CJK} $W$ \begin{CJK}{UTF8}{mj}为数域\end{CJK} $P$ \begin{CJK}{UTF8}{mj}上\end{CJK} $n$ \begin{CJK}{UTF8}{mj}维线性空间\end{CJK} $V$ \begin{CJK}{UTF8}{mj}的子空间\end{CJK}, $\mathscr{A}$ \begin{CJK}{UTF8}{mj}是\end{CJK} $V$ \begin{CJK}{UTF8}{mj}的一个线性变换\end{CJK}, $\mathscr{A} W$ \begin{CJK}{UTF8}{mj}表示\end{CJK} $W$ \begin{CJK}{UTF8}{mj}中向量的像组\end{CJK} \begin{CJK}{UTF8}{mj}成的子空间\end{CJK}. \begin{CJK}{UTF8}{mj}令\end{CJK} $W_{0}=W \cap \mathscr{A}^{-1}(0)$, \begin{CJK}{UTF8}{mj}证明\end{CJK}:
\end{enumerate}
$$
\operatorname{dim}(W)=\operatorname{dim}(\mathscr{A} W)+\operatorname{dim}\left(W_{0}\right) .
$$

\begin{enumerate}
  \setcounter{enumi}{8}
  \item ( 15 \begin{CJK}{UTF8}{mj}分\end{CJK}) \begin{CJK}{UTF8}{mj}设\end{CJK} $A, C$ \begin{CJK}{UTF8}{mj}是\end{CJK} $n$ \begin{CJK}{UTF8}{mj}阶正定矩阵\end{CJK}, \begin{CJK}{UTF8}{mj}已知\end{CJK} $B$ \begin{CJK}{UTF8}{mj}是矩阵方程\end{CJK}
\end{enumerate}
$$
A X+X A^{\prime}=C
$$
\begin{CJK}{UTF8}{mj}的唯一解\end{CJK}, \begin{CJK}{UTF8}{mj}证明\end{CJK}: $B$ \begin{CJK}{UTF8}{mj}是正定矩阵\end{CJK}.

\section{7. 华南理工大学 2015 年研究生入学考试试题高等代数 
 李扬 
 微信公众号: sxkyliyang}
\begin{enumerate}
  \item ( 20 \begin{CJK}{UTF8}{mj}分\end{CJK}) \begin{CJK}{UTF8}{mj}设\end{CJK}
\end{enumerate}
$$
\begin{aligned}
&f(x)=x^{5}+2 x^{4}-7 x^{3}-8 x-2 \\
&g(x)=2 x^{4}-2 x^{3}+5 x^{2}-2 x+3 .
\end{aligned}
$$
\begin{CJK}{UTF8}{mj}求\end{CJK} $(f(x), g(x))$, \begin{CJK}{UTF8}{mj}并求\end{CJK} $u(x), v(x)$ \begin{CJK}{UTF8}{mj}使得\end{CJK}
$$
(f(x), g(x))=u(x) f(x)+v(x) g(x)
$$

\begin{enumerate}
  \setcounter{enumi}{2}
  \item (15 \begin{CJK}{UTF8}{mj}分\end{CJK}) \begin{CJK}{UTF8}{mj}已知齐次方程组\end{CJK}
\end{enumerate}
$$
\left\{\begin{array}{l}
x_{1}+2 x_{2}+3 x_{3}=0 \\
2 x_{1}+3 x_{2}+5 x_{3}=0 \\
x_{1}+x_{2}+a x_{3}=0
\end{array}\right.
$$
$$
\left\{\begin{array}{l}
x_{1}+b x_{2}+c x_{3}=0 \\
2 x_{1}+b^{2} x_{2}+(c+1) x_{3}=0
\end{array},\right.
$$
\begin{CJK}{UTF8}{mj}同解\end{CJK}, \begin{CJK}{UTF8}{mj}求\end{CJK} $a, b, c$ \begin{CJK}{UTF8}{mj}的值\end{CJK}.

\begin{enumerate}
  \setcounter{enumi}{3}
  \item (15 \begin{CJK}{UTF8}{mj}分\end{CJK}) \begin{CJK}{UTF8}{mj}设\end{CJK} $\omega$ \begin{CJK}{UTF8}{mj}为任意一个\end{CJK} $n$ \begin{CJK}{UTF8}{mj}次单位根\end{CJK}, \begin{CJK}{UTF8}{mj}计算下列\end{CJK} $n(n \geq 2)$ \begin{CJK}{UTF8}{mj}阶行列式\end{CJK}:
\end{enumerate}
$$
D=\left|\begin{array}{ccccc}
1 & \omega^{-1} & \omega^{-2} & \cdots & \omega^{-n+1} \\
\omega^{-n+1} & 1 & \omega^{-1} & \cdots & \omega^{-n+2} \\
\omega^{-n+2} & \omega^{-n+1} & 1 & \cdots & \omega^{-n+3} \\
\vdots & \vdots & \vdots & & \vdots \\
\omega^{-1} & \omega^{-2} & \omega^{-3} & \cdots & 1
\end{array}\right|
$$

\begin{enumerate}
  \setcounter{enumi}{4}
  \item ( 15 \begin{CJK}{UTF8}{mj}分\end{CJK}) \begin{CJK}{UTF8}{mj}用\end{CJK} $\mathbb{C}[x]$ \begin{CJK}{UTF8}{mj}表示复数域\end{CJK} $\mathbb{C}$ \begin{CJK}{UTF8}{mj}上次数小于\end{CJK} $n$ \begin{CJK}{UTF8}{mj}的多项式以及零多项式组成的线性空间\end{CJK}. \begin{CJK}{UTF8}{mj}今有\end{CJK} $\mathbb{C}[x]$ \begin{CJK}{UTF8}{mj}到\end{CJK} $\mathbb{C}^{n}$ \begin{CJK}{UTF8}{mj}的映\end{CJK} \begin{CJK}{UTF8}{mj}射\end{CJK}
\end{enumerate}
$$
\mathscr{A}: f(x) \mapsto(f(0), f(1), f(2), \cdots, f(n-1)) .
$$
\begin{CJK}{UTF8}{mj}证明\end{CJK}: $\mathscr{A}$ \begin{CJK}{UTF8}{mj}为线性空间\end{CJK} $\mathbb{C}[x]$ \begin{CJK}{UTF8}{mj}到线性空间\end{CJK} $\mathbb{C}^{n}$ \begin{CJK}{UTF8}{mj}的同构映射\end{CJK}.

\begin{enumerate}
  \setcounter{enumi}{5}
  \item (20 \begin{CJK}{UTF8}{mj}分\end{CJK}) \begin{CJK}{UTF8}{mj}设实二次型\end{CJK}
\end{enumerate}
$$
f\left(x_{1}, x_{2}, x_{3}\right)=X^{\prime} A X=a x_{1}^{2}+2 x_{2}^{2}+2 x_{3}^{2}+2 b x_{1} x_{2}(b>0),
$$
\begin{CJK}{UTF8}{mj}其中二次型的矩阵\end{CJK} $A$ \begin{CJK}{UTF8}{mj}的特征值之和为\end{CJK} 1 , \begin{CJK}{UTF8}{mj}之积为\end{CJK} $-12$.

(1) \begin{CJK}{UTF8}{mj}求\end{CJK} $a, b$ \begin{CJK}{UTF8}{mj}的值\end{CJK};

(2) \begin{CJK}{UTF8}{mj}利用正交变换将二次型化为标准型\end{CJK}, \begin{CJK}{UTF8}{mj}并写出所用的正交变换和对应的正交矩阵\end{CJK}.

\begin{enumerate}
  \setcounter{enumi}{6}
  \item (20 \begin{CJK}{UTF8}{mj}分\end{CJK}) (1) \begin{CJK}{UTF8}{mj}设\end{CJK} $A, B$ \begin{CJK}{UTF8}{mj}是\end{CJK} $n$ \begin{CJK}{UTF8}{mj}阶方阵\end{CJK}, \begin{CJK}{UTF8}{mj}证明\end{CJK}:
\end{enumerate}
$$
\mathrm{r}(A B-E) \leq \mathrm{r}(A-E)+\mathrm{r}(B-E)
$$
(2) \begin{CJK}{UTF8}{mj}设当\end{CJK} $i=1,2, \cdots, k$ \begin{CJK}{UTF8}{mj}时\end{CJK}, $B_{i}$ \begin{CJK}{UTF8}{mj}为\end{CJK} $n$ \begin{CJK}{UTF8}{mj}阶幂等矩阵\end{CJK} (\begin{CJK}{UTF8}{mj}即满足\end{CJK} $B_{i}^{2}=B_{i}$ ), \begin{CJK}{UTF8}{mj}并且\end{CJK} $A=B_{1} B_{2} \cdots B_{k}$, \begin{CJK}{UTF8}{mj}证明\end{CJK}:
$$
\mathrm{r}(E-A) \leq k(n-\mathrm{r}(A))
$$

\begin{enumerate}
  \setcounter{enumi}{7}
  \item ( 20 \begin{CJK}{UTF8}{mj}分\end{CJK}) \begin{CJK}{UTF8}{mj}已知\end{CJK} $\mathbb{R}^{3}$ \begin{CJK}{UTF8}{mj}的线性变换\end{CJK} $\mathscr{A}$ \begin{CJK}{UTF8}{mj}在基\end{CJK} $\eta_{1}=(-1,1,1), \eta_{2}=(1,0,-1), \eta_{3}=(0,1,1)$ \begin{CJK}{UTF8}{mj}下的矩阵为\end{CJK}:
\end{enumerate}
$$
A=\left(\begin{array}{ccc}
1 & 0 & 1 \\
1 & 1 & 0 \\
-3 & 2 & 1
\end{array}\right)
$$
(1) \begin{CJK}{UTF8}{mj}求\end{CJK} $\mathscr{A}$ \begin{CJK}{UTF8}{mj}在基\end{CJK} $\varepsilon_{1}=(1,0,0), \varepsilon_{2}=(0,1,0), \varepsilon_{3}=(0,0,1)$ \begin{CJK}{UTF8}{mj}下的矩阵\end{CJK};

(2) \begin{CJK}{UTF8}{mj}求\end{CJK} $\mathscr{A}$ \begin{CJK}{UTF8}{mj}的值域和核\end{CJK}.

\begin{enumerate}
  \setcounter{enumi}{8}
  \item (20 \begin{CJK}{UTF8}{mj}分\end{CJK}) \begin{CJK}{UTF8}{mj}设\end{CJK} $V$ \begin{CJK}{UTF8}{mj}为\end{CJK} $n$ \begin{CJK}{UTF8}{mj}维欧式空间\end{CJK}, \begin{CJK}{UTF8}{mj}对任意单位向量\end{CJK} $\alpha$, \begin{CJK}{UTF8}{mj}定义变换\end{CJK}
\end{enumerate}
$$
\mathscr{A}_{\alpha}(x)=x-2(\alpha, x) \alpha, \forall x \in V
$$
\begin{CJK}{UTF8}{mj}证明\end{CJK}:

(1) $\mathscr{A}_{\alpha}$ \begin{CJK}{UTF8}{mj}为\end{CJK} $V$ \begin{CJK}{UTF8}{mj}的正交变换\end{CJK}, \begin{CJK}{UTF8}{mj}称它为镜面反射\end{CJK}.

(2) \begin{CJK}{UTF8}{mj}设\end{CJK} $\xi, \eta$ \begin{CJK}{UTF8}{mj}为\end{CJK} $V$ \begin{CJK}{UTF8}{mj}的任意两个不同的单位向量\end{CJK}, \begin{CJK}{UTF8}{mj}则存在镜面反射\end{CJK} $\mathscr{A}_{\alpha}$ \begin{CJK}{UTF8}{mj}使得\end{CJK} $\mathscr{A}_{\alpha}(\xi)=\eta$;

(3) $V$ \begin{CJK}{UTF8}{mj}的任意一个正交变换均可以表示为镜面反射的乘积\end{CJK}.

\section{8. 华南理工大学 2016 年研究生入学考试试题高等代数 
 李扬 
 微信公众号: sxkyliyang}
\begin{enumerate}
  \item (15 \begin{CJK}{UTF8}{mj}分\end{CJK}) \begin{CJK}{UTF8}{mj}设\end{CJK} $f(x)$ \begin{CJK}{UTF8}{mj}和\end{CJK} $g(x)$ \begin{CJK}{UTF8}{mj}都是数域\end{CJK} $P$ \begin{CJK}{UTF8}{mj}上的次数不小于\end{CJK} 1 \begin{CJK}{UTF8}{mj}的多项式\end{CJK}. \begin{CJK}{UTF8}{mj}证明\end{CJK}: $(f(x), g(x))=1$ \begin{CJK}{UTF8}{mj}当且仅当存在唯一\end{CJK} \begin{CJK}{UTF8}{mj}的多项式\end{CJK} $u(x), v(x) \in P[x]$ \begin{CJK}{UTF8}{mj}使得\end{CJK}
\end{enumerate}
$$
u(x) f(x)+v(x) g(x)=1
$$
\begin{CJK}{UTF8}{mj}这里\end{CJK} $\partial(u(x))<\partial(g(x))$ \begin{CJK}{UTF8}{mj}且\end{CJK} $\partial(v(x))<\partial(f(x))$.

\begin{enumerate}
  \setcounter{enumi}{2}
  \item ( 15 \begin{CJK}{UTF8}{mj}分\end{CJK}) \begin{CJK}{UTF8}{mj}设\end{CJK} $f(x)=a_{1}+a_{2} x+\cdots+a_{n} x^{n-1}, \varepsilon_{0}, \varepsilon_{1}, \cdots, \varepsilon_{n-1}$ \begin{CJK}{UTF8}{mj}为全部的\end{CJK} $n$ \begin{CJK}{UTF8}{mj}次单位根\end{CJK}, \begin{CJK}{UTF8}{mj}证明\end{CJK}:
\end{enumerate}
$$
D=\left|\begin{array}{cccccc}
a_{1} & a_{2} & a_{3} & \cdots & a_{n-1} & a_{n} \\
a_{n} & a_{1} & a_{2} & \cdots & a_{n-2} & a_{n-1} \\
a_{n-1} & a_{n} & a_{1} & \cdots & a_{n-3} & a_{n-2} \\
\vdots & \vdots & \vdots & & \vdots & \vdots \\
a_{3} & a_{4} & a_{5} & \cdots & a_{1} & a_{2} \\
a_{2} & a_{3} & a_{4} & \cdots & a_{n} & a_{1}
\end{array}\right|=\prod_{i=0}^{n-1} f\left(\varepsilon_{i}\right)
$$

\begin{enumerate}
  \setcounter{enumi}{3}
  \item ( 20 \begin{CJK}{UTF8}{mj}分\end{CJK}) \begin{CJK}{UTF8}{mj}设\end{CJK} $A$ \begin{CJK}{UTF8}{mj}为\end{CJK} $n$ \begin{CJK}{UTF8}{mj}阶方阵\end{CJK}, \begin{CJK}{UTF8}{mj}满足\end{CJK} $A^{2}=A$. \begin{CJK}{UTF8}{mj}证明\end{CJK}:
\end{enumerate}
(1) $\mathrm{r}(A)+\mathrm{r}(A-E)=n$;

(2) $\mathrm{r}\left(A^{k}\right)+\mathrm{r}(A-E)^{l}=n$, \begin{CJK}{UTF8}{mj}这里\end{CJK} $k, l$ \begin{CJK}{UTF8}{mj}为任意自然数\end{CJK}.

\begin{enumerate}
  \setcounter{enumi}{4}
  \item (15 \begin{CJK}{UTF8}{mj}分\end{CJK}) \begin{CJK}{UTF8}{mj}设\end{CJK} $l_{i}=c_{i 1} x_{1}+c_{i 2} x_{2}+\cdots+c_{i n} x_{n}, i=1,2, \cdots, p+q$, \begin{CJK}{UTF8}{mj}这里\end{CJK} $c_{i j} \in \mathbb{R}$. \begin{CJK}{UTF8}{mj}试证明实二次型\end{CJK}:
\end{enumerate}
$$
f\left(x_{1}, x_{2}, \cdots, x_{n}\right)=l_{1}^{2}+l_{2}^{2}+\cdots+l_{p}^{2}-l_{p+1}^{2}-\cdots-l_{p+q}^{2}
$$
\begin{CJK}{UTF8}{mj}的正惯性指数\end{CJK} $\leq p$, \begin{CJK}{UTF8}{mj}负惯性指数\end{CJK} $\leq q$.

\begin{enumerate}
  \setcounter{enumi}{5}
  \item (15 \begin{CJK}{UTF8}{mj}分\end{CJK}) \begin{CJK}{UTF8}{mj}对\end{CJK} $\lambda$ \begin{CJK}{UTF8}{mj}的不同的值判断下述方程组是否有解\end{CJK}? \begin{CJK}{UTF8}{mj}当有解时求出全部的解\end{CJK}.
\end{enumerate}
$$
\left\{\begin{array}{l}
(3-2 \lambda) x_{1}+(2-\lambda) x_{2}+x_{3}=\lambda \\
(2-\lambda) x_{1}+(2-\lambda) x_{2}+x_{3}=1, \\
x_{1}+x_{2}+(2-\lambda) x_{3}=1 .
\end{array}\right.
$$

\begin{enumerate}
  \setcounter{enumi}{6}
  \item (15 \begin{CJK}{UTF8}{mj}分\end{CJK}) \begin{CJK}{UTF8}{mj}设矩阵\end{CJK}
\end{enumerate}
$$
A=\left(\begin{array}{cccc}
-2 & -1 & -1 & 0 \\
3 & 1 & 2 & 0 \\
0 & 1 & -1 & -1 \\
3 & 0 & 4 & 1
\end{array}\right)
$$
\begin{CJK}{UTF8}{mj}求\end{CJK} $A^{2016}$.

\begin{enumerate}
  \setcounter{enumi}{7}
  \item ( 20 \begin{CJK}{UTF8}{mj}分\end{CJK}) \begin{CJK}{UTF8}{mj}设\end{CJK} $V=\mathbb{R}_{n}[x]$, \begin{CJK}{UTF8}{mj}若\end{CJK} $f(x)=\sum_{i=0}^{n} a_{i} x^{i}, g(x)=\sum_{i=0}^{n} b_{i} x^{i} \in V$, \begin{CJK}{UTF8}{mj}规定内积\end{CJK} $(f(x), g(x))=\sum_{i=0}^{n} a_{i} b_{i}$, \begin{CJK}{UTF8}{mj}则\end{CJK} $V$ \begin{CJK}{UTF8}{mj}为欧\end{CJK} \begin{CJK}{UTF8}{mj}式空间\end{CJK}. \begin{CJK}{UTF8}{mj}令\end{CJK}
\end{enumerate}
$$
W=\{f(x) \in V \mid f(1)=0\} .
$$
(1) \begin{CJK}{UTF8}{mj}证明\end{CJK} $W$ \begin{CJK}{UTF8}{mj}为\end{CJK} $V$ \begin{CJK}{UTF8}{mj}的子空间\end{CJK}, \begin{CJK}{UTF8}{mj}决定其维数并举出它的一组标准正交基\end{CJK};

(2) \begin{CJK}{UTF8}{mj}举出\end{CJK} $W^{\perp}$ \begin{CJK}{UTF8}{mj}的一组标准正交基\end{CJK}. 8. ( 20 \begin{CJK}{UTF8}{mj}分\end{CJK}) \begin{CJK}{UTF8}{mj}设\end{CJK} $\mathbb{R}^{2}$ \begin{CJK}{UTF8}{mj}中的线性变换\end{CJK} $\mathscr{A}$ \begin{CJK}{UTF8}{mj}在基\end{CJK} $\varepsilon_{1}=(1,2), \varepsilon_{2}=(2,1)$ \begin{CJK}{UTF8}{mj}下的矩阵为\end{CJK}
$$
\left(\begin{array}{ll}
1 & 2 \\
2 & 3
\end{array}\right)
$$
\begin{CJK}{UTF8}{mj}线性变换\end{CJK} $\mathscr{B}$ \begin{CJK}{UTF8}{mj}在基\end{CJK} $\eta_{1}=(1,1), \eta_{2}=(1,2)$ \begin{CJK}{UTF8}{mj}下的矩阵为\end{CJK}
$$
\left(\begin{array}{ll}
3 & 3 \\
2 & 4
\end{array}\right) \text {. }
$$
(1) \begin{CJK}{UTF8}{mj}求\end{CJK} $\mathscr{A}+\mathscr{B}$ \begin{CJK}{UTF8}{mj}在基\end{CJK} $\eta_{1}, \eta_{2}$ \begin{CJK}{UTF8}{mj}下的矩阵\end{CJK};

(2) \begin{CJK}{UTF8}{mj}求\end{CJK} $\mathscr{A} \mathscr{B}$ \begin{CJK}{UTF8}{mj}在基\end{CJK} $\varepsilon_{1}, \varepsilon_{2}$ \begin{CJK}{UTF8}{mj}下的矩阵\end{CJK};

(3) \begin{CJK}{UTF8}{mj}设\end{CJK} $\alpha=(3,3)$, \begin{CJK}{UTF8}{mj}求\end{CJK} $\mathscr{A} \alpha$ \begin{CJK}{UTF8}{mj}在基\end{CJK} $\varepsilon_{1}, \varepsilon_{2}$ \begin{CJK}{UTF8}{mj}下的坐标\end{CJK};

(4) \begin{CJK}{UTF8}{mj}求\end{CJK} $\mathscr{B} \alpha$ \begin{CJK}{UTF8}{mj}在基\end{CJK} $\eta_{1}, \eta_{2}$ \begin{CJK}{UTF8}{mj}下的坐标\end{CJK}.

\begin{enumerate}
  \setcounter{enumi}{9}
  \item (15 \begin{CJK}{UTF8}{mj}分\end{CJK}) \begin{CJK}{UTF8}{mj}设\end{CJK} $A, B$ \begin{CJK}{UTF8}{mj}均为\end{CJK} $n$ \begin{CJK}{UTF8}{mj}阶正交矩阵\end{CJK}, \begin{CJK}{UTF8}{mj}满足\end{CJK} $|A|+|B|=0$. \begin{CJK}{UTF8}{mj}证明\end{CJK}:
\end{enumerate}
$$
|A+B|=0 .
$$

\section{9. 华南理工大学 2017 年研究生入学考试试题高等代数 
 李扬 
 微信公众号: sxkyliyang}
\begin{enumerate}
  \item ( 15 \begin{CJK}{UTF8}{mj}分\end{CJK}) \begin{CJK}{UTF8}{mj}设\end{CJK} $g(x), h(x)$ \begin{CJK}{UTF8}{mj}是数域\end{CJK} $P$ \begin{CJK}{UTF8}{mj}上的多项式\end{CJK}, $\partial(g(x))=m, \partial(h(x))=n$, \begin{CJK}{UTF8}{mj}且\end{CJK} $(g(x), h(x))=1$, \begin{CJK}{UTF8}{mj}又设\end{CJK} $f(x)$ \begin{CJK}{UTF8}{mj}是\end{CJK} $P$ \begin{CJK}{UTF8}{mj}上的任一个次数\end{CJK} $<n+m$ \begin{CJK}{UTF8}{mj}的多项式\end{CJK}. \begin{CJK}{UTF8}{mj}证明\end{CJK}: \begin{CJK}{UTF8}{mj}存在\end{CJK} $r(x), s(x) \in P[x]$ \begin{CJK}{UTF8}{mj}使得\end{CJK}
\end{enumerate}
$$
f(x)=r(x) g(x)+s(x) h(x),
$$
\begin{CJK}{UTF8}{mj}其中\end{CJK} $r(x)=0$ \begin{CJK}{UTF8}{mj}或者\end{CJK} $\partial(r(x))<n, \partial(s(x))<m$.

\begin{enumerate}
  \setcounter{enumi}{2}
  \item ( 20 \begin{CJK}{UTF8}{mj}分\end{CJK}) \begin{CJK}{UTF8}{mj}设三元非齐次线性方程组\end{CJK} $A X=b$ \begin{CJK}{UTF8}{mj}的系数矩阵的秩\end{CJK} $r(A)=1$, \begin{CJK}{UTF8}{mj}这里\end{CJK} $A$ \begin{CJK}{UTF8}{mj}为三阶方阵\end{CJK}, $X=$ $\left(x_{1}, x_{2}, x_{3}\right)^{\prime}, b=\left(b_{1}, b_{2}, b_{3}\right)^{\prime} \neq 0$. \begin{CJK}{UTF8}{mj}已知\end{CJK} $\eta_{1}, \eta_{2}, \eta_{3}$ \begin{CJK}{UTF8}{mj}是\end{CJK} $A X=b$ \begin{CJK}{UTF8}{mj}的三个解向量\end{CJK}, $\eta_{1}+\eta_{2}=(1,2,3)^{\prime}, \eta_{2}+\eta_{3}=$ $(0,-1,1)^{\prime}, \eta_{3}+\eta_{1}=(1,0,-1)^{\prime}$, \begin{CJK}{UTF8}{mj}求该方程组的基础解系\end{CJK}.

  \item ( 20 \begin{CJK}{UTF8}{mj}分\end{CJK}) \begin{CJK}{UTF8}{mj}设\end{CJK} $A$ \begin{CJK}{UTF8}{mj}为\end{CJK} $n$ \begin{CJK}{UTF8}{mj}阶方阵\end{CJK}, $A$ \begin{CJK}{UTF8}{mj}的\end{CJK} $(i, j)-$ \begin{CJK}{UTF8}{mj}元素\end{CJK} $a_{i j}=|i-j|$, \begin{CJK}{UTF8}{mj}求行列式\end{CJK} $|A|$ \begin{CJK}{UTF8}{mj}的值\end{CJK}.

  \item ( 20 \begin{CJK}{UTF8}{mj}分\end{CJK}) \begin{CJK}{UTF8}{mj}已知\end{CJK} $\alpha_{1}, \alpha_{2}, \alpha_{3}$ \begin{CJK}{UTF8}{mj}是三维欧氏空间\end{CJK} $V$ \begin{CJK}{UTF8}{mj}的一组基\end{CJK}, \begin{CJK}{UTF8}{mj}且这组基的度量矩阵为\end{CJK}

\end{enumerate}
$$
A=\left(\begin{array}{ccc}
1 & -1 & 1 \\
-1 & 2 & 0 \\
1 & 0 & 4
\end{array}\right)
$$
\begin{CJK}{UTF8}{mj}求\end{CJK} $V$ \begin{CJK}{UTF8}{mj}的一组标准正交基\end{CJK} (\begin{CJK}{UTF8}{mj}用\end{CJK} $\alpha_{1}, \alpha_{2}, \alpha_{3}$ \begin{CJK}{UTF8}{mj}表示出来\end{CJK}).

\begin{enumerate}
  \setcounter{enumi}{5}
  \item ( 20 \begin{CJK}{UTF8}{mj}分\end{CJK}) \begin{CJK}{UTF8}{mj}设\end{CJK} $V$ \begin{CJK}{UTF8}{mj}是数域\end{CJK} $P$ \begin{CJK}{UTF8}{mj}上的\end{CJK} $n$ \begin{CJK}{UTF8}{mj}维线性空间\end{CJK}, $V_{1}$ \begin{CJK}{UTF8}{mj}是\end{CJK} $V$ \begin{CJK}{UTF8}{mj}的子空间且\end{CJK} $\operatorname{dim} V_{1} \geq \frac{n}{2}$.
\end{enumerate}
(1) \begin{CJK}{UTF8}{mj}证明\end{CJK}: \begin{CJK}{UTF8}{mj}存在\end{CJK} $V$ \begin{CJK}{UTF8}{mj}的子空间\end{CJK} $W_{1}, W_{2}$ \begin{CJK}{UTF8}{mj}使得\end{CJK}
$$
V=V_{1} \oplus W_{1}=V_{1} \oplus W_{2},
$$
\begin{CJK}{UTF8}{mj}而\end{CJK} $W_{1} \cap W_{2}=\{0\}$;

(2) \begin{CJK}{UTF8}{mj}问\end{CJK}: \begin{CJK}{UTF8}{mj}当\end{CJK} $\operatorname{dim} V_{1}<\frac{n}{2}$ \begin{CJK}{UTF8}{mj}时\end{CJK}, \begin{CJK}{UTF8}{mj}上述结论是否成立\end{CJK}? \begin{CJK}{UTF8}{mj}为什么\end{CJK}?

\begin{enumerate}
  \setcounter{enumi}{6}
  \item ( 20 \begin{CJK}{UTF8}{mj}分\end{CJK}) \begin{CJK}{UTF8}{mj}设\end{CJK} $f(x), g(x)$ \begin{CJK}{UTF8}{mj}是数域\end{CJK} $P$ \begin{CJK}{UTF8}{mj}上的多项式\end{CJK}, $(f(x), g(x))=1, A$ \begin{CJK}{UTF8}{mj}是\end{CJK} $P$ \begin{CJK}{UTF8}{mj}上的\end{CJK} $n$ \begin{CJK}{UTF8}{mj}阶方阵\end{CJK}. \begin{CJK}{UTF8}{mj}证明\end{CJK}: $f(A) g(A)=0$ \begin{CJK}{UTF8}{mj}当\end{CJK} \begin{CJK}{UTF8}{mj}且仅当\end{CJK}
\end{enumerate}
$$
r(f(A))+r(g(A))=n
$$

\begin{enumerate}
  \setcounter{enumi}{7}
  \item ( 20 \begin{CJK}{UTF8}{mj}分\end{CJK}) \begin{CJK}{UTF8}{mj}设\end{CJK} $A$ \begin{CJK}{UTF8}{mj}是数域\end{CJK} $\mathbb{C}$ \begin{CJK}{UTF8}{mj}上的任一个\end{CJK} $n$ \begin{CJK}{UTF8}{mj}阶方阵\end{CJK}.
\end{enumerate}
(1) \begin{CJK}{UTF8}{mj}设\end{CJK} $\lambda_{1}$ \begin{CJK}{UTF8}{mj}是\end{CJK} $A$ \begin{CJK}{UTF8}{mj}的一个特征值\end{CJK}, \begin{CJK}{UTF8}{mj}证明\end{CJK}: \begin{CJK}{UTF8}{mj}存在\end{CJK} $\mathbb{C}$ \begin{CJK}{UTF8}{mj}上的\end{CJK} $n$ \begin{CJK}{UTF8}{mj}阶可逆方阵\end{CJK} $T$ \begin{CJK}{UTF8}{mj}使得\end{CJK}
$$
T^{-1} A T=\left(\begin{array}{cccc}
\lambda_{1} & b_{12} & \cdots & b_{1 n} \\
0 & b_{22} & \cdots & b_{2 n} \\
\vdots & \vdots & & \vdots \\
0 & b_{n 2} & \cdots & b_{n n}
\end{array}\right)
$$
(2) \begin{CJK}{UTF8}{mj}对\end{CJK} $n$ \begin{CJK}{UTF8}{mj}作归纳法证明\end{CJK}, $A$ \begin{CJK}{UTF8}{mj}必相似于一个上三角矩阵\end{CJK}:
$$
\left(\begin{array}{cccc}
\lambda_{1} & c_{12} & \cdots & c_{1 n} \\
& \lambda_{2} & \cdots & c_{2 n} \\
& & \cdots & \vdots \\
& & & \lambda_{n}
\end{array}\right)
$$

\begin{enumerate}
  \setcounter{enumi}{8}
  \item (15 \begin{CJK}{UTF8}{mj}分\end{CJK}) \begin{CJK}{UTF8}{mj}设\end{CJK} $\mathscr{A}$ \begin{CJK}{UTF8}{mj}是\end{CJK} $n$ \begin{CJK}{UTF8}{mj}维欧式空间\end{CJK} $V$ \begin{CJK}{UTF8}{mj}的对称变换\end{CJK}. \begin{CJK}{UTF8}{mj}证明\end{CJK}: \begin{CJK}{UTF8}{mj}对\end{CJK} $\forall \alpha \in V$ \begin{CJK}{UTF8}{mj}都有\end{CJK}
\end{enumerate}
$$
(\mathscr{A} \alpha, \alpha) \geq 0
$$
\begin{CJK}{UTF8}{mj}的充分必要条件是\end{CJK} $\mathscr{A}$ \begin{CJK}{UTF8}{mj}的特征值全是非负实数\end{CJK}.

\section{0. 华南理工大学 2009 年研究生入学考试试题数学分析}
\begin{CJK}{UTF8}{mj}李扬\end{CJK}

\begin{CJK}{UTF8}{mj}微信公众号\end{CJK}: sxkyliyang

\begin{enumerate}
  \item (10 \begin{CJK}{UTF8}{mj}分\end{CJK}) \begin{CJK}{UTF8}{mj}设函数\end{CJK}
\end{enumerate}
$$
f(x)=\varphi(a+b x)-\varphi(a-b x),
$$
\begin{CJK}{UTF8}{mj}其中\end{CJK} $\varphi(x)$ \begin{CJK}{UTF8}{mj}在\end{CJK} $x=a$ \begin{CJK}{UTF8}{mj}的某个小邻域内有定义且在该点处可导\end{CJK}, \begin{CJK}{UTF8}{mj}求\end{CJK} $f^{\prime}(0)$.

\begin{enumerate}
  \setcounter{enumi}{2}
  \item (10 \begin{CJK}{UTF8}{mj}分\end{CJK}) \begin{CJK}{UTF8}{mj}设\end{CJK} $0<x<y<\pi$, \begin{CJK}{UTF8}{mj}证明\end{CJK}:
\end{enumerate}
$$
y \sin y+2 \cos y+\pi y>x \sin x+2 \cos x+\pi x .
$$

\begin{enumerate}
  \setcounter{enumi}{3}
  \item (10 \begin{CJK}{UTF8}{mj}分\end{CJK}) \begin{CJK}{UTF8}{mj}设\end{CJK} $x>0, y>0$, \begin{CJK}{UTF8}{mj}求\end{CJK}
\end{enumerate}
$$
f(x, y)=x^{2} y(4-x-y)
$$
\begin{CJK}{UTF8}{mj}的极值\end{CJK}.

\begin{enumerate}
  \setcounter{enumi}{4}
  \item ( 10 \begin{CJK}{UTF8}{mj}分\end{CJK}) \begin{CJK}{UTF8}{mj}设\end{CJK}
\end{enumerate}
$$
f(x)=\frac{\int_{0}^{x} \mathrm{~d} u \int_{0}^{u^{2}} \arctan (1+t) \mathrm{d} t}{x(1-\cos x)}
$$
\begin{CJK}{UTF8}{mj}求\end{CJK} $\lim _{x \rightarrow 0} f(x)$.

\begin{enumerate}
  \setcounter{enumi}{5}
  \item (10 \begin{CJK}{UTF8}{mj}分\end{CJK}) \begin{CJK}{UTF8}{mj}计算\end{CJK}
\end{enumerate}
$$
\oint_{c} x \mathrm{~d} y-y \mathrm{~d} x,
$$
\begin{CJK}{UTF8}{mj}其中\end{CJK} $c$ \begin{CJK}{UTF8}{mj}为椭圆\end{CJK} $(x+2 y)^{2}+(3 x+2 y)^{2}=1$ \begin{CJK}{UTF8}{mj}方向为逆时针方向\end{CJK}.

\begin{enumerate}
  \setcounter{enumi}{6}
  \item (10 \begin{CJK}{UTF8}{mj}分\end{CJK}) \begin{CJK}{UTF8}{mj}计算\end{CJK}
\end{enumerate}
$$
\iint_{S}(x-y) \mathrm{d} x \mathrm{~d} y+x(y-z) \mathrm{d} y \mathrm{~d} z,
$$
\begin{CJK}{UTF8}{mj}其中\end{CJK} $S$ \begin{CJK}{UTF8}{mj}为柱面\end{CJK} $x^{2}+y^{2}=1$ \begin{CJK}{UTF8}{mj}及平面\end{CJK} $z=0, z=3$ \begin{CJK}{UTF8}{mj}所围成的空间区域\end{CJK} $\Omega$ \begin{CJK}{UTF8}{mj}的整个边界曲面外侧\end{CJK}.

\begin{enumerate}
  \setcounter{enumi}{7}
  \item ( 15 \begin{CJK}{UTF8}{mj}分\end{CJK}) \begin{CJK}{UTF8}{mj}设\end{CJK}
\end{enumerate}
$$
f(x)=\sin \sqrt{x}
$$
\begin{CJK}{UTF8}{mj}判断\end{CJK} $f(x)$ \begin{CJK}{UTF8}{mj}在\end{CJK} $[0,+\infty)$ \begin{CJK}{UTF8}{mj}上是否一致连续\end{CJK}, \begin{CJK}{UTF8}{mj}并给出证明\end{CJK}.

\begin{enumerate}
  \setcounter{enumi}{8}
  \item (15 \begin{CJK}{UTF8}{mj}分\end{CJK}) \begin{CJK}{UTF8}{mj}计算积分\end{CJK}
\end{enumerate}
$$
I=\iint_{D} \min \left\{x^{2} y, 2\right\} \mathrm{d} x \mathrm{~d} y
$$
\begin{CJK}{UTF8}{mj}其中\end{CJK} $D=\{(x, y) \mid 0 \leq x \leq 4,0 \leq y \leq 3\}$.

\begin{enumerate}
  \setcounter{enumi}{9}
  \item ( 15 \begin{CJK}{UTF8}{mj}分\end{CJK}) \begin{CJK}{UTF8}{mj}计算积分\end{CJK}
\end{enumerate}
$$
I(y)=\int_{0}^{+\infty} \mathrm{e}^{-x^{2}} \sin 2 x y \mathrm{~d} x
$$

\begin{enumerate}
  \setcounter{enumi}{10}
  \item (15 \begin{CJK}{UTF8}{mj}分\end{CJK}) \begin{CJK}{UTF8}{mj}设\end{CJK}
\end{enumerate}
$$
f(x, y)= \begin{cases}\frac{x y^{2}}{x^{2}+y^{2}}, & x^{2}+y^{2} \neq 0 \\ 0, & x^{2}+y^{2}=0\end{cases}
$$
\begin{CJK}{UTF8}{mj}讨论以下性质\end{CJK}:

(1) $f(x, y)$ \begin{CJK}{UTF8}{mj}的连续性\end{CJK};

(2) $f_{x}, f_{y}$ \begin{CJK}{UTF8}{mj}的存在性及连续性\end{CJK};

(3) $f(x, y)$ \begin{CJK}{UTF8}{mj}的可微性\end{CJK}. 11. ( 15 \begin{CJK}{UTF8}{mj}分\end{CJK}) \begin{CJK}{UTF8}{mj}设\end{CJK} $x_{0}=\sqrt{6}, x_{n+1}=\sqrt{6+x_{n}}, n=0,1,2, \cdots$, \begin{CJK}{UTF8}{mj}判断级数\end{CJK}
$$
\sum_{n=0}^{\infty} \sqrt{3-x_{n}}
$$
\begin{CJK}{UTF8}{mj}的敛散性\end{CJK}.

\begin{enumerate}
  \setcounter{enumi}{12}
  \item ( 15 \begin{CJK}{UTF8}{mj}分\end{CJK}) \begin{CJK}{UTF8}{mj}设\end{CJK} $f(x)$ \begin{CJK}{UTF8}{mj}在\end{CJK} $(-\infty,+\infty)$ \begin{CJK}{UTF8}{mj}内有连续的一阶导数\end{CJK}, \begin{CJK}{UTF8}{mj}证明\end{CJK}:
\end{enumerate}
(1) \begin{CJK}{UTF8}{mj}若\end{CJK} $\lim _{|x| \rightarrow+\infty} f^{\prime}(x)=\alpha>0$, \begin{CJK}{UTF8}{mj}则方程\end{CJK} $f(x)=0$ \begin{CJK}{UTF8}{mj}在\end{CJK} $(-\infty,+\infty)$ \begin{CJK}{UTF8}{mj}内至少有一个实根\end{CJK};

(2) \begin{CJK}{UTF8}{mj}若\end{CJK} $\lim _{|x| \rightarrow+\infty} f^{\prime}(x)=0$, \begin{CJK}{UTF8}{mj}则方程\end{CJK} $f^{\prime}(x)=0$ \begin{CJK}{UTF8}{mj}在\end{CJK} $(-\infty,+\infty)$ \begin{CJK}{UTF8}{mj}内至少有一个实根\end{CJK}.

\section{1. 华南理工大学 2010 年研究生入学考试试题数学分析}
\begin{CJK}{UTF8}{mj}李扬\end{CJK}

\begin{CJK}{UTF8}{mj}微信公众号\end{CJK}: sxkyliyang

\begin{CJK}{UTF8}{mj}一\end{CJK}. \begin{CJK}{UTF8}{mj}求解下列各题\end{CJK} (\begin{CJK}{UTF8}{mj}每小题\end{CJK} 10 \begin{CJK}{UTF8}{mj}分\end{CJK}, \begin{CJK}{UTF8}{mj}共\end{CJK} 60 \begin{CJK}{UTF8}{mj}分\end{CJK})

\begin{enumerate}
  \item \begin{CJK}{UTF8}{mj}确定\end{CJK} $\alpha$ \begin{CJK}{UTF8}{mj}与\end{CJK} $\beta$, \begin{CJK}{UTF8}{mj}使\end{CJK}
\end{enumerate}
$$
\lim _{n \rightarrow \infty}\left(\sqrt{3 n^{2}+4 n-2}-\alpha n-\beta\right)=0 .
$$

\begin{enumerate}
  \setcounter{enumi}{2}
  \item \begin{CJK}{UTF8}{mj}讨论函数\end{CJK} $f(x), g(x)$ \begin{CJK}{UTF8}{mj}在\end{CJK} $x=0$ \begin{CJK}{UTF8}{mj}处的可导性\end{CJK}, \begin{CJK}{UTF8}{mj}其中\end{CJK}
\end{enumerate}
$$
f(x)=\left\{\begin{array}{ll}
-x, & x \text { 为无理数 } \\
x, & x \text { 为有理数 }
\end{array} ; g(x)=\left\{\begin{array}{ll}
-x^{2}, & x \text { 为无理数 } \\
x^{2}, & x \text { 为有理数 }
\end{array} .\right.\right.
$$

\begin{enumerate}
  \setcounter{enumi}{3}
  \item \begin{CJK}{UTF8}{mj}已知\end{CJK} $f(x)$ \begin{CJK}{UTF8}{mj}在\end{CJK} $[0,+\infty)$ \begin{CJK}{UTF8}{mj}上连续\end{CJK}, \begin{CJK}{UTF8}{mj}且满足\end{CJK} $0 \leq f(x) \leq x, x \in[0,+\infty)$. \begin{CJK}{UTF8}{mj}设\end{CJK}
\end{enumerate}
$$
a_{1} \geq 0, a_{n+1}=f\left(a_{n}\right), n=1,2, \cdots
$$
\begin{CJK}{UTF8}{mj}证明\end{CJK}:

(1) $\left\{a_{n}\right\}$ \begin{CJK}{UTF8}{mj}收敛\end{CJK};

(2) \begin{CJK}{UTF8}{mj}若\end{CJK} $\lim _{n \rightarrow \infty} a_{n}=l$, \begin{CJK}{UTF8}{mj}则\end{CJK} $f(l)=l$.

\begin{enumerate}
  \setcounter{enumi}{4}
  \item \begin{CJK}{UTF8}{mj}判断下面级数的收敛性\end{CJK}
\end{enumerate}
$$
\sum_{n=1}^{\infty} \frac{x^{n}}{(1+x)\left(1+x^{2}\right) \cdots\left(1+x^{n}\right)}, x \geq 0 .
$$

\begin{enumerate}
  \setcounter{enumi}{5}
  \item \begin{CJK}{UTF8}{mj}讨论函数\end{CJK}
\end{enumerate}
$$
f(x, y)=\left(1+\mathrm{e}^{y}\right) \cos x-y \mathrm{e}^{y}
$$
\begin{CJK}{UTF8}{mj}的极大值和极小值\end{CJK}.

\begin{enumerate}
  \setcounter{enumi}{6}
  \item \begin{CJK}{UTF8}{mj}计算\end{CJK}
\end{enumerate}
$$
\iint_{S} x^{3} \mathrm{~d} y \mathrm{~d} z+2 y^{3} \mathrm{~d} z \mathrm{~d} x+3 z^{3} \mathrm{~d} x \mathrm{~d} y,
$$
\begin{CJK}{UTF8}{mj}其中\end{CJK} $S$ \begin{CJK}{UTF8}{mj}为球面\end{CJK} $x^{2}+y^{2}+z^{2}=a^{2}$ \begin{CJK}{UTF8}{mj}的外侧\end{CJK}.

\begin{CJK}{UTF8}{mj}二\end{CJK}. (15 \begin{CJK}{UTF8}{mj}分\end{CJK}) \begin{CJK}{UTF8}{mj}设\end{CJK} $p$ \begin{CJK}{UTF8}{mj}为正常数\end{CJK}, \begin{CJK}{UTF8}{mj}函数\end{CJK} $f(x)=\cos \left(x^{p}\right)$, \begin{CJK}{UTF8}{mj}证明\end{CJK}: \begin{CJK}{UTF8}{mj}当\end{CJK} $0<p \leq 1$ \begin{CJK}{UTF8}{mj}时\end{CJK}, $f(x)$ \begin{CJK}{UTF8}{mj}在\end{CJK} $[0,+\infty)$ \begin{CJK}{UTF8}{mj}上一致连续\end{CJK}.

\begin{CJK}{UTF8}{mj}三\end{CJK}. (15 \begin{CJK}{UTF8}{mj}分\end{CJK}) \begin{CJK}{UTF8}{mj}证明\end{CJK}
$$
\int_{a}^{b} \mathrm{e}^{-x y} \mathrm{~d} y=\frac{\mathrm{e}^{-a x}-\mathrm{e}^{-b x}}{x}
$$
\begin{CJK}{UTF8}{mj}并计算积分\end{CJK}
$$
\int_{0}^{+\infty} \frac{\mathrm{e}^{-a x}-\mathrm{e}^{-b x}}{x} \mathrm{~d} x,(b>a>0)
$$
\begin{CJK}{UTF8}{mj}四\end{CJK}. ( 15 \begin{CJK}{UTF8}{mj}分\end{CJK} $)$ \begin{CJK}{UTF8}{mj}令\end{CJK}
$$
f(x, y)= \begin{cases}\frac{\ln (1+x y)}{x}, & x \neq 0 \\ 0, & x=0\end{cases}
$$
\begin{CJK}{UTF8}{mj}证明\end{CJK} $f(x, y)$ \begin{CJK}{UTF8}{mj}在其定义域上是连续的\end{CJK}.

\begin{CJK}{UTF8}{mj}五\end{CJK}. (15 \begin{CJK}{UTF8}{mj}分\end{CJK}) \begin{CJK}{UTF8}{mj}求积分\end{CJK}
$$
I=\iint_{D}\left(\sqrt{\frac{x-c}{a}}+\sqrt{\frac{y-c}{b}}\right) \mathrm{d} x \mathrm{~d} y
$$
\begin{CJK}{UTF8}{mj}其中\end{CJK} $D$ \begin{CJK}{UTF8}{mj}由曲线\end{CJK} $\sqrt{\frac{x-c}{a}}+\sqrt{\frac{y-c}{b}}=1$ \begin{CJK}{UTF8}{mj}和\end{CJK} $x=c, y=c$ \begin{CJK}{UTF8}{mj}所围成\end{CJK}, \begin{CJK}{UTF8}{mj}且\end{CJK} $a, b, c>0$. \begin{CJK}{UTF8}{mj}六\end{CJK}. ( 15 \begin{CJK}{UTF8}{mj}分\end{CJK}) \begin{CJK}{UTF8}{mj}设\end{CJK} $f$ \begin{CJK}{UTF8}{mj}为定义在\end{CJK} $(a,+\infty)$ \begin{CJK}{UTF8}{mj}上的函数\end{CJK}, \begin{CJK}{UTF8}{mj}在每一有限区间\end{CJK} $(a, b)$ \begin{CJK}{UTF8}{mj}上有界\end{CJK}, \begin{CJK}{UTF8}{mj}且\end{CJK} $\lim _{x \rightarrow+\infty}[f(x+1)-f(x)]=A$. \begin{CJK}{UTF8}{mj}证\end{CJK} \begin{CJK}{UTF8}{mj}明\end{CJK}:
$$
\lim _{x \rightarrow+\infty} \frac{f(x)}{x}=A
$$
\begin{CJK}{UTF8}{mj}七\end{CJK}. (15 \begin{CJK}{UTF8}{mj}分\end{CJK}) \begin{CJK}{UTF8}{mj}设\end{CJK} $f(x), g(x)$ \begin{CJK}{UTF8}{mj}在\end{CJK} $[a, b]$ \begin{CJK}{UTF8}{mj}连续\end{CJK}, \begin{CJK}{UTF8}{mj}证明\end{CJK}
$$
\lim _{\lambda(\Delta) \rightarrow 0} \sum_{i=1}^{n} f\left(\xi_{i}\right) g\left(\theta_{i}\right) \Delta x_{i}=\int_{a}^{b} f(x) g(x) \mathrm{d} x
$$
\begin{CJK}{UTF8}{mj}其中\end{CJK} $\Delta$ \begin{CJK}{UTF8}{mj}为\end{CJK} $[a, b]$ \begin{CJK}{UTF8}{mj}的任一分割\end{CJK}, $\Delta: a=x_{0}<x_{1}<\cdots<x_{n}=b, \xi_{i}, \theta_{i} \in\left[x_{i}, x_{i+1}\right], i=1,2, \cdots, n, \Delta x_{i}=$ $x_{i}-x_{i-1}$.

\section{2. 华南理工大学 2011 年研究生入学考试试题数学分析 
 李扬 
 微信公众号: sxkyliyang}
\begin{enumerate}
  \item (10 \begin{CJK}{UTF8}{mj}分\end{CJK}) \begin{CJK}{UTF8}{mj}设函数\end{CJK} $y=f(x)$ \begin{CJK}{UTF8}{mj}在\end{CJK} $x=x_{0}$ \begin{CJK}{UTF8}{mj}的某个邻域内有定义且在该点处可导\end{CJK}, \begin{CJK}{UTF8}{mj}定义函数\end{CJK}
\end{enumerate}
$$
\psi(t)=f^{2}\left(x_{0}+a t\right)-f^{2}\left(x_{0}-a t\right)
$$
\begin{CJK}{UTF8}{mj}其中\end{CJK} $a$ \begin{CJK}{UTF8}{mj}为常数\end{CJK}, \begin{CJK}{UTF8}{mj}求导数\end{CJK} $\psi^{\prime}(0)$.

\begin{enumerate}
  \setcounter{enumi}{2}
  \item ( 10 \begin{CJK}{UTF8}{mj}分\end{CJK}) \begin{CJK}{UTF8}{mj}设\end{CJK} $k$ \begin{CJK}{UTF8}{mj}为自然数\end{CJK}, \begin{CJK}{UTF8}{mj}求数列\end{CJK}
\end{enumerate}
$$
x_{n}=\sqrt[k]{n^{k-1}}(\sqrt[k]{n+1}-\sqrt[k]{n})
$$
\begin{CJK}{UTF8}{mj}的极限\end{CJK}.

\begin{enumerate}
  \setcounter{enumi}{3}
  \item (10 \begin{CJK}{UTF8}{mj}分\end{CJK}) \begin{CJK}{UTF8}{mj}计算\end{CJK}
\end{enumerate}
$$
\sum_{n=1}^{\infty}(-1)^{n+1} \frac{1}{n}=1-\frac{1}{2}+\frac{1}{3}-\frac{1}{4}+\cdots
$$

\begin{enumerate}
  \setcounter{enumi}{4}
  \item (10 \begin{CJK}{UTF8}{mj}分\end{CJK}) \begin{CJK}{UTF8}{mj}讨论函数\end{CJK}
\end{enumerate}
$$
w=\left(\frac{x^{2}}{a^{2}}+\frac{y^{2}}{b^{2}}+\frac{z^{2}}{c^{2}}\right) \mathrm{e}^{-\left(\frac{x^{2}}{a^{2}}+\frac{y^{2}}{b^{2}}+\frac{z^{2}}{c^{2}}\right)}
$$
\begin{CJK}{UTF8}{mj}的极值\end{CJK}.

\begin{enumerate}
  \setcounter{enumi}{5}
  \item (10 \begin{CJK}{UTF8}{mj}分\end{CJK}) \begin{CJK}{UTF8}{mj}求二重积分\end{CJK}
\end{enumerate}
$$
I=\iint_{D} \mathrm{e}^{\frac{x-y}{x+y}} \mathrm{~d} x \mathrm{~d} y
$$
\begin{CJK}{UTF8}{mj}其中\end{CJK} $D$ \begin{CJK}{UTF8}{mj}是由\end{CJK} $x=0, y=0, x+y=1$ \begin{CJK}{UTF8}{mj}所围成的区域\end{CJK}.

\begin{enumerate}
  \setcounter{enumi}{6}
  \item (10 \begin{CJK}{UTF8}{mj}分\end{CJK}) \begin{CJK}{UTF8}{mj}计算积分\end{CJK}
\end{enumerate}
$$
\iint_{S} x \mathrm{~d} y \mathrm{~d} z+z \mathrm{~d} x \mathrm{~d} y
$$
\begin{CJK}{UTF8}{mj}其中\end{CJK} $S$ \begin{CJK}{UTF8}{mj}为抛物面\end{CJK} $z=x^{2}+y^{2}-1$ \begin{CJK}{UTF8}{mj}在\end{CJK} $z \leq 0$ \begin{CJK}{UTF8}{mj}部分\end{CJK}, \begin{CJK}{UTF8}{mj}方向取下侧\end{CJK}.

\begin{enumerate}
  \setcounter{enumi}{7}
  \item (12 \begin{CJK}{UTF8}{mj}分\end{CJK}) \begin{CJK}{UTF8}{mj}判断如下广义积分的收敛性\end{CJK}:
\end{enumerate}
$$
\int_{-\infty}^{+\infty} \frac{\mathrm{d} x}{\left|x-a_{1}\right|^{p_{1}} \cdot\left|x-a_{2}\right|^{p_{2}} \cdots\left|x-a_{n}\right|^{p_{n}}},
$$
\begin{CJK}{UTF8}{mj}其中\end{CJK} $a_{1}<a_{2}<\cdots<a_{n}, 0<p_{i}(i=1,2, \cdots, n)$.

\begin{enumerate}
  \setcounter{enumi}{8}
  \item ( 18 \begin{CJK}{UTF8}{mj}分\end{CJK}) \begin{CJK}{UTF8}{mj}设\end{CJK} $p \geq 1$ \begin{CJK}{UTF8}{mj}为正常数\end{CJK}, \begin{CJK}{UTF8}{mj}试证\end{CJK}:
\end{enumerate}
(1) \begin{CJK}{UTF8}{mj}如果\end{CJK} $0<a<b$, \begin{CJK}{UTF8}{mj}则\end{CJK}
$$
b^{\frac{1}{p}}-a^{\frac{1}{p}} \leq(b-a)^{\frac{1}{p}}
$$
(2) \begin{CJK}{UTF8}{mj}函数\end{CJK}
$$
f(x)=\sin \left(x^{-p}\right)
$$
\begin{CJK}{UTF8}{mj}在区间\end{CJK} $[c, 1)(1>c>0)$ \begin{CJK}{UTF8}{mj}上一致连续\end{CJK};

(3) \begin{CJK}{UTF8}{mj}函数\end{CJK}
$$
f(x)=\sin \left(x^{-p}\right)
$$
\begin{CJK}{UTF8}{mj}在开区间\end{CJK} $(0,1)$ \begin{CJK}{UTF8}{mj}上不一致连续\end{CJK}. 9. ( 15 \begin{CJK}{UTF8}{mj}分\end{CJK}) \begin{CJK}{UTF8}{mj}已知\end{CJK}
$$
\left(1+\frac{1}{n}\right)^{n}<\mathrm{e}<\left(1+\frac{1}{n}\right)^{n+1}
$$
\begin{CJK}{UTF8}{mj}试证明极限\end{CJK}
$$
\lim _{n \rightarrow \infty}\left(1+\frac{1}{2}+\frac{1}{3}+\cdots+\frac{1}{n}-\ln n\right)
$$
\begin{CJK}{UTF8}{mj}存在\end{CJK}

\begin{enumerate}
  \setcounter{enumi}{10}
  \item (15 \begin{CJK}{UTF8}{mj}分\end{CJK}) \begin{CJK}{UTF8}{mj}设\end{CJK} $f(x)$ \begin{CJK}{UTF8}{mj}在\end{CJK} $[a,+\infty)$ \begin{CJK}{UTF8}{mj}上连续\end{CJK}, \begin{CJK}{UTF8}{mj}且\end{CJK} $\int_{a}^{+\infty} f(x) \mathrm{d} x$ \begin{CJK}{UTF8}{mj}收玫\end{CJK}, \begin{CJK}{UTF8}{mj}则存在数列\end{CJK} $\left\{x_{n}\right\} \subset[a,+\infty)$, \begin{CJK}{UTF8}{mj}满足条件\end{CJK}
\end{enumerate}
$$
\lim _{n \rightarrow+\infty} x_{n}=+\infty, \lim _{n \rightarrow+\infty} f\left(x_{n}\right)=0 .
$$

\begin{enumerate}
  \setcounter{enumi}{11}
  \item (15 \begin{CJK}{UTF8}{mj}分\end{CJK}) \begin{CJK}{UTF8}{mj}计算积分\end{CJK}
\end{enumerate}
$$
\oint_{L} \frac{y \mathrm{~d} x-x \mathrm{~d} y}{x^{2}+2 y^{2}},
$$
\begin{CJK}{UTF8}{mj}其中\end{CJK} $L$ \begin{CJK}{UTF8}{mj}为以\end{CJK} $\left(0, \frac{1}{2}\right)$ \begin{CJK}{UTF8}{mj}为圆心\end{CJK}, \begin{CJK}{UTF8}{mj}以\end{CJK} $a$ \begin{CJK}{UTF8}{mj}为半径的圆周\end{CJK} $\left(a>0\right.$ \begin{CJK}{UTF8}{mj}且\end{CJK} $\left.a \neq \frac{1}{2}\right)$, \begin{CJK}{UTF8}{mj}顺时针方向\end{CJK}.

\begin{enumerate}
  \setcounter{enumi}{12}
  \item (15 \begin{CJK}{UTF8}{mj}分\end{CJK}) \begin{CJK}{UTF8}{mj}证明椭圆\end{CJK} $\frac{x^{2}}{a^{2}}+\frac{y^{2}}{b^{2}}=1(a>0, b>0)$ \begin{CJK}{UTF8}{mj}的周长为\end{CJK}
\end{enumerate}
$$
s=4 \int_{0}^{\frac{\pi}{2}} \sqrt{a^{2} \sin ^{2} t+b^{2} \cos ^{2} t} \mathrm{~d} t
$$
\begin{CJK}{UTF8}{mj}且\end{CJK}
$$
\pi(a+b) \leq s \leq \pi \sqrt{2 a^{2}+2 b^{2}}
$$

\section{3. 华南理工大学 2012 年研究生入学考试试题数学分析 
 李扬 
 微信公众号: sxkyliyang}
\begin{enumerate}
  \item (10 \begin{CJK}{UTF8}{mj}分\end{CJK}) \begin{CJK}{UTF8}{mj}设\end{CJK} $f(x)$ \begin{CJK}{UTF8}{mj}在\end{CJK} $[a, b]$ \begin{CJK}{UTF8}{mj}上连续可导且\end{CJK} $f(b)-f(a)=b-a$, \begin{CJK}{UTF8}{mj}证明\end{CJK}:
\end{enumerate}
$$
\int_{a}^{b}\left(f^{\prime}(x)\right)^{2} \mathrm{~d} x \geq b-a .
$$

\begin{enumerate}
  \setcounter{enumi}{2}
  \item (10 \begin{CJK}{UTF8}{mj}分\end{CJK}) \begin{CJK}{UTF8}{mj}求抛物线\end{CJK} $y=x^{2}$ \begin{CJK}{UTF8}{mj}和直线\end{CJK} $x-y-2=0$ \begin{CJK}{UTF8}{mj}之间的最短距离\end{CJK}.

  \item (10 \begin{CJK}{UTF8}{mj}分\end{CJK}) \begin{CJK}{UTF8}{mj}求\end{CJK}

\end{enumerate}
$$
\sum_{n=2}^{\infty}(-1)^{n} \frac{x^{n}}{n(n-1)}
$$
\begin{CJK}{UTF8}{mj}的收敛域及和函数\end{CJK}.

\begin{enumerate}
  \setcounter{enumi}{4}
  \item (10 \begin{CJK}{UTF8}{mj}分\end{CJK}) \begin{CJK}{UTF8}{mj}计算积分\end{CJK}
\end{enumerate}
$$
I=\iint_{D} \sqrt{\frac{x^{2}+y^{2}}{4 a^{2}-x^{2}-y^{2}}} \mathrm{~d} x \mathrm{~d} y,
$$
\begin{CJK}{UTF8}{mj}其中\end{CJK} $D$ \begin{CJK}{UTF8}{mj}是由曲线\end{CJK} $x^{2}+y^{2}+2 a y=0$ \begin{CJK}{UTF8}{mj}与直线\end{CJK} $y=-x$ \begin{CJK}{UTF8}{mj}所围的区域\end{CJK} $(y>-x)$.

\begin{enumerate}
  \setcounter{enumi}{5}
  \item (10 \begin{CJK}{UTF8}{mj}分\end{CJK}) \begin{CJK}{UTF8}{mj}计算积分\end{CJK}
\end{enumerate}
$$
\int_{L} z \mathrm{~d} x+2 x \mathrm{~d} y+3 y \mathrm{~d} z
$$
\begin{CJK}{UTF8}{mj}其中\end{CJK} $L$ \begin{CJK}{UTF8}{mj}是圆柱面\end{CJK} $x^{2}+y^{2}=a^{2}(a>0)$ \begin{CJK}{UTF8}{mj}与平面\end{CJK} $z=x+1$ \begin{CJK}{UTF8}{mj}的交线\end{CJK}, \begin{CJK}{UTF8}{mj}从\end{CJK} $z$ \begin{CJK}{UTF8}{mj}轴正向看是顺时针方向\end{CJK}.

\begin{enumerate}
  \setcounter{enumi}{6}
  \item (10 \begin{CJK}{UTF8}{mj}分\end{CJK}) \begin{CJK}{UTF8}{mj}计算积分\end{CJK}
\end{enumerate}
$$
\iint_{S}(y+z)^{2} \mathrm{~d} S
$$
\begin{CJK}{UTF8}{mj}其中\end{CJK} $S$ \begin{CJK}{UTF8}{mj}为区域\end{CJK} $\left\{(x, y, z) \mid \sqrt{y^{2}+z^{2}} \leq x \leq 2\right\}$ \begin{CJK}{UTF8}{mj}的边界\end{CJK}.

\begin{enumerate}
  \setcounter{enumi}{7}
  \item (15 \begin{CJK}{UTF8}{mj}分\end{CJK}) \begin{CJK}{UTF8}{mj}证明\end{CJK}
\end{enumerate}
$$
f(x)=\frac{1}{x^{2}}
$$
\begin{CJK}{UTF8}{mj}在\end{CJK} $(0,+\infty)$ \begin{CJK}{UTF8}{mj}上连续但不一致连续\end{CJK}.

\begin{enumerate}
  \setcounter{enumi}{8}
  \item (15 \begin{CJK}{UTF8}{mj}分\end{CJK}) \begin{CJK}{UTF8}{mj}设\end{CJK}
\end{enumerate}
$$
f(x, y)=\left\{\begin{array}{ll}
\frac{x^{2} y}{x^{2}+y^{2}}, & \text { 其它 } \\
0, & x=y=0
\end{array} .\right.
$$
\begin{CJK}{UTF8}{mj}研究\end{CJK} $f(x, y)$ \begin{CJK}{UTF8}{mj}的连续性\end{CJK}, \begin{CJK}{UTF8}{mj}一阶偏导的存在性以及可微性\end{CJK}.

\begin{enumerate}
  \setcounter{enumi}{9}
  \item (15 \begin{CJK}{UTF8}{mj}分\end{CJK}) \begin{CJK}{UTF8}{mj}设数列\end{CJK} $\left\{x_{n}\right\}$ \begin{CJK}{UTF8}{mj}满足条件\end{CJK}:
\end{enumerate}
(i) $\left\{x_{2 n}\right\}$ \begin{CJK}{UTF8}{mj}单调递增\end{CJK}, $\left\{x_{2 n+1}\right\}$ \begin{CJK}{UTF8}{mj}单调递减\end{CJK};

(ii) $\lim _{n \rightarrow+\infty}\left(x_{n+1}-x_{n}\right)=0$.

\begin{CJK}{UTF8}{mj}证明\end{CJK} $\left\{x_{n}\right\}$ \begin{CJK}{UTF8}{mj}收敛\end{CJK}.

\begin{enumerate}
  \setcounter{enumi}{10}
  \item (15 \begin{CJK}{UTF8}{mj}分\end{CJK}) \begin{CJK}{UTF8}{mj}设级数\end{CJK} $\sum_{n=1}^{\infty} a_{n}$ \begin{CJK}{UTF8}{mj}收敛\end{CJK}, $\sum_{n=1}^{\infty}\left(b_{n+1}-b_{n}\right)$ \begin{CJK}{UTF8}{mj}绝对收敛\end{CJK}. \begin{CJK}{UTF8}{mj}证明级数\end{CJK}
\end{enumerate}
$$
\sum_{n=1}^{\infty} a_{n} b_{n}
$$
\begin{CJK}{UTF8}{mj}也收敛\end{CJK}. 11. (15 \begin{CJK}{UTF8}{mj}分\end{CJK}) \begin{CJK}{UTF8}{mj}证明\end{CJK}
$$
\int_{0}^{+\infty} \frac{x}{\mathrm{e}^{x-1}} \mathrm{~d} x=\sum_{n=1}^{\infty} \frac{1}{n^{2}}=\frac{\pi^{2}}{6}
$$

\begin{enumerate}
  \setcounter{enumi}{12}
  \item ( 15 \begin{CJK}{UTF8}{mj}分\end{CJK}) \begin{CJK}{UTF8}{mj}设\end{CJK} $f_{n}(x)$ \begin{CJK}{UTF8}{mj}在\end{CJK} $[a, b]$ \begin{CJK}{UTF8}{mj}上可积\end{CJK} $(n=1,2, \cdots)$, \begin{CJK}{UTF8}{mj}且在\end{CJK} $[a, b]$ \begin{CJK}{UTF8}{mj}上\end{CJK} $f_{n}(x)$ \begin{CJK}{UTF8}{mj}一致收敛于\end{CJK} $f(x)$. \begin{CJK}{UTF8}{mj}证明\end{CJK} $f(x)$ \begin{CJK}{UTF8}{mj}在\end{CJK} $[a, b]$ \begin{CJK}{UTF8}{mj}上\end{CJK} \begin{CJK}{UTF8}{mj}可积\end{CJK}.
\end{enumerate}
\section{4. 华南理工大学 2013 年研究生入学考试试题数学分析}
\begin{CJK}{UTF8}{mj}李扬\end{CJK}

\begin{CJK}{UTF8}{mj}微信公众号\end{CJK}: sxkyliyang

\begin{enumerate}
  \item (12 \begin{CJK}{UTF8}{mj}分\end{CJK}) \begin{CJK}{UTF8}{mj}求极限\end{CJK}
\end{enumerate}
$$
\lim _{n \rightarrow \infty} \sqrt{n}\left(\sqrt[4]{n^{2}+1}-\sqrt{n+1}\right)
$$

\begin{enumerate}
  \setcounter{enumi}{2}
  \item (12 \begin{CJK}{UTF8}{mj}分\end{CJK}) \begin{CJK}{UTF8}{mj}确定函数项级数\end{CJK}
\end{enumerate}
$$
\sum_{n=1}^{\infty} \frac{x^{2}}{n}
$$
\begin{CJK}{UTF8}{mj}的收敛域\end{CJK}, \begin{CJK}{UTF8}{mj}并求其和函数\end{CJK}.

\begin{enumerate}
  \setcounter{enumi}{3}
  \item (12 \begin{CJK}{UTF8}{mj}分\end{CJK}) \begin{CJK}{UTF8}{mj}设函数\end{CJK} $f \in C^{2}(\mathbb{R})$, \begin{CJK}{UTF8}{mj}且\end{CJK}
\end{enumerate}
$$
f(x+h)+f(x-h)-2 f(x) \leq 0, \forall x \in \mathbb{R}, \forall h>0,
$$
\begin{CJK}{UTF8}{mj}证明\end{CJK}: \begin{CJK}{UTF8}{mj}对\end{CJK} $\forall x \in \mathbb{R}$, \begin{CJK}{UTF8}{mj}有\end{CJK} $f^{\prime \prime}(x) \leq 0$.

\begin{enumerate}
  \setcounter{enumi}{4}
  \item (12 \begin{CJK}{UTF8}{mj}分\end{CJK}) \begin{CJK}{UTF8}{mj}设\end{CJK} $\beta>0$, \begin{CJK}{UTF8}{mj}且\end{CJK}
\end{enumerate}
$$
x_{1}=\frac{1}{2}\left(2+\frac{\beta}{2}\right), x_{n+1}=\frac{1}{2}\left(x_{n}+\frac{\beta}{x_{n}}\right), n=1,2,3, \cdots .
$$
\begin{CJK}{UTF8}{mj}试证数列\end{CJK} $\left\{x_{n}\right\}$ \begin{CJK}{UTF8}{mj}收敛\end{CJK}, \begin{CJK}{UTF8}{mj}并求其极限\end{CJK}.

\begin{enumerate}
  \setcounter{enumi}{5}
  \item (12 \begin{CJK}{UTF8}{mj}分\end{CJK}) \begin{CJK}{UTF8}{mj}求极限\end{CJK}
\end{enumerate}
$$
\lim _{n \rightarrow \infty} \int_{-\frac{\pi}{2}}^{0} \cos ^{n} x \mathrm{~d} x
$$

\begin{enumerate}
  \setcounter{enumi}{6}
  \item (12 \begin{CJK}{UTF8}{mj}分\end{CJK}) \begin{CJK}{UTF8}{mj}求极限\end{CJK}
\end{enumerate}
$$
\lim _{x \rightarrow 0+0} \frac{\sin \sqrt{x}}{\sqrt{1+x \tan x-\sqrt{\cos x}}} .
$$

\begin{enumerate}
  \setcounter{enumi}{7}
  \item (13 \begin{CJK}{UTF8}{mj}分\end{CJK}) \begin{CJK}{UTF8}{mj}设函数\end{CJK} $g(x, y)$ \begin{CJK}{UTF8}{mj}在\end{CJK} $(0,0)$ \begin{CJK}{UTF8}{mj}点可微且在该点的函数值及微分均为零\end{CJK}, \begin{CJK}{UTF8}{mj}定义函数\end{CJK}
\end{enumerate}
$$
f(x, y)= \begin{cases}g(x, y) \sin \frac{1}{x^{2}+y^{2}}, & x^{2}+y^{2} \neq 0 \\ 0, & x^{2}+y^{2}=0\end{cases}
$$
\begin{CJK}{UTF8}{mj}试证\end{CJK} $f(x, y)$ \begin{CJK}{UTF8}{mj}在\end{CJK} $(0,0)$ \begin{CJK}{UTF8}{mj}点可微\end{CJK}.

\begin{enumerate}
  \setcounter{enumi}{8}
  \item (13 \begin{CJK}{UTF8}{mj}分\end{CJK}) \begin{CJK}{UTF8}{mj}计算曲面积分\end{CJK}
\end{enumerate}
$$
\iint_{S} y \mathrm{~d} x \mathrm{~d} z,
$$
\begin{CJK}{UTF8}{mj}其中\end{CJK} $S$ \begin{CJK}{UTF8}{mj}是球面\end{CJK} $x^{2}+y^{2}+z^{2}=1$ \begin{CJK}{UTF8}{mj}的上半部分\end{CJK}, \begin{CJK}{UTF8}{mj}并取外侧为正向\end{CJK}.

\begin{enumerate}
  \setcounter{enumi}{9}
  \item (13 \begin{CJK}{UTF8}{mj}分\end{CJK}) \begin{CJK}{UTF8}{mj}计算曲线积分\end{CJK}
\end{enumerate}
$$
\int_{C} \frac{x \mathrm{~d} y-y \mathrm{~d} x}{x^{2}+y^{2}},
$$
\begin{CJK}{UTF8}{mj}其中\end{CJK} $C$ \begin{CJK}{UTF8}{mj}是以\end{CJK} $(0,1)$ \begin{CJK}{UTF8}{mj}为圆心\end{CJK}, $R(R \neq 1)$ \begin{CJK}{UTF8}{mj}为半径的圆周\end{CJK}, \begin{CJK}{UTF8}{mj}方向为逆时针方向\end{CJK}.

\begin{enumerate}
  \setcounter{enumi}{10}
  \item (13 \begin{CJK}{UTF8}{mj}分\end{CJK}) \begin{CJK}{UTF8}{mj}计算曲面积分\end{CJK}
\end{enumerate}
$$
\iint_{\Sigma} \frac{\mathrm{d} S}{\sqrt{x^{2}+y^{2}+(z+2)^{2}}}
$$
\begin{CJK}{UTF8}{mj}其中\end{CJK} $\Sigma$ \begin{CJK}{UTF8}{mj}是以原点\end{CJK} $(0,0,0)$ \begin{CJK}{UTF8}{mj}为球心\end{CJK}, 2 \begin{CJK}{UTF8}{mj}为半径的球面被\end{CJK} $z=\sqrt{x^{2}+y^{2}}$ \begin{CJK}{UTF8}{mj}所截的上部分\end{CJK}. 11. (13 \begin{CJK}{UTF8}{mj}分\end{CJK}) \begin{CJK}{UTF8}{mj}设函数\end{CJK} $f(x)$ \begin{CJK}{UTF8}{mj}在\end{CJK} $[a, b]$ \begin{CJK}{UTF8}{mj}上连续\end{CJK}, \begin{CJK}{UTF8}{mj}且有唯一的最大值点\end{CJK} $x_{0} \in[a, b]$. \begin{CJK}{UTF8}{mj}试证\end{CJK}: \begin{CJK}{UTF8}{mj}若\end{CJK} $x_{n} \in[a, b]$ \begin{CJK}{UTF8}{mj}且\end{CJK} $\lim _{n \rightarrow \infty} f\left(x_{n}\right)=$ $f\left(x_{0}\right)$, \begin{CJK}{UTF8}{mj}则\end{CJK}
$$
\lim _{n \rightarrow \infty} x_{n}=x_{0} .
$$

\begin{enumerate}
  \setcounter{enumi}{12}
  \item (13 \begin{CJK}{UTF8}{mj}分\end{CJK}) \begin{CJK}{UTF8}{mj}设函数\end{CJK} $f(x, y)$ \begin{CJK}{UTF8}{mj}在闭区域\end{CJK} $\left|x-x_{0}\right| \leq a,\left|y-y_{0}\right| \leq b$ \begin{CJK}{UTF8}{mj}上连续\end{CJK}, \begin{CJK}{UTF8}{mj}函数列\end{CJK} $\left\{\varphi_{n}(x)\right\}$ \begin{CJK}{UTF8}{mj}在闭区间\end{CJK} $\left[x_{0}-a, x_{0}+a\right]$ \begin{CJK}{UTF8}{mj}上一致收敛于函数\end{CJK} $\varphi(x)$, \begin{CJK}{UTF8}{mj}且对任意的\end{CJK} $n$ \begin{CJK}{UTF8}{mj}及\end{CJK} $\forall x \in\left[x_{0}-a, x_{0}+a\right]$ \begin{CJK}{UTF8}{mj}有\end{CJK} $\left|\varphi_{n}(x)-y_{0}\right| \leq b$. \begin{CJK}{UTF8}{mj}试证\end{CJK}:
\end{enumerate}
$$
\lim _{n \rightarrow \infty} \int_{x_{0}}^{x} f\left(t, \varphi_{n}(t)\right) \mathrm{d} t=\int_{x_{0}}^{x} f(t, \varphi(t)) \mathrm{d} t,
$$
\begin{CJK}{UTF8}{mj}这里\end{CJK} $\left|x-x_{0}\right| \leq a$.

\section{5. 华南理工大学 2014 年研究生入学考试试题数学分析}
\begin{CJK}{UTF8}{mj}李扬\end{CJK}

\begin{CJK}{UTF8}{mj}微信公众号\end{CJK}: sxkyliyang

\begin{enumerate}
  \item (12 \begin{CJK}{UTF8}{mj}分\end{CJK}) \begin{CJK}{UTF8}{mj}求极限\end{CJK}
\end{enumerate}
$$
\lim _{x \rightarrow 0} \frac{\sin 2 x^{2}-x \tan x}{x^{2} \mathrm{e}^{x}-3 \cos 2 x+3} .
$$

\begin{enumerate}
  \setcounter{enumi}{2}
  \item (12 \begin{CJK}{UTF8}{mj}分\end{CJK}) \begin{CJK}{UTF8}{mj}求极限\end{CJK}
\end{enumerate}
$$
\lim _{n \rightarrow \infty} \frac{n}{\sqrt[n]{(n+1) !}}
$$

\begin{enumerate}
  \setcounter{enumi}{3}
  \item (12 \begin{CJK}{UTF8}{mj}分\end{CJK}) \begin{CJK}{UTF8}{mj}证明不等式\end{CJK}:
\end{enumerate}
$$
x-\frac{x^{2}}{2}<\ln (1+x)<x,(x>0)
$$

\begin{enumerate}
  \setcounter{enumi}{4}
  \item (12 \begin{CJK}{UTF8}{mj}分\end{CJK}) \begin{CJK}{UTF8}{mj}求定积分\end{CJK}
\end{enumerate}
$$
\int_{0}^{1}\left[\frac{x(1+x)}{\sqrt{1+2 x}}+\ln x\right] \mathrm{d} x .
$$

\begin{enumerate}
  \setcounter{enumi}{5}
  \item (12 \begin{CJK}{UTF8}{mj}分\end{CJK}) \begin{CJK}{UTF8}{mj}求\end{CJK}
\end{enumerate}
$$
\sum_{n=1}^{\infty}\left(\frac{n+1}{n}\right)^{n^{2}} x^{n}
$$
\begin{CJK}{UTF8}{mj}的收敛区间\end{CJK}, \begin{CJK}{UTF8}{mj}并说明其在收敛区间端点的敛散性\end{CJK}.

\begin{enumerate}
  \setcounter{enumi}{6}
  \item (12 \begin{CJK}{UTF8}{mj}分\end{CJK}) \begin{CJK}{UTF8}{mj}求曲面积分\end{CJK}
\end{enumerate}
$$
\iint_{S} z \mathrm{~d} x \mathrm{~d} y
$$
\begin{CJK}{UTF8}{mj}其中\end{CJK} $S$ \begin{CJK}{UTF8}{mj}是曲面\end{CJK} $\frac{x^{2}}{a^{2}}+\frac{y^{2}}{b^{2}}+\frac{z^{2}}{c^{2}}=1$, \begin{CJK}{UTF8}{mj}方向为外侧\end{CJK}.

\begin{enumerate}
  \setcounter{enumi}{7}
  \item (13 \begin{CJK}{UTF8}{mj}分\end{CJK}) \begin{CJK}{UTF8}{mj}设\end{CJK}
\end{enumerate}
$$
x_{1}=a>0, x_{n}=\sqrt{3+x_{n-1}}, n=2,3, \cdots
$$
\begin{CJK}{UTF8}{mj}试证\end{CJK} $\left\{x_{n}\right\}$ \begin{CJK}{UTF8}{mj}收敛\end{CJK}, \begin{CJK}{UTF8}{mj}并求其极限\end{CJK}.

\begin{enumerate}
  \setcounter{enumi}{8}
  \item (13 \begin{CJK}{UTF8}{mj}分\end{CJK}) \begin{CJK}{UTF8}{mj}求级数\end{CJK}
\end{enumerate}
$$
\sum_{n=0}^{\infty} \frac{1}{2 n+1}\left(\frac{1}{2}\right)^{2 n}
$$
\begin{CJK}{UTF8}{mj}之和\end{CJK}.

\begin{enumerate}
  \setcounter{enumi}{9}
  \item (13 \begin{CJK}{UTF8}{mj}分\end{CJK}) \begin{CJK}{UTF8}{mj}证明\end{CJK}
\end{enumerate}
$$
f(x)=\sqrt{x} \ln x
$$
\begin{CJK}{UTF8}{mj}在\end{CJK} $[1,+\infty)$ \begin{CJK}{UTF8}{mj}一致连续\end{CJK}.

\begin{enumerate}
  \setcounter{enumi}{10}
  \item (13 \begin{CJK}{UTF8}{mj}分\end{CJK}) \begin{CJK}{UTF8}{mj}设\end{CJK} $f(x, y)$ \begin{CJK}{UTF8}{mj}在\end{CJK} $[a,+\infty ; c, d]$ \begin{CJK}{UTF8}{mj}上连续\end{CJK}, $\int_{a}^{+\infty} f(x, y) \mathrm{d} x$ \begin{CJK}{UTF8}{mj}关于\end{CJK} $y \in[c, d]$ \begin{CJK}{UTF8}{mj}一致收敛\end{CJK}, \begin{CJK}{UTF8}{mj}证明\end{CJK}:
\end{enumerate}
$$
\int_{c}^{d} \mathrm{~d} y \int_{a}^{+\infty} f(x, y) \mathrm{d} x=\int_{a}^{+\infty} \mathrm{d} x \int_{c}^{d} f(x, y) \mathrm{d} y
$$

\begin{enumerate}
  \setcounter{enumi}{11}
  \item (13 \begin{CJK}{UTF8}{mj}分\end{CJK}) \begin{CJK}{UTF8}{mj}设\end{CJK} $f(x)$ \begin{CJK}{UTF8}{mj}在\end{CJK} $[a, b]$ \begin{CJK}{UTF8}{mj}上有二阶连续导数\end{CJK}, \begin{CJK}{UTF8}{mj}试证存在\end{CJK} $\xi \in[a, b]$, \begin{CJK}{UTF8}{mj}使得\end{CJK}
\end{enumerate}
$$
f(a)+f(b)=2 f\left(\frac{a+b}{2}\right)+\frac{1}{4}(b-a)^{2} f^{\prime \prime}(\xi) .
$$

\begin{enumerate}
  \setcounter{enumi}{12}
  \item ( 13 \begin{CJK}{UTF8}{mj}分\end{CJK}) \begin{CJK}{UTF8}{mj}设\end{CJK} $f(x, y)$ \begin{CJK}{UTF8}{mj}在区域\end{CJK} $G$ \begin{CJK}{UTF8}{mj}内对\end{CJK} $x$ \begin{CJK}{UTF8}{mj}和\end{CJK} $y$ \begin{CJK}{UTF8}{mj}分别连续\end{CJK}, \begin{CJK}{UTF8}{mj}且对\end{CJK} $x$ \begin{CJK}{UTF8}{mj}单增\end{CJK}, \begin{CJK}{UTF8}{mj}试证\end{CJK} $f(x, y)$ \begin{CJK}{UTF8}{mj}在\end{CJK} $G$ \begin{CJK}{UTF8}{mj}内连续\end{CJK}.
\end{enumerate}
\section{6. 华南理工大学 2015 年研究生入学考试试题数学分析 
 李扬 
 微信公众号: sxkyliyang}
\begin{enumerate}
  \item (12 \begin{CJK}{UTF8}{mj}分\end{CJK}) \begin{CJK}{UTF8}{mj}设\end{CJK}
\end{enumerate}
$$
f(x)=\frac{2 \sin (x-2)+([x]-1) x^{2}-2([x]+1) x+8}{(x-2)^{2}},
$$
\begin{CJK}{UTF8}{mj}求\end{CJK} $\lim _{x \rightarrow 2-0} f(x)$ \begin{CJK}{UTF8}{mj}及\end{CJK} $\lim _{x \rightarrow 2+0} f(x)$.

\begin{enumerate}
  \setcounter{enumi}{2}
  \item ( 12 \begin{CJK}{UTF8}{mj}分\end{CJK}) \begin{CJK}{UTF8}{mj}在曲线\end{CJK} $x=2 t^{2}, y=3 t^{2}, z=2 t$ \begin{CJK}{UTF8}{mj}上确定点\end{CJK}, \begin{CJK}{UTF8}{mj}使在该点处的切线平行于\end{CJK} $x^{2}+y^{2}+z^{2}=4$ \begin{CJK}{UTF8}{mj}在点\end{CJK} $(1,-1, \sqrt{2})$ \begin{CJK}{UTF8}{mj}处的切平面\end{CJK}.

  \item ( 12 \begin{CJK}{UTF8}{mj}分\end{CJK}) \begin{CJK}{UTF8}{mj}设\end{CJK}

\end{enumerate}
$$
f(x)=\left\{\begin{array}{cl}
x^{2} \ln x+1, & x>0 \\
1, & x=0 \\
\frac{2(1-\cos x)}{x^{2}}, & x<0
\end{array}\right.
$$
\begin{CJK}{UTF8}{mj}研究其导数的连续性\end{CJK}.

\begin{enumerate}
  \setcounter{enumi}{4}
  \item (12 \begin{CJK}{UTF8}{mj}分\end{CJK}) \begin{CJK}{UTF8}{mj}设\end{CJK} $l$ \begin{CJK}{UTF8}{mj}为圆周\end{CJK} $x^{2}+y^{2}+z^{2}=1, x+y+z=0$. \begin{CJK}{UTF8}{mj}从\end{CJK} $x$ \begin{CJK}{UTF8}{mj}轴正向看去\end{CJK}, \begin{CJK}{UTF8}{mj}按逆时针方向\end{CJK}, \begin{CJK}{UTF8}{mj}计算曲线积分\end{CJK}
\end{enumerate}
$$
\int_{l} 2 y \mathrm{~d} x+3 z \mathrm{~d} y+4 x \mathrm{~d} z
$$

\begin{enumerate}
  \setcounter{enumi}{5}
  \item (12 \begin{CJK}{UTF8}{mj}分\end{CJK}) \begin{CJK}{UTF8}{mj}计算\end{CJK}
\end{enumerate}
$$
\int_{0}^{+\infty} \frac{\sin x}{\mathrm{e}^{x y}} \mathrm{~d} x
$$
\begin{CJK}{UTF8}{mj}其中\end{CJK} $y>0$.

\begin{enumerate}
  \setcounter{enumi}{6}
  \item (12 \begin{CJK}{UTF8}{mj}分\end{CJK}) \begin{CJK}{UTF8}{mj}求\end{CJK}
\end{enumerate}
$$
\sum_{n=1}^{+\infty} n^{2} x^{n+1}
$$
\begin{CJK}{UTF8}{mj}的和函数\end{CJK}.

\begin{enumerate}
  \setcounter{enumi}{7}
  \item (13 \begin{CJK}{UTF8}{mj}分\end{CJK}) \begin{CJK}{UTF8}{mj}设\end{CJK} $f(x, y)$ \begin{CJK}{UTF8}{mj}在区域\end{CJK} $D: a \leq x \leq b, c \leq y \leq d$ \begin{CJK}{UTF8}{mj}上连续\end{CJK}. \begin{CJK}{UTF8}{mj}记\end{CJK} $D_{x y}: a \leq u \leq x, c \leq v \leq y$, \begin{CJK}{UTF8}{mj}及\end{CJK}
\end{enumerate}
$$
F(x, y)=\iint_{D_{x y}} f(u, v) \mathrm{d} u \mathrm{~d} v
$$
\begin{CJK}{UTF8}{mj}试在\end{CJK} $D$ \begin{CJK}{UTF8}{mj}内求二阶混合偏导数\end{CJK} $F_{x y}$ \begin{CJK}{UTF8}{mj}及\end{CJK} $F_{y x}$.

\begin{enumerate}
  \setcounter{enumi}{8}
  \item ( 13 \begin{CJK}{UTF8}{mj}分\end{CJK}) \begin{CJK}{UTF8}{mj}设\end{CJK} $y>0$. \begin{CJK}{UTF8}{mj}令\end{CJK}
\end{enumerate}
$$
\begin{gathered}
G(y)=\int_{0}^{1} \frac{y \mathrm{~d} x}{\sqrt{x^{2}+y^{2}}}, \\
F(y)=\frac{1}{2}\left[\sqrt{1+y^{2}}+y^{2} \ln \frac{1+\sqrt{1+y^{2}}}{y}\right] .
\end{gathered}
$$
\begin{CJK}{UTF8}{mj}证明\end{CJK} $G(y)=F^{\prime}(y)$.

\begin{enumerate}
  \setcounter{enumi}{9}
  \item ( 13 \begin{CJK}{UTF8}{mj}分\end{CJK}) \begin{CJK}{UTF8}{mj}设\end{CJK}
\end{enumerate}
$$
f(x, y)= \begin{cases}\frac{x^{2} y^{3}}{x^{4}+y^{4}}, & (x, y) \neq(0,0) \\ 0, & (x, y)=(0,0)\end{cases}
$$
\begin{CJK}{UTF8}{mj}研究以下性质\end{CJK}:

(1) \begin{CJK}{UTF8}{mj}该函数的连续性\end{CJK};

(2) \begin{CJK}{UTF8}{mj}一阶偏导的连续性\end{CJK};

(3) \begin{CJK}{UTF8}{mj}该函数的可微性\end{CJK}.

\begin{enumerate}
  \setcounter{enumi}{10}
  \item (13 \begin{CJK}{UTF8}{mj}分\end{CJK}) \begin{CJK}{UTF8}{mj}设\end{CJK}
\end{enumerate}
$$
x_{1}=\frac{1}{2}, x_{n+1}=\frac{1+2 x_{n}}{1+x_{n}}, n=1,2, \cdots
$$
\begin{CJK}{UTF8}{mj}试证\end{CJK} $\left\{x_{n}\right\}$ \begin{CJK}{UTF8}{mj}收敛\end{CJK}, \begin{CJK}{UTF8}{mj}并求\end{CJK} $\lim _{n \rightarrow \infty} x_{n}$.

\begin{enumerate}
  \setcounter{enumi}{11}
  \item (13 \begin{CJK}{UTF8}{mj}分\end{CJK}) \begin{CJK}{UTF8}{mj}证明\end{CJK}
\end{enumerate}
$$
f(x)=\frac{x-\sin x}{x}
$$
\begin{CJK}{UTF8}{mj}在\end{CJK} $(0,+\infty)$ \begin{CJK}{UTF8}{mj}一致连续\end{CJK}.

\begin{enumerate}
  \setcounter{enumi}{12}
  \item (13 \begin{CJK}{UTF8}{mj}分\end{CJK}) \begin{CJK}{UTF8}{mj}设\end{CJK} $f(x)$ \begin{CJK}{UTF8}{mj}在\end{CJK} $[a, b]$ \begin{CJK}{UTF8}{mj}上连续且满足\end{CJK}
\end{enumerate}
$$
0 \leq f(x) \leq m+n \int_{a}^{x} f(t) \mathrm{d} t(a \leq x \leq b)
$$
\begin{CJK}{UTF8}{mj}其中\end{CJK} $m \geq 0$ \begin{CJK}{UTF8}{mj}及\end{CJK} $n>0$ \begin{CJK}{UTF8}{mj}是两个常数\end{CJK}. \begin{CJK}{UTF8}{mj}试证对任意\end{CJK} $x \in[a, b]$ \begin{CJK}{UTF8}{mj}有\end{CJK}
$$
f(x) \leq m \mathrm{e}^{n(x-a)} .
$$

\section{7. 华南理工大学 2016 年研究生入学考试试题数学分析}
\begin{CJK}{UTF8}{mj}李扬\end{CJK}

\begin{CJK}{UTF8}{mj}微信公众号\end{CJK}: sxkyliyang

\begin{enumerate}
  \item (12 \begin{CJK}{UTF8}{mj}分\end{CJK}) \begin{CJK}{UTF8}{mj}求极限\end{CJK}
\end{enumerate}
$$
\lim _{x \rightarrow 0^{+}}\left(\sqrt{\frac{1}{x}+\sqrt{\frac{1}{x}+\sqrt{\frac{1}{x}}}}-\sqrt{\frac{1}{x}-\sqrt{\frac{1}{x}+\sqrt{\frac{1}{x}}}}\right)
$$

\begin{enumerate}
  \setcounter{enumi}{2}
  \item (12 \begin{CJK}{UTF8}{mj}分\end{CJK}) \begin{CJK}{UTF8}{mj}确定\end{CJK} $a, b$ \begin{CJK}{UTF8}{mj}的值\end{CJK}, \begin{CJK}{UTF8}{mj}使函数\end{CJK}
\end{enumerate}
$$
f(x)=\left\{\begin{array}{cl}
\frac{1}{x}(1-\cos a x), & x<0 \\
0, & x=0 . \\
\frac{1}{x} \ln \left(b+x^{2}\right), & x>0
\end{array}\right.
$$
\begin{CJK}{UTF8}{mj}在\end{CJK} $(-\infty,+\infty)$ \begin{CJK}{UTF8}{mj}内处处可导\end{CJK}, \begin{CJK}{UTF8}{mj}并求它的导数\end{CJK}.

\begin{enumerate}
  \setcounter{enumi}{3}
  \item (12 \begin{CJK}{UTF8}{mj}分\end{CJK}) \begin{CJK}{UTF8}{mj}计算极限\end{CJK}
\end{enumerate}
$$
\lim _{n \rightarrow \infty} \frac{1}{n} \sqrt[n]{n(n+1)(n+2) \cdots(2 n-1)}
$$

\begin{enumerate}
  \setcounter{enumi}{4}
  \item (12 \begin{CJK}{UTF8}{mj}分\end{CJK}) \begin{CJK}{UTF8}{mj}讨论函数\end{CJK}
\end{enumerate}
$$
y=\left\{\begin{array}{ll}
\sin \pi x, & x \text { 为有理数 } \\
0, & x \text { 为无理数 }
\end{array}\right. \text {. }
$$
\begin{CJK}{UTF8}{mj}的连续性\end{CJK}, \begin{CJK}{UTF8}{mj}并指出间断点类型\end{CJK}.

\begin{enumerate}
  \setcounter{enumi}{5}
  \item (12 \begin{CJK}{UTF8}{mj}分\end{CJK}) \begin{CJK}{UTF8}{mj}设\end{CJK} $a, b$ \begin{CJK}{UTF8}{mj}为常数\end{CJK}, \begin{CJK}{UTF8}{mj}讨论积分\end{CJK}
\end{enumerate}
$$
\int_{0}^{1} x^{a-1}(1-x)^{b-1} \mathrm{~d} x
$$
\begin{CJK}{UTF8}{mj}的收敛性\end{CJK}.

\begin{enumerate}
  \setcounter{enumi}{6}
  \item (12 \begin{CJK}{UTF8}{mj}分\end{CJK}) \begin{CJK}{UTF8}{mj}证明级数\end{CJK}
\end{enumerate}
$$
\sum_{n=0}^{\infty}\left(\frac{1}{3 n+1}+\frac{1}{3 n+2}-\frac{1}{3 n+3}\right)
$$
\begin{CJK}{UTF8}{mj}是发散的\end{CJK}.

\begin{enumerate}
  \setcounter{enumi}{7}
  \item ( 13 \begin{CJK}{UTF8}{mj}分\end{CJK}) \begin{CJK}{UTF8}{mj}设\end{CJK} $f(x)$ \begin{CJK}{UTF8}{mj}在\end{CJK} $[0,1]$ \begin{CJK}{UTF8}{mj}上连续可微\end{CJK}, \begin{CJK}{UTF8}{mj}证明\end{CJK}
\end{enumerate}
$$
\lim _{n \rightarrow \infty} n \int_{0}^{1} x^{n} f(x) \mathrm{d} x=f(1) .
$$

\begin{enumerate}
  \setcounter{enumi}{8}
  \item ( 13 \begin{CJK}{UTF8}{mj}分\end{CJK}) \begin{CJK}{UTF8}{mj}证明函数\end{CJK}
\end{enumerate}
$$
f(x, y)=\left\{\begin{array}{cc}
\left(x^{2}+y\right) \sin \frac{1}{\sqrt{x^{2}+y^{2}}}, & x^{2}+y^{2} \neq 0 \\
0, & x^{2}+y^{2}=0
\end{array}\right.
$$
\begin{CJK}{UTF8}{mj}在\end{CJK} $(0,0)$ \begin{CJK}{UTF8}{mj}处连续且偏导数存在\end{CJK}, \begin{CJK}{UTF8}{mj}但偏导数在点\end{CJK} $(0,0)$ \begin{CJK}{UTF8}{mj}处不连续\end{CJK}, \begin{CJK}{UTF8}{mj}而\end{CJK} $f(x, y)$ \begin{CJK}{UTF8}{mj}在点\end{CJK} $(0,0)$ \begin{CJK}{UTF8}{mj}处可微\end{CJK}.

\begin{enumerate}
  \setcounter{enumi}{9}
  \item ( 13 \begin{CJK}{UTF8}{mj}分\end{CJK}) (1) \begin{CJK}{UTF8}{mj}将\end{CJK} $\arctan x$ \begin{CJK}{UTF8}{mj}展开成幂级数\end{CJK}, \begin{CJK}{UTF8}{mj}求收玫半径\end{CJK}.
\end{enumerate}
(2) \begin{CJK}{UTF8}{mj}利用\end{CJK} (1) \begin{CJK}{UTF8}{mj}证明\end{CJK}
$$
\pi=4-\frac{4}{3}+\frac{4}{5}-\cdots+(-1)^{n} \frac{4}{2 n+1}+\cdots
$$
(3) \begin{CJK}{UTF8}{mj}利用\end{CJK} (2) \begin{CJK}{UTF8}{mj}中的公式近似计算\end{CJK} $\pi$ \begin{CJK}{UTF8}{mj}的值\end{CJK}, \begin{CJK}{UTF8}{mj}需要求多少项的和\end{CJK}, \begin{CJK}{UTF8}{mj}误差不会超过\end{CJK} $10^{-m}(m$ \begin{CJK}{UTF8}{mj}为自然数\end{CJK}). 10. (13 \begin{CJK}{UTF8}{mj}分\end{CJK}) \begin{CJK}{UTF8}{mj}设在某闭矩形域\end{CJK} $D$ \begin{CJK}{UTF8}{mj}内\end{CJK} $\frac{\partial P}{\partial y}=\frac{\partial Q}{\partial x}$. \begin{CJK}{UTF8}{mj}试证\end{CJK}
$$
u(x, y)=\int_{x_{0}}^{x} P(x, y) \mathrm{d} x+\int_{y_{0}}^{y} Q\left(x_{0}, y\right) \mathrm{d} y+C
$$
\begin{CJK}{UTF8}{mj}为\end{CJK} $P \mathrm{~d} x+Q \mathrm{~d} y$ \begin{CJK}{UTF8}{mj}的原函数\end{CJK}, \begin{CJK}{UTF8}{mj}其中\end{CJK} $C=u\left(x_{0}, y_{0}\right)$.

\begin{enumerate}
  \setcounter{enumi}{11}
  \item (13 \begin{CJK}{UTF8}{mj}分\end{CJK}) \begin{CJK}{UTF8}{mj}计算下列曲面\end{CJK}
\end{enumerate}
$$
\left(\frac{x^{2}}{a^{2}}+\frac{y^{2}}{b^{2}}+\frac{z^{2}}{c^{2}}\right)^{2}=\frac{x^{2}}{a^{2}}+\frac{y^{2}}{b^{2}}
$$
\begin{CJK}{UTF8}{mj}所围成的体积\end{CJK}.

\begin{enumerate}
  \setcounter{enumi}{12}
  \item (13 \begin{CJK}{UTF8}{mj}分\end{CJK}) \begin{CJK}{UTF8}{mj}设\end{CJK} $k$ \begin{CJK}{UTF8}{mj}是常数\end{CJK}, $L$ \begin{CJK}{UTF8}{mj}为曲线\end{CJK} $x^{2}+x y+y^{2}=r^{2}$, \begin{CJK}{UTF8}{mj}依逆时针方向\end{CJK}. \begin{CJK}{UTF8}{mj}定义\end{CJK}
\end{enumerate}
$$
I(r)=\oint_{L} \frac{x \mathrm{~d} y-y \mathrm{~d} x}{\left(x^{2}+y^{2}\right)^{k}}
$$
\begin{CJK}{UTF8}{mj}求\end{CJK} $\lim _{r \rightarrow+\infty} I(r)$.

\section{8. 华南理工大学 2017 年研究生入学考试试题数学分析}
\begin{CJK}{UTF8}{mj}李扬\end{CJK}

\begin{CJK}{UTF8}{mj}微信公众号\end{CJK}: sxkyliyang

\begin{enumerate}
  \item (10 \begin{CJK}{UTF8}{mj}分\end{CJK}) \begin{CJK}{UTF8}{mj}求\end{CJK}
\end{enumerate}
$$
\lim _{x \rightarrow 0} \frac{\int_{0}^{x} \mathrm{e}^{t^{2}} \mathrm{~d} t-x \cos x}{x-\sin x}
$$

\begin{enumerate}
  \setcounter{enumi}{2}
  \item (10 \begin{CJK}{UTF8}{mj}分\end{CJK}) \begin{CJK}{UTF8}{mj}求\end{CJK}
\end{enumerate}
$$
\lim _{n \rightarrow \infty} \frac{1}{n} \sum_{k=1}^{n} k^{\frac{1}{k}}
$$

\begin{enumerate}
  \setcounter{enumi}{3}
  \item (10 \begin{CJK}{UTF8}{mj}分\end{CJK}) \begin{CJK}{UTF8}{mj}计算积分\end{CJK}
\end{enumerate}
$$
\int_{0}^{1} \frac{x^{\alpha}-x_{\beta}}{\ln x} \sin (\ln x) \mathrm{d} x(\alpha>\beta>0)
$$

\begin{enumerate}
  \setcounter{enumi}{4}
  \item (10 \begin{CJK}{UTF8}{mj}分\end{CJK}) \begin{CJK}{UTF8}{mj}计算曲线积分\end{CJK}
\end{enumerate}
$$
\oint_{C} y^{2} \mathrm{~d} s,
$$
\begin{CJK}{UTF8}{mj}其中\end{CJK} $C$ \begin{CJK}{UTF8}{mj}为\end{CJK} $x^{2}+y^{2}+z^{2}=a^{2}$ \begin{CJK}{UTF8}{mj}与\end{CJK} $x+y+z=0$ \begin{CJK}{UTF8}{mj}的交线\end{CJK}.

\begin{enumerate}
  \setcounter{enumi}{5}
  \item (10 \begin{CJK}{UTF8}{mj}分\end{CJK}) \begin{CJK}{UTF8}{mj}计算曲面积分\end{CJK}
\end{enumerate}
$$
\iint_{\Sigma}(2 x+3) \mathrm{d} y \mathrm{~d} z+(3 z+4) \mathrm{d} x \mathrm{~d} y,
$$
\begin{CJK}{UTF8}{mj}其中\end{CJK} $\Sigma$ \begin{CJK}{UTF8}{mj}为顶点是\end{CJK} $(0,0,0),(1,0,0),(0,1,0)$ \begin{CJK}{UTF8}{mj}及\end{CJK} $(0,0,1)$ \begin{CJK}{UTF8}{mj}的四面体的外表面\end{CJK}.

\begin{enumerate}
  \setcounter{enumi}{6}
  \item ( 10 \begin{CJK}{UTF8}{mj}分\end{CJK}) \begin{CJK}{UTF8}{mj}设\end{CJK} $f(x)$ \begin{CJK}{UTF8}{mj}在\end{CJK} $(a, b)$ \begin{CJK}{UTF8}{mj}区间连续\end{CJK}, \begin{CJK}{UTF8}{mj}取\end{CJK} $a<x_{1}<x_{2}<\cdots<x_{n}<b$, \begin{CJK}{UTF8}{mj}试证存在\end{CJK} $\xi \in\left[x_{1}, x_{n}\right]$, \begin{CJK}{UTF8}{mj}使得\end{CJK}
\end{enumerate}
$$
f(\xi)=\frac{1}{n} \sum_{k=1}^{n} f\left(x_{k}\right)
$$

\begin{enumerate}
  \setcounter{enumi}{7}
  \item (15 \begin{CJK}{UTF8}{mj}分\end{CJK}) \begin{CJK}{UTF8}{mj}设\end{CJK} $f(x)$ \begin{CJK}{UTF8}{mj}在\end{CJK} $[0,+\infty)$ \begin{CJK}{UTF8}{mj}连续\end{CJK}, \begin{CJK}{UTF8}{mj}且对\end{CJK} $\forall \varepsilon>0$, \begin{CJK}{UTF8}{mj}积分\end{CJK} $\int_{\varepsilon}^{+\infty} \frac{f(x)}{x} \mathrm{~d} x$ \begin{CJK}{UTF8}{mj}都收敛\end{CJK}, \begin{CJK}{UTF8}{mj}试证对\end{CJK} $\forall z>y>0$, \begin{CJK}{UTF8}{mj}有\end{CJK}
\end{enumerate}
$$
\int_{0}^{+\infty} \frac{f(x y)-f(x z)}{x} \mathrm{~d} x=f(0) \ln \frac{z}{y}
$$

\begin{enumerate}
  \setcounter{enumi}{8}
  \item ( 15 \begin{CJK}{UTF8}{mj}分\end{CJK}) \begin{CJK}{UTF8}{mj}求\end{CJK}
\end{enumerate}
$$
\sum_{n=1}^{\infty} n^{2} x^{n}
$$
\begin{CJK}{UTF8}{mj}的收敛域及和函数\end{CJK}.

\begin{enumerate}
  \setcounter{enumi}{9}
  \item (15 \begin{CJK}{UTF8}{mj}分\end{CJK}) \begin{CJK}{UTF8}{mj}研究\end{CJK}
\end{enumerate}
$$
u=x y z
$$
\begin{CJK}{UTF8}{mj}在条件\end{CJK} $x^{2}+y^{2}-z^{2}=1$ \begin{CJK}{UTF8}{mj}及\end{CJK} $x+y+z=0$ \begin{CJK}{UTF8}{mj}之下是否有极值\end{CJK}.

\begin{enumerate}
  \setcounter{enumi}{10}
  \item (15 \begin{CJK}{UTF8}{mj}分\end{CJK}) \begin{CJK}{UTF8}{mj}设不收敛数列\end{CJK} $\left\{x_{n}\right\}$ \begin{CJK}{UTF8}{mj}有界\end{CJK}, \begin{CJK}{UTF8}{mj}试证存在\end{CJK} $\left\{x_{n}\right\}$ \begin{CJK}{UTF8}{mj}的两个收敛子列\end{CJK} $\left\{x_{n_{k}}^{(1)}\right\}$ \begin{CJK}{UTF8}{mj}及\end{CJK} $\left\{x_{n_{k}}^{(2)}\right\}$ \begin{CJK}{UTF8}{mj}满足\end{CJK}
\end{enumerate}
$$
\lim _{k \rightarrow \infty} x_{n_{k}}^{(1)} \neq \lim _{k \rightarrow \infty} x_{n_{k}}^{(2)}
$$

\begin{enumerate}
  \setcounter{enumi}{11}
  \item ( 15 \begin{CJK}{UTF8}{mj}分\end{CJK}) \begin{CJK}{UTF8}{mj}设\end{CJK} $f(x, y)$ \begin{CJK}{UTF8}{mj}的二阶混合偏导数在\end{CJK} $\left(x_{0}, y_{0}\right)$ \begin{CJK}{UTF8}{mj}的邻域内连续\end{CJK}, \begin{CJK}{UTF8}{mj}试证存在\end{CJK} $0<\theta_{i}<1(i=1,2,3,4)$, \begin{CJK}{UTF8}{mj}使得\end{CJK}
\end{enumerate}
$$
f_{x y}\left(x_{0}+\theta_{1} \Delta x, y_{0}+\theta_{2} \Delta y\right)=f_{y x}\left(x_{0}+\theta_{3} \Delta x, y_{0}+\theta_{4} \Delta y\right) .
$$

\begin{enumerate}
  \setcounter{enumi}{12}
  \item ( 15 \begin{CJK}{UTF8}{mj}分\end{CJK}) \begin{CJK}{UTF8}{mj}设\end{CJK} $f(x)$ \begin{CJK}{UTF8}{mj}在\end{CJK} $[a, b]$ \begin{CJK}{UTF8}{mj}上二阶可导且满足\end{CJK}
\end{enumerate}
(1) $f^{\prime}(x)>0, f^{\prime \prime}(x)>0$ \begin{CJK}{UTF8}{mj}对\end{CJK} $\forall x \in[a, b]$;

(2) $f(a)<0, f(b)>0$.

\begin{CJK}{UTF8}{mj}今\end{CJK}
$$
x_{1}=b-\frac{f(b)}{f^{\prime}(b)}, x_{n+1}=x_{n}-\frac{f\left(x_{n}\right)}{f^{\prime}\left(x_{n}\right)},(n=1,2, \cdots)
$$
\begin{CJK}{UTF8}{mj}试证\end{CJK} $\left\{x_{n}\right\}$ \begin{CJK}{UTF8}{mj}收敛到\end{CJK} $f(x)$ \begin{CJK}{UTF8}{mj}在\end{CJK} $(a, b)$ \begin{CJK}{UTF8}{mj}上的零点\end{CJK}.

\section{1. 华中科技大学 2009 年研究生入学考试试题高等代数 
 李扬 
 微信公众号: sxkyliyang}
\begin{enumerate}
  \item (15\begin{CJK}{UTF8}{mj}分\end{CJK}) \begin{CJK}{UTF8}{mj}计算如下行列式\end{CJK}(\begin{CJK}{UTF8}{mj}空白处为零\end{CJK})
\end{enumerate}
$$
\left|\begin{array}{ccccc}
x & & & & a_{n} \\
-1 & x & & & a_{n-1} \\
& -1 & \ddots & & \vdots \\
& & \ddots & x & a_{2} \\
& & & -1 & x+a_{1}
\end{array}\right| .
$$

\begin{enumerate}
  \setcounter{enumi}{2}
  \item (20\begin{CJK}{UTF8}{mj}分\end{CJK}) \begin{CJK}{UTF8}{mj}设\end{CJK} $A$ \begin{CJK}{UTF8}{mj}是\end{CJK} $n \times r$ \begin{CJK}{UTF8}{mj}矩阵\end{CJK}, $B$ \begin{CJK}{UTF8}{mj}是\end{CJK} $r \times s$ \begin{CJK}{UTF8}{mj}矩阵\end{CJK}, \begin{CJK}{UTF8}{mj}如果\end{CJK} $\operatorname{rank}(B)=r$. \begin{CJK}{UTF8}{mj}证明如下结论\end{CJK}:
\end{enumerate}
(1) \begin{CJK}{UTF8}{mj}如果\end{CJK} $A B=0$, \begin{CJK}{UTF8}{mj}则\end{CJK} $A=0$.

(2) \begin{CJK}{UTF8}{mj}如果\end{CJK} $A B=B$, \begin{CJK}{UTF8}{mj}则\end{CJK} $A=I$.

\begin{enumerate}
  \setcounter{enumi}{3}
  \item (20\begin{CJK}{UTF8}{mj}分\end{CJK}) \begin{CJK}{UTF8}{mj}设矩阵\end{CJK}
\end{enumerate}
$$
A=\left(\begin{array}{lll}
a_{11} & a_{12} & a_{13} \\
a_{21} & a_{22} & a_{23}
\end{array}\right)
$$
\begin{CJK}{UTF8}{mj}向量\end{CJK} $\alpha=\left(c_{1}, c_{2}, c_{3}\right)^{\prime}$, \begin{CJK}{UTF8}{mj}其中\end{CJK} $c_{1}=\operatorname{det}\left(\begin{array}{ll}a_{12} & a_{13} \\ a_{22} & a_{23}\end{array}\right), c_{2}=\operatorname{det}\left(\begin{array}{ll}a_{13} & a_{11} \\ a_{23} & a_{21}\end{array}\right), c_{3}=\operatorname{det}\left(\begin{array}{ll}a_{11} & a_{12} \\ a_{21} & a_{22}\end{array}\right)$. \begin{CJK}{UTF8}{mj}证\end{CJK} \begin{CJK}{UTF8}{mj}明如下结论\end{CJK}:

(1) $\operatorname{rank}(A)=2$ \begin{CJK}{UTF8}{mj}的充分必要条件是向量\end{CJK} $\alpha \neq 0$.

(2) \begin{CJK}{UTF8}{mj}如果\end{CJK} $\operatorname{rank}(A)=2$, \begin{CJK}{UTF8}{mj}那么\end{CJK} $\alpha$ \begin{CJK}{UTF8}{mj}是齐次线性方程组\end{CJK} $A X=0$ \begin{CJK}{UTF8}{mj}的基础解系\end{CJK}.

\begin{enumerate}
  \setcounter{enumi}{4}
  \item ( 20 \begin{CJK}{UTF8}{mj}分\end{CJK}) \begin{CJK}{UTF8}{mj}已知\end{CJK} $\alpha_{1}, \cdots, \alpha_{k}$ \begin{CJK}{UTF8}{mj}是其次线性方程组\end{CJK} $A X=0$ \begin{CJK}{UTF8}{mj}的一个基础解系\end{CJK}, \begin{CJK}{UTF8}{mj}向量\end{CJK} $\beta_{1}, \cdots, \beta_{k}$ \begin{CJK}{UTF8}{mj}都是\end{CJK} $A X=0$ \begin{CJK}{UTF8}{mj}的解\end{CJK}.
\end{enumerate}
\begin{CJK}{UTF8}{mj}令矩阵\end{CJK}
$$
S=\left(\alpha_{1}, \cdots, \alpha_{k}\right), T=\left(\beta_{1}, \cdots, \beta_{k}\right)
$$
\begin{CJK}{UTF8}{mj}证明如下结论\end{CJK}:

(1) \begin{CJK}{UTF8}{mj}存在\end{CJK} $k$ \begin{CJK}{UTF8}{mj}阶方阵\end{CJK} $C$, \begin{CJK}{UTF8}{mj}使得\end{CJK} $T=S C$.

(2) $\beta_{1}, \cdots, \beta_{k}$ \begin{CJK}{UTF8}{mj}也是\end{CJK} $A X=0$ \begin{CJK}{UTF8}{mj}的基础解系的充分必要条件是\end{CJK} $C$ \begin{CJK}{UTF8}{mj}可逆\end{CJK}.

\begin{enumerate}
  \setcounter{enumi}{5}
  \item (30\begin{CJK}{UTF8}{mj}分\end{CJK}) \begin{CJK}{UTF8}{mj}设\end{CJK} $\mathscr{A}$ \begin{CJK}{UTF8}{mj}是欧式空间\end{CJK} $E$ \begin{CJK}{UTF8}{mj}内的一个线性变换\end{CJK}, \begin{CJK}{UTF8}{mj}对任意的向量\end{CJK} $\alpha, \beta$ \begin{CJK}{UTF8}{mj}都有\end{CJK}
\end{enumerate}
$$
\langle\mathscr{A} \alpha, \beta\rangle=\langle\alpha, \mathscr{A} \beta\rangle
$$
(1) \begin{CJK}{UTF8}{mj}证明\end{CJK} $\mathscr{A}$ \begin{CJK}{UTF8}{mj}的特征值都是实数\end{CJK}.

(2) \begin{CJK}{UTF8}{mj}是否可以找到空间的基\end{CJK}, \begin{CJK}{UTF8}{mj}使得\end{CJK} $\mathscr{A}$ \begin{CJK}{UTF8}{mj}对应的矩阵是对角矩阵\end{CJK}? \begin{CJK}{UTF8}{mj}证明你的结论\end{CJK}.

(3) \begin{CJK}{UTF8}{mj}取\end{CJK} $E=\mathbb{R}^{3}$, \begin{CJK}{UTF8}{mj}以及它的一个标准正交基\end{CJK} $\left\{e_{1}, e_{2}, e_{3}\right\}$. \begin{CJK}{UTF8}{mj}已知\end{CJK} $\mathscr{A}\left(e_{1}\right)=\mathscr{A}\left(e_{2}\right)=\mathscr{A}\left(e_{3}\right)=e_{1}+e_{2}+e_{3}$, \begin{CJK}{UTF8}{mj}求\end{CJK} $\mathbb{R}^{3}$ \begin{CJK}{UTF8}{mj}的另一个标准正交基\end{CJK}, \begin{CJK}{UTF8}{mj}使得\end{CJK} $\mathscr{A}$ \begin{CJK}{UTF8}{mj}对应的矩阵是对角矩阵\end{CJK}.

\begin{enumerate}
  \setcounter{enumi}{6}
  \item ( 25 \begin{CJK}{UTF8}{mj}分\end{CJK}) \begin{CJK}{UTF8}{mj}设\end{CJK} $\mathscr{A}$ \begin{CJK}{UTF8}{mj}是\end{CJK} $n$ \begin{CJK}{UTF8}{mj}维线性空间\end{CJK} $V$ \begin{CJK}{UTF8}{mj}内的一个幂等线性变换\end{CJK}, \begin{CJK}{UTF8}{mj}即\end{CJK}
\end{enumerate}
$$
\mathscr{A}^{2}=\mathscr{A}
$$
(1) \begin{CJK}{UTF8}{mj}证明\end{CJK} $V$ \begin{CJK}{UTF8}{mj}是\end{CJK} $\operatorname{Im}(V)$ \begin{CJK}{UTF8}{mj}与\end{CJK} $\operatorname{ker}(V)$ \begin{CJK}{UTF8}{mj}的和空间\end{CJK}. \begin{CJK}{UTF8}{mj}这个和是直和吗\end{CJK}? \begin{CJK}{UTF8}{mj}说明你的理由\end{CJK}.

(2) \begin{CJK}{UTF8}{mj}设\end{CJK} $\mathscr{A}$ \begin{CJK}{UTF8}{mj}的秩为\end{CJK} $\gamma$. \begin{CJK}{UTF8}{mj}证明存在\end{CJK} $V$ \begin{CJK}{UTF8}{mj}的一组基\end{CJK}, \begin{CJK}{UTF8}{mj}使得对任意的向量\end{CJK} $\alpha$, \begin{CJK}{UTF8}{mj}如果\end{CJK} $\alpha$ \begin{CJK}{UTF8}{mj}在这组基下的坐标为\end{CJK} $\left(x_{1}, \cdots, x_{n}\right)^{\prime}$, \begin{CJK}{UTF8}{mj}则在这组基下的坐标为\end{CJK} $\left(x_{1}, \cdots, x_{\gamma}, 0, \cdots, 0\right)^{\prime}$.

(3) \begin{CJK}{UTF8}{mj}求\end{CJK} $\mathscr{A}$ \begin{CJK}{UTF8}{mj}的最小多项式\end{CJK}.

\begin{enumerate}
  \setcounter{enumi}{7}
  \item ( 20 \begin{CJK}{UTF8}{mj}分\end{CJK}) \begin{CJK}{UTF8}{mj}设\end{CJK} $A$ \begin{CJK}{UTF8}{mj}是\end{CJK} $n$ \begin{CJK}{UTF8}{mj}阶实对称矩阵\end{CJK}, $\alpha$ \begin{CJK}{UTF8}{mj}是\end{CJK} $n$ \begin{CJK}{UTF8}{mj}维实向量\end{CJK}, \begin{CJK}{UTF8}{mj}已知矩阵\end{CJK}
\end{enumerate}
$$
\left(\begin{array}{ll}
A & \alpha \\
\alpha^{\prime} & 1
\end{array}\right)
$$
\begin{CJK}{UTF8}{mj}是正定矩阵\end{CJK}.

(1) \begin{CJK}{UTF8}{mj}证明矩阵\end{CJK} $A$ \begin{CJK}{UTF8}{mj}可逆\end{CJK}. $A$ \begin{CJK}{UTF8}{mj}正定吗\end{CJK}? \begin{CJK}{UTF8}{mj}说明你的理由\end{CJK}.

(2) \begin{CJK}{UTF8}{mj}证明\end{CJK} $\alpha^{\prime} A^{-1} \alpha<1$.

\section{2. 华中科技大学 2010 年研究生入学考试试题高等代数 
 李扬 
 微信公众号: sxkyliyang}
\begin{enumerate}
  \item \begin{CJK}{UTF8}{mj}求\end{CJK} $n$ \begin{CJK}{UTF8}{mj}阶行列式\end{CJK} (\begin{CJK}{UTF8}{mj}空白处为零\end{CJK})
\end{enumerate}
$$
\left|\begin{array}{cccccc}
1-x & x & & & & \\
-1 & 1-x & x & & & \\
& -1 & 1-x & x & & \\
& & \ddots & \ddots & \ddots & \\
& & & -1 & 1-x & x \\
& & & & -1 & 1-x
\end{array}\right|
$$
\begin{CJK}{UTF8}{mj}的值\end{CJK}.

\begin{enumerate}
  \setcounter{enumi}{2}
  \item \begin{CJK}{UTF8}{mj}已知\end{CJK} $A, B, C, D$ \begin{CJK}{UTF8}{mj}均为\end{CJK} $n$ \begin{CJK}{UTF8}{mj}阶矩阵\end{CJK}, \begin{CJK}{UTF8}{mj}且\end{CJK} $A C=C A$, \begin{CJK}{UTF8}{mj}已知分块矩阵\end{CJK}
\end{enumerate}
$$
\left(\begin{array}{ll}
A & B \\
C & D
\end{array}\right)
$$
(1) \begin{CJK}{UTF8}{mj}若\end{CJK} $A$ \begin{CJK}{UTF8}{mj}可逆\end{CJK}, \begin{CJK}{UTF8}{mj}证明\end{CJK} $\left(\begin{array}{ll}A & B \\ C & D\end{array}\right)$ \begin{CJK}{UTF8}{mj}可逆的充要条件是\end{CJK}
$$
\operatorname{det}(A D-C B) \neq 0
$$
(2) \begin{CJK}{UTF8}{mj}若\end{CJK} $A$ \begin{CJK}{UTF8}{mj}不可逆\end{CJK}, \begin{CJK}{UTF8}{mj}上述结论是否仍旧成立\end{CJK}? \begin{CJK}{UTF8}{mj}为什么\end{CJK}?

\begin{enumerate}
  \setcounter{enumi}{3}
  \item \begin{CJK}{UTF8}{mj}已知矩阵\end{CJK}
\end{enumerate}
$$
A=\left(\begin{array}{cccc}
1 & -1 & 0 & -1 \\
1 & 1 & 2 & -1 \\
1 & 0 & 1 & -1 \\
1 & 3 & 4 & -1
\end{array}\right)
$$
\begin{CJK}{UTF8}{mj}求一个秩为\end{CJK} 2 \begin{CJK}{UTF8}{mj}的\end{CJK} 4 \begin{CJK}{UTF8}{mj}阶矩阵\end{CJK} $B$ \begin{CJK}{UTF8}{mj}使得\end{CJK} $A B=0$.

\begin{enumerate}
  \setcounter{enumi}{4}
  \item \begin{CJK}{UTF8}{mj}设\end{CJK} $A$ \begin{CJK}{UTF8}{mj}是\end{CJK} $n$ \begin{CJK}{UTF8}{mj}阶矩阵\end{CJK}, \begin{CJK}{UTF8}{mj}证明\end{CJK} $A$ \begin{CJK}{UTF8}{mj}的秩为\end{CJK} 1 \begin{CJK}{UTF8}{mj}当且仅当存在两个非零向量\end{CJK} $\alpha, \beta$ \begin{CJK}{UTF8}{mj}使得\end{CJK}
\end{enumerate}
$$
A=\alpha \beta^{\prime} .
$$

\begin{enumerate}
  \setcounter{enumi}{5}
  \item \begin{CJK}{UTF8}{mj}已知\end{CJK} $X$ \begin{CJK}{UTF8}{mj}为线性变换\end{CJK} $\mathscr{A}$ \begin{CJK}{UTF8}{mj}关于特征值\end{CJK} $\lambda$ \begin{CJK}{UTF8}{mj}的特征向量\end{CJK}, $Y$ \begin{CJK}{UTF8}{mj}为线性变换\end{CJK} $\mathscr{A}{ }^{\prime}$ \begin{CJK}{UTF8}{mj}关于特征值\end{CJK} $\mu$ \begin{CJK}{UTF8}{mj}的特征向量\end{CJK}, $\lambda \neq \mu$, \begin{CJK}{UTF8}{mj}证明\end{CJK}:
\end{enumerate}
(1) $Y^{\prime} X=0$;

(2) \begin{CJK}{UTF8}{mj}若\end{CJK} $X, Y$ \begin{CJK}{UTF8}{mj}均为实向量\end{CJK}, \begin{CJK}{UTF8}{mj}证明\end{CJK} $X$ \begin{CJK}{UTF8}{mj}与\end{CJK} $Y$ \begin{CJK}{UTF8}{mj}线性无关\end{CJK}.

\begin{enumerate}
  \setcounter{enumi}{6}
  \item \begin{CJK}{UTF8}{mj}在\end{CJK} $\mathbb{R}^{n}$ \begin{CJK}{UTF8}{mj}空间中\end{CJK}, \begin{CJK}{UTF8}{mj}已知线性变换\end{CJK} $\mathscr{A}$ \begin{CJK}{UTF8}{mj}在任一基\end{CJK} $e_{i}$ \begin{CJK}{UTF8}{mj}下的坐标均为\end{CJK} $(1,1, \cdots, 1)^{\prime}$, \begin{CJK}{UTF8}{mj}其中\end{CJK} $e_{i}$ \begin{CJK}{UTF8}{mj}为单位矩阵的第\end{CJK} $i$ \begin{CJK}{UTF8}{mj}列\end{CJK} \begin{CJK}{UTF8}{mj}的列向量\end{CJK}.
\end{enumerate}
(1) \begin{CJK}{UTF8}{mj}求\end{CJK} $T$ \begin{CJK}{UTF8}{mj}得特征值\end{CJK}.

(2) \begin{CJK}{UTF8}{mj}求\end{CJK} $\mathbb{R}^{n}$ \begin{CJK}{UTF8}{mj}的一组标准正交基\end{CJK}, \begin{CJK}{UTF8}{mj}使得\end{CJK} $T$ \begin{CJK}{UTF8}{mj}在这一组基下的矩阵为对角阵\end{CJK}.

\begin{enumerate}
  \setcounter{enumi}{7}
  \item \begin{CJK}{UTF8}{mj}已知\end{CJK} $A, B$ \begin{CJK}{UTF8}{mj}为对称矩阵\end{CJK}, \begin{CJK}{UTF8}{mj}且\end{CJK} $A$ \begin{CJK}{UTF8}{mj}为正定矩阵\end{CJK}, \begin{CJK}{UTF8}{mj}证明存在常数\end{CJK} $c$, \begin{CJK}{UTF8}{mj}使得\end{CJK} $c A+B$ \begin{CJK}{UTF8}{mj}为正定矩阵\end{CJK}. 8. \begin{CJK}{UTF8}{mj}已知\end{CJK} $\mathscr{A}$ \begin{CJK}{UTF8}{mj}和\end{CJK} $\mathscr{B}$ \begin{CJK}{UTF8}{mj}是复数域空间中的两个线性变换\end{CJK}, \begin{CJK}{UTF8}{mj}且\end{CJK}
\end{enumerate}
$$
\mathscr{A} \mathscr{B}=\mathscr{B} \mathscr{A}
$$
\begin{CJK}{UTF8}{mj}证明\end{CJK}:

(1) $\mathscr{A}$ \begin{CJK}{UTF8}{mj}和\end{CJK} $\mathscr{B}$ \begin{CJK}{UTF8}{mj}有一个公共的一维不变子空间\end{CJK} $U$;

(2) \begin{CJK}{UTF8}{mj}若\end{CJK} $\mathscr{A}$ \begin{CJK}{UTF8}{mj}和\end{CJK} $\mathscr{B}$ \begin{CJK}{UTF8}{mj}均可对角化\end{CJK}, \begin{CJK}{UTF8}{mj}证明存在一组基\end{CJK}, \begin{CJK}{UTF8}{mj}使得\end{CJK} $\mathscr{A}$ \begin{CJK}{UTF8}{mj}和\end{CJK} $\mathscr{B}$ \begin{CJK}{UTF8}{mj}在这一组基下的矩阵均为对角阵\end{CJK}.

\section{3. 华中科技大学 2011 年研究生入学考试试题高等代数 
 李扬 
 微信公众号: sxkyliyang}
\begin{enumerate}
  \item \begin{CJK}{UTF8}{mj}计算行列式\end{CJK}
\end{enumerate}
$$
\left|\begin{array}{ccccc}
1 & a_{1} & a_{2} & \cdots & a_{n-1} \\
b_{1} & x_{1} & 0 & \cdots & 0 \\
b_{2} & 0 & x_{2} & \cdots & 0 \\
\vdots & \vdots & \vdots & & \vdots \\
b_{n-1} & 0 & 0 & \cdots & x_{n-1}
\end{array}\right| .
$$

\begin{enumerate}
  \setcounter{enumi}{2}
  \item \begin{CJK}{UTF8}{mj}求齐次线性方程组\end{CJK}:
\end{enumerate}
$$
\left\{\begin{array}{l}
x_{1}-x_{2}+x_{3}+x_{4}-x_{5}=0 \\
x_{1}+x_{2}+x_{4}+x_{5}=0 \\
x_{1}-3 x_{2}+2 x_{3}+x_{4}-3 x_{5}=0 \\
3 x_{1}+x_{2}+x_{3}+3 x_{4}+x_{5}=0
\end{array}\right.
$$
\begin{CJK}{UTF8}{mj}的一组基础解系\end{CJK}.

3 . \begin{CJK}{UTF8}{mj}设\end{CJK} $A, B$ \begin{CJK}{UTF8}{mj}都是\end{CJK} $m \times n$ \begin{CJK}{UTF8}{mj}矩阵\end{CJK}, $C$ \begin{CJK}{UTF8}{mj}是\end{CJK} $n \times n$ \begin{CJK}{UTF8}{mj}矩阵\end{CJK}, \begin{CJK}{UTF8}{mj}且\end{CJK} $A=B C, \operatorname{rank}(B)=n$. \begin{CJK}{UTF8}{mj}证明\end{CJK}:
$$
\operatorname{rank}(A)=\operatorname{rank}(C)
$$

\begin{enumerate}
  \setcounter{enumi}{4}
  \item \begin{CJK}{UTF8}{mj}设\end{CJK} $\mathscr{A}$ \begin{CJK}{UTF8}{mj}是\end{CJK} $n$ \begin{CJK}{UTF8}{mj}维线性空间\end{CJK} $V$ \begin{CJK}{UTF8}{mj}的线性变换\end{CJK}, \begin{CJK}{UTF8}{mj}且\end{CJK} $\mathscr{A}^{2}=\mathscr{I}$ \begin{CJK}{UTF8}{mj}是单位变换\end{CJK}.\\
(1) \begin{CJK}{UTF8}{mj}证明\end{CJK}: $V=\operatorname{Im} \mathscr{A} \oplus \operatorname{ker} \mathscr{A}$;\\
(2) \begin{CJK}{UTF8}{mj}试求\end{CJK} $\mathscr{A}$ \begin{CJK}{UTF8}{mj}的最小多项式\end{CJK}.

  \item \begin{CJK}{UTF8}{mj}设\end{CJK} $A$ \begin{CJK}{UTF8}{mj}是\end{CJK} $n$ \begin{CJK}{UTF8}{mj}阶实矩阵\end{CJK}, $A$ \begin{CJK}{UTF8}{mj}的特征值为\end{CJK} 0 \begin{CJK}{UTF8}{mj}或\end{CJK} $-1$. \begin{CJK}{UTF8}{mj}证明\end{CJK}:\\
(1) $A$ \begin{CJK}{UTF8}{mj}及\end{CJK} $A+I$ \begin{CJK}{UTF8}{mj}可逆\end{CJK};\\
(2) $A$ \begin{CJK}{UTF8}{mj}正交\end{CJK} $\Leftrightarrow(A+I)^{-1}+\left(A^{\prime}+I\right)^{-1}=I$.

  \item \begin{CJK}{UTF8}{mj}设\end{CJK} $f(x)$ \begin{CJK}{UTF8}{mj}是正的多项式\end{CJK}, \begin{CJK}{UTF8}{mj}即对任意的\end{CJK} $x$ \begin{CJK}{UTF8}{mj}有\end{CJK} $f(x)>0$, \begin{CJK}{UTF8}{mj}又设\end{CJK} $A$ \begin{CJK}{UTF8}{mj}是实对称阵\end{CJK}, \begin{CJK}{UTF8}{mj}证明\end{CJK}:\\
(1) $f(A)$ \begin{CJK}{UTF8}{mj}是正定的\end{CJK};\\
(2) $A^{2}+I$ \begin{CJK}{UTF8}{mj}可逆\end{CJK}.

  \item \begin{CJK}{UTF8}{mj}设\end{CJK} $A$ \begin{CJK}{UTF8}{mj}是\end{CJK} $m \times n$ \begin{CJK}{UTF8}{mj}矩阵\end{CJK}, $B$ \begin{CJK}{UTF8}{mj}是\end{CJK} $n \times m$ \begin{CJK}{UTF8}{mj}矩阵\end{CJK}, \begin{CJK}{UTF8}{mj}且\end{CJK} $m \leq n$. \begin{CJK}{UTF8}{mj}证明\end{CJK}:

\end{enumerate}
$$
\operatorname{det}(\lambda E-B A)=\lambda^{n-m} \operatorname{det}(\lambda E-A B)
$$

\begin{enumerate}
  \setcounter{enumi}{8}
  \item \begin{CJK}{UTF8}{mj}设\end{CJK} $\mathscr{A}$ \begin{CJK}{UTF8}{mj}是线性空间\end{CJK} $V$ \begin{CJK}{UTF8}{mj}的线性变换\end{CJK}, \begin{CJK}{UTF8}{mj}对\end{CJK} $V$ \begin{CJK}{UTF8}{mj}中任意的\end{CJK} $\alpha, \beta$ \begin{CJK}{UTF8}{mj}有\end{CJK}
\end{enumerate}
$$
(\mathscr{A}(\alpha), \beta)=(\alpha, \mathscr{A}(\beta))
$$
\begin{CJK}{UTF8}{mj}证明\end{CJK}:\\
(1) $\operatorname{Im} \mathscr{A}=\operatorname{ker} \mathscr{A}^{\perp}$;\\
(2) $W$ \begin{CJK}{UTF8}{mj}是\end{CJK} $\mathscr{A}$ \begin{CJK}{UTF8}{mj}的不变子空间\end{CJK}, \begin{CJK}{UTF8}{mj}则\end{CJK} $W^{\perp}$ \begin{CJK}{UTF8}{mj}也是\end{CJK} $\mathscr{A}$ \begin{CJK}{UTF8}{mj}的不变子空间\end{CJK}.

\section{4. 华中科技大学 2012 年研究生入学考试试题高等代数 
 李扬 
 微信公众号: sxkyliyang}
\begin{enumerate}
  \item \begin{CJK}{UTF8}{mj}已知\end{CJK}
\end{enumerate}
$$
D=\left|\begin{array}{cccc}
1 & 1 & \cdots & 1 \\
0 & 1 & \cdots & 1 \\
\vdots & \vdots & \ddots & \vdots \\
0 & 0 & \cdots & 1
\end{array}\right|,
$$
\begin{CJK}{UTF8}{mj}求\end{CJK} $D$ \begin{CJK}{UTF8}{mj}的所有代数余子数之和\end{CJK}.

\begin{enumerate}
  \setcounter{enumi}{2}
  \item \begin{CJK}{UTF8}{mj}已知\end{CJK} $A$ \begin{CJK}{UTF8}{mj}为实矩阵\end{CJK}, \begin{CJK}{UTF8}{mj}证明\end{CJK}
\end{enumerate}
$$
\operatorname{rank}\left(A^{\prime} A\right)=\operatorname{rank}(A)=\operatorname{rank}\left(A A^{\prime}\right)
$$
\begin{CJK}{UTF8}{mj}其中\end{CJK} $A^{\prime}$ \begin{CJK}{UTF8}{mj}为\end{CJK} $A$ \begin{CJK}{UTF8}{mj}的转置\end{CJK}.

\begin{enumerate}
  \setcounter{enumi}{3}
  \item \begin{CJK}{UTF8}{mj}已知\end{CJK}
\end{enumerate}
$$
P=\left(\begin{array}{cc}
A & I \\
I & I
\end{array}\right) \text {, }
$$
\begin{CJK}{UTF8}{mj}证明\end{CJK} $P$ \begin{CJK}{UTF8}{mj}可逆的充要条件是\end{CJK} $I-A$ \begin{CJK}{UTF8}{mj}可逆\end{CJK}. \begin{CJK}{UTF8}{mj}并在\end{CJK} $(I-A)^{-1}$ \begin{CJK}{UTF8}{mj}已知情况下求\end{CJK} $P^{-1}$.

\begin{enumerate}
  \setcounter{enumi}{4}
  \item \begin{CJK}{UTF8}{mj}已知\end{CJK} $\mathscr{A}, \mathscr{B}, \mathscr{C}, \mathscr{D}$ \begin{CJK}{UTF8}{mj}为\end{CJK} $V$ \begin{CJK}{UTF8}{mj}上的线性变换\end{CJK}, \begin{CJK}{UTF8}{mj}并两两可交换\end{CJK}, \begin{CJK}{UTF8}{mj}并有\end{CJK} $\mathscr{A} \mathscr{C}+\mathscr{B} \mathscr{D}=\mathscr{I}($ \begin{CJK}{UTF8}{mj}单位变换\end{CJK}), \begin{CJK}{UTF8}{mj}证明\end{CJK}
\end{enumerate}
$$
\operatorname{ker} \mathscr{A} \mathscr{B}=\operatorname{ker} \mathscr{A}+\operatorname{ker} \mathscr{B}
$$
\begin{CJK}{UTF8}{mj}并且和为直和\end{CJK}.

\begin{enumerate}
  \setcounter{enumi}{5}
  \item \begin{CJK}{UTF8}{mj}已知\end{CJK} $A$ \begin{CJK}{UTF8}{mj}为全\end{CJK} 1 \begin{CJK}{UTF8}{mj}阵\end{CJK}.
\end{enumerate}
(1) \begin{CJK}{UTF8}{mj}求\end{CJK} $A$ \begin{CJK}{UTF8}{mj}的特征多项式与是小多项式\end{CJK};

(2) \begin{CJK}{UTF8}{mj}证明\end{CJK} $A$ \begin{CJK}{UTF8}{mj}可对角化\end{CJK}, \begin{CJK}{UTF8}{mj}并求\end{CJK} $P$, \begin{CJK}{UTF8}{mj}使得\end{CJK} $P^{-1} A P$ \begin{CJK}{UTF8}{mj}为对角阵\end{CJK}.

\begin{enumerate}
  \setcounter{enumi}{6}
  \item \begin{CJK}{UTF8}{mj}求正交变换化\end{CJK}
\end{enumerate}
$$
x y+y z+z x=1
$$
\begin{CJK}{UTF8}{mj}为标准方程\end{CJK}, \begin{CJK}{UTF8}{mj}并指出曲面类型\end{CJK}.

\begin{enumerate}
  \setcounter{enumi}{7}
  \item \begin{CJK}{UTF8}{mj}已知\end{CJK} $A, B$ \begin{CJK}{UTF8}{mj}为实对称矩阵\end{CJK}
\end{enumerate}
(1) \begin{CJK}{UTF8}{mj}若\end{CJK} $A, B$ \begin{CJK}{UTF8}{mj}正定\end{CJK}, $A B=B A$, \begin{CJK}{UTF8}{mj}证明\end{CJK} $A B$ \begin{CJK}{UTF8}{mj}也正定\end{CJK};

(2) \begin{CJK}{UTF8}{mj}若\end{CJK} $A, B$ \begin{CJK}{UTF8}{mj}半正定\end{CJK}, \begin{CJK}{UTF8}{mj}证明\end{CJK} $A+B$ \begin{CJK}{UTF8}{mj}也半正定\end{CJK}, \begin{CJK}{UTF8}{mj}若还有\end{CJK} $A$ \begin{CJK}{UTF8}{mj}正定\end{CJK}, \begin{CJK}{UTF8}{mj}证明\end{CJK} $A+B$ \begin{CJK}{UTF8}{mj}也正定\end{CJK}.

\begin{enumerate}
  \setcounter{enumi}{8}
  \item $V$ \begin{CJK}{UTF8}{mj}为实数域上的\end{CJK} $2 n+1$ \begin{CJK}{UTF8}{mj}维空间\end{CJK}, $\mathscr{A}, \mathscr{B}$ \begin{CJK}{UTF8}{mj}为\end{CJK} $V$ \begin{CJK}{UTF8}{mj}上的线性变换\end{CJK}, \begin{CJK}{UTF8}{mj}且\end{CJK} $\mathscr{A} \mathscr{B}=\mathscr{B} \mathscr{A}$, \begin{CJK}{UTF8}{mj}证明存在\end{CJK} $\lambda, \mu \in \mathbb{R}, v \in V$ \begin{CJK}{UTF8}{mj}且\end{CJK} $v \neq 0$, \begin{CJK}{UTF8}{mj}使得\end{CJK}
\end{enumerate}
$$
\mathscr{A}(v)=\lambda v, \mathscr{B}(v)=\mu v
$$

\section{5. 华中科技大学 2013 年研究生入学考试试题高等代数 
 李扬 
 微信公众号: sxkyliyang}
\begin{enumerate}
  \item \begin{CJK}{UTF8}{mj}设\end{CJK} $n$ \begin{CJK}{UTF8}{mj}阶行列式\end{CJK}
\end{enumerate}
$$
D_{n}=\left|\begin{array}{cccc}
x & a & \cdots & a \\
b & x & \cdots & a \\
\vdots & \vdots & \ddots & \vdots \\
b & b & \cdots & x
\end{array}\right| .
$$
(1) \begin{CJK}{UTF8}{mj}求\end{CJK} $D_{n}$.

(2) \begin{CJK}{UTF8}{mj}计算\end{CJK} $\sum_{i=1}^{n} D_{n i}$, \begin{CJK}{UTF8}{mj}其中\end{CJK} $D_{n i}$ \begin{CJK}{UTF8}{mj}表示相应元素的代数余子式\end{CJK}.

2 . \begin{CJK}{UTF8}{mj}设其次线性方程组\end{CJK}
$$
\begin{gathered}
\left\{\begin{array}{l}
x_{1}-x_{2}=0 \\
x_{2}+x_{4}=0
\end{array}\right. \\
\left\{\begin{array}{l}
x_{1}+x_{2}+x_{3}=0 \\
x_{2}+x_{3}-x_{4}=0
\end{array}\right.
\end{gathered}
$$
(1) \begin{CJK}{UTF8}{mj}分别给出方程组\end{CJK} (i) \begin{CJK}{UTF8}{mj}与\end{CJK} (ii) \begin{CJK}{UTF8}{mj}的一个基础解系\end{CJK};

(2) \begin{CJK}{UTF8}{mj}给出\end{CJK} (i) \begin{CJK}{UTF8}{mj}和\end{CJK} (ii) \begin{CJK}{UTF8}{mj}的全部公共解\end{CJK}.

\begin{enumerate}
  \setcounter{enumi}{3}
  \item \begin{CJK}{UTF8}{mj}设\end{CJK} $A, B$ \begin{CJK}{UTF8}{mj}是\end{CJK} $n$ \begin{CJK}{UTF8}{mj}阶幂等矩阵\end{CJK} (\begin{CJK}{UTF8}{mj}即\end{CJK} $A^{2}=A, B^{2}=B$ ) \begin{CJK}{UTF8}{mj}并且\end{CJK} $I-A-B$ \begin{CJK}{UTF8}{mj}可逆\end{CJK}, \begin{CJK}{UTF8}{mj}证明\end{CJK}
\end{enumerate}
$$
\operatorname{rank}(A)=\operatorname{rank}(B)
$$

\begin{enumerate}
  \setcounter{enumi}{4}
  \item \begin{CJK}{UTF8}{mj}设\end{CJK} $V$ \begin{CJK}{UTF8}{mj}是域\end{CJK} $\mathbb{F}$ \begin{CJK}{UTF8}{mj}上所有\end{CJK} $n$ \begin{CJK}{UTF8}{mj}阶矩阵所构成的\end{CJK} $n \times n$ \begin{CJK}{UTF8}{mj}维线性空间\end{CJK}, \begin{CJK}{UTF8}{mj}固定\end{CJK} $M$, \begin{CJK}{UTF8}{mj}定义映射\end{CJK}
\end{enumerate}
$$
\mathscr{A}: V \rightarrow V, \mathscr{A}(A)=M A-A M
$$
(1)\begin{CJK}{UTF8}{mj}证明\end{CJK} $\mathscr{A}$ \begin{CJK}{UTF8}{mj}是线性空间\end{CJK} $V$ \begin{CJK}{UTF8}{mj}的线性变换\end{CJK};
$$
M=\left[\begin{array}{cccc}
d_{1} & & & \\
& d_{2} & & \\
& & \ddots & \\
& & & d_{\gamma}
\end{array}\right]
$$
\begin{CJK}{UTF8}{mj}为对角矩阵\end{CJK}, \begin{CJK}{UTF8}{mj}并且\end{CJK} $d_{i}=d_{j}(i \neq j), 1 \leq i, j \leq \gamma$, \begin{CJK}{UTF8}{mj}求\end{CJK} $\mathscr{A}$ \begin{CJK}{UTF8}{mj}的核\end{CJK} $\operatorname{ker}(\mathscr{A})$;

(3) \begin{CJK}{UTF8}{mj}令\end{CJK} $M=\left[\begin{array}{ll}d_{1} & \\ & d_{2}\end{array}\right]$ \begin{CJK}{UTF8}{mj}为对角阵\end{CJK}, \begin{CJK}{UTF8}{mj}取\end{CJK} $V$ \begin{CJK}{UTF8}{mj}的基\end{CJK}
$$
E_{11}=\left[\begin{array}{ll}
1 & 0 \\
0 & 1
\end{array}\right], E_{12}=\left[\begin{array}{ll}
1 & 0 \\
0 & 0
\end{array}\right], E_{21}=\left[\begin{array}{ll}
0 & 0 \\
0 & 1
\end{array}\right], E_{22}=\left[\begin{array}{ll}
0 & 0 \\
1 & 0
\end{array}\right]
$$
\begin{CJK}{UTF8}{mj}求\end{CJK} $\mathscr{A}$ \begin{CJK}{UTF8}{mj}关于这组基下的矩阵\end{CJK}. 5. \begin{CJK}{UTF8}{mj}设\end{CJK} $A=\left(a_{i j}\right)$ \begin{CJK}{UTF8}{mj}是\end{CJK} $n$ \begin{CJK}{UTF8}{mj}阶方阵\end{CJK}, \begin{CJK}{UTF8}{mj}并且\end{CJK} $a_{i j}>0(1 \leq i, j \leq n)$,
$$
\sum_{k=1}^{n} a_{i k}=1(1 \leq i \leq n)
$$
(1) \begin{CJK}{UTF8}{mj}证明\end{CJK}: \begin{CJK}{UTF8}{mj}对于\end{CJK} $A$ \begin{CJK}{UTF8}{mj}的任一特征值\end{CJK} $\lambda$, \begin{CJK}{UTF8}{mj}均有\end{CJK} $|\lambda| \leq 1$ \begin{CJK}{UTF8}{mj}并且\end{CJK} 1 \begin{CJK}{UTF8}{mj}是\end{CJK} $A$ \begin{CJK}{UTF8}{mj}的特征值\end{CJK};

(2) \begin{CJK}{UTF8}{mj}若\end{CJK} $A$ \begin{CJK}{UTF8}{mj}可逆\end{CJK}, \begin{CJK}{UTF8}{mj}求\end{CJK} $A^{-1}$ \begin{CJK}{UTF8}{mj}的每行之和\end{CJK}.

\begin{enumerate}
  \setcounter{enumi}{6}
  \item \begin{CJK}{UTF8}{mj}设\end{CJK} $n$ \begin{CJK}{UTF8}{mj}阶矩阵\end{CJK} $A$ \begin{CJK}{UTF8}{mj}的\end{CJK} $n$ \begin{CJK}{UTF8}{mj}个特征值两两不同\end{CJK}, $A B=B A$ \begin{CJK}{UTF8}{mj}证明\end{CJK}:
\end{enumerate}
(1) $A$ \begin{CJK}{UTF8}{mj}与\end{CJK} $B$ \begin{CJK}{UTF8}{mj}有相同的特征向量\end{CJK};

(2) $A, B$ \begin{CJK}{UTF8}{mj}可对角化\end{CJK}.

\begin{enumerate}
  \setcounter{enumi}{7}
  \item \begin{CJK}{UTF8}{mj}设\end{CJK} $A, B$ \begin{CJK}{UTF8}{mj}都是\end{CJK} $n$ \begin{CJK}{UTF8}{mj}阶实对称矩阵\end{CJK}, \begin{CJK}{UTF8}{mj}并且\end{CJK} $A$ \begin{CJK}{UTF8}{mj}是正定矩阵\end{CJK}. \begin{CJK}{UTF8}{mj}证明存在可逆矩阵\end{CJK} $P$ \begin{CJK}{UTF8}{mj}使得\end{CJK}
\end{enumerate}
$$
P^{\prime} A P=I
$$
\begin{CJK}{UTF8}{mj}并且\end{CJK} $P^{\prime} B P$ \begin{CJK}{UTF8}{mj}是对角矩阵\end{CJK}.

\begin{enumerate}
  \setcounter{enumi}{8}
  \item \begin{CJK}{UTF8}{mj}设\end{CJK} $\alpha, \beta, \gamma$ \begin{CJK}{UTF8}{mj}是欧式空间\end{CJK} $\mathbb{R}^{n}$ \begin{CJK}{UTF8}{mj}的向量\end{CJK}, \begin{CJK}{UTF8}{mj}并且\end{CJK}
\end{enumerate}
$$
\alpha+\beta+\gamma=0
$$
(1) \begin{CJK}{UTF8}{mj}如果\end{CJK} $(\alpha, \beta)>0$, \begin{CJK}{UTF8}{mj}证明\end{CJK} $(\alpha, \gamma)>0,(\beta, \gamma)>0$, \begin{CJK}{UTF8}{mj}并且\end{CJK} $|\gamma|>\max \{|\alpha|,|\beta|\}$.

(2) \begin{CJK}{UTF8}{mj}如果\end{CJK} $(\alpha, \beta)<0$, \begin{CJK}{UTF8}{mj}证明\end{CJK} $(\alpha, \beta)>\max \{(\alpha, \beta),(\beta, \gamma)\}$, \begin{CJK}{UTF8}{mj}并且\end{CJK} $|\gamma|>\max \{|\alpha|,|\beta|\}$.

(3) \begin{CJK}{UTF8}{mj}试说明\end{CJK} (1) \begin{CJK}{UTF8}{mj}与\end{CJK} (2) \begin{CJK}{UTF8}{mj}的几何含义\end{CJK}.

\section{6. 华中科技大学 2015 年研究生入学考试试题高等代数 
 李扬 
 微信公众号: sxkyliyang}
\begin{enumerate}
  \item \begin{CJK}{UTF8}{mj}证明\end{CJK}
\end{enumerate}
$$
\left|\begin{array}{cccc}
A_{22} & A_{23} & \cdots & A_{2 n} \\
A_{32} & A_{33} & \cdots & A_{3 n} \\
\vdots & \vdots & & \vdots \\
A_{n 2} & A_{n 3} & \cdots & A_{n n}
\end{array}\right|=a_{11}|A|^{n-2}
$$
$A=\left(a_{i j}\right)_{n \times n}, A_{i j}$ \begin{CJK}{UTF8}{mj}为\end{CJK} $A$ \begin{CJK}{UTF8}{mj}的代数余子式\end{CJK}.

\begin{enumerate}
  \setcounter{enumi}{2}
  \item \begin{CJK}{UTF8}{mj}做初等行变换\end{CJK}, \begin{CJK}{UTF8}{mj}求极大线性无关组\end{CJK}.

  \item \begin{CJK}{UTF8}{mj}实质为\end{CJK}

\end{enumerate}
$$
\left[\begin{array}{lll}
1 & 1 & 1 \\
1 & 1 & 1 \\
1 & 1 & 1
\end{array}\right]
$$
\begin{CJK}{UTF8}{mj}求正交阵使其对角化\end{CJK}.

\begin{enumerate}
  \setcounter{enumi}{4}
  \item $A$ \begin{CJK}{UTF8}{mj}为\end{CJK} $n$ \begin{CJK}{UTF8}{mj}阶反对称矩阵\end{CJK}, $b$ \begin{CJK}{UTF8}{mj}为\end{CJK} $n$ \begin{CJK}{UTF8}{mj}维列向量\end{CJK}, $A X=b$ \begin{CJK}{UTF8}{mj}有解\end{CJK}, \begin{CJK}{UTF8}{mj}则\end{CJK}
\end{enumerate}
$$
\mathrm{r}(A)=\mathrm{r}\left(\begin{array}{cc}
A & b \\
-b^{\prime} & 0
\end{array}\right)
$$

\begin{enumerate}
  \setcounter{enumi}{5}
  \item \begin{CJK}{UTF8}{mj}设\end{CJK} $\mathscr{A}$ \begin{CJK}{UTF8}{mj}是欧式空间\end{CJK} $V$ \begin{CJK}{UTF8}{mj}的线性变换\end{CJK}, $\mathscr{B}$ \begin{CJK}{UTF8}{mj}是\end{CJK} $V$ \begin{CJK}{UTF8}{mj}的一个变换\end{CJK}, \begin{CJK}{UTF8}{mj}且对\end{CJK} $\forall \alpha, \beta \in V$,
\end{enumerate}
\begin{CJK}{UTF8}{mj}都有\end{CJK}
$$
(\mathscr{A}(\alpha), \beta)+(\alpha, \mathscr{B}(\beta))=0
$$
\begin{CJK}{UTF8}{mj}证明\end{CJK}:

(1) $\mathscr{B}$ \begin{CJK}{UTF8}{mj}是\end{CJK} $V$ \begin{CJK}{UTF8}{mj}的线性变换\end{CJK};

(2) $\mathscr{B}$ \begin{CJK}{UTF8}{mj}的值域\end{CJK} $\operatorname{Im} \mathscr{B}$ \begin{CJK}{UTF8}{mj}等于\end{CJK} $\operatorname{ker} \mathscr{A}$ \begin{CJK}{UTF8}{mj}的正交补\end{CJK}.

\begin{enumerate}
  \setcounter{enumi}{6}
  \item \begin{CJK}{UTF8}{mj}对线性空间\end{CJK} $V$, \begin{CJK}{UTF8}{mj}有线性变换\end{CJK} $\mathscr{A}$ \begin{CJK}{UTF8}{mj}的不同特征值\end{CJK} $\lambda_{1}, \cdots, \lambda_{k}$ \begin{CJK}{UTF8}{mj}的相应的特征向量\end{CJK} $\alpha_{1}, \cdots, \alpha_{k}$, \begin{CJK}{UTF8}{mj}若有\end{CJK} $\alpha_{1}+$ $\cdots+\alpha_{k} \in W$, \begin{CJK}{UTF8}{mj}而\end{CJK} $W$ \begin{CJK}{UTF8}{mj}是\end{CJK} $\mathscr{A}$ \begin{CJK}{UTF8}{mj}的不变子空间\end{CJK}, \begin{CJK}{UTF8}{mj}求证\end{CJK}:
\end{enumerate}
$$
\operatorname{dim}(W) \geq k
$$

\begin{enumerate}
  \setcounter{enumi}{7}
  \item \begin{CJK}{UTF8}{mj}设\end{CJK} $A$ \begin{CJK}{UTF8}{mj}为\end{CJK} $n$ \begin{CJK}{UTF8}{mj}阶正定矩阵\end{CJK}, \begin{CJK}{UTF8}{mj}求证\end{CJK}
\end{enumerate}
$$
|A| \leq a_{11} a_{22} \cdots a_{n n}
$$
$a_{i i}$ \begin{CJK}{UTF8}{mj}为矩阵\end{CJK} $A$ \begin{CJK}{UTF8}{mj}主对角元素\end{CJK}.

\begin{enumerate}
  \setcounter{enumi}{8}
  \item (1) \begin{CJK}{UTF8}{mj}不难\end{CJK};
\end{enumerate}
(2) \begin{CJK}{UTF8}{mj}设\end{CJK} $A, B$ \begin{CJK}{UTF8}{mj}是数域\end{CJK} $P$ \begin{CJK}{UTF8}{mj}上的\end{CJK} $n$ \begin{CJK}{UTF8}{mj}阶方阵\end{CJK}, $A, B$ \begin{CJK}{UTF8}{mj}的特征值均在数域\end{CJK} $P$ \begin{CJK}{UTF8}{mj}中\end{CJK}, $A B=B A$, \begin{CJK}{UTF8}{mj}则存在可逆阵\end{CJK} $T$, \begin{CJK}{UTF8}{mj}使\end{CJK} \begin{CJK}{UTF8}{mj}得\end{CJK} $T^{-1} A T, T^{-1} B T$ \begin{CJK}{UTF8}{mj}均为上三角阵\end{CJK}.

(3)\begin{CJK}{UTF8}{mj}用\end{CJK} (1), (2) \begin{CJK}{UTF8}{mj}问结论\end{CJK}.

\section{7. 华中科技大学 2016 年研究生入学考试试题高等代数 
 李扬 
 微信公众号: sxkyliyang}
$1 .$
$$
D=\left|\begin{array}{ccccc}
a_{11} & a_{12} & \cdots & a_{1 n-1} & 1 \\
a_{21} & a_{22} & \cdots & a_{2 n-1} & 1 \\
\vdots & \vdots & & \vdots & \vdots \\
a_{n 1} & a_{n 2} & \cdots & a_{n n-1} & 1
\end{array}\right|,
$$
\begin{CJK}{UTF8}{mj}把\end{CJK} $D$ \begin{CJK}{UTF8}{mj}的第\end{CJK} $j$ \begin{CJK}{UTF8}{mj}行换为\end{CJK} $\alpha_{1}, \alpha_{2}, \cdots, \alpha_{n-1}, 1$ \begin{CJK}{UTF8}{mj}得\end{CJK} $D_{j}(j=1,2, \cdots, n)$, \begin{CJK}{UTF8}{mj}证明\end{CJK}: $D=D_{1}+D_{2}+\cdots+D_{n}$.

\begin{enumerate}
  \setcounter{enumi}{2}
  \item \begin{CJK}{UTF8}{mj}设\end{CJK} $A$ \begin{CJK}{UTF8}{mj}为\end{CJK} $n$ \begin{CJK}{UTF8}{mj}阶正定的实对称方阵\end{CJK}, $y \in \mathbb{R}^{n}$ \begin{CJK}{UTF8}{mj}且\end{CJK} $y \neq 0$. \begin{CJK}{UTF8}{mj}证明极限\end{CJK}
\end{enumerate}
$$
\lim _{m \rightarrow \infty} \frac{y^{\prime} A^{m+1} y}{y^{\prime} A^{m} y}
$$
\begin{CJK}{UTF8}{mj}存在\end{CJK}, \begin{CJK}{UTF8}{mj}且等于\end{CJK} $A$ \begin{CJK}{UTF8}{mj}的一个特征值\end{CJK}.

\begin{enumerate}
  \setcounter{enumi}{3}
  \item \begin{CJK}{UTF8}{mj}设\end{CJK} $A, B \in P^{n \times n}$, \begin{CJK}{UTF8}{mj}且\end{CJK} $A$ \begin{CJK}{UTF8}{mj}的\end{CJK} $n$ \begin{CJK}{UTF8}{mj}个特征值两两互异\end{CJK}, \begin{CJK}{UTF8}{mj}则\end{CJK}
\end{enumerate}
$$
A \text { 的特征向量恒为 } B \text { 的特征向量 } \Leftrightarrow A B=B A \text {. }
$$

\begin{enumerate}
  \setcounter{enumi}{4}
  \item \begin{CJK}{UTF8}{mj}设\end{CJK} $V$ \begin{CJK}{UTF8}{mj}是\end{CJK} $n$ \begin{CJK}{UTF8}{mj}维欧式空间\end{CJK}, $\alpha_{1}, \alpha_{2}, \cdots, \alpha_{n}$ \begin{CJK}{UTF8}{mj}是\end{CJK} $V$ \begin{CJK}{UTF8}{mj}的一组基\end{CJK}, \begin{CJK}{UTF8}{mj}证明\end{CJK}:\begin{CJK}{UTF8}{mj}对于任意一组实数\end{CJK} $b_{1}, b_{2}, \cdots, b_{n}$, \begin{CJK}{UTF8}{mj}恒有\end{CJK} \begin{CJK}{UTF8}{mj}一个向量\end{CJK} $\alpha \in V$, \begin{CJK}{UTF8}{mj}使得\end{CJK}
\end{enumerate}
$$
\left(\alpha, \alpha_{i}\right)=b_{i}, i=1,2, \cdots, n
$$

\begin{enumerate}
  \setcounter{enumi}{5}
  \item \begin{CJK}{UTF8}{mj}设\end{CJK} $A \in M_{n}(K), f_{1}(x), f_{2}(x) \in K[x]$, \begin{CJK}{UTF8}{mj}记\end{CJK} $f(x)=f_{1}(x) f_{2}(x)$. \begin{CJK}{UTF8}{mj}证明\end{CJK}: \begin{CJK}{UTF8}{mj}如果\end{CJK}
\end{enumerate}
$$
\left(f_{1}(x), f_{1}(x)\right)=1
$$
\begin{CJK}{UTF8}{mj}那么\end{CJK} $f(A) Z=0$ \begin{CJK}{UTF8}{mj}的任一解可以唯一的表示成\end{CJK} $f_{1}(A) Z=0$ \begin{CJK}{UTF8}{mj}的一个解与\end{CJK} $f_{2}(A) Z=0$ \begin{CJK}{UTF8}{mj}的一个解的和\end{CJK}.

\begin{enumerate}
  \setcounter{enumi}{6}
  \item \begin{CJK}{UTF8}{mj}证明\end{CJK}:
\end{enumerate}
(1) \begin{CJK}{UTF8}{mj}当\end{CJK} $A$ \begin{CJK}{UTF8}{mj}可逆时\end{CJK}, \begin{CJK}{UTF8}{mj}有\end{CJK}
$$
\left|\begin{array}{ll}
A & B \\
C & D
\end{array}\right|=|A|\left|D-C A^{-1} B\right|
$$
(2) \begin{CJK}{UTF8}{mj}当\end{CJK} $A$ \begin{CJK}{UTF8}{mj}不可逆时\end{CJK}, \begin{CJK}{UTF8}{mj}等式成立吗\end{CJK}?

\begin{enumerate}
  \setcounter{enumi}{7}
  \item \begin{CJK}{UTF8}{mj}设空间\end{CJK} $C^{n \times n}$ \begin{CJK}{UTF8}{mj}上的变换\end{CJK}
\end{enumerate}
$$
\mathscr{B}_{A}(X): X A-A X
$$
\begin{CJK}{UTF8}{mj}其中\end{CJK} $A$ \begin{CJK}{UTF8}{mj}为复矩阵\end{CJK}, \begin{CJK}{UTF8}{mj}证明\end{CJK} $\mathscr{B}_{A}(X)$ \begin{CJK}{UTF8}{mj}的秩至多为\end{CJK} $n^{2}-n$.

\begin{enumerate}
  \setcounter{enumi}{8}
  \item \begin{CJK}{UTF8}{mj}设\end{CJK} $\alpha_{1}, \alpha_{2}, \cdots, \alpha_{n}$ \begin{CJK}{UTF8}{mj}为一组基\end{CJK}, \begin{CJK}{UTF8}{mj}证明\end{CJK} $\forall b_{1}, b_{2}, \cdots, b_{n}$ \begin{CJK}{UTF8}{mj}存在唯一的\end{CJK} $\beta_{1}, \beta_{2}, \cdots, \beta_{n}$, \begin{CJK}{UTF8}{mj}使得\end{CJK}
\end{enumerate}
$$
\left(\beta_{i}, \alpha_{i}\right)=b_{i} .
$$

\section{8. 华中科技大学 2017 年研究生入学考试试题高等代数 
 李扬 
 微信公众号: sxkyliyang}
\begin{enumerate}
  \item \begin{CJK}{UTF8}{mj}已知\end{CJK}
\end{enumerate}
$$
A=\left[\begin{array}{ccc}
3 & 2 & 2 \\
k & -1 & k \\
4 & 2 & -3
\end{array}\right],
$$
\begin{CJK}{UTF8}{mj}可对角化\end{CJK}, \begin{CJK}{UTF8}{mj}求\end{CJK} $k$, \begin{CJK}{UTF8}{mj}并求\end{CJK} $P$ \begin{CJK}{UTF8}{mj}使得\end{CJK} $P^{-1} A P$ \begin{CJK}{UTF8}{mj}为对角阵\end{CJK}.

\begin{enumerate}
  \setcounter{enumi}{2}
  \item \begin{CJK}{UTF8}{mj}设\end{CJK}
\end{enumerate}
$$
A=\left[\begin{array}{cc}
1 & 4 \\
-1 & 2
\end{array}\right], B=\left[\begin{array}{ll}
2 & 3 \\
0 & 6
\end{array}\right]
$$
\begin{CJK}{UTF8}{mj}在空间\end{CJK} $P^{2 \times 2}$ \begin{CJK}{UTF8}{mj}中定义变换\end{CJK} $\mathscr{A}(X)=A X B$ \begin{CJK}{UTF8}{mj}证明\end{CJK}: $\mathscr{A}(X)$ \begin{CJK}{UTF8}{mj}为线性变换\end{CJK}, \begin{CJK}{UTF8}{mj}并求\end{CJK} $\mathscr{A}(X)$ \begin{CJK}{UTF8}{mj}的特征多项式\end{CJK}.

\begin{enumerate}
  \setcounter{enumi}{3}
  \item \begin{CJK}{UTF8}{mj}假设空间\end{CJK} $\mathbb{Q}$ (\begin{CJK}{UTF8}{mj}有理数域\end{CJK}) \begin{CJK}{UTF8}{mj}内有\end{CJK}
\end{enumerate}
$$
\mathscr{A}(x)=y, \mathscr{A}(y)=z, \mathscr{A}(z)=x+y
$$
\begin{CJK}{UTF8}{mj}求满足条件的变换\end{CJK} $\mathscr{A}$ \begin{CJK}{UTF8}{mj}生成空间的维数\end{CJK}.

\begin{enumerate}
  \setcounter{enumi}{4}
  \item \begin{CJK}{UTF8}{mj}设\end{CJK} $A, B$ \begin{CJK}{UTF8}{mj}都为正交矩阵\end{CJK}, \begin{CJK}{UTF8}{mj}证明\end{CJK}
\end{enumerate}
$$
|\operatorname{det}(A+B)| \leq 2^{n}
$$

\begin{enumerate}
  \setcounter{enumi}{5}
  \item \begin{CJK}{UTF8}{mj}假设\end{CJK} $A B=B A=0, \mathrm{r}\left(A^{2}\right)=\mathrm{r}(A)$, \begin{CJK}{UTF8}{mj}证明\end{CJK}:
\end{enumerate}
$$
\mathrm{r}(A+B)=\mathrm{r}(A)+\mathrm{r}(B)
$$

\begin{enumerate}
  \setcounter{enumi}{6}
  \item \begin{CJK}{UTF8}{mj}假设一个矩阵\end{CJK} $A$ \begin{CJK}{UTF8}{mj}的前\end{CJK} $n-1$ \begin{CJK}{UTF8}{mj}个顺序主子式都是大于\end{CJK} 0 , \begin{CJK}{UTF8}{mj}但\end{CJK} $\operatorname{det}(A)=0$, \begin{CJK}{UTF8}{mj}证明\end{CJK} $A$ \begin{CJK}{UTF8}{mj}半正定\end{CJK}.

  \item \begin{CJK}{UTF8}{mj}设\end{CJK} $V$ \begin{CJK}{UTF8}{mj}为\end{CJK} $n$ \begin{CJK}{UTF8}{mj}维欧式空间\end{CJK},\begin{CJK}{UTF8}{mj}一组标准正交基为\end{CJK} $\alpha_{1}, \alpha_{2}, \cdots, \alpha_{n}$, \begin{CJK}{UTF8}{mj}证明\end{CJK}:\begin{CJK}{UTF8}{mj}对任意\end{CJK} $r$, \begin{CJK}{UTF8}{mj}存在空间\end{CJK} $w_{1}, w_{2}, \cdots, w_{\gamma}$, \begin{CJK}{UTF8}{mj}使得所有的\end{CJK} $\alpha_{i} \notin w_{1} \cup w_{2} \cup \cdots w_{\gamma}$.

  \item \begin{CJK}{UTF8}{mj}设\end{CJK} $V$ \begin{CJK}{UTF8}{mj}为\end{CJK} $n$ \begin{CJK}{UTF8}{mj}维线性空间\end{CJK}, \begin{CJK}{UTF8}{mj}且\end{CJK} $W=\{x \in V, \delta(x)=2 x\}$. \begin{CJK}{UTF8}{mj}且有\end{CJK} $\delta^{n}(x)=2^{n} \varepsilon$ (\begin{CJK}{UTF8}{mj}恒等变换\end{CJK} $)$. \begin{CJK}{UTF8}{mj}证明\end{CJK}: $W$ \begin{CJK}{UTF8}{mj}为\end{CJK} $V$ \begin{CJK}{UTF8}{mj}的一个\end{CJK} \begin{CJK}{UTF8}{mj}子空间\end{CJK}, \begin{CJK}{UTF8}{mj}且\end{CJK}

\end{enumerate}
$$
\operatorname{dim} W=\frac{\operatorname{tr}(\delta)}{2 n}+\frac{\operatorname{tr}\left(\delta^{2}\right)}{2^{2} n}+\cdots+\frac{\operatorname{tr}\left(\delta^{n}\right)}{2^{n} n}
$$

\section{9. 华中科技大学 2009 年研究生入学考试试题数学分析 
 李扬 
 微信公众号: sxkyliyang}
\begin{enumerate}
  \item (15 \begin{CJK}{UTF8}{mj}分\end{CJK}) \begin{CJK}{UTF8}{mj}设\end{CJK} $G(s, t)$ \begin{CJK}{UTF8}{mj}是二元连续可微函数满足\end{CJK} $a \frac{\partial G}{\partial S}+b \frac{\partial G}{\partial t} \neq 0$, \begin{CJK}{UTF8}{mj}又设\end{CJK} $z=f(x, y)$ \begin{CJK}{UTF8}{mj}是由方程\end{CJK} $G(c x-a z, c y-$ $b z)=0$ \begin{CJK}{UTF8}{mj}所确定的隐函数\end{CJK}, \begin{CJK}{UTF8}{mj}其中\end{CJK} $a, b, c$ \begin{CJK}{UTF8}{mj}为非零常数\end{CJK}. \begin{CJK}{UTF8}{mj}求\end{CJK}
\end{enumerate}
$$
a \frac{\partial z}{\partial x}+b \frac{\partial z}{\partial y} .
$$

\begin{enumerate}
  \setcounter{enumi}{2}
  \item ( 15 \begin{CJK}{UTF8}{mj}分\end{CJK}) \begin{CJK}{UTF8}{mj}计算曲线积分\end{CJK}
\end{enumerate}
$$
I=\int_{C} \frac{(x-y) \mathrm{d} x+(x+4 y) \mathrm{d} y}{x^{2}+4 y^{2}},
$$
\begin{CJK}{UTF8}{mj}其中积分路径\end{CJK} $C$ \begin{CJK}{UTF8}{mj}为单位圆周\end{CJK} $x^{2}+y^{2}=1$ (\begin{CJK}{UTF8}{mj}逆时针方向\end{CJK}).

\begin{enumerate}
  \setcounter{enumi}{3}
  \item (15 \begin{CJK}{UTF8}{mj}分\end{CJK}) \begin{CJK}{UTF8}{mj}计算三重积分\end{CJK}
\end{enumerate}
$$
I=\iiint_{\Omega}(x+y-z)(x-y+z)(y+z-x) \mathrm{d} x \mathrm{~d} y \mathrm{~d} z
$$
\begin{CJK}{UTF8}{mj}其中\end{CJK} $\Omega=\{(x, y, z) \mid 0 \leq x+y-z \leq 1,0 \leq x-y+z \leq 1,0 \leq y+z-x \leq 1\}$.

\begin{enumerate}
  \setcounter{enumi}{4}
  \item ( 15 \begin{CJK}{UTF8}{mj}分\end{CJK}) \begin{CJK}{UTF8}{mj}将函数\end{CJK}
\end{enumerate}
$$
f(x)=x(\pi-x)
$$
\begin{CJK}{UTF8}{mj}在区间\end{CJK} $[0, \pi]$ \begin{CJK}{UTF8}{mj}上展开为余弦级数并求该级数在区间\end{CJK} $[-\pi, \pi]$ \begin{CJK}{UTF8}{mj}上的和函数\end{CJK}.

\begin{enumerate}
  \setcounter{enumi}{5}
  \item (15 \begin{CJK}{UTF8}{mj}分\end{CJK})
\end{enumerate}
(1) (10 \begin{CJK}{UTF8}{mj}分\end{CJK}) \begin{CJK}{UTF8}{mj}求函数\end{CJK}
$$
f(x, y, z)=\ln x+2 \ln y+3 \ln z,(\text { 其中 } x, y, z>0 \text { ) }
$$
\begin{CJK}{UTF8}{mj}在球面\end{CJK} $x^{2}+y^{2}+z^{2}=6 R^{2}$ \begin{CJK}{UTF8}{mj}的最大值\end{CJK}.

(2) ( 5 \begin{CJK}{UTF8}{mj}分\end{CJK}) \begin{CJK}{UTF8}{mj}设\end{CJK} $a, b, c$ \begin{CJK}{UTF8}{mj}为正数\end{CJK}, \begin{CJK}{UTF8}{mj}证明不等式\end{CJK}:
$$
a b^{2} c^{3} \leq 108\left(\frac{a+b+c}{6}\right)^{6}
$$

\begin{enumerate}
  \setcounter{enumi}{6}
  \item (15 \begin{CJK}{UTF8}{mj}分\end{CJK}) \begin{CJK}{UTF8}{mj}讨论函数项级数\end{CJK}
\end{enumerate}
$$
\sum_{n=0}^{\infty} \frac{x^{2}}{\left(1+x^{2}\right)^{n}}
$$
\begin{CJK}{UTF8}{mj}在区间\end{CJK} $(-\infty,+\infty)$ \begin{CJK}{UTF8}{mj}上的一致收敛性\end{CJK}.

\begin{enumerate}
  \setcounter{enumi}{7}
  \item (15 \begin{CJK}{UTF8}{mj}分\end{CJK}) \begin{CJK}{UTF8}{mj}设\end{CJK} $f(x)$ \begin{CJK}{UTF8}{mj}在区间\end{CJK} $(0,1]$ \begin{CJK}{UTF8}{mj}上有连续的右导数\end{CJK}, \begin{CJK}{UTF8}{mj}极限\end{CJK} $\lim _{x \rightarrow 0^{+}} \sqrt{x} \frac{\mathrm{d} f(x)}{\mathrm{d} x}$ \begin{CJK}{UTF8}{mj}存在且有限\end{CJK}. \begin{CJK}{UTF8}{mj}证明函数\end{CJK} $f(x)$ \begin{CJK}{UTF8}{mj}在区\end{CJK} \begin{CJK}{UTF8}{mj}间\end{CJK} $(0,1]$ \begin{CJK}{UTF8}{mj}上一致连续\end{CJK}.

  \item (15 \begin{CJK}{UTF8}{mj}分\end{CJK}) \begin{CJK}{UTF8}{mj}设正项级数\end{CJK} $\sum_{n=1}^{\infty} a_{n}$ \begin{CJK}{UTF8}{mj}发散\end{CJK}, \begin{CJK}{UTF8}{mj}且\end{CJK} $\lim _{n \rightarrow \infty} \frac{a_{n}}{A_{n}}=0$, \begin{CJK}{UTF8}{mj}其中\end{CJK} $A_{n}=\sum_{k=1}^{n} a_{k}$. \begin{CJK}{UTF8}{mj}证明幂级数\end{CJK}

\end{enumerate}
$$
\sum_{n=1}^{\infty} a_{n} x^{n}
$$
\begin{CJK}{UTF8}{mj}的收敛半径\end{CJK} $R=1$. 9. ( 15 \begin{CJK}{UTF8}{mj}分\end{CJK}) \begin{CJK}{UTF8}{mj}设\end{CJK} $f(x) \in C[0,1]$, \begin{CJK}{UTF8}{mj}证明\end{CJK}
$$
\lim _{n \rightarrow \infty} \int_{0}^{1} n x^{n-1} f(x) \mathrm{d} x=f(1)
$$

\begin{enumerate}
  \setcounter{enumi}{10}
  \item ( 15 \begin{CJK}{UTF8}{mj}分\end{CJK}) \begin{CJK}{UTF8}{mj}设\end{CJK} $f(x)$ \begin{CJK}{UTF8}{mj}在区间\end{CJK} $(0,+\infty)$ \begin{CJK}{UTF8}{mj}上连续\end{CJK}, \begin{CJK}{UTF8}{mj}又设广义积分\end{CJK} $\int_{0}^{+\infty} x^{a} f(x) \mathrm{d} x$ \begin{CJK}{UTF8}{mj}和\end{CJK} $\int_{0}^{+\infty} x^{b} f(x) \mathrm{d} x$ \begin{CJK}{UTF8}{mj}均收敛\end{CJK}, \begin{CJK}{UTF8}{mj}其\end{CJK} \begin{CJK}{UTF8}{mj}中\end{CJK} $a<b$. \begin{CJK}{UTF8}{mj}证明\end{CJK}: \begin{CJK}{UTF8}{mj}含参变量广义积分\end{CJK}
\end{enumerate}
$$
J(y)=\int_{0}^{+\infty} x^{y} f(x) \mathrm{d} x
$$
\begin{CJK}{UTF8}{mj}关于参量\end{CJK} $y \in[a, b]$ \begin{CJK}{UTF8}{mj}一致收敛\end{CJK}.

\section{0. 华中科技大学 2010 年研究生入学考试试题数学分析 
 李扬 
 微信公众号: sxkyliyang}
\begin{enumerate}
  \item (15 \begin{CJK}{UTF8}{mj}分\end{CJK}) \begin{CJK}{UTF8}{mj}求极限\end{CJK}
\end{enumerate}
$$
I=\lim _{x \rightarrow+\infty}\left(\sqrt[5]{x^{5}+x^{4}}-\sqrt[5]{x^{5}-x^{4}}\right) .
$$

\begin{enumerate}
  \setcounter{enumi}{2}
  \item ( 15 \begin{CJK}{UTF8}{mj}分\end{CJK}) \begin{CJK}{UTF8}{mj}设函数\end{CJK} $Q(x, y)$ \begin{CJK}{UTF8}{mj}在\end{CJK} $\mathbb{R}^{2}$ \begin{CJK}{UTF8}{mj}上连续可微\end{CJK}, \begin{CJK}{UTF8}{mj}曲线积分\end{CJK} $\int_{L} 2 x y \mathrm{~d} x+Q(x, y) \mathrm{d} y$ \begin{CJK}{UTF8}{mj}与路径无关\end{CJK}, \begin{CJK}{UTF8}{mj}并且对任意的\end{CJK} $t$ \begin{CJK}{UTF8}{mj}均有\end{CJK}:
\end{enumerate}
$$
\int_{(0,0)}^{(t, 1)} 2 x y \mathrm{~d} x+Q(x, y) \mathrm{d} y=\int_{(0,0)}^{(1, t)} 2 x y \mathrm{~d} x+Q(x, y) \mathrm{d} y,
$$
\begin{CJK}{UTF8}{mj}求\end{CJK} $Q(x, y)$.

\begin{enumerate}
  \setcounter{enumi}{3}
  \item (15 \begin{CJK}{UTF8}{mj}分\end{CJK}) \begin{CJK}{UTF8}{mj}设\end{CJK} $f(r)$ \begin{CJK}{UTF8}{mj}在区间\end{CJK} $(-\infty,+\infty)$ \begin{CJK}{UTF8}{mj}上可微\end{CJK}, \begin{CJK}{UTF8}{mj}且\end{CJK} $f(0)=0, f^{\prime}(0)=1$. \begin{CJK}{UTF8}{mj}求极限\end{CJK}:
\end{enumerate}
$$
I=\lim _{r \rightarrow 0^{+}} \frac{1}{r^{4}} \iiint_{x^{2}+y^{2}+z^{2} \leq r^{2}} f\left(\sqrt{x^{2}+y^{2}+z^{2}}\right) \mathrm{d} x \mathrm{~d} y \mathrm{~d} z .
$$

\begin{enumerate}
  \setcounter{enumi}{4}
  \item (15 \begin{CJK}{UTF8}{mj}分\end{CJK}) \begin{CJK}{UTF8}{mj}已知\end{CJK} $\int_{0}^{+\infty} \mathrm{e}^{-y^{2}} \mathrm{~d} y=\frac{\sqrt{\pi}}{2}$, \begin{CJK}{UTF8}{mj}计算含参量广义积分\end{CJK}:
\end{enumerate}
$$
I(x)=\int_{0}^{+\infty} \mathrm{e}^{-y^{2}} \cos (2 x y) \mathrm{d} y, x \in(-\infty,+\infty) .
$$

\begin{enumerate}
  \setcounter{enumi}{5}
  \item ( 15 \begin{CJK}{UTF8}{mj}分\end{CJK}) \begin{CJK}{UTF8}{mj}设\end{CJK} $a_{1}>0$, \begin{CJK}{UTF8}{mj}令\end{CJK}
\end{enumerate}
$$
a_{n+1}=\frac{1}{2}\left(a_{n}+\frac{1}{a_{n}}\right), n=1,2, \cdots
$$
\begin{CJK}{UTF8}{mj}证明数列\end{CJK} $\left\{a_{n}\right\}$ \begin{CJK}{UTF8}{mj}收敛并求极限\end{CJK}.

\begin{enumerate}
  \setcounter{enumi}{6}
  \item (15 \begin{CJK}{UTF8}{mj}分\end{CJK}) \begin{CJK}{UTF8}{mj}设广义积分\end{CJK}
\end{enumerate}
$$
\int_{1}^{+\infty} x f(x) \mathrm{d} x
$$
\begin{CJK}{UTF8}{mj}收敛\end{CJK}, \begin{CJK}{UTF8}{mj}证明广义积分\end{CJK} $\int_{1}^{+\infty} f(x) \mathrm{d} x$ \begin{CJK}{UTF8}{mj}也收敛\end{CJK}.

\begin{enumerate}
  \setcounter{enumi}{7}
  \item ( 15 \begin{CJK}{UTF8}{mj}分\end{CJK}) \begin{CJK}{UTF8}{mj}设\end{CJK} $f(x)$ \begin{CJK}{UTF8}{mj}在区间\end{CJK} $[-1,1]$ \begin{CJK}{UTF8}{mj}上二阶连续可微\end{CJK}, \begin{CJK}{UTF8}{mj}且\end{CJK} $\lim _{x \rightarrow 0} \frac{f(x)}{x}=0$. \begin{CJK}{UTF8}{mj}证明数项级数\end{CJK}
\end{enumerate}
$$
\sum_{n=1}^{\infty} f\left(\frac{1}{n}\right)
$$
\begin{CJK}{UTF8}{mj}绝对收敛\end{CJK}.

\begin{enumerate}
  \setcounter{enumi}{8}
  \item ( 15 \begin{CJK}{UTF8}{mj}分\end{CJK}) \begin{CJK}{UTF8}{mj}设函数项级数\end{CJK} $\sum_{n=1}^{\infty} u_{n}(x)$ \begin{CJK}{UTF8}{mj}在有界闭区间\end{CJK} $[a, b]$ \begin{CJK}{UTF8}{mj}上逐点收敛\end{CJK},\begin{CJK}{UTF8}{mj}其中\end{CJK} $u_{n}(x)$ \begin{CJK}{UTF8}{mj}在区间\end{CJK} $[a, b]$ \begin{CJK}{UTF8}{mj}上可导\end{CJK} $(n=1,2, \cdots)$, \begin{CJK}{UTF8}{mj}并且存在正常数\end{CJK} $M$, \begin{CJK}{UTF8}{mj}使得\end{CJK}:
\end{enumerate}
$$
\left|\sum_{k=1}^{n} u_{k}^{\prime}(x)\right| \leq M, \forall x \in[a, b], \forall n \geq 1
$$
\begin{CJK}{UTF8}{mj}证明函数项级数\end{CJK} $\sum_{n=1}^{\infty} u_{n}(x)$ \begin{CJK}{UTF8}{mj}在区间\end{CJK} $[a, b]$ \begin{CJK}{UTF8}{mj}上一致收敛\end{CJK}. 9. ( 15 \begin{CJK}{UTF8}{mj}分\end{CJK}) \begin{CJK}{UTF8}{mj}设二元函数\end{CJK} $u(t, x)$ \begin{CJK}{UTF8}{mj}在区域\end{CJK} $[0, T) \times[a, b]$ \begin{CJK}{UTF8}{mj}上二阶连续可微\end{CJK}, \begin{CJK}{UTF8}{mj}并且满足\end{CJK}:
$$
\begin{gathered}
\frac{\partial u(t, x)}{\partial t}=\frac{\partial^{2} u(t, x)}{\partial x^{2}}-u(t, x),(t, x) \in[0, T) \times[a, b] \\
\frac{\partial u(t, a)}{\partial x}=\frac{\partial u(t, b)}{\partial x}=0, t \in[0, T)
\end{gathered}
$$
\begin{CJK}{UTF8}{mj}定义函数\end{CJK}
$$
f(t)=\int_{a}^{b}\left[\left(\frac{\partial u(t, x)}{\partial x}\right)^{2}+u^{2}(t, x)\right] \mathrm{d} x, t \in[0, T)
$$
\begin{CJK}{UTF8}{mj}证明\end{CJK}: $f(x)$ \begin{CJK}{UTF8}{mj}在\end{CJK} $[0, T)$ \begin{CJK}{UTF8}{mj}上单调递减\end{CJK}.

\begin{enumerate}
  \setcounter{enumi}{10}
  \item ( 15 \begin{CJK}{UTF8}{mj}分\end{CJK}) \begin{CJK}{UTF8}{mj}设\end{CJK} $f(x)$ \begin{CJK}{UTF8}{mj}在区间\end{CJK} $[0,1]$ \begin{CJK}{UTF8}{mj}上二阶连续可微\end{CJK}, \begin{CJK}{UTF8}{mj}则\end{CJK}
\end{enumerate}
$$
\int_{0}^{1}\left|f^{\prime}(x)\right| \mathrm{d} x \leq 9 \int_{0}^{1}|f(x)| \mathrm{d} x+\int_{0}^{1}\left|f^{\prime \prime}(x)\right| \mathrm{d} x
$$

\section{1. 华中科技大学 2011 年研究生入学考试试题数学分析 
 李扬 
 微信公众号: sxkyliyang}
\begin{enumerate}
  \item \begin{CJK}{UTF8}{mj}求极限\end{CJK}
\end{enumerate}
$$
\lim _{x \rightarrow+\infty} x^{\frac{3}{2}}(2 \sqrt{x}-\sqrt{x+1}-\sqrt{x-1}) .
$$

\begin{enumerate}
  \setcounter{enumi}{2}
  \item \begin{CJK}{UTF8}{mj}设\end{CJK} $f(x)$ \begin{CJK}{UTF8}{mj}一阶连续可微\end{CJK}, $f(0)=0$, \begin{CJK}{UTF8}{mj}且\end{CJK} $D: x^{2}+y^{2} \leq 2 t x$, \begin{CJK}{UTF8}{mj}求极限\end{CJK}
\end{enumerate}
$$
\lim _{t \rightarrow 0^{+}} \frac{\iint_{D} y f\left(\sqrt{x^{2}+y^{2}}\right) \mathrm{d} x \mathrm{~d} y}{t^{4}} .
$$

\begin{enumerate}
  \setcounter{enumi}{3}
  \item \begin{CJK}{UTF8}{mj}设曲面\end{CJK} $S$ \begin{CJK}{UTF8}{mj}是椭球面\end{CJK} $z=\sqrt{2\left(1-x^{2}-y^{2}\right)}$ \begin{CJK}{UTF8}{mj}的上半部分\end{CJK}, \begin{CJK}{UTF8}{mj}设\end{CJK} $\rho$ \begin{CJK}{UTF8}{mj}是原点到椭球面上任一点的切平面的距离\end{CJK}, \begin{CJK}{UTF8}{mj}求\end{CJK}
\end{enumerate}
$$
\iint_{S} \frac{z}{\rho} \mathrm{d} S .
$$

\begin{enumerate}
  \setcounter{enumi}{4}
  \item \begin{CJK}{UTF8}{mj}计算积分\end{CJK}
\end{enumerate}
$$
I=\oint_{L^{+}} y \mathrm{~d} x+z \mathrm{~d} y+x \mathrm{~d} z
$$
\begin{CJK}{UTF8}{mj}其中\end{CJK} $L^{+}$\begin{CJK}{UTF8}{mj}为圆周\end{CJK} $x^{2}+y^{2}+z^{2}=a^{2}, a>0, x+y+z=0$,\begin{CJK}{UTF8}{mj}从\end{CJK} $Z$ \begin{CJK}{UTF8}{mj}轴正向看为逆时针方向\end{CJK}.

\begin{enumerate}
  \setcounter{enumi}{5}
  \item \begin{CJK}{UTF8}{mj}已知\end{CJK} $\sum_{n=1}^{\infty} \frac{a_{n}}{n+1}$ \begin{CJK}{UTF8}{mj}收敛\end{CJK}, \begin{CJK}{UTF8}{mj}试证明等式\end{CJK}
\end{enumerate}
$$
\int_{0}^{1} \sum_{n=1}^{\infty} a_{n} x^{n} \mathrm{~d} x=\sum_{n=1}^{\infty} \frac{a_{n}}{n+1}
$$
\begin{CJK}{UTF8}{mj}并利用之求\end{CJK} $1-\frac{1}{2}+\frac{1}{3}-\frac{1}{4}+\frac{1}{5} \cdots$.

\begin{enumerate}
  \setcounter{enumi}{6}
  \item \begin{CJK}{UTF8}{mj}求无穷积分\end{CJK}
\end{enumerate}
$$
\int_{0}^{+\infty} \frac{\mathrm{e}^{-a x^{2}}-\mathrm{e}^{-b x^{2}}}{x} \mathrm{~d} x(b>a>0)
$$

\begin{enumerate}
  \setcounter{enumi}{7}
  \item \begin{CJK}{UTF8}{mj}设\end{CJK} $a_{n}>0(n=1,2,3 \cdots)$ \begin{CJK}{UTF8}{mj}级数\end{CJK} $\sum_{n=1}^{\infty} a_{n}$ \begin{CJK}{UTF8}{mj}收敛\end{CJK}, $r_{n}=\sum_{k=n}^{\infty} a_{k}$, \begin{CJK}{UTF8}{mj}证明\end{CJK}:
\end{enumerate}
$$
\sum_{n=1}^{\infty} \frac{a_{n}}{r_{n}}
$$
\begin{CJK}{UTF8}{mj}发散\end{CJK}.

\begin{enumerate}
  \setcounter{enumi}{8}
  \item \begin{CJK}{UTF8}{mj}设函数\end{CJK} $f(x)$ \begin{CJK}{UTF8}{mj}在区间\end{CJK} $[0,2 \pi]$ \begin{CJK}{UTF8}{mj}上可积\end{CJK}, \begin{CJK}{UTF8}{mj}证明\end{CJK}:
\end{enumerate}
$$
\frac{1}{2 \pi} \int_{0}^{2 \pi} f(x)(\pi-x) \mathrm{d} x=\sum_{n=1}^{\infty} \frac{b_{n}}{n}
$$
\begin{CJK}{UTF8}{mj}其中\end{CJK} $b_{n}=\frac{1}{\pi} \int_{0}^{2 \pi} f(x) \sin n x \mathrm{~d} x,(n=1,2,3, \cdots)$.

\begin{enumerate}
  \setcounter{enumi}{9}
  \item \begin{CJK}{UTF8}{mj}设\end{CJK} $f(x)$ \begin{CJK}{UTF8}{mj}在\end{CJK} $[0,1]$ \begin{CJK}{UTF8}{mj}上二阶连续可微\end{CJK}, \begin{CJK}{UTF8}{mj}证明\end{CJK}:
\end{enumerate}
$$
\int_{0}^{1}\left|f^{\prime}(x)\right| \mathrm{d} x \leq 9 \int_{0}^{1}|f(x)| \mathrm{d} x+\int_{0}^{1}\left|f^{\prime \prime}(x)\right| \mathrm{d} x
$$

\section{2. 华中科技大学 2012 年研究生入学考试试题数学分析 
 李扬 
 微信公众号: sxkyliyang}
\begin{enumerate}
  \item (15 \begin{CJK}{UTF8}{mj}分\end{CJK}) (1) \begin{CJK}{UTF8}{mj}求极限\end{CJK}
\end{enumerate}
$$
\lim _{x \rightarrow 0} \frac{1}{x}\left(\frac{1}{x}-\frac{1}{\sin x}\right) ;
$$
(2) \begin{CJK}{UTF8}{mj}设\end{CJK} $x_{1}=\sqrt{2}, x_{n+1}=\sqrt{2 x_{n}}$. \begin{CJK}{UTF8}{mj}证明\end{CJK} $\left\{x_{n}\right\}$ \begin{CJK}{UTF8}{mj}收玫并求其极限\end{CJK}.

\begin{enumerate}
  \setcounter{enumi}{2}
  \item (15 \begin{CJK}{UTF8}{mj}分\end{CJK}) \begin{CJK}{UTF8}{mj}求下列曲线在第一象限围成的图像的面积\end{CJK}:
\end{enumerate}
$$
y=x^{2}, 2 y=x^{2}, x y=1, x y=2
$$

\begin{enumerate}
  \setcounter{enumi}{3}
  \item ( 15 \begin{CJK}{UTF8}{mj}分\end{CJK}) \begin{CJK}{UTF8}{mj}求下列圆环\end{CJK} $L$ \begin{CJK}{UTF8}{mj}的质量\end{CJK}: \begin{CJK}{UTF8}{mj}已知圆环\end{CJK}
\end{enumerate}
$$
L:\left\{\begin{array}{l}
x^{2}+y^{2}+z^{2}=1 \\
x+y+z=0
\end{array}\right.
$$
\begin{CJK}{UTF8}{mj}其线密度为\end{CJK} $\rho(x, y, z)=(x-1)^{2}+(y-1)^{2}+(z-1)^{2}$.

\begin{enumerate}
  \setcounter{enumi}{4}
  \item ( 15 \begin{CJK}{UTF8}{mj}分\end{CJK}) \begin{CJK}{UTF8}{mj}展开\end{CJK}
\end{enumerate}
$$
f(x)=|\cos x|
$$
\begin{CJK}{UTF8}{mj}为\end{CJK} $[-\pi, \pi]$ \begin{CJK}{UTF8}{mj}上的\end{CJK} Fourier \begin{CJK}{UTF8}{mj}级数\end{CJK}.

\begin{enumerate}
  \setcounter{enumi}{5}
  \item ( 15 \begin{CJK}{UTF8}{mj}分\end{CJK}) \begin{CJK}{UTF8}{mj}求幂级数\end{CJK}
\end{enumerate}
$$
\sum_{n=0}^{\infty} \frac{n+1}{n !} x^{n}
$$
\begin{CJK}{UTF8}{mj}的收敛域与和函数\end{CJK}.

\begin{enumerate}
  \setcounter{enumi}{6}
  \item ( 15 \begin{CJK}{UTF8}{mj}分\end{CJK}) \begin{CJK}{UTF8}{mj}已知\end{CJK} $\sum_{n=1}^{\infty} a_{n}$ \begin{CJK}{UTF8}{mj}为发散的正项级数\end{CJK}, $S_{n}$ \begin{CJK}{UTF8}{mj}为其部分和\end{CJK}, \begin{CJK}{UTF8}{mj}用\end{CJK} Cauchy \begin{CJK}{UTF8}{mj}收敛原理证明\end{CJK}
\end{enumerate}
$$
\sum_{n=1}^{\infty} \frac{a_{n}}{S_{n}}
$$
\begin{CJK}{UTF8}{mj}发散\end{CJK}.

\begin{enumerate}
  \setcounter{enumi}{7}
  \item ( 15 \begin{CJK}{UTF8}{mj}分\end{CJK}) \begin{CJK}{UTF8}{mj}已知\end{CJK} $f(x)$ \begin{CJK}{UTF8}{mj}在\end{CJK} $[0,+\infty)$ \begin{CJK}{UTF8}{mj}上连续\end{CJK}, $\lim _{x \rightarrow+\infty} f(x)$ \begin{CJK}{UTF8}{mj}存在且有限\end{CJK}, \begin{CJK}{UTF8}{mj}证明\end{CJK} $f(x)$ \begin{CJK}{UTF8}{mj}在\end{CJK} $[0,+\infty)$ \begin{CJK}{UTF8}{mj}上有界\end{CJK}.

  \item ( 15 \begin{CJK}{UTF8}{mj}分\end{CJK}) \begin{CJK}{UTF8}{mj}已知反常积分\end{CJK} $\int_{1}^{+\infty} x f(x) \mathrm{d} x$ \begin{CJK}{UTF8}{mj}收敛\end{CJK}, \begin{CJK}{UTF8}{mj}证明含参量反常积分\end{CJK}

\end{enumerate}
$$
I(y)=\int_{1}^{+\infty} x^{y} f(x) \mathrm{d} x
$$
\begin{CJK}{UTF8}{mj}在\end{CJK} $[0,1]$ \begin{CJK}{UTF8}{mj}上一致收敛\end{CJK}.

\begin{enumerate}
  \setcounter{enumi}{9}
  \item ( 15 \begin{CJK}{UTF8}{mj}分\end{CJK}) \begin{CJK}{UTF8}{mj}已知\end{CJK} $\Omega$ \begin{CJK}{UTF8}{mj}为三维空间中的有界区域\end{CJK}, $\Omega$ \begin{CJK}{UTF8}{mj}的边界为分片光滑的曲面\end{CJK}, $\mathbf{n}$ \begin{CJK}{UTF8}{mj}为外法向量\end{CJK}, $u(x, y, z)$ \begin{CJK}{UTF8}{mj}在\end{CJK} $\Omega$ \begin{CJK}{UTF8}{mj}上二阶连续可偏导\end{CJK}. \begin{CJK}{UTF8}{mj}求证\end{CJK}:
\end{enumerate}
$$
\iiint_{\Omega}\left(\frac{\partial^{2} u}{\partial x^{2}}+\frac{\partial^{2} u}{\partial y^{2}}+\frac{\partial^{2} u}{\partial z^{2}}\right) \mathrm{d} x \mathrm{~d} y \mathrm{~d} z=\iint_{\partial \Omega} \frac{\partial u}{\partial \mathbf{n}} \mathrm{dS}
$$

\begin{enumerate}
  \setcounter{enumi}{10}
  \item (15 \begin{CJK}{UTF8}{mj}分\end{CJK}) \begin{CJK}{UTF8}{mj}已知\end{CJK} $f(x)$ \begin{CJK}{UTF8}{mj}在\end{CJK} $[0,1]$ \begin{CJK}{UTF8}{mj}上二阶连续可导\end{CJK}, \begin{CJK}{UTF8}{mj}证明\end{CJK}:
\end{enumerate}
$$
\max _{x \in[0,1]}\left|f^{\prime}(x)\right| \leq|f(1)-f(0)|+\int_{0}^{1} f^{\prime \prime}(x) \mathrm{d} x
$$

\section{3. 华中科技大学 2013 年研究生入学考试试题数学分析 
 李扬 
 微信公众号: sxkyliyang}
\begin{enumerate}
  \item (15 \begin{CJK}{UTF8}{mj}分\end{CJK}) (1) \begin{CJK}{UTF8}{mj}求极限\end{CJK}
\end{enumerate}
$$
I=\lim _{x \rightarrow 0^{+}} \frac{1-\cos x}{\int_{0}^{x} \frac{\ln (1+x y)}{y} \mathrm{~d} y} .
$$
(2) \begin{CJK}{UTF8}{mj}计算含参变量广义积分\end{CJK}
$$
F(x)=\int_{0}^{+\infty} \frac{\sin (x y)}{y \mathrm{e}^{y}} \mathrm{~d} y .
$$

\begin{enumerate}
  \setcounter{enumi}{2}
  \item ( 15 \begin{CJK}{UTF8}{mj}分\end{CJK}) \begin{CJK}{UTF8}{mj}设\end{CJK} $R(x, y)$ \begin{CJK}{UTF8}{mj}是曲线\end{CJK} $C: \frac{x^{2}}{a^{2}}+\frac{y^{2}}{b^{2}}=1$ \begin{CJK}{UTF8}{mj}上点\end{CJK} $(x, y)$ \begin{CJK}{UTF8}{mj}的曲率半径\end{CJK} $(a, b>0)$, \begin{CJK}{UTF8}{mj}计算第一型曲线积分\end{CJK}
\end{enumerate}
$$
\int_{C} R(x, y) \mathrm{ds} .
$$

\begin{enumerate}
  \setcounter{enumi}{3}
  \item (15 \begin{CJK}{UTF8}{mj}分\end{CJK}) \begin{CJK}{UTF8}{mj}设\end{CJK} $\Omega$ \begin{CJK}{UTF8}{mj}是椭球体\end{CJK} $C: \frac{x^{2}}{a^{2}}+\frac{y^{2}}{b^{2}}+\frac{z^{2}}{c^{2}} \leq 1$ \begin{CJK}{UTF8}{mj}在第一卦限中的部分\end{CJK} $(a, b, c>0)$. \begin{CJK}{UTF8}{mj}计算三重积分\end{CJK}
\end{enumerate}
$$
I=\iiint_{\Omega} \frac{x^{2}}{a^{2}}+\frac{y^{2}}{b^{2}}+\frac{z^{2}}{c^{2}} \mathrm{~d} x \mathrm{~d} y \mathrm{~d} z .
$$

\begin{enumerate}
  \setcounter{enumi}{4}
  \item (15 \begin{CJK}{UTF8}{mj}分\end{CJK}) \begin{CJK}{UTF8}{mj}求幂级数\end{CJK}
\end{enumerate}
$$
\sum_{n=1}^{\infty}(-1)^{n-1} n^{2} x^{n}
$$
\begin{CJK}{UTF8}{mj}的收敛域及其和函数\end{CJK}.

\begin{enumerate}
  \setcounter{enumi}{5}
  \item (15 \begin{CJK}{UTF8}{mj}分\end{CJK}) \begin{CJK}{UTF8}{mj}在椭球面\end{CJK}
\end{enumerate}
$$
C: \frac{x^{2}}{a^{2}}+\frac{y^{2}}{b^{2}}+\frac{z^{2}}{c^{2}}=1
$$
\begin{CJK}{UTF8}{mj}上确定一点\end{CJK} $\left(x_{0}, y_{0}, z_{0}\right)$ (\begin{CJK}{UTF8}{mj}其中\end{CJK} $x_{0}, y_{0}, z_{0}>0$ ) \begin{CJK}{UTF8}{mj}使得过该点的切平面与三个坐标面所围成的四边体的体积\end{CJK} \begin{CJK}{UTF8}{mj}是小\end{CJK}, \begin{CJK}{UTF8}{mj}并求出最小体积\end{CJK}.

\begin{enumerate}
  \setcounter{enumi}{6}
  \item ( 15 \begin{CJK}{UTF8}{mj}分\end{CJK}) \begin{CJK}{UTF8}{mj}设\end{CJK} $f(x)$ \begin{CJK}{UTF8}{mj}在区间\end{CJK} $[0,+\infty)$ \begin{CJK}{UTF8}{mj}上连续\end{CJK}, \begin{CJK}{UTF8}{mj}又设极限\end{CJK} $\lim _{x \rightarrow \infty} f(x)$ \begin{CJK}{UTF8}{mj}存在且有限\end{CJK}, \begin{CJK}{UTF8}{mj}证明\end{CJK}: $f(x)$ \begin{CJK}{UTF8}{mj}在区间\end{CJK} $[0,+\infty)$ \begin{CJK}{UTF8}{mj}上有界\end{CJK}.

  \item ( 15 \begin{CJK}{UTF8}{mj}分\end{CJK}) \begin{CJK}{UTF8}{mj}设函数\end{CJK} $f(x)$ \begin{CJK}{UTF8}{mj}和\end{CJK} $g(x)$ \begin{CJK}{UTF8}{mj}在区间\end{CJK} $[a, b]$ \begin{CJK}{UTF8}{mj}上可微且\end{CJK} $g(x) \neq 0(x \in(a, b))$, \begin{CJK}{UTF8}{mj}证明存在\end{CJK} $\xi \in(a, b)$ \begin{CJK}{UTF8}{mj}使得\end{CJK}

\end{enumerate}
$$
\frac{f(a)-f(\xi)}{g(\xi)-g(b)}=\frac{f^{\prime}(\xi)}{g^{\prime}(\xi)}
$$

\begin{enumerate}
  \setcounter{enumi}{8}
  \item ( 15 \begin{CJK}{UTF8}{mj}分\end{CJK}) \begin{CJK}{UTF8}{mj}设函数列\end{CJK} $\left\{u_{n}(x)\right\}$ \begin{CJK}{UTF8}{mj}在\end{CJK} $[a, b]$ \begin{CJK}{UTF8}{mj}上一致收敛于函数\end{CJK} $u(x)$. \begin{CJK}{UTF8}{mj}并且对\end{CJK} $n=1,2, \cdots, u_{n}(x)$ \begin{CJK}{UTF8}{mj}在区间\end{CJK} $[a, b]$ \begin{CJK}{UTF8}{mj}上至少存在一个零点\end{CJK}, \begin{CJK}{UTF8}{mj}证明\end{CJK} $u(x)$ \begin{CJK}{UTF8}{mj}在区间\end{CJK} $[a, b]$ \begin{CJK}{UTF8}{mj}也至少存在一个零点\end{CJK}.

  \item ( 15 \begin{CJK}{UTF8}{mj}分\end{CJK}) \begin{CJK}{UTF8}{mj}设函数\end{CJK} $f(x)$ \begin{CJK}{UTF8}{mj}在区间\end{CJK} $[0,1]$ \begin{CJK}{UTF8}{mj}上二阶连续可微\end{CJK}, \begin{CJK}{UTF8}{mj}且在区间\end{CJK} $(0,1)$ \begin{CJK}{UTF8}{mj}内存在极值\end{CJK}, \begin{CJK}{UTF8}{mj}证明\end{CJK}:

\end{enumerate}
$$
\max _{x \in[0,1]}\left|f^{\prime}(x)\right| \leq \int_{0}^{1}\left|f^{\prime \prime}(x)\right| \mathrm{dx}
$$

\begin{enumerate}
  \setcounter{enumi}{10}
  \item (15\begin{CJK}{UTF8}{mj}分\end{CJK}) \begin{CJK}{UTF8}{mj}证明\end{CJK}: \begin{CJK}{UTF8}{mj}对每个自然数\end{CJK} $n=1,2, \cdots$, \begin{CJK}{UTF8}{mj}方程\end{CJK} $x^{n+1}+x^{n}+\cdots x=1$ \begin{CJK}{UTF8}{mj}在开区间\end{CJK} $(0,1)$ \begin{CJK}{UTF8}{mj}内存在唯一实根\end{CJK} $x_{n}$, \begin{CJK}{UTF8}{mj}并证明\end{CJK} $\left\{x_{n}\right\}$ \begin{CJK}{UTF8}{mj}收敛\end{CJK}.
\end{enumerate}
\section{4. 华中科技大学 2015 年研究生入学考试试题数学分析 
 李扬 
 微信公众号: sxkyliyang}
\begin{enumerate}
  \item (15 \begin{CJK}{UTF8}{mj}分\end{CJK}) \begin{CJK}{UTF8}{mj}设\end{CJK}
\end{enumerate}
$$
x_{1}>0, x_{n+1}=\ln \left(1+x_{n}\right),
$$
\begin{CJK}{UTF8}{mj}证明\end{CJK} $\left\{x_{n}\right\}$ \begin{CJK}{UTF8}{mj}收玫\end{CJK}, \begin{CJK}{UTF8}{mj}且\end{CJK} $\lim _{n \rightarrow \infty} x_{n}=0$.

\begin{enumerate}
  \setcounter{enumi}{2}
  \item (15 \begin{CJK}{UTF8}{mj}分\end{CJK}) \begin{CJK}{UTF8}{mj}求\end{CJK}
\end{enumerate}
$$
I=\int_{L} \frac{x \mathrm{~d} y-y \mathrm{~d} x}{x^{2}+x y+y^{2}},
$$
$L$ \begin{CJK}{UTF8}{mj}是取逆时针方向的圆周\end{CJK} $x^{2}+y^{2}=1$.

\begin{enumerate}
  \setcounter{enumi}{3}
  \item ( 15 \begin{CJK}{UTF8}{mj}分\end{CJK}) \begin{CJK}{UTF8}{mj}计算闭曲面\end{CJK}
\end{enumerate}
$$
\left(\frac{x^{2}}{a^{2}}+\frac{y^{2}}{b^{2}}+\frac{z^{2}}{c^{2}}\right)^{2}=c x
$$
\begin{CJK}{UTF8}{mj}所围之体积\end{CJK}.

\begin{enumerate}
  \setcounter{enumi}{4}
  \item ( 15 \begin{CJK}{UTF8}{mj}分\end{CJK} $)$
\end{enumerate}
(1) \begin{CJK}{UTF8}{mj}将\end{CJK} $f(x)=x, x \in[0, \pi]$ \begin{CJK}{UTF8}{mj}展开为余弦级数与正弦级数\end{CJK}.

(2) \begin{CJK}{UTF8}{mj}证明\end{CJK}: $\sum_{n=1}^{\infty} \frac{1}{(2 n-1)^{2}}=\frac{\pi^{2}}{8}$

\begin{enumerate}
  \setcounter{enumi}{5}
  \item ( 15 \begin{CJK}{UTF8}{mj}分\end{CJK})
\end{enumerate}
(1)\begin{CJK}{UTF8}{mj}求\end{CJK}
$$
\sum_{n=1}^{\infty} n^{2} x^{n}
$$
\begin{CJK}{UTF8}{mj}的收敛域\end{CJK}.

(2) \begin{CJK}{UTF8}{mj}求\end{CJK}
$$
\sum_{n=1}^{\infty} \frac{n^{2}+2 n}{2^{n}}
$$
\begin{CJK}{UTF8}{mj}的和\end{CJK}.

\begin{enumerate}
  \setcounter{enumi}{6}
  \item ( 15 \begin{CJK}{UTF8}{mj}分\end{CJK}) $f(x)$ \begin{CJK}{UTF8}{mj}在\end{CJK} $(0,+\infty)$ \begin{CJK}{UTF8}{mj}连续可微\end{CJK}, \begin{CJK}{UTF8}{mj}且\end{CJK} $f(0)=0, \sum_{n=1}^{\infty} a_{n}$ \begin{CJK}{UTF8}{mj}收敛\end{CJK}, \begin{CJK}{UTF8}{mj}则\end{CJK}
\end{enumerate}
$$
\sum_{n=1}^{\infty} f\left(a_{n}\right)
$$
\begin{CJK}{UTF8}{mj}收敛\end{CJK}.

\begin{enumerate}
  \setcounter{enumi}{7}
  \item ( 15 \begin{CJK}{UTF8}{mj}分\end{CJK}) $f(x)$ \begin{CJK}{UTF8}{mj}在\end{CJK} $(0,+\infty)$ \begin{CJK}{UTF8}{mj}上\end{CJK} 2 \begin{CJK}{UTF8}{mj}阶可微\end{CJK}, $f(0)>0, f^{\prime \prime}(x)<0, f^{\prime}(0)=0$, \begin{CJK}{UTF8}{mj}则\end{CJK} $f(x)$ \begin{CJK}{UTF8}{mj}在\end{CJK} $(0,+\infty)$ \begin{CJK}{UTF8}{mj}至少有一根\end{CJK}.

  \item ( 15 \begin{CJK}{UTF8}{mj}分\end{CJK}) $f(x)=\int_{0}^{x} t^{n} \mathrm{e}^{-t} \mathrm{dt}, \sum_{n=1}^{\infty} a_{n}$ \begin{CJK}{UTF8}{mj}收敛\end{CJK}, \begin{CJK}{UTF8}{mj}证明\end{CJK}:

\end{enumerate}
$$
\sum_{n=1}^{\infty} \frac{a_{n}}{n !} f(x)
$$
\begin{CJK}{UTF8}{mj}在\end{CJK} $(0,+\infty)$ \begin{CJK}{UTF8}{mj}一致收敛\end{CJK}. 9. ( 15 \begin{CJK}{UTF8}{mj}分\end{CJK}) \begin{CJK}{UTF8}{mj}证明\end{CJK}: \begin{CJK}{UTF8}{mj}若\end{CJK} $f(x)$ \begin{CJK}{UTF8}{mj}在\end{CJK} $(a,+\infty)$ \begin{CJK}{UTF8}{mj}连续\end{CJK}, \begin{CJK}{UTF8}{mj}且\end{CJK} $\int_{a}^{+\infty} f(x) \mathrm{d} x$ \begin{CJK}{UTF8}{mj}与\end{CJK} $\int_{a}^{+\infty} f^{\prime}(x) \mathrm{d} x$ \begin{CJK}{UTF8}{mj}均收敛\end{CJK}, \begin{CJK}{UTF8}{mj}则\end{CJK}
$$
\lim _{x \rightarrow+\infty} f(x)=0 .
$$

\begin{enumerate}
  \setcounter{enumi}{10}
  \item ( 15 \begin{CJK}{UTF8}{mj}分\end{CJK}) \begin{CJK}{UTF8}{mj}设\end{CJK} $f(x)$ \begin{CJK}{UTF8}{mj}在\end{CJK} $[a, b]$ \begin{CJK}{UTF8}{mj}连续\end{CJK}, \begin{CJK}{UTF8}{mj}在\end{CJK} $(a, b)$ \begin{CJK}{UTF8}{mj}可导\end{CJK}, \begin{CJK}{UTF8}{mj}且存在\end{CJK} $c \in(a, b)$, \begin{CJK}{UTF8}{mj}使得\end{CJK} $f(a) f(c)<0, f(b) f(c)<0$, \begin{CJK}{UTF8}{mj}证明\end{CJK}: $\exists \xi \in(a, b)$, \begin{CJK}{UTF8}{mj}使得\end{CJK}
\end{enumerate}
$$
f(\xi)+f^{\prime}(\xi)=0
$$

\section{5. 华中科技大学 2016 年研究生入学考试试题数学分析 
 李扬 
 微信公众号: sxkyliyang}
\begin{enumerate}
  \item \begin{CJK}{UTF8}{mj}二重积分\end{CJK}.

  \item \begin{CJK}{UTF8}{mj}三重积分\end{CJK}, Gauss \begin{CJK}{UTF8}{mj}公式\end{CJK}.

  \item \begin{CJK}{UTF8}{mj}多元函数证明等式\end{CJK}.

  \item \begin{CJK}{UTF8}{mj}用区间套法证明零点存在定理\end{CJK}.

  \item \begin{CJK}{UTF8}{mj}证明\end{CJK}

\end{enumerate}
$$
\sum n f\left(a_{n}\right)
$$
\begin{CJK}{UTF8}{mj}收玫\end{CJK}, \begin{CJK}{UTF8}{mj}其中\end{CJK} $f(x)$ \begin{CJK}{UTF8}{mj}在\end{CJK} 0 \begin{CJK}{UTF8}{mj}点连续可微\end{CJK}.

\begin{enumerate}
  \setcounter{enumi}{6}
  \item \begin{CJK}{UTF8}{mj}求\end{CJK}
\end{enumerate}
$$
\sum\left(\frac{n}{\mathrm{e}}\right)^{n} \frac{(-1)^{n}}{n !} x^{n}
$$
\begin{CJK}{UTF8}{mj}的收敛半径\end{CJK}, \begin{CJK}{UTF8}{mj}收敛区间\end{CJK}, \begin{CJK}{UTF8}{mj}说明在端点处的收敛情况\end{CJK}.

\begin{enumerate}
  \setcounter{enumi}{7}
  \item \begin{CJK}{UTF8}{mj}已知\end{CJK} $f\left(\frac{a+b}{2}\right)=0$, \begin{CJK}{UTF8}{mj}证明\end{CJK}:
\end{enumerate}
$$
\int_{a}^{b}|f(x)| \mathrm{d} x \leq \frac{b-a}{2} \int_{a}^{b}\left|f^{\prime}(x)\right| \mathrm{d} x .
$$

\begin{enumerate}
  \setcounter{enumi}{8}
  \item $f(x)$ \begin{CJK}{UTF8}{mj}在\end{CJK} $(a, b)$ \begin{CJK}{UTF8}{mj}有最大值与最小值\end{CJK}, \begin{CJK}{UTF8}{mj}则存在\end{CJK} $\xi \in(a, b)$, \begin{CJK}{UTF8}{mj}使得\end{CJK}
\end{enumerate}
$$
f^{\prime}(\xi)-\frac{\xi f^{\prime \prime}(\xi)}{2}=0 .
$$

\begin{enumerate}
  \setcounter{enumi}{9}
  \item \begin{CJK}{UTF8}{mj}傅里叶展开\end{CJK}
\end{enumerate}
$$
f(x)=\operatorname{sgn} x, x \in[-\pi, \pi] .
$$

\begin{enumerate}
  \setcounter{enumi}{10}
  \item \begin{CJK}{UTF8}{mj}证明\end{CJK}:
\end{enumerate}
$$
\int_{0}^{1} \sin \left(x^{3}\right) f(x) \mathrm{d} x
$$
\begin{CJK}{UTF8}{mj}一致收敛\end{CJK}, \begin{CJK}{UTF8}{mj}已知\end{CJK} $\int_{0}^{1} f(x) \mathrm{d} x$ \begin{CJK}{UTF8}{mj}收敛\end{CJK}.

\section{6. 华中科技大学 2017 年研究生入学考试试题数学分析 
 李扬 
 微信公众号: sxkyliyang}
\begin{enumerate}
  \item \begin{CJK}{UTF8}{mj}计算\end{CJK}
\end{enumerate}
$$
\lim _{x \rightarrow \infty}\left(x-x^{2}\left[\ln \left(1+\frac{1}{x}\right)\right]\right) .
$$

\begin{enumerate}
  \setcounter{enumi}{2}
  \item \begin{CJK}{UTF8}{mj}计算\end{CJK}
\end{enumerate}
$$
\iint_{S} x \mathrm{~d} y \mathrm{~d} z+y \mathrm{~d} x \mathrm{~d} z+z \mathrm{~d} x \mathrm{~d} y,
$$
\begin{CJK}{UTF8}{mj}其中\end{CJK} $S$ \begin{CJK}{UTF8}{mj}表示\end{CJK} $z=\sqrt{1-c^{2}-y^{2}}$.

\begin{enumerate}
  \setcounter{enumi}{3}
  \item \begin{CJK}{UTF8}{mj}计算\end{CJK}
\end{enumerate}
$$
I=\int_{0}^{\infty} \frac{\mathrm{e}^{-a x}-\mathrm{e}^{-b x}}{x} \mathrm{~d} x .
$$

\begin{enumerate}
  \setcounter{enumi}{4}
  \item \begin{CJK}{UTF8}{mj}求\end{CJK}
\end{enumerate}
$$
\sum_{n=0}^{\infty} \frac{x^{2 n}}{2 n+1}
$$
\begin{CJK}{UTF8}{mj}的收敛域\end{CJK}, \begin{CJK}{UTF8}{mj}并求其和\end{CJK}.

\begin{enumerate}
  \setcounter{enumi}{5}
  \item \begin{CJK}{UTF8}{mj}在\end{CJK} $[-\pi, \pi]$ \begin{CJK}{UTF8}{mj}上\end{CJK}, \begin{CJK}{UTF8}{mj}将\end{CJK} $\arcsin (\cos x)$ \begin{CJK}{UTF8}{mj}用傅里叶级数展开\end{CJK}, \begin{CJK}{UTF8}{mj}并计算\end{CJK}
\end{enumerate}
$$
\sum_{n=0}^{\infty} \frac{1}{(2 n+1)^{2}}
$$
\begin{CJK}{UTF8}{mj}的值\end{CJK}.

\begin{enumerate}
  \setcounter{enumi}{6}
  \item \begin{CJK}{UTF8}{mj}用确界定理证明单调有界数列必有极限\end{CJK}.

  \item \begin{CJK}{UTF8}{mj}已知\end{CJK} $\lim _{x \rightarrow a^{+}} f^{\prime}$ \begin{CJK}{UTF8}{mj}存在\end{CJK}, $f(x)$ \begin{CJK}{UTF8}{mj}在\end{CJK} $(a, b)$ \begin{CJK}{UTF8}{mj}内连续可微\end{CJK}, \begin{CJK}{UTF8}{mj}求证\end{CJK} $f(x)$ \begin{CJK}{UTF8}{mj}在\end{CJK} $x=a$ \begin{CJK}{UTF8}{mj}处右可导\end{CJK}.

  \item \begin{CJK}{UTF8}{mj}已知\end{CJK} $\sum_{n=0}^{\infty} a_{n}$ \begin{CJK}{UTF8}{mj}条件收敛\end{CJK}, \begin{CJK}{UTF8}{mj}设\end{CJK} $a_{n}^{+}=\max \left\{a_{n}, 0\right\}, a_{n}^{-}=\max \left\{-a_{n}, 0\right\}$, \begin{CJK}{UTF8}{mj}证明\end{CJK}

\end{enumerate}
$$
\lim _{n \rightarrow \infty} \frac{\sum_{k=0}^{n} a_{k}^{+}}{\sum_{k=0}^{n} a_{k}^{-}}=1 .
$$

\begin{enumerate}
  \setcounter{enumi}{9}
  \item \begin{CJK}{UTF8}{mj}已知\end{CJK} $f(x, t)$ \begin{CJK}{UTF8}{mj}在\end{CJK} $\mathbb{R}^{2 \times 2}$ \begin{CJK}{UTF8}{mj}上连续且二阶可微\end{CJK}, \begin{CJK}{UTF8}{mj}且满足\end{CJK} $f_{x x}(x, t)=f_{t t}(x, t)$, \begin{CJK}{UTF8}{mj}设\end{CJK}
\end{enumerate}
$$
E(t)=\frac{1}{2} \int_{t-1}^{1-t}\left[f_{x}(x, t)\right]^{2}+\left[f_{t}(x, t)\right]^{2} \mathrm{~d} t
$$
\begin{CJK}{UTF8}{mj}证明\end{CJK} $E(t)$ \begin{CJK}{UTF8}{mj}单调递减\end{CJK}.

\begin{enumerate}
  \setcounter{enumi}{10}
  \item \begin{CJK}{UTF8}{mj}证明\end{CJK}
\end{enumerate}
$$
\iint_{D}|[f(x, y)-f(0,0)]| \mathrm{d} x \mathrm{~d} y \leq \iint_{D} \frac{\sqrt{f_{x}^{2}(x, y)+f_{y}^{2}(x, y)}}{2 \sqrt{x^{2}+y^{2}}} \mathrm{~d} x \mathrm{~d} y,
$$
\begin{CJK}{UTF8}{mj}其中\end{CJK} $D$ \begin{CJK}{UTF8}{mj}表示\end{CJK} $\left\{(x, y) \mid x^{2}+y^{2} \leq 1\right\}$ \begin{CJK}{UTF8}{mj}平面\end{CJK}.

\section{1. 华中师范大学 2009 年研究生入学考试试题高等代数 
 李扬 
 微信公众号: sxkyliyang}
\begin{enumerate}
  \item (20 \begin{CJK}{UTF8}{mj}分\end{CJK}) \begin{CJK}{UTF8}{mj}设\end{CJK} $a_{1}, \cdots, a_{n}$ \begin{CJK}{UTF8}{mj}是\end{CJK} $n$ \begin{CJK}{UTF8}{mj}个复数\end{CJK}, $x$ \begin{CJK}{UTF8}{mj}是复变元\end{CJK}. \begin{CJK}{UTF8}{mj}求解\end{CJK}: $x$ \begin{CJK}{UTF8}{mj}取哪些复数值时下述等式\end{CJK} (\begin{CJK}{UTF8}{mj}等式左边是\end{CJK} $n+1$ \begin{CJK}{UTF8}{mj}阶行列式\end{CJK}) \begin{CJK}{UTF8}{mj}成立\end{CJK}:
\end{enumerate}
$$
\left|\begin{array}{ccccc}
1 & 1 & 1 & \cdots & 1 \\
x & a_{1} & a_{2} & \cdots & a_{n} \\
x^{2} & a_{1}^{2} & a_{2}^{2} & \cdots & a_{n}^{2} \\
\vdots & \vdots & \vdots & & \vdots \\
x^{n} & a_{1}^{n} & a_{2}^{n} & \cdots & a_{n}^{n}
\end{array}\right|=0 .
$$

\begin{enumerate}
  \setcounter{enumi}{2}
  \item ( 20 \begin{CJK}{UTF8}{mj}分\end{CJK}) \begin{CJK}{UTF8}{mj}设\end{CJK} $f(x)$ \begin{CJK}{UTF8}{mj}是\end{CJK} $n$ \begin{CJK}{UTF8}{mj}次实系数多项式\end{CJK}, $n>1$. \begin{CJK}{UTF8}{mj}设\end{CJK} $f^{\prime}(x)$ \begin{CJK}{UTF8}{mj}是\end{CJK} $f(x)$ \begin{CJK}{UTF8}{mj}的导数多项式\end{CJK}. \begin{CJK}{UTF8}{mj}证明\end{CJK}:
\end{enumerate}
(1) \begin{CJK}{UTF8}{mj}如果\end{CJK} $r$ \begin{CJK}{UTF8}{mj}是\end{CJK} $f(x)$ \begin{CJK}{UTF8}{mj}的\end{CJK} $m$ \begin{CJK}{UTF8}{mj}重根\end{CJK}, $m>0$, \begin{CJK}{UTF8}{mj}则\end{CJK} $r$ \begin{CJK}{UTF8}{mj}是\end{CJK} $f^{\prime}(x)$ \begin{CJK}{UTF8}{mj}的\end{CJK} $m-1$ \begin{CJK}{UTF8}{mj}重根\end{CJK} (\begin{CJK}{UTF8}{mj}若\end{CJK} $r$ \begin{CJK}{UTF8}{mj}是\end{CJK} $f(x)$ \begin{CJK}{UTF8}{mj}的零重根则表示\end{CJK} $r$ \begin{CJK}{UTF8}{mj}不\end{CJK} \begin{CJK}{UTF8}{mj}是\end{CJK} $f^{\prime}(x)$ \begin{CJK}{UTF8}{mj}的根\end{CJK}).

(2) \begin{CJK}{UTF8}{mj}如果\end{CJK} $f(x)$ \begin{CJK}{UTF8}{mj}的根都是实数\end{CJK}, \begin{CJK}{UTF8}{mj}则\end{CJK} $f^{\prime}(x)$ \begin{CJK}{UTF8}{mj}的根也都是实数\end{CJK}.

\begin{enumerate}
  \setcounter{enumi}{3}
  \item ( 20 \begin{CJK}{UTF8}{mj}分\end{CJK}) \begin{CJK}{UTF8}{mj}设\end{CJK} $A$ \begin{CJK}{UTF8}{mj}是秩为\end{CJK} $r$ \begin{CJK}{UTF8}{mj}的\end{CJK} $m \times n$ \begin{CJK}{UTF8}{mj}阶矩阵\end{CJK}, $B$ \begin{CJK}{UTF8}{mj}是非零的\end{CJK} $m \times 1$ \begin{CJK}{UTF8}{mj}阶矩阵\end{CJK}. \begin{CJK}{UTF8}{mj}考虑线性方程组\end{CJK} $A X=B$, \begin{CJK}{UTF8}{mj}其中\end{CJK} $X$ \begin{CJK}{UTF8}{mj}是变元\end{CJK} $x_{1}, \cdots, x_{n}$ \begin{CJK}{UTF8}{mj}的列向量\end{CJK}. \begin{CJK}{UTF8}{mj}证明\end{CJK}:
\end{enumerate}
(1) \begin{CJK}{UTF8}{mj}线性方程组\end{CJK} $A X=B$ \begin{CJK}{UTF8}{mj}的任意有限个解向量\end{CJK} $X_{1}, \cdots, X_{k}$ \begin{CJK}{UTF8}{mj}的向量组的秩\end{CJK} $\leqslant n-r+1$.

(2) \begin{CJK}{UTF8}{mj}若线性方程组\end{CJK} $A X=B$ \begin{CJK}{UTF8}{mj}有解\end{CJK}, \begin{CJK}{UTF8}{mj}则它有\end{CJK} $n-r+1$ \begin{CJK}{UTF8}{mj}个解向量是线性无关的\end{CJK}.

\begin{enumerate}
  \setcounter{enumi}{4}
  \item (30 \begin{CJK}{UTF8}{mj}分\end{CJK}) \begin{CJK}{UTF8}{mj}设\end{CJK} $A, B, C$ \begin{CJK}{UTF8}{mj}都是\end{CJK} $n$ \begin{CJK}{UTF8}{mj}阶方阵\end{CJK}, \begin{CJK}{UTF8}{mj}令\end{CJK} $\left(\begin{array}{cc}A & B \\ C & 0\end{array}\right)$ \begin{CJK}{UTF8}{mj}是分块构成的\end{CJK} $2 n$ \begin{CJK}{UTF8}{mj}阶方阵\end{CJK}, \begin{CJK}{UTF8}{mj}其中右下块\end{CJK} 0 \begin{CJK}{UTF8}{mj}表示\end{CJK} $n$ \begin{CJK}{UTF8}{mj}阶\end{CJK} \begin{CJK}{UTF8}{mj}零方阵\end{CJK}.
\end{enumerate}
(1) \begin{CJK}{UTF8}{mj}证明\end{CJK}:
$$
\operatorname{rank}\left(\begin{array}{cc}
A & B \\
C & 0
\end{array}\right) \geqslant \operatorname{rank}(B)+\operatorname{rank}(C)
$$
\begin{CJK}{UTF8}{mj}这里\end{CJK} $\operatorname{rank}(B)$ \begin{CJK}{UTF8}{mj}表示矩阵\end{CJK} $B$ \begin{CJK}{UTF8}{mj}的秩\end{CJK}.

(2) \begin{CJK}{UTF8}{mj}举例说明\end{CJK}: (1) \begin{CJK}{UTF8}{mj}中的等号和不等号都可能成立\end{CJK}.

\begin{enumerate}
  \setcounter{enumi}{5}
  \item (30 \begin{CJK}{UTF8}{mj}分\end{CJK}) \begin{CJK}{UTF8}{mj}设\end{CJK} $V$ \begin{CJK}{UTF8}{mj}是有限维向量空间\end{CJK}, \begin{CJK}{UTF8}{mj}设\end{CJK} $U, W$ \begin{CJK}{UTF8}{mj}是\end{CJK} $V$ \begin{CJK}{UTF8}{mj}的两个子空间\end{CJK}.
\end{enumerate}
(1)\begin{CJK}{UTF8}{mj}什么是\end{CJK} $U$ \begin{CJK}{UTF8}{mj}与\end{CJK} $W$ \begin{CJK}{UTF8}{mj}的和子空间\end{CJK} $U+W ?$ \begin{CJK}{UTF8}{mj}请叙述关于\end{CJK} $U+W$ \begin{CJK}{UTF8}{mj}的维数公式\end{CJK}.

(2) \begin{CJK}{UTF8}{mj}证明关于和子空间的维数公式\end{CJK}.

\begin{enumerate}
  \setcounter{enumi}{6}
  \item ( 30 \begin{CJK}{UTF8}{mj}分\end{CJK}) \begin{CJK}{UTF8}{mj}设\end{CJK} $A$ \begin{CJK}{UTF8}{mj}为\end{CJK} $n$ \begin{CJK}{UTF8}{mj}阶实矩阵\end{CJK}, $\lambda_{i}=r+s i$ \begin{CJK}{UTF8}{mj}是\end{CJK} $A$ \begin{CJK}{UTF8}{mj}的特征根\end{CJK}, \begin{CJK}{UTF8}{mj}其中\end{CJK} $r, s$ \begin{CJK}{UTF8}{mj}是实数\end{CJK}, $i$ \begin{CJK}{UTF8}{mj}是虚数单位\end{CJK}.
\end{enumerate}
(1) \begin{CJK}{UTF8}{mj}证明\end{CJK}: $\frac{1}{2}\left(A+A^{\prime}\right)$ \begin{CJK}{UTF8}{mj}的特征根都是实数\end{CJK}, \begin{CJK}{UTF8}{mj}令\end{CJK} $\mu_{1} \leqslant \cdots \leqslant \mu_{n}$ \begin{CJK}{UTF8}{mj}是\end{CJK} $\frac{1}{2}\left(A+A^{\prime}\right)$ \begin{CJK}{UTF8}{mj}的全部特征根\end{CJK}.

(2) \begin{CJK}{UTF8}{mj}证明\end{CJK}: $\mu_{1} \leqslant r \leqslant \mu_{n}$.

(3) \begin{CJK}{UTF8}{mj}你有类似的估计\end{CJK} $s$ \begin{CJK}{UTF8}{mj}的办法吗\end{CJK}?

\section{2. 华中师范大学 2010 年研究生入学考试试题高等代数 
 李扬 
 微信公众号: sxkyliyang}
\begin{enumerate}
  \item ( 20 \begin{CJK}{UTF8}{mj}分\end{CJK}) \begin{CJK}{UTF8}{mj}设\end{CJK} $\mathbb{F}$ \begin{CJK}{UTF8}{mj}是任意数域\end{CJK}, $p(x) \in \mathbb{F}[x]$. \begin{CJK}{UTF8}{mj}证明\end{CJK}: $p(x)$ \begin{CJK}{UTF8}{mj}是不可约多项式当且仅当\end{CJK} $p(x)$ \begin{CJK}{UTF8}{mj}是素多项式\end{CJK}.

  \item (20 \begin{CJK}{UTF8}{mj}分\end{CJK})

\end{enumerate}
(1) \begin{CJK}{UTF8}{mj}设\end{CJK} $A$ \begin{CJK}{UTF8}{mj}是\end{CJK} $n$ \begin{CJK}{UTF8}{mj}阶方阵\end{CJK}, $E$ \begin{CJK}{UTF8}{mj}是单位矩阵\end{CJK}, $k \neq 0$. \begin{CJK}{UTF8}{mj}证明\end{CJK}: $A^{2}=k A$ \begin{CJK}{UTF8}{mj}当且仅当\end{CJK}
$$
\operatorname{rank}(A)+\operatorname{rank}(A-k E)=n .
$$
(2) \begin{CJK}{UTF8}{mj}证明\end{CJK}: \begin{CJK}{UTF8}{mj}任意方阵可以表示为满秩矩阵和幂等矩阵的乘积\end{CJK}.

\begin{enumerate}
  \setcounter{enumi}{3}
  \item ( 20 \begin{CJK}{UTF8}{mj}分\end{CJK}) \begin{CJK}{UTF8}{mj}设\end{CJK} $\mathbb{R}$ \begin{CJK}{UTF8}{mj}表示实数域\end{CJK}, $V=M_{3}(\mathbb{R})$ \begin{CJK}{UTF8}{mj}表示所有\end{CJK} $3 \times 3$ \begin{CJK}{UTF8}{mj}实矩阵构成的向量空间\end{CJK}. \begin{CJK}{UTF8}{mj}对给定的\end{CJK} $A \in M_{3}(\mathbb{R})$, \begin{CJK}{UTF8}{mj}定义\end{CJK} $V$ \begin{CJK}{UTF8}{mj}上的线性变换\end{CJK} $\mathscr{A}: V \rightarrow V$ \begin{CJK}{UTF8}{mj}为\end{CJK}
\end{enumerate}
$$
\mathscr{A}(B)=A B-B A \text {, 对任意的 } B \in M_{3}(\mathbb{R}) \text {. }
$$
$$
A=\left(\begin{array}{lll}
0 & 0 & 0 \\
0 & 1 & 0 \\
0 & 0 & 2
\end{array}\right) .
$$
\begin{CJK}{UTF8}{mj}求\end{CJK} $\mathscr{A}$ \begin{CJK}{UTF8}{mj}的特征值和相应的特征子空间\end{CJK}; \begin{CJK}{UTF8}{mj}并求此时\end{CJK} $\mathscr{A}$ \begin{CJK}{UTF8}{mj}的极小多项式\end{CJK}.

\begin{enumerate}
  \setcounter{enumi}{4}
  \item (30 \begin{CJK}{UTF8}{mj}分\end{CJK}) \begin{CJK}{UTF8}{mj}设有三元实二次型\end{CJK}
\end{enumerate}
$$
f(x, y, z)=x^{2}+3 y^{2}+z^{2}+4 x z .
$$
\begin{CJK}{UTF8}{mj}并设\end{CJK} $x, y, z$ \begin{CJK}{UTF8}{mj}满足\end{CJK} $x^{2}+y^{2}+z^{2}=1$. \begin{CJK}{UTF8}{mj}试求\end{CJK} $f$ \begin{CJK}{UTF8}{mj}的最大值和最小值\end{CJK}, \begin{CJK}{UTF8}{mj}并求当\end{CJK} $x, y, z$ \begin{CJK}{UTF8}{mj}取什么值时\end{CJK}, $f$ \begin{CJK}{UTF8}{mj}分别达到\end{CJK}

\begin{enumerate}
  \setcounter{enumi}{5}
  \item (30 \begin{CJK}{UTF8}{mj}分\end{CJK}) \begin{CJK}{UTF8}{mj}设\end{CJK} $\mathbb{R}$ \begin{CJK}{UTF8}{mj}是实数域\end{CJK}, $V=C^{1}[0,1]$ \begin{CJK}{UTF8}{mj}是闭区间\end{CJK} $[0,1]$ \begin{CJK}{UTF8}{mj}上的实连续可微函数的集合\end{CJK}. $V$ \begin{CJK}{UTF8}{mj}在函数的加法和数\end{CJK}
\end{enumerate}
(1) \begin{CJK}{UTF8}{mj}证明函数\end{CJK} $f(x)=\cos x, g(x)=2 x, h(x)=\mathrm{e}^{x}$ \begin{CJK}{UTF8}{mj}在\end{CJK} $V$ \begin{CJK}{UTF8}{mj}中线性无关\end{CJK}.

\begin{enumerate}
  \setcounter{enumi}{6}
  \item (30 \begin{CJK}{UTF8}{mj}分\end{CJK})
\end{enumerate}
(1) \begin{CJK}{UTF8}{mj}设\end{CJK} $A$ \begin{CJK}{UTF8}{mj}和\end{CJK} $B$ \begin{CJK}{UTF8}{mj}均为\end{CJK} $n$ \begin{CJK}{UTF8}{mj}阶复方阵\end{CJK}, \begin{CJK}{UTF8}{mj}证明\end{CJK}: $A$ \begin{CJK}{UTF8}{mj}与\end{CJK} $B$ \begin{CJK}{UTF8}{mj}相似当且仅当作为\end{CJK} $\lambda$ \begin{CJK}{UTF8}{mj}一矩阵\end{CJK}, \begin{CJK}{UTF8}{mj}有\end{CJK} $\lambda E-A$ \begin{CJK}{UTF8}{mj}等价于\end{CJK} $\lambda E-B$.

(3) \begin{CJK}{UTF8}{mj}试说明上述结论\end{CJK} (2) \begin{CJK}{UTF8}{mj}对\end{CJK} 4 \begin{CJK}{UTF8}{mj}阶幂零矩阵是否成立\end{CJK}, \begin{CJK}{UTF8}{mj}为什么\end{CJK}?

\section{3. 华中师范大学 2011 年研究生入学考试试题高等代数 
 李扬 
 微信公众号: sxkyliyang}
\begin{enumerate}
  \item (20 \begin{CJK}{UTF8}{mj}分\end{CJK}) \begin{CJK}{UTF8}{mj}计算下列行列式\end{CJK}
\end{enumerate}
$$
D=\left|\begin{array}{cccc}
2+x_{1} & 2+x_{1}^{2} & \cdots & 2+x_{1}^{n} \\
2+x_{2} & 2+x_{2}^{2} & \cdots & 2+x_{2}^{n} \\
\vdots & \vdots & & \vdots \\
2+x_{n} & 2+x_{n}^{2} & \cdots & 2+x_{n}^{n}
\end{array}\right| .
$$

\begin{enumerate}
  \setcounter{enumi}{2}
  \item ( 20 \begin{CJK}{UTF8}{mj}分\end{CJK}) \begin{CJK}{UTF8}{mj}设多项式\end{CJK} $f(x)$ \begin{CJK}{UTF8}{mj}与\end{CJK} $g(x)$ \begin{CJK}{UTF8}{mj}互素\end{CJK}, \begin{CJK}{UTF8}{mj}并设\end{CJK} $f^{2}(x)+g^{2}(x)$ \begin{CJK}{UTF8}{mj}有重根\end{CJK}, \begin{CJK}{UTF8}{mj}令\end{CJK} $f^{\prime}(x), g^{\prime}(x)$ \begin{CJK}{UTF8}{mj}分别表示\end{CJK} $f(x), g(x)$ \begin{CJK}{UTF8}{mj}的导数多项式\end{CJK}. \begin{CJK}{UTF8}{mj}证明\end{CJK}: $f^{2}(x)+g^{2}(x)$ \begin{CJK}{UTF8}{mj}的重根是\end{CJK} $f^{\prime 2}(x)+g^{\prime 2}(x)$ \begin{CJK}{UTF8}{mj}的根\end{CJK}.

  \item (30 \begin{CJK}{UTF8}{mj}分\end{CJK}) \begin{CJK}{UTF8}{mj}设\end{CJK} $\mathbb{F}$ \begin{CJK}{UTF8}{mj}是数域\end{CJK}, $V$ \begin{CJK}{UTF8}{mj}表示数域\end{CJK} $\mathbb{F}$ \begin{CJK}{UTF8}{mj}上所有次数小于\end{CJK} $n$ \begin{CJK}{UTF8}{mj}的多项式加上零多项式构成的线性空间\end{CJK}. \begin{CJK}{UTF8}{mj}令\end{CJK}

\end{enumerate}
$$
\mathscr{A}: V \rightarrow V, f(x) \mapsto f(x+1)-f(x)
$$
(1) \begin{CJK}{UTF8}{mj}证明\end{CJK}: $\mathscr{A}$ \begin{CJK}{UTF8}{mj}是\end{CJK} $V$ \begin{CJK}{UTF8}{mj}上的线性变换\end{CJK}.

(2) \begin{CJK}{UTF8}{mj}证明\end{CJK}:
$$
\gamma_{0}=1, \gamma_{1}=\frac{x}{1}, \gamma_{2}=\frac{x(x-1)}{2 !}, \cdots, \gamma_{n-1}=\frac{x(x-1) \cdots(x-n+2)}{(n-1) !}
$$
\begin{CJK}{UTF8}{mj}是\end{CJK} $V$ \begin{CJK}{UTF8}{mj}的基底\end{CJK}.

(3) \begin{CJK}{UTF8}{mj}求\end{CJK} $\mathscr{A}$ \begin{CJK}{UTF8}{mj}在上述基底下的矩阵\end{CJK}.

\begin{enumerate}
  \setcounter{enumi}{4}
  \item ( 20 \begin{CJK}{UTF8}{mj}分\end{CJK}) \begin{CJK}{UTF8}{mj}设\end{CJK} $\mathscr{A}$ \begin{CJK}{UTF8}{mj}是实向量空间\end{CJK} $V$ \begin{CJK}{UTF8}{mj}上的线性变换\end{CJK}, \begin{CJK}{UTF8}{mj}且满足\end{CJK} $\mathscr{A}^{2}=\mathrm{idv}$, \begin{CJK}{UTF8}{mj}这里\end{CJK} idv \begin{CJK}{UTF8}{mj}表示\end{CJK} $V$ \begin{CJK}{UTF8}{mj}上的恒等变换\end{CJK}. \begin{CJK}{UTF8}{mj}定义\end{CJK}
\end{enumerate}
$$
\begin{gathered}
V_{1}=\{v \in V \mid \mathscr{A}(v)=v\} \\
V_{2}=\{v \in V \mid \mathscr{A}(v)=-v\}
\end{gathered}
$$

\begin{enumerate}
  \setcounter{enumi}{5}
  \item ( 20 \begin{CJK}{UTF8}{mj}分\end{CJK}) \begin{CJK}{UTF8}{mj}设\end{CJK} $\mathbb{R}$ \begin{CJK}{UTF8}{mj}表示实数域\end{CJK}, $M_{n}(\mathbb{R})$ \begin{CJK}{UTF8}{mj}表示\end{CJK} $\mathbb{R}$ \begin{CJK}{UTF8}{mj}上所有实矩阵的集合\end{CJK}.
\end{enumerate}
(1) \begin{CJK}{UTF8}{mj}设\end{CJK} $A \in M_{n}(\mathbb{R})$ \begin{CJK}{UTF8}{mj}是\end{CJK} $n$ \begin{CJK}{UTF8}{mj}阶实对称矩阵\end{CJK}, $E$ \begin{CJK}{UTF8}{mj}是单位矩阵\end{CJK}, \begin{CJK}{UTF8}{mj}且\end{CJK} $A$ \begin{CJK}{UTF8}{mj}满足\end{CJK} $A^{3}+6 A-7 E=0$. \begin{CJK}{UTF8}{mj}证明\end{CJK}: $A$ \begin{CJK}{UTF8}{mj}是正定矩\end{CJK} \begin{CJK}{UTF8}{mj}阵\end{CJK}.
$$
X^{T} A X=X^{T} B X
$$

\begin{enumerate}
  \setcounter{enumi}{6}
  \item ( 20 \begin{CJK}{UTF8}{mj}分\end{CJK}) \begin{CJK}{UTF8}{mj}设\end{CJK} $W_{1}, W_{2}$ \begin{CJK}{UTF8}{mj}是\end{CJK} $n$ \begin{CJK}{UTF8}{mj}维欧式空间\end{CJK} $V$ \begin{CJK}{UTF8}{mj}的子空间\end{CJK}, \begin{CJK}{UTF8}{mj}且\end{CJK} $W_{1}$ \begin{CJK}{UTF8}{mj}的维数小于\end{CJK} $W_{2}$ \begin{CJK}{UTF8}{mj}的维数\end{CJK}. \begin{CJK}{UTF8}{mj}证明\end{CJK}: \begin{CJK}{UTF8}{mj}在子空间\end{CJK} $W_{2}$ \begin{CJK}{UTF8}{mj}中\end{CJK}
\end{enumerate}
\section{4. 华中师范大学 2012 年研究生入学考试试题高等代数 
 李扬 
 微信公众号: sxkyliyang}
\begin{enumerate}
  \item ( 15 \begin{CJK}{UTF8}{mj}分\end{CJK}) \begin{CJK}{UTF8}{mj}设\end{CJK} $\mathbb{F}$ \begin{CJK}{UTF8}{mj}是任意数域\end{CJK}, $\mathbb{F}[x]$ \begin{CJK}{UTF8}{mj}表示数域\end{CJK} $\mathbb{F}$ \begin{CJK}{UTF8}{mj}上的所有\end{CJK} $\mathbb{F}$ \begin{CJK}{UTF8}{mj}一多项式的集合\end{CJK}, \begin{CJK}{UTF8}{mj}设\end{CJK} $f(x) \in \mathbb{F}[x]$. \begin{CJK}{UTF8}{mj}证明\end{CJK}: $f(x)$ \begin{CJK}{UTF8}{mj}是\end{CJK} \begin{CJK}{UTF8}{mj}一个不可约多项式的幂当且仅当对任意互素的多项式\end{CJK} $g(x), h(x) \in \mathbb{F}[x]$, \begin{CJK}{UTF8}{mj}只要\end{CJK} $f(x) \mid g(x) h(x)$, \begin{CJK}{UTF8}{mj}则或者\end{CJK} $f(x) \mid g(x)$ \begin{CJK}{UTF8}{mj}或者\end{CJK} $f(x) \mid h(x) .$

  \item (10 \begin{CJK}{UTF8}{mj}分\end{CJK}) \begin{CJK}{UTF8}{mj}计算行列式\end{CJK}

\end{enumerate}
$$
D_{n+1}=\left|\begin{array}{cccc}
a^{n} & (a-1)^{n} & \cdots & (a-n)^{n} \\
a^{n-1} & (a-1)^{n-1} & \cdots & (a-n)^{n-1} \\
\vdots & \vdots & & \vdots \\
a & a-1 & \cdots & a-n \\
1 & 1 & \cdots & 1
\end{array}\right|
$$

\begin{enumerate}
  \setcounter{enumi}{3}
  \item ( 20 \begin{CJK}{UTF8}{mj}分\end{CJK}) \begin{CJK}{UTF8}{mj}设\end{CJK} $n, k$ \begin{CJK}{UTF8}{mj}是整数\end{CJK}, $n>2,1 \leqslant k \leqslant n$, \begin{CJK}{UTF8}{mj}设复数\end{CJK} $\omega$ \begin{CJK}{UTF8}{mj}满足\end{CJK} $\omega^{n}=1$ \begin{CJK}{UTF8}{mj}但\end{CJK} $\omega^{t} \neq 1$ \begin{CJK}{UTF8}{mj}对任意\end{CJK} $t=1, \cdots, n-1$ (\begin{CJK}{UTF8}{mj}称这样的\end{CJK} $\omega$ \begin{CJK}{UTF8}{mj}为\end{CJK} $n$ \begin{CJK}{UTF8}{mj}次本原单位根\end{CJK}). \begin{CJK}{UTF8}{mj}令\end{CJK} $A=\left(\omega^{i j}\right)_{0 \leqslant i, j \leqslant n-1}$ \begin{CJK}{UTF8}{mj}是一个\end{CJK} $n$ \begin{CJK}{UTF8}{mj}阶方阵\end{CJK}, \begin{CJK}{UTF8}{mj}令\end{CJK}
\end{enumerate}
$$
A\left(\begin{array}{ccc}
i_{1} & \cdots & i_{k} \\
j_{1} & \cdots & j_{k}
\end{array}\right)
$$
\begin{CJK}{UTF8}{mj}是由\end{CJK} $A$ \begin{CJK}{UTF8}{mj}的位于第\end{CJK} $i_{1}, \cdots, i_{k}$ \begin{CJK}{UTF8}{mj}行和第\end{CJK} $j_{1}, \cdots, j_{k}$ \begin{CJK}{UTF8}{mj}列的交叉位置的元素构成的\end{CJK} $k$ \begin{CJK}{UTF8}{mj}阶子矩阵\end{CJK}, \begin{CJK}{UTF8}{mj}这里\end{CJK} $1 \leqslant i_{1}<$

(1) \begin{CJK}{UTF8}{mj}证明\end{CJK}: \begin{CJK}{UTF8}{mj}矩阵\end{CJK} $A$ \begin{CJK}{UTF8}{mj}是行满秩矩阵当且仅当\end{CJK} $m \leqslant n$, \begin{CJK}{UTF8}{mj}并且存在可逆的\end{CJK} $n \times n$ \begin{CJK}{UTF8}{mj}矩阵\end{CJK} $B$ \begin{CJK}{UTF8}{mj}使得\end{CJK} $A B=(E \mid 0)$, \begin{CJK}{UTF8}{mj}这\end{CJK}
$$
X=(A+P)(A-P)^{-1}, Y=(A+P)^{-1}(A-P)
$$
\begin{CJK}{UTF8}{mj}是矩阵方程\end{CJK} $X A Y=A$ \begin{CJK}{UTF8}{mj}的解\end{CJK}.

(2) \begin{CJK}{UTF8}{mj}设\end{CJK} $C \in M_{m \times n}(\mathbb{F})$, \begin{CJK}{UTF8}{mj}设\end{CJK} $V$ \begin{CJK}{UTF8}{mj}为\end{CJK} $\mathbb{F}^{n}$ \begin{CJK}{UTF8}{mj}的子空间\end{CJK}. \begin{CJK}{UTF8}{mj}令\end{CJK} $\operatorname{rank}(C)$ \begin{CJK}{UTF8}{mj}表示矩阵\end{CJK} $C$ \begin{CJK}{UTF8}{mj}的秩\end{CJK}, $\operatorname{dim}(V)$ \begin{CJK}{UTF8}{mj}表示向量空间\end{CJK} $V$ \begin{CJK}{UTF8}{mj}的\end{CJK} \begin{CJK}{UTF8}{mj}维数\end{CJK}. \begin{CJK}{UTF8}{mj}设\end{CJK} $W$ \begin{CJK}{UTF8}{mj}是齐次线性方程组\end{CJK} $C X=0$ \begin{CJK}{UTF8}{mj}的解子空间\end{CJK}. \begin{CJK}{UTF8}{mj}证明\end{CJK}: \begin{CJK}{UTF8}{mj}如果\end{CJK} $W \cap V=0$, \begin{CJK}{UTF8}{mj}则\end{CJK}
$$
\operatorname{rank}(C) \geqslant \operatorname{dim}(V)
$$
(1) \begin{CJK}{UTF8}{mj}证明\end{CJK}: \begin{CJK}{UTF8}{mj}当\end{CJK} $B$ \begin{CJK}{UTF8}{mj}是\end{CJK} $n$ \begin{CJK}{UTF8}{mj}阶正定矩阵时\end{CJK}, $\operatorname{det}(A+B) \geqslant \operatorname{det}(B)$, \begin{CJK}{UTF8}{mj}等号成立当且仅当\end{CJK} $A=0$.

(2) \begin{CJK}{UTF8}{mj}当\end{CJK} $A \neq 0$ \begin{CJK}{UTF8}{mj}时\end{CJK}, $\operatorname{det}(A+E)>1$.

\begin{enumerate}
  \setcounter{enumi}{7}
  \item ( 15 \begin{CJK}{UTF8}{mj}分\end{CJK}) \begin{CJK}{UTF8}{mj}令\end{CJK} $\mathbb{R}_{2}$ \begin{CJK}{UTF8}{mj}表示实数域\end{CJK} $\mathbb{R}$ \begin{CJK}{UTF8}{mj}上的次数不超过\end{CJK} 2 \begin{CJK}{UTF8}{mj}次的多项式构成的实向量空间\end{CJK}.
\end{enumerate}
(1) \begin{CJK}{UTF8}{mj}证明\end{CJK}: \begin{CJK}{UTF8}{mj}对任意的\end{CJK} $p(x), q(x) \in \mathbb{R}_{2}$,
$$
\langle p(x), q(x)\rangle=\int_{-1}^{1} p(x) q(x) \mathrm{d} x
$$
\begin{CJK}{UTF8}{mj}是\end{CJK} $\mathbb{R}_{2}$ \begin{CJK}{UTF8}{mj}上的一个内积\end{CJK}.

(2) \begin{CJK}{UTF8}{mj}将\end{CJK} $\mathbb{R}_{2}$ \begin{CJK}{UTF8}{mj}的基底\end{CJK} $\left\{\frac{1}{\sqrt{2}}, \frac{\sqrt{3}}{\sqrt{2}} x, x^{2}\right\}$ \begin{CJK}{UTF8}{mj}标准正交化\end{CJK}, \begin{CJK}{UTF8}{mj}求出标准正交基\end{CJK}.

\begin{enumerate}
  \setcounter{enumi}{8}
  \item (25 \begin{CJK}{UTF8}{mj}分\end{CJK})
\end{enumerate}
(1)\begin{CJK}{UTF8}{mj}设\end{CJK} $\mathbb{F}$ \begin{CJK}{UTF8}{mj}是任意数域\end{CJK}, $A \in M_{n \times m}(\mathbb{F}), B \in M_{m \times n}(\mathbb{F})$. \begin{CJK}{UTF8}{mj}证明\end{CJK}:
$$
\operatorname{det}\left(\lambda E_{n \times n}-A B\right)=\lambda^{n-m} \operatorname{det}\left(\lambda E_{m \times m}-B A\right)
$$
(2) \begin{CJK}{UTF8}{mj}设\end{CJK} $\alpha=\left(a_{1}, \cdots, a_{n}\right)$ \begin{CJK}{UTF8}{mj}是非零实向量\end{CJK}, \begin{CJK}{UTF8}{mj}令\end{CJK} $\alpha^{T}$ \begin{CJK}{UTF8}{mj}表示向量\end{CJK} $\alpha$ \begin{CJK}{UTF8}{mj}的转置\end{CJK}. \begin{CJK}{UTF8}{mj}求\end{CJK} $C=\alpha^{T} \alpha$ \begin{CJK}{UTF8}{mj}的特征值和特征向量\end{CJK}.

(2) \begin{CJK}{UTF8}{mj}矩阵\end{CJK} $C$ \begin{CJK}{UTF8}{mj}可以相似对角化吗\end{CJK}? \begin{CJK}{UTF8}{mj}为什么\end{CJK}?

\section{5. 华中师范大学 2013 年研究生入学考试试题高等代数 
 李扬 
 微信公众号: sxkyliyang}
\begin{enumerate}
  \item ( 15 \begin{CJK}{UTF8}{mj}分\end{CJK}) \begin{CJK}{UTF8}{mj}设\end{CJK} $\mathbb{F}$ \begin{CJK}{UTF8}{mj}是一个数域\end{CJK}, $f(x), g(x) \in \mathbb{F}[x]$. \begin{CJK}{UTF8}{mj}证明\end{CJK}: $f(x)$ \begin{CJK}{UTF8}{mj}与\end{CJK} $g(x)$ \begin{CJK}{UTF8}{mj}互素当且仅当\end{CJK} $f\left(x^{2}\right)$ \begin{CJK}{UTF8}{mj}与\end{CJK} $g\left(x^{2}\right)$ \begin{CJK}{UTF8}{mj}互素\end{CJK}.

  \item (20 \begin{CJK}{UTF8}{mj}分\end{CJK})

\end{enumerate}
(1) \begin{CJK}{UTF8}{mj}设\end{CJK} $a_{1}, a_{2}, a_{3}, a_{4}$ \begin{CJK}{UTF8}{mj}是\end{CJK} 4 \begin{CJK}{UTF8}{mj}个数\end{CJK}. \begin{CJK}{UTF8}{mj}计算如下\end{CJK} 4 \begin{CJK}{UTF8}{mj}阶行列式\end{CJK}:
$$
\left|\begin{array}{llll}
a_{1} & a_{2} & a_{3} & a_{4} \\
a_{2} & a_{3} & a_{4} & a_{1} \\
a_{3} & a_{4} & a_{1} & a_{2} \\
a_{4} & a_{1} & a_{2} & a_{3}
\end{array}\right| .
$$
(2) \begin{CJK}{UTF8}{mj}设\end{CJK} $D=\left|a_{i j}\right|_{n \times n}$ \begin{CJK}{UTF8}{mj}是一个\end{CJK} $n$ \begin{CJK}{UTF8}{mj}阶行列式\end{CJK}, \begin{CJK}{UTF8}{mj}且每个\end{CJK} $a_{i j}$ \begin{CJK}{UTF8}{mj}均为整数\end{CJK}. \begin{CJK}{UTF8}{mj}设\end{CJK} $b_{i j}$ \begin{CJK}{UTF8}{mj}为\end{CJK} $a_{i j}$ \begin{CJK}{UTF8}{mj}被\end{CJK} 2013 \begin{CJK}{UTF8}{mj}除的余数\end{CJK}, \begin{CJK}{UTF8}{mj}即\end{CJK} $a_{i j}=2013 \cdot q_{i j}+b_{i j}$ \begin{CJK}{UTF8}{mj}这里\end{CJK} $q_{i j}$ \begin{CJK}{UTF8}{mj}为整数\end{CJK}, $0 \leqslant b_{i j}<2013,1 \leqslant i, j \leqslant n$. \begin{CJK}{UTF8}{mj}设\end{CJK} $D_{1}=\left|b_{i j}\right|_{n \times n}$ \begin{CJK}{UTF8}{mj}是\end{CJK} $(i, j)$ \begin{CJK}{UTF8}{mj}元为\end{CJK} $b_{i j}$ \begin{CJK}{UTF8}{mj}的\end{CJK} $n$ \begin{CJK}{UTF8}{mj}阶行列式\end{CJK}. \begin{CJK}{UTF8}{mj}证明\end{CJK}: 2013 \begin{CJK}{UTF8}{mj}整除\end{CJK} $D-D_{1}$.
$$
A=\left(\begin{array}{ccccc}
x & y & \cdots & y & y \\
y & x & \cdots & y & y \\
\vdots & \vdots & & \vdots & \vdots \\
y & y & \cdots & x & y \\
y & y & \cdots & y & x
\end{array}\right)
$$

\begin{enumerate}
  \setcounter{enumi}{4}
  \item (20 \begin{CJK}{UTF8}{mj}分\end{CJK})
\end{enumerate}
(1) \begin{CJK}{UTF8}{mj}设\end{CJK} $A$ \begin{CJK}{UTF8}{mj}是\end{CJK} $n$ \begin{CJK}{UTF8}{mj}阶方阵且\end{CJK} $A^{2}=E$, \begin{CJK}{UTF8}{mj}但是\end{CJK} $A \neq E$ \begin{CJK}{UTF8}{mj}且\end{CJK} $A \neq-E$, \begin{CJK}{UTF8}{mj}这里\end{CJK} $n \geqslant 3, E$ \begin{CJK}{UTF8}{mj}是\end{CJK} $n$ \begin{CJK}{UTF8}{mj}阶单位矩阵\end{CJK}. \begin{CJK}{UTF8}{mj}求\end{CJK} $\operatorname{tr}(A)$

(2) \begin{CJK}{UTF8}{mj}设\end{CJK} $n$ \begin{CJK}{UTF8}{mj}阶方阵\end{CJK} $A$ \begin{CJK}{UTF8}{mj}满足\end{CJK}
$$
A^{3}-6 A^{2}+11 A-6 E=0 .
$$
\begin{CJK}{UTF8}{mj}试确定使得\end{CJK} $A^{2}+A+k E$ \begin{CJK}{UTF8}{mj}可逆的数\end{CJK} $k$ \begin{CJK}{UTF8}{mj}的范围\end{CJK}.

\begin{enumerate}
  \setcounter{enumi}{5}
  \item (30 \begin{CJK}{UTF8}{mj}分\end{CJK})
\end{enumerate}
(1) \begin{CJK}{UTF8}{mj}证明\end{CJK}: \begin{CJK}{UTF8}{mj}实对称矩阵的任何特征根总是实数\end{CJK}.
$$
A=B^{T} B
$$
\begin{CJK}{UTF8}{mj}的前\end{CJK} $l_{1}$ \begin{CJK}{UTF8}{mj}列列向量均为\end{CJK} $v_{1}$, \begin{CJK}{UTF8}{mj}它的\end{CJK} $l_{1}+1$ \begin{CJK}{UTF8}{mj}列到\end{CJK} $l_{2}$ \begin{CJK}{UTF8}{mj}列列向量均为\end{CJK} $v_{2}$, \begin{CJK}{UTF8}{mj}它的最后\end{CJK} $l_{p}$ \begin{CJK}{UTF8}{mj}列列向量均为\end{CJK} $v_{p}$. \begin{CJK}{UTF8}{mj}即\end{CJK} $M=\left(v_{1}, \cdots, v_{1}, v_{2}, \cdots, v_{2}, \cdots, v_{p}, \cdots, v_{p}\right)$ \begin{CJK}{UTF8}{mj}为\end{CJK} $M$ \begin{CJK}{UTF8}{mj}按列向量分块\end{CJK}. $\mathbb{C}^{m}$, $\mathbb{C}^{n}$ \begin{CJK}{UTF8}{mj}分别表示\end{CJK} $m$ \begin{CJK}{UTF8}{mj}维和\end{CJK} $n$ \begin{CJK}{UTF8}{mj}维复列向\end{CJK}
$$
\mathscr{A}: \mathbb{C}^{m} \rightarrow \mathbb{C}^{n}, \alpha \rightarrow M \alpha
$$
(1)\begin{CJK}{UTF8}{mj}证明\end{CJK}: $U(M)$ \begin{CJK}{UTF8}{mj}是\end{CJK} $\mathbb{C}^{m}$ \begin{CJK}{UTF8}{mj}的一个子空间并求这个子空间的维数和一组基底\end{CJK}.

(2) \begin{CJK}{UTF8}{mj}证明\end{CJK}: $U(M)$ \begin{CJK}{UTF8}{mj}是\end{CJK} $\mathscr{A}$ \begin{CJK}{UTF8}{mj}的核空间\end{CJK} $\operatorname{ker}(\mathscr{A})$ \begin{CJK}{UTF8}{mj}的子空间\end{CJK}, \begin{CJK}{UTF8}{mj}并给出\end{CJK} $U(M)=\operatorname{ker}(\mathscr{A})$ \begin{CJK}{UTF8}{mj}的充分必要条件\end{CJK}.

\begin{enumerate}
  \setcounter{enumi}{7}
  \item ( 20 \begin{CJK}{UTF8}{mj}分\end{CJK}) \begin{CJK}{UTF8}{mj}设\end{CJK} $n$ \begin{CJK}{UTF8}{mj}为正整数\end{CJK}, $1 \leqslant k, l \leqslant n$.
\end{enumerate}
$$
a_{1}, a_{2}, \cdots, a_{k-1}, a_{k+1}, \cdots, a_{n}
$$
\begin{CJK}{UTF8}{mj}是不全为\end{CJK} 0 \begin{CJK}{UTF8}{mj}的复数\end{CJK},
$$
b_{1}, b_{2}, \cdots, b_{l-1}, b_{l+1}, \cdots, b_{n}
$$
\begin{CJK}{UTF8}{mj}也是\end{CJK} $n-1$ \begin{CJK}{UTF8}{mj}个不全零的复数\end{CJK}. \begin{CJK}{UTF8}{mj}设\end{CJK} $E$ \begin{CJK}{UTF8}{mj}为\end{CJK} $n$ \begin{CJK}{UTF8}{mj}阶单位矩阵\end{CJK}.\begin{CJK}{UTF8}{mj}把\end{CJK} $E$ \begin{CJK}{UTF8}{mj}的第\end{CJK} $k$ \begin{CJK}{UTF8}{mj}行用行向量\end{CJK} $\left(a_{1}, \cdots, a_{k-1}, 1, a_{k+1}, \cdots, a_{n}\right)$ \begin{CJK}{UTF8}{mj}代替得矩阵\end{CJK} $A$; \begin{CJK}{UTF8}{mj}把\end{CJK} $E$ \begin{CJK}{UTF8}{mj}的第\end{CJK} $l$ \begin{CJK}{UTF8}{mj}列用列向量\end{CJK} $\left(b_{1}, \cdots, b_{l-1}, 1, b_{l+1}, \cdots, b_{n}\right)^{T}$ \begin{CJK}{UTF8}{mj}来代替得矩阵\end{CJK} $B$.

(1) \begin{CJK}{UTF8}{mj}求\end{CJK} $A, B$ \begin{CJK}{UTF8}{mj}的若当标准形\end{CJK}.

(2) \begin{CJK}{UTF8}{mj}证明\end{CJK}: \begin{CJK}{UTF8}{mj}作为复矩阵\end{CJK} $A$ \begin{CJK}{UTF8}{mj}与\end{CJK} $B$ \begin{CJK}{UTF8}{mj}是相似的\end{CJK}.

\section{6. 华中师范大学 2014 年研究生入学考试试题高等代数 
 李扬 
 微信公众号: sxkyliyang}
\begin{enumerate}
  \item (15 \begin{CJK}{UTF8}{mj}分\end{CJK}) \begin{CJK}{UTF8}{mj}计算行列式\end{CJK}
\end{enumerate}
$$
D(x)=\left|\begin{array}{cccc}
1+x & x & \cdots & x \\
x & 2+x & \cdots & x \\
\vdots & \vdots & & \vdots \\
x & x & \cdots & n+x
\end{array}\right|
$$
\begin{CJK}{UTF8}{mj}的值\end{CJK}.

\begin{enumerate}
  \setcounter{enumi}{2}
  \item ( 15 \begin{CJK}{UTF8}{mj}分\end{CJK}) \begin{CJK}{UTF8}{mj}设\end{CJK} $\mathbb{R}$ \begin{CJK}{UTF8}{mj}为实数域\end{CJK}, \begin{CJK}{UTF8}{mj}证明\end{CJK}: \begin{CJK}{UTF8}{mj}实数系多项式环\end{CJK} $\mathbb{R}[x]$ \begin{CJK}{UTF8}{mj}中的不可约多项式的次数为\end{CJK} 1 \begin{CJK}{UTF8}{mj}或者\end{CJK} 2 .

  \item (30 \begin{CJK}{UTF8}{mj}分\end{CJK}) \begin{CJK}{UTF8}{mj}设\end{CJK} $\mathbb{F}$ \begin{CJK}{UTF8}{mj}是个数域\end{CJK}, $\mathbb{F}^{n}$ \begin{CJK}{UTF8}{mj}表示\end{CJK} $n$ \begin{CJK}{UTF8}{mj}维\end{CJK} $\mathbb{F}$ \begin{CJK}{UTF8}{mj}一列构成的向量空间\end{CJK}. \begin{CJK}{UTF8}{mj}证明\end{CJK}:

\end{enumerate}
(1) \begin{CJK}{UTF8}{mj}对\end{CJK} $\mathbb{F}^{n}$ \begin{CJK}{UTF8}{mj}的任意一个子空间\end{CJK} $V$,\begin{CJK}{UTF8}{mj}一定存在\end{CJK} $n$ \begin{CJK}{UTF8}{mj}阶\end{CJK} $\mathbb{F}$ \begin{CJK}{UTF8}{mj}一方阵\end{CJK} $A$, \begin{CJK}{UTF8}{mj}使得\end{CJK} $V$ \begin{CJK}{UTF8}{mj}恰好是齐次线性方程组\end{CJK} $A X=0$ \begin{CJK}{UTF8}{mj}的\end{CJK} \begin{CJK}{UTF8}{mj}解子空间\end{CJK}.

(2) \begin{CJK}{UTF8}{mj}设\end{CJK} $\mathbb{F}^{n}$ \begin{CJK}{UTF8}{mj}的两个子空间\end{CJK} $V_{1}=\{x \mid A X=0\}, V_{2}=\{x \mid B X=0\}$, \begin{CJK}{UTF8}{mj}其中\end{CJK} $A B$ \begin{CJK}{UTF8}{mj}均为\end{CJK} $n$ \begin{CJK}{UTF8}{mj}阶\end{CJK} $\mathbb{F}$ - \begin{CJK}{UTF8}{mj}方阵\end{CJK}. \begin{CJK}{UTF8}{mj}证明\end{CJK}: $\mathbb{F}^{n}=V_{1} \oplus V_{2}$ \begin{CJK}{UTF8}{mj}当且仅当\end{CJK}
$$
\operatorname{rank}(A)+\operatorname{rank}(B)=n \text { 且 } \operatorname{rank}\left(\begin{array}{c}
A \\
B
\end{array}\right)=n
$$

\begin{enumerate}
  \setcounter{enumi}{4}
  \item (30 \begin{CJK}{UTF8}{mj}分\end{CJK}) \begin{CJK}{UTF8}{mj}设\end{CJK} $\alpha_{1}, \alpha_{2}, \alpha_{3}$ \begin{CJK}{UTF8}{mj}是\end{CJK} 3 \begin{CJK}{UTF8}{mj}维复列向量空间\end{CJK} $\mathbb{C}^{3}$ \begin{CJK}{UTF8}{mj}的一组基\end{CJK}, \begin{CJK}{UTF8}{mj}令\end{CJK} $P=\left(\alpha_{1}, \alpha_{2}, \alpha_{3}\right), Q=\left(\alpha_{1}, \alpha_{2}+\alpha_{3}, \alpha_{1}+\alpha_{3}\right)$.
\end{enumerate}
$$
P^{-1} A P=\left[\begin{array}{lll}
1 & 1 & 0 \\
0 & 1 & 1 \\
0 & 0 & 1
\end{array}\right], Q^{-1} B Q=\left[\begin{array}{lll}
1 & 1 & 0 \\
0 & 1 & 1 \\
0 & 0 & 1
\end{array}\right]
$$
\begin{CJK}{UTF8}{mj}证明\end{CJK}:

(1) $P, Q$ \begin{CJK}{UTF8}{mj}均为可逆矩阵\end{CJK}.

(3) $A B \neq B A$. \begin{CJK}{UTF8}{mj}从而说明两个方阵即使有完全相同的特征根与完全相同的特征向量\end{CJK}. \begin{CJK}{UTF8}{mj}它们也不一定相\end{CJK} \begin{CJK}{UTF8}{mj}乘可交换\end{CJK}.

\begin{enumerate}
  \setcounter{enumi}{5}
  \item (15 \begin{CJK}{UTF8}{mj}分\end{CJK}) \begin{CJK}{UTF8}{mj}设\end{CJK} $A$ \begin{CJK}{UTF8}{mj}为\end{CJK} $m \times n$ \begin{CJK}{UTF8}{mj}阶的实矩阵\end{CJK}, $A^{T}$ \begin{CJK}{UTF8}{mj}表示\end{CJK} $A$ \begin{CJK}{UTF8}{mj}的转置矩阵\end{CJK}. \begin{CJK}{UTF8}{mj}证明\end{CJK}:
\end{enumerate}
$$
\operatorname{rank}\left(A^{k}\right)=n-k .
$$

\begin{enumerate}
  \setcounter{enumi}{7}
  \item (30 \begin{CJK}{UTF8}{mj}分\end{CJK})
\end{enumerate}
(2) \begin{CJK}{UTF8}{mj}设\end{CJK} $V$ \begin{CJK}{UTF8}{mj}是一个\end{CJK} $n$ \begin{CJK}{UTF8}{mj}维复向量空间\end{CJK}, $\left(\varepsilon_{1}, \varepsilon_{2}, \cdots, \varepsilon_{n}\right)$ \begin{CJK}{UTF8}{mj}是\end{CJK} $V$ \begin{CJK}{UTF8}{mj}的任意一组基\end{CJK}. \begin{CJK}{UTF8}{mj}证明\end{CJK}: \begin{CJK}{UTF8}{mj}一定可以在\end{CJK} $V$ \begin{CJK}{UTF8}{mj}上定义一个\end{CJK}

\section{7. 华中师范大学 2015 年研究生入学考试试题高等代数 
 李扬 
 微信公众号: sxkyliyang}
\begin{enumerate}
  \item (15 \begin{CJK}{UTF8}{mj}分\end{CJK}) \begin{CJK}{UTF8}{mj}证明\end{CJK} $n$ \begin{CJK}{UTF8}{mj}阶范德蒙行列式\end{CJK}
\end{enumerate}
$$
\left|\begin{array}{cccc}
1 & 1 & \cdots & 1 \\
x_{1} & x_{2} & \cdots & x_{n} \\
\vdots & \vdots & & \vdots \\
x_{1}^{n-1} & x_{2}^{n-1} & \cdots & x_{n}^{n-1}
\end{array}\right|
$$
\begin{CJK}{UTF8}{mj}的值为\end{CJK} $\prod_{1 \leqslant i<j \leqslant n}\left(x_{j}-x_{i}\right)$.

\begin{enumerate}
  \setcounter{enumi}{2}
  \item ( 25 \begin{CJK}{UTF8}{mj}分\end{CJK}) \begin{CJK}{UTF8}{mj}设\end{CJK} $\mathbb{F}$ \begin{CJK}{UTF8}{mj}是一个数域\end{CJK}, $A$ \begin{CJK}{UTF8}{mj}是一个\end{CJK} $n$ \begin{CJK}{UTF8}{mj}阶\end{CJK} $\mathbb{F}$ \begin{CJK}{UTF8}{mj}一方阵\end{CJK}, \begin{CJK}{UTF8}{mj}这里\end{CJK} $n$ \begin{CJK}{UTF8}{mj}是大于\end{CJK} 1 \begin{CJK}{UTF8}{mj}的正整数\end{CJK}. \begin{CJK}{UTF8}{mj}用\end{CJK} $E_{i j}$ \begin{CJK}{UTF8}{mj}表示\end{CJK} $(i, j)$ \begin{CJK}{UTF8}{mj}位置为\end{CJK} 1 \begin{CJK}{UTF8}{mj}其余位置为\end{CJK} 0 \begin{CJK}{UTF8}{mj}的\end{CJK} $n$ \begin{CJK}{UTF8}{mj}阶\end{CJK} $\mathbb{F}$-\begin{CJK}{UTF8}{mj}方阵\end{CJK}. \begin{CJK}{UTF8}{mj}证明以下\end{CJK} 3 \begin{CJK}{UTF8}{mj}条等价\end{CJK}:
\end{enumerate}
(1) $A$ \begin{CJK}{UTF8}{mj}和所有\end{CJK} $\mathbb{F}$ \begin{CJK}{UTF8}{mj}一方阵相乘可交换\end{CJK}.

(2) $A$ \begin{CJK}{UTF8}{mj}和所有可逆\end{CJK} $\mathbb{F}$ \begin{CJK}{UTF8}{mj}一方阵相乘可交换\end{CJK}.

(3) $A$ \begin{CJK}{UTF8}{mj}和所有的\end{CJK} $E_{i j}$ (\begin{CJK}{UTF8}{mj}其中\end{CJK} $1 \leqslant i, j \leqslant n$ \begin{CJK}{UTF8}{mj}但是\end{CJK} $i \neq j$ ) \begin{CJK}{UTF8}{mj}相乘可交换\end{CJK}.

\begin{enumerate}
  \setcounter{enumi}{3}
  \item ( 15 \begin{CJK}{UTF8}{mj}分\end{CJK}) \begin{CJK}{UTF8}{mj}求多项式\end{CJK} $f(x)=x^{3}+1$ \begin{CJK}{UTF8}{mj}与\end{CJK} $g(x)=x^{4}+3 x+2$ \begin{CJK}{UTF8}{mj}的首一最大公因式\end{CJK} $d(x)$, \begin{CJK}{UTF8}{mj}并求多项式\end{CJK} $u(x)$ \begin{CJK}{UTF8}{mj}与多\end{CJK} \begin{CJK}{UTF8}{mj}项式\end{CJK} $v(x)$ \begin{CJK}{UTF8}{mj}使得\end{CJK}
\end{enumerate}
$$
f(x) u(x)+g(x) v(x)=d(x)
$$

\begin{enumerate}
  \setcounter{enumi}{4}
  \item ( 15 \begin{CJK}{UTF8}{mj}分\end{CJK}) \begin{CJK}{UTF8}{mj}设\end{CJK} $\mathbb{F} \subseteq \mathbb{K}$ \begin{CJK}{UTF8}{mj}是两个数域\end{CJK}, $\mathbb{F}^{n}$ \begin{CJK}{UTF8}{mj}表示所有\end{CJK} $n$ \begin{CJK}{UTF8}{mj}维\end{CJK} $\mathbb{F}-$ \begin{CJK}{UTF8}{mj}列向量构成的向量空间\end{CJK}. $\alpha_{1}, \cdots, \alpha_{r}$ \begin{CJK}{UTF8}{mj}为\end{CJK} $\mathbb{F}^{n}$ \begin{CJK}{UTF8}{mj}中的\end{CJK} $r$ \begin{CJK}{UTF8}{mj}个向量\end{CJK}. \begin{CJK}{UTF8}{mj}证明\end{CJK}: $\alpha_{1}, \cdots, \alpha_{r}$ \begin{CJK}{UTF8}{mj}作为\end{CJK} $\mathbb{F}^{n}$ \begin{CJK}{UTF8}{mj}中的向量线性无关当且仅当\end{CJK} $\alpha_{1}, \cdots, \alpha_{r}$ \begin{CJK}{UTF8}{mj}作为\end{CJK} $\mathbb{K}^{n}$ \begin{CJK}{UTF8}{mj}中的向量线性无\end{CJK} \begin{CJK}{UTF8}{mj}关\end{CJK}.

  \item ( 20 \begin{CJK}{UTF8}{mj}分\end{CJK}) \begin{CJK}{UTF8}{mj}设\end{CJK} $\mathbb{F}$ \begin{CJK}{UTF8}{mj}是一个数域\end{CJK}, $M_{n}(\mathbb{F})$ \begin{CJK}{UTF8}{mj}是由所有\end{CJK} $n$ \begin{CJK}{UTF8}{mj}阶\end{CJK} $\mathbb{F}$ \begin{CJK}{UTF8}{mj}一矩阵在矩阵加法和数乘矩阵之下构成的\end{CJK} $\mathbb{F}-$ \begin{CJK}{UTF8}{mj}向量空\end{CJK} \begin{CJK}{UTF8}{mj}间\end{CJK}. \begin{CJK}{UTF8}{mj}设\end{CJK} $V$ \begin{CJK}{UTF8}{mj}是\end{CJK} $M_{n}(\mathbb{F})$ \begin{CJK}{UTF8}{mj}的一个非零子空间\end{CJK}, \begin{CJK}{UTF8}{mj}且满足\end{CJK} $V$ \begin{CJK}{UTF8}{mj}中的任何非零矩阵都是可逆矩阵\end{CJK}.

\end{enumerate}
(1)\begin{CJK}{UTF8}{mj}举出一个这样的子空间\end{CJK} $V$ \begin{CJK}{UTF8}{mj}的例子从而说明这样的子空间确实存在\end{CJK}.

(2) \begin{CJK}{UTF8}{mj}证明\end{CJK} $V$ \begin{CJK}{UTF8}{mj}的维数满足\end{CJK}: $\operatorname{dim}(V) \leqslant n$.

\begin{enumerate}
  \setcounter{enumi}{6}
  \item ( 20 \begin{CJK}{UTF8}{mj}分\end{CJK}) \begin{CJK}{UTF8}{mj}设\end{CJK} $q(X)$ \begin{CJK}{UTF8}{mj}为一个二次型\end{CJK}, \begin{CJK}{UTF8}{mj}且满足只要\end{CJK} $X$ \begin{CJK}{UTF8}{mj}的各分量均非零\end{CJK}, \begin{CJK}{UTF8}{mj}即\end{CJK} $x_{1} \neq 0, \cdots, x_{n} \neq 0$ \begin{CJK}{UTF8}{mj}时就有\end{CJK} $q(X)$
\end{enumerate}
(1) \begin{CJK}{UTF8}{mj}叙述\end{CJK} $\mathscr{A}$ \begin{CJK}{UTF8}{mj}的核子空间\end{CJK} $\operatorname{ker}(\mathscr{A})$ \begin{CJK}{UTF8}{mj}和像子空间\end{CJK} $\operatorname{Im}(\mathscr{A})$ \begin{CJK}{UTF8}{mj}的定义\end{CJK}.

\section{8. 华中师范大学 2016 年研究生入学考试试题高等代数 
 李扬 
 微信公众号: sxkyliyang}
\begin{enumerate}
  \item (15 \begin{CJK}{UTF8}{mj}分\end{CJK}) \begin{CJK}{UTF8}{mj}证明\end{CJK} $n$ \begin{CJK}{UTF8}{mj}阶行列式\end{CJK}
\end{enumerate}
$$
\left|\begin{array}{cccc}
a+b & a b & & \\
1 & a+b & \ddots & \\
& \ddots & \ddots & a b \\
& & 1 & a+b
\end{array}\right|=\frac{a^{n+1}-b^{n+1}}{a-b}
$$
\begin{CJK}{UTF8}{mj}其中\end{CJK} $a, b$ \begin{CJK}{UTF8}{mj}是互不相等的两个数\end{CJK}.

\begin{enumerate}
  \setcounter{enumi}{2}
  \item (15 \begin{CJK}{UTF8}{mj}分\end{CJK}) \begin{CJK}{UTF8}{mj}求多项式\end{CJK}
\end{enumerate}
$$
f(x)=x^{2016}+x^{2015}+x^{2014}
$$
\begin{CJK}{UTF8}{mj}除以多项式\end{CJK} $g(x)=(x-1)^{2}(x+1)$ \begin{CJK}{UTF8}{mj}的余式\end{CJK}.

\begin{enumerate}
  \setcounter{enumi}{3}
  \item ( 15 \begin{CJK}{UTF8}{mj}分\end{CJK}) \begin{CJK}{UTF8}{mj}设\end{CJK} $a_{1}, \cdots, a_{n}, b_{1}, \cdots, b_{n}$ \begin{CJK}{UTF8}{mj}是\end{CJK} $2 n$ \begin{CJK}{UTF8}{mj}个满足\end{CJK} $\sum_{i=1}^{n} a_{i} b_{i}=2$ \begin{CJK}{UTF8}{mj}的数\end{CJK}. \begin{CJK}{UTF8}{mj}设\end{CJK} $n$ \begin{CJK}{UTF8}{mj}阶矩阵\end{CJK}
\end{enumerate}
$$
A=\left(\begin{array}{cccc}
a_{1} b_{1} & a_{1} b_{2} & \cdots & a_{1} b_{n} \\
a_{2} b_{1} & a_{2} b_{2} & \cdots & a_{2} b_{n} \\
\vdots & \vdots & & \vdots \\
a_{n} b_{1} & a_{n} b_{2} & \cdots & a_{n} b_{n}
\end{array}\right)
$$
\begin{CJK}{UTF8}{mj}若\end{CJK} $k$ \begin{CJK}{UTF8}{mj}为正整数\end{CJK}, \begin{CJK}{UTF8}{mj}求\end{CJK} $A^{k}$.

\begin{enumerate}
  \setcounter{enumi}{4}
  \item (15 \begin{CJK}{UTF8}{mj}分\end{CJK}) \begin{CJK}{UTF8}{mj}设\end{CJK} $\mathscr{A}$ \begin{CJK}{UTF8}{mj}是从\end{CJK} $\mathbb{F}$-\begin{CJK}{UTF8}{mj}向量空间\end{CJK} $V$ \begin{CJK}{UTF8}{mj}到\end{CJK} $U$ \begin{CJK}{UTF8}{mj}的线性映射\end{CJK}, \begin{CJK}{UTF8}{mj}若对\end{CJK} $V$ \begin{CJK}{UTF8}{mj}的任意生成组\end{CJK} $\Omega$ \begin{CJK}{UTF8}{mj}都有\end{CJK} $\mathscr{A}(\Omega)=\{\mathscr{A}(\omega) \mid \omega \in \Omega\}$ \begin{CJK}{UTF8}{mj}是\end{CJK} $U$ \begin{CJK}{UTF8}{mj}的生成组\end{CJK}. \begin{CJK}{UTF8}{mj}证明\end{CJK} $\mathscr{A}$ \begin{CJK}{UTF8}{mj}一定是满的线性映射\end{CJK}.

  \item (30 \begin{CJK}{UTF8}{mj}分\end{CJK})

\end{enumerate}
$$
\left(\begin{array}{ccc}
4 & 6 & 0 \\
-3 & -5 & 0 \\
-3 & -6 & 1
\end{array}\right)
$$
$$
\left\{\begin{array}{l}
2 x_{1}+\lambda x_{2}-x_{3}=1 \\
\lambda x_{1}-x_{2}+x_{3}=2 \\
4 x_{1}+5 x_{2}-5 x_{3}=-1
\end{array}\right.
$$
\begin{CJK}{UTF8}{mj}解的情形\end{CJK}, \begin{CJK}{UTF8}{mj}并且在有解的情况之下写出它的通解\end{CJK}.

\begin{enumerate}
  \setcounter{enumi}{6}
  \item (30 \begin{CJK}{UTF8}{mj}分\end{CJK}) \begin{CJK}{UTF8}{mj}设\end{CJK} $\mathbb{F}$ \begin{CJK}{UTF8}{mj}是一个数域\end{CJK}. $M_{n}(\mathbb{F})$ \begin{CJK}{UTF8}{mj}表示所有\end{CJK} $n \times n$ \begin{CJK}{UTF8}{mj}的\end{CJK} $\mathbb{F}$ \begin{CJK}{UTF8}{mj}矩阵的复合\end{CJK}. \begin{CJK}{UTF8}{mj}设\end{CJK} $f: M_{n}(\mathbb{F}) \rightarrow \mathbb{F}$ \begin{CJK}{UTF8}{mj}是一个\end{CJK} $\mathbb{F}$ \begin{CJK}{UTF8}{mj}线性函\end{CJK} (1) \begin{CJK}{UTF8}{mj}对\end{CJK} $n$ \begin{CJK}{UTF8}{mj}阶零矩阵\end{CJK} $0_{n \times n}$ \begin{CJK}{UTF8}{mj}有\end{CJK} $f\left(0_{n \times n}\right)=0$.
\end{enumerate}
(2) \begin{CJK}{UTF8}{mj}对任意的\end{CJK} $1 \leqslant k, l \leqslant n$, \begin{CJK}{UTF8}{mj}设\end{CJK} $E_{k l}$ \begin{CJK}{UTF8}{mj}是\end{CJK} $(k, l)$ \begin{CJK}{UTF8}{mj}位置为\end{CJK} 1 \begin{CJK}{UTF8}{mj}其余位置为\end{CJK} 0 \begin{CJK}{UTF8}{mj}的\end{CJK} $n$ \begin{CJK}{UTF8}{mj}阶\end{CJK} $\mathbb{F}$ \begin{CJK}{UTF8}{mj}方阵\end{CJK}. \begin{CJK}{UTF8}{mj}则当\end{CJK} $k \neq l$ \begin{CJK}{UTF8}{mj}时有\end{CJK} $f\left(E_{k l}\right)=0$; \begin{CJK}{UTF8}{mj}对任意的\end{CJK} $1 \leqslant i, j \leqslant n$ \begin{CJK}{UTF8}{mj}有\end{CJK} $f\left(E_{i i}\right)=f\left(E_{j j}\right)$.

(3) \begin{CJK}{UTF8}{mj}一定存在一个数\end{CJK} $c_{0}$ \begin{CJK}{UTF8}{mj}使得\end{CJK}: $f(A)=c_{0} \cdot \operatorname{tr}(A)$ \begin{CJK}{UTF8}{mj}对任意的\end{CJK} $A \in M_{n}(\mathbb{F})$ \begin{CJK}{UTF8}{mj}成立\end{CJK}. \begin{CJK}{UTF8}{mj}这里\end{CJK} $\operatorname{tr}(A)$ \begin{CJK}{UTF8}{mj}表示矩阵\end{CJK} $A$ \begin{CJK}{UTF8}{mj}的\end{CJK} \begin{CJK}{UTF8}{mj}䢍\end{CJK}.

\begin{enumerate}
  \setcounter{enumi}{7}
  \item (30 \begin{CJK}{UTF8}{mj}分\end{CJK}) \begin{CJK}{UTF8}{mj}设\end{CJK} $\mathbb{R}^{n}$ \begin{CJK}{UTF8}{mj}表示所有\end{CJK} $n$ \begin{CJK}{UTF8}{mj}维实列向量构成的实向量空间\end{CJK}, $A$ \begin{CJK}{UTF8}{mj}是一个实正定矩阵\end{CJK}.
\end{enumerate}
(1) \begin{CJK}{UTF8}{mj}证明\end{CJK}: \begin{CJK}{UTF8}{mj}由\end{CJK} $(X, Y)=X^{T} A Y$, \begin{CJK}{UTF8}{mj}这里\end{CJK} $X, Y \in \mathbb{R}^{n}$ \begin{CJK}{UTF8}{mj}且\end{CJK} $X^{T}$ \begin{CJK}{UTF8}{mj}表示\end{CJK} $X$ \begin{CJK}{UTF8}{mj}的转置\end{CJK}.\begin{CJK}{UTF8}{mj}定义了\end{CJK} $\mathbb{R}^{n}$ \begin{CJK}{UTF8}{mj}上的一个正定的\end{CJK}, \begin{CJK}{UTF8}{mj}对\end{CJK} \begin{CJK}{UTF8}{mj}称的双线性型\end{CJK}. \begin{CJK}{UTF8}{mj}从而使得\end{CJK} $\mathbb{R}^{n}$ \begin{CJK}{UTF8}{mj}成为一个欧式空间\end{CJK}.

(2)\begin{CJK}{UTF8}{mj}求上述欧式空间\end{CJK} $\mathbb{R}^{n}$ \begin{CJK}{UTF8}{mj}的一组标准正交基\end{CJK}.

\section{9. 华中师范大学 2017 年研究生入学考试试题高等代数 
 李扬 
 微信公众号: sxkyliyang}
\begin{enumerate}
  \item (20 \begin{CJK}{UTF8}{mj}分\end{CJK})
\end{enumerate}
(1) \begin{CJK}{UTF8}{mj}设\end{CJK} $\alpha_{1}, \cdots, \alpha_{n-1}$ \begin{CJK}{UTF8}{mj}是\end{CJK} $n-1$ \begin{CJK}{UTF8}{mj}个\end{CJK} $n$ \begin{CJK}{UTF8}{mj}维实向量\end{CJK}. \begin{CJK}{UTF8}{mj}设\end{CJK} $n$ \begin{CJK}{UTF8}{mj}阶行列式\end{CJK} $D=\left|\beta_{1}, \cdots, \beta_{n}\right|$, \begin{CJK}{UTF8}{mj}其中\end{CJK} $\beta_{1}, \cdots, \beta_{n}$ \begin{CJK}{UTF8}{mj}是\end{CJK} $D$ \begin{CJK}{UTF8}{mj}的列向量\end{CJK}, \begin{CJK}{UTF8}{mj}且每个向量\end{CJK} $\beta_{i}$ \begin{CJK}{UTF8}{mj}都是\end{CJK} $\alpha_{1}, \cdots, \alpha_{n-1}$ \begin{CJK}{UTF8}{mj}的线性组合\end{CJK}. \begin{CJK}{UTF8}{mj}证明\end{CJK}:
$$
D=0 .
$$
(2) \begin{CJK}{UTF8}{mj}设正整数\end{CJK} $n>2$, \begin{CJK}{UTF8}{mj}设\end{CJK} $f_{1}(x), \cdots, f_{n}(x)$ \begin{CJK}{UTF8}{mj}是\end{CJK} $n$ \begin{CJK}{UTF8}{mj}个次数至多为\end{CJK} $n-2$ \begin{CJK}{UTF8}{mj}的实多项式\end{CJK}, \begin{CJK}{UTF8}{mj}而\end{CJK} $a_{1}, \cdots, a_{n}$ \begin{CJK}{UTF8}{mj}是\end{CJK} $n$ \begin{CJK}{UTF8}{mj}个\end{CJK} \begin{CJK}{UTF8}{mj}实数\end{CJK}. \begin{CJK}{UTF8}{mj}利用\end{CJK} (1) \begin{CJK}{UTF8}{mj}的结论证明如下行列式\end{CJK}
$$
\left|\begin{array}{cccc}
f_{1}\left(a_{1}\right) & f_{1}\left(a_{2}\right) & \cdots & f_{1}\left(a_{n}\right) \\
f_{2}\left(a_{1}\right) & f_{2}\left(a_{2}\right) & \cdots & f_{2}\left(a_{n}\right) \\
\vdots & \vdots & & \vdots \\
f_{n}\left(a_{1}\right) & f_{n}\left(a_{2}\right) & \cdots & f_{n}\left(a_{n}\right)
\end{array}\right|
$$
\begin{CJK}{UTF8}{mj}的值为零\end{CJK}.

\begin{enumerate}
  \setcounter{enumi}{2}
  \item (30 \begin{CJK}{UTF8}{mj}分\end{CJK}) \begin{CJK}{UTF8}{mj}设\end{CJK} $n$ \begin{CJK}{UTF8}{mj}阶实方阵\end{CJK}
\end{enumerate}
$$
C=\left(\begin{array}{ccccc}
a_{0} & a_{1} & \cdots & a_{n-2} & a_{n-1} \\
a_{n-1} & a_{0} & a_{1} & \cdots & a_{n-2} \\
\vdots & \ddots & \ddots & \ddots & \vdots \\
a_{2} & \cdots & \ddots & \ddots & a_{1} \\
a_{1} & a_{2} & \cdots & a_{n-1} & a_{0}
\end{array}\right), T=\left(\begin{array}{cccc}
0 & 1 & \cdots & 0 \\
\vdots & \ddots & \ddots & \vdots \\
\vdots & & \ddots & 1 \\
1 & \cdots & \cdots & 0
\end{array}\right) .
$$
\begin{CJK}{UTF8}{mj}多项式\end{CJK}
$$
f(\lambda)=a_{0}+a_{1} \lambda+\cdots+a_{n-2} \lambda^{n-2}+a_{n-1} \lambda^{n-1} .
$$
(1) \begin{CJK}{UTF8}{mj}证明\end{CJK}: $C=f(T)$.

(3) \begin{CJK}{UTF8}{mj}求\end{CJK} $C$ \begin{CJK}{UTF8}{mj}的所有特征根\end{CJK}.
$$
\left\{\begin{array}{l}
x_{2}+x_{3}+x_{4}+7 x_{5}=0 \\
2 x_{1}+x_{2}+5 x_{3}-x_{4}-x_{5}=0 \\
x_{1}-x_{2}+x_{3}+2 x_{4}+x_{5}=0
\end{array}\right.
$$

\begin{enumerate}
  \setcounter{enumi}{4}
  \item (25 \begin{CJK}{UTF8}{mj}分\end{CJK})
\end{enumerate}
(2) \begin{CJK}{UTF8}{mj}把实数域\end{CJK} $\mathbb{R}$ \begin{CJK}{UTF8}{mj}看成是有理数域\end{CJK} $\mathbb{Q}$ \begin{CJK}{UTF8}{mj}上的向量空间\end{CJK}, \begin{CJK}{UTF8}{mj}证明实数域上的\end{CJK} 2 \begin{CJK}{UTF8}{mj}个数\end{CJK} $\sqrt{2}, \sqrt{3}$ \begin{CJK}{UTF8}{mj}在有理数域上是线\end{CJK} \begin{CJK}{UTF8}{mj}性无关的\end{CJK}. 5. ( 15 \begin{CJK}{UTF8}{mj}分\end{CJK}) \begin{CJK}{UTF8}{mj}设\end{CJK} $V, U$ \begin{CJK}{UTF8}{mj}是数域\end{CJK} $\mathbb{F}$ \begin{CJK}{UTF8}{mj}上的有限维向量空间\end{CJK}, \begin{CJK}{UTF8}{mj}且\end{CJK} $V$ \begin{CJK}{UTF8}{mj}的维数是\end{CJK} $n$. \begin{CJK}{UTF8}{mj}设\end{CJK} $\alpha_{1}, \cdots, \alpha_{n}$ \begin{CJK}{UTF8}{mj}是\end{CJK} $V$ \begin{CJK}{UTF8}{mj}的一组基底\end{CJK}, \begin{CJK}{UTF8}{mj}而\end{CJK} $\gamma_{1}, \cdots, \gamma_{n}$ \begin{CJK}{UTF8}{mj}是\end{CJK} $U$ \begin{CJK}{UTF8}{mj}中任意\end{CJK} $n$ \begin{CJK}{UTF8}{mj}个向量\end{CJK}. \begin{CJK}{UTF8}{mj}证明\end{CJK}: \begin{CJK}{UTF8}{mj}存在唯一的从\end{CJK} $V$ \begin{CJK}{UTF8}{mj}到\end{CJK} $U$ \begin{CJK}{UTF8}{mj}的线性映射\end{CJK} $\mathscr{A}$ \begin{CJK}{UTF8}{mj}满足\end{CJK}
$$
\mathscr{A}\left(\alpha_{i}\right)=\gamma_{i}, i=1, \cdots, n
$$

\begin{enumerate}
  \setcounter{enumi}{6}
  \item (30 \begin{CJK}{UTF8}{mj}分\end{CJK})
\end{enumerate}
(1) \begin{CJK}{UTF8}{mj}设二次型\end{CJK}
$$
f\left(x_{1}, x_{2}, x_{3}, x_{4}\right)=x_{1} x_{2}+x_{2} x_{3}+x_{3} x_{4} .
$$
\begin{CJK}{UTF8}{mj}利用满秩的线性替换把二次型化为平方和\end{CJK}, \begin{CJK}{UTF8}{mj}并求二次型的正和负惯性指数\end{CJK}.

(2) \begin{CJK}{UTF8}{mj}设\end{CJK} $V$ \begin{CJK}{UTF8}{mj}是一个欧式空间\end{CJK}, \begin{CJK}{UTF8}{mj}它上面的内积记为\end{CJK} $\langle *, *\rangle$. \begin{CJK}{UTF8}{mj}证明施瓦茨不等式\end{CJK}:
$$
\langle\alpha, \beta\rangle^{2} \leqslant\langle\alpha, \alpha\rangle\langle\beta, \beta\rangle
$$
\begin{CJK}{UTF8}{mj}对任意的\end{CJK} $\alpha, \beta \in V$ \begin{CJK}{UTF8}{mj}成立\end{CJK}.

\begin{enumerate}
  \setcounter{enumi}{7}
  \item (15 \begin{CJK}{UTF8}{mj}分\end{CJK}) \begin{CJK}{UTF8}{mj}利用若尔当标准形定理证明\end{CJK}: \begin{CJK}{UTF8}{mj}对任意的\end{CJK} $n$ \begin{CJK}{UTF8}{mj}阶复方阵\end{CJK} $A$ \begin{CJK}{UTF8}{mj}一定存在一个正整数\end{CJK} $r$ \begin{CJK}{UTF8}{mj}满足\end{CJK}
\end{enumerate}
$$
\operatorname{rank}\left(A^{r}\right)=\operatorname{rank}\left(A^{r+1}\right)
$$
\begin{CJK}{UTF8}{mj}并求这样的最小正整数\end{CJK} $r$.

\section{0. 华中师范大学 2009 年研究生入学考试试题数学分析}
\begin{CJK}{UTF8}{mj}李扬\end{CJK}

\begin{CJK}{UTF8}{mj}微信公众号\end{CJK}: sxkyliyang

\begin{CJK}{UTF8}{mj}一\end{CJK}. ( 30 \begin{CJK}{UTF8}{mj}分\end{CJK}) \begin{CJK}{UTF8}{mj}计算题\end{CJK}

(1)
$$
\lim _{x \rightarrow 0^{+}} \frac{\sin \left(x^{\alpha}\right) \cos \left[\sin \left(\frac{1}{\ln x}\right)\right]}{(1+x)^{\beta}-1} .
$$
\begin{CJK}{UTF8}{mj}其中\end{CJK} $\alpha>1, \beta>0$ \begin{CJK}{UTF8}{mj}均为常数\end{CJK}.

(2) \begin{CJK}{UTF8}{mj}计算二重积分\end{CJK}
$$
\iint_{D} \frac{\sin y}{y} \mathrm{~d} x \mathrm{~d} y .
$$
\begin{CJK}{UTF8}{mj}其中\end{CJK} $D$ \begin{CJK}{UTF8}{mj}是由\end{CJK} $y=x, y=1$ \begin{CJK}{UTF8}{mj}和\end{CJK} $x=0$ \begin{CJK}{UTF8}{mj}所围成的区域\end{CJK}

(3) \begin{CJK}{UTF8}{mj}求曲线积分\end{CJK}
$$
\oint_{C} \frac{(x-1) \mathrm{d} y-(y-2) \mathrm{d} x}{4(x-1)^{2}+(y-2)^{2}}
$$
\begin{CJK}{UTF8}{mj}其中\end{CJK} $C$ \begin{CJK}{UTF8}{mj}为平面内任意一条不过点\end{CJK} $(1,2)$ \begin{CJK}{UTF8}{mj}的正向光滑封闭简单曲线\end{CJK}.

\begin{CJK}{UTF8}{mj}二\end{CJK}. (12 \begin{CJK}{UTF8}{mj}分\end{CJK}) \begin{CJK}{UTF8}{mj}设函数\end{CJK} $f(x)$ \begin{CJK}{UTF8}{mj}定义在开区间\end{CJK} $(a, b)$ \begin{CJK}{UTF8}{mj}内\end{CJK}, \begin{CJK}{UTF8}{mj}若对任意\end{CJK} $c \in(a, b)$, \begin{CJK}{UTF8}{mj}都有\end{CJK} $\lim _{x \rightarrow c} f(x)$ \begin{CJK}{UTF8}{mj}存在\end{CJK}, \begin{CJK}{UTF8}{mj}且\end{CJK} $\lim _{x \rightarrow a^{+}} f(x)$ \begin{CJK}{UTF8}{mj}和\end{CJK} $\lim _{x \rightarrow b^{-}} f(x)$ \begin{CJK}{UTF8}{mj}也存在\end{CJK}, \begin{CJK}{UTF8}{mj}则\end{CJK} $f(x)$ \begin{CJK}{UTF8}{mj}在开区间\end{CJK} $(a, b)$ \begin{CJK}{UTF8}{mj}内有界\end{CJK}.

\begin{CJK}{UTF8}{mj}三\end{CJK}. (12 \begin{CJK}{UTF8}{mj}分\end{CJK}) \begin{CJK}{UTF8}{mj}证明含参量反常积分\end{CJK}
$$
\int_{0}^{+\infty} x \mathrm{e}^{-x y} \mathrm{~d} y
$$
\begin{CJK}{UTF8}{mj}在\end{CJK} $[\delta,+\infty)$ \begin{CJK}{UTF8}{mj}上一致收敛\end{CJK} (\begin{CJK}{UTF8}{mj}其中\end{CJK} $\delta>0)$, \begin{CJK}{UTF8}{mj}但在\end{CJK} $(0,+\infty)$ \begin{CJK}{UTF8}{mj}内不一致收敛\end{CJK}.

\begin{CJK}{UTF8}{mj}四\end{CJK}. ( 20 \begin{CJK}{UTF8}{mj}分\end{CJK}) \begin{CJK}{UTF8}{mj}设函数\end{CJK} $f(x)$ \begin{CJK}{UTF8}{mj}在\end{CJK} $[0,1]$ \begin{CJK}{UTF8}{mj}上连续\end{CJK}, \begin{CJK}{UTF8}{mj}在\end{CJK} $(0,1)$ \begin{CJK}{UTF8}{mj}内可微\end{CJK}, \begin{CJK}{UTF8}{mj}且存在\end{CJK} $M>0$, \begin{CJK}{UTF8}{mj}使得\end{CJK} $\forall x \in(0,1)$,
$$
\left|x f^{\prime}(x)-f(x)\right|<x^{2} M .
$$
\begin{CJK}{UTF8}{mj}证明\end{CJK}:\\
(1) $\frac{f(x)}{x}$ \begin{CJK}{UTF8}{mj}在\end{CJK} $(0,1)$ \begin{CJK}{UTF8}{mj}内一致连续\end{CJK}.\\
(2) $\lim _{x \rightarrow 0^{+}} f^{\prime}(x)$ \begin{CJK}{UTF8}{mj}存在\end{CJK}.

\begin{CJK}{UTF8}{mj}五\end{CJK}. ( 20 \begin{CJK}{UTF8}{mj}分\end{CJK}) \begin{CJK}{UTF8}{mj}证明下面的结论\end{CJK}:

(1) \begin{CJK}{UTF8}{mj}若\end{CJK} $f(x)$ \begin{CJK}{UTF8}{mj}在\end{CJK} $[0,1]$ \begin{CJK}{UTF8}{mj}上连续\end{CJK}, \begin{CJK}{UTF8}{mj}则\end{CJK}
$$
\lim _{n \rightarrow \infty} \int_{0}^{1} x^{n} f(x) \mathrm{d} x=0
$$
(2) \begin{CJK}{UTF8}{mj}若\end{CJK} $f(x)$ \begin{CJK}{UTF8}{mj}在\end{CJK} $[0,1]$ \begin{CJK}{UTF8}{mj}上连续可微\end{CJK}, \begin{CJK}{UTF8}{mj}则\end{CJK}
$$
\lim _{n \rightarrow \infty} n \int_{0}^{1} x^{n} f(x) \mathrm{d} x=f(1)
$$
\begin{CJK}{UTF8}{mj}六\end{CJK}. ( 18 \begin{CJK}{UTF8}{mj}分\end{CJK}) \begin{CJK}{UTF8}{mj}设\end{CJK}
$$
f(x, y)= \begin{cases}\frac{x^{2} y}{x^{2}+y^{2}} \sin \sqrt{x^{2}+y^{2}}, & x^{2}+y^{2} \neq 0 \\ 0, & x^{2}+y^{2}=0 .\end{cases}
$$
\begin{CJK}{UTF8}{mj}讨论\end{CJK} $f(x, y)$ \begin{CJK}{UTF8}{mj}在原点\end{CJK} $(0,0)$ \begin{CJK}{UTF8}{mj}处的连续性\end{CJK}, \begin{CJK}{UTF8}{mj}偏导数的存在性以及可微性\end{CJK}. \begin{CJK}{UTF8}{mj}七\end{CJK}. ( 20 \begin{CJK}{UTF8}{mj}分\end{CJK}) \begin{CJK}{UTF8}{mj}设函数列\end{CJK} $\left\{f_{n}(x)\right\}$ \begin{CJK}{UTF8}{mj}中的每一项函数\end{CJK} $f_{n}(x)$ \begin{CJK}{UTF8}{mj}都是\end{CJK} $[a, b]$ \begin{CJK}{UTF8}{mj}上的单调函数\end{CJK}, \begin{CJK}{UTF8}{mj}试证明\end{CJK}:

(1) \begin{CJK}{UTF8}{mj}若\end{CJK} $\sum_{n=1}^{\infty} f_{n}(a)$ \begin{CJK}{UTF8}{mj}和\end{CJK} $\sum_{n=1}^{\infty} f_{n}(b)$ \begin{CJK}{UTF8}{mj}都绝对收敛\end{CJK}, \begin{CJK}{UTF8}{mj}则\end{CJK} $\sum_{n=1}^{\infty} f_{n}(x)$ \begin{CJK}{UTF8}{mj}在\end{CJK} $[a, b]$ \begin{CJK}{UTF8}{mj}上一致收敛\end{CJK}.

(2) \begin{CJK}{UTF8}{mj}若每一项函数\end{CJK} $f_{n}(x)$ \begin{CJK}{UTF8}{mj}的单调性相同\end{CJK}, \begin{CJK}{UTF8}{mj}且\end{CJK} $\sum_{n=1}^{\infty} f_{n}(a)$ \begin{CJK}{UTF8}{mj}和\end{CJK} $\sum_{n=1}^{\infty} f_{n}(b)$ \begin{CJK}{UTF8}{mj}都收敛\end{CJK}, \begin{CJK}{UTF8}{mj}则\end{CJK} $\sum_{n=1}^{\infty} f_{n}(x)$ \begin{CJK}{UTF8}{mj}在\end{CJK} $[a, b]$ \begin{CJK}{UTF8}{mj}上也一\end{CJK} \begin{CJK}{UTF8}{mj}致收敛\end{CJK}.

\begin{CJK}{UTF8}{mj}八\end{CJK}. (18 \begin{CJK}{UTF8}{mj}分\end{CJK}) \begin{CJK}{UTF8}{mj}设\end{CJK} $f$ \begin{CJK}{UTF8}{mj}连续\end{CJK}, \begin{CJK}{UTF8}{mj}证明\end{CJK}:

(1) \begin{CJK}{UTF8}{mj}证明\end{CJK}:
$$
\iiint_{V} f(z) \mathrm{d} x \mathrm{~d} y \mathrm{~d} z=\pi \int_{-1}^{1} f(x)\left(1-x^{2}\right) \mathrm{d} x
$$
\begin{CJK}{UTF8}{mj}其中\end{CJK} $V: x^{2}+y^{2}+z^{2} \leqslant 1$.

(2) \begin{CJK}{UTF8}{mj}记函数\end{CJK}
$$
F(a, b, c)=\iint_{V} f(a x+b y+c z) \mathrm{d} x \mathrm{~d} y \mathrm{~d} z
$$
\begin{CJK}{UTF8}{mj}其中\end{CJK} $V: x^{2}+y^{2}+z^{2} \leqslant 1$, \begin{CJK}{UTF8}{mj}证明\end{CJK}: \begin{CJK}{UTF8}{mj}球面\end{CJK} $a^{2}+b^{2}+c^{2}=1$ \begin{CJK}{UTF8}{mj}为函数\end{CJK} $F(a, b, c)$ \begin{CJK}{UTF8}{mj}的等值面\end{CJK}, \begin{CJK}{UTF8}{mj}即\end{CJK} $F(a, b, c)$ \begin{CJK}{UTF8}{mj}在\end{CJK} \begin{CJK}{UTF8}{mj}球面\end{CJK} $a^{2}+b^{2}+c^{2}=1$ \begin{CJK}{UTF8}{mj}上恒为常数\end{CJK}, \begin{CJK}{UTF8}{mj}并求出此常数\end{CJK}.

\section{1. 华中师范大学 2010 年研究生入学考试试题数学分析}
\begin{CJK}{UTF8}{mj}李扬\end{CJK}

\begin{CJK}{UTF8}{mj}微信公众号\end{CJK}: sxkyliyang

\begin{CJK}{UTF8}{mj}一\end{CJK}. (30 \begin{CJK}{UTF8}{mj}分\end{CJK}) \begin{CJK}{UTF8}{mj}计算题\end{CJK}

(1) \begin{CJK}{UTF8}{mj}设函数\end{CJK} $f(x)$ \begin{CJK}{UTF8}{mj}定义在\end{CJK} $(-\infty,+\infty)$ \begin{CJK}{UTF8}{mj}上\end{CJK}, \begin{CJK}{UTF8}{mj}满足\end{CJK}:
$$
f(2 x)=f(x) \cos x, \lim _{x \rightarrow 0} f(x)=f(0)=1 .
$$
\begin{CJK}{UTF8}{mj}求\end{CJK} $f(x)$.

(2) \begin{CJK}{UTF8}{mj}设\end{CJK} $a_{n}=\int_{0}^{\frac{\pi}{4}} \tan ^{n} x \mathrm{~d} x$, \begin{CJK}{UTF8}{mj}求\end{CJK}
$$
\sum_{n=1}^{\infty} \frac{1}{n}\left(a_{n}+a_{n+2}\right)
$$
\begin{CJK}{UTF8}{mj}的值\end{CJK}.

(3) \begin{CJK}{UTF8}{mj}求曲线积分\end{CJK}
$$
\oint_{L}(y-z) \mathrm{d} x+(z-x) \mathrm{d} y+(x-y) \mathrm{d} z .
$$
\begin{CJK}{UTF8}{mj}其中\end{CJK} $L$ \begin{CJK}{UTF8}{mj}为平面\end{CJK} $x+y+z=0$ \begin{CJK}{UTF8}{mj}与球面\end{CJK} $x^{2}+y^{2}+z^{2}=1$ \begin{CJK}{UTF8}{mj}相交的交线\end{CJK}, \begin{CJK}{UTF8}{mj}方向从\end{CJK} $z$ \begin{CJK}{UTF8}{mj}轴正向看是逆时针\end{CJK}.

\begin{CJK}{UTF8}{mj}二\end{CJK}. (12 \begin{CJK}{UTF8}{mj}分\end{CJK}) \begin{CJK}{UTF8}{mj}设\end{CJK} $f(x)=x^{\alpha}, \alpha>0$, \begin{CJK}{UTF8}{mj}证明\end{CJK}: \begin{CJK}{UTF8}{mj}当\end{CJK} $0<\alpha \leqslant 1$ \begin{CJK}{UTF8}{mj}时\end{CJK}, $f(x)$ \begin{CJK}{UTF8}{mj}在\end{CJK} $(0,+\infty)$ \begin{CJK}{UTF8}{mj}上一致连续\end{CJK}; \begin{CJK}{UTF8}{mj}当\end{CJK} $\alpha>1$ \begin{CJK}{UTF8}{mj}时\end{CJK}, $f(x)$ \begin{CJK}{UTF8}{mj}在\end{CJK} $(0,+\infty)$ \begin{CJK}{UTF8}{mj}上不一致连续\end{CJK}.

\begin{CJK}{UTF8}{mj}三\end{CJK}. (12 \begin{CJK}{UTF8}{mj}分\end{CJK}) \begin{CJK}{UTF8}{mj}证明含参量\end{CJK} $x$ \begin{CJK}{UTF8}{mj}的反常积分\end{CJK}
$$
\int_{0}^{+\infty} \frac{\sin x y}{y} \mathrm{~d} y
$$
\begin{CJK}{UTF8}{mj}在\end{CJK} $[\delta,+\infty)$ \begin{CJK}{UTF8}{mj}上一致收敛\end{CJK} (\begin{CJK}{UTF8}{mj}其中\end{CJK} $\delta>0)$, \begin{CJK}{UTF8}{mj}但在\end{CJK} $(0,+\infty)$ \begin{CJK}{UTF8}{mj}内不一致收敛\end{CJK}.

\begin{CJK}{UTF8}{mj}四\end{CJK}. ( 20 \begin{CJK}{UTF8}{mj}分\end{CJK}) \begin{CJK}{UTF8}{mj}设函数\end{CJK} $f(x)$ \begin{CJK}{UTF8}{mj}在\end{CJK} $[a, b]$ \begin{CJK}{UTF8}{mj}上连续\end{CJK}, \begin{CJK}{UTF8}{mj}在\end{CJK} $(a, b)$ \begin{CJK}{UTF8}{mj}内二阶可导\end{CJK}, \begin{CJK}{UTF8}{mj}且过点\end{CJK} $A(a, f(a))$ \begin{CJK}{UTF8}{mj}和\end{CJK} $B(b, f(b))$ \begin{CJK}{UTF8}{mj}的直线与曲\end{CJK} \begin{CJK}{UTF8}{mj}线\end{CJK} $y=f(x)$ \begin{CJK}{UTF8}{mj}相交于点\end{CJK} $C(c, f(c))$, \begin{CJK}{UTF8}{mj}其中\end{CJK} $a<c<b$, \begin{CJK}{UTF8}{mj}证明\end{CJK}: \begin{CJK}{UTF8}{mj}存在\end{CJK} $\xi \in(a, b)$, \begin{CJK}{UTF8}{mj}使得\end{CJK}
$$
f^{\prime \prime}(\xi)=0
$$
\begin{CJK}{UTF8}{mj}五\end{CJK}. ( 20 \begin{CJK}{UTF8}{mj}分\end{CJK}) \begin{CJK}{UTF8}{mj}设可微函数列\end{CJK} $\left\{f_{n}(x)\right\}$ \begin{CJK}{UTF8}{mj}在\end{CJK} $[a, b]$ \begin{CJK}{UTF8}{mj}上逐点收敛\end{CJK}, \begin{CJK}{UTF8}{mj}且对任意\end{CJK} $x \in[a, b]$, \begin{CJK}{UTF8}{mj}存在\end{CJK} $x$ \begin{CJK}{UTF8}{mj}的邻域\end{CJK} $U(x)$, \begin{CJK}{UTF8}{mj}使得\end{CJK}

\begin{CJK}{UTF8}{mj}六\end{CJK}. (20 \begin{CJK}{UTF8}{mj}分\end{CJK}) \begin{CJK}{UTF8}{mj}设\end{CJK}
$$
f(x, y)= \begin{cases}x y \ln \left(x^{2}+y^{2}\right), & x^{2}+y^{2} \neq 0 \\ 0, & x^{2}+y^{2}=0\end{cases}
$$
\begin{CJK}{UTF8}{mj}讨论\end{CJK} $f(x, y)$ \begin{CJK}{UTF8}{mj}在原点\end{CJK} $(0,0)$ \begin{CJK}{UTF8}{mj}处的连续性\end{CJK}, \begin{CJK}{UTF8}{mj}偏导数的存在性以及可微性\end{CJK}.
$$
\int_{0}^{+\infty} \frac{1}{f(x)} \mathrm{d} x<+\infty
$$
\begin{CJK}{UTF8}{mj}证明\end{CJK}: (1) \begin{CJK}{UTF8}{mj}存在数列\end{CJK} $x_{n} \in[0,+\infty)(n=1,2, \cdots)$, \begin{CJK}{UTF8}{mj}满足\end{CJK}: $\left\{x_{n}\right\}$ \begin{CJK}{UTF8}{mj}严格单调递增\end{CJK}, $\lim _{n \rightarrow \infty} x_{n}=+\infty, \lim _{n \rightarrow \infty} f\left(x_{n}\right)=+\infty$.

(2) $\lim _{x \rightarrow+\infty} \frac{1}{x^{2}} \int_{0}^{x} f(t) \mathrm{d} t=+\infty$.

\begin{CJK}{UTF8}{mj}八\end{CJK}. ( 18 \begin{CJK}{UTF8}{mj}分\end{CJK}) \begin{CJK}{UTF8}{mj}已知\end{CJK} $f(x, y, z)$ \begin{CJK}{UTF8}{mj}和\end{CJK} $g(x, y, z)$ \begin{CJK}{UTF8}{mj}在\end{CJK} $V: x^{2}+y^{2}+z^{2} \leqslant 1$ \begin{CJK}{UTF8}{mj}上具有二阶连续偏导数\end{CJK}, \begin{CJK}{UTF8}{mj}记\end{CJK}
$$
\Delta=\frac{\partial^{2}}{\partial x^{2}}+\frac{\partial^{2}}{\partial y^{2}}+\frac{\partial^{2}}{\partial z^{2}}, \nabla=\left(\frac{\partial}{\partial x}, \frac{\partial}{\partial y}, \frac{\partial}{\partial z}\right)
$$
(1) \begin{CJK}{UTF8}{mj}证明\end{CJK}:
$$
\iiint_{V}(\nabla g \cdot \nabla f) \mathrm{d} x \mathrm{~d} y \mathrm{~d} z=\oiint_{S} g \frac{\partial f}{\partial \mathbf{n}} \mathrm{d} S-\iint_{V} g \Delta f \mathrm{~d} x \mathrm{~d} y \mathrm{~d} z
$$
\begin{CJK}{UTF8}{mj}其中\end{CJK} $\mathbf{n}$ \begin{CJK}{UTF8}{mj}表示\end{CJK} $S$ \begin{CJK}{UTF8}{mj}的外法线方向\end{CJK}, $S$ \begin{CJK}{UTF8}{mj}为球面\end{CJK} $x^{2}+y^{2}+z^{2}=1$.

(2) \begin{CJK}{UTF8}{mj}若\end{CJK} $\Delta f=x^{2}+y^{2}+z^{2}$, \begin{CJK}{UTF8}{mj}试计算\end{CJK}:
$$
I=\iiint_{V}\left(\frac{x}{\sqrt{x^{2}+y^{2}+z^{2}}} \frac{\partial f}{\partial x}+\frac{y}{\sqrt{x^{2}+y^{2}+z^{2}}} \frac{\partial f}{\partial y}+\frac{z}{\sqrt{x^{2}+y^{2}+z^{2}}} \frac{\partial f}{\partial z}\right) \mathrm{d} x \mathrm{~d} y \mathrm{~d} z .
$$

\section{2. 华中师范大学 2011 年研究生入学考试试题数学分析}
\begin{CJK}{UTF8}{mj}李扬\end{CJK}

\begin{CJK}{UTF8}{mj}微信公众号\end{CJK}: sxkyliyang

\begin{CJK}{UTF8}{mj}一\end{CJK}. (24 \begin{CJK}{UTF8}{mj}分\end{CJK}) \begin{CJK}{UTF8}{mj}设\end{CJK} $x_{1} \in\left(0, \frac{\pi}{2}\right), x_{n+1}=\sin x_{n}$, \begin{CJK}{UTF8}{mj}证明\end{CJK}:

(1) $\lim _{n \rightarrow \infty} x_{n}=0$;

(2) $\lim _{n \rightarrow \infty} n \sin ^{2} x_{n}=3$;

(3) \begin{CJK}{UTF8}{mj}级数\end{CJK} $\sum_{n=1}^{\infty} x_{n}^{p}$, \begin{CJK}{UTF8}{mj}当\end{CJK} $p>2$ \begin{CJK}{UTF8}{mj}时收敛\end{CJK}, \begin{CJK}{UTF8}{mj}当\end{CJK} $p \leqslant 2$ \begin{CJK}{UTF8}{mj}时发散\end{CJK}.

\begin{CJK}{UTF8}{mj}二\end{CJK}. ( 20 \begin{CJK}{UTF8}{mj}分\end{CJK})

(1) \begin{CJK}{UTF8}{mj}设\end{CJK} $\left\{a_{n}\right\}$ \begin{CJK}{UTF8}{mj}和\end{CJK} $\left\{b_{n}\right\}$ \begin{CJK}{UTF8}{mj}都是有界数列\end{CJK}, \begin{CJK}{UTF8}{mj}且\end{CJK} $\alpha a_{n+1}+\beta a_{n}=b_{n}$, \begin{CJK}{UTF8}{mj}其中\end{CJK} $\alpha \neq \beta$, \begin{CJK}{UTF8}{mj}证明\end{CJK}: $\lim _{n \rightarrow \infty} b_{n}$ \begin{CJK}{UTF8}{mj}存在的充分必要条\end{CJK} \begin{CJK}{UTF8}{mj}件是\end{CJK} $\lim _{n \rightarrow \infty} a_{n}$ \begin{CJK}{UTF8}{mj}存在\end{CJK}.

(2) \begin{CJK}{UTF8}{mj}设\end{CJK} $\left\{a_{n}\right\}$ \begin{CJK}{UTF8}{mj}和\end{CJK} $\left\{b_{n}\right\}$ \begin{CJK}{UTF8}{mj}都是有界数列\end{CJK}, \begin{CJK}{UTF8}{mj}且\end{CJK} $\alpha a_{n+1}+\beta a_{n}=b_{n}$, \begin{CJK}{UTF8}{mj}其中\end{CJK} $\alpha=\beta$, \begin{CJK}{UTF8}{mj}试问\end{CJK} (1) \begin{CJK}{UTF8}{mj}的结论是否还成立\end{CJK}, \begin{CJK}{UTF8}{mj}请\end{CJK} \begin{CJK}{UTF8}{mj}据理说明你的结论\end{CJK}.

\begin{CJK}{UTF8}{mj}三\end{CJK}. (20 \begin{CJK}{UTF8}{mj}分\end{CJK}) \begin{CJK}{UTF8}{mj}据理说明下面的问题\end{CJK}:

(1) \begin{CJK}{UTF8}{mj}是否存在\end{CJK} $\mathbb{R} \rightarrow \mathbb{R}$ \begin{CJK}{UTF8}{mj}的连续可微函数\end{CJK} $f(x)$ \begin{CJK}{UTF8}{mj}满足\end{CJK}:
$$
f(x)>0, f^{\prime}(x)=f(f(x)) .
$$
(2) \begin{CJK}{UTF8}{mj}若将\end{CJK} (1) \begin{CJK}{UTF8}{mj}中的条件\end{CJK} “ $f(x)>0 "$ \begin{CJK}{UTF8}{mj}改为\end{CJK} " $f(x) \geqslant 0 "$, \begin{CJK}{UTF8}{mj}结论又如何\end{CJK}?

\begin{CJK}{UTF8}{mj}四\end{CJK}. ( 20 \begin{CJK}{UTF8}{mj}分\end{CJK}) \begin{CJK}{UTF8}{mj}设\end{CJK}
$$
F(x, y)=y+\sin (|x| y)
$$
\begin{CJK}{UTF8}{mj}试问\end{CJK}:

(1) $F(x, y)$ \begin{CJK}{UTF8}{mj}在\end{CJK} $(0,0)$ \begin{CJK}{UTF8}{mj}附近是否满足\end{CJK} $F(x, y)=0$;

(2) \begin{CJK}{UTF8}{mj}在\end{CJK} $(0,0)$ \begin{CJK}{UTF8}{mj}附近是否存在过点\end{CJK} $(0,0)$ \begin{CJK}{UTF8}{mj}的唯一连续可微的函数\end{CJK} $y=f(x)$, \begin{CJK}{UTF8}{mj}使得\end{CJK} $F(x, f(x))=0$, \begin{CJK}{UTF8}{mj}若存在\end{CJK}, \begin{CJK}{UTF8}{mj}求出\end{CJK} $y=f(x)$ \begin{CJK}{UTF8}{mj}和\end{CJK} $f^{\prime}(x)$.

\begin{CJK}{UTF8}{mj}杢\end{CJK}. (24 \begin{CJK}{UTF8}{mj}分\end{CJK}) \begin{CJK}{UTF8}{mj}设\end{CJK}
$$
f(x, y)= \begin{cases}\frac{x^{3}-y^{3}}{x^{2}+y^{2}}, & x^{2}+y^{2} \neq 0 \\ 0, & x^{2}+y^{2}=0 .\end{cases}
$$
\begin{CJK}{UTF8}{mj}坌\end{CJK}. (12 \begin{CJK}{UTF8}{mj}分\end{CJK}) \begin{CJK}{UTF8}{mj}证明\end{CJK}:
$$
\sum_{n=1}^{\infty} \frac{\ln (1+n x)}{n^{x}}
$$
\begin{CJK}{UTF8}{mj}七\end{CJK}. ( 20 \begin{CJK}{UTF8}{mj}分\end{CJK}) \begin{CJK}{UTF8}{mj}若\end{CJK} $f(x)$ \begin{CJK}{UTF8}{mj}是\end{CJK} $[0,+\infty)$ \begin{CJK}{UTF8}{mj}上的单调函数\end{CJK}, \begin{CJK}{UTF8}{mj}则对任意固定的\end{CJK} $a$, \begin{CJK}{UTF8}{mj}有\end{CJK}
$$
\lim _{n \rightarrow \infty} \int_{0}^{a} f(x) \sin n x \mathrm{~d} x=0
$$
$$
\lim _{n \rightarrow \infty} \int_{0}^{+\infty} f(x) \sin n x \mathrm{~d} x=0
$$
\begin{CJK}{UTF8}{mj}八\end{CJK}. (10 \begin{CJK}{UTF8}{mj}分\end{CJK}) \begin{CJK}{UTF8}{mj}设\end{CJK} $\Omega$ \begin{CJK}{UTF8}{mj}是\end{CJK} $x y$ \begin{CJK}{UTF8}{mj}平面上具有光滑边界\end{CJK} $\partial \Omega$ \begin{CJK}{UTF8}{mj}的有界区域\end{CJK}, $u \in C^{2}(\bar{\Omega})$ \begin{CJK}{UTF8}{mj}为非常值函数\end{CJK}, \begin{CJK}{UTF8}{mj}且\end{CJK} $\left.u\right|_{\partial \Omega}=0$, \begin{CJK}{UTF8}{mj}证明\end{CJK}:
$$
\iint_{\bar{\Omega}} u \Delta u \mathrm{~d} x \mathrm{~d} y<0
$$
\begin{CJK}{UTF8}{mj}其中\end{CJK} $\Delta u=\frac{\partial^{2} u}{\partial x^{2}}+\frac{\partial^{2} u}{\partial y^{2}}, \bar{\Omega}=\Omega \cup \partial \Omega$.

\section{3. 华中师范大学 2012 年研究生入学考试试题数学分析 
 李扬 
 微信公众号:sxkyliyang}
\begin{CJK}{UTF8}{mj}一\end{CJK}. (20 \begin{CJK}{UTF8}{mj}分\end{CJK}) \begin{CJK}{UTF8}{mj}设数列\end{CJK} $\left\{x_{n}\right\}$ \begin{CJK}{UTF8}{mj}满足\end{CJK}: $x_{1}=\frac{1}{2}, x_{n+1}=\frac{1}{2}-\frac{1}{2} x_{n}^{2}(n \geqslant 1), A=\sqrt{2}-1$, \begin{CJK}{UTF8}{mj}证明\end{CJK}:

(1) $\lim _{n \rightarrow \infty} x_{n}=A$;

(2) $\sum_{n=1}^{\infty}\left(x_{n}-A\right)$ \begin{CJK}{UTF8}{mj}绝对收敛\end{CJK}.

\begin{CJK}{UTF8}{mj}二\end{CJK}. (20 \begin{CJK}{UTF8}{mj}分\end{CJK}) \begin{CJK}{UTF8}{mj}据理说明下面的结论是否成立\end{CJK} (\begin{CJK}{UTF8}{mj}成立请给出证明\end{CJK}, \begin{CJK}{UTF8}{mj}不成立请举出反例\end{CJK}):

(1) \begin{CJK}{UTF8}{mj}设\end{CJK} $f(x)$ \begin{CJK}{UTF8}{mj}在区间\end{CJK} $I$ \begin{CJK}{UTF8}{mj}上连续\end{CJK}, \begin{CJK}{UTF8}{mj}若\end{CJK} $f(x)$ \begin{CJK}{UTF8}{mj}在区间\end{CJK} $I$ \begin{CJK}{UTF8}{mj}上一致连续\end{CJK}, \begin{CJK}{UTF8}{mj}则\end{CJK} $\sin ^{3}|f(x)|$ \begin{CJK}{UTF8}{mj}也在区间\end{CJK} $I$ \begin{CJK}{UTF8}{mj}上一致连续\end{CJK};

(2) \begin{CJK}{UTF8}{mj}设\end{CJK} $f(x)$ \begin{CJK}{UTF8}{mj}在区间\end{CJK} $I$ \begin{CJK}{UTF8}{mj}上连续\end{CJK}, \begin{CJK}{UTF8}{mj}若\end{CJK} $|f(x)|$ \begin{CJK}{UTF8}{mj}在区间\end{CJK} $I$ \begin{CJK}{UTF8}{mj}上一致连续\end{CJK}, \begin{CJK}{UTF8}{mj}则\end{CJK} $\sin ^{3} f(x)$ \begin{CJK}{UTF8}{mj}也在区间\end{CJK} $I$ \begin{CJK}{UTF8}{mj}上一致连续\end{CJK}.

\begin{CJK}{UTF8}{mj}三\end{CJK}. ( 20 \begin{CJK}{UTF8}{mj}分\end{CJK}) \begin{CJK}{UTF8}{mj}设\end{CJK} $f(x)$ \begin{CJK}{UTF8}{mj}在\end{CJK} $[a, b]$ \begin{CJK}{UTF8}{mj}上连续\end{CJK}, \begin{CJK}{UTF8}{mj}在\end{CJK} $(a, b)$ \begin{CJK}{UTF8}{mj}内可导\end{CJK}, \begin{CJK}{UTF8}{mj}证明\end{CJK}: \begin{CJK}{UTF8}{mj}存在\end{CJK} $\xi \in(a, b)$, \begin{CJK}{UTF8}{mj}使得\end{CJK}
$$
f^{\prime}(\xi)>\frac{f(b)-f(a)}{b-a}
$$
\begin{CJK}{UTF8}{mj}的充分必要条件是\end{CJK} $f(x)$ \begin{CJK}{UTF8}{mj}不为常函数或线性函数\end{CJK}.

\begin{CJK}{UTF8}{mj}四\end{CJK}. ( 20 \begin{CJK}{UTF8}{mj}分\end{CJK}) \begin{CJK}{UTF8}{mj}设\end{CJK}
$$
F(x, y)=x^{2}+y-\cos (x y)
$$
\begin{CJK}{UTF8}{mj}证明\end{CJK}:

(1) \begin{CJK}{UTF8}{mj}在\end{CJK} $(0,1)$ \begin{CJK}{UTF8}{mj}附近存在过点\end{CJK} $(0,1)$ \begin{CJK}{UTF8}{mj}的唯一连续可微函数\end{CJK} $y=f(x)$, \begin{CJK}{UTF8}{mj}使得\end{CJK} $F(x, f(x)) \equiv 0$;

(2) \begin{CJK}{UTF8}{mj}在\end{CJK} $x=0$ \begin{CJK}{UTF8}{mj}附近\end{CJK}, \begin{CJK}{UTF8}{mj}当\end{CJK} $x>0$ \begin{CJK}{UTF8}{mj}时\end{CJK}, $y=f(x)$ \begin{CJK}{UTF8}{mj}单调递减\end{CJK}; \begin{CJK}{UTF8}{mj}当\end{CJK} $x<0$ \begin{CJK}{UTF8}{mj}时\end{CJK}, $y=f(x)$ \begin{CJK}{UTF8}{mj}单调递减\end{CJK}.

\begin{CJK}{UTF8}{mj}五\end{CJK}. (21 \begin{CJK}{UTF8}{mj}分\end{CJK}) \begin{CJK}{UTF8}{mj}设\end{CJK}
$$
f(x, y)= \begin{cases}\left(x^{2}+y^{2}\right) \cos \frac{1}{\sqrt{x^{2}+y^{2}}}, & x^{2}+y^{2} \neq 0 \\ 0, & x^{2}+y^{2}=0\end{cases}
$$
(1) \begin{CJK}{UTF8}{mj}求\end{CJK} $f_{x}^{\prime}(x, y), f_{y}^{\prime}(x, y)$.

(2) \begin{CJK}{UTF8}{mj}证明\end{CJK}: $f_{x}^{\prime}(x, y), f_{y}^{\prime}(x, y)$ \begin{CJK}{UTF8}{mj}在点\end{CJK} $(0,0)$ \begin{CJK}{UTF8}{mj}不连续\end{CJK}.

\begin{CJK}{UTF8}{mj}六\end{CJK}. (15 \begin{CJK}{UTF8}{mj}分\end{CJK}) \begin{CJK}{UTF8}{mj}求幂级数\end{CJK}
$$
\sum_{n=1}^{\infty}\left[y\left(\frac{1}{n}\right)-\frac{1}{n}-1\right] x^{n}
$$
\begin{CJK}{UTF8}{mj}的收敛域\end{CJK}, \begin{CJK}{UTF8}{mj}其中\end{CJK} $y=y(x)$ \begin{CJK}{UTF8}{mj}满足\end{CJK}: $y^{\prime}=x+y$ \begin{CJK}{UTF8}{mj}且\end{CJK} $\left.y\right|_{x=0}=1$.
$$
\left|\int_{x}^{x+c} \sin t^{2} \mathrm{~d} t\right| \leqslant \frac{1}{x}(\text { 其中 } c>0) \text {. }
$$
\begin{CJK}{UTF8}{mj}八\end{CJK}. (14 \begin{CJK}{UTF8}{mj}分\end{CJK}) \begin{CJK}{UTF8}{mj}设\end{CJK} $P(x, y)$ \begin{CJK}{UTF8}{mj}和\end{CJK} $Q(x, y)$ \begin{CJK}{UTF8}{mj}在\end{CJK} $\mathbb{R}^{2}$ \begin{CJK}{UTF8}{mj}上具有连续的偏导数\end{CJK}, \begin{CJK}{UTF8}{mj}且对任意\end{CJK} $\left(x_{0}, y_{0}\right) \in \mathbb{R}^{2}$, \begin{CJK}{UTF8}{mj}以及任意\end{CJK} $r>0$, \begin{CJK}{UTF8}{mj}总有\end{CJK}
$$
\int_{L} P \mathrm{~d} x+Q \mathrm{~d} y=0
$$
\begin{CJK}{UTF8}{mj}其中\end{CJK} $L$ \begin{CJK}{UTF8}{mj}为以\end{CJK} $\left(x_{0}, y_{0}\right)$ \begin{CJK}{UTF8}{mj}为心\end{CJK} $r>0$ \begin{CJK}{UTF8}{mj}为半径的上半圆周\end{CJK}, \begin{CJK}{UTF8}{mj}方向是逆时针\end{CJK}, \begin{CJK}{UTF8}{mj}证明\end{CJK}: \begin{CJK}{UTF8}{mj}在\end{CJK} $\mathbb{R}^{2}$ \begin{CJK}{UTF8}{mj}上\end{CJK}, $P(x, y) \equiv 0, \frac{\partial Q}{\partial x} \equiv 0$.

\section{4. 华中师范大学 2013 年研究生入学考试试题数学分析}
\begin{CJK}{UTF8}{mj}李扬\end{CJK}

\begin{CJK}{UTF8}{mj}微信公众号\end{CJK}: sxkyliyang

\begin{CJK}{UTF8}{mj}一\end{CJK}. (20 \begin{CJK}{UTF8}{mj}分\end{CJK})

(1) \begin{CJK}{UTF8}{mj}写出\end{CJK} $\lim _{x \rightarrow+\infty} f(x)=A$ \begin{CJK}{UTF8}{mj}的归结原则\end{CJK}.

(2) \begin{CJK}{UTF8}{mj}设\end{CJK} $f(x)$ \begin{CJK}{UTF8}{mj}定义在\end{CJK} $[0,+\infty)$ \begin{CJK}{UTF8}{mj}上\end{CJK}, \begin{CJK}{UTF8}{mj}且满足\end{CJK}: $f(x)$ \begin{CJK}{UTF8}{mj}在\end{CJK} $x=0$ \begin{CJK}{UTF8}{mj}连续\end{CJK}, \begin{CJK}{UTF8}{mj}对任意\end{CJK} $x \in(0,+\infty), f(a x)=f(x),($ \begin{CJK}{UTF8}{mj}其中\end{CJK} $a>1$ \begin{CJK}{UTF8}{mj}为常数\end{CJK} $)$, \begin{CJK}{UTF8}{mj}若\end{CJK} $\lim _{x \rightarrow+\infty} f(x)=1$, \begin{CJK}{UTF8}{mj}则在\end{CJK} $[0,+\infty)$ \begin{CJK}{UTF8}{mj}上\end{CJK}, $f(x) \equiv 1$.

\begin{CJK}{UTF8}{mj}二\end{CJK}. (20 \begin{CJK}{UTF8}{mj}分\end{CJK})

(1) \begin{CJK}{UTF8}{mj}若\end{CJK} $f(u, v)$ \begin{CJK}{UTF8}{mj}在点集\end{CJK} $G \subset \mathbb{R}^{2}$ \begin{CJK}{UTF8}{mj}上一致收敛\end{CJK}, $u=g(x)$ \begin{CJK}{UTF8}{mj}和\end{CJK} $v=h(x)$ \begin{CJK}{UTF8}{mj}都在区间\end{CJK} $I$ \begin{CJK}{UTF8}{mj}上一致收敛\end{CJK}, \begin{CJK}{UTF8}{mj}且对任意\end{CJK} $x \in I,(g(x), h(x)) \in G$, \begin{CJK}{UTF8}{mj}则\end{CJK} $f(g(x), h(x))$ \begin{CJK}{UTF8}{mj}在区间\end{CJK} $I$ \begin{CJK}{UTF8}{mj}上一致连续\end{CJK}.

(2) \begin{CJK}{UTF8}{mj}试讨论\end{CJK}
$$
\cos (\sin x+\sqrt{x})
$$
\begin{CJK}{UTF8}{mj}在区间\end{CJK} $[0,+\infty)$ \begin{CJK}{UTF8}{mj}上的一致连续性\end{CJK}.

\begin{CJK}{UTF8}{mj}三\end{CJK}. ( 20 \begin{CJK}{UTF8}{mj}分\end{CJK}) \begin{CJK}{UTF8}{mj}设\end{CJK} $f(x)$ \begin{CJK}{UTF8}{mj}在\end{CJK} $[a, b]$ \begin{CJK}{UTF8}{mj}上连续\end{CJK}, \begin{CJK}{UTF8}{mj}在\end{CJK} $(a, b)$ \begin{CJK}{UTF8}{mj}内可导\end{CJK}, \begin{CJK}{UTF8}{mj}若\end{CJK} $f(x)$ \begin{CJK}{UTF8}{mj}不是常函数或一次函数\end{CJK}, \begin{CJK}{UTF8}{mj}则存在\end{CJK} $\xi \in(a, b)$, \begin{CJK}{UTF8}{mj}使\end{CJK} \begin{CJK}{UTF8}{mj}得\end{CJK}
$$
\left|f^{\prime}(\xi)\right|>\left|\frac{f(b)-f(a)}{b-a}\right|
$$
\begin{CJK}{UTF8}{mj}四\end{CJK}. ( 20 \begin{CJK}{UTF8}{mj}分\end{CJK}) \begin{CJK}{UTF8}{mj}设\end{CJK} $f(x)$ \begin{CJK}{UTF8}{mj}在\end{CJK} 1 \begin{CJK}{UTF8}{mj}的某领域\end{CJK} $U(1)$ \begin{CJK}{UTF8}{mj}内可导\end{CJK}, $f^{\prime}(x)$ \begin{CJK}{UTF8}{mj}在\end{CJK} $U(1)$ \begin{CJK}{UTF8}{mj}内连续\end{CJK}, \begin{CJK}{UTF8}{mj}且\end{CJK} $f^{\prime}(x) \neq 0$, \begin{CJK}{UTF8}{mj}据理说明\end{CJK}: \begin{CJK}{UTF8}{mj}方程\end{CJK}
$$
2 f(x y)=f(x)+f(y)
$$
\begin{CJK}{UTF8}{mj}在点\end{CJK} $(1,1)$ \begin{CJK}{UTF8}{mj}的邻域内能确定唯一的连续可微隐函数\end{CJK} $y=g(x)$, \begin{CJK}{UTF8}{mj}并求出\end{CJK} $g^{\prime}(1)$.

\begin{CJK}{UTF8}{mj}五\end{CJK}. ( 21 \begin{CJK}{UTF8}{mj}分\end{CJK}) \begin{CJK}{UTF8}{mj}设\end{CJK}
$$
f(x, y)=\left\{\begin{array}{lr}
x y \cos \frac{1}{\sqrt{x^{2}+y^{2}}}, & x^{2}+y^{2} \neq 0 \\
0, & x^{2}+y^{2}=0
\end{array}\right.
$$
(1) \begin{CJK}{UTF8}{mj}求\end{CJK} $f_{x}^{\prime}(x, y), f_{y}^{\prime}(x, y)$.

(2) \begin{CJK}{UTF8}{mj}讨论\end{CJK} $f_{x}^{\prime}(x, y), f_{y}^{\prime}(x, y)$ \begin{CJK}{UTF8}{mj}在点\end{CJK} $(0,0)$ \begin{CJK}{UTF8}{mj}的连续性\end{CJK}.

(3) \begin{CJK}{UTF8}{mj}讨论\end{CJK} $f(x, y)$ \begin{CJK}{UTF8}{mj}在点\end{CJK} $(0,0)$ \begin{CJK}{UTF8}{mj}的可微性\end{CJK}.

\begin{CJK}{UTF8}{mj}六\end{CJK}. (15 \begin{CJK}{UTF8}{mj}分\end{CJK}) \begin{CJK}{UTF8}{mj}设\end{CJK} $\left\{f_{n}(x)\right\}$ \begin{CJK}{UTF8}{mj}为\end{CJK} $[a, b]$ \begin{CJK}{UTF8}{mj}上的连续函数列\end{CJK}, $f(x)$ \begin{CJK}{UTF8}{mj}为\end{CJK} $[a, b]$ \begin{CJK}{UTF8}{mj}上的连续函数\end{CJK}, \begin{CJK}{UTF8}{mj}若对任意\end{CJK} $x \in[a, b], f_{n}(x) \leqslant$ $f_{n+1}(x)(n=1,2, \cdots)$, \begin{CJK}{UTF8}{mj}且\end{CJK} $\lim _{n \rightarrow \infty} f_{n}(x)=f(x)$, \begin{CJK}{UTF8}{mj}则\end{CJK} $\left\{f_{n}(x)\right\}$ \begin{CJK}{UTF8}{mj}在\end{CJK} $[a, b]$ \begin{CJK}{UTF8}{mj}上一致收敛于\end{CJK} $f(x)$.

\begin{CJK}{UTF8}{mj}七\end{CJK}. ( 20 \begin{CJK}{UTF8}{mj}分\end{CJK})

(1) \begin{CJK}{UTF8}{mj}证明\end{CJK}: \begin{CJK}{UTF8}{mj}若\end{CJK} $f(x)$ \begin{CJK}{UTF8}{mj}为\end{CJK} $[a, b]$ \begin{CJK}{UTF8}{mj}上黎曼可积\end{CJK}, \begin{CJK}{UTF8}{mj}则\end{CJK}
$$
\lim _{n \rightarrow \infty} \int_{a}^{b} f(x) \sin n x \mathrm{~d} x=0 .
$$
(2) \begin{CJK}{UTF8}{mj}证明\end{CJK}: \begin{CJK}{UTF8}{mj}若无穷限反常积分\end{CJK} $\int_{a}^{+\infty} f(x) \mathrm{d} x$ \begin{CJK}{UTF8}{mj}绝对收敛\end{CJK}, \begin{CJK}{UTF8}{mj}则\end{CJK}
$$
\lim _{n \rightarrow \infty} \int_{a}^{+\infty} f(x) \sin n x \mathrm{~d} x=0
$$
\begin{CJK}{UTF8}{mj}八\end{CJK}. ( 14 \begin{CJK}{UTF8}{mj}分\end{CJK}) \begin{CJK}{UTF8}{mj}设\end{CJK} $f(x, y)$ \begin{CJK}{UTF8}{mj}和\end{CJK} $g(x, y)$ \begin{CJK}{UTF8}{mj}在光滑闭曲线\end{CJK} $L$ \begin{CJK}{UTF8}{mj}围成的有界闭区域\end{CJK} $D$ \begin{CJK}{UTF8}{mj}上具有二阶连续的偏导数\end{CJK}, \begin{CJK}{UTF8}{mj}则\end{CJK}

(1)
$$
\int_{L}\left|\begin{array}{cc}
\frac{\partial f}{\partial \vec{n}} & \frac{\partial g}{\partial \vec{n}} \\
f & g
\end{array}\right| \mathrm{d} s=\iint_{D}\left|\begin{array}{cc}
\Delta f & \Delta g \\
f & g
\end{array}\right| \mathrm{d} x \mathrm{~d} y
$$
\begin{CJK}{UTF8}{mj}其中\end{CJK} $\vec{n}$ \begin{CJK}{UTF8}{mj}为\end{CJK} $L$ \begin{CJK}{UTF8}{mj}的外法线方向\end{CJK}, $\frac{\partial f}{\partial \vec{n}}$ \begin{CJK}{UTF8}{mj}和\end{CJK} $\frac{\partial g}{\partial \vec{n}}$ \begin{CJK}{UTF8}{mj}分别是\end{CJK} $f(x, y)$ \begin{CJK}{UTF8}{mj}和\end{CJK} $g(x, y)$ \begin{CJK}{UTF8}{mj}沿\end{CJK} $\vec{n}$ \begin{CJK}{UTF8}{mj}的方向导数\end{CJK}, $\Delta f=$ $\frac{\partial^{2} f}{\partial x^{2}}+\frac{\partial^{2} f}{\partial y^{2}}, \Delta g=\frac{\partial^{2} g}{\partial x^{2}}+\frac{\partial^{2} g}{\partial y^{2}} .$

(2) \begin{CJK}{UTF8}{mj}若进一步有\end{CJK} $f(x, y)$ \begin{CJK}{UTF8}{mj}为\end{CJK} $D$ \begin{CJK}{UTF8}{mj}上的调和函数\end{CJK}, $\left(x_{0}, y_{0}\right)$ \begin{CJK}{UTF8}{mj}为\end{CJK} $D$ \begin{CJK}{UTF8}{mj}的内点\end{CJK}, $g(x, y)=\ln r, r=\sqrt{\left(x-x_{0}\right)^{2}+\left(y-y_{0}\right)^{2}}$, \begin{CJK}{UTF8}{mj}则\end{CJK}
$$
f\left(x_{0}, y_{0}\right)=\frac{1}{2 \pi} \int_{L}\left[f \cdot \frac{\partial \ln r}{\partial \vec{n}}-\ln r \cdot \frac{\partial f}{\partial \vec{n}}\right] \mathrm{d} s
$$

\section{5. 华中师范大学 2014 年研究生入学考试试题数学分析}
\begin{CJK}{UTF8}{mj}李扬\end{CJK}

\begin{CJK}{UTF8}{mj}微信公众号\end{CJK}: sxkyliyang

\begin{CJK}{UTF8}{mj}一\end{CJK}. \begin{CJK}{UTF8}{mj}计算题\end{CJK}(\begin{CJK}{UTF8}{mj}共\end{CJK} 4 \begin{CJK}{UTF8}{mj}小题\end{CJK}, \begin{CJK}{UTF8}{mj}总计\end{CJK} 40 \begin{CJK}{UTF8}{mj}分\end{CJK})

(1) \begin{CJK}{UTF8}{mj}求极限\end{CJK}
$$
\lim _{x \rightarrow 0}\left[\frac{(1+x)^{\frac{1}{x}}}{\mathrm{e}}\right]^{\frac{x}{\ln (1+x)}} .
$$
(2)\begin{CJK}{UTF8}{mj}求极限\end{CJK}
$$
\lim _{n \rightarrow+\infty} \frac{1}{n^{2}} \sum_{i=1}^{n} \sum_{j=1}^{n}\left[\frac{2 i}{n}+\frac{j}{n}\right] .
$$
\begin{CJK}{UTF8}{mj}这里\end{CJK} [.] \begin{CJK}{UTF8}{mj}表示取整\end{CJK}.

(3) \begin{CJK}{UTF8}{mj}计算积分\end{CJK}
$$
\oint_{L} \frac{x \mathrm{~d} y-y \mathrm{~d} x}{x^{2}+2 y^{2}}
$$
\begin{CJK}{UTF8}{mj}其中\end{CJK} $L$ \begin{CJK}{UTF8}{mj}为平面内任意一条不过原点的正向光滑封闭曲线\end{CJK}.

(4) \begin{CJK}{UTF8}{mj}求极限\end{CJK}
$$
\lim _{r \rightarrow 0^{+}} \frac{1}{r^{5}} \iiint_{x^{2}+y^{2}+z^{2} \leqslant r^{2}} \ln \left(1+x^{2}+y^{2}+z^{2}\right) \mathrm{d} x \mathrm{~d} y \mathrm{~d} z
$$
\begin{CJK}{UTF8}{mj}一\end{CJK}. ( 20 \begin{CJK}{UTF8}{mj}分\end{CJK}) \begin{CJK}{UTF8}{mj}设\end{CJK}
$$
f(x, y)= \begin{cases}\left(x^{2}+y^{2}\right) \cos \frac{1}{\sqrt{x^{2}+y^{2}}}, & (x, y) \neq(0,0) \\ 0, & (x, y)=(0,0) .\end{cases}
$$
\begin{CJK}{UTF8}{mj}讨论\end{CJK} $f_{x}(x, y), f_{y}(x, y)$ \begin{CJK}{UTF8}{mj}在点\end{CJK} $(0,0)$ \begin{CJK}{UTF8}{mj}的连续性\end{CJK}, \begin{CJK}{UTF8}{mj}偏导数的存在性\end{CJK}, \begin{CJK}{UTF8}{mj}可微性及偏导函数的连续性\end{CJK}.

\begin{CJK}{UTF8}{mj}三\end{CJK}. ( 15 \begin{CJK}{UTF8}{mj}分\end{CJK}) \begin{CJK}{UTF8}{mj}设\end{CJK} $f(x)$ \begin{CJK}{UTF8}{mj}在\end{CJK} $[0,1]$ \begin{CJK}{UTF8}{mj}上可导\end{CJK}, $f(0)=0, f(1)=1, a>0, b>0$ \begin{CJK}{UTF8}{mj}为常数\end{CJK}.

(1) \begin{CJK}{UTF8}{mj}证明\end{CJK}: \begin{CJK}{UTF8}{mj}存在\end{CJK} $\xi \in(0,1)$, \begin{CJK}{UTF8}{mj}使得\end{CJK}
$$
f(\xi)=\frac{a}{a+b} .
$$
(2) \begin{CJK}{UTF8}{mj}证明\end{CJK}: \begin{CJK}{UTF8}{mj}存在\end{CJK} $(0,1)$ \begin{CJK}{UTF8}{mj}内两个互异的点\end{CJK} $\xi_{1}, \xi_{2}$, \begin{CJK}{UTF8}{mj}使得\end{CJK}
$$
\frac{a}{f^{\prime}\left(\xi_{1}\right)}+\frac{b}{f^{\prime}\left(\xi_{2}\right)}=a+b .
$$
\begin{CJK}{UTF8}{mj}四\end{CJK}. ( 10 \begin{CJK}{UTF8}{mj}分\end{CJK}) \begin{CJK}{UTF8}{mj}证明\end{CJK}: \begin{CJK}{UTF8}{mj}方程\end{CJK}
$$
x^{3}+y+1=\cos (x y)
$$
\begin{CJK}{UTF8}{mj}在\end{CJK} $(0,0)$ \begin{CJK}{UTF8}{mj}的某个领域内可以唯一确定隐函数\end{CJK} $y=f(x)$, \begin{CJK}{UTF8}{mj}并求\end{CJK} $f^{\prime}(0)$ \begin{CJK}{UTF8}{mj}的值\end{CJK}.

\begin{CJK}{UTF8}{mj}五\end{CJK}. (15 \begin{CJK}{UTF8}{mj}分\end{CJK}) \begin{CJK}{UTF8}{mj}求幂级数\end{CJK}
$$
\sum_{n=1}^{\infty} \frac{(n-1)^{2}}{n+1} x^{n}
$$
\begin{CJK}{UTF8}{mj}的收敛域及和函数\end{CJK} $S(x)$, \begin{CJK}{UTF8}{mj}并计算\end{CJK}
$$
\sum_{n=1}^{+\infty} \frac{(n-1)^{2}}{n+1} 3^{-n}
$$
\begin{CJK}{UTF8}{mj}六\end{CJK}. (15 \begin{CJK}{UTF8}{mj}分\end{CJK}) \begin{CJK}{UTF8}{mj}证明\end{CJK}: \begin{CJK}{UTF8}{mj}含参量反常积分\end{CJK}
$$
\int_{0}^{+\infty} x \mathrm{e}^{-x y} \mathrm{~d} x
$$
\begin{CJK}{UTF8}{mj}在\end{CJK} $[\delta,+\infty)$ \begin{CJK}{UTF8}{mj}上一致收敛\end{CJK}.

\begin{CJK}{UTF8}{mj}七\end{CJK}. (15 \begin{CJK}{UTF8}{mj}分\end{CJK}) \begin{CJK}{UTF8}{mj}设点\end{CJK} $M\left(x_{0}, y_{0}, z_{0}\right)$ \begin{CJK}{UTF8}{mj}是椭球面\end{CJK} $x^{2}+\frac{y^{2}}{2}+\frac{z^{2}}{3}=1$ \begin{CJK}{UTF8}{mj}上位于第一卦限的点\end{CJK}, $S$ \begin{CJK}{UTF8}{mj}是该椭球面在\end{CJK} $M\left(x_{0}, y_{0}, z_{0}\right)$ \begin{CJK}{UTF8}{mj}处的切平面被三个坐标面所截得的三角形上侧\end{CJK}, \begin{CJK}{UTF8}{mj}求\end{CJK} $M\left(x_{0}, y_{0}, z_{0}\right)$ \begin{CJK}{UTF8}{mj}使曲面积分\end{CJK}
$$
\iint_{S} x \mathrm{~d} y \mathrm{~d} z+2 y \mathrm{~d} z \mathrm{~d} x+3 z \mathrm{~d} x \mathrm{~d} y
$$
\begin{CJK}{UTF8}{mj}为最小\end{CJK}, \begin{CJK}{UTF8}{mj}并求此最小值\end{CJK}.

\begin{CJK}{UTF8}{mj}八\end{CJK}. ( 14 \begin{CJK}{UTF8}{mj}分\end{CJK}) \begin{CJK}{UTF8}{mj}设\end{CJK} $D$ \begin{CJK}{UTF8}{mj}为\end{CJK} $\mathbb{R}^{2}$ \begin{CJK}{UTF8}{mj}有界区域\end{CJK}, \begin{CJK}{UTF8}{mj}其边界\end{CJK} $L$ \begin{CJK}{UTF8}{mj}光滑\end{CJK}, \begin{CJK}{UTF8}{mj}函数\end{CJK} $u(x, y)$ \begin{CJK}{UTF8}{mj}和\end{CJK} $v(x, y)$ \begin{CJK}{UTF8}{mj}在\end{CJK} $\bar{D}=D \cup L$ \begin{CJK}{UTF8}{mj}上具有一阶连续偏\end{CJK} \begin{CJK}{UTF8}{mj}导数\end{CJK}, \begin{CJK}{UTF8}{mj}在\end{CJK} $D$ \begin{CJK}{UTF8}{mj}上具有二阶连续的偏导数\end{CJK}
$$
\iint_{D}\left|\begin{array}{cc}
\Delta u & \Delta v \\
u & v
\end{array}\right| \mathrm{d} x \mathrm{~d} y=\oint_{L}\left|\begin{array}{cc}
\frac{\partial u}{\partial \vec{n}} & \frac{\partial v}{\partial \vec{n}} \\
u & v
\end{array}\right| \mathrm{d} s .
$$
\begin{CJK}{UTF8}{mj}其中\end{CJK} $\vec{n}$ \begin{CJK}{UTF8}{mj}为\end{CJK} $L$ \begin{CJK}{UTF8}{mj}的外法线方向\end{CJK}, $\frac{\partial f}{\partial \vec{n}}$ \begin{CJK}{UTF8}{mj}和\end{CJK} $\frac{\partial g}{\partial \vec{n}}$ \begin{CJK}{UTF8}{mj}分别是\end{CJK} $f(x, y)$ \begin{CJK}{UTF8}{mj}和\end{CJK} $g(x, y)$ \begin{CJK}{UTF8}{mj}沿\end{CJK} $\vec{n}$ \begin{CJK}{UTF8}{mj}的方向导数\end{CJK}, $\Delta u=$ $\frac{\partial^{2} u}{\partial x^{2}}+\frac{\partial^{2} u}{\partial y^{2}}, \Delta v=\frac{\partial^{2} v}{\partial x^{2}}+\frac{\partial^{2} v}{\partial y^{2}} .$

(2) \begin{CJK}{UTF8}{mj}利用\end{CJK} (1) \begin{CJK}{UTF8}{mj}证明\end{CJK}: \begin{CJK}{UTF8}{mj}若在\end{CJK} $D$ \begin{CJK}{UTF8}{mj}内\end{CJK}, $\Delta u=0$, \begin{CJK}{UTF8}{mj}即\end{CJK} $u(x, y)$ \begin{CJK}{UTF8}{mj}为\end{CJK} $D$ \begin{CJK}{UTF8}{mj}上的调和函数\end{CJK}, $\left(x_{0}, y_{0}\right)$ \begin{CJK}{UTF8}{mj}为\end{CJK} $D$ \begin{CJK}{UTF8}{mj}的内点\end{CJK}, $g(x, y)=$ $\ln r, r=\sqrt{\left(x-x_{0}\right)^{2}+\left(y-y_{0}\right)^{2}}$, \begin{CJK}{UTF8}{mj}则\end{CJK}
$$
u\left(x_{0}, y_{0}\right)=\frac{1}{2 \pi} \oint_{L}\left[u \cdot \frac{\partial \ln r}{\partial \vec{n}}-\ln r \cdot \frac{\partial u}{\partial \vec{n}}\right] \mathrm{d} s
$$
(3) \begin{CJK}{UTF8}{mj}证明\end{CJK}: \begin{CJK}{UTF8}{mj}若在\end{CJK} $D$ \begin{CJK}{UTF8}{mj}内\end{CJK}, $\Delta u=0, C_{R}$ \begin{CJK}{UTF8}{mj}是\end{CJK} $D$ \begin{CJK}{UTF8}{mj}内以\end{CJK} $\left(x_{0}, y_{0}\right)$ \begin{CJK}{UTF8}{mj}为圆心\end{CJK}, $R$ \begin{CJK}{UTF8}{mj}为半径的任意圆周\end{CJK}, \begin{CJK}{UTF8}{mj}则\end{CJK}
$$
u\left(x_{0}, y_{0}\right)=\frac{1}{2 \pi R} \oint_{L} u(x, y) \mathrm{d} s
$$

\section{6. 华中师范大学 2015 年研究生入学考试试题数学分析 
 李扬 
 微信公众号: sxkyliyang}
\begin{CJK}{UTF8}{mj}一\end{CJK}. \begin{CJK}{UTF8}{mj}计算题\end{CJK}(\begin{CJK}{UTF8}{mj}共\end{CJK} 5 \begin{CJK}{UTF8}{mj}小题\end{CJK}, \begin{CJK}{UTF8}{mj}每小题\end{CJK} 10 \begin{CJK}{UTF8}{mj}分\end{CJK}, \begin{CJK}{UTF8}{mj}共\end{CJK} 50 \begin{CJK}{UTF8}{mj}分\end{CJK})

(1)
$$
\lim _{x \rightarrow 0^{+}}\left[\frac{(1+x)^{\frac{1}{x}}}{\mathrm{e}}\right]^{\frac{1}{\sin x}} .
$$
(2)
$$
\lim _{n \rightarrow+\infty} \sum_{i=1}^{n} \frac{1}{n+\frac{1}{i}} \sqrt{1+\cos \left(\frac{i}{n} \pi\right)} .
$$
$(3)$
$$
\int_{0}^{1} \frac{x^{b}-x^{a}}{\ln x} \mathrm{~d} x,(b>a>0) .
$$
(4) \begin{CJK}{UTF8}{mj}试用幂级数求级数\end{CJK}
$$
\sum_{n=0}^{+\infty} \frac{(-1)^{n}}{(2 n+1) \cdot 3^{n}}
$$
\begin{CJK}{UTF8}{mj}的和\end{CJK}.

(5) \begin{CJK}{UTF8}{mj}设隐函数\end{CJK} $y=y(x)$ \begin{CJK}{UTF8}{mj}由方程\end{CJK}
$$
y+\frac{1}{3} y^{3}=x-\frac{1}{3} x^{3}+\frac{2}{3}
$$
\begin{CJK}{UTF8}{mj}确定\end{CJK}, \begin{CJK}{UTF8}{mj}求\end{CJK} $y(x)$ \begin{CJK}{UTF8}{mj}的极值\end{CJK}.

\begin{CJK}{UTF8}{mj}二\end{CJK}. ( 20 \begin{CJK}{UTF8}{mj}分\end{CJK}) \begin{CJK}{UTF8}{mj}设函数\end{CJK} $f(x)$ \begin{CJK}{UTF8}{mj}满足\end{CJK}: $f(x)=f(x+2 \pi)$, \begin{CJK}{UTF8}{mj}且\end{CJK} $f(x)=x^{2},(-\pi \leqslant x<\pi)$.

(1) \begin{CJK}{UTF8}{mj}求\end{CJK} $f(x)$ \begin{CJK}{UTF8}{mj}的\end{CJK} Fourier \begin{CJK}{UTF8}{mj}级数展开式\end{CJK}.

(2) \begin{CJK}{UTF8}{mj}证明\end{CJK}: $f(x)$ \begin{CJK}{UTF8}{mj}的\end{CJK} Fourier \begin{CJK}{UTF8}{mj}级数展开式在\end{CJK} $(-\infty,+\infty)$ \begin{CJK}{UTF8}{mj}一致收敛\end{CJK}.

(3) \begin{CJK}{UTF8}{mj}试利用\end{CJK} (2) \begin{CJK}{UTF8}{mj}的结论求级数\end{CJK} $\sum_{n=1}^{+\infty} \frac{1}{n^{4}}$ \begin{CJK}{UTF8}{mj}的和\end{CJK}.

\begin{CJK}{UTF8}{mj}三\end{CJK}. ( 15 \begin{CJK}{UTF8}{mj}分\end{CJK}) \begin{CJK}{UTF8}{mj}设二元函数\end{CJK} $u(x, y)$ \begin{CJK}{UTF8}{mj}在\end{CJK} $\mathbb{R}^{2}$ \begin{CJK}{UTF8}{mj}上具有二阶连续偏导数\end{CJK}, \begin{CJK}{UTF8}{mj}且满足方程\end{CJK}:
$$
\frac{\partial^{2} u}{\partial x^{2}}-4 \frac{\partial^{2} u}{\partial x \partial y}+3 \frac{\partial^{2} u}{\partial y^{2}}=0,
$$
(1)\begin{CJK}{UTF8}{mj}线性变换\end{CJK}
$$
\left\{\begin{array}{l}
\xi=x+a y ; \\
\eta=x+b y
\end{array} \quad(a \neq b)\right.
$$
\begin{CJK}{UTF8}{mj}将方程\end{CJK} $(*)$ \begin{CJK}{UTF8}{mj}变换成为\end{CJK} $\frac{\partial^{2} u}{\partial \xi \partial \eta}=0$, \begin{CJK}{UTF8}{mj}求\end{CJK} $a, b$ \begin{CJK}{UTF8}{mj}的值\end{CJK};

(2) \begin{CJK}{UTF8}{mj}试利用\end{CJK} (1) \begin{CJK}{UTF8}{mj}的结果求级数\end{CJK} (*) \begin{CJK}{UTF8}{mj}的通解\end{CJK}.

\begin{CJK}{UTF8}{mj}四\end{CJK}. ( 15 \begin{CJK}{UTF8}{mj}分\end{CJK}) \begin{CJK}{UTF8}{mj}证明\end{CJK}: \begin{CJK}{UTF8}{mj}函数列\end{CJK} $\left\{f_{n}\right\}$ \begin{CJK}{UTF8}{mj}在区间\end{CJK} $I$ \begin{CJK}{UTF8}{mj}上内闭一致收敛于\end{CJK} $f$ \begin{CJK}{UTF8}{mj}的充要条件是\end{CJK}: \begin{CJK}{UTF8}{mj}对任意\end{CJK} $x_{0} \in I$, \begin{CJK}{UTF8}{mj}存在\end{CJK} $x_{0}$ \begin{CJK}{UTF8}{mj}的一\end{CJK} \begin{CJK}{UTF8}{mj}个领域\end{CJK} $U\left(x_{0}\right)$, \begin{CJK}{UTF8}{mj}使得\end{CJK} $\left\{f_{n}\right\}$ \begin{CJK}{UTF8}{mj}在\end{CJK} $U\left(x_{0}\right) \cap I$ \begin{CJK}{UTF8}{mj}上一致收玫于\end{CJK} $f$.

\begin{CJK}{UTF8}{mj}五\end{CJK}. ( 15 \begin{CJK}{UTF8}{mj}分\end{CJK}) \begin{CJK}{UTF8}{mj}设\end{CJK} $f(x)$ \begin{CJK}{UTF8}{mj}在\end{CJK} $[0,1]$ \begin{CJK}{UTF8}{mj}上具有连续导数\end{CJK}, $f^{\prime}(x)$ \begin{CJK}{UTF8}{mj}在\end{CJK} $[0,1]$ \begin{CJK}{UTF8}{mj}上最大值为\end{CJK} $M$, \begin{CJK}{UTF8}{mj}最小值为\end{CJK} $m$. (1) \begin{CJK}{UTF8}{mj}证明\end{CJK}:
$$
\frac{m}{2 n} \leqslant \frac{1}{n} \sum_{i=1}^{n} f\left(\frac{i}{n}\right)-\int_{0}^{1} f(x) \mathrm{d} x \leqslant \frac{M}{2 n}
$$
(2) \begin{CJK}{UTF8}{mj}证明\end{CJK}:
$$
\lim _{n \rightarrow+\infty} n\left[\frac{1}{n} \sum_{i=1}^{n} f\left(\frac{i}{n}\right)-\int_{0}^{1} f(x) \mathrm{d} x\right]=\frac{1}{2}(f(1)-f(0))
$$
(3) \begin{CJK}{UTF8}{mj}试利用\end{CJK} (2) \begin{CJK}{UTF8}{mj}求极限\end{CJK}
$$
\lim _{n \rightarrow+\infty} n\left(\frac{\pi}{4}-\sum_{i=1}^{n} \frac{n}{n^{2}+i^{2}}\right) \text {. }
$$
\begin{CJK}{UTF8}{mj}六\end{CJK}. $(15$ \begin{CJK}{UTF8}{mj}分\end{CJK} $)$ \begin{CJK}{UTF8}{mj}设\end{CJK} $y_{0}(x)=0, y_{n}(x)=\int_{0}^{x} 2 t\left[y_{n-1}(t)+1\right] \mathrm{d} t,(n=1,2, \cdots)$

(1)\begin{CJK}{UTF8}{mj}用归纳法证明\end{CJK}:
$$
y_{n}(x)=\sum_{k=1}^{n} \frac{x^{2 k}}{k !},(n=1,2, \cdots)
$$
(2) \begin{CJK}{UTF8}{mj}设\end{CJK} $y(x)=\lim _{n \rightarrow+\infty} y_{n}(x)$, \begin{CJK}{UTF8}{mj}证明\end{CJK}: $y(x)$ \begin{CJK}{UTF8}{mj}满足方程\end{CJK}
$$
\left\{\begin{array}{l}
y^{\prime}(x)=2 x(y(x)+1) \\
y(0)=0
\end{array}\right.
$$
\begin{CJK}{UTF8}{mj}七\end{CJK}. (10 \begin{CJK}{UTF8}{mj}分\end{CJK}) \begin{CJK}{UTF8}{mj}设\end{CJK} $\Sigma$ \begin{CJK}{UTF8}{mj}是光滑闭曲线\end{CJK}, \begin{CJK}{UTF8}{mj}取外侧\end{CJK}.
$$
I=\oiint_{\Sigma}\left(x^{3}-x\right) \mathrm{d} y \mathrm{~d} z+\left(y^{3}-y\right) \mathrm{d} z \mathrm{~d} x+\left(z^{3}-z\right) \mathrm{d} x \mathrm{~d} y
$$
(1) \begin{CJK}{UTF8}{mj}将上述积分\end{CJK} $I$ \begin{CJK}{UTF8}{mj}化为三重积分\end{CJK};

(2) \begin{CJK}{UTF8}{mj}求使得\end{CJK} $I$ \begin{CJK}{UTF8}{mj}达到最小值的曲面\end{CJK} $\Sigma$ \begin{CJK}{UTF8}{mj}的方程\end{CJK}, \begin{CJK}{UTF8}{mj}并求出最小值\end{CJK}.

\begin{CJK}{UTF8}{mj}八\end{CJK}. ( 10 \begin{CJK}{UTF8}{mj}分\end{CJK}) \begin{CJK}{UTF8}{mj}设\end{CJK} $\Omega$ \begin{CJK}{UTF8}{mj}是由\end{CJK} $\mathbb{R}^{3}$ \begin{CJK}{UTF8}{mj}中简单光滑闭曲线\end{CJK} $\Sigma$ \begin{CJK}{UTF8}{mj}所围的有界连通区域\end{CJK}.

(1) \begin{CJK}{UTF8}{mj}考察问题\end{CJK}
$$
(I) \begin{cases}\Delta u=g(x, y, z), & (x, y, z) \in \Omega \\ \left.\frac{\partial u}{\partial \vec{n}}\right|_{\Sigma}=f(x, y, z), & (x, y, z) \in \Sigma .\end{cases}
$$
\begin{CJK}{UTF8}{mj}其中\end{CJK} $f(x, y, z), g(x, y, z)$ \begin{CJK}{UTF8}{mj}为已知连续函数\end{CJK}, $\vec{n}$ \begin{CJK}{UTF8}{mj}为\end{CJK} $\Sigma$ \begin{CJK}{UTF8}{mj}的外法向\end{CJK}, $\Delta=\frac{\partial^{2}}{\partial x^{2}}+\frac{\partial^{2}}{\partial y^{2}}+\frac{\partial^{2}}{\partial z^{2}}, u(x, y, z)$ \begin{CJK}{UTF8}{mj}为\end{CJK} \begin{CJK}{UTF8}{mj}具有二阶连续偏导的末知函数\end{CJK}. \begin{CJK}{UTF8}{mj}证明\end{CJK}: \begin{CJK}{UTF8}{mj}问题\end{CJK} $(I)$ \begin{CJK}{UTF8}{mj}有解的必要条件是\end{CJK}:
$$
\iiint_{\Omega} g(x, y, z) \mathrm{d} x \mathrm{~d} y \mathrm{~d} z=\oiint_{\Sigma} f(x, y, z) \mathrm{d} s
$$
(2) \begin{CJK}{UTF8}{mj}考察问题\end{CJK}
$$
(I I) \begin{cases}\Delta u=0, & (x, y, z) \in \Omega \\ \left.u\right|_{\Sigma}=f(x, y, z), & (x, y, z) \in \Sigma .\end{cases}
$$
\begin{CJK}{UTF8}{mj}其中\end{CJK} $f(x, y, z)$ \begin{CJK}{UTF8}{mj}为已知连续函数\end{CJK}, $u(x, y, z)$ \begin{CJK}{UTF8}{mj}具有二阶连续偏导的末知函数\end{CJK}. \begin{CJK}{UTF8}{mj}证明\end{CJK}: \begin{CJK}{UTF8}{mj}问题\end{CJK} $(I I)$ \begin{CJK}{UTF8}{mj}有解\end{CJK}, \begin{CJK}{UTF8}{mj}则\end{CJK} \begin{CJK}{UTF8}{mj}其解是唯一的\end{CJK}, \begin{CJK}{UTF8}{mj}即若\end{CJK} $u(x, y, z), v(x, y, z)$ \begin{CJK}{UTF8}{mj}皆满足\end{CJK} $(I I)$, \begin{CJK}{UTF8}{mj}则有\end{CJK}:
$$
u(x, y, z) \equiv v(x, y, z)
$$

\section{7. 华中师范大学 2016 年研究生入学考试试题数学分析 
 李扬 
 微信公众号: sxkyliyang}
\begin{CJK}{UTF8}{mj}一\end{CJK}. \begin{CJK}{UTF8}{mj}计算题\end{CJK}(\begin{CJK}{UTF8}{mj}共\end{CJK} 5 \begin{CJK}{UTF8}{mj}小题\end{CJK}, \begin{CJK}{UTF8}{mj}每小题\end{CJK} 10 \begin{CJK}{UTF8}{mj}分\end{CJK}, \begin{CJK}{UTF8}{mj}共\end{CJK} 50 \begin{CJK}{UTF8}{mj}分\end{CJK})

(1)
$$
\lim _{x \rightarrow 0^{+}} \frac{1}{\sin \left(x^{3}\right)}\left[\left(\frac{2+\cos x}{3}\right)^{x}-1\right] .
$$
(2)
$$
\lim _{n \rightarrow \infty} \sqrt[n]{\left(1+\frac{1}{n}\right)^{2}\left(1+\frac{2}{n}\right)^{2} \cdots\left(1+\frac{n}{n}\right)^{2}}
$$
(3)
$$
\int_{0}^{1} \sin (\ln x) \frac{x^{b}-x^{a}}{\ln x} \mathrm{~d} x,(b>a>0) .
$$
(4) \begin{CJK}{UTF8}{mj}求幂级数\end{CJK}
$$
\sum_{n=0}^{+\infty}\left(n^{2}+3 n+4\right) x^{n}
$$
\begin{CJK}{UTF8}{mj}的和函数并指出其收敛域\end{CJK}.

(5) \begin{CJK}{UTF8}{mj}设函数\end{CJK} $f(x)$ \begin{CJK}{UTF8}{mj}满足\end{CJK}: $f(x)=f(x+2 \pi)$ \begin{CJK}{UTF8}{mj}且\end{CJK} $f(x)=|x|, x \in(-\pi, \pi]$, \begin{CJK}{UTF8}{mj}求\end{CJK} $f(x)$ \begin{CJK}{UTF8}{mj}的傅里叶级数展开式\end{CJK}.

\begin{CJK}{UTF8}{mj}二\end{CJK}. ( 15 \begin{CJK}{UTF8}{mj}分\end{CJK}) \begin{CJK}{UTF8}{mj}是否存在区间\end{CJK} $[0,2]$ \begin{CJK}{UTF8}{mj}上的连续可微函数\end{CJK} $f(x)$, \begin{CJK}{UTF8}{mj}满足\end{CJK}
$$
f(0)=f(2)=1,\left|f^{\prime}(x)\right| \leqslant 1,\left|\int_{0}^{1} f(x) \mathrm{d} x\right| \geqslant 3 ?
$$
\begin{CJK}{UTF8}{mj}请说明理由\end{CJK}.

\begin{CJK}{UTF8}{mj}三\end{CJK}. ( 20 \begin{CJK}{UTF8}{mj}分\end{CJK} $)$ \begin{CJK}{UTF8}{mj}设定义在\end{CJK} $[a, b]$ \begin{CJK}{UTF8}{mj}上的连续函数列\end{CJK} $\left\{\varphi_{n}\right\}$ \begin{CJK}{UTF8}{mj}满足关系\end{CJK}:
$$
\int_{a}^{b} \varphi_{n}(x) \varphi_{m}(x) \mathrm{d} x=\left\{\begin{array}{l}
0, n \neq m ; \\
1, n=m .
\end{array}\right.
$$
$f(x)$ \begin{CJK}{UTF8}{mj}为\end{CJK} $[a, b]$ \begin{CJK}{UTF8}{mj}上的可积函数\end{CJK}.

(1) \begin{CJK}{UTF8}{mj}记\end{CJK} $T_{n}(x)=\sum_{k=1}^{n} A_{k} \varphi_{k}(x)$, \begin{CJK}{UTF8}{mj}求使得积分\end{CJK} $\int_{a}^{b}\left[f(x)-T_{n}(x)\right]^{2} \mathrm{~d} x$ \begin{CJK}{UTF8}{mj}达到最小的系数\end{CJK} $A_{1}, A_{2}, \cdots, A_{n}$;

(2) \begin{CJK}{UTF8}{mj}记\end{CJK} $a_{n}=\int_{a}^{b} f(x) \cdot \varphi_{n}(x) \mathrm{d} x, n=1,2, \cdots$, \begin{CJK}{UTF8}{mj}证明\end{CJK}: $\sum_{n=1}^{+\infty} a_{n}^{2}$ \begin{CJK}{UTF8}{mj}收敛\end{CJK}, \begin{CJK}{UTF8}{mj}且有不等式\end{CJK}
$$
\sum_{n=1}^{+\infty} a_{n}^{2} \leqslant \int_{a}^{b}[f(x)]^{2} \mathrm{~d} x
$$
\begin{CJK}{UTF8}{mj}四\end{CJK}. ( 10 \begin{CJK}{UTF8}{mj}分\end{CJK}) \begin{CJK}{UTF8}{mj}已知函数\end{CJK} $f(x, y)$ \begin{CJK}{UTF8}{mj}具有二阶连续偏导数\end{CJK}, \begin{CJK}{UTF8}{mj}且\end{CJK} $f(1, y)=0, f(x, 1)=0, \iint_{D} f(x, y) \mathrm{d} x \mathrm{~d} y=A$, \begin{CJK}{UTF8}{mj}其中\end{CJK} $D=\{(x, y) \mid 0 \leqslant x \leqslant 1,0 \leqslant y \leqslant 1\}$, \begin{CJK}{UTF8}{mj}计算二重积分\end{CJK}
$$
I=\iint_{D} x y \frac{\partial^{2} f(x, y)}{\partial x \partial y} \mathrm{~d} x \mathrm{~d} y
$$
\begin{CJK}{UTF8}{mj}五\end{CJK}. (15 \begin{CJK}{UTF8}{mj}分\end{CJK}) \begin{CJK}{UTF8}{mj}设二元函数列\end{CJK} $\left\{f_{n}\right\}$ \begin{CJK}{UTF8}{mj}定义在平面区域\end{CJK} $\Omega$ \begin{CJK}{UTF8}{mj}上\end{CJK}, \begin{CJK}{UTF8}{mj}证明\end{CJK}: $\left\{f_{n}\right\}$ \begin{CJK}{UTF8}{mj}在\end{CJK} $\Omega$ \begin{CJK}{UTF8}{mj}上内闭一致收敛于\end{CJK} $f$ \begin{CJK}{UTF8}{mj}的充要条件是\end{CJK}: \begin{CJK}{UTF8}{mj}对于任意\end{CJK} $P_{0} \in \Omega$, \begin{CJK}{UTF8}{mj}存在\end{CJK} $P_{0}$ \begin{CJK}{UTF8}{mj}的一个邻域\end{CJK} $U\left(P_{0}\right)$, \begin{CJK}{UTF8}{mj}使得\end{CJK} $\left\{f_{n}\right\}$ \begin{CJK}{UTF8}{mj}在\end{CJK} $U\left(P_{0}\right) \cap \Omega$ \begin{CJK}{UTF8}{mj}上一致收敛于\end{CJK} $f$.

\begin{CJK}{UTF8}{mj}六\end{CJK}. (15 \begin{CJK}{UTF8}{mj}分\end{CJK}) \begin{CJK}{UTF8}{mj}设\end{CJK} $\Sigma$ \begin{CJK}{UTF8}{mj}是光滑闭曲线\end{CJK}, \begin{CJK}{UTF8}{mj}取外侧\end{CJK}.
$$
I=\oiint_{\Sigma}\left(x-x^{3}+\sin y\right) \mathrm{d} y \mathrm{~d} z+\left(y-2 y^{2}+\mathrm{e}^{x} z\right) \mathrm{d} z \mathrm{~d} x+\left(z-3 z^{3}+x^{2} y\right) \mathrm{d} x \mathrm{~d} y .
$$
\begin{CJK}{UTF8}{mj}求使得\end{CJK} $I$ \begin{CJK}{UTF8}{mj}达到最大值的曲面\end{CJK} $\Sigma$ \begin{CJK}{UTF8}{mj}的方程\end{CJK}, \begin{CJK}{UTF8}{mj}并求最大值\end{CJK}.

\begin{CJK}{UTF8}{mj}七\end{CJK}. ( 10 \begin{CJK}{UTF8}{mj}分\end{CJK}) \begin{CJK}{UTF8}{mj}设函数\end{CJK} $f(x)$ \begin{CJK}{UTF8}{mj}连续且恒大于零\end{CJK},
$$
\begin{array}{r}
F(t)=\frac{\iiint_{\Omega(t)} f\left(x^{2}+y^{2}+z^{2}\right) \mathrm{d} v}{\iint_{D(t)} f\left(x^{2}+y^{2}\right) \mathrm{d} \sigma}, \\
G(t)=\frac{\iint_{D(t)} f\left(x^{2}+y^{2}\right) \mathrm{d} \sigma}{\int_{-t}^{t} f\left(x^{2}\right) \mathrm{d} x},
\end{array}
$$
\begin{CJK}{UTF8}{mj}其中\end{CJK} $\Omega(t)=\left\{(x, y, z) \mid x^{2}+y^{2}+z^{2} \leqslant t^{2}\right\}, D(t)=\left\{(x, y) \mid x^{2}+y^{2} \leqslant t^{2}\right\}$.

(1) \begin{CJK}{UTF8}{mj}讨论\end{CJK} $F(t)$ \begin{CJK}{UTF8}{mj}在区间\end{CJK} $(0,+\infty)$ \begin{CJK}{UTF8}{mj}内的单调性\end{CJK}.

(2) \begin{CJK}{UTF8}{mj}证明\end{CJK}: \begin{CJK}{UTF8}{mj}当\end{CJK} $t>0, F(t)>\frac{2}{\pi} G(t)$.

\begin{CJK}{UTF8}{mj}八\end{CJK}. (15 \begin{CJK}{UTF8}{mj}分\end{CJK}) \begin{CJK}{UTF8}{mj}证明\end{CJK}: \begin{CJK}{UTF8}{mj}函数\end{CJK}
$$
\zeta(x)=\sum_{n=1}^{+\infty} \frac{1}{n^{x}}
$$
\begin{CJK}{UTF8}{mj}在\end{CJK} $(1,+\infty)$ \begin{CJK}{UTF8}{mj}上有连续的各阶导函数\end{CJK}.

\section{8. 华中师范大学 2017 年研究生入学考试试题数学分析 
 李扬 
 微信公众号: sxkyliyang}
\begin{CJK}{UTF8}{mj}一\end{CJK}. \begin{CJK}{UTF8}{mj}计算题\end{CJK}(\begin{CJK}{UTF8}{mj}共\end{CJK} 5 \begin{CJK}{UTF8}{mj}小题\end{CJK}, \begin{CJK}{UTF8}{mj}每小题\end{CJK} 10 \begin{CJK}{UTF8}{mj}分\end{CJK}, \begin{CJK}{UTF8}{mj}共\end{CJK} 50 \begin{CJK}{UTF8}{mj}分\end{CJK})

(1) \begin{CJK}{UTF8}{mj}设函数\end{CJK} $f(x)$ \begin{CJK}{UTF8}{mj}连续\end{CJK}, \begin{CJK}{UTF8}{mj}且满足\end{CJK} $\lim _{x \rightarrow 0} \frac{f(x)}{x}=1$, \begin{CJK}{UTF8}{mj}求极限\end{CJK}
$$
\lim _{x \rightarrow 0}\left[1+\int_{0}^{x} t f\left(x^{2}-t^{2}\right) \mathrm{d} t\right]^{\frac{1}{x^{x}}} .
$$
(2)
$$
\lim _{n \rightarrow \infty} \frac{2 n+\sqrt{n}-1}{n^{2}} \sum_{i=1}^{\infty} \sin ^{4} \frac{i \pi}{n} .
$$
(3) \begin{CJK}{UTF8}{mj}将函数\end{CJK}
$$
f(x)=\frac{x}{2+x-x^{2}}
$$
\begin{CJK}{UTF8}{mj}展开成关于\end{CJK} $(x-1)$ \begin{CJK}{UTF8}{mj}的幂级数\end{CJK}.

(4) \begin{CJK}{UTF8}{mj}将函数\end{CJK} $f(x)=2-x^{2}(0 \leqslant x \leqslant \pi)$ \begin{CJK}{UTF8}{mj}展开成余弦级数\end{CJK}, \begin{CJK}{UTF8}{mj}并求级数\end{CJK}
$$
\sum_{n=1}^{+\infty} \frac{(-1)^{n}}{n^{2}}
$$
\begin{CJK}{UTF8}{mj}的和\end{CJK}.

(5) \begin{CJK}{UTF8}{mj}计算曲面积分\end{CJK}
$$
I=\iint_{\Sigma} x z \mathrm{~d} y \mathrm{~d} z+2 z y \mathrm{~d} z \mathrm{~d} x+4 x y \mathrm{~d} x \mathrm{~d} y
$$
\begin{CJK}{UTF8}{mj}其中\end{CJK} $\Sigma$ \begin{CJK}{UTF8}{mj}为曲面\end{CJK} $z=1-x^{2}-\frac{y^{2}}{4}(0 \leqslant z \leqslant 1)$ \begin{CJK}{UTF8}{mj}的上侧\end{CJK}.

\begin{CJK}{UTF8}{mj}二\end{CJK}. ( 15 \begin{CJK}{UTF8}{mj}分\end{CJK}) \begin{CJK}{UTF8}{mj}设函数\end{CJK} $f, g$ \begin{CJK}{UTF8}{mj}在有限闭区间\end{CJK} $[a, b]$ \begin{CJK}{UTF8}{mj}上\end{CJK} Riemann \begin{CJK}{UTF8}{mj}可积\end{CJK}, \begin{CJK}{UTF8}{mj}证明\end{CJK}:

(1) $f \cdot g$ \begin{CJK}{UTF8}{mj}在\end{CJK} $[a, b]$ \begin{CJK}{UTF8}{mj}上\end{CJK} Riemann \begin{CJK}{UTF8}{mj}可积\end{CJK}.

(2) $\lim _{|x| \rightarrow 1} \sum_{i=1}^{n} f\left(\xi_{i}\right) g\left(\eta_{i}\right) \Delta x_{i}=\int_{a}^{b} f(x) g(x) \mathrm{d} x$. \begin{CJK}{UTF8}{mj}其中\end{CJK} $\xi_{i}, \eta_{i}$ \begin{CJK}{UTF8}{mj}是\end{CJK} $T$ \begin{CJK}{UTF8}{mj}所属小区间\end{CJK} $\Delta_{i}$ \begin{CJK}{UTF8}{mj}中的任意两点\end{CJK}, $i=$ $1,2, \cdots, n$

\begin{CJK}{UTF8}{mj}三\end{CJK}. ( 15 \begin{CJK}{UTF8}{mj}分\end{CJK}) \begin{CJK}{UTF8}{mj}设\end{CJK} $f(x)$ \begin{CJK}{UTF8}{mj}在\end{CJK} $[a, b]$ \begin{CJK}{UTF8}{mj}上连续\end{CJK}.

(1) \begin{CJK}{UTF8}{mj}证明\end{CJK}: \begin{CJK}{UTF8}{mj}若\end{CJK}
$$
\int_{a}^{b} f(x) \mathrm{d} x=0, \int_{a}^{b} x f(x) \mathrm{d} x=0
$$
\begin{CJK}{UTF8}{mj}则\end{CJK} $f$ \begin{CJK}{UTF8}{mj}在\end{CJK} $(a, b)$ \begin{CJK}{UTF8}{mj}内至少有两个零点\end{CJK}.

(2) \begin{CJK}{UTF8}{mj}证明\end{CJK}:\begin{CJK}{UTF8}{mj}若\end{CJK}
$$
\int_{a}^{b} x^{i} f(x) \mathrm{d} x=0, i=0,1,2, \cdots, n(n \geqslant 1) .
$$
\begin{CJK}{UTF8}{mj}则\end{CJK} $f$ \begin{CJK}{UTF8}{mj}在\end{CJK} $(a, b)$ \begin{CJK}{UTF8}{mj}内至少有\end{CJK} $n+1$ \begin{CJK}{UTF8}{mj}个零点\end{CJK}. \begin{CJK}{UTF8}{mj}四\end{CJK}. (15 \begin{CJK}{UTF8}{mj}分\end{CJK}) \begin{CJK}{UTF8}{mj}证明\end{CJK}: \begin{CJK}{UTF8}{mj}函数\end{CJK}
$$
f(x, y)= \begin{cases}\left(x^{2}+x y+y^{2}\right) \sin \frac{1}{\sqrt{x^{2}+y^{2}}}, & (x, y) \neq(0,0) \\ 0, & (x, y)=(0,0) .\end{cases}
$$
\begin{CJK}{UTF8}{mj}在\end{CJK} $(0,0)$ \begin{CJK}{UTF8}{mj}处连续且偏导数存在\end{CJK}, \begin{CJK}{UTF8}{mj}但偏导数在\end{CJK} $(0,0)$ \begin{CJK}{UTF8}{mj}处不连续\end{CJK}, \begin{CJK}{UTF8}{mj}而\end{CJK} $f(x, y)$ \begin{CJK}{UTF8}{mj}在\end{CJK} $(0,0)$ \begin{CJK}{UTF8}{mj}处可微\end{CJK}.

\begin{CJK}{UTF8}{mj}五\end{CJK}. ( 15 \begin{CJK}{UTF8}{mj}分\end{CJK}) \begin{CJK}{UTF8}{mj}设二元函数\end{CJK} $z(x, y)$ \begin{CJK}{UTF8}{mj}在上半平面\end{CJK} $D=\{(x, y) \mid y>0\}$ \begin{CJK}{UTF8}{mj}内具有二阶连续偏导数\end{CJK}, \begin{CJK}{UTF8}{mj}且满足方程\end{CJK}:
$$
\frac{\partial^{2} z}{\partial x^{2}}-y \frac{\partial^{2} z}{\partial y^{2}}-\frac{1}{2} \frac{\partial z}{\partial y}=0
$$
(1) \begin{CJK}{UTF8}{mj}变换\end{CJK}
$$
\left\{\begin{array}{l}
u=x+a \sqrt{y} \\
v=x+b \sqrt{y} .
\end{array} \quad(a \neq b)\right.
$$
\begin{CJK}{UTF8}{mj}将上述方程\end{CJK} $(*)$ \begin{CJK}{UTF8}{mj}变换为\end{CJK} $\frac{\partial^{2} z}{\partial u \partial v}=0$, \begin{CJK}{UTF8}{mj}求\end{CJK} $a, b$ \begin{CJK}{UTF8}{mj}的值\end{CJK};

(2) \begin{CJK}{UTF8}{mj}利用\end{CJK} (1) \begin{CJK}{UTF8}{mj}的结果求方程\end{CJK} $(*)$ \begin{CJK}{UTF8}{mj}的通解\end{CJK}.

\begin{CJK}{UTF8}{mj}六\end{CJK}. ( 10 \begin{CJK}{UTF8}{mj}分\end{CJK}) \begin{CJK}{UTF8}{mj}计算二重积分\end{CJK}:
$$
I=\iint_{D} \mathrm{e}^{\frac{y}{x+y}} \mathrm{~d} x \mathrm{~d} y
$$
\begin{CJK}{UTF8}{mj}其中\end{CJK} $D=\{(x, y) \mid x+y \leqslant 1, x \geqslant 0, y \geqslant 0\}$.

\begin{CJK}{UTF8}{mj}七\end{CJK}. ( 15 \begin{CJK}{UTF8}{mj}分\end{CJK}) \begin{CJK}{UTF8}{mj}设上半平面\end{CJK} $D=\{(x+y) \mid y>0\}$ \begin{CJK}{UTF8}{mj}内\end{CJK}, \begin{CJK}{UTF8}{mj}函数\end{CJK} $f(x, y)$ \begin{CJK}{UTF8}{mj}具有连续偏导数\end{CJK}, \begin{CJK}{UTF8}{mj}证明\end{CJK}: \begin{CJK}{UTF8}{mj}在\end{CJK} $D$ \begin{CJK}{UTF8}{mj}内曲线积分\end{CJK} $\int_{L} y f(x, y) \mathrm{d} x-x f(x, y) \mathrm{d} y$ \begin{CJK}{UTF8}{mj}与路径无关的充要条件是\end{CJK} $f(x, y)$ \begin{CJK}{UTF8}{mj}满足恒等式\end{CJK}
$$
f(t x, t y)=t^{-2} f(x, y),(t>0)
$$
$$
\frac{a_{0}^{2}}{2}+\sum_{n=1}^{+\infty}\left(a_{n}^{2}+b_{n}^{2}\right)=\frac{1}{\pi} \int_{-\pi}^{\pi} f^{2}(x) \mathrm{d} x
$$
$$
\frac{1}{\pi} \int_{-\pi}^{\pi} f(x) g(x) \mathrm{d} x=\frac{a_{0} \alpha_{0}}{2}+\sum_{n=1}^{\infty}\left(a_{n} \alpha_{n}+b_{n} \beta_{n}\right) .
$$
(2) \begin{CJK}{UTF8}{mj}利用\end{CJK} Parserval \begin{CJK}{UTF8}{mj}等式证明\end{CJK}: \begin{CJK}{UTF8}{mj}如果\end{CJK} $[-\pi, \pi]$ \begin{CJK}{UTF8}{mj}上的连续函数\end{CJK} $f$ \begin{CJK}{UTF8}{mj}和三角函数系\end{CJK}
$$
\{1, \cos x, \sin x, \cdots, \cos n x, \sin n x, \cdots\}
$$

\section{1. 兰州大学 2009 年研究生入学考试试题高等代数}
\begin{CJK}{UTF8}{mj}李扬\end{CJK}

\begin{CJK}{UTF8}{mj}微信公众号\end{CJK}: sxkyliyang

\begin{CJK}{UTF8}{mj}一\end{CJK}. (25 \begin{CJK}{UTF8}{mj}分\end{CJK}) \begin{CJK}{UTF8}{mj}证明以下问题\end{CJK}.

(1) \begin{CJK}{UTF8}{mj}设\end{CJK} $a_{1}, \cdots, a_{n}$ \begin{CJK}{UTF8}{mj}是互不相同的整数\end{CJK}. \begin{CJK}{UTF8}{mj}证明\end{CJK}: $f(x)=\left(x-a_{1}\right)^{2}+\left(x-a_{2}\right)^{2}+\cdots+\left(x-a_{n}\right)^{2}+1$ \begin{CJK}{UTF8}{mj}在有理数域\end{CJK} \begin{CJK}{UTF8}{mj}上不可约\end{CJK}.

(2) \begin{CJK}{UTF8}{mj}证明数域\end{CJK} $F$ \begin{CJK}{UTF8}{mj}上的\end{CJK} $n(n>0)$ \begin{CJK}{UTF8}{mj}次多项式\end{CJK} $f(x)$ \begin{CJK}{UTF8}{mj}能被它的微商\end{CJK} $f^{\prime}(x)$ \begin{CJK}{UTF8}{mj}整除的充要条件是\end{CJK} $f(x)=a(x-b)^{n}$, \begin{CJK}{UTF8}{mj}其\end{CJK} \begin{CJK}{UTF8}{mj}中\end{CJK} $a, b \in F$.

\begin{CJK}{UTF8}{mj}二\end{CJK}. ( 15 \begin{CJK}{UTF8}{mj}分\end{CJK}) \begin{CJK}{UTF8}{mj}计算下列\end{CJK} $n(n \geqslant 2)$ \begin{CJK}{UTF8}{mj}阶行列式的值\end{CJK}.

(1)
$$
D_{n}=\left|\begin{array}{ccccc}
x & a & a & \cdots & a \\
-a & x & a & \cdots & a \\
-a & -a & x & \cdots & a \\
\vdots & \vdots & \vdots & & \vdots \\
-a & -a & -a & \cdots & x
\end{array}\right| ;
$$
(2)
$$
D_{n}=\left|\begin{array}{ccccc}
1 & 3 & 3 & \cdots & 3 \\
3 & 2 & 3 & \cdots & 3 \\
3 & 3 & 3 & \cdots & 3 \\
\vdots & \vdots & \vdots & & \vdots \\
3 & 3 & 3 & \cdots & n
\end{array}\right| .
$$
\begin{CJK}{UTF8}{mj}三\end{CJK}. ( 15 \begin{CJK}{UTF8}{mj}分\end{CJK}) \begin{CJK}{UTF8}{mj}设\end{CJK} $A, B, C, D$ \begin{CJK}{UTF8}{mj}都是\end{CJK} $n$ \begin{CJK}{UTF8}{mj}级实矩阵\end{CJK}, \begin{CJK}{UTF8}{mj}且\end{CJK} $A C=C A$. \begin{CJK}{UTF8}{mj}证明\end{CJK}:
$$
\left|\begin{array}{ll}
A & B \\
C & D
\end{array}\right|=|A D-C B| .
$$
\begin{CJK}{UTF8}{mj}四\end{CJK}. ( 20 \begin{CJK}{UTF8}{mj}分\end{CJK}) \begin{CJK}{UTF8}{mj}证明\end{CJK}: $n$ \begin{CJK}{UTF8}{mj}级矩阵\end{CJK} $A$ \begin{CJK}{UTF8}{mj}为幂等矩阵\end{CJK} $\left(A^{2}=A\right)$ \begin{CJK}{UTF8}{mj}的充分必要条件是\end{CJK} $\mathrm{r}(A)+\mathrm{r}(E-A)=n$.

\begin{CJK}{UTF8}{mj}五\end{CJK}. ( 13 \begin{CJK}{UTF8}{mj}分\end{CJK}) \begin{CJK}{UTF8}{mj}设\end{CJK} $A$ \begin{CJK}{UTF8}{mj}是\end{CJK} $n$ \begin{CJK}{UTF8}{mj}级正交矩阵\end{CJK}, \begin{CJK}{UTF8}{mj}其特征值均为实数\end{CJK}. \begin{CJK}{UTF8}{mj}证明\end{CJK}: $A$ \begin{CJK}{UTF8}{mj}是对称矩阵\end{CJK}.

\begin{CJK}{UTF8}{mj}六\end{CJK}. (15 \begin{CJK}{UTF8}{mj}分\end{CJK}) \begin{CJK}{UTF8}{mj}设\end{CJK} $A, B$ \begin{CJK}{UTF8}{mj}都是\end{CJK} $n$ \begin{CJK}{UTF8}{mj}级正定矩阵\end{CJK}. \begin{CJK}{UTF8}{mj}证明\end{CJK}: $A^{-1}, A+B$ \begin{CJK}{UTF8}{mj}是正定矩阵\end{CJK}.

(1) $V=W \oplus \sigma^{-1}(0)$;

\begin{CJK}{UTF8}{mj}八\end{CJK}. ( 25 \begin{CJK}{UTF8}{mj}分\end{CJK}) \begin{CJK}{UTF8}{mj}已知矩阵\end{CJK}
$$
A=\left(\begin{array}{lll}
1 & 2 & 2 \\
2 & a & 2 \\
2 & 2 & 1
\end{array}\right), B=\left(\begin{array}{ccc}
-1 & 0 & 0 \\
0 & -1 & 0 \\
0 & 0 & b
\end{array}\right)
$$
\begin{CJK}{UTF8}{mj}问\end{CJK} $a, b$ \begin{CJK}{UTF8}{mj}取何值时\end{CJK}, $A$ \begin{CJK}{UTF8}{mj}与\end{CJK} $B$ \begin{CJK}{UTF8}{mj}相似\end{CJK}, \begin{CJK}{UTF8}{mj}并求出可逆矩阵\end{CJK} $P$ \begin{CJK}{UTF8}{mj}使得\end{CJK} $P^{-1} A P=B$.

\section{2. 兰州大学 2010 年研究生入学考试试题高等代数}
\begin{CJK}{UTF8}{mj}李扬\end{CJK}

\begin{CJK}{UTF8}{mj}微信公众号\end{CJK}: sxkyliyang

\begin{CJK}{UTF8}{mj}一\end{CJK}. (15 \begin{CJK}{UTF8}{mj}分\end{CJK}) \begin{CJK}{UTF8}{mj}证明以下问题\end{CJK}.

(1) \begin{CJK}{UTF8}{mj}设\end{CJK} $f(x), g(x)$ \begin{CJK}{UTF8}{mj}都是多项式\end{CJK}, \begin{CJK}{UTF8}{mj}且\end{CJK} $F(x)=\frac{f(x)}{(f(x), g(x))}$ \begin{CJK}{UTF8}{mj}和\end{CJK} $G(x)=\frac{g(x)}{(f(x), g(x))}$ \begin{CJK}{UTF8}{mj}的次数都大于零\end{CJK}. \begin{CJK}{UTF8}{mj}证明\end{CJK}: \begin{CJK}{UTF8}{mj}存在\end{CJK} \begin{CJK}{UTF8}{mj}唯一的多项式\end{CJK} $u(x), v(x)$ \begin{CJK}{UTF8}{mj}使\end{CJK}:
$$
u(x) F(x)+v(x) G(x)=(f(x), g(x)),
$$
\begin{CJK}{UTF8}{mj}并且\end{CJK} $\partial(u(x))<\partial(G(x)), \partial(v(x))<\partial(F(x))$.

(2) \begin{CJK}{UTF8}{mj}设\end{CJK} $f(x)=\left(x-a_{1}\right)\left(x-a_{2}\right) \cdots\left(x-a_{n}\right)-1$, \begin{CJK}{UTF8}{mj}其中\end{CJK} $a_{1}, a_{2}, \cdots, a_{n}$ \begin{CJK}{UTF8}{mj}是\end{CJK} $n$ \begin{CJK}{UTF8}{mj}个两两不等的整数\end{CJK}. \begin{CJK}{UTF8}{mj}证明\end{CJK}: $f(x)$ \begin{CJK}{UTF8}{mj}在\end{CJK} \begin{CJK}{UTF8}{mj}有理数域上不可约\end{CJK}.

\begin{CJK}{UTF8}{mj}二\end{CJK}. (16 \begin{CJK}{UTF8}{mj}分\end{CJK}) \begin{CJK}{UTF8}{mj}计算下列行列式的值\end{CJK}.

(1)
$$
D_{n}=\left|\begin{array}{cccc}
1+x_{1} & 1+x_{1}^{2} & \cdots & 1+x_{1}^{n} \\
1+x_{2} & 1+x_{2}^{2} & \cdots & 1+x_{2}^{n} \\
\vdots & \vdots & & \vdots \\
1+x_{n} & 1+x_{n}^{2} & \cdots & 1+x_{n}^{n}
\end{array}\right|
$$
(2)
$$
\text { 设 } A=\left(\begin{array}{cccc}
a_{11} & a_{12} & \cdots & a_{1 n} \\
a_{21} & a_{22} & \cdots & a_{2 n} \\
\vdots & \vdots & & \vdots \\
a_{n 1} & a_{n 2} & \cdots & a_{n n}
\end{array}\right) \text { 是实矩阵. 证明: 当 }\left|a_{i i}\right|>\left|\begin{array}{ccccc}
x & b & b & \cdots & b \\
a & x & b & \cdots & b \\
a & a & x & \cdots & b \\
\vdots & \vdots & \vdots & & \vdots \\
a & a & a & \cdots & x
\end{array}\right| \cdot
$$
\begin{CJK}{UTF8}{mj}四\end{CJK}. (12 \begin{CJK}{UTF8}{mj}分\end{CJK}) \begin{CJK}{UTF8}{mj}设\end{CJK} $A$ \begin{CJK}{UTF8}{mj}与\end{CJK} $B$ \begin{CJK}{UTF8}{mj}是数域上的\end{CJK} $n$ \begin{CJK}{UTF8}{mj}级矩阵且\end{CJK} $A B=B A$. \begin{CJK}{UTF8}{mj}证明\end{CJK}:
$$
\mathrm{r}(A+B) \leqslant \mathrm{r}(A)+\mathrm{r}(\mathrm{B})-\mathrm{r}(A B) .
$$
\begin{CJK}{UTF8}{mj}五\end{CJK}. (16 \begin{CJK}{UTF8}{mj}分\end{CJK}) \begin{CJK}{UTF8}{mj}设\end{CJK} $A, B$ \begin{CJK}{UTF8}{mj}是可交换的实对称矩阵\end{CJK}. \begin{CJK}{UTF8}{mj}证明\end{CJK}: \begin{CJK}{UTF8}{mj}存在正交矩阵\end{CJK} $T$ \begin{CJK}{UTF8}{mj}使得\end{CJK} $T^{-1} A T, T^{-1} B T$ \begin{CJK}{UTF8}{mj}都是对角矩阵\end{CJK}.

\begin{CJK}{UTF8}{mj}六\end{CJK}. ( 20 \begin{CJK}{UTF8}{mj}分\end{CJK}) \begin{CJK}{UTF8}{mj}设\end{CJK} $\sigma$ \begin{CJK}{UTF8}{mj}为\end{CJK} $n$ \begin{CJK}{UTF8}{mj}维线性空间\end{CJK} $V$ \begin{CJK}{UTF8}{mj}上的线性变换\end{CJK}, \begin{CJK}{UTF8}{mj}并且\end{CJK}
$$
V_{1}=\left\{x \in V \mid \text { 存在正整数 } m \text { 使得 } \sigma^{m} x=0\right\}, V_{2}=\bigcap_{i=1}^{\infty} \sigma^{i} V \text {. }
$$
\begin{CJK}{UTF8}{mj}证明\end{CJK}:

(1) $V_{1}, V_{2}$ \begin{CJK}{UTF8}{mj}都是\end{CJK} $\sigma$ \begin{CJK}{UTF8}{mj}的不变子空间\end{CJK}.

(2) $V=V_{1} \oplus V_{2}$. \begin{CJK}{UTF8}{mj}八\end{CJK}. ( 25 \begin{CJK}{UTF8}{mj}分\end{CJK}) \begin{CJK}{UTF8}{mj}已知\end{CJK} 3 \begin{CJK}{UTF8}{mj}维列向量\end{CJK} $(2,0,1)^{T}$ \begin{CJK}{UTF8}{mj}是\end{CJK} 3 \begin{CJK}{UTF8}{mj}级实对称矩阵\end{CJK} $A=\left(\begin{array}{ccc}2 & 2 & -2 \\ 2 & 5 & b \\ -2 & b & a\end{array}\right)$ \begin{CJK}{UTF8}{mj}的特征向量\end{CJK}.

(1) \begin{CJK}{UTF8}{mj}求\end{CJK} $a, b$ \begin{CJK}{UTF8}{mj}的值\end{CJK};

(2) \begin{CJK}{UTF8}{mj}求正交矩阵\end{CJK} $P$ \begin{CJK}{UTF8}{mj}使得\end{CJK} $P^{-1} A P$ \begin{CJK}{UTF8}{mj}为对角矩阵\end{CJK}, \begin{CJK}{UTF8}{mj}并给出这个对角矩阵\end{CJK}.

\section{3. 兰州大学 2011 年研究生入学考试试题高等代数}
\begin{CJK}{UTF8}{mj}李扬\end{CJK}

\begin{CJK}{UTF8}{mj}微信公众号\end{CJK}: sxkyliyang

\begin{CJK}{UTF8}{mj}一\end{CJK}.\begin{CJK}{UTF8}{mj}证明以下问题\end{CJK}.

(1) (15 \begin{CJK}{UTF8}{mj}分\end{CJK}) \begin{CJK}{UTF8}{mj}设\end{CJK} $p(x)$ \begin{CJK}{UTF8}{mj}是数域\end{CJK} $P$ \begin{CJK}{UTF8}{mj}上的次数大于零的多项式\end{CJK}. \begin{CJK}{UTF8}{mj}证明\end{CJK}: $p(x)$ \begin{CJK}{UTF8}{mj}是一个不可约多项式的充分必要条件\end{CJK} \begin{CJK}{UTF8}{mj}是对任意\end{CJK} $f(x), g(x) \in P[x]$, \begin{CJK}{UTF8}{mj}由\end{CJK} $p(x) \mid f(x) g(x)$ \begin{CJK}{UTF8}{mj}一定推出\end{CJK} $p(x) \mid f(x)$ \begin{CJK}{UTF8}{mj}或\end{CJK} $p(x) \mid g(x)$.

(2) (10 \begin{CJK}{UTF8}{mj}分\end{CJK}) \begin{CJK}{UTF8}{mj}设\end{CJK} $n$ \begin{CJK}{UTF8}{mj}是大于等于\end{CJK} 2 \begin{CJK}{UTF8}{mj}的正整数\end{CJK}. \begin{CJK}{UTF8}{mj}证明\end{CJK}: \begin{CJK}{UTF8}{mj}整系数多项式\end{CJK} $x^{n}+2$ \begin{CJK}{UTF8}{mj}不能分解为两个次数小于\end{CJK} $n$ \begin{CJK}{UTF8}{mj}的整系数\end{CJK} \begin{CJK}{UTF8}{mj}多项式的乘积\end{CJK}.

\begin{CJK}{UTF8}{mj}二\end{CJK}. (1) (10 \begin{CJK}{UTF8}{mj}分\end{CJK})\begin{CJK}{UTF8}{mj}设\end{CJK} $A_{i j}$ \begin{CJK}{UTF8}{mj}是\end{CJK} $n$ \begin{CJK}{UTF8}{mj}级矩阵\end{CJK} $A=\left(a_{i j}\right)_{n \times n}$ \begin{CJK}{UTF8}{mj}的元素\end{CJK} $a_{i j}$ \begin{CJK}{UTF8}{mj}的代数余子式\end{CJK}. \begin{CJK}{UTF8}{mj}证明\end{CJK}:
$$
\left|\begin{array}{ccccc}
a_{11} & a_{12} & \cdots & a_{1 n} & x_{1} \\
a_{21} & a_{22} & \cdots & a_{2 n} & x_{2} \\
\vdots & \vdots & & \vdots & \vdots \\
a_{n 1} & a_{n 2} & \cdots & a_{n n} & x_{n} \\
y_{1} & y_{2} & \cdots & y_{n} & z
\end{array}\right|=|A| z-\sum_{i, j=1}^{n} A_{i j} x_{i} y_{j} .
$$
(2) (8 \begin{CJK}{UTF8}{mj}分\end{CJK}) \begin{CJK}{UTF8}{mj}计算\end{CJK} $n$ \begin{CJK}{UTF8}{mj}级行列式\end{CJK}
$$
\left|\begin{array}{cccccc}
9 & 5 & 0 & \cdots & 0 & 0 \\
4 & 9 & 5 & \cdots & 0 & 0 \\
0 & 4 & 9 & \cdots & 0 & 0 \\
\vdots & \vdots & \vdots & & \vdots & \vdots \\
0 & 0 & 0 & \cdots & 9 & 5 \\
0 & 0 & 0 & \cdots & 4 & 9
\end{array}\right| \text { 的值. }
$$
\begin{CJK}{UTF8}{mj}三\end{CJK}. (13 \begin{CJK}{UTF8}{mj}分\end{CJK}) \begin{CJK}{UTF8}{mj}设矩阵\end{CJK} $A=\left(a_{i j}\right)_{m n}, B=\left(b_{i j}\right)_{n s}$ \begin{CJK}{UTF8}{mj}分别为数域\end{CJK} $P$ \begin{CJK}{UTF8}{mj}上的\end{CJK} $m \times n, n \times s$ \begin{CJK}{UTF8}{mj}矩阵\end{CJK}. \begin{CJK}{UTF8}{mj}证明\end{CJK}:
$$
\operatorname{rank}(A B) \geqslant \operatorname{rank}(A)+\operatorname{rank}(B)-n .
$$
\begin{CJK}{UTF8}{mj}四\end{CJK}. (1) ( 20 \begin{CJK}{UTF8}{mj}分\end{CJK}) \begin{CJK}{UTF8}{mj}设\end{CJK} $A$ \begin{CJK}{UTF8}{mj}是\end{CJK} $n$ \begin{CJK}{UTF8}{mj}级实对称矩阵\end{CJK}, \begin{CJK}{UTF8}{mj}证明\end{CJK}: $A$ \begin{CJK}{UTF8}{mj}是半正定的充分必要条件是对任意的实数\end{CJK} $k>0, k E+A$ \begin{CJK}{UTF8}{mj}是\end{CJK} \begin{CJK}{UTF8}{mj}正定矩阵\end{CJK}.

(2) \begin{CJK}{UTF8}{mj}设\end{CJK} $A$ \begin{CJK}{UTF8}{mj}是\end{CJK} $n$ \begin{CJK}{UTF8}{mj}级正交矩阵且\end{CJK} $|A|=-1$. \begin{CJK}{UTF8}{mj}证明\end{CJK}: $|E+A|=0$.

\begin{CJK}{UTF8}{mj}五\end{CJK}. $f(x)$ \begin{CJK}{UTF8}{mj}是数域\end{CJK} $P$ \begin{CJK}{UTF8}{mj}上的\end{CJK} $m$ \begin{CJK}{UTF8}{mj}维线性空间\end{CJK} $U$ \begin{CJK}{UTF8}{mj}到\end{CJK} $n$ \begin{CJK}{UTF8}{mj}维线性空间\end{CJK} $V$ \begin{CJK}{UTF8}{mj}的线性映射\end{CJK}(\begin{CJK}{UTF8}{mj}即保持线性运算的映射\end{CJK}). \begin{CJK}{UTF8}{mj}证明\end{CJK}:
$$
\operatorname{dim}(f(U))=m-\operatorname{dim}\left(f^{-1}(0)\right) .
$$
\begin{CJK}{UTF8}{mj}其中\end{CJK}, $f(U)=\{f(\alpha) \mid \alpha \in V\}, f^{-1}(0)=\{\alpha \in U \mid f(\alpha)=0\}$.

\begin{CJK}{UTF8}{mj}六\end{CJK}. (20 \begin{CJK}{UTF8}{mj}分\end{CJK}) \begin{CJK}{UTF8}{mj}设\end{CJK} $A$ \begin{CJK}{UTF8}{mj}是数域\end{CJK} $P$ \begin{CJK}{UTF8}{mj}上的\end{CJK} $n$ \begin{CJK}{UTF8}{mj}级矩阵\end{CJK}, \begin{CJK}{UTF8}{mj}其特征多项式\end{CJK} $f(\lambda)$ \begin{CJK}{UTF8}{mj}可分解为一次因式的乘积\end{CJK}
$$
f(\lambda)=\left(\lambda-\lambda_{1}\right)^{\gamma_{1}}\left(\lambda-\lambda_{2}\right)^{\gamma_{2}} \cdots\left(\lambda-\lambda_{s}\right)^{\gamma_{s}} .
$$
\begin{CJK}{UTF8}{mj}证明\end{CJK}: $P^{n}=V_{1} \oplus V_{2} \oplus \cdots \oplus V_{s}$, \begin{CJK}{UTF8}{mj}其中\end{CJK} $V_{i}=\left\{\alpha \in P^{n} \mid\left(A-\lambda_{i} E\right)^{r_{i}} \alpha=0\right\}, i=1,2, \cdots, s$.

\begin{CJK}{UTF8}{mj}七\end{CJK}. (15 \begin{CJK}{UTF8}{mj}分\end{CJK}) \begin{CJK}{UTF8}{mj}设\end{CJK} $\alpha_{1}, \alpha_{2}, \cdots, \alpha_{m}$ \begin{CJK}{UTF8}{mj}是\end{CJK} $n$ \begin{CJK}{UTF8}{mj}维欧式空间\end{CJK} $V$ \begin{CJK}{UTF8}{mj}中的一组向量\end{CJK},
$$
\Delta=\left(\begin{array}{cccc}
\left(\alpha_{1}, \alpha_{1}\right) & \left(\alpha_{1}, \alpha_{2}\right) & \cdots & \left(\alpha_{1}, \alpha_{m}\right) \\
\left(\alpha_{2}, \alpha_{1}\right) & \left(\alpha_{2}, \alpha_{2}\right) & \cdots & \left(\alpha_{2}, \alpha_{m}\right) \\
\vdots & \vdots & & \vdots \\
\left(\alpha_{m}, \alpha_{1}\right) & \left(\alpha_{m}, \alpha_{2}\right) & \cdots & \left(\alpha_{m}, \alpha_{m}\right)
\end{array}\right)
$$
\begin{CJK}{UTF8}{mj}证明\end{CJK}: $\alpha_{1}, \alpha_{2}, \cdots, \alpha_{m}$ \begin{CJK}{UTF8}{mj}线性无关当且仅当\end{CJK} $|\Delta| \neq 0$. \begin{CJK}{UTF8}{mj}八\end{CJK}. ( 25 \begin{CJK}{UTF8}{mj}分\end{CJK}) \begin{CJK}{UTF8}{mj}设\end{CJK} 1 \begin{CJK}{UTF8}{mj}是对称矩阵\end{CJK} $A=\left(\begin{array}{ccc}2 & 2 & -2 \\ 2 & 5 & -4 \\ -2 & -4 & a\end{array}\right)$ \begin{CJK}{UTF8}{mj}的二重特征根\end{CJK}, \begin{CJK}{UTF8}{mj}求\end{CJK} $a$ \begin{CJK}{UTF8}{mj}的值\end{CJK}, \begin{CJK}{UTF8}{mj}并求正交矩阵\end{CJK} $T$ \begin{CJK}{UTF8}{mj}使得\end{CJK} $T^{T} A T$ \begin{CJK}{UTF8}{mj}为\end{CJK} \begin{CJK}{UTF8}{mj}对角矩阵\end{CJK}.

\section{4. 兰州大学 2012 年研究生入学考试试题高等代数}
\begin{CJK}{UTF8}{mj}李扬\end{CJK}

\begin{CJK}{UTF8}{mj}微信公众号\end{CJK}: sxkyliyang

\begin{CJK}{UTF8}{mj}一\end{CJK}. \begin{CJK}{UTF8}{mj}证明以下问题\end{CJK}.

(1) (15 \begin{CJK}{UTF8}{mj}分\end{CJK}) \begin{CJK}{UTF8}{mj}设\end{CJK} $f(x)$ \begin{CJK}{UTF8}{mj}是次数大于零的首项系数为\end{CJK} 1 \begin{CJK}{UTF8}{mj}的多项式\end{CJK}. \begin{CJK}{UTF8}{mj}证明\end{CJK}: $f(x)$ \begin{CJK}{UTF8}{mj}是一个不可约多项式的方幂的\end{CJK} \begin{CJK}{UTF8}{mj}充分必要条件为\end{CJK}: \begin{CJK}{UTF8}{mj}对任意多项式\end{CJK} $g(x), h(x)$, \begin{CJK}{UTF8}{mj}由\end{CJK} $f(x) \mid g(x) h(x)$ \begin{CJK}{UTF8}{mj}可以推出\end{CJK} $f(x) \mid g(x)$, \begin{CJK}{UTF8}{mj}或者对某一正整数\end{CJK} $m, f(x) \mid h^{m}(x)$.

(2)(10 \begin{CJK}{UTF8}{mj}分\end{CJK}) \begin{CJK}{UTF8}{mj}设\end{CJK} $f(x), g(x)$ \begin{CJK}{UTF8}{mj}是数域\end{CJK} $P$ \begin{CJK}{UTF8}{mj}上的两个多项式\end{CJK}. \begin{CJK}{UTF8}{mj}证明\end{CJK}: $g^{2}(x) \mid f^{2}(x)$ \begin{CJK}{UTF8}{mj}当且仅当\end{CJK} $g(x) \mid f(x)$.

\begin{CJK}{UTF8}{mj}二\end{CJK}. \begin{CJK}{UTF8}{mj}计算下列行列式的值\end{CJK}

(1) (8 \begin{CJK}{UTF8}{mj}分\end{CJK})
$$
\left|\begin{array}{ccccc}
a_{1} & b & b & \cdots & b \\
b & a_{2} & b & \cdots & b \\
b & b & a_{3} & \cdots & b \\
\vdots & \vdots & \vdots & & \vdots \\
b & b & b & \cdots & a_{n}
\end{array}\right|
$$
(2) (10 \begin{CJK}{UTF8}{mj}分\end{CJK})
$$
\left|\begin{array}{cccc}
\left(a_{1}+b_{1}\right)^{-1} & \left(a_{1}+b_{2}\right)^{-1} & \cdots & \left(a_{1}+b_{n}\right)^{-1} \\
\left(a_{2}+b_{1}\right)^{-1} & \left(a_{2}+b_{2}\right)^{-1} & \cdots & \left(a_{2}+b_{n}\right)^{-1} \\
\vdots & \vdots & & \vdots \\
\left(a_{n}+b_{1}\right)^{-1} & \left(a_{n}+b_{2}\right)^{-1} & \cdots & \left(a_{n}+b_{n}\right)^{-1}
\end{array}\right| .
$$
\begin{CJK}{UTF8}{mj}三\end{CJK}. ( 20 \begin{CJK}{UTF8}{mj}分\end{CJK}) \begin{CJK}{UTF8}{mj}设\end{CJK} $\alpha_{1}, \alpha_{2}, \cdots, \alpha_{r}$ \begin{CJK}{UTF8}{mj}与\end{CJK} $\beta_{1}, \beta_{2}, \cdots, \beta_{s}$ \begin{CJK}{UTF8}{mj}是两个向量组\end{CJK}. \begin{CJK}{UTF8}{mj}证明\end{CJK}: \begin{CJK}{UTF8}{mj}如果\end{CJK}

(1)\begin{CJK}{UTF8}{mj}向量组\end{CJK} $\alpha_{1}, \alpha_{2}, \cdots, \alpha_{r}$ \begin{CJK}{UTF8}{mj}可以由\end{CJK} $\beta_{1}, \beta_{2}, \cdots, \beta_{s}$ \begin{CJK}{UTF8}{mj}线性表示\end{CJK},

(2) $r>s$,

\begin{CJK}{UTF8}{mj}那么向量组\end{CJK} $\alpha_{1}, \alpha_{2}, \cdots, \alpha_{r}$ \begin{CJK}{UTF8}{mj}必线性相关\end{CJK}.

\begin{CJK}{UTF8}{mj}四\end{CJK}. (15 \begin{CJK}{UTF8}{mj}分\end{CJK}) \begin{CJK}{UTF8}{mj}设\end{CJK} $A, B$ \begin{CJK}{UTF8}{mj}是两个正定矩阵\end{CJK}, $A^{2}=B^{2}$. \begin{CJK}{UTF8}{mj}证明\end{CJK}: $A$ \begin{CJK}{UTF8}{mj}相似于\end{CJK} $B$.

\begin{CJK}{UTF8}{mj}五\end{CJK}. (16 \begin{CJK}{UTF8}{mj}分\end{CJK}) \begin{CJK}{UTF8}{mj}设\end{CJK} $P^{n}$ \begin{CJK}{UTF8}{mj}是数域\end{CJK} $P$ \begin{CJK}{UTF8}{mj}上所有\end{CJK} $n$ \begin{CJK}{UTF8}{mj}维列向量作成的线性空间\end{CJK}, $U, V$ \begin{CJK}{UTF8}{mj}是\end{CJK} $P^{n}$ \begin{CJK}{UTF8}{mj}的真子空间\end{CJK}, $P^{n}=U \oplus V$. \begin{CJK}{UTF8}{mj}证明存\end{CJK} \begin{CJK}{UTF8}{mj}在唯一的幂等变换\end{CJK} $f\left(\right.$ \begin{CJK}{UTF8}{mj}即\end{CJK} $\left.f^{2}=f\right)$, \begin{CJK}{UTF8}{mj}使得\end{CJK} $f\left(P^{n}\right)=U, f^{-1}(0)=V$.

\begin{CJK}{UTF8}{mj}六\end{CJK}. (16 \begin{CJK}{UTF8}{mj}分\end{CJK}) \begin{CJK}{UTF8}{mj}设\end{CJK} $V$ \begin{CJK}{UTF8}{mj}是数域\end{CJK} $P$ \begin{CJK}{UTF8}{mj}上的\end{CJK} $n$ \begin{CJK}{UTF8}{mj}维线性空间\end{CJK}, $f$ \begin{CJK}{UTF8}{mj}是\end{CJK} $V$ \begin{CJK}{UTF8}{mj}上的线性变换\end{CJK}. \begin{CJK}{UTF8}{mj}证明\end{CJK}: $f$ \begin{CJK}{UTF8}{mj}在某一组基下的矩阵是对角矩阵\end{CJK} \begin{CJK}{UTF8}{mj}的充分必要条件是\end{CJK} $f$ \begin{CJK}{UTF8}{mj}的特征多项式的根都属于\end{CJK} $P$, \begin{CJK}{UTF8}{mj}并且对\end{CJK} $f$ \begin{CJK}{UTF8}{mj}的每一个特征值\end{CJK} $\lambda$, \begin{CJK}{UTF8}{mj}对应的特征子空间\end{CJK} $V_{\lambda}$ \begin{CJK}{UTF8}{mj}的维数\end{CJK} \begin{CJK}{UTF8}{mj}等于\end{CJK} $\lambda$ \begin{CJK}{UTF8}{mj}的代数重数\end{CJK} (\begin{CJK}{UTF8}{mj}即作为特征值的重数\end{CJK}).

\begin{CJK}{UTF8}{mj}七\end{CJK}. ( 15 \begin{CJK}{UTF8}{mj}分\end{CJK}) \begin{CJK}{UTF8}{mj}设\end{CJK} $A$ \begin{CJK}{UTF8}{mj}是\end{CJK} $n$ \begin{CJK}{UTF8}{mj}级实矩阵\end{CJK}. \begin{CJK}{UTF8}{mj}证明\end{CJK}: \begin{CJK}{UTF8}{mj}存在\end{CJK} $n$ \begin{CJK}{UTF8}{mj}级正交矩阵\end{CJK} $T_{1}$ \begin{CJK}{UTF8}{mj}和\end{CJK} $T_{2}$ \begin{CJK}{UTF8}{mj}使得\end{CJK}
$$
T_{1} A T_{2}=\left(\begin{array}{cccc}
\gamma_{1} & 0 & \cdots & 0 \\
0 & \gamma_{2} & \cdots & 0 \\
\vdots & \vdots & & \vdots \\
0 & 0 & \cdots & \gamma_{n}
\end{array}\right),
$$
\begin{CJK}{UTF8}{mj}其中\end{CJK} $\gamma_{1}^{2}, \gamma_{2}^{2}, \cdots, \gamma_{n}^{2}$ \begin{CJK}{UTF8}{mj}是\end{CJK} $A^{T} A$ \begin{CJK}{UTF8}{mj}的特征值\end{CJK}. \begin{CJK}{UTF8}{mj}八\end{CJK}. $\left(25\right.$ \begin{CJK}{UTF8}{mj}分\end{CJK}) \begin{CJK}{UTF8}{mj}设\end{CJK} $(2,0,1)^{T}$ \begin{CJK}{UTF8}{mj}是实对称矩阵\end{CJK}
$$
A=\left(\begin{array}{ccc}
1 & -2 & 2 \\
-2 & -2 & a \\
2 & a & b
\end{array}\right)
$$
\begin{CJK}{UTF8}{mj}的特征向量\end{CJK}. \begin{CJK}{UTF8}{mj}求\end{CJK} $a, b$ \begin{CJK}{UTF8}{mj}的值\end{CJK}, \begin{CJK}{UTF8}{mj}并求正交矩阵\end{CJK} $T$ \begin{CJK}{UTF8}{mj}使得\end{CJK} $T^{T} A T$ \begin{CJK}{UTF8}{mj}为对角矩阵\end{CJK}.

\section{5. 兰州大学 2013 年研究生入学考试试题高等代数}
\begin{CJK}{UTF8}{mj}李扬\end{CJK}

\begin{CJK}{UTF8}{mj}微信公众号\end{CJK}: sxkyliyang

\begin{CJK}{UTF8}{mj}一\end{CJK}. (25 \begin{CJK}{UTF8}{mj}分\end{CJK}) \begin{CJK}{UTF8}{mj}设\end{CJK} $\alpha$ \begin{CJK}{UTF8}{mj}是数域\end{CJK} $P$ \begin{CJK}{UTF8}{mj}上一个正次数多项式的一个复根\end{CJK}. \begin{CJK}{UTF8}{mj}令\end{CJK}
$$
I=\{f(x) \in P[x] \mid f(\alpha)=0\} .
$$
\begin{CJK}{UTF8}{mj}证明\end{CJK}:

(1) I \begin{CJK}{UTF8}{mj}中存在正次数首\end{CJK} 1 \begin{CJK}{UTF8}{mj}多项式\end{CJK} $p(x)$ \begin{CJK}{UTF8}{mj}使得对任意\end{CJK} $f(x) \in I$, \begin{CJK}{UTF8}{mj}有\end{CJK} $p(x) \mid f(x)$, \begin{CJK}{UTF8}{mj}进而有\end{CJK} $I=p(x) P[x]$;

(2) $p(x)$ \begin{CJK}{UTF8}{mj}是不可约多项式\end{CJK};

(3) \begin{CJK}{UTF8}{mj}进一步\end{CJK}, \begin{CJK}{UTF8}{mj}若\end{CJK} $p(-\alpha)=0=p(\alpha)$, \begin{CJK}{UTF8}{mj}则对任意复数\end{CJK} $\beta, p(\beta)=0$ \begin{CJK}{UTF8}{mj}当且仅当\end{CJK} $p(-\beta)=0$.

\begin{CJK}{UTF8}{mj}二\end{CJK}. \begin{CJK}{UTF8}{mj}计算下列\end{CJK} $n$ \begin{CJK}{UTF8}{mj}级行列式的值\end{CJK}
$$
\begin{aligned}
&\text { (1) ( } 8 \text { 分) }\left|\begin{array}{ccccc}
2 n & n & \cdots & 0 & 0 \\
n & 2 n & \cdots & 0 & 0 \\
\vdots & \vdots & & \vdots & \vdots \\
0 & 0 & \cdots & 2 n & n \\
0 & 0 & \cdots & n & 2 n
\end{array}\right| ; \\
&\text { (2) ( } 10 \text { 分) }\left|\begin{array}{ccccc} 
& 0 & a_{1}+a_{2} & \cdots & a_{1}+a_{n} \\
a_{2}+a_{1} & 0 & \cdots & a_{2}+a_{n} \\
\vdots & & \vdots & & \vdots \\
a_{n}+a_{1} & a_{n}+a_{2} & \cdots & 0
\end{array}\right| \text {, 其中 } a_{1} a_{2} \cdots a_{n} \neq 0 .
\end{aligned}
$$
\begin{CJK}{UTF8}{mj}三\end{CJK}. ( 15 \begin{CJK}{UTF8}{mj}分\end{CJK}) \begin{CJK}{UTF8}{mj}设\end{CJK} $A, B$ \begin{CJK}{UTF8}{mj}分别是数域\end{CJK} $P$ \begin{CJK}{UTF8}{mj}上的\end{CJK} $m \times n, n \times m$ \begin{CJK}{UTF8}{mj}矩阵\end{CJK}. \begin{CJK}{UTF8}{mj}证明\end{CJK}: $E_{m}-A B$ \begin{CJK}{UTF8}{mj}可逆当且仅当\end{CJK} $E_{n}-B A$ \begin{CJK}{UTF8}{mj}可逆\end{CJK}.

\begin{CJK}{UTF8}{mj}四\end{CJK}. ( 20 \begin{CJK}{UTF8}{mj}分\end{CJK}) \begin{CJK}{UTF8}{mj}设\end{CJK} $V$ \begin{CJK}{UTF8}{mj}是数域\end{CJK} $P$ \begin{CJK}{UTF8}{mj}上一有限维线性空间\end{CJK}, $\sigma$ \begin{CJK}{UTF8}{mj}是\end{CJK} $V$ \begin{CJK}{UTF8}{mj}上一线性变换\end{CJK}. \begin{CJK}{UTF8}{mj}显然\end{CJK}
$$
\operatorname{Ker} \sigma=\{\alpha \in V \mid \sigma(\alpha)=0\} \text { 和 } \operatorname{Im} \sigma=\{\sigma(\alpha) \mid \alpha \in V\}
$$
\begin{CJK}{UTF8}{mj}都是\end{CJK} $V$ \begin{CJK}{UTF8}{mj}的子空间\end{CJK}. \begin{CJK}{UTF8}{mj}证明\end{CJK}: $\operatorname{dim}(\operatorname{Ker} \sigma)+\operatorname{dim}(\operatorname{Im} \sigma)=\operatorname{dim}(V)$; \begin{CJK}{UTF8}{mj}并问是否有\end{CJK} $\operatorname{Ker} \sigma \cap \operatorname{Im} \sigma=\{0\}$ ?

\begin{CJK}{UTF8}{mj}五\end{CJK}. (16 \begin{CJK}{UTF8}{mj}分\end{CJK}) \begin{CJK}{UTF8}{mj}设\end{CJK} $f\left(x_{1}, x_{2}, \cdots, x_{n}\right)$ \begin{CJK}{UTF8}{mj}是一秩为\end{CJK} $n$ \begin{CJK}{UTF8}{mj}的二次型\end{CJK}, \begin{CJK}{UTF8}{mj}符号差为\end{CJK} $s$. \begin{CJK}{UTF8}{mj}证明\end{CJK}: \begin{CJK}{UTF8}{mj}存在\end{CJK} $\mathbb{R}^{n}$ \begin{CJK}{UTF8}{mj}的一个\end{CJK} $\frac{1}{2}(n-|s|)$ \begin{CJK}{UTF8}{mj}维子空间\end{CJK} $V_{1}$ \begin{CJK}{UTF8}{mj}使得对任一\end{CJK} $\left(x_{1}, x_{2}, \cdots, x_{n}\right)^{T} \in V_{1}$, \begin{CJK}{UTF8}{mj}都有\end{CJK} $f\left(x_{1}, x_{2}, \cdots, x_{n}\right)=0$.

\begin{CJK}{UTF8}{mj}六\end{CJK}. (16 \begin{CJK}{UTF8}{mj}分\end{CJK}) \begin{CJK}{UTF8}{mj}设\end{CJK} $A$ \begin{CJK}{UTF8}{mj}是\end{CJK} $n$ \begin{CJK}{UTF8}{mj}级实矩阵\end{CJK}. \begin{CJK}{UTF8}{mj}证明\end{CJK}: $A$ \begin{CJK}{UTF8}{mj}正交相似于一个上三角矩阵的充分必要条件是\end{CJK} $A$ \begin{CJK}{UTF8}{mj}的特征多项式在复数\end{CJK} \begin{CJK}{UTF8}{mj}域中的根都是实数\end{CJK}.

\begin{CJK}{UTF8}{mj}七\end{CJK}. ( 15 \begin{CJK}{UTF8}{mj}分\end{CJK}) \begin{CJK}{UTF8}{mj}设\end{CJK} $P$ \begin{CJK}{UTF8}{mj}是一个数域\end{CJK}, $V \in P^{n \times n}$ \begin{CJK}{UTF8}{mj}是\end{CJK} $P$ \begin{CJK}{UTF8}{mj}上所有\end{CJK} $n$ \begin{CJK}{UTF8}{mj}级矩阵构成的\end{CJK} $P$ \begin{CJK}{UTF8}{mj}上的线性空间\end{CJK}, $f$ \begin{CJK}{UTF8}{mj}是\end{CJK} $V$ \begin{CJK}{UTF8}{mj}上的线性变换\end{CJK}. \begin{CJK}{UTF8}{mj}证明\end{CJK}: \begin{CJK}{UTF8}{mj}若\end{CJK} $f$ \begin{CJK}{UTF8}{mj}保持矩阵的乘法运算\end{CJK}, \begin{CJK}{UTF8}{mj}即\end{CJK}: \begin{CJK}{UTF8}{mj}对任意\end{CJK} $A, B \in V, f(A B)=f(A) f(B)$, \begin{CJK}{UTF8}{mj}则存在\end{CJK} $n$ \begin{CJK}{UTF8}{mj}级可逆矩阵\end{CJK} $Q$ \begin{CJK}{UTF8}{mj}使得对\end{CJK}
$$
A=\left(\begin{array}{ccc}
2 & -2 & 2 \\
-2 & -1 & a \\
2 & a & b
\end{array}\right)
$$

\section{6. 兰州大学 2014 年研究生入学考试试题高等代数}
\begin{CJK}{UTF8}{mj}李扬\end{CJK}

\begin{CJK}{UTF8}{mj}微信公众号\end{CJK}: sxkyliyang

\begin{CJK}{UTF8}{mj}一\end{CJK}. \begin{CJK}{UTF8}{mj}证明以下问题\end{CJK}.

(1) ( 10 \begin{CJK}{UTF8}{mj}分\end{CJK}) \begin{CJK}{UTF8}{mj}设\end{CJK} $f(x), g(x)$ \begin{CJK}{UTF8}{mj}为复数域上两个首项系数为\end{CJK} 1 \begin{CJK}{UTF8}{mj}的互异\end{CJK} 3 \begin{CJK}{UTF8}{mj}次多项式\end{CJK}. \begin{CJK}{UTF8}{mj}且\end{CJK}
$$
x^{4}+x^{2}+1 \mid f\left(x^{3}\right)+x^{4} g\left(x^{3}\right) .
$$
\begin{CJK}{UTF8}{mj}证明\end{CJK} $(f(x), g(x))=(x+1)(x-1)$.

(2) (13 \begin{CJK}{UTF8}{mj}分\end{CJK}) \begin{CJK}{UTF8}{mj}设\end{CJK} $f(x)$ \begin{CJK}{UTF8}{mj}是一个实系数多项式\end{CJK}. \begin{CJK}{UTF8}{mj}证明\end{CJK}: $f(x)$ \begin{CJK}{UTF8}{mj}根全是实根的充要条件是\end{CJK} $f^{2}(x)$ \begin{CJK}{UTF8}{mj}不能表示成两个次数\end{CJK} \begin{CJK}{UTF8}{mj}不同的实系数多项式的平方和\end{CJK}.

\begin{CJK}{UTF8}{mj}二\end{CJK}. \begin{CJK}{UTF8}{mj}计算行列式的值\end{CJK}, \begin{CJK}{UTF8}{mj}其中\end{CJK} $a \neq b$.

(1) (8 \begin{CJK}{UTF8}{mj}分\end{CJK})
$$
\left|\begin{array}{ccccc}
1 & 2 & 2 & \cdots & 2 \\
2 & 2 & 2 & \cdots & 2 \\
2 & 2 & 3 & \cdots & 2 \\
\vdots & \vdots & \vdots & & \vdots \\
2 & 2 & 2 & \cdots & n
\end{array}\right|
$$
(2) (10 \begin{CJK}{UTF8}{mj}分\end{CJK})
$$
\left|\begin{array}{ccccc}
x_{1} & a & a & \cdots & a \\
b & x_{2} & a & \cdots & a \\
b & b & x_{3} & \cdots & a \\
\vdots & \vdots & \vdots & & \vdots \\
b & b & b & \cdots & x_{n}
\end{array}\right| \text {. }
$$
\begin{CJK}{UTF8}{mj}三\end{CJK}. ( 20 \begin{CJK}{UTF8}{mj}分\end{CJK}) \begin{CJK}{UTF8}{mj}设\end{CJK} $n$ \begin{CJK}{UTF8}{mj}元齐次线性方程组\end{CJK} $A x=0$ \begin{CJK}{UTF8}{mj}有非零解\end{CJK}. \begin{CJK}{UTF8}{mj}证明\end{CJK}: \begin{CJK}{UTF8}{mj}它有基础解系\end{CJK}, \begin{CJK}{UTF8}{mj}并且基础解系含解的个数等于\end{CJK} $n-r$, \begin{CJK}{UTF8}{mj}其中\end{CJK} $r$ \begin{CJK}{UTF8}{mj}表示系数矩阵\end{CJK} $A$ \begin{CJK}{UTF8}{mj}的秩\end{CJK}.

\begin{CJK}{UTF8}{mj}四\end{CJK}. (1) (10 \begin{CJK}{UTF8}{mj}分\end{CJK}) \begin{CJK}{UTF8}{mj}设\end{CJK} $A$ \begin{CJK}{UTF8}{mj}是数域\end{CJK} $P$ \begin{CJK}{UTF8}{mj}上的一个\end{CJK} $n$ \begin{CJK}{UTF8}{mj}级矩阵\end{CJK}. \begin{CJK}{UTF8}{mj}证明\end{CJK}: \begin{CJK}{UTF8}{mj}存在数域\end{CJK} $P$ \begin{CJK}{UTF8}{mj}上的\end{CJK} $n$ \begin{CJK}{UTF8}{mj}级矩阵\end{CJK} $B$ \begin{CJK}{UTF8}{mj}使得\end{CJK} $A B A=A, B A B=$ $B$.

(2) (10 \begin{CJK}{UTF8}{mj}分\end{CJK}) \begin{CJK}{UTF8}{mj}设\end{CJK} $A$ \begin{CJK}{UTF8}{mj}为\end{CJK} $n$ \begin{CJK}{UTF8}{mj}级实对称矩阵\end{CJK}, $B$ \begin{CJK}{UTF8}{mj}为\end{CJK} $n$ \begin{CJK}{UTF8}{mj}级实矩阵\end{CJK}, \begin{CJK}{UTF8}{mj}且\end{CJK} $B A+A B^{T}$ \begin{CJK}{UTF8}{mj}的特征值全大于\end{CJK} 0 . \begin{CJK}{UTF8}{mj}证明\end{CJK}: $A$ \begin{CJK}{UTF8}{mj}可逆\end{CJK}. \begin{CJK}{UTF8}{mj}五\end{CJK}. ( 20 \begin{CJK}{UTF8}{mj}分\end{CJK}) \begin{CJK}{UTF8}{mj}设\end{CJK} $A$ \begin{CJK}{UTF8}{mj}为\end{CJK} $n$ \begin{CJK}{UTF8}{mj}级幂等矩阵\end{CJK}, \begin{CJK}{UTF8}{mj}即存在正整数\end{CJK} $m$ \begin{CJK}{UTF8}{mj}使得\end{CJK} $A^{m}=0$. \begin{CJK}{UTF8}{mj}证明\end{CJK}:

(2) \begin{CJK}{UTF8}{mj}若\end{CJK} $A$ \begin{CJK}{UTF8}{mj}的秩等于\end{CJK} $r$, \begin{CJK}{UTF8}{mj}则\end{CJK} $A^{r+1}=0$.

\begin{CJK}{UTF8}{mj}六\end{CJK}. (12 \begin{CJK}{UTF8}{mj}分\end{CJK}) \begin{CJK}{UTF8}{mj}设\end{CJK} $V$ \begin{CJK}{UTF8}{mj}是数域\end{CJK} $P$ \begin{CJK}{UTF8}{mj}上的\end{CJK} $n$ \begin{CJK}{UTF8}{mj}维线性空间\end{CJK}, $\sigma$ \begin{CJK}{UTF8}{mj}是\end{CJK} $V$ \begin{CJK}{UTF8}{mj}的线性变换\end{CJK}, $\lambda_{1}, \lambda_{2}, \cdots, \lambda_{k}$ \begin{CJK}{UTF8}{mj}是\end{CJK} $\sigma$ \begin{CJK}{UTF8}{mj}的互不相同的特征值\end{CJK},
$$
W=\left(W \cap V_{1}\right) \oplus\left(W \cap V_{2}\right) \oplus \cdots \oplus\left(W \cap V_{k}\right) .
$$
\begin{CJK}{UTF8}{mj}七\end{CJK}. (12 \begin{CJK}{UTF8}{mj}分\end{CJK}) \begin{CJK}{UTF8}{mj}设\end{CJK} $\sigma$ \begin{CJK}{UTF8}{mj}为\end{CJK} $n$ \begin{CJK}{UTF8}{mj}维欧几里得空间\end{CJK} $V$ \begin{CJK}{UTF8}{mj}上的一个对称变换\end{CJK}, $\sigma$ \begin{CJK}{UTF8}{mj}的特征多项式为\end{CJK}
$$
f(x)=\left(x-\lambda_{1}\right)^{r_{1}}\left(x-\lambda_{2}\right)^{r_{2}} \cdots\left(x-\lambda_{m}\right)^{r_{m}},
$$
$$
\operatorname{Ker}\left(\sigma-\lambda_{i} \varepsilon\right)=\operatorname{Ker}\left(\sigma-\lambda_{i} \varepsilon\right)^{r_{i}}
$$
\begin{CJK}{UTF8}{mj}其中\end{CJK} $\varepsilon$ \begin{CJK}{UTF8}{mj}表示\end{CJK} $V$ \begin{CJK}{UTF8}{mj}上的恒等变换\end{CJK}. \begin{CJK}{UTF8}{mj}八\end{CJK}. ( 25 \begin{CJK}{UTF8}{mj}分\end{CJK}) \begin{CJK}{UTF8}{mj}设实二次型\end{CJK}
$$
f\left(x_{1}, x_{2}, x_{3}\right)=x_{1}^{2}+x_{2}^{2}+x_{3}^{2}+2 a x_{1} x_{2}+4 x_{1} x_{3}+2 b x_{2} x_{3}, a>0
$$
\begin{CJK}{UTF8}{mj}通过正交变换化为标准形\end{CJK} $-y_{1}^{2}-y_{2}^{2}+5 y_{3}^{2}$, \begin{CJK}{UTF8}{mj}求参数\end{CJK} $a, b$ \begin{CJK}{UTF8}{mj}及所用的正交变换\end{CJK}.

\section{7. 兰州大学 2015 年研究生入学考试试题高等代数}
\begin{CJK}{UTF8}{mj}李扬\end{CJK}

\begin{CJK}{UTF8}{mj}微信公众号\end{CJK}: sxkyliyang

\begin{CJK}{UTF8}{mj}一\end{CJK}. \begin{CJK}{UTF8}{mj}证明以下问题\end{CJK}.

(1) (10 \begin{CJK}{UTF8}{mj}分\end{CJK}) \begin{CJK}{UTF8}{mj}设\end{CJK} $P$ \begin{CJK}{UTF8}{mj}是一个数域\end{CJK}, $m$ \begin{CJK}{UTF8}{mj}是任一正整数\end{CJK}. \begin{CJK}{UTF8}{mj}证明\end{CJK}: \begin{CJK}{UTF8}{mj}如果在\end{CJK} $P[x]$ \begin{CJK}{UTF8}{mj}中\end{CJK}, $x-a \mid f\left(x^{m}\right)$, \begin{CJK}{UTF8}{mj}那么\end{CJK} $x^{m}-a^{m} \mid f\left(x^{m}\right)$.

(2) (10 \begin{CJK}{UTF8}{mj}分\end{CJK}) \begin{CJK}{UTF8}{mj}证明\end{CJK}: \begin{CJK}{UTF8}{mj}有理系数多项式\end{CJK} $f(x)$ \begin{CJK}{UTF8}{mj}在有理数域上不可约的充分必要条件是\end{CJK}, \begin{CJK}{UTF8}{mj}对任意有理数\end{CJK} $a \neq 0$ \begin{CJK}{UTF8}{mj}和\end{CJK} $b$, \begin{CJK}{UTF8}{mj}多项式\end{CJK} $g(x)=f(a x+b)$ \begin{CJK}{UTF8}{mj}在有理数域上不可约\end{CJK}.

\begin{CJK}{UTF8}{mj}二\end{CJK}. \begin{CJK}{UTF8}{mj}计算下列行列式的值\end{CJK}.

(1) ( 10 \begin{CJK}{UTF8}{mj}分\end{CJK})
$$
D_{n}=\left|\begin{array}{cccc}
1+a_{1} & 1+a_{1}^{2} & \cdots & 1+a_{1}^{n} \\
1+a_{2} & 1+a_{2}^{2} & \cdots & 1+a_{2}^{n} \\
\vdots & \vdots & & \vdots \\
1+a_{n} & 1+a_{n}^{2} & \cdots & 1+a_{n}^{n}
\end{array}\right|
$$
(2) (10 \begin{CJK}{UTF8}{mj}分\end{CJK})
$$
D_{n+1}=\left|\begin{array}{cccccc}
b & a_{1} & a_{2} & a_{3} & \cdots & a_{n} \\
a_{1} & b & a_{2} & a_{3} & \cdots & a_{n} \\
a_{1} & a_{2} & b & a_{3} & \cdots & a_{n} \\
a_{1} & a_{2} & a_{3} & b & \cdots & a_{n} \\
\vdots & \vdots & \vdots & \vdots & & \vdots \\
a_{1} & a_{2} & a_{3} & a_{4} & \cdots & b
\end{array}\right| .
$$
\begin{CJK}{UTF8}{mj}三\end{CJK}. ( 20 \begin{CJK}{UTF8}{mj}分\end{CJK}) \begin{CJK}{UTF8}{mj}设\end{CJK} $A, B, C, D$ \begin{CJK}{UTF8}{mj}都是\end{CJK} $n$ \begin{CJK}{UTF8}{mj}级矩阵\end{CJK}, \begin{CJK}{UTF8}{mj}且\end{CJK} $A C=C A$, \begin{CJK}{UTF8}{mj}证明\end{CJK}:
$$
\left|\begin{array}{ll}
A & B \\
C & D
\end{array}\right|=|A D-C B| .
$$
\begin{CJK}{UTF8}{mj}四\end{CJK}. ( 20 \begin{CJK}{UTF8}{mj}分\end{CJK}) \begin{CJK}{UTF8}{mj}证明\end{CJK}: \begin{CJK}{UTF8}{mj}一个矩阵\end{CJK} $A$ \begin{CJK}{UTF8}{mj}的秩等于\end{CJK} $r$ \begin{CJK}{UTF8}{mj}的充分必要条件是\end{CJK} $A$ \begin{CJK}{UTF8}{mj}有一个\end{CJK} $r$ \begin{CJK}{UTF8}{mj}级子式不为零\end{CJK}, \begin{CJK}{UTF8}{mj}并且所有\end{CJK}(\begin{CJK}{UTF8}{mj}如果存在的\end{CJK} \begin{CJK}{UTF8}{mj}话\end{CJK})\begin{CJK}{UTF8}{mj}的\end{CJK} $r+1$ \begin{CJK}{UTF8}{mj}级子式全为零\end{CJK}.

\begin{CJK}{UTF8}{mj}五\end{CJK}. ( 15 \begin{CJK}{UTF8}{mj}分\end{CJK}) \begin{CJK}{UTF8}{mj}设\end{CJK} $A$ \begin{CJK}{UTF8}{mj}是实对称正定矩阵\end{CJK}. \begin{CJK}{UTF8}{mj}证明\end{CJK}: $A$ \begin{CJK}{UTF8}{mj}可唯一分解为\end{CJK} $A=B^{2}$, \begin{CJK}{UTF8}{mj}其中\end{CJK} $B$ \begin{CJK}{UTF8}{mj}正定\end{CJK}.

\begin{CJK}{UTF8}{mj}六\end{CJK}. (15 \begin{CJK}{UTF8}{mj}分\end{CJK}) \begin{CJK}{UTF8}{mj}设\end{CJK} $V$ \begin{CJK}{UTF8}{mj}是复数域上的\end{CJK} $n$ \begin{CJK}{UTF8}{mj}维线性空间\end{CJK}, $\sigma, \tau$ \begin{CJK}{UTF8}{mj}是\end{CJK} $V$ \begin{CJK}{UTF8}{mj}的线性变换\end{CJK}, \begin{CJK}{UTF8}{mj}且\end{CJK} $\sigma \tau=\tau \sigma$. \begin{CJK}{UTF8}{mj}证明\end{CJK}:

(1) \begin{CJK}{UTF8}{mj}如果\end{CJK} $\lambda$ \begin{CJK}{UTF8}{mj}是\end{CJK} $\sigma$ \begin{CJK}{UTF8}{mj}的特征值\end{CJK}, \begin{CJK}{UTF8}{mj}那么\end{CJK} $\sigma$ \begin{CJK}{UTF8}{mj}的特征子空间\end{CJK} $V_{\lambda}$ \begin{CJK}{UTF8}{mj}是\end{CJK} $\tau$ \begin{CJK}{UTF8}{mj}的不变子空间\end{CJK};

(2) $\sigma, \tau$ \begin{CJK}{UTF8}{mj}至少有一个公共的特征向量\end{CJK}.

\begin{CJK}{UTF8}{mj}七\end{CJK}. (15 \begin{CJK}{UTF8}{mj}分\end{CJK}) \begin{CJK}{UTF8}{mj}设\end{CJK} $\sigma$ \begin{CJK}{UTF8}{mj}是\end{CJK} $n$ \begin{CJK}{UTF8}{mj}维线性空间\end{CJK} $V$ \begin{CJK}{UTF8}{mj}上的线性变换\end{CJK}. \begin{CJK}{UTF8}{mj}证明\end{CJK}: $r\left(\sigma^{2}\right)=r(\sigma)$ \begin{CJK}{UTF8}{mj}的充分必要条件为\end{CJK}
$$
V=\sigma(V) \oplus \sigma^{-1}(0) .
$$
\begin{CJK}{UTF8}{mj}八\end{CJK}. ( 25 \begin{CJK}{UTF8}{mj}分\end{CJK}) \begin{CJK}{UTF8}{mj}设实二次型\end{CJK}
$$
f\left(x_{1}, x_{2}, x_{3}\right)=x_{1}^{2}+x_{2}^{2}+x_{3}^{2}+2 a x_{1} x_{2}+4 x_{1} x_{3}+4 x_{2} x_{3}
$$
\begin{CJK}{UTF8}{mj}通过正交变换化为标准形\end{CJK} $5 y_{1}^{2}-y_{2}^{2}-y_{3}^{2}$, \begin{CJK}{UTF8}{mj}求参数\end{CJK} $a$ \begin{CJK}{UTF8}{mj}及所用的正交变换\end{CJK}.

\section{8. 兰州大学 2016 年研究生入学考试试题高等代数}
\begin{CJK}{UTF8}{mj}李扬\end{CJK}

\begin{CJK}{UTF8}{mj}微信公众号\end{CJK}: sxkyliyang

\begin{CJK}{UTF8}{mj}一\end{CJK}. \begin{CJK}{UTF8}{mj}证明以下问题\end{CJK}.

(1) (10 \begin{CJK}{UTF8}{mj}分\end{CJK}) \begin{CJK}{UTF8}{mj}设\end{CJK} $f_{1}(x)=a f(x)+b g(x), g_{1}(x)=c f(x)+d g(x)$, \begin{CJK}{UTF8}{mj}且\end{CJK} $\left|\begin{array}{ll}a & b \\ c & d\end{array}\right| \neq 0$. \begin{CJK}{UTF8}{mj}证明\end{CJK}:
$$
(f(x), g(x))=\left(f_{1}(x), g_{1}(x)\right) .
$$
(2) (12 \begin{CJK}{UTF8}{mj}分\end{CJK}) \begin{CJK}{UTF8}{mj}设\end{CJK} $a_{1}, \cdots, a_{n}$ \begin{CJK}{UTF8}{mj}为互不相同的整数\end{CJK}, $g(x)=\left(x-a_{1}\right) \cdots\left(x-a_{n}\right)-1$. \begin{CJK}{UTF8}{mj}证明\end{CJK}: $g(x)$ \begin{CJK}{UTF8}{mj}在有理数域\end{CJK} $\mathbb{Q}$ \begin{CJK}{UTF8}{mj}上不可约\end{CJK}.

\begin{CJK}{UTF8}{mj}二\end{CJK}. \begin{CJK}{UTF8}{mj}计算下列\end{CJK} $n$ \begin{CJK}{UTF8}{mj}级行列式的值\end{CJK}:

(1) (8 \begin{CJK}{UTF8}{mj}分\end{CJK})
$$
\left|\begin{array}{ccccc}
1 & 2 & 3 & \cdots & n \\
a & 1 & 2 & \cdots & n-1 \\
a & a & 1 & \cdots & n-2 \\
\vdots & \vdots & \vdots & & \vdots \\
a & a & a & \cdots & 1
\end{array}\right|
$$
(2) (10 \begin{CJK}{UTF8}{mj}分\end{CJK})
$$
\left|\begin{array}{ccccc}
2 & -1 & 0 & \cdots & 0 \\
-1 & 2 & -1 & \cdots & 0 \\
0 & -1 & 2 & \cdots & 0 \\
\vdots & \vdots & \vdots & & \vdots \\
0 & 0 & 0 & \cdots & 2
\end{array}\right| .
$$
\begin{CJK}{UTF8}{mj}三\end{CJK}. ( 20 \begin{CJK}{UTF8}{mj}分\end{CJK}) \begin{CJK}{UTF8}{mj}设\end{CJK} $\alpha_{1}, \alpha_{2}, \cdots, \alpha_{n}$ \begin{CJK}{UTF8}{mj}是\end{CJK} $n$ \begin{CJK}{UTF8}{mj}维线性空间\end{CJK} $V$ \begin{CJK}{UTF8}{mj}的一组基\end{CJK}, $A$ \begin{CJK}{UTF8}{mj}是\end{CJK} $n \times s$ \begin{CJK}{UTF8}{mj}矩阵\end{CJK},
$$
\left(\beta_{1}, \beta_{2}, \cdots, \beta_{s}\right)=\left(\alpha_{1}, \alpha_{2}, \cdots, \alpha_{n}\right) A .
$$
\begin{CJK}{UTF8}{mj}证明\end{CJK}: $L\left(\beta_{1}, \beta_{2}, \cdots, \beta_{s}\right)$ \begin{CJK}{UTF8}{mj}的维数等于\end{CJK} $A$ \begin{CJK}{UTF8}{mj}的秩\end{CJK}.

\begin{CJK}{UTF8}{mj}四\end{CJK}. (15 \begin{CJK}{UTF8}{mj}分\end{CJK}) \begin{CJK}{UTF8}{mj}设\end{CJK} $A, B$ \begin{CJK}{UTF8}{mj}都是\end{CJK} $n$ \begin{CJK}{UTF8}{mj}阶非零实矩阵\end{CJK}, \begin{CJK}{UTF8}{mj}且\end{CJK} $A$ \begin{CJK}{UTF8}{mj}为正定矩阵\end{CJK}, $B$ \begin{CJK}{UTF8}{mj}为半正定矩阵\end{CJK}. \begin{CJK}{UTF8}{mj}证明\end{CJK}:

(1) $\left|E_{n}+B\right|>1$

(2) \begin{CJK}{UTF8}{mj}若\end{CJK} $B$ \begin{CJK}{UTF8}{mj}和\end{CJK} $A-B$ \begin{CJK}{UTF8}{mj}也是正定矩阵\end{CJK}, \begin{CJK}{UTF8}{mj}则\end{CJK} $B^{-1}-A^{-1}$ \begin{CJK}{UTF8}{mj}也是正定阵\end{CJK}.

\begin{CJK}{UTF8}{mj}五\end{CJK}. ( 15 \begin{CJK}{UTF8}{mj}分\end{CJK}) \begin{CJK}{UTF8}{mj}设\end{CJK} $A, B$ \begin{CJK}{UTF8}{mj}都是\end{CJK} $n$ \begin{CJK}{UTF8}{mj}阶复矩阵\end{CJK}. \begin{CJK}{UTF8}{mj}证明\end{CJK}: \begin{CJK}{UTF8}{mj}如果\end{CJK} $A$ \begin{CJK}{UTF8}{mj}与\end{CJK} $B$ \begin{CJK}{UTF8}{mj}乘积可交换\end{CJK}, \begin{CJK}{UTF8}{mj}那么存在\end{CJK} $n$ \begin{CJK}{UTF8}{mj}阶可逆复矩阵\end{CJK} $P$ \begin{CJK}{UTF8}{mj}使得\end{CJK} $P^{-1} A P$ \begin{CJK}{UTF8}{mj}和\end{CJK} $P^{-1} B P$ \begin{CJK}{UTF8}{mj}都是上三角矩阵\end{CJK}.

\begin{CJK}{UTF8}{mj}六\end{CJK}. ( 20 \begin{CJK}{UTF8}{mj}分\end{CJK}) \begin{CJK}{UTF8}{mj}设\end{CJK} $V$ \begin{CJK}{UTF8}{mj}是数域\end{CJK} $P$ \begin{CJK}{UTF8}{mj}上的有限维线性空间\end{CJK}, $\sigma$ \begin{CJK}{UTF8}{mj}是\end{CJK} $V$ \begin{CJK}{UTF8}{mj}上的一个线性变换\end{CJK}. \begin{CJK}{UTF8}{mj}若在\end{CJK} $P[x]$ \begin{CJK}{UTF8}{mj}中\end{CJK} $f(x)=f_{1}(x) f_{2}(x)$ \begin{CJK}{UTF8}{mj}且\end{CJK} $\left(f_{1}(x), f_{2}(x)\right)=1$, \begin{CJK}{UTF8}{mj}则\end{CJK}
$$
(f(\sigma))^{-1}(0)=\left(f_{1}(\sigma)\right)^{-1}(0) \oplus\left(f_{2}(\sigma)\right)^{-1}(0)
$$
(2) $\sigma$ \begin{CJK}{UTF8}{mj}保持向量的长度不变\end{CJK}, \begin{CJK}{UTF8}{mj}即对于任意\end{CJK} $\alpha \in V,|\sigma(\alpha)|=|\alpha|$; (3) \begin{CJK}{UTF8}{mj}若\end{CJK} $\varepsilon_{1}, \varepsilon_{2}, \cdots, \varepsilon_{n}$ \begin{CJK}{UTF8}{mj}是标准正交基\end{CJK}, \begin{CJK}{UTF8}{mj}则\end{CJK} $\sigma\left(\varepsilon_{1}\right), \sigma\left(\varepsilon_{2}\right), \cdots, \sigma\left(\varepsilon_{n}\right)$ \begin{CJK}{UTF8}{mj}也是标准正交基\end{CJK};

(4) $\sigma$ \begin{CJK}{UTF8}{mj}在任一组标准正交基下的矩阵是正交矩阵\end{CJK}.

\begin{CJK}{UTF8}{mj}八\end{CJK}. $(20$ \begin{CJK}{UTF8}{mj}分\end{CJK} $)$ \begin{CJK}{UTF8}{mj}设\end{CJK} $\xi=(1,1,2)^{T}$ \begin{CJK}{UTF8}{mj}是实对称矩阵\end{CJK}
$$
A=\left(\begin{array}{lll}
a & b & 2 \\
b & 0 & 2 \\
2 & 2 & 3
\end{array}\right)
$$
\begin{CJK}{UTF8}{mj}的一个特征向量\end{CJK}.

(1) \begin{CJK}{UTF8}{mj}求\end{CJK} $a, b$ \begin{CJK}{UTF8}{mj}的值\end{CJK}.

(2) \begin{CJK}{UTF8}{mj}求正交矩阵\end{CJK} $T$ \begin{CJK}{UTF8}{mj}使得\end{CJK} $T^{-1} A T$ \begin{CJK}{UTF8}{mj}为对角矩阵\end{CJK}.

\section{9. 兰州大学 2017 年研究生入学考试试题高等代数}
\begin{CJK}{UTF8}{mj}李扬\end{CJK}

\begin{CJK}{UTF8}{mj}微信公众号\end{CJK}: sxkyliyang

\begin{CJK}{UTF8}{mj}一\end{CJK}. \begin{CJK}{UTF8}{mj}证明以下问题\end{CJK}.

(1) ( 12 \begin{CJK}{UTF8}{mj}分\end{CJK}) \begin{CJK}{UTF8}{mj}设\end{CJK} $f(x), g(x)$ \begin{CJK}{UTF8}{mj}是数域\end{CJK} $P$ \begin{CJK}{UTF8}{mj}上的两个不全为零的多项式\end{CJK}. \begin{CJK}{UTF8}{mj}证明集合\end{CJK}
$$
M=\{u(x) f(x)+v(x) g(x) \mid u(x), v(x) \in P[x]\}
$$
\begin{CJK}{UTF8}{mj}中存在次数最小的首项系数为\end{CJK} 1 \begin{CJK}{UTF8}{mj}的多项式\end{CJK} $d(x)$, \begin{CJK}{UTF8}{mj}并且\end{CJK} $d(x)$ \begin{CJK}{UTF8}{mj}是\end{CJK} $f(x)$ \begin{CJK}{UTF8}{mj}与\end{CJK} $g(x)$ \begin{CJK}{UTF8}{mj}的最大公因式\end{CJK}.

(2) ( 8 \begin{CJK}{UTF8}{mj}分\end{CJK}) \begin{CJK}{UTF8}{mj}设\end{CJK} $m \geqslant 2, p_{1}, p_{2}, \cdots, p_{r}$ \begin{CJK}{UTF8}{mj}是两两不同的素数\end{CJK}. \begin{CJK}{UTF8}{mj}证明\end{CJK}: $\sqrt[m]{p_{1} p_{2} \cdots p_{r}}$ \begin{CJK}{UTF8}{mj}不是有理数\end{CJK}.

\begin{CJK}{UTF8}{mj}二\end{CJK}. \begin{CJK}{UTF8}{mj}计算下列\end{CJK} $n$ \begin{CJK}{UTF8}{mj}级行列式的值\end{CJK}:

(1) (8 \begin{CJK}{UTF8}{mj}分\end{CJK})
$$
\left|\begin{array}{ccccc}
1 & 2 & 3 & \cdots & n \\
2 & 3 & 4 & \cdots & 1 \\
3 & 4 & 5 & \cdots & 2 \\
\vdots & \vdots & \vdots & & \vdots \\
n & 1 & 2 & \cdots & n-1
\end{array}\right|
$$
(2) (10 \begin{CJK}{UTF8}{mj}分\end{CJK})
$$
\left|\begin{array}{ccccc}
\alpha+\beta & \alpha \beta & 0 & \cdots & 0 \\
1 & \alpha+\beta & \alpha \beta & \cdots & 0 \\
0 & 1 & \alpha+\beta & \cdots & 0 \\
\vdots & \vdots & \vdots & & \vdots \\
0 & 0 & 0 & \cdots & \alpha \beta \\
0 & 0 & 0 & \cdots & \alpha+\beta
\end{array}\right|
$$
\begin{CJK}{UTF8}{mj}三\end{CJK}. $\left(20\right.$ \begin{CJK}{UTF8}{mj}分\end{CJK}) \begin{CJK}{UTF8}{mj}设\end{CJK} $A$ \begin{CJK}{UTF8}{mj}是一个\end{CJK} $n$ \begin{CJK}{UTF8}{mj}级实矩阵\end{CJK}. \begin{CJK}{UTF8}{mj}证明\end{CJK}: $A$ \begin{CJK}{UTF8}{mj}满足\end{CJK} $A^{2}-2 A-3 E=0$ \begin{CJK}{UTF8}{mj}的充要条件是\end{CJK} $\mathrm{r}(A+E)+\mathrm{r}(A-3 E)=n$.
$$
f\left(x_{1}, \cdots, x_{n}\right)=a \sum_{i=1}^{n} x_{i}^{2}+2 b \sum_{i<j}^{n} x_{i} x_{j}
$$
\begin{CJK}{UTF8}{mj}五\end{CJK}. ( 17 \begin{CJK}{UTF8}{mj}分\end{CJK}) \begin{CJK}{UTF8}{mj}设\end{CJK} $V_{1}, V_{2}$ \begin{CJK}{UTF8}{mj}是数域\end{CJK} $P$ \begin{CJK}{UTF8}{mj}上线性空间\end{CJK} $V$ \begin{CJK}{UTF8}{mj}的两个有限维子空间\end{CJK}. \begin{CJK}{UTF8}{mj}证明\end{CJK}:
$$
\operatorname{dim}\left(V_{1}\right)+\operatorname{dim}\left(V_{2}\right)=\operatorname{dim}\left(V_{1}+V_{2}\right)+\operatorname{dim}\left(V_{1} \cap V_{2}\right) .
$$
\begin{CJK}{UTF8}{mj}六\end{CJK}. ( 15 \begin{CJK}{UTF8}{mj}分\end{CJK}) \begin{CJK}{UTF8}{mj}设\end{CJK} $V$ \begin{CJK}{UTF8}{mj}是实数域上的\end{CJK} $n$ \begin{CJK}{UTF8}{mj}维线性空间\end{CJK}. \begin{CJK}{UTF8}{mj}证明\end{CJK}: $V$ \begin{CJK}{UTF8}{mj}上的任一线性变换\end{CJK} $\sigma$ \begin{CJK}{UTF8}{mj}必有一个\end{CJK} 1 \begin{CJK}{UTF8}{mj}维不变子空间或者\end{CJK} 2 \begin{CJK}{UTF8}{mj}维\end{CJK} \begin{CJK}{UTF8}{mj}不变子空间\end{CJK}.

\begin{CJK}{UTF8}{mj}七\end{CJK}. ( 20 \begin{CJK}{UTF8}{mj}分\end{CJK}) \begin{CJK}{UTF8}{mj}设\end{CJK} $V$ \begin{CJK}{UTF8}{mj}是数域\end{CJK} $P$ \begin{CJK}{UTF8}{mj}上的\end{CJK} $n$ \begin{CJK}{UTF8}{mj}维线性空间\end{CJK}, $\sigma$ \begin{CJK}{UTF8}{mj}是\end{CJK} $V$ \begin{CJK}{UTF8}{mj}上的线性变换\end{CJK}, $f(x), g(x) \in P[x],(f(x), g(x))=1, h(x)=$
$$
\operatorname{ker} h(\sigma)=\operatorname{ker} f(\sigma) \oplus \operatorname{ker} g(\sigma) \text {. }
$$
(1) \begin{CJK}{UTF8}{mj}求\end{CJK} $a$ \begin{CJK}{UTF8}{mj}的值\end{CJK};

\section{0. 兰州大学 2018 年研究生入学考试试题高等代数}
\begin{CJK}{UTF8}{mj}李扬\end{CJK}

\begin{CJK}{UTF8}{mj}微信公众号\end{CJK}: sxkyliyang

\begin{CJK}{UTF8}{mj}一\end{CJK}. (1) \begin{CJK}{UTF8}{mj}证明\end{CJK}: \begin{CJK}{UTF8}{mj}次数大于\end{CJK} 0 \begin{CJK}{UTF8}{mj}且首项系数为\end{CJK} 1 \begin{CJK}{UTF8}{mj}的多项式\end{CJK} $f$ \begin{CJK}{UTF8}{mj}是一个不可约多项式的方幂的充分必要条件是对任意的\end{CJK} \begin{CJK}{UTF8}{mj}多项式\end{CJK} $g(x), h(x)$, \begin{CJK}{UTF8}{mj}由\end{CJK} $f(x) \mid g(x) h(x)$ \begin{CJK}{UTF8}{mj}推出\end{CJK} $f(x) \mid g(x)$, \begin{CJK}{UTF8}{mj}或存在某一正整数\end{CJK} $m$ \begin{CJK}{UTF8}{mj}使得\end{CJK} $f(x) \mid h^{m}(x)$.

(2) \begin{CJK}{UTF8}{mj}已知\end{CJK} $f(x)$ \begin{CJK}{UTF8}{mj}是\end{CJK} $\mathbb{Q}$ \begin{CJK}{UTF8}{mj}上的\end{CJK} $n$ \begin{CJK}{UTF8}{mj}次不可约多项式\end{CJK}, \begin{CJK}{UTF8}{mj}且非零复数\end{CJK} $\alpha$ \begin{CJK}{UTF8}{mj}满足\end{CJK} $f(\alpha)=f\left(\frac{1}{\alpha}\right)=0$, \begin{CJK}{UTF8}{mj}证明\end{CJK}: $f(x)$ \begin{CJK}{UTF8}{mj}的每个根\end{CJK} \begin{CJK}{UTF8}{mj}的倒数仍旧是\end{CJK} $f(x)$ \begin{CJK}{UTF8}{mj}的根\end{CJK}.

\begin{CJK}{UTF8}{mj}二\end{CJK}. \begin{CJK}{UTF8}{mj}计算下列\end{CJK} $n$ \begin{CJK}{UTF8}{mj}级行列式的值\end{CJK}:
$$
\left|\begin{array}{cccc}
0 & \alpha_{1}+\alpha_{2} & \cdots & \alpha_{1}+\alpha_{n} \\
\alpha_{2}+\alpha_{1} & 0 & \cdots & \alpha_{2}+\alpha_{n} \\
\vdots & \vdots & & \vdots \\
\alpha_{n}+\alpha_{1} & \alpha_{n}+\alpha_{2} & \cdots & 0
\end{array}\right|
$$
$$
\left|\begin{array}{cccc}
x_{1} & \alpha & \cdots & \alpha \\
\beta & x_{2} & \cdots & \alpha \\
\vdots & \vdots & & \vdots \\
\beta & \beta & \cdots & x_{n}
\end{array}\right|
$$
\begin{CJK}{UTF8}{mj}三\end{CJK}. \begin{CJK}{UTF8}{mj}设矩阵\end{CJK} $A_{n \times m}, B_{m \times s}$ \begin{CJK}{UTF8}{mj}是数域\end{CJK} $P$ \begin{CJK}{UTF8}{mj}上的\end{CJK} $n \times m$ \begin{CJK}{UTF8}{mj}与\end{CJK} $m \times s$ \begin{CJK}{UTF8}{mj}矩阵\end{CJK}, \begin{CJK}{UTF8}{mj}记\end{CJK} $N(A), R(B)$ \begin{CJK}{UTF8}{mj}分别为\end{CJK} $A$ \begin{CJK}{UTF8}{mj}的列向量与\end{CJK} $B$ \begin{CJK}{UTF8}{mj}的行向\end{CJK} \begin{CJK}{UTF8}{mj}量生成的线性空间\end{CJK}. \begin{CJK}{UTF8}{mj}证明\end{CJK}:

(1) \begin{CJK}{UTF8}{mj}由\end{CJK} $V=\left.[N(A) \cap R(B)]^{\perp}\right|_{R(B)}$. \begin{CJK}{UTF8}{mj}可推出\end{CJK} $\mathrm{r}(A B)=\operatorname{dim} V$.

\begin{CJK}{UTF8}{mj}四\end{CJK}. \begin{CJK}{UTF8}{mj}设\end{CJK} $A$ \begin{CJK}{UTF8}{mj}为\end{CJK} $n$ \begin{CJK}{UTF8}{mj}级正交矩阵\end{CJK}, $B$ \begin{CJK}{UTF8}{mj}为\end{CJK} $n$ \begin{CJK}{UTF8}{mj}级实对称矩阵\end{CJK}. \begin{CJK}{UTF8}{mj}证明\end{CJK}: \begin{CJK}{UTF8}{mj}存在实可逆矩阵\end{CJK} $P$ \begin{CJK}{UTF8}{mj}使得\end{CJK} $P^{\prime} A P=E_{n}, P^{\prime} B P$ \begin{CJK}{UTF8}{mj}为对角矩\end{CJK}

\begin{CJK}{UTF8}{mj}五\end{CJK}. \begin{CJK}{UTF8}{mj}已知\end{CJK} $A, B$ \begin{CJK}{UTF8}{mj}是复数域上的\end{CJK} $n$ \begin{CJK}{UTF8}{mj}级方阵\end{CJK}, \begin{CJK}{UTF8}{mj}满足\end{CJK} $A B=B A$, \begin{CJK}{UTF8}{mj}证明存在可逆矩阵\end{CJK} $P$ \begin{CJK}{UTF8}{mj}使得\end{CJK} $P^{-1} A P$ \begin{CJK}{UTF8}{mj}与\end{CJK} $P^{-1} B P$ \begin{CJK}{UTF8}{mj}同时为\end{CJK}

\begin{CJK}{UTF8}{mj}六\end{CJK}. $V$ \begin{CJK}{UTF8}{mj}是复数域\end{CJK} $\mathbb{C}$ \begin{CJK}{UTF8}{mj}上的线性空间\end{CJK}, $\mathscr{A}$ \begin{CJK}{UTF8}{mj}是空间\end{CJK} $V$ \begin{CJK}{UTF8}{mj}上的线性变换\end{CJK}, $f(\lambda)$ \begin{CJK}{UTF8}{mj}是\end{CJK} $\mathscr{A}$ \begin{CJK}{UTF8}{mj}的特征多项式\end{CJK}, \begin{CJK}{UTF8}{mj}且\end{CJK} $f(\lambda)=$
$$
V=\operatorname{Ker}\left(\mathscr{A}-\lambda_{1} \mathscr{E}\right)^{r_{1}} \oplus \operatorname{Ker}\left(\mathscr{A}-\lambda_{2} \mathscr{E}\right)^{r_{2}} \oplus \cdots \oplus \operatorname{Ker}\left(\mathscr{A}-\lambda_{s} \mathscr{E}\right)^{r_{s}} .
$$
\begin{CJK}{UTF8}{mj}七\end{CJK}. \begin{CJK}{UTF8}{mj}设\end{CJK} $\mathscr{A}$ \begin{CJK}{UTF8}{mj}是有限维线性空间\end{CJK} $V$ \begin{CJK}{UTF8}{mj}的线性变换\end{CJK}, $W$ \begin{CJK}{UTF8}{mj}是\end{CJK} $V$ \begin{CJK}{UTF8}{mj}的子空间\end{CJK}, $\mathscr{A} W$ \begin{CJK}{UTF8}{mj}表示\end{CJK} $W$ \begin{CJK}{UTF8}{mj}中向量的像组成的子空间\end{CJK}. \begin{CJK}{UTF8}{mj}证明\end{CJK}
$$
\operatorname{dim} \mathscr{A} W+\operatorname{dim}\left(\mathscr{A}^{-1}(0) \cap W\right)=\operatorname{dim} W .
$$

\section{1. 兰州大学 2009 年研究生入学考试试题数学分析}
\begin{CJK}{UTF8}{mj}李扬\end{CJK}

\begin{CJK}{UTF8}{mj}微信公众号\end{CJK}: sxkyliyang

\begin{CJK}{UTF8}{mj}一\end{CJK}. \begin{CJK}{UTF8}{mj}计算题\end{CJK}.

\begin{enumerate}
  \item $\lim _{x \rightarrow 0} \frac{\int_{0}^{x^{2}} \sin ^{\frac{3}{2}} t \mathrm{~d} t}{\int_{0}^{x} t(t-\sin t) \mathrm{d} t}$.

  \item $\int \arctan \sqrt{x} \mathrm{~d} x$.

  \item $\int_{1}^{2} \mathrm{~d} x \int_{\sqrt{x}}^{x} \frac{1}{y} e^{-x} \mathrm{~d} y+\int_{2}^{4} \mathrm{~d} x \int_{\sqrt{x}}^{2} \frac{1}{y} e^{-x} \mathrm{~d} y$.

  \item \begin{CJK}{UTF8}{mj}求抛物线\end{CJK} $y^{2}=4 x$ \begin{CJK}{UTF8}{mj}与它在\end{CJK} $(1,2)$ \begin{CJK}{UTF8}{mj}处的法线所围成的有限区域的面积\end{CJK}.

\end{enumerate}
5 . \begin{CJK}{UTF8}{mj}求幂级数\end{CJK} $\sum_{n=1}^{\infty}(-1)^{n-1} \frac{x^{2 n-1}}{n}$ \begin{CJK}{UTF8}{mj}的收敛域与和函数\end{CJK}.

\begin{enumerate}
  \setcounter{enumi}{6}
  \item \begin{CJK}{UTF8}{mj}计算曲面积分\end{CJK}
\end{enumerate}
$$
\int_{L}\left(e^{x} \sin y-b(x-y)\right) \mathrm{d} x+\left(e^{x} \cos y-a x\right) \mathrm{d} y,
$$
\begin{CJK}{UTF8}{mj}其中\end{CJK} $L$ \begin{CJK}{UTF8}{mj}是从\end{CJK} $(2 a, 0)$ \begin{CJK}{UTF8}{mj}沿曲线\end{CJK} $y=\sqrt{2 a x-x^{2}}$ \begin{CJK}{UTF8}{mj}到点\end{CJK} $(0,0)$ \begin{CJK}{UTF8}{mj}的一段\end{CJK}.

\begin{CJK}{UTF8}{mj}二\end{CJK}. \begin{CJK}{UTF8}{mj}证明\end{CJK}: $\lim _{n \rightarrow \infty} \sin n$ \begin{CJK}{UTF8}{mj}不存在\end{CJK}.

\begin{CJK}{UTF8}{mj}三\end{CJK}. \begin{CJK}{UTF8}{mj}设函数\end{CJK} $f:[a, b] \rightarrow[a, b]$ \begin{CJK}{UTF8}{mj}满足\end{CJK} $|f(x)-f(y)| \leqslant L|x-y|^{\alpha}, x, y \in[a, b]$, \begin{CJK}{UTF8}{mj}其中\end{CJK} $L, \alpha$ \begin{CJK}{UTF8}{mj}为正常数\end{CJK}. \begin{CJK}{UTF8}{mj}证明\end{CJK}:

(1) \begin{CJK}{UTF8}{mj}当\end{CJK} $\alpha>1$ \begin{CJK}{UTF8}{mj}时\end{CJK}, $f(x)$ \begin{CJK}{UTF8}{mj}恒为常数\end{CJK};

(2) \begin{CJK}{UTF8}{mj}当\end{CJK} $L<1, \alpha=1$ \begin{CJK}{UTF8}{mj}时\end{CJK}, \begin{CJK}{UTF8}{mj}存在唯一的\end{CJK} $\xi \in[a, b]$, \begin{CJK}{UTF8}{mj}使得\end{CJK} $f(\xi)=\xi$.

\begin{CJK}{UTF8}{mj}四\end{CJK}. \begin{CJK}{UTF8}{mj}证明\end{CJK}: \begin{CJK}{UTF8}{mj}有界函数\end{CJK} $f(x)$ \begin{CJK}{UTF8}{mj}在区间\end{CJK} $I$ \begin{CJK}{UTF8}{mj}上一致连续的充分必要条件是对任给的\end{CJK} $\varepsilon>0$ \begin{CJK}{UTF8}{mj}和\end{CJK} $x, y \in I$, \begin{CJK}{UTF8}{mj}总存在正数\end{CJK} $M$, \begin{CJK}{UTF8}{mj}使\end{CJK} \begin{CJK}{UTF8}{mj}得当\end{CJK} $\left|\frac{f(x)-f(y)}{x-y}\right|>M$ \begin{CJK}{UTF8}{mj}时就有\end{CJK} $|f(x)-f(y)|<\varepsilon$.

\begin{CJK}{UTF8}{mj}五\end{CJK}. \begin{CJK}{UTF8}{mj}设\end{CJK} $f: \mathbb{R}^{2} \rightarrow \mathbb{R}^{2}$ \begin{CJK}{UTF8}{mj}是连续映射\end{CJK}, \begin{CJK}{UTF8}{mj}若对\end{CJK} $\mathbb{R}^{2}$ \begin{CJK}{UTF8}{mj}中任何有界闭集\end{CJK} $K, f^{-1}(K)$ \begin{CJK}{UTF8}{mj}均是有界的\end{CJK}, \begin{CJK}{UTF8}{mj}证明\end{CJK}: $f\left(\mathbb{R}^{2}\right)$ \begin{CJK}{UTF8}{mj}是闭集\end{CJK}.

\begin{CJK}{UTF8}{mj}六\end{CJK}. \begin{CJK}{UTF8}{mj}证明\end{CJK}: \begin{CJK}{UTF8}{mj}二元函数\end{CJK} $f(x, y)=\sqrt{|x y|}$ \begin{CJK}{UTF8}{mj}在点\end{CJK} $(0,0)$ \begin{CJK}{UTF8}{mj}处连续\end{CJK}, $f_{x}(0,0), f_{y}(0,0)$ \begin{CJK}{UTF8}{mj}存在\end{CJK}, \begin{CJK}{UTF8}{mj}但\end{CJK} $f(x, y)$ \begin{CJK}{UTF8}{mj}在点\end{CJK} $(0,0)$ \begin{CJK}{UTF8}{mj}处不可微\end{CJK}.

\begin{CJK}{UTF8}{mj}七\end{CJK}. \begin{CJK}{UTF8}{mj}设\end{CJK} $f(x)=\sum_{n=1}^{\infty} \frac{1}{2^{n}+x}$, \begin{CJK}{UTF8}{mj}证明\end{CJK}:

(1) $f(x)$ \begin{CJK}{UTF8}{mj}在\end{CJK} $[0,+\infty)$ \begin{CJK}{UTF8}{mj}上可导\end{CJK}, \begin{CJK}{UTF8}{mj}且一致连续\end{CJK};

(2) \begin{CJK}{UTF8}{mj}反常积分\end{CJK} $\int_{0}^{+\infty} f(x) \mathrm{d} x$ \begin{CJK}{UTF8}{mj}发散\end{CJK}.

\section{2. 兰州大学 2010 年研究生入学考试试题数学分析}
\begin{CJK}{UTF8}{mj}李扬\end{CJK}

\begin{CJK}{UTF8}{mj}微信公众号\end{CJK}: sxkyliyang

\begin{CJK}{UTF8}{mj}一\end{CJK}. \begin{CJK}{UTF8}{mj}计算下列各题\end{CJK}:

\begin{enumerate}
  \item \begin{CJK}{UTF8}{mj}求极限\end{CJK}
\end{enumerate}
$$
\lim _{x \rightarrow 0} \frac{e^{-x^{2}}+1-2 \sqrt{1-x^{2}}}{\sin x^{4}+3 \tan ^{5} x}
$$

\begin{enumerate}
  \setcounter{enumi}{2}
  \item \begin{CJK}{UTF8}{mj}求定积分\end{CJK}
\end{enumerate}
$$
\int_{1}^{e} \sin (\ln x) \mathrm{d} x
$$
3 . \begin{CJK}{UTF8}{mj}设\end{CJK}
$$
f(x, y)= \begin{cases}\frac{x y\left(x^{2}-y^{2}\right)}{x^{2}+y^{2}}, & x^{2}+y^{2} \neq 0 \\ 0, & x^{2}+y^{2}=0\end{cases}
$$
\begin{CJK}{UTF8}{mj}求\end{CJK} $f_{x y}(0,0)$ \begin{CJK}{UTF8}{mj}和\end{CJK} $f_{y x}(0,0)$.

\begin{enumerate}
  \setcounter{enumi}{4}
  \item \begin{CJK}{UTF8}{mj}计算积分\end{CJK}
\end{enumerate}
$$
\int_{1}^{2} \mathrm{~d} x \int_{\sqrt{x}}^{x} \sin \frac{\pi x}{2 y} \mathrm{~d} y+\int_{2}^{4} \mathrm{~d} x \int_{\sqrt{x}}^{4} \sin \frac{\pi x}{2 y} \mathrm{~d} y .
$$

\begin{enumerate}
  \setcounter{enumi}{5}
  \item \begin{CJK}{UTF8}{mj}设\end{CJK} $C$ \begin{CJK}{UTF8}{mj}是柱面\end{CJK} $x^{2}+y^{2}=a^{2}$ \begin{CJK}{UTF8}{mj}与平面\end{CJK} $\frac{x}{a}+\frac{z}{h}=1$ \begin{CJK}{UTF8}{mj}的交线\end{CJK} $(a, h>0)$, \begin{CJK}{UTF8}{mj}且从\end{CJK} $x$ \begin{CJK}{UTF8}{mj}轴正向看为逆时针方向\end{CJK}. \begin{CJK}{UTF8}{mj}计算曲\end{CJK} \begin{CJK}{UTF8}{mj}线积分\end{CJK}
\end{enumerate}
$$
I=\oint_{C}(z-x) \mathrm{d} y+(x-y) \mathrm{d} z
$$

\begin{enumerate}
  \setcounter{enumi}{6}
  \item \begin{CJK}{UTF8}{mj}设\end{CJK}
\end{enumerate}
$$
f(x, y, z)= \begin{cases}x^{2}+y^{2}, & z \geqslant \sqrt{x^{2}+y^{2}}, \\ 0, & z<\sqrt{x^{2}+y^{2}}\end{cases}
$$
$\Sigma$ \begin{CJK}{UTF8}{mj}为单位球面\end{CJK} $x^{2}+y^{2}+z^{2}=1$. \begin{CJK}{UTF8}{mj}计算曲面积分\end{CJK}
$$
I=\iint_{\Sigma} f(x, y, z) \mathrm{d} S .
$$
\begin{CJK}{UTF8}{mj}二\end{CJK}. \begin{CJK}{UTF8}{mj}设实函数\end{CJK} $f(x)=\lim _{t \rightarrow x}\left(\frac{\sin t}{\sin x}\right)^{\frac{x}{\sin t-\sin x}}$. \begin{CJK}{UTF8}{mj}讨论\end{CJK} $f(x)$ \begin{CJK}{UTF8}{mj}的连续性并说明是否可在\end{CJK} $x=0$ \begin{CJK}{UTF8}{mj}处定义\end{CJK} $f(0)$ \begin{CJK}{UTF8}{mj}的值\end{CJK}, \begin{CJK}{UTF8}{mj}使得\end{CJK} $f(x)$ \begin{CJK}{UTF8}{mj}在该点可导\end{CJK}.

\begin{CJK}{UTF8}{mj}三\end{CJK}. \begin{CJK}{UTF8}{mj}已知函数\end{CJK} $f(x)$ \begin{CJK}{UTF8}{mj}在\end{CJK} $[a, b]$ \begin{CJK}{UTF8}{mj}上有二阶导数并且\end{CJK} $f(x)>0, f^{\prime \prime}(x)<0$. \begin{CJK}{UTF8}{mj}记\end{CJK} $f(x)$ \begin{CJK}{UTF8}{mj}的图像曲线为\end{CJK} $C$, \begin{CJK}{UTF8}{mj}过\end{CJK} $C$ \begin{CJK}{UTF8}{mj}上点\end{CJK} $M(t, f(t))(t \in[a, b])$ \begin{CJK}{UTF8}{mj}引切线\end{CJK}. \begin{CJK}{UTF8}{mj}证明\end{CJK}: \begin{CJK}{UTF8}{mj}当\end{CJK} $t$ \begin{CJK}{UTF8}{mj}变动时\end{CJK}, \begin{CJK}{UTF8}{mj}由该切线与曲线\end{CJK} $C$ \begin{CJK}{UTF8}{mj}以及直线\end{CJK} $x=a, x=b$ \begin{CJK}{UTF8}{mj}围成的平面图形\end{CJK} \begin{CJK}{UTF8}{mj}面积可取到最小值\end{CJK}, \begin{CJK}{UTF8}{mj}并求出此值\end{CJK}.

\begin{CJK}{UTF8}{mj}四\end{CJK}. \begin{CJK}{UTF8}{mj}用一致连续的定义验证\end{CJK} $f(x)=\sin \left(x^{2}\right)$ \begin{CJK}{UTF8}{mj}在\end{CJK} $(-\infty,+\infty)$ \begin{CJK}{UTF8}{mj}上不一致连续\end{CJK}.

\begin{CJK}{UTF8}{mj}五\end{CJK}. \begin{CJK}{UTF8}{mj}在区间\end{CJK} $[0,1]$ \begin{CJK}{UTF8}{mj}上\end{CJK}, \begin{CJK}{UTF8}{mj}函数\end{CJK} $f(x)$ \begin{CJK}{UTF8}{mj}定义为\end{CJK}
$$
f(x)= \begin{cases}\frac{1}{x}-\left[\frac{1}{x}\right], & x \in(0,1], \\ 0, & x=0 .\end{cases}
$$
\begin{CJK}{UTF8}{mj}试讨论\end{CJK} $f(x)$ \begin{CJK}{UTF8}{mj}在\end{CJK} $[0,1]$ \begin{CJK}{UTF8}{mj}上的\end{CJK} Riemann \begin{CJK}{UTF8}{mj}可积性\end{CJK}.

\begin{CJK}{UTF8}{mj}六\end{CJK}. \begin{CJK}{UTF8}{mj}设\end{CJK} $f(x)$ \begin{CJK}{UTF8}{mj}是闭区间\end{CJK} $[a, b]$ \begin{CJK}{UTF8}{mj}上的连续可导函数\end{CJK}. \begin{CJK}{UTF8}{mj}记\end{CJK}
$$
f^{-1}(0)=\{x \in[a, b] \mid f(x)=0\} .
$$
\begin{CJK}{UTF8}{mj}假设\end{CJK} $f^{-1}(0) \neq \emptyset$ \begin{CJK}{UTF8}{mj}且\end{CJK} $x \in f^{-1}(0) \Rightarrow f^{\prime}(0) \neq 0$. \begin{CJK}{UTF8}{mj}证明\end{CJK}: $f^{-1}(0)$ \begin{CJK}{UTF8}{mj}是有限集\end{CJK}.

\begin{CJK}{UTF8}{mj}七\end{CJK}. \begin{CJK}{UTF8}{mj}设\end{CJK} $D \subset \mathbb{R}^{2}$ \begin{CJK}{UTF8}{mj}是有界闭集\end{CJK}, $f(x, y)$ \begin{CJK}{UTF8}{mj}是\end{CJK} $D$ \begin{CJK}{UTF8}{mj}上的连续函数\end{CJK}. \begin{CJK}{UTF8}{mj}证明\end{CJK}: $f(x, y)$ \begin{CJK}{UTF8}{mj}在\end{CJK} $D$ \begin{CJK}{UTF8}{mj}上有界\end{CJK}, \begin{CJK}{UTF8}{mj}且一定取到最大值和最小值\end{CJK}.

\section{3. 兰州大学 2011 年研究生入学考试试题数学分析}
\begin{CJK}{UTF8}{mj}李扬\end{CJK}

\begin{CJK}{UTF8}{mj}微信公众号\end{CJK}: sxkyliyang

\begin{CJK}{UTF8}{mj}一\end{CJK}. \begin{CJK}{UTF8}{mj}计算题\end{CJK} $\left(6 \times 10^{\prime}=60^{\prime}\right)$.

\begin{enumerate}
  \item \begin{CJK}{UTF8}{mj}求极限\end{CJK}
\end{enumerate}
$$
\lim _{x \rightarrow 0} \frac{e^{x-\sin x}-e^{\frac{x^{3}}{6}}}{x^{5}}
$$

\begin{enumerate}
  \setcounter{enumi}{2}
  \item \begin{CJK}{UTF8}{mj}设\end{CJK} $x \geqslant 0, f(x)=\int_{0}^{x}\left(u-u^{2}\right) \sin u \mathrm{~d} u$, \begin{CJK}{UTF8}{mj}求\end{CJK} $f(x)$ \begin{CJK}{UTF8}{mj}的最大值\end{CJK}.

  \item \begin{CJK}{UTF8}{mj}设\end{CJK} $\Omega=[0,1]^{3} \subset \mathbb{R}^{3}, p$ \begin{CJK}{UTF8}{mj}是实数\end{CJK}, \begin{CJK}{UTF8}{mj}判断广义积分\end{CJK}

\end{enumerate}
$$
\iint_{\Omega} \frac{\mathrm{d} x \mathrm{~d} y \mathrm{~d} z}{\left(x^{2}+y^{2}+z^{2}\right)^{\frac{p}{2}}}
$$
\begin{CJK}{UTF8}{mj}的敛散性\end{CJK}.

\begin{enumerate}
  \setcounter{enumi}{4}
  \item \begin{CJK}{UTF8}{mj}求幂级数\end{CJK} $\sum_{n=0}^{\infty}\left(\frac{1}{3^{n+1}}-(-2)^{n+1}\right) x^{n}$ \begin{CJK}{UTF8}{mj}的收敛域与和函数\end{CJK}.

  \item \begin{CJK}{UTF8}{mj}计算\end{CJK}

\end{enumerate}
$$
\int_{(1,0)}^{(3,2)} \frac{x \mathrm{~d} x+y \mathrm{~d} y}{\sqrt{x^{2}+y^{2}}}
$$
\begin{CJK}{UTF8}{mj}路径沿曲线\end{CJK} $y^{4}=2 x^{2}-2(x>0, y>0)$.

\begin{enumerate}
  \setcounter{enumi}{6}
  \item \begin{CJK}{UTF8}{mj}计算\end{CJK}
\end{enumerate}
$$
\iint_{\Sigma} z \mathrm{~d} x \mathrm{~d} y+x \mathrm{~d} y \mathrm{~d} z+y \mathrm{~d} x \mathrm{~d} z,
$$
\begin{CJK}{UTF8}{mj}其中\end{CJK} $\Sigma$ \begin{CJK}{UTF8}{mj}为柱面\end{CJK} $x^{2}+y^{2}=1$ \begin{CJK}{UTF8}{mj}被平面\end{CJK} $z=0$ \begin{CJK}{UTF8}{mj}及\end{CJK} $z=3$ \begin{CJK}{UTF8}{mj}所截部分的外侧\end{CJK}.

\begin{CJK}{UTF8}{mj}二\end{CJK}. \begin{CJK}{UTF8}{mj}判断题\end{CJK} $\left(3 \times 6^{\prime}=18^{\prime}\right)$.

\begin{enumerate}
  \item \begin{CJK}{UTF8}{mj}若一元函数\end{CJK} $f(x)$ \begin{CJK}{UTF8}{mj}在\end{CJK} $x=x_{0}$ \begin{CJK}{UTF8}{mj}附近有定义且存在\end{CJK} $\delta>0$ \begin{CJK}{UTF8}{mj}和正常数\end{CJK} $\alpha, L$, \begin{CJK}{UTF8}{mj}使得对任何\end{CJK} $x \in\left(x_{0}-\delta, x_{0}+\delta\right)$ \begin{CJK}{UTF8}{mj}成\end{CJK} \begin{CJK}{UTF8}{mj}立\end{CJK}
\end{enumerate}
$$
\left|f(x)-f\left(x_{0}\right)\right| \leqslant L\left|x-x_{0}\right|^{\alpha}
$$
\begin{CJK}{UTF8}{mj}则\end{CJK} $f(x)$ \begin{CJK}{UTF8}{mj}在\end{CJK} $x_{0}$ \begin{CJK}{UTF8}{mj}点可导\end{CJK}.

\begin{enumerate}
  \setcounter{enumi}{2}
  \item \begin{CJK}{UTF8}{mj}若二元函数\end{CJK} $f(x, y)$ \begin{CJK}{UTF8}{mj}在\end{CJK} $\left(x_{0}, y_{0}\right)$ \begin{CJK}{UTF8}{mj}的某邻域内有定义且\end{CJK} $\lim _{(x, y) \rightarrow\left(x_{0}, y_{0}\right)} f(x, y)$ \begin{CJK}{UTF8}{mj}存在\end{CJK}, \begin{CJK}{UTF8}{mj}则\end{CJK} $\lim _{y \rightarrow y_{0}} \lim _{x \rightarrow x_{0}} f(x, y)$ \begin{CJK}{UTF8}{mj}一定\end{CJK} \begin{CJK}{UTF8}{mj}存在\end{CJK}.

  \item \begin{CJK}{UTF8}{mj}二元函数\end{CJK} $f(x, y)$ \begin{CJK}{UTF8}{mj}在\end{CJK} $\left(x_{0}, y_{0}\right)$ \begin{CJK}{UTF8}{mj}可微\end{CJK}, \begin{CJK}{UTF8}{mj}则\end{CJK} $f(x, y)$ \begin{CJK}{UTF8}{mj}的偏导数在\end{CJK} $\left(x_{0}, y_{0}\right)$ \begin{CJK}{UTF8}{mj}点都连续\end{CJK}.

\end{enumerate}
\begin{CJK}{UTF8}{mj}三\end{CJK}. (15 \begin{CJK}{UTF8}{mj}分\end{CJK}) \begin{CJK}{UTF8}{mj}已知\end{CJK} $x \in[-\pi, \pi]$, \begin{CJK}{UTF8}{mj}证明\end{CJK}: $\left|\sin \frac{x}{2}\right| \leqslant \frac{|x|}{\pi}$.

\begin{CJK}{UTF8}{mj}四\end{CJK}. (15 \begin{CJK}{UTF8}{mj}分\end{CJK}) \begin{CJK}{UTF8}{mj}证明\end{CJK}: \begin{CJK}{UTF8}{mj}符号函数\end{CJK}
$$
\operatorname{sgn}(x)= \begin{cases}-1, & x<0 \\ 0, & x=0 \\ 1, & x>0\end{cases}
$$
\begin{CJK}{UTF8}{mj}在\end{CJK} $[-1,1]$ \begin{CJK}{UTF8}{mj}上是\end{CJK} Riemann \begin{CJK}{UTF8}{mj}可积的\end{CJK}, \begin{CJK}{UTF8}{mj}但在\end{CJK} $[-1,1]$ \begin{CJK}{UTF8}{mj}上不存在原函数\end{CJK}.

\begin{CJK}{UTF8}{mj}五\end{CJK}. ( 15 \begin{CJK}{UTF8}{mj}分\end{CJK}) \begin{CJK}{UTF8}{mj}设\end{CJK} $f(x)$ \begin{CJK}{UTF8}{mj}是\end{CJK} $\mathbb{R}$ \begin{CJK}{UTF8}{mj}上的实函数\end{CJK}, $F(x, y)=\frac{f(y-x)}{2 x}$ \begin{CJK}{UTF8}{mj}且\end{CJK} $F(1, y)=\frac{1}{2} y^{2}-y+5$. \begin{CJK}{UTF8}{mj}取\end{CJK} $x_{0}>0$, \begin{CJK}{UTF8}{mj}定义\end{CJK}
$$
x_{n+1}=F\left(x_{n}, 2 x_{n}\right)(n \geqslant 0) .
$$
\begin{CJK}{UTF8}{mj}求证\end{CJK}: $\lim _{n \rightarrow \infty} x_{n}$ \begin{CJK}{UTF8}{mj}存在并求其值\end{CJK}. \begin{CJK}{UTF8}{mj}六\end{CJK}. ( 15 \begin{CJK}{UTF8}{mj}分\end{CJK}) \begin{CJK}{UTF8}{mj}设\end{CJK} $f(x, y)$ \begin{CJK}{UTF8}{mj}在区域\end{CJK} $D \subset \mathbb{R}^{2}$ \begin{CJK}{UTF8}{mj}上分别对每一自变量\end{CJK} $x$ \begin{CJK}{UTF8}{mj}和\end{CJK} $y$ \begin{CJK}{UTF8}{mj}是连续的\end{CJK}, \begin{CJK}{UTF8}{mj}并且当固定\end{CJK} $x$ \begin{CJK}{UTF8}{mj}时\end{CJK}, \begin{CJK}{UTF8}{mj}对\end{CJK} $y$ \begin{CJK}{UTF8}{mj}是单调的\end{CJK}. \begin{CJK}{UTF8}{mj}证明\end{CJK}: $f$ \begin{CJK}{UTF8}{mj}是区域\end{CJK} $D$ \begin{CJK}{UTF8}{mj}上的二元连续函数\end{CJK}.

\begin{CJK}{UTF8}{mj}七\end{CJK}. ( 12 \begin{CJK}{UTF8}{mj}分\end{CJK}) \begin{CJK}{UTF8}{mj}设函数列\end{CJK} $\left\{f_{n}(x)\right\}_{n=1}^{\infty}$ \begin{CJK}{UTF8}{mj}在\end{CJK} $[0,1]$ \begin{CJK}{UTF8}{mj}上连续\end{CJK}, \begin{CJK}{UTF8}{mj}在\end{CJK} $(0,1)$ \begin{CJK}{UTF8}{mj}内可导\end{CJK}, \begin{CJK}{UTF8}{mj}并且\end{CJK} $\left\{f_{n}(x)\right\}_{n=1}^{\infty}$ \begin{CJK}{UTF8}{mj}在\end{CJK} $[0,1]$ \begin{CJK}{UTF8}{mj}上一致有界\end{CJK}, $\left\{f_{n}^{\prime}(x)\right\}_{n=1}^{\infty}$ \begin{CJK}{UTF8}{mj}在\end{CJK} $(0,1)$ \begin{CJK}{UTF8}{mj}上一致有界\end{CJK}. \begin{CJK}{UTF8}{mj}证明\end{CJK}: \begin{CJK}{UTF8}{mj}函数列\end{CJK} $\left\{f_{n}(x)\right\}_{n=1}^{\infty}$ \begin{CJK}{UTF8}{mj}有一致收敛的子列\end{CJK}.

\section{4. 兰州大学 2012 年研究生入学考试试题数学分析}
\begin{CJK}{UTF8}{mj}李扬\end{CJK}

\begin{CJK}{UTF8}{mj}微信公众号\end{CJK}: sxkyliyang

\begin{CJK}{UTF8}{mj}一\end{CJK}. \begin{CJK}{UTF8}{mj}计算题\end{CJK} $\left(6 \times 10^{\prime}=60^{\prime}\right)$.

\begin{enumerate}
  \item \begin{CJK}{UTF8}{mj}求\end{CJK} $\lim _{x \rightarrow+\infty}\left[\left(x^{3}-x^{2}+\frac{x}{2}\right) e^{\frac{1}{x}}-\sqrt{x^{6}+1}\right]$.

  \item \begin{CJK}{UTF8}{mj}求\end{CJK} $\lim _{n \rightarrow \infty} \int_{n}^{n+a} \frac{\cos t}{\sqrt{t}} \mathrm{~d} t(n$ \begin{CJK}{UTF8}{mj}为正整数\end{CJK}, $a>0)$.

  \item \begin{CJK}{UTF8}{mj}求曲线\end{CJK} $y=e^{-x}$ \begin{CJK}{UTF8}{mj}的切线\end{CJK}, \begin{CJK}{UTF8}{mj}使其和两坐标轴的正半轴围成的面积最大\end{CJK}.

  \item \begin{CJK}{UTF8}{mj}设\end{CJK} $\Omega$ \begin{CJK}{UTF8}{mj}是平面\end{CJK} $\mathbb{R}^{2}$ \begin{CJK}{UTF8}{mj}中由曲线\end{CJK} $x y=1, x y=2, y=x$ \begin{CJK}{UTF8}{mj}和\end{CJK} $y=4 x$ \begin{CJK}{UTF8}{mj}围成的区域\end{CJK}. \begin{CJK}{UTF8}{mj}计算\end{CJK}

\end{enumerate}
$$
\iint_{\Omega} \frac{1}{x y} \mathrm{~d} x \mathrm{~d} y
$$

\begin{enumerate}
  \setcounter{enumi}{5}
  \item \begin{CJK}{UTF8}{mj}计算曲线\end{CJK} $y(x)=\int_{0}^{x} \sqrt{\sin x} \mathrm{~d} x$ \begin{CJK}{UTF8}{mj}在\end{CJK} $[0, \pi]$ \begin{CJK}{UTF8}{mj}上的弧长\end{CJK}.

  \item \begin{CJK}{UTF8}{mj}计算\end{CJK}

\end{enumerate}
$$
I=\int_{L}\left(y^{2}+z^{2}\right) \mathrm{d} x+\left(z^{2}+x^{2}\right) \mathrm{d} y+\left(x^{2}+y^{2}\right) \mathrm{d} z
$$
\begin{CJK}{UTF8}{mj}其中\end{CJK} $L$ \begin{CJK}{UTF8}{mj}为上半球面\end{CJK} $x^{2}+y^{2}+z^{2}=2 R x(z \geqslant 0)$ \begin{CJK}{UTF8}{mj}与圆柱面\end{CJK} $x^{2}+y^{2}=2 r x(R>r>0)$ \begin{CJK}{UTF8}{mj}的交线\end{CJK}, \begin{CJK}{UTF8}{mj}从\end{CJK} $z$ \begin{CJK}{UTF8}{mj}轴的\end{CJK} \begin{CJK}{UTF8}{mj}正向看为逆时针方向\end{CJK}.

\begin{CJK}{UTF8}{mj}二\end{CJK}. ( 20 \begin{CJK}{UTF8}{mj}分\end{CJK}) \begin{CJK}{UTF8}{mj}设\end{CJK} $f(x)$ \begin{CJK}{UTF8}{mj}是定义在\end{CJK} $[1,+\infty)$ \begin{CJK}{UTF8}{mj}上的实函数\end{CJK}, \begin{CJK}{UTF8}{mj}满足\end{CJK} $f(1)=1$ \begin{CJK}{UTF8}{mj}且\end{CJK} $f^{\prime}(x)=\frac{1}{x^{2}+f^{2}(x)}$. \begin{CJK}{UTF8}{mj}证明\end{CJK}: $\lim _{x \rightarrow \infty} f(x)$ \begin{CJK}{UTF8}{mj}存在\end{CJK}, \begin{CJK}{UTF8}{mj}且\end{CJK} $\lim _{x \rightarrow \infty} f(x) \leqslant 1+\frac{\pi}{4}$.

\begin{CJK}{UTF8}{mj}三\end{CJK}. $\left(20\right.$ \begin{CJK}{UTF8}{mj}分\end{CJK}) \begin{CJK}{UTF8}{mj}设\end{CJK} $f(x)$ \begin{CJK}{UTF8}{mj}在\end{CJK} $(0,+\infty)$ \begin{CJK}{UTF8}{mj}上可导\end{CJK}, \begin{CJK}{UTF8}{mj}且\end{CJK} $\lim _{x \rightarrow+\infty} f^{\prime}(x)=+\infty$. \begin{CJK}{UTF8}{mj}证明\end{CJK}: $f(x)$ \begin{CJK}{UTF8}{mj}在\end{CJK} $(0,+\infty)$ \begin{CJK}{UTF8}{mj}上不一致连续\end{CJK}.

\begin{CJK}{UTF8}{mj}四\end{CJK}. (15 \begin{CJK}{UTF8}{mj}分\end{CJK}) \begin{CJK}{UTF8}{mj}设\end{CJK} $p>0$, \begin{CJK}{UTF8}{mj}讨论级数\end{CJK} $\sum_{n=2}^{\infty} \sin \left(n \pi+\sin \frac{1}{n^{p}}\right)$ \begin{CJK}{UTF8}{mj}的敛散性\end{CJK}.

\begin{CJK}{UTF8}{mj}五\end{CJK}. ( 15 \begin{CJK}{UTF8}{mj}分\end{CJK}) \begin{CJK}{UTF8}{mj}设\end{CJK} $f(x, y)=|x-y| \phi(x, y)$, \begin{CJK}{UTF8}{mj}其中\end{CJK} $\phi(x, y)$ \begin{CJK}{UTF8}{mj}在\end{CJK} $(0,0)$ \begin{CJK}{UTF8}{mj}的一个邻域内连续\end{CJK}. \begin{CJK}{UTF8}{mj}证明\end{CJK}:\begin{CJK}{UTF8}{mj}函数\end{CJK} $f(x, y)$ \begin{CJK}{UTF8}{mj}在点\end{CJK} $(0,0)$ \begin{CJK}{UTF8}{mj}处可微的充分必要条件是\end{CJK} $\phi(0,0)=0$.

\begin{CJK}{UTF8}{mj}六\end{CJK}. (10 \begin{CJK}{UTF8}{mj}分\end{CJK}) \begin{CJK}{UTF8}{mj}设\end{CJK} $f(x, y)$ \begin{CJK}{UTF8}{mj}在平面区域\end{CJK} $\Omega$ \begin{CJK}{UTF8}{mj}中连续\end{CJK}. \begin{CJK}{UTF8}{mj}如果对任意\end{CJK} $(x, y) \in \Omega$, \begin{CJK}{UTF8}{mj}都成立\end{CJK} $\frac{\partial f(x, y)}{\partial x}=0$, \begin{CJK}{UTF8}{mj}问能否断定在\end{CJK} $\Omega$ \begin{CJK}{UTF8}{mj}中\end{CJK} $f(x, y)$ \begin{CJK}{UTF8}{mj}不依赖于\end{CJK} $x$ ? \begin{CJK}{UTF8}{mj}请说明理由\end{CJK}.

\begin{CJK}{UTF8}{mj}七\end{CJK}. (10 \begin{CJK}{UTF8}{mj}分\end{CJK}) \begin{CJK}{UTF8}{mj}设函数\end{CJK} $f:(-\infty,+\infty) \rightarrow[0,+\infty)$ \begin{CJK}{UTF8}{mj}是双射\end{CJK}. \begin{CJK}{UTF8}{mj}证明\end{CJK}: $f$ \begin{CJK}{UTF8}{mj}有无穷多个不连续点\end{CJK}.

\section{5. 兰州大学 2013 年研究生入学考试试题数学分析}
\begin{CJK}{UTF8}{mj}李扬\end{CJK}

\begin{CJK}{UTF8}{mj}微信公众号\end{CJK}: sxkyliyang

\begin{CJK}{UTF8}{mj}一\end{CJK}. (30 \begin{CJK}{UTF8}{mj}分\end{CJK}) \begin{CJK}{UTF8}{mj}判断下列命题是否正确\end{CJK}, \begin{CJK}{UTF8}{mj}并说明理由\end{CJK} (\begin{CJK}{UTF8}{mj}每小题\end{CJK} 10 \begin{CJK}{UTF8}{mj}分\end{CJK})

\begin{enumerate}
  \item \begin{CJK}{UTF8}{mj}若\end{CJK} $f(x)$ \begin{CJK}{UTF8}{mj}和\end{CJK} $g(x)$ \begin{CJK}{UTF8}{mj}都是\end{CJK} $[a, b]$ \begin{CJK}{UTF8}{mj}上的单增函数\end{CJK}, \begin{CJK}{UTF8}{mj}则\end{CJK} $f(x) g(x)$ \begin{CJK}{UTF8}{mj}也是\end{CJK} $[a, b]$ \begin{CJK}{UTF8}{mj}上的单增函数\end{CJK}.

  \item \begin{CJK}{UTF8}{mj}设\end{CJK} $f$ \begin{CJK}{UTF8}{mj}在\end{CJK} $[0,1]$ \begin{CJK}{UTF8}{mj}有定义且对任意\end{CJK} $c \in(0,1), \lim _{x \rightarrow c} f(x)$ \begin{CJK}{UTF8}{mj}存在\end{CJK}, \begin{CJK}{UTF8}{mj}则\end{CJK} $f$ \begin{CJK}{UTF8}{mj}在\end{CJK} $[0,1]$ \begin{CJK}{UTF8}{mj}上黎曼可积\end{CJK}.

  \item \begin{CJK}{UTF8}{mj}若函数列\end{CJK} $\left\{f_{n}(x)\right\}$ \begin{CJK}{UTF8}{mj}在\end{CJK} $[0,+\infty)$ \begin{CJK}{UTF8}{mj}上一致收敛到\end{CJK} 0 , \begin{CJK}{UTF8}{mj}则\end{CJK} $\lim _{n \rightarrow \infty} \int_{0}^{+\infty} f_{n}(x) \mathrm{d} x=0$.

\end{enumerate}
\begin{CJK}{UTF8}{mj}二\end{CJK}. (60 \begin{CJK}{UTF8}{mj}分\end{CJK}) \begin{CJK}{UTF8}{mj}计算下列各题\end{CJK} (\begin{CJK}{UTF8}{mj}每小题\end{CJK} 10 \begin{CJK}{UTF8}{mj}分\end{CJK})

\begin{enumerate}
  \item \begin{CJK}{UTF8}{mj}求\end{CJK} $\lim _{x \rightarrow 0} \frac{\cos \left(\frac{\pi}{2} \cos x\right)}{\sin (\sin x)}$.

  \item \begin{CJK}{UTF8}{mj}求\end{CJK} $f(x)=\frac{e^{x}+e^{-x}}{2}$ \begin{CJK}{UTF8}{mj}在\end{CJK} $[1,+\infty)$ \begin{CJK}{UTF8}{mj}上的反函数的导数\end{CJK}.

  \item \begin{CJK}{UTF8}{mj}设\end{CJK}

\end{enumerate}
$$
f(x)= \begin{cases}e^{-\frac{1}{x^{2}}}, & x \neq 0 \\ 0, & x=0\end{cases}
$$
\begin{CJK}{UTF8}{mj}对任意自然数\end{CJK} $k$, \begin{CJK}{UTF8}{mj}求\end{CJK} $f^{(k)}(0)$.

\begin{enumerate}
  \setcounter{enumi}{4}
  \item \begin{CJK}{UTF8}{mj}设\end{CJK} $k, n$ \begin{CJK}{UTF8}{mj}是自然数\end{CJK}, \begin{CJK}{UTF8}{mj}求\end{CJK} $\lim _{n \rightarrow \infty}\left(\lim _{k \rightarrow \infty}(\cos n ! \pi x)^{2 k}\right)$.

  \item \begin{CJK}{UTF8}{mj}求\end{CJK} $\int \frac{\sqrt{1+\sin x}}{\cos x} \mathrm{~d} x$.

  \item \begin{CJK}{UTF8}{mj}计算\end{CJK}

\end{enumerate}
$$
\iint_{\Omega} \frac{e^{\sqrt{y}}}{\sqrt{z^{2}+x^{2}}} \mathrm{~d} z \mathrm{~d} x
$$
\begin{CJK}{UTF8}{mj}其中\end{CJK} $\Omega$ \begin{CJK}{UTF8}{mj}是抛物面\end{CJK} $y=x^{2}+z^{2}$ \begin{CJK}{UTF8}{mj}与平面\end{CJK} $y=1, y=2$ \begin{CJK}{UTF8}{mj}所围立体的表面\end{CJK}, \begin{CJK}{UTF8}{mj}方向取外侧\end{CJK}.

\begin{CJK}{UTF8}{mj}三\end{CJK}. ( 15 \begin{CJK}{UTF8}{mj}分\end{CJK}) \begin{CJK}{UTF8}{mj}设\end{CJK} $a_{n}=\cos n(n \geqslant 0)$. \begin{CJK}{UTF8}{mj}证明\end{CJK}: \begin{CJK}{UTF8}{mj}对任意给定的实数\end{CJK} $r \in[-1,1]$, \begin{CJK}{UTF8}{mj}总存在数列\end{CJK} $\left\{a_{n}\right\}$ \begin{CJK}{UTF8}{mj}的子列\end{CJK} $\left\{a_{n_{j}}\right\}$, \begin{CJK}{UTF8}{mj}使得\end{CJK} $\lim _{j \rightarrow \infty} a_{n_{j}}=r .$

\begin{CJK}{UTF8}{mj}四\end{CJK}. (15 \begin{CJK}{UTF8}{mj}分\end{CJK}) \begin{CJK}{UTF8}{mj}证明\end{CJK}: \begin{CJK}{UTF8}{mj}函数\end{CJK} $f(x)=\sum_{n=1}^{\infty} \frac{(-1)^{n}}{n^{x}}$ \begin{CJK}{UTF8}{mj}在\end{CJK} $(0,+\infty)$ \begin{CJK}{UTF8}{mj}上连续\end{CJK}, \begin{CJK}{UTF8}{mj}且有各阶连续导数\end{CJK}.

\begin{CJK}{UTF8}{mj}五\end{CJK}. ( 10 \begin{CJK}{UTF8}{mj}分\end{CJK}) \begin{CJK}{UTF8}{mj}设函数\end{CJK} $f:[0,1] \rightarrow \mathbb{R}$ \begin{CJK}{UTF8}{mj}是连续的且在\end{CJK} $(0,1)$ \begin{CJK}{UTF8}{mj}上可微\end{CJK}. \begin{CJK}{UTF8}{mj}若\end{CJK} $f$ \begin{CJK}{UTF8}{mj}满足\end{CJK}:

(1) $f(0)=0$;

(2) \begin{CJK}{UTF8}{mj}存在常数\end{CJK} $M>0$ \begin{CJK}{UTF8}{mj}使得\end{CJK} $\left|f^{\prime}(x)\right| \leqslant M|f(x)|$ \begin{CJK}{UTF8}{mj}对任意\end{CJK} $x \in(0,1)$ \begin{CJK}{UTF8}{mj}成立\end{CJK}.

\begin{CJK}{UTF8}{mj}证明\end{CJK}: \begin{CJK}{UTF8}{mj}在\end{CJK} $[0,1]$ \begin{CJK}{UTF8}{mj}上\end{CJK} $f(x)=0$.

\begin{CJK}{UTF8}{mj}六\end{CJK}. (10 \begin{CJK}{UTF8}{mj}分\end{CJK}) \begin{CJK}{UTF8}{mj}设\end{CJK} $D$ \begin{CJK}{UTF8}{mj}是平面上的一个闭圆盘\end{CJK}, $u$ \begin{CJK}{UTF8}{mj}是\end{CJK} $D$ \begin{CJK}{UTF8}{mj}上的连续函数\end{CJK}, \begin{CJK}{UTF8}{mj}并且在\end{CJK} $D$ \begin{CJK}{UTF8}{mj}的内部可微\end{CJK}, \begin{CJK}{UTF8}{mj}在\end{CJK} $D$ \begin{CJK}{UTF8}{mj}的边界上为常值\end{CJK}. \begin{CJK}{UTF8}{mj}证明\end{CJK}: \begin{CJK}{UTF8}{mj}存在\end{CJK} $D$ \begin{CJK}{UTF8}{mj}内一点\end{CJK} $P_{0}$, \begin{CJK}{UTF8}{mj}使得\end{CJK} $\nabla u\left(P_{0}\right)=0$.

\begin{CJK}{UTF8}{mj}七\end{CJK}. (10 \begin{CJK}{UTF8}{mj}分\end{CJK}) \begin{CJK}{UTF8}{mj}设\end{CJK} $D \subset \mathbb{R}^{2}$ \begin{CJK}{UTF8}{mj}是有界闭集\end{CJK}. \begin{CJK}{UTF8}{mj}映射\end{CJK} $f: D \rightarrow D$ \begin{CJK}{UTF8}{mj}满足\end{CJK}:
$$
\|f(x)-f(y)\|<\|x-y\|, \forall x, y \in D
$$
\begin{CJK}{UTF8}{mj}证明\end{CJK}: \begin{CJK}{UTF8}{mj}存在\end{CJK} $x_{0} \in D$, \begin{CJK}{UTF8}{mj}使得\end{CJK} $f\left(x_{0}\right)=x_{0}$. \begin{CJK}{UTF8}{mj}其中\end{CJK} $\|\cdot\|$ \begin{CJK}{UTF8}{mj}定义为\end{CJK}: \begin{CJK}{UTF8}{mj}对\end{CJK} $x=\left(x_{1}, x_{2}\right),\|x\|=\sqrt{x_{1}^{2}+x_{2}^{2}}$.

\section{6. 兰州大学 2014 年研究生入学考试试题数学分析}
\begin{CJK}{UTF8}{mj}李扬\end{CJK}

\begin{CJK}{UTF8}{mj}微信公众号\end{CJK}: sxkyliyang

\begin{CJK}{UTF8}{mj}一\end{CJK}. \begin{CJK}{UTF8}{mj}计算题\end{CJK} (\begin{CJK}{UTF8}{mj}每题\end{CJK} 10 \begin{CJK}{UTF8}{mj}分\end{CJK}, \begin{CJK}{UTF8}{mj}共\end{CJK} 50 \begin{CJK}{UTF8}{mj}分\end{CJK})

\begin{enumerate}
  \item \begin{CJK}{UTF8}{mj}求极限\end{CJK}
\end{enumerate}
$$
\lim _{n \rightarrow \infty}\left(\int_{0}^{\frac{\pi}{2}} \sin ^{n} x \mathrm{~d} x\right)^{\frac{1}{n}} .
$$

\begin{enumerate}
  \setcounter{enumi}{2}
  \item \begin{CJK}{UTF8}{mj}求积分\end{CJK}
\end{enumerate}
$$
\int_{0}^{\ln 2} \sqrt{e^{x}-1} d x
$$

\begin{enumerate}
  \setcounter{enumi}{3}
  \item \begin{CJK}{UTF8}{mj}求曲线积分\end{CJK}
\end{enumerate}
$$
\oint_{L} \frac{(x+4 y) \mathrm{d} y+(x-y) \mathrm{d} x}{x^{2}+4 y^{2}}
$$
\begin{CJK}{UTF8}{mj}其中\end{CJK} $L$ \begin{CJK}{UTF8}{mj}为单位圆\end{CJK} $x^{2}+y^{2}=1$ \begin{CJK}{UTF8}{mj}取正向\end{CJK}.

\begin{enumerate}
  \setcounter{enumi}{4}
  \item \begin{CJK}{UTF8}{mj}计算积分\end{CJK}
\end{enumerate}
$$
\iint_{\Sigma} x(4 y+3) \mathrm{d} y \mathrm{~d} z+\left(2-y^{2}\right) \mathrm{d} z \mathrm{~d} x-2 y z \mathrm{~d} x \mathrm{~d} y
$$
\begin{CJK}{UTF8}{mj}其中\end{CJK} $\Sigma$ \begin{CJK}{UTF8}{mj}是由曲线\end{CJK} $z=\sqrt{y-1}, x=0(1 \leqslant y \leqslant 3)$ \begin{CJK}{UTF8}{mj}绕\end{CJK} $y$ \begin{CJK}{UTF8}{mj}轴旋转一周而成的曲面\end{CJK}, \begin{CJK}{UTF8}{mj}它在点\end{CJK} $(x, y, z)$ \begin{CJK}{UTF8}{mj}处的法向\end{CJK} \begin{CJK}{UTF8}{mj}量与\end{CJK} $y$ \begin{CJK}{UTF8}{mj}轴正向的夹角恒大于\end{CJK} $\frac{\pi}{2}$.

\begin{enumerate}
  \setcounter{enumi}{5}
  \item \begin{CJK}{UTF8}{mj}求幂级数\end{CJK} $\sum_{n=1}^{\infty}(3 n-2) x^{n}$ \begin{CJK}{UTF8}{mj}的和函数\end{CJK}, \begin{CJK}{UTF8}{mj}并求级数\end{CJK} $\sum_{n=1}^{\infty} \frac{3 n-2}{3^{n}}$ \begin{CJK}{UTF8}{mj}的和\end{CJK}.
\end{enumerate}
\begin{CJK}{UTF8}{mj}二\end{CJK}. (15 \begin{CJK}{UTF8}{mj}分\end{CJK}) \begin{CJK}{UTF8}{mj}设\end{CJK} $f_{n}(x)=n \sin \sqrt{4 \pi^{2} n^{2}+x^{2}}$, \begin{CJK}{UTF8}{mj}求证\end{CJK}:

\begin{enumerate}
  \item $\left\{f_{n}(x)\right\}$ \begin{CJK}{UTF8}{mj}在有界闭区间\end{CJK} $[0, a]$ \begin{CJK}{UTF8}{mj}上一致收敛\end{CJK}, \begin{CJK}{UTF8}{mj}其中\end{CJK} $a>0$ \begin{CJK}{UTF8}{mj}为常数\end{CJK}.

  \item \begin{CJK}{UTF8}{mj}讨论\end{CJK} $\left\{f_{n}(x)\right\}$ \begin{CJK}{UTF8}{mj}在\end{CJK} $(-\infty,+\infty)$ \begin{CJK}{UTF8}{mj}上是否一致收敛\end{CJK}.

\end{enumerate}
\begin{CJK}{UTF8}{mj}三\end{CJK}. ( 15 \begin{CJK}{UTF8}{mj}分\end{CJK}) \begin{CJK}{UTF8}{mj}证明\end{CJK}: \begin{CJK}{UTF8}{mj}当\end{CJK} $0<x<\frac{\pi}{2}$ \begin{CJK}{UTF8}{mj}时\end{CJK}, $\frac{\pi}{2}-x>\cos x>1-\frac{2}{\pi} x$.

\begin{CJK}{UTF8}{mj}四\end{CJK}. ( 15 \begin{CJK}{UTF8}{mj}分\end{CJK}) \begin{CJK}{UTF8}{mj}设\end{CJK}
$$
f(x, y)= \begin{cases}\frac{x^{2} y^{2}}{\left(x^{2}+4 y^{2}\right)^{\frac{3}{2}}}, & x^{2}+y^{2} \neq 0, \\ 0, & x^{2}+y^{2}=0 .\end{cases}
$$
\begin{CJK}{UTF8}{mj}证明\end{CJK}: $f(x, y)$ \begin{CJK}{UTF8}{mj}在点\end{CJK} $(0,0)$ \begin{CJK}{UTF8}{mj}处连续\end{CJK}, \begin{CJK}{UTF8}{mj}且偏导数存在\end{CJK}, \begin{CJK}{UTF8}{mj}但不可微\end{CJK}.

\begin{CJK}{UTF8}{mj}五\end{CJK}. (15 \begin{CJK}{UTF8}{mj}分\end{CJK}) \begin{CJK}{UTF8}{mj}求证\end{CJK}: $f(x)=x^{p} \ln x$ \begin{CJK}{UTF8}{mj}在\end{CJK} $(0,+\infty)$ \begin{CJK}{UTF8}{mj}上一致连续\end{CJK}, \begin{CJK}{UTF8}{mj}其中\end{CJK} $0<p<1$ \begin{CJK}{UTF8}{mj}为定常数\end{CJK}.

\begin{CJK}{UTF8}{mj}六\end{CJK}. (15 \begin{CJK}{UTF8}{mj}分\end{CJK}) \begin{CJK}{UTF8}{mj}设函数\end{CJK} $f$ \begin{CJK}{UTF8}{mj}在整个数轴上二次可微\end{CJK}, \begin{CJK}{UTF8}{mj}且\end{CJK}
$$
M_{0}=\sup \{|f(x)| \mid x \in(-\infty,+\infty)\}, M_{2}=\sup \left\{\left|f^{\prime \prime}(x)\right| \mid x \in(-\infty,+\infty)\right\}
$$
\begin{CJK}{UTF8}{mj}为有限数\end{CJK}. \begin{CJK}{UTF8}{mj}证明\end{CJK}: \begin{CJK}{UTF8}{mj}对任意的\end{CJK} $x \in(-\infty,+\infty)$, \begin{CJK}{UTF8}{mj}有\end{CJK} $\left|f^{\prime}(x)\right| \leqslant \sqrt{2 M_{0} M_{2}}$.

\begin{CJK}{UTF8}{mj}七\end{CJK}. (10 \begin{CJK}{UTF8}{mj}分\end{CJK}) \begin{CJK}{UTF8}{mj}设函数\end{CJK} $f(x)$ \begin{CJK}{UTF8}{mj}定义在\end{CJK} $[0,1]$ \begin{CJK}{UTF8}{mj}上\end{CJK}, \begin{CJK}{UTF8}{mj}其中对所有的正整数\end{CJK} $n, f\left(\frac{1}{n}\right)=\sin \frac{(-1)^{n}}{n}$, \begin{CJK}{UTF8}{mj}而在\end{CJK} $[0,1]$ \begin{CJK}{UTF8}{mj}上其余点处取\end{CJK} \begin{CJK}{UTF8}{mj}值为零\end{CJK}. \begin{CJK}{UTF8}{mj}试用定积分定义证明函数\end{CJK} $f(x)$ \begin{CJK}{UTF8}{mj}在\end{CJK} $[0,1]$ \begin{CJK}{UTF8}{mj}上可积\end{CJK}.

\begin{CJK}{UTF8}{mj}八\end{CJK}. (15 \begin{CJK}{UTF8}{mj}分\end{CJK}) \begin{CJK}{UTF8}{mj}设\end{CJK} $f(x)$ \begin{CJK}{UTF8}{mj}是定义在\end{CJK} $(0,+\infty)$ \begin{CJK}{UTF8}{mj}上的有界连续函数\end{CJK}. \begin{CJK}{UTF8}{mj}定义\end{CJK}
$$
I=\left\{a \in(-\infty,+\infty) \mid \text { 存在非负数列 }\left\{x_{n}\right\} \text {, 当 } n \rightarrow \infty \text { 时, } x_{n} \rightarrow+\infty \text {, 且 } f\left(x_{n}\right) \rightarrow a\right\} .
$$
\begin{CJK}{UTF8}{mj}证明\end{CJK}: $I$ \begin{CJK}{UTF8}{mj}是一个闭区间或一个单点集\end{CJK}.


\end{document}