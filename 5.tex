\documentclass[10pt]{article}
\usepackage[utf8]{inputenc}
\usepackage[T1]{fontenc}
\usepackage{CJKutf8}
\usepackage{amsmath}
\usepackage{amsfonts}
\usepackage{amssymb}
\usepackage{mhchem}
\usepackage{stmaryrd}
\usepackage{bbold}
\usepackage{mathrsfs}
\usepackage{graphicx}
\usepackage[export]{adjustbox}
\graphicspath{ {./images/} }

\begin{document}
\begin{enumerate}
  \setcounter{enumi}{8}
  \item (15 \begin{CJK}{UTF8}{mj}分\end{CJK}) \begin{CJK}{UTF8}{mj}求矩阵\end{CJK} $A=\left(\begin{array}{cccc}2 & 1 & -1 & 0 \\ 20 & 3 & -1 & -20 \\ 20 & 1 & 1 & -20 \\ 5 & 1 & -1 & -3\end{array}\right)$ \begin{CJK}{UTF8}{mj}的特征值\end{CJK}、\begin{CJK}{UTF8}{mj}极小多项式以及\end{CJK} Jordan \begin{CJK}{UTF8}{mj}标准型\end{CJK}.

  \item (15 \begin{CJK}{UTF8}{mj}分\end{CJK}) \begin{CJK}{UTF8}{mj}设\end{CJK}

\end{enumerate}
$$
f\left(x_{1}, x_{2}, \cdots, x_{n}\right)=\sum_{i=1}^{n} \sum_{j=1}^{n} a_{i j} x_{i} x_{j}
$$
\begin{CJK}{UTF8}{mj}是一个实一次型\end{CJK}, \begin{CJK}{UTF8}{mj}其中\end{CJK} $A=\left(a_{i j}\right)$ \begin{CJK}{UTF8}{mj}是实对称矩阵\end{CJK}. \begin{CJK}{UTF8}{mj}将二次型看作\end{CJK} $n$ \begin{CJK}{UTF8}{mj}元实函数\end{CJK}, \begin{CJK}{UTF8}{mj}用代数方法确定它在\end{CJK}
$$
S=\left\{X=\left(x_{1}, x_{2}, \cdots, x_{n}\right)^{T} \in \mathbb{R}^{n} \mid x_{1}^{2}+x_{2}^{2}+\cdots+x_{n}^{2}=1\right\}
$$
\begin{CJK}{UTF8}{mj}上的取值范围\end{CJK}.

\section{$2.172012$ 年}
\begin{enumerate}
  \item (10 \begin{CJK}{UTF8}{mj}分\end{CJK}) \begin{CJK}{UTF8}{mj}计算行列式\end{CJK}:
\end{enumerate}
$$
D=\left|\begin{array}{lllll}
1 & 4 & 6 & 4 & 1 \\
1 & 1 & 4 & 6 & 4 \\
4 & 1 & 1 & 4 & 6 \\
6 & 4 & 1 & 1 & 4 \\
4 & 6 & 4 & 1 & 1
\end{array}\right|
$$

\begin{enumerate}
  \setcounter{enumi}{2}
  \item (15 \begin{CJK}{UTF8}{mj}分\end{CJK}) \begin{CJK}{UTF8}{mj}试给出三次方程\end{CJK} $x^{3}-a_{1} x^{2}+a_{2} x-a_{3}=0$ \begin{CJK}{UTF8}{mj}的三个根成等比数列的充分必要条件\end{CJK} (\begin{CJK}{UTF8}{mj}用\end{CJK} $a_{1}, a_{2}, a_{3}$ \begin{CJK}{UTF8}{mj}的形式表示\end{CJK}, \begin{CJK}{UTF8}{mj}且尽可能简单\end{CJK}), \begin{CJK}{UTF8}{mj}并证明你的结论\end{CJK}.

  \item (20 \begin{CJK}{UTF8}{mj}分\end{CJK}) \begin{CJK}{UTF8}{mj}设\end{CJK} $K$ \begin{CJK}{UTF8}{mj}是数域\end{CJK}, $W \in K^{n}$ \begin{CJK}{UTF8}{mj}是\end{CJK} $K$ \begin{CJK}{UTF8}{mj}上的线性方程组\end{CJK} $A X=B$ \begin{CJK}{UTF8}{mj}的非空解集\end{CJK}, \begin{CJK}{UTF8}{mj}其中\end{CJK} $A \in$ $M_{m \times n}(K), X=\left(x_{1}, x_{2}, \cdots, x_{n}\right)^{T}, B \in M_{m \times 1}(K)$. \begin{CJK}{UTF8}{mj}证明\end{CJK}:

\end{enumerate}
(1) \begin{CJK}{UTF8}{mj}存在该方程组的特解\end{CJK} $\gamma_{0}$, \begin{CJK}{UTF8}{mj}及\end{CJK} $K^{n}$ \begin{CJK}{UTF8}{mj}的子空间\end{CJK} $V$, \begin{CJK}{UTF8}{mj}使\end{CJK} $W=\gamma_{0}+V=\left\{\gamma_{0}+\eta \mid \eta \in V\right\}$;

(2) \begin{CJK}{UTF8}{mj}若取\end{CJK} $\gamma_{0}=(2,0,1,2)^{T}, V$ \begin{CJK}{UTF8}{mj}是由\end{CJK} $(2,1,0,0)^{T},(4,0,-1,0)^{T},(1,0,0,1)^{T},(3,0,-1,-1)^{T}$ \begin{CJK}{UTF8}{mj}生成的\end{CJK} $K^{n}$ \begin{CJK}{UTF8}{mj}的子空间\end{CJK}. \begin{CJK}{UTF8}{mj}试求线性方程组\end{CJK}, \begin{CJK}{UTF8}{mj}使其解集等于\end{CJK} $\gamma_{0}+V$.

\begin{enumerate}
  \setcounter{enumi}{4}
  \item (20 \begin{CJK}{UTF8}{mj}分\end{CJK}) \begin{CJK}{UTF8}{mj}设\end{CJK} $A$ \begin{CJK}{UTF8}{mj}是\end{CJK} $n$ \begin{CJK}{UTF8}{mj}阶实对称矩阵\end{CJK}. \begin{CJK}{UTF8}{mj}证明\end{CJK}:
\end{enumerate}
(1) \begin{CJK}{UTF8}{mj}存在正定矩阵\end{CJK} $B$ \begin{CJK}{UTF8}{mj}和负定矩阵\end{CJK} $C$, \begin{CJK}{UTF8}{mj}使得\end{CJK} $A=B+C$. \begin{CJK}{UTF8}{mj}这样的分解唯一吗\end{CJK}? \begin{CJK}{UTF8}{mj}说明理由\end{CJK}.

(2) \begin{CJK}{UTF8}{mj}如果\end{CJK} $A$ \begin{CJK}{UTF8}{mj}正定\end{CJK}, \begin{CJK}{UTF8}{mj}且\end{CJK} $n>1$, \begin{CJK}{UTF8}{mj}则存在不定矩阵\end{CJK} $D$, \begin{CJK}{UTF8}{mj}使得\end{CJK} $A=D^{2}$.

\begin{enumerate}
  \setcounter{enumi}{5}
  \item (14 \begin{CJK}{UTF8}{mj}分\end{CJK}) \begin{CJK}{UTF8}{mj}统计学中将各元素为非负数\end{CJK}, \begin{CJK}{UTF8}{mj}且每行元素之和为\end{CJK} 1 \begin{CJK}{UTF8}{mj}的方阵\end{CJK} $A$ \begin{CJK}{UTF8}{mj}称为\end{CJK}"\begin{CJK}{UTF8}{mj}转移概率矩阵\end{CJK}". \begin{CJK}{UTF8}{mj}证明\end{CJK}:
\end{enumerate}
(1) \begin{CJK}{UTF8}{mj}转移概率矩阵\end{CJK} $A$ \begin{CJK}{UTF8}{mj}必以\end{CJK} 1 \begin{CJK}{UTF8}{mj}为其一个特征值\end{CJK};

(2) \begin{CJK}{UTF8}{mj}转移概率矩阵\end{CJK} $A$ \begin{CJK}{UTF8}{mj}与\end{CJK} $B$ \begin{CJK}{UTF8}{mj}的乘积仍是转移概率矩阵\end{CJK}.

\begin{enumerate}
  \setcounter{enumi}{6}
  \item (20 \begin{CJK}{UTF8}{mj}分\end{CJK}) \begin{CJK}{UTF8}{mj}设\end{CJK} $K^{n}$ \begin{CJK}{UTF8}{mj}是数域\end{CJK} $K$ \begin{CJK}{UTF8}{mj}上的线性空间\end{CJK}.
\end{enumerate}
(1) \begin{CJK}{UTF8}{mj}若\end{CJK} $K^{n}=V_{1} \oplus V_{2}$, \begin{CJK}{UTF8}{mj}其中\end{CJK} $V_{1}, V_{2}$ \begin{CJK}{UTF8}{mj}是\end{CJK} $K^{n}$ \begin{CJK}{UTF8}{mj}的两个非平凡子空间\end{CJK}. \begin{CJK}{UTF8}{mj}证明\end{CJK}: \begin{CJK}{UTF8}{mj}存在唯一幂等矩阵\end{CJK} $A \in M_{n}(K)$, \begin{CJK}{UTF8}{mj}使\end{CJK}
$$
V_{1}=\left\{X \in K^{n} \mid A X=0\right\}, \quad V_{2}=\left\{X \in K^{n} \mid A X=X\right\}
$$
(2) \begin{CJK}{UTF8}{mj}若取子空间\end{CJK} $V_{1}=\left\{X=\left(a_{1}, \cdots, a_{n}\right)^{T} \in K^{n} \mid a_{1}+a_{2}+\cdots+a_{n}=0\right\}, \quad V_{2}=\left\{X \in K^{n} \mid X=\right.$ $\left.(a, a, \cdots, a)^{T}\right\}$, \begin{CJK}{UTF8}{mj}证明\end{CJK}: $K^{n}=V_{1} \oplus V_{2}$,\begin{CJK}{UTF8}{mj}并求\end{CJK} $(1)$ \begin{CJK}{UTF8}{mj}小题中对应的幕等矩阵\end{CJK} $A$.

\begin{enumerate}
  \setcounter{enumi}{7}
  \item (16 \begin{CJK}{UTF8}{mj}分\end{CJK}) \begin{CJK}{UTF8}{mj}求矩阵\end{CJK} $A=\left(\begin{array}{cccc}5 & 0 & -4 & -4 \\ 6 & 8 & 1 & 8 \\ 14 & 7 & -6 & 0 \\ -6 & -7 & -1 & -7\end{array}\right)$ \begin{CJK}{UTF8}{mj}的特征多项式\end{CJK}、\begin{CJK}{UTF8}{mj}初等因子组\end{CJK}、\begin{CJK}{UTF8}{mj}极小多项式以及\end{CJK}
\end{enumerate}
Jordan \begin{CJK}{UTF8}{mj}标准型\end{CJK}.

\begin{enumerate}
  \setcounter{enumi}{8}
  \item (20 \begin{CJK}{UTF8}{mj}分\end{CJK}) \begin{CJK}{UTF8}{mj}证明\end{CJK}: \begin{CJK}{UTF8}{mj}对每个正整数\end{CJK} $n$, \begin{CJK}{UTF8}{mj}均存在\end{CJK} $A \in M_{n}(\mathbb{R})$, \begin{CJK}{UTF8}{mj}使得\end{CJK} $A^{3}=A+2 E_{n}$. \begin{CJK}{UTF8}{mj}对任何满足条件\end{CJK} $A^{3}=A+2 E_{n}$ \begin{CJK}{UTF8}{mj}的矩阵\end{CJK} $A$, \begin{CJK}{UTF8}{mj}均有\end{CJK} $|A|>0$.

  \item (15 \begin{CJK}{UTF8}{mj}分\end{CJK}) \begin{CJK}{UTF8}{mj}求所有\end{CJK} 3 \begin{CJK}{UTF8}{mj}阶复矩阵\end{CJK} $A$, \begin{CJK}{UTF8}{mj}使得\end{CJK} $A$ \begin{CJK}{UTF8}{mj}与\end{CJK} $A^{2}$ \begin{CJK}{UTF8}{mj}相似\end{CJK}.

\end{enumerate}
\section{$2.182013$ 年}
\begin{enumerate}
  \item (15 \begin{CJK}{UTF8}{mj}分\end{CJK}) \begin{CJK}{UTF8}{mj}计算下列行列式的值\end{CJK}:
\end{enumerate}
$$
\left|\begin{array}{cccc}
2 & -1 & 1 & -2 \\
-1 & 2 & -1 & 0 \\
0 & -1 & 2 & -1 \\
-1 & 2 & -2 & 2
\end{array}\right| ; \quad(2)\left|\begin{array}{cccccc}
1 & 2 & 2 & 2 & \cdots & 2 \\
2 & 2 & 2 & 2 & \cdots & 2 \\
2 & 2 & 3 & 2 & \cdots & 2 \\
2 & 2 & 2 & 4 & \cdots & 2 \\
\vdots & \vdots & \vdots & \vdots & \ddots & \vdots \\
2 & 2 & 2 & 2 & \cdots & n
\end{array}\right|
$$

\begin{enumerate}
  \setcounter{enumi}{2}
  \item (12 \begin{CJK}{UTF8}{mj}分\end{CJK}) \begin{CJK}{UTF8}{mj}设\end{CJK} $A$ \begin{CJK}{UTF8}{mj}为\end{CJK} $4 \times 2$ \begin{CJK}{UTF8}{mj}矩阵\end{CJK}, $B$ \begin{CJK}{UTF8}{mj}为\end{CJK} $2 \times 4$ \begin{CJK}{UTF8}{mj}矩阵\end{CJK}, \begin{CJK}{UTF8}{mj}且满足\end{CJK}
\end{enumerate}
$$
A B=\left(\begin{array}{cccc}
1 & 0 & -1 & 0 \\
0 & 1 & 0 & -1 \\
-1 & 0 & 1 & 0 \\
0 & -1 & 0 & 1
\end{array}\right)
$$
\begin{CJK}{UTF8}{mj}求\end{CJK} BA.
$$
\text { 3. (20 分) 设 } A=\left(\begin{array}{ccccc}
-3 & 1 & 0 & 0 & 0 \\
0 & -3 & 0 & 0 & 0 \\
0 & 0 & 0 & 0 & 0 \\
0 & 0 & 1 & 0 & 0 \\
0 & 0 & 0 & 1 & 0
\end{array}\right) \text {, 求 } A^{2} \text { 的不变因子组, 初等因子组, 极小多项式, }
$$
Jordan \begin{CJK}{UTF8}{mj}标准开攵\end{CJK}.

\begin{enumerate}
  \setcounter{enumi}{4}
  \item (15 \begin{CJK}{UTF8}{mj}分\end{CJK}) \begin{CJK}{UTF8}{mj}设多项式\end{CJK} $f(x)=x^{4}-x^{3}+2 x^{2}-x+1, g(x)=x^{3}-2 x^{2}+2 x-1$. \begin{CJK}{UTF8}{mj}求\end{CJK} $f(x), g(x)$ \begin{CJK}{UTF8}{mj}的首一\end{CJK} \begin{CJK}{UTF8}{mj}最大公因式\end{CJK} $(f(x), g(x))$ \begin{CJK}{UTF8}{mj}以及多项式\end{CJK} $u(x), v(x)$, \begin{CJK}{UTF8}{mj}使得\end{CJK}
\end{enumerate}
$$
(f(x), g(x))=u(x) f(x)+v(x) g(x)
$$

\begin{enumerate}
  \setcounter{enumi}{5}
  \item (15 \begin{CJK}{UTF8}{mj}分\end{CJK}) \begin{CJK}{UTF8}{mj}求次数是低的多项式\end{CJK} $f(x)$, \begin{CJK}{UTF8}{mj}使得\end{CJK}
\end{enumerate}
$$
f(1)=1, f(-1)=-1, f(2)=2, f(-2)=-8
$$

\begin{enumerate}
  \setcounter{enumi}{6}
  \item (20 \begin{CJK}{UTF8}{mj}分\end{CJK}) \begin{CJK}{UTF8}{mj}用正交线性替换化下列二次型为标准型\end{CJK}
\end{enumerate}
$$
2 x_{1}^{2}+4 x_{1} x_{2}-4 x_{1} x_{3}+5 x_{2}^{2}-8 x_{2} x_{3}+5 x_{3}^{2} .
$$

\begin{enumerate}
  \setcounter{enumi}{7}
  \item (18 \begin{CJK}{UTF8}{mj}分\end{CJK}) \begin{CJK}{UTF8}{mj}证明\end{CJK}: (1). \begin{CJK}{UTF8}{mj}矩阵\end{CJK} $A$ \begin{CJK}{UTF8}{mj}幂零的充要条件为\end{CJK} 0 \begin{CJK}{UTF8}{mj}是\end{CJK} $A$ \begin{CJK}{UTF8}{mj}的唯一特征值\end{CJK}.
\end{enumerate}
(2). \begin{CJK}{UTF8}{mj}对于任何矩阵\end{CJK} $A \in M_{n}(\mathbb{C})$, \begin{CJK}{UTF8}{mj}存在可对角化矩阵\end{CJK} $B$ \begin{CJK}{UTF8}{mj}以及幂零阵\end{CJK} $C$, \begin{CJK}{UTF8}{mj}使得\end{CJK} $B C=C B$, \begin{CJK}{UTF8}{mj}且\end{CJK} $A$ \begin{CJK}{UTF8}{mj}有分\end{CJK} \begin{CJK}{UTF8}{mj}解\end{CJK} $A=B+C$, \begin{CJK}{UTF8}{mj}并且这样的分解是唯一的\end{CJK}.

\begin{enumerate}
  \setcounter{enumi}{8}
  \item (15 \begin{CJK}{UTF8}{mj}分\end{CJK}) \begin{CJK}{UTF8}{mj}设\end{CJK} $(\alpha, \beta)$ \begin{CJK}{UTF8}{mj}是欧几里得空间\end{CJK} $V$ \begin{CJK}{UTF8}{mj}的内积函数\end{CJK}. \begin{CJK}{UTF8}{mj}对于任何给定的\end{CJK} $\gamma \in V$, \begin{CJK}{UTF8}{mj}定义\end{CJK} $V$ \begin{CJK}{UTF8}{mj}的函数\end{CJK} $f_{\gamma}: \alpha \mapsto(\alpha, \gamma)$, \begin{CJK}{UTF8}{mj}即\end{CJK} $f_{\gamma}(\alpha)=(\alpha, \gamma)$. \begin{CJK}{UTF8}{mj}证明\end{CJK}:
\end{enumerate}
(1). $f_{\gamma}$ \begin{CJK}{UTF8}{mj}是\end{CJK} $V$ \begin{CJK}{UTF8}{mj}的线性函数\end{CJK};

(2). $V$ \begin{CJK}{UTF8}{mj}的线性函数都具有\end{CJK} $f_{\gamma}$ \begin{CJK}{UTF8}{mj}的形式\end{CJK}.

\begin{enumerate}
  \setcounter{enumi}{9}
  \item (20 \begin{CJK}{UTF8}{mj}分\end{CJK}) \begin{CJK}{UTF8}{mj}证明\end{CJK}: \begin{CJK}{UTF8}{mj}对任何实系数可逆矩阵\end{CJK} $A$, \begin{CJK}{UTF8}{mj}存在正交矩阵\end{CJK} $Q$ \begin{CJK}{UTF8}{mj}以及上三角矩阵\end{CJK} $R$, \begin{CJK}{UTF8}{mj}使得\end{CJK} $A=Q R$, \begin{CJK}{UTF8}{mj}且如果要求\end{CJK} $T$ \begin{CJK}{UTF8}{mj}的主对角线上的元素均大于零\end{CJK},\begin{CJK}{UTF8}{mj}则此分解是唯一的\end{CJK}. \begin{CJK}{UTF8}{mj}对\end{CJK} $A=\left(\begin{array}{rrr}1 & 1 & 0 \\ 2 & -1 & 5 \\ -2 & 4\end{array}\right)$, \begin{CJK}{UTF8}{mj}求这样的\end{CJK} \begin{CJK}{UTF8}{mj}分解\end{CJK}.
\end{enumerate}
\section{$2.192014$ 年}
\begin{enumerate}
  \item (10 \begin{CJK}{UTF8}{mj}分\end{CJK}) \begin{CJK}{UTF8}{mj}计算行列式\end{CJK} $D=\left|\begin{array}{cccc}a^{2}+m & b a & c a & d a \\ a b & b^{2}+m & c b & d b \\ a c & b c & c^{2}+m & d c \\ a d & b d & c d & d^{2}+m\end{array}\right|$.

  \item (15 \begin{CJK}{UTF8}{mj}分\end{CJK}) \begin{CJK}{UTF8}{mj}设矩阵\end{CJK} $A=\left(a_{i j}\right) \in M_{m \times n}(\mathbb{R}), B=\left(b_{1}, \cdots, b_{m}\right)^{T} \in M_{m \times 1}(\mathbb{R})$. \begin{CJK}{UTF8}{mj}证明\end{CJK}: \begin{CJK}{UTF8}{mj}线性方程组\end{CJK} $A^{T} A X=A^{T} B$ 。\begin{CJK}{UTF8}{mj}定有解\end{CJK}.

  \item (15 \begin{CJK}{UTF8}{mj}分\end{CJK}) \begin{CJK}{UTF8}{mj}设矩阵\end{CJK} $A \in M_{n}(\mathbb{C})$ \begin{CJK}{UTF8}{mj}的特征值互不相同\end{CJK}. \begin{CJK}{UTF8}{mj}定义\end{CJK}

\end{enumerate}
$$
C(A)=\left\{B \in M_{n}(\mathbb{C}) \mid A B=B A\right\}
$$
(1). \begin{CJK}{UTF8}{mj}验证\end{CJK}: $C(A)$ \begin{CJK}{UTF8}{mj}是复线性空间\end{CJK} $M_{n}(\mathbb{C})$ \begin{CJK}{UTF8}{mj}的线性子空间\end{CJK};

(2). \begin{CJK}{UTF8}{mj}证明\end{CJK}: \begin{CJK}{UTF8}{mj}对于任意\end{CJK} $B, C \in C(A)$, \begin{CJK}{UTF8}{mj}有\end{CJK} $B C=C B$.

\begin{enumerate}
  \setcounter{enumi}{4}
  \item (20 \begin{CJK}{UTF8}{mj}分\end{CJK}) \begin{CJK}{UTF8}{mj}设\end{CJK} $V$ \begin{CJK}{UTF8}{mj}是数域\end{CJK} ( $\mathbb{K})$ \begin{CJK}{UTF8}{mj}上的\end{CJK} 4 \begin{CJK}{UTF8}{mj}维线性空间\end{CJK}, $\alpha_{1}, \alpha_{2}, \alpha_{3}, \alpha_{4}$ \begin{CJK}{UTF8}{mj}是\end{CJK} $V$ \begin{CJK}{UTF8}{mj}的一组基\end{CJK}. \begin{CJK}{UTF8}{mj}若\end{CJK} $\mathscr{A}$ \begin{CJK}{UTF8}{mj}是\end{CJK} $V$ \begin{CJK}{UTF8}{mj}上的线\end{CJK} \begin{CJK}{UTF8}{mj}性变换\end{CJK}, \begin{CJK}{UTF8}{mj}且在基\end{CJK} $\alpha_{1}, \alpha_{2}, \alpha_{3}, \alpha_{4}$ \begin{CJK}{UTF8}{mj}下的矩阵为准对角阵\end{CJK} $\left(\begin{array}{cccc}1 & 1 & 0 & 0 \\ 0 & 1 & 0 & 0 \\ 0 & 0 & 3 & 0 \\ 0 & 0 & 0 & 3\end{array}\right)$, \begin{CJK}{UTF8}{mj}试求所有\end{CJK} $\mathscr{A}$ - \begin{CJK}{UTF8}{mj}不变子空间\end{CJK}.

  \item (15 \begin{CJK}{UTF8}{mj}分\end{CJK}) \begin{CJK}{UTF8}{mj}设\end{CJK} $A \in M_{n}(\mathbb{R})$ \begin{CJK}{UTF8}{mj}是半正定矩阵\end{CJK}, \begin{CJK}{UTF8}{mj}且存在整数\end{CJK} $m>1$, \begin{CJK}{UTF8}{mj}使得\end{CJK} $A^{m}=E_{n}$, \begin{CJK}{UTF8}{mj}求\end{CJK} $A$; \begin{CJK}{UTF8}{mj}若将上述\end{CJK} “\begin{CJK}{UTF8}{mj}半\end{CJK} \begin{CJK}{UTF8}{mj}正定\end{CJK}”\begin{CJK}{UTF8}{mj}的条件改为\end{CJK} “\begin{CJK}{UTF8}{mj}半负定\end{CJK}”, \begin{CJK}{UTF8}{mj}你能得出什么结论\end{CJK}?

  \item (20 \begin{CJK}{UTF8}{mj}分\end{CJK}) \begin{CJK}{UTF8}{mj}设\end{CJK} $V$ \begin{CJK}{UTF8}{mj}是实数域上的\end{CJK} $n$ \begin{CJK}{UTF8}{mj}维欧式空间\end{CJK}, $e_{1}, \cdots, e_{n}$ \begin{CJK}{UTF8}{mj}是\end{CJK} \begin{CJK}{UTF8}{mj}组基\end{CJK}, \begin{CJK}{UTF8}{mj}满足内积\end{CJK} $\left(e_{i}, e_{j}\right) \leqslant 0(i \neq j)$.

\end{enumerate}
(1). \begin{CJK}{UTF8}{mj}证明\end{CJK}: \begin{CJK}{UTF8}{mj}存在一个非零向量\end{CJK} $v \in V$, \begin{CJK}{UTF8}{mj}满足\end{CJK} $\left(e_{i}, v\right) \geqslant 0, \forall i$.

(2). \begin{CJK}{UTF8}{mj}假设\end{CJK} $v=a_{1} e_{1}+\cdots+a_{n} e_{n} \in V$ \begin{CJK}{UTF8}{mj}是任何满足\end{CJK} $(1)$ \begin{CJK}{UTF8}{mj}的向量\end{CJK}, \begin{CJK}{UTF8}{mj}证明\end{CJK}: $a_{i} \geqslant 0, i=1,2, \cdots, n$.

(3). \begin{CJK}{UTF8}{mj}设\end{CJK} $u=b_{1} e_{1}+\cdots+b_{n} e_{n} \in V$ \begin{CJK}{UTF8}{mj}是另一个满足\end{CJK} (1) \begin{CJK}{UTF8}{mj}的向量\end{CJK}, \begin{CJK}{UTF8}{mj}并定义\end{CJK} $w=c_{1} e_{1}+\cdots+c_{n} e_{n} \in V$, \begin{CJK}{UTF8}{mj}其\end{CJK}

\includegraphics[max width=\textwidth]{2022_04_18_7db0708508f26638f054g-005}
$$
c_{i}=\min \left\{a_{i}, b_{i}\right\}, i=1,2, \cdots, n,
$$
\begin{CJK}{UTF8}{mj}证明\end{CJK}: \begin{CJK}{UTF8}{mj}向量\end{CJK} $w$ \begin{CJK}{UTF8}{mj}也满足\end{CJK} (1).

\begin{enumerate}
  \setcounter{enumi}{7}
  \item (25 \begin{CJK}{UTF8}{mj}分\end{CJK}) \begin{CJK}{UTF8}{mj}设\end{CJK} $n$ \begin{CJK}{UTF8}{mj}阶矩阵\end{CJK}
\end{enumerate}
\includegraphics[max width=\textwidth]{2022_04_18_7db0708508f26638f054g-005(1)}

\begin{CJK}{UTF8}{mj}其特征多项式记为\end{CJK} $f_{n}(\lambda)$.

(1). \begin{CJK}{UTF8}{mj}证明\end{CJK}: $f_{n}(\lambda)=(\lambda+2) f_{n-1}(\lambda)-f_{n-2}(\lambda)$.

(2). \begin{CJK}{UTF8}{mj}求\end{CJK} $f_{1}(\lambda), f_{2}(\lambda), f_{3}(\lambda)$, \begin{CJK}{UTF8}{mj}并求相应的特征值及特征向量\end{CJK}.

(3). \begin{CJK}{UTF8}{mj}试写出\end{CJK} $A_{3}$ \begin{CJK}{UTF8}{mj}的若尔当典范型\end{CJK}.

\begin{enumerate}
  \setcounter{enumi}{8}
  \item (15 \begin{CJK}{UTF8}{mj}分\end{CJK}) \begin{CJK}{UTF8}{mj}设\end{CJK} $A \in M_{n}(\mathbb{C})$ \begin{CJK}{UTF8}{mj}是一个幂零矩阵\end{CJK} (\begin{CJK}{UTF8}{mj}即\end{CJK}, \begin{CJK}{UTF8}{mj}存在正整数\end{CJK} $m$, \begin{CJK}{UTF8}{mj}使得\end{CJK} $A^{m}=0$ ), \begin{CJK}{UTF8}{mj}定义矩阵\end{CJK} $\exp (A)=\sum_{k=0}^{\infty} \frac{A^{k}}{k !}$. \begin{CJK}{UTF8}{mj}证明\end{CJK}: $\exp (A)$ \begin{CJK}{UTF8}{mj}是可逆矩阵\end{CJK}, \begin{CJK}{UTF8}{mj}且\end{CJK} $\exp (A)^{-1}=\exp (-A)$.

  \item (15 \begin{CJK}{UTF8}{mj}分\end{CJK}) \begin{CJK}{UTF8}{mj}设\end{CJK} $A_{1}, A_{2}, \cdots, A_{n}$ \begin{CJK}{UTF8}{mj}都是数域\end{CJK} $\mathbb{K}$ \begin{CJK}{UTF8}{mj}上的\end{CJK} $n$ \begin{CJK}{UTF8}{mj}阶非零矩阵\end{CJK},

\end{enumerate}
$$
A_{i}^{2}=A_{i}(i=1,2, \cdots, n), A_{i} A_{j}=0(i \neq j ; i, j=1,2, \cdots, n) .
$$
(1). \begin{CJK}{UTF8}{mj}证明\end{CJK}: $A_{i}(i=1,2, \cdots, n)$ \begin{CJK}{UTF8}{mj}都可以对角化\end{CJK};

(2). \begin{CJK}{UTF8}{mj}求数域\end{CJK} $\mathbb{K}$ \begin{CJK}{UTF8}{mj}上的\end{CJK} $n$ \begin{CJK}{UTF8}{mj}阶可逆矩阵\end{CJK} $P$, \begin{CJK}{UTF8}{mj}使得\end{CJK} $P^{-1} A_{1} P, P^{-1} A_{2} P, \cdots, P^{-1} A_{n} P$ \begin{CJK}{UTF8}{mj}为对角矩阵\end{CJK}.

\section{$2.202015$ 年}
\begin{enumerate}
  \item (20 \begin{CJK}{UTF8}{mj}分\end{CJK}) \begin{CJK}{UTF8}{mj}求一个\end{CJK} 3 \begin{CJK}{UTF8}{mj}阶实对称矩阵\end{CJK} $A$, \begin{CJK}{UTF8}{mj}满足\end{CJK}:\begin{CJK}{UTF8}{mj}特征值为\end{CJK} $6,3,3$, \begin{CJK}{UTF8}{mj}且\end{CJK} 6 \begin{CJK}{UTF8}{mj}对应的特征向量为\end{CJK} $\alpha_{1}=$ $(1,1,1)^{T}$.

  \item (20 \begin{CJK}{UTF8}{mj}分\end{CJK}) \begin{CJK}{UTF8}{mj}设矩阵\end{CJK} $A=\left(\begin{array}{cccc}1 & -2 & -2 & -2 \\ -2 & 1 & -2 & -2 \\ -2 & -2 & 1 & -2 \\ -2 & -2 & -2 & 1\end{array}\right)$, \begin{CJK}{UTF8}{mj}求圭正交矩阵\end{CJK} $T$, \begin{CJK}{UTF8}{mj}使\end{CJK} $T{ }^{-1} A T$ \begin{CJK}{UTF8}{mj}为对角阵\end{CJK}.

  \item (15 \begin{CJK}{UTF8}{mj}分\end{CJK}) \begin{CJK}{UTF8}{mj}求解下面的方程组\end{CJK}

\end{enumerate}
\includegraphics[max width=\textwidth]{2022_04_18_7db0708508f26638f054g-007}

(1) \begin{CJK}{UTF8}{mj}证明\end{CJK}: $f_{1}(x)=x, f_{2}(x)=x^{2}-1, f_{n}(x)=x f_{n-1}(x)-f_{n-2}(x),(n>2)$,

(2) \begin{CJK}{UTF8}{mj}求\end{CJK} $f_{n}(2)$ \begin{CJK}{UTF8}{mj}的值\end{CJK},

(3) \begin{CJK}{UTF8}{mj}证明\end{CJK}: $f_{n}(x)=0$ \begin{CJK}{UTF8}{mj}的根是绝对值不超过\end{CJK} 2 \begin{CJK}{UTF8}{mj}的实数\end{CJK}.
$$
\text { 的方程组 }\left\{\begin{array}{l}
x_{1}+x_{2}+\cdots+x_{n}=0 \\
x_{1}^{2}+x_{2}^{2}+\cdots+x_{n}^{2}=0 \\
\cdots \cdots \cdots \cdots \cdots \\
x_{1}^{n}+x_{2}^{n}+\cdots+x_{n}^{n}=0
\end{array}\right. \text { 只有零解. }
$$

\begin{enumerate}
  \setcounter{enumi}{6}
  \item (15 \begin{CJK}{UTF8}{mj}分\end{CJK}) \begin{CJK}{UTF8}{mj}设\end{CJK} $A, B$ \begin{CJK}{UTF8}{mj}为复数域的\end{CJK} $n$ \begin{CJK}{UTF8}{mj}阶矩阵\end{CJK}, \begin{CJK}{UTF8}{mj}且\end{CJK} $A^{2}=A, B^{2}=B, \operatorname{rank}(A)=\operatorname{rank}(B)$, \begin{CJK}{UTF8}{mj}证明\end{CJK}: $A$ \begin{CJK}{UTF8}{mj}与\end{CJK} $B$ \begin{CJK}{UTF8}{mj}相\end{CJK} \begin{CJK}{UTF8}{mj}似\end{CJK}.

  \item (20 \begin{CJK}{UTF8}{mj}分\end{CJK}) \begin{CJK}{UTF8}{mj}设\end{CJK} $A, B \in R^{2 \times 2}$, \begin{CJK}{UTF8}{mj}且\end{CJK}

\end{enumerate}
$$
A^{2}=B^{2}=E, A B+B A=0
$$
\begin{CJK}{UTF8}{mj}证明\end{CJK}: \begin{CJK}{UTF8}{mj}存在可逆矩阵\end{CJK} $P$ \begin{CJK}{UTF8}{mj}使得\end{CJK}
$$
P^{-1} A P=\left(\begin{array}{cc}
1 & 0 \\
0 & -1
\end{array}\right), P^{-1} B P=\left(\begin{array}{ll}
0 & 1 \\
1 & 0
\end{array}\right) \text {. }
$$

\begin{enumerate}
  \setcounter{enumi}{8}
  \item (20 \begin{CJK}{UTF8}{mj}分\end{CJK}) \begin{CJK}{UTF8}{mj}域\end{CJK} $\mathbb{C}$ \begin{CJK}{UTF8}{mj}上全体\end{CJK} $n$ \begin{CJK}{UTF8}{mj}阶矩阵组成一个\end{CJK} $n^{2}$ \begin{CJK}{UTF8}{mj}维线性空间\end{CJK} $M_{n}(\mathbb{C}), A \in M_{n}(\mathbb{C}), A$ \begin{CJK}{UTF8}{mj}可对角化\end{CJK}, \begin{CJK}{UTF8}{mj}特征\end{CJK} \begin{CJK}{UTF8}{mj}值为\end{CJK} $\lambda_{1}, \lambda_{2}, \ldots, \lambda_{n}$ (\begin{CJK}{UTF8}{mj}不一定不相等\end{CJK}), \begin{CJK}{UTF8}{mj}设\end{CJK} $\mathscr{A}$ \begin{CJK}{UTF8}{mj}是域\end{CJK} $\mathbb{C}$ \begin{CJK}{UTF8}{mj}的变换\end{CJK}, $\mathscr{A}(B)=A B-B A$,
\end{enumerate}
(1). \begin{CJK}{UTF8}{mj}证明\end{CJK}: $\mathscr{A}$ \begin{CJK}{UTF8}{mj}为域\end{CJK} $\mathbb{C}$ \begin{CJK}{UTF8}{mj}上的线性变换\end{CJK},

(2). \begin{CJK}{UTF8}{mj}求\end{CJK} $\mathscr{A}$ \begin{CJK}{UTF8}{mj}的所有\end{CJK} $n^{2}$ \begin{CJK}{UTF8}{mj}个特征值\end{CJK}.

\section{$2.212016$ 年}
\begin{enumerate}
  \item (15 \begin{CJK}{UTF8}{mj}分\end{CJK}) \begin{CJK}{UTF8}{mj}设\end{CJK} $M$ \begin{CJK}{UTF8}{mj}是二阶矩阵\end{CJK}, \begin{CJK}{UTF8}{mj}求证\end{CJK}:
\end{enumerate}
$$
M\left(\begin{array}{cc}
0 & 1 \\
-1 & 0
\end{array}\right) M^{T}=\left(\begin{array}{cc}
0 & 1 \\
-1 & 0
\end{array}\right) \Leftrightarrow|M|=1
$$

\begin{enumerate}
  \setcounter{enumi}{2}
  \item (15 \begin{CJK}{UTF8}{mj}分\end{CJK}) \begin{CJK}{UTF8}{mj}在矩阵\end{CJK}
\end{enumerate}
$$
A=\left(\begin{array}{cccc}
1 & 2 & \cdots & n \\
n+1 & n+2 & \cdots & 2 n \\
\cdots & \cdots & & \cdots \\
(n-1) n+1 & (n-1) n+2 & \cdots & n^{2}
\end{array}\right)
$$
\begin{CJK}{UTF8}{mj}中取\end{CJK} $n$ \begin{CJK}{UTF8}{mj}个数\end{CJK}, \begin{CJK}{UTF8}{mj}使得每行每列都恰好只被取到一个数\end{CJK}. \begin{CJK}{UTF8}{mj}问\end{CJK}: \begin{CJK}{UTF8}{mj}这些取出的数相加之和会有哪些可能的值\end{CJK}?

\begin{enumerate}
  \setcounter{enumi}{3}
  \item (30 \begin{CJK}{UTF8}{mj}分\end{CJK}) \begin{CJK}{UTF8}{mj}已知矩阵\end{CJK}
\end{enumerate}
$$
A=\left(\begin{array}{cccc}
3 & 1 & 0 & -1 \\
1 & 3 & -1 & 0 \\
0 & -1 & 3 & 1 \\
-1 & 0 & 1 & 3
\end{array}\right)
$$
\begin{CJK}{UTF8}{mj}求正交矩阵\end{CJK} $Q$, \begin{CJK}{UTF8}{mj}使得\end{CJK} $Q^{-1} A Q$ \begin{CJK}{UTF8}{mj}为对角矩阵\end{CJK}, \begin{CJK}{UTF8}{mj}并写出得到的对角矩阵\end{CJK}.

\begin{enumerate}
  \setcounter{enumi}{4}
  \item (15 \begin{CJK}{UTF8}{mj}分\end{CJK}) \begin{CJK}{UTF8}{mj}设\end{CJK} $\varphi$ \begin{CJK}{UTF8}{mj}是\end{CJK} $n$ \begin{CJK}{UTF8}{mj}维线性空间\end{CJK} $V$ \begin{CJK}{UTF8}{mj}上的线性变换\end{CJK}, $\alpha$ \begin{CJK}{UTF8}{mj}是\end{CJK} $V$ \begin{CJK}{UTF8}{mj}中的向量\end{CJK}. \begin{CJK}{UTF8}{mj}已知整数\end{CJK} $m$ \begin{CJK}{UTF8}{mj}满足\end{CJK} $\varphi^{m}(\alpha) \neq 0$, \begin{CJK}{UTF8}{mj}但\end{CJK} $\varphi^{m+1}(\alpha)=0$. \begin{CJK}{UTF8}{mj}求证\end{CJK} $\alpha, \varphi(\alpha), \cdots, \varphi^{m}(\alpha)$ \begin{CJK}{UTF8}{mj}线性无关\end{CJK}.

  \item (20 \begin{CJK}{UTF8}{mj}分\end{CJK}) \begin{CJK}{UTF8}{mj}设\end{CJK} $V$ \begin{CJK}{UTF8}{mj}是数域\end{CJK} $K$ \begin{CJK}{UTF8}{mj}上的线性空间\end{CJK}, $X$ \begin{CJK}{UTF8}{mj}是一个集合\end{CJK}. \begin{CJK}{UTF8}{mj}已知存在一个双射\end{CJK} $\varphi: X \rightarrow V$. \begin{CJK}{UTF8}{mj}先在\end{CJK} $X$ \begin{CJK}{UTF8}{mj}上定义加法和数乘运算如下\end{CJK}:

\end{enumerate}
$$
\begin{aligned}
&x \oplus y=\varphi^{-1}(\varphi(x)+\varphi(y)), \quad \forall x, y \in X, \\
&x \circ y=\varphi^{-1}(\lambda \varphi(x)), \quad \forall \lambda \in K, x \in X .
\end{aligned}
$$
\begin{CJK}{UTF8}{mj}验证\end{CJK} $X$ \begin{CJK}{UTF8}{mj}关于上述定义的加法与数乘构成\end{CJK} $K$ \begin{CJK}{UTF8}{mj}上的一个线性空间\end{CJK},\begin{CJK}{UTF8}{mj}并且\end{CJK} $\varphi$ \begin{CJK}{UTF8}{mj}是线性空间之间的一个同构\end{CJK}.

\begin{enumerate}
  \setcounter{enumi}{6}
  \item (20 \begin{CJK}{UTF8}{mj}分\end{CJK}) \begin{CJK}{UTF8}{mj}设\end{CJK} $V$ \begin{CJK}{UTF8}{mj}是全体\end{CJK} $n$ \begin{CJK}{UTF8}{mj}阶实系数矩阵构成的线性空间\end{CJK}, \begin{CJK}{UTF8}{mj}定义运算\end{CJK}
\end{enumerate}
$$
(A, B)=\operatorname{Tr}\left(A^{T} B\right), \quad A, B \in V
$$
(1) \begin{CJK}{UTF8}{mj}证明\end{CJK}: $(,$, \begin{CJK}{UTF8}{mj}是内积\end{CJK}, $V$ \begin{CJK}{UTF8}{mj}是\end{CJK} $n^{2}$ \begin{CJK}{UTF8}{mj}维欧式空间\end{CJK}.

(2) \begin{CJK}{UTF8}{mj}设\end{CJK} $T \in V$ \begin{CJK}{UTF8}{mj}是给定矩阵\end{CJK}, \begin{CJK}{UTF8}{mj}定义映射\end{CJK}
$$
\phi(A)=T A, \quad A \in V
$$
\begin{CJK}{UTF8}{mj}证明\end{CJK}: $\phi$ \begin{CJK}{UTF8}{mj}是\end{CJK} $V$ \begin{CJK}{UTF8}{mj}上的线性映射\end{CJK}.

(3) \begin{CJK}{UTF8}{mj}求\end{CJK} $\phi$ \begin{CJK}{UTF8}{mj}的伴随算子\end{CJK}.

\begin{enumerate}
  \setcounter{enumi}{7}
  \item (15 \begin{CJK}{UTF8}{mj}分\end{CJK}) \begin{CJK}{UTF8}{mj}证明\end{CJK}: \begin{CJK}{UTF8}{mj}下列二次型\end{CJK}
\end{enumerate}
$$
f\left(x_{1}, x_{2}, \cdots, x_{n}\right)=n \sum_{i=1}^{n} x_{i}^{2}-\left(\sum_{i=1}^{n} x_{i}\right)^{2}
$$
\begin{CJK}{UTF8}{mj}是半正定型\end{CJK}. 8. (20 \begin{CJK}{UTF8}{mj}分\end{CJK}) \begin{CJK}{UTF8}{mj}已知实矩阵\end{CJK}
$$
A=\left(\begin{array}{cccccc}
a_{1} & b_{1} & & & & \\
c_{1} & a_{2} & b_{2} & & \\
& \ddots & \ddots & \ddots & \\
& \ddots & \ddots & b_{n-1} \\
& & c_{n-1} & a_{n}
\end{array}\right)
$$
\begin{CJK}{UTF8}{mj}满足\end{CJK} $b_{i} c_{i}>0,(i=1,2, \cdots, n-1)$. \begin{CJK}{UTF8}{mj}求证\end{CJK}: $A$ \begin{CJK}{UTF8}{mj}有\end{CJK} $n$ \begin{CJK}{UTF8}{mj}个两两不同的实特征值\end{CJK}.

(\begin{CJK}{UTF8}{mj}提示\end{CJK}: \begin{CJK}{UTF8}{mj}先考虑\end{CJK} $b_{i}=c_{i}(i=1,2, \cdots, n-1$ ) \begin{CJK}{UTF8}{mj}的特殊情况\end{CJK}; \begin{CJK}{UTF8}{mj}对一般情形\end{CJK}, \begin{CJK}{UTF8}{mj}试找出一个实对角可逆矩\end{CJK} \begin{CJK}{UTF8}{mj}阵\end{CJK} $D$ \begin{CJK}{UTF8}{mj}使得\end{CJK} $D^{-1} A D$ \begin{CJK}{UTF8}{mj}符合该特殊情形\end{CJK}.)

\section{$2.222017$ 年}
\begin{enumerate}
  \item (20 \begin{CJK}{UTF8}{mj}分\end{CJK}) \begin{CJK}{UTF8}{mj}当实数\end{CJK} $\lambda$ \begin{CJK}{UTF8}{mj}为何值时\end{CJK}, \begin{CJK}{UTF8}{mj}方程组\end{CJK}
\end{enumerate}
$$
\left\{\begin{array}{l}
(\lambda-2) x_{1}-x_{2}-x_{3}=-2 \\
4 x_{1}+(\lambda-1) x_{2}+4 x_{3}=7 \\
x_{1}+x_{2}+x_{3}=2
\end{array}\right.
$$
\begin{CJK}{UTF8}{mj}有唯一解\end{CJK}, \begin{CJK}{UTF8}{mj}无解\end{CJK}, \begin{CJK}{UTF8}{mj}有无穷多个解\end{CJK}; \begin{CJK}{UTF8}{mj}有解时\end{CJK}, \begin{CJK}{UTF8}{mj}请求出求全部解\end{CJK}.

\begin{enumerate}
  \setcounter{enumi}{2}
  \item (12 \begin{CJK}{UTF8}{mj}分\end{CJK}) \begin{CJK}{UTF8}{mj}已知二次型\end{CJK}
\end{enumerate}
$$
f\left(x_{1}, x_{2}, x_{3}\right)=2 x_{1}^{2}+x_{2}^{2}+3 x_{3}^{2}+2 \lambda x_{1} x_{2}+2 x_{1} x_{3}
$$
\begin{CJK}{UTF8}{mj}正定\end{CJK}, \begin{CJK}{UTF8}{mj}求\end{CJK} $\lambda$ \begin{CJK}{UTF8}{mj}的取值范围\end{CJK}.

\begin{enumerate}
  \setcounter{enumi}{3}
  \item (20 \begin{CJK}{UTF8}{mj}分\end{CJK}) \begin{CJK}{UTF8}{mj}已知实对称矩阵\end{CJK}
\end{enumerate}
$$
A=\left(\begin{array}{lll}
4 & 1 & 1 \\
1 & 4 & 1 \\
1 & 1 & 4
\end{array}\right)
$$
\begin{CJK}{UTF8}{mj}求正交矩阵\end{CJK} $T$, \begin{CJK}{UTF8}{mj}使得\end{CJK} $T^{-1} A T$ \begin{CJK}{UTF8}{mj}为对角矩阵\end{CJK}.

\begin{enumerate}
  \setcounter{enumi}{4}
  \item (20 \begin{CJK}{UTF8}{mj}分\end{CJK}) \begin{CJK}{UTF8}{mj}设\end{CJK} $\mathbb{K}$ \begin{CJK}{UTF8}{mj}是数域\end{CJK},
\end{enumerate}
(1) \begin{CJK}{UTF8}{mj}证明\end{CJK}: \begin{CJK}{UTF8}{mj}一元多项式\end{CJK} $x^{2}+x^{3}$ \begin{CJK}{UTF8}{mj}不能写成另一多项式的平方\end{CJK};

(2) \begin{CJK}{UTF8}{mj}证明\end{CJK}: \begin{CJK}{UTF8}{mj}二元多项式\end{CJK} $y^{2}-x^{2}-x^{3}$ \begin{CJK}{UTF8}{mj}是二元多项式环\end{CJK} $K[x, y]$ \begin{CJK}{UTF8}{mj}中的不可约多项式\end{CJK}, \begin{CJK}{UTF8}{mj}也就是说它不能\end{CJK} \begin{CJK}{UTF8}{mj}㝍成两个非常数多项式的乘积\end{CJK}.

\begin{enumerate}
  \setcounter{enumi}{5}
  \item (13 \begin{CJK}{UTF8}{mj}分\end{CJK}) \begin{CJK}{UTF8}{mj}设\end{CJK} $A, B$ \begin{CJK}{UTF8}{mj}是同阶方阵\end{CJK}, \begin{CJK}{UTF8}{mj}若\end{CJK} $A$ \begin{CJK}{UTF8}{mj}可逆\end{CJK}, \begin{CJK}{UTF8}{mj}证明\end{CJK} $A B$ \begin{CJK}{UTF8}{mj}与\end{CJK} $B A$ \begin{CJK}{UTF8}{mj}相似\end{CJK}. \begin{CJK}{UTF8}{mj}问当\end{CJK} $A$ \begin{CJK}{UTF8}{mj}不可逆时\end{CJK},\begin{CJK}{UTF8}{mj}结论是否成\end{CJK} iे?

  \item (10 \begin{CJK}{UTF8}{mj}分\end{CJK}) \begin{CJK}{UTF8}{mj}给定\end{CJK} $m+n$ \begin{CJK}{UTF8}{mj}阶分块方阵\end{CJK}

\end{enumerate}
$$
A=\left(\begin{array}{cc}
0_{m} & B_{m \times n} \\
C_{n \times m} & 0_{n}
\end{array}\right)
$$
\begin{CJK}{UTF8}{mj}证明\end{CJK}: \begin{CJK}{UTF8}{mj}若\end{CJK} $\lambda$ \begin{CJK}{UTF8}{mj}为\end{CJK} $A$ \begin{CJK}{UTF8}{mj}的特征值\end{CJK}, \begin{CJK}{UTF8}{mj}则\end{CJK} $-\lambda$ \begin{CJK}{UTF8}{mj}也为\end{CJK} $A$ \begin{CJK}{UTF8}{mj}的特征值\end{CJK}.

\begin{enumerate}
  \setcounter{enumi}{7}
  \item (20 \begin{CJK}{UTF8}{mj}分\end{CJK}) (1) \begin{CJK}{UTF8}{mj}求证\end{CJK}: 3 \begin{CJK}{UTF8}{mj}阶复矩阵\end{CJK} $A$ \begin{CJK}{UTF8}{mj}与\end{CJK} $B$ \begin{CJK}{UTF8}{mj}相似的充要条件是它们有相同的特征多项式和极小多\end{CJK} \begin{CJK}{UTF8}{mj}项式\end{CJK};
\end{enumerate}
(2) \begin{CJK}{UTF8}{mj}挙例说明\end{CJK} 4 \begin{CJK}{UTF8}{mj}阶复矩阵即使韦相同的特征多项式和极小多项式也不一定相似\end{CJK}.

\begin{enumerate}
  \setcounter{enumi}{8}
  \item (15 \begin{CJK}{UTF8}{mj}分\end{CJK}) \begin{CJK}{UTF8}{mj}设\end{CJK} $f: U \rightarrow V, g: V \rightarrow W$ \begin{CJK}{UTF8}{mj}是数域\end{CJK} $\mathbb{K}$ \begin{CJK}{UTF8}{mj}上有限维的线性映射\end{CJK}, \begin{CJK}{UTF8}{mj}证明\end{CJK}:
\end{enumerate}
$$
\operatorname{dim}(\operatorname{Ker} f)+\operatorname{dim}(\operatorname{Im} f \cap \operatorname{Ker} g)=\operatorname{dim}(\operatorname{Ker}(g f)) .
$$

\begin{enumerate}
  \setcounter{enumi}{9}
  \item (20 \begin{CJK}{UTF8}{mj}分\end{CJK}) \begin{CJK}{UTF8}{mj}设\end{CJK} $f(x), g(x)$ \begin{CJK}{UTF8}{mj}是数域\end{CJK} $\mathbb{K}$ \begin{CJK}{UTF8}{mj}上的非零多项式\end{CJK}, $A$ \begin{CJK}{UTF8}{mj}是\end{CJK} $n(n \geqslant 2)$ \begin{CJK}{UTF8}{mj}阶方阵\end{CJK}.
\end{enumerate}
(1) \begin{CJK}{UTF8}{mj}证明\end{CJK}: \begin{CJK}{UTF8}{mj}若\end{CJK} $g(A)$ \begin{CJK}{UTF8}{mj}可逆\end{CJK}, \begin{CJK}{UTF8}{mj}则\end{CJK}
$$
f(A) g(A)^{*}=g(A)^{*} f(A)
$$
\begin{CJK}{UTF8}{mj}其中\end{CJK} $g(A)^{*}$ \begin{CJK}{UTF8}{mj}为\end{CJK} $g(A)$ \begin{CJK}{UTF8}{mj}的伴随矩阵\end{CJK}.

(2) $g(A)$ \begin{CJK}{UTF8}{mj}不可逆时\end{CJK}, \begin{CJK}{UTF8}{mj}结论是否成立\end{CJK}?

\section{$2.232018$ 年}
\begin{enumerate}
  \item (15 \begin{CJK}{UTF8}{mj}分\end{CJK}) \begin{CJK}{UTF8}{mj}当实数\end{CJK} $a, d$ \begin{CJK}{UTF8}{mj}取何值时\end{CJK}, \begin{CJK}{UTF8}{mj}下列方程无解\end{CJK}、\begin{CJK}{UTF8}{mj}有唯一解\end{CJK}、\begin{CJK}{UTF8}{mj}有无穷多个解\end{CJK}? \begin{CJK}{UTF8}{mj}有解时\end{CJK}, \begin{CJK}{UTF8}{mj}求出所有\end{CJK} \begin{CJK}{UTF8}{mj}解\end{CJK}.
\end{enumerate}
$$
\left\{\begin{array}{l}
-x_{2}-2 x_{3}-2 x_{4}-6 x_{5}=a-3 \\
x_{1}-x_{3}-x_{4}+(d-5) x_{5}=-4 \\
2 x_{1}+2 x_{2}+2 x_{3}+2 x_{4}+(d+2) x_{5}=-a \\
2 x_{2}+4 x_{3}+4 x_{4}+12 x_{5}=-a+6
\end{array}\right.
$$

\begin{enumerate}
  \setcounter{enumi}{2}
  \item (10 \begin{CJK}{UTF8}{mj}分\end{CJK}) \begin{CJK}{UTF8}{mj}已知\end{CJK} 5 \begin{CJK}{UTF8}{mj}阶复方阵\end{CJK} $A$ \begin{CJK}{UTF8}{mj}的特征多项式为\end{CJK} $f_{A}(\lambda)$ \begin{CJK}{UTF8}{mj}与极小多项式\end{CJK} $m_{A}(\lambda)$ \begin{CJK}{UTF8}{mj}分别为\end{CJK}
\end{enumerate}
$$
f_{A}(\lambda)=(\lambda-1)^{3}(\lambda+2)^{2}, \quad m_{A}(\lambda)=(\lambda-1)^{2}(\lambda+2) .
$$
\begin{CJK}{UTF8}{mj}求\end{CJK} $A$ \begin{CJK}{UTF8}{mj}的\end{CJK} Jordan \begin{CJK}{UTF8}{mj}典范型\end{CJK}.

\begin{enumerate}
  \setcounter{enumi}{3}
  \item (10 \begin{CJK}{UTF8}{mj}分\end{CJK}) \begin{CJK}{UTF8}{mj}已知实二次型\end{CJK} $Q$ \begin{CJK}{UTF8}{mj}满足\end{CJK} $Q(\alpha)=0 \Leftrightarrow \alpha=0$. \begin{CJK}{UTF8}{mj}求证\end{CJK}: $Q$ \begin{CJK}{UTF8}{mj}或者正定或者负定\end{CJK}.

  \item (15 \begin{CJK}{UTF8}{mj}分\end{CJK}) \begin{CJK}{UTF8}{mj}设\end{CJK} $A=\frac{1}{3}\left(\begin{array}{ccc}2 & -1 & 2 \\ -1 & 2 & 2 \\ 2 & 2 & 1\end{array}\right)$. \begin{CJK}{UTF8}{mj}求一个正交矩阵\end{CJK} $P$ \begin{CJK}{UTF8}{mj}使得\end{CJK} $P^{T} A P$ \begin{CJK}{UTF8}{mj}为对角阵\end{CJK}, \begin{CJK}{UTF8}{mj}并写出该对角\end{CJK} \begin{CJK}{UTF8}{mj}阵\end{CJK}.

  \item (20 \begin{CJK}{UTF8}{mj}分\end{CJK}) (1) \begin{CJK}{UTF8}{mj}利用初等变换将下列矩阵化成简化的行阶梯形矩阵\end{CJK}.

\end{enumerate}
$$
\left(\begin{array}{ccccccc}
1 & 2 & -1 & 0 & 2 & 1 & 5 \\
-1 & -2 & 0 & 0 & 1 & -2 & -3 \\
1 & 2 & -3 & 0 & 5 & 1 & 6
\end{array}\right)
$$
(2) \begin{CJK}{UTF8}{mj}设\end{CJK} $V$ \begin{CJK}{UTF8}{mj}数域\end{CJK} $\mathbb{K}$ \begin{CJK}{UTF8}{mj}上的有限维线性空间\end{CJK}, \begin{CJK}{UTF8}{mj}给定他的一组基\end{CJK} $e_{1}, e_{2}, \ldots, e_{n}$. \begin{CJK}{UTF8}{mj}对于\end{CJK} $V$ \begin{CJK}{UTF8}{mj}中的一个非零向\end{CJK} \begin{CJK}{UTF8}{mj}量\end{CJK} $\alpha=\sum_{i=1}^{n} \lambda_{i} \alpha i$, \begin{CJK}{UTF8}{mj}若\end{CJK} $i$ \begin{CJK}{UTF8}{mj}是最小正整数使得\end{CJK} $\lambda_{i}$ \begin{CJK}{UTF8}{mj}不为\end{CJK} 0 , \begin{CJK}{UTF8}{mj}则称\end{CJK} $e_{i}$ \begin{CJK}{UTF8}{mj}为它的\end{CJK} tip, \begin{CJK}{UTF8}{mj}记为\end{CJK} $e_{i}=\operatorname{tip}(\alpha)$, \begin{CJK}{UTF8}{mj}对于\end{CJK} $V$ \begin{CJK}{UTF8}{mj}的一\end{CJK} \begin{CJK}{UTF8}{mj}个了空间\end{CJK} $W$, \begin{CJK}{UTF8}{mj}定义\end{CJK}
$$
\begin{gathered}
\operatorname{Tip}(W)=\{\operatorname{tip}(\alpha): \alpha \in W, \alpha \neq 0\} \\
\operatorname{NonTip}(W)=\left\{e_{1}, e_{2}, \ldots, e_{n}\right\}-\operatorname{Tip}(W) .
\end{gathered}
$$
\begin{CJK}{UTF8}{mj}现设\end{CJK} $v=\mathbb{K}^{7}$ \begin{CJK}{UTF8}{mj}是\end{CJK} 7 \begin{CJK}{UTF8}{mj}维行向量组成的空间\end{CJK}, \begin{CJK}{UTF8}{mj}取它的标准基\end{CJK} $e_{1}, e_{2}, \ldots, e_{7}$. \begin{CJK}{UTF8}{mj}令\end{CJK} $W$ \begin{CJK}{UTF8}{mj}为\end{CJK} (1) \begin{CJK}{UTF8}{mj}中矩阵的行向量张\end{CJK} \begin{CJK}{UTF8}{mj}成的子空间\end{CJK}. \begin{CJK}{UTF8}{mj}求\end{CJK} $\operatorname{Tip}(W)$ \begin{CJK}{UTF8}{mj}和\end{CJK} $\operatorname{NonTip}(W)$.

(3) \begin{CJK}{UTF8}{mj}设\end{CJK} $V$ \begin{CJK}{UTF8}{mj}是数域\end{CJK} $\mathbb{K}$ \begin{CJK}{UTF8}{mj}上的有限维线性空间\end{CJK},\begin{CJK}{UTF8}{mj}给定它的一组基\end{CJK} $e_{1}, e_{2}, \ldots, e_{n}$, \begin{CJK}{UTF8}{mj}设\end{CJK} $W$ \begin{CJK}{UTF8}{mj}是\end{CJK} $V$ \begin{CJK}{UTF8}{mj}的一个子空\end{CJK} \begin{CJK}{UTF8}{mj}间\end{CJK}. \begin{CJK}{UTF8}{mj}证\end{CJK}:
$$
V=W \oplus \operatorname{Span}_{k}(\operatorname{NonTip}(W))
$$
\begin{CJK}{UTF8}{mj}这里\end{CJK} $\operatorname{Span}_{k}(\operatorname{NonTip}(W))$ \begin{CJK}{UTF8}{mj}是\end{CJK} $\operatorname{NonTip}(W)$ \begin{CJK}{UTF8}{mj}张成的子空间\end{CJK}.

\begin{enumerate}
  \setcounter{enumi}{6}
  \item (15 \begin{CJK}{UTF8}{mj}分\end{CJK}) \begin{CJK}{UTF8}{mj}设\end{CJK} $f$ \begin{CJK}{UTF8}{mj}与\end{CJK} $g$ \begin{CJK}{UTF8}{mj}是从有限维线性空间\end{CJK} $U$ \begin{CJK}{UTF8}{mj}到有限维线性空间\end{CJK} $W$ \begin{CJK}{UTF8}{mj}的两个线性映射\end{CJK}. \begin{CJK}{UTF8}{mj}若\end{CJK} $\operatorname{Im}(f)=$ $\operatorname{Im}(g)$, \begin{CJK}{UTF8}{mj}这里\end{CJK} $\operatorname{Im}(f)$ \begin{CJK}{UTF8}{mj}是\end{CJK} $f$ \begin{CJK}{UTF8}{mj}的像\end{CJK}, \begin{CJK}{UTF8}{mj}证明\end{CJK}: \begin{CJK}{UTF8}{mj}存在\end{CJK} $U$ \begin{CJK}{UTF8}{mj}上的可逆线性变换\end{CJK} $h$, \begin{CJK}{UTF8}{mj}使得\end{CJK} $g=f \circ h$.

  \item (25 \begin{CJK}{UTF8}{mj}分\end{CJK}) \begin{CJK}{UTF8}{mj}设\end{CJK} $\mathbb{K}$ \begin{CJK}{UTF8}{mj}是一个数域\end{CJK}, $m, n$ \begin{CJK}{UTF8}{mj}为自然数\end{CJK}, $M_{m, n}(\mathbb{K}), M_{m}(\mathbb{K})$ \begin{CJK}{UTF8}{mj}分别是数域\end{CJK} $\mathbb{K}$ \begin{CJK}{UTF8}{mj}上\end{CJK} $m \times n$ \begin{CJK}{UTF8}{mj}阶与\end{CJK} $m$ \begin{CJK}{UTF8}{mj}阶\end{CJK} \begin{CJK}{UTF8}{mj}矩阵生成的空间\end{CJK}, $A$ \begin{CJK}{UTF8}{mj}是秩为\end{CJK} $r$ \begin{CJK}{UTF8}{mj}的\end{CJK} $m \times n$ \begin{CJK}{UTF8}{mj}阶矩阵\end{CJK}. \begin{CJK}{UTF8}{mj}定义\end{CJK}

\end{enumerate}
$$
f: M_{m}(\mathbb{K}) \rightarrow M_{m, n}(\mathbb{K}), \quad f(X)=X A .
$$
(1). \begin{CJK}{UTF8}{mj}证明\end{CJK}: $f$ \begin{CJK}{UTF8}{mj}是一个线性映射\end{CJK};

(2). \begin{CJK}{UTF8}{mj}设\end{CJK} $m=n=2, A=\left(\begin{array}{ll}1 & 2 \\ 2 & 4\end{array}\right)$, \begin{CJK}{UTF8}{mj}分别求\end{CJK} $f$ \begin{CJK}{UTF8}{mj}的核\end{CJK} $\operatorname{ker}(f)$ \begin{CJK}{UTF8}{mj}的一组基与\end{CJK} $f$ \begin{CJK}{UTF8}{mj}的像\end{CJK} $\operatorname{Im}(f)$ \begin{CJK}{UTF8}{mj}的一组基\end{CJK};

(3). \begin{CJK}{UTF8}{mj}对于任意的\end{CJK} $m, n, r$, \begin{CJK}{UTF8}{mj}求\end{CJK} $f$ \begin{CJK}{UTF8}{mj}的秩\end{CJK};

(4). \begin{CJK}{UTF8}{mj}对于任意的\end{CJK} $m, n, r$, \begin{CJK}{UTF8}{mj}求\end{CJK} $f$ \begin{CJK}{UTF8}{mj}的核\end{CJK} $\operatorname{ker}(f)$ \begin{CJK}{UTF8}{mj}的维数\end{CJK}.

\begin{enumerate}
  \setcounter{enumi}{8}
  \item (20 \begin{CJK}{UTF8}{mj}分\end{CJK}) \begin{CJK}{UTF8}{mj}设\end{CJK} $M_{k, n}$ \begin{CJK}{UTF8}{mj}是所有\end{CJK} $k \times n$ \begin{CJK}{UTF8}{mj}阶复矩阵的集合\end{CJK}, $N_{k}^{-}$\begin{CJK}{UTF8}{mj}是所有\end{CJK} $k$ \begin{CJK}{UTF8}{mj}阶下三角幂么方阵的集合\end{CJK}, $N_{k}^{+}$ \begin{CJK}{UTF8}{mj}是所有\end{CJK} $n$ \begin{CJK}{UTF8}{mj}阶上三角幕么方阵的集合\end{CJK}.\begin{CJK}{UTF8}{mj}这里的幂么矩阵是指对角线上全为\end{CJK} 1 \begin{CJK}{UTF8}{mj}的上三角或下三角\end{CJK}.\begin{CJK}{UTF8}{mj}在\end{CJK} $M_{k, n}$ \begin{CJK}{UTF8}{mj}中定义如下关系\end{CJK}
\end{enumerate}
$$
A \sim B \Leftrightarrow \exists P \in N_{k}^{-}, Q \in N_{k}^{+}, \text {s.t. } A=P B Q
$$
(1). \begin{CJK}{UTF8}{mj}求证\end{CJK} \begin{CJK}{UTF8}{mj}是\end{CJK} $M_{k, n}$ \begin{CJK}{UTF8}{mj}上的等价关系\end{CJK}.

(2). \begin{CJK}{UTF8}{mj}设\end{CJK} $r=\min \{k, n\}$, \begin{CJK}{UTF8}{mj}求证\end{CJK} $\Delta_{1}, \cdots, \Delta_{r}$ \begin{CJK}{UTF8}{mj}是上述等价关系的不变量\end{CJK}, \begin{CJK}{UTF8}{mj}也就是说\end{CJK}, \begin{CJK}{UTF8}{mj}两个满足该等价关\end{CJK} \begin{CJK}{UTF8}{mj}系的矩阵具有相同的\end{CJK} $\Delta_{1}, \cdots, \Delta_{r}$ \begin{CJK}{UTF8}{mj}值\end{CJK}, \begin{CJK}{UTF8}{mj}这里\end{CJK} $\Delta_{i}(i=1, \cdots, r)$ \begin{CJK}{UTF8}{mj}是矩阵的第\end{CJK} $i$ \begin{CJK}{UTF8}{mj}个顺序主子式\end{CJK}.

\begin{enumerate}
  \setcounter{enumi}{9}
  \item (20 \begin{CJK}{UTF8}{mj}分\end{CJK}) \begin{CJK}{UTF8}{mj}设\end{CJK} $\lambda_{1}, \cdots, \lambda_{n}$ \begin{CJK}{UTF8}{mj}是数域\end{CJK} $\mathbb{K}$ \begin{CJK}{UTF8}{mj}上的\end{CJK} $n$ \begin{CJK}{UTF8}{mj}个两两不同的数\end{CJK}, $V$ \begin{CJK}{UTF8}{mj}是\end{CJK} $\mathbb{K}$ \begin{CJK}{UTF8}{mj}上线性空间\end{CJK}, $\varphi$ \begin{CJK}{UTF8}{mj}是\end{CJK} $V$ \begin{CJK}{UTF8}{mj}上的线\end{CJK} \begin{CJK}{UTF8}{mj}性变换\end{CJK}, \begin{CJK}{UTF8}{mj}且它在基\end{CJK} $\xi_{1}, \cdots, \xi_{n}$ \begin{CJK}{UTF8}{mj}下的矩阵为对角矩阵\end{CJK} $A=\operatorname{diag}\left(\lambda_{1}, \cdots, \lambda_{n}\right)$.
\end{enumerate}
(1). \begin{CJK}{UTF8}{mj}设\end{CJK} $W$ \begin{CJK}{UTF8}{mj}是\end{CJK} $\varphi$ \begin{CJK}{UTF8}{mj}的不变子空间\end{CJK}, $x_{1} \xi_{1}+\cdots+x_{n} \xi_{n} \in W$, \begin{CJK}{UTF8}{mj}其中\end{CJK} $x_{1}, \cdots, x_{n} \in \mathbb{K}$, \begin{CJK}{UTF8}{mj}证明\end{CJK}: \begin{CJK}{UTF8}{mj}若某个\end{CJK} $x_{i}$ \begin{CJK}{UTF8}{mj}不为\end{CJK} 0 , \begin{CJK}{UTF8}{mj}则\end{CJK} $\xi_{i} \in W$.

(2). \begin{CJK}{UTF8}{mj}求\end{CJK} $\varphi$ \begin{CJK}{UTF8}{mj}的不变子空间个数\end{CJK}.

\section{$2.242019$ 年}
\begin{enumerate}
  \item (20 \begin{CJK}{UTF8}{mj}分\end{CJK}) $m \times n$ \begin{CJK}{UTF8}{mj}实矩阵\end{CJK} $A=\left(a_{i j}\right)$ \begin{CJK}{UTF8}{mj}以\end{CJK} $a_{11}$ \begin{CJK}{UTF8}{mj}为圆心逆时针旋转\end{CJK} $90^{\circ}$ \begin{CJK}{UTF8}{mj}得到矩阵\end{CJK} $B$.
\end{enumerate}
(1). \begin{CJK}{UTF8}{mj}求\end{CJK} $B$ \begin{CJK}{UTF8}{mj}的行数和列数\end{CJK}.

(2). $\operatorname{rank}(A)$ \begin{CJK}{UTF8}{mj}与\end{CJK} $\operatorname{rank}(B)$ \begin{CJK}{UTF8}{mj}的关系\end{CJK}, \begin{CJK}{UTF8}{mj}并解释原因\end{CJK}.

(3). \begin{CJK}{UTF8}{mj}设\end{CJK} $m=n,|A|$ \begin{CJK}{UTF8}{mj}与\end{CJK} $|B|$ \begin{CJK}{UTF8}{mj}的关系\end{CJK}? \begin{CJK}{UTF8}{mj}并证明\end{CJK}.

\begin{enumerate}
  \setcounter{enumi}{2}
  \item (20 \begin{CJK}{UTF8}{mj}分\end{CJK}) \begin{CJK}{UTF8}{mj}当实数\end{CJK} $\lambda$ \begin{CJK}{UTF8}{mj}取何值时\end{CJK}, \begin{CJK}{UTF8}{mj}下列方程无解\end{CJK}、\begin{CJK}{UTF8}{mj}有唯一解\end{CJK}、\begin{CJK}{UTF8}{mj}有无穷多个解\end{CJK}? \begin{CJK}{UTF8}{mj}有解时\end{CJK}, \begin{CJK}{UTF8}{mj}求出所有解\end{CJK}.
\end{enumerate}
$$
\begin{cases}\lambda x_{1}+x_{2}+x_{3} & =1, \\ \left(\lambda^{2}+1\right) x_{1}+2 \lambda x_{2}+(\lambda+1) x_{3} & =\lambda+1 \\ x_{1}+x_{2}+\lambda x_{3} & =1 \\ 2 x_{1}+(\lambda+1) x_{2}+(\lambda+1) x_{3} & =2 .\end{cases}
$$

\begin{enumerate}
  \setcounter{enumi}{3}
  \item (20 \begin{CJK}{UTF8}{mj}分\end{CJK}) \begin{CJK}{UTF8}{mj}已知矩阵\end{CJK} $\left(\begin{array}{ccc}0 & 1 & -1 \\ -2 & -3 & a \\ 3 & 3 & -4\end{array}\right)$ \begin{CJK}{UTF8}{mj}的特征多项式有一重根\end{CJK}, \begin{CJK}{UTF8}{mj}求\end{CJK} $a$ \begin{CJK}{UTF8}{mj}的值\end{CJK}, \begin{CJK}{UTF8}{mj}并讨论可否对角化\end{CJK}.

  \item (15 \begin{CJK}{UTF8}{mj}分\end{CJK}) \begin{CJK}{UTF8}{mj}已知\end{CJK} 2019 \begin{CJK}{UTF8}{mj}阶实对称矩阵\end{CJK} $A$ \begin{CJK}{UTF8}{mj}满足\end{CJK} $A^{2}=2019 A$, \begin{CJK}{UTF8}{mj}证明\end{CJK}: $E+A+\cdots+A^{2019}$ \begin{CJK}{UTF8}{mj}为正定矩阵\end{CJK}.

  \item (20 \begin{CJK}{UTF8}{mj}分\end{CJK}) \begin{CJK}{UTF8}{mj}已知\end{CJK} $\mathscr{A}: V \rightarrow V$ \begin{CJK}{UTF8}{mj}是有限维复线性空间\end{CJK} $V$ \begin{CJK}{UTF8}{mj}上的线性变换\end{CJK}. \begin{CJK}{UTF8}{mj}设\end{CJK} $v \in V$, \begin{CJK}{UTF8}{mj}存在\end{CJK} $f(\lambda) \in \mathbb{C}[\lambda]$ \begin{CJK}{UTF8}{mj}使得\end{CJK} $f(\mathscr{A})(v)=0$, \begin{CJK}{UTF8}{mj}则称\end{CJK} $f(\lambda)$ \begin{CJK}{UTF8}{mj}为\end{CJK} $\mathscr{A}$ \begin{CJK}{UTF8}{mj}对\end{CJK} $v$ \begin{CJK}{UTF8}{mj}的零化多项式\end{CJK}.

\end{enumerate}
(1). \begin{CJK}{UTF8}{mj}证明\end{CJK}: $\mathscr{A}$ \begin{CJK}{UTF8}{mj}对\end{CJK} $v$ \begin{CJK}{UTF8}{mj}的非零零化多项式存在\end{CJK}.

(2). $\mathscr{A}$ \begin{CJK}{UTF8}{mj}对\end{CJK} $v$ \begin{CJK}{UTF8}{mj}的次数最低的首项系数为\end{CJK} 1 \begin{CJK}{UTF8}{mj}的零化多项式称为极小多项式\end{CJK}, \begin{CJK}{UTF8}{mj}记为\end{CJK} $m_{\mathscr{A}, v}(\lambda)$. \begin{CJK}{UTF8}{mj}证明\end{CJK}: \begin{CJK}{UTF8}{mj}零化多项式均能被\end{CJK} $m_{\mathscr{A}, v}(\lambda)$ \begin{CJK}{UTF8}{mj}整除\end{CJK}.

(3). \begin{CJK}{UTF8}{mj}记\end{CJK} $\mathscr{A}$ \begin{CJK}{UTF8}{mj}的极小多项式为\end{CJK} $m_{\mathscr{A}}(\lambda)$, \begin{CJK}{UTF8}{mj}证明\end{CJK}: \begin{CJK}{UTF8}{mj}存在\end{CJK} $v \in V$ \begin{CJK}{UTF8}{mj}使得\end{CJK} $m_{\mathscr{A}, v}(\lambda)=m_{\mathscr{A}}(\lambda)$.

\begin{enumerate}
  \setcounter{enumi}{6}
  \item (15 \begin{CJK}{UTF8}{mj}分\end{CJK}) \begin{CJK}{UTF8}{mj}证明\end{CJK}: \begin{CJK}{UTF8}{mj}任意复二阶矩阵\end{CJK} $A, B, C$ \begin{CJK}{UTF8}{mj}满足\end{CJK}
\end{enumerate}
$$
\left[[A, B]^{2}, C\right]=0
$$
\begin{CJK}{UTF8}{mj}其中\end{CJK} $[X, Y]=X Y-Y X$.

\begin{enumerate}
  \setcounter{enumi}{7}
  \item (20 \begin{CJK}{UTF8}{mj}分\end{CJK}) \begin{CJK}{UTF8}{mj}记\end{CJK} $V_{n}(n \geqslant 0)$ \begin{CJK}{UTF8}{mj}为次数不大于\end{CJK} $n$ \begin{CJK}{UTF8}{mj}的关于\end{CJK} $x, y$ \begin{CJK}{UTF8}{mj}的实系数二元多项式生成的空间\end{CJK}. \begin{CJK}{UTF8}{mj}求\end{CJK} $V_{2}$ \begin{CJK}{UTF8}{mj}上\end{CJK} \begin{CJK}{UTF8}{mj}线性变换\end{CJK}
\end{enumerate}
$$
\mathscr{A}=2 \frac{\partial}{\partial x}+\frac{\partial}{\partial y}
$$
\begin{CJK}{UTF8}{mj}的\end{CJK} Jordan \begin{CJK}{UTF8}{mj}标准型\end{CJK}, \begin{CJK}{UTF8}{mj}并推广到一般情形\end{CJK}.

\begin{enumerate}
  \setcounter{enumi}{8}
  \item (20 \begin{CJK}{UTF8}{mj}分\end{CJK}) $G L_{2}(\mathbb{C})$ \begin{CJK}{UTF8}{mj}为\end{CJK} 2 \begin{CJK}{UTF8}{mj}阶可逆复矩阵集合\end{CJK}, $V$ \begin{CJK}{UTF8}{mj}是迹为\end{CJK} 0 \begin{CJK}{UTF8}{mj}的\end{CJK} 2 \begin{CJK}{UTF8}{mj}阶复矩阵构成的复线性空间\end{CJK}. \begin{CJK}{UTF8}{mj}若\end{CJK} $V$ \begin{CJK}{UTF8}{mj}的一个线性子空间\end{CJK} $W$ \begin{CJK}{UTF8}{mj}满足\end{CJK}: $\forall P \in G L_{2}(\mathbb{C})$ \begin{CJK}{UTF8}{mj}与\end{CJK} $\forall A \in W$, \begin{CJK}{UTF8}{mj}总有\end{CJK} $P^{-1} A P$ \begin{CJK}{UTF8}{mj}落在\end{CJK} $W$ \begin{CJK}{UTF8}{mj}中\end{CJK}, \begin{CJK}{UTF8}{mj}称\end{CJK} $W$ \begin{CJK}{UTF8}{mj}为\end{CJK} $G L_{2}(\mathbb{C})-$ \begin{CJK}{UTF8}{mj}不\end{CJK} \begin{CJK}{UTF8}{mj}变子空间\end{CJK}. \begin{CJK}{UTF8}{mj}求证\end{CJK}: $V$ \begin{CJK}{UTF8}{mj}的\end{CJK} $G L_{2}(\mathbb{C})-$ \begin{CJK}{UTF8}{mj}不变子空间只有零空间和\end{CJK} $V$.
\end{enumerate}
\section{$2.252020$ 年}
\begin{enumerate}
  \item (15 \begin{CJK}{UTF8}{mj}分\end{CJK}) \begin{CJK}{UTF8}{mj}设\end{CJK} $A=\left(\begin{array}{ccc}-2 & 0 & -1 \\ 1 & 2 & b \\ a & 2 / 3 & 0\end{array}\right)$, \begin{CJK}{UTF8}{mj}求所有\end{CJK} $a, b$ \begin{CJK}{UTF8}{mj}的值\end{CJK}, \begin{CJK}{UTF8}{mj}使得\end{CJK} $A$ \begin{CJK}{UTF8}{mj}是幂零矩阵\end{CJK}. (\begin{CJK}{UTF8}{mj}矩阵\end{CJK} $A$ \begin{CJK}{UTF8}{mj}称为幕零矩\end{CJK} \begin{CJK}{UTF8}{mj}阵是指存在正整数\end{CJK} $k$ \begin{CJK}{UTF8}{mj}使得\end{CJK} $A^{k}=0$.)

  \item (15 \begin{CJK}{UTF8}{mj}分\end{CJK}) \begin{CJK}{UTF8}{mj}设\end{CJK} $\alpha_{1}, \alpha_{2}, \ldots, \alpha_{n}, \beta_{1}, \beta_{2}, \ldots, \beta_{n}$, \begin{CJK}{UTF8}{mj}是线性空间\end{CJK} $V$ \begin{CJK}{UTF8}{mj}中的\end{CJK} $2 n$ \begin{CJK}{UTF8}{mj}个向量\end{CJK}. \begin{CJK}{UTF8}{mj}已知对任意的\end{CJK} $1 \leqslant k \leqslant n$ \begin{CJK}{UTF8}{mj}以及\end{CJK} $1 \leqslant i_{1}<\cdots<i_{k} \leqslant n, \alpha_{i_{1}}, \alpha_{i_{2}}, \ldots, \alpha_{i_{k}}$ \begin{CJK}{UTF8}{mj}线性相关当且仅当\end{CJK} $\beta_{i_{1}}, \beta_{i_{2}}, \ldots, \beta_{i_{k}}$ \begin{CJK}{UTF8}{mj}线性相关\end{CJK}. \begin{CJK}{UTF8}{mj}求证向量组\end{CJK} $\alpha_{1}, \alpha_{2}, \ldots, \alpha_{n}$ \begin{CJK}{UTF8}{mj}的秩与向量组\end{CJK} $\beta_{1}, \beta_{2}, \ldots, \beta_{n}$ \begin{CJK}{UTF8}{mj}的秩相同\end{CJK}.

  \item (15 \begin{CJK}{UTF8}{mj}分\end{CJK}) \begin{CJK}{UTF8}{mj}已知\end{CJK} $n \geqslant 2, a, b \in \mathbb{C}$. \begin{CJK}{UTF8}{mj}求矩阵\end{CJK} $A=\left(\begin{array}{cccccc}a & a & a & a & a & \cdots \\ a & b & b & b & b & \cdots \\ a & b & a & a & a & \cdots \\ a & b & a & b & b & \cdots \\ a & b & a & b & a & \cdots \\ \vdots & \vdots & \vdots & \vdots & \vdots & \ddots\end{array}\right)_{n \times n}$ \begin{CJK}{UTF8}{mj}的行列式\end{CJK}.

  \item (25 \begin{CJK}{UTF8}{mj}分\end{CJK}) \begin{CJK}{UTF8}{mj}设\end{CJK} $A=\left(\begin{array}{ccc}\frac{7}{4} & -\frac{3}{4} & \frac{\sqrt{6}}{4} \\ -\frac{3}{4} & \frac{7}{4} & -\frac{\sqrt{6}}{4} \\ \frac{\sqrt{6}}{4} & -\frac{\sqrt{6}}{4} & \frac{7}{2}\end{array}\right)$.

\end{enumerate}
(1). \begin{CJK}{UTF8}{mj}求一个正交矩阵\end{CJK} $P$, \begin{CJK}{UTF8}{mj}使得\end{CJK} $P^{T} A P$ \begin{CJK}{UTF8}{mj}是对角矩阵\end{CJK}.

(2). \begin{CJK}{UTF8}{mj}求\end{CJK} $f\left(x_{1}, x_{2}, x_{3}\right)=\frac{7}{4} x_{1}^{2}+\frac{7}{4} x_{2}^{2}+\frac{7}{2} x_{3}^{2}-\frac{3}{2} x_{1} x_{2}-\frac{\sqrt{6}}{2} x_{2} x_{3}+\frac{\sqrt{6}}{2} x_{1} x_{3}$ \begin{CJK}{UTF8}{mj}在单位球面\end{CJK} $S^{2}=\left\{\left(x_{1}, x_{2}, x_{3}\right) \in\right.$ $\left.\mathbb{R}^{3} \mid x_{1}^{2}+x_{2}^{2}+x_{3}^{2}=1\right\}$ \begin{CJK}{UTF8}{mj}上能取到的最大值\end{CJK}, \begin{CJK}{UTF8}{mj}并求出能取到该最大值的所有\end{CJK} $\left(x_{1}, x_{2}, x_{3}\right)$.

\begin{enumerate}
  \setcounter{enumi}{5}
  \item (15 \begin{CJK}{UTF8}{mj}分\end{CJK}) \begin{CJK}{UTF8}{mj}已知矩阵\end{CJK} $A \in M_{n}(\mathbb{C})$ \begin{CJK}{UTF8}{mj}满足\end{CJK}
\end{enumerate}
$$
A+A^{2}+\frac{1}{2 !} A^{3}+\frac{1}{3 !} A^{4}+\cdots+\frac{1}{2019 !} A^{2020}=0
$$
\begin{CJK}{UTF8}{mj}求证\end{CJK}: $A$ \begin{CJK}{UTF8}{mj}可对角化\end{CJK}.

\begin{enumerate}
  \setcounter{enumi}{6}
  \item $\left(20\right.$ \begin{CJK}{UTF8}{mj}分\end{CJK}) \begin{CJK}{UTF8}{mj}设\end{CJK} $A, B \in M_{n}(\mathbb{C})$, \begin{CJK}{UTF8}{mj}令\end{CJK} $L(A, B)=\left\{X \in M_{n}(\mathbb{C}) \mid A X B=0\right\}$.
\end{enumerate}
(1). \begin{CJK}{UTF8}{mj}验证\end{CJK} $L(A, B)$ \begin{CJK}{UTF8}{mj}是\end{CJK} $M_{n}(\mathbb{C})$ \begin{CJK}{UTF8}{mj}的线性子空间\end{CJK}.

(2). \begin{CJK}{UTF8}{mj}设\end{CJK} $\operatorname{rank}(A)=r, \operatorname{rank}(B)=s$. \begin{CJK}{UTF8}{mj}求\end{CJK} $\operatorname{dim} L(A, B)$. (\begin{CJK}{UTF8}{mj}用\end{CJK} $n, r, s$ \begin{CJK}{UTF8}{mj}表示\end{CJK}).

\begin{enumerate}
  \setcounter{enumi}{7}
  \item (15 \begin{CJK}{UTF8}{mj}分\end{CJK}) \begin{CJK}{UTF8}{mj}设\end{CJK} $A, B, C$ \begin{CJK}{UTF8}{mj}是二阶复方阵\end{CJK}, \begin{CJK}{UTF8}{mj}且\end{CJK} $A, B, C$ \begin{CJK}{UTF8}{mj}在\end{CJK} $M_{2}(\mathbb{C})$ \begin{CJK}{UTF8}{mj}中线性无关\end{CJK}. \begin{CJK}{UTF8}{mj}求证\end{CJK}: \begin{CJK}{UTF8}{mj}存在复数\end{CJK} $x_{1}, x_{2}, x_{3}$ \begin{CJK}{UTF8}{mj}使得\end{CJK} $x_{1} A+x_{2} B+x_{3} C$ \begin{CJK}{UTF8}{mj}是可逆矩阵\end{CJK}.

  \item (20 \begin{CJK}{UTF8}{mj}分\end{CJK}) (1). \begin{CJK}{UTF8}{mj}设\end{CJK} $A \in M_{n}(\mathbb{C})$. \begin{CJK}{UTF8}{mj}求证\end{CJK}:\begin{CJK}{UTF8}{mj}若存在可逆矩阵\end{CJK} $B \in M_{n}(\mathbb{C})$, \begin{CJK}{UTF8}{mj}使得\end{CJK} $A=B^{-1} \bar{B}$, \begin{CJK}{UTF8}{mj}则\end{CJK} $A^{-1}=\bar{A}$.

\end{enumerate}
(2). \begin{CJK}{UTF8}{mj}设可逆矩阵\end{CJK} $A \in M_{n}$ ( $\left.\mathbb{C}\right)$ \begin{CJK}{UTF8}{mj}满足\end{CJK} $A^{-1}=\bar{A}$. \begin{CJK}{UTF8}{mj}求证\end{CJK}: \begin{CJK}{UTF8}{mj}存在可逆矩阵\end{CJK} $B \in\{a \bar{A}+b E \mid a, b \in \mathbb{C}\}$ \begin{CJK}{UTF8}{mj}使得\end{CJK} $A=B^{-1} \bar{B}$. ( $\bar{A}$ \begin{CJK}{UTF8}{mj}为\end{CJK} $A$ \begin{CJK}{UTF8}{mj}的共轭矩阵\end{CJK}, $E$ \begin{CJK}{UTF8}{mj}是指单位矩阵\end{CJK}. $)$

\begin{enumerate}
  \setcounter{enumi}{9}
  \item (10 \begin{CJK}{UTF8}{mj}分\end{CJK}) \begin{CJK}{UTF8}{mj}设\end{CJK} $n$ \begin{CJK}{UTF8}{mj}为奇数\end{CJK}, $A, B \in M_{n}(\mathbb{C})$ \begin{CJK}{UTF8}{mj}且\end{CJK} $A^{2}=0$, \begin{CJK}{UTF8}{mj}求证\end{CJK}: $A B-B A$ \begin{CJK}{UTF8}{mj}不可逆\end{CJK}.
\end{enumerate}
\section{$2.262021$ 年}
\begin{enumerate}
  \item (15 \begin{CJK}{UTF8}{mj}分\end{CJK}) \begin{CJK}{UTF8}{mj}设\end{CJK} $A \in \mathbb{M}_{m \times n}, \beta \in \mathbb{M}_{m \times 1}$, \begin{CJK}{UTF8}{mj}问\end{CJK}: \begin{CJK}{UTF8}{mj}线性方程组\end{CJK} $A x=\beta$ \begin{CJK}{UTF8}{mj}有多少个线性无关的解\end{CJK} (\begin{CJK}{UTF8}{mj}用\end{CJK} $r=\operatorname{rank}(A)$ \begin{CJK}{UTF8}{mj}表示\end{CJK} $)$,\begin{CJK}{UTF8}{mj}并说明理由\end{CJK}.

  \item (15 \begin{CJK}{UTF8}{mj}分\end{CJK}) \begin{CJK}{UTF8}{mj}设\end{CJK} $2 n$ \begin{CJK}{UTF8}{mj}阶方阵\end{CJK} $S=\left(\begin{array}{cc}0 & E_{n} \\ -E_{n} & 0\end{array}\right)$, \begin{CJK}{UTF8}{mj}给出复线性空间\end{CJK}

\end{enumerate}
$$
S P_{n}=\left\{X \in \mathbb{M}_{2 n \times 2 n}(\mathbb{C}) \mid S X=-X^{T} S\right\}
$$
\begin{CJK}{UTF8}{mj}的一组基\end{CJK}, \begin{CJK}{UTF8}{mj}并计算其维数\end{CJK}.

\begin{enumerate}
  \setcounter{enumi}{3}
  \item (15 \begin{CJK}{UTF8}{mj}分\end{CJK}) \begin{CJK}{UTF8}{mj}设\end{CJK} $n$ \begin{CJK}{UTF8}{mj}阶矩阵\end{CJK} $A(t)=\left(a_{i j}(t)\right)_{n \times n}$ \begin{CJK}{UTF8}{mj}中元素\end{CJK} $\left(a_{i j}(t)\right.$ \begin{CJK}{UTF8}{mj}为实变量\end{CJK} $t$ \begin{CJK}{UTF8}{mj}的可微函数\end{CJK}. \begin{CJK}{UTF8}{mj}记\end{CJK} $A^{\prime}(t)=$ $\left(\frac{\mathrm{d}}{\mathrm{dt}} a_{i j}(t)\right)_{n \times n}$. \begin{CJK}{UTF8}{mj}证明\end{CJK}: \begin{CJK}{UTF8}{mj}若对\end{CJK} $\forall t \in \mathbb{R},|A(t)|>0$, \begin{CJK}{UTF8}{mj}则\end{CJK}
\end{enumerate}
$$
\frac{\mathrm{d}}{\mathrm{dt}} \ln |A(t)|=\operatorname{tr}\left(A^{-1}(t) A^{\prime}(t)\right) .
$$

\begin{enumerate}
  \setcounter{enumi}{4}
  \item (15 \begin{CJK}{UTF8}{mj}分\end{CJK}) \begin{CJK}{UTF8}{mj}证明\end{CJK}: \begin{CJK}{UTF8}{mj}若\end{CJK} $n$ \begin{CJK}{UTF8}{mj}阶复矩阵\end{CJK} $A, B$ \begin{CJK}{UTF8}{mj}满足\end{CJK} $A B=B A$, \begin{CJK}{UTF8}{mj}且\end{CJK} $B$ \begin{CJK}{UTF8}{mj}有\end{CJK} $n$ \begin{CJK}{UTF8}{mj}个不同的特征值\end{CJK}, \begin{CJK}{UTF8}{mj}则\end{CJK} $A$ \begin{CJK}{UTF8}{mj}可对角化\end{CJK}.

  \item (15 \begin{CJK}{UTF8}{mj}分\end{CJK}) \begin{CJK}{UTF8}{mj}设\end{CJK} $c_{1}, c_{2}, c_{3}$ \begin{CJK}{UTF8}{mj}是多项式\end{CJK} $f(x)=2 x^{3}-4 x^{2}+6 x-1$ \begin{CJK}{UTF8}{mj}的三个复根\end{CJK}. \begin{CJK}{UTF8}{mj}求\end{CJK}

\end{enumerate}
$$
\left(c_{1} c_{2}+c_{3}^{2}\right)\left(c_{2} c_{3}+c_{1}^{2}\right)\left(c_{1} c_{3}+c_{2}^{2}\right) .
$$

\begin{enumerate}
  \setcounter{enumi}{6}
  \item (20 \begin{CJK}{UTF8}{mj}分\end{CJK}) \begin{CJK}{UTF8}{mj}令\end{CJK} $f(x, y)=a_{11} x^{2}+2 a_{12} x y+a_{22} y^{2}+2 b_{1} x+2 b_{2} y+c$,
\end{enumerate}
$$
A_{f}=\left(\begin{array}{ll}
a_{11} & a_{12} \\
a_{12} & a_{22}
\end{array}\right),, \quad B_{f}=\left(\begin{array}{lll}
a_{11} & a_{12} & b_{1} \\
a_{12} & a_{22} & b_{2} \\
b_{1} & b_{2} & c
\end{array}\right)
$$
\begin{CJK}{UTF8}{mj}证明\end{CJK}: \begin{CJK}{UTF8}{mj}函数\end{CJK} $f$ \begin{CJK}{UTF8}{mj}在坐标变换\end{CJK} $\left(\begin{array}{c}x^{\prime} \\ y^{\prime}\end{array}\right)=Q\left(\begin{array}{c}x \\ y\end{array}\right)+\left(\begin{array}{c}d_{1} \\ d_{2}\end{array}\right)$ \begin{CJK}{UTF8}{mj}下对应的\end{CJK} $\operatorname{Tr}\left(A_{f}\right), \operatorname{det}\left(A_{f}\right), \operatorname{det}\left(B_{f}\right)$ \begin{CJK}{UTF8}{mj}保持不变\end{CJK}, \begin{CJK}{UTF8}{mj}其中\end{CJK} $Q$ \begin{CJK}{UTF8}{mj}是二阶正交矩阵\end{CJK}.

\begin{enumerate}
  \setcounter{enumi}{7}
  \item (20 \begin{CJK}{UTF8}{mj}分\end{CJK}) \begin{CJK}{UTF8}{mj}设实矩阵\end{CJK} $A=\left(\begin{array}{ll}a & b \\ c & d\end{array}\right), a, b, c, d>0$. \begin{CJK}{UTF8}{mj}证明\end{CJK}: \begin{CJK}{UTF8}{mj}一定存在\end{CJK} $A$ \begin{CJK}{UTF8}{mj}的特征向量\end{CJK} $(x, y)^{\top} \in \mathbb{R}^{2}$ \begin{CJK}{UTF8}{mj}满\end{CJK} \begin{CJK}{UTF8}{mj}足\end{CJK} $x, y>0$.

  \item (15 \begin{CJK}{UTF8}{mj}分\end{CJK}) \begin{CJK}{UTF8}{mj}证明\end{CJK}: \begin{CJK}{UTF8}{mj}若\end{CJK} 6 \begin{CJK}{UTF8}{mj}阶复矩阵\end{CJK} $A, B$ \begin{CJK}{UTF8}{mj}是幂零矩阵\end{CJK}, \begin{CJK}{UTF8}{mj}且有相同的秩和最小多项式\end{CJK}, \begin{CJK}{UTF8}{mj}则\end{CJK} $A, B$ \begin{CJK}{UTF8}{mj}相似\end{CJK}.

  \item (20 \begin{CJK}{UTF8}{mj}分\end{CJK}) \begin{CJK}{UTF8}{mj}设\end{CJK} $A$ \begin{CJK}{UTF8}{mj}是\end{CJK} $n$ \begin{CJK}{UTF8}{mj}阶实矩阵\end{CJK}, $B$ \begin{CJK}{UTF8}{mj}是\end{CJK} $n$ \begin{CJK}{UTF8}{mj}阶实对称正定矩阵\end{CJK}. \begin{CJK}{UTF8}{mj}证明\end{CJK}:

\end{enumerate}
(1) \begin{CJK}{UTF8}{mj}存在唯\end{CJK} $n$ \begin{CJK}{UTF8}{mj}阶实矩阵\end{CJK} $C$ \begin{CJK}{UTF8}{mj}满足\end{CJK} $B C+C B=A$;

(2) \begin{CJK}{UTF8}{mj}对\end{CJK} (1) \begin{CJK}{UTF8}{mj}中实矩阵\end{CJK} $C$, \begin{CJK}{UTF8}{mj}有\end{CJK} $B C=C B$ \begin{CJK}{UTF8}{mj}当且仅当\end{CJK} $A B=B A$.

\section{$2.272022$ 年}
\begin{enumerate}
  \item (20 \begin{CJK}{UTF8}{mj}分\end{CJK}) \begin{CJK}{UTF8}{mj}考虑数域\end{CJK} $\mathbb{K}$ \begin{CJK}{UTF8}{mj}上的线性方程组\end{CJK}
\end{enumerate}
$$
\left\{\begin{array}{l}
x_{1}+x_{2}+2 a x_{3}=2 \\
x_{1}+3 b x_{2}+x_{3}=2 \\
x_{1}+x_{2}-a x_{3}=1
\end{array}\right.
$$
\begin{CJK}{UTF8}{mj}问在\end{CJK} $a, b$ \begin{CJK}{UTF8}{mj}取何值时\end{CJK}, \begin{CJK}{UTF8}{mj}方程组无解\end{CJK}, \begin{CJK}{UTF8}{mj}有唯一解\end{CJK}, \begin{CJK}{UTF8}{mj}有无穷多组解\end{CJK}. \begin{CJK}{UTF8}{mj}且在方程组有解时\end{CJK}, \begin{CJK}{UTF8}{mj}求出所有解\end{CJK}.

\begin{enumerate}
  \setcounter{enumi}{2}
  \item (20 \begin{CJK}{UTF8}{mj}分\end{CJK}) \begin{CJK}{UTF8}{mj}设\end{CJK} 3 \begin{CJK}{UTF8}{mj}阶实对称矩阵\end{CJK} $A$ \begin{CJK}{UTF8}{mj}的秩为\end{CJK} 2 , \begin{CJK}{UTF8}{mj}且\end{CJK} $-2$ \begin{CJK}{UTF8}{mj}是它的二重特征值\end{CJK}, \begin{CJK}{UTF8}{mj}若\end{CJK} $(1,0,0)^{\top},(2,1,1)^{\top}$ \begin{CJK}{UTF8}{mj}都是\end{CJK} $A$ \begin{CJK}{UTF8}{mj}的属于特征值\end{CJK} $-2$ \begin{CJK}{UTF8}{mj}的特征向量\end{CJK}, \begin{CJK}{UTF8}{mj}求矩阵\end{CJK} $A$.

  \item (20 \begin{CJK}{UTF8}{mj}分\end{CJK}) \begin{CJK}{UTF8}{mj}考虑末定元为\end{CJK} $x$ \begin{CJK}{UTF8}{mj}和\end{CJK} $y$ \begin{CJK}{UTF8}{mj}的次数至多为\end{CJK} 2 \begin{CJK}{UTF8}{mj}的复系数二元多项式空间\end{CJK}. \begin{CJK}{UTF8}{mj}求线性变换\end{CJK}

\end{enumerate}
$$
\mathscr{A}: f(x, y) \rightarrow f(2 x+1,2 y+1)
$$
\begin{CJK}{UTF8}{mj}的\end{CJK} Jordan \begin{CJK}{UTF8}{mj}标准型\end{CJK}.

\begin{enumerate}
  \setcounter{enumi}{4}
  \item (15 \begin{CJK}{UTF8}{mj}分\end{CJK}) \begin{CJK}{UTF8}{mj}设\end{CJK} $\sigma$ \begin{CJK}{UTF8}{mj}是有限维欧式空间\end{CJK} $V$ \begin{CJK}{UTF8}{mj}上的正交变换\end{CJK}, \begin{CJK}{UTF8}{mj}且满足\end{CJK} $\sigma^{m}=I$, \begin{CJK}{UTF8}{mj}其中\end{CJK} $m$ \begin{CJK}{UTF8}{mj}为大于\end{CJK} 1 \begin{CJK}{UTF8}{mj}的整数\end{CJK}, $I$ \begin{CJK}{UTF8}{mj}是恒等变换\end{CJK}. \begin{CJK}{UTF8}{mj}记\end{CJK} $V^{\sigma}=\{\theta \in V: \sigma(v)=v\}$, \begin{CJK}{UTF8}{mj}而\end{CJK} $V^{\sigma}$ \begin{CJK}{UTF8}{mj}的正交补记为\end{CJK} $V^{\sigma \perp}$.
\end{enumerate}
(a). \begin{CJK}{UTF8}{mj}求证\end{CJK} $V^{\sigma \perp}$ \begin{CJK}{UTF8}{mj}是\end{CJK} $\sigma$-\begin{CJK}{UTF8}{mj}不变子空间\end{CJK}.

(b). \begin{CJK}{UTF8}{mj}对于\end{CJK} $v \in V$, \begin{CJK}{UTF8}{mj}定义\end{CJK} $\bar{v}=\frac{1}{m} \sum_{i=1}^{m} \sigma^{i}(v)$. \begin{CJK}{UTF8}{mj}求证\end{CJK}: $\bar{v} \in V^{\sigma}$.

(c). \begin{CJK}{UTF8}{mj}证明\end{CJK}: \begin{CJK}{UTF8}{mj}若将\end{CJK} $v \in V$ \begin{CJK}{UTF8}{mj}展开成\end{CJK} $v=v_{1}+v_{2}$, \begin{CJK}{UTF8}{mj}其中\end{CJK} $v_{1} \in V^{\sigma}, v_{2} \in V^{\sigma \perp}$, \begin{CJK}{UTF8}{mj}则\end{CJK} $v_{1}=\bar{v}$.

\begin{enumerate}
  \setcounter{enumi}{5}
  \item (10 \begin{CJK}{UTF8}{mj}分\end{CJK}) \begin{CJK}{UTF8}{mj}设\end{CJK} $f(x)$ \begin{CJK}{UTF8}{mj}是次数大于\end{CJK} 0 \begin{CJK}{UTF8}{mj}的整系数多项式\end{CJK}, \begin{CJK}{UTF8}{mj}若\end{CJK} $2-\sqrt{3}$ \begin{CJK}{UTF8}{mj}是\end{CJK} $f(x)$ \begin{CJK}{UTF8}{mj}的根\end{CJK}, \begin{CJK}{UTF8}{mj}证明\end{CJK}: $2+\sqrt{3}$ \begin{CJK}{UTF8}{mj}也是\end{CJK} $f(x)$ \begin{CJK}{UTF8}{mj}的根\end{CJK}.
\end{enumerate}
$$
\operatorname{Im}\left(\mathscr{A}^{k}\right)=\operatorname{Im}\left(\mathscr{A}^{k+1}\right)=\cdots=\operatorname{Im}\left(\mathscr{A}^{n}\right) \text { 且 } \operatorname{ker}\left(\mathscr{A}^{k}\right)=\operatorname{ker}\left(\mathscr{A}^{k+1}\right)=\cdots=\operatorname{ker}\left(\mathscr{A}^{n}\right) \text {. }
$$
(i) $\mathscr{A}$ \begin{CJK}{UTF8}{mj}的秩\end{CJK} $r$;

(ii) $\mathscr{A}$ \begin{CJK}{UTF8}{mj}的特征值为\end{CJK} 0 \begin{CJK}{UTF8}{mj}的\end{CJK} Jordan \begin{CJK}{UTF8}{mj}块个数\end{CJK} $m$;

\begin{CJK}{UTF8}{mj}讨论这些数值之间的关系\end{CJK}, \begin{CJK}{UTF8}{mj}并证明你的结论\end{CJK}.
$$
<\phi(\phi(x)), y>=<x, \phi(y)>, \quad \forall x, y \in V
$$
\begin{CJK}{UTF8}{mj}证明\end{CJK}: $\phi$ \begin{CJK}{UTF8}{mj}是正交变换\end{CJK}.

\begin{enumerate}
  \setcounter{enumi}{8}
  \item (15 \begin{CJK}{UTF8}{mj}分\end{CJK}) \begin{CJK}{UTF8}{mj}设\end{CJK} $U, V, W$ \begin{CJK}{UTF8}{mj}是\end{CJK} 6 \begin{CJK}{UTF8}{mj}维线性空间的\end{CJK} 3 \begin{CJK}{UTF8}{mj}个\end{CJK} 3 \begin{CJK}{UTF8}{mj}维子空间\end{CJK}, \begin{CJK}{UTF8}{mj}设\end{CJK} $U \cap V=0$, \begin{CJK}{UTF8}{mj}求\end{CJK} $\operatorname{dim}((U+V) \cap(V+W))$ \begin{CJK}{UTF8}{mj}的是大值和最小值\end{CJK}.
\end{enumerate}
$$
A=\left(\begin{array}{ll}
A_{1} & A_{2} \\
A_{2}^{\top} & A_{4}
\end{array}\right)
$$
\begin{CJK}{UTF8}{mj}其中\end{CJK} $A_{1}$ \begin{CJK}{UTF8}{mj}是\end{CJK} $r$ \begin{CJK}{UTF8}{mj}阶方阵\end{CJK}. \begin{CJK}{UTF8}{mj}证明\end{CJK}: \begin{CJK}{UTF8}{mj}对\end{CJK} $x \in \mathbb{R}^{r}$, \begin{CJK}{UTF8}{mj}若\end{CJK} $A_{1} x=0$, \begin{CJK}{UTF8}{mj}则\end{CJK} $A_{2}^{\top} x=0$.

(c). \begin{CJK}{UTF8}{mj}设\end{CJK} $A, B$ \begin{CJK}{UTF8}{mj}是\end{CJK} $n$ \begin{CJK}{UTF8}{mj}阶半正定实对称矩阵\end{CJK}, \begin{CJK}{UTF8}{mj}且\end{CJK} $\operatorname{rank}(A)=r$. \begin{CJK}{UTF8}{mj}证明\end{CJK}: \begin{CJK}{UTF8}{mj}存在\end{CJK} $n$ \begin{CJK}{UTF8}{mj}阶可逆矩阵\end{CJK} $P$, \begin{CJK}{UTF8}{mj}使得\end{CJK}
$$
P^{-1} A\left(P^{-1}\right)^{\top}=\left(\begin{array}{cc}
I_{r} & 0 \\
0 & 0
\end{array}\right), \quad P^{\top} B P=\operatorname{diag}\left\{\lambda_{1}, \lambda_{2}, \cdots, \lambda_{n}\right\} .
$$

\section{第1章 北京大学}
\section{$1.12020$ 年数学分析考研真题}
\section{考生须知:}
\begin{enumerate}
  \item \begin{CJK}{UTF8}{mj}本试卷满分为\end{CJK} 150 \begin{CJK}{UTF8}{mj}分\end{CJK}, \begin{CJK}{UTF8}{mj}全部考试时间总计\end{CJK} 180 \begin{CJK}{UTF8}{mj}分钟\end{CJK};

  \item \begin{CJK}{UTF8}{mj}所有答案必须写在答题纸上\end{CJK}, \begin{CJK}{UTF8}{mj}写在试题纸上或草稿纸上一律无效\end{CJK}。

  \item (15 \begin{CJK}{UTF8}{mj}分\end{CJK}) \begin{CJK}{UTF8}{mj}定义在\end{CJK} $[a, b]$ \begin{CJK}{UTF8}{mj}上的函数\end{CJK} $f(x)$ \begin{CJK}{UTF8}{mj}满足\end{CJK}: \begin{CJK}{UTF8}{mj}任取\end{CJK} $x_{0} \in[a, b]$, \begin{CJK}{UTF8}{mj}均有\end{CJK} $\lim _{x \rightarrow x_{0}} \sup f(x) \leq f\left(x_{0}\right)$, \begin{CJK}{UTF8}{mj}试\end{CJK} \begin{CJK}{UTF8}{mj}问\end{CJK} $f(x)$ \begin{CJK}{UTF8}{mj}在\end{CJK} $[a, b]$ \begin{CJK}{UTF8}{mj}上是否有最大值\end{CJK}? \begin{CJK}{UTF8}{mj}若有给出证明\end{CJK}; \begin{CJK}{UTF8}{mj}若没有\end{CJK}, \begin{CJK}{UTF8}{mj}举出反例\end{CJK}.

  \item (15 \begin{CJK}{UTF8}{mj}分\end{CJK}) \begin{CJK}{UTF8}{mj}判断\end{CJK} $f(x)=\frac{x}{1+x \cos ^{2} x}$ \begin{CJK}{UTF8}{mj}在\end{CJK} $[0,+\infty)$ \begin{CJK}{UTF8}{mj}是否一致连续\end{CJK}? \begin{CJK}{UTF8}{mj}并说明理由\end{CJK}.

  \item (15 \begin{CJK}{UTF8}{mj}分\end{CJK}) \begin{CJK}{UTF8}{mj}已知函数\end{CJK} $f(x)$ \begin{CJK}{UTF8}{mj}在\end{CJK} $[1,+\infty)$ \begin{CJK}{UTF8}{mj}上连续且满足\end{CJK}: \begin{CJK}{UTF8}{mj}对任意的\end{CJK} $x, y \in[1,+\infty)$ \begin{CJK}{UTF8}{mj}有\end{CJK}

\end{enumerate}
$$
f(x+y) \leq f(x)+f(y)
$$
\begin{CJK}{UTF8}{mj}问\end{CJK}: $\lim _{x \rightarrow+\infty} \frac{f(x)}{x}$ \begin{CJK}{UTF8}{mj}是否存在\end{CJK}?\begin{CJK}{UTF8}{mj}若存在\end{CJK}, \begin{CJK}{UTF8}{mj}给出证明\end{CJK}; \begin{CJK}{UTF8}{mj}若不存在\end{CJK}, \begin{CJK}{UTF8}{mj}举出反例\end{CJK}.

\begin{enumerate}
  \setcounter{enumi}{4}
  \item (15 \begin{CJK}{UTF8}{mj}分\end{CJK}) \begin{CJK}{UTF8}{mj}已知函数\end{CJK} $f(x)$ \begin{CJK}{UTF8}{mj}在\end{CJK} $[0,1]$ \begin{CJK}{UTF8}{mj}上连续单调增加\end{CJK}, \begin{CJK}{UTF8}{mj}且\end{CJK} $f(x) \geq 0$, \begin{CJK}{UTF8}{mj}记\end{CJK}
\end{enumerate}
$$
s=\frac{\int_{0}^{1} x f(x) \mathrm{d} x}{\int_{0}^{1} f(x) \mathrm{d} x}
$$
(1). (7 \begin{CJK}{UTF8}{mj}分\end{CJK}) \begin{CJK}{UTF8}{mj}证明\end{CJK}: $s \geq \frac{1}{2}$;

(2). (8 \begin{CJK}{UTF8}{mj}分\end{CJK}) \begin{CJK}{UTF8}{mj}比较\end{CJK} $\int_{0}^{s} f(x) \mathrm{d} x$ \begin{CJK}{UTF8}{mj}与\end{CJK} $\int_{s}^{1} f(x) \mathrm{d} x$ \begin{CJK}{UTF8}{mj}的大小\end{CJK} (\begin{CJK}{UTF8}{mj}可以用物理或几何直觉\end{CJK} $)$.

\begin{enumerate}
  \setcounter{enumi}{5}
  \item (15 \begin{CJK}{UTF8}{mj}分\end{CJK}) \begin{CJK}{UTF8}{mj}根据\end{CJK} $\int_{0}^{+\infty} \frac{\sin x}{x} \mathrm{~d} x=\frac{\pi}{2}$, \begin{CJK}{UTF8}{mj}计算\end{CJK} $\int_{0}^{+\infty}\left(\frac{\sin x}{x}\right)^{2} \mathrm{~d} x$, \begin{CJK}{UTF8}{mj}并说明计算依据\end{CJK}.

  \item (15 \begin{CJK}{UTF8}{mj}分\end{CJK}) \begin{CJK}{UTF8}{mj}在承认平面\end{CJK} Green \begin{CJK}{UTF8}{mj}公式的前提下证明如下特殊情况下的\end{CJK} Stokes \begin{CJK}{UTF8}{mj}公式\end{CJK}

\end{enumerate}
$$
\oint_{\Gamma} R(x, y, z) \mathrm{d} z=\iint_{\Sigma} \frac{\partial R}{\partial y} \mathrm{~d} y \mathrm{~d} z-\frac{\partial R}{\partial x} \mathrm{~d} z \mathrm{~d} x
$$

\begin{enumerate}
  \setcounter{enumi}{7}
  \item $(20$ \begin{CJK}{UTF8}{mj}分\end{CJK} $)$
\end{enumerate}
(1). (10 \begin{CJK}{UTF8}{mj}分\end{CJK}) \begin{CJK}{UTF8}{mj}设\end{CJK} $0<p<1$, \begin{CJK}{UTF8}{mj}求\end{CJK} $f(x)=\cos p x$ \begin{CJK}{UTF8}{mj}在\end{CJK} $[-\pi, \pi]$ \begin{CJK}{UTF8}{mj}上的\end{CJK} Fourier \begin{CJK}{UTF8}{mj}级数\end{CJK},\begin{CJK}{UTF8}{mj}并求出其和函数\end{CJK}; (2). (10 \begin{CJK}{UTF8}{mj}分\end{CJK}) \begin{CJK}{UTF8}{mj}证明余元公式\end{CJK}
$$
\int_{0}^{1} x^{p-1}(1-x)^{-p} \mathrm{~d} x=\frac{\pi}{\sin (p \pi)}
$$

\begin{enumerate}
  \setcounter{enumi}{8}
  \item $\left(20\right.$ \begin{CJK}{UTF8}{mj}分\end{CJK}) \begin{CJK}{UTF8}{mj}设\end{CJK} $C_{r}$ \begin{CJK}{UTF8}{mj}为半径为\end{CJK} $r$ \begin{CJK}{UTF8}{mj}的圆周\end{CJK}, $f(x, y)$ \begin{CJK}{UTF8}{mj}满足\end{CJK} $f(0,0)=0, \frac{\partial^{2} f}{\partial x^{2}}+\frac{\partial^{2} f}{\partial y^{2}}=x^{2}+y^{2}, f(x, y)$ \begin{CJK}{UTF8}{mj}是\end{CJK} $\mathbb{C}^{2}$ \begin{CJK}{UTF8}{mj}的\end{CJK}, \begin{CJK}{UTF8}{mj}计算\end{CJK}
\end{enumerate}
$$
A(r)=\int_{C_{r}} f(x, y) \mathrm{d} S
$$

\begin{enumerate}
  \setcounter{enumi}{9}
  \item (20 \begin{CJK}{UTF8}{mj}分\end{CJK}) \begin{CJK}{UTF8}{mj}设\end{CJK} $q_{k} \geq p_{k}>0, q_{k+1}-q_{k} \geq p_{k}+p_{k+1}$ \begin{CJK}{UTF8}{mj}且\end{CJK} $\sum_{k=1}^{\infty} a_{k} \ln p_{k}=+\infty, \quad$ \begin{CJK}{UTF8}{mj}记\end{CJK}
\end{enumerate}
$$
\begin{aligned}
T_{p_{k}, q_{k}}(x) & \triangleq \frac{\cos \left(q_{k}+p_{k}\right) x}{p_{k}}+\frac{\cos \left(q_{k}+p_{k}-1\right) x}{p_{k}-1}+\cdots+\frac{\cos \left(q_{k}+1\right) x}{1} \\
&-\frac{\cos \left(q_{k}-1\right) x}{1}-\frac{\cos \left(q_{k}-2\right) x}{2}-\cdots-\frac{\cos \left(q_{k}-p_{k}\right) x}{p_{k}}
\end{aligned}
$$
\begin{CJK}{UTF8}{mj}设\end{CJK} $a_{k} \geq 0, \sum_{k=1}^{\infty} a_{k}<+\infty, f(x)=\sum_{k=1}^{\infty} a_{k} T_{p_{k}}, q_{k}(x)$.

(1). \begin{CJK}{UTF8}{mj}证明\end{CJK}: $f(x)$ \begin{CJK}{UTF8}{mj}是在\end{CJK} $\mathbb{R}$ \begin{CJK}{UTF8}{mj}上连续\end{CJK}, \begin{CJK}{UTF8}{mj}且以\end{CJK} $2 \pi$ \begin{CJK}{UTF8}{mj}为周期的周期函数\end{CJK}.

(2). \begin{CJK}{UTF8}{mj}判断并证明\end{CJK}: $f(x)$ \begin{CJK}{UTF8}{mj}的\end{CJK} Fourier \begin{CJK}{UTF8}{mj}级数在\end{CJK} $x=0$ \begin{CJK}{UTF8}{mj}处的收敛性\end{CJK}.

\section{$1.22020$ 年高等代数考研真题}
\section{考生须知:}
\begin{enumerate}
  \item \begin{CJK}{UTF8}{mj}本试卷满分为\end{CJK} 150 \begin{CJK}{UTF8}{mj}分\end{CJK}, \begin{CJK}{UTF8}{mj}全部考试时间总计\end{CJK} 180 \begin{CJK}{UTF8}{mj}分钟\end{CJK};

  \item \begin{CJK}{UTF8}{mj}所有答案必须写在答题纸上\end{CJK}, \begin{CJK}{UTF8}{mj}写在试题纸上或草稿纸上一律无效\end{CJK}。

  \item (15 \begin{CJK}{UTF8}{mj}分\end{CJK}) \begin{CJK}{UTF8}{mj}设\end{CJK} $V_{0}=\{0\}, V_{1}, V_{2}, \cdots, V_{n}=\{0\}$ \begin{CJK}{UTF8}{mj}是\end{CJK} $n+1$ \begin{CJK}{UTF8}{mj}有限维线性空间\end{CJK},\begin{CJK}{UTF8}{mj}定义线性变换\end{CJK} $\varphi_{i}: V_{i} \rightarrow$ $V_{i+1}, i=1,2, \cdots, n-1$, \begin{CJK}{UTF8}{mj}若对\end{CJK} $i=0,1,2, \cdots, n-1$ \begin{CJK}{UTF8}{mj}均有\end{CJK} $\operatorname{Ker} \varphi_{i+1}=\operatorname{Im} \varphi_{i}$ \begin{CJK}{UTF8}{mj}中\end{CJK}, \begin{CJK}{UTF8}{mj}证明\end{CJK}:

\end{enumerate}
$$
\sum_{i=0}^{n}(-1)^{i} \mathrm{~V}_{\mathrm{i}}=0
$$

\begin{enumerate}
  \setcounter{enumi}{2}
  \item (15 \begin{CJK}{UTF8}{mj}分\end{CJK}) \begin{CJK}{UTF8}{mj}设\end{CJK} $c_{0}, c_{1}, \cdots, c_{k+1}$ \begin{CJK}{UTF8}{mj}是\end{CJK} $k+1$ \begin{CJK}{UTF8}{mj}个复数\end{CJK}, \begin{CJK}{UTF8}{mj}证明\end{CJK}: \begin{CJK}{UTF8}{mj}存在唯一一个次数不超过\end{CJK} $k$ \begin{CJK}{UTF8}{mj}的复系数多项\end{CJK} \begin{CJK}{UTF8}{mj}式函数\end{CJK} $p(x)$ \begin{CJK}{UTF8}{mj}使得\end{CJK} $p(0)=c_{0}, p(1)=c_{1}, \cdots, p(k)=c_{k}$, \begin{CJK}{UTF8}{mj}且这样的多项式是唯一的\end{CJK}.

  \item (20 \begin{CJK}{UTF8}{mj}分\end{CJK}) \begin{CJK}{UTF8}{mj}设\end{CJK} $A$ \begin{CJK}{UTF8}{mj}是秩为\end{CJK} $r$ \begin{CJK}{UTF8}{mj}的实对称矩阵\end{CJK}, \begin{CJK}{UTF8}{mj}证明\end{CJK}:\begin{CJK}{UTF8}{mj}必存在一个非零的\end{CJK} $r$ \begin{CJK}{UTF8}{mj}阶主子式使得它的行列式\end{CJK} \begin{CJK}{UTF8}{mj}非零\end{CJK}, \begin{CJK}{UTF8}{mj}并且任意一个非零的\end{CJK} $r$ \begin{CJK}{UTF8}{mj}阶主子式符号相同\end{CJK}. 4. (20 \begin{CJK}{UTF8}{mj}分\end{CJK}) \begin{CJK}{UTF8}{mj}设\end{CJK} $n$ \begin{CJK}{UTF8}{mj}阶方阵\end{CJK} $A=\left(a_{i j}\right)_{n \times n}$ \begin{CJK}{UTF8}{mj}可相似对角化\end{CJK}, \begin{CJK}{UTF8}{mj}它的特征值为\end{CJK} $\lambda_{1}, \lambda_{2}, \cdots, \lambda_{n}$, \begin{CJK}{UTF8}{mj}每个特征值\end{CJK} $\lambda_{i}$ \begin{CJK}{UTF8}{mj}的特征子空间都由一族特征向量\end{CJK} $\alpha_{i j_{1}}, \cdots, \alpha_{i j_{n}}$ \begin{CJK}{UTF8}{mj}张成\end{CJK}, \begin{CJK}{UTF8}{mj}设\end{CJK} $A^{*}=\left(A_{i j}\right)_{n \times n}$ \begin{CJK}{UTF8}{mj}是\end{CJK} $a_{i j}$ \begin{CJK}{UTF8}{mj}对应的代数\end{CJK} \begin{CJK}{UTF8}{mj}余子式\end{CJK}, \begin{CJK}{UTF8}{mj}求\end{CJK} $A^{*}$ \begin{CJK}{UTF8}{mj}的特征值和特征向量\end{CJK}.

  \item (15 \begin{CJK}{UTF8}{mj}分\end{CJK}) \begin{CJK}{UTF8}{mj}设\end{CJK} $\varphi$ \begin{CJK}{UTF8}{mj}是一个线性变换\end{CJK}, $\lambda_{1}, \lambda_{2}, \cdots, \lambda_{k}$ \begin{CJK}{UTF8}{mj}是\end{CJK} $\varphi$ \begin{CJK}{UTF8}{mj}的特征值\end{CJK}, \begin{CJK}{UTF8}{mj}证明\end{CJK}: $\varphi$ \begin{CJK}{UTF8}{mj}可对角化的充分必要\end{CJK} \begin{CJK}{UTF8}{mj}条件是对\end{CJK} $\varphi$ \begin{CJK}{UTF8}{mj}的每一个特征值入\end{CJK} $\lambda$, \begin{CJK}{UTF8}{mj}均有\end{CJK}

\end{enumerate}
$$
\operatorname{dim}(\operatorname{Im}(\lambda i d-\varphi))=\operatorname{dim}\left(\operatorname{Im}(\lambda i d-\varphi)^{2}\right)
$$
\begin{CJK}{UTF8}{mj}其中\end{CJK} $i d$ \begin{CJK}{UTF8}{mj}为恒等变换\end{CJK}.

\begin{enumerate}
  \setcounter{enumi}{6}
  \item (15 \begin{CJK}{UTF8}{mj}分\end{CJK}) \begin{CJK}{UTF8}{mj}设\end{CJK} $n$ \begin{CJK}{UTF8}{mj}是欧氏空间\end{CJK} $V$ \begin{CJK}{UTF8}{mj}中的单位向量\end{CJK}, \begin{CJK}{UTF8}{mj}定义镜像变换\end{CJK} $\sigma: \sigma(\alpha)=\alpha-2(\alpha, \eta) \eta$, \begin{CJK}{UTF8}{mj}其中\end{CJK} $(-,-)$ \begin{CJK}{UTF8}{mj}表示内积\end{CJK}.
\end{enumerate}
(1). \begin{CJK}{UTF8}{mj}证明\end{CJK}: $\sigma$ \begin{CJK}{UTF8}{mj}为正交变换\end{CJK};

(2). \begin{CJK}{UTF8}{mj}证明\end{CJK}: $V$ \begin{CJK}{UTF8}{mj}的任意正交变换都可以表示成若干镜像变换的乘积\end{CJK}.

\begin{enumerate}
  \setcounter{enumi}{7}
  \item (15 \begin{CJK}{UTF8}{mj}分\end{CJK}) \begin{CJK}{UTF8}{mj}已知向量了\end{CJK} $|\vec{u}|,|\vec{v}|,|\vec{w}|$, \begin{CJK}{UTF8}{mj}或满足\end{CJK} $|\vec{u}|=|\vec{v}|=|\vec{w}|>0, \vec{u} \cdot \vec{v}=\vec{v} \cdot \vec{w}=\vec{w} \cdot \vec{u}$, \begin{CJK}{UTF8}{mj}若对任意\end{CJK} \begin{CJK}{UTF8}{mj}非零向量\end{CJK} $|\vec{x}|$, \begin{CJK}{UTF8}{mj}均存在实数\end{CJK} $a, b, c$, \begin{CJK}{UTF8}{mj}使得\end{CJK} $\vec{x} \times \vec{u}=a \vec{u}+b \vec{v}+c \vec{w}, \vec{x} \times \vec{v}=a \vec{v}+b \vec{w}+c \vec{u}$, \begin{CJK}{UTF8}{mj}证明\end{CJK}:
\end{enumerate}
\includegraphics[max width=\textwidth]{2022_04_18_7db0708508f26638f054g-020}

\begin{enumerate}
  \setcounter{enumi}{8}
  \item (20 \begin{CJK}{UTF8}{mj}分\end{CJK}) \begin{CJK}{UTF8}{mj}设平面直角坐标系下二次曲线的方程为\end{CJK}
\end{enumerate}
$$
x^{2}+2 y^{2}+6 x y+8 x+10 y+6=0
$$
(1). \begin{CJK}{UTF8}{mj}证明\end{CJK}: $\gamma$ \begin{CJK}{UTF8}{mj}是双曲线\end{CJK}.

(2). \begin{CJK}{UTF8}{mj}求的长半轴和短半轴的方程与长轴和短轴长\end{CJK}, \begin{CJK}{UTF8}{mj}并且说明哪条与\end{CJK} $\gamma$ \begin{CJK}{UTF8}{mj}相交\end{CJK}.

\begin{enumerate}
  \setcounter{enumi}{9}
  \item (20 \begin{CJK}{UTF8}{mj}分\end{CJK}) \begin{CJK}{UTF8}{mj}求粗圆\end{CJK} $x^{2}+8 y^{2}+4 x y+10 x+12 y+4=0$ \begin{CJK}{UTF8}{mj}的内接三角形的面积的最大值\end{CJK}.
\end{enumerate}
\section{第 2 章 中国科学院大学}
\section{$2.12020$ 年数学分析考研真题}
\section{考生须知:}
\begin{enumerate}
  \item \begin{CJK}{UTF8}{mj}本试卷满分为\end{CJK} 150 \begin{CJK}{UTF8}{mj}分\end{CJK}, \begin{CJK}{UTF8}{mj}全部考试时间总计\end{CJK} 180 \begin{CJK}{UTF8}{mj}分钟\end{CJK};

  \item \begin{CJK}{UTF8}{mj}所有答案必须写在答题纸上\end{CJK}, \begin{CJK}{UTF8}{mj}写在试题纸上或草稿纸上一律无效\end{CJK}。

  \item (15 \begin{CJK}{UTF8}{mj}分\end{CJK}) \begin{CJK}{UTF8}{mj}计算极限\end{CJK}

\end{enumerate}
$$
\lim _{x \rightarrow 0} \frac{\sqrt[5]{1+3 x^{4}}-\sqrt{1-2 x}}{\sqrt[3]{1+x}-\sqrt{1+x}}
$$

\begin{enumerate}
  \setcounter{enumi}{2}
  \item (15 \begin{CJK}{UTF8}{mj}分\end{CJK}) \begin{CJK}{UTF8}{mj}设\end{CJK}
\end{enumerate}
$$
x_{0}=\alpha, \quad x_{1}=\beta, \quad x_{n+1}=\frac{2}{3} x_{n}+\frac{1}{3} x_{n-1} \quad(n \geq 1)
$$
\begin{CJK}{UTF8}{mj}证明\end{CJK}:\begin{CJK}{UTF8}{mj}数列\end{CJK} $\left\{x_{n}\right\}$ \begin{CJK}{UTF8}{mj}收敛\end{CJK}, \begin{CJK}{UTF8}{mj}并求出极限值\end{CJK}

\begin{enumerate}
  \setcounter{enumi}{3}
  \item (15 \begin{CJK}{UTF8}{mj}分\end{CJK}) \begin{CJK}{UTF8}{mj}判断下列极限是否存在\end{CJK}, \begin{CJK}{UTF8}{mj}并说明理由\end{CJK}:
\end{enumerate}
$$
\lim _{x \rightarrow 0} \frac{1}{x} \int_{0}^{\sin x} \sin \frac{1}{t} \cos t^{2} \mathrm{~d} t
$$

\begin{enumerate}
  \setcounter{enumi}{4}
  \item (15 \begin{CJK}{UTF8}{mj}分\end{CJK}) \begin{CJK}{UTF8}{mj}设函数\end{CJK} $f(x)$ \begin{CJK}{UTF8}{mj}在区间\end{CJK} $[0, n](n$ \begin{CJK}{UTF8}{mj}是一个正整数\end{CJK} $)$ \begin{CJK}{UTF8}{mj}上连续\end{CJK}, \begin{CJK}{UTF8}{mj}并且\end{CJK} $f(0)=f(n)$. \begin{CJK}{UTF8}{mj}证明\end{CJK}:\begin{CJK}{UTF8}{mj}存在\end{CJK} \begin{CJK}{UTF8}{mj}点\end{CJK} $x_{0} \in[0, n-1]$, \begin{CJK}{UTF8}{mj}使得\end{CJK} $f\left(x_{0}=f\left(x_{0}+1\right)\right)$.

  \item (15 \begin{CJK}{UTF8}{mj}分\end{CJK})\\
(1). $I_{1}=\int_{0}^{1} \frac{x^{b}-x^{a}}{\ln x} \mathrm{~d} x(a, b>0)$;\\
(2). $I_{2}=\int_{0}^{1} \frac{\ln (1+x)}{1+x^{2}} \mathrm{~d} x$.

  \item (15 \begin{CJK}{UTF8}{mj}分\end{CJK}) \begin{CJK}{UTF8}{mj}设\end{CJK} $\mathscr{D}$ \begin{CJK}{UTF8}{mj}是\end{CJK} $O x y$ \begin{CJK}{UTF8}{mj}平面上由曲线\end{CJK} $y=\sqrt{x}$ \begin{CJK}{UTF8}{mj}和直线\end{CJK} $y=x$ \begin{CJK}{UTF8}{mj}所围成的图形\end{CJK}, \begin{CJK}{UTF8}{mj}求\end{CJK} $\mathscr{D}$ \begin{CJK}{UTF8}{mj}绕直线\end{CJK} $y=x$ \begin{CJK}{UTF8}{mj}旋转产生的旋转体体积\end{CJK}.

  \item (15 \begin{CJK}{UTF8}{mj}分\end{CJK}) \begin{CJK}{UTF8}{mj}求函数\end{CJK} $f(x, y, z)=\ln x+\ln y+3 \ln z$ \begin{CJK}{UTF8}{mj}在球面\end{CJK} $x^{2}+y^{2}+z^{2}=5 R^{2}(x, y, z>0)$ \begin{CJK}{UTF8}{mj}上的\end{CJK} \begin{CJK}{UTF8}{mj}最大值\end{CJK}.

  \item (15 \begin{CJK}{UTF8}{mj}分\end{CJK}) \begin{CJK}{UTF8}{mj}证明\end{CJK}:

\end{enumerate}
$$
\frac{\sqrt{3}}{2} \pi<\int_{0}^{1} \sqrt{\frac{x^{2}-x+1}{x-x^{2}}} \mathrm{~d} x<\pi
$$

\begin{enumerate}
  \setcounter{enumi}{9}
  \item (15 \begin{CJK}{UTF8}{mj}分\end{CJK}) \begin{CJK}{UTF8}{mj}讨论级数\end{CJK} $\sum_{n=1}^{\infty} \frac{1+(-1)^{n}}{n} x^{n}$ \begin{CJK}{UTF8}{mj}的敛散性\end{CJK}.

  \item (15 \begin{CJK}{UTF8}{mj}分\end{CJK}) \begin{CJK}{UTF8}{mj}证明\end{CJK}:

\end{enumerate}
$$
\left|\int_{100}^{200} \frac{x^{3}}{x^{4}+x-1} \mathrm{~d} x-\ln 2\right|<\frac{1}{3} \cdot 10^{-6}
$$

\section{$2.22020$ 年高等代数考研真题}
\section{考生须知:}
\begin{enumerate}
  \item \begin{CJK}{UTF8}{mj}本试卷满分为\end{CJK} 150 \begin{CJK}{UTF8}{mj}分\end{CJK}, \begin{CJK}{UTF8}{mj}全部考试时间总计\end{CJK} 180 \begin{CJK}{UTF8}{mj}分钟\end{CJK};

  \item \begin{CJK}{UTF8}{mj}所有答案必须写在答题纸上\end{CJK},\begin{CJK}{UTF8}{mj}写在试题纸上或草稿纸上一律无效\end{CJK}。

  \item (20 \begin{CJK}{UTF8}{mj}分\end{CJK}) \begin{CJK}{UTF8}{mj}若整系数多项式\end{CJK} $f(x)$ \begin{CJK}{UTF8}{mj}有根\end{CJK} $\frac{p}{q}$, \begin{CJK}{UTF8}{mj}其中\end{CJK} $p, q$ \begin{CJK}{UTF8}{mj}为互萦的整数\end{CJK}, \begin{CJK}{UTF8}{mj}证明\end{CJK}:

\end{enumerate}
(1). $q-p|f(1), q+p| f(-1)$.

(2). \begin{CJK}{UTF8}{mj}对任意整数\end{CJK} $m$, \begin{CJK}{UTF8}{mj}有\end{CJK} $(m q-p) \mid f(m)$.

\begin{enumerate}
  \setcounter{enumi}{2}
  \item (18 \begin{CJK}{UTF8}{mj}分\end{CJK}) \begin{CJK}{UTF8}{mj}记\end{CJK} $|M|$ \begin{CJK}{UTF8}{mj}为矩阵\end{CJK} $M$ \begin{CJK}{UTF8}{mj}的行列式\end{CJK}, \begin{CJK}{UTF8}{mj}证明下列结论\end{CJK}:
\end{enumerate}
(1). \begin{CJK}{UTF8}{mj}已知\end{CJK} $A, B$ \begin{CJK}{UTF8}{mj}为实\end{CJK} $n$ \begin{CJK}{UTF8}{mj}阶方阵\end{CJK}, \begin{CJK}{UTF8}{mj}证明\end{CJK}
$$
\left|\begin{array}{cc}
A & B \\
-B & A
\end{array}\right|=|A+\sqrt{-1} B| \cdot|A-\sqrt{-1} B|
$$
$(2)$. \begin{CJK}{UTF8}{mj}设\end{CJK} $A$ \begin{CJK}{UTF8}{mj}是\end{CJK} $m \times n$ \begin{CJK}{UTF8}{mj}矩阵\end{CJK}, $B$ \begin{CJK}{UTF8}{mj}是\end{CJK} $n \times n$ \begin{CJK}{UTF8}{mj}矩阵\end{CJK}, $I_{k}$ \begin{CJK}{UTF8}{mj}表示\end{CJK} $k$ \begin{CJK}{UTF8}{mj}阶单位矩阵\end{CJK}, \begin{CJK}{UTF8}{mj}则\end{CJK}
$$
\lambda^{n}\left|\lambda I_{m}-A B\right|=\lambda^{m}\left|\lambda I_{n}-B A\right|
$$
\begin{CJK}{UTF8}{mj}其中\end{CJK} $\lambda$ \begin{CJK}{UTF8}{mj}是复数\end{CJK}.

\begin{enumerate}
  \setcounter{enumi}{3}
  \item (18 \begin{CJK}{UTF8}{mj}分\end{CJK}) \begin{CJK}{UTF8}{mj}已知\end{CJK} $n$ \begin{CJK}{UTF8}{mj}阶方阵\end{CJK} $A$ \begin{CJK}{UTF8}{mj}满足\end{CJK} $A^{2}=I_{n}$, \begin{CJK}{UTF8}{mj}试问\end{CJK}:
\end{enumerate}
$$
r\left(I_{n}+A\right)+r\left(I_{n}-A\right)=?
$$
\begin{CJK}{UTF8}{mj}并证明结论\end{CJK}.

\begin{enumerate}
  \setcounter{enumi}{4}
  \item (18 \begin{CJK}{UTF8}{mj}分\end{CJK}) \begin{CJK}{UTF8}{mj}设\end{CJK} $A$ \begin{CJK}{UTF8}{mj}是\end{CJK} $n$ \begin{CJK}{UTF8}{mj}阶实对称正定矩阵\end{CJK}, $B$ \begin{CJK}{UTF8}{mj}是\end{CJK} $n$ \begin{CJK}{UTF8}{mj}阶实对称半正定矩阵\end{CJK}.
\end{enumerate}
(1). \begin{CJK}{UTF8}{mj}证明\end{CJK}: $|A+B| \geq|A|+|B|$

(2). \begin{CJK}{UTF8}{mj}当\end{CJK} $n \geq 2$ \begin{CJK}{UTF8}{mj}时\end{CJK}, \begin{CJK}{UTF8}{mj}试问\end{CJK}: \begin{CJK}{UTF8}{mj}在什么条件下有\end{CJK} $|A+B|>|A|+|B|$, \begin{CJK}{UTF8}{mj}并证明\end{CJK}.

\begin{enumerate}
  \setcounter{enumi}{5}
  \item (18 \begin{CJK}{UTF8}{mj}分\end{CJK}) \begin{CJK}{UTF8}{mj}设\end{CJK} $n$ \begin{CJK}{UTF8}{mj}阶复方阵\end{CJK} $A$ \begin{CJK}{UTF8}{mj}的全部特征值为\end{CJK} $\lambda_{1}, \lambda_{2}, \cdots, \lambda_{n}$, \begin{CJK}{UTF8}{mj}求\end{CJK} $A$ \begin{CJK}{UTF8}{mj}的伴随矩阵\end{CJK} $A^{*}$ \begin{CJK}{UTF8}{mj}的全部特征值\end{CJK}. 6. (18 \begin{CJK}{UTF8}{mj}分\end{CJK}) \begin{CJK}{UTF8}{mj}已知实对称矩阵\end{CJK}
\end{enumerate}
$$
A=\left(\begin{array}{ccc}
2 & 2 & -2 \\
2 & 5 & -4 \\
-2 & -4 & 5
\end{array}\right)
$$
(1). \begin{CJK}{UTF8}{mj}求正交矩阵\end{CJK} $Q$, \begin{CJK}{UTF8}{mj}使得\end{CJK} $Q^{-1} A Q$ \begin{CJK}{UTF8}{mj}为对角阵\end{CJK};

(2). \begin{CJK}{UTF8}{mj}求解矩阵方程\end{CJK} $X^{2}=A$.

\begin{enumerate}
  \setcounter{enumi}{7}
  \item (18 \begin{CJK}{UTF8}{mj}分\end{CJK}) \begin{CJK}{UTF8}{mj}设\end{CJK} $\lambda$ \begin{CJK}{UTF8}{mj}是非零复数\end{CJK}, $k$ \begin{CJK}{UTF8}{mj}是正整数\end{CJK}, $J_{n}(\lambda)$ \begin{CJK}{UTF8}{mj}表示特征值为\end{CJK} $\lambda$ \begin{CJK}{UTF8}{mj}的\end{CJK} $n$ \begin{CJK}{UTF8}{mj}阶若尔当块\end{CJK}.
\end{enumerate}
(1). \begin{CJK}{UTF8}{mj}求\end{CJK} $\left(J_{n}(\lambda)\right)^{k}$ \begin{CJK}{UTF8}{mj}的若当标准型\end{CJK};

(2). \begin{CJK}{UTF8}{mj}证明\end{CJK}: $J_{n}(\lambda)$ \begin{CJK}{UTF8}{mj}有\end{CJK} $k$ \begin{CJK}{UTF8}{mj}次方根\end{CJK}, \begin{CJK}{UTF8}{mj}即存在\end{CJK} $n$ \begin{CJK}{UTF8}{mj}阶复方阵\end{CJK} $B$ \begin{CJK}{UTF8}{mj}使得\end{CJK} $B^{k}=J_{n}(\lambda)$;

(3). \begin{CJK}{UTF8}{mj}证明\end{CJK}: \begin{CJK}{UTF8}{mj}任意\end{CJK} $n$ \begin{CJK}{UTF8}{mj}阶可逆复方阵\end{CJK} $A$ \begin{CJK}{UTF8}{mj}都有\end{CJK} $k$ \begin{CJK}{UTF8}{mj}次方根\end{CJK}.

\begin{enumerate}
  \setcounter{enumi}{8}
  \item (20 \begin{CJK}{UTF8}{mj}分\end{CJK}) $n$ \begin{CJK}{UTF8}{mj}阶实方阵\end{CJK} $P$ \begin{CJK}{UTF8}{mj}称为正交矩阵\end{CJK}, \begin{CJK}{UTF8}{mj}如果\end{CJK} $P P^{\prime}=I_{n} ; n$ \begin{CJK}{UTF8}{mj}阶实方阵\end{CJK} $R$ \begin{CJK}{UTF8}{mj}称为反射矩阵\end{CJK}, \begin{CJK}{UTF8}{mj}如果\end{CJK} $R$ \begin{CJK}{UTF8}{mj}正交相似于对角\end{CJK} $\operatorname{diag}(-1,1, \cdots, 1)$, \begin{CJK}{UTF8}{mj}证明\end{CJK}: \begin{CJK}{UTF8}{mj}每个一阶正交矩阵都能写成反射矩阵的乘积\end{CJK}.

  \item (20 \begin{CJK}{UTF8}{mj}分\end{CJK}) \begin{CJK}{UTF8}{mj}设\end{CJK} $\mathbb{R}[x]_{n}$ \begin{CJK}{UTF8}{mj}表示实数域\end{CJK} $\mathbb{R}$ \begin{CJK}{UTF8}{mj}上所有次数小于\end{CJK} $n(>1)$ \begin{CJK}{UTF8}{mj}的多项式之集\end{CJK}, \begin{CJK}{UTF8}{mj}它是实数域上\end{CJK} $n$ \begin{CJK}{UTF8}{mj}维线\end{CJK} \begin{CJK}{UTF8}{mj}性空间\end{CJK}, \begin{CJK}{UTF8}{mj}导算子\end{CJK} $\mathscr{D}: \mathscr{D} f(x)=f^{\prime}(x), \forall f(x) \in \mathbb{R}[x]_{n}$ \begin{CJK}{UTF8}{mj}是\end{CJK} $\mathbb{R}[x]_{n}$ \begin{CJK}{UTF8}{mj}上的线性变换\end{CJK}.

\end{enumerate}
(1). \begin{CJK}{UTF8}{mj}对于任意实数\end{CJK} $a$, \begin{CJK}{UTF8}{mj}证明\end{CJK}: \begin{CJK}{UTF8}{mj}平移算子\end{CJK} $\mathscr{S}_{a} f(x)=f(x+a), \forall f(x) \in \mathbb{R}[x]_{n}$ \begin{CJK}{UTF8}{mj}是\end{CJK} $\mathbb{R}[x]_{n}$ \begin{CJK}{UTF8}{mj}上的线\end{CJK} \begin{CJK}{UTF8}{mj}性变换\end{CJK},\begin{CJK}{UTF8}{mj}且存在一个多项式\end{CJK} $g(x) \in \mathbb{R}[x]_{n}$ \begin{CJK}{UTF8}{mj}使得\end{CJK} $\mathscr{S}_{a}=g(\mathscr{D})$;

(2). \begin{CJK}{UTF8}{mj}分别求出\end{CJK} $\mathscr{S}_{a}, \mathscr{D}$ \begin{CJK}{UTF8}{mj}在基\end{CJK} $1, x, \frac{x^{2}}{2 !}, \cdots, \frac{x^{n-1}}{(n-1) !}$ \begin{CJK}{UTF8}{mj}下的矩阵\end{CJK}.

\section{第3 章 北京师范大学}
\section{$3.12020$ 年数学分析真题}
\section{考生须知:}
\begin{enumerate}
  \item \begin{CJK}{UTF8}{mj}本试卷满分为\end{CJK} 150 \begin{CJK}{UTF8}{mj}分\end{CJK}, \begin{CJK}{UTF8}{mj}全部考试时间总计\end{CJK} 180 \begin{CJK}{UTF8}{mj}分钟\end{CJK};

  \item \begin{CJK}{UTF8}{mj}所有答案必须写在答题纸上\end{CJK}, \begin{CJK}{UTF8}{mj}写在试题纸上或草稿纸上一律无效\end{CJK}。

  \item \begin{CJK}{UTF8}{mj}计算极限\end{CJK}

\end{enumerate}
$$
\lim _{x \rightarrow+\infty} x^{3}\left(\sqrt[3]{\frac{x^{3}+x}{x^{6}+x^{5}+1}}-\sin \frac{1}{x}\right)
$$

\begin{enumerate}
  \setcounter{enumi}{2}
  \item \begin{CJK}{UTF8}{mj}证明级数\end{CJK} $\sum_{n=1}^{+\infty} a_{n}$ \begin{CJK}{UTF8}{mj}与\end{CJK} $\sum_{n=1}^{+\infty} 2^{n-1} a_{2^{n}}$ \begin{CJK}{UTF8}{mj}同敛散\end{CJK}.

  \item \begin{CJK}{UTF8}{mj}证明\end{CJK} $f(x)=\sqrt{x} \ln x$ \begin{CJK}{UTF8}{mj}在\end{CJK} $(0,+\infty)$ \begin{CJK}{UTF8}{mj}上二致连续\end{CJK}.

  \item \begin{CJK}{UTF8}{mj}证明曲面方程\end{CJK} $x^{\frac{2}{3}}+y^{\frac{2}{3}}+z^{\frac{2}{3}}=a^{\frac{2}{3}}$ \begin{CJK}{UTF8}{mj}的切平面在各坐标轴上的截距是常数\end{CJK}.

  \item \begin{CJK}{UTF8}{mj}计算三重积分\end{CJK}

\end{enumerate}
$$
\iiint_{\Omega}\left|x^{2}+y^{2}-1\right| \mathrm{d} x \mathrm{~d} y \mathrm{~d} z
$$
\begin{CJK}{UTF8}{mj}其中\end{CJK} $\Omega$ \begin{CJK}{UTF8}{mj}是\end{CJK} $x^{2}+y^{2}=2 z$ \begin{CJK}{UTF8}{mj}与\end{CJK} $z=2$ \begin{CJK}{UTF8}{mj}所围成的区域\end{CJK}.

\begin{enumerate}
  \setcounter{enumi}{6}
  \item \begin{CJK}{UTF8}{mj}计算含参积分\end{CJK}
\end{enumerate}
$$
\int_{0}^{+\infty} \frac{\cos b x-\cos a x}{x} d x
$$
\begin{CJK}{UTF8}{mj}其中\end{CJK} $b>a>0$ \begin{CJK}{UTF8}{mj}为常数\end{CJK}.

\begin{enumerate}
  \setcounter{enumi}{7}
  \item \begin{CJK}{UTF8}{mj}若\end{CJK} $\vec{A}$ \begin{CJK}{UTF8}{mj}表示区域\end{CJK} $A$ \begin{CJK}{UTF8}{mj}的闭包\end{CJK}, $\vec{A}$ \begin{CJK}{UTF8}{mj}有界\end{CJK}, \begin{CJK}{UTF8}{mj}且\end{CJK} $f(x)$ \begin{CJK}{UTF8}{mj}在\end{CJK} $\vec{A}$ \begin{CJK}{UTF8}{mj}上连续\end{CJK}, \begin{CJK}{UTF8}{mj}在\end{CJK} $A$ \begin{CJK}{UTF8}{mj}上可导\end{CJK}, \begin{CJK}{UTF8}{mj}对\end{CJK} $\forall x \in \vec{A} \backslash A$ \begin{CJK}{UTF8}{mj}有\end{CJK} $f(x)=0$, \begin{CJK}{UTF8}{mj}试证\end{CJK}:\begin{CJK}{UTF8}{mj}存在\end{CJK} $\theta \in A$ \begin{CJK}{UTF8}{mj}使得\end{CJK} $f^{\prime}(\theta)=0$.

  \item \begin{CJK}{UTF8}{mj}若对\end{CJK} $\forall x_{0} \in[a, b]$, \begin{CJK}{UTF8}{mj}且\end{CJK} $\lim _{x \rightarrow x_{0}} f(x)$ \begin{CJK}{UTF8}{mj}存在且有限\end{CJK}, \begin{CJK}{UTF8}{mj}试证\end{CJK}: $f$ \begin{CJK}{UTF8}{mj}在\end{CJK} $[a, b]$ \begin{CJK}{UTF8}{mj}上至多有可数多个间断点\end{CJK}, \begin{CJK}{UTF8}{mj}并\end{CJK} \begin{CJK}{UTF8}{mj}且在\end{CJK} $[a, b]$ \begin{CJK}{UTF8}{mj}上黎蔓可积\end{CJK}.

  \item \begin{CJK}{UTF8}{mj}若函数\end{CJK} $f(x)=\frac{\mathrm{e}^{x}-\mathrm{e}^{-x}}{\mathrm{e}^{\pi}+\mathrm{e}^{\pi}}$

\end{enumerate}
(1). \begin{CJK}{UTF8}{mj}计算\end{CJK} $f(x)$ \begin{CJK}{UTF8}{mj}在\end{CJK} $[-\pi, \pi]$ \begin{CJK}{UTF8}{mj}上的\end{CJK} Fourier \begin{CJK}{UTF8}{mj}级数\end{CJK};

(2). \begin{CJK}{UTF8}{mj}计算级数\end{CJK} (\begin{CJK}{UTF8}{mj}具体记不清\end{CJK}, \begin{CJK}{UTF8}{mj}是利用第\end{CJK} $(1)$ \begin{CJK}{UTF8}{mj}问级数求\end{CJK} $f(\pi)$ \begin{CJK}{UTF8}{mj}的值\end{CJK}).

\section{$3.22019$ 年高等代数真题}
\section{考生须知:}
\begin{enumerate}
  \item \begin{CJK}{UTF8}{mj}本试卷满分为\end{CJK} 150 \begin{CJK}{UTF8}{mj}分\end{CJK}, \begin{CJK}{UTF8}{mj}全部考试时间总计\end{CJK} 180 \begin{CJK}{UTF8}{mj}分钟\end{CJK};

  \item \begin{CJK}{UTF8}{mj}所有答案必须写在答题纸上\end{CJK}, \begin{CJK}{UTF8}{mj}写在试题纸上或草稿纸上一律无效\end{CJK}。

\end{enumerate}
\section{解析几何}
\begin{enumerate}
  \item \begin{CJK}{UTF8}{mj}已知点在直线上的垂足\end{CJK}, \begin{CJK}{UTF8}{mj}求坐标和线段的长度\end{CJK}.

  \item \begin{CJK}{UTF8}{mj}将直线绕轴旋转\end{CJK}, \begin{CJK}{UTF8}{mj}求这个旋转曲面的方程\end{CJK}, \begin{CJK}{UTF8}{mj}并就可能的值讨论这是什么曲面\end{CJK}.

  \item \begin{CJK}{UTF8}{mj}求经过直线的直圆柱面方程\end{CJK}.

  \item \begin{CJK}{UTF8}{mj}已知二次曲面的方程\end{CJK}, \begin{CJK}{UTF8}{mj}判断曲面类型并求对称平面的方程\end{CJK}.

\end{enumerate}
\section{高等代数}
\begin{enumerate}
  \item \begin{CJK}{UTF8}{mj}若\end{CJK} $f$ \begin{CJK}{UTF8}{mj}是域\end{CJK} $\mathbb{E}$ \begin{CJK}{UTF8}{mj}上的多项式\end{CJK}, \begin{CJK}{UTF8}{mj}对\end{CJK} $\forall x$ \begin{CJK}{UTF8}{mj}有\end{CJK} $f(x)=f(x+a)$, \begin{CJK}{UTF8}{mj}其中\end{CJK} $a$ \begin{CJK}{UTF8}{mj}为常数且不为\end{CJK} 0 , \begin{CJK}{UTF8}{mj}试证\end{CJK}: $f(x)$ \begin{CJK}{UTF8}{mj}是常数多项式\end{CJK}.
\end{enumerate}
2 . \begin{CJK}{UTF8}{mj}若\end{CJK} $A$ \begin{CJK}{UTF8}{mj}是\end{CJK} $m \times n$ \begin{CJK}{UTF8}{mj}矩阵\end{CJK}, $B$ \begin{CJK}{UTF8}{mj}是\end{CJK} $m$ \begin{CJK}{UTF8}{mj}维列向量\end{CJK}.\\
(1). $\operatorname{rank}\left(A^{T} A\right)=\operatorname{rank}(A)$\\
(2). \begin{CJK}{UTF8}{mj}形如\end{CJK} $A^{T} A X=A^{T} B$ \begin{CJK}{UTF8}{mj}的方程一定有解\end{CJK}.

\begin{enumerate}
  \setcounter{enumi}{3}
  \item \begin{CJK}{UTF8}{mj}若\end{CJK} $A, B$ \begin{CJK}{UTF8}{mj}均为\end{CJK} $n$ \begin{CJK}{UTF8}{mj}阶方阵\end{CJK}, \begin{CJK}{UTF8}{mj}试证\end{CJK}: $(A B)^{*}=B^{*} A^{*}$.

  \item \begin{CJK}{UTF8}{mj}已知实对称矩阵\end{CJK} $A$, \begin{CJK}{UTF8}{mj}且\end{CJK} $|A|<0$, \begin{CJK}{UTF8}{mj}证明\end{CJK}: $\exists x$, \begin{CJK}{UTF8}{mj}有\end{CJK} $x^{T} A x<0$

  \item \begin{CJK}{UTF8}{mj}设\end{CJK} $\sigma$ \begin{CJK}{UTF8}{mj}有限维向量空间上的线性变换\end{CJK}, \begin{CJK}{UTF8}{mj}证明\end{CJK}: $\sigma$ \begin{CJK}{UTF8}{mj}是同构映射\end{CJK},\begin{CJK}{UTF8}{mj}当且仅当是单射与满射\end{CJK}.

\end{enumerate}
\includegraphics[max width=\textwidth]{2022_04_18_7db0708508f26638f054g-025}

\section{第 4 章 同济大学}
\section{$4.12020$ 年数学分析考研真题}
\section{考生须知:}
\begin{enumerate}
  \item \begin{CJK}{UTF8}{mj}本试卷满分为\end{CJK} 150 \begin{CJK}{UTF8}{mj}分\end{CJK}, \begin{CJK}{UTF8}{mj}全部考试时间总计\end{CJK} 180 \begin{CJK}{UTF8}{mj}分钟\end{CJK};

  \item \begin{CJK}{UTF8}{mj}所有答案必须写在答题纸上\end{CJK}, \begin{CJK}{UTF8}{mj}写在试题纸上或草稿纸上一律无效\end{CJK}。

  \item $(10$ \begin{CJK}{UTF8}{mj}分\end{CJK} $)$ \begin{CJK}{UTF8}{mj}设函数\end{CJK} $f(x)$ \begin{CJK}{UTF8}{mj}在\end{CJK} $\mathbb{R}$ \begin{CJK}{UTF8}{mj}上连续且\end{CJK} $f(0)=1$, \begin{CJK}{UTF8}{mj}计算\end{CJK}

\end{enumerate}
$$
\lim _{x \rightarrow 0} \frac{\int_{0}^{x^{2}}(x-\sqrt{t}) f(t) \mathrm{d} t}{x^{2} \ln (1+x)}
$$

\begin{enumerate}
  \setcounter{enumi}{2}
  \item (15 \begin{CJK}{UTF8}{mj}分\end{CJK}) \begin{CJK}{UTF8}{mj}用任一实数理论完备性定理证明闭区间连续函数的有界性定理\end{CJK}.

  \item (15 \begin{CJK}{UTF8}{mj}分\end{CJK}) \begin{CJK}{UTF8}{mj}已知\end{CJK} $f(x)$ \begin{CJK}{UTF8}{mj}在\end{CJK} $[0,+\infty)$ \begin{CJK}{UTF8}{mj}上一致连续\end{CJK}, \begin{CJK}{UTF8}{mj}且对牜一个固定的\end{CJK} $x \in[0,+\infty)$ \begin{CJK}{UTF8}{mj}成立\end{CJK} $\lim _{n \rightarrow \infty} f(x+$ $n)=0$, \begin{CJK}{UTF8}{mj}证明\end{CJK}: \begin{CJK}{UTF8}{mj}当\end{CJK} $n \rightarrow \infty$ \begin{CJK}{UTF8}{mj}时\end{CJK}, \begin{CJK}{UTF8}{mj}函数列\end{CJK} $\{f(x+n)\}$ \begin{CJK}{UTF8}{mj}在\end{CJK} $[0,1]$ \begin{CJK}{UTF8}{mj}上一致收敛至\end{CJK} 0 .

  \item (15 \begin{CJK}{UTF8}{mj}分\end{CJK}) \begin{CJK}{UTF8}{mj}确定实数\end{CJK} $p$ \begin{CJK}{UTF8}{mj}的取值范围\end{CJK}, \begin{CJK}{UTF8}{mj}使平面无界区或\end{CJK} $D=\left(x, y \mid x \geq 1,0 \leq y \leq \frac{1}{x p}\right)$ \begin{CJK}{UTF8}{mj}绕\end{CJK} $x$ \begin{CJK}{UTF8}{mj}轴旋转\end{CJK} \begin{CJK}{UTF8}{mj}一周所得旋转体的体积有限而其表面积无限\end{CJK}, \begin{CJK}{UTF8}{mj}并计算该旋转体的体积\end{CJK}.

  \item (15 \begin{CJK}{UTF8}{mj}分\end{CJK}) \begin{CJK}{UTF8}{mj}曲面\end{CJK} $\Sigma: \frac{x^{2}}{a^{2}}+\frac{y^{2}}{b^{2}}+\frac{z^{2}}{c^{2}}=1$ \begin{CJK}{UTF8}{mj}围成的有限区域记为\end{CJK} $\Omega$, \begin{CJK}{UTF8}{mj}其中\end{CJK} $a, b, c>0$, \begin{CJK}{UTF8}{mj}记\end{CJK} $\Sigma$ \begin{CJK}{UTF8}{mj}在\end{CJK} $P_{0}\left(x_{0}, y_{0}, z_{0}\right)$ \begin{CJK}{UTF8}{mj}在点处的切平面\end{CJK} $\prod 0$, \begin{CJK}{UTF8}{mj}计算积分\end{CJK}

\end{enumerate}
$$
\int_{\Omega} f(p) \mathrm{d} V
$$
\begin{CJK}{UTF8}{mj}其中\end{CJK} $f(p)$ \begin{CJK}{UTF8}{mj}是\end{CJK} $p$ \begin{CJK}{UTF8}{mj}到平面\end{CJK} $\prod 0$ \begin{CJK}{UTF8}{mj}距离的平方\end{CJK}.

\begin{enumerate}
  \setcounter{enumi}{6}
  \item (20 \begin{CJK}{UTF8}{mj}分\end{CJK}) \begin{CJK}{UTF8}{mj}设函数\end{CJK} $f(x)$ \begin{CJK}{UTF8}{mj}在\end{CJK} $[0,1]$ \begin{CJK}{UTF8}{mj}上二阶可导\end{CJK}, \begin{CJK}{UTF8}{mj}且\end{CJK} $f^{\prime \prime}(x)<0$, \begin{CJK}{UTF8}{mj}并满足\end{CJK} $f(0)=f(1)=0$, \begin{CJK}{UTF8}{mj}试证明\end{CJK}:
\end{enumerate}
(1) \begin{CJK}{UTF8}{mj}对任意整数\end{CJK} $n$, \begin{CJK}{UTF8}{mj}存在唯\end{CJK} $-x_{n} \in(0,1)$ \begin{CJK}{UTF8}{mj}使得\end{CJK} $f^{\prime}\left(x_{n}\right)=\frac{M}{n}$, \begin{CJK}{UTF8}{mj}其中\end{CJK} $M$ \begin{CJK}{UTF8}{mj}是\end{CJK} $f(x)$ \begin{CJK}{UTF8}{mj}在\end{CJK} $[0,1]$ \begin{CJK}{UTF8}{mj}上的\end{CJK} \begin{CJK}{UTF8}{mj}最大值\end{CJK};

(2) $\left\{x_{n}\right\}$ \begin{CJK}{UTF8}{mj}收敛且\end{CJK} $\lim _{n \rightarrow \infty} f\left(x_{n}\right)=M$

\begin{enumerate}
  \setcounter{enumi}{7}
  \item (20 \begin{CJK}{UTF8}{mj}分\end{CJK}) \begin{CJK}{UTF8}{mj}设\end{CJK} $a_{0}=1, a_{n+1}=a_{n}+\frac{1}{(n+1) !}, n \in \mathbb{N}$, \begin{CJK}{UTF8}{mj}求幂级数\end{CJK} $\sum_{n=0}^{\infty} a_{n} x_{n}$ \begin{CJK}{UTF8}{mj}的收敛域以及和函数\end{CJK}. 8. (20 \begin{CJK}{UTF8}{mj}分\end{CJK}) \begin{CJK}{UTF8}{mj}说明方程\end{CJK} $2 x^{2}+y^{2}+z+\mathrm{e}^{x}=1$ \begin{CJK}{UTF8}{mj}可确定唯一的具有连续偏导数\end{CJK}, \begin{CJK}{UTF8}{mj}并且定义在全平面上\end{CJK} \begin{CJK}{UTF8}{mj}的隐函数\end{CJK} $z=z(x, y)$, \begin{CJK}{UTF8}{mj}求\end{CJK} $z=z(x, y)$ \begin{CJK}{UTF8}{mj}的极值\end{CJK}.

  \item (20 \begin{CJK}{UTF8}{mj}分\end{CJK}) \begin{CJK}{UTF8}{mj}指出函数\end{CJK}

\end{enumerate}
$$
\phi(r)=\int_{0}^{+\infty} \mathrm{e}^{-x^{2}} \cos (x r) \mathrm{d} x
$$
\begin{CJK}{UTF8}{mj}定义域并用初等函数表示\end{CJK} $\phi(r)$.

\section{$4.22020$ 年高等代数考研真题}
\section{考生须知:}
\begin{enumerate}
  \item \begin{CJK}{UTF8}{mj}本试卷满分为\end{CJK} 150 \begin{CJK}{UTF8}{mj}分\end{CJK}, \begin{CJK}{UTF8}{mj}全部考试时间总计\end{CJK} 180 \begin{CJK}{UTF8}{mj}分钟\end{CJK};

  \item \begin{CJK}{UTF8}{mj}所有答案必须写在答题纸上\end{CJK},\begin{CJK}{UTF8}{mj}写在试题纸上或草稿纸上一律无效\end{CJK}。

  \item \begin{CJK}{UTF8}{mj}已知\end{CJK} $\sqrt{2}-\sqrt{3}$ \begin{CJK}{UTF8}{mj}是首项系数为\end{CJK} 1 \begin{CJK}{UTF8}{mj}且不可约的有理多项式\end{CJK} $f(x)$ \begin{CJK}{UTF8}{mj}的根\end{CJK}, \begin{CJK}{UTF8}{mj}求\end{CJK} $f(x)$ \begin{CJK}{UTF8}{mj}且证明\end{CJK} $f(x)$ \begin{CJK}{UTF8}{mj}不\end{CJK} \begin{CJK}{UTF8}{mj}可约\end{CJK}.

  \item \begin{CJK}{UTF8}{mj}已知\end{CJK} $F(x)=x^{n}-1$, \begin{CJK}{UTF8}{mj}试证\end{CJK}:

\end{enumerate}
(1). $F(x)$ \begin{CJK}{UTF8}{mj}没有重根\end{CJK};

(2). \begin{CJK}{UTF8}{mj}已知\end{CJK} $w_{1}, w_{2}, \cdots, w_{n}$ \begin{CJK}{UTF8}{mj}是\end{CJK} $F(x)$ \begin{CJK}{UTF8}{mj}的根\end{CJK} $\left(w_{i} \neq 1, i=1,2, \cdots, n\right)$, \begin{CJK}{UTF8}{mj}证明\end{CJK}:
$$
\left(1-w_{1}\right)\left(1-w_{2}\right) \cdots\left(1-w_{n}\right)=n
$$

\begin{enumerate}
  \setcounter{enumi}{3}
  \item \begin{CJK}{UTF8}{mj}已知一个\end{CJK} $m \times n$ \begin{CJK}{UTF8}{mj}的矩阵\end{CJK} $A=a(i, j)$, \begin{CJK}{UTF8}{mj}且\end{CJK} $a(i, j)=\min (i, j)$, \begin{CJK}{UTF8}{mj}求\end{CJK} $|A|$ \begin{CJK}{UTF8}{mj}及\end{CJK} $A^{-1}$.

  \item \begin{CJK}{UTF8}{mj}已知\end{CJK} $A$ \begin{CJK}{UTF8}{mj}和\end{CJK} $B$ \begin{CJK}{UTF8}{mj}都是矩阵\end{CJK}, \begin{CJK}{UTF8}{mj}证明\end{CJK} $r(A)=r(B)$ \begin{CJK}{UTF8}{mj}的充要条件是存在可逆矩阵\end{CJK} $C$ \begin{CJK}{UTF8}{mj}使得\end{CJK} $A=A B C$.

  \item \begin{CJK}{UTF8}{mj}已知矩阵\end{CJK} $\lambda E-A$ \begin{CJK}{UTF8}{mj}与以下矩阵等价\end{CJK}

\end{enumerate}
\includegraphics[max width=\textwidth]{2022_04_18_7db0708508f26638f054g-027}

\begin{CJK}{UTF8}{mj}求\end{CJK} $A$ \begin{CJK}{UTF8}{mj}的特征多项式\end{CJK}、\begin{CJK}{UTF8}{mj}最小多项式\end{CJK}、\begin{CJK}{UTF8}{mj}初等因子以及不变因子\end{CJK}.

\begin{enumerate}
  \setcounter{enumi}{6}
  \item \begin{CJK}{UTF8}{mj}已知\end{CJK} $f\left(x_{1}, x_{2}, x_{3}\right)=2 a x_{1}^{2}+2 a x_{2}^{2}+2 a x_{3}^{3}-2 x_{1} x_{2}-2 x_{1} x_{3}+2 x_{2} x_{3}$
\end{enumerate}
(1). \begin{CJK}{UTF8}{mj}求存在矩阵\end{CJK} $Q$ \begin{CJK}{UTF8}{mj}使得\end{CJK} $X=Q Y$ \begin{CJK}{UTF8}{mj}的标准形\end{CJK};

(2). \begin{CJK}{UTF8}{mj}求\end{CJK} $a$ \begin{CJK}{UTF8}{mj}为何值时\end{CJK}, $f\left(x_{1}, x_{2}, x_{3}\right)$ \begin{CJK}{UTF8}{mj}的二次型矩阵的秩为\end{CJK} 2 .

\section{第 5 章 兰州大学}
\section{$5.12020$ 年数学分析考研真题}
\section{考生须知:}
\begin{enumerate}
  \item \begin{CJK}{UTF8}{mj}本试卷满分为\end{CJK} 150 \begin{CJK}{UTF8}{mj}分\end{CJK}, \begin{CJK}{UTF8}{mj}全部考试时间总计\end{CJK} 180 \begin{CJK}{UTF8}{mj}分钟\end{CJK};

  \item \begin{CJK}{UTF8}{mj}所有答案必须写在答题纸上\end{CJK}, \begin{CJK}{UTF8}{mj}写在试题纸上或草稿纸上一律无效\end{CJK}。

  \item \begin{CJK}{UTF8}{mj}一\end{CJK}、\begin{CJK}{UTF8}{mj}计算题\end{CJK} (\begin{CJK}{UTF8}{mj}每小题\end{CJK} 5 \begin{CJK}{UTF8}{mj}分\end{CJK}, \begin{CJK}{UTF8}{mj}共\end{CJK} 50 \begin{CJK}{UTF8}{mj}分\end{CJK})

\end{enumerate}
(1). \begin{CJK}{UTF8}{mj}对\end{CJK} $\forall \alpha \in \mathbb{R}$, \begin{CJK}{UTF8}{mj}试证\end{CJK}:
$$
\lim _{n \rightarrow \infty} \int_{0}^{1}\left(1+x^{n}\right)^{\alpha} \mathrm{d} x=1
$$
(2). \begin{CJK}{UTF8}{mj}计算曲面积分\end{CJK}
$$
I=\iint_{S} \frac{d \sigma}{\left(x^{2}+y^{2}+z^{2}\right)^{\frac{3}{2}} \sqrt{\frac{x^{2}}{a^{4}}+\frac{y^{2}}{b^{4}}+\frac{z^{2}}{c^{4}}}}
$$
\begin{CJK}{UTF8}{mj}其中\end{CJK} $S$ \begin{CJK}{UTF8}{mj}为椭球面\end{CJK} $\frac{x^{2}}{a^{2}}+\frac{y^{2}}{b^{2}}+\frac{z^{2}}{c^{2}}=1(a, b, c>0)$.

(3). \begin{CJK}{UTF8}{mj}计算曲线积分\end{CJK}
$$
\oint_{L}\left(y^{2}+z^{2}\right) \mathrm{d} x+\left(z^{2}+x^{2}\right) \mathrm{d} y+\left(x^{2}+y^{2}\right) \mathrm{d} z
$$
\begin{CJK}{UTF8}{mj}其中\end{CJK} $L$ \begin{CJK}{UTF8}{mj}是\end{CJK} $x^{2}+y^{2}+z^{2}=2 R x$ \begin{CJK}{UTF8}{mj}与\end{CJK} $x^{2}+y^{2}=2 r x$ \begin{CJK}{UTF8}{mj}的交线\end{CJK}, \begin{CJK}{UTF8}{mj}且\end{CJK} $0<r<R$.

(4). \begin{CJK}{UTF8}{mj}若\end{CJK} $F(x)=\int_{0}^{x} \sin \frac{1}{t} \mathrm{~d} t$, \begin{CJK}{UTF8}{mj}求\end{CJK} $F^{\prime}(0)$.

(5). \begin{CJK}{UTF8}{mj}计算极限\end{CJK} $\lim _{x \rightarrow \infty}\left(\int_{0}^{x} \mathrm{e}^{t^{2}} \mathrm{~d} t\right)^{\frac{1}{x^{2}}}$

\begin{enumerate}
  \setcounter{enumi}{2}
  \item (10 \begin{CJK}{UTF8}{mj}分\end{CJK}) \begin{CJK}{UTF8}{mj}若\end{CJK} $f(x)$ \begin{CJK}{UTF8}{mj}在\end{CJK} $[-1,1]$ \begin{CJK}{UTF8}{mj}上连续且二次可微\end{CJK}, \begin{CJK}{UTF8}{mj}且\end{CJK} $M=\sup _{x \in[-1,1]}\left|f^{\prime \prime}(x)\right|, f(0)=0$, \begin{CJK}{UTF8}{mj}试证\end{CJK}:
\end{enumerate}
$$
\left|\int_{-1}^{1} f(x) \mathrm{d} x\right| \leq \frac{M}{3}
$$

\begin{enumerate}
  \setcounter{enumi}{3}
  \item \begin{CJK}{UTF8}{mj}三\end{CJK}.(15 \begin{CJK}{UTF8}{mj}分\end{CJK}) \begin{CJK}{UTF8}{mj}若\end{CJK} $\int_{a}^{+\infty} f(x) \mathrm{d} x$ \begin{CJK}{UTF8}{mj}收敛\end{CJK}, \begin{CJK}{UTF8}{mj}且\end{CJK} $f(x)$ \begin{CJK}{UTF8}{mj}在\end{CJK} $[a,+\infty)$ \begin{CJK}{UTF8}{mj}上一致连续\end{CJK}, \begin{CJK}{UTF8}{mj}证明\end{CJK}: $\lim _{x \rightarrow \infty} f(x)=0$.

  \item \begin{CJK}{UTF8}{mj}四\end{CJK}.(15 \begin{CJK}{UTF8}{mj}分\end{CJK}) \begin{CJK}{UTF8}{mj}若\end{CJK} $\int_{a}^{+\infty} f(x) \mathrm{d} x$ \begin{CJK}{UTF8}{mj}收敛\end{CJK}, \begin{CJK}{UTF8}{mj}且\end{CJK} $f(x)$ \begin{CJK}{UTF8}{mj}在\end{CJK} $[a,+\infty)$ \begin{CJK}{UTF8}{mj}单调\end{CJK}, \begin{CJK}{UTF8}{mj}证明\end{CJK}: $\lim _{x \rightarrow \infty} x f(x)=0$. 5. \begin{CJK}{UTF8}{mj}五\end{CJK}.(15 \begin{CJK}{UTF8}{mj}分\end{CJK}) \begin{CJK}{UTF8}{mj}若连续函数列\end{CJK} $\left\{f_{n}(x)\right\}$ \begin{CJK}{UTF8}{mj}在\end{CJK} $[a, b]$ \begin{CJK}{UTF8}{mj}上一致收敛于\end{CJK} $f(x)$, \begin{CJK}{UTF8}{mj}证明\end{CJK}: $\left\{f_{n}(x)\right\}$ \begin{CJK}{UTF8}{mj}在\end{CJK} $[a, b]$ \begin{CJK}{UTF8}{mj}上等\end{CJK} \begin{CJK}{UTF8}{mj}度连续\end{CJK}.

  \item \begin{CJK}{UTF8}{mj}六\end{CJK}.(15 \begin{CJK}{UTF8}{mj}分\end{CJK}) \begin{CJK}{UTF8}{mj}讨论\end{CJK}

\end{enumerate}
$$
\int_{0}^{\infty} \frac{\sin \left(x+\frac{1}{x}\right)}{x^{\alpha}} \mathrm{d} x
$$
\begin{CJK}{UTF8}{mj}收玫性与绝对收玫性\end{CJK}.

\begin{enumerate}
  \setcounter{enumi}{7}
  \item \begin{CJK}{UTF8}{mj}七\end{CJK}.(15 \begin{CJK}{UTF8}{mj}分\end{CJK}) \begin{CJK}{UTF8}{mj}若\end{CJK} $f(x)$ \begin{CJK}{UTF8}{mj}在\end{CJK} $[0,2 \pi]$ \begin{CJK}{UTF8}{mj}上连续可微\end{CJK}, \begin{CJK}{UTF8}{mj}试证\end{CJK}:
\end{enumerate}
$$
\lim _{n \rightarrow \infty} n \int_{0}^{2 \pi} f(x) \sin n x \mathrm{~d} x=f(0)-f(2 \pi)
$$

\section{$5.22020$ 年高等代数考研真题}
\section{考生须知:}
\begin{enumerate}
  \item \begin{CJK}{UTF8}{mj}本试卷满分为\end{CJK} 150 \begin{CJK}{UTF8}{mj}分\end{CJK}, \begin{CJK}{UTF8}{mj}全部考试时间总计\end{CJK} 180 \begin{CJK}{UTF8}{mj}分钟\end{CJK};

  \item \begin{CJK}{UTF8}{mj}所有答案必须写在答题纸上\end{CJK}, \begin{CJK}{UTF8}{mj}写在试题纸上或草稿纸上一律无效\end{CJK}。

  \item (22 \begin{CJK}{UTF8}{mj}分\end{CJK})

\end{enumerate}
(1). $f(x), g(x) \in P[x]$ \begin{CJK}{UTF8}{mj}分别是首项系数为\end{CJK} 1,3 \begin{CJK}{UTF8}{mj}次多项式\end{CJK}, \begin{CJK}{UTF8}{mj}且\end{CJK} $f(x) \neq g(x)$, \begin{CJK}{UTF8}{mj}证明\end{CJK}:
$$
x^{4}+x^{2}+1 \mid f\left(x^{3}\right)+x^{4} g\left(x^{3}\right)
$$
\begin{CJK}{UTF8}{mj}则\end{CJK} $(f(x), g(x))=(x-1)(x+1)$.

(2). \begin{CJK}{UTF8}{mj}证明\end{CJK}: \begin{CJK}{UTF8}{mj}实系数多项式\end{CJK} $f(x)$ \begin{CJK}{UTF8}{mj}可以拆分为两个实系数多项式平方和的充要条件是\end{CJK}, \begin{CJK}{UTF8}{mj}对\end{CJK} $\forall a \in \mathbb{R}$, \begin{CJK}{UTF8}{mj}有\end{CJK} $f(a) \geq 0$.

\begin{enumerate}
  \setcounter{enumi}{2}
  \item (18 \begin{CJK}{UTF8}{mj}分\end{CJK}) \begin{CJK}{UTF8}{mj}计算题\end{CJK}
\end{enumerate}
(1). (8 \begin{CJK}{UTF8}{mj}分\end{CJK})

\includegraphics[max width=\textwidth]{2022_04_18_7db0708508f26638f054g-029}
$$
D_{n}=\left|\begin{array}{cccc}
x_{1} & a & \cdots & a \\
b & x_{2} & \cdots & a \\
\vdots & \vdots & & \vdots \\
\vdots & \vdots & & \vdots \\
b & b & \cdots & x_{n}
\end{array}\right|
$$
(2). (10 \begin{CJK}{UTF8}{mj}分\end{CJK})
$$
D_{n}=\left|\begin{array}{cccc}
1+x_{1} & 1+x_{1}^{2} & \cdots & 1+x_{1}^{n} \\
1+x_{2} & 1+x_{2}^{2} & \cdots & 1+x_{2}^{n} \\
\vdots & \vdots & & \vdots \\
\vdots & \vdots & & \vdots \\
1+x_{n} & 1+x_{n}^{2} & \cdots & 1+x_{n}^{n}
\end{array}\right|
$$

\begin{enumerate}
  \setcounter{enumi}{3}
  \item (15 \begin{CJK}{UTF8}{mj}分\end{CJK})
\end{enumerate}
(1). \begin{CJK}{UTF8}{mj}若\end{CJK} $n$ \begin{CJK}{UTF8}{mj}阶实矩阵\end{CJK} $A$ \begin{CJK}{UTF8}{mj}满足\end{CJK} $\left|a_{i j}\right|>\sum_{j \neq i}\left|a_{i j}\right|$, \begin{CJK}{UTF8}{mj}则\end{CJK} $|A| \neq 0$;

(2). \begin{CJK}{UTF8}{mj}若\end{CJK} $n$ \begin{CJK}{UTF8}{mj}阶实矩阵\end{CJK} $A$ \begin{CJK}{UTF8}{mj}满足\end{CJK} $a_{i j}>\sum_{j \neq i}\left|a_{i j}\right|$, \begin{CJK}{UTF8}{mj}则\end{CJK} $|A|>0$;

\begin{enumerate}
  \setcounter{enumi}{4}
  \item (23 \begin{CJK}{UTF8}{mj}分\end{CJK})
\end{enumerate}
(1). (10 \begin{CJK}{UTF8}{mj}分\end{CJK}) \begin{CJK}{UTF8}{mj}设\end{CJK} $A$ \begin{CJK}{UTF8}{mj}为\end{CJK} $m \times n$ \begin{CJK}{UTF8}{mj}矩阵\end{CJK}, $B$ \begin{CJK}{UTF8}{mj}为\end{CJK} $n \times s$ \begin{CJK}{UTF8}{mj}矩阵\end{CJK}, $C$ \begin{CJK}{UTF8}{mj}为\end{CJK} $s \times t$ \begin{CJK}{UTF8}{mj}矩阵\end{CJK}, \begin{CJK}{UTF8}{mj}试证\end{CJK}:
$$
\operatorname{rank}(A B)+\operatorname{rank}(B C)-\operatorname{rank}(B) \leq \operatorname{rank}(A B C)
$$
(2). (13 \begin{CJK}{UTF8}{mj}分\end{CJK}) \begin{CJK}{UTF8}{mj}证明\end{CJK}: \begin{CJK}{UTF8}{mj}实系数矩阵特征值均为实根的充要条件是存在正交矩阵\end{CJK} $T$, \begin{CJK}{UTF8}{mj}使得\end{CJK} $T^{-1} A T$ \begin{CJK}{UTF8}{mj}为二角阵\end{CJK}.

\begin{enumerate}
  \setcounter{enumi}{5}
  \item \begin{CJK}{UTF8}{mj}已知\end{CJK} $P$ \begin{CJK}{UTF8}{mj}为一数域\end{CJK}, \begin{CJK}{UTF8}{mj}且\end{CJK} $f(x), g(x) \in P[x],(f(x), g(x))=1$, \begin{CJK}{UTF8}{mj}若\end{CJK} $A$ \begin{CJK}{UTF8}{mj}是\end{CJK} $n$ \begin{CJK}{UTF8}{mj}阶矩阵\end{CJK}, $V=$ $\{\alpha \mid f(A) g(A) \alpha\}, V_{1}=\{\alpha \mid f(A) \alpha\}, V_{2}=\{\alpha \mid g(A) \alpha\}$, \begin{CJK}{UTF8}{mj}证明\end{CJK}: $V$ \begin{CJK}{UTF8}{mj}是\end{CJK} $V_{1} 与 V_{2}$ \begin{CJK}{UTF8}{mj}的直和\end{CJK}.

  \item \begin{CJK}{UTF8}{mj}若二次型\end{CJK} $f\left(x_{1}, \cdots, x_{n}\right)=l_{1}^{2}+\cdots+l_{p}^{2}-l_{p+1}^{2}-\cdots-l_{p+q}^{2}$, \begin{CJK}{UTF8}{mj}其中\end{CJK} $l_{i}$ \begin{CJK}{UTF8}{mj}是\end{CJK} $x_{1}, \cdots, x_{n}$ \begin{CJK}{UTF8}{mj}的齐次线性\end{CJK} \begin{CJK}{UTF8}{mj}函数\end{CJK}, \begin{CJK}{UTF8}{mj}证明\end{CJK}: \begin{CJK}{UTF8}{mj}二次型\end{CJK} $f\left(x_{1}, \cdots, x_{n}\right)$ \begin{CJK}{UTF8}{mj}正惯性指数小于等于\end{CJK} $p$.

  \item \begin{CJK}{UTF8}{mj}设\end{CJK} $\sigma$ \begin{CJK}{UTF8}{mj}是\end{CJK} $n$ \begin{CJK}{UTF8}{mj}维几里得空间\end{CJK} $V$ \begin{CJK}{UTF8}{mj}上的一个对称变换\end{CJK}, $\varepsilon$ \begin{CJK}{UTF8}{mj}是单位变换\end{CJK}, $\sigma$ \begin{CJK}{UTF8}{mj}的特征多项式为\end{CJK}

\end{enumerate}
$$
P(x)=\left(x-\lambda_{1}\right)^{r_{1}}\left(x-\lambda_{2}\right)^{r_{2}} \cdots\left(x-\lambda_{m}\right)^{r_{m}}
$$
\begin{CJK}{UTF8}{mj}其中\end{CJK} $\lambda_{1}, \lambda_{2}, \cdots, \lambda_{m}$ \begin{CJK}{UTF8}{mj}互不相同\end{CJK}, $r_{i} \geq 1, i=1,2, \cdots, m$, \begin{CJK}{UTF8}{mj}证明\end{CJK}: \begin{CJK}{UTF8}{mj}对\end{CJK} $i=1,2, \cdots, m$, \begin{CJK}{UTF8}{mj}且\end{CJK} $\varepsilon$ \begin{CJK}{UTF8}{mj}表示\end{CJK} $V$ \begin{CJK}{UTF8}{mj}上的恒等变换\end{CJK}.
$$
\operatorname{ker}\left(\sigma-\lambda_{i} \varepsilon\right)=\operatorname{ker}\left(\sigma-\lambda_{i} \varepsilon\right)^{r_{i}}
$$

\begin{enumerate}
  \setcounter{enumi}{8}
  \item \begin{CJK}{UTF8}{mj}已知二次曲面方程\end{CJK} $x^{2}+a y^{2}+z^{2}+2 b x y+2 x z+2 y z=4$, \begin{CJK}{UTF8}{mj}经正交变换后得到椭圆柱面方程\end{CJK} $y_{1}^{2}+4 z_{1}^{2}=4$, \begin{CJK}{UTF8}{mj}求\end{CJK} $a, b$ \begin{CJK}{UTF8}{mj}的值以及所做的正交变换\end{CJK}.
\end{enumerate}
\section{第6章 浙江大学}
\section{$6.12020$ 年数学分析考研真题}
\section{考生须知:}
\begin{enumerate}
  \item \begin{CJK}{UTF8}{mj}本试卷满分为\end{CJK} 150 \begin{CJK}{UTF8}{mj}分\end{CJK}, \begin{CJK}{UTF8}{mj}全部考试时间总计\end{CJK} 180 \begin{CJK}{UTF8}{mj}分钟\end{CJK};

  \item \begin{CJK}{UTF8}{mj}所有答案必须写在答题纸上\end{CJK}, \begin{CJK}{UTF8}{mj}写在试题纸上或草稿纸上一律无效\end{CJK}。

  \item \begin{CJK}{UTF8}{mj}计算题\end{CJK}

\end{enumerate}
(1). \begin{CJK}{UTF8}{mj}求极限\end{CJK} $\lim _{x \rightarrow 0^{+}} \frac{\cos \frac{\pi \cos x}{2}}{\sin \sin ^{2} x}$;

(2). \begin{CJK}{UTF8}{mj}求广\end{CJK} \begin{CJK}{UTF8}{mj}义积分\end{CJK} $\int_{0}^{+\infty} \frac{\sin ^{2} x}{x^{2}} \mathrm{~d} x$

(3). \begin{CJK}{UTF8}{mj}求和\end{CJK} $\sum_{n=1}^{\infty}(-1)^{n} \frac{1}{3 n-1}$

(4). \begin{CJK}{UTF8}{mj}求曲线积分\end{CJK}
$$
\int_{L}\left(y^{2}-z^{2}\right) \mathrm{d} x+\left(x^{2}-z^{2}\right) \mathrm{d} y+\left(x^{2}-y^{2}\right) \mathrm{d} z
$$
\begin{CJK}{UTF8}{mj}其中\end{CJK} $L$ \begin{CJK}{UTF8}{mj}是\end{CJK} $\{(x, y, z) \mid 0 \leq x \leq a, 0 \leq y \leq a, 0 \leq z \leq a\}$ \begin{CJK}{UTF8}{mj}与\end{CJK} $x+y+z=\frac{3 a}{2}$ \begin{CJK}{UTF8}{mj}的交线\end{CJK}, \begin{CJK}{UTF8}{mj}从\end{CJK} $z$ \begin{CJK}{UTF8}{mj}轴\end{CJK} \begin{CJK}{UTF8}{mj}正向看为逆时针\end{CJK}.

\begin{enumerate}
  \setcounter{enumi}{2}
  \item \begin{CJK}{UTF8}{mj}利用闭区间套定理证明确界原理\end{CJK}.

  \item \begin{CJK}{UTF8}{mj}设\end{CJK} $f(x)$ \begin{CJK}{UTF8}{mj}在\end{CJK} $[a, b]$ \begin{CJK}{UTF8}{mj}上可导\end{CJK}, $f(0)=0$, \begin{CJK}{UTF8}{mj}对\end{CJK} $\forall x \in[a, b]$, \begin{CJK}{UTF8}{mj}有\end{CJK} $|f(x)| \geq\left|f^{\prime}(x)\right|$, \begin{CJK}{UTF8}{mj}试证\end{CJK}: $f(x) \equiv 0, x \in$ $[a, b]$.

  \item \begin{CJK}{UTF8}{mj}若在\end{CJK} $[a, b]$ \begin{CJK}{UTF8}{mj}上的黎曼可积函数列\end{CJK} $\left\{f_{n}(x)\right\}$ \begin{CJK}{UTF8}{mj}在\end{CJK} $[a, b]$ \begin{CJK}{UTF8}{mj}上一致收敛于\end{CJK} $f(x)$. \begin{CJK}{UTF8}{mj}证明\end{CJK}: $f(x)$ \begin{CJK}{UTF8}{mj}在\end{CJK} $[a, b]$ \begin{CJK}{UTF8}{mj}上也黎曼可积\end{CJK}, \begin{CJK}{UTF8}{mj}并且\end{CJK}

\end{enumerate}
$$
\lim _{n \rightarrow \infty} \int_{a}^{b} f_{n}(x) \mathrm{d} x=\int_{a}^{b} f(x) \mathrm{d} x
$$

\begin{enumerate}
  \setcounter{enumi}{5}
  \item \begin{CJK}{UTF8}{mj}设函数列\end{CJK} $\left\{f_{n}(x)\right\}$ \begin{CJK}{UTF8}{mj}在有限闭区间\end{CJK} $[a, b]$ \begin{CJK}{UTF8}{mj}上连续\end{CJK}, \begin{CJK}{UTF8}{mj}如果对每一个\end{CJK} $x \in[a, b]$, \begin{CJK}{UTF8}{mj}数列\end{CJK} $\left\{f_{n}(x)\right\}$ \begin{CJK}{UTF8}{mj}单调递\end{CJK} \begin{CJK}{UTF8}{mj}减趋于\end{CJK} $f(x)$. \begin{CJK}{UTF8}{mj}证明\end{CJK}: $\left\{f_{n}(x)\right\}$ \begin{CJK}{UTF8}{mj}在\end{CJK} $[a, b]$ \begin{CJK}{UTF8}{mj}上一致收敛于\end{CJK} $f(x)$.

  \item \begin{CJK}{UTF8}{mj}若函数\end{CJK} $f(x)$ \begin{CJK}{UTF8}{mj}在\end{CJK} $[a, b]$ \begin{CJK}{UTF8}{mj}上可导并且\end{CJK} $f_{+}^{\prime}(a)<f_{-}^{\prime}(b)$. \begin{CJK}{UTF8}{mj}证明\end{CJK}: \begin{CJK}{UTF8}{mj}对任意\end{CJK} $A: f_{+}^{\prime}(a)<A<f_{-}^{\prime}(b)$, \begin{CJK}{UTF8}{mj}均存在\end{CJK} $\xi \in(a, b)$ \begin{CJK}{UTF8}{mj}使得\end{CJK} $f^{\prime}(\xi)=A$. 7. \begin{CJK}{UTF8}{mj}证明\end{CJK}: \begin{CJK}{UTF8}{mj}积分\end{CJK} $\int_{a}^{+\infty} f(x, u) \mathrm{d} x$ \begin{CJK}{UTF8}{mj}在\end{CJK} $[\alpha, \beta]$ \begin{CJK}{UTF8}{mj}上一致收敛的充要条件是\end{CJK}: \begin{CJK}{UTF8}{mj}对任意单调递增趋于正无穷\end{CJK} \begin{CJK}{UTF8}{mj}的数列\end{CJK} $\left\{A_{n}\right\}\left(A_{1}=a\right)$, \begin{CJK}{UTF8}{mj}函数项级数\end{CJK}

\end{enumerate}
$$
\sum_{n=1}^{\infty} \int_{A_{n}}^{A_{n+1}} f(x, u) \mathrm{d} x
$$
\begin{CJK}{UTF8}{mj}在\end{CJK} $[\alpha, \beta]$ \begin{CJK}{UTF8}{mj}上一致收敛\end{CJK}.

\section{$6.22020$ 年高等代数考研真题}
\section{考生须知:}
\begin{enumerate}
  \item \begin{CJK}{UTF8}{mj}本试卷满分为\end{CJK} 150 \begin{CJK}{UTF8}{mj}分\end{CJK}, \begin{CJK}{UTF8}{mj}全部考试时间总计\end{CJK} 180 \begin{CJK}{UTF8}{mj}分钟\end{CJK};

  \item \begin{CJK}{UTF8}{mj}所有答案必须写在答题纸上\end{CJK},\begin{CJK}{UTF8}{mj}写在试题纸上或草稿纸上\end{CJK}―\begin{CJK}{UTF8}{mj}律无效\end{CJK}。

  \item \begin{CJK}{UTF8}{mj}若\end{CJK}

\end{enumerate}
$$
s(x)= \begin{cases}\frac{x}{|x|}, & x \neq 0 \\ 0, & x=0\end{cases}
$$
\begin{CJK}{UTF8}{mj}已知\end{CJK} $A=\left(a_{i j}\right), a_{i j}=s(i-j)$, \begin{CJK}{UTF8}{mj}求\end{CJK} $|A|$.

\begin{enumerate}
  \setcounter{enumi}{2}
  \item \begin{CJK}{UTF8}{mj}若\end{CJK} $x, y, z$ \begin{CJK}{UTF8}{mj}为复数\end{CJK}, \begin{CJK}{UTF8}{mj}证明\end{CJK}: $x, y, z$ \begin{CJK}{UTF8}{mj}在复平面上共线当且仅当\end{CJK} $\left|\begin{array}{ccc}x & y & z \\ \bar{x} & \bar{y} & \bar{z} \\ 1 & 1 & 1\end{array}\right|=0$

  \item \begin{CJK}{UTF8}{mj}若\end{CJK} $A, B$ \begin{CJK}{UTF8}{mj}为实方阵\end{CJK}, \begin{CJK}{UTF8}{mj}存在\end{CJK} $n$ \begin{CJK}{UTF8}{mj}阶可逆复方阵\end{CJK} $X$, \begin{CJK}{UTF8}{mj}使得\end{CJK} $X A+2 B X=O$, \begin{CJK}{UTF8}{mj}证明\end{CJK}: \begin{CJK}{UTF8}{mj}存在\end{CJK} $n$ \begin{CJK}{UTF8}{mj}阶可逆实\end{CJK} \begin{CJK}{UTF8}{mj}方阵\end{CJK} $Y$, \begin{CJK}{UTF8}{mj}使得\end{CJK} $Y A+2 B Y=O$.

  \item \begin{CJK}{UTF8}{mj}若\end{CJK} $A, B$ \begin{CJK}{UTF8}{mj}为可逆实方阵\end{CJK}, \begin{CJK}{UTF8}{mj}证明\end{CJK}: \begin{CJK}{UTF8}{mj}存在唯一的负定阵\end{CJK} $P$ \begin{CJK}{UTF8}{mj}和正交阵\end{CJK} $Q$, \begin{CJK}{UTF8}{mj}使得\end{CJK} $A=P Q$.

  \item \begin{CJK}{UTF8}{mj}若\end{CJK} $A_{i}(i=1,2, \cdots, s)$ \begin{CJK}{UTF8}{mj}为\end{CJK} $n$ \begin{CJK}{UTF8}{mj}阶复方阵\end{CJK}, $\sum_{i=1}^{s} A_{i}=I_{n}, \operatorname{rank}\left(A_{i}\right)=r_{j}(i=1,2, \cdots, s)$ \begin{CJK}{UTF8}{mj}证明\end{CJK}:

\end{enumerate}
$$
A_{i} A_{j}=\delta_{i j} A_{j}(i, j=1,2, \cdots, s) \text { 当且仅当 } \sum_{i=1}^{s} r_{i}=n
$$

\begin{enumerate}
  \setcounter{enumi}{6}
  \item \begin{CJK}{UTF8}{mj}设\end{CJK} $V$ \begin{CJK}{UTF8}{mj}是复向量空间\end{CJK}, $U, W$ \begin{CJK}{UTF8}{mj}是\end{CJK} $V$ \begin{CJK}{UTF8}{mj}的子空间\end{CJK}, $p, q$ \begin{CJK}{UTF8}{mj}是\end{CJK} $W$ \begin{CJK}{UTF8}{mj}中的向量\end{CJK}, \begin{CJK}{UTF8}{mj}且\end{CJK}
\end{enumerate}
$$
p+U=\{p+u \mid u \in U\}, q+W=\{p+w \mid w \in W\}
$$
\begin{CJK}{UTF8}{mj}若\end{CJK} $p+U=q+W$, \begin{CJK}{UTF8}{mj}证明\end{CJK}: $U=W$. 7. \begin{CJK}{UTF8}{mj}设\end{CJK} $A$ \begin{CJK}{UTF8}{mj}为\end{CJK} $n$ \begin{CJK}{UTF8}{mj}阶复方阵\end{CJK}, \begin{CJK}{UTF8}{mj}且\end{CJK} $A$ \begin{CJK}{UTF8}{mj}为幂零阵\end{CJK}, \begin{CJK}{UTF8}{mj}即存在正整数\end{CJK} $s$, \begin{CJK}{UTF8}{mj}使得\end{CJK} $A^{s}=O$, \begin{CJK}{UTF8}{mj}令\end{CJK} $\mathrm{e}^{A}=\sum_{i=0}^{\infty} \frac{1}{i !} A^{i}$, \begin{CJK}{UTF8}{mj}证\end{CJK} \begin{CJK}{UTF8}{mj}明\end{CJK}: $\mathrm{e}^{A}$ \begin{CJK}{UTF8}{mj}与\end{CJK} $I_{n}+A$ \begin{CJK}{UTF8}{mj}相似\end{CJK}.

\begin{enumerate}
  \setcounter{enumi}{8}
  \item \begin{CJK}{UTF8}{mj}设\end{CJK} $V$ \begin{CJK}{UTF8}{mj}是\end{CJK} $n$ \begin{CJK}{UTF8}{mj}维复线性空间\end{CJK}, $B=\left\{v_{1}, v_{2}, \cdots, \ldots v_{n}\right\}$ \begin{CJK}{UTF8}{mj}是\end{CJK} $V$ \begin{CJK}{UTF8}{mj}的一组基\end{CJK}, $J=\left(a_{i j}\right)(i, j=1,2, \cdots, s)$ \begin{CJK}{UTF8}{mj}为\end{CJK} $n$ \begin{CJK}{UTF8}{mj}阶方阵\end{CJK}, $a_{i j}=1$ \begin{CJK}{UTF8}{mj}当且仅当\end{CJK} $j=i+1(i=1,22, \cdots, n-1)$, \begin{CJK}{UTF8}{mj}否则\end{CJK} $a_{i j}=0$. \begin{CJK}{UTF8}{mj}已知\end{CJK} $V$ \begin{CJK}{UTF8}{mj}上线性\end{CJK} \begin{CJK}{UTF8}{mj}变换\end{CJK} $\mathscr{A}$ \begin{CJK}{UTF8}{mj}关于基\end{CJK} $B$ \begin{CJK}{UTF8}{mj}的矩阵为\end{CJK} $\lambda I n+J$. \begin{CJK}{UTF8}{mj}令\end{CJK} $V_{s}$ \begin{CJK}{UTF8}{mj}为\end{CJK} $V$ \begin{CJK}{UTF8}{mj}的由基\end{CJK} $B$ \begin{CJK}{UTF8}{mj}中的前\end{CJK} $s$ \begin{CJK}{UTF8}{mj}个向量\end{CJK} $v_{1}, v_{2}, \cdots, v_{s}$ \begin{CJK}{UTF8}{mj}生\end{CJK} \begin{CJK}{UTF8}{mj}成的子空间\end{CJK} $(s=1,2, \cdots, n)$, \begin{CJK}{UTF8}{mj}证明\end{CJK}:
\end{enumerate}
(1). $V_{s}$ \begin{CJK}{UTF8}{mj}为\end{CJK} $V$ \begin{CJK}{UTF8}{mj}上的\end{CJK} $\mathscr{A}$ \begin{CJK}{UTF8}{mj}的不变子空间\end{CJK}, \begin{CJK}{UTF8}{mj}且\end{CJK} $(\mathscr{A}-\lambda \mathscr{E})^{s}(v)=0$ \begin{CJK}{UTF8}{mj}当且仅当\end{CJK} $v \in V_{s}$.

(2). \begin{CJK}{UTF8}{mj}若\end{CJK} $W$ \begin{CJK}{UTF8}{mj}是\end{CJK} $V$ \begin{CJK}{UTF8}{mj}的\end{CJK} $s$ \begin{CJK}{UTF8}{mj}维\end{CJK} $\mathscr{A}$ \begin{CJK}{UTF8}{mj}的不变子空间\end{CJK} $(1 \leq s \leq n)$, \begin{CJK}{UTF8}{mj}则\end{CJK} $W=V$.

\begin{enumerate}
  \setcounter{enumi}{9}
  \item \begin{CJK}{UTF8}{mj}设\end{CJK} $\mathrm{e}_{1}, \mathrm{e}_{2}, \cdots, \mathrm{e}_{n}$ \begin{CJK}{UTF8}{mj}是欧氏空间\end{CJK} $V$ \begin{CJK}{UTF8}{mj}的一组标准正交基\end{CJK} (\begin{CJK}{UTF8}{mj}每个\end{CJK} $\mathrm{e}_{i}$ \begin{CJK}{UTF8}{mj}长度为\end{CJK} 1 \begin{CJK}{UTF8}{mj}且这\end{CJK} $n$ \begin{CJK}{UTF8}{mj}个向量两两正交\end{CJK}), $v_{1}, v_{2}, \cdots, v_{n}$ \begin{CJK}{UTF8}{mj}是\end{CJK} $V$ \begin{CJK}{UTF8}{mj}中的\end{CJK} $n$ \begin{CJK}{UTF8}{mj}个向量\end{CJK}, \begin{CJK}{UTF8}{mj}且\end{CJK} $\left\|\mathrm{e}_{i}-v_{i}\right\|<\frac{1}{\sqrt{n}}(i=1,2, \cdots, n)$, \begin{CJK}{UTF8}{mj}证明\end{CJK}: $v_{1}, v_{2}, \cdots, v_{n}$ \begin{CJK}{UTF8}{mj}是\end{CJK} $V$ \begin{CJK}{UTF8}{mj}的一组基\end{CJK}.

  \item \begin{CJK}{UTF8}{mj}设\end{CJK} $f_{i}(x)=a_{i 0}+a_{i 1} x+\cdots+a_{i, n-1} x^{n-1}+a_{i n} x^{n}(i=0,1, \cdots, n)$, \begin{CJK}{UTF8}{mj}且这些\end{CJK} $a_{i j}$ \begin{CJK}{UTF8}{mj}皆为整数\end{CJK}, \begin{CJK}{UTF8}{mj}设\end{CJK} $A=\left(a_{i j}\right)(i, j=0,1, \cdots, n)$ \begin{CJK}{UTF8}{mj}为相应的\end{CJK} $n+1$ \begin{CJK}{UTF8}{mj}阶方阵\end{CJK}. \begin{CJK}{UTF8}{mj}证明\end{CJK}:

\end{enumerate}
(1). \begin{CJK}{UTF8}{mj}对于任意正整数\end{CJK} $k,\left(f_{0}(k), f_{1}(k), \cdots, f_{n}(k)\right) \| A \mid$

(2). \begin{CJK}{UTF8}{mj}存在\end{CJK} $n+1$ \begin{CJK}{UTF8}{mj}阶方阵\end{CJK} $B$ \begin{CJK}{UTF8}{mj}使得\end{CJK} $A B=I_{n+1}$ \begin{CJK}{UTF8}{mj}的充分必要条件是存在\end{CJK} $n+1$ \begin{CJK}{UTF8}{mj}个互不相同的整数\end{CJK} $b_{0}, b_{1}, \cdots, b_{n}$, \begin{CJK}{UTF8}{mj}使得\end{CJK} $n+1$ \begin{CJK}{UTF8}{mj}阶方阵\end{CJK} $\left(f_{i}\left(b_{j}\right)\right)(i, j=0,1, \cdots, n)$ \begin{CJK}{UTF8}{mj}满足\end{CJK} $|D|=\prod_{0 \leq i<j \leq n}\left(b_{j}-b_{i}\right)$

\section{第 7 章 哈尔滨工业大学}
\section{$7.12020$ 年数学分析考研真题}
\section{考生须知:}
\begin{enumerate}
  \item \begin{CJK}{UTF8}{mj}本试卷满分为\end{CJK} 150 \begin{CJK}{UTF8}{mj}分\end{CJK}, \begin{CJK}{UTF8}{mj}全部考试时间总计\end{CJK} 180 \begin{CJK}{UTF8}{mj}分钟\end{CJK};

  \item \begin{CJK}{UTF8}{mj}所有答案必须写在答题纸上\end{CJK}, \begin{CJK}{UTF8}{mj}写在试题纸上或草稿纸上一律无效\end{CJK}。

  \item \begin{CJK}{UTF8}{mj}判断下列命题成立与否\end{CJK}, \begin{CJK}{UTF8}{mj}并给出证明\end{CJK}.

\end{enumerate}
(1). $f(x)$ \begin{CJK}{UTF8}{mj}在\end{CJK} $x=0$ \begin{CJK}{UTF8}{mj}的任意邻域上无界\end{CJK}, \begin{CJK}{UTF8}{mj}则当\end{CJK} $x \rightarrow 0$ \begin{CJK}{UTF8}{mj}时\end{CJK}, $f(x)$ \begin{CJK}{UTF8}{mj}为无穷大\end{CJK};

(2). \begin{CJK}{UTF8}{mj}数列\end{CJK} $\left\{a_{n}\right\}$ \begin{CJK}{UTF8}{mj}的无穷多个子列都收敛于\end{CJK} $a$, \begin{CJK}{UTF8}{mj}是否可以判定\end{CJK} $\lim _{n \rightarrow+\infty} a_{n}=a$;

(3). \begin{CJK}{UTF8}{mj}设\end{CJK} $f(x)$ \begin{CJK}{UTF8}{mj}在有限闭区间\end{CJK} $[a, b]$ \begin{CJK}{UTF8}{mj}上处处可导\end{CJK}, \begin{CJK}{UTF8}{mj}则\end{CJK} $f(x)$ \begin{CJK}{UTF8}{mj}有界\end{CJK}.

(4). \begin{CJK}{UTF8}{mj}非负数列\end{CJK} $\left\{u_{n}\right\}$ \begin{CJK}{UTF8}{mj}满足\end{CJK} $u_{n}=o\left(\frac{1}{n}\right)$, \begin{CJK}{UTF8}{mj}则\end{CJK} $\sum_{n=1}^{\infty} u_{n}$ \begin{CJK}{UTF8}{mj}收敛\end{CJK}.

(5). \begin{CJK}{UTF8}{mj}设\end{CJK} $f(x, y)$ \begin{CJK}{UTF8}{mj}在\end{CJK} $(0,0)$ \begin{CJK}{UTF8}{mj}上的偏导数都存在\end{CJK}, \begin{CJK}{UTF8}{mj}则\end{CJK} $f(x, y)$ \begin{CJK}{UTF8}{mj}在\end{CJK} $(0,0)$ \begin{CJK}{UTF8}{mj}处连续\end{CJK}.

2 . \begin{CJK}{UTF8}{mj}设\end{CJK} $\lim _{n \rightarrow+\infty} a_{n}=a$, \begin{CJK}{UTF8}{mj}证明\end{CJK}:
$$
\lim _{n \rightarrow \infty} \frac{\frac{1}{2} a_{1}+\frac{2}{3} a_{2}+\cdots+\frac{n}{n+1} a_{n}}{n}=a
$$

\begin{enumerate}
  \setcounter{enumi}{3}
  \item \begin{CJK}{UTF8}{mj}叙述闭区间连续函数的\end{CJK} Cantor \begin{CJK}{UTF8}{mj}定理\end{CJK}, \begin{CJK}{UTF8}{mj}并证明\end{CJK}.

  \item \begin{CJK}{UTF8}{mj}若函数\end{CJK} $f(x)$ \begin{CJK}{UTF8}{mj}在\end{CJK} $(0, a]$ \begin{CJK}{UTF8}{mj}上可导\end{CJK}, \begin{CJK}{UTF8}{mj}且\end{CJK}\\
(1). $\sqrt{x} f^{\prime}(x)$ \begin{CJK}{UTF8}{mj}在\end{CJK} $(0, a]$ \begin{CJK}{UTF8}{mj}上有界\end{CJK}, \begin{CJK}{UTF8}{mj}求证\end{CJK}: $f(x)$ \begin{CJK}{UTF8}{mj}在\end{CJK} $(0, a]$ \begin{CJK}{UTF8}{mj}上一致连续\end{CJK};\\
(2). $\lim _{x \rightarrow 0^{+}} \sqrt{x} f^{\prime}(x)$ \begin{CJK}{UTF8}{mj}在\end{CJK} $(0, a]$ \begin{CJK}{UTF8}{mj}上存在\end{CJK}, \begin{CJK}{UTF8}{mj}求证\end{CJK}: $f(x)$ \begin{CJK}{UTF8}{mj}在\end{CJK} $(0, a]$ \begin{CJK}{UTF8}{mj}上一致连续\end{CJK};

  \item \begin{CJK}{UTF8}{mj}函数\end{CJK} $f(x)$ \begin{CJK}{UTF8}{mj}在\end{CJK} $[a, b]$ \begin{CJK}{UTF8}{mj}上具有连续的二阶导数\end{CJK}, \begin{CJK}{UTF8}{mj}则存在\end{CJK} $c \in[a, b]$, \begin{CJK}{UTF8}{mj}使得\end{CJK}

\end{enumerate}
$$
\int_{a}^{b} f(x) \mathrm{d} x=(b-a) f\left(\frac{a+b}{2}\right)+\frac{1}{24}(b-a)^{3} f^{\prime \prime}(c)
$$

\begin{enumerate}
  \setcounter{enumi}{6}
  \item \begin{CJK}{UTF8}{mj}讨论\end{CJK} $\sum_{n=1}^{\infty} \ln \left(1+\frac{(-1)^{n}}{n^{p}}\right)$ \begin{CJK}{UTF8}{mj}的收敛性和绝对收敛性\end{CJK} $(p>0)$. 7. (1). \begin{CJK}{UTF8}{mj}函数项级数\end{CJK} $\sum_{n=1}^{\infty} u_{n}(x)$ \begin{CJK}{UTF8}{mj}在区间\end{CJK} $I$ \begin{CJK}{UTF8}{mj}上一致收玫\end{CJK}, \begin{CJK}{UTF8}{mj}求证其通项\end{CJK} $u_{n}(x)$ \begin{CJK}{UTF8}{mj}在\end{CJK} I \begin{CJK}{UTF8}{mj}上一致收玫于\end{CJK} 0 . (2). \begin{CJK}{UTF8}{mj}讨论级数\end{CJK} $\sum_{n=1}^{\infty} 2^{n} \sin \frac{1}{3^{n} x}$ \begin{CJK}{UTF8}{mj}在\end{CJK} $x>0$ \begin{CJK}{UTF8}{mj}上的一致收玫性\end{CJK}.

  \item (1). \begin{CJK}{UTF8}{mj}试证\end{CJK}:\begin{CJK}{UTF8}{mj}方程在\end{CJK} $(0,0)$ \begin{CJK}{UTF8}{mj}的充分小邻域上确定唯一的连续函数\end{CJK} $y=y(x)$, \begin{CJK}{UTF8}{mj}使得\end{CJK} $y(0)=0$.

\end{enumerate}
(2). \begin{CJK}{UTF8}{mj}讨论\end{CJK} $y=y(x)$ \begin{CJK}{UTF8}{mj}在\end{CJK} $x=0$ \begin{CJK}{UTF8}{mj}处的可微性\end{CJK}.

(3). \begin{CJK}{UTF8}{mj}求极限\end{CJK} $\lim _{x \rightarrow 0} \frac{y(x)}{x}$.

\begin{enumerate}
  \setcounter{enumi}{9}
  \item \begin{CJK}{UTF8}{mj}求极限\end{CJK}
\end{enumerate}
$$
\lim _{t \rightarrow 0^{+}} \frac{1}{t^{4}} \iiint_{x^{2}+y^{2}+z^{2} \leq t^{2}} \sin \sqrt{x^{2}+y^{2}+z^{2}} \mathrm{~d} x \mathrm{~d} y \mathrm{~d} z
$$

\begin{enumerate}
  \setcounter{enumi}{10}
  \item \begin{CJK}{UTF8}{mj}计算曲面积分\end{CJK}
\end{enumerate}
$$
I=\iint_{S}\left(x^{2}+a z^{2}\right) \mathrm{d} y \mathrm{~d} z+\left(y^{2}+a x^{2}\right) \mathrm{d} z \mathrm{~d} x+\left(z^{2}+a y^{2}\right) \mathrm{d} x \mathrm{~d} y
$$
\begin{CJK}{UTF8}{mj}其中\end{CJK} $S$ \begin{CJK}{UTF8}{mj}为上半球\end{CJK} $z=\sqrt{a^{2}-x^{2}-y^{2}}$ \begin{CJK}{UTF8}{mj}的上侧\end{CJK}.

\section{$7.22020$ 年高等代数考研真题}
\section{考生须知:}
\begin{enumerate}
  \item \begin{CJK}{UTF8}{mj}本试卷满分为\end{CJK} 150 \begin{CJK}{UTF8}{mj}分\end{CJK}, \begin{CJK}{UTF8}{mj}全部考试时间总计\end{CJK} 180 \begin{CJK}{UTF8}{mj}分钟\end{CJK};

  \item \begin{CJK}{UTF8}{mj}所有答案必须写在答题纸上\end{CJK}, \begin{CJK}{UTF8}{mj}写在试题纸上或草稿纸上一律无效\end{CJK}。

  \item \begin{CJK}{UTF8}{mj}当\end{CJK} $n$ \begin{CJK}{UTF8}{mj}满足什么条件时\end{CJK}, $x^{2} n+x^{n}+1$ \begin{CJK}{UTF8}{mj}不可约\end{CJK}.

  \item \begin{CJK}{UTF8}{mj}已知\end{CJK} $f(x)=x^{3}-49 x-120$ \begin{CJK}{UTF8}{mj}的根为\end{CJK} $a, b, c$, \begin{CJK}{UTF8}{mj}求\end{CJK}

\end{enumerate}
\includegraphics[max width=\textwidth]{2022_04_18_7db0708508f26638f054g-035}

\begin{CJK}{UTF8}{mj}其中\end{CJK} $s_{i}=a^{i}+b^{i}+c^{i}$.

\begin{enumerate}
  \setcounter{enumi}{3}
  \item \begin{CJK}{UTF8}{mj}求\end{CJK}
\end{enumerate}
$$
\left(\begin{array}{ccccc}
3 & 2 m & 0 & 0 & 0 \\
0 & 3 & 2 m & 0 & 0 \\
0 & 0 & 3 & 2 m & 0 \\
1 & m & 0 & 4 & 0 \\
0 & 1 & m & 0 & 4
\end{array}\right)
$$
\begin{CJK}{UTF8}{mj}的秩\end{CJK}, \begin{CJK}{UTF8}{mj}其中\end{CJK} $m \in \mathbb{R}$.

\begin{enumerate}
  \setcounter{enumi}{4}
  \item \begin{CJK}{UTF8}{mj}已知\end{CJK} $A$ \begin{CJK}{UTF8}{mj}为\end{CJK} $n \times n$ \begin{CJK}{UTF8}{mj}介实矩阵\end{CJK}, \begin{CJK}{UTF8}{mj}且\end{CJK} $A^{2}=-A A^{\prime}$, \begin{CJK}{UTF8}{mj}是否有\end{CJK} $A=-A^{\prime}$ ? \begin{CJK}{UTF8}{mj}若正确给出证明\end{CJK}, \begin{CJK}{UTF8}{mj}错误举出反\end{CJK} \begin{CJK}{UTF8}{mj}例\end{CJK}.

  \item \begin{CJK}{UTF8}{mj}已知\end{CJK} $A$ \begin{CJK}{UTF8}{mj}为实矩阵\end{CJK}, \begin{CJK}{UTF8}{mj}问\end{CJK}:\begin{CJK}{UTF8}{mj}是否存在可逆矩阵\end{CJK} $P$, \begin{CJK}{UTF8}{mj}使得\end{CJK} $P^{\prime}\left(A A^{\prime}+A+\bar{A}^{\prime}\right) P=A^{\prime} A+A+A^{\prime}$ ? \begin{CJK}{UTF8}{mj}若\end{CJK} \begin{CJK}{UTF8}{mj}正确给出证明\end{CJK}, \begin{CJK}{UTF8}{mj}错误举出反例\end{CJK}.

  \item \begin{CJK}{UTF8}{mj}已知\end{CJK} $A, B$ \begin{CJK}{UTF8}{mj}为\end{CJK} $4 \times 4$ \begin{CJK}{UTF8}{mj}阶的正定矩阵\end{CJK}, \begin{CJK}{UTF8}{mj}且\end{CJK} $A^{4}=B^{4}$, \begin{CJK}{UTF8}{mj}问\end{CJK}; \begin{CJK}{UTF8}{mj}是否有\end{CJK} $A=B$ \begin{CJK}{UTF8}{mj}成立\end{CJK}? \begin{CJK}{UTF8}{mj}若正确给出证明\end{CJK}, \begin{CJK}{UTF8}{mj}错误举出反例\end{CJK}.

\end{enumerate}
\section{第8章 华东师范大学}
\section{$8.12020$ 年数学分析考研真题}
\section{考生须知:}
\begin{enumerate}
  \item \begin{CJK}{UTF8}{mj}本试卷满分为\end{CJK} 150 \begin{CJK}{UTF8}{mj}分\end{CJK}, \begin{CJK}{UTF8}{mj}全部考试时间总计\end{CJK} 180 \begin{CJK}{UTF8}{mj}分钟\end{CJK};

  \item \begin{CJK}{UTF8}{mj}所有答案必须写在答题纸上\end{CJK}, \begin{CJK}{UTF8}{mj}写在试题纸上或草稿纸上一律无效\end{CJK}。

  \item \begin{CJK}{UTF8}{mj}判断下列命题是否正确\end{CJK}, \begin{CJK}{UTF8}{mj}若正确给出证明\end{CJK}, \begin{CJK}{UTF8}{mj}若错误举出反例\end{CJK} (\begin{CJK}{UTF8}{mj}每小题\end{CJK} 6 \begin{CJK}{UTF8}{mj}分\end{CJK}, \begin{CJK}{UTF8}{mj}共\end{CJK} 36 \begin{CJK}{UTF8}{mj}分\end{CJK})

\end{enumerate}
(1). $\lim _{n \rightarrow \infty} a_{n}=A$ \begin{CJK}{UTF8}{mj}的充要条件\end{CJK}:\begin{CJK}{UTF8}{mj}对任意的正整数\end{CJK} $k$, \begin{CJK}{UTF8}{mj}存在\end{CJK} $n>N$, \begin{CJK}{UTF8}{mj}当\end{CJK} $n>N$ \begin{CJK}{UTF8}{mj}时\end{CJK}, \begin{CJK}{UTF8}{mj}有\end{CJK}
$$
\left|a_{n}-A\right|<\frac{k}{k^{2}+1}
$$
(2). \begin{CJK}{UTF8}{mj}若\end{CJK} $f(x)$ \begin{CJK}{UTF8}{mj}在\end{CJK} $x=0$ \begin{CJK}{UTF8}{mj}的某邻域内有定义\end{CJK}, \begin{CJK}{UTF8}{mj}且\end{CJK}
$$
\lim _{x \rightarrow 0} \frac{f(x)-f(-x)}{2 x}
$$
\begin{CJK}{UTF8}{mj}存在\end{CJK}, \begin{CJK}{UTF8}{mj}则\end{CJK} $f^{\prime}(0)$ \begin{CJK}{UTF8}{mj}存在\end{CJK}.

(3). \begin{CJK}{UTF8}{mj}若\end{CJK} $f(x)$ \begin{CJK}{UTF8}{mj}在\end{CJK} $[a, b]$ \begin{CJK}{UTF8}{mj}上可积\end{CJK}, \begin{CJK}{UTF8}{mj}则\end{CJK} $f(x)$ \begin{CJK}{UTF8}{mj}在\end{CJK} $[a, b]$ \begin{CJK}{UTF8}{mj}上有原函数\end{CJK}.

(4). \begin{CJK}{UTF8}{mj}若\end{CJK} $f(x)$ \begin{CJK}{UTF8}{mj}在\end{CJK} $[0,1]$ \begin{CJK}{UTF8}{mj}上连续\end{CJK}, \begin{CJK}{UTF8}{mj}且\end{CJK} $\int_{0}^{1} f^{2}(x) \mathrm{d} x=0$, \begin{CJK}{UTF8}{mj}则\end{CJK} $f(x) \equiv 0, x \in[0,1]$.

(5). \begin{CJK}{UTF8}{mj}若级数\end{CJK} $\sum_{n=1}^{\infty} a_{n}$ \begin{CJK}{UTF8}{mj}与\end{CJK} $\sum_{n=1}^{\infty} b_{n}$ \begin{CJK}{UTF8}{mj}都收敛\end{CJK}, \begin{CJK}{UTF8}{mj}则\end{CJK} $\sum_{n=1}^{\infty} a_{n} b_{n}$ \begin{CJK}{UTF8}{mj}也收敛\end{CJK}.

$(6)$. \begin{CJK}{UTF8}{mj}设\end{CJK} $x_{0}$ \begin{CJK}{UTF8}{mj}是\end{CJK} $f(x)$ \begin{CJK}{UTF8}{mj}的一个极小值点\end{CJK}, \begin{CJK}{UTF8}{mj}则一定存在\end{CJK} $\delta>0$, \begin{CJK}{UTF8}{mj}使得\end{CJK} $f(x)$ \begin{CJK}{UTF8}{mj}在\end{CJK} $\left(x_{0}-\delta, x_{0}\right)$ \begin{CJK}{UTF8}{mj}上单调递\end{CJK} \begin{CJK}{UTF8}{mj}减\end{CJK}, \begin{CJK}{UTF8}{mj}在\end{CJK} $\left(x_{0}, x_{0}+\delta\right)$ \begin{CJK}{UTF8}{mj}上单调递增\end{CJK}.

\begin{enumerate}
  \setcounter{enumi}{2}
  \item \begin{CJK}{UTF8}{mj}计算题\end{CJK} (\begin{CJK}{UTF8}{mj}每小题\end{CJK} 8 \begin{CJK}{UTF8}{mj}分\end{CJK}, \begin{CJK}{UTF8}{mj}共\end{CJK} 40 \begin{CJK}{UTF8}{mj}分\end{CJK})
\end{enumerate}
(1). $\int_{0}^{2 \pi} \sqrt{1+\sin x} d x$

(2). $\lim _{x \rightarrow 0} \frac{\sin \sin x-\tan \tan x}{x^{3}}$

(3). $\sum_{n=0}^{\infty} \frac{(-1)^{n}}{3^{n}(2 n+1)}$ (4). \begin{CJK}{UTF8}{mj}计算\end{CJK}
$$
\iint_{\Sigma}\left(z^{2}+x\right) \mathrm{d} y \mathrm{~d} z+\sqrt{z} \mathrm{~d} x \mathrm{~d} y
$$
\begin{CJK}{UTF8}{mj}其中\end{CJK} $\Sigma$ \begin{CJK}{UTF8}{mj}为抛物面\end{CJK} $z=\frac{x^{2}+y^{2}}{2}$ \begin{CJK}{UTF8}{mj}在平面\end{CJK} $z=0$ \begin{CJK}{UTF8}{mj}与\end{CJK} $z=2$ \begin{CJK}{UTF8}{mj}之间的部分\end{CJK},\begin{CJK}{UTF8}{mj}方向取下侧\end{CJK}.

(5). \begin{CJK}{UTF8}{mj}已知\end{CJK} $\lim _{n \rightarrow+\infty} a_{n}=A$, \begin{CJK}{UTF8}{mj}求\end{CJK} $\lim _{n \rightarrow \infty} \sum_{k=1}^{n} \frac{a_{n+k}}{n+k}$

\begin{enumerate}
  \setcounter{enumi}{3}
  \item \begin{CJK}{UTF8}{mj}证明题\end{CJK} (\begin{CJK}{UTF8}{mj}第\end{CJK} 1 \begin{CJK}{UTF8}{mj}题\end{CJK} 14 \begin{CJK}{UTF8}{mj}分\end{CJK}, $2-5$ \begin{CJK}{UTF8}{mj}题\end{CJK} 15 \begin{CJK}{UTF8}{mj}分\end{CJK}, \begin{CJK}{UTF8}{mj}共\end{CJK} 74 \begin{CJK}{UTF8}{mj}分\end{CJK}) \begin{CJK}{UTF8}{mj}计算题\end{CJK}
\end{enumerate}
(1). \begin{CJK}{UTF8}{mj}设\end{CJK} $a_{n}>0, n=0,1,2, \cdots, S_{n}=a_{1}+a_{2}+\cdots+a_{n}$, \begin{CJK}{UTF8}{mj}试证\end{CJK}: $\sum_{n=1}^{\infty} a_{n}$ \begin{CJK}{UTF8}{mj}与\end{CJK} $\sum_{n=1}^{\infty} \frac{a_{n}}{s_{n}}$ \begin{CJK}{UTF8}{mj}具有相同的\end{CJK} \begin{CJK}{UTF8}{mj}敛散性\end{CJK}.

(2). \begin{CJK}{UTF8}{mj}已知数列\end{CJK} $\left\{a_{n}\right\}$ \begin{CJK}{UTF8}{mj}非负其有界\end{CJK}, \begin{CJK}{UTF8}{mj}证明\end{CJK}:
$$
\lim _{n \rightarrow \infty} \sqrt[n]{a_{1}^{n}+a_{2}^{n}+\cdots+a_{n}^{n}}=\sup _{n \geq 1} a_{n}
$$
(3). \begin{CJK}{UTF8}{mj}谋\end{CJK}
$$
Q(x)=\left\{\begin{array}{l}
q, x=\frac{p}{q} \in(0,1), p, q \text { 为互素的正整数 } \\
0,(0,1) \text { 上的其它点 }
\end{array}\right.
$$
\begin{CJK}{UTF8}{mj}证明\end{CJK}: \begin{CJK}{UTF8}{mj}对任意的\end{CJK} $x_{0} \in(0,1)$ \begin{CJK}{UTF8}{mj}以及任意的\end{CJK} $\delta>0$, \begin{CJK}{UTF8}{mj}在\end{CJK} $U\left(x_{0}, \delta\right) \cap(0,1)$ \begin{CJK}{UTF8}{mj}上无界\end{CJK}.

(4). \begin{CJK}{UTF8}{mj}若\end{CJK} $f(x)$ \begin{CJK}{UTF8}{mj}在\end{CJK} $[0,1]$ \begin{CJK}{UTF8}{mj}上连续\end{CJK}, \begin{CJK}{UTF8}{mj}且\end{CJK} $\int_{0}^{1} f^{2}(x) \mathrm{d} x=0$, \begin{CJK}{UTF8}{mj}则\end{CJK} $f(x) \equiv 0, x \in[0,1]$.

(5). \begin{CJK}{UTF8}{mj}设\end{CJK} $u_{n}(x)$ \begin{CJK}{UTF8}{mj}在\end{CJK} $[a, b]$ \begin{CJK}{UTF8}{mj}上连续\end{CJK}, \begin{CJK}{UTF8}{mj}且\end{CJK} $u_{n}(x) \geq 0, n=1,2, \cdots$. \begin{CJK}{UTF8}{mj}设\end{CJK} $\sum_{n=1}^{\infty} u_{n}$ \begin{CJK}{UTF8}{mj}在\end{CJK} $[a, b]$ \begin{CJK}{UTF8}{mj}上收敛于\end{CJK} $f(x)$ . \begin{CJK}{UTF8}{mj}证明\end{CJK}: $f(x)$ \begin{CJK}{UTF8}{mj}在\end{CJK} $[a, b]$ \begin{CJK}{UTF8}{mj}上有最小值\end{CJK}.

$(6)$. \begin{CJK}{UTF8}{mj}设\end{CJK} $f(x)$ \begin{CJK}{UTF8}{mj}是定义在\end{CJK} $[0,+\infty)$ \begin{CJK}{UTF8}{mj}上的非负函数且可微\end{CJK}, \begin{CJK}{UTF8}{mj}满足\end{CJK} $\int_{0}^{+\infty} f(x) \mathrm{d} x$ \begin{CJK}{UTF8}{mj}收敛\end{CJK}. \begin{CJK}{UTF8}{mj}证明\end{CJK}:\begin{CJK}{UTF8}{mj}存在\end{CJK} \begin{CJK}{UTF8}{mj}趋于正无穷的数列\end{CJK} $\left\{x_{n}\right\}$ \begin{CJK}{UTF8}{mj}使得\end{CJK}
$$
\lim _{n \rightarrow \infty}\left[f^{2}\left(x_{n}\right)+\left|f^{\prime}(x)\right|^{2}\right]=0
$$

\section{$8.22020$ 年高等代数考研真题}
\section{考生须知:}
\begin{enumerate}
  \item \begin{CJK}{UTF8}{mj}本试卷满分为\end{CJK} 150 \begin{CJK}{UTF8}{mj}分\end{CJK}, \begin{CJK}{UTF8}{mj}全部考试时间总计\end{CJK} 180 \begin{CJK}{UTF8}{mj}分钟\end{CJK}; 1. \begin{CJK}{UTF8}{mj}设\end{CJK} $A=\left(\begin{array}{ccc}-2 & 0 & -1 \\ 1 & 2 & b \\ a & \frac{2}{3} & 0\end{array}\right)$, \begin{CJK}{UTF8}{mj}求所有\end{CJK} $a, b$ \begin{CJK}{UTF8}{mj}的值\end{CJK}, \begin{CJK}{UTF8}{mj}使得\end{CJK} $A$ \begin{CJK}{UTF8}{mj}是幂零矩阵\end{CJK} (\begin{CJK}{UTF8}{mj}矩阵\end{CJK} $A$ \begin{CJK}{UTF8}{mj}称为幂零矩阵是指\end{CJK} \begin{CJK}{UTF8}{mj}存在正整数\end{CJK} $k$ \begin{CJK}{UTF8}{mj}使得\end{CJK} $\left.A^{k}=O\right)$.

  \item \begin{CJK}{UTF8}{mj}设\end{CJK} $\alpha_{1}, \alpha_{2}, \cdots, \alpha_{n}, \beta_{1}, \beta_{2}, \cdots, \beta_{n}$ \begin{CJK}{UTF8}{mj}是线性空间\end{CJK} $V$ \begin{CJK}{UTF8}{mj}中的\end{CJK} $2 n$ \begin{CJK}{UTF8}{mj}个向量\end{CJK}. \begin{CJK}{UTF8}{mj}已知对任意的\end{CJK} $1 \leq k \leq n$ \begin{CJK}{UTF8}{mj}及\end{CJK} $1 \leq i_{i}<\cdots<i_{k} \leq n$, \begin{CJK}{UTF8}{mj}有\end{CJK} $\alpha_{i 1}, \alpha_{i 2}, \cdots, \alpha_{i k}$ \begin{CJK}{UTF8}{mj}线性相关当且仅当\end{CJK} $\beta_{i 1}, \beta_{i 2}, \cdots, \beta_{i k}$ \begin{CJK}{UTF8}{mj}线性相关\end{CJK}:\begin{CJK}{UTF8}{mj}证\end{CJK} \begin{CJK}{UTF8}{mj}明\end{CJK}:\begin{CJK}{UTF8}{mj}向量组\end{CJK} $\alpha_{1}, \alpha_{2}, \cdots, \alpha_{n}$ \begin{CJK}{UTF8}{mj}的秩与向量组\end{CJK} $\beta_{1}, \beta_{2}, \cdots, \beta_{n}$ \begin{CJK}{UTF8}{mj}的秩相同\end{CJK}.

  \item \begin{CJK}{UTF8}{mj}已知\end{CJK} $n \geq 2, a, b \in \mathbb{C}$, \begin{CJK}{UTF8}{mj}求\end{CJK}

\end{enumerate}
$$
A=\left|\begin{array}{ccccc}
a & a & a & a & \cdots \\
a & b & b & b & \cdots \\
a & b & a & a & \cdots \\
a & b & b & b & \cdots \\
a & b & a & b & a \\
\vdots & \vdots & \vdots & \vdots & \ddots
\end{array}\right|
$$

\begin{enumerate}
  \setcounter{enumi}{4}
  \item \begin{CJK}{UTF8}{mj}设\end{CJK} $A=\left(\begin{array}{ccc}\frac{7}{4} & -\frac{3}{4} & \frac{\sqrt{6}}{4} \\ -\frac{3}{4} & \frac{7}{4} & -\frac{\sqrt{6}}{4} \\ \frac{\sqrt{6}}{4} & -\frac{\sqrt{6}}{4} & \frac{7}{2}\end{array}\right)$
\end{enumerate}
(1). \begin{CJK}{UTF8}{mj}求一个正交矩阵\end{CJK} $P$, \begin{CJK}{UTF8}{mj}使得\end{CJK} $P^{\prime} A P$ \begin{CJK}{UTF8}{mj}是对角矩阵\end{CJK}.

(2). \begin{CJK}{UTF8}{mj}求\end{CJK}
$$
f\left(x_{1}, x_{2}, x_{3}\right)=\frac{7}{4} x_{1}^{2}+\frac{7}{4} x_{2}^{2}+\frac{7}{2} x_{3}^{2}-\frac{3}{2} x_{1} x_{2}-\frac{\sqrt{6}}{2} x_{2} x_{3}+\frac{\sqrt{6}}{2} x_{1} x_{3}
$$
\begin{CJK}{UTF8}{mj}在单位球面\end{CJK} $S^{2}=\left\{\left(x_{1}, x_{2}, x_{3}\right) \in \mathbb{R}^{3} \mid x_{1}^{2}+x_{2}^{2}+x_{3}^{2}=1\right\}$ \begin{CJK}{UTF8}{mj}上能取到的最大值\end{CJK}, \begin{CJK}{UTF8}{mj}并求出能取\end{CJK} \begin{CJK}{UTF8}{mj}到最大值的所有\end{CJK} $\left(x_{1}, x_{2}, x_{3}\right)$.

\begin{enumerate}
  \setcounter{enumi}{5}
  \item \begin{CJK}{UTF8}{mj}已知矩阵\end{CJK} $A \in M_{n} \in(\mathbb{C})$ \begin{CJK}{UTF8}{mj}满足\end{CJK} $A+A^{2}+\frac{1}{2 !} A^{3}+\frac{1}{3 !} A^{4}+\cdots+\frac{1}{2019 !} A^{2020}=O$, \begin{CJK}{UTF8}{mj}证明\end{CJK}: $A$ \begin{CJK}{UTF8}{mj}可对\end{CJK} \begin{CJK}{UTF8}{mj}角化\end{CJK}.

  \item \begin{CJK}{UTF8}{mj}设\end{CJK} $A, B \in M_{n}(\mathbb{C})$, \begin{CJK}{UTF8}{mj}令\end{CJK} $L(A, B)=\left\{X \in M_{n}(\mathbb{C}) \mid A X B=O\right\}$

\end{enumerate}
(1). \begin{CJK}{UTF8}{mj}验证\end{CJK} $L(A, B)$ \begin{CJK}{UTF8}{mj}是\end{CJK} $M_{n} \in(\mathbb{C})$ \begin{CJK}{UTF8}{mj}的线性子空间\end{CJK};

(2). \begin{CJK}{UTF8}{mj}设\end{CJK} $\operatorname{rank}(A)=r, \operatorname{rank}(B)=s$ ,\begin{CJK}{UTF8}{mj}求\end{CJK} $\operatorname{dim} L(A, B)$ (\begin{CJK}{UTF8}{mj}用\end{CJK} $m, r, s$ \begin{CJK}{UTF8}{mj}表示\end{CJK}).

\begin{enumerate}
  \setcounter{enumi}{7}
  \item \begin{CJK}{UTF8}{mj}设\end{CJK} $A, B, C$ \begin{CJK}{UTF8}{mj}是二阶复方阵\end{CJK},\begin{CJK}{UTF8}{mj}且\end{CJK} $A, B, C$ \begin{CJK}{UTF8}{mj}在\end{CJK} $M_{2}(\mathbb{C})$ \begin{CJK}{UTF8}{mj}中线性无关\end{CJK}. \begin{CJK}{UTF8}{mj}证明\end{CJK}:\begin{CJK}{UTF8}{mj}存在复数\end{CJK} $x_{1}, x_{2}, x_{3}$ \begin{CJK}{UTF8}{mj}使得\end{CJK} $x_{1} A+x_{2} B+x_{3} C$ \begin{CJK}{UTF8}{mj}是可逆矩阵\end{CJK}.

  \item \begin{CJK}{UTF8}{mj}证明\end{CJK}: (1). \begin{CJK}{UTF8}{mj}已知\end{CJK} $A \in M_{n} \in(\mathbb{C})$, \begin{CJK}{UTF8}{mj}若存在可逆矩阵\end{CJK} $B \in M_{n} \in(\mathbb{C})$, \begin{CJK}{UTF8}{mj}使得\end{CJK} $A=B^{-1} \bar{B}$, \begin{CJK}{UTF8}{mj}则\end{CJK} $A^{-1}=\bar{A}$.

\end{enumerate}
(2). \begin{CJK}{UTF8}{mj}设可逆矩阵\end{CJK} $A \in M_{n}(\mathbb{C})$ \begin{CJK}{UTF8}{mj}满足\end{CJK} $A^{-1}=\bar{A}$, \begin{CJK}{UTF8}{mj}证明\end{CJK}: \begin{CJK}{UTF8}{mj}存在可逆矩阵\end{CJK} $B \in\{a \bar{A}+b E \mid a, b \in \mathbb{C}\}$ \begin{CJK}{UTF8}{mj}使\end{CJK} \begin{CJK}{UTF8}{mj}得\end{CJK} $A=B^{-1} \bar{B}$, \begin{CJK}{UTF8}{mj}其中\end{CJK} $\bar{A}$ \begin{CJK}{UTF8}{mj}为\end{CJK} $A$ \begin{CJK}{UTF8}{mj}的共轭矩阵\end{CJK}, $E$ \begin{CJK}{UTF8}{mj}为单位矩阵\end{CJK}。

\begin{enumerate}
  \setcounter{enumi}{9}
  \item \begin{CJK}{UTF8}{mj}设\end{CJK} $n$ \begin{CJK}{UTF8}{mj}为奇数\end{CJK}, $A, B \in M_{n}(\mathbb{C})$ \begin{CJK}{UTF8}{mj}且\end{CJK} $A^{2}=O$, \begin{CJK}{UTF8}{mj}证明\end{CJK}: $A B-B A$ \begin{CJK}{UTF8}{mj}不可逆\end{CJK}.
\end{enumerate}
\section{第 9 章 上海交通大学}
\section{$9.12020$ 年数学分析考研真题}
\section{考生须知:}
\begin{enumerate}
  \item \begin{CJK}{UTF8}{mj}本试卷满分为\end{CJK} 150 \begin{CJK}{UTF8}{mj}分\end{CJK}, \begin{CJK}{UTF8}{mj}全部考试时间总计\end{CJK} 180 \begin{CJK}{UTF8}{mj}分钟\end{CJK};

  \item \begin{CJK}{UTF8}{mj}所有答案必须写在答题纸上\end{CJK}, \begin{CJK}{UTF8}{mj}写在试题纸上或草稿纸上一律无效\end{CJK}。

  \item \begin{CJK}{UTF8}{mj}判断题\end{CJK} $(5 \times 6=30$ \begin{CJK}{UTF8}{mj}分\end{CJK}). \begin{CJK}{UTF8}{mj}若正确给出证明\end{CJK}, \begin{CJK}{UTF8}{mj}若错误举出反例\end{CJK}.

\end{enumerate}
(1). \begin{CJK}{UTF8}{mj}若数列\end{CJK} $\left\{x_{n}\right\}$ \begin{CJK}{UTF8}{mj}收玫且\end{CJK} $a_{n}>0$, \begin{CJK}{UTF8}{mj}则\end{CJK} $\lim _{n \rightarrow+\infty} \frac{a_{n+1}}{a_{n}}=1$.

(2). \begin{CJK}{UTF8}{mj}若函数\end{CJK} $f(x)$ \begin{CJK}{UTF8}{mj}在点\end{CJK} $x_{0}$ \begin{CJK}{UTF8}{mj}二阶可导\end{CJK}, \begin{CJK}{UTF8}{mj}则函数\end{CJK} $f(x)$ \begin{CJK}{UTF8}{mj}在点\end{CJK} $x_{0}$ \begin{CJK}{UTF8}{mj}附近连续\end{CJK};

(3). \begin{CJK}{UTF8}{mj}耇函数\end{CJK} $f(x), g(x)$ \begin{CJK}{UTF8}{mj}在\end{CJK} $[a, b]$ \begin{CJK}{UTF8}{mj}上可导\end{CJK}, \begin{CJK}{UTF8}{mj}且\end{CJK} $a=\min _{x \in[a, b]} g(x), b=\max _{x \in[a, b]} g(x)$, \begin{CJK}{UTF8}{mj}则存在\end{CJK} $\xi \in[a, b]$ \begin{CJK}{UTF8}{mj}使得\end{CJK}
$$
f(b)-f(a)=f^{\prime}(g(\xi))(b-a)
$$
(4). \begin{CJK}{UTF8}{mj}若级数\end{CJK} $\sum_{n=1}^{\infty} a_{n} 与 \sum_{n=1}^{\infty} b_{n}$ \begin{CJK}{UTF8}{mj}收敛\end{CJK}, \begin{CJK}{UTF8}{mj}则\end{CJK} $\sum_{n=1}^{\infty}\left(a_{n}+b_{n}\right)^{2}$ \begin{CJK}{UTF8}{mj}也收敛\end{CJK}.

$(5)$. \begin{CJK}{UTF8}{mj}设\end{CJK} $D$ \begin{CJK}{UTF8}{mj}为凸区域\end{CJK}, \begin{CJK}{UTF8}{mj}若函数\end{CJK} $f(x, y)$ \begin{CJK}{UTF8}{mj}在\end{CJK} $D$ \begin{CJK}{UTF8}{mj}内满足\end{CJK} $f_{x}(x, y)=0$, \begin{CJK}{UTF8}{mj}则\end{CJK} $f(x, y)$ \begin{CJK}{UTF8}{mj}的取值与\end{CJK} $x$ \begin{CJK}{UTF8}{mj}无\end{CJK} \begin{CJK}{UTF8}{mj}关\end{CJK}.

\begin{enumerate}
  \setcounter{enumi}{2}
  \item \begin{CJK}{UTF8}{mj}计算题\end{CJK} $(5 \times 10=50$ \begin{CJK}{UTF8}{mj}分\end{CJK} $)$.
\end{enumerate}
(1). \begin{CJK}{UTF8}{mj}计算极限\end{CJK}
$$
\lim _{x \rightarrow 0} \frac{\cos x-\mathrm{e}^{-\frac{x^{2}}{2}}}{\ln \left(\sin ^{4} x+1\right)}
$$
(2). \begin{CJK}{UTF8}{mj}已知\end{CJK} $\lim _{x \rightarrow 0} \frac{f(x)}{x}=A$, \begin{CJK}{UTF8}{mj}函数\end{CJK} $f(x)$ \begin{CJK}{UTF8}{mj}在\end{CJK} $(-\infty,+\infty)$ \begin{CJK}{UTF8}{mj}上连续\end{CJK}, \begin{CJK}{UTF8}{mj}且\end{CJK}
$$
g(x)= \begin{cases}\frac{\int_{0}^{x} f(t) \mathrm{d} t}{x}, & x \neq 0 \\ 0, & x=0\end{cases}
$$
\begin{CJK}{UTF8}{mj}求\end{CJK} $g^{\prime}(x)$, \begin{CJK}{UTF8}{mj}并讨论\end{CJK} $g^{\prime}(x)$ \begin{CJK}{UTF8}{mj}在\end{CJK} $x=0$ \begin{CJK}{UTF8}{mj}的连续性\end{CJK}. (3). \begin{CJK}{UTF8}{mj}计算第二型曲线积分\end{CJK}
$$
I_{\mathrm{e}}=\oint_{C} \frac{x \mathrm{~d} y-y \mathrm{~d} x}{4 x^{2}+y^{2}}
$$
\begin{CJK}{UTF8}{mj}其中\end{CJK} $C$ \begin{CJK}{UTF8}{mj}是以\end{CJK} $(1,0)$ \begin{CJK}{UTF8}{mj}为圆心\end{CJK}, \begin{CJK}{UTF8}{mj}以\end{CJK} 2 \begin{CJK}{UTF8}{mj}为半径的圆\end{CJK}, \begin{CJK}{UTF8}{mj}方向取逆时针\end{CJK}.

(4). \begin{CJK}{UTF8}{mj}无\end{CJK}

(5). \begin{CJK}{UTF8}{mj}设\end{CJK} $I_{n}=\int_{0}^{\frac{\pi}{2}} \frac{\sin ^{2}(n t)}{\sin t} \mathrm{~d} t$, \begin{CJK}{UTF8}{mj}计算\end{CJK} $\lim _{n \rightarrow \infty} \frac{I_{n}}{\ln n}$

\begin{enumerate}
  \setcounter{enumi}{3}
  \item \begin{CJK}{UTF8}{mj}证明题\end{CJK} $(5 \times 14=60$ \begin{CJK}{UTF8}{mj}分\end{CJK} $)$.
\end{enumerate}
(1). \begin{CJK}{UTF8}{mj}设数列\end{CJK} $\left\{x_{n}\right\}$ \begin{CJK}{UTF8}{mj}满足\end{CJK} $\lim _{n \rightarrow \infty}\left(x_{n+1}-x_{n}\right)=0$, \begin{CJK}{UTF8}{mj}用\end{CJK} $\varepsilon-N$ \begin{CJK}{UTF8}{mj}语言证明\end{CJK}: $\lim _{n \rightarrow \infty} \frac{x_{n}}{n}=0$.

(2). \begin{CJK}{UTF8}{mj}设函数\end{CJK} $f(x)$ \begin{CJK}{UTF8}{mj}在\end{CJK} $[a, b]$ \begin{CJK}{UTF8}{mj}恒正\end{CJK}, \begin{CJK}{UTF8}{mj}且对\end{CJK} $\forall x_{0} \in[a, b]$, \begin{CJK}{UTF8}{mj}有\end{CJK} $\lim _{x \rightarrow x_{0}} f(x)=c, x_{0}>0$. \begin{CJK}{UTF8}{mj}证明\end{CJK}: \begin{CJK}{UTF8}{mj}存在\end{CJK} $c>0$, \begin{CJK}{UTF8}{mj}使得对\end{CJK} $\forall x \in[a, b]$, \begin{CJK}{UTF8}{mj}有\end{CJK} $f(x) \leq c$.

(3). \begin{CJK}{UTF8}{mj}设函数\end{CJK} $f(x)$ \begin{CJK}{UTF8}{mj}二阶连续可导\end{CJK}, \begin{CJK}{UTF8}{mj}证明\end{CJK}:\begin{CJK}{UTF8}{mj}存在\end{CJK} $\xi \in(a, b)$, \begin{CJK}{UTF8}{mj}使得\end{CJK}
$$
\int_{a}^{b} f(x) \mathrm{d} x=f\left(\frac{a+b}{2}\right)(b-a)+\frac{f^{\prime \prime}(\xi)}{24}(b-a)^{3}
$$
(4). \begin{CJK}{UTF8}{mj}设数列\end{CJK} $\left\{x_{n}\right\}$ \begin{CJK}{UTF8}{mj}有极限\end{CJK} $L$, \begin{CJK}{UTF8}{mj}证明\end{CJK}: $f(x)=\sum_{n=1}^{\infty} a_{n} x^{n}$ \begin{CJK}{UTF8}{mj}在\end{CJK} $(-1,1)$ \begin{CJK}{UTF8}{mj}有定义\end{CJK}, \begin{CJK}{UTF8}{mj}且\end{CJK}
$$
\lim _{x \rightarrow 1^{-}}(1-x) f(x)=L
$$
$(5)$. \begin{CJK}{UTF8}{mj}设函数\end{CJK} $f(x, y, z)$ \begin{CJK}{UTF8}{mj}在\end{CJK} $\Omega$ \begin{CJK}{UTF8}{mj}内连续偏导数\end{CJK}, \begin{CJK}{UTF8}{mj}证明以下两个条件等价\end{CJK}.

I. \begin{CJK}{UTF8}{mj}对于\end{CJK} $\Omega$ \begin{CJK}{UTF8}{mj}内任意闭合双侧曲面\end{CJK} $\Sigma$ \begin{CJK}{UTF8}{mj}有\end{CJK}
$$
\iint_{\Sigma} f(x, y, z)(x \mathrm{~d} y \mathrm{~d} z+y \mathrm{~d} z \mathrm{~d} x+z \mathrm{~d} x \mathrm{~d} y)
$$
II. \begin{CJK}{UTF8}{mj}对\end{CJK} $\forall \lambda>0, \forall(x, y, z) \in \Omega$, \begin{CJK}{UTF8}{mj}有\end{CJK} $f(\lambda x, \lambda y, \lambda z)=\lambda^{-3} f(x, y, z)$ \begin{CJK}{UTF8}{mj}成立\end{CJK}.

$9.22020$ \begin{CJK}{UTF8}{mj}年高等代数考研真题\end{CJK}

\begin{CJK}{UTF8}{mj}考牡须知\end{CJK}:

\begin{enumerate}
  \item \begin{CJK}{UTF8}{mj}本试卷满分为\end{CJK} 150 \begin{CJK}{UTF8}{mj}分\end{CJK}, \begin{CJK}{UTF8}{mj}全部考试时间总计\end{CJK} 180 \begin{CJK}{UTF8}{mj}分钟\end{CJK};

  \item \begin{CJK}{UTF8}{mj}所有答案必须写在答题纸上\end{CJK}, \begin{CJK}{UTF8}{mj}写在试题纸上或草稿纸上一律无效\end{CJK}。

  \item \begin{CJK}{UTF8}{mj}已知复数域上不可约多项式均为一次多项式\end{CJK}. (1). \begin{CJK}{UTF8}{mj}证明\end{CJK}:\begin{CJK}{UTF8}{mj}实数域上正次数多项式均为一次多项式或二次多项式的乘积\end{CJK};

\end{enumerate}
(2). \begin{CJK}{UTF8}{mj}㝍出实数域上所有的不可约多项式\end{CJK}.

\begin{enumerate}
  \setcounter{enumi}{2}
  \item \begin{CJK}{UTF8}{mj}设\end{CJK} $A$ \begin{CJK}{UTF8}{mj}是数域\end{CJK} $F$ \begin{CJK}{UTF8}{mj}上的\end{CJK} $n$ \begin{CJK}{UTF8}{mj}阶矩阵\end{CJK}, \begin{CJK}{UTF8}{mj}证明\end{CJK}:
\end{enumerate}
(1). \begin{CJK}{UTF8}{mj}如果\end{CJK} $\mathrm{A}$ \begin{CJK}{UTF8}{mj}与所有对角矩阵可交换\end{CJK}, \begin{CJK}{UTF8}{mj}则\end{CJK} $\mathrm{A}$ \begin{CJK}{UTF8}{mj}也是对角矩阵\end{CJK};

(2). \begin{CJK}{UTF8}{mj}如果\end{CJK} $\mathrm{A}$ \begin{CJK}{UTF8}{mj}与所有矩阵可交换\end{CJK}, \begin{CJK}{UTF8}{mj}则\end{CJK} $\mathrm{A}$ \begin{CJK}{UTF8}{mj}是数量矩阵\end{CJK}.

\begin{enumerate}
  \setcounter{enumi}{3}
  \item \begin{CJK}{UTF8}{mj}设\end{CJK}
\end{enumerate}
$$
A=\left(\begin{array}{ccccc}
3 & -2 & 3 & 2 & 5 \\
5 & -3 & 2 & 3 & 4 \\
1 & -3 & -5 & 0 & -7 \\
2 & -2 & -1 & 1 & -1
\end{array}\right), B=\left(\begin{array}{ccccc}
1 & 0 & a & b & c \\
0 & 1 & d & e & f \\
0 & 0 & 0 & 0 & 0 \\
0 & 0 & 0 & 0 & 0
\end{array}\right)
$$
\begin{CJK}{UTF8}{mj}已知存在可逆矩阵\end{CJK} $P$, \begin{CJK}{UTF8}{mj}使得\end{CJK} $P A=B$.

(1). \begin{CJK}{UTF8}{mj}求下列两个子空间的各一组基\end{CJK}:
$$
N(A)=\left\{x \in \mathbb{R}^{5} \mid A X=O\right\} R\left(A^{\prime}\right) \text { 即 } A \text { 的行空间 } A^{\prime} \text { 的列空间 }
$$
(2). \begin{CJK}{UTF8}{mj}求\end{CJK} $R(A)$ \begin{CJK}{UTF8}{mj}的一组基\end{CJK}, \begin{CJK}{UTF8}{mj}并求\end{CJK} $A$ \begin{CJK}{UTF8}{mj}的每个列向量在该组基下的坐标\end{CJK};

(3). \begin{CJK}{UTF8}{mj}以\end{CJK} $W^{\perp}$ \begin{CJK}{UTF8}{mj}表示\end{CJK} $W$ \begin{CJK}{UTF8}{mj}的正交补\end{CJK}, \begin{CJK}{UTF8}{mj}求\end{CJK} $N^{\perp}\left(A^{\prime}\right)$ \begin{CJK}{UTF8}{mj}的一组标准正交基\end{CJK}.

(4). \begin{CJK}{UTF8}{mj}求一个\end{CJK} $4 \times 2$ \begin{CJK}{UTF8}{mj}的矩阵\end{CJK} $L$ \begin{CJK}{UTF8}{mj}与一个\end{CJK} $2 \times 5$ \begin{CJK}{UTF8}{mj}的矩阵\end{CJK} $R$, \begin{CJK}{UTF8}{mj}使得\end{CJK} $A=L R$.

\begin{enumerate}
  \setcounter{enumi}{4}
  \item \begin{CJK}{UTF8}{mj}设\end{CJK} $V=\mathbb{R}[x]_{n}$ \begin{CJK}{UTF8}{mj}是次数小于\end{CJK} $n$ \begin{CJK}{UTF8}{mj}的全体实系数多项式构成的实线性空间\end{CJK}, $f^{\prime}(x)$ \begin{CJK}{UTF8}{mj}表示多项式\end{CJK} $f(x)$ \begin{CJK}{UTF8}{mj}的导数\end{CJK}, \begin{CJK}{UTF8}{mj}定义\end{CJK} $V$ \begin{CJK}{UTF8}{mj}上的线性变换\end{CJK}. \begin{CJK}{UTF8}{mj}如下\end{CJK}:
\end{enumerate}
$$
\sigma: f(x) \mapsto x f^{\prime}(x)-f(x), \forall f(x) \in V
$$
(1). \begin{CJK}{UTF8}{mj}求\end{CJK} $\sigma$ \begin{CJK}{UTF8}{mj}的特征值与特征向量\end{CJK};

(2). \begin{CJK}{UTF8}{mj}记\end{CJK} $\sigma$ \begin{CJK}{UTF8}{mj}的核空间与像空间分别为\end{CJK} $\operatorname{Ker}(\sigma)$ \begin{CJK}{UTF8}{mj}与\end{CJK} $\operatorname{Im}(\sigma)$. \begin{CJK}{UTF8}{mj}判断\end{CJK} $V=\operatorname{Ker}(\sigma) \oplus \operatorname{Im}(\sigma)$ \begin{CJK}{UTF8}{mj}是否成立\end{CJK}? \begin{CJK}{UTF8}{mj}并说明理\end{CJK}.

\begin{enumerate}
  \setcounter{enumi}{5}
  \item \begin{CJK}{UTF8}{mj}设\end{CJK} $A=\left(\begin{array}{lll}1 & 0 & 0 \\ 1 & 0 & 1 \\ 0 & 1 & 0\end{array}\right)$
\end{enumerate}
(1). \begin{CJK}{UTF8}{mj}求\end{CJK} $A$ \begin{CJK}{UTF8}{mj}的若尔当标准型\end{CJK} $J$;

(2). \begin{CJK}{UTF8}{mj}设\end{CJK} $k$ \begin{CJK}{UTF8}{mj}是正整数\end{CJK}, \begin{CJK}{UTF8}{mj}求\end{CJK} $A^{k}$ \begin{CJK}{UTF8}{mj}的若尔当标准型\end{CJK};

(3). \begin{CJK}{UTF8}{mj}求\end{CJK} $B=\left(\begin{array}{ll}A & A \\ A & A\end{array}\right)$ \begin{CJK}{UTF8}{mj}的若尔当标准型\end{CJK}. 6. \begin{CJK}{UTF8}{mj}设\end{CJK} $A$ \begin{CJK}{UTF8}{mj}是可逆矩阵\end{CJK}.

(1). \begin{CJK}{UTF8}{mj}证明\end{CJK}: $A^{\prime} A$ \begin{CJK}{UTF8}{mj}是正定矩阵\end{CJK};

(2). \begin{CJK}{UTF8}{mj}证明\end{CJK}: \begin{CJK}{UTF8}{mj}是否成立\end{CJK}
$$
B=\left(\begin{array}{cc}
A A^{\prime}+A^{\prime} A & A A^{\prime} \\
A A^{\prime} & A A^{\prime}+A^{\prime} A
\end{array}\right)
$$
\begin{CJK}{UTF8}{mj}为正定矩阵\end{CJK}.

\begin{enumerate}
  \setcounter{enumi}{7}
  \item \begin{CJK}{UTF8}{mj}实方阵\end{CJK} $P$ \begin{CJK}{UTF8}{mj}称为正交投影矩阵\end{CJK}, \begin{CJK}{UTF8}{mj}如果\end{CJK} $p^{2}=P^{\prime}=P$, \begin{CJK}{UTF8}{mj}设\end{CJK} $\alpha_{1}, \alpha_{2}, \cdots, \alpha_{s}$ \begin{CJK}{UTF8}{mj}是实\end{CJK} $n$ \begin{CJK}{UTF8}{mj}维向量空间\end{CJK} $\mathbb{R}^{n}$ \begin{CJK}{UTF8}{mj}的一组线性无关的向量\end{CJK}.
\end{enumerate}
(1). \begin{CJK}{UTF8}{mj}证明\end{CJK}:\begin{CJK}{UTF8}{mj}存在唯一的正交投影矩阵\end{CJK} $P$, \begin{CJK}{UTF8}{mj}使得\end{CJK} $P$ \begin{CJK}{UTF8}{mj}的零空间\end{CJK} $N(P)$ (\begin{CJK}{UTF8}{mj}即方程组\end{CJK} $P X=O$ \begin{CJK}{UTF8}{mj}的解空\end{CJK} \begin{CJK}{UTF8}{mj}间\end{CJK}) \begin{CJK}{UTF8}{mj}由\end{CJK} $\alpha_{1}, \alpha_{2}, \cdots, \alpha_{s}$ \begin{CJK}{UTF8}{mj}生成\end{CJK}.

$(2)$. \begin{CJK}{UTF8}{mj}设\end{CJK} $s=2, \alpha_{1}=(1,0,0)^{\prime}, \alpha_{2}=(0,1,1)^{\prime}$, \begin{CJK}{UTF8}{mj}请写出第一问中的矩阵\end{CJK} $P$.

\begin{enumerate}
  \setcounter{enumi}{8}
  \item \begin{CJK}{UTF8}{mj}设\end{CJK} $A$ \begin{CJK}{UTF8}{mj}是\end{CJK} $m \times n$ \begin{CJK}{UTF8}{mj}阶实矩阵\end{CJK}, \begin{CJK}{UTF8}{mj}证明\end{CJK}: \begin{CJK}{UTF8}{mj}存在\end{CJK} $m \times n$ \begin{CJK}{UTF8}{mj}阶非负实对角矩阵\end{CJK} $D($ \begin{CJK}{UTF8}{mj}即所有行标与列标相同的\end{CJK} \begin{CJK}{UTF8}{mj}元素均非负\end{CJK}, \begin{CJK}{UTF8}{mj}其余元素均为\end{CJK} 0$)$ \begin{CJK}{UTF8}{mj}与两个正交矩阵\end{CJK} $U, V$, \begin{CJK}{UTF8}{mj}使得\end{CJK} $A=U D V($ \begin{CJK}{UTF8}{mj}即矩阵\end{CJK} $A$ \begin{CJK}{UTF8}{mj}的奇异值分\end{CJK} \begin{CJK}{UTF8}{mj}解\end{CJK}).

  \item \begin{CJK}{UTF8}{mj}记\end{CJK} $M_{n}(F)$ \begin{CJK}{UTF8}{mj}是数域\end{CJK} $F$ \begin{CJK}{UTF8}{mj}上全体\end{CJK} $n$ \begin{CJK}{UTF8}{mj}阶矩阵构成的向量空间\end{CJK}, \begin{CJK}{UTF8}{mj}符号\end{CJK} $\operatorname{tr}(X)$ \begin{CJK}{UTF8}{mj}表示\end{CJK} $X$ \begin{CJK}{UTF8}{mj}的迹\end{CJK}. \begin{CJK}{UTF8}{mj}证明\end{CJK}: \begin{CJK}{UTF8}{mj}映射\end{CJK} $\operatorname{tr}: X \mapsto \operatorname{tr}(X)$ \begin{CJK}{UTF8}{mj}是\end{CJK} $M_{n}(F)$ \begin{CJK}{UTF8}{mj}到\end{CJK} $F$ \begin{CJK}{UTF8}{mj}的满足性质\end{CJK} $\sigma(X Y)=\sigma(Y X)$ \begin{CJK}{UTF8}{mj}以及\end{CJK} $\sigma(I)=n$ \begin{CJK}{UTF8}{mj}的唯一线性变换\end{CJK} $\sigma$.

  \item \begin{CJK}{UTF8}{mj}设\end{CJK} $A, B$ \begin{CJK}{UTF8}{mj}是向量空间\end{CJK} $V$ \begin{CJK}{UTF8}{mj}的两个子空间\end{CJK}.

\end{enumerate}
(1). \begin{CJK}{UTF8}{mj}证明\end{CJK}:\begin{CJK}{UTF8}{mj}商空间\end{CJK} $A /(A \cap B)$ \begin{CJK}{UTF8}{mj}与商空间\end{CJK} $(A+B) / B$ \begin{CJK}{UTF8}{mj}同构\end{CJK}.

(2). \begin{CJK}{UTF8}{mj}证明\end{CJK}: $(A+B) /(A \cap B)=A /(A \cap B) \oplus B /(A \cap B)$.

\section{第 10 章 中国科学技术大㥯}
\section{$10.12020$ 年数学分析考研真题}
\section{考生须知:}
\begin{enumerate}
  \item \begin{CJK}{UTF8}{mj}本试卷满分为\end{CJK} 150 \begin{CJK}{UTF8}{mj}分\end{CJK}, \begin{CJK}{UTF8}{mj}全部考试时间总计\end{CJK} 180 \begin{CJK}{UTF8}{mj}分钟\end{CJK};

  \item \begin{CJK}{UTF8}{mj}所有答案必须写在答题纸上\end{CJK}, \begin{CJK}{UTF8}{mj}写在试题纸上或草稿纸上一律无效\end{CJK}。

  \item \begin{CJK}{UTF8}{mj}计算题\end{CJK} (\begin{CJK}{UTF8}{mj}每小题\end{CJK} 10 \begin{CJK}{UTF8}{mj}分\end{CJK})

\end{enumerate}
(1). \begin{CJK}{UTF8}{mj}求极限\end{CJK} $\lim _{x \rightarrow 0} \frac{\left(1+x^{2}\right)^{\frac{3}{2}}-\cos x}{\sin \left(x^{2}\right)}$

(2). \begin{CJK}{UTF8}{mj}设\end{CJK} $f(x)$ \begin{CJK}{UTF8}{mj}连续\end{CJK}, \begin{CJK}{UTF8}{mj}已知\end{CJK} $f(0) \neq 0$, \begin{CJK}{UTF8}{mj}计算\end{CJK}

\includegraphics[max width=\textwidth]{2022_04_18_7db0708508f26638f054g-045}

(3). \begin{CJK}{UTF8}{mj}已知\end{CJK} $z=f\left(x^{2}-2 y, g(x y)\right)$, \begin{CJK}{UTF8}{mj}且\end{CJK} $f$ \begin{CJK}{UTF8}{mj}二阶连续可微\end{CJK}, $g$ \begin{CJK}{UTF8}{mj}二阶可微\end{CJK}, \begin{CJK}{UTF8}{mj}求\end{CJK} $\frac{\partial^{2} z}{\partial x \partial y}$.

(4). \begin{CJK}{UTF8}{mj}求积分\end{CJK}
$$
\iiint_{V}\left(\sqrt{x^{2}+y^{2}+z^{2}}+x^{5}+y^{2} \sin y\right) d V
$$
\begin{CJK}{UTF8}{mj}其中\end{CJK} $V$ \begin{CJK}{UTF8}{mj}为\end{CJK} $x^{2}+y^{2}+z^{2}=2 z$ \begin{CJK}{UTF8}{mj}所围成的区域\end{CJK}.

\begin{enumerate}
  \setcounter{enumi}{2}
  \item (15 \begin{CJK}{UTF8}{mj}分\end{CJK}) \begin{CJK}{UTF8}{mj}计算曲面积分\end{CJK}
\end{enumerate}
$$
\iint_{S} \frac{x \mathrm{~d} y \mathrm{~d} z+y \mathrm{~d} z \mathrm{~d} x+z \mathrm{~d} x \mathrm{~d} y}{\left(4 x^{2}+2 y^{2}+z^{2}\right)^{\frac{3}{2}}}
$$
\begin{CJK}{UTF8}{mj}其中\end{CJK} $S$ \begin{CJK}{UTF8}{mj}是曲面\end{CJK} $x^{2}+y^{2}+z^{2}=1$, \begin{CJK}{UTF8}{mj}取外侧\end{CJK}.

\begin{enumerate}
  \setcounter{enumi}{3}
  \item (15 \begin{CJK}{UTF8}{mj}分\end{CJK}) \begin{CJK}{UTF8}{mj}已知\end{CJK} $f$ \begin{CJK}{UTF8}{mj}是\end{CJK} $(0,+\infty)$ \begin{CJK}{UTF8}{mj}的连续函数\end{CJK}, \begin{CJK}{UTF8}{mj}满足\end{CJK} $x=f(x) \mathrm{e}^{f(x)}$, \begin{CJK}{UTF8}{mj}求证\end{CJK}:
\end{enumerate}
(1). $f(x)$ \begin{CJK}{UTF8}{mj}单调递增\end{CJK};

(2). $\lim _{x \rightarrow+\infty} f(x)=+\infty$;

(3). $\lim _{x \rightarrow+\infty} \frac{f(x)}{\ln x}=1$. 4. (15 \begin{CJK}{UTF8}{mj}分\end{CJK}) \begin{CJK}{UTF8}{mj}证明\end{CJK}:
$$
\sum_{n=0}^{\infty} \frac{\cos [(2 n+1) \pi x]}{(2 n+1)^{2}}=\frac{\pi^{2}}{8}-\frac{\pi^{2}}{4}|x|, \quad|x| \leq 1
$$
\begin{CJK}{UTF8}{mj}并且计算\end{CJK} $\sum_{n=1}^{\infty} \frac{1}{n^{2}}$ \begin{CJK}{UTF8}{mj}和\end{CJK} $\sum_{n=1}^{\infty} \frac{1}{n^{4}}$.

\begin{enumerate}
  \setcounter{enumi}{5}
  \item (15 \begin{CJK}{UTF8}{mj}分\end{CJK}) \begin{CJK}{UTF8}{mj}讨论积分\end{CJK} $\int_{0}^{+\infty} \frac{\sin x}{x^{u}}$ \begin{CJK}{UTF8}{mj}的绝对收敛性与条件收敛性\end{CJK}.

  \item (15 \begin{CJK}{UTF8}{mj}分\end{CJK}) \begin{CJK}{UTF8}{mj}设\end{CJK} $f(x)$ \begin{CJK}{UTF8}{mj}在\end{CJK} $[0,1]$ \begin{CJK}{UTF8}{mj}连续\end{CJK}, \begin{CJK}{UTF8}{mj}并且满足对\end{CJK} $\forall x \in[0,1], \int_{x}^{1} f(t) \mathrm{d} t \geq \frac{1-x^{2}}{2}$, \begin{CJK}{UTF8}{mj}证明\end{CJK}

\end{enumerate}
$$
\int_{0}^{1} f^{2}(t) \mathrm{d} t \geq \frac{1}{3}
$$

\begin{enumerate}
  \setcounter{enumi}{7}
  \item (15 \begin{CJK}{UTF8}{mj}分\end{CJK}) \begin{CJK}{UTF8}{mj}设\end{CJK} $f \in C^{\infty}[-1,1]$
\end{enumerate}
(1). \begin{CJK}{UTF8}{mj}证明\end{CJK}:
$$
f(x)=\sum_{k=0}^{n} \frac{f^{(k)}(0)}{k !} x^{k}+\frac{1}{n !} \int_{0}^{x}(x-t)^{n} f^{(n+1)}(t) \mathrm{d} t
$$
\begin{CJK}{UTF8}{mj}其中\end{CJK} $n$ \begin{CJK}{UTF8}{mj}是正整数\end{CJK}.

(2). \begin{CJK}{UTF8}{mj}若\end{CJK} $f^{(n)}(x) \geq 0$ \begin{CJK}{UTF8}{mj}对所有\end{CJK} $n \in \mathbb{N}, x \in[-1,1]$ \begin{CJK}{UTF8}{mj}都成立\end{CJK}, \begin{CJK}{UTF8}{mj}证明\end{CJK}: $f$ \begin{CJK}{UTF8}{mj}在\end{CJK} $[-1,1]$ \begin{CJK}{UTF8}{mj}上可以展开成幂级\end{CJK} \begin{CJK}{UTF8}{mj}数\end{CJK}.

\begin{enumerate}
  \setcounter{enumi}{8}
  \item (15 \begin{CJK}{UTF8}{mj}分\end{CJK}) \begin{CJK}{UTF8}{mj}设\end{CJK} $f_{n}(x)$ \begin{CJK}{UTF8}{mj}在\end{CJK} $[a, b]$ \begin{CJK}{UTF8}{mj}上可微\end{CJK}, \begin{CJK}{UTF8}{mj}并且存在\end{CJK} $M>0$ \begin{CJK}{UTF8}{mj}使得对所有的\end{CJK} $x \in[a, b]$ \begin{CJK}{UTF8}{mj}及所有的\end{CJK} $n$ \begin{CJK}{UTF8}{mj}都\end{CJK} \begin{CJK}{UTF8}{mj}有\end{CJK} $\left|f_{n}^{\prime}(x)\right| \leq M$. \begin{CJK}{UTF8}{mj}若\end{CJK} $f_{n}(x)$ \begin{CJK}{UTF8}{mj}在\end{CJK} $[a, b]$ \begin{CJK}{UTF8}{mj}上收敛到\end{CJK} $f(x)$, \begin{CJK}{UTF8}{mj}求证\end{CJK}: $f_{n}(x)$ \begin{CJK}{UTF8}{mj}在\end{CJK} $[-1,1]$ \begin{CJK}{UTF8}{mj}上是一致收敛到\end{CJK} $f(x)$ \begin{CJK}{UTF8}{mj}的\end{CJK}.

  \item (10 \begin{CJK}{UTF8}{mj}分\end{CJK}) \begin{CJK}{UTF8}{mj}设\end{CJK} $g(x)$ \begin{CJK}{UTF8}{mj}在\end{CJK} $[0,1]$ \begin{CJK}{UTF8}{mj}上单调递增\end{CJK}, \begin{CJK}{UTF8}{mj}证明\end{CJK}

\end{enumerate}
$$
\lim _{y \rightarrow+\infty} \int_{0}^{1} g(x) \frac{\sin (x y)}{x} \mathrm{~d} x=\frac{\pi}{2} g\left(0^{+}\right)
$$

\section{$10.22020$ 年高等代数考研真题}
\section{考生须知:}
\begin{enumerate}
  \item \begin{CJK}{UTF8}{mj}本试卷满分为\end{CJK} 150 \begin{CJK}{UTF8}{mj}分\end{CJK}, \begin{CJK}{UTF8}{mj}全部考试时间总计\end{CJK} 180 \begin{CJK}{UTF8}{mj}分钟\end{CJK}; (1). \begin{CJK}{UTF8}{mj}直线\end{CJK} $l$ \begin{CJK}{UTF8}{mj}与平面\end{CJK} $\pi[?]$, \begin{CJK}{UTF8}{mj}则\end{CJK} $l$ \begin{CJK}{UTF8}{mj}与\end{CJK} $\pi$ \begin{CJK}{UTF8}{mj}的夹角是\end{CJK} ,\begin{CJK}{UTF8}{mj}直线到平面的投影方程方程是\end{CJK} \begin{CJK}{UTF8}{mj}直线\end{CJK} $l$ \begin{CJK}{UTF8}{mj}绕\end{CJK} $y$ \begin{CJK}{UTF8}{mj}轴旋转而得的曲面方程是\end{CJK}
\end{enumerate}
$(2)$. \begin{CJK}{UTF8}{mj}矩阵\end{CJK} $\left(\begin{array}{ll}1 & -3 \\ 1 & -4\end{array}\right)^{2020}=$ ;\begin{CJK}{UTF8}{mj}行列式\end{CJK} $\operatorname{det}\left(I_{4}-\alpha \alpha^{T}\right)=$ ,\begin{CJK}{UTF8}{mj}其中\end{CJK} $\alpha=(1,1,0,-2)^{T}$. (3). \begin{CJK}{UTF8}{mj}若矩阵\end{CJK}
$$
\left(\begin{array}{cccc}
2 & -1 & & \\
-1 & a & 1 & \\
& 0 & 1 & 2
\end{array}\right)
$$
\begin{CJK}{UTF8}{mj}的正惯性指数为\end{CJK} 2 , \begin{CJK}{UTF8}{mj}则\end{CJK} $a$ \begin{CJK}{UTF8}{mj}的范围是\end{CJK}

(4). \begin{CJK}{UTF8}{mj}考虑标准欧式空间\end{CJK} $\mathbb{R}^{4}$, \begin{CJK}{UTF8}{mj}设\end{CJK} $U$ \begin{CJK}{UTF8}{mj}是\end{CJK} $\cdots$ \begin{CJK}{UTF8}{mj}张成的子空间\end{CJK}, $V$ \begin{CJK}{UTF8}{mj}是\end{CJK} $\cdots$ \begin{CJK}{UTF8}{mj}张成的子空间\end{CJK}, \begin{CJK}{UTF8}{mj}则\end{CJK} $U \cap V$ \begin{CJK}{UTF8}{mj}的维数是\end{CJK} ,$U^{\perp} \cap V$ \begin{CJK}{UTF8}{mj}的维数是\end{CJK}

(5). \begin{CJK}{UTF8}{mj}设方阵\end{CJK}
$$
A=\left(\begin{array}{lll}
1 & 0 & 1 \\
1 & 2 & a \\
1 & 0 & 1
\end{array}\right)
$$
\begin{CJK}{UTF8}{mj}可以对角化\end{CJK}, \begin{CJK}{UTF8}{mj}则\end{CJK} $a=$ ,$A$ \begin{CJK}{UTF8}{mj}的最小多项式是\end{CJK}

\begin{enumerate}
  \setcounter{enumi}{2}
  \item \begin{CJK}{UTF8}{mj}给定二次曲面\end{CJK} $\cdots$, \begin{CJK}{UTF8}{mj}试用正交变换和平移将它化为标准方程\end{CJK}, \begin{CJK}{UTF8}{mj}并判断其曲面类型\end{CJK}.

  \item \begin{CJK}{UTF8}{mj}设\end{CJK} $A$ \begin{CJK}{UTF8}{mj}是不可对角化的\end{CJK} $n$ \begin{CJK}{UTF8}{mj}阶复方阵\end{CJK}, \begin{CJK}{UTF8}{mj}试证\end{CJK}: \begin{CJK}{UTF8}{mj}存在非零方阵\end{CJK} $B$ \begin{CJK}{UTF8}{mj}使得\end{CJK} $A B=B A$ \begin{CJK}{UTF8}{mj}且\end{CJK} $A^{n}=O$.

  \item \begin{CJK}{UTF8}{mj}在\end{CJK} $M_{2}(\mathbb{C})$ \begin{CJK}{UTF8}{mj}上定义线性变换\end{CJK}

\end{enumerate}
$$
X \longrightarrow A X, A=\left(\begin{array}{ll}
1 & 2 \\
0 & 1
\end{array}\right)
$$
\begin{CJK}{UTF8}{mj}求\end{CJK} $\mathscr{A}$ \begin{CJK}{UTF8}{mj}的特征值\end{CJK}, \begin{CJK}{UTF8}{mj}特征向量以及若当标准型\end{CJK}.

\begin{enumerate}
  \setcounter{enumi}{5}
  \item \begin{CJK}{UTF8}{mj}给定行向量组集合\end{CJK} $S=?$, \begin{CJK}{UTF8}{mj}试求它的极大无关组并证明其结论\end{CJK}.

  \item \begin{CJK}{UTF8}{mj}设\end{CJK} $A$ \begin{CJK}{UTF8}{mj}为\end{CJK} $n$ \begin{CJK}{UTF8}{mj}阶实对称矩阵\end{CJK}, $b$ \begin{CJK}{UTF8}{mj}是\end{CJK} $n$ \begin{CJK}{UTF8}{mj}维列向量\end{CJK}, \begin{CJK}{UTF8}{mj}证明\end{CJK}: $A-b b^{T}$ \begin{CJK}{UTF8}{mj}正定当且仅当\end{CJK} $A$ \begin{CJK}{UTF8}{mj}正定且\end{CJK} $b^{T} A^{-1} b<1$.

\end{enumerate}
\section{第11章 武汉大学}
\section{$11.12020$ 年数学分析考研真题}
\section{考生须知:}
\begin{enumerate}
  \item \begin{CJK}{UTF8}{mj}本试卷满分为\end{CJK} 150 \begin{CJK}{UTF8}{mj}分\end{CJK}, \begin{CJK}{UTF8}{mj}全部考试时间总计\end{CJK} 180 \begin{CJK}{UTF8}{mj}分钟\end{CJK};

  \item \begin{CJK}{UTF8}{mj}所有答案必须写在答题纸上\end{CJK}, \begin{CJK}{UTF8}{mj}写在试题纸上或草稿纸上一律无效\end{CJK}。

  \item (15 \begin{CJK}{UTF8}{mj}分\end{CJK}) \begin{CJK}{UTF8}{mj}若\end{CJK} $\lim _{n \rightarrow \infty} a_{n}=a(a>0), \lim _{n \rightarrow \infty} b_{n}=0$, \begin{CJK}{UTF8}{mj}计算\end{CJK} $\lim _{n \rightarrow \infty} \frac{a_{n}^{b_{n}}-1}{b_{n}}$

  \item (15 \begin{CJK}{UTF8}{mj}分\end{CJK}) \begin{CJK}{UTF8}{mj}设\end{CJK} $f(x, y, z)=x^{y} y^{z} z^{x}$, \begin{CJK}{UTF8}{mj}求\end{CJK} $f(x, y, z)$ \begin{CJK}{UTF8}{mj}的全微分以及二阶偏导数\end{CJK}.

  \item (15 \begin{CJK}{UTF8}{mj}分\end{CJK}) \begin{CJK}{UTF8}{mj}计算不定积分\end{CJK}

\end{enumerate}
$$
\int \frac{x^{2}+1}{\left(x^{2}-2 x+2\right)^{2}} \mathrm{~d} x
$$

\begin{enumerate}
  \setcounter{enumi}{4}
  \item (15 \begin{CJK}{UTF8}{mj}分\end{CJK}) \begin{CJK}{UTF8}{mj}计算\end{CJK}
\end{enumerate}
$$
\iiint_{\Omega}(x+y-z)(y+z-x)(z+x-y) \mathrm{d} x \mathrm{~d} y \mathrm{~d} z
$$
\begin{CJK}{UTF8}{mj}其中\end{CJK} $\Omega=\{(x, y, z) \mid 0 \leq x+y-z \leq 1,0 \leq y+z-x \leq 1,0 \leq z+x-y \leq 1\}$.

\begin{enumerate}
  \setcounter{enumi}{5}
  \item (15 \begin{CJK}{UTF8}{mj}分\end{CJK}) \begin{CJK}{UTF8}{mj}讨论级数\end{CJK} $\sum_{n=1}^{\infty} \frac{n^{n}}{a^{n} n !}$.

  \item (15 \begin{CJK}{UTF8}{mj}分\end{CJK}) \begin{CJK}{UTF8}{mj}若级数\end{CJK} $\sum_{n=1}^{\infty} a_{n} \cos ^{n} x$ \begin{CJK}{UTF8}{mj}在\end{CJK} $[0,2 \pi]$ \begin{CJK}{UTF8}{mj}上收敛\end{CJK}, \begin{CJK}{UTF8}{mj}试问是否一致收敛并说明理由\end{CJK}.

  \item (15 \begin{CJK}{UTF8}{mj}分\end{CJK}) \begin{CJK}{UTF8}{mj}设\end{CJK} $f(x)=\sin , x \in[0, \pi]$, \begin{CJK}{UTF8}{mj}试讲\end{CJK} $f(x)$ \begin{CJK}{UTF8}{mj}展开成余弦级数\end{CJK}, \begin{CJK}{UTF8}{mj}并讨论其收敛性\end{CJK}.

  \item (15 \begin{CJK}{UTF8}{mj}分\end{CJK}) \begin{CJK}{UTF8}{mj}证明\end{CJK}: \begin{CJK}{UTF8}{mj}当\end{CJK} $x \in(0,1)$ \begin{CJK}{UTF8}{mj}时\end{CJK}, \begin{CJK}{UTF8}{mj}存在\end{CJK} $\theta_{n}$, \begin{CJK}{UTF8}{mj}使得\end{CJK}

\end{enumerate}
$$
\frac{1}{1-x}=\frac{\left(1-\theta_{n}\right)^{n} \cdot(1+n)}{\left(1-\theta_{n} x\right)^{n+2}}
$$
\begin{CJK}{UTF8}{mj}并求\end{CJK} $\lim _{x \rightarrow 0} \theta_{n}$

\begin{enumerate}
  \setcounter{enumi}{9}
  \item $(15$ \begin{CJK}{UTF8}{mj}分\end{CJK} $)$ \begin{CJK}{UTF8}{mj}若\end{CJK} $f(x)$ \begin{CJK}{UTF8}{mj}在\end{CJK} $[a, b]$ \begin{CJK}{UTF8}{mj}上有连续的二阶导数\end{CJK}, $f(a) f(b)<0, f^{\prime}(x) f^{\prime \prime}(x) \neq 0(x \in[a, b])$
\end{enumerate}
(1). \begin{CJK}{UTF8}{mj}证明\end{CJK}: $f(x)$ \begin{CJK}{UTF8}{mj}在\end{CJK} $[a, b]$ \begin{CJK}{UTF8}{mj}上有唯一的零点\end{CJK}; (2). \begin{CJK}{UTF8}{mj}假设\end{CJK}
$$
x_{n}=x_{n-1}-\frac{f\left(x_{n-1}\right)}{f^{\prime}\left(x_{n-1}\right)}, x_{0}= \begin{cases}a, & f^{\prime}(x) f^{\prime \prime}(x)<0 \\ b, & f^{\prime}(x) f^{\prime \prime}(x)>0\end{cases}
$$
\begin{CJK}{UTF8}{mj}设\end{CJK} $f(x)$ \begin{CJK}{UTF8}{mj}在\end{CJK} $[a, b]$ \begin{CJK}{UTF8}{mj}上的零点为\end{CJK} $\xi$, \begin{CJK}{UTF8}{mj}证明\end{CJK}: $\lim _{n \rightarrow \infty} x_{n}=\xi$ \begin{CJK}{UTF8}{mj}且\end{CJK} $\left|x_{n}-\xi\right| \leq \frac{\left|f\left(x_{n}\right)\right|}{m}$, \begin{CJK}{UTF8}{mj}其中\end{CJK} $m=$ $\min _{x \in[a, b]}\left\{\left|f^{\prime}(x)\right|\right\}$.

\begin{enumerate}
  \setcounter{enumi}{10}
  \item (15 \begin{CJK}{UTF8}{mj}分\end{CJK}) \begin{CJK}{UTF8}{mj}设\end{CJK} $B^{2}=\left\{(x, y) \mid x^{2}+y^{2} \leq 1\right\}, \partial B^{2}=\left\{(x, y) \mid x^{2}+y^{2}=1\right\}$, \begin{CJK}{UTF8}{mj}证明\end{CJK}: \begin{CJK}{UTF8}{mj}不存在连续可微的\end{CJK} \begin{CJK}{UTF8}{mj}映射\end{CJK} $g: B^{2} \rightarrow \mathbb{R}^{2}$ \begin{CJK}{UTF8}{mj}满足\end{CJK}: $g\left(B^{2}\right) \subseteq \partial B^{2}$ \begin{CJK}{UTF8}{mj}且\end{CJK} $g(x, y)=g\left(x y, x^{2}-y^{2}\right)$.
\end{enumerate}
\section{$11.22020$ 年高等代数考研真题}
\section{考生须知:}
\begin{enumerate}
  \item \begin{CJK}{UTF8}{mj}本试卷满分为\end{CJK} 150 \begin{CJK}{UTF8}{mj}分\end{CJK}, \begin{CJK}{UTF8}{mj}全部考试时间总计\end{CJK} 180 \begin{CJK}{UTF8}{mj}分钟\end{CJK};

  \item \begin{CJK}{UTF8}{mj}所有答案必须与在答题纸上\end{CJK}, \begin{CJK}{UTF8}{mj}写在试题纸上或草稿纸上一律无效\end{CJK}。

  \item \begin{CJK}{UTF8}{mj}计算\end{CJK} $n$ \begin{CJK}{UTF8}{mj}阶行列式\end{CJK}

\end{enumerate}
$\left|\begin{array}{cccccc}0 & 1 & 2 & 3 & \cdots & n-1 \\ 1 & 0 & 1 & 2 & \cdots & n-2 \\ 2 & 1 & 0 & 1 & \cdots & n-3 \\ 3 & 2 & 1 & 0 & \cdots & n-4 \\ \vdots & \vdots & \vdots & \vdots & & \vdots \\ n-1 & n-2 & n-3 & n-4 & \cdots & 0\end{array}\right|$

\begin{enumerate}
  \setcounter{enumi}{2}
  \item \begin{CJK}{UTF8}{mj}设\end{CJK} $A_{n}=\left(a_{i j}\right)_{n \times n}, i, j \in \mathbb{Z}^{+}, a_{i j} \in \mathbb{Z}, k \in \mathbb{Z}$ \begin{CJK}{UTF8}{mj}且\end{CJK} $k \geq 2$, \begin{CJK}{UTF8}{mj}证明\end{CJK}:
\end{enumerate}
$$
D_{n}=\left|\begin{array}{ccccc}
a_{11}-\frac{1}{k} & a_{12} & a_{13} & \cdots & a_{1 n} \\
a_{21} & a_{22}-\frac{1}{k} & a_{23} & \cdots & a_{2 n} \\
\vdots & \vdots & \vdots & & \vdots \\
a_{n 1} & a_{n 2} & a_{n 3} & \cdots & a_{n n}-\frac{1}{k}
\end{array}\right| \neq 0
$$

\begin{enumerate}
  \setcounter{enumi}{3}
  \item \begin{CJK}{UTF8}{mj}已知\end{CJK} $A, C$ \begin{CJK}{UTF8}{mj}为正定矩阵\end{CJK}, \begin{CJK}{UTF8}{mj}若矩阵方程\end{CJK} $A X+X A=C$ \begin{CJK}{UTF8}{mj}有唯一解\end{CJK} $B$, \begin{CJK}{UTF8}{mj}证明\end{CJK}: $B$ \begin{CJK}{UTF8}{mj}是正定矩阵\end{CJK}.

  \item \begin{CJK}{UTF8}{mj}已知\end{CJK} $\mathscr{M}$ \begin{CJK}{UTF8}{mj}是\end{CJK} $M_{2 \times 2}(\mathbb{R})$ \begin{CJK}{UTF8}{mj}上的线性变换\end{CJK}, $\forall \alpha \in M_{2 \times 2}(\mathbb{R}), \mathscr{M}(\alpha)=\left(\begin{array}{cc}1 \\ -1 & 4\end{array}\right) \alpha$, \begin{CJK}{UTF8}{mj}试求\end{CJK} $\mathscr{M}$ \begin{CJK}{UTF8}{mj}的特征\end{CJK} \begin{CJK}{UTF8}{mj}值和特征子空间\end{CJK}. 5. \begin{CJK}{UTF8}{mj}已知\end{CJK} $A$ \begin{CJK}{UTF8}{mj}的特征值都是实数\end{CJK}, \begin{CJK}{UTF8}{mj}且\end{CJK} $A$ \begin{CJK}{UTF8}{mj}的一阶主子式之和和二阶主子式之和都为零\end{CJK}, \begin{CJK}{UTF8}{mj}证明\end{CJK}: $A^{n}=O$.

  \item \begin{CJK}{UTF8}{mj}已知多项式\end{CJK} $f_{1}(x), f_{2}(x)$ \begin{CJK}{UTF8}{mj}互素\end{CJK} (\begin{CJK}{UTF8}{mj}即\end{CJK} $\left.\left(f_{1}(x), f_{2}(x)\right)=1\right)$ ), \begin{CJK}{UTF8}{mj}证明\end{CJK}: \begin{CJK}{UTF8}{mj}对\end{CJK} $\forall g(x) \in K[x]$, \begin{CJK}{UTF8}{mj}有\end{CJK}

\end{enumerate}
$$
\left(f_{1}(x) f_{2}(x), g(x)\right)=\left(f_{1}(x), g(x)\right) \cdot\left(f_{2}(x), g(x)\right)
$$

\begin{enumerate}
  \setcounter{enumi}{7}
  \item \begin{CJK}{UTF8}{mj}已知\end{CJK} $A$ \begin{CJK}{UTF8}{mj}是可逆矩阵\end{CJK}, \begin{CJK}{UTF8}{mj}证明\end{CJK}: $A=Q T$, \begin{CJK}{UTF8}{mj}其中\end{CJK} $Q$ \begin{CJK}{UTF8}{mj}为正交矩阵\end{CJK}, $T$ \begin{CJK}{UTF8}{mj}为非奇异上三角矩阵\end{CJK}.

  \item \begin{CJK}{UTF8}{mj}已知\end{CJK} $\tau$ \begin{CJK}{UTF8}{mj}是\end{CJK} $n$ \begin{CJK}{UTF8}{mj}维欧氏空间上的线性变换\end{CJK}, \begin{CJK}{UTF8}{mj}且\end{CJK} $\tau^{3}+\tau=0$, \begin{CJK}{UTF8}{mj}证明\end{CJK}: $\tau$ \begin{CJK}{UTF8}{mj}在基下对应的矩阵的迹为\end{CJK} 0 .

  \item \begin{CJK}{UTF8}{mj}已知\end{CJK} $V$ \begin{CJK}{UTF8}{mj}为\end{CJK} $n$ \begin{CJK}{UTF8}{mj}维向量空间\end{CJK}, $\alpha_{1}, \alpha_{2}, \cdots, \alpha_{n}$ \begin{CJK}{UTF8}{mj}为一组基\end{CJK}, \begin{CJK}{UTF8}{mj}证明\end{CJK}: \begin{CJK}{UTF8}{mj}对\end{CJK} $\forall m>n$, \begin{CJK}{UTF8}{mj}必存在\end{CJK} $m$ \begin{CJK}{UTF8}{mj}个向量组\end{CJK} \begin{CJK}{UTF8}{mj}成的向量组\end{CJK}, \begin{CJK}{UTF8}{mj}其中任意\end{CJK} $n$ \begin{CJK}{UTF8}{mj}个线性无关\end{CJK}.

  \item \begin{CJK}{UTF8}{mj}已知\end{CJK} $A$ \begin{CJK}{UTF8}{mj}的特征值全为\end{CJK} 1 , \begin{CJK}{UTF8}{mj}证明\end{CJK}: \begin{CJK}{UTF8}{mj}对任意自然数\end{CJK} $s$, \begin{CJK}{UTF8}{mj}证明\end{CJK}: $A$ \begin{CJK}{UTF8}{mj}与\end{CJK} $A^{s}$ \begin{CJK}{UTF8}{mj}相似\end{CJK}.

\end{enumerate}
\section{第12章 南开大学}
\section{$12.12020$ 年数学分析考研真题}
\section{考生须知:}
\begin{enumerate}
  \item \begin{CJK}{UTF8}{mj}本试卷满分为\end{CJK} 150 \begin{CJK}{UTF8}{mj}分\end{CJK}, \begin{CJK}{UTF8}{mj}全部考试时间总计\end{CJK} 180 \begin{CJK}{UTF8}{mj}分钟\end{CJK};

  \item \begin{CJK}{UTF8}{mj}所有答案必须写在答题纸上\end{CJK}, \begin{CJK}{UTF8}{mj}写在试题纸上或草稿纸上一律无效\end{CJK}。

  \item (20 \begin{CJK}{UTF8}{mj}分\end{CJK}) \begin{CJK}{UTF8}{mj}求极限\end{CJK}

\end{enumerate}
$$
\lim _{x \rightarrow 0} \frac{e^{\arctan x}-e^{x}}{\tan ^{3} x}
$$

\begin{enumerate}
  \setcounter{enumi}{2}
  \item (20 \begin{CJK}{UTF8}{mj}分\end{CJK}) \begin{CJK}{UTF8}{mj}判定函数\end{CJK} $f(x)=\frac{x^{3}}{x+1} \sin \frac{1}{x}$ \begin{CJK}{UTF8}{mj}在\end{CJK} $(0,+\infty)$ \begin{CJK}{UTF8}{mj}上是否一致连续\end{CJK}、\begin{CJK}{UTF8}{mj}连续\end{CJK}, \begin{CJK}{UTF8}{mj}并说明理由\end{CJK}.

  \item (30 \begin{CJK}{UTF8}{mj}分\end{CJK}) \begin{CJK}{UTF8}{mj}设\end{CJK} $p>0$, \begin{CJK}{UTF8}{mj}讨论广义积分\end{CJK}

\end{enumerate}
$$
\int_{0}^{+\infty} \frac{x^{p-1}}{x^{p}+1} \cos x \mathrm{~d} x
$$
\begin{CJK}{UTF8}{mj}的绝对收敛性和条件收敛性\end{CJK}.

\begin{enumerate}
  \setcounter{enumi}{4}
  \item (30 \begin{CJK}{UTF8}{mj}分\end{CJK}) \begin{CJK}{UTF8}{mj}求级数\end{CJK}
\end{enumerate}
$$
\sum_{n=1}^{\infty} \frac{n^{2}+1}{n \cdot 2^{n}}
$$

\begin{enumerate}
  \setcounter{enumi}{5}
  \item (30 \begin{CJK}{UTF8}{mj}分\end{CJK}) \begin{CJK}{UTF8}{mj}求函数\end{CJK} $f(x, y)=y\left(x^{2}+y^{2}+\sqrt{2} x-2\right)$ \begin{CJK}{UTF8}{mj}在闭区域\end{CJK} $D=\left\{(x, y) \mid x^{2}+y^{2} \leq 3\right\}$ \begin{CJK}{UTF8}{mj}中的最\end{CJK} \begin{CJK}{UTF8}{mj}大值和最小值\end{CJK}.

  \item (10 \begin{CJK}{UTF8}{mj}分\end{CJK}) \begin{CJK}{UTF8}{mj}设函数\end{CJK} $f(x)$ \begin{CJK}{UTF8}{mj}在\end{CJK} $[0,2]$ \begin{CJK}{UTF8}{mj}可导\end{CJK}, \begin{CJK}{UTF8}{mj}在\end{CJK} $(0,2)$ \begin{CJK}{UTF8}{mj}三次可导\end{CJK}, \begin{CJK}{UTF8}{mj}并且\end{CJK}

\end{enumerate}
$$
\int_{0}^{2} f(x) \mathrm{d} x=8 \int_{0}^{1} f(x) \mathrm{d} x
$$
\begin{CJK}{UTF8}{mj}证明\end{CJK}: \begin{CJK}{UTF8}{mj}存在\end{CJK} $\xi \in(0,2)$, \begin{CJK}{UTF8}{mj}使得\end{CJK} $f^{\prime \prime \prime}(\xi)=0$.

\begin{enumerate}
  \setcounter{enumi}{7}
  \item (10 \begin{CJK}{UTF8}{mj}分\end{CJK}) \begin{CJK}{UTF8}{mj}设\end{CJK} $G$ \begin{CJK}{UTF8}{mj}是\end{CJK} $\mathbb{R}$ \begin{CJK}{UTF8}{mj}中的有界闭区域\end{CJK},\begin{CJK}{UTF8}{mj}其边界\end{CJK} $\partial G$ \begin{CJK}{UTF8}{mj}是由有限个珠片光滑曲面构成的\end{CJK}. $u=\mathbb{C}^{2}(G)$ \begin{CJK}{UTF8}{mj}且有\end{CJK} $u$ \begin{CJK}{UTF8}{mj}在\end{CJK} $\partial G$ \begin{CJK}{UTF8}{mj}上恒等于\end{CJK} 0 ,
\end{enumerate}
$$
\Delta u=\frac{\partial^{2} u}{\partial x^{2}}+\frac{\partial^{2} u}{\partial y^{2}}+\frac{\partial^{2} u}{\partial z^{2}}
$$
$\nabla u$ \begin{CJK}{UTF8}{mj}是\end{CJK} $u$ \begin{CJK}{UTF8}{mj}的梯度\end{CJK}. \begin{CJK}{UTF8}{mj}证明\end{CJK}: \begin{CJK}{UTF8}{mj}对任意\end{CJK} $\lambda>0$ \begin{CJK}{UTF8}{mj}都有\end{CJK}
$$
\lambda \iiint_{G} u^{2} \mathrm{~d} y \mathrm{~d} z+\frac{1}{\lambda} \iiint_{G}|\Delta u|^{2} \mathrm{~d} x \mathrm{~d} y \mathrm{~d} z \geq 2 \iiint_{G}|\nabla u|^{2} \mathrm{~d} x \mathrm{~d} y \mathrm{~d} z
$$

\section{$12.22020$ 年高等代数考研真题}
\section{考生须知:}
\begin{enumerate}
  \item \begin{CJK}{UTF8}{mj}本试卷满分为\end{CJK} 150 \begin{CJK}{UTF8}{mj}分\end{CJK}, \begin{CJK}{UTF8}{mj}全部考试时间总计\end{CJK} 180 \begin{CJK}{UTF8}{mj}分钟\end{CJK};

  \item \begin{CJK}{UTF8}{mj}所有答案必须写在答题纸上\end{CJK}, \begin{CJK}{UTF8}{mj}写在试题纸上或草稿纸上工律无效\end{CJK}。

  \item $(20$ \begin{CJK}{UTF8}{mj}分\end{CJK} $)$ \begin{CJK}{UTF8}{mj}求矩阵\end{CJK}

\end{enumerate}
$$
\left(\begin{array}{llll}
1 & 1 & 1 & 1 \\
2 & 1 & 1 & 1 \\
1 & 2 & 1 & 1 \\
1 & 1 & 2 & 1
\end{array}\right)
$$
\begin{CJK}{UTF8}{mj}的逆矩阵\end{CJK}.

\begin{enumerate}
  \setcounter{enumi}{2}
  \item (30 \begin{CJK}{UTF8}{mj}分\end{CJK}) \begin{CJK}{UTF8}{mj}设矩阵\end{CJK}
\end{enumerate}
$$
A=\left(\begin{array}{cccc}
0 & 1 & 1 & -1 \\
1 & 0 & -1 & 1 \\
1 & -1 & 0 & 1 \\
-1 & 1 & 1 & 0
\end{array}\right)
$$
\begin{CJK}{UTF8}{mj}求正交矩阵\end{CJK} $Q$ \begin{CJK}{UTF8}{mj}与对角矩阵\end{CJK} $D$, \begin{CJK}{UTF8}{mj}使得\end{CJK} $A=Q D Q^{\prime}$.

\begin{enumerate}
  \setcounter{enumi}{3}
  \item (30 \begin{CJK}{UTF8}{mj}分\end{CJK}) \begin{CJK}{UTF8}{mj}证明矩阵\end{CJK} $A=\left(\begin{array}{rrr}0 & 1 & 2 \\ 1 & 0 & -2 \\ -1 & 1 & 3\end{array}\right)$ \begin{CJK}{UTF8}{mj}与矩阵\end{CJK} $B=\left(\begin{array}{rrr}-1 & 1 & 3 \\ 3 & 0 & -4 \\ -2 & 1 & 4\end{array}\right)$ \begin{CJK}{UTF8}{mj}不相似\end{CJK}.

  \item (15 \begin{CJK}{UTF8}{mj}分\end{CJK}) \begin{CJK}{UTF8}{mj}已知\end{CJK} $\mathscr{A}, \mathscr{B}$ \begin{CJK}{UTF8}{mj}是有限维欧氏空间\end{CJK} $V$ \begin{CJK}{UTF8}{mj}中的线性变换\end{CJK}, $\mathscr{B}^{*}$ \begin{CJK}{UTF8}{mj}是\end{CJK} $\mathscr{B}$ \begin{CJK}{UTF8}{mj}的共辄变换\end{CJK}, \begin{CJK}{UTF8}{mj}满足\end{CJK} $\mathscr{B}{ }^{*} \mathscr{A}=\mathscr{O}$, \begin{CJK}{UTF8}{mj}证明\end{CJK} $\mathscr{A}+\mathscr{B}$ \begin{CJK}{UTF8}{mj}的秩等于\end{CJK} $\mathscr{A}$ \begin{CJK}{UTF8}{mj}的秩加\end{CJK} $\mathscr{B}$ \begin{CJK}{UTF8}{mj}的秩\end{CJK}.

  \item (15 \begin{CJK}{UTF8}{mj}分\end{CJK}) \begin{CJK}{UTF8}{mj}设\end{CJK} $\mu_{1}, \mu_{2}, \cdots, \mu_{n}$ \begin{CJK}{UTF8}{mj}是\end{CJK} $\mathbb{C}^{n \times 1}$ \begin{CJK}{UTF8}{mj}的一组基\end{CJK}, $l_{1}, l_{2}, \cdots, l_{m}$ \begin{CJK}{UTF8}{mj}是\end{CJK} $\mathbb{C}^{1 \times m}$ \begin{CJK}{UTF8}{mj}的一组基\end{CJK},\begin{CJK}{UTF8}{mj}证明\end{CJK}

\end{enumerate}
$$
\left\{\mu_{i} l_{j} \mid 1 \leq i \leq n, 1 \leq j \leq m\right\}
$$
\begin{CJK}{UTF8}{mj}是\end{CJK} $\mathbb{C}^{n \times m}$ \begin{CJK}{UTF8}{mj}的一组基\end{CJK}.

\begin{enumerate}
  \setcounter{enumi}{6}
  \item (10 \begin{CJK}{UTF8}{mj}分\end{CJK}) \begin{CJK}{UTF8}{mj}设\end{CJK} $A \in \mathbb{C}^{n \times n}$ \begin{CJK}{UTF8}{mj}有\end{CJK} $n$ \begin{CJK}{UTF8}{mj}个互不相同的特征值\end{CJK} $\lambda_{1}, \lambda_{2}, \cdots, \lambda_{n}$, \begin{CJK}{UTF8}{mj}定义\end{CJK} $\mathbb{C}^{n \times n}$ \begin{CJK}{UTF8}{mj}上的线性变换\end{CJK} $\mathrm{ad} A$ \begin{CJK}{UTF8}{mj}为\end{CJK}
\end{enumerate}
$$
\operatorname{ad}_{A}(B)=A B-B A, \quad B \in \mathbb{C}^{n \times n}
$$
\begin{CJK}{UTF8}{mj}证明入\end{CJK}; $\lambda_{i}-\lambda_{j}(1 \leq i, j \leq n, i \neq j)$ \begin{CJK}{UTF8}{mj}是\end{CJK} $\operatorname{ad}_{A}$ \begin{CJK}{UTF8}{mj}的特征值\end{CJK} 7. (10 \begin{CJK}{UTF8}{mj}分\end{CJK}) \begin{CJK}{UTF8}{mj}设\end{CJK} $p(x), g(x)$ \begin{CJK}{UTF8}{mj}是次数不超过\end{CJK} $n(n>1)$ \begin{CJK}{UTF8}{mj}的实系数多项式\end{CJK}. \begin{CJK}{UTF8}{mj}证明\end{CJK}: \begin{CJK}{UTF8}{mj}存在次数不超过\end{CJK} $2 n-2$ \begin{CJK}{UTF8}{mj}的实系数多项式\end{CJK} $F(x, y)$, \begin{CJK}{UTF8}{mj}使得\end{CJK} $F(p(t), g(t))=0$ \begin{CJK}{UTF8}{mj}对任意的实数\end{CJK} $t$ \begin{CJK}{UTF8}{mj}都成立\end{CJK}.

\begin{enumerate}
  \setcounter{enumi}{8}
  \item (10 \begin{CJK}{UTF8}{mj}分\end{CJK}) \begin{CJK}{UTF8}{mj}设\end{CJK} $B, C$ \begin{CJK}{UTF8}{mj}为\end{CJK} $n$ \begin{CJK}{UTF8}{mj}阶实对称矩阵\end{CJK}, \begin{CJK}{UTF8}{mj}且\end{CJK} $|B| \neq 0$, \begin{CJK}{UTF8}{mj}证明存在\end{CJK} $n$ \begin{CJK}{UTF8}{mj}阶实矩阵\end{CJK} $A$, \begin{CJK}{UTF8}{mj}使得\end{CJK}
\end{enumerate}
$$
A B+B A^{\prime}=C
$$

\begin{enumerate}
  \setcounter{enumi}{9}
  \item (10 \begin{CJK}{UTF8}{mj}分\end{CJK}) \begin{CJK}{UTF8}{mj}设\end{CJK} $V$ \begin{CJK}{UTF8}{mj}为\end{CJK} $n$ \begin{CJK}{UTF8}{mj}维实线性空间\end{CJK}, \begin{CJK}{UTF8}{mj}如果存在\end{CJK} $V$ \begin{CJK}{UTF8}{mj}上的可逆线性变换\end{CJK} $\mathscr{A}, \mathscr{B}$, \begin{CJK}{UTF8}{mj}使得等式\end{CJK}
\end{enumerate}
$$
\mathscr{A} \mathscr{B}-\mathscr{B} \mathscr{A}=\mathscr{B}^{2} \mathscr{A}
$$
\begin{CJK}{UTF8}{mj}成立\end{CJK}, \begin{CJK}{UTF8}{mj}求所有正整数\end{CJK} $n$ \begin{CJK}{UTF8}{mj}的可能取值\end{CJK}.

\section{第13章 华南理工大学}
\section{$13.12020$ 年数学分析考研真题}
\section{考生须知:}
\begin{enumerate}
  \item \begin{CJK}{UTF8}{mj}本试卷满分为\end{CJK} 150 \begin{CJK}{UTF8}{mj}分\end{CJK}, \begin{CJK}{UTF8}{mj}全部考试时间总计\end{CJK} 180 \begin{CJK}{UTF8}{mj}分钟\end{CJK};

  \item \begin{CJK}{UTF8}{mj}所有答案必须写在答题纸上\end{CJK}, \begin{CJK}{UTF8}{mj}写在试题纸上或草稿纸上一律无效\end{CJK}。

  \item (11 \begin{CJK}{UTF8}{mj}分\end{CJK}) \begin{CJK}{UTF8}{mj}计算\end{CJK}

\end{enumerate}
$$
\lim _{n \rightarrow \infty}\left(\frac{1+\sqrt[n]{3}}{2}\right)^{n}
$$

\begin{enumerate}
  \setcounter{enumi}{2}
  \item (11 \begin{CJK}{UTF8}{mj}分\end{CJK}) \begin{CJK}{UTF8}{mj}若\end{CJK} $\alpha>0$, \begin{CJK}{UTF8}{mj}求\end{CJK} $\lim _{x \rightarrow+\infty}\left((x+1)^{\alpha}-x^{\alpha}\right)$.

  \item (11 \begin{CJK}{UTF8}{mj}分\end{CJK}) \begin{CJK}{UTF8}{mj}求级数\end{CJK} $\sum_{n=1}^{\infty} n^{2} x^{n-1}$ \begin{CJK}{UTF8}{mj}的和函数\end{CJK}.

  \item (11 \begin{CJK}{UTF8}{mj}分\end{CJK}) \begin{CJK}{UTF8}{mj}计算\end{CJK}

\end{enumerate}
$$
\int_{0}^{+\infty} \frac{[\sin (5 x)-\sin (3 x)] \mathrm{e}^{-2 x}}{x} \mathrm{~d} x
$$

\begin{enumerate}
  \setcounter{enumi}{5}
  \item (13 \begin{CJK}{UTF8}{mj}分\end{CJK}) \begin{CJK}{UTF8}{mj}设\end{CJK} $L$ \begin{CJK}{UTF8}{mj}为\end{CJK} $y=0$ \begin{CJK}{UTF8}{mj}与\end{CJK} $y=\sin x, 0 \leq x \leq \pi$ \begin{CJK}{UTF8}{mj}所围区域的边界\end{CJK}, \begin{CJK}{UTF8}{mj}积分方向取正\end{CJK}, \begin{CJK}{UTF8}{mj}计算\end{CJK}
\end{enumerate}
$$
\int_{L} e^{x}(1-\cos y) \mathrm{d} x-\mathrm{e}^{x}(y-\sin y) \mathrm{d} y
$$

\begin{enumerate}
  \setcounter{enumi}{6}
  \item (13 \begin{CJK}{UTF8}{mj}分\end{CJK}) \begin{CJK}{UTF8}{mj}设函数\end{CJK} $f(x)$ \begin{CJK}{UTF8}{mj}在闭区间\end{CJK} $[a, b]$ \begin{CJK}{UTF8}{mj}上可积\end{CJK}, \begin{CJK}{UTF8}{mj}证明\end{CJK}: $\int_{a}^{b} f^{2}(x) \mathrm{d} x=0$ \begin{CJK}{UTF8}{mj}当且仅当对\end{CJK} $f(x)$ \begin{CJK}{UTF8}{mj}的任\end{CJK} \begin{CJK}{UTF8}{mj}意连续点\end{CJK}, \begin{CJK}{UTF8}{mj}有\end{CJK} $f(x)=0$.

  \item $\left(13\right.$ \begin{CJK}{UTF8}{mj}分\end{CJK}) \begin{CJK}{UTF8}{mj}设\end{CJK} $f(x)$ \begin{CJK}{UTF8}{mj}在\end{CJK} $[0, \infty)$ \begin{CJK}{UTF8}{mj}上单调递减\end{CJK}, \begin{CJK}{UTF8}{mj}且\end{CJK} $\lim _{x \rightarrow+\infty} f(x)=0, f^{\prime}(x)$ \begin{CJK}{UTF8}{mj}连续\end{CJK}, \begin{CJK}{UTF8}{mj}证明\end{CJK}: \begin{CJK}{UTF8}{mj}积分\end{CJK} $\int_{0}^{+\infty} f^{\prime}(x) \sin ^{2} x \mathrm{~d} x$ \begin{CJK}{UTF8}{mj}收敛\end{CJK}.

  \item (13 \begin{CJK}{UTF8}{mj}分\end{CJK}) \begin{CJK}{UTF8}{mj}设\end{CJK} $f(x)$ \begin{CJK}{UTF8}{mj}在\end{CJK} $[a, b]$ \begin{CJK}{UTF8}{mj}上可微且导函数连续\end{CJK}, $f(a)=0$, \begin{CJK}{UTF8}{mj}试证\end{CJK}:

\end{enumerate}
$$
M^{2} \leq(b-a) \int_{a}^{b}\left(f^{\prime}(x)\right)^{2} \mathrm{~d} x
$$
\begin{CJK}{UTF8}{mj}其中\end{CJK} $M=\sup _{x \in[a, b]}|f(x)|$. 9. (13 \begin{CJK}{UTF8}{mj}分\end{CJK}) \begin{CJK}{UTF8}{mj}求曲面\end{CJK} $2 x^{2}+y^{2}+3 z^{2}=84$ \begin{CJK}{UTF8}{mj}的切平面\end{CJK}, \begin{CJK}{UTF8}{mj}使得该切平面与平面\end{CJK} $4 x+y+6 z=12$ \begin{CJK}{UTF8}{mj}平行\end{CJK}.

\begin{enumerate}
  \setcounter{enumi}{10}
  \item (13 \begin{CJK}{UTF8}{mj}分\end{CJK}) \begin{CJK}{UTF8}{mj}设\end{CJK} $f(x)$ \begin{CJK}{UTF8}{mj}在\end{CJK} $[0,2]$ \begin{CJK}{UTF8}{mj}上二阶可微\end{CJK}, \begin{CJK}{UTF8}{mj}且\end{CJK} $|f(x)| \leq 1,\left|f^{\prime \prime}(x)\right| \leq 1$, \begin{CJK}{UTF8}{mj}证明\end{CJK}: $\left|f^{\prime}(x)\right| \leq 2$.

  \item (13 \begin{CJK}{UTF8}{mj}分\end{CJK}) \begin{CJK}{UTF8}{mj}已知\end{CJK} $\lim _{n \rightarrow \infty} a_{n}=0, \lim _{n \rightarrow \infty} b_{n}=b$, \begin{CJK}{UTF8}{mj}试证\end{CJK}:

\end{enumerate}
$$
\lim _{n \rightarrow \infty} \frac{a_{1} b_{n}+a_{2} b_{n-1}+\cdots+a_{n} b_{1}}{n}=0
$$

\begin{enumerate}
  \setcounter{enumi}{12}
  \item (15 \begin{CJK}{UTF8}{mj}分\end{CJK}) \begin{CJK}{UTF8}{mj}设\end{CJK} $f(x)$ \begin{CJK}{UTF8}{mj}在\end{CJK} $[0,+\infty)$ \begin{CJK}{UTF8}{mj}上一致连续\end{CJK}, \begin{CJK}{UTF8}{mj}对\end{CJK} $\forall x \geq 0$ \begin{CJK}{UTF8}{mj}有\end{CJK} $\lim _{n \rightarrow \infty} f(x+n)=0$, \begin{CJK}{UTF8}{mj}证明\end{CJK}: $\lim _{x \rightarrow+\infty} f(x)=$ 0 .
\end{enumerate}
\section{$13.22020$ 年高等代数考研真题}
\section{考生须知:}
\begin{enumerate}
  \item \begin{CJK}{UTF8}{mj}本试卷满分为\end{CJK} 150 \begin{CJK}{UTF8}{mj}分\end{CJK},\begin{CJK}{UTF8}{mj}全部考试时间总计\end{CJK} 180 \begin{CJK}{UTF8}{mj}分钟\end{CJK};

  \item \begin{CJK}{UTF8}{mj}所有答案必须写在答题纸上\end{CJK}, \begin{CJK}{UTF8}{mj}写在试题纸上或草稿纸上一律无效\end{CJK}。

  \item (20 \begin{CJK}{UTF8}{mj}分\end{CJK}) \begin{CJK}{UTF8}{mj}已知\end{CJK} $f(x), g(x) \in P[x]$, \begin{CJK}{UTF8}{mj}证明\end{CJK}: $(f(x), g(x))=1$ \begin{CJK}{UTF8}{mj}的充分必要条件是对\end{CJK} $\forall r(x), s(x), \exists q(x), p(x)$ \begin{CJK}{UTF8}{mj}使得\end{CJK} $p(x) f(x)+r(x)=q(x) g(x)+s(x)$

  \item (15 \begin{CJK}{UTF8}{mj}分\end{CJK}) \begin{CJK}{UTF8}{mj}计算\end{CJK} $n+1$ \begin{CJK}{UTF8}{mj}阶行列式\end{CJK}

\end{enumerate}
$$
D=\left|\begin{array}{ccccc}
a^{n} & (a-1)^{n} & (a-2)^{n} & \cdots & (a-n)^{n} \\
a^{n-1} & (a-1)^{n-1} & (a-2)^{n-1} & \cdots & (a-n)^{n-1} \\
\vdots & \vdots & \vdots & & \vdots \\
a & a-1 & a-2 & \cdots & a-n \\
1 & 1 & 1 & \cdots & 1
\end{array}\right|
$$

\begin{enumerate}
  \setcounter{enumi}{3}
  \item (20 \begin{CJK}{UTF8}{mj}分\end{CJK}) \begin{CJK}{UTF8}{mj}已知\end{CJK}
\end{enumerate}
$$
A=\left(\begin{array}{lll}
1 & 0 & 0 \\
1 & 0 & 1 \\
0 & 1 & 0
\end{array}\right)
$$
(1). \begin{CJK}{UTF8}{mj}证明\end{CJK}: $A^{n}=A^{n-2}+A^{2}-E(n \geq 3)$

(2). $A^{2020}$.

\begin{enumerate}
  \setcounter{enumi}{4}
  \item (20 \begin{CJK}{UTF8}{mj}分\end{CJK}) \begin{CJK}{UTF8}{mj}已知\end{CJK} $\alpha_{1}=(1,2,0), \alpha_{2}=(1, a+2,-3 a), \alpha_{3}=(-1,-b-2, a+2 b), \beta=(1,3,-3)$ \begin{CJK}{UTF8}{mj}求\end{CJK} $a, b$ \begin{CJK}{UTF8}{mj}的值使得\end{CJK}
\end{enumerate}
(1). $\beta$ \begin{CJK}{UTF8}{mj}不可被\end{CJK} $\alpha_{1}, \alpha_{2}, \alpha_{3}$ \begin{CJK}{UTF8}{mj}线性表出\end{CJK}; (2). $\beta$ \begin{CJK}{UTF8}{mj}可被\end{CJK} $\alpha_{1}, \alpha_{2}, \alpha_{3}$ \begin{CJK}{UTF8}{mj}唯一线性表出\end{CJK}, \begin{CJK}{UTF8}{mj}并求表达式\end{CJK};

(3). $\beta$ \begin{CJK}{UTF8}{mj}可被\end{CJK} $\alpha_{1}, \alpha_{2}, \alpha_{3}$ \begin{CJK}{UTF8}{mj}线性表出且不唯一\end{CJK}, \begin{CJK}{UTF8}{mj}并求表达式\end{CJK}.

\begin{enumerate}
  \setcounter{enumi}{5}
  \item (20 \begin{CJK}{UTF8}{mj}分\end{CJK}) \begin{CJK}{UTF8}{mj}已知\end{CJK} $\left.f\left(x_{1}, x_{2}, \cdots\right), x_{n}\right)=\sum_{i=1}^{n} \sum_{j=1}^{n} a_{i j} x_{i} x_{j}$ \begin{CJK}{UTF8}{mj}正负惯性指数分别为\end{CJK} $p, q$, \begin{CJK}{UTF8}{mj}且\end{CJK}
\end{enumerate}
$$
\alpha_{1}, \alpha_{2}, \cdots, \alpha_{p}, \beta_{1}, \beta_{2}, \cdots, \beta_{q}
$$
\begin{CJK}{UTF8}{mj}为任意\end{CJK} $p+q$ \begin{CJK}{UTF8}{mj}个正数\end{CJK}, \begin{CJK}{UTF8}{mj}证明\end{CJK}: \begin{CJK}{UTF8}{mj}存在非退化线性替换\end{CJK} $X=C Y$ \begin{CJK}{UTF8}{mj}使得\end{CJK}
$$
f(X)=\alpha_{1} y_{1}^{2}+\cdots+\alpha_{p} y_{p}^{2}-\beta_{1} y_{p+1}^{2}-\cdots-\beta_{q} y_{p+q}^{2}
$$

\begin{enumerate}
  \setcounter{enumi}{6}
  \item (20 \begin{CJK}{UTF8}{mj}分\end{CJK}) \begin{CJK}{UTF8}{mj}设\end{CJK} $V$ \begin{CJK}{UTF8}{mj}是数域\end{CJK} $P$ \begin{CJK}{UTF8}{mj}上的\end{CJK} $n$ \begin{CJK}{UTF8}{mj}维线性空间\end{CJK}, \begin{CJK}{UTF8}{mj}且\end{CJK} $V=P[x]_{n},(f(x))=x f^{\prime}(x)-f(x), f(x) \in V$
\end{enumerate}
(1). \begin{CJK}{UTF8}{mj}证明\end{CJK}: $\mathscr{A}$ \begin{CJK}{UTF8}{mj}为线性变换\end{CJK};

(2). \begin{CJK}{UTF8}{mj}求\end{CJK} $\mathscr{A}^{-1}(0)$ \begin{CJK}{UTF8}{mj}与\end{CJK} $\mathscr{A}(V)$;

(3). \begin{CJK}{UTF8}{mj}证明\end{CJK}: $V=\mathscr{A}^{-1}(0) \oplus \mathscr{A}(V)$.

\begin{enumerate}
  \setcounter{enumi}{7}
  \item (20 \begin{CJK}{UTF8}{mj}分\end{CJK}) \begin{CJK}{UTF8}{mj}已知\end{CJK} $A, B$ \begin{CJK}{UTF8}{mj}为数域\end{CJK} $P$ \begin{CJK}{UTF8}{mj}上的\end{CJK} $n$ \begin{CJK}{UTF8}{mj}阶方阵\end{CJK}, $A$ \begin{CJK}{UTF8}{mj}有\end{CJK} $n$ \begin{CJK}{UTF8}{mj}个互异的特征值\end{CJK}, \begin{CJK}{UTF8}{mj}证明\end{CJK}: $A$ \begin{CJK}{UTF8}{mj}的特征向量\end{CJK} \begin{CJK}{UTF8}{mj}是\end{CJK} $B$ \begin{CJK}{UTF8}{mj}的特征向量的充分必要条件是\end{CJK} $A B=B A$.

  \item (15 \begin{CJK}{UTF8}{mj}分\end{CJK}) \begin{CJK}{UTF8}{mj}已知\end{CJK} $V$ \begin{CJK}{UTF8}{mj}是数域\end{CJK} $P$ \begin{CJK}{UTF8}{mj}上的\end{CJK} $n$ \begin{CJK}{UTF8}{mj}维线性空间\end{CJK}, $\varepsilon_{1}, \varepsilon_{2}, \varepsilon_{3}, \varepsilon_{4}$ \begin{CJK}{UTF8}{mj}是\end{CJK} $V$ \begin{CJK}{UTF8}{mj}的一组基\end{CJK}, $\mathscr{A}$ \begin{CJK}{UTF8}{mj}在\end{CJK} $\varepsilon_{1}, \varepsilon_{2}, \varepsilon_{3}, \varepsilon_{4}$ F\begin{CJK}{UTF8}{mj}的矩阵为\end{CJK}

\end{enumerate}
$$
A=\left(\begin{array}{cccc}
1 & -1 & -1 & 2 \\
0 & 1 & 0 & 0 \\
2 & 3 & 1 & 1 \\
1 & -2 & -2 & -1
\end{array}\right)
$$
\begin{CJK}{UTF8}{mj}求包含\end{CJK} $\varepsilon_{1}$ \begin{CJK}{UTF8}{mj}的最小不变子空间\end{CJK}.

\includegraphics[max width=\textwidth]{2022_04_18_7db0708508f26638f054g-056}

\section{第14章 天津大学}
\section{$14.12020$ 年数学分析考研真题}
\section{考生须知:}
\begin{enumerate}
  \item \begin{CJK}{UTF8}{mj}本试卷满分为\end{CJK} 150 \begin{CJK}{UTF8}{mj}分\end{CJK}, \begin{CJK}{UTF8}{mj}全部考试时间总计\end{CJK} 180 \begin{CJK}{UTF8}{mj}分钟\end{CJK};

  \item \begin{CJK}{UTF8}{mj}所有答案必须写在答题纸上\end{CJK}, \begin{CJK}{UTF8}{mj}写在试题纸上或草稿纸上一律无效\end{CJK}。

  \item \begin{CJK}{UTF8}{mj}填空题\end{CJK} (\begin{CJK}{UTF8}{mj}每小题\end{CJK} 5 \begin{CJK}{UTF8}{mj}分\end{CJK}, \begin{CJK}{UTF8}{mj}共\end{CJK} 50 \begin{CJK}{UTF8}{mj}分\end{CJK})

\end{enumerate}
(1). \begin{CJK}{UTF8}{mj}计算\end{CJK} $\lim _{n \rightarrow \infty} n\left(\sqrt{n^{2}+1}-n\right)=$

(2). \begin{CJK}{UTF8}{mj}计算\end{CJK} $\lim _{x \rightarrow 0} \frac{\mathrm{e}^{x}-1-x}{\cos x-1}=$

(3). \begin{CJK}{UTF8}{mj}若点\end{CJK} $(0,1)$ \begin{CJK}{UTF8}{mj}满足\end{CJK} $y=x^{3}+b x^{2}+c$ \begin{CJK}{UTF8}{mj}的拐点\end{CJK}, \begin{CJK}{UTF8}{mj}则\end{CJK} $b, c$ \begin{CJK}{UTF8}{mj}所满足的条件是\end{CJK} $=$

(4). \begin{CJK}{UTF8}{mj}级数\end{CJK} $\sum_{n=1}^{\infty} n(x-1)^{n}$ \begin{CJK}{UTF8}{mj}的收敛域\end{CJK}.

(5). \begin{CJK}{UTF8}{mj}当且仅当\end{CJK} $p$ \begin{CJK}{UTF8}{mj}属于\end{CJK} \begin{CJK}{UTF8}{mj}何时\end{CJK}, \begin{CJK}{UTF8}{mj}无穷积分\end{CJK} $\int_{0}^{+\infty} \frac{\mathrm{d} x}{x \ln x[\ln (\ln x)] p}$ \begin{CJK}{UTF8}{mj}收敛\end{CJK} $=$

(6). \begin{CJK}{UTF8}{mj}曲线\end{CJK} $x=\cos ^{3} t, y=\sin ^{3} t, t \in(0,2 \pi)$ \begin{CJK}{UTF8}{mj}的弧长为\end{CJK}

(7). \begin{CJK}{UTF8}{mj}曲线\end{CJK} $\mathrm{e}^{z}-z+x y=3$ \begin{CJK}{UTF8}{mj}在点\end{CJK} $(2,1,0)$ \begin{CJK}{UTF8}{mj}处的切平面为\end{CJK}

(8). \begin{CJK}{UTF8}{mj}求\end{CJK} $f(x)=x$ \begin{CJK}{UTF8}{mj}在\end{CJK} $(-\pi, \pi]$ \begin{CJK}{UTF8}{mj}上的俌里叶展开\end{CJK}

(9). \begin{CJK}{UTF8}{mj}计算\end{CJK} $\int_{0}^{\frac{\pi}{4}} \frac{x}{1+\cos 2 x} \mathrm{~d} x=$

$(10)$. \begin{CJK}{UTF8}{mj}若\end{CJK} $L$ \begin{CJK}{UTF8}{mj}为不经过原点的简单闭曲线\end{CJK}, \begin{CJK}{UTF8}{mj}方向为逆时针\end{CJK}, \begin{CJK}{UTF8}{mj}则\end{CJK} $\int_{L} \frac{x \mathrm{~d} y-y \mathrm{~d} x}{x^{2}+y^{2}}=$

\begin{enumerate}
  \setcounter{enumi}{2}
  \item (15 \begin{CJK}{UTF8}{mj}分\end{CJK}) \begin{CJK}{UTF8}{mj}计算题\end{CJK}
\end{enumerate}
(1). \begin{CJK}{UTF8}{mj}求\end{CJK} $\sum_{n=1}^{\infty} \frac{x^{n}}{n}$ \begin{CJK}{UTF8}{mj}在\end{CJK} $(-1,1)$ \begin{CJK}{UTF8}{mj}上的和函数\end{CJK};

(2). \begin{CJK}{UTF8}{mj}求\end{CJK} $\lim _{x \rightarrow+\infty} \sum_{n=1}^{\infty} \frac{1}{n}\left(\frac{1-x}{x}\right)^{n}$.

\begin{enumerate}
  \setcounter{enumi}{3}
  \item (15 \begin{CJK}{UTF8}{mj}分\end{CJK}) \begin{CJK}{UTF8}{mj}计算题\end{CJK} (1). \begin{CJK}{UTF8}{mj}求\end{CJK} $\iiint_{\Omega} x^{2}+y^{2}+z^{2} \mathrm{~d} x \mathrm{~d} y \mathrm{~d} z$, \begin{CJK}{UTF8}{mj}其中\end{CJK} $\Omega=\left\{(x, y, z) \mid x^{2}+y^{2}+z^{2} \leq 1\right\}$;
\end{enumerate}
(2). \begin{CJK}{UTF8}{mj}求\end{CJK} $\iint_{\Sigma} x \mathrm{~d} y \mathrm{~d} z+y \mathrm{~d} z \mathrm{~d} x+z \mathrm{~d} x \mathrm{~d} y$, \begin{CJK}{UTF8}{mj}其中\end{CJK} $\Sigma$ \begin{CJK}{UTF8}{mj}为\end{CJK} $z=x^{2}+y^{2}$ \begin{CJK}{UTF8}{mj}外侧\end{CJK} $z \in[0,1]$.

\begin{enumerate}
  \setcounter{enumi}{4}
  \item (10 \begin{CJK}{UTF8}{mj}分\end{CJK}) \begin{CJK}{UTF8}{mj}若\end{CJK} $f(x)$ \begin{CJK}{UTF8}{mj}在\end{CJK} $[a,+\infty)$ \begin{CJK}{UTF8}{mj}上连续\end{CJK}, \begin{CJK}{UTF8}{mj}且\end{CJK} $\lim _{x \rightarrow+\infty} f(x)$ \begin{CJK}{UTF8}{mj}存在\end{CJK}, $a$ \begin{CJK}{UTF8}{mj}为常数\end{CJK}, \begin{CJK}{UTF8}{mj}试证\end{CJK}: $f(x)$ \begin{CJK}{UTF8}{mj}在\end{CJK} $[a,+\infty)$ \begin{CJK}{UTF8}{mj}上一致连续\end{CJK}.

  \item (10 \begin{CJK}{UTF8}{mj}分\end{CJK}) \begin{CJK}{UTF8}{mj}若\end{CJK} $f(x)$ \begin{CJK}{UTF8}{mj}在\end{CJK} $[1,+\infty)$ \begin{CJK}{UTF8}{mj}上单调\end{CJK}, \begin{CJK}{UTF8}{mj}且\end{CJK} $\int_{1}^{+\infty} f(x) \mathrm{d} x$ \begin{CJK}{UTF8}{mj}收敛\end{CJK}, \begin{CJK}{UTF8}{mj}试证\end{CJK}: $\lim _{x \rightarrow+\infty} x f(x)=0$.

  \item (10 \begin{CJK}{UTF8}{mj}分\end{CJK}) \begin{CJK}{UTF8}{mj}已知函数\end{CJK} $f(x, y), x=x(u, v), y=y(u, v)$ \begin{CJK}{UTF8}{mj}二阶可导且满足\end{CJK} $\frac{\partial x}{\partial u}=\frac{\partial y}{\partial v}, \frac{\partial x}{\partial v}=-\frac{\partial y}{\partial u}($ \begin{CJK}{UTF8}{mj}柯\end{CJK} \begin{CJK}{UTF8}{mj}西黎曼方程\end{CJK}), \begin{CJK}{UTF8}{mj}证明\end{CJK}:

\end{enumerate}
$$
\frac{\partial^{2} f}{\partial u^{2}}+\frac{\partial^{2} f}{\partial v^{2}}=\left(\frac{\partial x}{\partial u} \frac{\partial y}{\partial v}-\frac{\partial x}{\partial v} \frac{\partial y}{\partial u}\right) \cdot\left(\frac{\partial^{2} f}{\partial x^{2}}+\frac{\partial^{2} f}{\partial y^{2}}\right)
$$

\begin{enumerate}
  \setcounter{enumi}{7}
  \item (10 \begin{CJK}{UTF8}{mj}分\end{CJK}) \begin{CJK}{UTF8}{mj}求由方程\end{CJK} $x^{2}+y^{2}+z^{4}-2 x-2 y-5 z-4=0$ \begin{CJK}{UTF8}{mj}所确定的\end{CJK} $z=z(x, y)$ \begin{CJK}{UTF8}{mj}的整数极值点并\end{CJK} \begin{CJK}{UTF8}{mj}半䉼极值点类型\end{CJK}.

  \item $(10$ \begin{CJK}{UTF8}{mj}分\end{CJK} $)$ \begin{CJK}{UTF8}{mj}若\end{CJK} $f(x)$ \begin{CJK}{UTF8}{mj}在\end{CJK} $(-\infty,+\infty)$ \begin{CJK}{UTF8}{mj}上二阶可导\end{CJK}, \begin{CJK}{UTF8}{mj}且有\end{CJK} $|f(x)| \leq M_{0},\left|f^{\prime \prime}(x)\right| \leq M_{1}$, \begin{CJK}{UTF8}{mj}证明\end{CJK}: $\left|f^{\prime}(x)\right| \leq$ $\sqrt{2 M_{0} M_{1}}$

  \item (10 \begin{CJK}{UTF8}{mj}分\end{CJK}) \begin{CJK}{UTF8}{mj}设\end{CJK} $\theta \in(-1,1)$, \begin{CJK}{UTF8}{mj}试证\end{CJK}:

\end{enumerate}
$$
\int_{0}^{\pi} \ln (1+\theta \cos x) \mathrm{d} x=\pi \ln \frac{1+\sqrt{1-\theta^{2}}}{2}
$$

\section{$14.22020$ 年高等代数考研真题}
\section{考生须知:}
\begin{enumerate}
  \item \begin{CJK}{UTF8}{mj}本试卷满分为\end{CJK} 150 \begin{CJK}{UTF8}{mj}分\end{CJK}, \begin{CJK}{UTF8}{mj}王部考试时间总计\end{CJK} 180 \begin{CJK}{UTF8}{mj}分钟\end{CJK};

  \item \begin{CJK}{UTF8}{mj}所有答案必须写在答题纸上\end{CJK}, \begin{CJK}{UTF8}{mj}写在试题纸上或草稿纸上一律无效\end{CJK}。

  \item \begin{CJK}{UTF8}{mj}已知\end{CJK} $f(x), g(x)$ \begin{CJK}{UTF8}{mj}为数域\end{CJK} $K$ \begin{CJK}{UTF8}{mj}上的多项式\end{CJK}, $m$ \begin{CJK}{UTF8}{mj}为大于\end{CJK} 1 \begin{CJK}{UTF8}{mj}的正整数\end{CJK}, \begin{CJK}{UTF8}{mj}证明\end{CJK}:

\end{enumerate}
$$
g^{m}(x) \mid f^{m}(x) \text { 当且仅当 } g(x) \mid f(x)
$$

\begin{enumerate}
  \setcounter{enumi}{2}
  \item \begin{CJK}{UTF8}{mj}已知\end{CJK} $A=\left(a_{i j}\right)$ \begin{CJK}{UTF8}{mj}为\end{CJK} $n$ \begin{CJK}{UTF8}{mj}阶矩阵\end{CJK}, \begin{CJK}{UTF8}{mj}且\end{CJK} $a_{i j}=\left\{\begin{array}{ll}n, & i=j \\ -1, & i \neq j\end{array}(1 \leq i, j \leq n)\right.$, \begin{CJK}{UTF8}{mj}求\end{CJK} $|A|$.

  \item \begin{CJK}{UTF8}{mj}已知\end{CJK} $A B A=C$, \begin{CJK}{UTF8}{mj}其中\end{CJK} $A=\left(\begin{array}{ccc}1 & 0 & 1 \\ 0 & 4 & 2 \\ 1 & -1 & 0\end{array}\right), C=\left(\begin{array}{ccc}1 & 1 & 0 \\ 0 & 1 & 0 \\ 0 & 0 & 1\end{array}\right)$, \begin{CJK}{UTF8}{mj}试求\end{CJK} $B$ \begin{CJK}{UTF8}{mj}的伴随矩阵\end{CJK} $B^{*}$. 4. \begin{CJK}{UTF8}{mj}已知\end{CJK} $A=\left(\begin{array}{ccc}1 & -2020 & 2020 \\ 0 & 2 & -1 \\ 0 & a & 4\end{array}\right)$ \begin{CJK}{UTF8}{mj}有一个二重特征根\end{CJK}.

\end{enumerate}
(1). \begin{CJK}{UTF8}{mj}求\end{CJK} $A$ \begin{CJK}{UTF8}{mj}的最小多项式以及若尔当标准形\end{CJK}.

(2). \begin{CJK}{UTF8}{mj}若\end{CJK} $A$ \begin{CJK}{UTF8}{mj}相似于对角矩阵\end{CJK}, \begin{CJK}{UTF8}{mj}求\end{CJK} $A^{2020}$.

\begin{enumerate}
  \setcounter{enumi}{5}
  \item \begin{CJK}{UTF8}{mj}已知\end{CJK} $A$ \begin{CJK}{UTF8}{mj}为\end{CJK} $n$ \begin{CJK}{UTF8}{mj}阶方阵\end{CJK}, \begin{CJK}{UTF8}{mj}且\end{CJK} $A^{r}=0\left(r \in \mathbb{N}^{+}\right)$, \begin{CJK}{UTF8}{mj}证明\end{CJK}: $E+2 A$ \begin{CJK}{UTF8}{mj}为可逆矩阵\end{CJK}.

  \item \begin{CJK}{UTF8}{mj}已知\end{CJK} $\tau$ \begin{CJK}{UTF8}{mj}为线性空间\end{CJK} $\mathbb{R}^{3}$ \begin{CJK}{UTF8}{mj}上的线性变换\end{CJK}, \begin{CJK}{UTF8}{mj}且\end{CJK} $\tau$ \begin{CJK}{UTF8}{mj}在基\end{CJK} $v_{1}, v_{2}, v_{3}$ \begin{CJK}{UTF8}{mj}下的矩阵为\end{CJK} $\left(\begin{array}{ccccc}3 & 0 & 3 \\ 3 & 3 & 0 \\ 6 & 3\end{array}\right)$, \begin{CJK}{UTF8}{mj}求\end{CJK} $\tau$ \begin{CJK}{UTF8}{mj}在基\end{CJK} $u_{1}=(2,0,0), u_{2}=(0,2,0), u_{3}=(0,0,2)$ \begin{CJK}{UTF8}{mj}下的矩阵\end{CJK}.

  \item \begin{CJK}{UTF8}{mj}设\end{CJK} $K^{n \times n}$ \begin{CJK}{UTF8}{mj}为数域\end{CJK} $K$ \begin{CJK}{UTF8}{mj}上全体\end{CJK} $n$ \begin{CJK}{UTF8}{mj}阶方阵构成的线性空间\end{CJK}. $V_{1}$ \begin{CJK}{UTF8}{mj}为\end{CJK} $K$ \begin{CJK}{UTF8}{mj}上全体\end{CJK} $n$ \begin{CJK}{UTF8}{mj}阶对称方阵构成的子\end{CJK} \begin{CJK}{UTF8}{mj}空间\end{CJK}, $V_{2}$ \begin{CJK}{UTF8}{mj}为\end{CJK} $K$ \begin{CJK}{UTF8}{mj}上全体\end{CJK} $n$ \begin{CJK}{UTF8}{mj}阶反对称方阵构成的子空间\end{CJK}, \begin{CJK}{UTF8}{mj}证明\end{CJK}:

\end{enumerate}
$$
K^{n \times n}=V_{1} \oplus V_{2}
$$

\begin{enumerate}
  \setcounter{enumi}{8}
  \item \begin{CJK}{UTF8}{mj}在\end{CJK} $\mathbb{R}^{n}$ \begin{CJK}{UTF8}{mj}中定义又线性函数\end{CJK}
\end{enumerate}
$$
f(x, y)=\sum_{i=1}^{n} x_{i} y_{i}+\frac{1}{2} \sum_{\substack{i \neq j \\ 1 \leq i, j \leq n}} x_{i} x_{j}
$$
\begin{CJK}{UTF8}{mj}其中\end{CJK} $x=\left(x_{1}, x_{2}, \cdots, x_{n}\right), y=\left(y_{1}, y_{2}, \cdots, y_{n}\right)$, \begin{CJK}{UTF8}{mj}问\end{CJK}: $f(x, y)$ \begin{CJK}{UTF8}{mj}是否为\end{CJK} $\mathbb{R}^{n}$ \begin{CJK}{UTF8}{mj}上的内积\end{CJK}? \begin{CJK}{UTF8}{mj}若是请证\end{CJK} \begin{CJK}{UTF8}{mj}明\end{CJK}, \begin{CJK}{UTF8}{mj}若不是\end{CJK}, \begin{CJK}{UTF8}{mj}请说明理由\end{CJK}.

\begin{enumerate}
  \setcounter{enumi}{9}
  \item \begin{CJK}{UTF8}{mj}已知\end{CJK} $A$ \begin{CJK}{UTF8}{mj}为复数域上的\end{CJK} $n$ \begin{CJK}{UTF8}{mj}阶矩阵\end{CJK}, \begin{CJK}{UTF8}{mj}且存在正整数\end{CJK} $m$ \begin{CJK}{UTF8}{mj}使得\end{CJK} $A^{m}=E$. \begin{CJK}{UTF8}{mj}证明\end{CJK}:
\end{enumerate}
(1). $A$ \begin{CJK}{UTF8}{mj}与对称矩阵相似\end{CJK}.

(2). $A$ \begin{CJK}{UTF8}{mj}的特征值为\end{CJK} $m$ \begin{CJK}{UTF8}{mj}次单位根\end{CJK}.

\begin{enumerate}
  \setcounter{enumi}{10}
  \item \begin{CJK}{UTF8}{mj}已知\end{CJK} $f(x), g(x)$ \begin{CJK}{UTF8}{mj}为数域\end{CJK} $K$ \begin{CJK}{UTF8}{mj}上的两个一元多项式\end{CJK}, \begin{CJK}{UTF8}{mj}证明\end{CJK}: \begin{CJK}{UTF8}{mj}存在\end{CJK} $K$ \begin{CJK}{UTF8}{mj}上非零二元多项式\end{CJK} $p(x, y)$, \begin{CJK}{UTF8}{mj}使得\end{CJK} $p(f(x), g(x)) \equiv 0$.
\end{enumerate}
\section{第15章 中国海洋大学}
\section{$15.12020$ 年数学分析考研真题}
\section{考生须知:}
\begin{enumerate}
  \item \begin{CJK}{UTF8}{mj}本试卷满分为\end{CJK} 150 \begin{CJK}{UTF8}{mj}分\end{CJK}, \begin{CJK}{UTF8}{mj}全部考试时间总计\end{CJK} 180 \begin{CJK}{UTF8}{mj}分钟\end{CJK};

  \item \begin{CJK}{UTF8}{mj}所有答案必须写在答题纸上\end{CJK}, \begin{CJK}{UTF8}{mj}写在试题纸上或草稿纸上一律无效\end{CJK}。

  \item \begin{CJK}{UTF8}{mj}计算题\end{CJK}\\
(1). $\lim _{x \rightarrow 0} \lim _{n \rightarrow \infty}\left(\cos x \cdot \cos \frac{x}{2} \cdot \cos \frac{x}{2^{2}} \cdots \cos \frac{x}{2^{n}}\right)$\\
(2). $\lim _{\substack{x \rightarrow 0 \\ y \rightarrow 0}}\left(x^{2}+y^{2}\right)^{x^{2} y^{2}}$

  \item \begin{CJK}{UTF8}{mj}判断下列命题是否正确\end{CJK}, \begin{CJK}{UTF8}{mj}正确的给予证明\end{CJK}, \begin{CJK}{UTF8}{mj}错误的给出反例\end{CJK}.

\end{enumerate}
(1). \begin{CJK}{UTF8}{mj}若函数\end{CJK} $f(x)$ \begin{CJK}{UTF8}{mj}在\end{CJK} $[a, b]$ \begin{CJK}{UTF8}{mj}上连续\end{CJK}, \begin{CJK}{UTF8}{mj}且对\end{CJK} $\forall x \in[a, b]$, \begin{CJK}{UTF8}{mj}都有\end{CJK} $f(x)>0$, \begin{CJK}{UTF8}{mj}则\end{CJK} $\exists r>0$, \begin{CJK}{UTF8}{mj}对\end{CJK} $\forall x \in[a, b]$, \begin{CJK}{UTF8}{mj}有\end{CJK} $f(x)>r$.

(2). \begin{CJK}{UTF8}{mj}若函数\end{CJK} $f(x)$ \begin{CJK}{UTF8}{mj}可导\end{CJK}, \begin{CJK}{UTF8}{mj}则其导函数\end{CJK} $f^{\prime}(x)$ \begin{CJK}{UTF8}{mj}不存在第二类间断点\end{CJK}.

(3). \begin{CJK}{UTF8}{mj}若级数\end{CJK} $\sum_{n=1}^{\infty}\left(a_{2 n-1}+a_{2 n}\right)$ \begin{CJK}{UTF8}{mj}收敛\end{CJK}, \begin{CJK}{UTF8}{mj}且\end{CJK} $\lim _{n \rightarrow \infty} a_{n}=0$, \begin{CJK}{UTF8}{mj}则\end{CJK} $\sum_{n=1}^{\infty} a_{n}$ \begin{CJK}{UTF8}{mj}收敛\end{CJK}.

\begin{enumerate}
  \setcounter{enumi}{3}
  \item \begin{CJK}{UTF8}{mj}设函数\end{CJK} $f(x)$ \begin{CJK}{UTF8}{mj}在\end{CJK} $(a,+\infty)$ \begin{CJK}{UTF8}{mj}内可导\end{CJK}, \begin{CJK}{UTF8}{mj}且\end{CJK} $\lim _{x \rightarrow a^{+}} f(x)=\lim _{x \rightarrow+\infty} f(x)=A$, \begin{CJK}{UTF8}{mj}证明\end{CJK}: \begin{CJK}{UTF8}{mj}在\end{CJK} $(a,+\infty)$ \begin{CJK}{UTF8}{mj}内至\end{CJK} \begin{CJK}{UTF8}{mj}少存在一点\end{CJK} $\xi$, \begin{CJK}{UTF8}{mj}使得\end{CJK} $f^{\prime}(\xi)=0$.

  \item \begin{CJK}{UTF8}{mj}若\end{CJK} $E$ \begin{CJK}{UTF8}{mj}是非空有上界的实数集\end{CJK}, \begin{CJK}{UTF8}{mj}设\end{CJK} $\sup E=a$, \begin{CJK}{UTF8}{mj}且\end{CJK} $a \neq E$, \begin{CJK}{UTF8}{mj}证明\end{CJK}:\begin{CJK}{UTF8}{mj}存在数列\end{CJK} $\left\{x_{n}\right\} \subseteq E$, \begin{CJK}{UTF8}{mj}使得\end{CJK} $x_{n}<x_{n+1}, n=1,2, \cdots$, \begin{CJK}{UTF8}{mj}且\end{CJK} $\lim _{n \rightarrow \infty} x_{n}=a$.

  \item \begin{CJK}{UTF8}{mj}设\end{CJK} $f(x)=\int_{0}^{+\infty} e^{-t^{2}} \cos 2 x t d t$, \begin{CJK}{UTF8}{mj}证明\end{CJK}:

\end{enumerate}
(1). $f(x)$ \begin{CJK}{UTF8}{mj}满足微分方程\end{CJK} $f^{\prime}(x)+2 x f(x)=0$;

(2). $f(x)=\frac{\sqrt{\pi}}{2} \mathrm{e}^{-x^{2}}$

\begin{enumerate}
  \setcounter{enumi}{6}
  \item \begin{CJK}{UTF8}{mj}设函数\end{CJK}
\end{enumerate}
$$
f(x, y)= \begin{cases}\frac{\sqrt{|x y|}}{x^{2}+y^{2}} \sin \left(x^{2}+y^{2}\right), & x^{2}+y^{2} \neq 0 \\ 0, & x^{2}+y^{2}=0\end{cases}
$$
\begin{CJK}{UTF8}{mj}试问\end{CJK}

(1). \begin{CJK}{UTF8}{mj}函数\end{CJK} $f(x, y)$ \begin{CJK}{UTF8}{mj}在点\end{CJK} $(0,0)$ \begin{CJK}{UTF8}{mj}处是否连续\end{CJK}? \begin{CJK}{UTF8}{mj}为什么\end{CJK}?

(2). \begin{CJK}{UTF8}{mj}函数\end{CJK} $f(x, y)$ \begin{CJK}{UTF8}{mj}在点\end{CJK} $(0,0)$ \begin{CJK}{UTF8}{mj}处是否可微\end{CJK}? \begin{CJK}{UTF8}{mj}为什么\end{CJK}?

\begin{enumerate}
  \setcounter{enumi}{7}
  \item \begin{CJK}{UTF8}{mj}若函数列\end{CJK} $\left\{g_{n}(x)\right\}$ \begin{CJK}{UTF8}{mj}满足下列条件\end{CJK}:
\end{enumerate}
(1). $g_{n}(x)$ \begin{CJK}{UTF8}{mj}在\end{CJK} $[-1,1]$ \begin{CJK}{UTF8}{mj}上非负连续\end{CJK}, \begin{CJK}{UTF8}{mj}且\end{CJK} $\lim _{n \rightarrow+\infty} \int_{-1}^{1} g_{n}(x) \mathrm{d} x=1$.

(2). $\forall c \in(0,1),\left\{g_{n}(x)\right\}$ \begin{CJK}{UTF8}{mj}在\end{CJK} $[-1,-c]$ \begin{CJK}{UTF8}{mj}与\end{CJK} $[c, 1]$ \begin{CJK}{UTF8}{mj}上一致收敛于\end{CJK} 0 .

\begin{CJK}{UTF8}{mj}证明\end{CJK}:\begin{CJK}{UTF8}{mj}对\end{CJK} $[-1,1]$ \begin{CJK}{UTF8}{mj}上的任意连续函数\end{CJK} $f(x)$, \begin{CJK}{UTF8}{mj}有\end{CJK}
$$
\lim _{n \rightarrow+\infty} \int_{-1}^{1} f(x) g_{n}(x) \mathrm{d} x=f(0)
$$

\begin{enumerate}
  \setcounter{enumi}{8}
  \item \begin{CJK}{UTF8}{mj}设\end{CJK} $\left\{a_{n}\right\}$ \begin{CJK}{UTF8}{mj}是正的单调递增数列\end{CJK}, \begin{CJK}{UTF8}{mj}证明\end{CJK}:\begin{CJK}{UTF8}{mj}级数\end{CJK} $\sum_{n=1}^{\infty}\left(1-\frac{a_{n}}{a_{n+1}}\right)$ \begin{CJK}{UTF8}{mj}收敛的充分必要条件是数列\end{CJK} $\left\{a_{n}\right\}$ \begin{CJK}{UTF8}{mj}有界\end{CJK}。

  \item \begin{CJK}{UTF8}{mj}计算封闭曲面\end{CJK} $\left(x^{2}+y^{2}+z^{2}\right)^{2}=a^{3} z(a>0)$ \begin{CJK}{UTF8}{mj}所围立体的体积\end{CJK}.

\end{enumerate}
\section{$15.22020$ 年高等代数考研真题}
\section{考生须知:}
\begin{enumerate}
  \item \begin{CJK}{UTF8}{mj}本试卷满分为\end{CJK} 150 \begin{CJK}{UTF8}{mj}分\end{CJK}, \begin{CJK}{UTF8}{mj}全部考试时间总计\end{CJK} 180 \begin{CJK}{UTF8}{mj}分钟\end{CJK};

  \item \begin{CJK}{UTF8}{mj}所有答案必须写在答题纸上\end{CJK},\begin{CJK}{UTF8}{mj}写在试题纸上或草稿纸上一律无效\end{CJK}。

  \item \begin{CJK}{UTF8}{mj}填空题\end{CJK} $(8 \times 5=40$ \begin{CJK}{UTF8}{mj}分\end{CJK} $)$

\end{enumerate}
(1). \begin{CJK}{UTF8}{mj}若多项式\end{CJK} $f(x)$ \begin{CJK}{UTF8}{mj}除以\end{CJK} $x-2$ \begin{CJK}{UTF8}{mj}的余式为\end{CJK} 3 , \begin{CJK}{UTF8}{mj}除以\end{CJK} $x-3$ \begin{CJK}{UTF8}{mj}的余式为\end{CJK} 4 , \begin{CJK}{UTF8}{mj}则\end{CJK} $f(x)$ \begin{CJK}{UTF8}{mj}除以\end{CJK} $x^{2}-5 x+6$ \begin{CJK}{UTF8}{mj}的余式\end{CJK}

(2). \begin{CJK}{UTF8}{mj}设四阶行列式\end{CJK} $D_{4}$ \begin{CJK}{UTF8}{mj}的第三行元素为\end{CJK} $-1,0,2,3$, \begin{CJK}{UTF8}{mj}第四行元素对应的余子式分别是\end{CJK} $5,10, a, 5$, \begin{CJK}{UTF8}{mj}则\end{CJK} $a=$

(3). \begin{CJK}{UTF8}{mj}设\end{CJK} $\alpha, \beta, \gamma_{1}, \gamma_{2}, \gamma_{3}, \gamma_{4}$ \begin{CJK}{UTF8}{mj}都是四维行向量\end{CJK}, \begin{CJK}{UTF8}{mj}四阶矩阵\end{CJK} $A=\left(\begin{array}{c}\alpha \\ 2 \gamma_{2} \\ \gamma_{3} \\ \gamma_{4}\end{array}\right), \quad B=\left(\begin{array}{c}\beta \\ \gamma_{2} \\ p_{3} \\ 2 \gamma_{4}\end{array}\right)$, \begin{CJK}{UTF8}{mj}且行\end{CJK} \begin{CJK}{UTF8}{mj}列式\end{CJK} $|A|=2,|B|=1$, \begin{CJK}{UTF8}{mj}则行列式\end{CJK} $|2 A-B|=$ (4). \begin{CJK}{UTF8}{mj}若实对称矩阵\end{CJK} $A$ \begin{CJK}{UTF8}{mj}与\end{CJK} $B=\left(\begin{array}{lll}1 & 0 & 0 \\ 0 & 0 & 2 \\ 0 & 2 & 0\end{array}\right)$ \begin{CJK}{UTF8}{mj}矩阵合同\end{CJK}, \begin{CJK}{UTF8}{mj}则二次型\end{CJK} $X^{\prime} A X$ \begin{CJK}{UTF8}{mj}的正惯性指数\end{CJK} \begin{CJK}{UTF8}{mj}是\end{CJK}

$(5)$. \begin{CJK}{UTF8}{mj}当\end{CJK} $t$ \begin{CJK}{UTF8}{mj}满足\end{CJK} $\quad$ \begin{CJK}{UTF8}{mj}时\end{CJK}, \begin{CJK}{UTF8}{mj}二次型\end{CJK} $f\left(x_{1}, x_{2}, x_{3}\right)=x_{1}^{2}+2 x_{2}^{2}+(1-t) x_{3}^{2}+2 t x_{1} x_{2}+2 x_{1} x_{3}$ \begin{CJK}{UTF8}{mj}是正\end{CJK} \begin{CJK}{UTF8}{mj}定白\end{CJK}.

(6). \begin{CJK}{UTF8}{mj}已知\end{CJK} $\alpha_{1}, \alpha_{2}, \alpha_{3}$ \begin{CJK}{UTF8}{mj}与\end{CJK} $\beta_{1}, \beta_{2}, \beta_{3}$ \begin{CJK}{UTF8}{mj}是\end{CJK} 3 \begin{CJK}{UTF8}{mj}维向量空间的两组基\end{CJK},\begin{CJK}{UTF8}{mj}若向量\end{CJK} $\gamma$ \begin{CJK}{UTF8}{mj}在这两组基下的坐标\end{CJK} \begin{CJK}{UTF8}{mj}分别为\end{CJK} $\left(x_{1}, x_{2}, x_{3}\right)^{\prime}$ \begin{CJK}{UTF8}{mj}与\end{CJK} $\left(y_{1}, y_{2}, y_{3}\right)^{\prime}$, \begin{CJK}{UTF8}{mj}且\end{CJK} $y_{1}=x_{1}, y_{2}=x_{1}+x_{2}, y_{3}=x_{1}+x_{2}+x_{3}$, \begin{CJK}{UTF8}{mj}则由基\end{CJK} $\alpha_{1}, \alpha_{2}, \alpha_{3}$ \begin{CJK}{UTF8}{mj}到\end{CJK} $\beta_{1}, \beta_{2}, \beta_{3}$ \begin{CJK}{UTF8}{mj}的过渡矩阵为\end{CJK}

(7). \begin{CJK}{UTF8}{mj}已知\end{CJK} $P$ \begin{CJK}{UTF8}{mj}是数域\end{CJK}, \begin{CJK}{UTF8}{mj}向量空间\end{CJK} $P^{3}$ \begin{CJK}{UTF8}{mj}上的线性变换\end{CJK} $\mathscr{A}$ \begin{CJK}{UTF8}{mj}为\end{CJK}: \begin{CJK}{UTF8}{mj}对\end{CJK} $\forall(a, b, c) \in P^{3}$, \begin{CJK}{UTF8}{mj}有\end{CJK}
$$
\mathscr{A}(a, b, c=(a+2 b-c, b+c, a+b-2 c))
$$
\begin{CJK}{UTF8}{mj}则线性变换\end{CJK} $\mathscr{A}$ \begin{CJK}{UTF8}{mj}的秩为\end{CJK}

(8). \begin{CJK}{UTF8}{mj}在二维实向量空间\end{CJK} $\mathbb{R}^{2}$ \begin{CJK}{UTF8}{mj}中定义内积如下\end{CJK}: \begin{CJK}{UTF8}{mj}对\end{CJK} $\forall x=\left(x_{1}, x_{2}\right) \in \mathbb{R}^{2}, \forall Y=\left(y_{1}, y_{2}\right) \in \mathbb{R}^{2}$, \begin{CJK}{UTF8}{mj}有\end{CJK} $(X, Y)=3 x_{1} y_{1}+x_{2} y_{2}$, \begin{CJK}{UTF8}{mj}求向量\end{CJK} $\alpha=(1,1,), \beta=(0,2$, \begin{CJK}{UTF8}{mj}的大角余弦\end{CJK} $\cos \langle\widehat{\alpha, \beta}\rangle=$

\begin{enumerate}
  \setcounter{enumi}{2}
  \item \begin{CJK}{UTF8}{mj}设\end{CJK}
\end{enumerate}
$$
f(x)=(x-1)(x-2) \cdots(x-(2 n-1))+1
$$
\begin{CJK}{UTF8}{mj}其中\end{CJK} $n$ \begin{CJK}{UTF8}{mj}为大于\end{CJK} 1 \begin{CJK}{UTF8}{mj}的非负整数\end{CJK}, \begin{CJK}{UTF8}{mj}证明\end{CJK}: $f(x)$ \begin{CJK}{UTF8}{mj}在有理数域上不可约\end{CJK}.

\begin{enumerate}
  \setcounter{enumi}{3}
  \item \begin{CJK}{UTF8}{mj}设\end{CJK} $A$ \begin{CJK}{UTF8}{mj}为\end{CJK} $n$ \begin{CJK}{UTF8}{mj}阶方阵\end{CJK}, \begin{CJK}{UTF8}{mj}证明\end{CJK}:
\end{enumerate}
(1). \begin{CJK}{UTF8}{mj}若\end{CJK} $k$ \begin{CJK}{UTF8}{mj}是正整数\end{CJK}, $\alpha$ \begin{CJK}{UTF8}{mj}是\end{CJK} $A^{k+1} X=0$ \begin{CJK}{UTF8}{mj}的解\end{CJK}, $\alpha$ \begin{CJK}{UTF8}{mj}不是\end{CJK} $A^{k+1} X=0$ \begin{CJK}{UTF8}{mj}的解\end{CJK},\begin{CJK}{UTF8}{mj}则\end{CJK} $\alpha, A \alpha, A^{2} \alpha \cdots, A^{k} \alpha$ \begin{CJK}{UTF8}{mj}线性无关\end{CJK}.

(2). \begin{CJK}{UTF8}{mj}当正整数\end{CJK} $k \geq n$ \begin{CJK}{UTF8}{mj}时\end{CJK}, \begin{CJK}{UTF8}{mj}必有\end{CJK} $r\left(A^{k+1}\right)=r\left(A^{k}\right)$

\begin{enumerate}
  \setcounter{enumi}{4}
  \item \begin{CJK}{UTF8}{mj}设矩阵\end{CJK} $A$ \begin{CJK}{UTF8}{mj}的伴随矩阵\end{CJK}
\end{enumerate}
$$
A^{*}=\left(\begin{array}{cccc}
1 & 0 & 0 & 0 \\
0 & 1 & 0 & 0 \\
1 & 0 & 1 & 0 \\
0 & -3 & 0 & 8
\end{array}\right)
$$
\begin{CJK}{UTF8}{mj}且\end{CJK} $A B A^{-1}=B A^{-1}+3 E$

(1). \begin{CJK}{UTF8}{mj}计算行列式\end{CJK} $|A|$;

(2). \begin{CJK}{UTF8}{mj}求矩阵\end{CJK} $B$ 5. \begin{CJK}{UTF8}{mj}已知矩阵\end{CJK}
$$
A=\left(\begin{array}{llll}
1 & 0 & 0 & 0 \\
a & 1 & 0 & 0 \\
a_{1} & b & 2 & 0 \\
a_{2} & b_{1} & c & 2
\end{array}\right)
$$
\includegraphics[max width=\textwidth]{2022_04_18_7db0708508f26638f054g-063}

\begin{enumerate}
  \setcounter{enumi}{6}
  \item \begin{CJK}{UTF8}{mj}已知矩阵\end{CJK}
\end{enumerate}
$$
A=\left(\begin{array}{rrr}
1 & -1 & 2 \\
3 & -3 & 6 \\
2 & -2 & 4
\end{array}\right)
$$
(1). \begin{CJK}{UTF8}{mj}求出\end{CJK} $A$ \begin{CJK}{UTF8}{mj}的特征矩阵的等价标准形\end{CJK};

(2). \begin{CJK}{UTF8}{mj}写出\end{CJK} $A$ \begin{CJK}{UTF8}{mj}的不变因子\end{CJK}, \begin{CJK}{UTF8}{mj}行列式因子\end{CJK}, \begin{CJK}{UTF8}{mj}初等因子\end{CJK};

(3). \begin{CJK}{UTF8}{mj}写出\end{CJK} $A$ \begin{CJK}{UTF8}{mj}的特征多项式和是小多项\end{CJK};

(4). \begin{CJK}{UTF8}{mj}写出\end{CJK} $A$ \begin{CJK}{UTF8}{mj}的有理标准形和若尔当标准形\end{CJK}.

\begin{enumerate}
  \setcounter{enumi}{7}
  \item \begin{CJK}{UTF8}{mj}设\end{CJK} $\mathscr{F}$ \begin{CJK}{UTF8}{mj}是\end{CJK} $n$ \begin{CJK}{UTF8}{mj}维欧几里得空间\end{CJK} $V$ \begin{CJK}{UTF8}{mj}的对称变换\end{CJK},\begin{CJK}{UTF8}{mj}证明\end{CJK}: $\mathscr{F}$ \begin{CJK}{UTF8}{mj}的像子空间\end{CJK} $\operatorname{Im} \mathscr{F}$ \begin{CJK}{UTF8}{mj}是\end{CJK} $\mathscr{F}$ \begin{CJK}{UTF8}{mj}的核子空间\end{CJK} Ker $\mathscr{F}$ \begin{CJK}{UTF8}{mj}的正交补\end{CJK}.
\end{enumerate}
\section{第16章 南京大学}
\section{$16.12020$ 年数学分析考研真题}
\section{考生须知:}
\begin{enumerate}
  \item \begin{CJK}{UTF8}{mj}本试卷满分为\end{CJK} 150 \begin{CJK}{UTF8}{mj}分\end{CJK}, \begin{CJK}{UTF8}{mj}全部考试时间总计\end{CJK} 180 \begin{CJK}{UTF8}{mj}分钟\end{CJK};

  \item \begin{CJK}{UTF8}{mj}所有答案必须写在答题纸上\end{CJK}, \begin{CJK}{UTF8}{mj}写在试题纸上或草稿纸上一律无效\end{CJK}。

  \item \begin{CJK}{UTF8}{mj}解答题\end{CJK} $(15 \times 7=105$ \begin{CJK}{UTF8}{mj}分\end{CJK} $)$

\end{enumerate}
(1). \begin{CJK}{UTF8}{mj}计算极限\end{CJK}
$$
\lim _{x \rightarrow 0} \frac{\ln (1+x)-\ln (1+\sin x)}{x^{3}}
$$
(2). \begin{CJK}{UTF8}{mj}已知函数\end{CJK} $f(x)$ \begin{CJK}{UTF8}{mj}满足\end{CJK} $\left|f^{\prime}(x)\right|<\frac{1}{2}$, \begin{CJK}{UTF8}{mj}证明\end{CJK}: \begin{CJK}{UTF8}{mj}存在唯一的\end{CJK} $x$, \begin{CJK}{UTF8}{mj}使得\end{CJK} $f(x)=x$.

(3). \begin{CJK}{UTF8}{mj}举例\end{CJK}: $f(x)$ \begin{CJK}{UTF8}{mj}在\end{CJK} $[0,1]$ \begin{CJK}{UTF8}{mj}上可积\end{CJK}, \begin{CJK}{UTF8}{mj}但\end{CJK} $F(x)=\int_{0}^{x} f(x) \mathrm{d} x$ \begin{CJK}{UTF8}{mj}在\end{CJK} $[0,1]$ \begin{CJK}{UTF8}{mj}上不可导\end{CJK}; \begin{CJK}{UTF8}{mj}若增加条件\end{CJK} $f(x)$ \begin{CJK}{UTF8}{mj}单调\end{CJK}, \begin{CJK}{UTF8}{mj}证明\end{CJK} $F(x)$ \begin{CJK}{UTF8}{mj}存在单侧导数\end{CJK} (\begin{CJK}{UTF8}{mj}不需要举例\end{CJK} $)$.

(4). \begin{CJK}{UTF8}{mj}已知函数\end{CJK} $f(x)$ \begin{CJK}{UTF8}{mj}一阶连续可微\end{CJK}, \begin{CJK}{UTF8}{mj}且是周期为\end{CJK} 1 \begin{CJK}{UTF8}{mj}的周期函数\end{CJK}, \begin{CJK}{UTF8}{mj}设\end{CJK} $a_{n}=\int_{0}^{1} f(x) \cos 2 \pi n x \mathrm{~d} x$, \begin{CJK}{UTF8}{mj}证明\end{CJK}: $\sum_{n=1}^{\infty}\left|a_{n}\right|$ \begin{CJK}{UTF8}{mj}收敛\end{CJK}.

(5). \begin{CJK}{UTF8}{mj}已知幂级数\end{CJK} $\sum_{n=1}^{\infty} \frac{(2 n-1) ! !}{(2 n) ! !} x^{n}$, \begin{CJK}{UTF8}{mj}求收敛半径\end{CJK}, \begin{CJK}{UTF8}{mj}并判断端点处是否收敛\end{CJK}。

(6). \begin{CJK}{UTF8}{mj}计算第二型曲线积分\end{CJK}
$$
\int_{L} \frac{x \mathrm{~d} y-(y-1) \mathrm{d} x}{x^{2}+(y-1)^{2}}
$$
\begin{CJK}{UTF8}{mj}其中\end{CJK} $L$ \begin{CJK}{UTF8}{mj}为\end{CJK} $x^{2}+\frac{y^{2}}{4}=1$.

(7). \begin{CJK}{UTF8}{mj}已知\end{CJK}
$$
f_{n}(x)= \begin{cases}x^{n} \sqrt{\ln \frac{1}{x},} & 0<x \leqslant 1 \\ 0, & x=0\end{cases}
$$
\begin{CJK}{UTF8}{mj}证明\end{CJK}: $\sum_{n=1}^{\infty} f_{n}(x)$ \begin{CJK}{UTF8}{mj}一致收敛\end{CJK}. 2. (15 \begin{CJK}{UTF8}{mj}分\end{CJK}) \begin{CJK}{UTF8}{mj}计算第一型曲面积分\end{CJK}
$$
\iint x_{x^{2}+y^{2}+z^{2}=1} x \cos (x+y+z) \mathrm{d} \sigma
$$

\begin{enumerate}
  \setcounter{enumi}{3}
  \item (15 \begin{CJK}{UTF8}{mj}分\end{CJK}) \begin{CJK}{UTF8}{mj}设\end{CJK} $r>0, B_{r}$ \begin{CJK}{UTF8}{mj}是以\end{CJK} $r$ \begin{CJK}{UTF8}{mj}为半径\end{CJK}, \begin{CJK}{UTF8}{mj}原点为圆心的\end{CJK} $n$ \begin{CJK}{UTF8}{mj}维球域\end{CJK}, $F$ \begin{CJK}{UTF8}{mj}是\end{CJK} $B_{r}(0)$ \begin{CJK}{UTF8}{mj}上\end{CJK} $C^{\prime}$ \begin{CJK}{UTF8}{mj}函数\end{CJK}, \begin{CJK}{UTF8}{mj}且\end{CJK} $F(\overrightarrow{0})=0, J F(\overrightarrow{0})=E$. \begin{CJK}{UTF8}{mj}证明\end{CJK}:\begin{CJK}{UTF8}{mj}对\end{CJK} $\forall \varepsilon>0, \exists \delta_{0}>0, \forall 0<\delta<\delta_{0}$, \begin{CJK}{UTF8}{mj}有\end{CJK}
\end{enumerate}
$$
B_{(1-\varepsilon) \delta}(0)<F\left(B_{\delta}(0)\right) \subset B_{(1+\varepsilon) \delta}(0)
$$

\begin{enumerate}
  \setcounter{enumi}{4}
  \item (15 \begin{CJK}{UTF8}{mj}分\end{CJK}) \begin{CJK}{UTF8}{mj}已知\end{CJK} $u(\vec{x})$ \begin{CJK}{UTF8}{mj}是\end{CJK} $B_{1}(0)$ \begin{CJK}{UTF8}{mj}连续\end{CJK}, \begin{CJK}{UTF8}{mj}在\end{CJK} $B_{1}(0)$ \begin{CJK}{UTF8}{mj}有连续偏导\end{CJK}, $B_{1}(0)$ \begin{CJK}{UTF8}{mj}是\end{CJK} $n$ \begin{CJK}{UTF8}{mj}维单位圆球\end{CJK}, \begin{CJK}{UTF8}{mj}且\end{CJK} $\left.u\right|_{\partial B}=0, u(\overrightarrow{0})=-l<0$, \begin{CJK}{UTF8}{mj}证明\end{CJK}: $B_{L}(0) \subset \nabla u\left(B_{1}(0)\right)$, \begin{CJK}{UTF8}{mj}其中\end{CJK} $\nabla u$ \begin{CJK}{UTF8}{mj}为梯度\end{CJK}: $B_{1}(0) \rightarrow \mathbb{R}^{n}$.

  \item (15 \begin{CJK}{UTF8}{mj}分\end{CJK}) \begin{CJK}{UTF8}{mj}已知函数\end{CJK} $f(x)$ \begin{CJK}{UTF8}{mj}是\end{CJK} $(a, b)$ \begin{CJK}{UTF8}{mj}上的解析函数\end{CJK}, \begin{CJK}{UTF8}{mj}即存在一点\end{CJK} $x_{0} \in(a, b)$, \begin{CJK}{UTF8}{mj}使得\end{CJK} $f(x)$ \begin{CJK}{UTF8}{mj}敛于其\end{CJK} $x_{0}$ \begin{CJK}{UTF8}{mj}处的泰勒级数\end{CJK}. \begin{CJK}{UTF8}{mj}若存在一点列\end{CJK} $x_{n} \in(a, b)$, \begin{CJK}{UTF8}{mj}使得\end{CJK} $\lim _{n \rightarrow+\infty} x_{n}=x_{0}$, \begin{CJK}{UTF8}{mj}且对\end{CJK} $\forall n \in \mathbb{N}^{+}$, \begin{CJK}{UTF8}{mj}有\end{CJK} $f\left(x_{n}\right)=0$, \begin{CJK}{UTF8}{mj}证明\end{CJK}: $f(x) \equiv 0$

\end{enumerate}
\section{$16.22020$ 年高等代数考研真题}
\section{考生须知:}
\begin{enumerate}
  \item \begin{CJK}{UTF8}{mj}本试卷满分为\end{CJK} 150 \begin{CJK}{UTF8}{mj}分\end{CJK}, \begin{CJK}{UTF8}{mj}全部考试时间总计\end{CJK} 180 \begin{CJK}{UTF8}{mj}分钟\end{CJK};

  \item \begin{CJK}{UTF8}{mj}所有答案必须写在答题纸上\end{CJK},\begin{CJK}{UTF8}{mj}写在试题纸上或草稿纸上一律无效\end{CJK}。

  \item \begin{CJK}{UTF8}{mj}给出一个具体的\end{CJK} $4 \times 4$ \begin{CJK}{UTF8}{mj}的矩阵\end{CJK}, \begin{CJK}{UTF8}{mj}求\end{CJK} $r(A)$ \begin{CJK}{UTF8}{mj}以及\end{CJK} $A^{*}$.

  \item \begin{CJK}{UTF8}{mj}已知\end{CJK} $\alpha=\sqrt{2}+\sqrt{-5}$

\end{enumerate}
(1). \begin{CJK}{UTF8}{mj}求包含\end{CJK} $\alpha$ \begin{CJK}{UTF8}{mj}的最小数域\end{CJK} $F$;

(2). \begin{CJK}{UTF8}{mj}证明\end{CJK}:\begin{CJK}{UTF8}{mj}上问所求的\end{CJK} $\mathbb{F}$ \begin{CJK}{UTF8}{mj}为数域\end{CJK} $\mathbb{Q}$ \begin{CJK}{UTF8}{mj}上的线性空间\end{CJK}, \begin{CJK}{UTF8}{mj}并求一组基\end{CJK}.

\begin{enumerate}
  \setcounter{enumi}{3}
  \item \begin{CJK}{UTF8}{mj}已知多项式\end{CJK} $f(x)=x^{7}+7 x^{2}+2$
\end{enumerate}
(1). \begin{CJK}{UTF8}{mj}证明\end{CJK}: $f(x)$ \begin{CJK}{UTF8}{mj}在\end{CJK} $\mathbb{Q}$ \begin{CJK}{UTF8}{mj}上不可约\end{CJK};

(2). \begin{CJK}{UTF8}{mj}证明\end{CJK}: $f(x)$ \begin{CJK}{UTF8}{mj}王少存在一个实根\end{CJK};

(3). \begin{CJK}{UTF8}{mj}对\end{CJK} $\beta \in \mathbb{Q}$, \begin{CJK}{UTF8}{mj}存在\end{CJK} $u(x) \in \mathbb{Q}[x]$, \begin{CJK}{UTF8}{mj}使得\end{CJK} $u(\alpha)=\frac{1}{\alpha-\beta}$

\begin{enumerate}
  \setcounter{enumi}{4}
  \item \begin{CJK}{UTF8}{mj}已知复对称矩阵\end{CJK} $A$ \begin{CJK}{UTF8}{mj}的二次型\end{CJK} $f\left(x_{1}, x_{2}, \cdots, x_{n}\right)=X^{\prime} A X$ \begin{CJK}{UTF8}{mj}可约\end{CJK}, \begin{CJK}{UTF8}{mj}即存在两个非常数多项式\end{CJK}, \begin{CJK}{UTF8}{mj}使\end{CJK} \begin{CJK}{UTF8}{mj}得\end{CJK}
\end{enumerate}
$$
f\left(x_{1}, x_{2}, \cdots, x_{n}\right)=f_{1}\left(x_{1}, x_{2}, \cdots, x_{n}\right) f_{2}\left(x_{1}, x_{2}, \cdots, x_{n}\right)
$$
\begin{CJK}{UTF8}{mj}证明\end{CJK}: $|A| \equiv 0$. 5. \begin{CJK}{UTF8}{mj}已知\end{CJK} $A$ \begin{CJK}{UTF8}{mj}是\end{CJK} $m \times n$ \begin{CJK}{UTF8}{mj}的矩阵\end{CJK}, \begin{CJK}{UTF8}{mj}且\end{CJK} $r(A)=r$, \begin{CJK}{UTF8}{mj}并且定义\end{CJK}
$$
M(A)=\left\{\beta \in P^{n \times n} \mid A X=\beta \text { 有解 }\right\}
$$
(1). \begin{CJK}{UTF8}{mj}证明\end{CJK}: $M(A)$ \begin{CJK}{UTF8}{mj}是\end{CJK} $P^{n}$ \begin{CJK}{UTF8}{mj}的子空间\end{CJK}, \begin{CJK}{UTF8}{mj}并求其维数\end{CJK};

(2). \begin{CJK}{UTF8}{mj}给出一个具体的\end{CJK} $4 \times 4$ \begin{CJK}{UTF8}{mj}的矩阵\end{CJK} $A$, \begin{CJK}{UTF8}{mj}求\end{CJK} $M(A)$ \begin{CJK}{UTF8}{mj}的一组基\end{CJK}.

\begin{enumerate}
  \setcounter{enumi}{6}
  \item \begin{CJK}{UTF8}{mj}给出一个具体的秩为\end{CJK} 1 \begin{CJK}{UTF8}{mj}的矩阵\end{CJK}, \begin{CJK}{UTF8}{mj}求若尔当标准形和最小多项式\end{CJK}.

  \item \begin{CJK}{UTF8}{mj}已知\end{CJK} $A=I_{n}$, \begin{CJK}{UTF8}{mj}且\end{CJK} $r\left(I_{n}-A\right)=r$, \begin{CJK}{UTF8}{mj}对于\end{CJK} $n \times n$ \begin{CJK}{UTF8}{mj}阶矩阵\end{CJK} $X$ \begin{CJK}{UTF8}{mj}定义线性变换\end{CJK}

\end{enumerate}
$$
\sigma(A)=A X
$$
\begin{CJK}{UTF8}{mj}求\end{CJK} $\sigma$ \begin{CJK}{UTF8}{mj}的特征多项式以及所有特征子空间的维数\end{CJK}.

\begin{enumerate}
  \setcounter{enumi}{8}
  \item \begin{CJK}{UTF8}{mj}设\end{CJK} $A$ \begin{CJK}{UTF8}{mj}是\end{CJK} $n$ \begin{CJK}{UTF8}{mj}级正交矩阵且\end{CJK} $|A|=1$,
\end{enumerate}
$$
f(x)=a_{0} x^{n}+a_{1} x^{n-1}+\cdots+a_{n-1} x+a_{n}
$$
\begin{CJK}{UTF8}{mj}是\end{CJK} $A$ \begin{CJK}{UTF8}{mj}的特征多项式\end{CJK}, \begin{CJK}{UTF8}{mj}证明\end{CJK}:

(1). \begin{CJK}{UTF8}{mj}当\end{CJK} $n$ \begin{CJK}{UTF8}{mj}为偶数\end{CJK}, \begin{CJK}{UTF8}{mj}对任意的\end{CJK} $0 \leqslant i \leq n$, \begin{CJK}{UTF8}{mj}有\end{CJK} $a_{i}=a_{n-i}$;

(2). \begin{CJK}{UTF8}{mj}当\end{CJK} $n$ \begin{CJK}{UTF8}{mj}为奇数\end{CJK}, \begin{CJK}{UTF8}{mj}对任意的\end{CJK} $0 \leqslant i \leqslant n$, \begin{CJK}{UTF8}{mj}有\end{CJK} $a_{i}=-a_{n-i}$

(3). \begin{CJK}{UTF8}{mj}当\end{CJK} $n=2$ \begin{CJK}{UTF8}{mj}时\end{CJK}, \begin{CJK}{UTF8}{mj}存在正交矩阵\end{CJK} $B$ \begin{CJK}{UTF8}{mj}使得\end{CJK} $A=B^{2}$.

\begin{enumerate}
  \setcounter{enumi}{9}
  \item \begin{CJK}{UTF8}{mj}已知\end{CJK} $A, B, C$ \begin{CJK}{UTF8}{mj}是正定矩阵\end{CJK}, \begin{CJK}{UTF8}{mj}且\end{CJK} $A B C$ \begin{CJK}{UTF8}{mj}是对称矩阵\end{CJK}, \begin{CJK}{UTF8}{mj}证明\end{CJK}: $A B C$ \begin{CJK}{UTF8}{mj}是正定矩阵\end{CJK}.
\end{enumerate}
\section{第17章 中山大学}
\section{$17.12020$ 年数学分析考研真题}
\section{考生须知:}
\begin{enumerate}
  \item \begin{CJK}{UTF8}{mj}本试卷满分为\end{CJK} 150 \begin{CJK}{UTF8}{mj}分\end{CJK}, \begin{CJK}{UTF8}{mj}全部考试时间总计\end{CJK} 180 \begin{CJK}{UTF8}{mj}分钟\end{CJK};

  \item \begin{CJK}{UTF8}{mj}所有答案必须写在答题纸上\end{CJK}, \begin{CJK}{UTF8}{mj}写在试题纸上或草稿纸上一律无效\end{CJK}。

  \item \begin{CJK}{UTF8}{mj}求不定积分\end{CJK} $\int x^{2} \arctan x \mathrm{~d} x$.

  \item \begin{CJK}{UTF8}{mj}已知\end{CJK} $\lim _{x \rightarrow 0} f(x)=0, \lim _{x \rightarrow 0} g(x)=0$, \begin{CJK}{UTF8}{mj}试证\end{CJK}:

\end{enumerate}
$$
\lim _{x \rightarrow 0} \frac{(1+f(x))^{\frac{1}{f(x)}}-(1+g(x))^{\frac{1}{g(x)}}}{f(x)-g(x)}=-\frac{\mathrm{e}}{2}
$$

\begin{enumerate}
  \setcounter{enumi}{3}
  \item \begin{CJK}{UTF8}{mj}求幂级数\end{CJK} $\sum_{n=1}^{\infty}\left(\frac{a^{n}}{n}+\frac{b^{n}}{n^{2}}\right) x^{n}(a>0, b>0)$ \begin{CJK}{UTF8}{mj}的收敛区间\end{CJK}.

  \item \begin{CJK}{UTF8}{mj}若\end{CJK} $f(x)$ \begin{CJK}{UTF8}{mj}在区间\end{CJK} $[a, b]$ \begin{CJK}{UTF8}{mj}上可导\end{CJK}, $f^{\prime}(x)$ \begin{CJK}{UTF8}{mj}在\end{CJK} $[a, b]$ \begin{CJK}{UTF8}{mj}上连续\end{CJK}, \begin{CJK}{UTF8}{mj}证明\end{CJK}:

\end{enumerate}
$$
f_{n}(x)=n\left[f\left(x+\frac{1}{n}\right)-f(x)\right]
$$
\begin{CJK}{UTF8}{mj}在\end{CJK} $[a, b]$ \begin{CJK}{UTF8}{mj}上一致收敛\end{CJK}.

\begin{enumerate}
  \setcounter{enumi}{5}
  \item \begin{CJK}{UTF8}{mj}已知\end{CJK} $z=z(x, y)$, \begin{CJK}{UTF8}{mj}且\end{CJK} $x^{2}+y^{2}+z^{2}+x z+y z+x y=6$, \begin{CJK}{UTF8}{mj}求\end{CJK} $z$ \begin{CJK}{UTF8}{mj}的极值\end{CJK}。

  \item \begin{CJK}{UTF8}{mj}证明曲线的法向量为\end{CJK}: $\pm\left(F_{x_{0}}^{\prime}\left(x_{0}, y_{0}, z_{0}\right), F_{y_{0}}^{\prime}\left(x_{0}, y_{0}, z_{0}\right), F_{z_{0}}^{\prime}\left(x_{0}, y_{0}, z_{0}\right)\right)$

  \item \begin{CJK}{UTF8}{mj}若\end{CJK}

\end{enumerate}
$$
f(x, y)= \begin{cases}\left(x^{2}+y^{2}\right)^{a} \sin \frac{1}{\sqrt{x^{2}+y^{2}}} & (x, y) \neq(0,0) \\ 0 & (x, y)=(0,0)\end{cases}
$$
(1). \begin{CJK}{UTF8}{mj}当\end{CJK} $a$ \begin{CJK}{UTF8}{mj}取何值时\end{CJK}, $f(x, y)$ \begin{CJK}{UTF8}{mj}在\end{CJK} $(0,0)$ \begin{CJK}{UTF8}{mj}处可微\end{CJK}.

(2). \begin{CJK}{UTF8}{mj}当\end{CJK} $a$ \begin{CJK}{UTF8}{mj}取何值时\end{CJK}, $\frac{\partial f}{\partial x}$ \begin{CJK}{UTF8}{mj}和\end{CJK} $\frac{\partial f}{\partial y}$ \begin{CJK}{UTF8}{mj}在\end{CJK} $(0,0)$ \begin{CJK}{UTF8}{mj}处连续\end{CJK}.

\begin{enumerate}
  \setcounter{enumi}{8}
  \item \begin{CJK}{UTF8}{mj}一个补平面然后高斯公式的计算题\end{CJK} (\begin{CJK}{UTF8}{mj}补的那个平面积分为\end{CJK} 0 ) 9. \begin{CJK}{UTF8}{mj}计算\end{CJK}
\end{enumerate}
$$
\int_{L} y\left(\frac{y}{2 x^{2}}+1\right) \mathrm{d} x-\left(\frac{y}{x}+x\right) \mathrm{d} y
$$
\begin{CJK}{UTF8}{mj}其中\end{CJK} $L$ \begin{CJK}{UTF8}{mj}为从点\end{CJK} $A(1,1)$ \begin{CJK}{UTF8}{mj}沿\end{CJK} $(x-2)^{2}+(y-2)^{2}=1$ \begin{CJK}{UTF8}{mj}逆时针到点\end{CJK} $B(3,3)$.

\begin{enumerate}
  \setcounter{enumi}{10}
  \item \begin{CJK}{UTF8}{mj}若\end{CJK} $\Sigma$ \begin{CJK}{UTF8}{mj}为曲面\end{CJK} $z+1=\sqrt{x^{2}+y^{2}}$ \begin{CJK}{UTF8}{mj}与\end{CJK} $z=0$ \begin{CJK}{UTF8}{mj}和\end{CJK} $z=1$ \begin{CJK}{UTF8}{mj}所围\end{CJK}, \begin{CJK}{UTF8}{mj}求该曲面的表面积\end{CJK}.

  \item \begin{CJK}{UTF8}{mj}求\end{CJK} $\lim _{n \rightarrow+\infty} \sum_{k=1}^{n} \frac{\sin \frac{k \pi}{n}}{\sqrt{n^{2}+k}}$ \begin{CJK}{UTF8}{mj}的极限\end{CJK}.

  \item \begin{CJK}{UTF8}{mj}证明\end{CJK}: $\tan x \sin ^{2} x>x^{3}, x \in\left(0, \frac{\pi}{2}\right)$.

  \item \begin{CJK}{UTF8}{mj}判断级数\end{CJK} $\sum_{n=1}^{\infty} \sqrt[n]{\frac{1}{n}} \sin n \sin \frac{1}{n}$ \begin{CJK}{UTF8}{mj}的玫散性\end{CJK}.

\end{enumerate}
\section{$17.22020$ 年高等代数考研真题}
\section{考生须知:}
\begin{enumerate}
  \item \begin{CJK}{UTF8}{mj}本试卷满分为\end{CJK} 150 \begin{CJK}{UTF8}{mj}分\end{CJK}, \begin{CJK}{UTF8}{mj}全部考试时间总计\end{CJK} 180 \begin{CJK}{UTF8}{mj}分钟\end{CJK};

  \item \begin{CJK}{UTF8}{mj}所有答案必须写在答题纸上\end{CJK}, \begin{CJK}{UTF8}{mj}写在试题纸上或草稿纸上一律无效\end{CJK}。

  \item \begin{CJK}{UTF8}{mj}已知矩阵\end{CJK} $A=\left(\begin{array}{lll}1 & 0 & 1 \\ 0 & 2 & 0 \\ 1 & 0 & 1\end{array}\right)$

\end{enumerate}
(1). \begin{CJK}{UTF8}{mj}求所有与\end{CJK} $A$ \begin{CJK}{UTF8}{mj}可交换的矩阵\end{CJK};

(2). \begin{CJK}{UTF8}{mj}若\end{CJK} $A B+E=A^{2}+B$, \begin{CJK}{UTF8}{mj}求\end{CJK} $B$.

\begin{enumerate}
  \setcounter{enumi}{2}
  \item \begin{CJK}{UTF8}{mj}已知\end{CJK} 2 \begin{CJK}{UTF8}{mj}阶矩阵\end{CJK} $A$ \begin{CJK}{UTF8}{mj}的特征值为\end{CJK} 1,2 , \begin{CJK}{UTF8}{mj}求\end{CJK} $\left|A^{*}-2 A+E\right|$.

  \item \begin{CJK}{UTF8}{mj}已知\end{CJK} 5 \begin{CJK}{UTF8}{mj}阶矩阵\end{CJK}

\end{enumerate}
$$
A=\left(\begin{array}{lllll}
0 & 1 & 1 & 1 & 1 \\
1 & 0 & 1 & 1 & 1 \\
1 & 1 & 0 & 1 & 1 \\
1 & 1 & 1 & 0 & 1 \\
1 & 1 & 1 & 1 & 0
\end{array}\right)
$$
\begin{CJK}{UTF8}{mj}求正交矩阵\end{CJK} $P$, \begin{CJK}{UTF8}{mj}使得\end{CJK} $P^{-1} A P$ \begin{CJK}{UTF8}{mj}为对角阵\end{CJK}. 4. \begin{CJK}{UTF8}{mj}已知\end{CJK} 5 \begin{CJK}{UTF8}{mj}阶矩阵\end{CJK}
$$
J=\left(\begin{array}{lllll}
0 & 1 & 0 & 0 & 0 \\
0 & 0 & 1 & 0 & 0 \\
0 & 0 & 0 & 1 & 0 \\
0 & 0 & 0 & 0 & 1 \\
0 & 0 & 0 & 0 & 0
\end{array}\right)
$$
\begin{CJK}{UTF8}{mj}求\end{CJK} $J^{2}$ \begin{CJK}{UTF8}{mj}的若尔当标准型\end{CJK}.

\begin{enumerate}
  \setcounter{enumi}{5}
  \item \begin{CJK}{UTF8}{mj}已知\end{CJK} $A_{1}, A_{2}, \cdots, A_{s}(s \geq 2)$ \begin{CJK}{UTF8}{mj}为数域\end{CJK} $\mathbb{K}$ \begin{CJK}{UTF8}{mj}上的\end{CJK} $n$ \begin{CJK}{UTF8}{mj}阶矩阵\end{CJK}, \begin{CJK}{UTF8}{mj}且\end{CJK} $A_{1} A_{2} \cdots A_{s}=O$, \begin{CJK}{UTF8}{mj}证明\end{CJK}:

  \item \begin{CJK}{UTF8}{mj}已知\end{CJK} $C$ \begin{CJK}{UTF8}{mj}为\end{CJK} $Q$ \begin{CJK}{UTF8}{mj}上的线性空间\end{CJK}, $f(x)$ \begin{CJK}{UTF8}{mj}是\end{CJK} $Q[x]$ \begin{CJK}{UTF8}{mj}中的一个\end{CJK} $n$ \begin{CJK}{UTF8}{mj}次不可约因式\end{CJK}, $\alpha \in \mathbb{C}$ \begin{CJK}{UTF8}{mj}是\end{CJK} $f(x)$ \begin{CJK}{UTF8}{mj}的一\end{CJK} \begin{CJK}{UTF8}{mj}个根\end{CJK}, \begin{CJK}{UTF8}{mj}且\end{CJK}.

\end{enumerate}
$$
Q[x]=\left\{a_{0}+a_{1} \alpha+a_{2} \alpha^{2}+\cdots+a_{n-1} \alpha^{n-1} \mid \alpha_{i} \in Q, i=1,2, \cdots, n-1\right\}
$$
(1). \begin{CJK}{UTF8}{mj}证明\end{CJK}: $Q[x]$ \begin{CJK}{UTF8}{mj}是\end{CJK} $C$ \begin{CJK}{UTF8}{mj}的一个有限维子空间\end{CJK},\begin{CJK}{UTF8}{mj}求\end{CJK} $Q[x]$ \begin{CJK}{UTF8}{mj}的一组基\end{CJK};

$(2)$. \begin{CJK}{UTF8}{mj}设\end{CJK} $\beta \in Q[x], \beta \neq 0$, \begin{CJK}{UTF8}{mj}证明\end{CJK}:\begin{CJK}{UTF8}{mj}存在\end{CJK} $r \in Q[x]$, \begin{CJK}{UTF8}{mj}使得\end{CJK} $\beta_{r}=1$.

\begin{enumerate}
  \setcounter{enumi}{7}
  \item \begin{CJK}{UTF8}{mj}设\end{CJK} $x_{1}, x_{2}, \cdots, x_{n}$ \begin{CJK}{UTF8}{mj}是互不相同的数\end{CJK}, $V \in R_{n}[x]$, \begin{CJK}{UTF8}{mj}定义\end{CJK} $V$ \begin{CJK}{UTF8}{mj}到\end{CJK} $R$ \begin{CJK}{UTF8}{mj}上的映射\end{CJK} $L_{i}$ \begin{CJK}{UTF8}{mj}为\end{CJK} $L_{i}(f(x))=f\left(x_{i}\right)$, \begin{CJK}{UTF8}{mj}且对\end{CJK} $\forall f(x) \in \mathbb{R}_{n}(x), i=1,2, \cdots, n$, \begin{CJK}{UTF8}{mj}试证\end{CJK}:
\end{enumerate}
(1). $L_{1}, L_{2}, \cdots, L_{n}$ \begin{CJK}{UTF8}{mj}为\end{CJK} $V$ \begin{CJK}{UTF8}{mj}的对偶空间\end{CJK} $V^{*}$ \begin{CJK}{UTF8}{mj}的一组基\end{CJK};

(2). \begin{CJK}{UTF8}{mj}存在\end{CJK} $\lambda_{1}, \lambda_{2}, \cdots, \lambda_{n}$, \begin{CJK}{UTF8}{mj}使得对\end{CJK} $\forall f(x) \in V$ \begin{CJK}{UTF8}{mj}有\end{CJK}
$$
\int_{0}^{1} f(x) \mathrm{d} x=\sum_{i=1}^{m} \lambda_{i} f(x)
$$

\begin{enumerate}
  \setcounter{enumi}{8}
  \item \begin{CJK}{UTF8}{mj}已知\end{CJK} $\sigma$ \begin{CJK}{UTF8}{mj}为\end{CJK} $n$ \begin{CJK}{UTF8}{mj}维欧氏空间\end{CJK} $V$ \begin{CJK}{UTF8}{mj}上的可逆线性变换\end{CJK}, \begin{CJK}{UTF8}{mj}并且具有性质\end{CJK} $\alpha$ \begin{CJK}{UTF8}{mj}与\end{CJK} $\beta$ \begin{CJK}{UTF8}{mj}正交\end{CJK}, \begin{CJK}{UTF8}{mj}则\end{CJK} $\sigma(\alpha), \sigma(\beta)$ \begin{CJK}{UTF8}{mj}正\end{CJK} \begin{CJK}{UTF8}{mj}交\end{CJK}, \begin{CJK}{UTF8}{mj}证明\end{CJK}: \begin{CJK}{UTF8}{mj}存在\end{CJK} $K \in \mathbb{R}$, \begin{CJK}{UTF8}{mj}使得\end{CJK} $k \sigma$ \begin{CJK}{UTF8}{mj}为正交变换\end{CJK}.

  \item \begin{CJK}{UTF8}{mj}定义\end{CJK} $V=M_{2}(\mathbb{R})$ \begin{CJK}{UTF8}{mj}的二元函数如下\end{CJK}: $\forall A, B \in M_{2}(\mathbb{R}), f(A, B)=|A+B|-|A|-|B|$

\end{enumerate}
(1). \begin{CJK}{UTF8}{mj}证明\end{CJK}: $f$ \begin{CJK}{UTF8}{mj}是\end{CJK} $V$ \begin{CJK}{UTF8}{mj}上的对称双线性函数\end{CJK};

(2). \begin{CJK}{UTF8}{mj}求\end{CJK} $f$ \begin{CJK}{UTF8}{mj}在基\end{CJK} $E_{11}=\left(\begin{array}{ll}1 & 0 \\ 0 & 0\end{array}\right), E_{12}=\left(\begin{array}{ll}0 & 1 \\ 0 & 0\end{array}\right), E_{21}=\left(\begin{array}{ll}0 & 0 \\ 1 & 0\end{array}\right), E_{22}=\left(\begin{array}{ll}0 & 0 \\ 0 & 1\end{array}\right)$ \begin{CJK}{UTF8}{mj}下的\end{CJK} \begin{CJK}{UTF8}{mj}矩阵以及\end{CJK} $f$ \begin{CJK}{UTF8}{mj}的符号差\end{CJK}.

\section{第18章 大连理工大学}
\section{$18.12020$ 年数学分析考研真}
\section{考生须知:}
\begin{enumerate}
  \item \begin{CJK}{UTF8}{mj}本试卷满分为\end{CJK} 150 \begin{CJK}{UTF8}{mj}分\end{CJK}, \begin{CJK}{UTF8}{mj}全部考试时间总计\end{CJK} 180 \begin{CJK}{UTF8}{mj}分钟\end{CJK};

  \item \begin{CJK}{UTF8}{mj}所有答案必须写在答题纸上\end{CJK}, \begin{CJK}{UTF8}{mj}写在试题纸上或草稿纸上一律无效\end{CJK}。

\end{enumerate}
\section{一、证明题 $(6 \times 10=\mathbf{6 0}$ 分 $)$}
\begin{enumerate}
  \item \begin{CJK}{UTF8}{mj}若\end{CJK} $h(x)$ \begin{CJK}{UTF8}{mj}在\end{CJK} $[0,1]$ \begin{CJK}{UTF8}{mj}上连续\end{CJK}, \begin{CJK}{UTF8}{mj}且\end{CJK} $h(0)=0$, \begin{CJK}{UTF8}{mj}求证\end{CJK}:
\end{enumerate}
$$
g_{n}(x)=\int_{0}^{x} h\left(s^{n}\right) \mathrm{d} s
$$
\begin{CJK}{UTF8}{mj}在\end{CJK} $[0,1]$ \begin{CJK}{UTF8}{mj}上一致收敛\end{CJK}.

\begin{enumerate}
  \setcounter{enumi}{2}
  \item \begin{CJK}{UTF8}{mj}已知\end{CJK} $g \in C[0,+\infty) \lim _{x \rightarrow+\infty} g(x)=c \in \mathbb{R}$, \begin{CJK}{UTF8}{mj}试证\end{CJK}:
\end{enumerate}
$$
\lim _{n \rightarrow \infty} n \int_{0}^{+\infty} \frac{g(x)}{n^{2}+x^{2}} \mathrm{~d} x=\frac{\pi}{2} c
$$

\begin{enumerate}
  \setcounter{enumi}{3}
  \item \begin{CJK}{UTF8}{mj}设\end{CJK} $r>1,\left\{a_{n}\right\}$ \begin{CJK}{UTF8}{mj}是正项数列\end{CJK}, \begin{CJK}{UTF8}{mj}满足\end{CJK} $n \ln \frac{a_{n}}{a_{n+1}}>r$, \begin{CJK}{UTF8}{mj}试证\end{CJK}: $\sum_{n=1}^{\infty} a_{n}$ \begin{CJK}{UTF8}{mj}收敛\end{CJK}.

  \item \begin{CJK}{UTF8}{mj}已知\end{CJK}

\end{enumerate}
$$
\left\{\begin{array}{l}
\mathrm{e}^{x}+f(x, y)=u^{2} \\
\mathrm{e}^{y}-f(x, y)=v^{2}
\end{array}\right.
$$
\begin{CJK}{UTF8}{mj}且\end{CJK}. $f(0,0)=0$, \begin{CJK}{UTF8}{mj}试证\end{CJK}: $(1,1)$ \begin{CJK}{UTF8}{mj}领域\end{CJK} $(u, v)$ \begin{CJK}{UTF8}{mj}唯一确定的\end{CJK} $(x, y)$.

\begin{enumerate}
  \setcounter{enumi}{5}
  \item \begin{CJK}{UTF8}{mj}设\end{CJK} $f(x)$ \begin{CJK}{UTF8}{mj}连续可微\end{CJK}, $J$ \begin{CJK}{UTF8}{mj}为\end{CJK} $f(x)$ \begin{CJK}{UTF8}{mj}其\end{CJK} Jacobi \begin{CJK}{UTF8}{mj}矩阵\end{CJK}. \begin{CJK}{UTF8}{mj}试证\end{CJK}: $f(x)$ \begin{CJK}{UTF8}{mj}是凸函数的充分必要条件是对\end{CJK} $\forall a$, \begin{CJK}{UTF8}{mj}都有\end{CJK} $f(x) \geq f(a)+J f(a) f^{\prime}(a)$.

  \item \begin{CJK}{UTF8}{mj}设\end{CJK} $L: \frac{x^{2}}{25}+\frac{y^{2}}{16}=1$, \begin{CJK}{UTF8}{mj}试求\end{CJK}

\end{enumerate}
$$
\int_{L} \frac{y^{3} \mathrm{~d} x-x y^{2} \mathrm{~d} y}{\left(x^{2}+y^{2}\right)^{2}}
$$

\begin{enumerate}
  \setcounter{enumi}{7}
  \item \begin{CJK}{UTF8}{mj}若\end{CJK} $f: \mathbb{R} \rightarrow \mathbb{R}, \lim _{x \rightarrow+\infty} f(x)=A$ \begin{CJK}{UTF8}{mj}存在且为有限数\end{CJK}. \begin{CJK}{UTF8}{mj}求证\end{CJK}: $A$ \begin{CJK}{UTF8}{mj}一致连续\end{CJK}. 8. \begin{CJK}{UTF8}{mj}试求\end{CJK}
\end{enumerate}
$$
\lim _{x \rightarrow 1} \sum_{n=0}^{\infty} \frac{x^{n}}{3^{n}} \cos \left(n \pi x^{3}\right)
$$

\begin{enumerate}
  \setcounter{enumi}{9}
  \item \begin{CJK}{UTF8}{mj}计算\end{CJK}
\end{enumerate}
$$
\lim _{x \rightarrow \infty} 2 x\left[\left(1+\frac{1}{x}\right)^{x}-\mathrm{e}\right]
$$

\begin{enumerate}
  \setcounter{enumi}{10}
  \item \begin{CJK}{UTF8}{mj}判断\end{CJK} $\int_{1}^{+\infty}(-1)^{[x]} \mathrm{d} x$ \begin{CJK}{UTF8}{mj}的玫散性\end{CJK}.
\end{enumerate}
\section{二、计算题}
\begin{enumerate}
  \item \begin{CJK}{UTF8}{mj}计算一重积分\end{CJK}
\end{enumerate}
$$
\iint_{D} \mathrm{e}^{\frac{x-y}{x+y}} \mathrm{~d} x \mathrm{~d} y
$$
\begin{CJK}{UTF8}{mj}其中\end{CJK} $D=\{(x, y) \mid x \geq 0, y \geq 0, x+y \geq 1\}$

\begin{enumerate}
  \setcounter{enumi}{2}
  \item \begin{CJK}{UTF8}{mj}设\end{CJK} $a$ \begin{CJK}{UTF8}{mj}是有理数\end{CJK}, \begin{CJK}{UTF8}{mj}且非整数\end{CJK}, \begin{CJK}{UTF8}{mj}求\end{CJK} $f(x)=\cos a x$ \begin{CJK}{UTF8}{mj}在\end{CJK} $[-\pi, \pi)$ \begin{CJK}{UTF8}{mj}的傅里叶级数\end{CJK}.
\end{enumerate}
\section{三、简答题}
\begin{enumerate}
  \item \begin{CJK}{UTF8}{mj}设\end{CJK} $f(x, u)$ \begin{CJK}{UTF8}{mj}在\end{CJK} $[0,+\infty) \times[a, b]$ \begin{CJK}{UTF8}{mj}上连续\end{CJK}, \begin{CJK}{UTF8}{mj}对\end{CJK} $\forall u \in[a, b), \int_{0}^{+\infty} f(x, u) \mathrm{d} x$ \begin{CJK}{UTF8}{mj}收敛\end{CJK}, $\int_{0}^{+\infty} f(x, b) \mathrm{d} x$ \begin{CJK}{UTF8}{mj}发散\end{CJK}, \begin{CJK}{UTF8}{mj}求证\end{CJK}: $\int_{0}^{+\infty} f(x, u) \mathrm{d} x$ \begin{CJK}{UTF8}{mj}在\end{CJK} $[a, b)$ \begin{CJK}{UTF8}{mj}上不一致收敛\end{CJK}.

  \item \begin{CJK}{UTF8}{mj}设\end{CJK} $g(x, y)=|x y|$, \begin{CJK}{UTF8}{mj}判定其在\end{CJK} $(0,0)$ \begin{CJK}{UTF8}{mj}处的可微性\end{CJK}.

  \item \begin{CJK}{UTF8}{mj}设\end{CJK} $f(x, y)=\mathrm{e}^{a y} \cos (\ln x)$, \begin{CJK}{UTF8}{mj}求\end{CJK}

\end{enumerate}
$$
\frac{\partial^{2} f}{\partial x^{2}}+\frac{1}{a^{2} x^{2}} \frac{\partial^{2} f}{\partial y^{2}}+\frac{\partial f}{\partial x}
$$

\begin{enumerate}
  \setcounter{enumi}{4}
  \item \begin{CJK}{UTF8}{mj}设\end{CJK} $f(x)$ \begin{CJK}{UTF8}{mj}在\end{CJK} $(a, b)$ \begin{CJK}{UTF8}{mj}上至多有第十类间断点\end{CJK}, \begin{CJK}{UTF8}{mj}且有\end{CJK}
\end{enumerate}
$$
f\left(\frac{x+y}{2}\right) \leq \frac{f(x)+f(y)}{2}
$$
\begin{CJK}{UTF8}{mj}求证\end{CJK}: $f(x)$ \begin{CJK}{UTF8}{mj}连续\end{CJK}.

\section{第19章 四川大学}
\section{$19.12020$ 年数学分析考研真题}
\section{考生须知:}
\begin{enumerate}
  \item \begin{CJK}{UTF8}{mj}本试卷满分为\end{CJK} 150 \begin{CJK}{UTF8}{mj}分\end{CJK}, \begin{CJK}{UTF8}{mj}全部考试时间总计\end{CJK} 180 \begin{CJK}{UTF8}{mj}分钟\end{CJK};

  \item \begin{CJK}{UTF8}{mj}所有答案必须写在答题纸上\end{CJK}, \begin{CJK}{UTF8}{mj}写在试题纸上或草稿纸上一律无效\end{CJK}。

\end{enumerate}
\section{一、计算题}
\begin{enumerate}
  \item \begin{CJK}{UTF8}{mj}设\end{CJK} $f^{\prime}(x)$ \begin{CJK}{UTF8}{mj}在\end{CJK} $x=0$ \begin{CJK}{UTF8}{mj}的无穷邻域内存在\end{CJK}, \begin{CJK}{UTF8}{mj}且有\end{CJK} $f(0)=f^{\prime}(0)=0$, \begin{CJK}{UTF8}{mj}令\end{CJK}
\end{enumerate}
$$
x_{n}=f\left(\frac{1}{n^{2}}\right)+f\left(\frac{2}{n^{2}}\right)+\cdots+f\left(\frac{n}{n^{2}}\right)
$$
\begin{CJK}{UTF8}{mj}求\end{CJK} $\lim _{n \rightarrow \infty} x_{n}$.

\begin{enumerate}
  \setcounter{enumi}{2}
  \item \begin{CJK}{UTF8}{mj}计算\end{CJK} $\int \frac{1}{1+x^{4}} \mathrm{~d} x$.

  \item \begin{CJK}{UTF8}{mj}计算一重积分\end{CJK}

\end{enumerate}
$$
\iint_{D} \frac{(x+y) \ln \left(1+\frac{y}{x}\right)}{\sqrt{1-x-y}} \mathrm{~d} x \mathrm{~d} y
$$
\begin{CJK}{UTF8}{mj}其中\end{CJK} $D$ \begin{CJK}{UTF8}{mj}为\end{CJK} $x=0, y=0, x+y=1$ \begin{CJK}{UTF8}{mj}所围成的区域\end{CJK}.

\begin{enumerate}
  \setcounter{enumi}{4}
  \item \begin{CJK}{UTF8}{mj}计算曲线积分\end{CJK}
\end{enumerate}
$$
I=\oint_{L}[x \cos (n, x)+y \cos (n, y)] \mathrm{d} S
$$
\begin{CJK}{UTF8}{mj}其中\end{CJK} $(n, x),(n, y)$ \begin{CJK}{UTF8}{mj}分别是由\end{CJK} $x$ \begin{CJK}{UTF8}{mj}轴\end{CJK}, $y$ \begin{CJK}{UTF8}{mj}轴正向与\end{CJK} $L$ \begin{CJK}{UTF8}{mj}的外法向量\end{CJK} $n$ \begin{CJK}{UTF8}{mj}之间的夹角\end{CJK}, $L$ \begin{CJK}{UTF8}{mj}为逐渐光滑的\end{CJK} \begin{CJK}{UTF8}{mj}简单闭曲线\end{CJK}.

\begin{enumerate}
  \setcounter{enumi}{5}
  \item \begin{CJK}{UTF8}{mj}计算曲面积分\end{CJK}
\end{enumerate}
$$
\iint_{S} \frac{x \mathrm{~d} y \mathrm{~d} z+z^{2} \mathrm{~d} x \mathrm{~d} y}{x^{2}+y^{2}+z^{2}}
$$
\begin{CJK}{UTF8}{mj}其中\end{CJK} $S$ \begin{CJK}{UTF8}{mj}是曲面\end{CJK} $x^{2}+y^{2}=r^{2}$ \begin{CJK}{UTF8}{mj}以及\end{CJK} $z=r, z=-r(r>0)$ \begin{CJK}{UTF8}{mj}所围成的区域\end{CJK}

\section{二、证明题}
\begin{enumerate}
  \item $f(x)$ \begin{CJK}{UTF8}{mj}在\end{CJK} $[a, b]$ \begin{CJK}{UTF8}{mj}可积\end{CJK}, $F(x)$ \begin{CJK}{UTF8}{mj}在\end{CJK} $[a, b]$ \begin{CJK}{UTF8}{mj}连续\end{CJK}, \begin{CJK}{UTF8}{mj}在\end{CJK} $(a, b)$ \begin{CJK}{UTF8}{mj}内除有限个点外都有\end{CJK} $F^{\prime}(x)=f(x)$, \begin{CJK}{UTF8}{mj}则\end{CJK}
\end{enumerate}
$$
\int_{a}^{b} f(x) \mathrm{d} x=F(b)-F(a)
$$

\begin{enumerate}
  \setcounter{enumi}{2}
  \item \begin{CJK}{UTF8}{mj}若函数\end{CJK} $f(x)$ \begin{CJK}{UTF8}{mj}在\end{CJK} $(-\infty,+\infty)$ \begin{CJK}{UTF8}{mj}上三阶可导\end{CJK}, \begin{CJK}{UTF8}{mj}且\end{CJK} $f(x), f^{\prime \prime \prime}(x)$ \begin{CJK}{UTF8}{mj}在\end{CJK} $(\infty,+\infty)$ \begin{CJK}{UTF8}{mj}上有界\end{CJK}, \begin{CJK}{UTF8}{mj}证明\end{CJK} $f^{\prime}(x)$ \begin{CJK}{UTF8}{mj}与\end{CJK} $f^{\prime \prime}(x)$ \begin{CJK}{UTF8}{mj}在\end{CJK} $(-\infty,+\infty)$ \begin{CJK}{UTF8}{mj}上有界\end{CJK}.

  \item \begin{CJK}{UTF8}{mj}若\end{CJK} $\sum_{n=1}^{\infty} n \mathrm{e}^{-n x}$ \begin{CJK}{UTF8}{mj}在\end{CJK} $(0,+\infty)$ \begin{CJK}{UTF8}{mj}上收敛\end{CJK}, \begin{CJK}{UTF8}{mj}试证级数在\end{CJK} $[0,+\infty)$ \begin{CJK}{UTF8}{mj}上不一致收敛\end{CJK}.

  \item \begin{CJK}{UTF8}{mj}若\end{CJK} $x_{0}=a, 0<a<\frac{\pi}{2}$, \begin{CJK}{UTF8}{mj}且\end{CJK} $x_{n}=\sin x_{n-1}$, \begin{CJK}{UTF8}{mj}求证\end{CJK}: $\lim _{n \rightarrow \infty} \sqrt{\frac{n}{3}} x_{n}=1$.

  \item \begin{CJK}{UTF8}{mj}若\end{CJK} $f(x)=\frac{x+2}{x+1} \sin \frac{1}{x}, a>0$, \begin{CJK}{UTF8}{mj}试证\end{CJK}: $f(x)$ \begin{CJK}{UTF8}{mj}在\end{CJK} $(0, a)$ \begin{CJK}{UTF8}{mj}内非一致收敛又在\end{CJK} $[a,+\infty)$ \begin{CJK}{UTF8}{mj}上一致收敛\end{CJK}.

  \item \begin{CJK}{UTF8}{mj}证明广义积分\end{CJK} $\int_{0}^{+\infty} \frac{\mathrm{d} x}{1+x^{2} \sin ^{2} x}$ \begin{CJK}{UTF8}{mj}发散\end{CJK}.

  \item \begin{CJK}{UTF8}{mj}若\end{CJK} $f(x)$ \begin{CJK}{UTF8}{mj}在\end{CJK} $[0,+\infty)$ \begin{CJK}{UTF8}{mj}上可积\end{CJK}, \begin{CJK}{UTF8}{mj}试证\end{CJK}:

\end{enumerate}
$$
\lim _{y \rightarrow 0^{+}} \int_{0}^{+\infty} \mathrm{e}^{-x y} f(x) \mathrm{d} x=\int_{0}^{+\infty} f(x) \mathrm{d} x
$$

\section{第 20 章 电子科技大学}
\section{$20.12020$ 年数学分析考研真题}
\section{考生须知:}
\begin{enumerate}
  \item \begin{CJK}{UTF8}{mj}本试卷满分为\end{CJK} 150 \begin{CJK}{UTF8}{mj}分\end{CJK}, \begin{CJK}{UTF8}{mj}全部考试时间总计\end{CJK} 180 \begin{CJK}{UTF8}{mj}分钟\end{CJK};

  \item \begin{CJK}{UTF8}{mj}所有答案必须写在答题纸上\end{CJK}, \begin{CJK}{UTF8}{mj}写在试题纸上或草稿纸上一律无效\end{CJK}。

\end{enumerate}
\section{一、填空题}
\begin{enumerate}
  \item \begin{CJK}{UTF8}{mj}计算极限\end{CJK} $\lim _{n \rightarrow \infty} \frac{1^{2}+3^{2}+\cdots+(2 n+1)^{2}}{n^{3}}=$

  \item \begin{CJK}{UTF8}{mj}求\end{CJK} $\sum_{n=0}^{\infty} \frac{n^{n}}{n !} x^{n}$ \begin{CJK}{UTF8}{mj}的收敛半径\end{CJK}

  \item \begin{CJK}{UTF8}{mj}求极限\end{CJK} $\lim _{x \rightarrow 0^{+}} \frac{\int_{0}^{x^{2}} \sin \sqrt{t} \mathrm{~d} t}{x^{3}}=$

  \item \begin{CJK}{UTF8}{mj}右\end{CJK} $z=f\left(x^{2}+y^{2}+z^{2}, x y z\right)$, \begin{CJK}{UTF8}{mj}且\end{CJK} $f$ \begin{CJK}{UTF8}{mj}可微\end{CJK}, \begin{CJK}{UTF8}{mj}求\end{CJK} $\frac{\partial z}{\partial x}$

  \item \begin{CJK}{UTF8}{mj}叙述\end{CJK} $f_{n}(x)$ \begin{CJK}{UTF8}{mj}一致收敛到\end{CJK} $f(x)$ \begin{CJK}{UTF8}{mj}的定义\end{CJK}

\end{enumerate}
\section{二、计算题}
\begin{enumerate}
  \item \begin{CJK}{UTF8}{mj}若\end{CJK} $F(x)=\int_{0}^{y}(x+y) f(x) \mathrm{d} x$, \begin{CJK}{UTF8}{mj}其中\end{CJK} $f(x)$ \begin{CJK}{UTF8}{mj}为可微函数\end{CJK}, \begin{CJK}{UTF8}{mj}求\end{CJK} $F^{\prime \prime}(y)$.

  \item \begin{CJK}{UTF8}{mj}已知函数\end{CJK} $z=x^{2} y(3-x-y)$, \begin{CJK}{UTF8}{mj}且\end{CJK} $x, y \geq 0, x+y \leq 4$, \begin{CJK}{UTF8}{mj}求其最大值和最小值\end{CJK}.

  \item \begin{CJK}{UTF8}{mj}计算曲面积分\end{CJK}

\end{enumerate}
$$
\iint_{\Sigma} x^{2} \mathrm{~d} y \mathrm{~d} z+y^{2} \mathrm{~d} x \mathrm{~d} z+z^{2} \mathrm{~d} x \mathrm{~d} y
$$
\begin{CJK}{UTF8}{mj}其中\end{CJK} $\Sigma: z=\sqrt{x^{2}+y^{2}}(0 \leq z \leq 1)$

\begin{enumerate}
  \setcounter{enumi}{4}
  \item \begin{CJK}{UTF8}{mj}计算曲线积分\end{CJK}
\end{enumerate}
$$
\oint_{L} \frac{x \mathrm{~d} y-y \mathrm{~d} x}{x^{2}+y^{2}}
$$
\begin{CJK}{UTF8}{mj}其中\end{CJK} $L$ \begin{CJK}{UTF8}{mj}为不经过原点的简单闭曲线\end{CJK}, \begin{CJK}{UTF8}{mj}方向为逆时针\end{CJK}.

\section{三、证明题}
\begin{enumerate}
  \item \begin{CJK}{UTF8}{mj}已知函数\end{CJK} $f(x)$ \begin{CJK}{UTF8}{mj}为\end{CJK} $[0,1]$ \begin{CJK}{UTF8}{mj}上的连续函数\end{CJK}, \begin{CJK}{UTF8}{mj}且\end{CJK} $f(0)=f(1)=0, g(x)=x^{2} f(x)$, \begin{CJK}{UTF8}{mj}试证\end{CJK}: \begin{CJK}{UTF8}{mj}存在\end{CJK} $\xi \in(0,1)$, \begin{CJK}{UTF8}{mj}使得\end{CJK} $g^{\prime \prime}(\xi)=0$.

  \item \begin{CJK}{UTF8}{mj}若\end{CJK} $f(x)$ \begin{CJK}{UTF8}{mj}为\end{CJK} $[0,1]$ \begin{CJK}{UTF8}{mj}连续可微函数\end{CJK}, \begin{CJK}{UTF8}{mj}且\end{CJK} $f(0)-f(1)=1$, \begin{CJK}{UTF8}{mj}试证\end{CJK}: $\int_{0}^{1}\left[f^{\prime}(x)\right]^{2} \mathrm{~d} x=1$.

  \item \begin{CJK}{UTF8}{mj}证明\end{CJK}: $\sum_{n=1}^{\infty} x^{a} \mathrm{e}^{-n x}$ \begin{CJK}{UTF8}{mj}在\end{CJK} $x \in[\delta,+\infty),(\delta>0)$ \begin{CJK}{UTF8}{mj}上一致收敛\end{CJK}.

  \item \begin{CJK}{UTF8}{mj}证明\end{CJK}: \begin{CJK}{UTF8}{mj}含参变量积分\end{CJK} $\int_{0}^{+\infty} a \mathrm{e}^{-a x} \mathrm{~d} x$ \begin{CJK}{UTF8}{mj}在\end{CJK} $[\delta,+\infty),(\delta>0)$ \begin{CJK}{UTF8}{mj}致收玫\end{CJK}.

  \item \begin{CJK}{UTF8}{mj}若数列\end{CJK} $\left\{a_{n}\right\}$ \begin{CJK}{UTF8}{mj}单调递增且有界\end{CJK}, \begin{CJK}{UTF8}{mj}恒正\end{CJK}, \begin{CJK}{UTF8}{mj}试证\end{CJK}: $\sum_{n=1}^{\infty}\left(1-\frac{a_{n}}{a_{n+1}}\right)$ \begin{CJK}{UTF8}{mj}收敛\end{CJK}.

  \item \begin{CJK}{UTF8}{mj}若\end{CJK} $f(x)$ \begin{CJK}{UTF8}{mj}在闭区间\end{CJK} $[a, b]$ \begin{CJK}{UTF8}{mj}上连续\end{CJK}, $x_{n}$ \begin{CJK}{UTF8}{mj}为此区间的基本数列\end{CJK}, \begin{CJK}{UTF8}{mj}试证\end{CJK}: $f\left(x_{n}\right)$ \begin{CJK}{UTF8}{mj}也为基本数列\end{CJK}.

  \item \begin{CJK}{UTF8}{mj}设\end{CJK} $0<a_{n}<1, a_{n+1}=a_{n}\left(1-a_{n}\right)$, \begin{CJK}{UTF8}{mj}试证\end{CJK}:\\
(1) $\lim _{n \rightarrow \infty} n a_{n}=1$;\\
(2) $\lim _{n \rightarrow \infty} \frac{n\left(1-n a_{n}\right)}{\ln n}=1$.

\end{enumerate}
\section{第21章 湖南大学}
\section{$21.12020$ 年数学分析考研真题}
\section{考生须知:}
\begin{enumerate}
  \item \begin{CJK}{UTF8}{mj}本试卷满分为\end{CJK} 150 \begin{CJK}{UTF8}{mj}分\end{CJK}, \begin{CJK}{UTF8}{mj}全部考试时间总计\end{CJK} 180 \begin{CJK}{UTF8}{mj}分钟\end{CJK};

  \item \begin{CJK}{UTF8}{mj}所有答案必须写在答题纸上\end{CJK}, \begin{CJK}{UTF8}{mj}写在试题纸上或草稿纸上一律无效\end{CJK}。

  \item \begin{CJK}{UTF8}{mj}计算题\end{CJK}

\end{enumerate}
(1). \begin{CJK}{UTF8}{mj}已知函数\end{CJK} $f(x)$ \begin{CJK}{UTF8}{mj}的一个原函数为\end{CJK} $\arctan x$, \begin{CJK}{UTF8}{mj}求\end{CJK} $\int x f^{\prime}(x) \mathrm{d} x$.

(2). \begin{CJK}{UTF8}{mj}求\end{CJK} $\lim _{\alpha \rightarrow 0} \int_{\alpha}^{1+\alpha} \frac{1}{1+\alpha+\alpha^{2}} \mathrm{~d} x$.

(3). \begin{CJK}{UTF8}{mj}已知\end{CJK} $\left\{\begin{array}{l}x=\varphi(t) \\ y=\varphi(t)\end{array}\right.$ \begin{CJK}{UTF8}{mj}且满足\end{CJK} $\varphi^{\prime}(t)=\mathrm{e}^{t}+1, \frac{\mathrm{d}^{3} y}{\mathrm{~d} x^{3}}=\mathrm{e}^{t}+2 t$, \begin{CJK}{UTF8}{mj}求\end{CJK} $\frac{\mathrm{d}^{4} y}{\mathrm{~d} x^{4}}$.

(4). \begin{CJK}{UTF8}{mj}已知\end{CJK} $\lim _{x \rightarrow-\infty}\left(\sqrt{x^{2}-x+1}-a x-b\right)=0$, \begin{CJK}{UTF8}{mj}求\end{CJK} $a, b$ \begin{CJK}{UTF8}{mj}的值\end{CJK}.

\begin{enumerate}
  \setcounter{enumi}{2}
  \item \begin{CJK}{UTF8}{mj}已知数列\end{CJK} $\left\{x_{n}\right\}$ \begin{CJK}{UTF8}{mj}满足\end{CJK} $\sum_{k=1}^{n}\left|x_{k+1}-x_{k}\right| \leq M$, \begin{CJK}{UTF8}{mj}证明\end{CJK}:\begin{CJK}{UTF8}{mj}数列\end{CJK} $\left\{x_{n}\right\}$ \begin{CJK}{UTF8}{mj}收敛\end{CJK}.

  \item \begin{CJK}{UTF8}{mj}讨论\end{CJK} $\left\{\begin{array}{l}x=a(t-\sin t) \\ y=a(1-\cos t)\end{array} \quad(0 \leq t<2 \pi)\right.$ \begin{CJK}{UTF8}{mj}的凸性\end{CJK}.

  \item \begin{CJK}{UTF8}{mj}证明\end{CJK}: $f(x)=\sin \frac{1}{x}$ \begin{CJK}{UTF8}{mj}在\end{CJK} $(0,1)$ \begin{CJK}{UTF8}{mj}上不一致连续\end{CJK}.

  \item \begin{CJK}{UTF8}{mj}已知\end{CJK} $f(x) \in[a, b]$, \begin{CJK}{UTF8}{mj}且\end{CJK} $f(x)$ \begin{CJK}{UTF8}{mj}单调递增\end{CJK}, \begin{CJK}{UTF8}{mj}证明\end{CJK}: $\int_{a}^{b} x f(x) \mathrm{d} x \geq \frac{a+b}{2} \int_{a}^{b} f(x) \mathrm{d} x$.

  \item \begin{CJK}{UTF8}{mj}已知\end{CJK} $u_{n}(x)=\frac{\cos n x}{n \sqrt{n}}, x \in(-\infty,+\infty)$, \begin{CJK}{UTF8}{mj}证明\end{CJK}:

\end{enumerate}
(1). $\sum_{n=1}^{\infty} u_{n}(x)$ \begin{CJK}{UTF8}{mj}在\end{CJK} $(-\infty,+\infty)$ \begin{CJK}{UTF8}{mj}上一致收敛\end{CJK};

$(2)$. \begin{CJK}{UTF8}{mj}设\end{CJK} $S(x)=\sum_{n=1}^{\infty} u_{n}(x)$, \begin{CJK}{UTF8}{mj}求\end{CJK} $\int_{0}^{\pi} S(x) \mathrm{d} x$;

(3). \begin{CJK}{UTF8}{mj}在区间\end{CJK} $(0,2 \pi)$ \begin{CJK}{UTF8}{mj}上\end{CJK} $S(x)$ \begin{CJK}{UTF8}{mj}是否可以逐项求导\end{CJK},\begin{CJK}{UTF8}{mj}若可以\end{CJK}, \begin{CJK}{UTF8}{mj}求\end{CJK} $S^{\prime}(x), x \in(0,2 \pi)$; \begin{CJK}{UTF8}{mj}若不可\end{CJK}, \begin{CJK}{UTF8}{mj}说\end{CJK} \begin{CJK}{UTF8}{mj}明理由\end{CJK} 7. \begin{CJK}{UTF8}{mj}已知\end{CJK}
$$
f(x, y)= \begin{cases}\frac{x y}{\sqrt{x^{2}+y^{2}}}, & x^{2}+y^{2} \neq 0 \\ 0, & x^{2}+y^{2}=0\end{cases}
$$
\begin{CJK}{UTF8}{mj}证明\end{CJK}: $f(x, y)$ \begin{CJK}{UTF8}{mj}在\end{CJK} $(0,0)$ \begin{CJK}{UTF8}{mj}处的偏导数存在\end{CJK}, \begin{CJK}{UTF8}{mj}但不可微\end{CJK}.

\begin{enumerate}
  \setcounter{enumi}{8}
  \item \begin{CJK}{UTF8}{mj}计算第一型曲面积分\end{CJK}
\end{enumerate}
$$
I=\iint_{\Sigma} \frac{\mathrm{d} S}{\sqrt{x^{2}+y^{2}+(z+a)^{2}}}
$$
\begin{CJK}{UTF8}{mj}其中\end{CJK} $\sigma$ \begin{CJK}{UTF8}{mj}为\end{CJK} $x^{2}+y^{2}+z^{2}=a^{2}(z>0)$

\begin{enumerate}
  \setcounter{enumi}{9}
  \item \begin{CJK}{UTF8}{mj}已知\end{CJK} $f(x) \in C[0,1]$, \begin{CJK}{UTF8}{mj}一阶导数连续\end{CJK}, \begin{CJK}{UTF8}{mj}且\end{CJK} $f(0)=f(1)=0$, \begin{CJK}{UTF8}{mj}对\end{CJK} $\forall x \in(0,1)$, \begin{CJK}{UTF8}{mj}均有\end{CJK} $\left|f^{\prime}(x)\right| \leq A$, \begin{CJK}{UTF8}{mj}试证\end{CJK}: $\left|f^{\prime}(x)\right| \leq \frac{A}{2}$.
\end{enumerate}
\section{第 22 章 华中科技大学}
\section{$22.12020$ 年数学分析考研真题}
\section{考生须知:}
\begin{enumerate}
  \item \begin{CJK}{UTF8}{mj}本试卷满分为\end{CJK} 150 \begin{CJK}{UTF8}{mj}分\end{CJK}, \begin{CJK}{UTF8}{mj}全部考试时间总计\end{CJK} 180 \begin{CJK}{UTF8}{mj}分钟\end{CJK};

  \item \begin{CJK}{UTF8}{mj}所有答案必须写在答题纸上\end{CJK}, \begin{CJK}{UTF8}{mj}写在试题纸上或草稿纸上一律无效\end{CJK}。

  \item (15 \begin{CJK}{UTF8}{mj}分\end{CJK}) \begin{CJK}{UTF8}{mj}求极限\end{CJK}

\end{enumerate}
$$
\lim _{n \rightarrow \infty}\left[\frac{\sin \frac{x}{n}}{n+1}+\frac{2 \sin \frac{2 x}{n}}{2 n+1}+\frac{3 \sin \frac{3}{n} x}{3 n+1}+\cdots+\frac{n \sin \frac{n}{n} x}{n^{2}+1}\right]
$$

\begin{enumerate}
  \setcounter{enumi}{2}
  \item (15 \begin{CJK}{UTF8}{mj}分\end{CJK}) \begin{CJK}{UTF8}{mj}求定积分\end{CJK}
\end{enumerate}
$$
\int_{0}^{\frac{\pi}{2}} \frac{\sin x \cos x}{a^{2} \sin ^{2} x+b^{2} \cos ^{2} x} d x
$$

\begin{enumerate}
  \setcounter{enumi}{3}
  \item (15 \begin{CJK}{UTF8}{mj}分\end{CJK}) \begin{CJK}{UTF8}{mj}求二重积分\end{CJK}
\end{enumerate}
$$
\iint_{D} \frac{x}{y^{2}(1+x y)} \mathrm{d} x \mathrm{~d} y
$$
\begin{CJK}{UTF8}{mj}其中\end{CJK} $D$ \begin{CJK}{UTF8}{mj}是由\end{CJK} $x y=1, x y=3, y^{2}=x, y^{2}=3 x$ \begin{CJK}{UTF8}{mj}围成的区域\end{CJK}.

\begin{enumerate}
  \setcounter{enumi}{4}
  \item (15 \begin{CJK}{UTF8}{mj}分\end{CJK}) \begin{CJK}{UTF8}{mj}求第二型曲线积分\end{CJK}
\end{enumerate}
\begin{CJK}{UTF8}{mj}其中\end{CJK} $L$ \begin{CJK}{UTF8}{mj}是\end{CJK} $(x-1)^{2}+y^{2}=4$ \begin{CJK}{UTF8}{mj}取正向\end{CJK}.

\begin{enumerate}
  \setcounter{enumi}{5}
  \item (15 \begin{CJK}{UTF8}{mj}分\end{CJK}) \begin{CJK}{UTF8}{mj}求\end{CJK}
\end{enumerate}
$$
f(x)= \begin{cases}-\frac{\pi}{4}, & -\pi<x<0 \\ \frac{\pi}{4}, & 0 \leq x<\pi\end{cases}
$$
\begin{CJK}{UTF8}{mj}的\end{CJK} Fourier \begin{CJK}{UTF8}{mj}展开式\end{CJK}, \begin{CJK}{UTF8}{mj}并求\end{CJK} $\sum_{n=1}^{\infty} \frac{1}{(2 n-1)^{2}}$.

\begin{enumerate}
  \setcounter{enumi}{6}
  \item (15 \begin{CJK}{UTF8}{mj}分\end{CJK}) \begin{CJK}{UTF8}{mj}设\end{CJK} $f(x)$ \begin{CJK}{UTF8}{mj}在\end{CJK} $[0,+\infty)$ \begin{CJK}{UTF8}{mj}上一致连续\end{CJK}, \begin{CJK}{UTF8}{mj}在\end{CJK} $[0,+\infty)$ \begin{CJK}{UTF8}{mj}上连续且\end{CJK}
\end{enumerate}
$$
\lim _{x \rightarrow+\infty}[f(x)-g(x)]=0
$$
\begin{CJK}{UTF8}{mj}证明\end{CJK}: $g(x)$ \begin{CJK}{UTF8}{mj}也一致收敛\end{CJK}. 7. (15 \begin{CJK}{UTF8}{mj}分\end{CJK}) \begin{CJK}{UTF8}{mj}设\end{CJK} $f(x)$ \begin{CJK}{UTF8}{mj}在\end{CJK} $[0,1]$ \begin{CJK}{UTF8}{mj}上二阶可导\end{CJK}, \begin{CJK}{UTF8}{mj}且有\end{CJK} $f^{\prime}(0)=f^{\prime}(1)=0$, \begin{CJK}{UTF8}{mj}试证\end{CJK}:
$$
\exists \xi \in(0,1) \text {, s.t. }\left|f^{\prime \prime}(\xi)\right| \geq 4|f(1)-f(0)|
$$

\begin{enumerate}
  \setcounter{enumi}{8}
  \item (15 \begin{CJK}{UTF8}{mj}分\end{CJK}) \begin{CJK}{UTF8}{mj}设\end{CJK} $f(x)$ \begin{CJK}{UTF8}{mj}在\end{CJK} $[a, b]$ \begin{CJK}{UTF8}{mj}上连续\end{CJK}, \begin{CJK}{UTF8}{mj}且恒正\end{CJK}, \begin{CJK}{UTF8}{mj}试证\end{CJK}:
\end{enumerate}
$$
\int_{a}^{b} f(x) \mathrm{d} x \int_{a}^{b} \frac{1}{f(x)} \mathrm{d} x \geq(b-a)^{2}
$$

\begin{enumerate}
  \setcounter{enumi}{9}
  \item (15 \begin{CJK}{UTF8}{mj}分\end{CJK}) \begin{CJK}{UTF8}{mj}设\end{CJK} $\sum_{n=1}^{\infty} a_{n}$ \begin{CJK}{UTF8}{mj}绝对收敛\end{CJK}, \begin{CJK}{UTF8}{mj}证明\end{CJK}:
\end{enumerate}
$$
\sum_{n=1}^{\infty}\left(a_{1}+a_{2}+\cdots+a_{n}\right) a_{n}
$$

\begin{enumerate}
  \setcounter{enumi}{10}
  \item (15 \begin{CJK}{UTF8}{mj}分\end{CJK}) \begin{CJK}{UTF8}{mj}若\end{CJK} $\sum_{n=1}^{\infty} a_{n}$ \begin{CJK}{UTF8}{mj}收敛\end{CJK}, $\sum_{n=1}^{\infty} b_{n}$ \begin{CJK}{UTF8}{mj}是正项级数\end{CJK}, $b_{n}$ \begin{CJK}{UTF8}{mj}单调递增且\end{CJK} $\lim _{n \rightarrow \infty} b_{n}=+\infty$, \begin{CJK}{UTF8}{mj}试证\end{CJK}:
\end{enumerate}
$$
\lim _{n \rightarrow \infty} \frac{a_{1} b_{1}+a_{2} b_{2}+\cdots+a_{n} b_{n}}{b_{n}}=0
$$

\section{第23 章 厦门大学}
\section{$23.12020$ 年数学分析考研真题}
\section{考生须知:}
\begin{enumerate}
  \item \begin{CJK}{UTF8}{mj}本试卷满分为\end{CJK} 150 \begin{CJK}{UTF8}{mj}分\end{CJK}, \begin{CJK}{UTF8}{mj}全部考试时间总计\end{CJK} 180 \begin{CJK}{UTF8}{mj}分钟\end{CJK};

  \item \begin{CJK}{UTF8}{mj}所有答案必须写在答题纸上\end{CJK}, \begin{CJK}{UTF8}{mj}写在试题纸上或草稿纸上一律无效\end{CJK}。

  \item \begin{CJK}{UTF8}{mj}已知数列\end{CJK} $\left\{x_{k}\right\}$ \begin{CJK}{UTF8}{mj}满足\end{CJK} $x_{k+1}=\frac{x_{k}^{2}}{2\left(x_{k}-1\right)}(k=0,1, \cdots)$, \begin{CJK}{UTF8}{mj}且\end{CJK} $x_{1}>0$, \begin{CJK}{UTF8}{mj}证明\end{CJK}: \begin{CJK}{UTF8}{mj}数列\end{CJK} $\left\{x_{k}\right\}$ \begin{CJK}{UTF8}{mj}收玫\end{CJK}, \begin{CJK}{UTF8}{mj}并\end{CJK} \begin{CJK}{UTF8}{mj}求其极限\end{CJK}.

  \item \begin{CJK}{UTF8}{mj}已知\end{CJK} $a_{n}=\left(\mathrm{e}-\left(1+\frac{1}{n}\right)^{n}\right)^{p}, p$ \begin{CJK}{UTF8}{mj}为实数\end{CJK}, \begin{CJK}{UTF8}{mj}判断\end{CJK} $\sum_{n=1}^{\infty} a_{n}$ \begin{CJK}{UTF8}{mj}的收玫性并证明\end{CJK}.

  \item \begin{CJK}{UTF8}{mj}已知\end{CJK} $f(x)$ \begin{CJK}{UTF8}{mj}在\end{CJK} $\mathbb{R}$ \begin{CJK}{UTF8}{mj}上连续可微\end{CJK}, \begin{CJK}{UTF8}{mj}且\end{CJK} $\left|f^{\prime}(x)\right|<1$, \begin{CJK}{UTF8}{mj}证明\end{CJK}: \begin{CJK}{UTF8}{mj}存在\end{CJK} $0<M<1$, \begin{CJK}{UTF8}{mj}使不等式\end{CJK}

\end{enumerate}
$$
|f(x)-f(0)| \leq M|x|
$$
\begin{CJK}{UTF8}{mj}成立\end{CJK}.

\begin{enumerate}
  \setcounter{enumi}{4}
  \item \begin{CJK}{UTF8}{mj}已知\end{CJK} $f(x) \in \mathbb{C}^{2}(0,1)$, \begin{CJK}{UTF8}{mj}且\end{CJK} $f(0)=f(1)=0, x$ \begin{CJK}{UTF8}{mj}在\end{CJK} $(0,1)$ \begin{CJK}{UTF8}{mj}内且\end{CJK} $f(x) \neq 0$, \begin{CJK}{UTF8}{mj}证明\end{CJK}:
\end{enumerate}
$$
\int_{0}^{1} \frac{f^{\prime \prime}(x)}{f(x)} \mathrm{d} x>4
$$

\begin{enumerate}
  \setcounter{enumi}{5}
  \item \begin{CJK}{UTF8}{mj}已知\end{CJK} $f(x)$, \begin{CJK}{UTF8}{mj}且\end{CJK} $0<a<b, \lim _{x \rightarrow+\infty} f(x)=k$. \begin{CJK}{UTF8}{mj}证明\end{CJK}:
\end{enumerate}
$$
\int_{0}^{+\infty} \frac{f(a x)-f(b x)}{x} \mathrm{~d} x=(f(0)-k) \ln \frac{b}{a}
$$

\begin{enumerate}
  \setcounter{enumi}{6}
  \item \begin{CJK}{UTF8}{mj}证明不等式\end{CJK}
\end{enumerate}
$$
\sqrt{\frac{\pi}{2}\left(1-\mathrm{e}^{-\frac{u^{2}}{2}}\right)} \leq \int_{0}^{+\infty} \mathrm{e}^{-\frac{x^{2}}{2}} \mathrm{~d} x \leq \sqrt{\frac{\pi}{2}\left(1-\mathrm{e}^{-u^{2}}\right)(u>0)}
$$

\begin{enumerate}
  \setcounter{enumi}{7}
  \item \begin{CJK}{UTF8}{mj}已知\end{CJK} $2 x^{2}+y^{2}+z^{2}+2 x y-2 x-2 y+4 z-4=0$, \begin{CJK}{UTF8}{mj}求\end{CJK} $z=z(x, y)$ \begin{CJK}{UTF8}{mj}的极值\end{CJK}.

  \item \begin{CJK}{UTF8}{mj}已知\end{CJK} $\int_{-\infty}^{+\infty} f(x) \mathrm{d} x$ \begin{CJK}{UTF8}{mj}绝对收敛\end{CJK}, \begin{CJK}{UTF8}{mj}且\end{CJK} $f(x) \in \mathbb{C}(-\infty,+\infty)$, \begin{CJK}{UTF8}{mj}证明\end{CJK}:

\end{enumerate}
$$
g(x)=\int_{-\infty}^{+\infty} f(t) \sin (t x) \mathrm{d} t
$$
\begin{CJK}{UTF8}{mj}一致连续\end{CJK}.

\section{第24章 吉林大学}
\section{$24.12020$ 年数学分析考研真题}
\section{考生须知:}
\begin{enumerate}
  \item \begin{CJK}{UTF8}{mj}本试卷满分为\end{CJK} 150 \begin{CJK}{UTF8}{mj}分\end{CJK}, \begin{CJK}{UTF8}{mj}全部考试时间总计\end{CJK} 180 \begin{CJK}{UTF8}{mj}分钟\end{CJK};

  \item \begin{CJK}{UTF8}{mj}所有答案必须写在答题纸上\end{CJK}, \begin{CJK}{UTF8}{mj}写在试题纸上或草稿纸上一律无效\end{CJK}。

  \item \begin{CJK}{UTF8}{mj}计算题\end{CJK}

\end{enumerate}
(1). \begin{CJK}{UTF8}{mj}求极限\end{CJK} $\lim _{x \rightarrow 0} \frac{\mathrm{e}-(1+x)^{\frac{1}{x}}}{x}$

(2). \begin{CJK}{UTF8}{mj}求极限\end{CJK} $\lim _{n \rightarrow \infty}(n !)^{\frac{1}{n^{2}}}$

(3). \begin{CJK}{UTF8}{mj}求不定积分\end{CJK} $\int \frac{\mathrm{d} x}{(\sqrt{x}+1)(x+3)}$

(4). \begin{CJK}{UTF8}{mj}求定积分\end{CJK} $\int_{0}^{\frac{\pi}{2}} \frac{\sin x}{\sin x+\cos x} \mathrm{~d} x$

$(5)$. \begin{CJK}{UTF8}{mj}设\end{CJK} $u(x, y), v=v(x, y)$ \begin{CJK}{UTF8}{mj}是由方程组\end{CJK}
$$
\left\{\begin{array}{l}
3 x=3 \mathrm{e}^{v}+u^{3} \\
3 y=3 \mathrm{e}^{u}+v^{3}
\end{array}\right.
$$
\begin{CJK}{UTF8}{mj}确定的隐函数\end{CJK}, \begin{CJK}{UTF8}{mj}其中\end{CJK} $\mathrm{e}^{u+v} \neq u^{2} v^{2}$, \begin{CJK}{UTF8}{mj}求\end{CJK} $\frac{\partial u}{\partial x}, \frac{\partial u}{\partial y}, \frac{\partial v}{\partial x}, \frac{\partial v}{\partial y}$

(6). \begin{CJK}{UTF8}{mj}求极限\end{CJK}
$$
\lim _{n \rightarrow \infty} \frac{\ln n}{\ln \left(1^{2019}+2^{2019}+\cdots+n^{2019}\right)}
$$
(7). \begin{CJK}{UTF8}{mj}求数项级数\end{CJK} $\sum_{n=1}^{\infty} \frac{1}{n(2 n+1)}$ \begin{CJK}{UTF8}{mj}的和\end{CJK}.

(8). \begin{CJK}{UTF8}{mj}谋\end{CJK}
$$
f(t)=\iiint_{x^{2}+y^{2}+z^{2} \leq t^{2}} \mathrm{e}^{x^{2}+y^{2}+z^{2}} \mathrm{~d} x \mathrm{~d} y \mathrm{~d} z, t>0
$$
\begin{CJK}{UTF8}{mj}求\end{CJK} $f^{\prime}(1)$. $(9)$. \begin{CJK}{UTF8}{mj}计算第二型曲线\end{CJK}
$$
I=\int_{\Gamma}\left(2 x y^{3}-y^{2} \cos x\right) \mathrm{d} x+\left(1-2 y \sin x+3 x^{2} y^{2}\right) \mathrm{d} y
$$
\begin{CJK}{UTF8}{mj}其中\end{CJK} $\Gamma$ \begin{CJK}{UTF8}{mj}是抛物线\end{CJK} $2 x=\pi y^{2}$ \begin{CJK}{UTF8}{mj}上从\end{CJK} $(0,0)$ \begin{CJK}{UTF8}{mj}到\end{CJK} $\left(\frac{\pi}{2}, 1\right)$ \begin{CJK}{UTF8}{mj}的那一段\end{CJK}.

\begin{enumerate}
  \setcounter{enumi}{2}
  \item \begin{CJK}{UTF8}{mj}用\end{CJK} $\varepsilon-N$ \begin{CJK}{UTF8}{mj}语言证明\end{CJK}: $\lim _{n \rightarrow \infty} \sqrt[n]{2 n+1}=1$

  \item \begin{CJK}{UTF8}{mj}判断数项级数\end{CJK} $\sum_{n=1}^{\infty} \frac{\sin n}{\sqrt{n}}$ \begin{CJK}{UTF8}{mj}的敛散性\end{CJK}.

  \item \begin{CJK}{UTF8}{mj}求函数\end{CJK} $f(x, y, z)=x^{2}+y^{2}+2 z^{2}$ \begin{CJK}{UTF8}{mj}在区域\end{CJK} $D=\{(x, y, z) \||x|+|y|+|z| \leq 1\}$ \begin{CJK}{UTF8}{mj}上的最值\end{CJK}.

  \item \begin{CJK}{UTF8}{mj}设\end{CJK}

\end{enumerate}
$$
f(x, y)= \begin{cases}\frac{\ln (1+x y)}{x-\sin y}, & x \neq \sin y \\ 0, & x=\sin y\end{cases}
$$
\begin{CJK}{UTF8}{mj}试计算\end{CJK} $\lim _{x \rightarrow 0} \lim _{y \rightarrow 0} f(x, y)$ \begin{CJK}{UTF8}{mj}和\end{CJK} $\lim _{y \rightarrow 0} \lim _{x \rightarrow 0} f(x, y)$, \begin{CJK}{UTF8}{mj}并判断\end{CJK} $\lim _{(x, y) \rightarrow(0,0)} f(x, y)$ \begin{CJK}{UTF8}{mj}是否存在\end{CJK}?

\begin{enumerate}
  \setcounter{enumi}{6}
  \item \begin{CJK}{UTF8}{mj}设函数\end{CJK} $f(x)$ \begin{CJK}{UTF8}{mj}在\end{CJK} $(0,1)$ \begin{CJK}{UTF8}{mj}内可微\end{CJK}, \begin{CJK}{UTF8}{mj}且在\end{CJK} $(0,1)$ \begin{CJK}{UTF8}{mj}内恒有\end{CJK} $\ln x \leq \frac{f(x)}{x} \leq 0$, \begin{CJK}{UTF8}{mj}试证\end{CJK}: \begin{CJK}{UTF8}{mj}存在\end{CJK} $\xi \in(0,1)$, \begin{CJK}{UTF8}{mj}使得\end{CJK}
\end{enumerate}
$$
f^{\prime}(\xi)=1+\ln \xi
$$

\begin{enumerate}
  \setcounter{enumi}{7}
  \item \begin{CJK}{UTF8}{mj}证明\end{CJK}: \begin{CJK}{UTF8}{mj}方程\end{CJK} $y=x+\frac{1}{2} \arctan$ \begin{CJK}{UTF8}{mj}在\end{CJK} $\mathbb{R}^{2}$ \begin{CJK}{UTF8}{mj}内存在唯一连续解\end{CJK} $y=y(x)$.
\end{enumerate}
\section{第 25 章 重庆大学}
\section{$25.12020$ 年数学分析考研真题}
\section{考生须知:}
\begin{enumerate}
  \item \begin{CJK}{UTF8}{mj}本试卷满分为\end{CJK} 150 \begin{CJK}{UTF8}{mj}分\end{CJK}, \begin{CJK}{UTF8}{mj}全部考试时间总计\end{CJK} 180 \begin{CJK}{UTF8}{mj}分钟\end{CJK};

  \item \begin{CJK}{UTF8}{mj}所有答案必须写在答题纸上\end{CJK}, \begin{CJK}{UTF8}{mj}写在试题纸上或草稿纸上一律无效\end{CJK}。

  \item \begin{CJK}{UTF8}{mj}计算题\end{CJK}\\
(1). \begin{CJK}{UTF8}{mj}求极限\end{CJK} $\lim _{n \rightarrow \infty} \frac{\cos \frac{\pi}{n}}{n+\frac{1}{2}}+\frac{\cos \frac{\pi}{n}}{n+\frac{1}{2^{2}}}+\cdots$\\
(2). \begin{CJK}{UTF8}{mj}求极限\end{CJK} $\lim _{x \rightarrow 0} \frac{(\sin x-\sin (\sin x)) \tan x}{e^{x^{4}}-1}$\\
(3). \begin{CJK}{UTF8}{mj}求极限\end{CJK} $\lim _{n \rightarrow \infty} \tan ^{n}\left(\frac{\pi}{4}+\frac{1}{n}\right)$

\end{enumerate}
2 . \begin{CJK}{UTF8}{mj}设\end{CJK} $x_{1} \in(0,1)$, \begin{CJK}{UTF8}{mj}且\end{CJK} $x_{n+1}=\arctan x_{n}$

(1). \begin{CJK}{UTF8}{mj}证明\end{CJK}: $\left\{x_{n}\right\}$ \begin{CJK}{UTF8}{mj}极限存在\end{CJK}, \begin{CJK}{UTF8}{mj}并求极限\end{CJK}.

(2). \begin{CJK}{UTF8}{mj}求\end{CJK} $\lim _{n \rightarrow \infty}\left(\frac{x_{n}}{x_{n+1}}\right)^{\frac{1}{x_{n}^{2}}+1}$

\begin{enumerate}
  \setcounter{enumi}{3}
  \item \begin{CJK}{UTF8}{mj}判䉼\end{CJK}
\end{enumerate}
$$
f(x)= \begin{cases}\sin (\pi x), & x \text { 为有理数 } \\ 0, & x \text { 为无理数 }\end{cases}
$$
\begin{CJK}{UTF8}{mj}不连续点及其间断类型\end{CJK}.

\begin{enumerate}
  \setcounter{enumi}{4}
  \item \begin{CJK}{UTF8}{mj}设正项级数\end{CJK} $\sum_{n=1}^{\infty} a_{n}$ \begin{CJK}{UTF8}{mj}收敛于\end{CJK} $S$, \begin{CJK}{UTF8}{mj}求\end{CJK}\\
(1). $\lim _{n \rightarrow \infty} \frac{n a_{1}+(n-1) a_{2}+\cdots+a_{n}}{n}$\\
(2). $\lim _{n \rightarrow \infty} \frac{a_{1}+2 a_{2}+\cdots+n a_{n}}{n}$

  \item \begin{CJK}{UTF8}{mj}设\end{CJK} $f(x)$ \begin{CJK}{UTF8}{mj}是\end{CJK} $[0,+\infty)$ \begin{CJK}{UTF8}{mj}上的单调递减函数\end{CJK}, \begin{CJK}{UTF8}{mj}且\end{CJK} $\int_{0}^{+\infty} f(x) \mathrm{d} x<+\infty$, \begin{CJK}{UTF8}{mj}试证\end{CJK}: $\lim _{x \rightarrow+\infty} x f(x)=0$. 6. \begin{CJK}{UTF8}{mj}设\end{CJK} $\left\{f_{n}(x)\right\}$ \begin{CJK}{UTF8}{mj}中每一项\end{CJK} $f_{n}(x)$ \begin{CJK}{UTF8}{mj}都是\end{CJK} $[a, b]$ \begin{CJK}{UTF8}{mj}上的单调函数\end{CJK}, \begin{CJK}{UTF8}{mj}证明\end{CJK}: \begin{CJK}{UTF8}{mj}若每一项\end{CJK} $f_{n}(x)$ \begin{CJK}{UTF8}{mj}的单调性相同\end{CJK}, \begin{CJK}{UTF8}{mj}且\end{CJK} $\sum_{n=1}^{\infty} f_{n}(a)$ \begin{CJK}{UTF8}{mj}与\end{CJK} $\sum_{n=1}^{\infty} f_{n}(b)$ \begin{CJK}{UTF8}{mj}收玫\end{CJK}, \begin{CJK}{UTF8}{mj}则\end{CJK} $\left\{f_{n}(x)\right\}$ \begin{CJK}{UTF8}{mj}在\end{CJK} $[a, b]$ \begin{CJK}{UTF8}{mj}上一致收玫\end{CJK}.

  \item \begin{CJK}{UTF8}{mj}计算\end{CJK}

\end{enumerate}
$$
\sum_{n=1}^{\infty}(-1)^{n-1}\left(1+\frac{1}{n(n-2)}\right) x^{n}
$$
\begin{CJK}{UTF8}{mj}收玫区间与和函数\end{CJK} $f(x)$.

\begin{enumerate}
  \setcounter{enumi}{8}
  \item \begin{CJK}{UTF8}{mj}计算第二型曲面积分\end{CJK}
\end{enumerate}
$$
\iint_{S} z \mathrm{~d} x \mathrm{~d} y
$$

\begin{enumerate}
  \setcounter{enumi}{9}
  \item \begin{CJK}{UTF8}{mj}计算第一型曲线积分\end{CJK}
\end{enumerate}
$$
\int_{L} x^{2} \mathrm{~d} S
$$
\begin{CJK}{UTF8}{mj}其中\end{CJK} $L$ \begin{CJK}{UTF8}{mj}是\end{CJK} $x^{2}+y^{2}+z^{2}=a^{2}$ \begin{CJK}{UTF8}{mj}与\end{CJK} $x+y+z=0$ \begin{CJK}{UTF8}{mj}的交线\end{CJK}.

\begin{enumerate}
  \setcounter{enumi}{10}
  \item \begin{CJK}{UTF8}{mj}设\end{CJK} $f(x)$ \begin{CJK}{UTF8}{mj}是\end{CJK} $[0,1]$ \begin{CJK}{UTF8}{mj}上的连续函数\end{CJK}, \begin{CJK}{UTF8}{mj}对\end{CJK} $\forall \varepsilon>0, \exists \delta>0$, \begin{CJK}{UTF8}{mj}当\end{CJK} $x \in[0,1],\left|x-x_{0}\right|<\delta$ \begin{CJK}{UTF8}{mj}时\end{CJK}, \begin{CJK}{UTF8}{mj}有\end{CJK}
\end{enumerate}
$$
f(x)<f\left(x_{0}\right)+\varepsilon
$$
\begin{CJK}{UTF8}{mj}证明\end{CJK}: \begin{CJK}{UTF8}{mj}函数\end{CJK} $f(x)$ \begin{CJK}{UTF8}{mj}在\end{CJK} $[0,1]$ \begin{CJK}{UTF8}{mj}上有上界\end{CJK}, \begin{CJK}{UTF8}{mj}且能取到最大值\end{CJK}.

\section{第 26 章 东南大学}
\section{$26.12020$ 年数学分析考研真题}
\section{考生须知:}
\begin{enumerate}
  \item \begin{CJK}{UTF8}{mj}本试卷满分为\end{CJK} 150 \begin{CJK}{UTF8}{mj}分\end{CJK}, \begin{CJK}{UTF8}{mj}全部考试时间总计\end{CJK} 180 \begin{CJK}{UTF8}{mj}分钟\end{CJK};

  \item \begin{CJK}{UTF8}{mj}所有答案必须写在答题纸上\end{CJK}, \begin{CJK}{UTF8}{mj}写在试题纸上或草稿纸上一律无效\end{CJK}。

  \item \begin{CJK}{UTF8}{mj}计算题\end{CJK} $(7 \times 7=49$ \begin{CJK}{UTF8}{mj}分\end{CJK} $)$

\end{enumerate}
(1). \begin{CJK}{UTF8}{mj}计算极限\end{CJK}

(2). \begin{CJK}{UTF8}{mj}求极限\end{CJK}

\includegraphics[max width=\textwidth]{2022_04_18_7db0708508f26638f054g-085}

(3). \begin{CJK}{UTF8}{mj}计算三重积分\end{CJK}
$$
\iiint_{\Omega} x^{2} \sqrt{x^{2}+y^{2}} \mathrm{~d} x \mathrm{~d} y \mathrm{~d} z
$$
\begin{CJK}{UTF8}{mj}其中有界区域\end{CJK} $\Omega$ \begin{CJK}{UTF8}{mj}由\end{CJK} $z=\sqrt{x^{2}+y^{2}}$ \begin{CJK}{UTF8}{mj}和\end{CJK} $z=x^{2}+y^{2}$ \begin{CJK}{UTF8}{mj}围成\end{CJK}.

(4). \begin{CJK}{UTF8}{mj}计算\end{CJK}
$$
\int_{0}^{1} \frac{\mathrm{d} x}{\sqrt{1-x^{4}}} \int_{0}^{1} \frac{x^{2} \mathrm{~d} x}{\sqrt{1-x^{4}}}
$$
(5). \begin{CJK}{UTF8}{mj}无\end{CJK}

(6). \begin{CJK}{UTF8}{mj}计算\end{CJK}
$$
\int_{0}^{+\infty} \mathrm{e}^{-p x} \frac{\sin b x-\sin a x}{x} \mathrm{~d} x, \quad p>0, b>a>0
$$
(7). \begin{CJK}{UTF8}{mj}求级数\end{CJK}
$$
\sum_{n=0}^{\infty} a_{n}
$$
\begin{CJK}{UTF8}{mj}其中\end{CJK} $a_{n}=\int_{0}^{\frac{\pi}{4}} \sin ^{n} x \cos x \mathrm{~d} x$ 2 . \begin{CJK}{UTF8}{mj}解答题\end{CJK} $(5 \times 10=50$ \begin{CJK}{UTF8}{mj}分\end{CJK} $)$

(1). \begin{CJK}{UTF8}{mj}设\end{CJK} $\left\{a_{n}\right\}$ \begin{CJK}{UTF8}{mj}收敛\end{CJK}, $\left\{b_{n}\right\}$ \begin{CJK}{UTF8}{mj}有界发散\end{CJK}, \begin{CJK}{UTF8}{mj}求\end{CJK} $\left\{a_{n} b_{n}\right\}$ \begin{CJK}{UTF8}{mj}发散的充要条件\end{CJK}.

$(2)$. \begin{CJK}{UTF8}{mj}设\end{CJK} $g(x)$ \begin{CJK}{UTF8}{mj}二阶连续可导\end{CJK}, \begin{CJK}{UTF8}{mj}且\end{CJK} $g(0)=1$, \begin{CJK}{UTF8}{mj}令\end{CJK}
$$
f(x)= \begin{cases}g^{\prime}(0), & x=0 \\ \frac{g(x)-\cos x}{x}, & x \neq 0\end{cases}
$$
\begin{CJK}{UTF8}{mj}讨论\end{CJK}

I. $f(x)$ \begin{CJK}{UTF8}{mj}在\end{CJK} $x=0$ \begin{CJK}{UTF8}{mj}的连续性\end{CJK}.

II. $f^{\prime}(x)$ \begin{CJK}{UTF8}{mj}的表达式以及\end{CJK} $f^{\prime}(x)$ \begin{CJK}{UTF8}{mj}在\end{CJK} $x=0$ \begin{CJK}{UTF8}{mj}处的连续性\end{CJK}.

(3). \begin{CJK}{UTF8}{mj}设\end{CJK} $f(x)$ \begin{CJK}{UTF8}{mj}在\end{CJK} $[a,+\infty)$ \begin{CJK}{UTF8}{mj}上一致收玫\end{CJK}, $\varphi(x)$ \begin{CJK}{UTF8}{mj}在\end{CJK} $[a,+\infty)$ \begin{CJK}{UTF8}{mj}上连续\end{CJK}, \begin{CJK}{UTF8}{mj}且\end{CJK}
$$
\lim _{x \rightarrow+\infty} f(x)-\varphi(x)=0
$$
\begin{CJK}{UTF8}{mj}那么\end{CJK} $\varphi(x)$ \begin{CJK}{UTF8}{mj}在\end{CJK} $[a,+\infty)$ \begin{CJK}{UTF8}{mj}上一致连续吗\end{CJK}?

(4). \begin{CJK}{UTF8}{mj}设级数\end{CJK} $\sum_{n=1}^{\infty} a_{n}$ \begin{CJK}{UTF8}{mj}收敛\end{CJK}, \begin{CJK}{UTF8}{mj}同时\end{CJK} $\lim _{n \rightarrow \infty} b_{n}=1$

I. \begin{CJK}{UTF8}{mj}若\end{CJK} $a_{n} \geq 0$, \begin{CJK}{UTF8}{mj}请问\end{CJK} $\sum_{n=1}^{\infty} a_{n} b_{n}$ \begin{CJK}{UTF8}{mj}是否一定收敛\end{CJK}?

II. $\sum_{n=1}^{\infty} a_{n} b_{n}$ \begin{CJK}{UTF8}{mj}是否一定收敛\end{CJK}? \begin{CJK}{UTF8}{mj}若是\end{CJK}, \begin{CJK}{UTF8}{mj}给出证明\end{CJK}; \begin{CJK}{UTF8}{mj}若不是\end{CJK}, \begin{CJK}{UTF8}{mj}给出反例\end{CJK}.

$(5)$. \begin{CJK}{UTF8}{mj}求原点到\end{CJK} $x^{2}+y^{2}=4 z$ \begin{CJK}{UTF8}{mj}与\end{CJK} $x^{2}+x y+y^{2}=4$ \begin{CJK}{UTF8}{mj}交线的最近距离和最远距离\end{CJK}.

\begin{enumerate}
  \setcounter{enumi}{3}
  \item \begin{CJK}{UTF8}{mj}证明题\end{CJK} $(4 \times 10+11=51$ \begin{CJK}{UTF8}{mj}分\end{CJK} $)$
\end{enumerate}
(1). \begin{CJK}{UTF8}{mj}设\end{CJK} $f_{n}(x)=\sin x+\sin ^{2} x+\cdots+\sin ^{n} x, n \geqslant 2$

I. \begin{CJK}{UTF8}{mj}证明\end{CJK} $f_{n}(x)=1$ \begin{CJK}{UTF8}{mj}在\end{CJK} $\left(0, \frac{\pi}{2}\right)$ \begin{CJK}{UTF8}{mj}上必有解\end{CJK} $x_{n}$.

II. \begin{CJK}{UTF8}{mj}求\end{CJK} $\lim _{n \rightarrow \infty} x_{n}$.

$(2)$. \begin{CJK}{UTF8}{mj}设\end{CJK} $f(x)$ \begin{CJK}{UTF8}{mj}是非负连续函数\end{CJK}, \begin{CJK}{UTF8}{mj}同时\end{CJK}
$$
f(x) \leqslant \int_{0}^{x} f^{2}(t) \mathrm{d} t, \quad x \in \mathbb{R}
$$
\begin{CJK}{UTF8}{mj}证明\end{CJK}: $f(x) \equiv 0, x \in \mathbb{R}$.

$(3)$. \begin{CJK}{UTF8}{mj}设\end{CJK} $f(x)$ \begin{CJK}{UTF8}{mj}在\end{CJK} $[0,1]$ \begin{CJK}{UTF8}{mj}上二次可微\end{CJK}, $f(0)=f(1)=0, \min _{x \in[0,1]} f(x)=-1$, \begin{CJK}{UTF8}{mj}证明\end{CJK}:
$$
\max _{x \in[0,1]} f^{\prime \prime}(x) \geqslant 8
$$
(4). \begin{CJK}{UTF8}{mj}无\end{CJK}

(5). \begin{CJK}{UTF8}{mj}叙述有限覆盖定理\end{CJK}, \begin{CJK}{UTF8}{mj}并用有限覆盖定理证明有界数列必有收玫子列\end{CJK}. \begin{CJK}{UTF8}{mj}了两晩\end{CJK}, \begin{CJK}{UTF8}{mj}但在\end{CJK} 18 \begin{CJK}{UTF8}{mj}年武汉游第二次与他遇见之后\end{CJK}, \begin{CJK}{UTF8}{mj}我回南昌老家后\end{CJK}, \begin{CJK}{UTF8}{mj}却不幸听他高三同学说\end{CJK} \begin{CJK}{UTF8}{mj}郳胡涛得了抑郁症\end{CJK}, \begin{CJK}{UTF8}{mj}从此就再无他的踪迹\end{CJK}! \begin{CJK}{UTF8}{mj}现在每每想到\end{CJK} 16-18 \begin{CJK}{UTF8}{mj}年我们认识那三年的点滴\end{CJK} \begin{CJK}{UTF8}{mj}回忆\end{CJK}, \begin{CJK}{UTF8}{mj}就很是难受\end{CJK}.

\begin{CJK}{UTF8}{mj}下面是\end{CJK} 2021 \begin{CJK}{UTF8}{mj}年武汉大学数学分析与高等代数的解答\end{CJK}! \begin{CJK}{UTF8}{mj}也希望我的解答能给你有所启\end{CJK} \begin{CJK}{UTF8}{mj}发\end{CJK}, \begin{CJK}{UTF8}{mj}谢谢\end{CJK}!

\begin{enumerate}
  \item \begin{CJK}{UTF8}{mj}计算极限\end{CJK} $\lim _{x \rightarrow 0}(\cos x)^{\frac{1}{\sin ^{2} x}}$
\end{enumerate}
\begin{CJK}{UTF8}{mj}解\end{CJK}: \begin{CJK}{UTF8}{mj}取对数再等价洛必达\end{CJK}
$$
\begin{aligned}
\lim _{x \rightarrow 0}(\cos x)^{\frac{1}{\sin ^{2} x}} &=\exp \lim _{x \rightarrow 0} \frac{\ln \cos x}{\sin ^{2} x} \\
&=\exp \lim _{x \rightarrow 0} \frac{\ln \cos x}{x^{2}} \\
&=\exp \lim _{x \rightarrow 0} \frac{-\frac{\sin x}{\cos x}}{2 x} \\
&=\exp \left(-\frac{1}{2}\right)
\end{aligned}
$$
2 . \begin{CJK}{UTF8}{mj}求\end{CJK} $\lim _{n \rightarrow \infty} \frac{1}{\sqrt[n]{n !}}$

\begin{CJK}{UTF8}{mj}解\end{CJK}: \begin{CJK}{UTF8}{mj}由斯特林公式\end{CJK}
$$
n !=\sqrt{2 \pi n}\left(\frac{n}{\mathrm{e}}\right)^{n} \mathrm{e}^{\frac{\theta_{\mathrm{n}}}{12 n}}\left(0<\theta_{n}<1\right)
$$
\begin{CJK}{UTF8}{mj}也就\end{CJK}
$$
n ! \sim\left(\frac{n}{\mathrm{e}}\right)^{n} \sqrt{2 \pi n}
$$
\begin{CJK}{UTF8}{mj}即有\end{CJK}
$$
\begin{gathered}
\frac{1}{\sqrt[n]{n !}} \sim \frac{1}{\frac{n}{\mathrm{e}} \sqrt[n]{2 \pi n}}=\frac{\mathrm{e}}{n \sqrt[n]{2 \pi n}} \\
\Rightarrow \lim _{n \rightarrow \infty} \frac{1}{\sqrt[n]{n !}}=\lim _{n \rightarrow \infty} \frac{\mathrm{e}}{n \sqrt[n]{2 \pi n}}=0
\end{gathered}
$$

\begin{enumerate}
  \setcounter{enumi}{3}
  \item \begin{CJK}{UTF8}{mj}设函数\end{CJK}
\end{enumerate}
$$
f(x, y)=\left\{\begin{array}{cl}
\frac{1-\mathrm{e}^{x\left(x^{2}+y^{2}\right)}}{x^{2}+y^{2}}, & (x, y) \neq(0,0) \\
0, & (x, y)=(0,0)
\end{array}\right.
$$
\begin{CJK}{UTF8}{mj}试证\end{CJK}: $f(x, y)$ \begin{CJK}{UTF8}{mj}在\end{CJK} $(0,0)$ \begin{CJK}{UTF8}{mj}处连续性与可微性\end{CJK}.

\begin{CJK}{UTF8}{mj}证明\end{CJK}: \begin{CJK}{UTF8}{mj}对\end{CJK} $\forall \varepsilon>0, \exists \delta=\varepsilon$, \begin{CJK}{UTF8}{mj}当\end{CJK} $|x|<\delta$ \begin{CJK}{UTF8}{mj}时\end{CJK}, \begin{CJK}{UTF8}{mj}则有\end{CJK}
$$
|f(x, y)-f(0,0)|=\left|\frac{1-\mathrm{e}^{x\left(x^{2}+y^{2}\right)}}{x^{2}+y^{2}}\right| \leq|x|<\varepsilon
$$
\begin{CJK}{UTF8}{mj}故\end{CJK}
$$
\lim _{(x, y) \rightarrow(0.0)} f(x, y)=f(0,0)=0
$$
\begin{CJK}{UTF8}{mj}即\end{CJK} $f(x, y)$ \begin{CJK}{UTF8}{mj}在点\end{CJK} $(0,0)$ \begin{CJK}{UTF8}{mj}处连续\end{CJK}, \begin{CJK}{UTF8}{mj}又有\end{CJK}
$$
\begin{aligned}
f_{x}^{\prime}(0,0) &=\lim _{x \rightarrow 0} \frac{f(x, 0)-f(0,0)}{x-0} \\
&=\lim _{x \rightarrow 0} \frac{1-\mathrm{e}^{x^{3}}}{x^{3}}=-1 \\
f_{y}^{\prime}(0,0) &=\lim _{y \rightarrow 0} \frac{f(0, y)-f(0,0)}{y-0} \\
&=\lim _{y \rightarrow 0} \frac{1-1}{y^{3}}=0
\end{aligned}
$$
\begin{CJK}{UTF8}{mj}所以\end{CJK}

\begin{CJK}{UTF8}{mj}故\end{CJK}
$$
\Delta z=f(x, y)-f(0,0)-f_{x}^{\prime}(0,0) x-f_{y}^{\prime}(0,0) y=\frac{-\mathrm{e}^{x}\left(x^{2}+y^{2}\right)}{x^{2}+y^{2}}+x
$$
$$
\begin{aligned}
\lim _{\rho \rightarrow 0} \frac{\Delta z}{\rho} &=\lim _{\rho \rightarrow 0} \frac{\frac{1-\mathrm{e}^{x\left(x^{2}+y^{2}\right)}}{x^{2}+y^{2}}+x}{\sqrt{x^{2}+y^{2}}} \\
&=\lim _{(x, y) \rightarrow(0,0)} \frac{1-\mathrm{e}^{x\left(x^{2}+y^{2}\right)}+x\left(x^{2}+y^{2}\right)}{/\left(x^{2}+y^{2}\right)^{\frac{3}{2}}} \\
&=\lim _{r \rightarrow 0^{+}} \frac{1-\mathrm{e}^{r^{3} \cos \theta}+r^{3} \cos \theta}{r^{3}} \\
&=\lim _{r \rightarrow 0^{+}} \frac{r^{3} \cos \theta-r^{3} \cos \theta+o\left(r^{3}\right)}{r^{3}}=0
\end{aligned}
$$
\begin{CJK}{UTF8}{mj}故\end{CJK} $f(x, y)$ \begin{CJK}{UTF8}{mj}在点\end{CJK} $(0,0)$ \begin{CJK}{UTF8}{mj}处可微\end{CJK}.

\begin{enumerate}
  \setcounter{enumi}{4}
  \item \begin{CJK}{UTF8}{mj}利用条件极值方法证明\end{CJK}:
\end{enumerate}
$$
a b^{2} c^{3} \leqslant 108\left(\frac{a+b+c}{6}\right)^{6}
$$
\begin{CJK}{UTF8}{mj}证明\end{CJK}:\begin{CJK}{UTF8}{mj}令\end{CJK}
$$
L(x, y, z, \lambda)=x y^{2} z^{3}-\lambda\left(x^{2}+y^{2}+z^{2}-6 R^{2}\right)
$$
\begin{CJK}{UTF8}{mj}由\end{CJK} $L_{\lambda}=0$ \begin{CJK}{UTF8}{mj}得\end{CJK} $x^{2}+y^{2}+z^{2}=6 R^{2}$, \begin{CJK}{UTF8}{mj}再由\end{CJK}
$$
L_{x}=0, L_{x}=0, L_{z}=0
$$
\begin{CJK}{UTF8}{mj}可知\end{CJK} $x y^{2} y^{3}=2 \lambda R^{2}$ \begin{CJK}{UTF8}{mj}以及\end{CJK}
$$
x^{2}=R^{2}, y^{2}=2 R^{2}, z^{2}=3 R^{2}
$$
\begin{CJK}{UTF8}{mj}于是\end{CJK}
$$
x y^{2} z^{3} \leqslant 6 \sqrt{3} R^{6}
$$
\begin{CJK}{UTF8}{mj}因此\end{CJK}
$$
x y^{2} z^{3} \leqslant 6 \sqrt{3}\left(\frac{x^{2}+y^{2}+z^{2}}{6}\right)^{3}
$$
\begin{CJK}{UTF8}{mj}再令\end{CJK} $a=x^{2}, b=y^{2}, c=z^{2}$, \begin{CJK}{UTF8}{mj}即证\end{CJK}
$$
a b^{2} c^{3} \leqslant 108\left(\frac{a+b+c}{6}\right)^{6}
$$

\begin{enumerate}
  \setcounter{enumi}{5}
  \item \begin{CJK}{UTF8}{mj}讨论级数\end{CJK} $u_{n}(x)=\sum_{n=1}^{\infty} \frac{\mathrm{e}^{-n x}}{n}$ \begin{CJK}{UTF8}{mj}收敛区间与一致收敛区间\end{CJK}, \begin{CJK}{UTF8}{mj}并证明\end{CJK}. \begin{CJK}{UTF8}{mj}证明\end{CJK}:\begin{CJK}{UTF8}{mj}显然\end{CJK}
\end{enumerate}
$$
u_{n}(x)=\sum_{n=1}^{\infty} \frac{\mathrm{e}^{-n x}}{n}\left\{\begin{array}{l}
\geq \frac{1}{n}, x \leq 0 \\
\leq \frac{1}{n^{2} x}, x>0
\end{array}\right.
$$
\begin{CJK}{UTF8}{mj}故由比较判别法知当且仅当\end{CJK} $x>0$ \begin{CJK}{UTF8}{mj}时级数\end{CJK} $u_{n}(x)$ \begin{CJK}{UTF8}{mj}收敛\end{CJK}, \begin{CJK}{UTF8}{mj}又\end{CJK}
$$
\lim _{n \rightarrow \infty} \sup _{x>0} \sum_{k=n+1}^{\infty} \frac{\mathrm{e}^{-k x}}{k} \geqslant \lim _{n \rightarrow \infty} \sum_{k=n+1}^{2 n} \frac{\mathrm{e}^{-k \cdot \frac{1}{2 n}}}{k} \geqslant \lim _{n \rightarrow \infty} \frac{1}{\mathrm{e}} \sum_{k=n+1}^{2 n} \frac{1}{k}=+\infty
$$
\begin{CJK}{UTF8}{mj}可知在\end{CJK} $(0, \infty)$ \begin{CJK}{UTF8}{mj}上非一致收敛\end{CJK}.

\begin{CJK}{UTF8}{mj}对\end{CJK} $[a, b] \subset(0,+\infty)$, \begin{CJK}{UTF8}{mj}则有\end{CJK}
$$
\sup _{x \geq a} \sum_{k=n+1}^{\infty} \frac{\mathrm{e}^{-k x}}{k} \leq \sum_{k=n+1}^{\infty} \frac{\mathrm{e}^{-k a}}{k}<\frac{1}{a} \sum_{k=n+1}^{\infty} \frac{1}{k^{2}}
$$
\begin{CJK}{UTF8}{mj}可知在\end{CJK} $[a, b]$ \begin{CJK}{UTF8}{mj}上一致收敛\end{CJK}, \begin{CJK}{UTF8}{mj}即在\end{CJK} $(0, \infty)$ \begin{CJK}{UTF8}{mj}上内闭一致收敛\end{CJK}

\begin{enumerate}
  \setcounter{enumi}{6}
  \item \begin{CJK}{UTF8}{mj}计算二重积分\end{CJK}
\end{enumerate}
$$
\iint_{D} \frac{x^{2}-y^{2}}{\sqrt{x+y+4}} \mathrm{~d} x \mathrm{~d} y
$$
\begin{CJK}{UTF8}{mj}其中\end{CJK} $D:|x|+|y| \leq 1$.

\begin{CJK}{UTF8}{mj}解\end{CJK}: \begin{CJK}{UTF8}{mj}令\end{CJK} $u=x+y, v=x-y$, \begin{CJK}{UTF8}{mj}则\end{CJK}
$$
\begin{aligned}
D &=\{(x, y):-1 \leqslant x+y \leqslant 1,-1 \leqslant x-y\\
&=\{(u, v):-1 \leqslant u \leqslant 1,-1 \leqslant v \leqslant 1\} \\
J^{-1} &=\frac{\partial(u, v)}{\partial(x, y)}=\left|\begin{array}{cc}
1 & 1 \\
1 & -1
\end{array}\right|=-2 \\
J &=-\frac{1}{2}
\end{aligned}
$$
\begin{CJK}{UTF8}{mj}即有\end{CJK}
$$
\begin{aligned}
I &=\iint_{D} \frac{x^{2}-y^{2}}{\sqrt{x+y+4}} \mathrm{~d} x \mathrm{~d} y \\
&=\frac{1}{2} \int_{-1}^{1} \frac{u \mathrm{~d} u}{\sqrt{u+4}} \int_{-1}^{1} v \mathrm{~d} v=0
\end{aligned}
$$
\begin{CJK}{UTF8}{mj}当然你还可以利用轮换对称性做\end{CJK}
$$
\iint_{h} \frac{x^{2}}{\sqrt{x+y+4}} \mathrm{~d} x \mathrm{~d} y=\iint_{D} \frac{y^{2}}{\sqrt{x+y+4}} \mathrm{~d} x \mathrm{~d} y
$$
\begin{CJK}{UTF8}{mj}可知\end{CJK}
$$
\iint_{D} \frac{x^{2}-y^{2}}{\sqrt{x+y+4}} \mathrm{~d} x \mathrm{~d} y=0
$$

\begin{enumerate}
  \setcounter{enumi}{7}
  \item \begin{CJK}{UTF8}{mj}计算第二型曲面积分\end{CJK}
\end{enumerate}
$$
\oiint_{\Sigma \text { 外 }} \frac{x \mathrm{~d} y \mathrm{~d} z+y \mathrm{~d} z \mathrm{~d} x+z \mathrm{~d} x \mathrm{~d} y}{\left(x^{2}+y^{2}+z^{2}\right)^{\frac{3}{2}}}
$$
\begin{CJK}{UTF8}{mj}其中\end{CJK} $\Sigma$ \begin{CJK}{UTF8}{mj}是\end{CJK}
$$
V=\{(x, y, z)|| x|\leq 2,| y|\leq 2,| z \mid \leq 2\}
$$
\begin{CJK}{UTF8}{mj}的曲面\end{CJK}.

\begin{CJK}{UTF8}{mj}解\end{CJK}: \begin{CJK}{UTF8}{mj}这题出自\end{CJK} 1984 \begin{CJK}{UTF8}{mj}年安徽大学数学分析真题\end{CJK}, \begin{CJK}{UTF8}{mj}因为想找到原题\end{CJK}, \begin{CJK}{UTF8}{mj}我特意翻了下安徽大学历年真题\end{CJK}, \begin{CJK}{UTF8}{mj}再不巧看到一本\end{CJK} 1984 \begin{CJK}{UTF8}{mj}经典研究生数学分析题解\end{CJK}.

\begin{CJK}{UTF8}{mj}由于曲面\end{CJK} $\Sigma$ \begin{CJK}{UTF8}{mj}是包含原点\end{CJK}, \begin{CJK}{UTF8}{mj}故肯定要挖掉个半径为\end{CJK} 1 \begin{CJK}{UTF8}{mj}的球体记为\end{CJK} $\Sigma_{1}: x^{2}+y^{2}+z^{2} \leq 1$, \begin{CJK}{UTF8}{mj}取外侧\end{CJK}, \begin{CJK}{UTF8}{mj}才\end{CJK} \begin{CJK}{UTF8}{mj}能构成一个不包含原点的曲面\end{CJK}, \begin{CJK}{UTF8}{mj}这一类题也同时就体现了曲面积分与曲面无关的优越性\end{CJK}.

\begin{CJK}{UTF8}{mj}由于\end{CJK}
$$
\left\{\begin{aligned}
P &=\frac{x}{\left(x^{2}+y^{2}+z^{2}\right)^{\frac{3}{2}}} \\
Q &=\frac{y^{2}}{\left(x^{2}+y^{2}+z^{2}\right)^{\frac{3}{2}}} \\
R &=\frac{P_{x}+Q_{y}+R_{z}=0}{\left(x^{2}+y^{2}+z^{2}\right)^{\frac{3}{2}}}
\end{aligned}\right.
$$
$$
\begin{aligned}
I &=\left(\iint_{\Sigma+\Sigma_{1}}+\iint_{\Sigma_{1}}\right) P \mathrm{~d} y \mathrm{~d} z+Q \mathrm{~d} z \mathrm{~d} x+R \mathrm{~d} x \mathrm{~d} x \mathrm{~d} x+R \mathrm{~d} x \mathrm{~d} y \\
&=\iint_{\Sigma_{1}} P \mathrm{~d} y \mathrm{~d} z+Q \mathrm{~d} z \mathrm{~d} z+y \mathrm{~d} z \mathrm{~d} x+z \mathrm{~d} x \mathrm{~d} y \\
&=\iint_{x^{2}+y^{2}+z^{2}=1} x \mathrm{~d} y \mathrm{~d} y \mathrm{~d} z \\
&=3 \quad \iiint_{x^{2}+y^{2}+z^{2} \leq 1} \mathrm{~d} x
\end{aligned}
$$
\begin{CJK}{UTF8}{mj}或者你用对称性做也行\end{CJK}
$$
\begin{aligned}
&\iint_{\Sigma_{1}} P \mathrm{~d} y \mathrm{~d} z+Q \mathrm{~d} z \mathrm{~d} x+R \mathrm{~d} x \mathrm{~d} y \\
&\frac{\text { 称性 }}{2} 3 \iint_{\Sigma_{1}} R \mathrm{~d} x \mathrm{~d} y \\
&=3 \iint_{\Sigma_{1}} \frac{z}{\left(x^{2}+y^{2}+z^{2}\right)^{\frac{3}{2}}} \mathrm{~d} x \mathrm{~d} y \\
&\left\{\begin{array}{l}
x=\sin \varphi \cos \theta \\
y=\sin \varphi \sin \theta \\
z=\cos \varphi
\end{array}\right. \\
&=4 \pi
\end{aligned}
$$

\begin{enumerate}
  \setcounter{enumi}{8}
  \item \begin{CJK}{UTF8}{mj}若\end{CJK} $f(x)$ \begin{CJK}{UTF8}{mj}在\end{CJK} $[0,2]$ \begin{CJK}{UTF8}{mj}连续可微\end{CJK}, \begin{CJK}{UTF8}{mj}且\end{CJK}
\end{enumerate}
$$
f(0)=f(2)=1,\left|f^{\prime}(x)\right| \leqslant 1
$$
\begin{CJK}{UTF8}{mj}试证\end{CJK}:
$$
\int_{0}^{2} f(x) \mathrm{d} x>1
$$
\begin{CJK}{UTF8}{mj}证明\end{CJK}: \begin{CJK}{UTF8}{mj}对\end{CJK} $\forall x \in[0,1], \exists \xi \in(0, x)$, \begin{CJK}{UTF8}{mj}有\end{CJK}
$$
\begin{gathered}
f(0)=f(x)-f^{\prime}(\xi) x \\
\Rightarrow f(x)=f(0)+f^{\prime}(\xi) x \geq 1-x
\end{gathered}
$$
$$
\begin{aligned}
& \Rightarrow f(x)=f(0)+f^{\prime}(\xi) x \geq 1-x
\end{aligned}
$$
\begin{CJK}{UTF8}{mj}同理对\end{CJK} $\forall x \in[1,2], \exists \eta \in(x, 2)$ \begin{CJK}{UTF8}{mj}也有\end{CJK}
$$
\begin{gathered}
f(2)=f(x)+f^{\prime}(\eta)(2-x) \\
\Rightarrow f(x)=f(2)-f^{\prime}(\eta)(2-x) \geq x-1
\end{gathered}
$$
\begin{CJK}{UTF8}{mj}因此\end{CJK}
$$
\begin{aligned}
\int_{0}^{2} f(x) \mathrm{d} x & \geqslant \int_{0}^{1}(1-x) \mathrm{d} x+\int_{1}^{2}(x-1) \mathrm{d} x \\
&=\frac{1}{2}+\frac{1}{2}=1
\end{aligned}
$$
\begin{CJK}{UTF8}{mj}而等号成立条件是当\end{CJK} $0 \leq x<1$, \begin{CJK}{UTF8}{mj}则\end{CJK} $f^{\prime}(x)=-1$; \begin{CJK}{UTF8}{mj}当\end{CJK} $1<x \leq 2$, \begin{CJK}{UTF8}{mj}则\end{CJK} $f^{\prime}(x)=1$, \begin{CJK}{UTF8}{mj}与题设\end{CJK} $f(x)$ \begin{CJK}{UTF8}{mj}连续\end{CJK} \begin{CJK}{UTF8}{mj}矛盾\end{CJK}, \begin{CJK}{UTF8}{mj}故\end{CJK}
$$
\int_{0}^{2} f(x) \mathrm{d} x>1
$$
9 . \begin{CJK}{UTF8}{mj}设\end{CJK}

\includegraphics[max width=\textwidth]{2022_04_18_7db0708508f26638f054g-092}

\begin{CJK}{UTF8}{mj}试证\end{CJK}:
$$
\lim _{n \rightarrow \infty} n\left(\ln 2-A_{n}\right)=\frac{1}{4}
$$
\begin{CJK}{UTF8}{mj}证明\end{CJK}: \begin{CJK}{UTF8}{mj}首先考虑到\end{CJK}
$$
\ln 2=\int_{0}^{1} \frac{1}{1+x} \mathrm{~d} x=\sum_{k=1}^{n} \int_{\frac{k-1}{n}}^{\frac{k}{n}} \frac{1}{1+x} \mathrm{~d} x
$$
\begin{CJK}{UTF8}{mj}微信公众号\end{CJK}: \begin{CJK}{UTF8}{mj}八一考研数学竞赛\end{CJK} \begin{CJK}{UTF8}{mj}令\end{CJK} $f(x)=\frac{1}{1+x}$ \begin{CJK}{UTF8}{mj}可得\end{CJK}
$$
\begin{aligned}
&\lim _{n \rightarrow \infty} n\left(\ln 2-A_{n}\right) \\
&=\lim _{n \rightarrow \infty} n\left(\ln 2-\frac{1}{n} \sum_{k=1}^{n} \frac{n}{n+k}\right) \\
&=\lim _{n \rightarrow \infty} n\left(\sum_{k=1}^{n} \int_{\frac{k-1}{n}}^{\frac{k}{n}} f(x)-f\left(\frac{k}{n}\right) \mathrm{d} x\right) \\
&=\lim _{n \rightarrow \infty} n\left(\sum_{k=1}^{n} \int_{k-1}^{\frac{k}{n}} f^{\prime}\left(x_{k}\right)\left(x-\frac{k}{n}\right) \mathrm{d} x\right) \\
&=-\frac{1}{2} \int_{0}^{1} f^{\prime}(x) \mathrm{d} x=\frac{1}{4}
\end{aligned}
$$

\begin{enumerate}
  \setcounter{enumi}{10}
  \item \begin{CJK}{UTF8}{mj}当\end{CJK} $p \geq 1$ \begin{CJK}{UTF8}{mj}时\end{CJK}, \begin{CJK}{UTF8}{mj}试证\end{CJK}:
\end{enumerate}
$$
\sum_{n=1}^{\infty} \frac{1}{(n+1) \sqrt[p]{n}}<p
$$
\begin{CJK}{UTF8}{mj}证明\end{CJK}:\begin{CJK}{UTF8}{mj}由于\end{CJK}
$$
\begin{aligned}
\frac{1}{(n+1) \sqrt[p]{n}} &=\frac{n^{1-\frac{1}{p}}}{n(n+1)} \\
&=n^{\frac{p-1}{p}}\left(\frac{1}{n}-\frac{1}{n+1}\right) \\
&=n^{\frac{p-1}{p}}\left[\left(\frac{1}{\sqrt[p]{n}}\right)^{p}-\left(\frac{1}{\sqrt[p]{n+1}}\right)^{p}\right] \\
& \stackrel{\text { 拉式 }}{=} n^{\frac{p-1}{p}} \cdot p \xi_{n}^{p-1}\left(\frac{1}{\sqrt[p]{n}}-\frac{1}{\sqrt[p]{n+1}}\right) \\
&=\left(\sqrt[p]{n} \xi_{n}\right)^{p-1} \cdot p\left(\frac{1}{\sqrt[p]{n}}-\frac{1}{\sqrt[p]{n+1}}\right) \\
&<p\left(\frac{1}{\sqrt[p]{n}}-\frac{1}{\sqrt[p]{n+1}}\right)
\end{aligned}
$$
\begin{CJK}{UTF8}{mj}其中\end{CJK} $\xi_{n} \in\left(\frac{1}{\sqrt[p]{n+1}}, \frac{1}{\sqrt[p]{n}}\right)$.

\begin{CJK}{UTF8}{mj}所以\end{CJK}
$$
\sum_{n=1}^{\infty} \frac{1}{(n+1) \sqrt[p]{n}}<p \sum_{n=1}^{\infty}\left(\frac{1}{\sqrt[p]{n}}-\frac{1}{\sqrt[p]{n+1}}\right)=p
$$

\begin{enumerate}
  \setcounter{enumi}{11}
  \item \begin{CJK}{UTF8}{mj}若\end{CJK} $f(x)$ \begin{CJK}{UTF8}{mj}是\end{CJK} $[a, b]$ \begin{CJK}{UTF8}{mj}上的凸函数\end{CJK}, \begin{CJK}{UTF8}{mj}则\end{CJK} $f(x)$ \begin{CJK}{UTF8}{mj}在\end{CJK} $(a, b)$ \begin{CJK}{UTF8}{mj}内左右导数存在\end{CJK}.
\end{enumerate}
\begin{CJK}{UTF8}{mj}证明\end{CJK}: \begin{CJK}{UTF8}{mj}设对\end{CJK} $\forall x_{0} \in(a, b)$, \begin{CJK}{UTF8}{mj}以及\end{CJK}
$$
\forall 0<h_{1}<h_{2}<b-x_{0}
$$
\begin{CJK}{UTF8}{mj}由于\end{CJK} $f(x)$ \begin{CJK}{UTF8}{mj}是\end{CJK} $[a, b]$ \begin{CJK}{UTF8}{mj}上的凸函数\end{CJK}, \begin{CJK}{UTF8}{mj}故\end{CJK}
$$
\frac{f\left(x_{0}+h_{2}\right)-f\left(x_{0}\right)}{h_{2}} \leqslant \frac{f\left(x_{0}+h_{1}\right)-f\left(x_{0}\right)}{h_{1}}
$$
\begin{CJK}{UTF8}{mj}即\end{CJK}
$$
\frac{f\left(x_{0}+h\right)-f\left(x_{0}\right)}{h}
$$
\begin{CJK}{UTF8}{mj}是\end{CJK} $0<h<b-x_{0}$ \begin{CJK}{UTF8}{mj}的单调递增函数\end{CJK}, \begin{CJK}{UTF8}{mj}当\end{CJK} $a<x_{1}<x_{0}$ \begin{CJK}{UTF8}{mj}时\end{CJK}, \begin{CJK}{UTF8}{mj}有\end{CJK}
$$
\frac{f\left(x_{0}+h\right)-f\left(x_{0}\right)}{h} \geq \frac{f\left(x_{0}\right)-f\left(x_{1}\right)}{x_{0}-x_{1}}, \forall 0<h<b-x_{0}
$$
\begin{CJK}{UTF8}{mj}因此\end{CJK}
$$
\frac{f\left(x_{0}+h\right)-f\left(x_{0}\right)}{h}
$$
\begin{CJK}{UTF8}{mj}有下界\end{CJK}, \begin{CJK}{UTF8}{mj}故\end{CJK} $\lim _{h \rightarrow 0^{+}} \frac{f\left(x_{0}+h\right)-f\left(x_{0}\right)}{h}$ \begin{CJK}{UTF8}{mj}存在\end{CJK}, \begin{CJK}{UTF8}{mj}即\end{CJK} $f(x)$ \begin{CJK}{UTF8}{mj}在\end{CJK} $x_{0}$ \begin{CJK}{UTF8}{mj}内右导数存在\end{CJK}, \begin{CJK}{UTF8}{mj}同理\end{CJK} $f(x)$ \begin{CJK}{UTF8}{mj}在\end{CJK} $x_{0}$ \begin{CJK}{UTF8}{mj}内左导\end{CJK} \begin{CJK}{UTF8}{mj}数存在\end{CJK}, \begin{CJK}{UTF8}{mj}即由\end{CJK} $x_{0}$ \begin{CJK}{UTF8}{mj}任意性知\end{CJK} $f(x)$ \begin{CJK}{UTF8}{mj}在\end{CJK} $(a, b)$ \begin{CJK}{UTF8}{mj}内左右导数存在\end{CJK}.

\begin{CJK}{UTF8}{mj}附录\end{CJK}:\begin{CJK}{UTF8}{mj}一个大一的日经题\end{CJK}

\begin{CJK}{UTF8}{mj}计算\end{CJK} $\lim _{n \rightarrow \infty} \frac{\sqrt[n]{n !}}{n}$

\begin{CJK}{UTF8}{mj}解\end{CJK}: \begin{CJK}{UTF8}{mj}法\end{CJK} 1、\begin{CJK}{UTF8}{mj}根据\end{CJK} Stolz \begin{CJK}{UTF8}{mj}定理\end{CJK}
$$
\begin{aligned}
\lim _{n \rightarrow \infty} \frac{\sqrt[n]{n !}}{n} &=\exp \lim _{n \rightarrow \infty} \frac{\ln n !-n \ln n}{n} \\
&=\exp \lim _{n \rightarrow \infty} \frac{(\ln n !-n \ln n)-[\ln (n-1) !-(n-1) \ln (n-1)}{n-(n-1)} \\
&=\exp \lim _{n \rightarrow \infty}(\ln n !-n \ln n)-[\ln (n-1) !-(n-1) \ln (n-1)\\
&=\exp \lim _{n \rightarrow \infty}(n-1) \ln \left(1-\frac{1}{n}\right) \\
&=\frac{1}{\mathrm{e}}
\end{aligned}
$$
\begin{CJK}{UTF8}{mj}法\end{CJK} 2、\begin{CJK}{UTF8}{mj}定积分定义\end{CJK}
$$
\begin{aligned}
\lim _{n \rightarrow \infty} \frac{\sqrt[n]{n !}}{n} &=\lim _{n \rightarrow \infty} \sqrt[n]{\frac{n !}{n^{n}}} \\
&=\lim _{n \rightarrow \infty} \exp \frac{1}{n} \sum_{i=1}^{n} \frac{i}{n} \\
&=\exp \lim _{n \rightarrow \infty} \frac{1}{n} \sum_{i=1}^{n} \ln \frac{i}{n} \\
&=\exp \int_{0}^{1} \ln x \mathrm{~d} x=\frac{1}{\mathrm{e}}
\end{aligned}
$$
\begin{CJK}{UTF8}{mj}其中\end{CJK}
$$
\int_{0}^{1} \ln x \mathrm{~d} x=\left.x \ln x\right|_{0} ^{1}-\int_{0}^{1} \mathrm{~d} x=-1
$$
\begin{CJK}{UTF8}{mj}法\end{CJK} 3、\begin{CJK}{UTF8}{mj}将数列通项写成\end{CJK}
$$
\frac{n}{\sqrt[n]{n !}}=\sqrt[n]{\frac{n^{n}}{n !}}
$$
\begin{CJK}{UTF8}{mj}记\end{CJK} $x_{n}=\frac{n^{n}}{n !}$, \begin{CJK}{UTF8}{mj}则有\end{CJK}
$$
\frac{x_{n+1}}{x_{n}}=\frac{(n+1)^{n+1}}{(n+1) !} \cdot \frac{n !}{n^{n}}=\left(1+\frac{1}{n}\right)^{n} \rightarrow \mathrm{e}
$$
\begin{CJK}{UTF8}{mj}因此\end{CJK}
$$
\lim _{n \rightarrow \infty} \frac{\sqrt[n]{n !}}{n}=\lim _{n \rightarrow \infty} \frac{1}{\frac{n}{\sqrt[n]{n !}}}=\frac{1}{\mathrm{e}}
$$
\begin{CJK}{UTF8}{mj}法\end{CJK} 4、\begin{CJK}{UTF8}{mj}斯特林公式\end{CJK}
$$
n !=\sqrt{2 \pi n}\left(\frac{n}{\mathrm{e}}\right)^{n} \mathrm{e}^{\frac{\theta_{\mathrm{n}}}{12 n}}\left(0<\theta_{n}<1\right)
$$
\begin{CJK}{UTF8}{mj}也就\end{CJK} $n ! \sim\left(\frac{n}{\mathrm{e}}\right)^{n} \sqrt{2 \pi n}$, \begin{CJK}{UTF8}{mj}即有\end{CJK}
$$
\begin{gathered}
\frac{\sqrt[n]{n !}}{n} \sim \frac{\frac{n}{\mathrm{e}} \sqrt[n]{2 \pi n}}{n}=\frac{1}{\mathrm{e}} \sqrt[n]{2 \pi n} \\
\Rightarrow \lim _{n \rightarrow \infty} \frac{\sqrt[n]{n !}}{n}=\lim _{n \rightarrow \infty} \frac{1}{\mathrm{e}} \sqrt[n]{2 \pi n}=\frac{1}{\mathrm{e}}
\end{gathered}
$$
\begin{CJK}{UTF8}{mj}法\end{CJK} 5、\begin{CJK}{UTF8}{mj}考虑到\end{CJK}

\begin{CJK}{UTF8}{mj}法\end{CJK} 5、\begin{CJK}{UTF8}{mj}考虑到\end{CJK}
$$
\begin{aligned}
\frac{\sqrt[n]{n !}}{n} &=\mathrm{e}^{\frac{\ln n !}{n}-\ln n} \\
& \sim \mathrm{e}^{\frac{n \ln n-n+Q(\ln n)}{n}}-\ln n \\
&=\mathrm{e}^{-1+\frac{O(\ln n)}{n}}
\end{aligned}
$$
\begin{CJK}{UTF8}{mj}则有\end{CJK}
$$
\lim _{n \rightarrow \infty} \frac{\sqrt[n]{n !}}{n}=\lim _{n \rightarrow \infty} \mathrm{e}^{-1+\frac{1}{n}}=\frac{1}{\mathrm{e}}
$$
\begin{CJK}{UTF8}{mj}法\end{CJK} 6、\begin{CJK}{UTF8}{mj}令\end{CJK} $a_{n}=\frac{n !}{n^{n}}$, \begin{CJK}{UTF8}{mj}有\end{CJK} $\frac{a_{n+1}}{a_{n}}=\frac{1}{\left(1+\frac{1}{n}\right)^{n}}$, \begin{CJK}{UTF8}{mj}即\end{CJK}
$$
\lim _{n \rightarrow \infty} \frac{a_{n+1}}{a_{n}}=\frac{1}{\mathrm{e}} \Rightarrow \lim _{n \rightarrow \infty} \sqrt[n]{a_{n}}=\frac{1}{\mathrm{e}}
$$
\begin{CJK}{UTF8}{mj}法\end{CJK} 7 、\begin{CJK}{UTF8}{mj}考虑\end{CJK}
$$
\begin{aligned}
\lim _{n \rightarrow \infty} \frac{n+1}{\sqrt[n]{n !}} &=\lim _{n \rightarrow \infty} \sqrt[n]{\frac{(n+1)^{n}}{n !}} \\
&=\lim _{n \rightarrow \infty} \sqrt{\frac{2}{1} \cdots \frac{(n+1)^{n}}{n^{n}}} \\
&=\lim _{n \rightarrow \infty} \sqrt[n]{\left(1+\frac{1}{1}\right)^{1}\left(1+\frac{1}{2}\right)^{2} \cdots \cdots\left(1+\frac{1}{n}\right)^{n}}
\end{aligned}
$$
\begin{CJK}{UTF8}{mj}由正项数列可得到\end{CJK}
$$
\sqrt[n]{a_{1} a_{2} \cdots a_{n}} \longrightarrow a(n \rightarrow \infty)
$$
\begin{CJK}{UTF8}{mj}又因为\end{CJK}
$$
\lim _{n \rightarrow \infty}\left(1+\frac{1}{n}\right)^{n}=\mathrm{e}
$$
\begin{CJK}{UTF8}{mj}所以\end{CJK}
$$
\lim _{n \rightarrow \infty} \frac{\sqrt[n]{n !}}{n}=\lim _{n \rightarrow \infty} \frac{\sqrt[n]{n !}}{n+1}=\frac{1}{\mathrm{e}}
$$
\begin{CJK}{UTF8}{mj}法\end{CJK} 8、\begin{CJK}{UTF8}{mj}考虑到\end{CJK}
$$
\int_{k=1}^{k} \ln x \mathrm{~d} x<\ln k<\int_{k}^{k+1} \ln x \mathrm{~d} x, k=1,2, \ldots
$$
\begin{CJK}{UTF8}{mj}即\end{CJK}
$$
\int_{0}^{n} \ln x \mathrm{~d} x<\ln (n !)<\int_{1}^{n+1} \ln x \mathrm{~d} x
$$
\begin{CJK}{UTF8}{mj}则有\end{CJK}
$$
\begin{gathered}
n \ln n-n<\ln n !<(n+1) \ln (n+1)-n \\
\Rightarrow \frac{n^{n}}{\mathrm{e}^{n}}<n !<\frac{(n+1)^{n+1}}{\mathrm{e}^{n}}
\end{gathered}
$$
\begin{CJK}{UTF8}{mj}因此\end{CJK}
$$
\begin{aligned}
\frac{n}{\mathrm{e}}<\sqrt{n !} &<\frac{n+1}{\mathrm{e}} \cdot \sqrt[n]{n+1} \\
\Rightarrow & \frac{1}{\mathrm{e}}<\frac{\sqrt[n]{n !}}{n}<\frac{1}{\mathrm{e}} \cdot \frac{n+1}{n} \cdot \sqrt[n]{n+1}
\end{aligned}
$$
\begin{CJK}{UTF8}{mj}由夹逼准则可得\end{CJK} $\lim _{n \rightarrow \infty} \frac{\sqrt[n]{n !}}{n !}=\frac{1}{\mathrm{e}}$.

\begin{CJK}{UTF8}{mj}法\end{CJK} 9、\begin{CJK}{UTF8}{mj}考虑\end{CJK} $a_{n}=\frac{\sqrt[n]{n !}}{n}$, \begin{CJK}{UTF8}{mj}由于\end{CJK}
$$
\ln a_{n}=\frac{n \ln n-(\ln 1+\ln 2+\cdots+\ln n)}{n}=\frac{x_{n}}{n}
$$
\begin{CJK}{UTF8}{mj}其中\end{CJK}
$$
x_{n}=n \ln n-(\ln 1+\ln 2+\cdots+\ln n)
$$
\begin{CJK}{UTF8}{mj}又\end{CJK}
$$
\lim _{n \rightarrow \infty} \frac{x_{n+1}-x_{n}}{(n+1)-n}=\lim _{n \rightarrow \infty} n \ln \frac{n+1}{n}=1
$$
\begin{CJK}{UTF8}{mj}即\end{CJK}
$$
\lim _{n \rightarrow \infty} \ln a_{n}=1 \Rightarrow \lim _{n \rightarrow \infty} a_{n}=\mathrm{e}
$$
\begin{CJK}{UTF8}{mj}故\end{CJK}

\includegraphics[max width=\textwidth]{2022_04_18_7db0708508f26638f054g-096}

\section{$1.2$ 高等代数}
\begin{enumerate}
  \item \begin{CJK}{UTF8}{mj}若\end{CJK} $\alpha_{1}, \alpha_{2}$ \begin{CJK}{UTF8}{mj}为\end{CJK} 3 \begin{CJK}{UTF8}{mj}维列向量\end{CJK}, \begin{CJK}{UTF8}{mj}其夹角为\end{CJK} $\frac{\pi}{4}$, \begin{CJK}{UTF8}{mj}试证\end{CJK}: $\alpha_{1}, \alpha_{2}$ \begin{CJK}{UTF8}{mj}线性无关\end{CJK};
\end{enumerate}
\begin{CJK}{UTF8}{mj}证明\end{CJK}: \begin{CJK}{UTF8}{mj}反证法\end{CJK}: \begin{CJK}{UTF8}{mj}假设\end{CJK} $\alpha_{1}, \alpha_{2}$ \begin{CJK}{UTF8}{mj}线性相关\end{CJK}, \begin{CJK}{UTF8}{mj}则\end{CJK} $k \in \mathbb{R}, k \neq 0$ \begin{CJK}{UTF8}{mj}使得\end{CJK} $\alpha_{1}=k \alpha_{2}$, \begin{CJK}{UTF8}{mj}于是\end{CJK}
$$
\begin{gathered}
\left(\alpha_{1}, \alpha_{2}\right)=\left(k \alpha_{2}, \alpha_{2}\right) \\
\Rightarrow\left|\alpha_{1}\right|\left|\alpha_{2}\right| \cos \frac{\pi}{4}=k\left|\alpha_{2}\right|^{2} \Rightarrow k\left|\alpha_{2}\right|^{2} \cos \frac{\pi}{4}=k\left|\alpha_{2}\right|^{2}
\end{gathered}
$$
\begin{CJK}{UTF8}{mj}而\end{CJK} $k \neq 0$, \begin{CJK}{UTF8}{mj}即\end{CJK} $\cos \frac{\pi}{4}=1$ \begin{CJK}{UTF8}{mj}矛盾\end{CJK}, \begin{CJK}{UTF8}{mj}故\end{CJK} $\alpha_{1}, \alpha_{2}$ \begin{CJK}{UTF8}{mj}线性无关\end{CJK}.

\begin{CJK}{UTF8}{mj}暴力法\end{CJK}:\begin{CJK}{UTF8}{mj}由题设令\end{CJK}
$$
\alpha_{1}=\left(a_{1}, a_{2}, a_{3}\right)^{\prime}, \alpha_{2}=\left(b_{1}, b_{2}, b_{3}\right)^{\prime}
$$
\begin{CJK}{UTF8}{mj}又有\end{CJK}
$$
\begin{aligned}
\cos <\alpha_{1}, \alpha_{2}>&=\cos \frac{\pi}{4}=\frac{\left(\alpha_{1}, \alpha_{2}\right)}{\left|\alpha_{1}\right|\left|\alpha_{2}\right|} \\
&=\frac{\sum_{i=1}^{3} a_{i} b_{i}}{\sqrt{\sum_{i=1}^{3} a_{i}^{2} \sum_{i=1}^{3} b_{i}^{2}}} \\
&=\frac{\sqrt{2}}{2} \\
\Rightarrow 2 \sum_{i=1}^{3} a_{i} b_{i} &=\sqrt{2 \sum_{i=1}^{3} a_{i}^{2} \sum_{i=1}^{3} b_{i}^{2}}
\end{aligned}
$$
\begin{CJK}{UTF8}{mj}因此\end{CJK}
$$
2\left(\sum_{i=1}^{3} a_{i} b_{i}\right)^{2}=\sum_{i=1}^{3} a_{i}^{2} \sum_{i=1}^{3} b_{i}^{2}
$$
\begin{CJK}{UTF8}{mj}由\end{CJK} Lagrange \begin{CJK}{UTF8}{mj}恒等式可得\end{CJK}
$$
\sum_{i=1}^{3} a_{i}^{2} \sum_{i=1}^{3} b_{i}^{2}=\left(\sum_{i=1}^{3} a_{i} b_{i}\right)^{2}+\sum_{i=1}^{2} \sum_{j=i+1}^{3}\left(a_{i} b_{j}-a_{j} b_{i}\right)^{2}
$$
\begin{CJK}{UTF8}{mj}所以\end{CJK}
$$
\begin{gathered}
\sum_{i=1}^{2} \sum_{j=i+1}^{3}\left(a_{i} b_{j}-a_{j} b_{i}\right)^{2}=\left(\sum_{i=1}^{3} a_{i} b_{i}\right)^{2} \geq 0 \\
\Rightarrow a_{i} b_{j}-a_{j} b_{i}=0 \\
\Rightarrow a_{i} b_{j}=a_{j} b_{i} \Rightarrow\left\{\begin{array}{l}
a_{1} b_{2}=a_{2} b_{1} \\
a_{2} b_{3}=a_{3} b_{2} \\
a_{1} b_{3}=a_{3} b_{1}
\end{array}\right.
\end{gathered}
$$
\begin{CJK}{UTF8}{mj}故\end{CJK} $\alpha_{1}, \alpha_{2}$ \begin{CJK}{UTF8}{mj}线性无关\end{CJK}.

\begin{CJK}{UTF8}{mj}补充\end{CJK} Lagrange \begin{CJK}{UTF8}{mj}恒等式\end{CJK}
$$
\left(\sum_{k=1}^{n} a_{k}^{2}\right)\left(\sum_{k=1}^{n} b_{k}^{2}\right)-\left(\sum_{k=1}^{n} a_{k} b_{k}\right)^{2}=\sum_{i=1}^{n-1} \sum_{j=i+1}^{n}\left(a_{i} b_{j}-a_{j} b_{i}\right)^{2}
$$
\begin{CJK}{UTF8}{mj}其证明的代数方法\end{CJK}:\begin{CJK}{UTF8}{mj}由于\end{CJK}
$$
\text { (1) }\left(\sum_{k=1}^{n} a_{k}^{2}\right)\left(\sum_{k=1}^{n} b_{k}^{2}\right)=\sum_{i=1}^{n} \sum_{j=1}^{n} a_{i}^{2} b_{j}^{2}=\sum_{k=1}^{n} a_{k}^{2} b_{k}^{2}+\sum_{i=1}^{n} \sum_{j=i+1}^{n} a_{i}^{2} b_{j}^{2}+\sum_{j=1}^{n} \sum_{i=j+1}^{n} a_{i}^{2} b_{j}^{2}
$$
\begin{CJK}{UTF8}{mj}将其分解成对角线和对角线两侧的一对三角形\end{CJK}
$$
\text { (2) }\left(\sum_{k=1}^{n} a_{k} b_{k}\right)^{2}=\sum_{k=1}^{n} a_{k}^{2} b_{k}^{2}+2 \sum_{i=1}^{n-1} \sum_{j=i+1}^{n} a_{i} b_{i} a_{j} b_{j}
$$
\begin{CJK}{UTF8}{mj}将\end{CJK} (1) - (2) \begin{CJK}{UTF8}{mj}式可得\end{CJK}
$$
\sum_{i=1}^{n} \sum_{j=i+1}^{n}\left(a_{i} b_{j}-a_{j} b_{i}\right)^{2}=\sum_{i=1}^{n} \sum_{j=i+1}^{n} a_{i}^{2} b_{j}^{2}+\sum_{j=1}^{n} \sum_{i=j+1}^{n} a_{i}^{2} b_{j}^{2}-2 \sum_{i=1}^{n} \sum_{j=i+1}^{n} a_{i} b_{i} a_{j} b_{j}
$$

\begin{enumerate}
  \setcounter{enumi}{2}
  \item \begin{CJK}{UTF8}{mj}若\end{CJK} $\alpha_{1}, \alpha_{2}, \alpha_{3}$ \begin{CJK}{UTF8}{mj}为三维向量\end{CJK}, \begin{CJK}{UTF8}{mj}且\end{CJK}
\end{enumerate}
$$
A=\left(\alpha_{i}^{T} \alpha_{j}\right)_{1 \leqslant i_{j} j \leqslant 3}
$$
\begin{CJK}{UTF8}{mj}试证\end{CJK}: $\alpha_{1}, \alpha_{2}, \alpha_{3}$ \begin{CJK}{UTF8}{mj}线性无关的充要条件是\end{CJK} $A$ \begin{CJK}{UTF8}{mj}可逆\end{CJK}.

\begin{CJK}{UTF8}{mj}证明\end{CJK}:\begin{CJK}{UTF8}{mj}设有线性关系\end{CJK}
$$
k_{1} \alpha_{1}+k_{2} \alpha_{2}+k_{3} \alpha_{3}=0
$$
\begin{CJK}{UTF8}{mj}将其分别与\end{CJK} $\alpha_{i}$ \begin{CJK}{UTF8}{mj}取内积\end{CJK}, \begin{CJK}{UTF8}{mj}可得方程组\end{CJK}
$$
k_{1}\left(\alpha_{i}, \alpha_{1}\right)+k_{2}\left(\alpha_{i}, \alpha_{2}\right)+k_{3}\left(\alpha_{i}, \alpha_{3}\right)=0(i=1,2, \cdots, 3)
$$
\begin{CJK}{UTF8}{mj}由于上述方程组仅有零解的充要条件是系数行列式\end{CJK} $A$ \begin{CJK}{UTF8}{mj}不等于\end{CJK} 0 , \begin{CJK}{UTF8}{mj}丛而\end{CJK} $\alpha_{1}, \alpha_{2}, \alpha_{3}$ \begin{CJK}{UTF8}{mj}线性无关的充要条\end{CJK} \begin{CJK}{UTF8}{mj}件是\end{CJK} $A$ \begin{CJK}{UTF8}{mj}可逆\end{CJK}.

\begin{enumerate}
  \setcounter{enumi}{3}
  \item \begin{CJK}{UTF8}{mj}若\end{CJK} $A=\left(\begin{array}{rr}0 & -1 \\ 1 & 0\end{array}\right)$, \begin{CJK}{UTF8}{mj}求\end{CJK}
\end{enumerate}
$$
A^{2021}+A^{2019}=A
$$
\begin{CJK}{UTF8}{mj}解\end{CJK}: \begin{CJK}{UTF8}{mj}考场上原题是\end{CJK} $A^{2021}+A^{2019}-A$, \begin{CJK}{UTF8}{mj}这类题我在之前的推文中已经写过了每日一题\end{CJK} $047 \mid$ \begin{CJK}{UTF8}{mj}矩阵的\end{CJK} \begin{CJK}{UTF8}{mj}最小多项式简化矩阵计算一题\end{CJK}.

\begin{CJK}{UTF8}{mj}由题设可知\end{CJK} $A$ \begin{CJK}{UTF8}{mj}的特征多项式为\end{CJK}

\includegraphics[max width=\textwidth]{2022_04_18_7db0708508f26638f054g-098}

\begin{CJK}{UTF8}{mj}令\end{CJK} $g(\lambda)=\lambda^{2021}+\lambda^{2019}-\lambda$, \begin{CJK}{UTF8}{mj}可设\end{CJK}
$$
g(\lambda)=f(\lambda) q(\lambda)+a \lambda+b
$$
\begin{CJK}{UTF8}{mj}故有\end{CJK}

\begin{CJK}{UTF8}{mj}因此\end{CJK}
$$
\left\{\begin{array} { l } 
{ g ( i ) = a i + b = - i } \\
{ g ( - i ) = - a i + b = i }
\end{array} \Rightarrow \left\{\begin{array}{l}
a=-1 \\
b=0
\end{array}\right.\right.
$$
$$
g(A)=A^{2021}+A^{2019}-A=-A=\left(\begin{array}{cc}
0 & 1 \\
-1 & 0
\end{array}\right)
$$
\begin{CJK}{UTF8}{mj}提高一个难度\end{CJK}:\begin{CJK}{UTF8}{mj}若\end{CJK}
$$
\boldsymbol{A}=\left(\begin{array}{lll}
1 & 0 & 0 \\
1 & 0 & 1 \\
0 & 1 & 0
\end{array}\right)
$$
\begin{CJK}{UTF8}{mj}试求\end{CJK} $A^{99}$ \begin{CJK}{UTF8}{mj}和\end{CJK} $A^{100}$.

\begin{CJK}{UTF8}{mj}解\end{CJK}: \begin{CJK}{UTF8}{mj}很昃然可归纳得到\end{CJK}
$$
A^{n}=A^{n-2}+A^{2}-E
$$
\begin{CJK}{UTF8}{mj}从而有\end{CJK}
$$
A^{100-j}-A^{100-j-2}=A^{2}-E \quad(j=0,2,4, \cdots, 96)
$$
\begin{CJK}{UTF8}{mj}可将这\end{CJK} 49 \begin{CJK}{UTF8}{mj}个等式相加\end{CJK}, \begin{CJK}{UTF8}{mj}得到\end{CJK}
$$
A^{100}-A^{2}=49\left(A^{2}-E\right)
$$
\begin{CJK}{UTF8}{mj}于是\end{CJK}
$$
A^{100}=A^{2}+49\left(A^{2}-E\right)
$$
\begin{CJK}{UTF8}{mj}因为\end{CJK}
$$
\begin{aligned}
A^{2}-E &=\left(\begin{array}{lll}
1 & 0 & 0 \\
1 & 1 & 0 \\
1 & 0 & 1
\end{array}\right)-\left(\begin{array}{lll}
1 & 0 & 0 \\
0 & 1 & 0 \\
0 & 0 & 1
\end{array}\right) \\
&=\left(\begin{array}{lll}
0 & 0 & 0 \\
1 & 0 & 0 \\
1 & 0 & 0
\end{array}\right)
\end{aligned}
$$
\begin{CJK}{UTF8}{mj}所以\end{CJK}
$$
\begin{aligned}
A^{100} &=\left(\begin{array}{lll}
1 & 0 & 0 \\
1 & 1 & 0 \\
1 & 0 & 1
\end{array}\right)+49\left(\begin{array}{lll}
0 & 0 & 0 \\
1 & 0 & 0 \\
1 & 0 & 0
\end{array}\right) \\
&=\left(\begin{array}{ccc}
1 & 0 & 0 \\
50 & 1 & 0 \\
50 & 0 & 1
\end{array}\right)
\end{aligned}
$$
\begin{CJK}{UTF8}{mj}而\end{CJK}
$$
A^{-1}=\left(\begin{array}{ccc}
1 & 0 & 0 \\
-1 & 1 & 0 \\
-1 & 0 & 1
\end{array}\right)
$$
\begin{CJK}{UTF8}{mj}因此\end{CJK}
$$
\begin{aligned}
A^{99} &=\left(\begin{array}{ccc}
1 & 0 & 0 \\
50 & 1 & 0 \\
50 & 0 & 1
\end{array}\right)\left(\begin{array}{ccc}
1 & 0 & 0 \\
-1 & 1 & 0 \\
-1 & 0 & 1
\end{array}\right) \\
&=\left(\begin{array}{ccc}
1 & 0 & 0 \\
49 & 1 & 0 \\
49 & 0 & 1
\end{array}\right)
\end{aligned}
$$

\begin{enumerate}
  \setcounter{enumi}{4}
  \item \begin{CJK}{UTF8}{mj}设二阶矩阵\end{CJK} $A, B$, \begin{CJK}{UTF8}{mj}且\end{CJK} $B=2 A$, \begin{CJK}{UTF8}{mj}其中矩阵\end{CJK} $A$ \begin{CJK}{UTF8}{mj}的特征值为\end{CJK} $-\frac{1}{2}, 2$, \begin{CJK}{UTF8}{mj}若\end{CJK} $M=\left(\begin{array}{cc}B & A \\ A\end{array}\right)$, \begin{CJK}{UTF8}{mj}求\end{CJK} $\operatorname{det}(M)$.
\end{enumerate}
\begin{CJK}{UTF8}{mj}解\end{CJK}: \begin{CJK}{UTF8}{mj}由题设知\end{CJK} $|A|=-1$, \begin{CJK}{UTF8}{mj}对\end{CJK} $M$ \begin{CJK}{UTF8}{mj}做初等变换\end{CJK}
$$
\left(\begin{array}{ll}
B & A \\
A & B
\end{array}\right) \rightarrow\left(\begin{array}{cc}
B+A & B+A \\
A & B
\end{array}\right) \rightarrow\left(\begin{array}{cc}
B+A & 0 \\
A & B-A
\end{array}\right)
$$
\begin{CJK}{UTF8}{mj}因此\end{CJK}
$$
\begin{aligned}
\operatorname{det}(M) &=\left|\begin{array}{lr}
B & A \\
A & B
\end{array}\right|=|B-A||B+A| \\
&=|A||3 A|=9|A|^{2}=9
\end{aligned}
$$

\begin{enumerate}
  \setcounter{enumi}{5}
  \item \begin{CJK}{UTF8}{mj}设\end{CJK} $n$ \begin{CJK}{UTF8}{mj}维列向量\end{CJK} $\alpha$, \begin{CJK}{UTF8}{mj}且\end{CJK} $|\alpha|=3, A=\alpha \alpha^{\prime}$, \begin{CJK}{UTF8}{mj}求\end{CJK} $A$ \begin{CJK}{UTF8}{mj}的所有特征值\end{CJK}.
\end{enumerate}
\begin{CJK}{UTF8}{mj}解\end{CJK}: \begin{CJK}{UTF8}{mj}先证一个打洞结论\end{CJK}, \begin{CJK}{UTF8}{mj}而且这个结论曾经在\end{CJK} 2012 \begin{CJK}{UTF8}{mj}年武汉大学高等代数真题以\end{CJK} 16 \begin{CJK}{UTF8}{mj}分大题出现\end{CJK}.
$$
\begin{aligned}
\left|\lambda E_{n}+A B\right| &=\left|\begin{array}{cc}
\lambda E_{n} & A \\
-B & E_{m}
\end{array}\right| \\
& \stackrel{\text { 前 } n \text { 行提一个 } \lambda}{ } \lambda^{n}\left|\begin{array}{cc}
E_{n} & \frac{A}{\lambda} \\
-B & E_{m}
\end{array}\right| \\
& \stackrel{\text { 后 } m \text { 列乘 } \lambda}{=} \frac{\lambda_{n}}{\lambda_{m}}\left|\begin{array}{cc}
E_{n} & A \\
-B & \lambda E_{m}
\end{array}\right| \\
&=\lambda_{n-m}\left|\lambda E_{m}+B A\right|
\end{aligned}
$$
\begin{CJK}{UTF8}{mj}由题设可知\end{CJK} $A=\alpha \alpha^{\prime}$ \begin{CJK}{UTF8}{mj}为秩\end{CJK} 1 \begin{CJK}{UTF8}{mj}矩阵\end{CJK}, \begin{CJK}{UTF8}{mj}故有\end{CJK}
$$
\begin{aligned}
|\lambda E-A| &=\left|\lambda E-\alpha \alpha^{\prime}\right| \\
&=\lambda^{n-1}\left(\lambda-\alpha^{\prime} \alpha\right) \\
&=\lambda^{n-1}\left(\lambda-|\alpha|^{2}\right) \\
&=0
\end{aligned}
$$
\begin{CJK}{UTF8}{mj}因此\end{CJK} $A$ \begin{CJK}{UTF8}{mj}的所有特征值\end{CJK}
$$
\lambda \lambda=0(n-1 \text { 重 }), \lambda=|\alpha|^{2}=9
$$

\begin{enumerate}
  \setcounter{enumi}{6}
  \item \begin{CJK}{UTF8}{mj}若\end{CJK} $\mathscr{T}: \mathbb{R}^{n} \rightarrow \mathbb{R}^{n}$ \begin{CJK}{UTF8}{mj}的线性映射\end{CJK}, \begin{CJK}{UTF8}{mj}试证\end{CJK}: $\mathscr{T}$ \begin{CJK}{UTF8}{mj}是单射的充要条件是\end{CJK} $v$ \begin{CJK}{UTF8}{mj}是满射\end{CJK}.
\end{enumerate}
\begin{CJK}{UTF8}{mj}证明\end{CJK}: \begin{CJK}{UTF8}{mj}由于\end{CJK} $\mathscr{T}$ \begin{CJK}{UTF8}{mj}是单射的充要条件是\end{CJK} $\operatorname{Ker} T=\{0\}$, \begin{CJK}{UTF8}{mj}从而有\end{CJK}
$$
\operatorname{dim}(\operatorname{Im} T)=\operatorname{dim}\left(\mathbb{R}^{n}\right)
$$
\begin{CJK}{UTF8}{mj}而\end{CJK} $\mathscr{T}$ \begin{CJK}{UTF8}{mj}是满射的充要条件是\end{CJK} $\operatorname{Im} T=\mathbb{R}^{n}$, \begin{CJK}{UTF8}{mj}故证\end{CJK} $\mathscr{T}$ \begin{CJK}{UTF8}{mj}是单射的充要条件是\end{CJK} $\mathscr{T}$ \begin{CJK}{UTF8}{mj}是满射\end{CJK}.

\begin{CJK}{UTF8}{mj}重要结论\end{CJK}: \begin{CJK}{UTF8}{mj}对于有限维线性空间\end{CJK} $V$ \begin{CJK}{UTF8}{mj}上的线性变换\end{CJK} $\mathscr{A}$, \begin{CJK}{UTF8}{mj}则\end{CJK} $\mathscr{A}$ \begin{CJK}{UTF8}{mj}是单射的充要条件是\end{CJK} $\mathscr{A}$ \begin{CJK}{UTF8}{mj}是满射\end{CJK}.

\begin{CJK}{UTF8}{mj}先给出单射与满射的定义\end{CJK}:

(1) $\mathscr{A}(\alpha)=\mathscr{A}(\beta) \Leftrightarrow \alpha=\beta \Leftrightarrow \mathscr{A}$ \begin{CJK}{UTF8}{mj}是单射\end{CJK};

(2) $\mathscr{A} V=V \Leftrightarrow \mathscr{A}^{-1}(0)=\{0\} \Leftrightarrow \mathscr{A}$ \begin{CJK}{UTF8}{mj}是满射\end{CJK}。

\begin{CJK}{UTF8}{mj}设\end{CJK} $\mathscr{A}$ \begin{CJK}{UTF8}{mj}是\end{CJK} $n$ \begin{CJK}{UTF8}{mj}维线性空间\end{CJK} $V$ \begin{CJK}{UTF8}{mj}的线性变换\end{CJK}, \begin{CJK}{UTF8}{mj}则有\end{CJK}
$$
\operatorname{dim}(\mathscr{A} V)+\operatorname{dim}\left(\mathscr{A}^{-1}(0)\right)=n
$$
\begin{CJK}{UTF8}{mj}微信公众号\end{CJK}: \begin{CJK}{UTF8}{mj}八一考研数学竞赛\end{CJK} \begin{CJK}{UTF8}{mj}这也就是我们熟知的维数公式\end{CJK} - \begin{CJK}{UTF8}{mj}若\end{CJK} $\mathscr{A}$ \begin{CJK}{UTF8}{mj}是单射\end{CJK}, \begin{CJK}{UTF8}{mj}下证\end{CJK} $\mathscr{A}^{-1}(0)=\{0\}$, \begin{CJK}{UTF8}{mj}对\end{CJK} $\forall \alpha \in \mathscr{A}^{-1}(0)$ \begin{CJK}{UTF8}{mj}有\end{CJK}
$$
\mathscr{A}(\alpha+\beta)=\mathscr{A}(\alpha)+\mathscr{A}(\beta)=\mathscr{A}(\beta)
$$
\begin{CJK}{UTF8}{mj}即\end{CJK} $\alpha+\beta=\beta$ \begin{CJK}{UTF8}{mj}也就有\end{CJK} $\alpha=0$ \begin{CJK}{UTF8}{mj}故\end{CJK} $\mathscr{A}$ \begin{CJK}{UTF8}{mj}是满射\end{CJK}.

\begin{itemize}
  \item \begin{CJK}{UTF8}{mj}若\end{CJK} $\mathscr{A}$ \begin{CJK}{UTF8}{mj}是满射\end{CJK}, \begin{CJK}{UTF8}{mj}即\end{CJK} $\mathscr{A} V=V$, \begin{CJK}{UTF8}{mj}可得\end{CJK} $\mathscr{A}^{-1}(0)=0$, \begin{CJK}{UTF8}{mj}反证法\end{CJK}: \begin{CJK}{UTF8}{mj}假设\end{CJK}
\end{itemize}
$$
\exists \alpha \neq \beta, \quad \text { s.t. } \quad \mathscr{A}(\alpha)=\mathscr{A}(\beta)
$$
\begin{CJK}{UTF8}{mj}则有\end{CJK} $\mathscr{A}(\alpha-\beta)=0$, \begin{CJK}{UTF8}{mj}而\end{CJK} $\alpha-\beta \neq 0 \in \mathscr{A}^{-1}(0)$, \begin{CJK}{UTF8}{mj}矛盾\end{CJK}, \begin{CJK}{UTF8}{mj}则证\end{CJK} $\alpha=\beta$, \begin{CJK}{UTF8}{mj}即\end{CJK} $\mathscr{A}$ \begin{CJK}{UTF8}{mj}是单射\end{CJK};

\begin{CJK}{UTF8}{mj}因此\end{CJK} $\mathscr{A}$ \begin{CJK}{UTF8}{mj}是单射的充要条件是\end{CJK} $\mathscr{A}$ \begin{CJK}{UTF8}{mj}是满射\end{CJK}.

\begin{enumerate}
  \setcounter{enumi}{7}
  \item \begin{CJK}{UTF8}{mj}若\end{CJK} $C$ \begin{CJK}{UTF8}{mj}为\end{CJK} $n$ \begin{CJK}{UTF8}{mj}阶实对称矩阵\end{CJK}, $I$ \begin{CJK}{UTF8}{mj}为恒等矩阵\end{CJK}, \begin{CJK}{UTF8}{mj}且满足\end{CJK}
\end{enumerate}
$$
A=C^{2}+C+I, B=C^{4}+C+I
$$
\begin{CJK}{UTF8}{mj}试证\end{CJK}:
$$
\operatorname{det}(A+B)>\operatorname{det}(A)+\operatorname{det}(B)
$$
\begin{CJK}{UTF8}{mj}证明\end{CJK}: \begin{CJK}{UTF8}{mj}由于\end{CJK} $C$ \begin{CJK}{UTF8}{mj}是实对称矩阵\end{CJK}, \begin{CJK}{UTF8}{mj}故\end{CJK} $A, B$ \begin{CJK}{UTF8}{mj}也为实对称矩阵\end{CJK}, \begin{CJK}{UTF8}{mj}而多项式\end{CJK}
$$
x^{2}+x+1, x^{4}+x+1
$$
\begin{CJK}{UTF8}{mj}均在实数域不可因式分解\end{CJK}, \begin{CJK}{UTF8}{mj}无实根\end{CJK}, \begin{CJK}{UTF8}{mj}且大于\end{CJK} 0 , \begin{CJK}{UTF8}{mj}继而\end{CJK} $A, B$ \begin{CJK}{UTF8}{mj}为正定矩阵\end{CJK}.

\begin{CJK}{UTF8}{mj}由\end{CJK} $A$ \begin{CJK}{UTF8}{mj}是正定阵\end{CJK}, \begin{CJK}{UTF8}{mj}即合同于单位阵\end{CJK} $I$, \begin{CJK}{UTF8}{mj}则存在可逆阵\end{CJK} $P$, \begin{CJK}{UTF8}{mj}使\end{CJK} $P^{\prime} A P=E$, \begin{CJK}{UTF8}{mj}而\end{CJK} $P^{\prime} B P$ \begin{CJK}{UTF8}{mj}仍为实对称阵\end{CJK}, \begin{CJK}{UTF8}{mj}从\end{CJK} \begin{CJK}{UTF8}{mj}而存在正交阵\end{CJK} $Q$, \begin{CJK}{UTF8}{mj}使\end{CJK}
$$
Q^{\prime}\left(P^{\prime} B P\right) Q=\left(\begin{array}{lll}
\lambda_{1} & & \\
& \ddots & \\
& & \lambda_{n}
\end{array}\right)
$$
\begin{CJK}{UTF8}{mj}其中\end{CJK} $\lambda_{1}, \cdots, \lambda_{n}$ \begin{CJK}{UTF8}{mj}为\end{CJK} $P^{\prime} B P$ \begin{CJK}{UTF8}{mj}的特征值\end{CJK}.

\begin{CJK}{UTF8}{mj}令\end{CJK} $T=P Q$, \begin{CJK}{UTF8}{mj}则有\end{CJK}
$$
T^{\prime}(A+B) T=\left(\begin{array}{ccc}
1+\lambda_{1} & & \\
& \ddots & \\
& & 1+\lambda_{n}
\end{array}\right)
$$
\begin{CJK}{UTF8}{mj}故有\end{CJK}
$$
\begin{gathered}
|A| \cdot|P|^{2}=1 \\
|B| \cdot|P|^{2}=\prod_{i=1}^{n} \lambda_{i} \\
|T|^{2}=|P|^{2} \cdot|Q|^{2}=|P|^{2} \\
|A+B| \cdot|T|^{2}=\prod_{i=1}^{n}\left(1+\lambda_{i}\right)
\end{gathered}
$$
\begin{CJK}{UTF8}{mj}由于\end{CJK} $B$ \begin{CJK}{UTF8}{mj}为正定阵\end{CJK}, \begin{CJK}{UTF8}{mj}故\end{CJK} $P^{\prime} B P$ \begin{CJK}{UTF8}{mj}也为正定阵\end{CJK}, \begin{CJK}{UTF8}{mj}从而\end{CJK} $\lambda_{i}>0(i=1, \cdots, n)$, \begin{CJK}{UTF8}{mj}即\end{CJK}
$$
\begin{aligned}
|A+B| \cdot|P|^{2} &=\prod_{i=1}^{n}\left(1+\lambda_{i}\right)>1+\prod_{i=1}^{n} \lambda_{i} \\
&=|A| \cdot|P|^{2}+|B| \cdot|P|^{2} \\
& \Rightarrow|A+B|>|A|+|B|
\end{aligned}
$$
[Hadamard \begin{CJK}{UTF8}{mj}矩阵\end{CJK}] \begin{CJK}{UTF8}{mj}每个主元都大于其所在行其他元的绝对值之和的方阵称为\end{CJK} Hadamard \begin{CJK}{UTF8}{mj}矩阵\end{CJK}.

\begin{CJK}{UTF8}{mj}设\end{CJK} $A, B$ \begin{CJK}{UTF8}{mj}都是\end{CJK} $n$ \begin{CJK}{UTF8}{mj}阶\end{CJK} Hadamard \begin{CJK}{UTF8}{mj}矩阵\end{CJK}, \begin{CJK}{UTF8}{mj}试证\end{CJK}:
$$
|A+B|>|A|+|B|
$$

\begin{enumerate}
  \setcounter{enumi}{8}
  \item \begin{CJK}{UTF8}{mj}设\end{CJK} 3 \begin{CJK}{UTF8}{mj}维列向量\end{CJK} $\alpha$, \begin{CJK}{UTF8}{mj}且\end{CJK} $|\alpha|=1$, \begin{CJK}{UTF8}{mj}若\end{CJK} $f(x)=\alpha^{\prime} A \alpha$, \begin{CJK}{UTF8}{mj}其中\end{CJK} $>$
\end{enumerate}
$$
A=\left(\begin{array}{ccc}
1 & 0 & -1 \\
0 & 1 & 1 \\
-1 & 1 & -1
\end{array}\right)
$$
\begin{CJK}{UTF8}{mj}试证\end{CJK}: \begin{CJK}{UTF8}{mj}是否存在一个\end{CJK} $[a, b]$, \begin{CJK}{UTF8}{mj}使得\end{CJK} $[a, b]$ \begin{CJK}{UTF8}{mj}为\end{CJK} $f(x)$ \begin{CJK}{UTF8}{mj}的值域\end{CJK}.

\begin{CJK}{UTF8}{mj}证明\end{CJK}:\begin{CJK}{UTF8}{mj}这个题在我之前的推送也写到过每日一题\end{CJK} $050 \mid$ \begin{CJK}{UTF8}{mj}三道融合了分析与代数的考研经典题\end{CJK}, \begin{CJK}{UTF8}{mj}所以\end{CJK} \begin{CJK}{UTF8}{mj}说这道题你既可以从代数角度\end{CJK}(\begin{CJK}{UTF8}{mj}正交合同对角化\end{CJK})\begin{CJK}{UTF8}{mj}出发也可以利用分析手段\end{CJK}(\begin{CJK}{UTF8}{mj}拉格朗日乘数法\end{CJK}), \begin{CJK}{UTF8}{mj}当\end{CJK} \begin{CJK}{UTF8}{mj}然这是高等代数科目考试\end{CJK}, \begin{CJK}{UTF8}{mj}我们还是用第一种为好\end{CJK}.

\begin{CJK}{UTF8}{mj}由题设知\end{CJK}

\begin{CJK}{UTF8}{mj}故有\end{CJK}
$$
\begin{aligned}
|\lambda E-A| &=\left|\begin{array}{ccc}
\lambda-1 & 0 & 1 \\
0 & \lambda-1 & -1 \\
1 & -1 & \lambda+1
\end{array}\right| \\
&=(\lambda-1)\left[\lambda^{2}-2\right]-(\lambda-1) \\
&=(\lambda-1)(\lambda-\sqrt{3})(\lambda+\sqrt{3}) \\
\lambda_{1} &=-\sqrt{3}, \lambda_{2}=1, \lambda_{3}=\sqrt{3}
\end{aligned}
$$
\begin{CJK}{UTF8}{mj}且\end{CJK} $\lambda_{1}<\lambda_{2}<\lambda_{3}$

\begin{CJK}{UTF8}{mj}由于\end{CJK} $A$ \begin{CJK}{UTF8}{mj}实对称矩阵\end{CJK}, \begin{CJK}{UTF8}{mj}故存在正交矩阵\end{CJK} $T$ \begin{CJK}{UTF8}{mj}使得\end{CJK}
$$
A=T^{\prime} \operatorname{diag}\left\{\lambda_{1}, \lambda_{2}, \lambda_{3}\right\} T
$$
\includegraphics[max width=\textwidth]{2022_04_18_7db0708508f26638f054g-102}

\begin{CJK}{UTF8}{mj}由于\end{CJK} $|\alpha|=1$, \begin{CJK}{UTF8}{mj}故\end{CJK} $\alpha \neq 0$, \begin{CJK}{UTF8}{mj}则有\end{CJK}
$$
\beta^{\prime} \beta=\alpha^{\prime} \alpha=\sum_{i=1}^{3} b_{i}
$$
$$
\Leftrightarrow|\beta|^{2}=|\alpha|^{2}=\sum_{i=1}^{3} b_{i}=1
$$
\begin{CJK}{UTF8}{mj}而\end{CJK}
$$
\begin{aligned}
f(x) &=\alpha^{\prime} A \alpha=\beta^{\prime} \operatorname{diag}\left\{\lambda_{1}, \lambda_{2}, \lambda_{3}\right\} \beta \\
&=\lambda_{1} b_{1}^{2}+\lambda_{2} b_{2}^{2}+\lambda_{3} b_{3}^{2}
\end{aligned}
$$
\begin{CJK}{UTF8}{mj}则有\end{CJK}
$$
\begin{gathered}
\lambda_{1} \sum_{i=1}^{3} b_{i} \leq f(x) \leq \lambda_{3} \sum_{i=1}^{3} b_{i} \\
\Leftrightarrow \lambda_{1} \leq f(x) \leq \lambda_{3}
\end{gathered}
$$
\begin{CJK}{UTF8}{mj}从而\end{CJK} $[a, b]=[-\sqrt{3}, \sqrt{3}]$, \begin{CJK}{UTF8}{mj}由于\end{CJK} $|\alpha|=1$ \begin{CJK}{UTF8}{mj}是可以使结论成立\end{CJK}, \begin{CJK}{UTF8}{mj}且端点是可以取到\end{CJK}.

\section{第二章 2021 年四川大学真题}
\section{$2.1$ 数学分析}
\begin{enumerate}
  \item \begin{CJK}{UTF8}{mj}计算题\end{CJK}
\end{enumerate}
(1) \begin{CJK}{UTF8}{mj}求积分\end{CJK}
$$
\int_{0}^{\frac{\pi}{2}} \frac{\sin 2021 x}{\sin x} d x
$$
(2) \begin{CJK}{UTF8}{mj}计算极限\end{CJK}
$$
\lim _{x \rightarrow+\infty} \mathrm{e}^{-\sqrt{x}}\left(1+\frac{1}{\sqrt{x}}\right)^{x}
$$
(3) \begin{CJK}{UTF8}{mj}求\end{CJK} $y=\mathrm{e}^{-x^{2}}$ \begin{CJK}{UTF8}{mj}绕\end{CJK} $x$ \begin{CJK}{UTF8}{mj}轴旋转一周所围成的体积\end{CJK}.

(4) \begin{CJK}{UTF8}{mj}求\end{CJK} $\left\{\begin{array}{l}x+y+z=1 \\ x^{2}+y^{2}+z^{2}=9\end{array}\right.$ \begin{CJK}{UTF8}{mj}在点\end{CJK} $(2,1,-2)$ \begin{CJK}{UTF8}{mj}的切线方程\end{CJK}.

(5) \begin{CJK}{UTF8}{mj}计算第二型曲面积分\end{CJK}
$$
\iiint_{\Sigma} x^{3} \mathrm{~d} y \mathrm{~d} z+y^{3} \mathrm{~d} z \mathrm{~d} x+z^{3} \mathrm{~d} x \mathrm{~d} y
$$
\begin{CJK}{UTF8}{mj}其中\end{CJK} $\Sigma: z=\sqrt{1-x^{2}-y^{2}} \geq 0$, \begin{CJK}{UTF8}{mj}方向指向外侧\end{CJK}.

\begin{enumerate}
  \setcounter{enumi}{2}
  \item \begin{CJK}{UTF8}{mj}判断\end{CJK} $f(x)=x \cos \frac{1}{x}$ \begin{CJK}{UTF8}{mj}在\end{CJK} $(0,+\infty)$ \begin{CJK}{UTF8}{mj}上是否一致连续\end{CJK}, \begin{CJK}{UTF8}{mj}并给出其证明\end{CJK}.

  \item \begin{CJK}{UTF8}{mj}若\end{CJK} $f(x), g(x)$ \begin{CJK}{UTF8}{mj}都是黎曼可积函数\end{CJK}, \begin{CJK}{UTF8}{mj}且\end{CJK} $0 \leq f(x) \leq 1$, \begin{CJK}{UTF8}{mj}试问\end{CJK}: \begin{CJK}{UTF8}{mj}复合函数\end{CJK} $g(f(x))$ \begin{CJK}{UTF8}{mj}在\end{CJK} $[0,1]$ \begin{CJK}{UTF8}{mj}上是否黎曼可积\end{CJK}.

  \item \begin{CJK}{UTF8}{mj}若\end{CJK} $x_{n}=\sum_{n=1}^{\infty} \frac{(-1)^{n}}{n}$, \begin{CJK}{UTF8}{mj}试将\end{CJK} $x_{n}$ \begin{CJK}{UTF8}{mj}重排后序列记为\end{CJK} $x_{n}^{\prime}$, \begin{CJK}{UTF8}{mj}则\end{CJK} $\lim _{n \rightarrow \infty} x_{n}^{\prime}=+\infty$.

  \item \begin{CJK}{UTF8}{mj}将\end{CJK} \begin{CJK}{UTF8}{mj}米的绳子分成三段\end{CJK}, \begin{CJK}{UTF8}{mj}分别围成圆\end{CJK}、\begin{CJK}{UTF8}{mj}正方形与等边三角形\end{CJK}, \begin{CJK}{UTF8}{mj}问三者面积之和何时最\end{CJK} 6. \begin{CJK}{UTF8}{mj}讨论二重积分\end{CJK}

\end{enumerate}
$$
\iint_{\mathbb{R} \times \mathbb{R}} \frac{\sin \left(x^{2}+y^{2}\right)}{\left(x^{2}+y^{2}\right)^{p}} \mathrm{~d} x \mathrm{~d} y
$$
\begin{CJK}{UTF8}{mj}敛散性\end{CJK}.

\begin{enumerate}
  \setcounter{enumi}{7}
  \item \begin{CJK}{UTF8}{mj}若\end{CJK} $x_{1} \in \mathbb{R}$, \begin{CJK}{UTF8}{mj}且\end{CJK} $x_{n+1}=x_{n}-x_{n}^{3}$, \begin{CJK}{UTF8}{mj}讨论\end{CJK} $\left\{x_{n}\right\}$ \begin{CJK}{UTF8}{mj}玫散性\end{CJK}.

  \item \begin{CJK}{UTF8}{mj}若\end{CJK} $f(x)$ \begin{CJK}{UTF8}{mj}在\end{CJK} $[0,1]$ \begin{CJK}{UTF8}{mj}上二阶可导\end{CJK}, \begin{CJK}{UTF8}{mj}且满足\end{CJK}

\end{enumerate}
$$
f(0)=f(1)=0\left|f^{\prime \prime}(x)\right| \leq 1
$$
\begin{CJK}{UTF8}{mj}试证\end{CJK}:
$$
|f(x)| \leq \frac{1}{8},\left|f^{\prime}(x)\right| \leq \frac{1}{2}
$$

\begin{enumerate}
  \setcounter{enumi}{9}
  \item \begin{CJK}{UTF8}{mj}求函数项级数\end{CJK}
\end{enumerate}
$$
S_{n}(x)=\sum_{n=1}^{\infty} \frac{\sin n x}{n+1}
$$
\begin{CJK}{UTF8}{mj}收玫域\end{CJK} $D$, \begin{CJK}{UTF8}{mj}并讨论和函数\end{CJK} $S(x)$ \begin{CJK}{UTF8}{mj}在\end{CJK} $D$ \begin{CJK}{UTF8}{mj}中点的连续性\end{CJK}.

\begin{enumerate}
  \setcounter{enumi}{10}
  \item \begin{CJK}{UTF8}{mj}若\end{CJK} $f(r)$ \begin{CJK}{UTF8}{mj}是\end{CJK} $(0,+\infty)$ \begin{CJK}{UTF8}{mj}上的连续函数\end{CJK}, $L$ \begin{CJK}{UTF8}{mj}是一条不经过原点的简单闭曲线\end{CJK}, \begin{CJK}{UTF8}{mj}试问\end{CJK}:
\end{enumerate}
$$
\oint_{L} f\left(x^{2}+y^{2}\right)(x \mathrm{~d} y+y \mathrm{~d} x)=0
$$
\begin{CJK}{UTF8}{mj}是否一定成立\end{CJK}? \begin{CJK}{UTF8}{mj}并说明你的理由\end{CJK}.

\section{$2.2$ 高等代数}
\begin{enumerate}
  \item \begin{CJK}{UTF8}{mj}若多项式\end{CJK} $f(x)=x^{6}-2 x^{4}+2 x^{2}-1$, \begin{CJK}{UTF8}{mj}且\end{CJK}
\end{enumerate}
$$
\alpha_{1}, \alpha_{2}, \alpha_{3}, \alpha_{4}, \alpha_{5}, \alpha_{6}
$$
\begin{CJK}{UTF8}{mj}为其全部复数根\end{CJK}, \begin{CJK}{UTF8}{mj}试问\end{CJK}:

(1) \begin{CJK}{UTF8}{mj}将\end{CJK} $f(x)$ \begin{CJK}{UTF8}{mj}化为\end{CJK} $\mathbb{Q}$ \begin{CJK}{UTF8}{mj}上的不可约多项式\end{CJK};

(2) \begin{CJK}{UTF8}{mj}计算\end{CJK} $\sum_{i=1}^{6} \alpha_{i}^{2021}$;

(3) \begin{CJK}{UTF8}{mj}设\end{CJK} $g(x)=\prod_{i=1}^{6}\left(x-\alpha_{i}^{12}\right)$ \begin{CJK}{UTF8}{mj}为方阵\end{CJK} $A$ \begin{CJK}{UTF8}{mj}的特征多项式\end{CJK}, \begin{CJK}{UTF8}{mj}求\end{CJK} $\left|A^{2}+A+E_{6}\right|$, \begin{CJK}{UTF8}{mj}其中\end{CJK} $E_{6}$ \begin{CJK}{UTF8}{mj}为\end{CJK} \begin{CJK}{UTF8}{mj}六阶单位阵\end{CJK};

(4) \begin{CJK}{UTF8}{mj}设\end{CJK} $\mathbb{K}$ \begin{CJK}{UTF8}{mj}为\end{CJK} $\alpha_{1}, \alpha_{2}, \alpha_{3}, \alpha_{4}, \alpha_{5}, \alpha_{6}$ \begin{CJK}{UTF8}{mj}的最小数域\end{CJK}, \begin{CJK}{UTF8}{mj}求\end{CJK} $\mathbb{K}$ \begin{CJK}{UTF8}{mj}作为\end{CJK} $\mathbb{Q}$ \begin{CJK}{UTF8}{mj}上线性空间的维数\end{CJK}.

\begin{enumerate}
  \setcounter{enumi}{2}
  \item \begin{CJK}{UTF8}{mj}设\end{CJK} $\mathbb{K}$ \begin{CJK}{UTF8}{mj}为数域\end{CJK}, $M_{n \times n}(\mathbb{K})$ \begin{CJK}{UTF8}{mj}为\end{CJK} $\mathbb{K}$ \begin{CJK}{UTF8}{mj}上所有\end{CJK} $m \times n$ \begin{CJK}{UTF8}{mj}阶矩阵组成的集合\end{CJK}, \begin{CJK}{UTF8}{mj}试问\end{CJK}: (1) \begin{CJK}{UTF8}{mj}设\end{CJK} $n$ \begin{CJK}{UTF8}{mj}元非齐次线性方程组\end{CJK} $A X=\beta$ \begin{CJK}{UTF8}{mj}有解\end{CJK}, $A$ \begin{CJK}{UTF8}{mj}的秩为\end{CJK} $r$, \begin{CJK}{UTF8}{mj}证明\end{CJK}: $A X=\beta$ \begin{CJK}{UTF8}{mj}的解集秩\end{CJK} \begin{CJK}{UTF8}{mj}为\end{CJK} $n-r+1$;
\end{enumerate}
(2) \begin{CJK}{UTF8}{mj}设非齐次线性方程组\end{CJK} $A X=B$ \begin{CJK}{UTF8}{mj}有无穷多解\end{CJK}, \begin{CJK}{UTF8}{mj}且任意解都可表示为\end{CJK}
$$
(1,0,1,-1)^{\prime},(1,1,-1,-1)^{\prime},(3,2,-1,-3)^{\prime} \text { 川 }
$$
\begin{CJK}{UTF8}{mj}的线性组合\end{CJK}, \begin{CJK}{UTF8}{mj}求\end{CJK} $A$ \begin{CJK}{UTF8}{mj}的秩取值范围\end{CJK};

(3) \begin{CJK}{UTF8}{mj}设\end{CJK} $A \in M_{m \times n}(\mathbb{K}), B \in M_{n \times t}(\mathbb{K})$, \begin{CJK}{UTF8}{mj}证明\end{CJK}: \begin{CJK}{UTF8}{mj}矩阵方程\end{CJK} $A X=B$ \begin{CJK}{UTF8}{mj}有解当且仅当\end{CJK} $A$ \begin{CJK}{UTF8}{mj}的\end{CJK} \begin{CJK}{UTF8}{mj}秩与分块矩阵\end{CJK} $(A \mid B)$ \begin{CJK}{UTF8}{mj}的秩相等\end{CJK};

(4) \begin{CJK}{UTF8}{mj}设\end{CJK} $A, B \in M_{n \times n}(\mathbb{K})$ \begin{CJK}{UTF8}{mj}且\end{CJK} $A, B$ \begin{CJK}{UTF8}{mj}的秩都为\end{CJK} 1 , \begin{CJK}{UTF8}{mj}设\end{CJK} $A$ \begin{CJK}{UTF8}{mj}的一个行向量与\end{CJK} $B$ \begin{CJK}{UTF8}{mj}的一个行向\end{CJK} \begin{CJK}{UTF8}{mj}量线性无关\end{CJK}, \begin{CJK}{UTF8}{mj}且\end{CJK} $A$ \begin{CJK}{UTF8}{mj}的一个列向量与\end{CJK} $B$ \begin{CJK}{UTF8}{mj}的一个列向量线性无关\end{CJK}, \begin{CJK}{UTF8}{mj}求\end{CJK} $A+B$ \begin{CJK}{UTF8}{mj}的秩\end{CJK}.

\begin{enumerate}
  \setcounter{enumi}{3}
  \item \begin{CJK}{UTF8}{mj}设\end{CJK} $V$ \begin{CJK}{UTF8}{mj}为内积空间\end{CJK}
\end{enumerate}
(1) \begin{CJK}{UTF8}{mj}若\end{CJK} $U$ \begin{CJK}{UTF8}{mj}为\end{CJK} $V$ \begin{CJK}{UTF8}{mj}的有限维子空间\end{CJK}, $U^{\perp}$ \begin{CJK}{UTF8}{mj}为\end{CJK} $U$ \begin{CJK}{UTF8}{mj}的正交补\end{CJK}, \begin{CJK}{UTF8}{mj}证明\end{CJK}: $V=U \oplus U^{\perp}$;

(2) \begin{CJK}{UTF8}{mj}设\end{CJK} $U$ \begin{CJK}{UTF8}{mj}为\end{CJK} $V$ \begin{CJK}{UTF8}{mj}的无限维子空间\end{CJK}, \begin{CJK}{UTF8}{mj}则\end{CJK} $V=U \oplus U^{\perp}$ \begin{CJK}{UTF8}{mj}是否仍然成立\end{CJK}? \begin{CJK}{UTF8}{mj}并给出其证明\end{CJK}.

\begin{enumerate}
  \setcounter{enumi}{4}
  \item \begin{CJK}{UTF8}{mj}设实二次型\end{CJK}
\end{enumerate}
$$
f\left(x_{1}, x_{2}, x_{3}\right)=a x_{1}^{2}+3 x_{2}^{2}-3 x_{3}^{2}+2 b x_{1} x_{3}
$$
\begin{CJK}{UTF8}{mj}矩阵的全部特征值和为\end{CJK} 1 , \begin{CJK}{UTF8}{mj}乘积为\end{CJK} $-48$, \begin{CJK}{UTF8}{mj}试求\end{CJK} $a, b$, \begin{CJK}{UTF8}{mj}并用非退化线性变换将\end{CJK} $f$ \begin{CJK}{UTF8}{mj}化为标\end{CJK} \begin{CJK}{UTF8}{mj}准型\end{CJK}.

\begin{enumerate}
  \setcounter{enumi}{5}
  \item \begin{CJK}{UTF8}{mj}设任意数域\end{CJK} $\mathbb{F}$ \begin{CJK}{UTF8}{mj}上的线性空间\end{CJK} $l_{1}, l_{2}$, \begin{CJK}{UTF8}{mj}且\end{CJK} $\operatorname{Hom}\left(l_{1}, l_{2}\right)$ \begin{CJK}{UTF8}{mj}表示\end{CJK} $l_{1}$ \begin{CJK}{UTF8}{mj}向\end{CJK} $I_{2}$ \begin{CJK}{UTF8}{mj}的所有线性映射组成\end{CJK} \begin{CJK}{UTF8}{mj}的线性空间\end{CJK}, \begin{CJK}{UTF8}{mj}若\end{CJK} $U, V, W$ \begin{CJK}{UTF8}{mj}为\end{CJK} $\mathbb{F}$ \begin{CJK}{UTF8}{mj}上有限维线性空间\end{CJK}, \begin{CJK}{UTF8}{mj}维度分别为\end{CJK} $l, m, n$ \begin{CJK}{UTF8}{mj}且\end{CJK}
\end{enumerate}
$$
f \in \operatorname{Hom}(U, V), g \in \operatorname{Hom}(U, W)
$$
\begin{CJK}{UTF8}{mj}有\end{CJK} $\operatorname{Ker} f \subseteq \operatorname{Ker} g$

(1) \begin{CJK}{UTF8}{mj}试证\end{CJK}: \begin{CJK}{UTF8}{mj}存在\end{CJK} $h \in \operatorname{Hom}(V, W)$, \begin{CJK}{UTF8}{mj}使得\end{CJK} $g=h \circ f, h$ \begin{CJK}{UTF8}{mj}唯一当且仅当\end{CJK} $f$ \begin{CJK}{UTF8}{mj}满射\end{CJK};

(2) $f$ \begin{CJK}{UTF8}{mj}的像空间维度为\end{CJK} $t$,
$$
\begin{gathered}
S=\{h \in \operatorname{Hom}(V, W) \mid g=h \circ f\} \\
T=\left\{\rho \in \operatorname{Hom}(V, W) \mid \exists h_{1}, h_{2} \in S, \text { s.t. } \rho=h_{1}-h_{2}\right\}
\end{gathered}
$$
$\sqrt{1}$

$\operatorname{Hom}(V, W)$ \begin{CJK}{UTF8}{mj}的子空间\end{CJK}, \begin{CJK}{UTF8}{mj}并求维数\end{CJK}.

\begin{enumerate}
  \setcounter{enumi}{6}
  \item \begin{CJK}{UTF8}{mj}设\end{CJK} $V$ \begin{CJK}{UTF8}{mj}为实数域\end{CJK} $\mathbb{R}$ \begin{CJK}{UTF8}{mj}上\end{CJK} $n(n>1)$ \begin{CJK}{UTF8}{mj}维线性空间\end{CJK}
\end{enumerate}
(1) \begin{CJK}{UTF8}{mj}举出有无穷多个不变子空间的\end{CJK} $V$ \begin{CJK}{UTF8}{mj}上的线性变换的例子\end{CJK}; (2) \begin{CJK}{UTF8}{mj}若\end{CJK} $\mathscr{A}$ \begin{CJK}{UTF8}{mj}为\end{CJK} $V$ \begin{CJK}{UTF8}{mj}上线性变换\end{CJK}, \begin{CJK}{UTF8}{mj}且\end{CJK} $\mathscr{A}$ \begin{CJK}{UTF8}{mj}有\end{CJK} $n$ \begin{CJK}{UTF8}{mj}个实特征值\end{CJK} (\begin{CJK}{UTF8}{mj}重根按重数计\end{CJK}), $\mathscr{A}$ \begin{CJK}{UTF8}{mj}有有限多\end{CJK} \begin{CJK}{UTF8}{mj}不变子空间\end{CJK}, \begin{CJK}{UTF8}{mj}写出\end{CJK} $\mathscr{A}$ \begin{CJK}{UTF8}{mj}所有可能的\end{CJK} Jordan \begin{CJK}{UTF8}{mj}标准型并说明\end{CJK};

(3) \begin{CJK}{UTF8}{mj}若\end{CJK} $\mathscr{B}$ \begin{CJK}{UTF8}{mj}为\end{CJK} $V$ \begin{CJK}{UTF8}{mj}上线性变换\end{CJK}, \begin{CJK}{UTF8}{mj}且\end{CJK} $\mathscr{B}$ \begin{CJK}{UTF8}{mj}有\end{CJK} $n$ \begin{CJK}{UTF8}{mj}个实特征值\end{CJK} (\begin{CJK}{UTF8}{mj}重根按重数计\end{CJK}), $\mathscr{B}$ \begin{CJK}{UTF8}{mj}有有限多不\end{CJK} \begin{CJK}{UTF8}{mj}变子空间\end{CJK}, \begin{CJK}{UTF8}{mj}请用\end{CJK} $\mathscr{B}$ \begin{CJK}{UTF8}{mj}的初等因子给出\end{CJK} $\mathscr{B}=\mathscr{C}^{2}$ \begin{CJK}{UTF8}{mj}成立的充要条件\end{CJK}, \begin{CJK}{UTF8}{mj}其中\end{CJK} $\mathscr{C}$ \begin{CJK}{UTF8}{mj}为\end{CJK} $V$ \begin{CJK}{UTF8}{mj}上\end{CJK} \begin{CJK}{UTF8}{mj}线性变换\end{CJK}.

\begin{enumerate}
  \setcounter{enumi}{7}
  \item \begin{CJK}{UTF8}{mj}设\end{CJK} $V$ \begin{CJK}{UTF8}{mj}为数域\end{CJK} $F$ \begin{CJK}{UTF8}{mj}上的\end{CJK} $n$ \begin{CJK}{UTF8}{mj}维线性空间\end{CJK}, $V^{*}$ \begin{CJK}{UTF8}{mj}为\end{CJK} $V$ \begin{CJK}{UTF8}{mj}的对偶空间\end{CJK}, $\mathscr{A}$ \begin{CJK}{UTF8}{mj}为\end{CJK} $V$ \begin{CJK}{UTF8}{mj}上线性变换\end{CJK}, \begin{CJK}{UTF8}{mj}对\end{CJK} \begin{CJK}{UTF8}{mj}任意的\end{CJK} $g \in V^{*}$, \begin{CJK}{UTF8}{mj}定义\end{CJK} $\mathscr{B}(g)=g \circ A$
\end{enumerate}
(1) \begin{CJK}{UTF8}{mj}试证\end{CJK}: $\mathscr{B}$ \begin{CJK}{UTF8}{mj}为\end{CJK} $V^{*}$ \begin{CJK}{UTF8}{mj}上线性变换\end{CJK};

(2) \begin{CJK}{UTF8}{mj}证明\end{CJK}: $\mathscr{A}$ \begin{CJK}{UTF8}{mj}为线性同构当且仅当\end{CJK} $\mathscr{B}$ \begin{CJK}{UTF8}{mj}为线性同构\end{CJK};

(3) \begin{CJK}{UTF8}{mj}设\end{CJK} $f \in V^{*}$, \begin{CJK}{UTF8}{mj}证明\end{CJK}:
$$
f, \mathscr{B}(f), \mathscr{B}^{2}(f), \ldots, \mathscr{B}^{n-1}(f)
$$
\begin{CJK}{UTF8}{mj}为\end{CJK} $V^{*}$ \begin{CJK}{UTF8}{mj}的基当且仅当\end{CJK} $\mathscr{A}$ \begin{CJK}{UTF8}{mj}的任意非零不变子空间不是\end{CJK} Ker \begin{CJK}{UTF8}{mj}子空间\end{CJK}.

\section{第三章 2021 年湖南大学真题}
\section{$3.1$ 数学分析}
\begin{enumerate}
  \item (15 \begin{CJK}{UTF8}{mj}分\end{CJK}) \begin{CJK}{UTF8}{mj}设\end{CJK}
\end{enumerate}
$$
x_{0} \in\left(-\frac{\pi}{2}, \frac{\pi}{2}\right), x_{n}=\frac{\pi}{2} \sin x_{n-1},(n=1,2, \cdots)
$$
\begin{CJK}{UTF8}{mj}试证\end{CJK}: $\left\{x_{n}\right\}$ \begin{CJK}{UTF8}{mj}收玫\end{CJK}, \begin{CJK}{UTF8}{mj}并求\end{CJK} $\lim _{n \rightarrow \infty} x_{n}$.

\begin{enumerate}
  \setcounter{enumi}{2}
  \item (15 \begin{CJK}{UTF8}{mj}分\end{CJK}) \begin{CJK}{UTF8}{mj}证明\end{CJK}: $f(x)=\cos \sqrt{x}$ \begin{CJK}{UTF8}{mj}在\end{CJK} $[0,+\infty)$ \begin{CJK}{UTF8}{mj}上一致连续\end{CJK}.

  \item (20 \begin{CJK}{UTF8}{mj}分\end{CJK}) \begin{CJK}{UTF8}{mj}设\end{CJK} $f(x)$ \begin{CJK}{UTF8}{mj}在点\end{CJK} $a$ \begin{CJK}{UTF8}{mj}处二阶可导\end{CJK}, \begin{CJK}{UTF8}{mj}且\end{CJK} $f^{\prime \prime}(a) \neq 0$, \begin{CJK}{UTF8}{mj}证明\end{CJK}: \begin{CJK}{UTF8}{mj}当\end{CJK} $h$ \begin{CJK}{UTF8}{mj}充分小时成立\end{CJK}

\end{enumerate}
$$
f(a+h)-f(a)=f^{\prime}(a+\theta h) h
$$
\begin{CJK}{UTF8}{mj}其中的\end{CJK} $\theta$ \begin{CJK}{UTF8}{mj}具有性质\end{CJK} $\lim _{h \rightarrow 0} \theta(h)=\frac{1}{2}$.

\begin{enumerate}
  \setcounter{enumi}{4}
  \item (17 \begin{CJK}{UTF8}{mj}分\end{CJK}) \begin{CJK}{UTF8}{mj}设\end{CJK} $f(x)$ \begin{CJK}{UTF8}{mj}在\end{CJK} $[a, b]$ \begin{CJK}{UTF8}{mj}上可积\end{CJK}, \begin{CJK}{UTF8}{mj}且在\end{CJK} $[a, b]$ \begin{CJK}{UTF8}{mj}满足\end{CJK}
\end{enumerate}
$$
|f(x)| \geqslant m>0
$$
\begin{CJK}{UTF8}{mj}证明\end{CJK}: $\frac{1}{f(x)}$ \begin{CJK}{UTF8}{mj}在\end{CJK} $[a, b]$ \begin{CJK}{UTF8}{mj}也可积\end{CJK}.

\begin{enumerate}
  \setcounter{enumi}{5}
  \item (18 \begin{CJK}{UTF8}{mj}分\end{CJK}) \begin{CJK}{UTF8}{mj}设连续函数列\end{CJK} $\left\{f_{n}(x)\right\}$ \begin{CJK}{UTF8}{mj}在闭区间\end{CJK} $[a, b]$ \begin{CJK}{UTF8}{mj}致连续于\end{CJK} $f(x)$, \begin{CJK}{UTF8}{mj}若\end{CJK} $x_{n} \in[a, b],(n=$ $1,2, \cdots)$ \begin{CJK}{UTF8}{mj}且\end{CJK} $x_{n} \rightarrow x_{0}(n \rightarrow \infty)$, \begin{CJK}{UTF8}{mj}证明\end{CJK}: $\lim _{n \rightarrow \infty} f_{n}(x)=f\left(x_{0}\right)$.

  \item (19 \begin{CJK}{UTF8}{mj}分\end{CJK}) \begin{CJK}{UTF8}{mj}证明\end{CJK}: \begin{CJK}{UTF8}{mj}在点\end{CJK} $(1,1)$ \begin{CJK}{UTF8}{mj}的某一邻域内存在唯一的连续可微函数\end{CJK} $y=f(x)$ \begin{CJK}{UTF8}{mj}满足\end{CJK} $f(1)=1$ \begin{CJK}{UTF8}{mj}以及\end{CJK}

\end{enumerate}
$$
x f(x)+2 \ln x+3 \ln f(x)-1=0
$$
\begin{CJK}{UTF8}{mj}并求\end{CJK} $f^{\prime}(x)$

\begin{enumerate}
  \setcounter{enumi}{7}
  \item (15 \begin{CJK}{UTF8}{mj}分\end{CJK}) \begin{CJK}{UTF8}{mj}设\end{CJK}
\end{enumerate}
$$
F(x)=\frac{1}{h^{2}} \int_{0}^{h} \mathrm{~d} \xi \int_{0}^{h} f(x+\xi+\eta) \mathrm{d} \eta(h>0)
$$
\begin{CJK}{UTF8}{mj}其中\end{CJK} $f(x)$ \begin{CJK}{UTF8}{mj}为连续函数\end{CJK}, \begin{CJK}{UTF8}{mj}求\end{CJK} $F^{\prime \prime}(x)$. 8. (15 \begin{CJK}{UTF8}{mj}分\end{CJK}) \begin{CJK}{UTF8}{mj}求区域\end{CJK}
$$
\left\{\begin{array}{l}
0 \leq x \leq 1 \\
0 \leq y \leq x \\
x+y \leq z \leq \mathrm{e}^{x+y}
\end{array}\right.
$$
\begin{CJK}{UTF8}{mj}所围成的体积\end{CJK}.

\begin{enumerate}
  \setcounter{enumi}{9}
  \item (15 \begin{CJK}{UTF8}{mj}分\end{CJK}) \begin{CJK}{UTF8}{mj}计算曲线积分\end{CJK}
\end{enumerate}
$$
\oint_{L^{+}}(x-1) \mathrm{d} y-(y+1) \mathrm{d} x+z \mathrm{~d} z
$$
\begin{CJK}{UTF8}{mj}其中\end{CJK} $L^{+}$\begin{CJK}{UTF8}{mj}为上半球面\end{CJK} $x^{2}+y^{2}+z^{2}=1, z \geqslant 0$ \begin{CJK}{UTF8}{mj}与柱面\end{CJK} $x^{2}+y^{2}=x$ \begin{CJK}{UTF8}{mj}的交线沿\end{CJK} $z$ \begin{CJK}{UTF8}{mj}轴从上\end{CJK} \begin{CJK}{UTF8}{mj}往下看为逆时针方向\end{CJK}.

\section{$3.2$ 高等代数}
\begin{enumerate}
  \item (15 \begin{CJK}{UTF8}{mj}分\end{CJK}) \begin{CJK}{UTF8}{mj}求多项式\end{CJK}
\end{enumerate}
$$
f(x)=\frac{2}{3} x^{5}-x^{4}+2 x^{3}-\frac{8}{3} x^{2}+1
$$
\begin{CJK}{UTF8}{mj}在复数域中的标准分解式\end{CJK}.

\begin{enumerate}
  \setcounter{enumi}{2}
  \item (15 \begin{CJK}{UTF8}{mj}分\end{CJK}) \begin{CJK}{UTF8}{mj}设\end{CJK}
\end{enumerate}
$$
D_{n}(t)=\operatorname{det}\left(\left[\begin{array}{cccc}
1+t & t & \cdots & t \\
t & 2+t & \ddots & \vdots \\
\vdots & \ddots & \ddots & t \\
t & \cdots & t & n+t
\end{array}\right]\right)
$$
\begin{CJK}{UTF8}{mj}计算\end{CJK} $D_{n}(t)$ \begin{CJK}{UTF8}{mj}的一阶导数\end{CJK} $\frac{\mathrm{d} D_{n}(t)}{\sqrt[\mathrm{d} t]{ } \text {. }}$

\begin{enumerate}
  \setcounter{enumi}{3}
  \item (30 \begin{CJK}{UTF8}{mj}分\end{CJK}) \begin{CJK}{UTF8}{mj}对于数域\end{CJK} $\mathbb{K}$ \begin{CJK}{UTF8}{mj}上的线性方程组\end{CJK}
\end{enumerate}
$$
\left\{\begin{array}{l}
x_{1}+2 x_{2}+5 x_{4}=1 \\
x_{1}-x_{2}-6 x_{3}-x_{4}=-2 \\
4 x_{1}+x_{2}-\lambda x_{3}+6 x_{4}=-3 \\
2 x_{1}+x_{2}-6 x_{3}+4 x_{4}=\mu
\end{array}\right.
$$
(1) \begin{CJK}{UTF8}{mj}在\end{CJK} $\lambda, \mu$ \begin{CJK}{UTF8}{mj}满足什么条件时\end{CJK}, \begin{CJK}{UTF8}{mj}方程组无解\end{CJK};

(2) \begin{CJK}{UTF8}{mj}在\end{CJK} $\lambda, \mu$ \begin{CJK}{UTF8}{mj}满足什么条件时\end{CJK}, \begin{CJK}{UTF8}{mj}方程组有解\end{CJK}, \begin{CJK}{UTF8}{mj}并求出方程组的通解\end{CJK}.

\begin{enumerate}
  \setcounter{enumi}{4}
  \item (15 \begin{CJK}{UTF8}{mj}分\end{CJK}) \begin{CJK}{UTF8}{mj}设\end{CJK} $A$ \begin{CJK}{UTF8}{mj}是数域\end{CJK} $\mathbb{K}$ \begin{CJK}{UTF8}{mj}上的\end{CJK} $n$ \begin{CJK}{UTF8}{mj}阶方阵\end{CJK} $(n \geqslant 2)$, \begin{CJK}{UTF8}{mj}求\end{CJK} $A$ \begin{CJK}{UTF8}{mj}的伴随矩阵\end{CJK} $A^{*}$ \begin{CJK}{UTF8}{mj}的秩\end{CJK} $\operatorname{rank}\left(A^{*}\right)$.

  \item \begin{CJK}{UTF8}{mj}设\end{CJK} $A, B$ \begin{CJK}{UTF8}{mj}为\end{CJK} $n$ \begin{CJK}{UTF8}{mj}阶实矩阵\end{CJK}, \begin{CJK}{UTF8}{mj}即\end{CJK} $A, B \in M_{n}(\mathbb{R})$, \begin{CJK}{UTF8}{mj}试证\end{CJK}: (1) \begin{CJK}{UTF8}{mj}若\end{CJK} $A$ \begin{CJK}{UTF8}{mj}是正定矩阵\end{CJK}, \begin{CJK}{UTF8}{mj}则\end{CJK} $A^{-1}, A^{*}$ \begin{CJK}{UTF8}{mj}均为正定矩阵\end{CJK};

\end{enumerate}
(2) \begin{CJK}{UTF8}{mj}若\end{CJK} $A B$ \begin{CJK}{UTF8}{mj}均为正定矩阵\end{CJK}, \begin{CJK}{UTF8}{mj}则\end{CJK} $A+B$ \begin{CJK}{UTF8}{mj}也为正定矩阵\end{CJK};

(3) \begin{CJK}{UTF8}{mj}若\end{CJK} $A B$ \begin{CJK}{UTF8}{mj}均为正定矩阵\end{CJK}, \begin{CJK}{UTF8}{mj}且\end{CJK} $A B=B A$, \begin{CJK}{UTF8}{mj}则存在正交矩阵\end{CJK} $P$ \begin{CJK}{UTF8}{mj}使得\end{CJK} $P{ }^{-1} A P, P^{-1} B P$ \begin{CJK}{UTF8}{mj}均为对角阵\end{CJK};

(4) \begin{CJK}{UTF8}{mj}若\end{CJK} $A B$ \begin{CJK}{UTF8}{mj}均为正定矩阵\end{CJK}, \begin{CJK}{UTF8}{mj}则\end{CJK} $A B$ \begin{CJK}{UTF8}{mj}是正定矩阵充要条件是\end{CJK} $A B=B A$.

\begin{enumerate}
  \setcounter{enumi}{6}
  \item (30 \begin{CJK}{UTF8}{mj}分\end{CJK}) \begin{CJK}{UTF8}{mj}设\end{CJK} $V$ \begin{CJK}{UTF8}{mj}是数域\end{CJK} $\mathbb{K}$ \begin{CJK}{UTF8}{mj}上的\end{CJK} $n$ \begin{CJK}{UTF8}{mj}维向量空间\end{CJK}, $V$ \begin{CJK}{UTF8}{mj}分解成\end{CJK} $k$ \begin{CJK}{UTF8}{mj}维子空间\end{CJK} $P$ \begin{CJK}{UTF8}{mj}与\end{CJK} $n-k$ \begin{CJK}{UTF8}{mj}维\end{CJK} \begin{CJK}{UTF8}{mj}子空间\end{CJK} $Q$ \begin{CJK}{UTF8}{mj}的直和\end{CJK} $(1 \leq k \leq n-1)$, \begin{CJK}{UTF8}{mj}即\end{CJK} $V=P \oplus Q$, \begin{CJK}{UTF8}{mj}记\end{CJK} $P$ \begin{CJK}{UTF8}{mj}到\end{CJK} $Q$ \begin{CJK}{UTF8}{mj}的全体线性映射为\end{CJK} $\operatorname{Hom}(P, Q)$, \begin{CJK}{UTF8}{mj}定义\end{CJK} $T \in \operatorname{Hom}(P, Q)$ \begin{CJK}{UTF8}{mj}在\end{CJK} $V$ \begin{CJK}{UTF8}{mj}中的图像为\end{CJK}
\end{enumerate}
$$
\Gamma(T)=\{p+T \mathcal{P}: \forall p \in P\}
$$
\begin{CJK}{UTF8}{mj}试证\end{CJK}:

(1) \begin{CJK}{UTF8}{mj}对任意\end{CJK} $T \in \operatorname{Hom}(P, Q), \Gamma(T)$ \begin{CJK}{UTF8}{mj}均为\end{CJK} $V$ \begin{CJK}{UTF8}{mj}上的\end{CJK} $k$ \begin{CJK}{UTF8}{mj}维子空间\end{CJK};

(2) $V$ \begin{CJK}{UTF8}{mj}中的\end{CJK} $k$ \begin{CJK}{UTF8}{mj}维子空间\end{CJK} $S$ \begin{CJK}{UTF8}{mj}是某个\end{CJK} $T \in \operatorname{Hom}(P, Q)$ \begin{CJK}{UTF8}{mj}的图像\end{CJK}, \begin{CJK}{UTF8}{mj}即\end{CJK} $S=\Gamma(T)$ \begin{CJK}{UTF8}{mj}充要条件是\end{CJK} $S \cap Q=\{0\}$

\begin{enumerate}
  \setcounter{enumi}{7}
  \item (15 \begin{CJK}{UTF8}{mj}分\end{CJK}) \begin{CJK}{UTF8}{mj}设\end{CJK} $V$ \begin{CJK}{UTF8}{mj}是复数域上的\end{CJK} $n$ \begin{CJK}{UTF8}{mj}维向量空间\end{CJK}, $\operatorname{End}(\mathrm{V})$ \begin{CJK}{UTF8}{mj}表示\end{CJK} $V$ \begin{CJK}{UTF8}{mj}上的全体线性变换\end{CJK}, \begin{CJK}{UTF8}{mj}证明\end{CJK}:
\end{enumerate}
(1) \begin{CJK}{UTF8}{mj}若\end{CJK} $\mathscr{A} \mathscr{B} \in \operatorname{End}(\mathrm{V}), \mathscr{A} \mathscr{B}=\mathscr{B} \mathscr{A}$, \begin{CJK}{UTF8}{mj}则\end{CJK} $\operatorname{Ker}(\mathscr{A}) 、 \operatorname{Im}(\mathscr{A}) 、 \mathscr{A}$ \begin{CJK}{UTF8}{mj}的特征子空间均为\end{CJK} $\mathscr{B}$ \begin{CJK}{UTF8}{mj}的不变子空间\end{CJK};

(2) \begin{CJK}{UTF8}{mj}若\end{CJK} $f(x) \in \mathbb{C}[x], \mathscr{A} \in \operatorname{End}(\mathrm{V})$, \begin{CJK}{UTF8}{mj}则\end{CJK} $\operatorname{Ker}(f(\mathscr{A})) 、 \operatorname{Im}(f(\mathscr{A})) 、 f(\mathscr{A})$ \begin{CJK}{UTF8}{mj}的特征子空间\end{CJK} \begin{CJK}{UTF8}{mj}均为\end{CJK} $\mathscr{A}$ \begin{CJK}{UTF8}{mj}的不变子空间\end{CJK};

(3) \begin{CJK}{UTF8}{mj}设\end{CJK} $f(x), f_{1}(x), f_{2}(x) \in \mathbb{C}[x]$ \begin{CJK}{UTF8}{mj}满足\end{CJK} $f(x)=f_{1}(x) f_{2}(x)$, \begin{CJK}{UTF8}{mj}且\end{CJK} $\left(f_{1}(x), f_{2}(x)\right)=1$
$$
\operatorname{Ker}(f(\mathscr{A}))=\operatorname{Ker}\left(f_{1}(\mathscr{A})\right) \oplus \operatorname{Ker}\left(f_{2}(\mathscr{A})\right)
$$
\begin{CJK}{UTF8}{mj}注\end{CJK}: \begin{CJK}{UTF8}{mj}对于\end{CJK} $\mathscr{A} \in \operatorname{End}(\mathrm{V})$, \begin{CJK}{UTF8}{mj}有\end{CJK}
$$
\left\{\begin{array}{l}
\operatorname{ker}(\mathscr{A})=\{v \in V: \mathscr{A}(v)=0\} \\
\operatorname{Im}(\mathscr{A})=\{\mathscr{A}(v) \in V: v \in V\}
\end{array}\right.
$$

\section{第四章 2021 年中南大学真题}
\section{$4.1$ 数学分析}
\begin{enumerate}
  \item \begin{CJK}{UTF8}{mj}计算题\end{CJK} (\begin{CJK}{UTF8}{mj}每小题\end{CJK} 10 \begin{CJK}{UTF8}{mj}分\end{CJK}, \begin{CJK}{UTF8}{mj}共\end{CJK} 40 \begin{CJK}{UTF8}{mj}分\end{CJK})\\
(1) $\lim _{n \rightarrow \infty} \sin ^{2}\left(\pi \sqrt{n^{2}+n}\right)$\\
(2) $\lim _{x \rightarrow 0} \frac{\int_{0}^{\sin ^{2} x} \ln (1+t) \mathrm{d} t}{\sqrt{1+x^{4}}-1}$
\end{enumerate}
\includegraphics[max width=\textwidth]{2022_04_18_7db0708508f26638f054g-111}\\
(3) \begin{CJK}{UTF8}{mj}计算二重积分\end{CJK} $\iint_{D}\left(x^{2}+y\right) \mathrm{d} x \mathrm{~d} y$, \begin{CJK}{UTF8}{mj}其中\end{CJK} $D: x^{2}+2 y^{2} \leq 1$\\
(4) \begin{CJK}{UTF8}{mj}计算曲面积分\end{CJK}
$$
\begin{aligned}
& \iint_{\Sigma} x^{3} \mathrm{~d} y \mathrm{~d} z+y^{3} \mathrm{~d} z \mathrm{~d} x+z^{3} \mathrm{~d} x \mathrm{~d} y
\end{aligned}
$$
\begin{CJK}{UTF8}{mj}其中\end{CJK} $\Sigma: x^{2}+y^{2}+z^{2}=a^{2}$, \begin{CJK}{UTF8}{mj}方向指向外侧\end{CJK}.

\begin{enumerate}
  \setcounter{enumi}{2}
  \item (15 \begin{CJK}{UTF8}{mj}分\end{CJK}) \begin{CJK}{UTF8}{mj}若\end{CJK} $f(x) \in \mathbb{C}^{2}[a, b]$, \begin{CJK}{UTF8}{mj}且\end{CJK} $f^{\prime \prime}(x)<0$, \begin{CJK}{UTF8}{mj}试证\end{CJK}:
\end{enumerate}
$$
\int \sqrt{\frac{f(a)+f(b)}{2}}<\frac{1}{b-a} \int_{a}^{b} f(x) \mathrm{d} x
$$

\begin{enumerate}
  \setcounter{enumi}{3}
  \item (15 \begin{CJK}{UTF8}{mj}分\end{CJK}) \begin{CJK}{UTF8}{mj}若函数列\end{CJK} $\left\{f_{n}(x)\right\} \subset C[a, b]$, \begin{CJK}{UTF8}{mj}且关于\end{CJK} $n$ \begin{CJK}{UTF8}{mj}单调递增\end{CJK}, \begin{CJK}{UTF8}{mj}当\end{CJK} $n \rightarrow \infty$ \begin{CJK}{UTF8}{mj}时\end{CJK}, $f_{n}(x)$ \begin{CJK}{UTF8}{mj}在\end{CJK} $[a, b]$ \begin{CJK}{UTF8}{mj}上逐点收敛于连续函数\end{CJK} $f(x)$, \begin{CJK}{UTF8}{mj}试证\end{CJK}: $f_{n}(x)$ \begin{CJK}{UTF8}{mj}一致收玫于\end{CJK} $f(x)$.

  \item (20 \begin{CJK}{UTF8}{mj}分\end{CJK})

\end{enumerate}
(1) \begin{CJK}{UTF8}{mj}求幂级数\end{CJK}
$$
\sum_{n=1}^{\infty}\left(1+\frac{1}{2}+\cdots+\frac{1}{n}\right) x^{n}
$$
\begin{CJK}{UTF8}{mj}收玫域\end{CJK};

(2) \begin{CJK}{UTF8}{mj}设\end{CJK} $f(x)$ \begin{CJK}{UTF8}{mj}在\end{CJK} $[0,2 \pi]$ \begin{CJK}{UTF8}{mj}上单调递增\end{CJK}, \begin{CJK}{UTF8}{mj}且\end{CJK} $a_{n}, b_{n}$ \begin{CJK}{UTF8}{mj}为\end{CJK} $f(x)$ \begin{CJK}{UTF8}{mj}的\end{CJK} Fourier \begin{CJK}{UTF8}{mj}级数\end{CJK}, \begin{CJK}{UTF8}{mj}利用积分第二\end{CJK} \begin{CJK}{UTF8}{mj}中值定理\end{CJK}, \begin{CJK}{UTF8}{mj}证明\end{CJK}: $\left\{n a_{n}\right\},\left\{n b_{n}\right\}$ \begin{CJK}{UTF8}{mj}有界\end{CJK}. 5. (20 \begin{CJK}{UTF8}{mj}分\end{CJK}) \begin{CJK}{UTF8}{mj}若\end{CJK} $n \geq 1, x=\left(x_{1}, x_{2}, \cdots, x_{n}\right) \in \mathbb{R}^{n}, U \subset \mathbb{R}$, \begin{CJK}{UTF8}{mj}且\end{CJK} $U$ \begin{CJK}{UTF8}{mj}为凸集\end{CJK}, \begin{CJK}{UTF8}{mj}即对任意\end{CJK} $x, y \in U$, \begin{CJK}{UTF8}{mj}任意的\end{CJK} $\lambda \in[0,1]$ \begin{CJK}{UTF8}{mj}有\end{CJK} $(1-\lambda) x+\lambda y \in u$, \begin{CJK}{UTF8}{mj}若\end{CJK} $f(x)$ \begin{CJK}{UTF8}{mj}为定义在\end{CJK} $U$ \begin{CJK}{UTF8}{mj}上的凸函数\end{CJK}, \begin{CJK}{UTF8}{mj}即对任意\end{CJK} \begin{CJK}{UTF8}{mj}的\end{CJK} $x, y \in U$, \begin{CJK}{UTF8}{mj}任意的\end{CJK} $\lambda \in[0,1]$ \begin{CJK}{UTF8}{mj}有\end{CJK}
$$
f((1-\lambda) x+\lambda y) \leqslant(1-\lambda) f(x)+\lambda f(y)
$$
\begin{CJK}{UTF8}{mj}定义\end{CJK}
$$
\phi(t)=\phi(t ; x, y)=f((1-t) x+t y)
$$
\begin{CJK}{UTF8}{mj}试证\end{CJK}:

(1) $f(x)$ \begin{CJK}{UTF8}{mj}为\end{CJK} $U$ \begin{CJK}{UTF8}{mj}上的凸函数充要条件是\end{CJK} $\phi(t)$ \begin{CJK}{UTF8}{mj}在\end{CJK} $[0,1]$ \begin{CJK}{UTF8}{mj}上的凸函数\end{CJK};

(2) $f(x)$ \begin{CJK}{UTF8}{mj}为\end{CJK} $U$ \begin{CJK}{UTF8}{mj}上的凸函数\end{CJK}, \begin{CJK}{UTF8}{mj}且在\end{CJK} $U$ \begin{CJK}{UTF8}{mj}上的一个内点处取最大值\end{CJK}, \begin{CJK}{UTF8}{mj}试证\end{CJK}: $f(x)$ \begin{CJK}{UTF8}{mj}在\end{CJK} $U$ \begin{CJK}{UTF8}{mj}上\end{CJK} \begin{CJK}{UTF8}{mj}的取值为某一常数\end{CJK}.

\begin{enumerate}
  \setcounter{enumi}{6}
  \item (15 \begin{CJK}{UTF8}{mj}分\end{CJK}) \begin{CJK}{UTF8}{mj}若函数\end{CJK}
\end{enumerate}
$$
f(x, y)= \begin{cases}x(1-y), & x \leqslant y \\ y(1-x), & x>y\end{cases}
$$
\begin{CJK}{UTF8}{mj}求\end{CJK} $f(x, y)$ \begin{CJK}{UTF8}{mj}在\end{CJK} $D:[0,1] \times[0,1]$ \begin{CJK}{UTF8}{mj}上的最大值和最小值\end{CJK}.

\begin{enumerate}
  \setcounter{enumi}{7}
  \item (15 \begin{CJK}{UTF8}{mj}分\end{CJK}) \begin{CJK}{UTF8}{mj}设函数\end{CJK}
\end{enumerate}
$$
f(x, y)=\left\{\begin{array}{cc}
\frac{|x|^{a}|y|^{a}}{x^{2}+y^{2}} & (x, y) \neq(0,0) \\
0 & (x, y)=(0,0)
\end{array}\right.
$$
\begin{CJK}{UTF8}{mj}证明\end{CJK}:

(1) \begin{CJK}{UTF8}{mj}当且仅当\end{CJK} $a>1$ \begin{CJK}{UTF8}{mj}时\end{CJK}, $f(x, y)$ \begin{CJK}{UTF8}{mj}在原点连续\end{CJK};

(2) \begin{CJK}{UTF8}{mj}当且仅当\end{CJK} $a>\frac{3}{2}$ \begin{CJK}{UTF8}{mj}时\end{CJK}, $f(x, y)$ \begin{CJK}{UTF8}{mj}在原点可微\end{CJK};

\begin{enumerate}
  \setcounter{enumi}{8}
  \item (10 \begin{CJK}{UTF8}{mj}分\end{CJK}) \begin{CJK}{UTF8}{mj}若函数\end{CJK} $y=y(x)$ \begin{CJK}{UTF8}{mj}是由方程\end{CJK}
\end{enumerate}
$$
x^{3}+y^{3}+x y-1=0
$$
\begin{CJK}{UTF8}{mj}所确定\end{CJK}, \begin{CJK}{UTF8}{mj}求\end{CJK}
$$
\lim _{x \rightarrow 0} \frac{3 y+x-3}{x^{3}}
$$

\section{$4.2$ 高等代数}
\begin{enumerate}
  \item \begin{CJK}{UTF8}{mj}设\end{CJK} $M$ \begin{CJK}{UTF8}{mj}是数域\end{CJK} $\mathbb{P}$ \begin{CJK}{UTF8}{mj}上的一元多项式环\end{CJK} $\mathbb{P}[x]$ \begin{CJK}{UTF8}{mj}的一个子集\end{CJK}, \begin{CJK}{UTF8}{mj}且满足\end{CJK}
\end{enumerate}
(1) $\forall f(x), g(x) \in M$, \begin{CJK}{UTF8}{mj}有\end{CJK} $f(x)+g(x) \in M$; (2) $\forall f(x) \in M, q(x) \in P[x]$, \begin{CJK}{UTF8}{mj}有\end{CJK} $\quad q(x) f(x) \in M$.

\begin{CJK}{UTF8}{mj}证明\end{CJK}: \begin{CJK}{UTF8}{mj}存在\end{CJK} $d(x) \in M$ \begin{CJK}{UTF8}{mj}使得\end{CJK}
$$
M=\{d(x) q(x) \mid q(x) \in P[x]\}
$$

\begin{enumerate}
  \setcounter{enumi}{2}
  \item (16 \begin{CJK}{UTF8}{mj}分\end{CJK}) \begin{CJK}{UTF8}{mj}设\end{CJK} $A=\left(a_{i j}\right)$ \begin{CJK}{UTF8}{mj}为\end{CJK} $n$ \begin{CJK}{UTF8}{mj}阶方阵\end{CJK}, \begin{CJK}{UTF8}{mj}定义\end{CJK} $A$ \begin{CJK}{UTF8}{mj}的行列式\end{CJK}
\end{enumerate}
$$
|A|=\sum_{j_{1} j_{2} \cdots j_{n}}(-1)^{\tau\left(j_{1}, j_{2} \cdots j_{n}\right)} a_{1 j} a_{2 j_{2}} \cdots a_{n j_{n}}
$$
\begin{CJK}{UTF8}{mj}其中\end{CJK} $\tau\left(j_{1}, j_{2} \cdots j_{n}\right)$ \begin{CJK}{UTF8}{mj}表示数字\end{CJK} $1,2, \cdots, n$ \begin{CJK}{UTF8}{mj}的全排列\end{CJK} $j_{1}, j_{2},{ }^{2}, j_{n}$ \begin{CJK}{UTF8}{mj}的逆序数\end{CJK}, \begin{CJK}{UTF8}{mj}证明\end{CJK}:

(1) $|A|=\sum_{i_{1} i_{2} \cdots i_{n}}(-1)^{\tau\left(i_{1}, i_{2} \cdots j_{n}\right)} a_{i_{1} 1} a_{i_{2} 2} \cdots a_{i_{n} n}$

(2) \begin{CJK}{UTF8}{mj}任何\end{CJK} $1,2, \cdots, n$ \begin{CJK}{UTF8}{mj}的一个全排列\end{CJK} $l_{1}, l_{2}, \cdots, l_{n}$ \begin{CJK}{UTF8}{mj}都有\end{CJK}
$$
|A|=\sum_{k_{1}} \sum_{k_{2} \cdots k_{n}}(-1)^{\tau\left(l_{1}, l_{2} \cdots l_{n}\right)+\tau\left(k_{1}, k_{2}, \cdots, k_{n}\right)} a_{l_{1} k_{1}} a_{l_{2} k_{2}} \cdots a_{l_{n} k_{n}}
$$

\begin{enumerate}
  \setcounter{enumi}{3}
  \item (16 \begin{CJK}{UTF8}{mj}分\end{CJK}) \begin{CJK}{UTF8}{mj}若\end{CJK} $A$ \begin{CJK}{UTF8}{mj}为一个三阶实矩阵\end{CJK}, \begin{CJK}{UTF8}{mj}且第一行为\end{CJK} $(a, b, c)$
\end{enumerate}
$$
B=\left(\begin{array}{lll}
1 & 2 & 3 \\
2 & 4 & 6 \\
3 & 6 & t
\end{array}\right)
$$
\begin{CJK}{UTF8}{mj}且\end{CJK} $A B=0$, \begin{CJK}{UTF8}{mj}求方程组\end{CJK} $A X=0$ \begin{CJK}{UTF8}{mj}的通解\end{CJK}.

\begin{enumerate}
  \setcounter{enumi}{4}
  \item (16 \begin{CJK}{UTF8}{mj}分\end{CJK}) \begin{CJK}{UTF8}{mj}设\end{CJK} $A, B$ \begin{CJK}{UTF8}{mj}分别为\end{CJK} $m \times n, n \times p$ \begin{CJK}{UTF8}{mj}矩阵\end{CJK}, \begin{CJK}{UTF8}{mj}且\end{CJK} $\operatorname{rank}(A B)=\operatorname{rank}(A)$, \begin{CJK}{UTF8}{mj}试证\end{CJK}: \begin{CJK}{UTF8}{mj}存在\end{CJK} $p \times n$ \begin{CJK}{UTF8}{mj}矩阵\end{CJK} $W$ \begin{CJK}{UTF8}{mj}使得\end{CJK} $A=A B W$.

  \item (16 \begin{CJK}{UTF8}{mj}分\end{CJK}) \begin{CJK}{UTF8}{mj}若\end{CJK} $A=\left(a_{i j}\right)$ \begin{CJK}{UTF8}{mj}为\end{CJK} $n$ \begin{CJK}{UTF8}{mj}阶实方阵\end{CJK}, \begin{CJK}{UTF8}{mj}且满足\end{CJK}

\end{enumerate}
(1) $a_{11}=a_{22}=\cdots=a_{n n}=a>0$;

(2) \begin{CJK}{UTF8}{mj}对\end{CJK} $\forall i(i=1,2, \cdots, n)$ \begin{CJK}{UTF8}{mj}有\end{CJK}
$$
\sum_{j=1}^{n}\left|a_{i j}\right|+\sum_{j=1}^{n}\left|a_{j i}\right|<4 a
$$
\includegraphics[max width=\textwidth]{2022_04_18_7db0708508f26638f054g-113}
$$
f\left(x_{1}, x_{2}, \cdots, x_{n}\right)=\left(x_{1}, x_{2}, \cdots, x_{n}\right) A\left(\begin{array}{c}
x_{1} \\
x_{2} \\
\vdots \\
x_{n}
\end{array}\right)
$$
\begin{CJK}{UTF8}{mj}的规范型\end{CJK}. 6. (16 \begin{CJK}{UTF8}{mj}分\end{CJK}) \begin{CJK}{UTF8}{mj}设\end{CJK} $\varepsilon_{1}, \varepsilon_{2}, \cdots, \varepsilon_{n}$ \begin{CJK}{UTF8}{mj}是\end{CJK} $n$ \begin{CJK}{UTF8}{mj}维实线性空间\end{CJK} $V$ \begin{CJK}{UTF8}{mj}的一组基\end{CJK}, \begin{CJK}{UTF8}{mj}且\end{CJK}
$$
\varepsilon_{n+1}=-\varepsilon_{1}-\varepsilon_{2}-\cdots-\varepsilon_{n}
$$
\begin{CJK}{UTF8}{mj}试证\end{CJK}:

(1) \begin{CJK}{UTF8}{mj}对任意\end{CJK} $i(i=1,2, \cdots, n+1)$, \begin{CJK}{UTF8}{mj}则\end{CJK}
$$
\varepsilon_{1}, \cdots \varepsilon_{i-1}, \varepsilon_{i+1}, \cdots \varepsilon_{n+1}
$$
\begin{CJK}{UTF8}{mj}构成\end{CJK} $V$ \begin{CJK}{UTF8}{mj}的一组基\end{CJK};

(2) \begin{CJK}{UTF8}{mj}对任意\end{CJK} $\alpha \in V$, \begin{CJK}{UTF8}{mj}在\end{CJK} (1) \begin{CJK}{UTF8}{mj}中的\end{CJK} $n+1$ \begin{CJK}{UTF8}{mj}组基中\end{CJK}, \begin{CJK}{UTF8}{mj}存在一组基使得\end{CJK} $\alpha$ \begin{CJK}{UTF8}{mj}在这组基下坐标都\end{CJK} \begin{CJK}{UTF8}{mj}非负\end{CJK}。

\begin{enumerate}
  \setcounter{enumi}{7}
  \item (22 \begin{CJK}{UTF8}{mj}分\end{CJK}) \begin{CJK}{UTF8}{mj}若\end{CJK} $n$ \begin{CJK}{UTF8}{mj}阶实矩阵\end{CJK}
\end{enumerate}
$$
A=\left(\begin{array}{ccccc}
a_{1} & b_{1} & 0 & \cdots & 0 \\
* & a_{2} & b_{2} & \ddots & \vdots \\
* & * & a_{3} & \ddots & 0 \\
\vdots & \vdots & \% & \ddots & b_{n-1} \\
* & * & * & \cdots & a_{n}
\end{array}\right)
$$
\begin{CJK}{UTF8}{mj}有\end{CJK} $n$ \begin{CJK}{UTF8}{mj}个线性无关的特征向量\end{CJK}, \begin{CJK}{UTF8}{mj}且\end{CJK} $b_{1}, b_{2}, \cdots, b_{n-1}$ \begin{CJK}{UTF8}{mj}都非零\end{CJK}, \begin{CJK}{UTF8}{mj}试证\end{CJK}:

(1) $A$ \begin{CJK}{UTF8}{mj}有\end{CJK} $n$ \begin{CJK}{UTF8}{mj}个互异的特征值\end{CJK};

(2) $W=\left\{x \in \mathbb{R}^{n \times n} \mid x A=A x\right\}$ \begin{CJK}{UTF8}{mj}都是\end{CJK} $\mathbb{R}$ \begin{CJK}{UTF8}{mj}上的线性空间\end{CJK};

(3) \begin{CJK}{UTF8}{mj}若\end{CJK}
$$
V=\left\{\left(\begin{array}{cccc}
d_{1} & 0 & \cdots & 0 \\
0 & d_{2} & \cdots & \vdots \\
\vdots & \vdots & \ddots & \vdots \\
0 & 0 & \cdots & d_{n}
\end{array}\right) \mid d_{1}, d_{2}, \cdots, d_{n} \in R\right\}
$$
\begin{CJK}{UTF8}{mj}则\end{CJK} $W, V$ \begin{CJK}{UTF8}{mj}同构\end{CJK}.

\begin{enumerate}
  \setcounter{enumi}{8}
  \item (16 \begin{CJK}{UTF8}{mj}分\end{CJK}) \begin{CJK}{UTF8}{mj}若\end{CJK} $\sigma$ \begin{CJK}{UTF8}{mj}是\end{CJK} $n$ \begin{CJK}{UTF8}{mj}维线性空间\end{CJK} $V$ \begin{CJK}{UTF8}{mj}上的线性变换\end{CJK}, $\mathscr{E}$ \begin{CJK}{UTF8}{mj}是\end{CJK} $V$ \begin{CJK}{UTF8}{mj}上的恒等变换\end{CJK}, \begin{CJK}{UTF8}{mj}证明\end{CJK}: $\sigma^{3}=\mathscr{E}$ \begin{CJK}{UTF8}{mj}充要条件是\end{CJK}
\end{enumerate}
$$
\operatorname{Im}(\sigma-\mathscr{E}) \oplus \operatorname{Im}\left(\sigma^{2}+\sigma+\mathscr{E}\right)=V
$$

\begin{enumerate}
  \setcounter{enumi}{9}
  \item (16 \begin{CJK}{UTF8}{mj}分\end{CJK}) \begin{CJK}{UTF8}{mj}设\end{CJK} $M_{n}(\mathbb{R})$ \begin{CJK}{UTF8}{mj}为实数域\end{CJK} $\mathbb{R}$ \begin{CJK}{UTF8}{mj}上的\end{CJK} $n$ \begin{CJK}{UTF8}{mj}阶方阵全体构成的线性空间\end{CJK}, $\varphi: M_{n}(\mathbb{R}) \rightarrow \mathbb{R}$ \begin{CJK}{UTF8}{mj}的非零线性映射\end{CJK}, \begin{CJK}{UTF8}{mj}满足\end{CJK}
\end{enumerate}
$$
\forall X, Y \in M_{n}(\mathbb{R}), \varphi(X Y)=\varphi(Y X)
$$
\begin{CJK}{UTF8}{mj}在\end{CJK} $M_{n}(\mathbb{R})$ \begin{CJK}{UTF8}{mj}上定义\end{CJK} $(\cdot, \cdot):(X, Y)=\varphi(X Y)$. (1) $(\cdot, \cdot)$ \begin{CJK}{UTF8}{mj}是\end{CJK} $M_{n}(\mathbb{R})$ \begin{CJK}{UTF8}{mj}上的内积吗\end{CJK}? \begin{CJK}{UTF8}{mj}若是\end{CJK}, \begin{CJK}{UTF8}{mj}请给出证明\end{CJK}; \begin{CJK}{UTF8}{mj}若不是\end{CJK}, \begin{CJK}{UTF8}{mj}请说明理由\end{CJK};

(2) \begin{CJK}{UTF8}{mj}证明\end{CJK}: $(\cdot, \cdot)$ \begin{CJK}{UTF8}{mj}是非退化的\end{CJK}, \begin{CJK}{UTF8}{mj}即若\end{CJK} $(X, Y)=0\left(\forall Y \in M_{n}(\mathbb{R})\right)$, \begin{CJK}{UTF8}{mj}则\end{CJK} $X=0$.

\section{第五章 2021 年南开大学真题}
\section{$5.1$ 数学分析}
\begin{enumerate}
  \item (20 \begin{CJK}{UTF8}{mj}分\end{CJK}) \begin{CJK}{UTF8}{mj}已知数列\end{CJK}
\end{enumerate}
$$
a_{1}=-\frac{1}{2}, a_{n+1}=-\frac{1}{2}+\frac{a_{n}^{2}}{2}(n \geqslant 1)
$$
\begin{CJK}{UTF8}{mj}证明数列\end{CJK} $\left\{a_{n}\right\}$ \begin{CJK}{UTF8}{mj}收玫\end{CJK}, \begin{CJK}{UTF8}{mj}并求其极限\end{CJK}.

\begin{enumerate}
  \setcounter{enumi}{2}
  \item (20 \begin{CJK}{UTF8}{mj}分\end{CJK}) \begin{CJK}{UTF8}{mj}设函数\end{CJK} $f(x)$ \begin{CJK}{UTF8}{mj}在\end{CJK} $[a,+\infty)$ \begin{CJK}{UTF8}{mj}上连续\end{CJK}, \begin{CJK}{UTF8}{mj}且\end{CJK}
\end{enumerate}
$$
\lim _{x \rightarrow+\infty} f(x)=+\infty
$$
$c \in(a,+\infty)$ \begin{CJK}{UTF8}{mj}为\end{CJK} $f(x)$ \begin{CJK}{UTF8}{mj}的最小值点\end{CJK}, $a \leqslant f(c)<c<f(a)$ \begin{CJK}{UTF8}{mj}试证\end{CJK}; $f(f(x))$ \begin{CJK}{UTF8}{mj}至少在两个点\end{CJK} \begin{CJK}{UTF8}{mj}处取得最小值\end{CJK}.

\begin{enumerate}
  \setcounter{enumi}{3}
  \item (30 \begin{CJK}{UTF8}{mj}分\end{CJK}) \begin{CJK}{UTF8}{mj}假设函数\end{CJK} $z=z(x, y), w=w(x, y)$ \begin{CJK}{UTF8}{mj}具有二阶连续偏导数\end{CJK}, \begin{CJK}{UTF8}{mj}在变换\end{CJK}
\end{enumerate}
$$
x=u y, v=x, w=x z-y
$$
\begin{CJK}{UTF8}{mj}下\end{CJK}, \begin{CJK}{UTF8}{mj}将\end{CJK}
$$
-x^{3} \frac{\partial^{2} z}{\partial x^{2}}+2 x^{2} y \frac{\partial^{2} z}{\partial x \partial y}+x y^{2} \frac{\partial^{2} z}{\partial y^{2}}=2(x z-y)
$$
\begin{CJK}{UTF8}{mj}变为\end{CJK} $w=w(u, v)$ \begin{CJK}{UTF8}{mj}的方程\end{CJK}.

\begin{enumerate}
  \setcounter{enumi}{4}
  \item (20 \begin{CJK}{UTF8}{mj}分\end{CJK}) \begin{CJK}{UTF8}{mj}求第二型曲线积分\end{CJK}
\end{enumerate}
$$
I=\int_{L} y \mathrm{~d} x+2 z \mathrm{~d} y+3 z \mathrm{~d} z
$$
\begin{CJK}{UTF8}{mj}其中\end{CJK} $L$ \begin{CJK}{UTF8}{mj}为球面\end{CJK} $x^{2}+y^{2}+z^{2}=4$ \begin{CJK}{UTF8}{mj}与平面\end{CJK} $x+y+z=0$ \begin{CJK}{UTF8}{mj}的交线\end{CJK}, $x$ \begin{CJK}{UTF8}{mj}正向看去是顺时针\end{CJK} \begin{CJK}{UTF8}{mj}方向\end{CJK}.

\begin{enumerate}
  \setcounter{enumi}{5}
  \item (30 \begin{CJK}{UTF8}{mj}分\end{CJK}) \begin{CJK}{UTF8}{mj}已知不定积分\end{CJK}
\end{enumerate}
$$
\int \frac{\mathrm{d} x}{1+b \cos x}=\frac{2}{\sqrt{1-b^{2}}} \arctan \left(\sqrt{\frac{1-b}{1+b}} \tan \frac{x}{2}\right)+C,|b|<1
$$
\begin{CJK}{UTF8}{mj}以及广义积分\end{CJK}
$$
\int_{0}^{\frac{\pi}{2}} \ln \sin x \mathrm{~d} x=-\frac{\pi}{2} \ln 2
$$
\begin{CJK}{UTF8}{mj}计算含参量积分\end{CJK}
$$
I(a)=\int_{0}^{\pi} \ln \left(1-2 a \cos x+a^{2}\right) \mathrm{d} x
$$
\begin{CJK}{UTF8}{mj}其中\end{CJK} $a$ \begin{CJK}{UTF8}{mj}为实数\end{CJK}.

\begin{enumerate}
  \setcounter{enumi}{6}
  \item (15 \begin{CJK}{UTF8}{mj}分\end{CJK}) \begin{CJK}{UTF8}{mj}已知\end{CJK} $n$ \begin{CJK}{UTF8}{mj}为正整数\end{CJK}, \begin{CJK}{UTF8}{mj}讨论广义积分\end{CJK}
\end{enumerate}
$$
\int_{0}^{+\infty} x^{n} \mathrm{e}^{-x^{12} \sin ^{2} x} \mathrm{~d} x
$$
\begin{CJK}{UTF8}{mj}的玫散性\end{CJK}.

\begin{enumerate}
  \setcounter{enumi}{7}
  \item (15 \begin{CJK}{UTF8}{mj}分\end{CJK}) \begin{CJK}{UTF8}{mj}已知\end{CJK} $r \in(0,1)$, \begin{CJK}{UTF8}{mj}函绥\end{CJK} $f(x)$ \begin{CJK}{UTF8}{mj}在\end{CJK} $(0, a]$ \begin{CJK}{UTF8}{mj}可导\end{CJK}, \begin{CJK}{UTF8}{mj}且\end{CJK} $\lim _{x \rightarrow 0^{+}} x^{r} f^{\prime}(x)$ \begin{CJK}{UTF8}{mj}存在\end{CJK}, \begin{CJK}{UTF8}{mj}证明\end{CJK} $f(x)$ \begin{CJK}{UTF8}{mj}在\end{CJK} $(0, a]$ \begin{CJK}{UTF8}{mj}上一致连续\end{CJK}.
\end{enumerate}
\section{$5.2$ 高等代数}
\begin{enumerate}
  \item (20 \begin{CJK}{UTF8}{mj}分\end{CJK}) \begin{CJK}{UTF8}{mj}计算行列式\end{CJK}
\end{enumerate}
$$
\left|\begin{array}{cccc}
a & -a & -1 & 0 \\
a & -a & 0 & -1 \\
1 & 0 & a & -a \\
0 & 1 & a & -a
\end{array}\right|
$$

\begin{enumerate}
  \setcounter{enumi}{2}
  \item (20 \begin{CJK}{UTF8}{mj}分\end{CJK}) \begin{CJK}{UTF8}{mj}设\end{CJK} $A$ \begin{CJK}{UTF8}{mj}为\end{CJK} $n$ \begin{CJK}{UTF8}{mj}阶方阵\end{CJK}, $n \geqslant 3$, \begin{CJK}{UTF8}{mj}且\end{CJK} $A$ \begin{CJK}{UTF8}{mj}第\end{CJK} $i$ \begin{CJK}{UTF8}{mj}行第\end{CJK} $j$ \begin{CJK}{UTF8}{mj}列元素为\end{CJK} $(i-j)^{2}$, \begin{CJK}{UTF8}{mj}求\end{CJK} $A$ \begin{CJK}{UTF8}{mj}的秩\end{CJK}.

  \item (20 \begin{CJK}{UTF8}{mj}分\end{CJK}) \begin{CJK}{UTF8}{mj}设\end{CJK} $A$ \begin{CJK}{UTF8}{mj}为三阶实对称方阵\end{CJK}, $A$ \begin{CJK}{UTF8}{mj}的特征值为\end{CJK} $-1,1$ (\begin{CJK}{UTF8}{mj}二重\end{CJK}), \begin{CJK}{UTF8}{mj}且\end{CJK}

\end{enumerate}
$$
\alpha_{1}=(-1,2,2)^{\mathrm{T}}, \alpha_{2}=(1,1,4)^{\mathrm{T}}
$$
\begin{CJK}{UTF8}{mj}是属于特征值\end{CJK} 1 \begin{CJK}{UTF8}{mj}的特征向量\end{CJK}.

(1) \begin{CJK}{UTF8}{mj}求属于特征值\end{CJK} $-1$ \begin{CJK}{UTF8}{mj}的特征向量\end{CJK};

(2) \begin{CJK}{UTF8}{mj}求方阵\end{CJK} $A$.

\begin{enumerate}
  \setcounter{enumi}{4}
  \item (10 \begin{CJK}{UTF8}{mj}分\end{CJK}) \begin{CJK}{UTF8}{mj}设\end{CJK} $A, B$ \begin{CJK}{UTF8}{mj}为\end{CJK} $n$ \begin{CJK}{UTF8}{mj}阶实可计方阵\end{CJK}, \begin{CJK}{UTF8}{mj}且\end{CJK} $A+B$ \begin{CJK}{UTF8}{mj}也可逆\end{CJK}. \begin{CJK}{UTF8}{mj}如果\end{CJK}
\end{enumerate}
$$
(A+B)^{-1}=A^{-1}+B^{-1}
$$
\begin{CJK}{UTF8}{mj}求证\end{CJK}: $|A|=|B|$ 5. (20 \begin{CJK}{UTF8}{mj}分\end{CJK}) \begin{CJK}{UTF8}{mj}在\end{CJK} $\mathbb{R}^{4}$ \begin{CJK}{UTF8}{mj}中\end{CJK}, \begin{CJK}{UTF8}{mj}设线性方程组\end{CJK}
$$
\left\{\begin{aligned}
x_{1}-7 x_{3}-8 x_{4} &=0 \\
s x_{2}+5 x_{3}+6 x_{4} &=0
\end{aligned} \text { 和 }\left\{\begin{aligned}
x_{1}+2 x_{2}+3 x-3 &=0 \\
x_{4} &=0
\end{aligned}\right\}\right.
$$
\begin{CJK}{UTF8}{mj}的解空间分别为\end{CJK} $V$ \begin{CJK}{UTF8}{mj}和\end{CJK} $W$.

(1) \begin{CJK}{UTF8}{mj}证明\end{CJK} $V+W$ \begin{CJK}{UTF8}{mj}是\end{CJK} $\mathbb{R}^{4}$ \begin{CJK}{UTF8}{mj}的三维子空间\end{CJK};

(2) \begin{CJK}{UTF8}{mj}求线性函数\end{CJK} $l$, \begin{CJK}{UTF8}{mj}使得\end{CJK}
$$
V+W=\left\{x \in \mathbb{R}^{4} ; l(x)=0\right\}
$$

\begin{enumerate}
  \setcounter{enumi}{6}
  \item (20 \begin{CJK}{UTF8}{mj}分\end{CJK}) \begin{CJK}{UTF8}{mj}给定\end{CJK} $A \in \mathbb{C}^{m \times m}, B \in \mathbb{C}^{n \times n}$, \begin{CJK}{UTF8}{mj}在线性空间\end{CJK} $V=\mathbb{C}^{n \times n}$ \begin{CJK}{UTF8}{mj}上定义线性变换\end{CJK} $\varphi$ \begin{CJK}{UTF8}{mj}如下\end{CJK}:
\end{enumerate}
$$
\varphi(X)=A X-X B, \forall X \in \mathbb{C}^{m \times n}
$$
\begin{CJK}{UTF8}{mj}如果\end{CJK} $A, B$ \begin{CJK}{UTF8}{mj}没有公共特征值\end{CJK}, \begin{CJK}{UTF8}{mj}证明\end{CJK}: $\varphi$ \begin{CJK}{UTF8}{mj}是可逆的\end{CJK}.

\begin{enumerate}
  \setcounter{enumi}{7}
  \item (15 \begin{CJK}{UTF8}{mj}分\end{CJK}) \begin{CJK}{UTF8}{mj}在实线性空间\end{CJK} $V=\mathbb{R}^{n \times n}$ \begin{CJK}{UTF8}{mj}上定义二次型\end{CJK}
\end{enumerate}
$$
q(A)=\operatorname{tr}\left(A^{2}\right), \forall A \in V
$$
\begin{CJK}{UTF8}{mj}试计算\end{CJK} $q$ \begin{CJK}{UTF8}{mj}的正惯性指数和负惯性指数\end{CJK}.

\begin{enumerate}
  \setcounter{enumi}{8}
  \item (15 \begin{CJK}{UTF8}{mj}分\end{CJK}) \begin{CJK}{UTF8}{mj}设\end{CJK} $\mathscr{T}$ \begin{CJK}{UTF8}{mj}是\end{CJK} $n$ \begin{CJK}{UTF8}{mj}维线性空间\end{CJK} $V$ \begin{CJK}{UTF8}{mj}上的线性变换\end{CJK}, $U$ \begin{CJK}{UTF8}{mj}和\end{CJK} $W$ \begin{CJK}{UTF8}{mj}分别是\end{CJK} $\mathscr{T}^{n}$ \begin{CJK}{UTF8}{mj}的值域和核\end{CJK}. \begin{CJK}{UTF8}{mj}证\end{CJK} \begin{CJK}{UTF8}{mj}明\end{CJK}: $V$ \begin{CJK}{UTF8}{mj}是\end{CJK} $U$ \begin{CJK}{UTF8}{mj}与\end{CJK} $W$ \begin{CJK}{UTF8}{mj}的直和\end{CJK}.

  \item (10 \begin{CJK}{UTF8}{mj}分\end{CJK}) \begin{CJK}{UTF8}{mj}如果\end{CJK} $n$ \begin{CJK}{UTF8}{mj}阶方阵\end{CJK} $A_{1}, \cdots, A-m$ \begin{CJK}{UTF8}{mj}满足\end{CJK}

\end{enumerate}
\begin{CJK}{UTF8}{mj}求证\end{CJK}: $m \leqslant n$.
$$
\left\{\begin{array}{l}
A_{i}^{2}=0,1 \leq i \leq m \\
A_{i} A_{j}=0,1 \leq i \neq j \leq m
\end{array}\right.
$$

\section{第六章 2021 年东南大学真题}
\section{$6.1$ 数学分析}
\begin{enumerate}
  \item \begin{CJK}{UTF8}{mj}计算极限\end{CJK}
\end{enumerate}
$$
\lim _{n \rightarrow \infty} \sum_{k=1}^{n} \frac{1}{\sqrt{n^{2}+k}}
$$

\begin{enumerate}
  \setcounter{enumi}{2}
  \item \begin{CJK}{UTF8}{mj}计算极限\end{CJK}
\end{enumerate}
$$
\lim _{x \rightarrow 0} \frac{\mathrm{e}^{x}+\cos \sqrt{x}-2}{x}
$$

\begin{enumerate}
  \setcounter{enumi}{3}
  \item \begin{CJK}{UTF8}{mj}微分方程\end{CJK} $y+x \mathrm{e}^{y}=1$ \begin{CJK}{UTF8}{mj}在何处有解\end{CJK}, \begin{CJK}{UTF8}{mj}若有则求出它的解\end{CJK}.

  \item \begin{CJK}{UTF8}{mj}求三元函数\end{CJK}

\end{enumerate}
\begin{CJK}{UTF8}{mj}的极值\end{CJK}.
$$
f(x, y, z)=\frac{x}{y}+\frac{y}{z}+\frac{z}{x}, x, y, z>0
$$

\begin{enumerate}
  \setcounter{enumi}{5}
  \item \begin{CJK}{UTF8}{mj}计算累次积分\end{CJK}
\end{enumerate}
$$
\iint_{0}^{1} \mathrm{~d} y \int_{y}^{1}\left(\frac{\mathrm{e}^{x^{2}}}{x}-\mathrm{e}^{y^{2}}\right) \mathrm{d} x
$$

\begin{enumerate}
  \setcounter{enumi}{6}
  \item \begin{CJK}{UTF8}{mj}计算曲线积分\end{CJK}
\end{enumerate}
$$
\oint_{l} \frac{x}{x^{2}+y^{2}-2} \mathrm{~d} y+\frac{y}{x^{2}+y^{2}-2} \mathrm{~d} x
$$
\begin{CJK}{UTF8}{mj}其中\end{CJK} $l: x^{2}+y^{2}=4$, \begin{CJK}{UTF8}{mj}取逆时针方向\end{CJK}.

\begin{enumerate}
  \setcounter{enumi}{7}
  \item \begin{CJK}{UTF8}{mj}计算三重积分\end{CJK}
\end{enumerate}
$$
\iiint_{V} y \sqrt{1-x^{2}} \mathrm{~d} x \mathrm{~d} y \mathrm{~d} z
$$
\begin{CJK}{UTF8}{mj}其中\end{CJK} $V: x^{2}+z^{2}=1,0 \leqslant y \leqslant$\\
\begin{CJK}{UTF8}{mj}求幂级数\end{CJK} $\sum_{n=1}^{\infty} \frac{x^{n-1}}{n \cdot 2^{n}}$ \begin{CJK}{UTF8}{mj}的收玫域与和函数\end{CJK}. 9. \begin{CJK}{UTF8}{mj}函数项级数\end{CJK} $\sum_{n=1}^{\infty}(-1)^{n}(1-x) x^{n}$ \begin{CJK}{UTF8}{mj}是否在\end{CJK} $[0,1]$ \begin{CJK}{UTF8}{mj}上一致收敛\end{CJK}?

\begin{enumerate}
  \setcounter{enumi}{10}
  \item \begin{CJK}{UTF8}{mj}已知函数\end{CJK} $f(x)$ \begin{CJK}{UTF8}{mj}在\end{CJK} $[a, b]$ \begin{CJK}{UTF8}{mj}上单调递增\end{CJK}, \begin{CJK}{UTF8}{mj}且值域为\end{CJK} $[f(a), f(b)]$. \begin{CJK}{UTF8}{mj}证明\end{CJK}: \begin{CJK}{UTF8}{mj}此函数之致连续\end{CJK}.

  \item \begin{CJK}{UTF8}{mj}已知\end{CJK} $a_{1}>0$, \begin{CJK}{UTF8}{mj}且\end{CJK}

\end{enumerate}
$$
a_{n+1}=\frac{1}{2}\left(a_{n}+\frac{4}{a_{n}}\right)(n=1.2 \cdots)
$$
\begin{CJK}{UTF8}{mj}求证\end{CJK}:

(1) \begin{CJK}{UTF8}{mj}数列\end{CJK} $\left\{a_{n}\right\}$ \begin{CJK}{UTF8}{mj}有极限\end{CJK};

(2) \begin{CJK}{UTF8}{mj}考查级数\end{CJK} $\sum_{n=1}^{\infty}\left(\frac{a_{n}}{a_{n+1}}-1\right)$ \begin{CJK}{UTF8}{mj}的敛散性\end{CJK}.

\begin{enumerate}
  \setcounter{enumi}{12}
  \item \begin{CJK}{UTF8}{mj}已知二阶连续混合偏导数\end{CJK} $f_{x y}(x, y), f_{y x}(x, y)$ \begin{CJK}{UTF8}{mj}在\end{CJK} $\left(x_{0}, y_{0}\right)$ \begin{CJK}{UTF8}{mj}连续\end{CJK}, \begin{CJK}{UTF8}{mj}求证\end{CJK}:
\end{enumerate}
$$
f_{x y}\left(x_{0}, y_{0}\right)=f_{y x}\left(x_{0}, y_{0}\right)
$$

\begin{enumerate}
  \setcounter{enumi}{13}
  \item \begin{CJK}{UTF8}{mj}已知连续函数\end{CJK} $f(x) \geqslant 0$ \begin{CJK}{UTF8}{mj}满足\end{CJK} $\int_{0}^{\infty} f(x) \mathrm{d} x$ \begin{CJK}{UTF8}{mj}收敘\end{CJK}. \begin{CJK}{UTF8}{mj}求证\end{CJK}:
\end{enumerate}
(1) \begin{CJK}{UTF8}{mj}存在数列\end{CJK} $\left\{x_{n}\right\}, \lim _{x \rightarrow \infty} x_{n}=+\infty$ \begin{CJK}{UTF8}{mj}使得\end{CJK} $\lim _{n \rightarrow \infty} f\left(x_{n}\right)=0$;

(2) \begin{CJK}{UTF8}{mj}在基础上\end{CJK}, $\lim _{x \rightarrow+\infty} f(x)=0$ \begin{CJK}{UTF8}{mj}是否成立\end{CJK}?

\begin{enumerate}
  \setcounter{enumi}{14}
  \item \begin{CJK}{UTF8}{mj}已知连续函数\end{CJK} $f(x)$ \begin{CJK}{UTF8}{mj}滿足\end{CJK}
\end{enumerate}
$$
\int_{0}^{1} f(x) \mathrm{d} x=\int_{0}^{1} x f(x) \mathrm{d} x=0
$$
\begin{CJK}{UTF8}{mj}求证\end{CJK}: \begin{CJK}{UTF8}{mj}此函数在\end{CJK} $[0,1]$ \begin{CJK}{UTF8}{mj}上至䒚有两个零点\end{CJK}.

\begin{enumerate}
  \setcounter{enumi}{15}
  \item \begin{CJK}{UTF8}{mj}用有限覆盖定理证明闭区间上的连续函数有最大值与最小值\end{CJK}.
\end{enumerate}
\section{$6.2$ 言等代数}
\begin{enumerate}
  \item \begin{CJK}{UTF8}{mj}已知方程组\end{CJK}
\end{enumerate}
$$
\left(\begin{array}{lllll}
a & 1 & 1 & 1 & 1 \\
1 & a & 1 & 1 & 1 \\
1 & 1 & a & 1 & 1 \\
1 & 1 & 1 & a & 1 \\
1 & 1 & 1 & 1 & a
\end{array}\right)\left(\begin{array}{l}
x_{1} \\
x_{2} \\
x_{3} \\
x_{4} \\
x_{5}
\end{array}\right)=\left(\begin{array}{l}
0 \\
1 \\
1 \\
b \\
0
\end{array}\right)
$$
\begin{CJK}{UTF8}{mj}问此方程组何时有唯一解\end{CJK}, \begin{CJK}{UTF8}{mj}何时有无穷多组解\end{CJK}, \begin{CJK}{UTF8}{mj}并写出基础解系\end{CJK}. 2. \begin{CJK}{UTF8}{mj}定义复数域线性空间\end{CJK} $\boldsymbol{V}=\mathbb{C}^{2 \times 2}$ \begin{CJK}{UTF8}{mj}上的变换\end{CJK}
$$
\mathscr{T}(X)=X+X^{\mathrm{T}}, \forall X \in V
$$
(1) \begin{CJK}{UTF8}{mj}证明\end{CJK} $\mathscr{T}$ \begin{CJK}{UTF8}{mj}为线性变换\end{CJK};

(2) \begin{CJK}{UTF8}{mj}写出\end{CJK} $\mathscr{T}$ \begin{CJK}{UTF8}{mj}在基\end{CJK} $E_{11}, E_{12}, E_{21}, E_{22}$ \begin{CJK}{UTF8}{mj}下的矩阵\end{CJK} $T$;

(3) \begin{CJK}{UTF8}{mj}求出\end{CJK} $\mathscr{T}$ \begin{CJK}{UTF8}{mj}的特征值\end{CJK}, \begin{CJK}{UTF8}{mj}并写出对应特征子空间的基\end{CJK};

(4) $\mathscr{T}$ \begin{CJK}{UTF8}{mj}是否可以对角化\end{CJK};

(5) \begin{CJK}{UTF8}{mj}计算\end{CJK} $T$ \begin{CJK}{UTF8}{mj}的中心化子空间\end{CJK}
$$
\{X \in V ; T X=X T\}
$$
\begin{CJK}{UTF8}{mj}的维数\end{CJK}.

\begin{enumerate}
  \setcounter{enumi}{3}
  \item \begin{CJK}{UTF8}{mj}已知复数域上的两个三阶矩阵\end{CJK}
\end{enumerate}
$$
\left(\begin{array}{lll}
1 & 2 & 5 \\
0 & a & 7 \\
0 & 0 & 1
\end{array}\right),\left(\begin{array}{lll}
1 & 3 & 7 \\
0 & b & c \\
0 & 0 & 2
\end{array}\right)
$$
(1) \begin{CJK}{UTF8}{mj}写出第一个矩阵的若尔当标准形\end{CJK};

(2) \begin{CJK}{UTF8}{mj}若两矩阵相似\end{CJK}, \begin{CJK}{UTF8}{mj}求出\end{CJK} $a, b, c$.

\begin{enumerate}
  \setcounter{enumi}{4}
  \item \begin{CJK}{UTF8}{mj}已知复数域上的多项式\end{CJK} $h(x)=f(x) g(x), \mathscr{A}$ \begin{CJK}{UTF8}{mj}是复数域上线性空间\end{CJK} $V$ \begin{CJK}{UTF8}{mj}上的线性变换\end{CJK}, \begin{CJK}{UTF8}{mj}定义\end{CJK}
\end{enumerate}
$$
W=\operatorname{ker}(h(\mathscr{A})), W_{1}=\operatorname{ker}(f(\mathscr{A})), W_{2}=\operatorname{ker}(g(\mathscr{A}))
$$
\begin{CJK}{UTF8}{mj}求证\end{CJK}:

(1) $W_{1}, W_{2} \subset W$

(2) $W=W_{1} \oplus W_{2}$

\begin{enumerate}
  \setcounter{enumi}{5}
  \item \begin{CJK}{UTF8}{mj}已知两实对称矩阵\end{CJK} $A, B$ \begin{CJK}{UTF8}{mj}相似\end{CJK}, \begin{CJK}{UTF8}{mj}求证其在实数域上合同\end{CJK}.

  \item \begin{CJK}{UTF8}{mj}已知\end{CJK} $s \times n$ \begin{CJK}{UTF8}{mj}矩阵\end{CJK} $A$ \begin{CJK}{UTF8}{mj}列满秩\end{CJK}, $n \times m$ \begin{CJK}{UTF8}{mj}矩阵\end{CJK} $B$. \begin{CJK}{UTF8}{mj}求证\end{CJK}:

\end{enumerate}
$$
\operatorname{rank}(A B)=\operatorname{rank}(B)
$$

\begin{enumerate}
  \setcounter{enumi}{7}
  \item \begin{CJK}{UTF8}{mj}在复空间\end{CJK} $K$ \begin{CJK}{UTF8}{mj}上有变换\end{CJK} $f, \alpha \in V, W$ \begin{CJK}{UTF8}{mj}是\end{CJK} $f$ \begin{CJK}{UTF8}{mj}的不变子空间\end{CJK}. \begin{CJK}{UTF8}{mj}若有多项式\end{CJK} $p(f)$, \begin{CJK}{UTF8}{mj}使得\end{CJK} $p(f) \alpha \in W$, \begin{CJK}{UTF8}{mj}则称此多项式为\end{CJK} $\alpha$ \begin{CJK}{UTF8}{mj}到\end{CJK} $W$ \begin{CJK}{UTF8}{mj}的导向多项式\end{CJK}. \begin{CJK}{UTF8}{mj}将\end{CJK} $\alpha$ \begin{CJK}{UTF8}{mj}的首一最高次且最低多项\end{CJK} \begin{CJK}{UTF8}{mj}式\end{CJK} $m(x)$ \begin{CJK}{UTF8}{mj}称为极小\end{CJK} $\alpha$ \begin{CJK}{UTF8}{mj}型多项式\end{CJK}.
\end{enumerate}
(1) \begin{CJK}{UTF8}{mj}求证\end{CJK}: \begin{CJK}{UTF8}{mj}对于任意导向多项式\end{CJK} $p(x)$, \begin{CJK}{UTF8}{mj}都有\end{CJK} $m(x) \mid p(x)$; (2) \begin{CJK}{UTF8}{mj}求证\end{CJK}: \begin{CJK}{UTF8}{mj}导向多项式存在且唯一\end{CJK};

(3) \begin{CJK}{UTF8}{mj}若\end{CJK} $W$ \begin{CJK}{UTF8}{mj}是真不变子空间\end{CJK}, \begin{CJK}{UTF8}{mj}则存在\end{CJK} $\alpha \notin W$, \begin{CJK}{UTF8}{mj}多项式\end{CJK} $q(x)$ \begin{CJK}{UTF8}{mj}使得\end{CJK}
$$
q(f) \alpha-c \alpha \in W
$$
\begin{CJK}{UTF8}{mj}其中\end{CJK} $c$ \begin{CJK}{UTF8}{mj}为一常数\end{CJK}.

\begin{enumerate}
  \setcounter{enumi}{8}
  \item \begin{CJK}{UTF8}{mj}已知线性变换\end{CJK} $\mathscr{A}$ \begin{CJK}{UTF8}{mj}有\end{CJK} $s$ \begin{CJK}{UTF8}{mj}个互不相同的特征值\end{CJK} $\lambda_{1}, \cdots, \lambda_{s}$, \begin{CJK}{UTF8}{mj}且\end{CJK} $\eta_{1}, \cdots, \eta_{s}$ \begin{CJK}{UTF8}{mj}为其对应的特征\end{CJK} \begin{CJK}{UTF8}{mj}向量\end{CJK}, $W$ \begin{CJK}{UTF8}{mj}为\end{CJK} $\mathscr{A}$ \begin{CJK}{UTF8}{mj}的不变子空间\end{CJK}. \begin{CJK}{UTF8}{mj}记\end{CJK}
\end{enumerate}
\includegraphics[max width=\textwidth]{2022_04_18_7db0708508f26638f054g-122}

\begin{CJK}{UTF8}{mj}求证\end{CJK}: $\operatorname{dim} W \geqslant s$

\begin{enumerate}
  \setcounter{enumi}{9}
  \item \begin{CJK}{UTF8}{mj}在欧氏空间中有两线性变换\end{CJK}, \begin{CJK}{UTF8}{mj}满足\end{CJK}
\end{enumerate}
\includegraphics[max width=\textwidth]{2022_04_18_7db0708508f26638f054g-122(1)}

\begin{CJK}{UTF8}{mj}若\end{CJK} $f$ \begin{CJK}{UTF8}{mj}为正交变换\end{CJK}, \begin{CJK}{UTF8}{mj}求证\end{CJK} $g$ \begin{CJK}{UTF8}{mj}也为正交变换\end{CJK}.

\begin{enumerate}
  \setcounter{enumi}{10}
  \item \begin{CJK}{UTF8}{mj}已知矩阵\end{CJK} $A, B$ \begin{CJK}{UTF8}{mj}均为半正定矩阵\end{CJK}, \begin{CJK}{UTF8}{mj}求证存在可逆矩阵\end{CJK} $C$, \begin{CJK}{UTF8}{mj}使得\end{CJK} $A, B$ \begin{CJK}{UTF8}{mj}同时合同对角化\end{CJK}.
\end{enumerate}
\section{第七暗 2021 年中国科学院大学真题}
\section{$7.1$ 数学分析}
\begin{enumerate}
  \item \begin{CJK}{UTF8}{mj}计算极限\end{CJK}\\
(1) $\lim _{n \rightarrow \infty} \frac{\left(1+\frac{1}{n}\right)^{n^{2}}}{\mathrm{e}^{n}}$\\
(2) $\lim _{x \rightarrow 0} \frac{(1+x)^{\frac{1}{x}}-(1+2 x)^{\frac{1}{2 x}}}{\sin x}$
\end{enumerate}
\includegraphics[max width=\textwidth]{2022_04_18_7db0708508f26638f054g-123}

\begin{enumerate}
  \setcounter{enumi}{2}
  \item \begin{CJK}{UTF8}{mj}设\end{CJK} $f(x)$ \begin{CJK}{UTF8}{mj}在\end{CJK} $\mathbb{R}$ \begin{CJK}{UTF8}{mj}上连续可微\end{CJK}, \begin{CJK}{UTF8}{mj}且\end{CJK} $f(0)=0, f(1)=1$, \begin{CJK}{UTF8}{mj}试证明\end{CJK}:
\end{enumerate}
$$
\int_{0}^{1}\left|f(x)-f^{\prime}(x)\right| \mathrm{d} x \geq \frac{1}{\mathrm{e}}
$$

\begin{enumerate}
  \setcounter{enumi}{3}
  \item \begin{CJK}{UTF8}{mj}设\end{CJK}
\end{enumerate}
$$
f_{n}(x)=x+x^{2}+\cdots+x^{n}(n=2,3,-\cdots)
$$
\begin{CJK}{UTF8}{mj}证明\end{CJK}: $f_{n}(x)=1$ \begin{CJK}{UTF8}{mj}在\end{CJK} $[0,+\infty)$ \begin{CJK}{UTF8}{mj}内有唯一的解\end{CJK}, \begin{CJK}{UTF8}{mj}并求\end{CJK} $\lim _{n \rightarrow \infty} x_{n}$.

\begin{enumerate}
  \setcounter{enumi}{4}
  \item \begin{CJK}{UTF8}{mj}计算积分\end{CJK}
\end{enumerate}
(1) $I=\int_{0}^{+\infty} \int_{0}^{+\infty}\left(\mathrm{e}^{-\left(x^{2}+y^{2}\right)} \mathrm{d} x \mathrm{~d} y\right.$

(2) $J=\int_{0}^{+\infty} \mathrm{e}^{-x^{2}} \mathrm{~d} x$

\begin{enumerate}
  \setcounter{enumi}{5}
  \item \begin{CJK}{UTF8}{mj}设\end{CJK} $f(x)$ \begin{CJK}{UTF8}{mj}在\end{CJK} $[a,+\infty)$ \begin{CJK}{UTF8}{mj}内有界可微\end{CJK}, \begin{CJK}{UTF8}{mj}且\end{CJK} $\lim _{x \rightarrow \infty} f^{\prime}(x)$ \begin{CJK}{UTF8}{mj}存在\end{CJK}, \begin{CJK}{UTF8}{mj}求证\end{CJK}: $\lim _{x \rightarrow \infty} f^{\prime}(x)=0$.

  \item \begin{CJK}{UTF8}{mj}判断\end{CJK}

\end{enumerate}
$$
\sum_{n=1}^{\infty}\left(1-\frac{x_{n}}{x_{n+1}}\right)
$$
\begin{CJK}{UTF8}{mj}的敛散性\end{CJK}, \begin{CJK}{UTF8}{mj}其中\end{CJK} $x_{n}(n \geq 1)$ \begin{CJK}{UTF8}{mj}是有界递增的正数列\end{CJK}.

\begin{enumerate}
  \setcounter{enumi}{7}
  \item \begin{CJK}{UTF8}{mj}设\end{CJK} $u$ \begin{CJK}{UTF8}{mj}关于\end{CJK} $x, y$ \begin{CJK}{UTF8}{mj}的偏导数存在\end{CJK}, \begin{CJK}{UTF8}{mj}且\end{CJK} $u=x+y \sin u$, \begin{CJK}{UTF8}{mj}证明\end{CJK}:
\end{enumerate}
$$
\frac{\partial u}{\partial y}=\sin u \frac{\partial u}{\partial x}
$$

\begin{enumerate}
  \setcounter{enumi}{8}
  \item \begin{CJK}{UTF8}{mj}计算第二类曲面积分\end{CJK}
\end{enumerate}
$$
I=\iint_{D} \frac{x^{2}+y^{2}-2}{\left(x^{2}+y^{2}\right)^{\frac{5}{2}}} \mathrm{~d} x \mathrm{~d} y
$$
\begin{CJK}{UTF8}{mj}其中\end{CJK} $D: x^{2}+y^{2} \geq 2, x \leq 1$.

\begin{enumerate}
  \setcounter{enumi}{9}
  \item \begin{CJK}{UTF8}{mj}证明\end{CJK}:
\end{enumerate}
$$
\left|\int_{a}^{a+1} \sin t^{2} \mathrm{~d} t\right| \leq \frac{1}{a}(a>0)
$$

\section{$7.2$ 高等代数}
\begin{enumerate}
  \item (15 \begin{CJK}{UTF8}{mj}分\end{CJK}) \begin{CJK}{UTF8}{mj}确定满足下面厇个条件的多项式\end{CJK}:
\end{enumerate}
$$
f(0)=3, f^{\prime}(0)=-1, f(1)=2, f^{\prime}(1)=3, f^{\prime \prime}(1)=-1
$$

\begin{enumerate}
  \setcounter{enumi}{2}
  \item (20 \begin{CJK}{UTF8}{mj}分\end{CJK}) \begin{CJK}{UTF8}{mj}计算行列式\end{CJK} $(n \geqslant 2)$
\end{enumerate}
$$
\left|\begin{array}{cccc}
2+a_{1} c_{1}+b_{1} d_{1} & a_{2} c_{1}+b_{2} d_{1} & \cdots & a_{n} c_{1}+b_{n} d_{1} \\
a_{1} c_{2}+b_{1} d_{2} & 2+a_{2} c_{2}+b_{2} d_{2} & \cdots & a_{n} c_{2}+b_{n} d_{2} \\
\vdots & \vdots & \ddots & \vdots \\
a_{1} c_{n}+b_{1} d_{n} & a_{2} c_{2}+b_{2} d_{2} & \cdots & 2+a_{n} c_{n}+b_{n} d_{n}
\end{array}\right|
$$

\begin{enumerate}
  \setcounter{enumi}{3}
  \item (20 \begin{CJK}{UTF8}{mj}分\end{CJK}) \begin{CJK}{UTF8}{mj}用正交线性变换将下面二次型化为标准形\end{CJK}
\end{enumerate}
$$
f\left(x_{1}, x_{2}, x_{3}\right)=x_{1}^{2}+2 x_{2}^{2}+3 x_{3}^{2}-4 x_{1} x_{2}-4 x_{2} x_{3}
$$

\begin{enumerate}
  \setcounter{enumi}{4}
  \item (15 \begin{CJK}{UTF8}{mj}分\end{CJK}) \begin{CJK}{UTF8}{mj}已知\end{CJK} $A$ \begin{CJK}{UTF8}{mj}是\end{CJK} $n$ \begin{CJK}{UTF8}{mj}阶实对称半正定矩阵\end{CJK}, \begin{CJK}{UTF8}{mj}证明\end{CJK} $A$ \begin{CJK}{UTF8}{mj}的伴随矩阵\end{CJK} $A^{*}$ \begin{CJK}{UTF8}{mj}也是半正定的\end{CJK}.

  \item (20 \begin{CJK}{UTF8}{mj}分\end{CJK}) \begin{CJK}{UTF8}{mj}设\end{CJK} $A=\left(a_{i j}\right)$ \begin{CJK}{UTF8}{mj}是\end{CJK} \begin{CJK}{UTF8}{mj}个\end{CJK} $n \times n$ \begin{CJK}{UTF8}{mj}秩为\end{CJK} $r$ \begin{CJK}{UTF8}{mj}的复矩阵\end{CJK}, \begin{CJK}{UTF8}{mj}且\end{CJK} $A$ \begin{CJK}{UTF8}{mj}的第\end{CJK} $r$ \begin{CJK}{UTF8}{mj}个顺序主子式不为\end{CJK} \begin{CJK}{UTF8}{mj}零\end{CJK}, \begin{CJK}{UTF8}{mj}即\end{CJK}

\end{enumerate}
$$
A\left(\begin{array}{c}
1,2, \cdots, r \\
1,2, \cdots, r
\end{array}\right) \neq 0
$$

\begin{enumerate}
  \setcounter{enumi}{6}
  \item (15 \begin{CJK}{UTF8}{mj}分\end{CJK}) \begin{CJK}{UTF8}{mj}已知\end{CJK} $v_{1}, \cdots, v_{m}$ \begin{CJK}{UTF8}{mj}为一组向量\end{CJK}, $\mathscr{A}$ \begin{CJK}{UTF8}{mj}为可逆复线性变换\end{CJK}, \begin{CJK}{UTF8}{mj}且对任意\end{CJK} $i=1, \cdots, m$, \begin{CJK}{UTF8}{mj}有\end{CJK}
\end{enumerate}
$$
\mathscr{A} v_{i} \in\left\{v_{1}, \cdots, v_{i}\right\}
$$
\begin{CJK}{UTF8}{mj}证明\end{CJK}: $\mathscr{A}$ \begin{CJK}{UTF8}{mj}可对角化\end{CJK}, \begin{CJK}{UTF8}{mj}且特征值为单位根\end{CJK}.

\begin{enumerate}
  \setcounter{enumi}{7}
  \item (20 \begin{CJK}{UTF8}{mj}分\end{CJK}) \begin{CJK}{UTF8}{mj}设\end{CJK} $M_{n}(\mathbb{C})$ \begin{CJK}{UTF8}{mj}是复数域上所有\end{CJK} $n$ \begin{CJK}{UTF8}{mj}阶方阵组成的线性空间\end{CJK}
\end{enumerate}
$$
\mathscr{T}: M_{n}(\mathbb{C}) \rightarrow \mathbb{C}
$$
\begin{CJK}{UTF8}{mj}是线性变换\end{CJK}, \begin{CJK}{UTF8}{mj}满足\end{CJK}
$$
\mathscr{T}(A B)=\mathscr{T}(B A)
$$
\begin{CJK}{UTF8}{mj}求证\end{CJK}:
$$
\exists \lambda \in \mathbb{C} \text {, s.t. } \forall A \in M_{n}(\mathrm{C}), \mathscr{T}(A)=\lambda \operatorname{tr}(A)
$$

\begin{enumerate}
  \setcounter{enumi}{8}
  \item (15 \begin{CJK}{UTF8}{mj}分\end{CJK}) \begin{CJK}{UTF8}{mj}已知\end{CJK} $A, B$ \begin{CJK}{UTF8}{mj}为\end{CJK} $n$ \begin{CJK}{UTF8}{mj}阶实对称矩阵且\end{CJK} $A B=B A$. \begin{CJK}{UTF8}{mj}证明\end{CJK}: \begin{CJK}{UTF8}{mj}存在正交矩阵\end{CJK} $Q$, \begin{CJK}{UTF8}{mj}使得\end{CJK} $Q^{\mathrm{T}} A Q$ \begin{CJK}{UTF8}{mj}与\end{CJK} $Q^{\mathrm{T}} B Q$ \begin{CJK}{UTF8}{mj}均为对角矩阵\end{CJK}.

  \item (20 \begin{CJK}{UTF8}{mj}分\end{CJK}) \begin{CJK}{UTF8}{mj}已知\end{CJK} $E$ \begin{CJK}{UTF8}{mj}为元素全为\end{CJK} 1 \begin{CJK}{UTF8}{mj}的\end{CJK} $n$ \begin{CJK}{UTF8}{mj}阶复方阵\end{CJK}, $A, B$ \begin{CJK}{UTF8}{mj}均为\end{CJK} $n$ \begin{CJK}{UTF8}{mj}阶复可逆方阵\end{CJK}, $W$ \begin{CJK}{UTF8}{mj}表示\end{CJK} $W$ \begin{CJK}{UTF8}{mj}的所有元素之和\end{CJK}, $m$ \begin{CJK}{UTF8}{mj}是不为\end{CJK} 1 \begin{CJK}{UTF8}{mj}的复数\end{CJK}.

\end{enumerate}
(1) \begin{CJK}{UTF8}{mj}若\end{CJK} $A+B=m E$, \begin{CJK}{UTF8}{mj}证明\end{CJK}:
$$
\left[1-m \sigma\left(B^{-1}\right)\right]=1
$$
(2) (1) \begin{CJK}{UTF8}{mj}的逆命題是否成立\end{CJK}? \begin{CJK}{UTF8}{mj}若成立\end{CJK}, \begin{CJK}{UTF8}{mj}请给出证明\end{CJK}; \begin{CJK}{UTF8}{mj}若不成立\end{CJK}, \begin{CJK}{UTF8}{mj}请举出反例\end{CJK}.

\section{第八章 2021 年浙江大学真题}
\section{$8.1$ 数学分析}
\begin{enumerate}
  \item (1) \begin{CJK}{UTF8}{mj}计算极限\end{CJK}
\end{enumerate}
$$
\lim _{n \rightarrow+\infty}\left[\frac{(2 n) ! !}{(2 n-1) ! !}\right]^{2} \frac{1}{2 n+1}
$$
(2) \begin{CJK}{UTF8}{mj}计算含参积分\end{CJK}
$$
I(a)=\int_{0}^{\frac{\pi}{2}} \frac{1}{\cos x} \ln \frac{1+a \cos x}{1-a \cos x} \mathrm{~d} x, \quad|a|<1
$$
(3) \begin{CJK}{UTF8}{mj}计算第二型曲面积分\end{CJK}
$$
\iint_{S} y(x-z) \mathrm{d} y \mathrm{~d} z+x^{2} \mathrm{~d} z \mathrm{~d} x+\left(y^{2}+x z\right) \mathrm{d} x \mathrm{~d} y
$$
\begin{CJK}{UTF8}{mj}其中\end{CJK} $S$ \begin{CJK}{UTF8}{mj}为\end{CJK} $z=5-x^{2}-y^{2}, z \geq 1$ \begin{CJK}{UTF8}{mj}的外侧\end{CJK}.

(4) \begin{CJK}{UTF8}{mj}求\end{CJK}
$$
f(x)=\arctan \frac{1-2 x}{1+2 x}
$$
\begin{CJK}{UTF8}{mj}在\end{CJK} $x=0$ \begin{CJK}{UTF8}{mj}处的幂级数展开\end{CJK}, \begin{CJK}{UTF8}{mj}并求\end{CJK} $\sum_{n=0}^{\infty} \frac{(-1)^{n}}{2 n+1}$.

\begin{enumerate}
  \setcounter{enumi}{2}
  \item (1) \begin{CJK}{UTF8}{mj}叙述这个精准定义\end{CJK}
\end{enumerate}
$$
\lim _{(x, y) \rightarrow(0, \infty)} f(x, y) \neq A
$$
(2) \begin{CJK}{UTF8}{mj}按定义证明\end{CJK}:
$$
\lim _{(x, y) \rightarrow(0,0)} \frac{x^{2}+y^{2}}{|x|+|y|}=0
$$

\begin{enumerate}
  \setcounter{enumi}{3}
  \item \begin{CJK}{UTF8}{mj}设\end{CJK} $f(x)$ \begin{CJK}{UTF8}{mj}在\end{CJK} $[a, b]$ \begin{CJK}{UTF8}{mj}上单调递增\end{CJK}, \begin{CJK}{UTF8}{mj}且\end{CJK} $f(a)>a, f(b)<b$, \begin{CJK}{UTF8}{mj}试证\end{CJK}: \begin{CJK}{UTF8}{mj}存在\end{CJK} $c \in(a, b)$ \begin{CJK}{UTF8}{mj}使得\end{CJK} $f(c)=c$.

  \item \begin{CJK}{UTF8}{mj}若\end{CJK} $f(x)$ \begin{CJK}{UTF8}{mj}在\end{CJK} $[1,+\infty)$ \begin{CJK}{UTF8}{mj}上一致连续\end{CJK}, \begin{CJK}{UTF8}{mj}且\end{CJK} $\int_{1}^{+\infty} f(x) \mathrm{d} x$ \begin{CJK}{UTF8}{mj}收玫\end{CJK}, \begin{CJK}{UTF8}{mj}则\end{CJK} $\lim _{x \rightarrow+\infty} f(x)=0$. 5. \begin{CJK}{UTF8}{mj}若\end{CJK} $f(x)$ \begin{CJK}{UTF8}{mj}在\end{CJK} $(-\infty,+\infty)$ \begin{CJK}{UTF8}{mj}上连续\end{CJK}, \begin{CJK}{UTF8}{mj}且\end{CJK} $g(x)=f(x) \int_{0}^{x} f(t) \mathrm{d} t$ \begin{CJK}{UTF8}{mj}在\end{CJK} $(-\infty,+\infty)$ \begin{CJK}{UTF8}{mj}上单调递减\end{CJK}, \begin{CJK}{UTF8}{mj}试证\end{CJK}: $f(x) \equiv 0$.

  \item \begin{CJK}{UTF8}{mj}若\end{CJK} $\sum_{i=1}^{n} p_{i}=1,0<p_{i}<1$, \begin{CJK}{UTF8}{mj}对任意实数\end{CJK} $x_{1}, x_{2}, \cdots, x_{n}$, \begin{CJK}{UTF8}{mj}试证\end{CJK}:

\end{enumerate}
\includegraphics[max width=\textwidth]{2022_04_18_7db0708508f26638f054g-127}
$$
\sum_{i=1}^{n} p_{i}\left(x_{i}-\ln p_{i}\right) \leq \ln \left(\sum_{i=1}^{n} \mathrm{e}^{x_{i}}\right)
$$

\begin{enumerate}
  \setcounter{enumi}{7}
  \item \begin{CJK}{UTF8}{mj}若\end{CJK} $\sum_{n=0}^{\infty} a_{n}$ \begin{CJK}{UTF8}{mj}绝对收敛\end{CJK}, \begin{CJK}{UTF8}{mj}和为\end{CJK} $A, \sum_{n=0}^{\infty} b_{n}$ \begin{CJK}{UTF8}{mj}收玫于\end{CJK} $B$, \begin{CJK}{UTF8}{mj}设\end{CJK} $C_{n}=\sum_{k=0}^{n} a_{k} b_{n-k}$, \begin{CJK}{UTF8}{mj}求证\end{CJK}:
\end{enumerate}
$$
\sum_{n=0}^{\infty} C_{n}=A B
$$

\begin{enumerate}
  \setcounter{enumi}{8}
  \item \begin{CJK}{UTF8}{mj}设\end{CJK} $f(x)$ \begin{CJK}{UTF8}{mj}在\end{CJK} $[a, b]$ \begin{CJK}{UTF8}{mj}上黎曼可积\end{CJK}, \begin{CJK}{UTF8}{mj}且在\end{CJK} $[a, b]$ \begin{CJK}{UTF8}{mj}上连续\end{CJK}, \begin{CJK}{UTF8}{mj}有\end{CJK} $f(x) \in[m, M], g(x)$ \begin{CJK}{UTF8}{mj}为连续函数\end{CJK}, $g(x)$ \begin{CJK}{UTF8}{mj}在\end{CJK} $[m, M]$ \begin{CJK}{UTF8}{mj}上连续\end{CJK}, \begin{CJK}{UTF8}{mj}试证\end{CJK}: $g(f(x))$ \begin{CJK}{UTF8}{mj}在\end{CJK} $[a, b]$ \begin{CJK}{UTF8}{mj}上黎曼可积\end{CJK}.

  \item \begin{CJK}{UTF8}{mj}设\end{CJK} $f$ \begin{CJK}{UTF8}{mj}是\end{CJK} $n$ \begin{CJK}{UTF8}{mj}元函数\end{CJK}, \begin{CJK}{UTF8}{mj}在\end{CJK} $x_{0}$ \begin{CJK}{UTF8}{mj}的某个邻域\end{CJK} $U\left(x_{0} ; \delta_{0}\right)$ \begin{CJK}{UTF8}{mj}内二阶可微\end{CJK}, $\nabla f\left(x_{0}\right)=0$, \begin{CJK}{UTF8}{mj}且对于任\end{CJK} \begin{CJK}{UTF8}{mj}意的单位向量\end{CJK} $a \in \mathbb{R}^{n}$ \begin{CJK}{UTF8}{mj}有\end{CJK} $(a \cdot \nabla)^{2} f\left(x_{0}\right)>0$, \begin{CJK}{UTF8}{mj}试证明\end{CJK}:

\end{enumerate}
(1) \begin{CJK}{UTF8}{mj}存在\end{CJK} $\delta \in\left(0, \delta_{0}\right)$ \begin{CJK}{UTF8}{mj}使得\end{CJK} $\left(\left(x-x_{0}\right) \cdot \nabla\right) f(x)>0$ \begin{CJK}{UTF8}{mj}对一切\end{CJK} $x \in U\left(x_{0} ; \delta\right) \backslash\left\{x_{0}\right\}$ \begin{CJK}{UTF8}{mj}都成立\end{CJK};

(2) $x_{0}$ \begin{CJK}{UTF8}{mj}是\end{CJK} $f$ \begin{CJK}{UTF8}{mj}的极小值点\end{CJK}.

\section{$8.2$ 高等代数}
\begin{enumerate}
  \item \begin{CJK}{UTF8}{mj}试求\end{CJK} $t$ \begin{CJK}{UTF8}{mj}的值\end{CJK}, \begin{CJK}{UTF8}{mj}使得多项式\end{CJK}
\end{enumerate}
$$
f(x)=x^{3}+6 x^{2}+t x+8
$$
\begin{CJK}{UTF8}{mj}具有重根\end{CJK}, \begin{CJK}{UTF8}{mj}并求出相应重根\end{CJK}.

\begin{enumerate}
  \setcounter{enumi}{2}
  \item \begin{CJK}{UTF8}{mj}已知可逆方阵\end{CJK} $A$ \begin{CJK}{UTF8}{mj}的逆为\end{CJK}
\end{enumerate}
$$
A^{-1}=\left(\begin{array}{cccc}
1 & 1 & 1 & 1 \\
-1 & 2 & 1 & -2 \\
1 & 4 & 1 & 4 \\
-1 & 8 & 1 & -8
\end{array}\right)
$$
\begin{CJK}{UTF8}{mj}以\end{CJK} $A_{i j}$ \begin{CJK}{UTF8}{mj}表示\end{CJK} $A$ \begin{CJK}{UTF8}{mj}中的元素\end{CJK} $a_{i j}$ \begin{CJK}{UTF8}{mj}的代数余子式\end{CJK}, \begin{CJK}{UTF8}{mj}求\end{CJK} $\sum_{i=1}^{4} \sum_{j=1}^{4} i A_{i j}$ 3. \begin{CJK}{UTF8}{mj}设\end{CJK} $\alpha_{1}, \cdots, \alpha_{s}$ \begin{CJK}{UTF8}{mj}为线性方程组\end{CJK} $A x=0$ \begin{CJK}{UTF8}{mj}的一组基础解系\end{CJK}, \begin{CJK}{UTF8}{mj}另有一组向量\end{CJK}
$$
\left\{\begin{array}{l}
\beta_{i}=\alpha_{i}+\alpha_{i+1} \\
\beta_{s}=\alpha_{s}+\alpha_{1}
\end{array}, i=1,2, \cdots, s-1\right.
$$
\begin{CJK}{UTF8}{mj}问何时\end{CJK} $\beta_{1}, \cdots, \beta_{s}$ \begin{CJK}{UTF8}{mj}也是\end{CJK} $A x=0$ \begin{CJK}{UTF8}{mj}的基础解系\end{CJK}? \begin{CJK}{UTF8}{mj}若\end{CJK} $\beta_{1}, \cdots, \beta_{s}$ \begin{CJK}{UTF8}{mj}不是\end{CJK} $A x=0$ \begin{CJK}{UTF8}{mj}的基础解系\end{CJK}, \begin{CJK}{UTF8}{mj}试求出其极大线性无关组\end{CJK}, \begin{CJK}{UTF8}{mj}并将其扩充为基础解系\end{CJK}.

\begin{enumerate}
  \setcounter{enumi}{4}
  \item \begin{CJK}{UTF8}{mj}已知\end{CJK} $A$ \begin{CJK}{UTF8}{mj}为\end{CJK} 3 \begin{CJK}{UTF8}{mj}行\end{CJK} 2 \begin{CJK}{UTF8}{mj}列的实矩阵\end{CJK}, $B$ \begin{CJK}{UTF8}{mj}为\end{CJK} 2 \begin{CJK}{UTF8}{mj}行\end{CJK} 3 \begin{CJK}{UTF8}{mj}列的实矩阵\end{CJK},
\end{enumerate}
$$
A B=\left(\begin{array}{ccc}
8 & 2 & -2 \\
2 & 5 & 4 \\
-2 & 4 & 5
\end{array}\right)(x
$$
\begin{CJK}{UTF8}{mj}试求\end{CJK}

(1) $(A B)^{2}$;

(2) $B A$ \begin{CJK}{UTF8}{mj}的极小多项式\end{CJK};

\includegraphics[max width=\textwidth]{2022_04_18_7db0708508f26638f054g-128}

(3) $B A$.

\begin{enumerate}
  \setcounter{enumi}{5}
  \item $n$ \begin{CJK}{UTF8}{mj}阶实方阵\end{CJK} $A=\left(a_{i j}\right)$, \begin{CJK}{UTF8}{mj}其中\end{CJK}
\end{enumerate}
$$
a_{i i}=0, i=1, \cdots, n ; a_{i j}+a_{j i}=1,1 \leqslant i<j \leqslant n
$$
\begin{CJK}{UTF8}{mj}证明\end{CJK}: $\operatorname{rank}(A) \geqslant n-1$.

\begin{enumerate}
  \setcounter{enumi}{6}
  \item \begin{CJK}{UTF8}{mj}设\end{CJK} $A$ \begin{CJK}{UTF8}{mj}为\end{CJK} $m \times n$ \begin{CJK}{UTF8}{mj}矩阵\end{CJK}, $b$ \begin{CJK}{UTF8}{mj}为\end{CJK} $m$ \begin{CJK}{UTF8}{mj}维列向量\end{CJK}, \begin{CJK}{UTF8}{mj}证明\end{CJK}: $A x=b$ \begin{CJK}{UTF8}{mj}有解的充要条件是满足\end{CJK} $A^{\mathrm{T}} z=0$ \begin{CJK}{UTF8}{mj}的\end{CJK} $m$ \begin{CJK}{UTF8}{mj}维列向量\end{CJK} $z$ \begin{CJK}{UTF8}{mj}也一定满足\end{CJK} $b^{\mathrm{T}} z=0$.

  \item \begin{CJK}{UTF8}{mj}若四阶实对称方阵\end{CJK} $A$ \begin{CJK}{UTF8}{mj}的行列式等于\end{CJK} $2, A$ \begin{CJK}{UTF8}{mj}有两个特征值\end{CJK} $1,-1$, \begin{CJK}{UTF8}{mj}相应的特征子空间为\end{CJK}

\end{enumerate}
$$
L_{1}=L\left(\alpha_{1}, \alpha_{2}\right), L_{2}=L\left(\alpha_{3}\right)
$$
\begin{CJK}{UTF8}{mj}其中\end{CJK}
$$
\alpha_{1}=(1,1,-1,-1)^{\mathrm{T}}, \alpha_{2}=(1,-1,1,1)^{\mathrm{T}}, \alpha_{3}=(0,1,1,0)^{\mathrm{T}}
$$
\begin{CJK}{UTF8}{mj}求\end{CJK} $A$ \begin{CJK}{UTF8}{mj}的伴随矩阵\end{CJK} $A^{*}$, \begin{CJK}{UTF8}{mj}并用正交阵对角化\end{CJK} $A$.

\begin{enumerate}
  \setcounter{enumi}{8}
  \item \begin{CJK}{UTF8}{mj}线性空间\end{CJK} $V$ \begin{CJK}{UTF8}{mj}上线性变换\end{CJK} $\varphi$ \begin{CJK}{UTF8}{mj}的特征多项式为\end{CJK}
\end{enumerate}
$$
(\lambda-2)^{6}(\lambda+2)^{4}
$$
\begin{CJK}{UTF8}{mj}试将\end{CJK} $V$ \begin{CJK}{UTF8}{mj}分解成里两个非平凡的\end{CJK} $\varphi$-\begin{CJK}{UTF8}{mj}不变子空间的直和\end{CJK}. 9. \begin{CJK}{UTF8}{mj}对于实矩阵\end{CJK}
$$
A=\left(\begin{array}{llll}
1 & x & 4 & 2 \\
0 & 1 & 3 & 3 \\
0 & 0 & 2 & y \\
0 & 0 & 0 & 2
\end{array}\right)
$$
(1) \begin{CJK}{UTF8}{mj}当\end{CJK} $x$ \begin{CJK}{UTF8}{mj}和\end{CJK} $y$ \begin{CJK}{UTF8}{mj}为何值时\end{CJK}, \begin{CJK}{UTF8}{mj}方阵\end{CJK} $A$ \begin{CJK}{UTF8}{mj}可对角化\end{CJK}?

(2) \begin{CJK}{UTF8}{mj}当\end{CJK} $x=0, y=1$ \begin{CJK}{UTF8}{mj}时\end{CJK}, \begin{CJK}{UTF8}{mj}求\end{CJK} $A$ \begin{CJK}{UTF8}{mj}的初等因子\end{CJK}、\begin{CJK}{UTF8}{mj}不变因子和\end{CJK} Jordan \begin{CJK}{UTF8}{mj}标准形\end{CJK}.

\begin{enumerate}
  \setcounter{enumi}{10}
  \item \begin{CJK}{UTF8}{mj}设\end{CJK} $V$ \begin{CJK}{UTF8}{mj}和\end{CJK} $W$ \begin{CJK}{UTF8}{mj}都是有限维线性空间\end{CJK}, $\varphi: V \rightarrow W$ \begin{CJK}{UTF8}{mj}是线性映射\end{CJK} \begin{CJK}{UTF8}{mj}证明\end{CJK}:
\end{enumerate}
(1) $\varphi$ \begin{CJK}{UTF8}{mj}为满映射的充分必要条件是存在映射\end{CJK} $\psi: W \nrightarrow V$, \begin{CJK}{UTF8}{mj}使得\end{CJK} $\varphi \psi$ \begin{CJK}{UTF8}{mj}是\end{CJK} $W$ \begin{CJK}{UTF8}{mj}上的恒等映\end{CJK} \begin{CJK}{UTF8}{mj}射\end{CJK};

(2) $\varphi$ \begin{CJK}{UTF8}{mj}为单映射的充分必要条件是存在映射\end{CJK} $\tau: \boldsymbol{W} \rightarrow \boldsymbol{V}$, \begin{CJK}{UTF8}{mj}使得\end{CJK} $\tau \varphi$ \begin{CJK}{UTF8}{mj}是\end{CJK} $\boldsymbol{V}$ \begin{CJK}{UTF8}{mj}上的恒等映\end{CJK} \begin{CJK}{UTF8}{mj}射\end{CJK};

(3) $\varphi$ \begin{CJK}{UTF8}{mj}为同构的充分必要条件是存在映射\end{CJK} $\psi: W \rightarrow V$ \begin{CJK}{UTF8}{mj}和\end{CJK} $\tau: W \rightarrow V$, \begin{CJK}{UTF8}{mj}使得\end{CJK} $\varphi \psi$ \begin{CJK}{UTF8}{mj}是\end{CJK} $W$ \begin{CJK}{UTF8}{mj}上的恒等映射\end{CJK}, $\tau \varphi$ \begin{CJK}{UTF8}{mj}是\end{CJK} $V$ \begin{CJK}{UTF8}{mj}上的恒等映射\end{CJK}.

\section{第九章 2021 年华东师范大学真题}
\section{$9.1$ 数学分析}
\begin{CJK}{UTF8}{mj}一\end{CJK}、\begin{CJK}{UTF8}{mj}判断题\end{CJK} (\begin{CJK}{UTF8}{mj}每题\end{CJK} 6 \begin{CJK}{UTF8}{mj}分\end{CJK}, \begin{CJK}{UTF8}{mj}共\end{CJK} 30 \begin{CJK}{UTF8}{mj}分\end{CJK})

(1) \begin{CJK}{UTF8}{mj}数列\end{CJK} $\left\{a_{n}\right\}$ \begin{CJK}{UTF8}{mj}收敛的充要条件是\end{CJK} $\forall \varepsilon>0, \exists N>0$, \begin{CJK}{UTF8}{mj}当\end{CJK} $n>N$ \begin{CJK}{UTF8}{mj}时\end{CJK}, \begin{CJK}{UTF8}{mj}有\end{CJK} $\left|a_{n}-a_{2 n}\right|<\varepsilon$.

(2) \begin{CJK}{UTF8}{mj}若函数\end{CJK} $f(x)$ \begin{CJK}{UTF8}{mj}在闭区间\end{CJK} $[0,2]$ \begin{CJK}{UTF8}{mj}上连续\end{CJK}, \begin{CJK}{UTF8}{mj}且有\end{CJK} $f(0)=f(2)$, \begin{CJK}{UTF8}{mj}则方程\end{CJK} $f(x)-f(x+1)=0$ \begin{CJK}{UTF8}{mj}有解\end{CJK}.

(3) \begin{CJK}{UTF8}{mj}若函数\end{CJK} $f(x)$ \begin{CJK}{UTF8}{mj}在\end{CJK} $[a, b]$ \begin{CJK}{UTF8}{mj}上存在原函数\end{CJK}, \begin{CJK}{UTF8}{mj}则\end{CJK} $f(x)$ \begin{CJK}{UTF8}{mj}在\end{CJK} $[a, b]$ \begin{CJK}{UTF8}{mj}上黎曼可积\end{CJK}.

(4) \begin{CJK}{UTF8}{mj}若无穷积分\end{CJK} $\int_{1}^{+\infty} f(x) \mathrm{d} x$ \begin{CJK}{UTF8}{mj}收敛\end{CJK}, \begin{CJK}{UTF8}{mj}且\end{CJK} $f(x)$ \begin{CJK}{UTF8}{mj}在\end{CJK} $[1,+\infty)$ \begin{CJK}{UTF8}{mj}上连续\end{CJK}, \begin{CJK}{UTF8}{mj}则\end{CJK} $\lim _{x \rightarrow+\infty} f(x)=0$.

(5) \begin{CJK}{UTF8}{mj}若函数\end{CJK} $f(x)$ \begin{CJK}{UTF8}{mj}在\end{CJK} $(-1,1)$ \begin{CJK}{UTF8}{mj}上有定义\end{CJK}, \begin{CJK}{UTF8}{mj}在\end{CJK} $(-1,0) \cup(0,1)$ \begin{CJK}{UTF8}{mj}上可导\end{CJK}, \begin{CJK}{UTF8}{mj}且\end{CJK} $\lim _{x \rightarrow 0} f^{\prime}(x)$ \begin{CJK}{UTF8}{mj}存在\end{CJK}, \begin{CJK}{UTF8}{mj}则\end{CJK} $f^{\prime}(0)$ \begin{CJK}{UTF8}{mj}也存在\end{CJK}.

\begin{CJK}{UTF8}{mj}二\end{CJK}、\begin{CJK}{UTF8}{mj}计算题\end{CJK} (\begin{CJK}{UTF8}{mj}每题\end{CJK} 9 \begin{CJK}{UTF8}{mj}分\end{CJK}, \begin{CJK}{UTF8}{mj}共\end{CJK} 45 \begin{CJK}{UTF8}{mj}分\end{CJK})

(1) \begin{CJK}{UTF8}{mj}计算极限\end{CJK}
$$
\int \lim _{n \rightarrow \infty} \frac{\sqrt[n]{n(n+1)(n+2) \cdots(2 n-1)}}{n}
$$
(2) \begin{CJK}{UTF8}{mj}假设二元函数满足\end{CJK} $u_{x x}+u_{y y}=0$, \begin{CJK}{UTF8}{mj}令\end{CJK} $v=f\left(\frac{x}{x^{2}+y^{2}}, \frac{y}{x^{2}+y^{2}}\right)$, \begin{CJK}{UTF8}{mj}且\end{CJK} $x^{2}+y^{2} \neq 0$, \begin{CJK}{UTF8}{mj}试证\end{CJK}: $v_{x x}+v_{y y}=0$.

(3) \begin{CJK}{UTF8}{mj}计算第二类曲线积分\end{CJK}
$$
\int_{L} \frac{\mathrm{d} y-\mathrm{d} x}{x-y+1}
$$
\begin{CJK}{UTF8}{mj}其中\end{CJK} $L: x^{2}+y^{2}=2 x$ \begin{CJK}{UTF8}{mj}沿着\end{CJK} $x$ \begin{CJK}{UTF8}{mj}增长方向\end{CJK}.

(4) \begin{CJK}{UTF8}{mj}将\end{CJK} $f(x)=(x-1)^{2}$ \begin{CJK}{UTF8}{mj}在\end{CJK} $(0,1)$ \begin{CJK}{UTF8}{mj}展开余弦级数\end{CJK}, \begin{CJK}{UTF8}{mj}并证明\end{CJK}:
$$
\sum_{n=1}^{\infty} \frac{1}{n^{2}}=\frac{\pi^{2}}{6}
$$
(5) \begin{CJK}{UTF8}{mj}设\end{CJK} $f(x, y)$ \begin{CJK}{UTF8}{mj}在\end{CJK} $D=\left\{(x, y) \mid x^{2}+y^{2} \leqslant 1\right\}$ \begin{CJK}{UTF8}{mj}上连续函数\end{CJK}, \begin{CJK}{UTF8}{mj}计算\end{CJK}
$$
\lim _{n \rightarrow \infty}\left(\iint_{D} f^{n}(x, y) \mathrm{d} x \mathrm{~d} y\right)^{1 / n}
$$
\begin{CJK}{UTF8}{mj}三\end{CJK}、\begin{CJK}{UTF8}{mj}证明题\end{CJK} (\begin{CJK}{UTF8}{mj}每题\end{CJK} 15 \begin{CJK}{UTF8}{mj}分\end{CJK}, \begin{CJK}{UTF8}{mj}共\end{CJK} 75 \begin{CJK}{UTF8}{mj}分\end{CJK})

(1) \begin{CJK}{UTF8}{mj}若数项级数\end{CJK} $\sum_{n=0}^{+\infty} a_{n}$ \begin{CJK}{UTF8}{mj}收玫\end{CJK}, \begin{CJK}{UTF8}{mj}且\end{CJK} $\left\{a_{n}\right\}$ \begin{CJK}{UTF8}{mj}单调\end{CJK}, \begin{CJK}{UTF8}{mj}则\end{CJK} $\lim _{n \rightarrow \infty} n a_{n}=0$.

(2) \begin{CJK}{UTF8}{mj}证明\end{CJK}:
$$
\int_{0}^{+\infty} \frac{\mathrm{e}^{-t} \sin (t x)}{t} \mathrm{~d} t=\arctan x
$$
(3) \begin{CJK}{UTF8}{mj}设函数\end{CJK} $f(u)$ \begin{CJK}{UTF8}{mj}在闭区间\end{CJK} $I$ \begin{CJK}{UTF8}{mj}上连续\end{CJK}, $\left\{g_{n}\right\}$ \begin{CJK}{UTF8}{mj}在\end{CJK} $[a, b]$ \begin{CJK}{UTF8}{mj}上致收玫\end{CJK}, \begin{CJK}{UTF8}{mj}且\end{CJK} $\forall n>N, x \in[a, b]$ \begin{CJK}{UTF8}{mj}有\end{CJK} $\left\{g_{n}(x)\right\} \in I$, \begin{CJK}{UTF8}{mj}试证\end{CJK}: $\left\{f\left(g_{n}(x)\right)\right\}$ \begin{CJK}{UTF8}{mj}一致收敛\end{CJK}.

(4) \begin{CJK}{UTF8}{mj}若函数\end{CJK} $f(x)$ \begin{CJK}{UTF8}{mj}在\end{CJK} $[a, b]$ \begin{CJK}{UTF8}{mj}连续\end{CJK}, \begin{CJK}{UTF8}{mj}且存在常数\end{CJK} $r \in(0,1)$ \begin{CJK}{UTF8}{mj}满足\end{CJK} $\forall x \in[a, b], \exists y \in[a, b]$ \begin{CJK}{UTF8}{mj}使\end{CJK} \begin{CJK}{UTF8}{mj}得\end{CJK} $|f(x)| \leq r|f(y)|$, \begin{CJK}{UTF8}{mj}试证\end{CJK}: \begin{CJK}{UTF8}{mj}存在\end{CJK} $\xi \in[a, b]$ \begin{CJK}{UTF8}{mj}使得\end{CJK} $f(\xi)=0$.

(5) \begin{CJK}{UTF8}{mj}设\end{CJK} $f(x)$ \begin{CJK}{UTF8}{mj}在\end{CJK} $[a,+\infty)$ \begin{CJK}{UTF8}{mj}上连续可微\end{CJK}, \begin{CJK}{UTF8}{mj}且\end{CJK}
$$
f(x+1)-f(x)=f^{\prime}(x), \quad \forall x \in[a,+\infty), \lim _{x \rightarrow+\infty} f^{\prime}(x)=A
$$
\begin{CJK}{UTF8}{mj}试证\end{CJK}: $f^{\prime}(x) \equiv A, \forall x \in[a,+\infty)$

\section{$9.2$ 高等代数}
\begin{enumerate}
  \item (15 \begin{CJK}{UTF8}{mj}分\end{CJK}) \begin{CJK}{UTF8}{mj}设\end{CJK} $\mathbb{F}$ \begin{CJK}{UTF8}{mj}为数域\end{CJK}, \begin{CJK}{UTF8}{mj}且\end{CJK}
\end{enumerate}
$$
A \in M^{m \times n}(\mathbb{F}), \beta \in M^{m \times 1}(\mathbb{F})
$$
\begin{CJK}{UTF8}{mj}记\end{CJK} $\operatorname{rank}(A)=r$, \begin{CJK}{UTF8}{mj}则线性方程组\end{CJK} $A x=\beta$ \begin{CJK}{UTF8}{mj}有多少个线性无关的解\end{CJK}, \begin{CJK}{UTF8}{mj}并说明理由\end{CJK}.

\begin{enumerate}
  \setcounter{enumi}{2}
  \item (15 \begin{CJK}{UTF8}{mj}分\end{CJK}) \begin{CJK}{UTF8}{mj}设\end{CJK} $2 n$ \begin{CJK}{UTF8}{mj}阶方阵\end{CJK}
\end{enumerate}
$$
S=\left(\begin{array}{cc}
0 & E_{n} \\
-E_{n} & 0
\end{array}\right)
$$
\begin{CJK}{UTF8}{mj}给出复线性空间\end{CJK}
$$
S P_{n}=\left\{X \in M^{2 n \times 2 n}(\mathbb{C}) ; S X=-X^{\mathrm{T}} S\right\}
$$
\begin{CJK}{UTF8}{mj}的一组基\end{CJK}, \begin{CJK}{UTF8}{mj}井计算其维数\end{CJK}.

\begin{enumerate}
  \setcounter{enumi}{3}
  \item (15 \begin{CJK}{UTF8}{mj}分\end{CJK}) \begin{CJK}{UTF8}{mj}设\end{CJK} $n$ \begin{CJK}{UTF8}{mj}阶矩阵\end{CJK} $A(t)=\left(a_{i j}(t)\right)_{n \times n}$ \begin{CJK}{UTF8}{mj}中元素\end{CJK} $a_{i j}(t)$ \begin{CJK}{UTF8}{mj}是实变量\end{CJK} $t$ \begin{CJK}{UTF8}{mj}的可微函数\end{CJK}. \begin{CJK}{UTF8}{mj}记\end{CJK}
\end{enumerate}
$$
A^{\prime}(t)=\left(\frac{\mathrm{d}}{\mathrm{d} t} a_{i j}(t)\right)_{n \times n}
$$
\begin{CJK}{UTF8}{mj}证明\end{CJK}: \begin{CJK}{UTF8}{mj}若对\end{CJK} $\forall t \in \mathbb{R},|A(t)|>0$, \begin{CJK}{UTF8}{mj}则\end{CJK}
$$
|A(t)| \frac{\mathrm{d}}{\mathrm{d} t} \ln |A(t)|=\operatorname{tr}\left(A^{-1}(t) A^{\prime}(t)\right)
$$

\begin{enumerate}
  \setcounter{enumi}{4}
  \item (15 \begin{CJK}{UTF8}{mj}分\end{CJK}) \begin{CJK}{UTF8}{mj}若\end{CJK} $n$ \begin{CJK}{UTF8}{mj}阶复矩阵\end{CJK} $A, B$ \begin{CJK}{UTF8}{mj}满足\end{CJK} $A B=B A$, \begin{CJK}{UTF8}{mj}且\end{CJK} $B$ \begin{CJK}{UTF8}{mj}有\end{CJK} $n$ \begin{CJK}{UTF8}{mj}个不同的特征值\end{CJK}, \begin{CJK}{UTF8}{mj}证明\end{CJK}: $A$ \begin{CJK}{UTF8}{mj}可对角化\end{CJK}.

  \item (15 \begin{CJK}{UTF8}{mj}分\end{CJK}) \begin{CJK}{UTF8}{mj}设\end{CJK} $c_{1}, c_{2}, c_{3}$ \begin{CJK}{UTF8}{mj}是多项式\end{CJK}

\end{enumerate}
$$
f(x)=2 x^{3}-4 x^{2}+6 x-1
$$
\begin{CJK}{UTF8}{mj}的三个复根\end{CJK}. \begin{CJK}{UTF8}{mj}求\end{CJK}
$$
\left(c_{1} c_{2}+c_{3}^{2}\right)\left(c_{2} c_{3}+c_{1}^{2}\right)\left(c_{1} c_{3}+c_{2}^{2}\right)
$$

\begin{enumerate}
  \setcounter{enumi}{6}
  \item (20 \begin{CJK}{UTF8}{mj}分\end{CJK}) \begin{CJK}{UTF8}{mj}在\end{CJK} $\mathbb{R}^{2}$ \begin{CJK}{UTF8}{mj}上\end{CJK}
\end{enumerate}
$$
f(x, y)=a_{11} x^{2}+2 a_{12} x y+a_{22} y^{2}+2 b_{1} x+2 b_{2} y+c .
$$
\begin{CJK}{UTF8}{mj}令\end{CJK}
$$
A_{f}=\left(\begin{array}{cc}
a_{11} & a_{12} \\
a_{21} & a_{22}
\end{array}\right), B_{f}=\left(\begin{array}{ccc}
a_{11} & a_{12} & b_{1} \\
a_{21} & a_{22} & b_{2} \\
b_{1} & b_{2} & c
\end{array}\right)
$$
\begin{CJK}{UTF8}{mj}证明\end{CJK}: \begin{CJK}{UTF8}{mj}函数\end{CJK} $f$ \begin{CJK}{UTF8}{mj}在坐标变换\end{CJK}
$$
\left(\begin{array}{l}
x^{\prime} \\
y^{\prime}
\end{array}\right)=Q\left(\begin{array}{l}
x \\
y
\end{array}\right)+\left(\begin{array}{l}
d_{1} \\
d_{2}
\end{array}\right)
$$
\begin{CJK}{UTF8}{mj}下\end{CJK},
$$
\operatorname{tr}\left(A_{f}\right), \operatorname{det}\left(A_{f}\right), \operatorname{det}\left(B_{f}\right)
$$
\begin{CJK}{UTF8}{mj}保持不变\end{CJK}, \begin{CJK}{UTF8}{mj}其中\end{CJK} $A$ \begin{CJK}{UTF8}{mj}是二阶正交矩阵\end{CJK}.

\begin{enumerate}
  \setcounter{enumi}{7}
  \item (20 \begin{CJK}{UTF8}{mj}分\end{CJK}) \begin{CJK}{UTF8}{mj}设实矩阵\end{CJK}
\end{enumerate}
$$
A=\left(\begin{array}{ll}
a & b \\
c & d
\end{array}\right), a, b, c, d>0
$$
\begin{CJK}{UTF8}{mj}证明\end{CJK}: \begin{CJK}{UTF8}{mj}一定存在\end{CJK} $A$ \begin{CJK}{UTF8}{mj}的特征向量\end{CJK} $\left(\begin{array}{l}x \\ y\end{array}\right) \in \mathbb{R}^{2}$, \begin{CJK}{UTF8}{mj}其中\end{CJK} $x, y>0$.

\begin{enumerate}
  \setcounter{enumi}{8}
  \item (15 \begin{CJK}{UTF8}{mj}分\end{CJK}) \begin{CJK}{UTF8}{mj}设\end{CJK} 6 \begin{CJK}{UTF8}{mj}阶复矩阵\end{CJK} $A, B$ \begin{CJK}{UTF8}{mj}是幂零矩阵\end{CJK}, \begin{CJK}{UTF8}{mj}且有相同的秩和最小多项式\end{CJK}, \begin{CJK}{UTF8}{mj}证明\end{CJK}: $A, B$ \begin{CJK}{UTF8}{mj}相\end{CJK} \begin{CJK}{UTF8}{mj}似\end{CJK}.

  \item (20 \begin{CJK}{UTF8}{mj}分\end{CJK}) \begin{CJK}{UTF8}{mj}设\end{CJK} $A$ \begin{CJK}{UTF8}{mj}是\end{CJK} $n$ \begin{CJK}{UTF8}{mj}阶实矩阵\end{CJK}, $B$ \begin{CJK}{UTF8}{mj}是\end{CJK} $n$ \begin{CJK}{UTF8}{mj}阶实对称矩阵\end{CJK}.

\end{enumerate}
(1) \begin{CJK}{UTF8}{mj}证明\end{CJK}: \begin{CJK}{UTF8}{mj}存在唯一\end{CJK} $n$ \begin{CJK}{UTF8}{mj}阶实矩阵\end{CJK} $C$ \begin{CJK}{UTF8}{mj}满足\end{CJK} $B C+C B=A$.

(2) \begin{CJK}{UTF8}{mj}证明\end{CJK}: \begin{CJK}{UTF8}{mj}对\end{CJK} (1) \begin{CJK}{UTF8}{mj}中实矩阵\end{CJK} $C$ \begin{CJK}{UTF8}{mj}有\end{CJK} $B C=C B$ \begin{CJK}{UTF8}{mj}当且仅当\end{CJK} $A B=B A$.

\section{第十章 2021 年中国科学技术大学真题}
\section{$10.1$ 数学分析}
\begin{enumerate}
  \item \begin{CJK}{UTF8}{mj}计算题\end{CJK} (\begin{CJK}{UTF8}{mj}每题\end{CJK} 10 \begin{CJK}{UTF8}{mj}分\end{CJK}, \begin{CJK}{UTF8}{mj}共\end{CJK} 50 \begin{CJK}{UTF8}{mj}分\end{CJK})\\
(1) $\lim _{x \rightarrow 0} \frac{\left(1+x^{2}\right)^{2}-\cos x}{\sin ^{2} x}$\\
(3) \begin{CJK}{UTF8}{mj}若\end{CJK}
\end{enumerate}
$$
z=\frac{1}{x^{2}} f\left(x^{2} y\right)+x y g(x+y)
$$
\begin{CJK}{UTF8}{mj}求\end{CJK} $\frac{\partial^{2} z}{\partial x^{2}}, \frac{\partial z}{\partial x \partial y}$.

(4) \begin{CJK}{UTF8}{mj}设\end{CJK} $f(t)=\int_{x}^{4 x} \sin \left((x-t)^{2}\right) \mathrm{d} t$, \begin{CJK}{UTF8}{mj}求\end{CJK} $f^{\prime}(t)$

(5) \begin{CJK}{UTF8}{mj}已知\end{CJK} $\gamma$ \begin{CJK}{UTF8}{mj}是由\end{CJK}
$$
y^{2}=\frac{1}{3} x^{2} \sqrt{1-4 x}
$$
\begin{CJK}{UTF8}{mj}确定\end{CJK}, \begin{CJK}{UTF8}{mj}且\end{CJK} $y \geq 0, x \in\left[0, \frac{1}{4}\right]$, \begin{CJK}{UTF8}{mj}求\end{CJK} $\gamma$ \begin{CJK}{UTF8}{mj}的弧长\end{CJK}.

\begin{enumerate}
  \setcounter{enumi}{2}
  \item (15 \begin{CJK}{UTF8}{mj}分\end{CJK}) \begin{CJK}{UTF8}{mj}计算第二型曲面积分\end{CJK}
\end{enumerate}
$$
\iint_{S} \frac{x \mathrm{~d} y \mathrm{~d} z+z^{4} \mathrm{~d} x \mathrm{~d} y}{x^{2}+y^{2}+z^{2}}
$$
\begin{CJK}{UTF8}{mj}其中\end{CJK} $S$ \begin{CJK}{UTF8}{mj}是由\end{CJK} $x^{2}+y^{2}=1, z=1, z=-1$ \begin{CJK}{UTF8}{mj}所围成的立体外侧\end{CJK}.

\begin{enumerate}
  \setcounter{enumi}{3}
  \item (15 \begin{CJK}{UTF8}{mj}分\end{CJK}) \begin{CJK}{UTF8}{mj}若\end{CJK} $\sum_{n=1}^{\infty} \frac{(1-x) x^{n}}{1-x^{2 n}} \cos n x$, \begin{CJK}{UTF8}{mj}试证\end{CJK}:
\end{enumerate}
(1) \begin{CJK}{UTF8}{mj}级数在\end{CJK} $\left[0, \frac{1}{2}\right]$ \begin{CJK}{UTF8}{mj}上一致收敛\end{CJK};

(2) \begin{CJK}{UTF8}{mj}级数在\end{CJK} $\left(\frac{1}{2}, 1\right)$ \begin{CJK}{UTF8}{mj}上一致收玫\end{CJK}. 4. (20 \begin{CJK}{UTF8}{mj}分\end{CJK}) \begin{CJK}{UTF8}{mj}试求\end{CJK} $f(x)=\cos (\alpha x)$ \begin{CJK}{UTF8}{mj}在\end{CJK} $x \in[-\pi, \pi]$ \begin{CJK}{UTF8}{mj}上的\end{CJK} Fourier \begin{CJK}{UTF8}{mj}级数展开\end{CJK}, \begin{CJK}{UTF8}{mj}其中\end{CJK} $\alpha$ \begin{CJK}{UTF8}{mj}不是整\end{CJK} \begin{CJK}{UTF8}{mj}数\end{CJK}, \begin{CJK}{UTF8}{mj}并由此证明\end{CJK}:\\
(1) $\sum_{n=-\infty}^{+\infty} \frac{(-1)^{n}}{n+\alpha}=\frac{\pi}{\sin (\alpha \pi)}$\\
(2) $\sum_{n=0}^{\infty} \frac{1}{\left(4 n^{2}-1\right)^{2}}=\frac{\pi^{2}-8}{16}$

\begin{enumerate}
  \setcounter{enumi}{5}
  \item (10 \begin{CJK}{UTF8}{mj}分\end{CJK}) \begin{CJK}{UTF8}{mj}若\end{CJK} $f(x)$ \begin{CJK}{UTF8}{mj}在\end{CJK} $[0,1]$ \begin{CJK}{UTF8}{mj}上连续\end{CJK}, \begin{CJK}{UTF8}{mj}则存在\end{CJK} $c \in(0,1)$ \begin{CJK}{UTF8}{mj}使得\end{CJK}
\end{enumerate}
$$
\int_{0}^{c} f(x) \mathrm{d} x=(1-c) f(c)
$$

\begin{enumerate}
  \setcounter{enumi}{6}
  \item (15 \begin{CJK}{UTF8}{mj}分\end{CJK}) \begin{CJK}{UTF8}{mj}求实数二次多项式\end{CJK} $p(x)$, \begin{CJK}{UTF8}{mj}使得\end{CJK}
\end{enumerate}
$$
\left|p(x)-\frac{1}{x-3}\right|<0.02
$$
\begin{CJK}{UTF8}{mj}成立\end{CJK}, \begin{CJK}{UTF8}{mj}对任意的\end{CJK} $x \in[-1,1]$.

\begin{enumerate}
  \setcounter{enumi}{7}
  \item (15 \begin{CJK}{UTF8}{mj}分\end{CJK}) \begin{CJK}{UTF8}{mj}若\end{CJK} $f(x)$ \begin{CJK}{UTF8}{mj}在\end{CJK} $[0,1]$ \begin{CJK}{UTF8}{mj}上的连续函数\end{CJK}, \begin{CJK}{UTF8}{mj}且满足\end{CJK}
\end{enumerate}
$$
\int_{0}^{c} f(x) \mathrm{d} x=1, \int_{0}^{1} x f(x) \mathrm{d} x=\frac{27}{2}
$$
\begin{CJK}{UTF8}{mj}试证\end{CJK}:
$$
\int_{0}^{1} f^{2}(x) \mathrm{d} x>2021
$$

\begin{enumerate}
  \setcounter{enumi}{8}
  \item (20 \begin{CJK}{UTF8}{mj}分\end{CJK}) \begin{CJK}{UTF8}{mj}若\end{CJK} $f(x)$ \begin{CJK}{UTF8}{mj}在\end{CJK} $\mathbb{R}$ \begin{CJK}{UTF8}{mj}上有界连续\end{CJK}, \begin{CJK}{UTF8}{mj}且\end{CJK}
\end{enumerate}
$$
\lim _{h \rightarrow 0} \sup _{x \in \mathbb{R}}|f(x+h)-2 f(x)+f(x-h)|=0
$$
\begin{CJK}{UTF8}{mj}则\end{CJK} $f(x)$ \begin{CJK}{UTF8}{mj}在\end{CJK} $\mathbb{R}$ \begin{CJK}{UTF8}{mj}上一致连续\end{CJK}.

\section{$10.2$ 高等代数}
\begin{CJK}{UTF8}{mj}一\end{CJK}、\begin{CJK}{UTF8}{mj}填空题\end{CJK} (\begin{CJK}{UTF8}{mj}每空\end{CJK} 5 \begin{CJK}{UTF8}{mj}分\end{CJK}, \begin{CJK}{UTF8}{mj}共\end{CJK} 40 \begin{CJK}{UTF8}{mj}分\end{CJK})

\begin{enumerate}
  \item \begin{CJK}{UTF8}{mj}已知三点\end{CJK} $A(1,1,1), B(2,1,2), C(1,-1,0)$, \begin{CJK}{UTF8}{mj}求\end{CJK} $S_{\Delta} A B C=$ , \begin{CJK}{UTF8}{mj}若\end{CJK} $P(0, a, 1)$ \begin{CJK}{UTF8}{mj}与\end{CJK} \begin{CJK}{UTF8}{mj}点\end{CJK} $A, B, C$ \begin{CJK}{UTF8}{mj}共面\end{CJK}, \begin{CJK}{UTF8}{mj}求\end{CJK} $a=$

  \item \begin{CJK}{UTF8}{mj}试求直线\end{CJK} $l_{1}: x-1=y-2=z$ \begin{CJK}{UTF8}{mj}绕直线\end{CJK} $l_{2}: y=z=0$ \begin{CJK}{UTF8}{mj}旋转后的曲面方程是\end{CJK}

\end{enumerate}
\begin{CJK}{UTF8}{mj}微信公众号\end{CJK}: \begin{CJK}{UTF8}{mj}八一考研数学竞赛\end{CJK} 3. \begin{CJK}{UTF8}{mj}已知四阶矩阵\end{CJK}
$$
A=\left(\begin{array}{llll}
1 & 0 & 1 & 1 \\
3 & 1 & 1 & 1 \\
0 & 0 & 1 & 0 \\
0 & 0 & 2 & 1
\end{array}\right)
$$
\begin{CJK}{UTF8}{mj}求\end{CJK} $A^{-1}=$ , \begin{CJK}{UTF8}{mj}以及\end{CJK}
$$
\operatorname{det}\left(\begin{array}{cc}
E_{n} & 2 E_{n} \\
2 E_{n} & 2 E_{n}
\end{array}\right)=
$$
\begin{CJK}{UTF8}{mj}其中\end{CJK} $E_{n}$ \begin{CJK}{UTF8}{mj}表示\end{CJK} $n$ \begin{CJK}{UTF8}{mj}阶单位方阵\end{CJK}.

\begin{enumerate}
  \setcounter{enumi}{4}
  \item \begin{CJK}{UTF8}{mj}若二次型\end{CJK}
\end{enumerate}
$$
2 x^{2}-x y+y^{2}-b y z+z^{2}
$$
\begin{CJK}{UTF8}{mj}正定\end{CJK}, \begin{CJK}{UTF8}{mj}求\end{CJK} $b$ \begin{CJK}{UTF8}{mj}的取值范围\end{CJK}

\begin{enumerate}
  \setcounter{enumi}{5}
  \item \begin{CJK}{UTF8}{mj}已知三阶矩阵\end{CJK} $A=\left(\begin{array}{lll}5 & 1 & 1 \\ 9 & 8 & 3 \\ 2 & 7 & 3\end{array}\right)$, \begin{CJK}{UTF8}{mj}若\end{CJK} $A_{i j}$ \begin{CJK}{UTF8}{mj}为\end{CJK} $A$ \begin{CJK}{UTF8}{mj}的代数余子式\end{CJK}, \begin{CJK}{UTF8}{mj}求\end{CJK} $A_{21}-A_{22}+$ $2 A_{33}=$

  \item \begin{CJK}{UTF8}{mj}设\end{CJK} $\mathbb{R}_{2}[x]$ \begin{CJK}{UTF8}{mj}为全体次数不超过\end{CJK} 2 \begin{CJK}{UTF8}{mj}的实系数多项式及零多项式生成的线性空间\end{CJK}, \begin{CJK}{UTF8}{mj}考虑\end{CJK} \begin{CJK}{UTF8}{mj}线性变换\end{CJK} $\mathscr{A}=x \frac{\mathrm{d}}{\mathrm{d} x}: \mathbb{R}_{2}[x] \rightarrow \mathbb{R}_{2}[x]$, \begin{CJK}{UTF8}{mj}则\end{CJK} $\mathscr{A}$ \begin{CJK}{UTF8}{mj}的最小多项式\end{CJK}

\end{enumerate}
\begin{CJK}{UTF8}{mj}二\end{CJK}、 \begin{CJK}{UTF8}{mj}解答题\end{CJK} (6 \begin{CJK}{UTF8}{mj}大题\end{CJK}, \begin{CJK}{UTF8}{mj}共\end{CJK} 110 \begin{CJK}{UTF8}{mj}分\end{CJK})

\begin{enumerate}
  \setcounter{enumi}{7}
  \item (15 \begin{CJK}{UTF8}{mj}分\end{CJK}) \begin{CJK}{UTF8}{mj}若\end{CJK}
\end{enumerate}
$$
\left\{\begin{array}{l}
\alpha_{1}=(1,2,-1,1) \\
\alpha_{2}=(1,3,-1,2) \\
\alpha_{3}=(2,5,0,5) \\
\alpha_{4}=(1,2,1,3) \\
\alpha_{5}=(5,12,1,13)
\end{array}\right.
$$
\begin{CJK}{UTF8}{mj}试求出其所有的极大线性无关组\end{CJK}.

\begin{enumerate}
  \setcounter{enumi}{8}
  \item (15 \begin{CJK}{UTF8}{mj}分\end{CJK}) \begin{CJK}{UTF8}{mj}已知二次曲面\end{CJK} $y^{2}+\sqrt{2} x y+y z-2 y+5=0$, \begin{CJK}{UTF8}{mj}试用正交变换和平移化为\end{CJK} \begin{CJK}{UTF8}{mj}标准形\end{CJK}, \begin{CJK}{UTF8}{mj}并说明这是一个什么类型曲面\end{CJK}?

  \item (20)\begin{CJK}{UTF8}{mj}分\end{CJK}) \begin{CJK}{UTF8}{mj}设\end{CJK} $\mathbb{R}_{3}[x]$ \begin{CJK}{UTF8}{mj}为由全体次数不超过\end{CJK} 3 \begin{CJK}{UTF8}{mj}的实系数多项式及零多项式生成的线性\end{CJK} \begin{CJK}{UTF8}{mj}空间\end{CJK}, \begin{CJK}{UTF8}{mj}对任意的\end{CJK} $f(x), g(x) \in \mathbb{R}_{3}[x]$, \begin{CJK}{UTF8}{mj}我们定义\end{CJK} $(f(x), g(x))=\int_{-1}^{1} f(x) g(x) \mathrm{d} x$

\end{enumerate}
(1) \begin{CJK}{UTF8}{mj}试证明\end{CJK}: $(f(x), g(x))$ \begin{CJK}{UTF8}{mj}定义了\end{CJK} $\mathbb{R}_{3}[x]$ \begin{CJK}{UTF8}{mj}上的内积\end{CJK};

(2) \begin{CJK}{UTF8}{mj}将一组基\end{CJK} $\left\{1, x, x^{2}, x^{3}\right\}$ \begin{CJK}{UTF8}{mj}作为\end{CJK} Gram-Schmidt \begin{CJK}{UTF8}{mj}正交化\end{CJK}, \begin{CJK}{UTF8}{mj}并求出其标准正交基\end{CJK}. 10. (20 \begin{CJK}{UTF8}{mj}分\end{CJK}) \begin{CJK}{UTF8}{mj}若\end{CJK} $A, B$ \begin{CJK}{UTF8}{mj}为实对称方阵\end{CJK}, \begin{CJK}{UTF8}{mj}且\end{CJK} $A$ \begin{CJK}{UTF8}{mj}正定\end{CJK}, \begin{CJK}{UTF8}{mj}则存在一个实数\end{CJK} $a$, \begin{CJK}{UTF8}{mj}当\end{CJK} $a$ \begin{CJK}{UTF8}{mj}充分大时\end{CJK}, \begin{CJK}{UTF8}{mj}证明\end{CJK}: $a A+B$ \begin{CJK}{UTF8}{mj}也正定\end{CJK}.

\begin{enumerate}
  \setcounter{enumi}{11}
  \item (20 \begin{CJK}{UTF8}{mj}分\end{CJK}) \begin{CJK}{UTF8}{mj}若\end{CJK} $n$ \begin{CJK}{UTF8}{mj}阶复方阵\end{CJK} $A$, \begin{CJK}{UTF8}{mj}试证\end{CJK}: \begin{CJK}{UTF8}{mj}对任意大于\end{CJK} $n$ \begin{CJK}{UTF8}{mj}的整数\end{CJK} $n, m$ \begin{CJK}{UTF8}{mj}都有\end{CJK}
\end{enumerate}
$$
\operatorname{rank}\left(A^{n}\right)=\operatorname{rank}\left(A^{m}\right)
$$

\begin{enumerate}
  \setcounter{enumi}{12}
  \item (20 \begin{CJK}{UTF8}{mj}分\end{CJK}) \begin{CJK}{UTF8}{mj}已知\end{CJK} $A, B, C, D$ \begin{CJK}{UTF8}{mj}为\end{CJK} $n$ \begin{CJK}{UTF8}{mj}阶实方阵\end{CJK}, \begin{CJK}{UTF8}{mj}且\end{CJK} $B D=D B$, \begin{CJK}{UTF8}{mj}试证\end{CJK}:
\end{enumerate}
$$
\left|\begin{array}{ll}
A & B \\
C & D
\end{array}\right|=|D A-C B|
$$
\includegraphics[max width=\textwidth]{2022_04_18_7db0708508f26638f054g-136}

\section{第十一章 2021 年重庆大学真题}
\section{$11.1$ 数学分析}
\begin{enumerate}
  \item \begin{CJK}{UTF8}{mj}计算题\end{CJK}.
\end{enumerate}
(1) \begin{CJK}{UTF8}{mj}已知\end{CJK} $a_{i} \in \mathbb{R}(i=1,2, \cdots, n)$ \begin{CJK}{UTF8}{mj}求极限\end{CJK}
$$
\lim _{x \rightarrow+\infty}\left[\sqrt[n]{\left(x+a_{1}\right)\left(x+a_{2}\right) \cdots\left(x+a_{n}\right)}-x\right]
$$
(2) \begin{CJK}{UTF8}{mj}求极限\end{CJK}
$$
\lim _{x \rightarrow 0} \frac{e^{\tan x}-e^{\sin x}}{x-\sin x}
$$
(3) \begin{CJK}{UTF8}{mj}求不定积分\end{CJK}
$$
\int\left(\frac{\ln x}{x}\right)^{2} d x
$$

\begin{enumerate}
  \setcounter{enumi}{2}
  \item \begin{CJK}{UTF8}{mj}已知定义在\end{CJK} $[0,1]$ \begin{CJK}{UTF8}{mj}上的黎曼函数\end{CJK}
\end{enumerate}
$$
f(x)= \begin{cases}\frac{1}{q}, & x=\frac{p}{q}\left(\frac{p}{q} \text { 为既约真分数 }\right) \\ 0, & x=0,1 \text { 和 }(0,1) \text { 内的无理数. }\end{cases}
$$
\begin{CJK}{UTF8}{mj}讨论\end{CJK} $f(x)$ \begin{CJK}{UTF8}{mj}间断点及其类型\end{CJK}, \begin{CJK}{UTF8}{mj}并求出\end{CJK} $f(x)$ \begin{CJK}{UTF8}{mj}的连续点\end{CJK}.

\begin{enumerate}
  \setcounter{enumi}{3}
  \item \begin{CJK}{UTF8}{mj}已知数列\end{CJK} $\left\{a_{n}\right\}$ \begin{CJK}{UTF8}{mj}满足\end{CJK} $a_{n} \mid<2$, \begin{CJK}{UTF8}{mj}且\end{CJK} $\left(2-a_{n}\right) a_{n+1} \geq 1$, \begin{CJK}{UTF8}{mj}证明\end{CJK}: $\left\{a_{n}\right\}$ \begin{CJK}{UTF8}{mj}收玫\end{CJK}, \begin{CJK}{UTF8}{mj}并求其极限\end{CJK}.

  \item \begin{CJK}{UTF8}{mj}已知数列\end{CJK} $\left\{a_{n}\right\}$ \begin{CJK}{UTF8}{mj}满足\end{CJK} $a_{0}=4, a_{1}=1, a_{n-2}=n(n-1) a_{n}(n \geq 2)$

\end{enumerate}
(1) \begin{CJK}{UTF8}{mj}求幂级数\end{CJK} $\sum_{n=0}^{\infty} a_{n} x^{n}$ \begin{CJK}{UTF8}{mj}的和函数\end{CJK} $S(x)$;

(2) \begin{CJK}{UTF8}{mj}求\end{CJK} $S(x)$ \begin{CJK}{UTF8}{mj}的极值\end{CJK}.

\begin{enumerate}
  \setcounter{enumi}{5}
  \item \begin{CJK}{UTF8}{mj}已知数列\end{CJK} $\left\{a_{n}\right\}$ \begin{CJK}{UTF8}{mj}满足\end{CJK} $a_{2 n-1}=\frac{1}{n}, a_{2 n}=\int_{n}^{n+1} \frac{1}{x} \mathrm{~d} x$, \begin{CJK}{UTF8}{mj}证明级数\end{CJK} $\sum_{n=1}^{\infty}(-1)^{n} a_{n}$ \begin{CJK}{UTF8}{mj}条件收玫\end{CJK}. 6. \begin{CJK}{UTF8}{mj}计算第二型曲面积分\end{CJK}
\end{enumerate}
$$
I=\iint_{S} \frac{x^{2} \mathrm{~d} y \mathrm{~d} z+y^{2} \mathrm{~d} z \mathrm{~d} x+z^{2} \mathrm{~d} x \mathrm{~d} y}{\sqrt{x^{2}+y^{2}+z^{2}}}
$$
\begin{CJK}{UTF8}{mj}其中\end{CJK} $S:(x-1)^{2}+(y-1)^{2}+\left(\frac{z}{2}\right)^{2}=1, y \geq 1$, \begin{CJK}{UTF8}{mj}取外侧\end{CJK}.

\begin{enumerate}
  \setcounter{enumi}{7}
  \item \begin{CJK}{UTF8}{mj}设函数列\end{CJK} $\left\{f_{n}(x)\right\}$ \begin{CJK}{UTF8}{mj}在区间\end{CJK} $I$ \begin{CJK}{UTF8}{mj}上一致收玫于\end{CJK} $f(x)$, \begin{CJK}{UTF8}{mj}且\end{CJK} $f(x)$ \begin{CJK}{UTF8}{mj}在\end{CJK} $I$ \begin{CJK}{UTF8}{mj}上有界\end{CJK}, \begin{CJK}{UTF8}{mj}同时对每个\end{CJK} $f_{n}(x)$, \begin{CJK}{UTF8}{mj}存在正数\end{CJK} $M_{n}>0$, \begin{CJK}{UTF8}{mj}使得对任意的\end{CJK} $x \in I$, \begin{CJK}{UTF8}{mj}有\end{CJK} $\left|f_{n}(x)\right| \leq M_{n}$, \begin{CJK}{UTF8}{mj}证明\end{CJK} $\left\{f_{n}(x)\right\}$ \begin{CJK}{UTF8}{mj}在\end{CJK} $I$ \begin{CJK}{UTF8}{mj}上一致有界\end{CJK}.

  \item \begin{CJK}{UTF8}{mj}设\end{CJK} $S$ \begin{CJK}{UTF8}{mj}为简单封闭光滑曲面\end{CJK}, $P(x, y, z), Q(x, y, z), R(x, y, z)$ \begin{CJK}{UTF8}{mj}为\end{CJK} $S$ \begin{CJK}{UTF8}{mj}上的连续函数\end{CJK}, \begin{CJK}{UTF8}{mj}证明\end{CJK}

\end{enumerate}
$$
\left|\iint_{S} P \mathrm{~d} y \mathrm{~d} z+Q \mathrm{~d} z \mathrm{~d} x+R \mathrm{~d} x \mathrm{~d} y\right| \leq M \Delta S
$$
\begin{CJK}{UTF8}{mj}其中\end{CJK} $M=\max _{(x, y, z) \in S} \sqrt{P^{2}+Q^{2}+R^{2}}, \Delta S$ \begin{CJK}{UTF8}{mj}为\end{CJK} $S$ \begin{CJK}{UTF8}{mj}的面积\end{CJK}.

\begin{enumerate}
  \setcounter{enumi}{9}
  \item \begin{CJK}{UTF8}{mj}已知\end{CJK} $f(x)$ \begin{CJK}{UTF8}{mj}在\end{CJK} $[0,1]$ \begin{CJK}{UTF8}{mj}上黎曼可积\end{CJK}, \begin{CJK}{UTF8}{mj}且\end{CJK} $0 \leq f(x) \leq 1$, \begin{CJK}{UTF8}{mj}对任意的\end{CJK} $[\alpha, \beta] \subseteq[0,1], \int_{\alpha}^{\beta} f(x) \mathrm{d} x$ \begin{CJK}{UTF8}{mj}存在\end{CJK}. \begin{CJK}{UTF8}{mj}证明\end{CJK}: \begin{CJK}{UTF8}{mj}对任意的\end{CJK} $\varepsilon>0$, \begin{CJK}{UTF8}{mj}存在\end{CJK} $[\alpha, \beta]$ \begin{CJK}{UTF8}{mj}上取值为\end{CJK} 0 \begin{CJK}{UTF8}{mj}或\end{CJK} 1 \begin{CJK}{UTF8}{mj}的分段函数\end{CJK} $g(x)$ (\begin{CJK}{UTF8}{mj}分有限\end{CJK} \begin{CJK}{UTF8}{mj}段\end{CJK}), \begin{CJK}{UTF8}{mj}使得\end{CJK}
\end{enumerate}
$$
\int_{\alpha}^{\beta}|f(x)-g(x)| \mathrm{d} x<\varepsilon
$$

\begin{enumerate}
  \setcounter{enumi}{10}
  \item \begin{CJK}{UTF8}{mj}设函数\end{CJK} $f(x)$ \begin{CJK}{UTF8}{mj}在\end{CJK} $(-1,1)$ \begin{CJK}{UTF8}{mj}上二阶导数存在且连续\end{CJK}, $f(0)=f^{\prime}(0)=0, f^{\prime \prime}(0) \neq 0$, \begin{CJK}{UTF8}{mj}对于\end{CJK} $x \in(-1,1), u$ \begin{CJK}{UTF8}{mj}为\end{CJK} $f(x)$ \begin{CJK}{UTF8}{mj}在\end{CJK} $(x, f(x))$ \begin{CJK}{UTF8}{mj}处的切线与\end{CJK} $x$ \begin{CJK}{UTF8}{mj}轴的交点的横坐标\end{CJK}, \begin{CJK}{UTF8}{mj}求\end{CJK} $\lim _{x \rightarrow 0} \frac{u f(x)}{x f(u)}$
\end{enumerate}
\section{$11.2$ 高等代数}
\begin{enumerate}
  \item \begin{CJK}{UTF8}{mj}已知\end{CJK} $n$ \begin{CJK}{UTF8}{mj}阶矩阵\end{CJK} $A$ \begin{CJK}{UTF8}{mj}的顺序主子式都不为零\end{CJK}, \begin{CJK}{UTF8}{mj}试证\end{CJK}: \begin{CJK}{UTF8}{mj}存在\end{CJK} $n$ \begin{CJK}{UTF8}{mj}阶下三角矩阵\end{CJK} $B$, \begin{CJK}{UTF8}{mj}使得\end{CJK} $B A$ \begin{CJK}{UTF8}{mj}为上三角矩阵\end{CJK}.

  \item \begin{CJK}{UTF8}{mj}已知\end{CJK} $f(x), g(x)$ \begin{CJK}{UTF8}{mj}为多项式\end{CJK}, \begin{CJK}{UTF8}{mj}且\end{CJK} $h(x)$ \begin{CJK}{UTF8}{mj}为首一多项式\end{CJK}, \begin{CJK}{UTF8}{mj}试证\end{CJK}:

\end{enumerate}
$$
(f(x) h(x), g(x) h(x))=(f(x), g(x)) h(x)
$$

\begin{enumerate}
  \setcounter{enumi}{3}
  \item \begin{CJK}{UTF8}{mj}已知\end{CJK} $A=\left(\alpha_{1}, \alpha_{2}, \cdots, \alpha_{n}\right)$ \begin{CJK}{UTF8}{mj}为\end{CJK} $n$ \begin{CJK}{UTF8}{mj}阶方阵\end{CJK}, \begin{CJK}{UTF8}{mj}且\end{CJK} $A$ \begin{CJK}{UTF8}{mj}的前\end{CJK} $n-1$ \begin{CJK}{UTF8}{mj}个列向量线性相关\end{CJK}, \begin{CJK}{UTF8}{mj}后\end{CJK} $n-1$ \begin{CJK}{UTF8}{mj}个向量线性无关\end{CJK}, \begin{CJK}{UTF8}{mj}记\end{CJK}
\end{enumerate}
$$
\beta=\alpha_{1}+\alpha_{2}+\cdots+\alpha_{n}
$$
(1) \begin{CJK}{UTF8}{mj}试证\end{CJK}: \begin{CJK}{UTF8}{mj}方程组\end{CJK} $A X=\beta$ \begin{CJK}{UTF8}{mj}有无穷多解\end{CJK};

(2) \begin{CJK}{UTF8}{mj}求方程组\end{CJK} $A X=\beta$ \begin{CJK}{UTF8}{mj}的通解\end{CJK}. 4. \begin{CJK}{UTF8}{mj}若\end{CJK} $A, B$ \begin{CJK}{UTF8}{mj}是\end{CJK} $n$ \begin{CJK}{UTF8}{mj}阶正定矩阵\end{CJK}, \begin{CJK}{UTF8}{mj}且\end{CJK} $A B=B A$, \begin{CJK}{UTF8}{mj}试证\end{CJK}: $A B$ \begin{CJK}{UTF8}{mj}也为正定矩阵\end{CJK}.

\begin{enumerate}
  \setcounter{enumi}{5}
  \item \begin{CJK}{UTF8}{mj}若矩阵\end{CJK} $A$ \begin{CJK}{UTF8}{mj}经过渡矩阵\end{CJK} $P$ \begin{CJK}{UTF8}{mj}化相似矩阵\end{CJK} $C$, \begin{CJK}{UTF8}{mj}且\end{CJK} $C$ \begin{CJK}{UTF8}{mj}为对角矩阵\end{CJK}, \begin{CJK}{UTF8}{mj}试证\end{CJK}: \begin{CJK}{UTF8}{mj}矩阵\end{CJK} $P$ \begin{CJK}{UTF8}{mj}是由\end{CJK} $A$ \begin{CJK}{UTF8}{mj}的\end{CJK} \begin{CJK}{UTF8}{mj}特征向量所构成\end{CJK}.

  \item \begin{CJK}{UTF8}{mj}若\end{CJK} $A$ \begin{CJK}{UTF8}{mj}为\end{CJK} $n$ \begin{CJK}{UTF8}{mj}阶矩阵\end{CJK}, \begin{CJK}{UTF8}{mj}试证\end{CJK}: \begin{CJK}{UTF8}{mj}若\end{CJK} $A$ \begin{CJK}{UTF8}{mj}为幂零矩阵\end{CJK}, \begin{CJK}{UTF8}{mj}则对任意的正整数\end{CJK} $k$, \begin{CJK}{UTF8}{mj}有\end{CJK} $\operatorname{tr}\left(A^{k}\right)=0$; \begin{CJK}{UTF8}{mj}反之若对任意的\end{CJK} $1 \leq k \leq n$ \begin{CJK}{UTF8}{mj}均有\end{CJK} $\operatorname{tr}\left(A^{k}\right)=0$, \begin{CJK}{UTF8}{mj}则\end{CJK} $A$ \begin{CJK}{UTF8}{mj}为幂零矩阵\end{CJK}.

  \item (1) \begin{CJK}{UTF8}{mj}若\end{CJK} $A$ \begin{CJK}{UTF8}{mj}是实反称矩阵\end{CJK}, \begin{CJK}{UTF8}{mj}试证\end{CJK}:

\end{enumerate}
$$
Q=(E-A)(E+A)^{-1}
$$
\begin{CJK}{UTF8}{mj}为正交矩阵\end{CJK};

(2) \begin{CJK}{UTF8}{mj}若\end{CJK} $Q$ \begin{CJK}{UTF8}{mj}为正交矩阵\end{CJK}, \begin{CJK}{UTF8}{mj}且\end{CJK} $E+Q$ \begin{CJK}{UTF8}{mj}可逆\end{CJK}, \begin{CJK}{UTF8}{mj}试证\end{CJK}: \begin{CJK}{UTF8}{mj}存在实反称矩阵\end{CJK} $A$, \begin{CJK}{UTF8}{mj}使得\end{CJK}
$$
\left.Q=(E-A)(E+A)^{-1}\right)
$$

\begin{enumerate}
  \setcounter{enumi}{8}
  \item \begin{CJK}{UTF8}{mj}若方程组\end{CJK} $x_{1}+x_{2}+\cdots+x_{n}=0$ \begin{CJK}{UTF8}{mj}的解空间是\end{CJK} $V_{1}$, \begin{CJK}{UTF8}{mj}方程组\end{CJK} $x_{1}=x_{2}=\cdots=x_{n}$ \begin{CJK}{UTF8}{mj}的解空\end{CJK} \begin{CJK}{UTF8}{mj}间是\end{CJK} $V_{2}$, \begin{CJK}{UTF8}{mj}试证\end{CJK}:
\end{enumerate}
$$
\mathrm{R}^{n}=V_{1} \oplus V_{2}
$$

\begin{enumerate}
  \setcounter{enumi}{9}
  \item \begin{CJK}{UTF8}{mj}试证\end{CJK}: \begin{CJK}{UTF8}{mj}一个实二次型可以分解为两个实系数的一次齐次多项式的乘积当且仅当它的秩\end{CJK} \begin{CJK}{UTF8}{mj}为\end{CJK} 2 \begin{CJK}{UTF8}{mj}且符号差为\end{CJK} 0 , \begin{CJK}{UTF8}{mj}或它的秩为\end{CJK} 1 .

  \item \begin{CJK}{UTF8}{mj}已知\end{CJK} $A, B$ \begin{CJK}{UTF8}{mj}分别为\end{CJK} $n \times m$ \begin{CJK}{UTF8}{mj}与\end{CJK} $m \times n$ \begin{CJK}{UTF8}{mj}阶矩阵\end{CJK}.

\end{enumerate}
(1) \begin{CJK}{UTF8}{mj}证明\end{CJK} $\left|I_{n}-A B\right|=\left|I_{m}-B A\right|$, \begin{CJK}{UTF8}{mj}其中\end{CJK} $I_{n}, I_{m}$ \begin{CJK}{UTF8}{mj}分别为\end{CJK} $n$ \begin{CJK}{UTF8}{mj}阶与\end{CJK} $m$ \begin{CJK}{UTF8}{mj}阶的单位矩阵\end{CJK};

(2) \begin{CJK}{UTF8}{mj}计算行列式\end{CJK}
$$
D_{n}=\left|\begin{array}{ccccc}
1+a_{1}+x_{1} & a_{1}+x_{2} & a_{1}+x_{3} & \cdots & a_{1}+x_{n} \\
a_{2}+x_{1} & 1+a_{2}+x_{2} & a_{2}+x_{3} & \cdots & a_{2}+x_{n} \\
a_{3}+x_{1} & a_{3}+x_{2} & 1+a_{3}+x_{3} & \cdots & a_{3}+x_{n} \\
\vdots & \vdots & \vdots & & \vdots \\
a_{n}+x_{1} & a_{n}+x_{2} & a_{n}+x_{3} & \cdots & 1+a_{n}+x_{n}
\end{array}\right|
$$

\begin{enumerate}
  \setcounter{enumi}{11}
  \item \begin{CJK}{UTF8}{mj}若\end{CJK} $\varphi$ \begin{CJK}{UTF8}{mj}为\end{CJK} $n$ \begin{CJK}{UTF8}{mj}维线性空间\end{CJK} $V$ \begin{CJK}{UTF8}{mj}上的线性变换\end{CJK}, $f(\lambda)$ \begin{CJK}{UTF8}{mj}是它的特征多项式\end{CJK}, \begin{CJK}{UTF8}{mj}且\end{CJK}
\end{enumerate}
$$
f(\lambda)=f_{1}(\lambda) f_{2}(\lambda)
$$
\begin{CJK}{UTF8}{mj}其中\end{CJK} $\left(f_{1}(\lambda), f_{2}(\lambda)\right)=1$, \begin{CJK}{UTF8}{mj}记\end{CJK}
$$
V_{1}=\operatorname{Im} f_{1}(\varphi), V_{2}=\operatorname{Im} f_{2}(\varphi)
$$
(1) $V_{1}=\operatorname{Ker} f_{2}(\varphi), V_{2}=\operatorname{Ker} f_{1}(\varphi)$;

(2) $V_{1}, V_{2}$ \begin{CJK}{UTF8}{mj}为\end{CJK} $\varphi-$ \begin{CJK}{UTF8}{mj}子空间\end{CJK}, \begin{CJK}{UTF8}{mj}且\end{CJK} $V=V_{1} \oplus V_{2}$.

\includegraphics[max width=\textwidth]{2022_04_18_7db0708508f26638f054g-140}

\section{第十二章 2021 年华中科技大学真题}
\section{$12.1$ 数学分析}
\begin{enumerate}
  \item \begin{CJK}{UTF8}{mj}计算极限\end{CJK}
\end{enumerate}
$$
\lim _{x \rightarrow \infty} x\left(1+x \ln \left(1-\frac{1}{x}\right)\right)
$$

\begin{enumerate}
  \setcounter{enumi}{2}
  \item \begin{CJK}{UTF8}{mj}设\end{CJK} $z=z(x, y)$, \begin{CJK}{UTF8}{mj}且满足\end{CJK}
\end{enumerate}
$$
F(c x-a z, c y-b z)=0
$$
\begin{CJK}{UTF8}{mj}试求\end{CJK}: $a \frac{\partial z}{\partial x}+b \frac{\partial z}{\partial y}$.

\begin{enumerate}
  \setcounter{enumi}{3}
  \item \begin{CJK}{UTF8}{mj}设曲线\end{CJK} $x^{2}+y^{2}=1(x>0, y>0)$, \begin{CJK}{UTF8}{mj}求该曲线上的一点\end{CJK} $(x, y)$ \begin{CJK}{UTF8}{mj}使得过该点的切线与\end{CJK} $x$ \begin{CJK}{UTF8}{mj}轴\end{CJK}, $y$ \begin{CJK}{UTF8}{mj}轴所围成的三角形面积最小\end{CJK}.

  \item \begin{CJK}{UTF8}{mj}求\end{CJK}

\end{enumerate}
$$
\int_{L}(x+y)^{2} \mathrm{~d} x-\left(x^{2}+y^{2}\right) \mathrm{d} y
$$
\begin{CJK}{UTF8}{mj}其中\end{CJK} $L$ \begin{CJK}{UTF8}{mj}是由\end{CJK} $A(1,1), B(2,5), C(3,2)$ \begin{CJK}{UTF8}{mj}围成的三角形区域\end{CJK}, \begin{CJK}{UTF8}{mj}方向逆时针\end{CJK}.

\begin{enumerate}
  \setcounter{enumi}{5}
  \item \begin{CJK}{UTF8}{mj}将\end{CJK} $f(x)=x(\pi-x)$ \begin{CJK}{UTF8}{mj}展开到\end{CJK} $[0, \pi]$ \begin{CJK}{UTF8}{mj}上的余弦级数\end{CJK}.

  \item \begin{CJK}{UTF8}{mj}证明\end{CJK}:

\end{enumerate}
$$
\int_{0}^{\pi} f(x) \mathrm{d} x=\frac{1}{2} \int_{0}^{\pi}[f(x)+f(\pi-x)
$$
$$
\begin{aligned}
& \text { 并计算 } \int_{0}^{\pi} \frac{x \sin x}{1+\cos ^{2} x} \mathrm{~d} x
\end{aligned}
$$

\begin{enumerate}
  \setcounter{enumi}{7}
  \item \begin{CJK}{UTF8}{mj}证明\end{CJK}: \begin{CJK}{UTF8}{mj}级数\end{CJK}
\end{enumerate}
$$
\sum_{n=3}^{\infty} \ln \left(1+\frac{x}{n \ln ^{2} n}\right)
$$
\begin{CJK}{UTF8}{mj}在\end{CJK} $x \in[-1,1]$ \begin{CJK}{UTF8}{mj}上一致收玫\end{CJK}.

\begin{enumerate}
  \setcounter{enumi}{8}
  \item \begin{CJK}{UTF8}{mj}证明\end{CJK}: \begin{CJK}{UTF8}{mj}反常积分\end{CJK}
\end{enumerate}
$$
\int_{1}^{+\infty} \frac{\mathrm{e}^{\sin x} \sin 2 x}{x} \mathrm{~d} x
$$

\begin{enumerate}
  \setcounter{enumi}{9}
  \item \begin{CJK}{UTF8}{mj}设\end{CJK} $f(x)$ \begin{CJK}{UTF8}{mj}在\end{CJK} $[a, b]$ \begin{CJK}{UTF8}{mj}可导\end{CJK}, \begin{CJK}{UTF8}{mj}且\end{CJK} $f^{\prime}(x)$ \begin{CJK}{UTF8}{mj}单调递增\end{CJK}, \begin{CJK}{UTF8}{mj}试证\end{CJK}:
\end{enumerate}
$$
\int_{a}^{b} f(x) \mathrm{d} x \leq \frac{b-a}{2}(f(a)+f(b))
$$

\begin{enumerate}
  \setcounter{enumi}{10}
  \item \begin{CJK}{UTF8}{mj}设\end{CJK} $f(x)$ \begin{CJK}{UTF8}{mj}在\end{CJK} $[0,1]$ \begin{CJK}{UTF8}{mj}上可导\end{CJK}, \begin{CJK}{UTF8}{mj}且\end{CJK} $\left|f^{\prime}(x)\right| \leq k<1$, \begin{CJK}{UTF8}{mj}数列\end{CJK} $\left\{x_{n}\right\}$ \begin{CJK}{UTF8}{mj}满足\end{CJK}:
\end{enumerate}
$$
x_{0}=0, x_{n+1}=f\left(x_{n}\right),(n=0,1, \cdots)
$$
\begin{CJK}{UTF8}{mj}试证\end{CJK}: \begin{CJK}{UTF8}{mj}数列\end{CJK} $\left\{x_{n}\right\}$ \begin{CJK}{UTF8}{mj}收敛\end{CJK}.

\section{$12.2$ 高等代数}
\begin{enumerate}
  \item \begin{CJK}{UTF8}{mj}若矩阵\end{CJK} $A$ \begin{CJK}{UTF8}{mj}的每个元素均为整数\end{CJK}, \begin{CJK}{UTF8}{mj}试证\end{CJK}: $\frac{1}{2}$ \begin{CJK}{UTF8}{mj}不是\end{CJK} $A$ \begin{CJK}{UTF8}{mj}的特征值\end{CJK},

  \item \begin{CJK}{UTF8}{mj}讨论方程组\end{CJK}

\end{enumerate}
$$
\left\{\begin{array}{l}
x_{1}+x_{2}+x_{3}+2 x_{4}=3 \\
x_{1}+x_{2}+(a+1) x_{3}+5 x_{4}=2 \\
x_{2}+a x_{3}+2 x_{4}=3 \\
x_{1}-x_{2}+x_{3}+(a-1) x_{4}=b-1
\end{array}\right.
$$

\begin{enumerate}
  \setcounter{enumi}{3}
  \item \begin{CJK}{UTF8}{mj}设\end{CJK} $A, B$ \begin{CJK}{UTF8}{mj}是同阶矩阵\end{CJK}, \begin{CJK}{UTF8}{mj}且\end{CJK} $A B=B A$, \begin{CJK}{UTF8}{mj}试证\end{CJK}:
\end{enumerate}
$$
\operatorname{rank}(A B)+\operatorname{rank}(A+B) \leq \operatorname{rank}(A)+\operatorname{rank}(B)
$$

\begin{enumerate}
  \setcounter{enumi}{4}
  \item \begin{CJK}{UTF8}{mj}设\end{CJK} $n$ \begin{CJK}{UTF8}{mj}维线性空间\end{CJK} $V$ \begin{CJK}{UTF8}{mj}的一组基为\end{CJK} $\alpha_{1}, \alpha_{2}, \cdots, \alpha_{n}$, \begin{CJK}{UTF8}{mj}而\end{CJK} $w_{1}, w_{2}, \cdots, w_{n}$ \begin{CJK}{UTF8}{mj}是\end{CJK} $V$ \begin{CJK}{UTF8}{mj}的非平凡子\end{CJK} \begin{CJK}{UTF8}{mj}空间\end{CJK}, \begin{CJK}{UTF8}{mj}证明\end{CJK}: $\alpha_{i} \notin w_{j},(i, j=1,2, \cdots, n)$

  \item \begin{CJK}{UTF8}{mj}设正交变换\end{CJK} $\sigma$ \begin{CJK}{UTF8}{mj}的特征值为实数\end{CJK}, \begin{CJK}{UTF8}{mj}试证\end{CJK}:

\end{enumerate}
$$
(\alpha, \sigma(\beta))=(\sigma(\alpha), \beta)
$$

\begin{enumerate}
  \setcounter{enumi}{6}
  \item \begin{CJK}{UTF8}{mj}设\end{CJK} $n$ \begin{CJK}{UTF8}{mj}阶矩阵\end{CJK} $A$ \begin{CJK}{UTF8}{mj}满足\end{CJK} $\operatorname{tr}(\mathrm{A})=0$, \begin{CJK}{UTF8}{mj}证明\end{CJK}: $A$ \begin{CJK}{UTF8}{mj}和一个主对角线全为\end{CJK} 0 \begin{CJK}{UTF8}{mj}的矩阵相似\end{CJK}.

  \item \begin{CJK}{UTF8}{mj}对于正定矩阵\end{CJK} $A$, \begin{CJK}{UTF8}{mj}有反对称阵\end{CJK} $S$, \begin{CJK}{UTF8}{mj}试证\end{CJK}:

\end{enumerate}
$$
\operatorname{det}(A+S) \geq \operatorname{det}(A)
$$
\begin{CJK}{UTF8}{mj}并求取等条件\end{CJK}.

\begin{enumerate}
  \setcounter{enumi}{8}
  \item \begin{CJK}{UTF8}{mj}设\end{CJK} $A /$ \begin{CJK}{UTF8}{mj}是\end{CJK} $n$ \begin{CJK}{UTF8}{mj}阶正定矩阵\end{CJK}, \begin{CJK}{UTF8}{mj}且\end{CJK} $x \in \mathbb{R}^{n}$ \begin{CJK}{UTF8}{mj}为非零向量\end{CJK}, \begin{CJK}{UTF8}{mj}试证\end{CJK}:
\end{enumerate}
(1) $A+x x^{\prime}$ \begin{CJK}{UTF8}{mj}可逆\end{CJK};

(2) $0<x^{\prime}\left(A+x x^{\prime}\right) x^{-1}<1$.

\section{第十三章 2021 年东北大学真题}
\section{$13.1$ 数学分析}
\begin{enumerate}
  \item \begin{CJK}{UTF8}{mj}计算题\end{CJK} (30 \begin{CJK}{UTF8}{mj}分\end{CJK})
\end{enumerate}
(1) \begin{CJK}{UTF8}{mj}求极限\end{CJK}
$$
\lim _{x \rightarrow 1} \frac{x^{n+1}-(n+1) x+n}{(x-1)^{2}}(n \text { 为正整数 }) .
$$
\begin{CJK}{UTF8}{mj}解\end{CJK}: \begin{CJK}{UTF8}{mj}两次洛必达即可\end{CJK}

(2) \begin{CJK}{UTF8}{mj}求极限\end{CJK}

\includegraphics[max width=\textwidth]{2022_04_18_7db0708508f26638f054g-143}

\begin{CJK}{UTF8}{mj}解\end{CJK}: \begin{CJK}{UTF8}{mj}设\end{CJK}
$$
f(x, y)=\frac{1}{1+(1+x y)^{\frac{1}{y}}}
$$
\begin{CJK}{UTF8}{mj}可知\end{CJK} $\lim _{y \rightarrow 0} f(x, y)=\frac{1}{1+\mathrm{e}^{x}}$, \begin{CJK}{UTF8}{mj}并定义\end{CJK} $f(x, 0)=\frac{1}{1+\mathrm{e}^{x}}$, \begin{CJK}{UTF8}{mj}则\end{CJK} $f(x, y)$ \begin{CJK}{UTF8}{mj}在\end{CJK} $[0,1] \times[0,1]$ \begin{CJK}{UTF8}{mj}上连续\end{CJK}, \begin{CJK}{UTF8}{mj}则\end{CJK} \begin{CJK}{UTF8}{mj}由含参昌积分的连续性知\end{CJK}
$$
\lim _{n \rightarrow \infty} \int_{0}^{1} \frac{1}{1+\left(1+\frac{x}{n}\right)^{n}} d x
$$
\includegraphics[max width=\textwidth]{2022_04_18_7db0708508f26638f054g-144}
$$
\begin{aligned}
& =\int_{0}^{1} \lim _{y \rightarrow 0} \frac{1}{1+(1+x y)^{\frac{1}{y}}} \mathrm{~d} x \\
& =\int_{0}^{1} \frac{1}{1+\mathrm{e}^{x}} \mathrm{~d} x
\end{aligned}
$$
$=\int_{0}^{1} \frac{1+\mathrm{e}^{x}-\mathrm{e}^{x}}{1+\mathrm{e}^{x}} \mathrm{~d} x$
$$
\begin{aligned}
&=\int_{0}^{1} \mathrm{~d} x- \\
&=1-\int_{0}^{1} \frac{\mathrm{d}}{} \\
&=1-\ln (1 \\
&=\ln \frac{2 \mathrm{e}}{1+\mathrm{e}}
\end{aligned}
$$
(3) \begin{CJK}{UTF8}{mj}求定积分\end{CJK}

\includegraphics[max width=\textwidth]{2022_04_18_7db0708508f26638f054g-144(1)}

\begin{CJK}{UTF8}{mj}角㸮\end{CJK}:\begin{CJK}{UTF8}{mj}利用区间再现公式\end{CJK}
$$
\int_{a}^{b} f(x) \mathrm{d} x=\int_{a}^{b} f(a+b-x) \mathrm{d} x
$$
\begin{CJK}{UTF8}{mj}㕝\end{CJK}
$$
\begin{aligned}
&\int_{0}^{\pi} \frac{x}{1+\cos ^{2} x} \mathrm{~d} x=\int_{0}^{\pi} \frac{\pi-x}{1+\cos ^{2}(\pi-x)} \mathrm{d} x \\
&\Rightarrow \int_{0}^{\pi} \frac{x}{1+\cos ^{2} x} \mathrm{~d} x=\frac{\pi}{2} \int_{0}^{\pi} \frac{1}{1+\cos ^{2} x} \mathrm{~d} x
\end{aligned}
$$
\begin{CJK}{UTF8}{mj}其中\end{CJK}

\includegraphics[max width=\textwidth]{2022_04_18_7db0708508f26638f054g-144(2)}
$$
\begin{aligned}
&\int_{0}^{\pi} \frac{1}{1+\cos ^{2} x} \mathrm{~d} x \\
&=2 \int_{0}^{\frac{\pi}{2}} \frac{1}{1+\cos ^{2} x} \mathrm{~d} x
\end{aligned}
$$
$$
\begin{aligned}
& =2 \int_{0}^{\frac{\pi}{2}} \frac{1}{2+\tan ^{2} x} d \tan x \\
& \stackrel{t=\tan x}{=} 2 \int_{0}^{+\infty} \frac{1}{2+t^{2}} \mathrm{~d} t \\
& \stackrel{t=u \sqrt{2}}{=} \sqrt{2} \int_{0}^{+\infty} \frac{1}{1+u^{2}} \mathrm{~d} u
\end{aligned}
$$
$=\frac{\pi}{\sqrt{2}}$ \begin{CJK}{UTF8}{mj}因此\end{CJK}
$$
\int_{0}^{\pi} \frac{x}{1+\cos ^{2} x} \mathrm{~d} x=\frac{\pi^{2}}{2 \sqrt{2}}
$$
\begin{CJK}{UTF8}{mj}变式\end{CJK}: \begin{CJK}{UTF8}{mj}计算极限\end{CJK}
$$
\lim _{n \rightarrow \infty} \int_{0}^{\pi} \frac{\sin x}{1+\cos ^{2}(n x)} \mathrm{d} x
$$
\begin{CJK}{UTF8}{mj}解\end{CJK}: \begin{CJK}{UTF8}{mj}由于\end{CJK}
$$
\begin{aligned}
&\int_{0}^{\pi} \frac{\sin x}{1+\cos ^{2}(n x)} \mathrm{d} x \\
&\underline{y=n x}=\frac{1}{n} \int_{0}^{n \pi} \frac{\sin \frac{y}{n}}{1+\cos ^{2} y} \mathrm{~d} y \\
&=\frac{1}{n} \sum_{k=0}^{n-1} \int_{k \pi}^{(k+1) \pi} \frac{\sin ^{\frac{y}{n}}}{1+\cos ^{2} y} \mathrm{~d} y
\end{aligned}
$$
\begin{CJK}{UTF8}{mj}故根据黎曼求和可得\end{CJK}
$$
\begin{aligned}
&\lim _{n \rightarrow \infty} \int_{0}^{\pi} \frac{\sin x}{1+\cos ^{2}(n x)} \mathrm{d} x \\
&=\lim _{n \rightarrow \infty} \frac{1}{\pi} \int_{0}^{\pi} \frac{1}{1+\cos ^{2} t}\left[\frac{\pi}{n} \sum_{k=0}^{n-1} \sin \left(\frac{t+k \pi}{n}\right)\right] \\
&=\frac{1}{\pi} \int_{0}^{\pi} \frac{1}{1+\cos ^{2} t} \mathrm{~d} t \int_{0}^{\pi} \sin t \mathrm{~d} t \\
&=\frac{1}{\pi} \cdot \frac{\pi}{\sqrt{2}}=\sqrt{2}
\end{aligned}
$$

\begin{enumerate}
  \setcounter{enumi}{2}
  \item (20 \begin{CJK}{UTF8}{mj}分\end{CJK})
\end{enumerate}
(1) \begin{CJK}{UTF8}{mj}设\end{CJK} $f(x)$ \begin{CJK}{UTF8}{mj}在\end{CJK} $(a,+\infty)$ \begin{CJK}{UTF8}{mj}上可导\end{CJK}, \begin{CJK}{UTF8}{mj}且\end{CJK} $\lim _{x \rightarrow+\infty} f(x)$ \begin{CJK}{UTF8}{mj}与\end{CJK} $\lim _{x \rightarrow+\infty} f^{\prime}(x)$ \begin{CJK}{UTF8}{mj}存在\end{CJK}, \begin{CJK}{UTF8}{mj}试证\end{CJK}: $\lim _{x \rightarrow+\infty} f^{\prime}(x)=$

(2) \begin{CJK}{UTF8}{mj}设\end{CJK} $f(x)$ \begin{CJK}{UTF8}{mj}在\end{CJK} $(a,+\infty)$ \begin{CJK}{UTF8}{mj}上\end{CJK} $n$ \begin{CJK}{UTF8}{mj}阶可导\end{CJK}, \begin{CJK}{UTF8}{mj}且\end{CJK} $\lim _{x \rightarrow+\infty} f(x)$ \begin{CJK}{UTF8}{mj}与\end{CJK} $\lim _{n \rightarrow+\infty} f^{(n)}(x)$ \begin{CJK}{UTF8}{mj}存在\end{CJK}, \begin{CJK}{UTF8}{mj}证明\end{CJK}:
$$
\lim _{x \rightarrow+\infty} f^{(k)}(x)=0, k=1,2, \cdots, n
$$
\begin{CJK}{UTF8}{mj}证明\end{CJK}: \begin{CJK}{UTF8}{mj}由\end{CJK} $\lim _{x \rightarrow+\infty} f(x)$ \begin{CJK}{UTF8}{mj}与\end{CJK} $\lim _{x \rightarrow+\infty} f^{\prime}(x)$ \begin{CJK}{UTF8}{mj}存在\end{CJK}, \begin{CJK}{UTF8}{mj}由柯西收敛定理\end{CJK}, \begin{CJK}{UTF8}{mj}对\end{CJK} $\forall \varepsilon>0, \exists N>0$, \begin{CJK}{UTF8}{mj}当\end{CJK} $x_{1}, x_{2}>N$ \begin{CJK}{UTF8}{mj}时\end{CJK}, \begin{CJK}{UTF8}{mj}则有\end{CJK}
$$
\begin{gathered}
\left|f\left(x_{1}\right)-f\left(x_{2}\right)\right|<\frac{\varepsilon}{2} \\
\left|f^{\prime}\left(x_{1}\right)-f^{\prime}\left(x_{2}\right)\right|<\frac{\varepsilon}{2}
\end{gathered}
$$
\begin{CJK}{UTF8}{mj}对\end{CJK} $\forall x>N$, \begin{CJK}{UTF8}{mj}由拉格朗日中值定理得\end{CJK}
$$
|f(x)-f(x+1)|=\left|f^{\prime}(\xi)\right|<\frac{\varepsilon}{2}
$$
\begin{CJK}{UTF8}{mj}则有\end{CJK}
$$
-\frac{\varepsilon}{2}<f^{\prime}(\xi)<\frac{\varepsilon}{2}, \quad \text { 其中 } \xi \in(x, x+1)
$$
\begin{CJK}{UTF8}{mj}因此\end{CJK}
$$
-\varepsilon<f^{\prime}(x)<\varepsilon \Rightarrow\left|f^{\prime}(x)\right|<\varepsilon
$$
\begin{CJK}{UTF8}{mj}即证\end{CJK} $\lim _{x \rightarrow+\infty} f^{\prime}(x)=0$.

\begin{CJK}{UTF8}{mj}同理\end{CJK} (2) \begin{CJK}{UTF8}{mj}问也是如此做法\end{CJK}.

\begin{CJK}{UTF8}{mj}提问\end{CJK}:\begin{CJK}{UTF8}{mj}设\end{CJK} $f(x)$ \begin{CJK}{UTF8}{mj}在\end{CJK} $(-\infty,+\infty)$ \begin{CJK}{UTF8}{mj}上的可微函数\end{CJK}, \begin{CJK}{UTF8}{mj}且\end{CJK} $\lim _{x \rightarrow+\infty} f(x)$ \begin{CJK}{UTF8}{mj}存在且有限\end{CJK}, \begin{CJK}{UTF8}{mj}试问\end{CJK}: $\lim _{x \rightarrow+\infty} f^{\prime}(x)$ \begin{CJK}{UTF8}{mj}是否\end{CJK} \begin{CJK}{UTF8}{mj}一定存在\end{CJK}?

\begin{CJK}{UTF8}{mj}解\end{CJK}: \begin{CJK}{UTF8}{mj}不一定\end{CJK}, \begin{CJK}{UTF8}{mj}例如\end{CJK} $f(x)=\frac{\sin x^{2}}{x}$.

\begin{enumerate}
  \setcounter{enumi}{3}
  \item (20 \begin{CJK}{UTF8}{mj}分\end{CJK}) \begin{CJK}{UTF8}{mj}计算\end{CJK}
\end{enumerate}
$$
I_{n}=\int \frac{\mathrm{d} x}{\sin ^{n} x}(n>2)
$$
\begin{CJK}{UTF8}{mj}的递推公式\end{CJK}, \begin{CJK}{UTF8}{mj}并计算\end{CJK} $\int \frac{\mathrm{d} x}{\sin ^{5} x}$

\begin{CJK}{UTF8}{mj}解\end{CJK}:\begin{CJK}{UTF8}{mj}显然利用分部积分\end{CJK}
$$
\begin{aligned}
I_{n} &=\int \frac{\mathrm{d} x}{\sin ^{n} x} \\
&=\int \csc ^{n} x \mathrm{~d} x \\
&=\int \csc ^{n-2} x \csc ^{2} x \mathrm{~d} x \\
&=-\int \csc ^{n-2} x \mathrm{~d}(\cot x) \\
&=-\cot x \csc ^{n-2} x+\int \cot x \mathrm{~d}\left(\csc ^{n-2} x\right) \\
&=\cot x \csc ^{n-2} x-(n-2) \int \csc ^{n} x \mathrm{~d} x+(n-2) \int \csc ^{n-2} x \mathrm{~d} x
\end{aligned}
$$
\begin{CJK}{UTF8}{mj}故\end{CJK}
$$
[1+(n-2)] \int \csc ^{n} x \mathrm{~d} x=-\cot x \csc ^{n-2} x+(n-2) \int \csc ^{n-2} x \mathrm{~d} x
$$
\begin{CJK}{UTF8}{mj}因此\end{CJK}
$$
I_{n}=\int \frac{\mathrm{d} x}{\sin ^{n} x}
$$
$$
=-\frac{\cos x}{(n-1) \sin ^{n-1} x}+\frac{n-2}{n-1} I_{n-2}
$$
\begin{CJK}{UTF8}{mj}所以可得\end{CJK}
$$
\begin{aligned}
\int \frac{\mathrm{d} x}{\sin ^{5} x} \\
&=-\frac{\cos x}{4 \sin ^{4} x}+\frac{3}{4} \int \frac{\mathrm{d} x}{\sin ^{3} x} \\
&=-\frac{\cos x}{4 \sin ^{4} x}+\frac{3}{4}\left(\frac{\cos x}{2 \sin ^{2} x}+\frac{1}{2} \int \frac{\mathrm{d} x}{\sin x}\right) \\
&=-\frac{\cos x}{4 \sin ^{4} x}+\frac{3}{4}\left(\frac{\cos x}{2 \sin ^{2} x}+\frac{1}{2} \ln |\csc x-\cot x|+C\right) \\
&=-\frac{\cos x}{4 \sin ^{4} x}+\frac{3 \cos x}{8 \sin ^{2} x}+\frac{3}{8} \ln |\csc x-\cot x|+C
\end{aligned}
$$

\begin{enumerate}
  \setcounter{enumi}{4}
  \item (20 \begin{CJK}{UTF8}{mj}分\end{CJK}) \begin{CJK}{UTF8}{mj}设函数\end{CJK} $f(x)$ \begin{CJK}{UTF8}{mj}在\end{CJK} $[a,+\infty)$ \begin{CJK}{UTF8}{mj}上连续\end{CJK}, \begin{CJK}{UTF8}{mj}且有渐进线\end{CJK} $y=c x$ (\begin{CJK}{UTF8}{mj}其中\end{CJK} $c$ \begin{CJK}{UTF8}{mj}为常数\end{CJK}), \begin{CJK}{UTF8}{mj}证明\end{CJK}: $f(x)$ \begin{CJK}{UTF8}{mj}在\end{CJK} $[a,+\infty)$ \begin{CJK}{UTF8}{mj}上一致连续\end{CJK}.
\end{enumerate}
\begin{CJK}{UTF8}{mj}证明\end{CJK}: \begin{CJK}{UTF8}{mj}由题设有渐进线\end{CJK} $y=c x$ \begin{CJK}{UTF8}{mj}可知\end{CJK}
$$
\lim _{x \rightarrow \infty}(f(x)-c x)=0
$$
\begin{CJK}{UTF8}{mj}故\end{CJK} $\exists N>a$, \begin{CJK}{UTF8}{mj}当\end{CJK} $x>N$ \begin{CJK}{UTF8}{mj}有\end{CJK}
$$
|f(x)-c x|<\frac{\varepsilon}{3}
$$
\begin{CJK}{UTF8}{mj}取\end{CJK} $\delta_{1}=\frac{\varepsilon}{3|c|+1}$, \begin{CJK}{UTF8}{mj}当\end{CJK} $x_{1}, x_{2}>N$, \begin{CJK}{UTF8}{mj}只要\end{CJK} $\left|x_{1}-x_{2}\right|<\delta_{1}$, \begin{CJK}{UTF8}{mj}则有\end{CJK}
$$
\begin{aligned}
\left|f\left(x_{1}\right)-f\left(x_{2}\right)\right| &=\left|f\left(x_{1}\right)-c x_{1}-f\left(x_{2}\right)-c x_{2}+c\left(x_{1}-x_{2}\right)\right| \\
& \leq\left|f\left(x_{1}\right)-c x_{1}\right|+\left|f\left(x_{2}\right)-c x_{2}\right|+c\left|x_{1}-x_{2}\right| \\
&<\frac{\varepsilon}{3}+\frac{\varepsilon}{3}+c \frac{\varepsilon}{3|c|+1} \\
&<\varepsilon
\end{aligned}
$$
$$
\begin{aligned}
& <\varepsilon
\end{aligned}
$$
\begin{CJK}{UTF8}{mj}又\end{CJK} $f(x)$ \begin{CJK}{UTF8}{mj}在\end{CJK} $[a, N+1]$ \begin{CJK}{UTF8}{mj}连续\end{CJK}, \begin{CJK}{UTF8}{mj}故一致连续\end{CJK}, \begin{CJK}{UTF8}{mj}则\end{CJK} $\exists \delta_{2}>0$, \begin{CJK}{UTF8}{mj}当\end{CJK} $x_{1}, x_{2} \in[a, N+1]$, \begin{CJK}{UTF8}{mj}只要\end{CJK} $\left|x_{1}-x_{2}\right|<\delta_{2}$, \begin{CJK}{UTF8}{mj}则有\end{CJK}
$$
\left|f\left(x_{1}\right)-f\left(x_{2}\right)\right|<\varepsilon
$$
\begin{CJK}{UTF8}{mj}令\end{CJK} $\delta=\min \left\{\delta_{1}, \delta_{2}, 1\right\}$, \begin{CJK}{UTF8}{mj}则对\end{CJK} $\forall x_{1}, x_{2} \in[a,+\infty)$ \begin{CJK}{UTF8}{mj}时\end{CJK}, \begin{CJK}{UTF8}{mj}则或者\end{CJK} $\left|x_{1}-x_{2}\right|<\delta$, \begin{CJK}{UTF8}{mj}则或者\end{CJK} $x_{1}, x_{2} \in[a, N+1]$, \begin{CJK}{UTF8}{mj}则或者\end{CJK} $x_{1}, x_{2}>N$ \begin{CJK}{UTF8}{mj}都有\end{CJK}
$$
\left|f\left(x_{1}\right)-f\left(x_{2}\right)\right|<\varepsilon
$$
\begin{CJK}{UTF8}{mj}从而\end{CJK} $f(x)$ \begin{CJK}{UTF8}{mj}在\end{CJK} $[a,+\infty)$ \begin{CJK}{UTF8}{mj}上一致连续\end{CJK}.

\begin{enumerate}
  \setcounter{enumi}{5}
  \item $(20$ \begin{CJK}{UTF8}{mj}分\end{CJK} $)$ \begin{CJK}{UTF8}{mj}问\end{CJK} $k$ \begin{CJK}{UTF8}{mj}取何值时\end{CJK}, \begin{CJK}{UTF8}{mj}函数列\end{CJK}
\end{enumerate}
$$
f_{n}(x)=x n^{k} \mathrm{e}^{-n x}
$$
\begin{CJK}{UTF8}{mj}在\end{CJK} $[0,+\infty)$ \begin{CJK}{UTF8}{mj}上一致收敛\end{CJK}?

\begin{CJK}{UTF8}{mj}解\end{CJK}: \begin{CJK}{UTF8}{mj}显然\end{CJK} $f_{n}(x)$ \begin{CJK}{UTF8}{mj}在\end{CJK} $[0,+\infty)$ \begin{CJK}{UTF8}{mj}收敛\end{CJK}, \begin{CJK}{UTF8}{mj}且\end{CJK} $\lim _{n \rightarrow \infty} f_{n}(x)=0$, \begin{CJK}{UTF8}{mj}可知\end{CJK}
$$
f_{n}^{\prime}(x)=n^{k}(1-n x) \mathrm{e}^{-n x}
$$
\begin{CJK}{UTF8}{mj}可知在\end{CJK} $x=\frac{1}{n}$ \begin{CJK}{UTF8}{mj}取最大值\end{CJK}, \begin{CJK}{UTF8}{mj}则\end{CJK}
$$
\begin{aligned}
& \lim _{n \rightarrow \infty} \sup _{x \in[0,+\infty)}\left|f_{n}(x)-0\right| \\
=& \lim _{n \rightarrow \infty} \sup _{x \in[0,+\infty)} f_{n}\left(\frac{1}{n}\right) \\
=& \lim _{n \rightarrow \infty} \frac{n^{k-1}}{\mathrm{e}} \\
=\left\{\begin{array}{l}
0, k<1 \\
\frac{1}{\mathrm{e}}, k=1 \\
+\infty, k>1
\end{array}\right.
\end{aligned}
$$
\begin{CJK}{UTF8}{mj}故当\end{CJK} $k<1$ \begin{CJK}{UTF8}{mj}时\end{CJK}, $f_{n}(x)$ \begin{CJK}{UTF8}{mj}在\end{CJK} $[0,+\infty)$ \begin{CJK}{UTF8}{mj}上一致收敛\end{CJK}, \begin{CJK}{UTF8}{mj}当\end{CJK} $k \geq$ \begin{CJK}{UTF8}{mj}时\end{CJK}, $f_{n}(x)$ \begin{CJK}{UTF8}{mj}在\end{CJK} $[0,+\infty)$ \begin{CJK}{UTF8}{mj}上不一致收敛\end{CJK}. 6. (20 \begin{CJK}{UTF8}{mj}分\end{CJK}) \begin{CJK}{UTF8}{mj}讨论方程\end{CJK} $\ln x=k x$ \begin{CJK}{UTF8}{mj}实根的情况\end{CJK}.

\begin{CJK}{UTF8}{mj}解\end{CJK}: \begin{CJK}{UTF8}{mj}设\end{CJK} $y=k x$ \begin{CJK}{UTF8}{mj}与\end{CJK} $y=\ln x$ \begin{CJK}{UTF8}{mj}相切\end{CJK}, \begin{CJK}{UTF8}{mj}其切点为\end{CJK} $P\left(x_{0}, y_{0}\right)$, \begin{CJK}{UTF8}{mj}则有\end{CJK}
$$
\left\{\begin{array} { l } 
{ y _ { 0 } = k x _ { 0 } } \\
{ y _ { 0 } = \operatorname { l n } x _ { 0 } } \\
{ \frac { 1 } { x _ { 0 } } = k }
\end{array} \Rightarrow \left\{\begin{array}{l}
k=\frac{1}{\mathrm{e}} \\
x_{0}=\mathrm{e} \\
y_{0}=1
\end{array}\right.\right.
$$
\begin{CJK}{UTF8}{mj}因此\end{CJK} - \begin{CJK}{UTF8}{mj}当\end{CJK} $k \leq 0$, \begin{CJK}{UTF8}{mj}方程\end{CJK} $\ln x-k x=0$ \begin{CJK}{UTF8}{mj}只有一个实根\end{CJK}; - \begin{CJK}{UTF8}{mj}当\end{CJK} $0 k \frac{1}{\mathrm{e}}$, \begin{CJK}{UTF8}{mj}方程\end{CJK} $\ln x-k x=0$ \begin{CJK}{UTF8}{mj}有两个不相等\end{CJK} \begin{CJK}{UTF8}{mj}的实数根\end{CJK}; - \begin{CJK}{UTF8}{mj}当\end{CJK} $k=\frac{1}{\mathrm{e}}$, \begin{CJK}{UTF8}{mj}方程\end{CJK} $\ln x-k x=0$ \begin{CJK}{UTF8}{mj}只有一个实根\end{CJK}; - \begin{CJK}{UTF8}{mj}当\end{CJK} $k>\frac{1}{\mathrm{e}}$, \begin{CJK}{UTF8}{mj}方程\end{CJK} $\ln x-k x=0$ \begin{CJK}{UTF8}{mj}无根\end{CJK}. \begin{CJK}{UTF8}{mj}变式\end{CJK}:\begin{CJK}{UTF8}{mj}设\end{CJK} $a>0$, \begin{CJK}{UTF8}{mj}讨论方程\end{CJK} $\mathrm{e}^{x}=a x^{2}$ \begin{CJK}{UTF8}{mj}有几个实根\end{CJK}.

\begin{CJK}{UTF8}{mj}解\end{CJK}:\begin{CJK}{UTF8}{mj}当\end{CJK} $0<a<\frac{\mathrm{e}^{2}}{4}$ \begin{CJK}{UTF8}{mj}时\end{CJK}, \begin{CJK}{UTF8}{mj}方程有\end{CJK} 1 \begin{CJK}{UTF8}{mj}个实根\end{CJK}; \begin{CJK}{UTF8}{mj}当\end{CJK} $a=\frac{\mathrm{e}^{2}}{4}$ \begin{CJK}{UTF8}{mj}时\end{CJK}, \begin{CJK}{UTF8}{mj}程有\end{CJK} 2 \begin{CJK}{UTF8}{mj}个实根\end{CJK}; \begin{CJK}{UTF8}{mj}当\end{CJK} $a>\frac{\mathrm{e}^{2}}{4}$ \begin{CJK}{UTF8}{mj}时\end{CJK}, \begin{CJK}{UTF8}{mj}方程有\end{CJK} 3 \begin{CJK}{UTF8}{mj}个\end{CJK} \begin{CJK}{UTF8}{mj}实根\end{CJK}.

\begin{enumerate}
  \setcounter{enumi}{7}
  \item (20 \begin{CJK}{UTF8}{mj}分\end{CJK}) \begin{CJK}{UTF8}{mj}试证\end{CJK}:\begin{CJK}{UTF8}{mj}数列\end{CJK} $\left\{a_{n}\right\}$ \begin{CJK}{UTF8}{mj}收敛\end{CJK}, \begin{CJK}{UTF8}{mj}其中\end{CJK}
\end{enumerate}
$$
a_{n}=\sum_{k=1}^{n} \frac{1}{k}-\ln (n+1)
$$
\begin{CJK}{UTF8}{mj}证日月\end{CJK}:\begin{CJK}{UTF8}{mj}考虑到\end{CJK}
$$
\begin{aligned}
\lim _{n \rightarrow \infty} a_{n} &=\lim _{n \rightarrow \infty} \sum_{k=1}^{n} \frac{1}{k}-\ln (1+n) \\
&=\lim _{n \rightarrow \infty}\left(\sum_{k=1}^{n} \frac{1}{k}-\ln n\right)+\lim _{n \rightarrow \infty}(\ln n-\ln (1\\
&=\gamma+\lim _{n \rightarrow \infty}\left(\ln \frac{n}{1+n}\right)
\end{aligned}
$$
\begin{CJK}{UTF8}{mj}其中\end{CJK} $\gamma$ \begin{CJK}{UTF8}{mj}为欧拉常数\end{CJK}, \begin{CJK}{UTF8}{mj}即证数列\end{CJK} $\left\{a_{n}\right\}$ \begin{CJK}{UTF8}{mj}收敛\end{CJK}.

\section{2 言等代敉}
\begin{enumerate}
  \item (15 \begin{CJK}{UTF8}{mj}分\end{CJK}) \begin{CJK}{UTF8}{mj}计算行列式\end{CJK}
\end{enumerate}
$$
D_{n}=\left|\begin{array}{cccc}
x_{1}+x & x_{2} & \cdots & x_{n} \\
x_{1} & x_{2}+x & \cdots & x_{n} \\
\vdots & \vdots & & \vdots \\
x_{1} & x_{2} & \cdots & x_{n}+x
\end{array}\right|
$$
\begin{CJK}{UTF8}{mj}微信公众号\end{CJK}: \begin{CJK}{UTF8}{mj}八一考研数学竞赛\end{CJK} 2. (15 \begin{CJK}{UTF8}{mj}分\end{CJK}) \begin{CJK}{UTF8}{mj}当\end{CJK} $a, b$ \begin{CJK}{UTF8}{mj}取何值时\end{CJK}, \begin{CJK}{UTF8}{mj}如下线性方程组有解\end{CJK}? \begin{CJK}{UTF8}{mj}在有解的情形下求通解\end{CJK}
$$
\left\{\begin{array}{l}
x_{1}+x_{2}+x_{3}+x_{4}+x_{5}=1 \\
3 x_{1}+2 x_{2}+x_{3}+x_{4}-3 x_{5}=a \\
x_{1}+2 x_{2}+3 x_{3}+3 x_{4}+7 x_{5}=4 \\
5 x_{1}+4 x_{2}+3 x_{3}+3 x_{4}+x_{5}=b
\end{array}\right.
$$

\begin{enumerate}
  \setcounter{enumi}{3}
  \item (15 \begin{CJK}{UTF8}{mj}分\end{CJK}) \begin{CJK}{UTF8}{mj}设四闭泩\end{CJK}
\end{enumerate}
$$
A=\left(\begin{array}{cccc}
1 & 2 & 0 & 0 \\
2 & 1 & 0 & 0 \\
0 & 0 & 4 & -1 \\
0 & 0 & 4 & 0
\end{array}\right)
$$
\begin{CJK}{UTF8}{mj}㲌\end{CJK} $A^{2020}$

\begin{enumerate}
  \setcounter{enumi}{4}
  \item (15 \begin{CJK}{UTF8}{mj}分\end{CJK}) \begin{CJK}{UTF8}{mj}设四维线性空间\end{CJK} $V$ \begin{CJK}{UTF8}{mj}的一组基为\end{CJK} $\varepsilon_{1}, \varepsilon_{2}, \varepsilon_{3}, \bar{\varepsilon}_{4}$, \begin{CJK}{UTF8}{mj}线性变换\end{CJK} $\mathscr{A}$ \begin{CJK}{UTF8}{mj}在这组基下的矩阵\end{CJK} \begin{CJK}{UTF8}{mj}为\end{CJK}
\end{enumerate}
$$
A=\left(\begin{array}{cccc}
1 & 0 & 2 & 1 \\
-1 & 2 & 1 & 3 \\
1 & 2 & 5 & 5 \\
2 & -2 & 1 & -2
\end{array}\right)
$$
\begin{CJK}{UTF8}{mj}求\end{CJK} $\mathscr{A}$ \begin{CJK}{UTF8}{mj}在基\end{CJK}
$$
\eta_{1}=\varepsilon_{1}-2 \varepsilon_{2}+\varepsilon_{4}, \quad r_{2}=3 \varepsilon_{2}-\varepsilon_{3}-\varepsilon_{4}, \eta_{3}=\varepsilon_{3}+\varepsilon_{4}, \eta_{4}=2 \varepsilon_{4}
$$
\begin{CJK}{UTF8}{mj}下的矩阵\end{CJK}, \begin{CJK}{UTF8}{mj}并求\end{CJK} $\mathscr{A}$ \begin{CJK}{UTF8}{mj}的核与值域\end{CJK}.

\begin{enumerate}
  \setcounter{enumi}{5}
  \item (15 \begin{CJK}{UTF8}{mj}分\end{CJK}) \begin{CJK}{UTF8}{mj}设向量组\end{CJK} I : $\alpha_{1}, \alpha_{2}, \cdots, \alpha_{r}$ \begin{CJK}{UTF8}{mj}与向量组\end{CJK} II : $\beta_{1}, \beta_{2}, \cdots, \beta_{s}$ \begin{CJK}{UTF8}{mj}有相同的秩\end{CJK}, \begin{CJK}{UTF8}{mj}且向量组\end{CJK} $I$ \begin{CJK}{UTF8}{mj}可由向量组\end{CJK} mathrmII \begin{CJK}{UTF8}{mj}线性表出\end{CJK}, \begin{CJK}{UTF8}{mj}证明向量组\end{CJK} I \begin{CJK}{UTF8}{mj}与向量组\end{CJK} II \begin{CJK}{UTF8}{mj}等价\end{CJK}.

  \item (15 \begin{CJK}{UTF8}{mj}分\end{CJK}) \begin{CJK}{UTF8}{mj}设\end{CJK} $A, C$ \begin{CJK}{UTF8}{mj}是实数域上的\end{CJK} $n$ \begin{CJK}{UTF8}{mj}阶正定矩阵\end{CJK}, $B$ \begin{CJK}{UTF8}{mj}是矩阵方程\end{CJK} $A X+X A=C$ \begin{CJK}{UTF8}{mj}的唯一\end{CJK} \begin{CJK}{UTF8}{mj}解\end{CJK}, \begin{CJK}{UTF8}{mj}证明\end{CJK} $B$ \begin{CJK}{UTF8}{mj}也是正定矩阵\end{CJK}.

  \item (15 \begin{CJK}{UTF8}{mj}分\end{CJK}) \begin{CJK}{UTF8}{mj}设\end{CJK} $b_{1}, b_{2}, \cdots, b_{r}$ \begin{CJK}{UTF8}{mj}是互不相同的\end{CJK} $r$ \begin{CJK}{UTF8}{mj}个实数\end{CJK}, \begin{CJK}{UTF8}{mj}且\end{CJK} $r \leq n$, \begin{CJK}{UTF8}{mj}证明向量组\end{CJK}

\end{enumerate}
$$
\because\left(\alpha_{1}=\left(\begin{array}{c}
1 \\
b_{1} \\
b_{1}^{2} \\
\vdots \\
b_{1}^{n-1}
\end{array}\right), \alpha_{2}=\left(\begin{array}{c}
1 \\
b_{2} \\
b_{2}^{2} \\
\vdots \\
b_{2}^{n-1}
\end{array}\right), \cdots, \alpha_{r}=\left(\begin{array}{c}
1 \\
b_{r} \\
b_{r}^{2} \\
\vdots \\
b_{r}^{n-1}
\end{array}\right)\right.
$$
\begin{CJK}{UTF8}{mj}线性无关\end{CJK}. 8. (15 \begin{CJK}{UTF8}{mj}分\end{CJK}) \begin{CJK}{UTF8}{mj}设\end{CJK} $A, B$ \begin{CJK}{UTF8}{mj}分别为\end{CJK} $m \times n$ \begin{CJK}{UTF8}{mj}与\end{CJK} $n \times m$ \begin{CJK}{UTF8}{mj}矩阵\end{CJK}, \begin{CJK}{UTF8}{mj}如果\end{CJK} $E_{m}-A B$ \begin{CJK}{UTF8}{mj}可逆\end{CJK}, \begin{CJK}{UTF8}{mj}证明\end{CJK}: $E_{n}-B A$ \begin{CJK}{UTF8}{mj}也可逆\end{CJK}, \begin{CJK}{UTF8}{mj}且\end{CJK}
$$
\left(E_{n}-B A\right)^{-1}=E_{n}+B\left(E_{m}-A B\right)^{-1} A
$$

\begin{enumerate}
  \setcounter{enumi}{9}
  \item ( 15 \begin{CJK}{UTF8}{mj}分\end{CJK}) \begin{CJK}{UTF8}{mj}已知\end{CJK} $A$ \begin{CJK}{UTF8}{mj}为\end{CJK} $n$ \begin{CJK}{UTF8}{mj}阶实矩阵\end{CJK}, \begin{CJK}{UTF8}{mj}且存在正整数\end{CJK} $m$ \begin{CJK}{UTF8}{mj}使得\end{CJK} $r\left(A^{m}\right)=r\left(A^{m+1}\right)$, \begin{CJK}{UTF8}{mj}证明\end{CJK}: \begin{CJK}{UTF8}{mj}对\end{CJK} \begin{CJK}{UTF8}{mj}所有的正整数\end{CJK} $k$, \begin{CJK}{UTF8}{mj}都有\end{CJK} $r\left(A^{m}\right)=r\left(A^{m+k}\right)$.

  \item ( 15 \begin{CJK}{UTF8}{mj}分\end{CJK}) \begin{CJK}{UTF8}{mj}设\end{CJK} $A, B, C$ \begin{CJK}{UTF8}{mj}分别为数域\end{CJK} $P$ \begin{CJK}{UTF8}{mj}上的\end{CJK} $m \times n, p \times q, m \times q$ \begin{CJK}{UTF8}{mj}矩阵\end{CJK}, \begin{CJK}{UTF8}{mj}证明\end{CJK}: \begin{CJK}{UTF8}{mj}矩阵方程\end{CJK} $A X-Y B=C$ \begin{CJK}{UTF8}{mj}有解的充要条件是\end{CJK}

\end{enumerate}
$$
r\left(\begin{array}{ll}
A & O \\
O & B
\end{array}\right)=r\left(\begin{array}{ll}
A & C \\
O & B
\end{array}\right)
$$

\section{第十四章 2021 年大连理工大学真题}
\section{$14.1$ 数学分析}
\begin{CJK}{UTF8}{mj}一\end{CJK}、\begin{CJK}{UTF8}{mj}解答题\end{CJK} (\begin{CJK}{UTF8}{mj}每题\end{CJK} 6 \begin{CJK}{UTF8}{mj}分\end{CJK}, \begin{CJK}{UTF8}{mj}共\end{CJK} 60 \begin{CJK}{UTF8}{mj}分\end{CJK})

\begin{enumerate}
  \item \begin{CJK}{UTF8}{mj}求极限\end{CJK}
\end{enumerate}
$$
\lim _{x \rightarrow+\infty}\left[\frac{x^{2}}{\mathrm{e}}\left(1+\frac{1}{x}\right)^{x}-x^{2}+\frac{x}{2}\right]
$$

\begin{enumerate}
  \setcounter{enumi}{2}
  \item \begin{CJK}{UTF8}{mj}设\end{CJK} $x_{1}, x_{2}, \cdots, x_{n}$ \begin{CJK}{UTF8}{mj}为正数\end{CJK}, \begin{CJK}{UTF8}{mj}证明\end{CJK}
\end{enumerate}
$$
\frac{x_{1}+x_{2}+\cdots+x_{n}}{n} \leq\left(x_{1}^{x_{1}} x_{2}^{x_{2}} \cdots x_{n}^{x_{n}}\right)^{\frac{1}{x_{1}+x_{2}+\cdots+x_{n}}}
$$

\begin{enumerate}
  \setcounter{enumi}{3}
  \item \begin{CJK}{UTF8}{mj}证明\end{CJK} $f(x)=\sum_{n=1}^{\infty} \frac{\mathrm{e}^{-n x}}{n}$ \begin{CJK}{UTF8}{mj}在\end{CJK} $(0,+\infty)$ \begin{CJK}{UTF8}{mj}上有任意阶导数\end{CJK}.

  \item \begin{CJK}{UTF8}{mj}问级数\end{CJK} $\sum_{n=1}^{\infty} \frac{\cos n}{n}$ \begin{CJK}{UTF8}{mj}是条件收玫还是绝对收敛\end{CJK}? \begin{CJK}{UTF8}{mj}为什么\end{CJK}?

  \item \begin{CJK}{UTF8}{mj}将\end{CJK} $f(x, y)=\frac{\cos y}{\cos x}$ \begin{CJK}{UTF8}{mj}在\end{CJK} $(0,0)$ \begin{CJK}{UTF8}{mj}做\end{CJK} Taylor \begin{CJK}{UTF8}{mj}展开\end{CJK}, \begin{CJK}{UTF8}{mj}要求展开到二次项\end{CJK}.

  \item \begin{CJK}{UTF8}{mj}对给定的正整数\end{CJK} $p$, \begin{CJK}{UTF8}{mj}有\end{CJK} $\lim _{n \rightarrow \infty}\left(a_{n+p}-a_{n}\right)=0$, \begin{CJK}{UTF8}{mj}问\end{CJK} $\left\{a_{n}\right\}$ \begin{CJK}{UTF8}{mj}是否收玫\end{CJK}, \begin{CJK}{UTF8}{mj}为什么\end{CJK}?

  \item \begin{CJK}{UTF8}{mj}设\end{CJK} $a_{n}>0(n=1,2, \cdots)$, \begin{CJK}{UTF8}{mj}证明\end{CJK} $\lim _{n \rightarrow \infty} \sup \sqrt[n]{a_{n}} \leq \lim _{n \rightarrow \infty} \sup \frac{a_{n+1}}{a_{n}}$.

  \item \begin{CJK}{UTF8}{mj}设\end{CJK} $f(x)$ \begin{CJK}{UTF8}{mj}在\end{CJK} $[0,+\infty)$ \begin{CJK}{UTF8}{mj}上连续\end{CJK}, \begin{CJK}{UTF8}{mj}且\end{CJK} $\lim _{x \rightarrow+\infty}[f(x)-a x-b]=0$, \begin{CJK}{UTF8}{mj}证明\end{CJK} $f(x)$ \begin{CJK}{UTF8}{mj}在\end{CJK} $[0,+\infty)$ \begin{CJK}{UTF8}{mj}上一致连续\end{CJK}.

  \item \begin{CJK}{UTF8}{mj}设函数\end{CJK} $f(x)$ \begin{CJK}{UTF8}{mj}在\end{CJK} $(-\infty,+\infty)$ \begin{CJK}{UTF8}{mj}上有三阶连续导数\end{CJK}, \begin{CJK}{UTF8}{mj}且\end{CJK} $f(0)=1, f^{\prime}(0)=0$, \begin{CJK}{UTF8}{mj}定义函数\end{CJK}

\end{enumerate}
$$
g(x)= \begin{cases}\frac{f(x)-1}{x^{2}}, & x \neq 0 \\ \frac{f^{\prime \prime}(0)}{2}, & x=0 .\end{cases}
$$
\begin{CJK}{UTF8}{mj}证明\end{CJK} $g(x)$ \begin{CJK}{UTF8}{mj}在\end{CJK} $(-\infty,+\infty)$ \begin{CJK}{UTF8}{mj}止有连续的导数\end{CJK}. 10. \begin{CJK}{UTF8}{mj}设\end{CJK} $A_{n} \subseteq[0,1]$ \begin{CJK}{UTF8}{mj}是有限集其中\end{CJK} $n=1,2, \cdots$, \begin{CJK}{UTF8}{mj}且当\end{CJK} $i \neq j$ \begin{CJK}{UTF8}{mj}吋\end{CJK}, $A_{i} \cap A_{j}=\emptyset$, \begin{CJK}{UTF8}{mj}定义函数\end{CJK}
$$
f(x)= \begin{cases}\frac{1}{n}, & x \in A_{n}, n=1,2, \cdots \\ 0, & x \in[0,1]-\bigcup_{n=1}^{\infty} A_{n} .\end{cases}
$$
\begin{CJK}{UTF8}{mj}对任意的\end{CJK} $a \in(0,1)$, \begin{CJK}{UTF8}{mj}求\end{CJK} $\lim _{x \rightarrow a} f(x)$.

\begin{CJK}{UTF8}{mj}二\end{CJK}、\begin{CJK}{UTF8}{mj}计算题\end{CJK} (\begin{CJK}{UTF8}{mj}每题\end{CJK} 10 \begin{CJK}{UTF8}{mj}分\end{CJK}, \begin{CJK}{UTF8}{mj}共\end{CJK} 30 \begin{CJK}{UTF8}{mj}分\end{CJK})

\begin{enumerate}
  \item \begin{CJK}{UTF8}{mj}设\end{CJK} $f(x, y), x(s, t), y(s, t)$ \begin{CJK}{UTF8}{mj}均具有二阶连续偏导数\end{CJK}, \begin{CJK}{UTF8}{mj}令\end{CJK} $u=f(x(s, t), y(s, t))$, \begin{CJK}{UTF8}{mj}求\end{CJK} $\frac{\partial^{2} u}{\partial s \partial t}$.

  \item \begin{CJK}{UTF8}{mj}求函数\end{CJK} $f(x)=\ln \left(x+\sqrt{1+x^{2}}\right)$ \begin{CJK}{UTF8}{mj}在\end{CJK} $x=0$ \begin{CJK}{UTF8}{mj}处的幕级数展开式\end{CJK}.

  \item \begin{CJK}{UTF8}{mj}计算第二型曲面积分\end{CJK}

\end{enumerate}
$$
I=\iint_{\Sigma}(z-x) \mathrm{d} x \mathrm{~d} y+(x-y) \mathrm{d} z \mathrm{~d} x+(y-z) \mathrm{d} x \mathrm{~d} y
$$
\begin{CJK}{UTF8}{mj}其中\end{CJK} $\Sigma$ \begin{CJK}{UTF8}{mj}为曲面\end{CJK} $z=\sqrt{x^{2}+y^{2}}, 0 \leq z \leq h(h>0)$, \begin{CJK}{UTF8}{mj}取上侧\end{CJK}.

\begin{CJK}{UTF8}{mj}三\end{CJK}、\begin{CJK}{UTF8}{mj}证明题\end{CJK} (\begin{CJK}{UTF8}{mj}每题\end{CJK} 12 \begin{CJK}{UTF8}{mj}分\end{CJK}, \begin{CJK}{UTF8}{mj}共\end{CJK} 60 \begin{CJK}{UTF8}{mj}分\end{CJK})

\begin{enumerate}
  \item \begin{CJK}{UTF8}{mj}设\end{CJK} $F(x, y)=(P(x, y), Q(x, y))$ \begin{CJK}{UTF8}{mj}在区域\end{CJK} $D \subseteq \mathbb{R}^{2}$ \begin{CJK}{UTF8}{mj}上连续可微\end{CJK}, \begin{CJK}{UTF8}{mj}对\end{CJK} $D$ \begin{CJK}{UTF8}{mj}内任意逆时针\end{CJK} \begin{CJK}{UTF8}{mj}方向的圆周\end{CJK} $C$, \begin{CJK}{UTF8}{mj}总有\end{CJK} $\int_{C} \boldsymbol{F} \cdot \boldsymbol{n} \mathrm{d} s=0$, \begin{CJK}{UTF8}{mj}其中\end{CJK} $\boldsymbol{n}$ \begin{CJK}{UTF8}{mj}为\end{CJK} $C$ \begin{CJK}{UTF8}{mj}的单位外法向\end{CJK}, $s$ \begin{CJK}{UTF8}{mj}为弧长参数\end{CJK}. \begin{CJK}{UTF8}{mj}证明\end{CJK}: \begin{CJK}{UTF8}{mj}在\end{CJK} $D$ \begin{CJK}{UTF8}{mj}上有\end{CJK} $\frac{\partial P}{\partial x}+\frac{\partial Q}{\partial y}=0$.

  \item \begin{CJK}{UTF8}{mj}定义函数\end{CJK}

\end{enumerate}
$$
F(u)=\frac{1}{2 \pi} \int_{0}^{2 \pi} \mathrm{e}^{u \cos x} \cos (u \sin x) \mathrm{d} x, u \in \mathbb{R}
$$
\begin{CJK}{UTF8}{mj}证明\end{CJK}

(1) $F(u)$ \begin{CJK}{UTF8}{mj}在\end{CJK} $(-\infty,+\infty)$ \begin{CJK}{UTF8}{mj}上有任意阶导数\end{CJK}, \begin{CJK}{UTF8}{mj}且存在与\end{CJK} $n$ \begin{CJK}{UTF8}{mj}无关的常数\end{CJK} $M(u)>0$, \begin{CJK}{UTF8}{mj}使\end{CJK} \begin{CJK}{UTF8}{mj}得\end{CJK}
$$
\left|F^{(n)}(u)\right| \leq M(u)
$$
(2) $F(u)=1, \forall u \in \mathbb{R}$

\begin{enumerate}
  \setcounter{enumi}{3}
  \item \begin{CJK}{UTF8}{mj}定义\end{CJK} $a_{n}=\int_{0}^{\frac{\pi}{2}} t\left|\frac{\sin n t}{\sin t}\right|^{3} \mathrm{~d} t, n=1,2, \cdots$, \begin{CJK}{UTF8}{mj}证明级数\end{CJK} $\sum_{n=1}^{\infty} \frac{1}{a_{n}}$ \begin{CJK}{UTF8}{mj}发散\end{CJK}.

  \item \begin{CJK}{UTF8}{mj}设\end{CJK} $f(x) \in C[a, b]$, \begin{CJK}{UTF8}{mj}且\end{CJK} $f(a)=f(b)=0$, \begin{CJK}{UTF8}{mj}若对任意的\end{CJK} $x \in(a, b)$, \begin{CJK}{UTF8}{mj}有\end{CJK}

\end{enumerate}
$$
\int \lim _{h \rightarrow 0} \frac{f(x+h)+f(x-h)-2 f(x)}{h^{2}}=0
$$
\begin{CJK}{UTF8}{mj}证明\end{CJK} $f(x) \equiv 0, x \in[a, b]$.

\begin{enumerate}
  \setcounter{enumi}{5}
  \item \begin{CJK}{UTF8}{mj}证明\end{CJK}
\end{enumerate}
$$
\lim _{n \rightarrow \infty} \int_{0}^{+\infty} \frac{n^{2} x}{1+x^{2}} \mathrm{e}^{-n^{2} x^{2}} \mathrm{~d} x=\frac{1}{2}
$$

\section{$14.2$ 高等代数}
\begin{CJK}{UTF8}{mj}一\end{CJK}、\begin{CJK}{UTF8}{mj}计算题\end{CJK} (\begin{CJK}{UTF8}{mj}每题\end{CJK} 10 \begin{CJK}{UTF8}{mj}分\end{CJK}, \begin{CJK}{UTF8}{mj}共\end{CJK} 30 \begin{CJK}{UTF8}{mj}分\end{CJK})

\begin{enumerate}
  \item \begin{CJK}{UTF8}{mj}设\end{CJK} $f(x)$ \begin{CJK}{UTF8}{mj}是数域\end{CJK} $\mathbb{P}$ \begin{CJK}{UTF8}{mj}上的\end{CJK} $n$ \begin{CJK}{UTF8}{mj}次多项式\end{CJK}, \begin{CJK}{UTF8}{mj}且当\end{CJK} $k=0,1, \ldots, n$ \begin{CJK}{UTF8}{mj}时\end{CJK}, \begin{CJK}{UTF8}{mj}有\end{CJK} $f(k)=\frac{k}{k+1}$, \begin{CJK}{UTF8}{mj}求\end{CJK} $f(n+1)$.

  \item \begin{CJK}{UTF8}{mj}求一个非退化线性替换\end{CJK}, \begin{CJK}{UTF8}{mj}将实二次型\end{CJK} $f\left(x_{1}, x_{2}, x_{3}\right)=2 x_{1} x_{2}+2 x_{1} x_{3}-6 x_{2} x_{3}$ \begin{CJK}{UTF8}{mj}化为\end{CJK} \begin{CJK}{UTF8}{mj}标准型\end{CJK}.

  \item \begin{CJK}{UTF8}{mj}在\end{CJK} $\mathbb{R}^{4}$ \begin{CJK}{UTF8}{mj}中\end{CJK}, \begin{CJK}{UTF8}{mj}设\end{CJK}

\end{enumerate}
$$
\alpha_{1}=(1,0,-1,0)^{\prime}, \alpha_{2}=(0,1,2,1)^{\prime}, \alpha_{3}=(2,1,0,1)^{\prime}
$$
\begin{CJK}{UTF8}{mj}生成的子空问为\end{CJK} $V_{1}$, \begin{CJK}{UTF8}{mj}设\end{CJK}
$$
\beta_{1}=(-1,1,1,1)^{\prime}, \beta_{2}=(1,-1,-3,-1)^{\prime}
$$
\begin{CJK}{UTF8}{mj}生成的子空间为\end{CJK} $V_{2}$, \begin{CJK}{UTF8}{mj}求\end{CJK} $\operatorname{dim}\left(V_{1}+V_{2}\right)$ \begin{CJK}{UTF8}{mj}和\end{CJK} $\operatorname{dim}\left(V_{1} \cap V_{2}\right)$.

\begin{CJK}{UTF8}{mj}二\end{CJK}、 \begin{CJK}{UTF8}{mj}证明题\end{CJK} (\begin{CJK}{UTF8}{mj}每题\end{CJK} 10 \begin{CJK}{UTF8}{mj}分\end{CJK}, \begin{CJK}{UTF8}{mj}共\end{CJK} 80 \begin{CJK}{UTF8}{mj}分\end{CJK})

\begin{enumerate}
  \item \begin{CJK}{UTF8}{mj}证明\end{CJK} $Q[\sqrt{2}]=\{a+b \sqrt{2} \mid a, b \in \mathbb{Q}\}$ \begin{CJK}{UTF8}{mj}为数域\end{CJK}, \begin{CJK}{UTF8}{mj}其中\end{CJK} $Q$ \begin{CJK}{UTF8}{mj}为有理数域\end{CJK}.

  \item \begin{CJK}{UTF8}{mj}设\end{CJK} $f_{i}(x)(i=1,2, \cdots, n)$ \begin{CJK}{UTF8}{mj}是数域\end{CJK} $P$ \begin{CJK}{UTF8}{mj}上两两互素的多项式\end{CJK}, $a_{i}(x)(2=1,2, \ldots, n)$ \begin{CJK}{UTF8}{mj}是\end{CJK} \begin{CJK}{UTF8}{mj}数域\end{CJK} $\mathbb{P}$ \begin{CJK}{UTF8}{mj}上任意\end{CJK}. $n$ \begin{CJK}{UTF8}{mj}个多项式\end{CJK}, \begin{CJK}{UTF8}{mj}证明\end{CJK}: \begin{CJK}{UTF8}{mj}存在\end{CJK} $g(x) \in P[x]$, \begin{CJK}{UTF8}{mj}使得对任意的\end{CJK} $i=1,2, \ldots, n$, \begin{CJK}{UTF8}{mj}均有\end{CJK}

\end{enumerate}
$$
g(x) \equiv a_{i}(x)\left(\bmod f_{i}(x)\right) \text { 即 } f_{i}(x) \mid\left\{g(x)-a_{i}(x)\right]
$$

\begin{enumerate}
  \setcounter{enumi}{3}
  \item \begin{CJK}{UTF8}{mj}已知实矩阵\end{CJK}
\end{enumerate}
$$
A=\left(\begin{array}{lll}
2 & 2 & 0 \\
2 & a & 0 \\
0 & 0 & 6
\end{array}\right), \quad B=\left(\begin{array}{ccc}
2 & b & b \\
1 & 4 & -2 \\
1 & -2 & 4
\end{array}\right)
$$
\begin{CJK}{UTF8}{mj}证明\end{CJK}: \begin{CJK}{UTF8}{mj}矩阵方程\end{CJK} $A X=B$ \begin{CJK}{UTF8}{mj}有解但\end{CJK} $B Y=A$ \begin{CJK}{UTF8}{mj}无解的充要条件是\end{CJK} $a \neq 2, b=2$.

\begin{enumerate}
  \setcounter{enumi}{4}
  \item \begin{CJK}{UTF8}{mj}实对称矩阵\end{CJK} $A$ \begin{CJK}{UTF8}{mj}称为负定\end{CJK}, \begin{CJK}{UTF8}{mj}如果二次型\end{CJK} $X^{\prime} A X$ \begin{CJK}{UTF8}{mj}负定\end{CJK}, \begin{CJK}{UTF8}{mj}设实对称矩阵\end{CJK} $A$ \begin{CJK}{UTF8}{mj}的阶数为偶\end{CJK} \begin{CJK}{UTF8}{mj}数\end{CJK}, \begin{CJK}{UTF8}{mj}且满足\end{CJK}
\end{enumerate}
$$
A^{3}+6 A^{2}+11 A+6 E=O
$$
\begin{CJK}{UTF8}{mj}证明\end{CJK}: $A$ \begin{CJK}{UTF8}{mj}的伴随矩阵\end{CJK} $A^{*}$ \begin{CJK}{UTF8}{mj}为负定矩阵\end{CJK}.

\begin{enumerate}
  \setcounter{enumi}{5}
  \item \begin{CJK}{UTF8}{mj}设\end{CJK} $V$ \begin{CJK}{UTF8}{mj}是复数域上的\end{CJK} $n$ \begin{CJK}{UTF8}{mj}维线性空间\end{CJK}, $f, g$ \begin{CJK}{UTF8}{mj}为\end{CJK} $V$ \begin{CJK}{UTF8}{mj}上的线性变换\end{CJK}, \begin{CJK}{UTF8}{mj}且\end{CJK} $f g=g f$, \begin{CJK}{UTF8}{mj}证明\end{CJK} $f, g$ \begin{CJK}{UTF8}{mj}有公共的特征向量\end{CJK}.

  \item \begin{CJK}{UTF8}{mj}设\end{CJK} $A, B$ \begin{CJK}{UTF8}{mj}是\end{CJK} $n$ \begin{CJK}{UTF8}{mj}级实方阵\end{CJK}, \begin{CJK}{UTF8}{mj}且\end{CJK}. $A$ \begin{CJK}{UTF8}{mj}是正定矩阵\end{CJK}, $B$ \begin{CJK}{UTF8}{mj}是实反称矩阵\end{CJK}, \begin{CJK}{UTF8}{mj}证明\end{CJK} $B^{\prime} A B$ \begin{CJK}{UTF8}{mj}的秩\end{CJK} \begin{CJK}{UTF8}{mj}为偶数\end{CJK}. 7. \begin{CJK}{UTF8}{mj}设\end{CJK} $V$ \begin{CJK}{UTF8}{mj}是实数域\end{CJK} $\mathbb{R}$ \begin{CJK}{UTF8}{mj}上的\end{CJK} $n(n>1)$ \begin{CJK}{UTF8}{mj}维线性空间\end{CJK}, $\tau$ \begin{CJK}{UTF8}{mj}是\end{CJK} $V$ \begin{CJK}{UTF8}{mj}上的线性变换\end{CJK}, \begin{CJK}{UTF8}{mj}证明\end{CJK} $V$ \begin{CJK}{UTF8}{mj}有\end{CJK} 1 \begin{CJK}{UTF8}{mj}维或\end{CJK} 2 \begin{CJK}{UTF8}{mj}维\end{CJK} $\tau$-\begin{CJK}{UTF8}{mj}子空间\end{CJK} (\begin{CJK}{UTF8}{mj}即不变子空间\end{CJK}).

  \item \begin{CJK}{UTF8}{mj}设\end{CJK} $V$ \begin{CJK}{UTF8}{mj}是\end{CJK} $n$ \begin{CJK}{UTF8}{mj}维欧氏空间\end{CJK}, $\varphi$ \begin{CJK}{UTF8}{mj}是\end{CJK} $V$ \begin{CJK}{UTF8}{mj}上的线性变换\end{CJK}, \begin{CJK}{UTF8}{mj}证明在\end{CJK} $V$ \begin{CJK}{UTF8}{mj}上存在唯一的线性变换\end{CJK} $\varphi^{*}$, \begin{CJK}{UTF8}{mj}使得对任意的\end{CJK} $\alpha, \beta \in V$, \begin{CJK}{UTF8}{mj}有\end{CJK} $(\varphi(\alpha), \beta)=\left(\alpha, \varphi^{*}(\beta)\right)$.

\end{enumerate}
\begin{CJK}{UTF8}{mj}三\end{CJK}、\begin{CJK}{UTF8}{mj}综合题\end{CJK} (\begin{CJK}{UTF8}{mj}每题\end{CJK} 20 \begin{CJK}{UTF8}{mj}分\end{CJK}, \begin{CJK}{UTF8}{mj}共\end{CJK} 40 \begin{CJK}{UTF8}{mj}分\end{CJK})

\begin{enumerate}
  \item \begin{CJK}{UTF8}{mj}设\end{CJK} $\mathfrak{s l}_{2}(\mathbb{R})$ \begin{CJK}{UTF8}{mj}表示实数域\end{CJK} $\mathbb{R}$ \begin{CJK}{UTF8}{mj}上迹为零的二级矩阵的集合\end{CJK}.
\end{enumerate}
(1) \begin{CJK}{UTF8}{mj}证明\end{CJK} $\mathfrak{s l}_{2}(\mathbb{R})$ \begin{CJK}{UTF8}{mj}是\end{CJK} $\mathbb{R}$ \begin{CJK}{UTF8}{mj}上的线性空间\end{CJK}, \begin{CJK}{UTF8}{mj}并求\end{CJK} $\mathrm{sl}_{2}(\mathbb{R})$ \begin{CJK}{UTF8}{mj}的一组基\end{CJK};

(2) \begin{CJK}{UTF8}{mj}对\end{CJK} $A \in \mathfrak{s l}_{2}(\mathbb{R})$, \begin{CJK}{UTF8}{mj}定义映射\end{CJK}
$$
\begin{aligned}
\tau_{A}: \mathfrak{s l}_{2}(\mathbb{R}) & \rightarrow \mathfrak{s l}_{2}(\mathbb{R}) \\
B & \mapsto A B-B A
\end{aligned}
$$
\begin{CJK}{UTF8}{mj}证明\end{CJK}: $\tau_{A}$ \begin{CJK}{UTF8}{mj}是\end{CJK} $\mathfrak{s l}_{2}(\mathbb{R})$ \begin{CJK}{UTF8}{mj}上的线代变换\end{CJK}.

(3) \begin{CJK}{UTF8}{mj}当\end{CJK} $A=\left(\begin{array}{ll}0 & 1 \\ 0 & 0\end{array}\right)$ \begin{CJK}{UTF8}{mj}时\end{CJK}, \begin{CJK}{UTF8}{mj}求\end{CJK} $\tau_{A}$ \begin{CJK}{UTF8}{mj}的所有特征值\end{CJK}, \begin{CJK}{UTF8}{mj}特征向量及最小多项式\end{CJK}.

\begin{enumerate}
  \setcounter{enumi}{2}
  \item \begin{CJK}{UTF8}{mj}设\end{CJK} $A$ \begin{CJK}{UTF8}{mj}为\end{CJK} $n$ \begin{CJK}{UTF8}{mj}阶可逆矩阵\end{CJK}.
\end{enumerate}
(1) \begin{CJK}{UTF8}{mj}求二次型\end{CJK} $f(X)=\operatorname{det}\left(\begin{array}{cc}0 & -X^{\prime} \\ X & A\end{array}\right)$ \begin{CJK}{UTF8}{mj}的矩阵\end{CJK}, \begin{CJK}{UTF8}{mj}其中\end{CJK} $X=\left(x_{1}, x_{2}, \cdots, x_{n}\right)^{\prime}$;

(2) \begin{CJK}{UTF8}{mj}证明\end{CJK}: \begin{CJK}{UTF8}{mj}当\end{CJK} $A$ \begin{CJK}{UTF8}{mj}是正定矩阵时\end{CJK}, $f(X)$ \begin{CJK}{UTF8}{mj}是正定二次型\end{CJK};

(3) \begin{CJK}{UTF8}{mj}当\end{CJK} $A$ \begin{CJK}{UTF8}{mj}是实对称矩阵时\end{CJK}, \begin{CJK}{UTF8}{mj}讨论\end{CJK} $A$ \begin{CJK}{UTF8}{mj}的正\end{CJK}、\begin{CJK}{UTF8}{mj}负惯性指数与\end{CJK} $f(X)$ \begin{CJK}{UTF8}{mj}的正\end{CJK}、\begin{CJK}{UTF8}{mj}负惯性指数\end{CJK} \begin{CJK}{UTF8}{mj}之间的关系\end{CJK}.

\section{第十五章 2021 年天津大学真题}
\section{$15.1$ 数学分析}
\begin{enumerate}
  \item (8 \begin{CJK}{UTF8}{mj}分\end{CJK}) \begin{CJK}{UTF8}{mj}设\end{CJK} $0<x_{1}<1$,
\end{enumerate}
$$
x_{n+1}=\sin x_{n}(n=1,2, \cdots)
$$
\begin{CJK}{UTF8}{mj}证明\end{CJK}: $\lim _{n \rightarrow \infty} x_{n}$ \begin{CJK}{UTF8}{mj}存在\end{CJK}, \begin{CJK}{UTF8}{mj}并求该极限\end{CJK}.

\begin{enumerate}
  \setcounter{enumi}{2}
  \item (10 \begin{CJK}{UTF8}{mj}分\end{CJK}) \begin{CJK}{UTF8}{mj}计算极限\end{CJK} $\lim _{x \rightarrow 0} \frac{x^{x}-(\sin x)^{x}}{x^{2} \ln (1+x)}$

  \item (12 \begin{CJK}{UTF8}{mj}分\end{CJK}) \begin{CJK}{UTF8}{mj}把函数\end{CJK}

\end{enumerate}
$$
f(x)= \begin{cases}x^{2}, & 0 \leqslant x<\pi \\ 0, & -\pi<x<0\end{cases}
$$
\begin{CJK}{UTF8}{mj}展开成\end{CJK} $[-\pi, \pi]$ \begin{CJK}{UTF8}{mj}上的傅里叶级数\end{CJK}, \begin{CJK}{UTF8}{mj}并求出该傅里叶级数在\end{CJK} $[-\pi, \pi]$ \begin{CJK}{UTF8}{mj}上的和函数\end{CJK}.

\begin{enumerate}
  \setcounter{enumi}{4}
  \item (12 \begin{CJK}{UTF8}{mj}分\end{CJK}) \begin{CJK}{UTF8}{mj}在曲面\end{CJK} $x^{2}+y^{2}+4 z^{2}=4$ \begin{CJK}{UTF8}{mj}上求一点\end{CJK}, \begin{CJK}{UTF8}{mj}使得其到平面\end{CJK} $x+2 y+2 z=10$ \begin{CJK}{UTF8}{mj}的距离最\end{CJK} \begin{CJK}{UTF8}{mj}短\end{CJK}, \begin{CJK}{UTF8}{mj}并求出该最短距离\end{CJK}.

  \item (12 \begin{CJK}{UTF8}{mj}分\end{CJK}) \begin{CJK}{UTF8}{mj}设\end{CJK} $\Sigma$ \begin{CJK}{UTF8}{mj}是单位球面\end{CJK} $x^{2}+y^{2}+z^{2}=1$ \begin{CJK}{UTF8}{mj}限制在\end{CJK}

\end{enumerate}
$$
\{(x, y, z): x \geqslant 0, y \geqslant 0, z \geqslant 0\}
$$
\begin{CJK}{UTF8}{mj}的部分\end{CJK}, \begin{CJK}{UTF8}{mj}求曲面积分\end{CJK}
$$
I=\iint_{\Sigma}\left[x^{2}+\left(y^{2} z^{2}+z^{2} x^{2}+x^{2} y^{2}\right) x y z\right] \mathrm{d} S
$$

\begin{enumerate}
  \setcounter{enumi}{6}
  \item (12 \begin{CJK}{UTF8}{mj}分\end{CJK}) \begin{CJK}{UTF8}{mj}设\end{CJK} $f(x)$ \begin{CJK}{UTF8}{mj}在\end{CJK} $[a, b]$ \begin{CJK}{UTF8}{mj}上连续\end{CJK}, \begin{CJK}{UTF8}{mj}在\end{CJK} $(a, b)$ \begin{CJK}{UTF8}{mj}内二阶可导\end{CJK}, \begin{CJK}{UTF8}{mj}证明\end{CJK}: \begin{CJK}{UTF8}{mj}存在\end{CJK} $\xi \in(a, b)$, \begin{CJK}{UTF8}{mj}使得\end{CJK}
\end{enumerate}
$$
f(b)-2 f\left(\frac{a+b}{2}\right)+f(a)=\frac{(b-a)^{2}}{4} f^{\prime \prime}(\xi)
$$

\begin{enumerate}
  \setcounter{enumi}{7}
  \item (12 \begin{CJK}{UTF8}{mj}分\end{CJK}) \begin{CJK}{UTF8}{mj}设\end{CJK} $0<a<b<+\infty$, \begin{CJK}{UTF8}{mj}函数\end{CJK} $f(x)$ \begin{CJK}{UTF8}{mj}在\end{CJK} $[a, b]$ \begin{CJK}{UTF8}{mj}上非负连续\end{CJK},
\end{enumerate}
$$
M=\sup _{x \in[a, b]} f(x)
$$
\begin{CJK}{UTF8}{mj}证明\end{CJK}:
$$
\lim _{n \rightarrow \infty}\left[\int_{a}^{b}|f(x)|^{n} \mathrm{~d} x\right]^{\frac{1}{n}}=M
$$

\begin{enumerate}
  \setcounter{enumi}{8}
  \item (12 \begin{CJK}{UTF8}{mj}分\end{CJK}) \begin{CJK}{UTF8}{mj}设一元函数\end{CJK} $f(x)$ \begin{CJK}{UTF8}{mj}在\end{CJK} $[0,+\infty)$ \begin{CJK}{UTF8}{mj}上连续可导\end{CJK}, \begin{CJK}{UTF8}{mj}并设\end{CJK}
\end{enumerate}
$$
u\left(x_{1}, x_{2}, \cdots, x_{n}\right)=f\left(x_{1}^{2}+x_{2}^{2}+\cdots+x_{n}^{2}\right)
$$
\begin{CJK}{UTF8}{mj}若存在常数\end{CJK} $c \neq 0$ \begin{CJK}{UTF8}{mj}使得\end{CJK} $\lim _{t \rightarrow \infty} f^{\prime}(t)=c$. \begin{CJK}{UTF8}{mj}证明\end{CJK}: $u$ \begin{CJK}{UTF8}{mj}在\end{CJK} $\mathbb{R}^{n}$ \begin{CJK}{UTF8}{mj}上不一致连续\end{CJK}.

\begin{enumerate}
  \setcounter{enumi}{9}
  \item (15 \begin{CJK}{UTF8}{mj}分\end{CJK}) \begin{CJK}{UTF8}{mj}解答如下问题\end{CJK}:
\end{enumerate}
(1) \begin{CJK}{UTF8}{mj}求幂级数\end{CJK} $\sum_{n=1}^{\infty} \frac{x^{4 n+1}}{n(4 n+1)}$ \begin{CJK}{UTF8}{mj}的收玫域\end{CJK};

(2) \begin{CJK}{UTF8}{mj}计算级数\end{CJK} $\sum_{n=1}^{\infty} \frac{1}{n(4 n+1)}$ \begin{CJK}{UTF8}{mj}的值\end{CJK}.

\begin{enumerate}
  \setcounter{enumi}{10}
  \item (15 \begin{CJK}{UTF8}{mj}分\end{CJK}) \begin{CJK}{UTF8}{mj}解答如下问题\end{CJK}:
\end{enumerate}
(1) \begin{CJK}{UTF8}{mj}证明广义积分\end{CJK}

\begin{CJK}{UTF8}{mj}关于\end{CJK} $r \in[0,+\infty)$ \begin{CJK}{UTF8}{mj}一致收敛\end{CJK};
$$
\int_{0}^{+\infty} \mathrm{e}^{-x^{2}} \sin (r x) \mathrm{d} x
$$
(2) \begin{CJK}{UTF8}{mj}计算极限\end{CJK} $\lim _{r \rightarrow+\infty} r \int_{0}^{+\infty} \mathrm{e}^{-x^{2}} \sin (r x) \mathrm{d} x$

\begin{enumerate}
  \setcounter{enumi}{11}
  \item (30 \begin{CJK}{UTF8}{mj}分\end{CJK}) \begin{CJK}{UTF8}{mj}证明如下结论\end{CJK}.
\end{enumerate}
(1) \begin{CJK}{UTF8}{mj}设\end{CJK} $\Omega \subset \mathbb{R}^{3}$ \begin{CJK}{UTF8}{mj}为边界分段光滑的有界闭区域\end{CJK}, \begin{CJK}{UTF8}{mj}函数\end{CJK} $u, v$ \begin{CJK}{UTF8}{mj}在\end{CJK} $\bar{\Omega}$ \begin{CJK}{UTF8}{mj}上具有二阶连续偏导\end{CJK} \begin{CJK}{UTF8}{mj}数\end{CJK}, \begin{CJK}{UTF8}{mj}证明\end{CJK}:
$$
\iiint_{\Omega}(u \Delta v-v \Delta u) \mathrm{d} x \mathrm{~d} y \mathrm{~d} z=\iint_{\partial \Omega}\left(u \frac{\partial u}{\partial n}-v \frac{\partial u}{\partial n}\right) \mathrm{d} S
$$
\begin{CJK}{UTF8}{mj}其中\end{CJK} $n$ \begin{CJK}{UTF8}{mj}为\end{CJK} $\partial \Omega$ \begin{CJK}{UTF8}{mj}的单位外法向量\end{CJK}
$$
\Delta=\frac{\partial^{2}}{\partial x^{2}}+\frac{\partial^{2}}{\partial y^{2}}+\frac{\partial^{2}}{\partial z^{2}}
$$
(2) \begin{CJK}{UTF8}{mj}设\end{CJK} $\Omega$ \begin{CJK}{UTF8}{mj}及\end{CJK} $u$ \begin{CJK}{UTF8}{mj}满足第\end{CJK} 1 \begin{CJK}{UTF8}{mj}问的条件\end{CJK}, \begin{CJK}{UTF8}{mj}证明\end{CJK}:
$$
\iiint_{\Omega} \Delta u \mathrm{~d} x \mathrm{~d} y \mathrm{~d} z=\iint_{\partial \Omega} \frac{\partial u}{\partial n} \mathrm{~d} S .
$$
(3) \begin{CJK}{UTF8}{mj}证明\end{CJK}: \begin{CJK}{UTF8}{mj}对任意非零的\end{CJK} $x \in \mathbb{R}^{3}$, \begin{CJK}{UTF8}{mj}有\end{CJK}
$$
\Delta\left(|x|^{-1}\right)=0
$$
\begin{CJK}{UTF8}{mj}对任意非零的\end{CJK} $x \in \mathbb{R}^{2}$, \begin{CJK}{UTF8}{mj}有\end{CJK}
$$
\Delta(\ln |x|)=0
$$
(4) \begin{CJK}{UTF8}{mj}记单位球\end{CJK}
$$
B=\left\{(x, y, z): x^{2}+y^{2}+z^{2} \leq 1\right\}
$$
\begin{CJK}{UTF8}{mj}设\end{CJK} $u$ \begin{CJK}{UTF8}{mj}在\end{CJK} $B$ \begin{CJK}{UTF8}{mj}上具有二阶连续偏导数\end{CJK}, \begin{CJK}{UTF8}{mj}且\end{CJK} $\Delta u=0$, \begin{CJK}{UTF8}{mj}证明\end{CJK}:
$$
u(0)=\frac{1}{4 \pi} \iint_{\partial B} u \mathrm{~d} S
$$

\section{$15.2$ 高等代数}
\begin{enumerate}
  \item (15 \begin{CJK}{UTF8}{mj}分\end{CJK}) \begin{CJK}{UTF8}{mj}设\end{CJK} $\alpha, \beta, \gamma$ \begin{CJK}{UTF8}{mj}为多项式\end{CJK} $x^{3}-x+1$ \begin{CJK}{UTF8}{mj}的三个根\end{CJK}, \begin{CJK}{UTF8}{mj}求首\end{CJK} $一$ \begin{CJK}{UTF8}{mj}三次多项式\end{CJK} $f(x)$, \begin{CJK}{UTF8}{mj}使得其三\end{CJK} \begin{CJK}{UTF8}{mj}个根分别为\end{CJK}
\end{enumerate}
$$
1+\alpha^{2}, 1+\beta^{2}, 1+\gamma^{2}
$$

\begin{enumerate}
  \setcounter{enumi}{2}
  \item (10 \begin{CJK}{UTF8}{mj}分\end{CJK}) \begin{CJK}{UTF8}{mj}计算行列式\end{CJK}
\end{enumerate}
$$
D_{n}=\left|\begin{array}{ccccc}
a_{1}+x_{1} & a_{2} & a_{3} & \cdots & a_{n} \\
a_{1} & a_{2}+x_{2} & a_{3} & \cdots & a_{n} \\
a_{1} & a_{2} & a_{3}+x_{3} & \cdots & a_{n} \\
\vdots & \vdots & \vdots & & \vdots \\
a_{1} & a_{2} & a_{3} & \cdots & a_{n}+x_{n}
\end{array}\right|
$$

\begin{enumerate}
  \setcounter{enumi}{3}
  \item (20 \begin{CJK}{UTF8}{mj}分\end{CJK}) \begin{CJK}{UTF8}{mj}设\end{CJK} $A, B$ \begin{CJK}{UTF8}{mj}分别为\end{CJK} $s \times n$ \begin{CJK}{UTF8}{mj}与\end{CJK} $n \times m$ \begin{CJK}{UTF8}{mj}阶矩阵\end{CJK}. \begin{CJK}{UTF8}{mj}解答如下问题\end{CJK}:
\end{enumerate}
(1) \begin{CJK}{UTF8}{mj}证明\end{CJK}: $\operatorname{rank}(A B)=\operatorname{rank}(B)$ \begin{CJK}{UTF8}{mj}当且仅当\end{CJK} $A B X=0$ \begin{CJK}{UTF8}{mj}的解都是\end{CJK} $B X=0$ \begin{CJK}{UTF8}{mj}的解\end{CJK};

(2) \begin{CJK}{UTF8}{mj}设\end{CJK} $C$ \begin{CJK}{UTF8}{mj}为\end{CJK} $m \times r$ \begin{CJK}{UTF8}{mj}矩阵\end{CJK}, \begin{CJK}{UTF8}{mj}证明\end{CJK}: \begin{CJK}{UTF8}{mj}若\end{CJK} $\operatorname{rank}(A B)=\operatorname{rank}(B)$, \begin{CJK}{UTF8}{mj}则\end{CJK} $\operatorname{rank}(A B C)=\operatorname{rank}(B C)$;

(3) \begin{CJK}{UTF8}{mj}设\end{CJK} $D$ \begin{CJK}{UTF8}{mj}为\end{CJK} $n$ \begin{CJK}{UTF8}{mj}阶实方阵\end{CJK}, $D^{\prime}$ \begin{CJK}{UTF8}{mj}表示\end{CJK} $D$ \begin{CJK}{UTF8}{mj}的转置\end{CJK}, \begin{CJK}{UTF8}{mj}证明\end{CJK}
$$
\operatorname{rank}\left(D D^{\prime}\right)=\operatorname{rank}\left(D^{\prime} D\right)=\operatorname{rank}(D) .
$$
\begin{CJK}{UTF8}{mj}并举例说明当\end{CJK} $D$ \begin{CJK}{UTF8}{mj}为\end{CJK} $n$ \begin{CJK}{UTF8}{mj}阶复方阵时\end{CJK}, \begin{CJK}{UTF8}{mj}结论不成立\end{CJK}.

\begin{enumerate}
  \setcounter{enumi}{4}
  \item \begin{CJK}{UTF8}{mj}设三维线性空间\end{CJK} $V$ \begin{CJK}{UTF8}{mj}上的线性变换\end{CJK} $\mathscr{T}$ \begin{CJK}{UTF8}{mj}在基\end{CJK} $\mathrm{e}_{1}, \mathrm{e}_{2}, \mathrm{e}_{3}$ \begin{CJK}{UTF8}{mj}下的矩阵为\end{CJK}
\end{enumerate}
$$
, \quad A=\left(\begin{array}{ccc}
1 & -1 & 0 \\
1 & 2 & 3 \\
-1 & 4 & 5
\end{array}\right)
$$
(1) \begin{CJK}{UTF8}{mj}求\end{CJK} $\mathscr{T}$ \begin{CJK}{UTF8}{mj}在基\end{CJK} $\mathrm{e}_{1}, \mathrm{e}_{2}, \mathrm{e}_{3}$ \begin{CJK}{UTF8}{mj}下的矩阵\end{CJK}.

(2) \begin{CJK}{UTF8}{mj}求\end{CJK} $\mathscr{T}$ \begin{CJK}{UTF8}{mj}在基\end{CJK} $\mathrm{e}_{1}-\mathrm{e}_{2}, 2 \mathrm{e}_{2}, \mathrm{e}_{3}$ \begin{CJK}{UTF8}{mj}下的矩阵\end{CJK}. 5. \begin{CJK}{UTF8}{mj}设\end{CJK} $\alpha_{1}, \alpha_{2}, \cdots, \alpha_{2021}$ \begin{CJK}{UTF8}{mj}为方程组\end{CJK} $A X=0$ \begin{CJK}{UTF8}{mj}的基础解系\end{CJK}, \begin{CJK}{UTF8}{mj}试证\end{CJK}:
$$
\left\{\begin{array}{l}
\beta_{1}=\alpha_{1}+\alpha_{2}+\cdots+\alpha_{2021} \\
\beta_{2}=2 \alpha_{1}+2^{2} \alpha_{2}+\cdots+2^{2021} \alpha_{2021} \\
\cdots \cdots \\
\beta_{2021}=2021 \alpha_{1}+2021^{2} \alpha_{2}+\cdots+2021^{2021} \alpha_{2021}
\end{array}\right.
$$
\begin{CJK}{UTF8}{mj}也是\end{CJK} $A X=0$ \begin{CJK}{UTF8}{mj}的基础解系\end{CJK}.

\begin{enumerate}
  \setcounter{enumi}{6}
  \item (20 \begin{CJK}{UTF8}{mj}分\end{CJK}) \begin{CJK}{UTF8}{mj}设\end{CJK} $A$ \begin{CJK}{UTF8}{mj}为\end{CJK} $n$ \begin{CJK}{UTF8}{mj}阶实方阵\end{CJK}, \begin{CJK}{UTF8}{mj}满足\end{CJK} $A^{2}+4 A+2021 I=O$, \begin{CJK}{UTF8}{mj}其中\end{CJK} $I$ \begin{CJK}{UTF8}{mj}表示\end{CJK} $n$ \begin{CJK}{UTF8}{mj}阶单位矩阵\end{CJK}
\end{enumerate}
(1) \begin{CJK}{UTF8}{mj}证明\end{CJK}: \begin{CJK}{UTF8}{mj}对任意的实数\end{CJK} $a$, \begin{CJK}{UTF8}{mj}则\end{CJK} $A+a I$ \begin{CJK}{UTF8}{mj}可逆\end{CJK};

(2) \begin{CJK}{UTF8}{mj}求\end{CJK} $A+2 I$ \begin{CJK}{UTF8}{mj}的逆和\end{CJK} $(A+2 I)^{2024}$.

\begin{enumerate}
  \setcounter{enumi}{7}
  \item (15 \begin{CJK}{UTF8}{mj}分\end{CJK}) \begin{CJK}{UTF8}{mj}设\end{CJK} $A$ \begin{CJK}{UTF8}{mj}为\end{CJK} $n$ \begin{CJK}{UTF8}{mj}阶实方阵\end{CJK}, \begin{CJK}{UTF8}{mj}并且满足\end{CJK} $A^{2}=A$.
\end{enumerate}
(1) \begin{CJK}{UTF8}{mj}求\end{CJK} $A$ \begin{CJK}{UTF8}{mj}的特征值\end{CJK};

(2) \begin{CJK}{UTF8}{mj}证明\end{CJK}: $A$ \begin{CJK}{UTF8}{mj}与对角矩阵相似\end{CJK}.

\begin{enumerate}
  \setcounter{enumi}{8}
  \item (15 \begin{CJK}{UTF8}{mj}分\end{CJK}) \begin{CJK}{UTF8}{mj}设\end{CJK} $I$ \begin{CJK}{UTF8}{mj}表示\end{CJK} $n$ \begin{CJK}{UTF8}{mj}阶单位矩阵\end{CJK}
\end{enumerate}
(1) \begin{CJK}{UTF8}{mj}求矩阵\end{CJK} $\left(\begin{array}{ll}0 & 1 \\ I & O\end{array}\right)$ \begin{CJK}{UTF8}{mj}对应二次型的正负惯性指数\end{CJK};

(2) \begin{CJK}{UTF8}{mj}设\end{CJK} $A$ \begin{CJK}{UTF8}{mj}为\end{CJK} $n$ \begin{CJK}{UTF8}{mj}阶实可逆矩阵\end{CJK}, \begin{CJK}{UTF8}{mj}求矩阵\end{CJK} $\left(\begin{array}{cc}O & A \\ A^{\prime} & O\end{array}\right)$ \begin{CJK}{UTF8}{mj}对应二次型的正负惯性指数\end{CJK}.

\begin{enumerate}
  \setcounter{enumi}{9}
  \item (10 \begin{CJK}{UTF8}{mj}分\end{CJK}) \begin{CJK}{UTF8}{mj}设\end{CJK} $\alpha_{1}, \alpha_{2}, \cdots, \alpha_{m}$ \begin{CJK}{UTF8}{mj}与\end{CJK} $\beta_{1}, \beta_{2}, \cdots, \beta_{m}$ \begin{CJK}{UTF8}{mj}为欧氏空间\end{CJK} $\mathbb{R}^{n}$ \begin{CJK}{UTF8}{mj}的两组向量\end{CJK}, \begin{CJK}{UTF8}{mj}并且对任意\end{CJK} \begin{CJK}{UTF8}{mj}的\end{CJK} $1 \leq i, j \leq m$, \begin{CJK}{UTF8}{mj}有\end{CJK} $\left(\alpha_{i}, \alpha_{j}\right)=\left(\beta_{i}, \beta_{j}\right)$. \begin{CJK}{UTF8}{mj}证明\end{CJK}
\end{enumerate}
(1) \begin{CJK}{UTF8}{mj}向量组\end{CJK} $\alpha_{1}, \alpha_{2}, \cdots \alpha_{m}$ \begin{CJK}{UTF8}{mj}与\end{CJK} $\beta_{1}, \beta_{2}, \cdots, \beta_{m}$ \begin{CJK}{UTF8}{mj}的秩相同\end{CJK};

(2) \begin{CJK}{UTF8}{mj}存在\end{CJK} $\mathbb{R}^{n}$ \begin{CJK}{UTF8}{mj}上的正交变换\end{CJK} $\mathscr{O}$, \begin{CJK}{UTF8}{mj}使得对任意的\end{CJK} $i=1,2, \cdots, m$, \begin{CJK}{UTF8}{mj}都有\end{CJK} $\mathscr{O}\left(\alpha_{i}\right)=\beta_{i}$.

\begin{enumerate}
  \setcounter{enumi}{10}
  \item (10 \begin{CJK}{UTF8}{mj}分\end{CJK}) \begin{CJK}{UTF8}{mj}设\end{CJK} $V_{1}, V_{2}, \cdots, V_{s}$ \begin{CJK}{UTF8}{mj}是实数域上线性空间\end{CJK} $V$ \begin{CJK}{UTF8}{mj}的\end{CJK} $s$ \begin{CJK}{UTF8}{mj}个真子空间\end{CJK}, \begin{CJK}{UTF8}{mj}证明\end{CJK}: $V$ \begin{CJK}{UTF8}{mj}中至少有\end{CJK} \begin{CJK}{UTF8}{mj}一个向量\end{CJK} $v$ \begin{CJK}{UTF8}{mj}不属于\end{CJK} $V_{1}, V_{2}, \cdots, V_{s}$ \begin{CJK}{UTF8}{mj}中的任何一个\end{CJK}.
\end{enumerate}
\section{第十六章 2021 年南京大学真题}
\section{$16.1$ 数学分析}
\begin{enumerate}
  \item \begin{CJK}{UTF8}{mj}计算题\end{CJK} (\begin{CJK}{UTF8}{mj}每题\end{CJK} 10 \begin{CJK}{UTF8}{mj}分\end{CJK}, \begin{CJK}{UTF8}{mj}共\end{CJK} 40 \begin{CJK}{UTF8}{mj}分\end{CJK})
\end{enumerate}
(1) \begin{CJK}{UTF8}{mj}求极限\end{CJK} $\lim _{n \rightarrow \infty} n^{2}\left[\sqrt[n]{5}-1-\ln \left(1+\frac{\ln 5}{n}\right)\right]$

(2) \begin{CJK}{UTF8}{mj}求无穷积分\end{CJK} $\int_{1}^{+\infty} \frac{\mathrm{d} x}{x \sqrt{x^{6}+x^{3}+1}}$;

(3) \begin{CJK}{UTF8}{mj}求累次积分\end{CJK} $\int_{0}^{1} \mathrm{~d} y \int_{0}^{y} \mathrm{~d} x \int_{0}^{x} \frac{\mathrm{e}^{z}}{1-z} \mathrm{~d} z$.

(4) \begin{CJK}{UTF8}{mj}求三重积分\end{CJK} $\iiint_{x^{2}+y^{2} \leq z \leq 1} \sqrt{x^{2}+y^{2}} \mathrm{~d} x \mathrm{~d} y \mathrm{~d} z$.

\begin{enumerate}
  \setcounter{enumi}{2}
  \item (10 \begin{CJK}{UTF8}{mj}分\end{CJK}) \begin{CJK}{UTF8}{mj}已知\end{CJK} $f(x)=\left\{\begin{array}{cc}\frac{\sin x}{x}, & x \neq 0 ; \\ 1, & x=0\end{array} \mid\right.$, \begin{CJK}{UTF8}{mj}证明\end{CJK} $f(x)$ \begin{CJK}{UTF8}{mj}在\end{CJK} $(-\infty,+\infty)$ \begin{CJK}{UTF8}{mj}上任意阶可导\end{CJK}.

  \item (10 \begin{CJK}{UTF8}{mj}分\end{CJK}) \begin{CJK}{UTF8}{mj}计算第二型曲线积分\end{CJK}

\end{enumerate}
$$
I=\oint_{L}(y-z) \mathrm{d} x+(z-x) \mathrm{d} y+(y-x) \mathrm{d} z
$$
\begin{CJK}{UTF8}{mj}其中\end{CJK} $L$ \begin{CJK}{UTF8}{mj}为\end{CJK} $x^{2}+y^{2}+z^{2}=2 a z$ \begin{CJK}{UTF8}{mj}与\end{CJK} $x+z=a$ \begin{CJK}{UTF8}{mj}的交线\end{CJK}, \begin{CJK}{UTF8}{mj}其中\end{CJK} $a>0$, \begin{CJK}{UTF8}{mj}从\end{CJK} $z$ \begin{CJK}{UTF8}{mj}轴\end{CJK} $+\infty$ \begin{CJK}{UTF8}{mj}看\end{CJK}, \begin{CJK}{UTF8}{mj}方向\end{CJK} \begin{CJK}{UTF8}{mj}为逆时针\end{CJK}.

\begin{enumerate}
  \setcounter{enumi}{4}
  \item (15 \begin{CJK}{UTF8}{mj}分\end{CJK}) \begin{CJK}{UTF8}{mj}计算第二型曲面积分\end{CJK}
\end{enumerate}
$$
I=\iint_{S} \frac{x}{r^{3}} \mathrm{~d} y \mathrm{~d} z+\frac{y}{r^{3}} \mathrm{~d} z \mathrm{~d} x+\frac{z}{r^{3}} \mathrm{~d} x \mathrm{~d} y
$$
\begin{CJK}{UTF8}{mj}其中\end{CJK} $r=\sqrt{x^{2}+y^{2}+z^{2}}, S$ \begin{CJK}{UTF8}{mj}为椭球面\end{CJK} $x^{2}+2 y^{2}+3 z^{2}=1$, \begin{CJK}{UTF8}{mj}取外侧\end{CJK}.

\begin{enumerate}
  \setcounter{enumi}{5}
  \item (15 \begin{CJK}{UTF8}{mj}分\end{CJK}) \begin{CJK}{UTF8}{mj}已知\end{CJK} $I(a)=\int_{0}^{+\infty} \frac{\sin x}{x} \mathrm{~d}^{-a x} \mathrm{~d} x$.
\end{enumerate}
(1) \begin{CJK}{UTF8}{mj}证明\end{CJK} $I(a)$ \begin{CJK}{UTF8}{mj}在\end{CJK} $[0,+\infty)$ \begin{CJK}{UTF8}{mj}上连续\end{CJK}: (2) \begin{CJK}{UTF8}{mj}求\end{CJK} $I(a)$.

\begin{enumerate}
  \setcounter{enumi}{6}
  \item (15 \begin{CJK}{UTF8}{mj}分\end{CJK}) \begin{CJK}{UTF8}{mj}已知函数项级数\end{CJK} $\sum_{n=1}^{\infty} \frac{a_{n}}{n^{x}}$ \begin{CJK}{UTF8}{mj}在\end{CJK} $x_{0}$ \begin{CJK}{UTF8}{mj}处收玫\end{CJK}, \begin{CJK}{UTF8}{mj}证明\end{CJK} $\sum_{n=1}^{\infty} \frac{a_{n}}{n^{x}}$ \begin{CJK}{UTF8}{mj}在\end{CJK} $\left(x_{0},+\infty\right)$ \begin{CJK}{UTF8}{mj}上一致收敘\end{CJK}, \begin{CJK}{UTF8}{mj}且在\end{CJK} $\left(x_{0}+1,+\infty\right)$ \begin{CJK}{UTF8}{mj}上连续可导\end{CJK}.

  \item (15 \begin{CJK}{UTF8}{mj}分\end{CJK}) \begin{CJK}{UTF8}{mj}设连续的函数列\end{CJK} $\left\{f_{n}(x)\right\}$ \begin{CJK}{UTF8}{mj}在\end{CJK} $[a, b]$ \begin{CJK}{UTF8}{mj}上收玫于连给函数\end{CJK} $f(x)$, \begin{CJK}{UTF8}{mj}若对每个固定的\end{CJK} $x \in[a, b]$, \begin{CJK}{UTF8}{mj}数列\end{CJK} $\left\{f_{n}(x)\right\}$ \begin{CJK}{UTF8}{mj}都关于\end{CJK} $n$ \begin{CJK}{UTF8}{mj}单调递减\end{CJK}, \begin{CJK}{UTF8}{mj}证明\end{CJK} $f_{n}(x)$ \begin{CJK}{UTF8}{mj}在\end{CJK} $[a, b]$ \begin{CJK}{UTF8}{mj}上致收玫于\end{CJK} $f(x)$.

  \item (15 \begin{CJK}{UTF8}{mj}分\end{CJK}) \begin{CJK}{UTF8}{mj}已知\end{CJK}

\end{enumerate}
$$
f(x, y)= \begin{cases}\sqrt{m n}, & (x, y) \in\left(\frac{1}{n+1}, \frac{1}{n}\right] \times\left(\frac{1}{m+1}, \frac{1}{m}\right] \\ 0, & \text { 其他. }\end{cases}
$$
\begin{CJK}{UTF8}{mj}其中\end{CJK} $m, n$ \begin{CJK}{UTF8}{mj}取遍所有正整数\end{CJK}, \begin{CJK}{UTF8}{mj}证明\end{CJK} $f(x, y)$ \begin{CJK}{UTF8}{mj}在\end{CJK} $[0,1] \times[0,1]$ \begin{CJK}{UTF8}{mj}上广义黎曼可积\end{CJK}.

\begin{enumerate}
  \setcounter{enumi}{9}
  \item (15 \begin{CJK}{UTF8}{mj}分\end{CJK}) \begin{CJK}{UTF8}{mj}已知\end{CJK} $f(x)$ \begin{CJK}{UTF8}{mj}在\end{CJK} $\mathbb{R}^{n}$ \begin{CJK}{UTF8}{mj}上任意阶可导\end{CJK}, \begin{CJK}{UTF8}{mj}其中\end{CJK} $x=\left(x_{1}, x_{2}, \cdots, x_{n}\right)$, \begin{CJK}{UTF8}{mj}且黑塞矩阵\end{CJK} $\left(\frac{\partial^{2} f}{\partial x_{i} \partial x_{j}}\right)$ \begin{CJK}{UTF8}{mj}处处为正定矩阵\end{CJK}.
\end{enumerate}
(1) \begin{CJK}{UTF8}{mj}证明\end{CJK} $\Phi(\boldsymbol{x})=\left(\frac{\partial f}{\partial x_{1}}, \frac{\partial f}{\partial x_{2}}, \cdots, \frac{\partial f}{\partial x_{n}}\right)$ \begin{CJK}{UTF8}{mj}为\end{CJK} $\mathbb{R}^{n}$ \begin{CJK}{UTF8}{mj}到\end{CJK} $\mathbb{R}^{n}$ \begin{CJK}{UTF8}{mj}的双射\end{CJK};

(2) \begin{CJK}{UTF8}{mj}记\end{CJK} $\Psi(\boldsymbol{y})$ \begin{CJK}{UTF8}{mj}为\end{CJK} $\Phi(\boldsymbol{x})$ \begin{CJK}{UTF8}{mj}的反函数\end{CJK}, \begin{CJK}{UTF8}{mj}证明\end{CJK} $u(\boldsymbol{y})=\langle\boldsymbol{y}, \Psi(\boldsymbol{y})\rangle-f(\Psi(\boldsymbol{y}))$ \begin{CJK}{UTF8}{mj}为凸函数\end{CJK}.

\section{$16.2$ 高等代数}
\begin{enumerate}
  \item \begin{CJK}{UTF8}{mj}已知\end{CJK} $f(x)$ \begin{CJK}{UTF8}{mj}是数域\end{CJK} $\mathbb{P}$ \begin{CJK}{UTF8}{mj}上的七次多项式\end{CJK}, \begin{CJK}{UTF8}{mj}且\end{CJK} $(x-1)^{4}\left|f(x)+1,(x+1)^{4}\right| f(x)-1$, \begin{CJK}{UTF8}{mj}求满足\end{CJK} \begin{CJK}{UTF8}{mj}条件的所有\end{CJK} $f(x)$.

  \item \begin{CJK}{UTF8}{mj}已知\end{CJK} $F=\{a+b \sqrt{2}+c \sqrt{3}+d \sqrt{6} \mid a, b, c, d \in \mathbb{Q}\}$.

\end{enumerate}
$$
\left\{\begin{array}{l}
\varphi(\alpha+\beta)=\varphi(\alpha)+\varphi(\beta) \\
\varphi(\alpha \beta)=\varphi(\alpha) \varphi(\beta)
\end{array}\right.
$$
\begin{CJK}{UTF8}{mj}求所有满足条件的\end{CJK} $\varphi$.

\begin{enumerate}
  \setcounter{enumi}{3}
  \item \begin{CJK}{UTF8}{mj}已知\end{CJK}
\end{enumerate}
$$
A=\left(\begin{array}{llll}
x_{1}-1 & x_{2}-1 & x_{3}-1 & x_{4}-1 \\
x_{1}^{2}-1 & x_{2}^{2}-1 & x_{3}^{2}-1 & x_{4}^{2}-1 \\
x_{1}^{3}-1 & x_{2}^{3}-1 & x_{3}^{3}-1 & x_{4}^{3}-1 \\
x_{1}^{4}-1 & x_{2}^{4}-1 & x_{3}^{4}-1 & x_{4}^{4}-1
\end{array}\right)
$$
\begin{CJK}{UTF8}{mj}求\end{CJK} $|A|$ \begin{CJK}{UTF8}{mj}及\end{CJK} $|A|$ \begin{CJK}{UTF8}{mj}的所有代数余子式\end{CJK}. 4. \begin{CJK}{UTF8}{mj}在复数域上解答如下问题\end{CJK}

(1) \begin{CJK}{UTF8}{mj}证明\end{CJK}: $n$ \begin{CJK}{UTF8}{mj}阶矩阵\end{CJK}
$$
A=\left(\begin{array}{llllll}
0 & 1 & & & \\
& 0 & 1 & & \\
& & \ddots & \ddots & \\
& & & 0 & 1 & \\
1 & & & & & 0
\end{array}\right)
$$
\begin{CJK}{UTF8}{mj}可对角化\end{CJK}.

(2) \begin{CJK}{UTF8}{mj}证明\end{CJK}: \begin{CJK}{UTF8}{mj}存在可逆矩阵\end{CJK} $P$ \begin{CJK}{UTF8}{mj}使得对任意的\end{CJK} $a_{i} \in \mathbb{C}(i=0,1, \cdots, n-1)$, \begin{CJK}{UTF8}{mj}都有\end{CJK} $P^{-1} B P$ \begin{CJK}{UTF8}{mj}为对角矩阵\end{CJK}, \begin{CJK}{UTF8}{mj}其中\end{CJK}
$$
B=\left(\begin{array}{ccccc}
a_{0} & a_{1} & a_{2} & \cdots & a_{n-1} \\
a_{n-1} & a_{0} & a_{1} & \ddots & \vdots \\
a_{n-2} & a_{n-1} & a_{0} & \ddots & a_{2} \\
\vdots & \ddots & \ddots & \ddots & a_{1} \\
a_{1} & \cdots & a_{n-2} & a_{n-1} & a_{0}
\end{array}\right),
$$

\begin{enumerate}
  \setcounter{enumi}{5}
  \item \begin{CJK}{UTF8}{mj}已知\end{CJK} $A, B$ \begin{CJK}{UTF8}{mj}为\end{CJK} 3 \begin{CJK}{UTF8}{mj}阶复矩阵\end{CJK}, \begin{CJK}{UTF8}{mj}且\end{CJK} 2 \begin{CJK}{UTF8}{mj}是\end{CJK} $A B$ \begin{CJK}{UTF8}{mj}的特征值\end{CJK}, $\alpha_{1}=(1,2,3)^{\prime}, \alpha_{2}=(0,1,-1)^{\prime}$ \begin{CJK}{UTF8}{mj}为对\end{CJK} \begin{CJK}{UTF8}{mj}应特征向量\end{CJK}, \begin{CJK}{UTF8}{mj}若\end{CJK} $B=\left(\begin{array}{lll}1 & 0 & 1 \\ 1 & 2 & 1 \\ 2 & 2 & 2\end{array}\right)$, \begin{CJK}{UTF8}{mj}证明\end{CJK}: 2 \begin{CJK}{UTF8}{mj}也是\end{CJK} $B A$ \begin{CJK}{UTF8}{mj}的特征值\end{CJK}, \begin{CJK}{UTF8}{mj}并求特征值\end{CJK} 2 \begin{CJK}{UTF8}{mj}对应\end{CJK} \begin{CJK}{UTF8}{mj}的特征值空间的一组基\end{CJK}.

  \item \begin{CJK}{UTF8}{mj}已知\end{CJK} $V$ \begin{CJK}{UTF8}{mj}为\end{CJK} $n$ \begin{CJK}{UTF8}{mj}维线性空间\end{CJK}, $\mathscr{A} \in \operatorname{End}(V)$, \begin{CJK}{UTF8}{mj}其中\end{CJK} $\operatorname{End}(V)$ \begin{CJK}{UTF8}{mj}表示\end{CJK} $V$ \begin{CJK}{UTF8}{mj}上所有线佐变换构成\end{CJK} \begin{CJK}{UTF8}{mj}的线代空间\end{CJK}. \begin{CJK}{UTF8}{mj}记\end{CJK}

\end{enumerate}
$$
K(\mathscr{A})=\{\mathscr{B} \in \operatorname{End}(V) \mid \mathscr{A} \mathscr{B}=\mathscr{O}\} .
$$
(1) \begin{CJK}{UTF8}{mj}证明\end{CJK}: $K(\mathscr{A})$ \begin{CJK}{UTF8}{mj}是\end{CJK} $\operatorname{End}(V)$ \begin{CJK}{UTF8}{mj}的一个线性子空间\end{CJK};

(2) \begin{CJK}{UTF8}{mj}试证\end{CJK}: \begin{CJK}{UTF8}{mj}是否存在\end{CJK} $\mathscr{A}$. \begin{CJK}{UTF8}{mj}满足\end{CJK} $\operatorname{dim} K(\mathscr{A})=n$.

\begin{enumerate}
  \setcounter{enumi}{7}
  \item \begin{CJK}{UTF8}{mj}在\end{CJK} $\mathbb{R}[x]_{n}$ \begin{CJK}{UTF8}{mj}上定义二元运算\end{CJK} $(f(x), g(x))=\int_{0}^{1} f(x) g(x) \mathrm{d} x$.
\end{enumerate}
(1) \begin{CJK}{UTF8}{mj}证明\end{CJK}: \begin{CJK}{UTF8}{mj}如上定义的运算\end{CJK} (,) \begin{CJK}{UTF8}{mj}为\end{CJK} $\mathbb{R}[x]_{n}$ \begin{CJK}{UTF8}{mj}上的内积\end{CJK};

(2) \begin{CJK}{UTF8}{mj}求基\end{CJK} $1, x, x^{2}, \cdots, x^{n-1}$ \begin{CJK}{UTF8}{mj}的度量矩阵\end{CJK};

(3) \begin{CJK}{UTF8}{mj}设\end{CJK} $\mathscr{D}$ \begin{CJK}{UTF8}{mj}为\end{CJK} $\mathbb{R}[x]_{n}$ \begin{CJK}{UTF8}{mj}上的微分变换\end{CJK}, \begin{CJK}{UTF8}{mj}即\end{CJK}
$$
\mathscr{D}(f(x))=f^{\prime}(x), f(x) \in \mathbb{R}[x]_{n}
$$
$\mathscr{D}^{*}$ \begin{CJK}{UTF8}{mj}为\end{CJK} $\mathscr{D}$ \begin{CJK}{UTF8}{mj}的共轭变换\end{CJK}, \begin{CJK}{UTF8}{mj}求\end{CJK} $\left(\operatorname{Ker} \mathscr{D}^{*}\right)^{\perp}$. 8. \begin{CJK}{UTF8}{mj}已知\end{CJK} $A, B$ \begin{CJK}{UTF8}{mj}为\end{CJK} $n$ \begin{CJK}{UTF8}{mj}阶复矩阵\end{CJK}, \begin{CJK}{UTF8}{mj}证明\end{CJK}
$$
r(A-A B A)=r(A)+r\left(I_{n}-B A\right)-n
$$

\begin{enumerate}
  \setcounter{enumi}{9}
  \item \begin{CJK}{UTF8}{mj}已知\end{CJK} $A, C$ \begin{CJK}{UTF8}{mj}为正定矩阵\end{CJK}, \begin{CJK}{UTF8}{mj}证明\end{CJK}: \begin{CJK}{UTF8}{mj}矩阵方程\end{CJK} $A X+X A=C$ \begin{CJK}{UTF8}{mj}存在唯一解\end{CJK} $B$, \begin{CJK}{UTF8}{mj}且\end{CJK} $B$ \begin{CJK}{UTF8}{mj}也为正\end{CJK} \begin{CJK}{UTF8}{mj}定矩阵\end{CJK}.
\end{enumerate}
\section{第十七章 2021 年吉林大学真题}
\section{$17.1$ 数学分析}
\begin{CJK}{UTF8}{mj}一\end{CJK}、\begin{CJK}{UTF8}{mj}计算题\end{CJK}

\begin{enumerate}
  \item \begin{CJK}{UTF8}{mj}设\end{CJK} $f(x)=2021 x^{2021}+x+1, f^{-1}(x)$ \begin{CJK}{UTF8}{mj}是\end{CJK} $f(x)$ \begin{CJK}{UTF8}{mj}的反函数\end{CJK}, \begin{CJK}{UTF8}{mj}求极限\end{CJK}
\end{enumerate}
$$
\lim _{x \rightarrow+\infty} \frac{f^{-1}(2021 x)-f^{-1}(x)}{\sqrt[2021]{x}}
$$

\begin{enumerate}
  \setcounter{enumi}{2}
  \item \begin{CJK}{UTF8}{mj}求极限\end{CJK}
\end{enumerate}
$$
\lim _{n \rightarrow \infty} \frac{1}{n}\left[\sqrt[3]{1^{3}+1^{2}}+\sqrt[3]{2^{3}+2^{2}}+\cdots+\sqrt[3]{n^{3}+n^{2}}-\frac{n(n+1)}{2}\right]
$$

\begin{enumerate}
  \setcounter{enumi}{3}
  \item \begin{CJK}{UTF8}{mj}求极限\end{CJK}
\end{enumerate}
$$
\lim _{x \rightarrow 0} \frac{x^{2}-\int_{0}^{x^{2}} \cos \left(t^{2}\right) \mathrm{d} t}{\sin ^{10} x}
$$

\begin{enumerate}
  \setcounter{enumi}{4}
  \item \begin{CJK}{UTF8}{mj}求极限\end{CJK}
\end{enumerate}
$$
\lim _{n \rightarrow \infty}\left[\frac{1}{\sqrt{n^{2}+1}}+\frac{1}{\sqrt{n^{2}+2}}+\cdots+\frac{1}{\sqrt{n^{2}+n}}\right]^{n}
$$

\begin{enumerate}
  \setcounter{enumi}{5}
  \item \begin{CJK}{UTF8}{mj}求不定积分\end{CJK}
\end{enumerate}
$$
\int \frac{\mathrm{d} x}{2+\tan x}
$$
6 . \begin{CJK}{UTF8}{mj}求定积分\end{CJK}
$$
\int_{0}^{3} \arcsin \frac{x}{x+1} \mathrm{~d} x
$$

\begin{enumerate}
  \setcounter{enumi}{7}
  \item \begin{CJK}{UTF8}{mj}求第一型曲面积分\end{CJK} $I=\iint_{\Gamma} z \mathrm{~d} S$, \begin{CJK}{UTF8}{mj}其中\end{CJK} $\Gamma$ \begin{CJK}{UTF8}{mj}为雉面\end{CJK} $z=\sqrt{x^{2}+y^{2}}$ \begin{CJK}{UTF8}{mj}位于\end{CJK} $0 \leq z \leq h$ \begin{CJK}{UTF8}{mj}的部分\end{CJK}.

  \item \begin{CJK}{UTF8}{mj}求曲线积分\end{CJK} $I=\int_{L} \frac{x y-y \mathrm{~d} x}{x^{2}+y^{2}}$, \begin{CJK}{UTF8}{mj}其中\end{CJK} $L:(x-1)^{2}+y^{2}=1$, \begin{CJK}{UTF8}{mj}取逆时针方向\end{CJK}. 9. \begin{CJK}{UTF8}{mj}求级数\end{CJK} $\sum_{n=1}^{\infty} \frac{1}{2^{n}(2 n-1)}$ \begin{CJK}{UTF8}{mj}的和\end{CJK}. \begin{CJK}{UTF8}{mj}二\end{CJK}、 \begin{CJK}{UTF8}{mj}证明题\end{CJK}

  \item \begin{CJK}{UTF8}{mj}用\end{CJK} $\varepsilon-\delta$ \begin{CJK}{UTF8}{mj}语言证明\end{CJK} $\lim _{x \rightarrow 1} \sin x^{2}=\sin 1$.

  \item \begin{CJK}{UTF8}{mj}证明一个多元微分恒等式\end{CJK}, \begin{CJK}{UTF8}{mj}具体数值忘了\end{CJK}.

  \item \begin{CJK}{UTF8}{mj}设曲线\end{CJK} $S:\left\{\begin{array}{l}x^{2}+y^{2}-x y-z^{2}=1 ; \\ x^{2}+y^{2}=1\end{array}\right.$, \begin{CJK}{UTF8}{mj}求\end{CJK} $(0,0,0)$ \begin{CJK}{UTF8}{mj}到\end{CJK} $S$ \begin{CJK}{UTF8}{mj}的是短距离\end{CJK}.

\end{enumerate}
\begin{CJK}{UTF8}{mj}三\end{CJK}、\begin{CJK}{UTF8}{mj}证明\end{CJK} $\int_{0}^{+\infty} \frac{\sin x y}{x} \mathrm{~d} x$ \begin{CJK}{UTF8}{mj}一致收敛\end{CJK}, \begin{CJK}{UTF8}{mj}但在\end{CJK} $y \in(0,+\infty)$ \begin{CJK}{UTF8}{mj}上不致收玫\end{CJK}, \begin{CJK}{UTF8}{mj}其中\end{CJK} $a>0$.

\begin{CJK}{UTF8}{mj}四\end{CJK}、\begin{CJK}{UTF8}{mj}已知函数\end{CJK} $f(x), g(x)$ \begin{CJK}{UTF8}{mj}在\end{CJK} $[a, b]$ \begin{CJK}{UTF8}{mj}上连续\end{CJK}, \begin{CJK}{UTF8}{mj}且对任意的\end{CJK} $x \in \widehat{a}, b], f(x)-g(x) \neq 0, g(x) \neq 0$, \begin{CJK}{UTF8}{mj}已知\end{CJK} $x_{0} \in(a, b)$ \begin{CJK}{UTF8}{mj}是\end{CJK} $\frac{f(x)+g(x)}{f(x)-g(x)}$ \begin{CJK}{UTF8}{mj}的极小值点\end{CJK}, \begin{CJK}{UTF8}{mj}证明\end{CJK} $x_{0}$ \begin{CJK}{UTF8}{mj}是\end{CJK} $\frac{f(x)}{g(x)}$ \begin{CJK}{UTF8}{mj}的极大值点\end{CJK}.

\begin{CJK}{UTF8}{mj}五\end{CJK}、\begin{CJK}{UTF8}{mj}设\end{CJK} $f(x)$ \begin{CJK}{UTF8}{mj}在\end{CJK} $[a, b]$ \begin{CJK}{UTF8}{mj}上连续\end{CJK}, \begin{CJK}{UTF8}{mj}在\end{CJK} $(a, b)$ \begin{CJK}{UTF8}{mj}上可导\end{CJK}, \begin{CJK}{UTF8}{mj}且\end{CJK} $f(a)=f(b),\left|f^{\prime}(x)\right| \leq 1$, \begin{CJK}{UTF8}{mj}证明\end{CJK}: \begin{CJK}{UTF8}{mj}对任意\end{CJK} $x_{1}, x_{2} \in[a, b]$, \begin{CJK}{UTF8}{mj}有\end{CJK}
$$
\left|f\left(x_{1}\right)-f\left(x_{2}\right)\right| \leq \frac{b-a}{2}
$$

\section{$17.2$ 高等代数}
\begin{enumerate}
  \item (20 \begin{CJK}{UTF8}{mj}分\end{CJK}) \begin{CJK}{UTF8}{mj}已知\end{CJK} $f(x)=\left(x^{2}-a\right)^{2021}+1$, \begin{CJK}{UTF8}{mj}其中\end{CJK} $a$ \begin{CJK}{UTF8}{mj}为实数\end{CJK}. \begin{CJK}{UTF8}{mj}证明\end{CJK} $f(x)$ \begin{CJK}{UTF8}{mj}有重因式的充要条件\end{CJK} \begin{CJK}{UTF8}{mj}是\end{CJK} $a=1$.

  \item (20 \begin{CJK}{UTF8}{mj}分\end{CJK}) \begin{CJK}{UTF8}{mj}已知\end{CJK} $A$ \begin{CJK}{UTF8}{mj}为\end{CJK} $n(n \geq 2)$ \begin{CJK}{UTF8}{mj}阶矩阵\end{CJK}, $\widetilde{A}=2 A$, \begin{CJK}{UTF8}{mj}其中\end{CJK} $\widetilde{A}$ \begin{CJK}{UTF8}{mj}表示\end{CJK} $A$ \begin{CJK}{UTF8}{mj}的伴随矩阵\end{CJK}, \begin{CJK}{UTF8}{mj}且\end{CJK} $|A|=0$, \begin{CJK}{UTF8}{mj}证明\end{CJK} $A=O$.

  \item (20 \begin{CJK}{UTF8}{mj}分\end{CJK}) \begin{CJK}{UTF8}{mj}已知\end{CJK} $A$ \begin{CJK}{UTF8}{mj}为\end{CJK} 3 \begin{CJK}{UTF8}{mj}阶复矩阵\end{CJK}, \begin{CJK}{UTF8}{mj}且\end{CJK} $(A-I)^{2} \neq O,(A-I)^{3}=O$, \begin{CJK}{UTF8}{mj}其中\end{CJK} $I$ \begin{CJK}{UTF8}{mj}为单位矩阵\end{CJK}.

\end{enumerate}
(1) \begin{CJK}{UTF8}{mj}求\end{CJK} $A$ \begin{CJK}{UTF8}{mj}的极小多项式和\end{CJK} Jordan \begin{CJK}{UTF8}{mj}标准型\end{CJK};

(2) \begin{CJK}{UTF8}{mj}记\end{CJK} $V$ \begin{CJK}{UTF8}{mj}是复数域上\end{CJK} 3 \begin{CJK}{UTF8}{mj}阶矩阵构成的线性空间\end{CJK}, \begin{CJK}{UTF8}{mj}证明\end{CJK}
$$
S=\{B \in V \mid A B=B A\}
$$
\begin{CJK}{UTF8}{mj}是\end{CJK} $V$ \begin{CJK}{UTF8}{mj}的子空间\end{CJK}, \begin{CJK}{UTF8}{mj}并求\end{CJK} $S$ \begin{CJK}{UTF8}{mj}的维数\end{CJK}.

\begin{enumerate}
  \setcounter{enumi}{4}
  \item (20 \begin{CJK}{UTF8}{mj}分\end{CJK}) \begin{CJK}{UTF8}{mj}设\end{CJK} $A, B$ \begin{CJK}{UTF8}{mj}均为\end{CJK} $n$ \begin{CJK}{UTF8}{mj}阶正定矩阵\end{CJK}, \begin{CJK}{UTF8}{mj}且\end{CJK} $A$ \begin{CJK}{UTF8}{mj}的每个特征向量均为\end{CJK} $B$ \begin{CJK}{UTF8}{mj}的特征向量\end{CJK}, \begin{CJK}{UTF8}{mj}证明\end{CJK} $A B$ \begin{CJK}{UTF8}{mj}也为正定矩阵\end{CJK}. \begin{CJK}{UTF8}{mj}当且仅当对\end{CJK} $V$ \begin{CJK}{UTF8}{mj}的任意子空间\end{CJK} $S$, \begin{CJK}{UTF8}{mj}均有\end{CJK} $\sigma\left(S^{\perp}\right) \subseteq \sigma(S)^{\perp}$, \begin{CJK}{UTF8}{mj}其中\end{CJK} $\sigma^{\prime}$ \begin{CJK}{UTF8}{mj}表示\end{CJK} $\sigma$ \begin{CJK}{UTF8}{mj}的伴随变换\end{CJK}, $a^{*}$ \begin{CJK}{UTF8}{mj}表示数乘变换\end{CJK}, $S^{\perp}$ \begin{CJK}{UTF8}{mj}表示\end{CJK} $S$ \begin{CJK}{UTF8}{mj}的正交补\end{CJK}. 6. (15 \begin{CJK}{UTF8}{mj}分\end{CJK}) \begin{CJK}{UTF8}{mj}设\end{CJK} $a, b, c$ \begin{CJK}{UTF8}{mj}为空问中的三个向量\end{CJK}, \begin{CJK}{UTF8}{mj}证明\end{CJK}
\end{enumerate}
$$
(a \times b) \times c=(\hat{a} \cdot c) b-(b \cdot c) a
$$

\begin{enumerate}
  \setcounter{enumi}{7}
  \item (15 \begin{CJK}{UTF8}{mj}分\end{CJK}) \begin{CJK}{UTF8}{mj}求过直线\end{CJK}
\end{enumerate}
$$
l:\left\{\begin{array}{l}
4 x-y+3 z-5=0 \\
x-y-z+z=0
\end{array}\right.
$$
\begin{CJK}{UTF8}{mj}且平行于\end{CJK} $z$ \begin{CJK}{UTF8}{mj}轴的平面方程\end{CJK}.

\begin{enumerate}
  \setcounter{enumi}{8}
  \item (20 \begin{CJK}{UTF8}{mj}分\end{CJK}) \begin{CJK}{UTF8}{mj}证明单叶双曲面\end{CJK} $\frac{x^{2}}{a^{2}}+\frac{y^{2}}{b^{2}}-\frac{z^{2}}{c^{2}}=1$ \begin{CJK}{UTF8}{mj}的异族直母线共面\end{CJK}.
\end{enumerate}
\section{第十八章 2021 年复旦学真题}
\section{$18.1$ 数学分析}
\begin{enumerate}
  \item \begin{CJK}{UTF8}{mj}计算题\end{CJK}
\end{enumerate}
(1) $\lim _{x \rightarrow 0}\left[\frac{\tan x}{\ln (1+x)}\right]^{\frac{1}{x}}=$

(2) \begin{CJK}{UTF8}{mj}已知\end{CJK} $f(x) \in C[0,1]$ \begin{CJK}{UTF8}{mj}满足\end{CJK}
$$
f(0)=1, f(1)=2, \int_{0}^{1} f(x) \mathrm{d} x=9
$$
\begin{CJK}{UTF8}{mj}则\end{CJK}
$$
\lim _{n \rightarrow \infty} \int_{0}^{1} f(x) \cos ^{n} 4 x \mathrm{~d} x=
$$
(3) \begin{CJK}{UTF8}{mj}已知级数\end{CJK} $\sum_{n=0}^{\infty} x^{\alpha}\left(1-x^{2}\right)^{n}$ \begin{CJK}{UTF8}{mj}在\end{CJK} $[0,1]$ \begin{CJK}{UTF8}{mj}上一致收玫\end{CJK}, \begin{CJK}{UTF8}{mj}求\end{CJK} $\alpha$ \begin{CJK}{UTF8}{mj}的取值范围为\end{CJK}

(4) \begin{CJK}{UTF8}{mj}定积分\end{CJK} $\int_{0}^{+\infty} \frac{\mathrm{d} x}{x^{3}+4}=$

(5) \begin{CJK}{UTF8}{mj}已知\end{CJK} $x_{0}$ \begin{CJK}{UTF8}{mj}是超越数\end{CJK}, \begin{CJK}{UTF8}{mj}并且数列\end{CJK} $\left\{x_{n}\right\}$ \begin{CJK}{UTF8}{mj}满足\end{CJK}
$$
x_{n+1}=\frac{3-x_{n}}{x_{n}^{2}+3 x_{n}-2}
$$
\begin{CJK}{UTF8}{mj}则\end{CJK} $\lim _{n \rightarrow \infty} x_{n}=$

\begin{enumerate}
  \setcounter{enumi}{2}
  \item \begin{CJK}{UTF8}{mj}证明\end{CJK}:\begin{CJK}{UTF8}{mj}闭区间不能写成两个无交闭集的并\end{CJK}.

  \item \begin{CJK}{UTF8}{mj}求\end{CJK} $a$ \begin{CJK}{UTF8}{mj}的取值范围\end{CJK}, \begin{CJK}{UTF8}{mj}使得不等式\end{CJK}

\end{enumerate}
$$
\ln (1+x)<a x+(1-a) \frac{x}{1+x}
$$
\begin{CJK}{UTF8}{mj}恒成立\end{CJK}. 4. \begin{CJK}{UTF8}{mj}已知函数\end{CJK} $f$ \begin{CJK}{UTF8}{mj}存在二阶连续偏导数\end{CJK}, $\Delta f \leqslant 0$,
$$
F(r)=\frac{1}{r^{2}} \iint_{x^{2}+y^{2}+z^{2}=r^{2}} f(x, y, z) \mathrm{d} S
$$
\includegraphics[max width=\textwidth]{2022_04_18_7db0708508f26638f054g-167}

\begin{CJK}{UTF8}{mj}求证\end{CJK}: $F(r)$ \begin{CJK}{UTF8}{mj}单调递增\end{CJK}.

\includegraphics[max width=\textwidth]{2022_04_18_7db0708508f26638f054g-167(1)}

\section{$18.2$ 高等代数}
\includegraphics[max width=\textwidth]{2022_04_18_7db0708508f26638f054g-167(2)}

\begin{enumerate}
  \item \begin{CJK}{UTF8}{mj}求解某个含参线性方程组有解情况\end{CJK}, \begin{CJK}{UTF8}{mj}具体数据无法回忆\end{CJK}.

  \item \begin{CJK}{UTF8}{mj}求正交变换将\end{CJK}

  \item \begin{CJK}{UTF8}{mj}求正交㝔抰将\end{CJK}

\end{enumerate}
$$
f\left(x_{1}, x_{2}, x_{3}\right)=x_{1}^{2}+x_{2}^{2}-2 x_{3}^{2}+8 x_{1} x_{2}+4 x_{1} x_{3}+4 x_{2} x_{3}
$$
\begin{CJK}{UTF8}{mj}化为标准型\end{CJK}.

\begin{enumerate}
  \setcounter{enumi}{3}
  \item \begin{CJK}{UTF8}{mj}已知\end{CJK} $A_{t}=\left(\begin{array}{ccc}t & t-2 & 4-2 t \\ 3 & -1 & 0 \\ 1+t & t-2 & 3-2 t\end{array}\right), f(x)=\left|x I_{3}-A_{t}\right|$
\end{enumerate}
(1) \begin{CJK}{UTF8}{mj}若\end{CJK} $f(x)$ \begin{CJK}{UTF8}{mj}与\end{CJK} $f^{\prime}(x)$ \begin{CJK}{UTF8}{mj}不互素\end{CJK}, \begin{CJK}{UTF8}{mj}求所有可能的\end{CJK} $t$ \begin{CJK}{UTF8}{mj}值\end{CJK};

(2) \begin{CJK}{UTF8}{mj}若\end{CJK} $A$ \begin{CJK}{UTF8}{mj}可对角化\end{CJK}, \begin{CJK}{UTF8}{mj}求此时的\end{CJK} $t$, \begin{CJK}{UTF8}{mj}并求出\end{CJK} $P$, \begin{CJK}{UTF8}{mj}使得\end{CJK} $P^{-1} A P$ \begin{CJK}{UTF8}{mj}是对角矩阵\end{CJK};

(3) \begin{CJK}{UTF8}{mj}若\end{CJK} $A$ \begin{CJK}{UTF8}{mj}不可对角化\end{CJK}, \begin{CJK}{UTF8}{mj}求此时的\end{CJK} $t$, \begin{CJK}{UTF8}{mj}并求出\end{CJK} $P$, \begin{CJK}{UTF8}{mj}使得\end{CJK} $P^{-1} A P$ \begin{CJK}{UTF8}{mj}是\end{CJK} Jordan \begin{CJK}{UTF8}{mj}标准型\end{CJK}.

\begin{enumerate}
  \setcounter{enumi}{4}
  \item \begin{CJK}{UTF8}{mj}已知\end{CJK} $A_{t}=\left(\begin{array}{ccc}1 & 0 & t \\ 1 & 1 & t \\ 0 & 0 & 1\end{array}\right)$, \begin{CJK}{UTF8}{mj}有线性变换\end{CJK} $\varphi_{t}(X)=A_{t} X-X A_{t}$, \begin{CJK}{UTF8}{mj}令\end{CJK} $f(t)=\operatorname{dim} \operatorname{Im} \varphi_{t}$, \begin{CJK}{UTF8}{mj}讨论实函数\end{CJK} $f(t)$ \begin{CJK}{UTF8}{mj}的连续性\end{CJK}.

  \item \begin{CJK}{UTF8}{mj}已知半正定矩阵\end{CJK} $M=\left(\begin{array}{cc}A & B \\ B & C\end{array}\right)$, \begin{CJK}{UTF8}{mj}其中\end{CJK} $A, B, C$ \begin{CJK}{UTF8}{mj}分别是\end{CJK} $m \times m, m \times n, n \times n$ \begin{CJK}{UTF8}{mj}阶矩阵\end{CJK}, $x=\left(x_{1}, \cdots, x_{m}\right)^{\prime},\left(y_{1}, \cdots, y_{n}\right)^{\prime}$ \begin{CJK}{UTF8}{mj}分别为\end{CJK} $m$ \begin{CJK}{UTF8}{mj}维和\end{CJK} $n$ \begin{CJK}{UTF8}{mj}维列向量\end{CJK}. \begin{CJK}{UTF8}{mj}证明\end{CJK}:

\end{enumerate}
(1) $\left(x^{\prime} B y\right)^{2} \leq\left(x^{\prime} A x\right)\left(y^{\prime} C y\right)$;

(2) \begin{CJK}{UTF8}{mj}上述等号成立的充分必要条件是存在非零常数\end{CJK} $\lambda$ \begin{CJK}{UTF8}{mj}使得\end{CJK}
$$
\left\{\begin{array} { l } 
{ \lambda A x = B y } \\
{ C y = \lambda B ^ { \prime } x }
\end{array} \text { 或 } \left\{\begin{array}{l}
A x=\lambda B y \\
\lambda C y=B^{\prime} x
\end{array}\right.\right.
$$
\begin{CJK}{UTF8}{mj}其中之一成立\end{CJK}.

\begin{enumerate}
  \setcounter{enumi}{6}
  \item \begin{CJK}{UTF8}{mj}已知\end{CJK} $A$ \begin{CJK}{UTF8}{mj}是每个元素都大于\end{CJK} 0 \begin{CJK}{UTF8}{mj}的矩阵\end{CJK}. (1) \begin{CJK}{UTF8}{mj}已知矩阵\end{CJK} $A$ \begin{CJK}{UTF8}{mj}的特征值模长都小于\end{CJK} 1 , \begin{CJK}{UTF8}{mj}证明\end{CJK}: $(I-A)^{-1}$ \begin{CJK}{UTF8}{mj}的每个元素都是正的\end{CJK}.
\end{enumerate}
(2) \begin{CJK}{UTF8}{mj}若\end{CJK} $A$ \begin{CJK}{UTF8}{mj}的每行和都是\end{CJK} $c>0$, \begin{CJK}{UTF8}{mj}证明\end{CJK}: \begin{CJK}{UTF8}{mj}在允许相差一个常数倍的情说下\end{CJK} $A$ \begin{CJK}{UTF8}{mj}只有一个\end{CJK} \begin{CJK}{UTF8}{mj}全为正的特征向量\end{CJK}.

\begin{enumerate}
  \setcounter{enumi}{7}
  \item \begin{CJK}{UTF8}{mj}已知复线性空间\end{CJK} $V$ \begin{CJK}{UTF8}{mj}上的线性变换\end{CJK} $\varphi, \psi$ \begin{CJK}{UTF8}{mj}满足\end{CJK} $\varphi \psi-\psi \varphi=\varphi+\psi$, \begin{CJK}{UTF8}{mj}证明\end{CJK}: \begin{CJK}{UTF8}{mj}存在一组基\end{CJK}, \begin{CJK}{UTF8}{mj}使得\end{CJK} $\varphi$ \begin{CJK}{UTF8}{mj}和\end{CJK} $\psi$ \begin{CJK}{UTF8}{mj}在这组基下的表示矩阵都是上三角阵\end{CJK}.\\

\includegraphics[max width=\textwidth]{2022_04_18_7db0708508f26638f054g-168}
\end{enumerate}
\section{第十九章 2021 年哈尔滨工业大学真题}
\section{$19.1$ 高等代数}
\begin{enumerate}
  \item \begin{CJK}{UTF8}{mj}判断如下说法是否正确\end{CJK}, \begin{CJK}{UTF8}{mj}并说明理由\end{CJK}.
\end{enumerate}
(1) \begin{CJK}{UTF8}{mj}设\end{CJK} $A$ \begin{CJK}{UTF8}{mj}是\end{CJK} 3 \begin{CJK}{UTF8}{mj}阶实矩阵\end{CJK}, \begin{CJK}{UTF8}{mj}则\end{CJK} $A$ \begin{CJK}{UTF8}{mj}的伴隨矩阵的行列式的值为非负实数\end{CJK};

(2) \begin{CJK}{UTF8}{mj}设\end{CJK} $S=\left\{A \in \mathbb{C}^{n \times n}|E+A| \neq 0\right\}, A \in S$, \begin{CJK}{UTF8}{mj}定义\end{CJK} $\varphi(A)=(E-A)(E+A)^{-1}$, \begin{CJK}{UTF8}{mj}则\end{CJK} $\varphi(\varphi(A))=A$.

\begin{enumerate}
  \setcounter{enumi}{2}
  \item \begin{CJK}{UTF8}{mj}解答如下问题\end{CJK}:
\end{enumerate}
(1) \begin{CJK}{UTF8}{mj}设矩阵\end{CJK} $A=\left(a_{i j}\right)_{6 \times 6}$, \begin{CJK}{UTF8}{mj}其中\end{CJK} $a_{i i}=2 i, i \neq j$ \begin{CJK}{UTF8}{mj}时\end{CJK}, $a_{i j}=i$, \begin{CJK}{UTF8}{mj}求\end{CJK} $A$ \begin{CJK}{UTF8}{mj}的行列式的值\end{CJK};

(2) \begin{CJK}{UTF8}{mj}设矩阵\end{CJK} $A=\left(a_{i j}\right)_{6 \times 6}$, \begin{CJK}{UTF8}{mj}其中\end{CJK} $a_{i j}=2 i j-i-j$, \begin{CJK}{UTF8}{mj}求\end{CJK} $A$ \begin{CJK}{UTF8}{mj}的特征值\end{CJK}.

\begin{enumerate}
  \setcounter{enumi}{3}
  \item \begin{CJK}{UTF8}{mj}设\end{CJK} 3 \begin{CJK}{UTF8}{mj}阶复矩阵\end{CJK} $A$ \begin{CJK}{UTF8}{mj}的秩为\end{CJK} 1 , \begin{CJK}{UTF8}{mj}求满足条件\end{CJK} $A B=2 B A$ \begin{CJK}{UTF8}{mj}的复矩阵\end{CJK} $B$ \begin{CJK}{UTF8}{mj}全体构成的复线性空\end{CJK} \begin{CJK}{UTF8}{mj}间\end{CJK} $V$ \begin{CJK}{UTF8}{mj}的维数\end{CJK}.

  \item \begin{CJK}{UTF8}{mj}设\end{CJK} $A$ \begin{CJK}{UTF8}{mj}是迹为\end{CJK} 0 \begin{CJK}{UTF8}{mj}行列式为\end{CJK} 1 \begin{CJK}{UTF8}{mj}的\end{CJK} 4 \begin{CJK}{UTF8}{mj}阶实矩阵\end{CJK}, \begin{CJK}{UTF8}{mj}求\end{CJK} $A^{2}+A^{4}$ \begin{CJK}{UTF8}{mj}的迹的是小值\end{CJK}.

  \item \begin{CJK}{UTF8}{mj}设一元三次实系数多项式\end{CJK} $f(x)$ \begin{CJK}{UTF8}{mj}满足\end{CJK}

\end{enumerate}
$$
f(-2)=-125, f(1)=1, f(0)=-1, f(-1)=-27
$$
\begin{CJK}{UTF8}{mj}求\end{CJK} $f(1)+f(2)+\cdots+f(100)$ \begin{CJK}{UTF8}{mj}的值\end{CJK}.

\begin{enumerate}
  \setcounter{enumi}{6}
  \item \begin{CJK}{UTF8}{mj}设\end{CJK} $V=\left\{a x^{3}+b x^{2}+c x+d \mid a, b, c, d \in \mathbb{R}\right\}$, \begin{CJK}{UTF8}{mj}在\end{CJK} $V$ \begin{CJK}{UTF8}{mj}上定义内积\end{CJK}
\end{enumerate}
$$
(f(x), g(x))=\int_{0}^{2} f(x) g(x) \mathrm{d} x
$$
\begin{CJK}{UTF8}{mj}求\end{CJK} $V$ \begin{CJK}{UTF8}{mj}的子空间\end{CJK} $W=\{f(x) \in V \mid f(1)=0\}$ \begin{CJK}{UTF8}{mj}的维数\end{CJK}, \begin{CJK}{UTF8}{mj}并求\end{CJK} $W$ \begin{CJK}{UTF8}{mj}的正交补\end{CJK}.

\section{第二十章 2021 年中国人民大学真题}
\section{$20.1$ 数学分析}
\begin{enumerate}
  \item \begin{CJK}{UTF8}{mj}若\end{CJK} $0<x_{1}<1$, \begin{CJK}{UTF8}{mj}且对\end{CJK} $\forall n \geq 1$ \begin{CJK}{UTF8}{mj}都有\end{CJK} $x_{n+1}=x_{n}\left(x_{n}-1\right)$, \begin{CJK}{UTF8}{mj}试证\end{CJK}: $\lim _{n \rightarrow \infty} n x_{n}=1$.

  \item \begin{CJK}{UTF8}{mj}利用闭区间套定理证明实数集是不可数的\end{CJK}.

  \item \begin{CJK}{UTF8}{mj}证明在\end{CJK} $(a, b)$ \begin{CJK}{UTF8}{mj}上的一致连续函数\end{CJK} $f(x)$ \begin{CJK}{UTF8}{mj}可以延拓为\end{CJK} $[a, b]$ \begin{CJK}{UTF8}{mj}上的连续函数\end{CJK}.

  \item \begin{CJK}{UTF8}{mj}举出一个在一点处可导但是这个点外的任何实数不连续的函数\end{CJK}, \begin{CJK}{UTF8}{mj}并给出其证明\end{CJK}.

  \item \begin{CJK}{UTF8}{mj}计算题\end{CJK}

\end{enumerate}
(1) \begin{CJK}{UTF8}{mj}计算极限\end{CJK} $\lim _{n \rightarrow+\infty} n\left(\frac{1^{k}+2^{k}+\cdots+n^{k}}{n^{k+1}}-\frac{1}{k+1}\right)$.

(2) \begin{CJK}{UTF8}{mj}若粗圆\end{CJK} $\Omega: \frac{x^{2}}{a^{2}}+\frac{y^{2}}{b^{2}}=1$, \begin{CJK}{UTF8}{mj}在第一象限\end{CJK}. \begin{CJK}{UTF8}{mj}求\end{CJK} $\int_{\Omega} x y \mathrm{~d} s$.

(3) \begin{CJK}{UTF8}{mj}曲面\end{CJK} $\Omega: x \geq 0, y \geq 0, x+y \leq 1$. \begin{CJK}{UTF8}{mj}求\end{CJK} $\iint_{A} \mathrm{e}^{\frac{x-y}{x+v}} \mathrm{~d} x \mathrm{~d} y$.

(4) \begin{CJK}{UTF8}{mj}曲面\end{CJK} $\Omega: 0 \leq z=2-x^{2}-y^{2}$, \begin{CJK}{UTF8}{mj}利用在曲面\end{CJK} $\Omega$ \begin{CJK}{UTF8}{mj}上定义的密度函数\end{CJK} $p(x, y)=x^{2}+y^{2}$ \begin{CJK}{UTF8}{mj}求曲面\end{CJK} $\Omega$ \begin{CJK}{UTF8}{mj}的质量\end{CJK}.

(5) \begin{CJK}{UTF8}{mj}求不定积分\end{CJK}
$$
\int \frac{1}{3 \sin x+4 \cos x+4} \mathrm{~d} x
$$
(6) \begin{CJK}{UTF8}{mj}求极限\end{CJK}
$$
\lim _{x \rightarrow 0} \frac{\ln (\cos x)}{\tan ^{2} x}
$$

\begin{enumerate}
  \setcounter{enumi}{6}
  \item \begin{CJK}{UTF8}{mj}若\end{CJK} $f(x)$ \begin{CJK}{UTF8}{mj}在\end{CJK} $[a, b]$ \begin{CJK}{UTF8}{mj}连续\end{CJK}, \begin{CJK}{UTF8}{mj}在\end{CJK} $(a, b)$ \begin{CJK}{UTF8}{mj}可导\end{CJK}, \begin{CJK}{UTF8}{mj}且\end{CJK} $b>a>0, f(a) \neq f(b)$, \begin{CJK}{UTF8}{mj}试证\end{CJK}: \begin{CJK}{UTF8}{mj}存在\end{CJK} $\xi, \eta \in(a, b)$, \begin{CJK}{UTF8}{mj}使得\end{CJK}
\end{enumerate}
$$
f^{\prime}(\xi)=\frac{a+b}{2 \eta} f^{\prime}(\eta)
$$

\begin{enumerate}
  \setcounter{enumi}{7}
  \item \begin{CJK}{UTF8}{mj}讨论\end{CJK} $\sum_{n=1}^{+\infty} \frac{\sin n x}{\sqrt{n}}$ \begin{CJK}{UTF8}{mj}的绝对收敛和条件收玫性\end{CJK}. 8. (1) \begin{CJK}{UTF8}{mj}将\end{CJK} $f(x)=(x-1)^{2}$ \begin{CJK}{UTF8}{mj}在\end{CJK} $(0,1)$ \begin{CJK}{UTF8}{mj}上展开为余弦函数\end{CJK}, \begin{CJK}{UTF8}{mj}并求和\end{CJK} $\sum_{n=1}^{+\infty} \frac{1}{n^{2}}$.
\end{enumerate}
(2) \begin{CJK}{UTF8}{mj}利用逐项积分求和\end{CJK} $\sum_{n=1}^{+\infty} \frac{1}{n^{4}}$.

\begin{enumerate}
  \setcounter{enumi}{9}
  \item \begin{CJK}{UTF8}{mj}证明\end{CJK} Tauber \begin{CJK}{UTF8}{mj}定理\end{CJK}: \begin{CJK}{UTF8}{mj}设幂级数\end{CJK} $\sum_{n=1}^{+\infty} a_{n} x^{n}$ \begin{CJK}{UTF8}{mj}的收敛半径为\end{CJK} 1 , \begin{CJK}{UTF8}{mj}且\end{CJK}
\end{enumerate}
$$
\lim _{x \rightarrow 1^{-}} \sum_{n=1}^{+\infty} a_{n} x^{n}=A
$$
\begin{CJK}{UTF8}{mj}存在\end{CJK}, \begin{CJK}{UTF8}{mj}如果\end{CJK} $a_{n}=o\left(\frac{1}{n}\right)$, \begin{CJK}{UTF8}{mj}试证\end{CJK}: $\sum_{n=1}^{+\infty} a_{n}=A$

\begin{enumerate}
  \setcounter{enumi}{10}
  \item \begin{CJK}{UTF8}{mj}设\end{CJK} $\Omega$ \begin{CJK}{UTF8}{mj}是\end{CJK} $\mathbb{R}^{3}$ \begin{CJK}{UTF8}{mj}中的一个闭区域\end{CJK}, \begin{CJK}{UTF8}{mj}向量\end{CJK} $n=(\cos \alpha, \cos \beta, \cos \gamma)$ \begin{CJK}{UTF8}{mj}是\end{CJK} $\partial \Omega$ \begin{CJK}{UTF8}{mj}的单位外法向量\end{CJK}, \begin{CJK}{UTF8}{mj}点\end{CJK} $(a, b, c) \notin \partial \Omega$, \begin{CJK}{UTF8}{mj}令\end{CJK}
\end{enumerate}
$$
\rho=(x-a, y-b, z-c), \quad p=\|\rho\|
$$
\begin{CJK}{UTF8}{mj}证明\end{CJK}:
$$
\iiint_{\Omega} \frac{\mathrm{d} x \mathrm{~d} y \mathrm{~d} z}{p}=\frac{1}{2} \iint_{\partial \Omega} \cos (\rho, n) \mathrm{d} \sigma
$$

\begin{enumerate}
  \setcounter{enumi}{11}
  \item \begin{CJK}{UTF8}{mj}计算\end{CJK}
\end{enumerate}
$$
\lim _{R \rightarrow+\infty} \oint_{x^{2}+y^{2} \leq R^{2}} \frac{y \mathrm{~d} x-x \mathrm{~d} y}{\left(x^{2}+x y+y^{2}\right)^{\frac{3}{2}}}
$$

\section{第二十一章 2021 年北京师范大学真题}
\section{$21.1$ 数学分析}
\begin{enumerate}
  \item \begin{CJK}{UTF8}{mj}若\end{CJK} $\lim _{n \rightarrow \infty}\left(n^{2 n \sin \frac{1}{n}} a_{n}\right)=1$, \begin{CJK}{UTF8}{mj}则级数\end{CJK} $\sum_{n=1}^{\infty} a_{n}$ \begin{CJK}{UTF8}{mj}是否收玫\end{CJK}? \begin{CJK}{UTF8}{mj}并给出证明\end{CJK}.

  \item \begin{CJK}{UTF8}{mj}设\end{CJK} $0<a<b<\infty$, \begin{CJK}{UTF8}{mj}试证\end{CJK}: \begin{CJK}{UTF8}{mj}存在\end{CJK} $\theta \in(a, b)$ \begin{CJK}{UTF8}{mj}使得\end{CJK}

\end{enumerate}
$$
b e^{a}-a e^{b}=(1-\theta) \mathrm{e}^{2}(b-a)
$$

\begin{enumerate}
  \setcounter{enumi}{3}
  \item \begin{CJK}{UTF8}{mj}设\end{CJK} $f(x)=\arctan x$, \begin{CJK}{UTF8}{mj}且\end{CJK}
\end{enumerate}
$$
B=\lim _{n \rightarrow \infty}\left(\sum_{k=1}^{n} f\left(\frac{k}{n}\right)-A n\right)
$$
\begin{CJK}{UTF8}{mj}存在时\end{CJK}, \begin{CJK}{UTF8}{mj}求\end{CJK} $A, B$ \begin{CJK}{UTF8}{mj}的值\end{CJK}.

\begin{enumerate}
  \setcounter{enumi}{4}
  \item (1) \begin{CJK}{UTF8}{mj}设\end{CJK} $f(x)$ \begin{CJK}{UTF8}{mj}在\end{CJK} $(a, b)$ \begin{CJK}{UTF8}{mj}连续\end{CJK}, $a, b \in \mathbb{R}$, \begin{CJK}{UTF8}{mj}试证\end{CJK}: $f(x)$ \begin{CJK}{UTF8}{mj}在\end{CJK} $(a, b)$ \begin{CJK}{UTF8}{mj}上一致连续的充要条件是\end{CJK} $f(a+0), f(b-0)$ \begin{CJK}{UTF8}{mj}都存在\end{CJK};
\end{enumerate}
(2) \begin{CJK}{UTF8}{mj}若\end{CJK} $a \in \mathbb{R}, b=+\infty$, \begin{CJK}{UTF8}{mj}上述命题依然存在\end{CJK}.

\begin{enumerate}
  \setcounter{enumi}{5}
  \item \begin{CJK}{UTF8}{mj}设幂级数\end{CJK} $f(x)=\sum_{n=1}^{\infty} \frac{x^{2 y}}{n(2 n+1)}$, \begin{CJK}{UTF8}{mj}试求收玫域\end{CJK}, \begin{CJK}{UTF8}{mj}以及求级数\end{CJK}
\end{enumerate}
$$
S=\sum_{n=1}^{\infty} \frac{1}{n(2 n+1) 2^{n}}
$$

\begin{enumerate}
  \setcounter{enumi}{6}
  \item \begin{CJK}{UTF8}{mj}求曲线\end{CJK}
\end{enumerate}
$$
\left(x^{2}+y^{2}\right)^{2}=32\left(x^{2}+y^{2}\right)^{2}=8 x y
$$
\begin{CJK}{UTF8}{mj}它们在第一象限围成的图形面积\end{CJK}.

\begin{enumerate}
  \setcounter{enumi}{7}
  \item \begin{CJK}{UTF8}{mj}计算第三型曲线积分\end{CJK}
\end{enumerate}
$$
\oint_{c} \frac{x \mathrm{~d} y-y \mathrm{~d} x}{x^{2}+8 y^{2}}
$$
\begin{CJK}{UTF8}{mj}其中\end{CJK} $C$ \begin{CJK}{UTF8}{mj}是以\end{CJK} $(1,0)$ \begin{CJK}{UTF8}{mj}为圆心\end{CJK}, \begin{CJK}{UTF8}{mj}半径为\end{CJK} $R(R \neq 1)$ \begin{CJK}{UTF8}{mj}的圆\end{CJK}. 8. \begin{CJK}{UTF8}{mj}判断积分\end{CJK}
$$
f(t)=\int_{0}^{1} \frac{1}{x^{t}} \sin \frac{1}{x} \mathrm{~d} x
$$
\begin{CJK}{UTF8}{mj}在\end{CJK} $0<t<2$ \begin{CJK}{UTF8}{mj}的一致收玫性\end{CJK}. 1. $\lim _{n \rightarrow \infty} a_{n}=+\infty$, \begin{CJK}{UTF8}{mj}证明\end{CJK} $\lim _{n \rightarrow \infty} \frac{a_{1}+a_{2}+\cdots+a_{n}}{n}=+\infty$. \begin{CJK}{UTF8}{mj}并说明反过来不成立\end{CJK}.

\begin{enumerate}
  \setcounter{enumi}{2}
  \item $f(x, y)=\left\{\begin{array}{l}\frac{x^{3}+2 x y^{2}}{x^{2}+y^{2}}, x^{2}+y^{2} \neq 0 \\ 0, x^{2}+y^{2}=0\end{array}\right.$ \begin{CJK}{UTF8}{mj}问\end{CJK}: $f(x, y)$ \begin{CJK}{UTF8}{mj}在\end{CJK} $(0,0)$ \begin{CJK}{UTF8}{mj}处连续吗\end{CJK}? \begin{CJK}{UTF8}{mj}方向可导吗\end{CJK}? \begin{CJK}{UTF8}{mj}可微吗\end{CJK}?

  \item \begin{CJK}{UTF8}{mj}计算\end{CJK}

\end{enumerate}
$$
\int_{0}^{\infty} \frac{\mathrm{e}^{-a x}-\mathrm{e}^{-b x}}{x} \sin x \mathrm{~d} x
$$

\begin{enumerate}
  \setcounter{enumi}{4}
  \item \begin{CJK}{UTF8}{mj}讨论\end{CJK} $\sum_{n=1}^{\infty} \frac{1}{a^{\ln n}}(a>0)$ \begin{CJK}{UTF8}{mj}的敛共孜性\end{CJK}.
\end{enumerate}
$5 . z^{3}-x y z=a^{3}$. \begin{CJK}{UTF8}{mj}求\end{CJK} $z_{x x}, z_{y x}$.

6.Green\begin{CJK}{UTF8}{mj}公式\end{CJK} + \begin{CJK}{UTF8}{mj}微分方程\end{CJK}.

\begin{enumerate}
  \setcounter{enumi}{7}
  \item \begin{CJK}{UTF8}{mj}计算\end{CJK} $\iint_{\Sigma} x^{3} \mathrm{~d} y \mathrm{~d} z$, \begin{CJK}{UTF8}{mj}其中\end{CJK} $\Sigma: \frac{x^{2}}{a^{2}}+\frac{y^{2}}{b^{2}}+\frac{z^{2}}{c^{2}}=1, z \geq 0$. \begin{CJK}{UTF8}{mj}取夕侧\end{CJK}.

  \item \begin{CJK}{UTF8}{mj}设\end{CJK} $f(x)$ \begin{CJK}{UTF8}{mj}在\end{CJK} $(0,1)$ \begin{CJK}{UTF8}{mj}可微\end{CJK}, \begin{CJK}{UTF8}{mj}且有\end{CJK} $\int_{\frac{1}{2}}^{1} x^{2} f(x) \mathrm{d} x=0$. \begin{CJK}{UTF8}{mj}证明\end{CJK}: \begin{CJK}{UTF8}{mj}存在\end{CJK} $\theta \in(0,1)$, \begin{CJK}{UTF8}{mj}使得\end{CJK} $f^{\prime}(\theta)=-\frac{f(\theta)}{\theta}$.

  \item \begin{CJK}{UTF8}{mj}求\end{CJK} $f(x, y, z)=x^{2} y^{2} z^{2}$ \begin{CJK}{UTF8}{mj}在单位球上的最值\end{CJK}.

  \item \begin{CJK}{UTF8}{mj}计算\end{CJK} $\iint_{\Omega} \mathrm{e}^{\frac{x-y}{x+y}} \mathrm{~d} \Omega$, \begin{CJK}{UTF8}{mj}其中\end{CJK} $\Omega: x \geq 0, y \geq 0, x+y \leq 1$.

\end{enumerate}
\includegraphics[max width=\textwidth]{2022_04_18_7db0708508f26638f054g-174}

1.(a) \begin{CJK}{UTF8}{mj}求\end{CJK} $\lim _{n \rightarrow \infty}\left(\frac{1}{n}+\frac{1}{n+1}+\frac{1}{n+2}+\cdots+\frac{1}{2 n}\right)$.

(b) \begin{CJK}{UTF8}{mj}用\end{CJK} (a) \begin{CJK}{UTF8}{mj}结论证明\end{CJK} $\sum_{n=1}^{\infty} \frac{1}{n}$ \begin{CJK}{UTF8}{mj}不收敛\end{CJK}.

\begin{enumerate}
  \setcounter{enumi}{2}
  \item $y-x-\frac{1}{2} \sin y=0$.
\end{enumerate}
(a) \begin{CJK}{UTF8}{mj}证明在\end{CJK} $\mathbb{R} 上 y 与 x$ \begin{CJK}{UTF8}{mj}与一映射\end{CJK}.

(b) \begin{CJK}{UTF8}{mj}证明\end{CJK} $\frac{\mathrm{d} y}{\mathrm{~d} x}$ \begin{CJK}{UTF8}{mj}存在\end{CJK}, \begin{CJK}{UTF8}{mj}并求\end{CJK} $\left.\frac{\mathrm{d} y}{\mathrm{~d} x}\right|_{x=0^{\circ}}$.

(c) \begin{CJK}{UTF8}{mj}证明\end{CJK} $\frac{\mathrm{d}^{2} y}{\mathrm{~d} x^{2}}$ \begin{CJK}{UTF8}{mj}存在\end{CJK}, \begin{CJK}{UTF8}{mj}㚔求\end{CJK} $\left.\frac{\mathrm{d}^{2} y}{\mathrm{~d} x^{2}}\right|_{x=0}$.

$3 . y=\left\{\begin{array}{l}\mathrm{e}^{x}, x \geq 0 \\ x+1, x<0\end{array}\right.$

(a) \begin{CJK}{UTF8}{mj}用导数定义证明\end{CJK} $x=0$ \begin{CJK}{UTF8}{mj}时导数存在\end{CJK};

(b) \begin{CJK}{UTF8}{mj}已知一种方法如下\end{CJK}:\begin{CJK}{UTF8}{mj}某点邻域内连续\end{CJK},\begin{CJK}{UTF8}{mj}厺心邻域导数存在导数且该点的导数极限连续\end{CJK}, \begin{CJK}{UTF8}{mj}则该点\end{CJK} \begin{CJK}{UTF8}{mj}该点的对应的导数值\end{CJK}.

\begin{enumerate}
  \setcounter{enumi}{4}
  \item $f(x, y)=\left\{\begin{array}{l}\left(x^{2}+x y+y^{2}\right) \sin \frac{1}{x^{2}+y^{2}}, x^{2}+y^{2} \neq 0 \\ 0, x^{2}+y^{2}=0\end{array}\right.$. \begin{CJK}{UTF8}{mj}证明\end{CJK} $f(x, y)$ \begin{CJK}{UTF8}{mj}在\end{CJK} $(0,0)$ \begin{CJK}{UTF8}{mj}处的可微性\end{CJK}.

  \item \begin{CJK}{UTF8}{mj}用\end{CJK} Lagrange \begin{CJK}{UTF8}{mj}乘数法求\end{CJK} $f(x, y)=x+4 y$ \begin{CJK}{UTF8}{mj}在椭圆\end{CJK} $\frac{x^{2}}{2}-x+y^{2}=1$ \begin{CJK}{UTF8}{mj}上的最大值和最小值\end{CJK}.

  \item \begin{CJK}{UTF8}{mj}求\end{CJK} $\iint_{S} z \mathrm{~d} S, S$ \begin{CJK}{UTF8}{mj}为螺旋面\end{CJK} $x=u \cos v, y=u \sin v, z=v$ \begin{CJK}{UTF8}{mj}的\end{CJK} $u \in(0, R), v \in(0,2 \pi)$ \begin{CJK}{UTF8}{mj}的区间上的值\end{CJK}.

  \item \begin{CJK}{UTF8}{mj}求\end{CJK} $\oint_{C} \frac{-(x+y) \mathrm{d} x+(x-y) \mathrm{d} y}{x^{2}+y^{2}}$ \begin{CJK}{UTF8}{mj}其中\end{CJK} $C:(-1,-1) ;(1,-1) ;(1,1) ;(-1,1)$. \begin{CJK}{UTF8}{mj}取逆时针方向\end{CJK}.

  \item \begin{CJK}{UTF8}{mj}求\end{CJK} $\iint_{\Sigma}\left(2 x z-y^{2}\right) \mathrm{d} y \mathrm{~d} z-z^{2} \mathrm{~d} x \mathrm{~d} y$, \begin{CJK}{UTF8}{mj}其中\end{CJK} $\Sigma: z=\frac{x^{2}+y^{2}}{4}$ \begin{CJK}{UTF8}{mj}的下表面\end{CJK}.

  \item \begin{CJK}{UTF8}{mj}证明敛散性\end{CJK}:(a) $\sum_{n=1}^{\infty} n !\left(\frac{2}{n}\right)^{n}$ (b) $\sum_{n=1}^{\infty} n !\left(\frac{e}{n}\right)^{n}$

  \item \begin{CJK}{UTF8}{mj}求级数\end{CJK} $\sum_{n=1}^{\infty} \frac{x^{n}}{n(1+n)}$ \begin{CJK}{UTF8}{mj}的收敛域与和函数\end{CJK}.

  \item $\sum_{n=1}^{\infty} \frac{\sin n x}{n^{a}}$

\end{enumerate}
(a) $a>1$ \begin{CJK}{UTF8}{mj}时\end{CJK}, \begin{CJK}{UTF8}{mj}证明其在\end{CJK} $\mathbb{R}$ \begin{CJK}{UTF8}{mj}上绝对收敛和一致收敛\end{CJK}.

(b) $a \leq 1$ \begin{CJK}{UTF8}{mj}且\end{CJK} $a>0$ \begin{CJK}{UTF8}{mj}时\end{CJK}, $x \neq k \pi(k \in Z)$ \begin{CJK}{UTF8}{mj}时条件收敛和在\end{CJK} $[-\delta, \pi-\delta], \delta \in(0, \pi)$ \begin{CJK}{UTF8}{mj}上一致收敛\end{CJK}.

(c) \begin{CJK}{UTF8}{mj}证明\end{CJK}: \begin{CJK}{UTF8}{mj}在\end{CJK} $a \leq 1, a>0$ \begin{CJK}{UTF8}{mj}时\end{CJK}, $x \in[0,2 \pi]$ \begin{CJK}{UTF8}{mj}不一致收敛\end{CJK}.
$$
D=\left|\begin{array}{ccccc}
1 & \frac{3}{2} & 1 & 1 & 1 \\
1 & 1 & \frac{4}{3} & 1 & 1 \\
1 & 1 & 1 & \frac{5}{4} & 1 \\
1 & 1 & 1 & 1 & \frac{6}{5}
\end{array}\right|
$$

\begin{enumerate}
  \setcounter{enumi}{2}
  \item \begin{CJK}{UTF8}{mj}设\end{CJK} $\mathbb{R}^{5}$ \begin{CJK}{UTF8}{mj}中向量组\end{CJK} $\left\{\alpha_{1}, \alpha_{2}, \alpha_{3}, \alpha_{4}\right\}$ \begin{CJK}{UTF8}{mj}如下\end{CJK}:
\end{enumerate}
$$
\alpha_{1}=(0,1,2,1,3)^{\prime}, \alpha_{2}=(1,0,1,2,3)^{\prime}, \alpha_{3}=(1,1,0,-1,4)^{\prime}, \alpha_{4}=(-1,4,4,-2,7)^{\prime}
$$
\begin{CJK}{UTF8}{mj}求一组极大线性无关组\end{CJK}, \begin{CJK}{UTF8}{mj}并且将余下的向量用它的线性组合表示出\end{CJK}.

\begin{enumerate}
  \setcounter{enumi}{3}
  \item \begin{CJK}{UTF8}{mj}设矩阵\end{CJK} $A=\left(\begin{array}{cccc}-3 & -4 & 4 & 4 \\ 1 & 3 & -1 & -1 \\ -1 & -1 & 2 & 1 \\ -2 & -1 & 2 & 3\end{array}\right)$. \begin{CJK}{UTF8}{mj}求\end{CJK}:
\end{enumerate}
(a) \begin{CJK}{UTF8}{mj}矩阵\end{CJK} $A$ \begin{CJK}{UTF8}{mj}的特征多项式\end{CJK} $f_{A}(x)$.

(b) \begin{CJK}{UTF8}{mj}矩阵\end{CJK} $A$ \begin{CJK}{UTF8}{mj}的最小多项式\end{CJK} $m_{A}(x)$.

(c) \begin{CJK}{UTF8}{mj}矩阵\end{CJK} $A$ \begin{CJK}{UTF8}{mj}的\end{CJK} Jordan \begin{CJK}{UTF8}{mj}标准型\end{CJK}.

\begin{enumerate}
  \setcounter{enumi}{4}
  \item \begin{CJK}{UTF8}{mj}设实数列\end{CJK} $\left\{x_{n}\right\}_{n=1}^{\infty}$ \begin{CJK}{UTF8}{mj}满足递推关系\end{CJK}:
\end{enumerate}
$$
\left\{\begin{array}{l}
x_{3 k+3}=x_{3 k-2} \\
x_{3 k+2}=x_{3 k}+x_{3 k-1} \\
x_{3 k+1}=x_{3 k}+x_{3 k-2}
\end{array}\right.
$$
\begin{CJK}{UTF8}{mj}其中\end{CJK} $k \geq 1$, \begin{CJK}{UTF8}{mj}并且\end{CJK} $x_{1}=x_{2}=0, x_{3}=1$. \begin{CJK}{UTF8}{mj}求\end{CJK} $x_{3} \bar{k}+2$ \begin{CJK}{UTF8}{mj}的通项公式\end{CJK}.

\begin{enumerate}
  \setcounter{enumi}{5}
  \item \begin{CJK}{UTF8}{mj}设\end{CJK} $\mathscr{F}: V \rightarrow V$ \begin{CJK}{UTF8}{mj}是\end{CJK} $n$ \begin{CJK}{UTF8}{mj}维复线焅宰旧\end{CJK} $V$ \begin{CJK}{UTF8}{mj}上的一个线性变换\end{CJK}, \begin{CJK}{UTF8}{mj}它的秩\end{CJK} $\operatorname{rank}(\mathscr{F})=r$. \begin{CJK}{UTF8}{mj}证明\end{CJK}: \begin{CJK}{UTF8}{mj}存在\end{CJK} $V$ \begin{CJK}{UTF8}{mj}的\end{CJK} $\left\{\alpha_{1}, \alpha_{2}, \cdots, \alpha_{n}\right\}$ \begin{CJK}{UTF8}{mj}和\end{CJK} $\left\{\beta_{1}, \beta_{2}, \cdots, \beta_{n}\right\}$ \begin{CJK}{UTF8}{mj}满足如下性质\end{CJK}: \begin{CJK}{UTF8}{mj}对任意向量\end{CJK} $\xi \in V$, \begin{CJK}{UTF8}{mj}若\end{CJK} $\xi=\sum_{j=1}^{n} k_{j} \alpha_{j}$, \begin{CJK}{UTF8}{mj}则\end{CJK} $F(\xi)=\sum_{j=1}^{r} k$

  \item \begin{CJK}{UTF8}{mj}设\end{CJK} $A=\left(a_{i j}\right)$ \begin{CJK}{UTF8}{mj}为\end{CJK} $n$ \begin{CJK}{UTF8}{mj}阶复方阵\end{CJK}, \begin{CJK}{UTF8}{mj}证明\end{CJK}: \begin{CJK}{UTF8}{mj}若\end{CJK} $A^{2} B=B A^{2}$ \begin{CJK}{UTF8}{mj}对任意\end{CJK} $n$ \begin{CJK}{UTF8}{mj}阶复方阵\end{CJK} $B$ \begin{CJK}{UTF8}{mj}成立\end{CJK}, \begin{CJK}{UTF8}{mj}则必有\end{CJK} $I_{n}+A$ \begin{CJK}{UTF8}{mj}可逆或\end{CJK} \begin{CJK}{UTF8}{mj}可逆戊立\end{CJK}.

  \item \begin{CJK}{UTF8}{mj}设\end{CJK} $A, B, C$ \begin{CJK}{UTF8}{mj}为\end{CJK} $n$ \begin{CJK}{UTF8}{mj}阶复方阵\end{CJK}, \begin{CJK}{UTF8}{mj}且\end{CJK} $C$ \begin{CJK}{UTF8}{mj}的所有特征值均为实数\end{CJK}.\begin{CJK}{UTF8}{mj}证明\end{CJK}: \begin{CJK}{UTF8}{mj}若\end{CJK} $A B-B A=C^{2}$, \begin{CJK}{UTF8}{mj}则\end{CJK} $C^{n}=O$.

  \item \begin{CJK}{UTF8}{mj}设\end{CJK} $A, B$ \begin{CJK}{UTF8}{mj}为\end{CJK} $n$ \begin{CJK}{UTF8}{mj}阶正交矩阵\end{CJK}, \begin{CJK}{UTF8}{mj}且\end{CJK} $A B=B A$. \begin{CJK}{UTF8}{mj}证明\end{CJK}: \begin{CJK}{UTF8}{mj}若\end{CJK} $A$ \begin{CJK}{UTF8}{mj}的特征值都严格小于\end{CJK} 1, \begin{CJK}{UTF8}{mj}则\end{CJK} $A B-A^{2} B$ \begin{CJK}{UTF8}{mj}是正\end{CJK},

  \item \begin{CJK}{UTF8}{mj}设\end{CJK} $A$ \begin{CJK}{UTF8}{mj}是\end{CJK} $n$ \begin{CJK}{UTF8}{mj}阶实方阵\end{CJK},$f_{A}(x)$ \begin{CJK}{UTF8}{mj}是\end{CJK} $A$ \begin{CJK}{UTF8}{mj}的特征多项式且\end{CJK} $f_{A}(x)$ \begin{CJK}{UTF8}{mj}在\end{CJK} $\mathbb{R}[x]$ \begin{CJK}{UTF8}{mj}上有不可约分解\end{CJK} $f_{A}(x)=p_{1}(x)^{l_{1}} p_{2}(x)$ \begin{CJK}{UTF8}{mj}其中\end{CJK} $\operatorname{deg} p_{i}(x) \in\{1,2\}, i=1,2, \cdots, k$. \begin{CJK}{UTF8}{mj}证明\end{CJK}: $\sum_{i=1}^{k} l_{i} \operatorname{rank}\left(p_{i}(A)\right) \leq n\left[\left(\sum_{i=1}^{k} l_{k}\right)-1\right]$, \begin{CJK}{UTF8}{mj}其中\end{CJK} $\operatorname{rank}\left(p_{i}(A)\right)$ \begin{CJK}{UTF8}{mj}是方\end{CJK} \begin{CJK}{UTF8}{mj}的秩\end{CJK}, $i=1,2, \cdots, k$. 1. \begin{CJK}{UTF8}{mj}证明\end{CJK}: \begin{CJK}{UTF8}{mj}多项式\end{CJK} $f(x)=1+x+x^{2}+x^{3}+x^{4}+x^{5}+x^{6}$ \begin{CJK}{UTF8}{mj}在\end{CJK} $\mathbb{Q}[x]$ \begin{CJK}{UTF8}{mj}上不可约\end{CJK}.

  \item \begin{CJK}{UTF8}{mj}已知线性方程组\end{CJK}

\end{enumerate}
$$
\left\{\begin{array}{l}
x_{1}+2 x_{2}+2 x_{4}=-1 \\
x_{1}-2 x_{2}+(b-1) x_{3}+x_{4}=2 \\
2 x_{1}-4 x_{2}+x_{3}+3 x_{4}=a
\end{array}\right.
$$
\begin{CJK}{UTF8}{mj}有\end{CJK} 3 \begin{CJK}{UTF8}{mj}个线性无关的解\end{CJK}, \begin{CJK}{UTF8}{mj}求\end{CJK} $a, b$ \begin{CJK}{UTF8}{mj}的值\end{CJK}, \begin{CJK}{UTF8}{mj}并求此时方程组的通解\end{CJK}.

\begin{enumerate}
  \setcounter{enumi}{3}
  \item \begin{CJK}{UTF8}{mj}已知\end{CJK} $\mathbb{R}^{4}$ \begin{CJK}{UTF8}{mj}中的向量组\end{CJK}
\end{enumerate}
$$
\begin{gathered}
\alpha_{1}=(1,-1,1,0)^{\prime}, \alpha_{2}=(1,1,0,2) \\
\alpha_{3}=(-2,1,1,3)^{\prime}, \alpha_{4}=(2,0,1,2)^{\prime}, \alpha_{5}=(1,2,-2,1)^{\prime}
\end{gathered}
$$
\begin{CJK}{UTF8}{mj}其中\end{CJK} $\alpha_{1}, \alpha_{2}$ \begin{CJK}{UTF8}{mj}是线性空间\end{CJK} $W_{1}$ \begin{CJK}{UTF8}{mj}中的向量\end{CJK}, $\alpha_{3}, \alpha_{4}, \alpha_{5}$ \begin{CJK}{UTF8}{mj}是线性空间\end{CJK} $W_{2}$ \begin{CJK}{UTF8}{mj}中的向量\end{CJK}, \begin{CJK}{UTF8}{mj}分别求\end{CJK} $W_{1}+W_{2}$ \begin{CJK}{UTF8}{mj}和\end{CJK} $W_{1} \cap W_{2}$ \begin{CJK}{UTF8}{mj}的\end{CJK} \begin{CJK}{UTF8}{mj}组基\end{CJK}.

\begin{enumerate}
  \setcounter{enumi}{4}
  \item \begin{CJK}{UTF8}{mj}已知实矩阵\end{CJK} $A$ \begin{CJK}{UTF8}{mj}的特征值不为\end{CJK} $-1$, \begin{CJK}{UTF8}{mj}证明\end{CJK}:
\end{enumerate}
(a) $A+I$ \begin{CJK}{UTF8}{mj}与\end{CJK} $A^{\prime}+I$ \begin{CJK}{UTF8}{mj}可逆\end{CJK};

(b) \begin{CJK}{UTF8}{mj}证明\end{CJK}: $A$ \begin{CJK}{UTF8}{mj}是正交矩阵当且仅当\end{CJK} $(A+I)^{-1}+\left(A^{\prime}+I\right)^{-1}=I$.

\begin{enumerate}
  \setcounter{enumi}{5}
  \item \begin{CJK}{UTF8}{mj}已知一次型\end{CJK} $f\left(x_{1}, x_{2}, x_{3}\right)=2 x_{1}^{2}+2 x_{2}^{2}+2 x_{3}^{2}-2 x_{1} x_{2}+2 a x_{2} x_{3}$, \begin{CJK}{UTF8}{mj}记\end{CJK} $x=\left(x_{1}, x_{2}, x_{3}\right)^{\prime}, y=\left(y_{1}, y_{2}, y_{3}\right)$ $f\left(x_{1}, x_{2}, x_{3}\right)$ \begin{CJK}{UTF8}{mj}是正定一次型\end{CJK}, \begin{CJK}{UTF8}{mj}求\end{CJK} $a$ \begin{CJK}{UTF8}{mj}的取值范围\end{CJK}; $(\mathrm{b})$ \begin{CJK}{UTF8}{mj}求正交变换\end{CJK} $P$, \begin{CJK}{UTF8}{mj}使得\end{CJK} $f\left(x_{1}, x_{2}, x_{3}\right)$ \begin{CJK}{UTF8}{mj}在变换\end{CJK} $x=P y$ \begin{CJK}{UTF8}{mj}下化为\end{CJK}

  \item \begin{CJK}{UTF8}{mj}已知\end{CJK} $n$ \begin{CJK}{UTF8}{mj}阶复矩阵\end{CJK} $A$ \begin{CJK}{UTF8}{mj}的极小多项式是\end{CJK} $(x-\lambda)^{m}$, \begin{CJK}{UTF8}{mj}其中\end{CJK} $\lambda$ \begin{CJK}{UTF8}{mj}是常数\end{CJK},

\end{enumerate}
(a) \begin{CJK}{UTF8}{mj}若\end{CJK} $n=4, m=2$, \begin{CJK}{UTF8}{mj}求\end{CJK} $A$ \begin{CJK}{UTF8}{mj}所有可能的\end{CJK} Jordan \begin{CJK}{UTF8}{mj}标准型并说明理由\end{CJK};

(b) \begin{CJK}{UTF8}{mj}证明\end{CJK}: \begin{CJK}{UTF8}{mj}存在对角阵\end{CJK} $D$, \begin{CJK}{UTF8}{mj}使得\end{CJK} $(A-D)^{m}=O$.

\begin{enumerate}
  \setcounter{enumi}{7}
  \item \begin{CJK}{UTF8}{mj}设\end{CJK} $V$ \begin{CJK}{UTF8}{mj}是全体次数不超过\end{CJK} 3 \begin{CJK}{UTF8}{mj}的多项式构成的线性空间\end{CJK}, \begin{CJK}{UTF8}{mj}定义\end{CJK} $V$ \begin{CJK}{UTF8}{mj}上的线性变换\end{CJK}
\end{enumerate}
$$
V, f(x) \mapsto f^{\prime}(x)+f(x+1)
$$
(a) \begin{CJK}{UTF8}{mj}证明\end{CJK}: $\varphi$ \begin{CJK}{UTF8}{mj}是可逆线性变换\end{CJK}, \begin{CJK}{UTF8}{mj}并求\end{CJK} $\varphi^{-1}\left(1+x+x^{2}+x^{3}\right)$;

(b) \begin{CJK}{UTF8}{mj}问\end{CJK} $\varphi$ \begin{CJK}{UTF8}{mj}是否可对角化\end{CJK}, \begin{CJK}{UTF8}{mj}并莌明理由\end{CJK}.

\begin{enumerate}
  \setcounter{enumi}{8}
  \item \begin{CJK}{UTF8}{mj}已知\end{CJK} $\varphi, \psi$ \begin{CJK}{UTF8}{mj}是复线性穴间\end{CJK} $V$ \begin{CJK}{UTF8}{mj}上的线性变换\end{CJK}, \begin{CJK}{UTF8}{mj}且\end{CJK} $\varphi \psi=\psi \varphi$, \begin{CJK}{UTF8}{mj}证明\end{CJK}:
\end{enumerate}
(a) \begin{CJK}{UTF8}{mj}证明\end{CJK}: $\varphi$ - \begin{CJK}{UTF8}{mj}特征子坴间是\end{CJK} $\psi-$ \begin{CJK}{UTF8}{mj}不变子空间\end{CJK};

(b) \begin{CJK}{UTF8}{mj}证明\end{CJK}: \begin{CJK}{UTF8}{mj}若\end{CJK} $\psi$-\begin{CJK}{UTF8}{mj}不变子空间只有零空间和其自身\end{CJK},\begin{CJK}{UTF8}{mj}则\end{CJK} $\varphi$ \begin{CJK}{UTF8}{mj}是数乘变换\end{CJK}.

9.(a) \begin{CJK}{UTF8}{mj}已知\end{CJK} $W_{1}, W_{2}, \cdots, W_{m}$ \begin{CJK}{UTF8}{mj}是\end{CJK} $V$ \begin{CJK}{UTF8}{mj}的\end{CJK} $m$ \begin{CJK}{UTF8}{mj}个直子空间\end{CJK}, \begin{CJK}{UTF8}{mj}则\end{CJK} $V \neq \bigcup_{i=1}^{m} W_{i}$;

(b)\begin{CJK}{UTF8}{mj}已知\end{CJK} $v_{1}, v_{2}, \cdots, v_{m}$ \begin{CJK}{UTF8}{mj}是\end{CJK} $V$ \begin{CJK}{UTF8}{mj}中的\end{CJK} $m$ \begin{CJK}{UTF8}{mj}个向荲\end{CJK}, \begin{CJK}{UTF8}{mj}证明\end{CJK}: \begin{CJK}{UTF8}{mj}存在\end{CJK} $v_{0} \in V$, \begin{CJK}{UTF8}{mj}使得\end{CJK} $\left(v_{0}, v_{i}\right) \neq 0(i=1,2, \cdots, m)$;

(c) \begin{CJK}{UTF8}{mj}已知\end{CJK} $f_{1}, f_{2}, \cdots, f_{m}$ \begin{CJK}{UTF8}{mj}是\end{CJK} $V$ \begin{CJK}{UTF8}{mj}上的两两不同的线性函数\end{CJK}, \begin{CJK}{UTF8}{mj}证明\end{CJK}: \begin{CJK}{UTF8}{mj}存在\end{CJK} $v_{0} \in V$, \begin{CJK}{UTF8}{mj}使得\end{CJK} $f_{1}\left(v_{0}\right), f_{2}\left(v_{0}\right), \cdots$, \begin{CJK}{UTF8}{mj}两不司\end{CJK}.

\begin{enumerate}
  \setcounter{enumi}{10}
  \item \begin{CJK}{UTF8}{mj}已知实矩阵\end{CJK} $A$, \begin{CJK}{UTF8}{mj}证明\end{CJK}: $\operatorname{tr}\left(A^{\prime} A\right)=0$ \begin{CJK}{UTF8}{mj}当且仅当\end{CJK} $A=O$, \begin{CJK}{UTF8}{mj}并且右\end{CJK} $A^{\prime} A=A^{2}$, \begin{CJK}{UTF8}{mj}则\end{CJK} $A=A^{\prime}$.
\end{enumerate}
\section{$3.2$ 数学分析}
\begin{enumerate}
  \item \begin{CJK}{UTF8}{mj}求下列极限\end{CJK}:
\end{enumerate}
(a) $\lim _{(x, y) \rightarrow(0,0)} x y \frac{3 x-4 y}{x^{2}+y^{2}}$;

(b) $\lim _{x \rightarrow 1} \frac{x^{k}-1}{x^{\lambda x}-1}$, \begin{CJK}{UTF8}{mj}其中\end{CJK} $\lambda$ \begin{CJK}{UTF8}{mj}是常数\end{CJK}, $k$ \begin{CJK}{UTF8}{mj}是正整数\end{CJK}. 5. \begin{CJK}{UTF8}{mj}已知\end{CJK} $f(x)$ \begin{CJK}{UTF8}{mj}在\end{CJK} $[a, b]$ \begin{CJK}{UTF8}{mj}上一阶连续可导\end{CJK}, $f(a)=f(b)=0, f^{\prime}(a) \cdot f^{\prime}(b)>0$, \begin{CJK}{UTF8}{mj}证明\end{CJK}:\begin{CJK}{UTF8}{mj}存在\end{CJK} $x_{1}, x_{2}, x_{3} \in(a$ $f\left(x_{1}\right)=0, f^{\prime}\left(x_{2}\right)=0, f^{\prime \prime}\left(x_{3}\right)=0 .$

6.(a) \begin{CJK}{UTF8}{mj}证明函数项级数\end{CJK} $\sum_{n=1}^{\infty} n \mathrm{e}^{-n x}$ \begin{CJK}{UTF8}{mj}在\end{CJK} $(0,+\infty)$ \begin{CJK}{UTF8}{mj}上收敛但不一致收敛\end{CJK};

(b) \begin{CJK}{UTF8}{mj}求\end{CJK} (1) \begin{CJK}{UTF8}{mj}中函数项级数的和函数\end{CJK};

(c) \begin{CJK}{UTF8}{mj}求\end{CJK} $\sum_{n=1}^{\infty} \frac{n}{\mathrm{e}^{3 n}}$ \begin{CJK}{UTF8}{mj}的值\end{CJK}.

\begin{enumerate}
  \setcounter{enumi}{7}
  \item \begin{CJK}{UTF8}{mj}已知\end{CJK} $g(x)$ \begin{CJK}{UTF8}{mj}是定义在\end{CJK} $(0, r)$ \begin{CJK}{UTF8}{mj}上的三阶可导函数\end{CJK}, \begin{CJK}{UTF8}{mj}其中\end{CJK} $g(0)=1$, \begin{CJK}{UTF8}{mj}定义\end{CJK}
\end{enumerate}
$$
f(r)=\iiint_{x^{2}+y^{2}+z^{2} \leq r^{2}} g\left(x^{2}+y^{2}+z^{2}\right) \mathrm{d} x \mathrm{~d} y \mathrm{~d} z
$$
\begin{CJK}{UTF8}{mj}证明\end{CJK}: $f(x)$ \begin{CJK}{UTF8}{mj}在\end{CJK} $x=0$ \begin{CJK}{UTF8}{mj}处三阶可导\end{CJK}, \begin{CJK}{UTF8}{mj}㚔求\end{CJK} $f_{+}^{\prime \prime \prime}(0)$.

\begin{enumerate}
  \setcounter{enumi}{8}
  \item \begin{CJK}{UTF8}{mj}求积分\end{CJK} $\int_{0}^{\pi} \ln \left(1-2 a \cos x+x^{2}\right) \mathrm{d} x$ \begin{CJK}{UTF8}{mj}的值\end{CJK}.

  \item \begin{CJK}{UTF8}{mj}已知\end{CJK} $\left\{a_{n}\right\}$ \begin{CJK}{UTF8}{mj}是一实数列\end{CJK} $0<|\lambda|<1$, \begin{CJK}{UTF8}{mj}证明\end{CJK}: $\lim _{n \rightarrow \infty} a_{n}=a$ \begin{CJK}{UTF8}{mj}的充分必要条件是\end{CJK} $\lim _{n \rightarrow \infty}\left(a_{n+1}-\lambda a_{n}\right)=(1-$, 10. \begin{CJK}{UTF8}{mj}证明级数\end{CJK} $\sum_{n=1}^{\infty} \frac{1+\frac{1}{2}+\cdots+\frac{1}{n}}{n(n+2)}$ \begin{CJK}{UTF8}{mj}收敛并求其值\end{CJK}. 1.\begin{CJK}{UTF8}{mj}计算\end{CJK}

\end{enumerate}
$$
\lim _{x \rightarrow 0} \frac{2 \sqrt{1+x}-x-2 \cos x}{x \mathrm{e}^{x}-\ln (1+x)}
$$

\begin{enumerate}
  \setcounter{enumi}{2}
  \item \begin{CJK}{UTF8}{mj}已知\end{CJK} $a_{n}=\int_{0}^{n^{2} \pi}(|\sin t|+|\cos t|) \mathrm{d} t, b_{n}=\int_{0}^{\frac{1}{n}} \mathrm{e}^{-t^{2}} \sin t \mathrm{~d} t$, \begin{CJK}{UTF8}{mj}求\end{CJK} $\lim _{n \rightarrow \infty} a_{n} b_{n}$.

  \item \begin{CJK}{UTF8}{mj}已知\end{CJK} $f(x)$ \begin{CJK}{UTF8}{mj}可微且\end{CJK} $f^{(n+i)}(0)=0, f^{(n+k)}(0) \neq 0, i=1,2, \cdots, k-1$.

\end{enumerate}
$$
f(x)=f(0)+f^{\prime}(0) x+\cdots+\frac{f^{(n-1)}(0)}{(n-1) !} x^{n-1}+\frac{f^{(n)}(\theta x)}{n !} x^{n}
$$
\begin{CJK}{UTF8}{mj}䇩\end{CJK} $\lim _{x \rightarrow 0} \theta$.

\begin{enumerate}
  \setcounter{enumi}{4}
  \item \begin{CJK}{UTF8}{mj}已知积分中值定理\end{CJK}: $\int_{a}^{b} f(x) g(x) \mathrm{d} x=f^{\prime}(\xi) \int_{a}^{b} g(x) \mathrm{d} x, \xi \in(a, b)$. \begin{CJK}{UTF8}{mj}这里\end{CJK} $f(x), g(x)$ \begin{CJK}{UTF8}{mj}满足积分中值衣\end{CJK} \begin{CJK}{UTF8}{mj}且\end{CJK} $f_{+}^{\prime}(a)$ \begin{CJK}{UTF8}{mj}存在且不为\end{CJK} 0 , \begin{CJK}{UTF8}{mj}求\end{CJK} $\lim _{x \rightarrow a^{+}} \frac{\xi-a}{x-a}$.

  \item \begin{CJK}{UTF8}{mj}已知\end{CJK} $V$ \begin{CJK}{UTF8}{mj}是由三个坐标平面以及\end{CJK} $x+y+2 z=1, x+y+2 z=2$ \begin{CJK}{UTF8}{mj}围成的封闭区域\end{CJK}. \begin{CJK}{UTF8}{mj}求\end{CJK} $\iiint_{V} \frac{\mathrm{d} V}{(x+y+2 z)^{3}}$

  \item \begin{CJK}{UTF8}{mj}已知\end{CJK} $S:\left\{(x, y, z) \mid x^{2}+4 y^{2}+9 z^{2}=1, z \leq 0\right\}$, \begin{CJK}{UTF8}{mj}取下侧\end{CJK}.\begin{CJK}{UTF8}{mj}求\end{CJK} $\iint_{S}[] \mathrm{d} y \mathrm{~d} z+[] \mathrm{d} z \mathrm{~d} x+[] \mathrm{d} x \mathrm{~d} y$.

  \item \begin{CJK}{UTF8}{mj}已知\end{CJK} $f(x)$ \begin{CJK}{UTF8}{mj}在\end{CJK} $[a, b]$ \begin{CJK}{UTF8}{mj}上三次可微\end{CJK}, \begin{CJK}{UTF8}{mj}且\end{CJK} $f(a)=f^{\prime}(a)=f(b)=0,\left|f^{\prime \prime \prime}(x)\right| \leq M$, \begin{CJK}{UTF8}{mj}证明\end{CJK}: $\left|\int_{a}^{b} f(x) \mathrm{d} x\right| \leq \frac{M}{7}$ ?

  \item \begin{CJK}{UTF8}{mj}已知含参变量积分\end{CJK} $\int_{e^{3}}^{\infty} \frac{\sin (x y)}{\ln (\ln y)} \mathrm{d} y$. \begin{CJK}{UTF8}{mj}证明\end{CJK}:

\end{enumerate}
(1) \begin{CJK}{UTF8}{mj}其在\end{CJK} $[\delta,+\infty)$ \begin{CJK}{UTF8}{mj}上关于\end{CJK} $x$ \begin{CJK}{UTF8}{mj}一致收敛\end{CJK} $(\delta>0)$.

(2) \begin{CJK}{UTF8}{mj}其在\end{CJK} $(0,+\infty)$ \begin{CJK}{UTF8}{mj}上关于\end{CJK} $x$ \begin{CJK}{UTF8}{mj}不一致抜敛\end{CJK}.\\
9. \begin{CJK}{UTF8}{mj}已知\end{CJK} $\left\{u_{n}(x)\right\}$ \begin{CJK}{UTF8}{mj}是可微函数序列\end{CJK}、\begin{CJK}{UTF8}{mj}至\end{CJK} $\sum_{n=1}^{\infty} u_{n}^{\prime}(x)$ \begin{CJK}{UTF8}{mj}在\end{CJK} $[a, b]$ \begin{CJK}{UTF8}{mj}上一致有界\end{CJK}. \begin{CJK}{UTF8}{mj}证明\end{CJK}: \begin{CJK}{UTF8}{mj}若\end{CJK} $\sum_{n=1}^{\infty} u_{n}(x)$ \begin{CJK}{UTF8}{mj}收敛\end{CJK},\begin{CJK}{UTF8}{mj}则\end{CJK} $\sum_{n=1}^{\infty}$ \begin{CJK}{UTF8}{mj}定一致收敛\end{CJK}.

\begin{enumerate}
  \setcounter{enumi}{10}
  \item \begin{CJK}{UTF8}{mj}已知\end{CJK} $I_{p}(u)=\left(\ln \iiint_{\Omega} u^{p} \mathrm{~d} x \mathrm{~d} y \mathrm{~d} z\right)^{p}, \Omega:[0,1] \times[0,1] \times[0,1]$. \begin{CJK}{UTF8}{mj}正明\end{CJK}: $\lim _{p \rightarrow 0} I_{p}(u)=\mathrm{e}^{\int} \int^{2}$
\end{enumerate}
\section{$4.2$ 高等代数}
\begin{enumerate}
  \item \begin{CJK}{UTF8}{mj}已知\end{CJK} $A=\left(a_{i j}\right)_{n \times n}, a_{i j}=-a_{j i}, \forall b$, \begin{CJK}{UTF8}{mj}求\end{CJK}
\end{enumerate}
$$
\left|\begin{array}{cccc}
a_{11}+b & a_{12}+b & \cdots & a_{1 n}+b \\
a_{21}+b & a_{22}+b & \cdots & a_{2 n}+b \\
\vdots & \vdots & & \vdots \\
a_{n 1}+b & a_{n 2}+b & \cdots & a_{n n}+b
\end{array}\right|
$$

\begin{enumerate}
  \setcounter{enumi}{2}
  \item \begin{CJK}{UTF8}{mj}已知\end{CJK} $\mathrm{r}(A B)=\mathrm{r}(B)$, \begin{CJK}{UTF8}{mj}证明\end{CJK}: $\mathrm{r}(A B C)=\mathrm{r}(B C)$.

  \item \begin{CJK}{UTF8}{mj}求\end{CJK} $A=\left(\begin{array}{ccc}2 & 6 & -15 \\ 1 & 1 & -5 \\ 1 & 2 & -6\end{array}\right)$ \begin{CJK}{UTF8}{mj}的\end{CJK} Jordan \begin{CJK}{UTF8}{mj}标准型\end{CJK}. (1) \begin{CJK}{UTF8}{mj}若\end{CJK} $B C=O$, \begin{CJK}{UTF8}{mj}则\end{CJK} $B=O$.

\end{enumerate}
(2) \begin{CJK}{UTF8}{mj}若\end{CJK} $B C=C$, \begin{CJK}{UTF8}{mj}则\end{CJK} $B=I$.

\begin{enumerate}
  \setcounter{enumi}{6}
  \item $f_{1}(x), f_{2}(x)$ \begin{CJK}{UTF8}{mj}是两个省项系数为\end{CJK} 1 \begin{CJK}{UTF8}{mj}的互异多项式\end{CJK}, $x^{4}+x^{2}+1$ \begin{CJK}{UTF8}{mj}敕除\end{CJK} $f_{1}\left(x^{3}\right)+x^{4} f_{2}\left(x^{3}\right)$, \begin{CJK}{UTF8}{mj}求\end{CJK} $f_{1}(x)$ \begin{CJK}{UTF8}{mj}和\end{CJK} $f$ \begin{CJK}{UTF8}{mj}大公式\end{CJK}.

  \item \begin{CJK}{UTF8}{mj}设\end{CJK} $A$ \begin{CJK}{UTF8}{mj}是\end{CJK} $n$ \begin{CJK}{UTF8}{mj}阶对称矩阵\end{CJK}, $A$ \begin{CJK}{UTF8}{mj}的秩为\end{CJK} $r$. \begin{CJK}{UTF8}{mj}证明\end{CJK}: \begin{CJK}{UTF8}{mj}存在秩为\end{CJK} $n-r$ \begin{CJK}{UTF8}{mj}的对称阵\end{CJK} $B$, \begin{CJK}{UTF8}{mj}使\end{CJK} $A B=O$.

  \item \begin{CJK}{UTF8}{mj}已知\end{CJK} $A, B$ \begin{CJK}{UTF8}{mj}为\end{CJK} $n$ \begin{CJK}{UTF8}{mj}阶方阵\end{CJK}, \begin{CJK}{UTF8}{mj}且\end{CJK} $A, B$ \begin{CJK}{UTF8}{mj}都可对角化\end{CJK}, $A B=B A$, \begin{CJK}{UTF8}{mj}证明\end{CJK}: $A, B$ \begin{CJK}{UTF8}{mj}可同时对角化\end{CJK}.

  \item \begin{CJK}{UTF8}{mj}设\end{CJK} $A$ \begin{CJK}{UTF8}{mj}是非奇异矩阵\end{CJK}, \begin{CJK}{UTF8}{mj}证明\end{CJK}: \begin{CJK}{UTF8}{mj}存在正交矩阵\end{CJK} $Q_{1}, Q_{2}$, \begin{CJK}{UTF8}{mj}使得\end{CJK} $Q_{1} A Q_{2}=\left(\begin{array}{c}a_{1} \\ a_{2}\end{array}\right), a_{i}>0, i=1$, 10. \begin{CJK}{UTF8}{mj}已知\end{CJK} $A, B$ \begin{CJK}{UTF8}{mj}是\end{CJK} $n$ \begin{CJK}{UTF8}{mj}阶矩阵\end{CJK}, $B$ \begin{CJK}{UTF8}{mj}半正定\end{CJK}, \begin{CJK}{UTF8}{mj}且存在\end{CJK} $m$, \begin{CJK}{UTF8}{mj}使得\end{CJK} $B^{m} A=A B^{m}$, \begin{CJK}{UTF8}{mj}证明\end{CJK} $A B=B A$. 1.\begin{CJK}{UTF8}{mj}计算\end{CJK}

\end{enumerate}
$$
\int_{0}^{\frac{\pi}{4}} \frac{\sin x \cos x}{\sin ^{4} x+\cos ^{4} x-5} d x
$$

\begin{enumerate}
  \setcounter{enumi}{2}
  \item \begin{CJK}{UTF8}{mj}计算\end{CJK}
\end{enumerate}
$$
\iint_{D} \sin \left\{\max \left(x^{2}, y^{2}\right)\right\} \mathrm{d} x \mathrm{~d} y
$$
\begin{CJK}{UTF8}{mj}其中\end{CJK} $D:\{(x, y) \mid 0 \leq x \leq \sqrt{2}, 0 \leq y \leq \sqrt{2}\}$.

\begin{enumerate}
  \setcounter{enumi}{3}
  \item \begin{CJK}{UTF8}{mj}计算级数\end{CJK}
\end{enumerate}
$$
\sum_{n=1}^{\infty} \frac{(-1)^{n-1}}{n(2 n+1)}
$$

\begin{enumerate}
  \setcounter{enumi}{4}
  \item \begin{CJK}{UTF8}{mj}判䉼隐函数极小值\end{CJK}.

  \item $f(x)$ \begin{CJK}{UTF8}{mj}不可导\end{CJK}, \begin{CJK}{UTF8}{mj}下凸\end{CJK}, \begin{CJK}{UTF8}{mj}证明\end{CJK}: \begin{CJK}{UTF8}{mj}对任意\end{CJK} $x \in \mathbb{R}, f\left(x+f^{\prime}(x)\right) \geq f(x)$.

  \item \begin{CJK}{UTF8}{mj}已知\end{CJK} $x_{n}=\sum_{k=0}^{n} \frac{1}{k !}$, \begin{CJK}{UTF8}{mj}求\end{CJK} $\lim _{n \rightarrow \infty}\left(\frac{\ln x_{n}}{\sqrt[n]{e}-1}-n\right)$.

  \item \begin{CJK}{UTF8}{mj}若\end{CJK} $f(x)$ \begin{CJK}{UTF8}{mj}与\end{CJK} $g(x)$ \begin{CJK}{UTF8}{mj}均在\end{CJK} $[a, b]$ \begin{CJK}{UTF8}{mj}上可积\end{CJK}, \begin{CJK}{UTF8}{mj}证明\end{CJK}: \begin{CJK}{UTF8}{mj}在\end{CJK} $(a, b)$ \begin{CJK}{UTF8}{mj}上存在一点\end{CJK} $x_{0}$, \begin{CJK}{UTF8}{mj}使得\end{CJK} $f(x)$ \begin{CJK}{UTF8}{mj}与\end{CJK} $g(x)$ \begin{CJK}{UTF8}{mj}在\end{CJK} $x_{0}$ \begin{CJK}{UTF8}{mj}处连\end{CJK}:

\end{enumerate}
\section{$5.2$ 高等代数}
\begin{enumerate}
  \item \begin{CJK}{UTF8}{mj}计算行列式\end{CJK}
\end{enumerate}
\includegraphics[max width=\textwidth]{2022_04_18_7db0708508f26638f054g-181}
$$
\begin{aligned}
&\text { 2. } A=\left(\begin{array}{ccc}
2 & -1 & 0 \\
1 & -1 & 1 \\
-1 & -1 & 1
\end{array}\right), \text { 戊 } \operatorname{tr}\left(A^{100}\right) . \\
&\text { 3. } A: \mathbb{R}[x]_{4} \rightarrow \mathbb{R}[x]_{4}, \mathscr{A}(f(x))=f(x)-f(0)+f^{\prime}(x) .
\end{aligned}
$$
(1)\begin{CJK}{UTF8}{mj}求\end{CJK} $\mathscr{A}$ \begin{CJK}{UTF8}{mj}在基\end{CJK} $1, x, x^{2}, x^{3}$ \begin{CJK}{UTF8}{mj}下的矩阵\end{CJK}.

(2) \begin{CJK}{UTF8}{mj}求\end{CJK} $\mathscr{A}$ \begin{CJK}{UTF8}{mj}的特征值与特征向荲\end{CJK}.

$4 . n>1, n$ \begin{CJK}{UTF8}{mj}为奇数\end{CJK}, $A \in \mathbb{R}^{n \times n}, A^{*}=A^{\prime}$. \begin{CJK}{UTF8}{mj}求\end{CJK} $\left|A^{2}-A\right|$.

\begin{enumerate}
  \setcounter{enumi}{5}
  \item $A, B \in \mathbb{R}^{n \times n}, A$ \begin{CJK}{UTF8}{mj}正定\end{CJK}, $B$ \begin{CJK}{UTF8}{mj}为反对称矩阵\end{CJK}, \begin{CJK}{UTF8}{mj}证明\end{CJK} $|A+B|>0$.

  \item $f \in \mathbb{R}[x], f$ \begin{CJK}{UTF8}{mj}的次数大于\end{CJK} 1 , \begin{CJK}{UTF8}{mj}证明\end{CJK}: \begin{CJK}{UTF8}{mj}存在非零的\end{CJK} $g(x) \in \mathbb{R}[x]$, \begin{CJK}{UTF8}{mj}使得\end{CJK} $f(x) \mid g\left(x^{8}\right)$.

  \item \begin{CJK}{UTF8}{mj}向旦组\end{CJK} $\alpha_{1}, \alpha_{2}, \cdots, \alpha_{r}$ \begin{CJK}{UTF8}{mj}与\end{CJK} $\beta_{1}, \beta_{2}, \cdots, \beta_{s}$ \begin{CJK}{UTF8}{mj}的秩分别为\end{CJK} $R_{1}$ \begin{CJK}{UTF8}{mj}和\end{CJK} $R_{2}, C=\left(c_{i j}\right) \in \mathbb{R}^{s \times r}, B_{j}=\sum_{i=1}^{r} c_{i j} \alpha_{i}(1$ \begin{CJK}{UTF8}{mj}证明\end{CJK}: $\mathrm{r}(\mathrm{C}) \leq R_{1}-R_{2}+r$.

  \item $A, B \in \mathbb{R}^{n \times n}$ \begin{CJK}{UTF8}{mj}且\end{CJK} $\mathrm{r}(A B A)=\mathrm{r}(B)$. \begin{CJK}{UTF8}{mj}证明\end{CJK}: $A B$ \begin{CJK}{UTF8}{mj}与\end{CJK} $B A$ \begin{CJK}{UTF8}{mj}相似\end{CJK}. 1.(20 \begin{CJK}{UTF8}{mj}分\end{CJK}) \begin{CJK}{UTF8}{mj}考虑数域\end{CJK} $\mathbb{K}$ \begin{CJK}{UTF8}{mj}上的线性方程组\end{CJK}

\end{enumerate}
$$
\left\{\begin{array}{l}
x_{1}+x_{2}+2 a x_{3}=2 \\
x_{1}+3 b x_{2}+x_{3}=2 \\
x_{1}+x_{2}-a x_{3}=1
\end{array}\right.
$$
\begin{CJK}{UTF8}{mj}问在\end{CJK} $a, b$ \begin{CJK}{UTF8}{mj}取何值时\end{CJK}, \begin{CJK}{UTF8}{mj}方程组无解\end{CJK}, \begin{CJK}{UTF8}{mj}有唯一解\end{CJK}, \begin{CJK}{UTF8}{mj}有无穷多组解\end{CJK}. \begin{CJK}{UTF8}{mj}且在方程组有解时\end{CJK}, \begin{CJK}{UTF8}{mj}求出所有解\end{CJK}.

2.(20 \begin{CJK}{UTF8}{mj}分\end{CJK}) \begin{CJK}{UTF8}{mj}设\end{CJK} 3 \begin{CJK}{UTF8}{mj}阶实对称矩阵\end{CJK} $A$ \begin{CJK}{UTF8}{mj}的秩为\end{CJK} 2 , \begin{CJK}{UTF8}{mj}且\end{CJK} $-2$ \begin{CJK}{UTF8}{mj}是它的二重特征值\end{CJK}, \begin{CJK}{UTF8}{mj}若\end{CJK} $(1,0,0)^{\prime},(2,1,1)^{\prime}$ \begin{CJK}{UTF8}{mj}都是\end{CJK} $A$ \begin{CJK}{UTF8}{mj}的\end{CJK} \begin{CJK}{UTF8}{mj}值\end{CJK} $-2$ \begin{CJK}{UTF8}{mj}的特征向量\end{CJK}, \begin{CJK}{UTF8}{mj}求矩阵\end{CJK} $A$.

3.(20 \begin{CJK}{UTF8}{mj}分\end{CJK}) \begin{CJK}{UTF8}{mj}考虑末定元为\end{CJK} $x$ \begin{CJK}{UTF8}{mj}和\end{CJK} $y$ \begin{CJK}{UTF8}{mj}的次数至多为\end{CJK} 2 \begin{CJK}{UTF8}{mj}的复系数二元多项式空间\end{CJK}. \begin{CJK}{UTF8}{mj}求线性变换\end{CJK} $\mathscr{A}: f$ $f(2 x+1,2 y+1)$ \begin{CJK}{UTF8}{mj}的\end{CJK} Jordan \begin{CJK}{UTF8}{mj}标准型\end{CJK}.

4.(15 \begin{CJK}{UTF8}{mj}分\end{CJK}) \begin{CJK}{UTF8}{mj}设\end{CJK} $\sigma$ \begin{CJK}{UTF8}{mj}是有限维欧式空间\end{CJK} $V$ \begin{CJK}{UTF8}{mj}上的正交变换\end{CJK}. \begin{CJK}{UTF8}{mj}且满足\end{CJK} $\sigma^{m}=I$. \begin{CJK}{UTF8}{mj}这里\end{CJK} $m>1$ \begin{CJK}{UTF8}{mj}是整数\end{CJK}, $I$ \begin{CJK}{UTF8}{mj}是恒等\end{CJK} $V^{\sigma}=\{\theta \in V: \sigma(v)=v\}$, \begin{CJK}{UTF8}{mj}而\end{CJK} $V^{\sigma}$ \begin{CJK}{UTF8}{mj}的正交补记为\end{CJK} $V^{\sigma}$.

(a). \begin{CJK}{UTF8}{mj}求证\end{CJK} $V^{\sigma \perp}$ \begin{CJK}{UTF8}{mj}是\end{CJK} $\sigma$ - \begin{CJK}{UTF8}{mj}不变子空间\end{CJK}.

(b). \begin{CJK}{UTF8}{mj}对于\end{CJK} $v \in V$, \begin{CJK}{UTF8}{mj}定义\end{CJK} $\bar{v}=\frac{1}{m} \sum_{i=1}^{m} \sigma^{i}(v)$. \begin{CJK}{UTF8}{mj}求证\end{CJK}: $\bar{v} \in V^{\sigma}$.

(c). \begin{CJK}{UTF8}{mj}证明若将\end{CJK} $v \in V$ \begin{CJK}{UTF8}{mj}展开成\end{CJK} $v=v_{1}+v_{2}$, \begin{CJK}{UTF8}{mj}其中\end{CJK} $v_{1} \in V^{\sigma}, v_{2} \in V^{\sigma}$. \begin{CJK}{UTF8}{mj}则\end{CJK} $v_{1}=\bar{v}$.

5.(10 \begin{CJK}{UTF8}{mj}分\end{CJK}). \begin{CJK}{UTF8}{mj}设\end{CJK} $f(x)$ \begin{CJK}{UTF8}{mj}是次数大于\end{CJK} 0 \begin{CJK}{UTF8}{mj}的整系数多项式\end{CJK}, \begin{CJK}{UTF8}{mj}若\end{CJK} $2-\sqrt{3}$ \begin{CJK}{UTF8}{mj}是\end{CJK} $f(x)$ \begin{CJK}{UTF8}{mj}的根\end{CJK}, \begin{CJK}{UTF8}{mj}证明\end{CJK} $2+\sqrt{3}$ \begin{CJK}{UTF8}{mj}也是\end{CJK} $f(x)$ \begin{CJK}{UTF8}{mj}的杻\end{CJK} 6. (15 \begin{CJK}{UTF8}{mj}分\end{CJK}). \begin{CJK}{UTF8}{mj}设\end{CJK} $V$ \begin{CJK}{UTF8}{mj}是复数域\end{CJK} $\mathbb{K}$ \begin{CJK}{UTF8}{mj}上的\end{CJK} $n$ \begin{CJK}{UTF8}{mj}维线性空间\end{CJK}, 4 \begin{CJK}{UTF8}{mj}是\end{CJK} $V$ \begin{CJK}{UTF8}{mj}上的线性变换\end{CJK}.

(a). \begin{CJK}{UTF8}{mj}证明存在正整数\end{CJK} $k \leq n$, \begin{CJK}{UTF8}{mj}使得\end{CJK} $\operatorname{Im}\left(\mathscr{A}^{k}\right)=\operatorname{Im}\left(\mathscr{A}^{k+1}\right)=\cdots=\operatorname{Im}\left(\mathscr{A}^{n}\right)$ \begin{CJK}{UTF8}{mj}且\end{CJK} $\operatorname{ker}\left(\mathscr{A}^{k}\right)=\operatorname{ker}\left(\mathscr{A}^{k+1}\right.$ $\operatorname{ker}\left(\mathscr{A}^{n}\right)$.

(b). \begin{CJK}{UTF8}{mj}考虑如下数值指标\end{CJK}:

\begin{CJK}{UTF8}{mj}粌\end{CJK}Jordan \begin{CJK}{UTF8}{mj}块个数\end{CJK} $m$.

(i) $\mathscr{A}$ \begin{CJK}{UTF8}{mj}的秩\end{CJK} $r$. (ii) $\mathscr{A}$ \begin{CJK}{UTF8}{mj}的特征值为\end{CJK} 0 \begin{CJK}{UTF8}{mj}的\end{CJK} Jordan\begin{CJK}{UTF8}{mj}块个数\end{CJK} $m$.

(iii) $\mathscr{A}$ \begin{CJK}{UTF8}{mj}的特征值为\end{CJK} 0 \begin{CJK}{UTF8}{mj}的\end{CJK} Jordan \begin{CJK}{UTF8}{mj}㘦的最大阶数\end{CJK} $n$. (iv)(a) \begin{CJK}{UTF8}{mj}中岕现最小的\end{CJK} $k$.

\begin{CJK}{UTF8}{mj}讨论这些数值之间的关系\end{CJK}, \begin{CJK}{UTF8}{mj}并证明你的结论\end{CJK}.

7.(15 \begin{CJK}{UTF8}{mj}分\end{CJK}) \begin{CJK}{UTF8}{mj}设\end{CJK} $V$ \begin{CJK}{UTF8}{mj}是实内积空间\end{CJK},$<,>$ \begin{CJK}{UTF8}{mj}是\end{CJK} $V$ \begin{CJK}{UTF8}{mj}上的内积\end{CJK}, $\phi$ \begin{CJK}{UTF8}{mj}是\end{CJK} $V$ \begin{CJK}{UTF8}{mj}上的可逆线性变换\end{CJK}. \begin{CJK}{UTF8}{mj}且满足\end{CJK} $<\phi(\phi$ $\langle x, \phi(y)\rangle, \forall x, y \in V$. \begin{CJK}{UTF8}{mj}证明\end{CJK} $\phi$ \begin{CJK}{UTF8}{mj}是正交变换\end{CJK}.

\begin{enumerate}
  \setcounter{enumi}{8}
  \item (15 \begin{CJK}{UTF8}{mj}分\end{CJK}). \begin{CJK}{UTF8}{mj}设\end{CJK} $U, V, W$ \begin{CJK}{UTF8}{mj}是\end{CJK} 6 \begin{CJK}{UTF8}{mj}维线性空间的\end{CJK} 3 \begin{CJK}{UTF8}{mj}个\end{CJK} 3 \begin{CJK}{UTF8}{mj}维子空间\end{CJK}, \begin{CJK}{UTF8}{mj}设\end{CJK} $U \cap V=0$, \begin{CJK}{UTF8}{mj}求\end{CJK} $\operatorname{dim}((U+V) \cap(V+W))$ \begin{CJK}{UTF8}{mj}和最小值\end{CJK}.

  \item (20 \begin{CJK}{UTF8}{mj}分\end{CJK}). (a). \begin{CJK}{UTF8}{mj}设\end{CJK} $A \in M_{n}(\mathbb{R})$ \begin{CJK}{UTF8}{mj}是半正定对称矩阵\end{CJK}, $x \in \mathbb{R}^{n}$. \begin{CJK}{UTF8}{mj}证明\end{CJK} $x^{T} A x=0$ \begin{CJK}{UTF8}{mj}等价于\end{CJK} $A x=0$.

\end{enumerate}
(b). \begin{CJK}{UTF8}{mj}设\end{CJK} $A$ \begin{CJK}{UTF8}{mj}是\end{CJK} $n$ \begin{CJK}{UTF8}{mj}阶半正定对称矩阵\end{CJK}, \begin{CJK}{UTF8}{mj}将其写成分块矩阵的形式\end{CJK}
$$
A=\left(\begin{array}{ll}
A_{1} & A_{2} \\
A_{2}^{T} & A_{4}
\end{array}\right) .
$$
\begin{CJK}{UTF8}{mj}其中\end{CJK} $A_{1}$ \begin{CJK}{UTF8}{mj}是\end{CJK} $r$ \begin{CJK}{UTF8}{mj}阶方阵\end{CJK}. \begin{CJK}{UTF8}{mj}证明对\end{CJK} $x \in \mathbb{R}^{r}$, \begin{CJK}{UTF8}{mj}若\end{CJK} $A_{1} x=0$, \begin{CJK}{UTF8}{mj}则\end{CJK} $A_{2}^{T} x=0$.

(c). \begin{CJK}{UTF8}{mj}设\end{CJK} A, B \begin{CJK}{UTF8}{mj}是\end{CJK} $n$ \begin{CJK}{UTF8}{mj}阶半正定实对称矩阵\end{CJK}, \begin{CJK}{UTF8}{mj}且\end{CJK} $\operatorname{rank}(A)=r$. \begin{CJK}{UTF8}{mj}证明存在\end{CJK} $n$ \begin{CJK}{UTF8}{mj}阶可逆矩阵\end{CJK} $P$, \begin{CJK}{UTF8}{mj}使得\end{CJK}:
$$
P^{-1} A\left(P^{-1}\right)^{T}=\left(\begin{array}{cc}
I_{r} & 0 \\
0 & 0
\end{array}\right), P^{T} B P=\operatorname{diag}\left\{\lambda_{1}, \lambda_{2}, \cdots, \lambda_{n}\right\} .
$$
\begin{CJK}{UTF8}{mj}其中\end{CJK} $I_{r}$ \begin{CJK}{UTF8}{mj}是\end{CJK} $r$ \begin{CJK}{UTF8}{mj}阶单位矩阵\end{CJK}, $\lambda_{i}$ \begin{CJK}{UTF8}{mj}是\end{CJK} $B$ \begin{CJK}{UTF8}{mj}的特征值\end{CJK}.
$$
\lim _{n \rightarrow \infty}\left(\frac{\left(n^{n}\right.}{f(a)}\right)
$$

\begin{enumerate}
  \setcounter{enumi}{2}
  \item \begin{CJK}{UTF8}{mj}判断极限\end{CJK} $\lim _{(x, y) \rightarrow(0,0)} \frac{x y}{\sqrt{x+y+1}-1}$ \begin{CJK}{UTF8}{mj}的存在性\end{CJK}. \begin{CJK}{UTF8}{mj}若存在\end{CJK}, \begin{CJK}{UTF8}{mj}求出此极限\end{CJK}, \begin{CJK}{UTF8}{mj}若不存在\end{CJK}, \begin{CJK}{UTF8}{mj}请说明理由\end{CJK}. 3. \begin{CJK}{UTF8}{mj}求不定积分\end{CJK}
\end{enumerate}
$$
\int \frac{\arctan x}{x^{2}\left(1+x^{2}\right)} \mathrm{d} x
$$

\begin{enumerate}
  \setcounter{enumi}{4}
  \item \begin{CJK}{UTF8}{mj}设函数\end{CJK} $z=z(x, y)$ \begin{CJK}{UTF8}{mj}由方程\end{CJK} $\mathrm{e}^{2 y z}+x+y^{2}+z=\frac{7}{4}$ \begin{CJK}{UTF8}{mj}所确定\end{CJK}, \begin{CJK}{UTF8}{mj}求\end{CJK} $\left.\mathrm{d} z\right|_{\left(\frac{1}{2}, \frac{1}{2}\right)}$ \begin{CJK}{UTF8}{mj}及\end{CJK} $\left.\frac{\partial^{2} z}{\partial x \partial y}\right|_{\left(\frac{1}{2}, \frac{1}{2}\right)}$.

  \item \begin{CJK}{UTF8}{mj}计算曲面积分\end{CJK}

\end{enumerate}
$$
\iint_{\Sigma}(2 x+z) \mathrm{d} y \mathrm{~d} z+z \mathrm{~d} x \mathrm{~d} y
$$
\begin{CJK}{UTF8}{mj}其中\end{CJK} $\Sigma$ \begin{CJK}{UTF8}{mj}为曲面\end{CJK} $z=x^{2}+y^{2}(0 \leq z \leq 1)$, \begin{CJK}{UTF8}{mj}取上侧\end{CJK}.

6.(1) \begin{CJK}{UTF8}{mj}证明\end{CJK}: \begin{CJK}{UTF8}{mj}当\end{CJK} $x>0$ \begin{CJK}{UTF8}{mj}时\end{CJK}, $\frac{x}{1+x}<\ln (x+1)<x$.

(2) \begin{CJK}{UTF8}{mj}设\end{CJK} $a_{n}=1+\frac{1}{2}+\frac{1}{3}+\cdots+\frac{1}{n}$, \begin{CJK}{UTF8}{mj}证明数列\end{CJK} $\left\{a_{n}\right\}$ \begin{CJK}{UTF8}{mj}收敛\end{CJK}.

\begin{enumerate}
  \setcounter{enumi}{7}
  \item \begin{CJK}{UTF8}{mj}讨论函数\end{CJK}
\end{enumerate}
$$
f(x)=\left\{\begin{array}{l}
x(1-x), x \text { 为有理数 } \\
x(1+x), x \text { 为无理数 }
\end{array}\right.
$$
\begin{CJK}{UTF8}{mj}的连续性与可微性\end{CJK}.

\begin{enumerate}
  \setcounter{enumi}{8}
  \item \begin{CJK}{UTF8}{mj}设\end{CJK} $n$ \begin{CJK}{UTF8}{mj}是正整数\end{CJK}, \begin{CJK}{UTF8}{mj}证明方程\end{CJK} $x^{n}+n x+1=0$ \begin{CJK}{UTF8}{mj}有唯一的正实根\end{CJK} $x_{n}$; \begin{CJK}{UTF8}{mj}进而\end{CJK}, \begin{CJK}{UTF8}{mj}当\end{CJK} $\alpha>1$ \begin{CJK}{UTF8}{mj}时\end{CJK}, \begin{CJK}{UTF8}{mj}判断级数\end{CJK} $\sum_{n=1}^{\infty} x$ \begin{CJK}{UTF8}{mj}性\end{CJK}, \begin{CJK}{UTF8}{mj}并证明你的结论\end{CJK}.

  \item \begin{CJK}{UTF8}{mj}已知函数\end{CJK} $f(x)$ \begin{CJK}{UTF8}{mj}在区间\end{CJK} $(-\delta, \delta) \forall$ \begin{CJK}{UTF8}{mj}有界\end{CJK} $(\delta>0)$, \begin{CJK}{UTF8}{mj}且对任意的\end{CJK} $x \in(-\delta, \delta)$, \begin{CJK}{UTF8}{mj}都有\end{CJK} $f(x)=3 f\left(\frac{x}{2}\right)$. \begin{CJK}{UTF8}{mj}证时\end{CJK} $x=0$ \begin{CJK}{UTF8}{mj}处连续\end{CJK}.

  \item \begin{CJK}{UTF8}{mj}设函数\end{CJK} $f(x, y)$ \begin{CJK}{UTF8}{mj}在\end{CJK} $[0,+\infty) \times[c, d]$ \begin{CJK}{UTF8}{mj}上连续\end{CJK}, \begin{CJK}{UTF8}{mj}且无穷积分\end{CJK} $\int_{0}^{\infty} f(x, y) \mathrm{d} x$ \begin{CJK}{UTF8}{mj}在\end{CJK} $[c, d)$ \begin{CJK}{UTF8}{mj}上一致收玫\end{CJK}. \begin{CJK}{UTF8}{mj}证明\end{CJK} $\int_{0}^{\infty} f(x, d) \mathrm{d} x$ \begin{CJK}{UTF8}{mj}收敛\end{CJK}.

  \item \begin{CJK}{UTF8}{mj}设函数\end{CJK} $f(x)$ \begin{CJK}{UTF8}{mj}在\end{CJK} $(-\infty,+\infty)$ \begin{CJK}{UTF8}{mj}上连续\end{CJK}, $f_{n}(x)=\sum_{k=1}^{n} \frac{1}{n} f\left(x+\frac{k}{n}\right), n=1,2, \cdots$

\end{enumerate}
(1) \begin{CJK}{UTF8}{mj}证明函数列\end{CJK} $\left\{f_{n}(x)\right\}$ \begin{CJK}{UTF8}{mj}在\end{CJK} $(-\infty,+\infty)$ \begin{CJK}{UTF8}{mj}上处处收玫\end{CJK}, \begin{CJK}{UTF8}{mj}并指出其极限函数\end{CJK}.

(2) \begin{CJK}{UTF8}{mj}证明\end{CJK} $\left\{f_{n}(x)\right\}$ \begin{CJK}{UTF8}{mj}在任意有限区间\end{CJK} $[a, b]$ \begin{CJK}{UTF8}{mj}上一致收玫\end{CJK}.

\begin{enumerate}
  \setcounter{enumi}{12}
  \item \begin{CJK}{UTF8}{mj}已知函数\end{CJK} $f(x)$ \begin{CJK}{UTF8}{mj}在\end{CJK} $[a, b]$ \begin{CJK}{UTF8}{mj}上非负\end{CJK}、\begin{CJK}{UTF8}{mj}连续且严格递增\end{CJK}, \begin{CJK}{UTF8}{mj}并对任意的正整数\end{CJK} $n$, \begin{CJK}{UTF8}{mj}存在\end{CJK} $x_{n} \in[a, b]$, \begin{CJK}{UTF8}{mj}使\end{CJK}
\end{enumerate}
$$
f^{n}\left(x_{n}\right)=\frac{1}{b-a} \int_{a}^{b} f^{n}(x) \mathrm{d} x
$$
\begin{CJK}{UTF8}{mj}证明数列\end{CJK} $\left\{x_{n}\right\}$ \begin{CJK}{UTF8}{mj}收敛\end{CJK}. 1. \begin{CJK}{UTF8}{mj}正明数列\end{CJK} $\{\sin n\}$ \begin{CJK}{UTF8}{mj}发散\end{CJK}.

\begin{enumerate}
  \setcounter{enumi}{2}
  \item $f(x)$ \begin{CJK}{UTF8}{mj}在\end{CJK} $x=0$ \begin{CJK}{UTF8}{mj}处连续\end{CJK}, \begin{CJK}{UTF8}{mj}且对仟音\end{CJK} $x \in \mathbb{R}$ \begin{CJK}{UTF8}{mj}有\end{CJK} $f(x)=f(3 x)$. \begin{CJK}{UTF8}{mj}证明\end{CJK}: $f(x)$ \begin{CJK}{UTF8}{mj}为常值函数\end{CJK}.

  \item $f(x)$ \begin{CJK}{UTF8}{mj}在\end{CJK} $(-1,1)$ \begin{CJK}{UTF8}{mj}上有任意阶导数\end{CJK}, $f(0)=0$, \begin{CJK}{UTF8}{mj}且对任意正整数\end{CJK} $n$, \begin{CJK}{UTF8}{mj}有\end{CJK} $f^{(n)}(0=0)$. \begin{CJK}{UTF8}{mj}设存在\end{CJK} $c \geq 0$, \begin{CJK}{UTF8}{mj}对\end{CJK} \begin{CJK}{UTF8}{mj}数\end{CJK} $n$ \begin{CJK}{UTF8}{mj}和\end{CJK} $x \in(-1,1)$ \begin{CJK}{UTF8}{mj}有\end{CJK} $\left|f^{(n)}(x)\right| \leq n ! c^{n}$. \begin{CJK}{UTF8}{mj}证明\end{CJK}: $f(x)$ \begin{CJK}{UTF8}{mj}在\end{CJK} $(-1,1)$ \begin{CJK}{UTF8}{mj}上恒为\end{CJK} 0 .

  \item $a, b, c, d$ \begin{CJK}{UTF8}{mj}皆为常数\end{CJK}, $c d \neq 0$, \begin{CJK}{UTF8}{mj}说明并给定理\end{CJK}: \begin{CJK}{UTF8}{mj}当\end{CJK} $a, b, c, d$ \begin{CJK}{UTF8}{mj}满足什么条件时\end{CJK}, $f(x)=\frac{a x+b}{c x+d}$ \begin{CJK}{UTF8}{mj}无极值\end{CJK}. $5 . f(x)$ \begin{CJK}{UTF8}{mj}在\end{CJK} $(0,1)$ \begin{CJK}{UTF8}{mj}内可导\end{CJK}, \begin{CJK}{UTF8}{mj}在\end{CJK} $[0,1]$ \begin{CJK}{UTF8}{mj}连续\end{CJK}, $f(1)-f(0)=\frac{2}{e}-1$. \begin{CJK}{UTF8}{mj}证明\end{CJK}: \begin{CJK}{UTF8}{mj}存在\end{CJK} $\xi \in(0,1)$, \begin{CJK}{UTF8}{mj}使得\end{CJK} $\mathrm{e}^{\xi^{2}} f^{\prime}(\xi)$ t

  \item \begin{CJK}{UTF8}{mj}已知\end{CJK} $f(x)=\left\{\begin{array}{l}x^{2}, x \in[0, \pi] \\ (x+2 \pi)^{3}, x \in[-\pi, 0)\end{array}\right.$.

\end{enumerate}
(1) \begin{CJK}{UTF8}{mj}求\end{CJK} $f(x)$ \begin{CJK}{UTF8}{mj}的\end{CJK} Fourier \begin{CJK}{UTF8}{mj}级数\end{CJK}.

(2) \begin{CJK}{UTF8}{mj}证明\end{CJK} $\sum_{n=1}^{\infty} \frac{1}{n^{2}}=\frac{\pi^{2}}{6}$.

\begin{enumerate}
  \setcounter{enumi}{7}
  \item \begin{CJK}{UTF8}{mj}级数\end{CJK} $\sum_{n=1}^{\infty} \frac{\sin (n x)}{1+n x^{2}}$.
\end{enumerate}
(1) $x$ \begin{CJK}{UTF8}{mj}取何值时函数绝对收敛\end{CJK}?

(2) $x$ \begin{CJK}{UTF8}{mj}取何值时函数条件收敛\end{CJK}?

\begin{enumerate}
  \setcounter{enumi}{8}
  \item \begin{CJK}{UTF8}{mj}求含参变是积分\end{CJK}
\end{enumerate}
$$
I(a)=\int_{0}^{\infty} \frac{\arctan (a x)}{x\left(1+x^{2}\right)} \mathrm{d} x, a>0
$$

\begin{enumerate}
  \setcounter{enumi}{9}
  \item \begin{CJK}{UTF8}{mj}求第一型曲面积分\end{CJK}
\end{enumerate}
$$
\int\left(z^{3}-x\right) \mathrm{d} y \mathrm{~d} z-x y \mathrm{~d} z \mathrm{~d} x-3 z \mathrm{~d} x \mathrm{~d} y
$$
\begin{CJK}{UTF8}{mj}其中\end{CJK} $S: z=4-y^{2}, x=0, x=3$ \begin{CJK}{UTF8}{mj}平面\end{CJK}, $O x y$ \begin{CJK}{UTF8}{mj}平面\end{CJK}, \begin{CJK}{UTF8}{mj}取夕侧\end{CJK}.

\begin{enumerate}
  \setcounter{enumi}{10}
  \item \begin{CJK}{UTF8}{mj}什么条件下\end{CJK}, \begin{CJK}{UTF8}{mj}对任意的\end{CJK} $j=1,2, \cdots, n, F\left(x_{1}, x_{2}, \cdots, x_{n}\right)=0$ \begin{CJK}{UTF8}{mj}都能在\end{CJK} $\overrightarrow{x_{0}} \in \mathbb{R}^{n}$ \begin{CJK}{UTF8}{mj}附近唯一确定\end{CJK} $x_{j}=x_{j}\left(x_{1}, x_{2}, \cdots, x_{j-1}, x_{j+1}, \cdots, x_{n}\right)$ \begin{CJK}{UTF8}{mj}在\end{CJK} $\overrightarrow{x_{0}}$ \begin{CJK}{UTF8}{mj}附近\end{CJK}. \begin{CJK}{UTF8}{mj}求\end{CJK} $\frac{\partial x_{1}}{\partial x_{2}}(\vec{x}) \cdot \frac{\partial x_{2}}{\partial x_{3}}(\vec{x}) \cdots \frac{\partial x_{n-1}}{\partial x_{n}}(\vec{x}) \cdot \frac{\partial x_{n}}{\partial x_{1}}(\vec{x})$.

  \item \begin{CJK}{UTF8}{mj}设函数\end{CJK} $u$ \begin{CJK}{UTF8}{mj}在\end{CJK} $\mathbb{R}^{n}$ \begin{CJK}{UTF8}{mj}上可微\end{CJK}, \begin{CJK}{UTF8}{mj}且存在常数\end{CJK} $c_{0}>0$, \begin{CJK}{UTF8}{mj}使得\end{CJK} $\lim _{|\vec{x}| \rightarrow+\infty}|\vec{x} \cdot \nabla u(\vec{x})| \geq c_{0}, \nabla=\left|\frac{\partial}{\partial x_{1}}, \frac{\partial}{\partial x_{2}}, \cdots, \frac{\partial}{\partial x_{n}}\right|$. $+\infty, \overrightarrow{\xi_{0}} \in \mathbb{R}^{n}, \nabla u\left(\overrightarrow{\xi_{0}}\right)=0 .$

  \item $\mathbb{R}^{3}$ \begin{CJK}{UTF8}{mj}上函数\end{CJK} $u$ \begin{CJK}{UTF8}{mj}具有\end{CJK} 2 \begin{CJK}{UTF8}{mj}阶连续偏导数不恒为常数\end{CJK}, $\Delta u+u^{5}=0, \Delta=\frac{\partial^{2}}{\partial x^{2}}+\frac{\partial^{2}}{\partial y^{2}}+\frac{\partial^{2}}{\partial z^{2}}, \lambda>0$. \begin{CJK}{UTF8}{mj}令\end{CJK} $u \lambda$ $\lambda^{\alpha} u(\lambda x, \lambda y, \lambda z), \alpha$ \begin{CJK}{UTF8}{mj}是某正常数\end{CJK}, \begin{CJK}{UTF8}{mj}使得对任意\end{CJK} $\lambda>0, u_{\lambda}$ \begin{CJK}{UTF8}{mj}满足\end{CJK} $\Delta u_{\lambda}+u_{\lambda}^{5}$.

\end{enumerate}
(1) \begin{CJK}{UTF8}{mj}求\end{CJK} $\alpha$.

(2) $\int_{\mathbb{R}^{3}}|\nabla u(x, y, z)|^{2} \mathrm{~d} x \mathrm{~d} y \mathrm{~d} z$ \begin{CJK}{UTF8}{mj}收敛\end{CJK}, $\quad \forall \lambda>0, \int_{\mathbb{R}^{3}}\left|\nabla u_{\lambda}(x, y, z)\right|^{2} \mathrm{~d} x \mathrm{~d} y \mathrm{~d} z=\int_{\mathbb{R}^{3}}|\nabla u(x, y, z)|^{2} \mathrm{~d} x \mathrm{~d} y \mathrm{~d} z, \nabla=\frac{\partial}{\partial x}+$

(3) $D$ \begin{CJK}{UTF8}{mj}是光滑边界\end{CJK} $\partial D$ \begin{CJK}{UTF8}{mj}的有界区域\end{CJK}, \begin{CJK}{UTF8}{mj}且\end{CJK} $\left.u\right|_{\partial D}=0$. \begin{CJK}{UTF8}{mj}证明\end{CJK}: $\int_{D}|u(x, y, z)|^{6} \mathrm{~d} x \mathrm{~d} y \mathrm{~d} z=\int_{D}|\nabla u(x, y, z)|^{2} \mathrm{~d} x \mathrm{~d} y \mathrm{~d} z$

\section{$7.2$ 高等代数}
\begin{enumerate}
  \item \begin{CJK}{UTF8}{mj}设\end{CJK} $m, n$ \begin{CJK}{UTF8}{mj}为正整数\end{CJK}.
\end{enumerate}
(1) \begin{CJK}{UTF8}{mj}证明\end{CJK}: \begin{CJK}{UTF8}{mj}存在多项式\end{CJK} $u(x)$ \begin{CJK}{UTF8}{mj}和\end{CJK} $v(x)$, \begin{CJK}{UTF8}{mj}使得\end{CJK} $u(x) x^{m}+v(x)\left(x^{n}-1\right)=1$. 3. \begin{CJK}{UTF8}{mj}设\end{CJK} $A, B, C, D$ \begin{CJK}{UTF8}{mj}均为\end{CJK} $n$ \begin{CJK}{UTF8}{mj}阶方阵\end{CJK}, \begin{CJK}{UTF8}{mj}且\end{CJK} $A C=C A$. \begin{CJK}{UTF8}{mj}证明\end{CJK}: $\left|\begin{array}{ll}A & B \\ C & D\end{array}\right|=|A D-C B|$. $4 . a, b$ \begin{CJK}{UTF8}{mj}取何值时以下方程组有解\end{CJK}? \begin{CJK}{UTF8}{mj}并求其解\end{CJK}.
$$
\left\{\begin{array}{l}
x_{1}+x_{2}-x_{3}+x_{4}+x_{5}=1 \\
5 x_{1}+4 x_{2}-3 x_{3}+3 x_{4}-x_{5}=a \\
x_{2}-2 x_{3}+2 x_{4}+6 x_{5}=3 \\
3 x_{1}+2 x_{2}-x_{3}+x_{4}-3 x_{5}=6
\end{array}\right.
$$

\begin{enumerate}
  \setcounter{enumi}{5}
  \item \begin{CJK}{UTF8}{mj}设\end{CJK} $\alpha_{1}, \alpha_{2}, \cdots, \alpha_{n}$ \begin{CJK}{UTF8}{mj}与\end{CJK} $\beta_{1}, \beta_{2}, \cdots, \beta_{n}$ \begin{CJK}{UTF8}{mj}是数域\end{CJK} $\mathbb{P}$ \begin{CJK}{UTF8}{mj}上的\end{CJK} $n$ \begin{CJK}{UTF8}{mj}维线性空间\end{CJK} $V$ \begin{CJK}{UTF8}{mj}的两组基\end{CJK}.
\end{enumerate}
(1) \begin{CJK}{UTF8}{mj}令\end{CJK} $V_{1}$ \begin{CJK}{UTF8}{mj}表示在上述两组基上坐标完全相同的全体向量组成的集合\end{CJK}, \begin{CJK}{UTF8}{mj}即\end{CJK} $V_{1}=\left\{\alpha \in V \mid \exists x_{1}, x_{2}, \cdots, x\right.$ \begin{CJK}{UTF8}{mj}得\end{CJK} $\left.\alpha=x_{1} \alpha_{1}+x_{2} \alpha_{2}+\cdots+x_{n} \alpha_{n}\right\}$. \begin{CJK}{UTF8}{mj}证明\end{CJK}: $V_{1}$ \begin{CJK}{UTF8}{mj}是\end{CJK} $V$ \begin{CJK}{UTF8}{mj}的一个子空间\end{CJK}.

(2) \begin{CJK}{UTF8}{mj}设由\end{CJK} $\alpha_{1}, \alpha_{2}, \cdots, \alpha_{n}$ \begin{CJK}{UTF8}{mj}到\end{CJK} $\beta_{1}, \beta_{2}, \cdots, \beta_{n}$ \begin{CJK}{UTF8}{mj}的过渡矩阵为\end{CJK} $A$. \begin{CJK}{UTF8}{mj}证明\end{CJK}: \begin{CJK}{UTF8}{mj}如果矩阵\end{CJK} $I_{n}-A$ \begin{CJK}{UTF8}{mj}的秩为\end{CJK} $r$, \begin{CJK}{UTF8}{mj}那么\end{CJK} $V_{1}$ $n-r$.

\begin{enumerate}
  \setcounter{enumi}{6}
  \item \begin{CJK}{UTF8}{mj}设\end{CJK} $V$ \begin{CJK}{UTF8}{mj}是数域\end{CJK} $\mathbb{P}$ \begin{CJK}{UTF8}{mj}上的线性空间\end{CJK}, $\mathscr{T}$ \begin{CJK}{UTF8}{mj}为\end{CJK} $V$ \begin{CJK}{UTF8}{mj}上的线性变换\end{CJK}.
\end{enumerate}
(1) \begin{CJK}{UTF8}{mj}设\end{CJK} $\lambda_{1}, \lambda_{2}$ \begin{CJK}{UTF8}{mj}是\end{CJK} $\mathscr{T}$ \begin{CJK}{UTF8}{mj}的两个不同特征值\end{CJK}, \begin{CJK}{UTF8}{mj}而\end{CJK} $\alpha_{1}, \alpha_{2}$ \begin{CJK}{UTF8}{mj}是分别属于\end{CJK} $\alpha_{1}, \alpha_{2}$ \begin{CJK}{UTF8}{mj}的特征向量\end{CJK}. \begin{CJK}{UTF8}{mj}证明\end{CJK}: $\alpha_{1}+\alpha_{2}$ \begin{CJK}{UTF8}{mj}不是\end{CJK} \begin{CJK}{UTF8}{mj}向量\end{CJK}.

(2) \begin{CJK}{UTF8}{mj}证明\end{CJK}: \begin{CJK}{UTF8}{mj}如果\end{CJK} $V$ \begin{CJK}{UTF8}{mj}中每个非零向量均为\end{CJK} $\mathscr{T}$ \begin{CJK}{UTF8}{mj}的特征向量\end{CJK}, \begin{CJK}{UTF8}{mj}那么\end{CJK} $\mathscr{T}$ \begin{CJK}{UTF8}{mj}为数乘变换\end{CJK}.

\begin{enumerate}
  \setcounter{enumi}{7}
  \item \begin{CJK}{UTF8}{mj}设\end{CJK} $V$ \begin{CJK}{UTF8}{mj}是\end{CJK} $n$ \begin{CJK}{UTF8}{mj}维欧氏空间\end{CJK}, $e_{1}, e_{2}, \cdots, e_{n}$ \begin{CJK}{UTF8}{mj}是\end{CJK} $V$ \begin{CJK}{UTF8}{mj}的一组标准正交基\end{CJK}, $\alpha_{1}, \alpha_{2}, \cdots, \alpha_{n} \in V$ \begin{CJK}{UTF8}{mj}是一组向量\end{CJK}, \begin{CJK}{UTF8}{mj}示\end{CJK} $e_{i}$ \begin{CJK}{UTF8}{mj}和\end{CJK} $\alpha_{j}$ \begin{CJK}{UTF8}{mj}的内积\end{CJK}. \begin{CJK}{UTF8}{mj}证明\end{CJK}: $\alpha_{1}, \alpha_{2}, \cdots, \alpha_{n}$ \begin{CJK}{UTF8}{mj}线性无关的充要条件是\end{CJK}
\end{enumerate}
\includegraphics[max width=\textwidth]{2022_04_18_7db0708508f26638f054g-185}

\begin{enumerate}
  \setcounter{enumi}{8}
  \item \begin{CJK}{UTF8}{mj}设\end{CJK} $\mathbb{P}$ \begin{CJK}{UTF8}{mj}是一个数域\end{CJK}, $a_{0}, a_{1}, \cdots, a_{n-1} \in \mathbb{P}$, \begin{CJK}{UTF8}{mj}求如下\end{CJK} $\lambda$-\begin{CJK}{UTF8}{mj}矩阵的不变因子\end{CJK}.
\end{enumerate}
$$
(\lambda)=\left(\begin{array}{cccccc}
\lambda & 0 & 0 & \cdots & 0 & a_{0} \\
-1 & \lambda & 0 & \cdots & 0 & a_{1} \\
0 & -1 & \lambda & \cdots & 0 & a_{2} \\
\vdots & \vdots & \vdots & \ddots & \vdots & \vdots \\
0 & 0 & 0 & \cdots & \lambda & a_{n-2} \\
0 & 0 & 0 & \cdots & -1 & \lambda+a_{n-1}
\end{array}\right)
$$

\begin{enumerate}
  \setcounter{enumi}{9}
  \item \begin{CJK}{UTF8}{mj}对于\end{CJK} $n$ \begin{CJK}{UTF8}{mj}阶方阵\end{CJK} $A=\left(a_{i j}\right)_{n \times n}$, \begin{CJK}{UTF8}{mj}定义\end{CJK} $\operatorname{tr}(A)=\sum_{i=1}^{n} a_{i i}$. \begin{CJK}{UTF8}{mj}求全体\end{CJK} $n$ \begin{CJK}{UTF8}{mj}阶实方阵构成的线性空间\end{CJK} $M_{n}(\mathbb{R})$ \begin{CJK}{UTF8}{mj}上\end{CJK} $x_{i j}(i, j=1,2, \cdots, n)$ \begin{CJK}{UTF8}{mj}的二次型\end{CJK} $\operatorname{tr}\left(X^{2}\right)$.(\begin{CJK}{UTF8}{mj}其中\end{CJK} $X=\left(x_{i j}\right)_{n \times n} \in M_{n}(\mathbb{R})$ \begin{CJK}{UTF8}{mj}的正惯性指数和负惯性指数\end{CJK}) 1.\begin{CJK}{UTF8}{mj}求极限\end{CJK}
\end{enumerate}
$$
\lim _{n \rightarrow \infty}(\arctan n+\sin n)^{n}
$$

\begin{enumerate}
  \setcounter{enumi}{2}
  \item \begin{CJK}{UTF8}{mj}求\end{CJK}
\end{enumerate}
$$
\iint_{\Omega} \frac{(z+1) \mathrm{d} y \mathrm{~d} z+(y+1) \mathrm{d} z \mathrm{~d} x+(x+1) \mathrm{d} x \mathrm{~d} y}{\sqrt{x^{2}+y^{2}+z^{2}}}
$$
\begin{CJK}{UTF8}{mj}其中\end{CJK}: $\Omega: z=\sqrt{9-x^{2}-y^{2}}$, \begin{CJK}{UTF8}{mj}取下侧\end{CJK}.

\begin{enumerate}
  \setcounter{enumi}{3}
  \item \begin{CJK}{UTF8}{mj}求\end{CJK} $2 x-y=1$ \begin{CJK}{UTF8}{mj}与\end{CJK} $x^{2}+y^{2}+z^{2}=1$ \begin{CJK}{UTF8}{mj}的交线上的点到原点的最近距离\end{CJK}.

  \item \begin{CJK}{UTF8}{mj}判䉼\end{CJK} $f(y)=\int_{0}^{\infty} \frac{\sin x^{2} y}{x} \mathrm{~d} x$ \begin{CJK}{UTF8}{mj}的一致收剑性\end{CJK}.

  \item \begin{CJK}{UTF8}{mj}证明\end{CJK}: \begin{CJK}{UTF8}{mj}若\end{CJK} $\lim _{n \rightarrow \infty} n\left(\frac{a_{n}}{a_{n+1}}-1\right)=\lambda>0$, \begin{CJK}{UTF8}{mj}则六错级数\end{CJK} $\sum_{n=1}^{\infty}(-1)^{n-1} a_{n}$ \begin{CJK}{UTF8}{mj}收敛\end{CJK}.

  \item \begin{CJK}{UTF8}{mj}设定义在\end{CJK} $\mathbb{R}$ \begin{CJK}{UTF8}{mj}上可导函数\end{CJK} $f(x)$ \begin{CJK}{UTF8}{mj}满足\end{CJK} $f(0)=0, f^{\prime}(x)$ \begin{CJK}{UTF8}{mj}单调递增\end{CJK}, \begin{CJK}{UTF8}{mj}证明\end{CJK} $\frac{f(x)}{x}$ \begin{CJK}{UTF8}{mj}单调递增\end{CJK}.

  \item \begin{CJK}{UTF8}{mj}设\end{CJK} $\Omega$ \begin{CJK}{UTF8}{mj}是\end{CJK} $\mathbb{R}^{3}$ \begin{CJK}{UTF8}{mj}中的光滑区域\end{CJK}, $u$ \begin{CJK}{UTF8}{mj}在\end{CJK} $\bar{\Omega}$ \begin{CJK}{UTF8}{mj}上连续\end{CJK}, \begin{CJK}{UTF8}{mj}在\end{CJK} $\Omega$ \begin{CJK}{UTF8}{mj}内连续可微\end{CJK}.\begin{CJK}{UTF8}{mj}试正\end{CJK}: \begin{CJK}{UTF8}{mj}存在\end{CJK} $x_{0} \in \Omega$, \begin{CJK}{UTF8}{mj}使得\end{CJK} $\nabla u\left(x_{0}\right)=$

  \item \begin{CJK}{UTF8}{mj}设\end{CJK} $\left\{f_{n}(x)\right\}$ \begin{CJK}{UTF8}{mj}是\end{CJK} $[a, b]$ \begin{CJK}{UTF8}{mj}上的可积函数列\end{CJK}, \begin{CJK}{UTF8}{mj}且一致收敛于\end{CJK} $f(x)$. \begin{CJK}{UTF8}{mj}试证\end{CJK}: $f(x)$ \begin{CJK}{UTF8}{mj}在\end{CJK} $[a, b]$ \begin{CJK}{UTF8}{mj}上可积\end{CJK}.

  \item \begin{CJK}{UTF8}{mj}用闭区间室定理证明确界存在定理\end{CJK}.

\end{enumerate}
\section{$8.2$ 高等代数}
\begin{enumerate}
  \item (1) \begin{CJK}{UTF8}{mj}求\end{CJK} $u(x), v(x)$, \begin{CJK}{UTF8}{mj}使得\end{CJK} $u(x)(x-1)^{3}+v(x) x^{2}=1$.
\end{enumerate}
(2) \begin{CJK}{UTF8}{mj}求次数\end{CJK}/\begin{CJK}{UTF8}{mj}于\end{CJK} 5 \begin{CJK}{UTF8}{mj}的多项式\end{CJK}, \begin{CJK}{UTF8}{mj}满足\end{CJK} $(x-1)^{3}\left|p(x)-1,(x-2)^{2}\right| p(x)+1$.

\begin{enumerate}
  \setcounter{enumi}{2}
  \item $A, B$ \begin{CJK}{UTF8}{mj}为\end{CJK} $n$ \begin{CJK}{UTF8}{mj}阶方阵\end{CJK}, $A B=A+B$, \begin{CJK}{UTF8}{mj}正明\end{CJK} $B A=B+A$.

  \item \begin{CJK}{UTF8}{mj}矩阵的满秩分解\end{CJK}. 1.(1)\begin{CJK}{UTF8}{mj}求极限\end{CJK}

\end{enumerate}
$$
\lim _{n \rightarrow \infty} \frac{1^{2} \sqrt[1]{1}+2^{2} \sqrt[2]{2}+3^{2} \sqrt[3]{3}+\cdots+n^{2} \sqrt[n]{n}}{n^{3}}
$$
(2) \begin{CJK}{UTF8}{mj}求极限\end{CJK}
$$
\lim _{x \rightarrow 0} \frac{\ln (1+x)-x}{x^{2}}
$$

\begin{enumerate}
  \setcounter{enumi}{2}
  \item $f$ \begin{CJK}{UTF8}{mj}在\end{CJK} $(a, b)$ \begin{CJK}{UTF8}{mj}上可导\end{CJK}, \begin{CJK}{UTF8}{mj}无界\end{CJK}, \begin{CJK}{UTF8}{mj}证明\end{CJK} $f^{\prime}$ \begin{CJK}{UTF8}{mj}无界\end{CJK}, \begin{CJK}{UTF8}{mj}又过來举反例\end{CJK}.

  \item \begin{CJK}{UTF8}{mj}讨论\end{CJK} $\int_{0}^{\infty} \frac{\sin b x}{x^{\lambda}}$ \begin{CJK}{UTF8}{mj}的收敛性和绝对收敛性\end{CJK}. $(b \neq 0)$

  \item \begin{CJK}{UTF8}{mj}示\end{CJK}.

  \item \begin{CJK}{UTF8}{mj}求\end{CJK} $\frac{x^{2}}{4}+y^{2}=1$ \begin{CJK}{UTF8}{mj}上一卣到\end{CJK} $3 x+4 y=12,3 x-4 y=-12, y=-1$ \begin{CJK}{UTF8}{mj}距离平方和的最小值\end{CJK}.

\end{enumerate}
6.(1) \begin{CJK}{UTF8}{mj}求\end{CJK}
$$
\int_{C} x^{2} \mathrm{~d} S
$$
\begin{CJK}{UTF8}{mj}其中\end{CJK} $C$ \begin{CJK}{UTF8}{mj}为\end{CJK} $x^{2}+y^{2}+z^{2}=a^{2} 与 \sqrt{x^{2}+y^{2}}=z$ \begin{CJK}{UTF8}{mj}交线\end{CJK}.

(2) \begin{CJK}{UTF8}{mj}某\end{CJK}
$$
\int_{C} \frac{x d x+y d y}{\sqrt{x^{2}+y^{2}}}
$$
\begin{CJK}{UTF8}{mj}其中\end{CJK} $C$ \begin{CJK}{UTF8}{mj}为\end{CJK} $(0,2)$ \begin{CJK}{UTF8}{mj}至\end{CJK} $(1,0)$ \begin{CJK}{UTF8}{mj}不经过\end{CJK} $(0,0)$ \begin{CJK}{UTF8}{mj}的一条光滑専线\end{CJK}.

\begin{enumerate}
  \setcounter{enumi}{7}
  \item $f(0)=0, f(1)=1, f$ \begin{CJK}{UTF8}{mj}连续\end{CJK}. \begin{CJK}{UTF8}{mj}求证\end{CJK}: \begin{CJK}{UTF8}{mj}对任意\end{CJK} $n$, \begin{CJK}{UTF8}{mj}存在\end{CJK} $\varepsilon_{n}$, \begin{CJK}{UTF8}{mj}使得\end{CJK} $f\left(\varepsilon_{n}+\frac{c}{n}\right)=f\left(\varepsilon_{n}\right)+\frac{1}{n}$.

  \item $f$ \begin{CJK}{UTF8}{mj}在\end{CJK} $[0,2 \pi]$ \begin{CJK}{UTF8}{mj}上连续\end{CJK}.\begin{CJK}{UTF8}{mj}求证\end{CJK}:

\end{enumerate}
\includegraphics[max width=\textwidth]{2022_04_18_7db0708508f26638f054g-187}

9.f \begin{CJK}{UTF8}{mj}一阶连续可微\end{CJK}, $\left(\frac{\partial^{2}}{\partial x^{2}}+\frac{\partial^{2}}{\partial y^{2}}+\frac{\partial^{2}}{\partial z^{2}}\right) f=\lambda f(\lambda>0)$. \begin{CJK}{UTF8}{mj}存在\end{CJK} $R$, \begin{CJK}{UTF8}{mj}对任意\end{CJK} $|x|>R, f(x) \equiv 0$. \begin{CJK}{UTF8}{mj}证明\end{CJK}: $f($

\begin{enumerate}
  \setcounter{enumi}{10}
  \item \begin{CJK}{UTF8}{mj}求证\end{CJK} $\int_{0}^{\frac{\pi}{2}} \ln (\sin x) \mathrm{d} x$ \begin{CJK}{UTF8}{mj}存在㚔证明其等于\end{CJK} $-\frac{\pi}{2} \ln 2$.
\end{enumerate}
\section{$9.2$ 高等代数}
\begin{enumerate}
  \item \begin{CJK}{UTF8}{mj}求一个次数较低的多项式满足下面的条件\end{CJK}:
\end{enumerate}
$$
f(-1)=0, f^{\prime}(-1)=3, f^{\prime \prime}(-1)=-12, f(0)=1, f^{\prime}(0)=1, f(1)=6
$$

\begin{enumerate}
  \setcounter{enumi}{2}
  \item \begin{CJK}{UTF8}{mj}设\end{CJK} $f(x)=x^{p}+p x+p$, \begin{CJK}{UTF8}{mj}其中\end{CJK} $p$ \begin{CJK}{UTF8}{mj}是系数\end{CJK}, \begin{CJK}{UTF8}{mj}证明\end{CJK}
\end{enumerate}
(a) $f(x)$ \begin{CJK}{UTF8}{mj}在\end{CJK} $\mathbb{C}$ \begin{CJK}{UTF8}{mj}无重根\end{CJK};

(b) $f(x)$ \begin{CJK}{UTF8}{mj}在\end{CJK} $\mathbb{Q}$ \begin{CJK}{UTF8}{mj}上不可约\end{CJK}.

\begin{enumerate}
  \setcounter{enumi}{3}
  \item \begin{CJK}{UTF8}{mj}设\end{CJK} $A \in M_{n}(\mathbb{R})$, \begin{CJK}{UTF8}{mj}卉有\end{CJK} $\left|a_{i j}\right|>\sum_{j \neq i}\left|a_{i j}\right|$,
\end{enumerate}
(a) \begin{CJK}{UTF8}{mj}证明\end{CJK} $\operatorname{det}(A) \neq 0$;

(b) \begin{CJK}{UTF8}{mj}若\end{CJK} $A$ \begin{CJK}{UTF8}{mj}的全部特征值为\end{CJK} $\lambda_{1}, \lambda_{2}, \ldots, \lambda_{n}$, \begin{CJK}{UTF8}{mj}求\end{CJK} $A$ \begin{CJK}{UTF8}{mj}的伴陏矩陎\end{CJK} $A^{*}$ \begin{CJK}{UTF8}{mj}的全部特征值\end{CJK}.

\begin{enumerate}
  \setcounter{enumi}{4}
  \item \begin{CJK}{UTF8}{mj}设一次型\end{CJK} $f=2 x_{1}^{2}+3 x_{2}^{2}+3 x_{3}^{2}+2 a x_{2} x_{3}$, \begin{CJK}{UTF8}{mj}其中\end{CJK} $a>0$, \begin{CJK}{UTF8}{mj}右\end{CJK} $f$ \begin{CJK}{UTF8}{mj}可以通过正交变换化为\end{CJK} $f=y_{1}^{2}+2 y_{2}^{2}+5 y$ \begin{CJK}{UTF8}{mj}值和这个正交变换\end{CJK}. 7. \begin{CJK}{UTF8}{mj}设\end{CJK} $V=\mathrm{M}_{n}(\mathrm{R}), A$ \begin{CJK}{UTF8}{mj}是\end{CJK} $V$ \begin{CJK}{UTF8}{mj}上可对角化的矩阵\end{CJK}, \begin{CJK}{UTF8}{mj}设\end{CJK} $A$ \begin{CJK}{UTF8}{mj}的特征值\end{CJK} $\lambda_{1}, \lambda_{2}, \ldots, \lambda_{n}$ (\begin{CJK}{UTF8}{mj}计重数\end{CJK}), \begin{CJK}{UTF8}{mj}考虑\end{CJK} $V$ \begin{CJK}{UTF8}{mj}上的\end{CJK}, $g(X)=A X+X A$, \begin{CJK}{UTF8}{mj}如果\end{CJK} $\lambda_{i}+\lambda_{j} \neq 0$ \begin{CJK}{UTF8}{mj}对所有的\end{CJK} $i, j=1,2, \ldots, n$ \begin{CJK}{UTF8}{mj}都成立\end{CJK}, \begin{CJK}{UTF8}{mj}计算\end{CJK} $g$ \begin{CJK}{UTF8}{mj}的像空间的维数\end{CJK}.

  \item \begin{CJK}{UTF8}{mj}设\end{CJK} $A$ \begin{CJK}{UTF8}{mj}是正交矩陎\end{CJK}, $\lambda_{0}$ \begin{CJK}{UTF8}{mj}为\end{CJK} $A$ \begin{CJK}{UTF8}{mj}的虛特征值\end{CJK} $\lambda_{0}=a+b \mathrm{i}(b \neq 0), \gamma=\alpha+\beta \mathrm{i}$ \begin{CJK}{UTF8}{mj}为\end{CJK} $\lambda_{0}$ \begin{CJK}{UTF8}{mj}对应的特征向的\end{CJK} $a, b \in \mathbb{R}, \alpha, \beta \in \mathbb{R}^{n}$, \begin{CJK}{UTF8}{mj}证明\end{CJK} $\alpha, \beta$ \begin{CJK}{UTF8}{mj}正交\end{CJK}, \begin{CJK}{UTF8}{mj}即\end{CJK} $\alpha^{\prime} \beta=0$, \begin{CJK}{UTF8}{mj}并且\end{CJK} $\alpha$ \begin{CJK}{UTF8}{mj}与\end{CJK} $\beta$ \begin{CJK}{UTF8}{mj}有相同的模长\end{CJK}.

  \item \begin{CJK}{UTF8}{mj}设\end{CJK} $\lambda$ \begin{CJK}{UTF8}{mj}是不为\end{CJK} 0 \begin{CJK}{UTF8}{mj}的复数\end{CJK}, $n, k$ \begin{CJK}{UTF8}{mj}是正整数\end{CJK}, $J_{n}(\lambda)$ \begin{CJK}{UTF8}{mj}是特征值为\end{CJK} $\lambda$ \begin{CJK}{UTF8}{mj}的\end{CJK} $n$ \begin{CJK}{UTF8}{mj}阶\end{CJK} Jordan \begin{CJK}{UTF8}{mj}块\end{CJK}.

\end{enumerate}
(a) \begin{CJK}{UTF8}{mj}求\end{CJK} $\left(J_{n}(\lambda)\right)^{k}$ \begin{CJK}{UTF8}{mj}的\end{CJK} Jordan \begin{CJK}{UTF8}{mj}标准型\end{CJK};

(b) \begin{CJK}{UTF8}{mj}证明\end{CJK} $J_{n}(\lambda)$ \begin{CJK}{UTF8}{mj}有\end{CJK} $k$ \begin{CJK}{UTF8}{mj}次方根\end{CJK}, \begin{CJK}{UTF8}{mj}即存在\end{CJK} $B \in M_{n}(\mathrm{C})$, \begin{CJK}{UTF8}{mj}使得\end{CJK} $B^{k}=J_{n}(\lambda)$;

(c) \begin{CJK}{UTF8}{mj}证明任意复矩阵\end{CJK} $A$ \begin{CJK}{UTF8}{mj}都有\end{CJK} $k$ \begin{CJK}{UTF8}{mj}次方根\end{CJK}. 1. \begin{CJK}{UTF8}{mj}设函数列\end{CJK} $f_{n}(x)=x+x^{2}+\cdots+x^{n}(n=1,2, \cdots$, ). \begin{CJK}{UTF8}{mj}证明\end{CJK}:

(1) \begin{CJK}{UTF8}{mj}方程\end{CJK} $f_{n}(x)=1$ \begin{CJK}{UTF8}{mj}在\end{CJK} $(0,1)$ \begin{CJK}{UTF8}{mj}有唯一的实根\end{CJK} $a_{n}$.

(2) $\left\{a_{n}\right\}_{n \geq 2}$ \begin{CJK}{UTF8}{mj}收敛且求\end{CJK} $\lim _{n \rightarrow \infty} a_{n}$.

\begin{enumerate}
  \setcounter{enumi}{2}
  \item \begin{CJK}{UTF8}{mj}已知函数\end{CJK} $f(x)$ \begin{CJK}{UTF8}{mj}在\end{CJK} $[a,+\infty)$ \begin{CJK}{UTF8}{mj}上一致收敛\end{CJK} , $g(x)$ \begin{CJK}{UTF8}{mj}在\end{CJK} $[a,+\infty)$ \begin{CJK}{UTF8}{mj}上连续\end{CJK}, \begin{CJK}{UTF8}{mj}且\end{CJK} $\lim _{x \rightarrow+\infty}[f(x)-g(x)]=0$. \begin{CJK}{UTF8}{mj}证旦\end{CJK} $[a,+\infty)$ \begin{CJK}{UTF8}{mj}上一致收剑\end{CJK}.

  \item \begin{CJK}{UTF8}{mj}设\end{CJK} $f(x)$ \begin{CJK}{UTF8}{mj}在\end{CJK} $[a, b]$ \begin{CJK}{UTF8}{mj}上三阶可导\end{CJK}, $f^{\prime}(a)=f^{\prime}(b)=0$, \begin{CJK}{UTF8}{mj}证明存在\end{CJK} $\xi \in(a, b)$, \begin{CJK}{UTF8}{mj}使得\end{CJK} $\left|f^{\prime \prime}(\xi)\right| \geq \frac{4}{(b-a)^{2}} \mid f(b$ 4. \begin{CJK}{UTF8}{mj}正明\end{CJK}

\end{enumerate}
$$
\int_{0}^{1} \frac{\cos x}{\sqrt{1-x^{2}}} d x>\int_{0}^{1} \frac{\sin x}{\sqrt{1-x^{2}}} d x
$$

\begin{enumerate}
  \setcounter{enumi}{5}
  \item \begin{CJK}{UTF8}{mj}求\end{CJK} $f(x, y)=\left(1+\mathrm{e}^{y}\right) \cos x-y \mathrm{e}^{y}$ \begin{CJK}{UTF8}{mj}在\end{CJK} $x y$ \begin{CJK}{UTF8}{mj}平面的极值\end{CJK}.

  \item \begin{CJK}{UTF8}{mj}平面上的函数\end{CJK} $f(x, y)$ \begin{CJK}{UTF8}{mj}满足\end{CJK} $\frac{\partial f}{\partial y}=2(y+1)$, \begin{CJK}{UTF8}{mj}且\end{CJK} $f(y, y)=(y+1)^{2}-(2-x) \ln x$, \begin{CJK}{UTF8}{mj}求证\end{CJK}: $f(x, y)=0$ s \begin{CJK}{UTF8}{mj}所旋转形成的旋转体体积\end{CJK}.

  \item \begin{CJK}{UTF8}{mj}区域\end{CJK} $D$ \begin{CJK}{UTF8}{mj}是由锥面\end{CJK} $z=\sqrt{x^{2}+y^{2}}$ \begin{CJK}{UTF8}{mj}与球体\end{CJK} $x^{2}+y^{2}+z^{1}=1$ \begin{CJK}{UTF8}{mj}所围成的区域\end{CJK}, \begin{CJK}{UTF8}{mj}求证\end{CJK}:

\end{enumerate}
\includegraphics[max width=\textwidth]{2022_04_18_7db0708508f26638f054g-189}

\begin{enumerate}
  \setcounter{enumi}{8}
  \item \begin{CJK}{UTF8}{mj}已知\end{CJK} $\Sigma$ \begin{CJK}{UTF8}{mj}为封闭光嗗曲面\end{CJK}, $\vec{n}$ \begin{CJK}{UTF8}{mj}为曲面\end{CJK} $\Sigma$ \begin{CJK}{UTF8}{mj}上的单位外法向量\end{CJK}, $\vec{l}$ \begin{CJK}{UTF8}{mj}为任意固定的已知向量\end{CJK},\begin{CJK}{UTF8}{mj}证明\end{CJK} $\iint_{\Sigma} \cos (\vec{n}$ $0 .$

  \item $f(x)$ \begin{CJK}{UTF8}{mj}为以\end{CJK} $2 \pi$ \begin{CJK}{UTF8}{mj}周期的函数\end{CJK}, \begin{CJK}{UTF8}{mj}且\end{CJK} $x \in[0,2 \pi]$ \begin{CJK}{UTF8}{mj}时\end{CJK}, $f(x)=\frac{1}{4} x(2 \pi-x)$, \begin{CJK}{UTF8}{mj}求证\end{CJK}: $\sum_{n=1}^{\infty} \frac{1}{n^{2}}=\frac{\pi^{2}}{6}$ \begin{CJK}{UTF8}{mj}和\end{CJK} $\sum_{n=1}^{\infty} \frac{1}{n^{4}}=\frac{1}{3}$ 10. \begin{CJK}{UTF8}{mj}求证函数项级数\end{CJK} $\sum_{n=1}^{\infty} \frac{x^{n} \cos n x}{1+\cdots+x^{2 n-1}}$ \begin{CJK}{UTF8}{mj}厓\end{CJK} $[0,1]$ \begin{CJK}{UTF8}{mj}上一致收剑\end{CJK}.

\end{enumerate}
\section{$10.2$ 高等代数}
\begin{enumerate}
  \item \begin{CJK}{UTF8}{mj}在\end{CJK} $\mathbb{R}[x]_{3}$ \begin{CJK}{UTF8}{mj}中定义和\end{CJK}
\end{enumerate}
$$
(f(x), g(x))=f(-1) g(-1)+f(0) g(0)+f(1) g(1)
$$

\begin{enumerate}
  \setcounter{enumi}{2}
  \item \begin{CJK}{UTF8}{mj}设\end{CJK} $A=\left(\begin{array}{ccc}0 & -1 & 0 \\ 1 & 0 & 0 \\ 0 & 0 & -1\end{array}\right), B=P^{-1} A P$, \begin{CJK}{UTF8}{mj}其中\end{CJK} $P$ \begin{CJK}{UTF8}{mj}为\end{CJK} 3 \begin{CJK}{UTF8}{mj}阶可逆矩阵\end{CJK}, \begin{CJK}{UTF8}{mj}求\end{CJK} $B^{2004}-2 A^{2}$ 。

  \item \begin{CJK}{UTF8}{mj}某线性空间\end{CJK} $V$ \begin{CJK}{UTF8}{mj}上的线性变换\end{CJK} $\varphi$ \begin{CJK}{UTF8}{mj}在某组基下的表示矩阵为\end{CJK}

\end{enumerate}
$$
A=\left(\begin{array}{llll}
1 & 1 & 1 & 1 \\
3 & 2 & 1 & 1 \\
0 & 1 & 2 & 2 \\
4 & 5 & 3 & 3
\end{array}\right)
$$
\begin{CJK}{UTF8}{mj}求\end{CJK} $\varphi$ \begin{CJK}{UTF8}{mj}的核空由维数\end{CJK} $\operatorname{dim} \operatorname{Ker}(\varphi)$.

\begin{enumerate}
  \setcounter{enumi}{4}
  \item \begin{CJK}{UTF8}{mj}设\end{CJK} $A=\left(\begin{array}{ccc}1 & -2 & 2 \\ -2 & -2 & 4 \\ 2 & 4 & -2\end{array}\right)$, \begin{CJK}{UTF8}{mj}若\end{CJK} $t I+A$ \begin{CJK}{UTF8}{mj}正定\end{CJK}, \begin{CJK}{UTF8}{mj}求实数\end{CJK} $t$ \begin{CJK}{UTF8}{mj}的取值范围\end{CJK}. $\mathbb{K}^{4}$ \begin{CJK}{UTF8}{mj}中的子空间\end{CJK}.
\end{enumerate}
(1) \begin{CJK}{UTF8}{mj}求以\end{CJK} $W$ \begin{CJK}{UTF8}{mj}的解空间的齐次线性方程组\end{CJK}.

(2) \begin{CJK}{UTF8}{mj}求以\end{CJK} $W^{\prime}=\{\eta+\alpha \mid \alpha \in W\}$ \begin{CJK}{UTF8}{mj}为解集的非齐次线性方程组\end{CJK}, \begin{CJK}{UTF8}{mj}其中\end{CJK} $\eta=(1,2,1,2,1)^{\prime}$.

\begin{enumerate}
  \setcounter{enumi}{7}
  \item \begin{CJK}{UTF8}{mj}变换\end{CJK} $\sigma: \mathbb{R}^{3} \rightarrow \mathbb{R}^{3}$ \begin{CJK}{UTF8}{mj}定义为\end{CJK}
\end{enumerate}
$$
\sigma\left(\begin{array}{l}
x \\
y \\
z
\end{array}\right)=\left(\begin{array}{c}
x+y+z \\
2 x-y+z \\
y-z
\end{array}\right)
$$
(1) \begin{CJK}{UTF8}{mj}试说明\end{CJK} $\sigma$ \begin{CJK}{UTF8}{mj}是一个线性变换\end{CJK}.

(2) \begin{CJK}{UTF8}{mj}求\end{CJK} $\sigma$ \begin{CJK}{UTF8}{mj}在基\end{CJK} $e_{1}=(1,0,0)^{\prime}, e_{2}=(0,1,0)^{\prime}, e_{3}=(0,0,1)^{\prime}$ \begin{CJK}{UTF8}{mj}下的矩阵\end{CJK}.

(3) \begin{CJK}{UTF8}{mj}求\end{CJK} $\sigma$ \begin{CJK}{UTF8}{mj}在基\end{CJK} $\alpha_{1}=(1,1,1)^{\prime}, \alpha_{2}=(1,-1,2)^{\prime}, \alpha_{3}=(0,1,1)^{\prime}$ \begin{CJK}{UTF8}{mj}下的矩阵\end{CJK}.

\begin{enumerate}
  \setcounter{enumi}{8}
  \item \begin{CJK}{UTF8}{mj}设\end{CJK} $\varepsilon_{1}, \varepsilon_{2}, \varepsilon_{3}$ \begin{CJK}{UTF8}{mj}为欧氏空间\end{CJK} $V$ \begin{CJK}{UTF8}{mj}的标准正交基\end{CJK}. $\alpha=\varepsilon_{1}-2 \varepsilon_{2}, \beta=2 \varepsilon_{1}+\varepsilon_{2}$. \begin{CJK}{UTF8}{mj}求昰交变换\end{CJK} $H$, \begin{CJK}{UTF8}{mj}使\end{CJK} $H(\alpha)=\beta$.

  \item \begin{CJK}{UTF8}{mj}已知\end{CJK} $A=\left(\begin{array}{lll}0 & 1 & 0 \\ 1 & 0 & 0 \\ 0 & 0 & 1\end{array}\right)$, \begin{CJK}{UTF8}{mj}试证\end{CJK}: \begin{CJK}{UTF8}{mj}当\end{CJK} $n \geq 3$ \begin{CJK}{UTF8}{mj}时\end{CJK}, \begin{CJK}{UTF8}{mj}有\end{CJK} $A^{n}=A^{n-2}+A^{2}-I$, \begin{CJK}{UTF8}{mj}衣\end{CJK} $A^{100}$.

  \item \begin{CJK}{UTF8}{mj}已知\end{CJK} $A, B$ \begin{CJK}{UTF8}{mj}是任意正定矩阵\end{CJK}, \begin{CJK}{UTF8}{mj}且\end{CJK} $A-B$ \begin{CJK}{UTF8}{mj}也正定\end{CJK}, \begin{CJK}{UTF8}{mj}试证\end{CJK} $B^{\prime}-A^{\prime}$ \begin{CJK}{UTF8}{mj}也企定\end{CJK}. 1. \begin{CJK}{UTF8}{mj}用\end{CJK}" $\varepsilon-N$ \begin{CJK}{UTF8}{mj}定义\end{CJK}" \begin{CJK}{UTF8}{mj}证明\end{CJK} $\lim _{n \rightarrow \infty} \sqrt[n]{n^{2}}=1$.

\end{enumerate}
2.(1) \begin{CJK}{UTF8}{mj}设\end{CJK} $f(x)$ \begin{CJK}{UTF8}{mj}在\end{CJK} $[a, b]$ \begin{CJK}{UTF8}{mj}上连续\end{CJK}, $x_{1}, x_{2}, \cdots, x_{n} \in[a, b]$. \begin{CJK}{UTF8}{mj}证明\end{CJK}: $\exists \xi \in[a, b]$, s.t. $f(\xi)=\frac{1}{n} \sum_{k=1}^{n} f\left(x_{n}\right)$.

(2) $g(x), h(x) \in C[a, b]$, \begin{CJK}{UTF8}{mj}存在\end{CJK} $\left\{x_{n}\right\}_{n=1}^{\infty}$, \begin{CJK}{UTF8}{mj}使得\end{CJK} $a \leq x_{n} \leq b(n=1,2, \cdots), g\left(x_{n}\right)=h\left(x_{n+1}\right)$. \begin{CJK}{UTF8}{mj}证明\end{CJK}: \begin{CJK}{UTF8}{mj}存在\end{CJK} $x_{0}$ s.t.g $g\left(x_{0}\right)=h\left(x_{0}\right)$.

$3 . n$ \begin{CJK}{UTF8}{mj}是一个给定的正整数\end{CJK}. $f(x)=\left\{\begin{array}{l}x^{n} \sin (\ln |x|), x \neq 0 \\ 0, x=0\end{array}\right.$.

(1) \begin{CJK}{UTF8}{mj}若\end{CJK} $f^{\prime}(x)=\sqrt{n^{2}+1} x^{n-k} \sin (\ln |x|+\alpha)(x \neq 0)$. \begin{CJK}{UTF8}{mj}求\end{CJK} $k$ \begin{CJK}{UTF8}{mj}和\end{CJK} $\tan \alpha$.

(2) $x \neq 0$ \begin{CJK}{UTF8}{mj}时求\end{CJK} $f^{(m)}(x), m$ \begin{CJK}{UTF8}{mj}是不超过\end{CJK} $n$ \begin{CJK}{UTF8}{mj}的任意正整数\end{CJK}.

(3) \begin{CJK}{UTF8}{mj}证明\end{CJK} $f$ \begin{CJK}{UTF8}{mj}在\end{CJK} 0 \begin{CJK}{UTF8}{mj}处存在\end{CJK} $1,2, \cdots, n-1$ \begin{CJK}{UTF8}{mj}阶导\end{CJK}, \begin{CJK}{UTF8}{mj}但不存在\end{CJK} $n$ \begin{CJK}{UTF8}{mj}阶导\end{CJK}.

\begin{enumerate}
  \setcounter{enumi}{4}
  \item $f(x) \in C[a, b]$, \begin{CJK}{UTF8}{mj}对\end{CJK} $\varphi(x) \in C[a, b]$. \begin{CJK}{UTF8}{mj}如果\end{CJK} $\int_{a}^{b} \varphi(x) \mathrm{d} x=0$, \begin{CJK}{UTF8}{mj}那么\end{CJK} $\int_{a}^{b} f(x) \varphi(x) \mathrm{d} x=0$. \begin{CJK}{UTF8}{mj}证明\end{CJK}:
\end{enumerate}
(1) \begin{CJK}{UTF8}{mj}存在\end{CJK} $c \in[a, b]$, \begin{CJK}{UTF8}{mj}使得\end{CJK} $\int_{a}^{b} f^{2}(x) \mathrm{d} x=f(c) \int_{a}^{b} f(x) \mathrm{d} x$.

(2) $f$ \begin{CJK}{UTF8}{mj}在\end{CJK} $[a, b]$ \begin{CJK}{UTF8}{mj}上为常值函数\end{CJK}.

\begin{enumerate}
  \setcounter{enumi}{5}
  \item $f(x)$ \begin{CJK}{UTF8}{mj}在\end{CJK} $[a, b]$ \begin{CJK}{UTF8}{mj}上有二阶导数\end{CJK}, $f^{\prime}(b)=-f^{\prime}(a)$.
\end{enumerate}
(1) \begin{CJK}{UTF8}{mj}分别用\end{CJK} $a, b$ \begin{CJK}{UTF8}{mj}点的\end{CJK} Taylor \begin{CJK}{UTF8}{mj}公式表达\end{CJK} $f\left(\frac{a+b}{2}\right)$.

(2) \begin{CJK}{UTF8}{mj}证明\end{CJK}: \begin{CJK}{UTF8}{mj}存在\end{CJK} $\xi \in[a, b]$, \begin{CJK}{UTF8}{mj}使得\end{CJK}
$$
\left|f^{\prime \prime}(\xi)\right|=\frac{4|f(b)-f(a)|}{(b-a)^{2}}
$$

\begin{enumerate}
  \setcounter{enumi}{6}
  \item $\sum_{n=1}^{\infty}\left(a_{n}-a_{n-1}\right)$ \begin{CJK}{UTF8}{mj}绝对收玫\end{CJK}, $\sum_{n=1}^{\infty} b_{n}$ \begin{CJK}{UTF8}{mj}收銈\end{CJK}, \begin{CJK}{UTF8}{mj}证明\end{CJK} $\sum_{n=1}^{\infty} a_{n} b_{n}$ \begin{CJK}{UTF8}{mj}收玫\end{CJK}.
\end{enumerate}
7.(1) $f(x)=\sum_{n=0}^{\infty} a_{n} x^{n}$ \begin{CJK}{UTF8}{mj}收玫半径为\end{CJK} $R>0$. \begin{CJK}{UTF8}{mj}求\end{CJK} $F(x)=\frac{f(x)}{1-x}$ \begin{CJK}{UTF8}{mj}的幂级数展开式\end{CJK}, \begin{CJK}{UTF8}{mj}并讨论二级数的收敛半隹\end{CJK}

(2) \begin{CJK}{UTF8}{mj}若\end{CJK} $\varlimsup_{n \rightarrow \infty} \sqrt[n]{\left|a_{n}\right|}=1, S_{n}=\sum_{k=1}^{n} a_{k}$, \begin{CJK}{UTF8}{mj}证明\end{CJK}: $\varlimsup_{n \rightarrow \infty} \sqrt[n]{\left|S_{n}\right|}=1$.

\begin{enumerate}
  \setcounter{enumi}{8}
  \item \begin{CJK}{UTF8}{mj}对参数\end{CJK} $\alpha \in(-\infty,+\infty)$, \begin{CJK}{UTF8}{mj}讨论\end{CJK} $f_{n}(x)=n^{\alpha} x \mathrm{e}^{-n x}$ \begin{CJK}{UTF8}{mj}在\end{CJK} $[0,1]$ \begin{CJK}{UTF8}{mj}上的收玫性和一致收玫性\end{CJK}.
\end{enumerate}
9.(1) $u=u(x, y)$ \begin{CJK}{UTF8}{mj}由以下方程组确定\end{CJK}:
$$
\left\{\begin{array}{l}
u=f(x, y, z, t) \\
y+\sin z-2 t=0 \\
2 z+\cos t=0
\end{array}\right.
$$
\begin{CJK}{UTF8}{mj}求\end{CJK} $\frac{\partial u}{\partial y}$.

(2) $z$ \begin{CJK}{UTF8}{mj}轴正向向上\end{CJK}, \begin{CJK}{UTF8}{mj}求\end{CJK} $2 x^{2}+2 y^{2}-4 x-2 y+4 x y+2 z^{2}-6 z+7=0$ \begin{CJK}{UTF8}{mj}的最高点和最低点\end{CJK}.

(3) \begin{CJK}{UTF8}{mj}设\end{CJK} $\Gamma$ \begin{CJK}{UTF8}{mj}是\end{CJK} $\mathbb{R}^{2}$ \begin{CJK}{UTF8}{mj}上没有自交点的曲线\end{CJK}, \begin{CJK}{UTF8}{mj}方向取逆时针\end{CJK}. \begin{CJK}{UTF8}{mj}求\end{CJK}
$$
\iint_{\Gamma} \frac{3 x x \mathrm{~d} y-3 y \mathrm{~d} x}{3 x^{2}+5 y^{2}}
$$
$\vec{F}=\left(z^{2}, \underline{x}^{2}, \underline{y}^{2}\right)$ \begin{CJK}{UTF8}{mj}绕\end{CJK} $\partial \Sigma$ \begin{CJK}{UTF8}{mj}一周所做功\end{CJK}.

\section{$11.2$ 高等代数}
\begin{enumerate}
  \item $V_{1}, V_{2}$ \begin{CJK}{UTF8}{mj}是线性空间\end{CJK} $V$ \begin{CJK}{UTF8}{mj}的两个子空间\end{CJK}. \begin{CJK}{UTF8}{mj}证明\end{CJK}:

  \item (1) $V_{1} \cap V_{2}=\{0\}$ \begin{CJK}{UTF8}{mj}等价于\end{CJK} $V_{1}+V_{2}$ \begin{CJK}{UTF8}{mj}是直和\end{CJK}.

\end{enumerate}
(2) $V_{1} \cup V_{2}=V$ \begin{CJK}{UTF8}{mj}等价于\end{CJK} $V_{1}=V$ \begin{CJK}{UTF8}{mj}或\end{CJK} $V_{2}=V$.

\includegraphics[max width=\textwidth]{2022_04_18_7db0708508f26638f054g-192}

(1) \begin{CJK}{UTF8}{mj}证明\end{CJK} $A$ \begin{CJK}{UTF8}{mj}的不变因子为\end{CJK} $1,1, \cdots, 1, f(\lambda)(2)$ \begin{CJK}{UTF8}{mj}证明\end{CJK} $A$ \begin{CJK}{UTF8}{mj}可对角化的充要条件是\end{CJK} $A$ \begin{CJK}{UTF8}{mj}有\end{CJK} $n$ \begin{CJK}{UTF8}{mj}个互不相同的\end{CJK}! 4. \begin{CJK}{UTF8}{mj}矩阵\end{CJK} $A$ \begin{CJK}{UTF8}{mj}的列向量分别为\end{CJK} $\alpha_{1}, \alpha_{2}, \cdots, \alpha_{n}$, \begin{CJK}{UTF8}{mj}行向旺分别为\end{CJK} $\beta_{1}, \beta_{2}, \cdots, \beta_{m}$.

\begin{CJK}{UTF8}{mj}证明\end{CJK}: $\forall \gamma \in \mathbb{R}$ \begin{CJK}{UTF8}{mj}方程\end{CJK} $k_{1} \alpha_{1}+k_{2} \alpha_{2}+\cdots+k_{n} \alpha_{n}=\gamma$ \begin{CJK}{UTF8}{mj}有解的充要条件是\end{CJK} $\beta_{1}, \beta_{2}, \cdots, \beta_{m}$ \begin{CJK}{UTF8}{mj}线性无关\end{CJK}.

\includegraphics[max width=\textwidth]{2022_04_18_7db0708508f26638f054g-192(1)}\\
$\cdots+x_{n} \alpha_{n}=\beta$,

(1) $\beta$ \begin{CJK}{UTF8}{mj}满足何条件时方程组有解\end{CJK}, \begin{CJK}{UTF8}{mj}并求解\end{CJK}.

(2) \begin{CJK}{UTF8}{mj}若方程组的解构成线性空间\end{CJK}, \begin{CJK}{UTF8}{mj}求需满足的条件和线性空间\end{CJK}.

\begin{enumerate}
  \setcounter{enumi}{6}
  \item \begin{CJK}{UTF8}{mj}矩阵\end{CJK} $A$ \begin{CJK}{UTF8}{mj}的任意\end{CJK} $k$ \begin{CJK}{UTF8}{mj}阶顺序主子式不为\end{CJK} $0, k=1,2, \cdots, n-1$.
\end{enumerate}
(1) \begin{CJK}{UTF8}{mj}证明存在下三角矩阵\end{CJK} $B$, \begin{CJK}{UTF8}{mj}使得\end{CJK} $B A$ \begin{CJK}{UTF8}{mj}为上三角矩阵\end{CJK}.

(2) \begin{CJK}{UTF8}{mj}证明\end{CJK} $A$ \begin{CJK}{UTF8}{mj}可分解为下三角矩阵\end{CJK} $L$, \begin{CJK}{UTF8}{mj}上三角矩阵\end{CJK} $U$ \begin{CJK}{UTF8}{mj}的积\end{CJK}.

\begin{enumerate}
  \setcounter{enumi}{7}
  \item $\sigma$ \begin{CJK}{UTF8}{mj}为\end{CJK} $\mathbb{R}^{n}$ \begin{CJK}{UTF8}{mj}中的线性卒换\end{CJK}, \begin{CJK}{UTF8}{mj}证明以下夾命题等价\end{CJK}.
\end{enumerate}
(1) $(\sigma(\alpha), \sigma(\beta))=(\alpha, \beta), \quad \forall \alpha, \beta \in \mathbb{R}^{n}$

(2) $\sigma$ \begin{CJK}{UTF8}{mj}将标准正交基映射为标准正交基\end{CJK}.

\begin{enumerate}
  \setcounter{enumi}{8}
  \item \begin{CJK}{UTF8}{mj}已知多项式\end{CJK} $f(x), g(x)$, \begin{CJK}{UTF8}{mj}设\end{CJK} $f(x)=(f(x), g(x)) f_{1}(x), g(x)=(f(x), g(x)) g_{1}(x)$. \begin{CJK}{UTF8}{mj}证明\end{CJK}: \begin{CJK}{UTF8}{mj}存在多项式\end{CJK} $u(x), v(x)$ \begin{CJK}{UTF8}{mj}使得\end{CJK}, $u(x) f_{1}(x)+v(x) g_{1}(x)=1$ \begin{CJK}{UTF8}{mj}且\end{CJK} $\partial(u(x))<\partial\left(g_{1}(x)\right), \partial(v(x))<$

  \item \begin{CJK}{UTF8}{mj}多项式\end{CJK} $f(x)=a_{n} x^{n}+a_{n-1} x^{n-1}+\cdots+a_{1} x+a_{0}$, \begin{CJK}{UTF8}{mj}除以多项式\end{CJK} $x-1$, \begin{CJK}{UTF8}{mj}得商式\end{CJK} $g(x)=b_{n-1} x^{n-1}+\cdots+b$ \begin{CJK}{UTF8}{mj}余式\end{CJK} $r$.

\end{enumerate}
(1) \begin{CJK}{UTF8}{mj}求\end{CJK} $M$ \begin{CJK}{UTF8}{mj}使得\end{CJK} $\left(b_{n-1}, b_{n-2}, \cdots b_{0}, r\right)=\left(a_{n}, a_{n-1}, \cdots, a_{0}\right) M$.

(2) \begin{CJK}{UTF8}{mj}求多项式\end{CJK} $x^{n}+x^{n-1}+\cdots+x+1$ \begin{CJK}{UTF8}{mj}除以多项式\end{CJK} $x-1$ \begin{CJK}{UTF8}{mj}所㥂商式和余式\end{CJK}.

\begin{enumerate}
  \setcounter{enumi}{10}
  \item \begin{CJK}{UTF8}{mj}已知\end{CJK} $A, B$ \begin{CJK}{UTF8}{mj}为实对称矩阵\end{CJK}, $B$ \begin{CJK}{UTF8}{mj}为正定矩阵\end{CJK}.
\end{enumerate}
(1) \begin{CJK}{UTF8}{mj}证明存在可逆矩阵\end{CJK} $C$, \begin{CJK}{UTF8}{mj}使得\end{CJK} $C^{\prime} A C, C^{\prime} B C$ \begin{CJK}{UTF8}{mj}同时对角化\end{CJK}.

(2) $A=\left(\begin{array}{lll}1 & 2 & 1 \\ 2 & 4 & 2 \\ 1 & 2 & 1\end{array}\right), B=\left(\begin{array}{ccc}1 & 1 & 0 \\ 1 & 2 & 1 \\ 0 & 1 & 2\end{array}\right)$. \begin{CJK}{UTF8}{mj}求矩阵\end{CJK} $C$, \begin{CJK}{UTF8}{mj}使得\end{CJK} $C^{\prime} A C, C^{\prime} B C$ \begin{CJK}{UTF8}{mj}同时对角化\end{CJK}. 1.(1) $\lim _{x \rightarrow 0} \frac{\left(1+\sin ^{2} x\right)^{2022}-(\cos x)^{2022}}{\ln \left(1+x^{2}\right)}$.

(2) \begin{CJK}{UTF8}{mj}求积分\end{CJK} $\int_{y \geq x^{2}+1} \frac{\mathrm{d} x \mathrm{~d} y}{y^{2}+x^{4}}$.

(3) \begin{CJK}{UTF8}{mj}求\end{CJK} $f(x)=\mathrm{e}^{-\frac{x^{2}}{3}}$ \begin{CJK}{UTF8}{mj}的\end{CJK} Maclaurin \begin{CJK}{UTF8}{mj}公式\end{CJK}, \begin{CJK}{UTF8}{mj}并求岜\end{CJK} $f^{(2022)}(0), f^{(2021)}(0)$.

(4) \begin{CJK}{UTF8}{mj}曲面积分\end{CJK} $\int_{S} \frac{x \mathrm{~d} y \mathrm{~d} z+y \mathrm{~d} z \mathrm{~d} x+z \mathrm{~d} x \mathrm{~d} y}{\left(x^{2}+y^{2}+z^{2}\right)^{3 / 2}}$

$S$ \begin{CJK}{UTF8}{mj}是区域\end{CJK} $V=\{(x, y, z)|| x|\leq z,| y|\leq 2,| z \mid \leq 2$,$} 的外侧.$

\begin{enumerate}
  \setcounter{enumi}{2}
  \item $x_{0}>0, x_{n}=\arctan x_{n-1}, n \geq 1$
\end{enumerate}
(1) \begin{CJK}{UTF8}{mj}证\end{CJK}: $\lim _{n \rightarrow \infty} x_{n}=0$ (\begin{CJK}{UTF8}{mj}求\end{CJK} $\left\{x_{n}\right\}$ \begin{CJK}{UTF8}{mj}的极限\end{CJK})

(2) \begin{CJK}{UTF8}{mj}证\end{CJK}:\begin{CJK}{UTF8}{mj}数列\end{CJK} $\left\{\sqrt{n} x_{n}\right\}$ \begin{CJK}{UTF8}{mj}收敛并求极限值\end{CJK}

\begin{enumerate}
  \setcounter{enumi}{3}
  \item $f(x)=x^{\alpha} \ln x$, \begin{CJK}{UTF8}{mj}在实数域\end{CJK} $\mathbb{R}$ \begin{CJK}{UTF8}{mj}上\end{CJK}, \begin{CJK}{UTF8}{mj}证明\end{CJK}: \begin{CJK}{UTF8}{mj}一致连续\end{CJK} $\Leftrightarrow \alpha>1$
\end{enumerate}
$4 . a_{n}>0, \frac{a_{n}}{a_{n+1}}=1+\frac{\alpha}{n}+O\left(\frac{1}{n^{2}}\right), O\left(\frac{1}{n^{2}}\right)$ \begin{CJK}{UTF8}{mj}表示存在\end{CJK} $M>0$, \begin{CJK}{UTF8}{mj}对所有的\end{CJK} $n$ \begin{CJK}{UTF8}{mj}戌立\end{CJK} $\left|O\left(\frac{1}{n^{2}}\right)\right|<\frac{M}{n^{2}}$.

\begin{CJK}{UTF8}{mj}证明\end{CJK}:\begin{CJK}{UTF8}{mj}级数\end{CJK} $\sum_{n=1}^{\infty} a_{n}$ \begin{CJK}{UTF8}{mj}收敛\end{CJK} $\Leftrightarrow \alpha>1$

$5 . f(x), g(x)$ \begin{CJK}{UTF8}{mj}在\end{CJK} $[0,+\infty)$ \begin{CJK}{UTF8}{mj}上连续\end{CJK}, \begin{CJK}{UTF8}{mj}且\end{CJK} $\lim _{x \rightarrow \infty} \frac{f(x)}{g(x)}=1$, \begin{CJK}{UTF8}{mj}雮积分\end{CJK} $\int_{0}^{+\infty} g(x) \mathrm{d} x$ \begin{CJK}{UTF8}{mj}收敛\end{CJK}.

(1) \begin{CJK}{UTF8}{mj}证明\end{CJK}: \begin{CJK}{UTF8}{mj}广义积分\end{CJK} $\int_{0}^{+\infty} f(x) \mathrm{d} x$ \begin{CJK}{UTF8}{mj}也收佥败\end{CJK}.

(2) \begin{CJK}{UTF8}{mj}如果没有连续性条件\end{CJK}, \begin{CJK}{UTF8}{mj}举例说明\end{CJK} $\int_{0}^{+\infty} f(x) \mathrm{d} x$ \begin{CJK}{UTF8}{mj}发散\end{CJK}.

6.(1) \begin{CJK}{UTF8}{mj}叙述在实数域\end{CJK} $\mathbb{R}$ \begin{CJK}{UTF8}{mj}上的有限覆直坓理和致筞性定理\end{CJK}

(2) \begin{CJK}{UTF8}{mj}用有限覆盖定理证明致密性定理\end{CJK}.

\begin{enumerate}
  \setcounter{enumi}{7}
  \item \begin{CJK}{UTF8}{mj}证\end{CJK}:\begin{CJK}{UTF8}{mj}函数\end{CJK} $G(p)=\int_{0}^{\pi} \frac{\sin x}{x^{p}(\pi-x)^{2-p}} \mathrm{~d} x$ \begin{CJK}{UTF8}{mj}在\end{CJK} $(0,2)$ \begin{CJK}{UTF8}{mj}内连续\end{CJK}.

  \item $f(x)$ \begin{CJK}{UTF8}{mj}单调连续\end{CJK}, \begin{CJK}{UTF8}{mj}在实数域上\end{CJK}, $f(x+1)=f(x)+1, f^{n}(x)$ \begin{CJK}{UTF8}{mj}表示\end{CJK} $f(x)$ \begin{CJK}{UTF8}{mj}的\end{CJK} $n$ \begin{CJK}{UTF8}{mj}次复合\end{CJK}, $\varphi=f^{n}(x)-x$.

\end{enumerate}
(1) \begin{CJK}{UTF8}{mj}证明\end{CJK}: $\forall n \geq 1, \varphi_{n}(x)$ \begin{CJK}{UTF8}{mj}为周期函数\end{CJK}

(2) \begin{CJK}{UTF8}{mj}证明\end{CJK}: $\forall x \in \mathbb{R}, \lim _{n \rightarrow \infty} \frac{\varphi_{n}(x)}{n}$ \begin{CJK}{UTF8}{mj}的极限值与\end{CJK} $x$ \begin{CJK}{UTF8}{mj}的取值无关\end{CJK}.

\section{$12.2$ 高等代数}
$1 . a_{n \times n}, A_{i j}$ \begin{CJK}{UTF8}{mj}代数余子式\end{CJK}. \begin{CJK}{UTF8}{mj}假设\end{CJK} $\forall i, j \leq n$, \begin{CJK}{UTF8}{mj}有\end{CJK} $A_{i j}=-i a_{i j}$, \begin{CJK}{UTF8}{mj}求\end{CJK} $|A|$.

$2 . b, c$ \begin{CJK}{UTF8}{mj}取何值时\end{CJK}, \begin{CJK}{UTF8}{mj}方程组有解\end{CJK}. \begin{CJK}{UTF8}{mj}若有解\end{CJK}, \begin{CJK}{UTF8}{mj}求解集和一组极大线性无关组\end{CJK}.
$$
\left\{\begin{array}{l}
x_{1}+x_{2}+x_{3}+x_{4}+x_{5}=1 \\
3 x_{1}+3 x_{2}+x_{3}+x_{4}-x_{5}=b \\
x_{2}+2 x_{3}+2 x_{4}+6 x_{5}=3 \\
5 x_{1}+4 x_{2}+3 x_{3}+3 x_{4}+x_{5}=c
\end{array}\right.
$$
3.R $\left(x_{1}, x_{2}, \cdots, x_{n}\right)=\frac{f\left(x_{1}, x_{2}, \cdots, x_{n}\right)}{g\left(x_{1}, x_{2}, \cdots, x_{n}\right)}, f, g$ \begin{CJK}{UTF8}{mj}为多元多项式\end{CJK}.\begin{CJK}{UTF8}{mj}假设\end{CJK} $\forall i, j \leq n$, \begin{CJK}{UTF8}{mj}成立\end{CJK} $R\left(x_{1}, \cdots, x_{i}, \cdots, x_{j} ,\right.$ $R\left(x_{1}, \cdots, x_{j}, \cdots, x_{i}, \cdots, x_{n}\right)$. \begin{CJK}{UTF8}{mj}证明\end{CJK}:\begin{CJK}{UTF8}{mj}存在初等对称多项式\end{CJK}, $u, v$, \begin{CJK}{UTF8}{mj}戒立\end{CJK} $R=\frac{u}{v}$. $5 . \varepsilon_{1}=(1,1,0)^{\prime}, \varepsilon_{2}=(0,1,1)^{\prime}, \varepsilon_{3}=(0,1,-1)^{\prime} . V$ \begin{CJK}{UTF8}{mj}是\end{CJK} $\mathbb{R}^{3}$ \begin{CJK}{UTF8}{mj}的对偶空间\end{CJK}. $\sigma$ \begin{CJK}{UTF8}{mj}是线性变换\end{CJK}, $\sigma\left(\begin{array}{l}x_{1} \\ x_{2} \\ x_{3}\end{array}\right)=\left(\begin{array}{l}1 \\ 0 \\ 2\end{array} \quad 1\right.$ \begin{CJK}{UTF8}{mj}对偶空间上的映射\end{CJK} $\tau, \tau(f)=f \circ \sigma, \forall f \in V$.

(1) \begin{CJK}{UTF8}{mj}验证\end{CJK} $\varepsilon_{1}, \varepsilon_{2}, \varepsilon_{3}$ \begin{CJK}{UTF8}{mj}是\end{CJK} $\mathbb{R}^{3}$ \begin{CJK}{UTF8}{mj}的一组其\end{CJK}.

(2) \begin{CJK}{UTF8}{mj}求\end{CJK} $\varepsilon_{1}, \varepsilon_{2}, \varepsilon_{3}$ \begin{CJK}{UTF8}{mj}的对偶基\end{CJK}.

(3) \begin{CJK}{UTF8}{mj}求变换\end{CJK} $\tau$ \begin{CJK}{UTF8}{mj}在该对偶基下的矩阵\end{CJK}.

\begin{enumerate}
  \setcounter{enumi}{6}
  \item $A=\left(\begin{array}{llll}0 & 1 & 2 & 3 \\ 0 & 0 & 1 & 2 \\ 0 & 0 & 0 & 1 \\ 0 & 0 & 0 & 0\end{array}\right)$. \begin{CJK}{UTF8}{mj}证明\end{CJK}:
\end{enumerate}
(1) \begin{CJK}{UTF8}{mj}不存在矩阵\end{CJK} $B$, \begin{CJK}{UTF8}{mj}使得\end{CJK} $B^{2}=A$.

(2) \begin{CJK}{UTF8}{mj}存在实矩阵\end{CJK} $S, S^{2}=I+A$.

\begin{enumerate}
  \setcounter{enumi}{7}
  \item $A_{n \times n}, a_{i j}=\left\{\begin{array}{l}2, i=j+1 \\ 1, \text { 其他 }\end{array}\right.$, \begin{CJK}{UTF8}{mj}求\end{CJK} $A$ \begin{CJK}{UTF8}{mj}㥂至特征多项式和极小多项式\end{CJK}.

  \item $A_{3 \times 3}$ \begin{CJK}{UTF8}{mj}实对称阵\end{CJK}, $|A|=-1$. \begin{CJK}{UTF8}{mj}已知\end{CJK} $A^{2}=I . A$ \begin{CJK}{UTF8}{mj}的两个特征向量为\end{CJK} $\alpha_{1}=(1,1,0)^{\prime}, \alpha_{2}=(0,1,1)^{\prime}$, \begin{CJK}{UTF8}{mj}且\end{CJK} $\alpha_{3}=$ \begin{CJK}{UTF8}{mj}不是特征向是\end{CJK}. \begin{CJK}{UTF8}{mj}求\end{CJK} $A$ \begin{CJK}{UTF8}{mj}及\end{CJK} $\left|A^{2022}+2021 A^{*}\right|$.

  \item $A_{m \times n}$ \begin{CJK}{UTF8}{mj}在数域\end{CJK} $\mathbb{F}$ \begin{CJK}{UTF8}{mj}上的秩为\end{CJK} $r, B_{n \times k}$ \begin{CJK}{UTF8}{mj}在在数域\end{CJK} $\mathbb{F}$ \begin{CJK}{UTF8}{mj}上的秩为\end{CJK} $s . V$ \begin{CJK}{UTF8}{mj}是所有在数域\end{CJK} $\mathbb{F}$ \begin{CJK}{UTF8}{mj}上的\end{CJK} $n \times n$ \begin{CJK}{UTF8}{mj}方阵生成\end{CJK} \begin{CJK}{UTF8}{mj}间\end{CJK}, $W$ \begin{CJK}{UTF8}{mj}是所有在数域\end{CJK} $\mathbb{F}$ \begin{CJK}{UTF8}{mj}上的\end{CJK} $m \times k$ \begin{CJK}{UTF8}{mj}的矩阵牛成的线性空间\end{CJK}. \begin{CJK}{UTF8}{mj}定义\end{CJK} $f: V \rightarrow W, \forall$ \begin{CJK}{UTF8}{mj}映射\end{CJK} $f, \forall X \in V, f(X)=A$ \begin{CJK}{UTF8}{mj}射\end{CJK} $f$ \begin{CJK}{UTF8}{mj}的\end{CJK} $\operatorname{dim} \operatorname{Ker} f$ \begin{CJK}{UTF8}{mj}和\end{CJK} $\operatorname{dim} \operatorname{Im} f$.

  \item $\forall \lambda_{1}, \cdots, \lambda_{n} \in \mathbb{R}$, \begin{CJK}{UTF8}{mj}证明\end{CJK}: \begin{CJK}{UTF8}{mj}存在如下形式的循环复矩阵\end{CJK} $A$, \begin{CJK}{UTF8}{mj}其特征值为\end{CJK} $\lambda_{1}, \lambda_{2}, \cdots, \lambda_{n}$.

\end{enumerate}
\includegraphics[max width=\textwidth]{2022_04_18_7db0708508f26638f054g-194}

\begin{CJK}{UTF8}{mj}一\end{CJK}、\begin{CJK}{UTF8}{mj}选择题\end{CJK} (\begin{CJK}{UTF8}{mj}正确选项可能不唯一\end{CJK}, \begin{CJK}{UTF8}{mj}请写出选项号\end{CJK}. \begin{CJK}{UTF8}{mj}每小题\end{CJK} 5 \begin{CJK}{UTF8}{mj}分\end{CJK}, \begin{CJK}{UTF8}{mj}共\end{CJK} 40 \begin{CJK}{UTF8}{mj}分\end{CJK})

\begin{enumerate}
  \item \begin{CJK}{UTF8}{mj}积分\end{CJK} $\int_{-1}^{1} \frac{\sin x}{e^{x}+e^{-x}} \mathrm{~d} x$ \begin{CJK}{UTF8}{mj}等于\end{CJK} ( )
\end{enumerate}
A. 2 B. 0 C. $2 \int_{0}^{1} \frac{\sin x}{e^{x}+e^{-x}} \mathrm{~d} x$ D. \begin{CJK}{UTF8}{mj}积分不存在\end{CJK}

\begin{enumerate}
  \setcounter{enumi}{2}
  \item \begin{CJK}{UTF8}{mj}设函数\end{CJK} $f: \mathbb{R} \rightarrow \mathbb{R}$ \begin{CJK}{UTF8}{mj}可微且导数有界\end{CJK}, \begin{CJK}{UTF8}{mj}则\end{CJK} ( )
\end{enumerate}
A. $f$ \begin{CJK}{UTF8}{mj}有界\end{CJK} B. $f$ \begin{CJK}{UTF8}{mj}单调增\end{CJK} C. $f$ \begin{CJK}{UTF8}{mj}一致连续\end{CJK} D. $f$ \begin{CJK}{UTF8}{mj}单调减\end{CJK}

\begin{enumerate}
  \setcounter{enumi}{3}
  \item \begin{CJK}{UTF8}{mj}如果\end{CJK} $y=\int_{\sin x}^{x^{2}}(x+t) \mathrm{d} t$, \begin{CJK}{UTF8}{mj}则\end{CJK} $\left.\frac{\mathrm{d} y}{\mathrm{~d} x}\right|_{x=0}$ \begin{CJK}{UTF8}{mj}等于\end{CJK} ( )
\end{enumerate}
A. 5 B. 0 C. 2 D. $-1$

\begin{enumerate}
  \setcounter{enumi}{4}
  \item \begin{CJK}{UTF8}{mj}设连续函数\end{CJK} $f$ \begin{CJK}{UTF8}{mj}满足\end{CJK} $f(a+b-x)=f(x), \forall x \in[a, b]$, \begin{CJK}{UTF8}{mj}则积分\end{CJK} $\int_{a}^{b} x f(x) \mathrm{d} x$ \begin{CJK}{UTF8}{mj}等于\end{CJK} ( )
\end{enumerate}
A. $\frac{a-b}{2} \int_{a}^{b} f(x) \mathrm{d} x$ B. $\int_{a}^{b} f(x) \mathrm{d} x$ C. $\frac{a+b}{2} \int_{a}^{b} f(x) \mathrm{d} x$ D. $(a+b) \int_{a}^{b} f(x) \mathrm{d} x$

\begin{enumerate}
  \setcounter{enumi}{5}
  \item \begin{CJK}{UTF8}{mj}函数\end{CJK} $\sum_{n=1}^{+\infty} \frac{(x+2)^{n}}{2^{n}(n+1)}$ \begin{CJK}{UTF8}{mj}的定义域是\end{CJK} ( )
\end{enumerate}
A. $[-4,0)$ B. $[-4,0]$ C. $(-4,0]$ D. \begin{CJK}{UTF8}{mj}以上都不是\end{CJK}

\begin{enumerate}
  \setcounter{enumi}{6}
  \item \begin{CJK}{UTF8}{mj}由下面哪个条件能够断定数列\end{CJK} $\left\{x_{n}\right\}$ \begin{CJK}{UTF8}{mj}收玫\end{CJK}? \begin{CJK}{UTF8}{mj}答\end{CJK}: ( )
\end{enumerate}
A. $\forall$ \begin{CJK}{UTF8}{mj}正整数\end{CJK} $p \lim _{n \rightarrow \infty}\left(x_{n+p}-x_{n}\right)=0$.

B. $\exists C>0 \forall$ \begin{CJK}{UTF8}{mj}正整数\end{CJK} $N \sum_{n=1}^{N}\left|x_{n}-x_{n+1}\right|<C$.

C. $\exists C>0 \exists r \in(0,1) \forall$ \begin{CJK}{UTF8}{mj}正整数\end{CJK} $n\left|x_{n}-x_{n+1}\right| \leqslant C r^{n}$.

D. \begin{CJK}{UTF8}{mj}对\end{CJK} $\left\{x_{n}\right\}$ \begin{CJK}{UTF8}{mj}的任意两个子列\end{CJK} $\left\{x_{n}^{(1)}\right\}$ \begin{CJK}{UTF8}{mj}和\end{CJK} $\left\{x_{n}^{(2)}\right\}$, \begin{CJK}{UTF8}{mj}都有\end{CJK} $\lim _{n \rightarrow \infty}\left(x_{n}^{(1)}-x_{n}^{(2)}\right)=0$.

\begin{enumerate}
  \setcounter{enumi}{7}
  \item \begin{CJK}{UTF8}{mj}对有界数列\end{CJK} $\left\{x_{n}\right\}$, \begin{CJK}{UTF8}{mj}下面哪个说法可作为\end{CJK} $\limsup _{n \rightarrow \infty} x_{n}=L$ \begin{CJK}{UTF8}{mj}的定义\end{CJK}? \begin{CJK}{UTF8}{mj}答\end{CJK}: ( )
\end{enumerate}
A. $\forall \varepsilon>0$, \begin{CJK}{UTF8}{mj}有无穷多个正整数\end{CJK} $n$ \begin{CJK}{UTF8}{mj}使\end{CJK} $x_{n}>L-\varepsilon$, \begin{CJK}{UTF8}{mj}同时存在至多有限个正整数\end{CJK} $n$ \begin{CJK}{UTF8}{mj}使得\end{CJK} $x_{n} \leqslant L+\varepsilon$.

B. $\lim _{k \rightarrow \infty} \sup _{n \geqslant k} x_{n}$.

C. $\inf _{k \geqslant 1} \sup _{n \geqslant k} x_{n}$.

D. \begin{CJK}{UTF8}{mj}在\end{CJK} $L$ \begin{CJK}{UTF8}{mj}右侧只有\end{CJK} $\left\{x_{n}\right\}$ \begin{CJK}{UTF8}{mj}的有限多项且存在\end{CJK} $\left\{x_{n}\right\}$ \begin{CJK}{UTF8}{mj}的一个子列单调增加趋于\end{CJK} $L$.

\begin{enumerate}
  \setcounter{enumi}{8}
  \item \begin{CJK}{UTF8}{mj}函数\end{CJK} $f$ \begin{CJK}{UTF8}{mj}在\end{CJK} $[a, b]$ \begin{CJK}{UTF8}{mj}上有定义\end{CJK}, \begin{CJK}{UTF8}{mj}在\end{CJK} $(a, b)$ \begin{CJK}{UTF8}{mj}上连续\end{CJK}。\begin{CJK}{UTF8}{mj}下面哪个条件能够断定函数\end{CJK} $f$ \begin{CJK}{UTF8}{mj}在\end{CJK} $[a, b]$ \begin{CJK}{UTF8}{mj}上有最大代\end{CJK}
\end{enumerate}
A. $f(a+0)$ \begin{CJK}{UTF8}{mj}和\end{CJK} $f(b-0)$ \begin{CJK}{UTF8}{mj}匀存在且有限\end{CJK}.

B. $f(a+0)=f(b-0)=-\infty$.

C. $f(a+0)=f(b-0)=L$, \begin{CJK}{UTF8}{mj}且\end{CJK} $\exists x_{0} \in(a, b)$ \begin{CJK}{UTF8}{mj}使得\end{CJK} $f\left(x_{0}\right)>L$.

D. $f(a+0)=f(a), f(b-0)=B$, \begin{CJK}{UTF8}{mj}且\end{CJK} $\exists x_{0} \in[a, b)$ \begin{CJK}{UTF8}{mj}使得\end{CJK} $f\left(x_{0}\right)>B$.

\begin{CJK}{UTF8}{mj}二\end{CJK}、\begin{CJK}{UTF8}{mj}计算题\end{CJK} (\begin{CJK}{UTF8}{mj}每小题\end{CJK} 10 \begin{CJK}{UTF8}{mj}分\end{CJK}, \begin{CJK}{UTF8}{mj}共\end{CJK} 30 \begin{CJK}{UTF8}{mj}分\end{CJK})

\begin{enumerate}
  \item \begin{CJK}{UTF8}{mj}计算积分\end{CJK}
\end{enumerate}
$$
\int_{0}^{+\infty} \frac{x^{m-1}}{1+x^{n}} \mathrm{~d} x
$$
\begin{CJK}{UTF8}{mj}其中\end{CJK} $0<m<n$.

\begin{enumerate}
  \setcounter{enumi}{2}
  \item \begin{CJK}{UTF8}{mj}计算\end{CJK} $\mathbb{R}^{3}$ \begin{CJK}{UTF8}{mj}中曲面\end{CJK} $\frac{x_{1}^{2}}{a_{1}^{2}}+\frac{x_{2}^{2}}{a_{2}^{2}}=\frac{x_{3}^{2}}{a_{3}^{2}}$ \begin{CJK}{UTF8}{mj}与超平面\end{CJK} $x_{3}=a_{3}$ \begin{CJK}{UTF8}{mj}所围雉体的体积\end{CJK}, $a_{1}, a_{2}, a_{3}>0$.

  \item \begin{CJK}{UTF8}{mj}计算积分\end{CJK}

\end{enumerate}
$$
\iint_{S} x^{3} \mathrm{~d} y \mathrm{~d} z+y^{3} \mathrm{~d} z \mathrm{~d} x+z^{3} \mathrm{~d} x \mathrm{~d} y
$$
\begin{CJK}{UTF8}{mj}五\end{CJK}、(\begin{CJK}{UTF8}{mj}本题\end{CJK} 15 \begin{CJK}{UTF8}{mj}分\end{CJK}) \begin{CJK}{UTF8}{mj}设\end{CJK} $f:[0,1] \rightarrow[0,1]$ \begin{CJK}{UTF8}{mj}是一个连续函数\end{CJK}。\begin{CJK}{UTF8}{mj}证明\end{CJK}: \begin{CJK}{UTF8}{mj}方程\end{CJK}
$$
2 x-\int_{0}^{x} f(t) \mathrm{d} t=1
$$
\begin{CJK}{UTF8}{mj}在\end{CJK} $[0,1]$ \begin{CJK}{UTF8}{mj}中有且仅有一个根\end{CJK}。

\begin{CJK}{UTF8}{mj}六\end{CJK}、(\begin{CJK}{UTF8}{mj}本题\end{CJK} 15 \begin{CJK}{UTF8}{mj}分\end{CJK}) \begin{CJK}{UTF8}{mj}设函数\end{CJK} $f: \mathbb{R} \rightarrow \mathbb{R}$ \begin{CJK}{UTF8}{mj}满足\end{CJK}:

(1) $f(1)=1$;

(2) $f^{\prime}(x)=\frac{1}{x^{2}+(f(x))^{2}}, \forall x \geqslant 1$

\begin{CJK}{UTF8}{mj}证明\end{CJK}: $\lim _{x \rightarrow+\infty} f(x)$ \begin{CJK}{UTF8}{mj}存在且小于\end{CJK} $1+\frac{\pi}{4}$.

\begin{CJK}{UTF8}{mj}七\end{CJK}、 (\begin{CJK}{UTF8}{mj}本题\end{CJK} 15 \begin{CJK}{UTF8}{mj}分\end{CJK}) \begin{CJK}{UTF8}{mj}设函数\end{CJK} $f: \mathbb{R} \rightarrow \mathbb{R}$ \begin{CJK}{UTF8}{mj}在\end{CJK} $\mathbb{R} \backslash\left\{x_{0}\right\}$ \begin{CJK}{UTF8}{mj}上有二阶导数\end{CJK}, \begin{CJK}{UTF8}{mj}满足\end{CJK}: \begin{CJK}{UTF8}{mj}当\end{CJK} $x \in\left(-\infty, x_{0}\right)$ \begin{CJK}{UTF8}{mj}时\end{CJK} $f^{\prime}(x)<0$ \begin{CJK}{UTF8}{mj}而当\end{CJK} $x \in\left(x_{0},+\infty\right)$ \begin{CJK}{UTF8}{mj}时\end{CJK} $f^{\prime}(x)>0>f^{\prime \prime}(x)$. \begin{CJK}{UTF8}{mj}证明\end{CJK}: $f$ \begin{CJK}{UTF8}{mj}在\end{CJK} $x_{0}$ \begin{CJK}{UTF8}{mj}处不可微\end{CJK}.

\begin{CJK}{UTF8}{mj}八\end{CJK}、(\begin{CJK}{UTF8}{mj}本题\end{CJK} 15 \begin{CJK}{UTF8}{mj}分\end{CJK}) \begin{CJK}{UTF8}{mj}设\end{CJK} $f(x)=\sin \left(a_{1} x\right)+\sin \left(a_{2} x\right)+\sin \left(a_{3} x\right), a_{1}, a_{2}, a_{3}>0$. \begin{CJK}{UTF8}{mj}证明\end{CJK}: \begin{CJK}{UTF8}{mj}存在数列\end{CJK} $\left\{t_{n}\right\}$ \begin{CJK}{UTF8}{mj}使得\end{CJK} $\lim _{n \rightarrow 0}$ \begin{CJK}{UTF8}{mj}且\end{CJK} $\lim _{n \rightarrow \infty} f\left(x+t_{n}\right)=f(x)$ \begin{CJK}{UTF8}{mj}对\end{CJK} $x \in \mathbb{R}$ \begin{CJK}{UTF8}{mj}一致成立\end{CJK}.

\section{$13.2$ 高等代数}
\begin{CJK}{UTF8}{mj}一\end{CJK}、\begin{CJK}{UTF8}{mj}填空题\end{CJK} (\begin{CJK}{UTF8}{mj}每空\end{CJK} 5 \begin{CJK}{UTF8}{mj}分\end{CJK}, \begin{CJK}{UTF8}{mj}共\end{CJK} 40 \begin{CJK}{UTF8}{mj}分\end{CJK})

1、\begin{CJK}{UTF8}{mj}平面上椭圆\end{CJK} $x^{2}+x y+y^{2}=1$ \begin{CJK}{UTF8}{mj}的长轴方程是\end{CJK} \begin{CJK}{UTF8}{mj}位于\end{CJK} $\mathrm{x}$ \begin{CJK}{UTF8}{mj}轴上方的焦点坐标是\end{CJK}

2、\begin{CJK}{UTF8}{mj}给定空间中四个点\end{CJK} $A B C D$ \begin{CJK}{UTF8}{mj}的坐标\end{CJK} (\begin{CJK}{UTF8}{mj}数据末知\end{CJK}), \begin{CJK}{UTF8}{mj}求\end{CJK} $A$ \begin{CJK}{UTF8}{mj}到\end{CJK} $B C D$ \begin{CJK}{UTF8}{mj}三点所形成的平面的距离\end{CJK}.

3、\begin{CJK}{UTF8}{mj}矩阵\end{CJK} $A=\left(\begin{array}{ll}1 & 3 \\ 0 & 2\end{array}\right)$ \begin{CJK}{UTF8}{mj}的逆\end{CJK} $A^{-1}=$ ,$A$ \begin{CJK}{UTF8}{mj}的最大奇异值\end{CJK} $\sigma=$

4、\begin{CJK}{UTF8}{mj}矩阵\end{CJK} $\left(\begin{array}{lll}0 & 0 & 3 \\ 0 & 2 & 2 \\ 1 & 0 & 5\end{array}\right)$

(\begin{CJK}{UTF8}{mj}第三列可能有误\end{CJK}) \begin{CJK}{UTF8}{mj}可以分解为正交矩阵和对角元为正数的上三角矩阵的乘利\end{CJK} \begin{CJK}{UTF8}{mj}上三角矩阵是\end{CJK}

5、\begin{CJK}{UTF8}{mj}给定线性变换\end{CJK} $\sigma$ \begin{CJK}{UTF8}{mj}在一组基下矩阵\end{CJK} $\left(\begin{array}{ccc}0 & 0 & 1 \\ 1 & 0 & -3 \\ 0 & 1 & 3\end{array}\right)$, \begin{CJK}{UTF8}{mj}求向量\end{CJK} $\alpha$ (\begin{CJK}{UTF8}{mj}给斗了坐标\end{CJK}, \begin{CJK}{UTF8}{mj}数据末知\end{CJK}) \begin{CJK}{UTF8}{mj}所形成的\end{CJK} $\sigma$ \begin{CJK}{UTF8}{mj}间的维数是\end{CJK}

\begin{CJK}{UTF8}{mj}一\end{CJK}、\begin{CJK}{UTF8}{mj}简答题\end{CJK} (\begin{CJK}{UTF8}{mj}炙题\end{CJK} 6 \begin{CJK}{UTF8}{mj}分\end{CJK}, \begin{CJK}{UTF8}{mj}共\end{CJK} 30 \begin{CJK}{UTF8}{mj}分\end{CJK}) \begin{CJK}{UTF8}{mj}需判断正误㚔给出证明或反例\end{CJK}.

1、 $A, B$ \begin{CJK}{UTF8}{mj}均为\end{CJK} $n(n \geqslant 2)$ \begin{CJK}{UTF8}{mj}阶复矩阵\end{CJK}, \begin{CJK}{UTF8}{mj}则\end{CJK} $\operatorname{rank}(A B)=\operatorname{rank}(B A)$.

2 、\begin{CJK}{UTF8}{mj}任意\end{CJK} 2022 \begin{CJK}{UTF8}{mj}阶可逆实对称方阵都存在\end{CJK} 2021 \begin{CJK}{UTF8}{mj}阶非零主子式\end{CJK}.

3 、\begin{CJK}{UTF8}{mj}任意\end{CJK} $n$ \begin{CJK}{UTF8}{mj}阶置换方阵都复相似于对角矩阵\end{CJK}.

4 、\begin{CJK}{UTF8}{mj}对任意矩阵\end{CJK} $A$, \begin{CJK}{UTF8}{mj}设\end{CJK} $V_{1}$ \begin{CJK}{UTF8}{mj}是\end{CJK} $A$ \begin{CJK}{UTF8}{mj}行向量组生成的子空间\end{CJK}, $V_{2}$ \begin{CJK}{UTF8}{mj}是\end{CJK} $A$ \begin{CJK}{UTF8}{mj}列向量组生成的子空间\end{CJK}, \begin{CJK}{UTF8}{mj}则有\end{CJK} $V_{1}=V_{2}$

5 、\begin{CJK}{UTF8}{mj}给定\end{CJK}(\begin{CJK}{UTF8}{mj}应该是有限维\end{CJK})\begin{CJK}{UTF8}{mj}线性空间\end{CJK} $V$ \begin{CJK}{UTF8}{mj}上的线性变换\end{CJK} $\sigma$, \begin{CJK}{UTF8}{mj}则\end{CJK} $V=\operatorname{Im} \sigma \oplus \operatorname{Ker} \sigma$.

\begin{CJK}{UTF8}{mj}三\end{CJK}、\begin{CJK}{UTF8}{mj}解䇼题\end{CJK} (\begin{CJK}{UTF8}{mj}每题\end{CJK} 20 \begin{CJK}{UTF8}{mj}分\end{CJK}, \begin{CJK}{UTF8}{mj}只\end{CJK} 80 \begin{CJK}{UTF8}{mj}分\end{CJK})

1、\begin{CJK}{UTF8}{mj}设\end{CJK} $n$ \begin{CJK}{UTF8}{mj}阶方阵\end{CJK} $A, B$, \begin{CJK}{UTF8}{mj}且\end{CJK} $C=A+B$, \begin{CJK}{UTF8}{mj}证明\end{CJK} $\operatorname{det}(C-A B)=\operatorname{det}(C-B A)$.

2、\begin{CJK}{UTF8}{mj}设\end{CJK} (\begin{CJK}{UTF8}{mj}实系数\end{CJK}?) \begin{CJK}{UTF8}{mj}多项式\end{CJK} $f(x)$ \begin{CJK}{UTF8}{mj}满足\end{CJK} $f^{\prime}(1)$ \begin{CJK}{UTF8}{mj}不等于\end{CJK} 0 , \begin{CJK}{UTF8}{mj}证明存在\end{CJK} $n$ \begin{CJK}{UTF8}{mj}阶\end{CJK} (\begin{CJK}{UTF8}{mj}复\end{CJK}?) \begin{CJK}{UTF8}{mj}方阵\end{CJK} $A$, \begin{CJK}{UTF8}{mj}使得\end{CJK} $f(A)=f($ 3、\begin{CJK}{UTF8}{mj}设线性空间\end{CJK} $V$ \begin{CJK}{UTF8}{mj}上的线性变换\end{CJK} $\sigma$ \begin{CJK}{UTF8}{mj}把基\end{CJK} $\alpha_{1}, \alpha_{2}, \cdots \cdots, \alpha_{n}$ \begin{CJK}{UTF8}{mj}分别映到\end{CJK} $\beta_{1}, \beta_{2}, \cdots \cdots, \beta_{n}$, \begin{CJK}{UTF8}{mj}证明\end{CJK} $\sigma$ \begin{CJK}{UTF8}{mj}在这两\end{CJK} \begin{CJK}{UTF8}{mj}矩阵相等\end{CJK}.

4、\begin{CJK}{UTF8}{mj}设\end{CJK} $V=\mathbb{R}^{2 \times 2}$, \begin{CJK}{UTF8}{mj}对\end{CJK} $V$ \begin{CJK}{UTF8}{mj}上的矩阵\end{CJK} $S$, \begin{CJK}{UTF8}{mj}定义\end{CJK} $V$ \begin{CJK}{UTF8}{mj}上的双线性函数\end{CJK} $\rho(X, Y)=\operatorname{tr}\left(X^{\top} S Y\right)$.

(1) \begin{CJK}{UTF8}{mj}求所有满足条件的\end{CJK} $S$, \begin{CJK}{UTF8}{mj}使得\end{CJK} $\rho$ \begin{CJK}{UTF8}{mj}能够作为\end{CJK} $V$ \begin{CJK}{UTF8}{mj}上的内积\end{CJK}.

(2) \begin{CJK}{UTF8}{mj}求所有满足条件的\end{CJK} $S$, \begin{CJK}{UTF8}{mj}使得\end{CJK} $V$ \begin{CJK}{UTF8}{mj}的线性变换\end{CJK} $\sigma(X)=X^{\top}$ \begin{CJK}{UTF8}{mj}为正交变换\end{CJK}. 1. \begin{CJK}{UTF8}{mj}已知\end{CJK}
$$
f(x, y)=\left\{\begin{array}{l}
x y \frac{x^{2}-y^{2}}{x^{2}+y^{2}},(x, y) \neq(0,0) \\
0,(x, y)=(0,0)
\end{array}\right.
$$
\begin{CJK}{UTF8}{mj}求\end{CJK} $f_{x y}(0,0)$ \begin{CJK}{UTF8}{mj}和\end{CJK} $f_{y x}(0,0)$.

\begin{enumerate}
  \setcounter{enumi}{2}
  \item $f(x, y)$ \begin{CJK}{UTF8}{mj}在\end{CJK} $x^{2}+y^{2} \leq 1$ \begin{CJK}{UTF8}{mj}上有连续一阶偏导\end{CJK}, \begin{CJK}{UTF8}{mj}且\end{CJK} $\frac{\partial^{2} f}{\partial x^{2}}+\frac{\partial^{2} f}{\partial y^{2}}=\mathrm{e}^{-\left(x^{2}+y^{2}\right)}$. \begin{CJK}{UTF8}{mj}证明\end{CJK}:
\end{enumerate}
$$
\iint_{x^{2}+y^{2} \leq 1}\left(x \frac{\partial f}{\partial x}+y \frac{\partial f}{\partial y}\right) \mathrm{d} x \mathrm{~d} y=\frac{\pi}{2 \mathrm{e}}
$$

\begin{enumerate}
  \setcounter{enumi}{3}
  \item \begin{CJK}{UTF8}{mj}求\end{CJK} $\sum_{n=1}^{\infty}(-1)^{n} \frac{1}{(2 n+1) !(2 n+1)}$ \begin{CJK}{UTF8}{mj}的和\end{CJK}.

  \item \begin{CJK}{UTF8}{mj}已知参数方程\end{CJK}

\end{enumerate}
\begin{CJK}{UTF8}{mj}求\end{CJK} $y=y(x)$ \begin{CJK}{UTF8}{mj}的极值\end{CJK}.
$$
\left\{\begin{array}{l}
x=\frac{t^{3}}{1+t^{3}} \\
y=\frac{t^{3}-2 t^{2}}{1+t^{2}}
\end{array}\right.
$$

\begin{enumerate}
  \setcounter{enumi}{5}
  \item $f(x)$ \begin{CJK}{UTF8}{mj}在\end{CJK} $(a,+\infty)$ \begin{CJK}{UTF8}{mj}可导\end{CJK}, $\lim _{x \rightarrow \infty}\left[f(x)+x f^{\prime}(x) \ln x\right]=l$, \begin{CJK}{UTF8}{mj}证明\end{CJK} $\lim _{x \rightarrow \infty} f(x)=l$.

  \item \begin{CJK}{UTF8}{mj}证明\end{CJK} $f(x)=\sum_{n=1}^{\infty} \frac{\sin n x}{n^{3}}$ \begin{CJK}{UTF8}{mj}在\end{CJK} $\mathbb{R}$ \begin{CJK}{UTF8}{mj}上连续\end{CJK}, \begin{CJK}{UTF8}{mj}具有连续的导数\end{CJK}.

  \item \begin{CJK}{UTF8}{mj}计算\end{CJK}

\end{enumerate}
$$
I(x)=\int_{0}^{\infty} \mathrm{e}^{-t^{2}} \cos (2 x t) \mathrm{d} t
$$

\begin{enumerate}
  \setcounter{enumi}{8}
  \item $f(x)$ \begin{CJK}{UTF8}{mj}在\end{CJK} $[0,1]$ \begin{CJK}{UTF8}{mj}上具有连续的二阶守数\end{CJK}, $f(0)=f(1)=f^{\prime}(0)=0, f^{\prime}(1)=1 .$ \begin{CJK}{UTF8}{mj}证明\end{CJK}: $\int_{0}^{1}\left[f^{\prime \prime}(x)\right]^{2} \mathrm{~d} x \geq 2$
\end{enumerate}
\section{2 线性代数}
\begin{enumerate}
  \item \begin{CJK}{UTF8}{mj}向量组\end{CJK} $\alpha_{1}, \alpha_{2}, \cdots, \alpha_{m}(m \geq 2), \alpha_{m} \neq 0$. \begin{CJK}{UTF8}{mj}证明\end{CJK}: \begin{CJK}{UTF8}{mj}对任意的\end{CJK} $k_{1}, k_{2} \cdots, k_{m-1}, \beta_{1}=\alpha_{1}+k_{1} \alpha_{m}, \beta$ $k_{2} \alpha_{m}, \cdots, \beta_{m-1}=\alpha_{m-1}+k_{m-1} \alpha_{m} \cdot \beta_{1}, \cdots, \beta_{m}$ \begin{CJK}{UTF8}{mj}线性无关的充要条件是\end{CJK} $\alpha_{1}, \cdots, \alpha_{m}$ \begin{CJK}{UTF8}{mj}线性无关\end{CJK}.
\end{enumerate}
\includegraphics[max width=\textwidth]{2022_04_18_7db0708508f26638f054g-198}

\begin{enumerate}
  \setcounter{enumi}{3}
  \item $\alpha_{1}, \alpha_{2}, \alpha_{3}$ \begin{CJK}{UTF8}{mj}是数域\end{CJK} $\mathbb{K}$ \begin{CJK}{UTF8}{mj}上的线性空间\end{CJK} $V$ \begin{CJK}{UTF8}{mj}的一组其\end{CJK}, $\sigma$ \begin{CJK}{UTF8}{mj}是\end{CJK} $V$ \begin{CJK}{UTF8}{mj}上的线性变换\end{CJK}. $\sigma \alpha_{1} \alpha_{1}, \sigma \alpha_{2}=\alpha_{1}+\alpha_{2}, \sigma \alpha_{3}=\alpha$ \begin{CJK}{UTF8}{mj}证明\end{CJK}:
\end{enumerate}
(1) $\sigma$ \begin{CJK}{UTF8}{mj}是可逆的线性变换\end{CJK}.

(2) \begin{CJK}{UTF8}{mj}求\end{CJK} $2 \sigma-\sigma^{-1}$ \begin{CJK}{UTF8}{mj}在基底\end{CJK} $\sigma_{1}, \sigma_{2}, \sigma_{3}$ \begin{CJK}{UTF8}{mj}下的矩阵\end{CJK}.

\begin{enumerate}
  \setcounter{enumi}{4}
  \item $A \in M_{n}(\mathbb{K}), \lambda_{1}, \cdots, \lambda_{n}$ \begin{CJK}{UTF8}{mj}是\end{CJK} $A$ \begin{CJK}{UTF8}{mj}的特征多项式\end{CJK} $f(\lambda)$ \begin{CJK}{UTF8}{mj}在\end{CJK} $\mathbb{C}$ \begin{CJK}{UTF8}{mj}上的根\end{CJK}. \begin{CJK}{UTF8}{mj}证明\end{CJK}: $\operatorname{tr}(A)=\lambda_{1}+\cdots+\lambda_{n},|A|=\lambda_{1}$.

  \item \begin{CJK}{UTF8}{mj}已知两组向量组的秩相同\end{CJK}, \begin{CJK}{UTF8}{mj}其中一个可由另一个线性表出\end{CJK}.\begin{CJK}{UTF8}{mj}证明两向荲组等价\end{CJK}.

\end{enumerate}
$6 . A=\left(\begin{array}{ccc}1 & 0 & 1 \\ 0 & 2 & 0 \\ -1 & 0 & 1\end{array}\right)$ \begin{CJK}{UTF8}{mj}满足\end{CJK} $A B+I=A^{2}+B$. \begin{CJK}{UTF8}{mj}求\end{CJK} $B$. (1) $A B=O$, \begin{CJK}{UTF8}{mj}则\end{CJK} $B=O$.

(2) $A B=A$, \begin{CJK}{UTF8}{mj}则\end{CJK} $B=I$.

\begin{enumerate}
  \setcounter{enumi}{9}
  \item $A(\lambda)$ \begin{CJK}{UTF8}{mj}是\end{CJK} $n$ \begin{CJK}{UTF8}{mj}阶\end{CJK} $\lambda$-\begin{CJK}{UTF8}{mj}矩阵\end{CJK}, \begin{CJK}{UTF8}{mj}证明\end{CJK} $A(\lambda)$ \begin{CJK}{UTF8}{mj}与\end{CJK} $A(\lambda)^{\prime}$ \begin{CJK}{UTF8}{mj}等价\end{CJK}.
\end{enumerate}
\section{$14.3$ 常微分方程}
\begin{enumerate}
  \item \begin{CJK}{UTF8}{mj}求解方程\end{CJK}
\end{enumerate}
$$
y=x\left(y^{\prime}+\sqrt{1+\left(y^{\prime}\right)^{2}}\right)
$$

\begin{enumerate}
  \setcounter{enumi}{2}
  \item \begin{CJK}{UTF8}{mj}常系数一阶线性方程\end{CJK} $y^{\prime \prime}+\alpha y^{\prime}+\beta y=\gamma \mathrm{e}^{x}$ \begin{CJK}{UTF8}{mj}有特解\end{CJK} $y=\mathrm{e}^{2 x}+(1+x) \mathrm{e}^{x}$. \begin{CJK}{UTF8}{mj}求\end{CJK} $\alpha, \beta, \gamma$, \begin{CJK}{UTF8}{mj}并求方程通解\end{CJK}.

  \item \begin{CJK}{UTF8}{mj}求\end{CJK} $X^{\prime}=A X, A=\left(\begin{array}{ccc}1 & -2 & 1 \\ 0 & -1 & 1 \\ 2 & 1 & 1\end{array}\right)$ \begin{CJK}{UTF8}{mj}的其础解组\end{CJK}.

  \item \begin{CJK}{UTF8}{mj}求解方程\end{CJK}

\end{enumerate}
$$
y=(1+x) y^{\prime}+\left(y^{\prime}\right)^{2}
$$
$5 . D=\{|x-1| \leq 2,|y| \leq 1\} . f(x, y)=x^{2}+2 y,(x, y) \in D$.\begin{CJK}{UTF8}{mj}水初值问题\end{CJK}
$$
\left\{\begin{array}{l}
\frac{\mathrm{d} y}{\mathrm{~d} x}=f(x, y) \\
y(1)=0
\end{array}\right.
$$
\begin{CJK}{UTF8}{mj}在\end{CJK} $D$ \begin{CJK}{UTF8}{mj}上的解的存在区间\end{CJK}、\begin{CJK}{UTF8}{mj}一次近似解和误差估计\end{CJK}.

\includegraphics[max width=\textwidth]{2022_04_18_7db0708508f26638f054g-199}

\begin{CJK}{UTF8}{mj}一\end{CJK} (16 \begin{CJK}{UTF8}{mj}分\end{CJK}) \begin{CJK}{UTF8}{mj}设\end{CJK} $\lim _{n \rightarrow \infty} x_{n}=a, \lim _{n \rightarrow \infty} y_{n}=b$, \begin{CJK}{UTF8}{mj}证明\end{CJK}:
$$
\begin{aligned}
&\lim _{n \rightarrow \infty} \frac{x_{1} y_{n}+x_{2} y_{n-1}+\cdots+x_{n} y_{1}}{n}=a b \\
&\text { ) 证明: } \lim _{x \rightarrow 1^{-}}\left(\sqrt{\frac{1}{1-x}+1}-\sqrt{\frac{1}{1-x}-1}\right)=0 .
\end{aligned}
$$
\begin{CJK}{UTF8}{mj}三\end{CJK} (16 \begin{CJK}{UTF8}{mj}分\end{CJK}) \begin{CJK}{UTF8}{mj}证明\end{CJK}: \begin{CJK}{UTF8}{mj}在实数轴\end{CJK} $R$ \begin{CJK}{UTF8}{mj}上满足方程\end{CJK}
$$
f(x+y)=f(x) f(y)
$$
\begin{CJK}{UTF8}{mj}的唯一不恒等于零的连续函数是\end{CJK} $f(x)=a^{x}(a>0$ \begin{CJK}{UTF8}{mj}为常数\end{CJK} $)$.

\begin{CJK}{UTF8}{mj}四\end{CJK} (16 \begin{CJK}{UTF8}{mj}分\end{CJK}) \begin{CJK}{UTF8}{mj}设\end{CJK} $f(x)$ \begin{CJK}{UTF8}{mj}在\end{CJK} $(0,1)$ \begin{CJK}{UTF8}{mj}上有定义\end{CJK}, \begin{CJK}{UTF8}{mj}且函数\end{CJK} $e^{x} f(x)$ \begin{CJK}{UTF8}{mj}与\end{CJK} $e^{-f(x)}$ \begin{CJK}{UTF8}{mj}在\end{CJK} $(0,1)$ \begin{CJK}{UTF8}{mj}上单调不减\end{CJK}, \begin{CJK}{UTF8}{mj}证明\end{CJK} $f(\mathrm{x})$ \begin{CJK}{UTF8}{mj}在\end{CJK} \begin{CJK}{UTF8}{mj}续\end{CJK}.

\begin{CJK}{UTF8}{mj}五\end{CJK} (16 \begin{CJK}{UTF8}{mj}分\end{CJK}) \begin{CJK}{UTF8}{mj}已知函数\end{CJK} $f(x)$ \begin{CJK}{UTF8}{mj}在区间\end{CJK} $[0,1]$ \begin{CJK}{UTF8}{mj}内具有二阶连续导数\end{CJK}, \begin{CJK}{UTF8}{mj}且满足\end{CJK} $f(0)=f(1),\left|f^{\prime \prime}(x)\right| \leq M$. \begin{CJK}{UTF8}{mj}试诗\end{CJK} $x \in[0,1]$ \begin{CJK}{UTF8}{mj}内有\end{CJK} $\left|f^{\prime}(x)\right| \leq \frac{M}{2}$.

\begin{CJK}{UTF8}{mj}六\end{CJK} (16 \begin{CJK}{UTF8}{mj}分\end{CJK}) \begin{CJK}{UTF8}{mj}证明\end{CJK}: \begin{CJK}{UTF8}{mj}对于任何正实数\end{CJK} $a, b, c$ \begin{CJK}{UTF8}{mj}有下面不等式成立\end{CJK}:
$$
a^{2} b^{2} c \leq 16\left(\frac{a+b+c}{5}\right)
$$
\begin{CJK}{UTF8}{mj}七\end{CJK} (20 \begin{CJK}{UTF8}{mj}分\end{CJK}) \begin{CJK}{UTF8}{mj}设\end{CJK} $\left\{f_{n}(x)\right\}$ \begin{CJK}{UTF8}{mj}是定义在\end{CJK} $[a, b]$ \begin{CJK}{UTF8}{mj}上的无穷次可微的函数列\end{CJK}, \begin{CJK}{UTF8}{mj}且逐点收玫\end{CJK}. \begin{CJK}{UTF8}{mj}并且在\end{CJK} $[a, b]$ \begin{CJK}{UTF8}{mj}上满足\end{CJK} $\mid f_{n}^{\prime}$

(1) \begin{CJK}{UTF8}{mj}证明\end{CJK}: $\left\{f_{n}(x)\right\}$ \begin{CJK}{UTF8}{mj}在\end{CJK} $[a, b]$ \begin{CJK}{UTF8}{mj}上一致收玫\end{CJK}.

(2) \begin{CJK}{UTF8}{mj}设\end{CJK} $f(x)=\lim _{n \rightarrow \infty} f_{n}(x)$, \begin{CJK}{UTF8}{mj}问\end{CJK}: $f(x)$ \begin{CJK}{UTF8}{mj}是否一定在\end{CJK} $[a, b]$ \begin{CJK}{UTF8}{mj}上处处可导\end{CJK}, \begin{CJK}{UTF8}{mj}如果是请给出证明\end{CJK}, \begin{CJK}{UTF8}{mj}如果不是请举\end{CJK} \begin{CJK}{UTF8}{mj}八\end{CJK} (18 \begin{CJK}{UTF8}{mj}分\end{CJK}) \begin{CJK}{UTF8}{mj}设函数\end{CJK} $f(x, y)$ \begin{CJK}{UTF8}{mj}在点\end{CJK} $\left(x_{0}, y_{0}\right)$ \begin{CJK}{UTF8}{mj}的邻域上二次连续可微\end{CJK}, \begin{CJK}{UTF8}{mj}且有\end{CJK} $f_{x}\left(x_{0}, y_{0}\right)=0, f_{x x}\left(x_{0}, y_{0}\right)>0$.

(1) \begin{CJK}{UTF8}{mj}证明\end{CJK}: \begin{CJK}{UTF8}{mj}存在\end{CJK} $y_{0}$ \begin{CJK}{UTF8}{mj}的\end{CJK} $\delta$ \begin{CJK}{UTF8}{mj}域\end{CJK} $U\left(y_{0}, \delta\right)$, \begin{CJK}{UTF8}{mj}使对任何\end{CJK} $y \in U\left(y_{0}, \delta\right)$ \begin{CJK}{UTF8}{mj}能求得\end{CJK} $f(x, y)$ \begin{CJK}{UTF8}{mj}关于\end{CJK} $x$ \begin{CJK}{UTF8}{mj}的一个极小值\end{CJK} $g(y)$.

(2) \begin{CJK}{UTF8}{mj}证明\end{CJK}: $g_{y}\left(y_{0}\right)=f_{y}\left(x_{0}, y_{0}\right)$.

\begin{CJK}{UTF8}{mj}九\end{CJK} (16 \begin{CJK}{UTF8}{mj}分\end{CJK}) \begin{CJK}{UTF8}{mj}计算曲面积分\end{CJK}:
$$
J=\iint_{S} y(x-z) \mathrm{d} y \mathrm{~d} z+x^{2} \mathrm{~d} z \mathrm{~d} x+\left(y^{2}+x z\right) \mathrm{d} x \mathrm{~d} y
$$
\begin{CJK}{UTF8}{mj}其中\end{CJK} $S$ : \begin{CJK}{UTF8}{mj}曲面\end{CJK} $z=5-x^{2}-y^{2}$ \begin{CJK}{UTF8}{mj}上\end{CJK} $z \geq 1$ \begin{CJK}{UTF8}{mj}的部分\end{CJK}, \begin{CJK}{UTF8}{mj}并取外侧\end{CJK}.

\section{$15.2$ 高等代数}
\begin{CJK}{UTF8}{mj}一\end{CJK} (20 \begin{CJK}{UTF8}{mj}分\end{CJK}) \begin{CJK}{UTF8}{mj}记\end{CJK} $K[x]$ \begin{CJK}{UTF8}{mj}是数域\end{CJK} $K$ \begin{CJK}{UTF8}{mj}上的全体多项式\end{CJK}, $f(x), g(x) \in K[x]$.

(1) \begin{CJK}{UTF8}{mj}若\end{CJK} $(f(x), g(x))=1$, \begin{CJK}{UTF8}{mj}且\end{CJK} $\operatorname{deg}(f(x)) \geq 1, \operatorname{deg}(g(x)) \geq 1$, \begin{CJK}{UTF8}{mj}证明\end{CJK}: \begin{CJK}{UTF8}{mj}存在唯一的一对多项式\end{CJK} $u(x), v(x) \in F$ $u(x) f(x)+v(x) g(x)=1$, \begin{CJK}{UTF8}{mj}其中\end{CJK} $\operatorname{deg}(u(x))<\operatorname{deg}(g(x))$, $\operatorname{deg}(v(x))<\operatorname{deg}(f(x))$.

(2) \begin{CJK}{UTF8}{mj}设\end{CJK} $f(x)=x^{3}-x^{2}-x+1, g(x)=x^{2}+x+2$, \begin{CJK}{UTF8}{mj}求多项式\end{CJK} $u(x), v(x)$ \begin{CJK}{UTF8}{mj}使得\end{CJK} $u(x) f(x)+v(x) g(x)=(f(x)$ $\operatorname{deg}(u(x))<\operatorname{deg}(g(x))$, $\operatorname{deg}(v(x))<\operatorname{deg}(f(x))$.

\begin{CJK}{UTF8}{mj}二\end{CJK} (30 \begin{CJK}{UTF8}{mj}分\end{CJK}) \begin{CJK}{UTF8}{mj}记\end{CJK} $M_{m, n}(K)$ \begin{CJK}{UTF8}{mj}是数域\end{CJK} $K$ \begin{CJK}{UTF8}{mj}上的全体\end{CJK} $m \times n$ \begin{CJK}{UTF8}{mj}阶矩阵\end{CJK}.

(1) \begin{CJK}{UTF8}{mj}设\end{CJK} $A \in M_{m, n}(K), B \in M_{m, n}(K)$, \begin{CJK}{UTF8}{mj}证明\end{CJK}: $\operatorname{det}\left(E_{m}+A B\right)=\operatorname{det}\left(E_{n}+B A\right)$ \begin{CJK}{UTF8}{mj}其中\end{CJK} $E_{m}, E_{n}$ \begin{CJK}{UTF8}{mj}分别是\end{CJK} $m$ \begin{CJK}{UTF8}{mj}阶和\end{CJK} \begin{CJK}{UTF8}{mj}位矩阵\end{CJK}, $\operatorname{det}(.)$ \begin{CJK}{UTF8}{mj}表示行列式\end{CJK}.

(2) \begin{CJK}{UTF8}{mj}设\end{CJK} $\alpha, \beta \in K^{n}$ \begin{CJK}{UTF8}{mj}是列向量\end{CJK}, \begin{CJK}{UTF8}{mj}请给出方阵\end{CJK} $A=E_{n}+\alpha \beta^{T}$ \begin{CJK}{UTF8}{mj}是可逆矩阵的充分必要条件\end{CJK}, \begin{CJK}{UTF8}{mj}说明理由并求出\end{CJK} \begin{CJK}{UTF8}{mj}三\end{CJK} (20 \begin{CJK}{UTF8}{mj}分\end{CJK}) \begin{CJK}{UTF8}{mj}设一次型\end{CJK} $f(x)=x A x^{T}$, \begin{CJK}{UTF8}{mj}其中\end{CJK} $x=\left(x_{1}, x_{2}, x_{3}, x_{4}\right), A=\left(\begin{array}{cccc}1 & 0 & 4 & -1 \\ -2 & 1 & -2 & 5 \\ 2 & -2 & 1 & -1 \\ -3 & 1 & -1 & 1\end{array}\right)$

(1)\begin{CJK}{UTF8}{mj}求出二次型\end{CJK} $f(x)$ \begin{CJK}{UTF8}{mj}对应的矩阵\end{CJK}.

(2) \begin{CJK}{UTF8}{mj}用正交线性替换化一次型\end{CJK} $f(x)$ \begin{CJK}{UTF8}{mj}为标准型\end{CJK}, \begin{CJK}{UTF8}{mj}并㐊本该替换\end{CJK}.

\begin{CJK}{UTF8}{mj}四\end{CJK}(30 \begin{CJK}{UTF8}{mj}分\end{CJK}) \begin{CJK}{UTF8}{mj}记\end{CJK} $M_{n}(K)$ \begin{CJK}{UTF8}{mj}是数域\end{CJK} $K$ \begin{CJK}{UTF8}{mj}上的全体\end{CJK} $n$ \begin{CJK}{UTF8}{mj}阶方阵\end{CJK}, $K[x]$ \begin{CJK}{UTF8}{mj}是数域\end{CJK} $K$ \begin{CJK}{UTF8}{mj}上的全体多项式\end{CJK}, \begin{CJK}{UTF8}{mj}给定\end{CJK} $A \in I$ $V=\left\{f(A) \in M_{n}(K) \mid \forall f(x) \in K[x]\right\} .$

(1) \begin{CJK}{UTF8}{mj}证明\end{CJK}: $V$ \begin{CJK}{UTF8}{mj}是\end{CJK} $M_{n}(K)$ \begin{CJK}{UTF8}{mj}的线性子空间\end{CJK}.

(2) \begin{CJK}{UTF8}{mj}若\end{CJK} $f_{0}(x)$ \begin{CJK}{UTF8}{mj}为\end{CJK} $A$ \begin{CJK}{UTF8}{mj}的最小零化多项式\end{CJK} (\begin{CJK}{UTF8}{mj}即\end{CJK} $f_{0}(x) \in K[x]$ \begin{CJK}{UTF8}{mj}是满足\end{CJK} $f(\mathrm{~A})=0$ \begin{CJK}{UTF8}{mj}的非零多项式中次数最\end{CJK} $m=\operatorname{deg}\left(f_{0}\right)$. \begin{CJK}{UTF8}{mj}证明\end{CJK}: $V$ \begin{CJK}{UTF8}{mj}的维数等于\end{CJK} $m$.

\begin{CJK}{UTF8}{mj}若\end{CJK} $A=\left(\begin{array}{ccccc}-1 & 1 & & & \\ & -1 & 1 & & \\ & & -1 & & \\ & & & 2 & 1 \\ & & & & 2\end{array}\right)$, \begin{CJK}{UTF8}{mj}求\end{CJK} $V$ \begin{CJK}{UTF8}{mj}的维数\end{CJK}.

\begin{CJK}{UTF8}{mj}五\end{CJK} (20 \begin{CJK}{UTF8}{mj}分\end{CJK}) \begin{CJK}{UTF8}{mj}设\end{CJK} $V$ \begin{CJK}{UTF8}{mj}是数域\end{CJK} $K$ \begin{CJK}{UTF8}{mj}上的\end{CJK} $n$ \begin{CJK}{UTF8}{mj}维线性空间\end{CJK}, $I$ \begin{CJK}{UTF8}{mj}是\end{CJK} $V$ \begin{CJK}{UTF8}{mj}上的怛等变换\end{CJK}, \begin{CJK}{UTF8}{mj}证明\end{CJK}: $V$ \begin{CJK}{UTF8}{mj}上的线性变换\end{CJK} $A$ \begin{CJK}{UTF8}{mj}可以对\end{CJK} \begin{CJK}{UTF8}{mj}仅当存在互不相同的数\end{CJK} $\lambda_{1}, \lambda_{2}, \cdots, \lambda_{k} \in K$ \begin{CJK}{UTF8}{mj}使得\end{CJK} $\left(\lambda_{1} I-A\right)\left(\lambda_{2} I-A\right) \cdots\left(\lambda_{k} I-A\right)=0$.

\begin{CJK}{UTF8}{mj}只\end{CJK}(30 \begin{CJK}{UTF8}{mj}分\end{CJK}) \begin{CJK}{UTF8}{mj}记\end{CJK} $M_{n}(K)$ \begin{CJK}{UTF8}{mj}是数域\end{CJK} $K$ \begin{CJK}{UTF8}{mj}上的全体\end{CJK} $n$ \begin{CJK}{UTF8}{mj}阶方阵\end{CJK}, $A \in M_{n}(K)$.

(1) \begin{CJK}{UTF8}{mj}证明\end{CJK}: \begin{CJK}{UTF8}{mj}若\end{CJK} $A$ \begin{CJK}{UTF8}{mj}是上三角的正交矩阵\end{CJK}, \begin{CJK}{UTF8}{mj}则\end{CJK} $A$ \begin{CJK}{UTF8}{mj}必定是对角矩阵\end{CJK}, \begin{CJK}{UTF8}{mj}且对角线元素为\end{CJK} $\pm 1$.

(2) \begin{CJK}{UTF8}{mj}证明\end{CJK}: \begin{CJK}{UTF8}{mj}右\end{CJK} $A$ \begin{CJK}{UTF8}{mj}是可逆矩阵\end{CJK}, \begin{CJK}{UTF8}{mj}则存在唯一的正交矩阵\end{CJK} $Q$ \begin{CJK}{UTF8}{mj}与对角线元素大于\end{CJK} 0 \begin{CJK}{UTF8}{mj}的上三角矩阵\end{CJK} $U$, \begin{CJK}{UTF8}{mj}使得\end{CJK}

(3) \begin{CJK}{UTF8}{mj}证明\end{CJK}: \begin{CJK}{UTF8}{mj}若\end{CJK} $A$ \begin{CJK}{UTF8}{mj}是正定矩阵\end{CJK}, \begin{CJK}{UTF8}{mj}则存在\end{CJK} $\underline{二}$ \begin{CJK}{UTF8}{mj}角矩阵\end{CJK} $B$, \begin{CJK}{UTF8}{mj}使得\end{CJK} $A=B^{T} B$. \begin{CJK}{UTF8}{mj}一\end{CJK}、\begin{CJK}{UTF8}{mj}判断题\end{CJK} $(5 \times 4=20$ \begin{CJK}{UTF8}{mj}分\end{CJK} $)$

\begin{enumerate}
  \item \begin{CJK}{UTF8}{mj}设数列\end{CJK} $\left\{x_{n}\right\}$ \begin{CJK}{UTF8}{mj}满足\end{CJK} $\lim _{n \rightarrow \infty} \prod_{k=1}^{n} x_{k}^{\frac{1}{n}}=1$, \begin{CJK}{UTF8}{mj}则\end{CJK} $\left\{x_{n}\right\}$ \begin{CJK}{UTF8}{mj}的极限存在且\end{CJK} $\lim _{n \rightarrow \infty} x_{n}=1$.

  \item \begin{CJK}{UTF8}{mj}设函数\end{CJK} $f(x)$ \begin{CJK}{UTF8}{mj}在\end{CJK} $x_{0}$ \begin{CJK}{UTF8}{mj}点的左\end{CJK}、\begin{CJK}{UTF8}{mj}右导数均存在\end{CJK}, \begin{CJK}{UTF8}{mj}则\end{CJK} $f(x)$ \begin{CJK}{UTF8}{mj}在\end{CJK} $x_{0}$ \begin{CJK}{UTF8}{mj}点连续\end{CJK}.

  \item \begin{CJK}{UTF8}{mj}记不清了\end{CJK}.

  \item \begin{CJK}{UTF8}{mj}设二元函数\end{CJK} $f(x, y)$ \begin{CJK}{UTF8}{mj}在\end{CJK} $\left(x_{0}, y_{0}\right)$ \begin{CJK}{UTF8}{mj}点的邻域内有定义且在\end{CJK} $\left(x_{0}, y_{0}\right)$ \begin{CJK}{UTF8}{mj}点可微\end{CJK}, \begin{CJK}{UTF8}{mj}并且偏导数\end{CJK} $f_{x}$ \begin{CJK}{UTF8}{mj}和\end{CJK} $f_{y}$ \begin{CJK}{UTF8}{mj}在\end{CJK} $\left(x_{0}\right.$ \begin{CJK}{UTF8}{mj}个邻域内存在\end{CJK}, \begin{CJK}{UTF8}{mj}则\end{CJK} $f_{x}$ \begin{CJK}{UTF8}{mj}和\end{CJK} $f_{y}$ \begin{CJK}{UTF8}{mj}在\end{CJK} $\left(x_{0}, y_{0}\right)$ \begin{CJK}{UTF8}{mj}点连续\end{CJK}.

\end{enumerate}
\begin{CJK}{UTF8}{mj}二\end{CJK}、\begin{CJK}{UTF8}{mj}计算\end{CJK} $(5 \times 14=70$ \begin{CJK}{UTF8}{mj}分\end{CJK} $)$

\begin{enumerate}
  \item \begin{CJK}{UTF8}{mj}计算极限\end{CJK}
\end{enumerate}
$$
\lim _{n \rightarrow \infty} \sum_{k=1}^{n} \frac{(2 k-1)^{4}}{n^{5}+k^{4}}
$$

\begin{enumerate}
  \setcounter{enumi}{2}
  \item \begin{CJK}{UTF8}{mj}求函数\end{CJK} $f(x, y, z)=z$ \begin{CJK}{UTF8}{mj}在条件\end{CJK} $2 x^{2}+y^{2}+z^{2}+2 x y-2 x-2 y-4 z+4=0$ \begin{CJK}{UTF8}{mj}下的极值\end{CJK}.

  \item \begin{CJK}{UTF8}{mj}设\end{CJK} $\Omega$ \begin{CJK}{UTF8}{mj}为曲面\end{CJK} $z=1+x^{2}+y^{2}$ \begin{CJK}{UTF8}{mj}和\end{CJK} $z=2\left(x^{2}+y^{2}\right)$ \begin{CJK}{UTF8}{mj}所围的封闭区域\end{CJK}, \begin{CJK}{UTF8}{mj}求重积分\end{CJK}

\end{enumerate}
$$
\iiint_{\Omega} x^{2} \mathrm{~d} x \mathrm{~d} y \mathrm{~d} z
$$

\begin{enumerate}
  \setcounter{enumi}{4}
  \item \begin{CJK}{UTF8}{mj}计算含参广义积分\end{CJK} $I(t)=\int_{1}^{+\infty} \frac{\arctan (t x)}{x^{2} \sqrt{x^{2}-1}} \mathrm{~d} x$.

  \item \begin{CJK}{UTF8}{mj}计算曲面积分\end{CJK}

\end{enumerate}
$$
\iint_{S} x^{3} \mathrm{~d} y \mathrm{~d} z+y^{3} \mathrm{~d} x \mathrm{~d} z+z^{3} \mathrm{~d} x \mathrm{~d} y
$$
\begin{CJK}{UTF8}{mj}其中\end{CJK} $S$ \begin{CJK}{UTF8}{mj}为球面\end{CJK} $x^{2}+y^{2}+z^{2}=2 z$ \begin{CJK}{UTF8}{mj}的外侧\end{CJK}.

\begin{CJK}{UTF8}{mj}三\end{CJK}、\begin{CJK}{UTF8}{mj}证明\end{CJK} $(5 \times 12=60$ \begin{CJK}{UTF8}{mj}分\end{CJK} $)$

\begin{enumerate}
  \item \begin{CJK}{UTF8}{mj}设函数\end{CJK} $f(x)$ \begin{CJK}{UTF8}{mj}为\end{CJK} $[a, b]$ \begin{CJK}{UTF8}{mj}上的凸函数\end{CJK}, \begin{CJK}{UTF8}{mj}证明极限\end{CJK} $\lim _{x \rightarrow a^{+}} f(x)$ \begin{CJK}{UTF8}{mj}和\end{CJK} $\lim _{x \rightarrow b^{-}} f(x)$ \begin{CJK}{UTF8}{mj}均存在\end{CJK}.

  \item \begin{CJK}{UTF8}{mj}已知函数\end{CJK} $f(x)$ \begin{CJK}{UTF8}{mj}在\end{CJK} $[0,+\infty)$ \begin{CJK}{UTF8}{mj}上连续且无穷积分\end{CJK} $\int_{0}^{+\infty} f(x) \mathrm{d} x$ \begin{CJK}{UTF8}{mj}收玫\end{CJK}, \begin{CJK}{UTF8}{mj}证明\end{CJK} $f(x)$ \begin{CJK}{UTF8}{mj}在\end{CJK} $[0,+\infty)$ \begin{CJK}{UTF8}{mj}上一致连续\end{CJK} \begin{CJK}{UTF8}{mj}要条件为\end{CJK} $\lim _{x \rightarrow+\infty} f(x)=0$.

  \item \begin{CJK}{UTF8}{mj}设函数\end{CJK} $f(x)$ \begin{CJK}{UTF8}{mj}在\end{CJK} $[a, b]$ \begin{CJK}{UTF8}{mj}上连续\end{CJK}.

\end{enumerate}
(1) \begin{CJK}{UTF8}{mj}若\end{CJK} $\int_{a}^{b} f(x) \mathrm{d} x=\int_{a}^{b} f(x) e^{x} \mathrm{~d} x=0$, \begin{CJK}{UTF8}{mj}证明\end{CJK} $f(x)$ \begin{CJK}{UTF8}{mj}在\end{CJK} $(a, b)$ \begin{CJK}{UTF8}{mj}内至少存在两个互异的零点\end{CJK};

(2) \begin{CJK}{UTF8}{mj}若\end{CJK} $\int_{a}^{b} f(x) e^{(k-1) x} \mathrm{~d} x=0(k=1,2, \cdots, n+1)$, \begin{CJK}{UTF8}{mj}证明\end{CJK} $f(x)$ \begin{CJK}{UTF8}{mj}在\end{CJK} $(a, b)$ \begin{CJK}{UTF8}{mj}内至少存在\end{CJK} $n+1$ \begin{CJK}{UTF8}{mj}个互异的零点\end{CJK}

\begin{enumerate}
  \setcounter{enumi}{4}
  \item \begin{CJK}{UTF8}{mj}已知函数\end{CJK} $f(x)$ \begin{CJK}{UTF8}{mj}在\end{CJK} $[a, b]$ \begin{CJK}{UTF8}{mj}上黎曼可积\end{CJK}, \begin{CJK}{UTF8}{mj}证明存在\end{CJK} $[a, b]$ \begin{CJK}{UTF8}{mj}上的函数列\end{CJK} $\left\{f_{n}(x)\right\}$ \begin{CJK}{UTF8}{mj}和点列\end{CJK} $a=x_{0}<x_{1}<x$ $x_{m_{n}}=b$, \begin{CJK}{UTF8}{mj}其中\end{CJK} $f_{n}(x)$ \begin{CJK}{UTF8}{mj}在区间\end{CJK} $\left[x_{k-1}, x_{k}\right)$ \begin{CJK}{UTF8}{mj}上的取值为常数\end{CJK} $c_{n}^{k}\left(k=1,2, \cdots, m_{n}\right)$, \begin{CJK}{UTF8}{mj}而且\end{CJK}
\end{enumerate}
$$
\lim _{n \rightarrow \infty} \int_{a}^{b}\left|f_{n}(x)-f(x)\right| \mathrm{d} x=0
$$

\begin{enumerate}
  \setcounter{enumi}{5}
  \item \begin{CJK}{UTF8}{mj}已知级数\end{CJK} $\sum_{k=1}^{\infty} a_{k}$ \begin{CJK}{UTF8}{mj}收玫\end{CJK}, \begin{CJK}{UTF8}{mj}证明\end{CJK} $\lim _{n \rightarrow \infty} \frac{\sum_{k=1}^{n} k a_{k}}{n}=0$.
\end{enumerate}
\section{$16.2$ 高等代数}
\section{一、证明以下问题}
\begin{enumerate}
  \item \begin{CJK}{UTF8}{mj}设\end{CJK} $f(x)$ \begin{CJK}{UTF8}{mj}为一个\end{CJK} $n$ \begin{CJK}{UTF8}{mj}次多项式\end{CJK}, \begin{CJK}{UTF8}{mj}证明\end{CJK} $f^{\prime}(x) \mid f(x)$ \begin{CJK}{UTF8}{mj}的充分必要条件为\end{CJK} $f(x)$ \begin{CJK}{UTF8}{mj}有\end{CJK} $n$ \begin{CJK}{UTF8}{mj}重根\end{CJK};

  \item \begin{CJK}{UTF8}{mj}设\end{CJK} $c$ \begin{CJK}{UTF8}{mj}为一个数\end{CJK} ( $c$ \begin{CJK}{UTF8}{mj}不一定是有理数\end{CJK}), $M=\{f(x) \in \mathbb{Q}[x] \mid f(c)=0\}$. \begin{CJK}{UTF8}{mj}证明\end{CJK} $M$ \begin{CJK}{UTF8}{mj}中存在唯一的首一不可约多项\end{CJK} \begin{CJK}{UTF8}{mj}式\end{CJK} $p(x)$, \begin{CJK}{UTF8}{mj}使得对\end{CJK} $M$ \begin{CJK}{UTF8}{mj}中任意多项式\end{CJK} $f(x)$, \begin{CJK}{UTF8}{mj}都存在多项式\end{CJK} $q(x)$, \begin{CJK}{UTF8}{mj}使得\end{CJK} $f(x)=p(x) q(x)$.

\end{enumerate}
\begin{CJK}{UTF8}{mj}二\end{CJK}、\begin{CJK}{UTF8}{mj}计算下列\end{CJK} $n$ \begin{CJK}{UTF8}{mj}阶行列式\end{CJK}
$$
\text { 1. }\left|\begin{array}{ccccc}
x & a & a & \cdots & a \\
-a & x & a & \cdots & a \\
-a & -a & x & \cdots & a \\
\vdots & \vdots & \vdots & & \vdots \\
-a & -a & -a & \cdots & x
\end{array}\right| \quad 2 .\left|\begin{array}{ccccc}
x_{1}+a_{1} b_{1} & x_{1}+a_{1} b_{2} & x_{1}+a_{1} b_{3} & \cdots & x_{1}+a_{1} b_{n} \\
x_{2}+a_{2} b_{1} & x_{2}+a_{2} b_{2} & x_{2}+a_{2} b_{3} & \cdots & x_{2}+a_{2} b_{n} \\
x_{3}+a_{3} b_{1} & x_{3}+a_{3} b_{2} & x_{3}+a_{3} b_{3} & \cdots & x_{3}+a_{3} b_{n} \\
\vdots & \vdots & \vdots & & \vdots \\
x_{n}+a_{n} b_{1} & x_{n}+a_{n} b_{2} & x_{n}+a_{n} b_{3} & \cdots & x_{n}+a_{n} b_{n}
\end{array}\right| .
$$
\begin{CJK}{UTF8}{mj}三\end{CJK}、\begin{CJK}{UTF8}{mj}设\end{CJK} $A$ \begin{CJK}{UTF8}{mj}为\end{CJK} $m \times n$ \begin{CJK}{UTF8}{mj}矩阵\end{CJK}, $B$ \begin{CJK}{UTF8}{mj}为\end{CJK} $m \times s$ \begin{CJK}{UTF8}{mj}矩阵\end{CJK}, $X$ \begin{CJK}{UTF8}{mj}为末知\end{CJK} $n \times s$ \begin{CJK}{UTF8}{mj}矩阵\end{CJK}.

(1) \begin{CJK}{UTF8}{mj}证明矩阵方程\end{CJK} $A X=B$ \begin{CJK}{UTF8}{mj}有解的充分必要条件为\end{CJK} $\mathrm{r}(A)=\mathrm{r}(A, B)$;

(2) \begin{CJK}{UTF8}{mj}给出矩阵方程有唯一解的充分必要条件\end{CJK}.

\begin{CJK}{UTF8}{mj}四\end{CJK}、(1) \begin{CJK}{UTF8}{mj}设\end{CJK} $A$ \begin{CJK}{UTF8}{mj}为\end{CJK} $s \times n$ \begin{CJK}{UTF8}{mj}矩阵\end{CJK}, $B$ \begin{CJK}{UTF8}{mj}为\end{CJK} $n \times k$ \begin{CJK}{UTF8}{mj}矩阵\end{CJK}, \begin{CJK}{UTF8}{mj}证明\end{CJK} $: \mathrm{r}(A)+\mathrm{r}(B)-n \leq \mathrm{r}(A B)$;

(2) \begin{CJK}{UTF8}{mj}设\end{CJK} $A, B$ \begin{CJK}{UTF8}{mj}为\end{CJK} $n$ \begin{CJK}{UTF8}{mj}级方阵\end{CJK}, \begin{CJK}{UTF8}{mj}满足\end{CJK} $A^{2}=B^{2}=E, A B=B A$, \begin{CJK}{UTF8}{mj}证明存在可逆矩阵使得\end{CJK} $P^{-Y} A P$ \begin{CJK}{UTF8}{mj}和\end{CJK} $P^{-1} B P$ \begin{CJK}{UTF8}{mj}同时为对角线\end{CJK} \begin{CJK}{UTF8}{mj}上元素为\end{CJK} $\pm 1$ \begin{CJK}{UTF8}{mj}的对角矩阵\end{CJK}.

\begin{CJK}{UTF8}{mj}五\end{CJK}、\begin{CJK}{UTF8}{mj}设\end{CJK} $\eta$ \begin{CJK}{UTF8}{mj}是\end{CJK} $n$ \begin{CJK}{UTF8}{mj}维欧式空间\end{CJK} $V$ \begin{CJK}{UTF8}{mj}中一单位向量\end{CJK}, \begin{CJK}{UTF8}{mj}定义变换\end{CJK} $\sigma \alpha=\alpha-2(\eta, \alpha) \eta, \sigma$ \begin{CJK}{UTF8}{mj}称为镜面反射\end{CJK}. \begin{CJK}{UTF8}{mj}证明\end{CJK}:

(1) $\sigma$ \begin{CJK}{UTF8}{mj}是正交变换\end{CJK};

(2) $\sigma$ \begin{CJK}{UTF8}{mj}在\end{CJK} $V$ \begin{CJK}{UTF8}{mj}的一组标准正交基下的矩阵的行列式值为\end{CJK} $-1$

(3) \begin{CJK}{UTF8}{mj}如果\end{CJK} $n$ \begin{CJK}{UTF8}{mj}维欧式空间\end{CJK} $V$ \begin{CJK}{UTF8}{mj}中\end{CJK}, \begin{CJK}{UTF8}{mj}正交变换\end{CJK} $\sigma^{\prime}$ \begin{CJK}{UTF8}{mj}属于特征值\end{CJK} 1 \begin{CJK}{UTF8}{mj}的特征子空间的维数为\end{CJK} $n-1$, \begin{CJK}{UTF8}{mj}那么\end{CJK} $\sigma^{\prime}$ \begin{CJK}{UTF8}{mj}为镜面反射\end{CJK}.

\begin{CJK}{UTF8}{mj}六\end{CJK}、\begin{CJK}{UTF8}{mj}设\end{CJK} $\sigma$ \begin{CJK}{UTF8}{mj}为\end{CJK} $n$ \begin{CJK}{UTF8}{mj}维线性空间\end{CJK} $V$ \begin{CJK}{UTF8}{mj}上的线性变换\end{CJK}, $f(x), g(x)$ \begin{CJK}{UTF8}{mj}为两个多项式\end{CJK}, $h(x)=f(x) g(x)$, \begin{CJK}{UTF8}{mj}证明\end{CJK}:

(1) $\operatorname{Ker}(f(\sigma))+\operatorname{Ker}(g(\sigma)) \subseteq \operatorname{Ker}(h(\sigma))$;

(2) \begin{CJK}{UTF8}{mj}若\end{CJK} $(f(x), g(x))=1$, \begin{CJK}{UTF8}{mj}则\end{CJK} $\operatorname{Ker}(f(\sigma)) \oplus \operatorname{Ker}(g(\sigma))=\operatorname{Ker}(h(\sigma))$.

\begin{CJK}{UTF8}{mj}七\end{CJK}、\begin{CJK}{UTF8}{mj}设\end{CJK} $\sigma$ \begin{CJK}{UTF8}{mj}为\end{CJK} $n$ \begin{CJK}{UTF8}{mj}维线性空间\end{CJK} $V$ \begin{CJK}{UTF8}{mj}上的线牲变换\end{CJK}, $\lambda_{1}, \lambda_{2}, \cdots, \lambda_{m}$ \begin{CJK}{UTF8}{mj}是\end{CJK} $\sigma$ \begin{CJK}{UTF8}{mj}的\end{CJK} $m$ \begin{CJK}{UTF8}{mj}个互不相同的特征值\end{CJK}, \begin{CJK}{UTF8}{mj}对应的特征向量\end{CJK} \begin{CJK}{UTF8}{mj}为\end{CJK} $\alpha_{1}, \alpha_{2}, \cdots, \alpha_{m}$, \begin{CJK}{UTF8}{mj}已知\end{CJK} $\alpha_{1}+\alpha_{2}+\cdots+\alpha_{m}=\alpha \in W$, \begin{CJK}{UTF8}{mj}其中\end{CJK} $W$ \begin{CJK}{UTF8}{mj}为\end{CJK} $\sigma$ \begin{CJK}{UTF8}{mj}的不变子空间\end{CJK}, \begin{CJK}{UTF8}{mj}证明\end{CJK} $\alpha_{i} \in W(i=1,2, \cdots, m)$.

\begin{CJK}{UTF8}{mj}八\end{CJK}、\begin{CJK}{UTF8}{mj}已知对称矩阵\end{CJK}
$$
A=\left(\begin{array}{ccc}
2 & 2 & -2 \\
2 & 5 & -4 \\
-2 & -4 & 5
\end{array}\right)
$$
(1) \begin{CJK}{UTF8}{mj}求正交矩阵\end{CJK} $P$ \begin{CJK}{UTF8}{mj}和对角矩阵\end{CJK} $\Lambda$, \begin{CJK}{UTF8}{mj}使得\end{CJK} $P^{T} A P=\Lambda$;

(2) \begin{CJK}{UTF8}{mj}求对称矩阵\end{CJK} $B$, \begin{CJK}{UTF8}{mj}使得\end{CJK} $A=B^{2}$.

\section{第 17 章 西安交通大学}
\section{$17.1$ 高等代数}
\begin{CJK}{UTF8}{mj}一\end{CJK}、1. \begin{CJK}{UTF8}{mj}计算行列式\end{CJK}
$$
D_{n}(x)=\left|\begin{array}{ccccc}
x & & & & a_{0} \\
-1 & x & & & a_{1} \\
& -1 & \ddots & & \vdots \\
& & \ddots & x & a_{n-2} \\
& & & -1 & a_{n-1}+x
\end{array}\right|
$$

\begin{enumerate}
  \setcounter{enumi}{2}
  \item \begin{CJK}{UTF8}{mj}若\end{CJK} $A$ \begin{CJK}{UTF8}{mj}为幂零矩阵\end{CJK}, \begin{CJK}{UTF8}{mj}即存在正整数\end{CJK} $l$, \begin{CJK}{UTF8}{mj}使得\end{CJK} $A^{l}=O$. \begin{CJK}{UTF8}{mj}证明\end{CJK}: $I-A$ \begin{CJK}{UTF8}{mj}可逆并求其逆\end{CJK}.
\end{enumerate}
\begin{CJK}{UTF8}{mj}二\end{CJK}、\begin{CJK}{UTF8}{mj}若\end{CJK} $A B=B A$, \begin{CJK}{UTF8}{mj}证明存在多项式\end{CJK} $f(x)=c_{0}+c_{1} x+\cdots+c_{n} x^{n}$, \begin{CJK}{UTF8}{mj}使得\end{CJK} $f(A)=B$.

\begin{CJK}{UTF8}{mj}三\end{CJK}、\begin{CJK}{UTF8}{mj}若\end{CJK} $A B-B A=C, A C=C A$. \begin{CJK}{UTF8}{mj}证明\end{CJK}: $\operatorname{tr}\left(C^{k}\right)=0$.

\begin{CJK}{UTF8}{mj}四\end{CJK}、 $A=\left(\begin{array}{ccccc}0 & 1 & & & \\ 1 & \ddots & \ddots & & \\ & \ddots & \ddots & \ddots & \\ & & \ddots & \ddots & 1 \\ & & & 1 & 0\end{array}\right)$. \begin{CJK}{UTF8}{mj}证明\end{CJK} $A$ \begin{CJK}{UTF8}{mj}的特征值均为实数\end{CJK}, \begin{CJK}{UTF8}{mj}査\end{CJK}. $A$ \begin{CJK}{UTF8}{mj}的特征值两两互不相同\end{CJK}.

\begin{CJK}{UTF8}{mj}五\end{CJK}、\begin{CJK}{UTF8}{mj}设\end{CJK} $V$ \begin{CJK}{UTF8}{mj}是实数域上的函数构成的集合\end{CJK}.

\begin{enumerate}
  \item \begin{CJK}{UTF8}{mj}证明\end{CJK} $V$ \begin{CJK}{UTF8}{mj}构成线性空间关于函数的加法和数乘封闭\end{CJK}.

  \item $V_{1}$ : \begin{CJK}{UTF8}{mj}偶数构成的集合\end{CJK}, $V_{2}$ : \begin{CJK}{UTF8}{mj}奇数构成的集合\end{CJK}. \begin{CJK}{UTF8}{mj}证明\end{CJK}: $V=V_{1} \oplus V_{2}$.

\end{enumerate}
\begin{CJK}{UTF8}{mj}六\end{CJK}、 $V$ \begin{CJK}{UTF8}{mj}为实数域上二阶矩阵构成的线性空间\end{CJK}. \begin{CJK}{UTF8}{mj}定义线性映射\end{CJK} $\sigma(x)=A x$.

\begin{enumerate}
  \item \begin{CJK}{UTF8}{mj}证明\end{CJK} $\sigma$ \begin{CJK}{UTF8}{mj}可逆的充要条件是\end{CJK} $A$ \begin{CJK}{UTF8}{mj}可逆\end{CJK}.

  \item $A=\left(\begin{array}{cc}1 & 4 \\ 4 & 16\end{array}\right)$. \begin{CJK}{UTF8}{mj}证明存在\end{CJK} $V$ \begin{CJK}{UTF8}{mj}的\end{CJK} \begin{CJK}{UTF8}{mj}组基\end{CJK}, \begin{CJK}{UTF8}{mj}使得\end{CJK} $A$ \begin{CJK}{UTF8}{mj}变为对角形\end{CJK}.

\end{enumerate}
\begin{CJK}{UTF8}{mj}七\end{CJK}、\begin{CJK}{UTF8}{mj}已知\end{CJK} $V$ \begin{CJK}{UTF8}{mj}是数域\end{CJK} $P$ \begin{CJK}{UTF8}{mj}上的四维空间\end{CJK} $, \alpha_{1}, \alpha_{2}, \alpha_{3}, \alpha_{4}$ \begin{CJK}{UTF8}{mj}为\end{CJK} $V$ \begin{CJK}{UTF8}{mj}的一组基\end{CJK}. \begin{CJK}{UTF8}{mj}定义\end{CJK} $V$ \begin{CJK}{UTF8}{mj}上的线性变换\end{CJK} $\mathscr{T}$. \begin{CJK}{UTF8}{mj}在\end{CJK} $\alpha_{1}, \alpha_{2}, \alpha_{3}, \alpha_{4}$ \begin{CJK}{UTF8}{mj}下的矩阵为\end{CJK} $A$.
$$
\eta_{1}=\alpha_{1}-2 \alpha_{2}+\alpha_{4}, \eta_{2}=\alpha_{2}+\alpha_{3}+2 \alpha_{4}, \eta_{3}=\cdots, \eta_{4}=2 \alpha_{4}
$$
\begin{CJK}{UTF8}{mj}求在\end{CJK} $\eta_{1}, \eta_{2}, \eta_{3}, \eta_{4}$ \begin{CJK}{UTF8}{mj}下的矩阵\end{CJK}

\begin{CJK}{UTF8}{mj}八\end{CJK}、 $M$ \begin{CJK}{UTF8}{mj}是\end{CJK} $n \times m$ \begin{CJK}{UTF8}{mj}阶矩阵\end{CJK}, \begin{CJK}{UTF8}{mj}若\end{CJK} $\lambda_{0}$ \begin{CJK}{UTF8}{mj}为\end{CJK} $\left(\begin{array}{cc}I_{n} & M \\ M^{\prime} & I_{m}\end{array}\right)$ \begin{CJK}{UTF8}{mj}的特征值\end{CJK}, \begin{CJK}{UTF8}{mj}则\end{CJK} $-\lambda_{0}$ \begin{CJK}{UTF8}{mj}也为它的特征值\end{CJK}.

\begin{CJK}{UTF8}{mj}九\end{CJK}、 $n$ \begin{CJK}{UTF8}{mj}阶矩阵\end{CJK} ( $n$ \begin{CJK}{UTF8}{mj}为奇数\end{CJK}) $A$ \begin{CJK}{UTF8}{mj}的主对角线元素为\end{CJK} 0 . \begin{CJK}{UTF8}{mj}每行元素均为\end{CJK} 1 \begin{CJK}{UTF8}{mj}或\end{CJK} $-1$. \begin{CJK}{UTF8}{mj}且\end{CJK} $-1$ \begin{CJK}{UTF8}{mj}与\end{CJK} 1 \begin{CJK}{UTF8}{mj}的个数相同\end{CJK}, \begin{CJK}{UTF8}{mj}证明\end{CJK} $\operatorname{rank}(A)=n-1 .$
$$
\text { 十、 } A=\left(\begin{array}{ccc}
2 & -1 & 1 \\
1 & 2 & -1 \\
-1 & 1 & 2
\end{array}\right) \text {. 证明 } A^{k} \text { 中所有元素之和可以被 } 6 \text { 整除. }
$$

\section{第 18 章 华中科技大学}
\section{$18.1$ 数学分析}
\begin{enumerate}
  \item \begin{CJK}{UTF8}{mj}已知\end{CJK}
\end{enumerate}
$$
\left\{\begin{array}{l}
2 y-t y^{2}+\mathrm{e}^{t}=5 \\
x=\arctan t
\end{array}\right.
$$
\begin{CJK}{UTF8}{mj}求\end{CJK} $\frac{\mathrm{d} y}{\mathrm{~d} x}, \frac{\mathrm{d}^{2} y}{\mathrm{~d} x^{2}}$.

\begin{enumerate}
  \setcounter{enumi}{2}
  \item \begin{CJK}{UTF8}{mj}讨论\end{CJK} $\int_{0}^{\infty} t^{2} \mathrm{e}^{-t} \ln t \mathrm{~d} t$ \begin{CJK}{UTF8}{mj}得到连续性\end{CJK}.

  \item \begin{CJK}{UTF8}{mj}求曲面\end{CJK} $S: x^{2}+y^{2}+z^{2}=4\left(x^{2}+y^{2}-z^{2}\right)$ \begin{CJK}{UTF8}{mj}围成的体积\end{CJK}.

  \item \begin{CJK}{UTF8}{mj}已知级数\end{CJK} $\sum_{n=1}^{\infty} \frac{(-1)^{n} n(n+1)}{n(n+1) x^{2}+2^{n}}$, \begin{CJK}{UTF8}{mj}判断其玫散性\end{CJK}, \begin{CJK}{UTF8}{mj}并求其和函数\end{CJK}.

\end{enumerate}
\includegraphics[max width=\textwidth]{2022_04_18_7db0708508f26638f054g-205}

\begin{enumerate}
  \setcounter{enumi}{5}
  \item \begin{CJK}{UTF8}{mj}用\end{CJK} $\varepsilon-N$ \begin{CJK}{UTF8}{mj}语言证明\end{CJK}: \begin{CJK}{UTF8}{mj}已知\end{CJK} $\lim _{n \rightarrow \infty} a_{n}=a$, \begin{CJK}{UTF8}{mj}证明\end{CJK} $\lim _{n \rightarrow \infty} \frac{1}{a_{n}}=\frac{1}{a}$.

  \item \begin{CJK}{UTF8}{mj}已知\end{CJK} $\lim _{n \rightarrow \infty} \sum_{k=1}^{n} \frac{k}{n} x_{k}=A$, \begin{CJK}{UTF8}{mj}求证\end{CJK}: $\lim _{n \rightarrow \infty} \sum_{k=1}^{n} \frac{k^{2}}{n^{2}} x_{k}=\frac{A}{2}$.

  \item \begin{CJK}{UTF8}{mj}计算曲面积分\end{CJK}

\end{enumerate}
$$
\iint_{S} \frac{(x-5) \mathrm{d} y \mathrm{~d} z+y \mathrm{~d} z \mathrm{~d} x+z \mathrm{~d} x \mathrm{~d} y}{(x-5)^{2}+y^{2}+z^{2}}
$$
\begin{CJK}{UTF8}{mj}其中\end{CJK} $S:(x-5)^{2}+2 y^{2}+(z+1)^{2}=3$.

8 . \begin{CJK}{UTF8}{mj}证明\end{CJK}:

\includegraphics[max width=\textwidth]{2022_04_18_7db0708508f26638f054g-205(1)}

\section{$18.2$ 高等代数}
\begin{enumerate}
  \item \begin{CJK}{UTF8}{mj}计算行列式\end{CJK}
\end{enumerate}
$$
D=\mid \begin{gathered}
1 \\
b_{1} \\
b_{2} \\
\vdots \\
b_{2022}
\end{gathered}
$$
$$
\begin{aligned}
& 2
\end{aligned}
$$

\begin{enumerate}
  \setcounter{enumi}{2}
  \item $A=\left(\begin{array}{lll}2 & 2 & a \\ 8 & 2 & 0 \\ 0 & 0 & 6\end{array}\right)$. \begin{CJK}{UTF8}{mj}确定\end{CJK} $a$ \begin{CJK}{UTF8}{mj}的值\end{CJK}, \begin{CJK}{UTF8}{mj}使得\end{CJK} $A$ \begin{CJK}{UTF8}{mj}可对角化\end{CJK}, \begin{CJK}{UTF8}{mj}并求出\end{CJK} $P$, \begin{CJK}{UTF8}{mj}使得\end{CJK} $P^{-1} A P$ \begin{CJK}{UTF8}{mj}为对角阵\end{CJK}.
\end{enumerate}
$3 . A=\left(a_{i j}\right)_{n \times n}$. \begin{CJK}{UTF8}{mj}若已知\end{CJK} $a_{i i}>\sum_{i+j}\left|a_{i j}\right|, i=1,2, \cdots, n$. \begin{CJK}{UTF8}{mj}证明\end{CJK}: $|A|>0$.

\begin{enumerate}
  \setcounter{enumi}{4}
  \item $A, B$ \begin{CJK}{UTF8}{mj}对称\end{CJK}, $C$ \begin{CJK}{UTF8}{mj}反称\end{CJK}. \begin{CJK}{UTF8}{mj}已知\end{CJK} $A^{2}+B^{2}=C^{2}$, \begin{CJK}{UTF8}{mj}证明\end{CJK} $A=B=C=O$.

  \item \begin{CJK}{UTF8}{mj}若\end{CJK} $A$ \begin{CJK}{UTF8}{mj}半正定\end{CJK}, $V=\left\{x \in \mathbb{R}^{n} \mid x^{\prime} A x=0\right\}$. \begin{CJK}{UTF8}{mj}证明\end{CJK}:

\end{enumerate}
(1) $V$ \begin{CJK}{UTF8}{mj}是\end{CJK} $\mathbb{R}^{n}$ \begin{CJK}{UTF8}{mj}的子空间\end{CJK}.

(2) \begin{CJK}{UTF8}{mj}求\end{CJK} $V$ \begin{CJK}{UTF8}{mj}的维数\end{CJK}.

\begin{enumerate}
  \setcounter{enumi}{6}
  \item \begin{CJK}{UTF8}{mj}设\end{CJK} $A$ \begin{CJK}{UTF8}{mj}是一个\end{CJK} $n$ \begin{CJK}{UTF8}{mj}阶可逆实对称矩阵\end{CJK}, \begin{CJK}{UTF8}{mj}证明\end{CJK} $A$ \begin{CJK}{UTF8}{mj}正定的充要条件是对任意的\end{CJK} $n$ \begin{CJK}{UTF8}{mj}阶方阵\end{CJK} $B$, \begin{CJK}{UTF8}{mj}都有\end{CJK} $\operatorname{tr}(A B)>0$.

  \item $A$ \begin{CJK}{UTF8}{mj}为\end{CJK} $n \times m$ \begin{CJK}{UTF8}{mj}矩阵\end{CJK}, \begin{CJK}{UTF8}{mj}证明\end{CJK}: $\mathrm{r}\left(I_{n}-A^{\prime} A\right)-\mathrm{r}\left(I_{m}-A A^{\prime}\right)=n-m$. 8. \begin{CJK}{UTF8}{mj}欧氏空间中\end{CJK}, $\mathscr{A}$ \begin{CJK}{UTF8}{mj}是反称变换\end{CJK}. \begin{CJK}{UTF8}{mj}对任意\end{CJK} $\alpha, \beta \in \mathbb{R}^{n}$, \begin{CJK}{UTF8}{mj}有\end{CJK}

\end{enumerate}
$$
(\mathscr{A}(\alpha), \beta)+(\alpha, \beta)=0
$$
(1) \begin{CJK}{UTF8}{mj}证明\end{CJK} $\mathscr{A}$ \begin{CJK}{UTF8}{mj}是线性变换\end{CJK}.\\
(2) \begin{CJK}{UTF8}{mj}证明\end{CJK} $\mathscr{A}-\mathscr{E}$ \begin{CJK}{UTF8}{mj}是可逆变换\end{CJK}.\\
(3) \begin{CJK}{UTF8}{mj}证明\end{CJK} $(\mathscr{A}+\mathscr{E})(\mathscr{A}-\mathscr{E})^{-1}$ \begin{CJK}{UTF8}{mj}是正交变换\end{CJK}.

\section{第 19 章 中南大学}
\section{$19.1$ 高等代数}
\begin{CJK}{UTF8}{mj}一\end{CJK}、1. \begin{CJK}{UTF8}{mj}设四阶方阵\end{CJK} $A$ \begin{CJK}{UTF8}{mj}的第二列元素依次为\end{CJK} $2, m, k, 3, A$ \begin{CJK}{UTF8}{mj}的行列式\end{CJK} $|A|$ \begin{CJK}{UTF8}{mj}的第二列元素的余子式为\end{CJK} $1,-1,1,-1$, \begin{CJK}{UTF8}{mj}第四列元素的代数余子式依次为\end{CJK} $3,1,4,2$, \begin{CJK}{UTF8}{mj}且\end{CJK} $|A|=1$. \begin{CJK}{UTF8}{mj}求\end{CJK} $m, k$ \begin{CJK}{UTF8}{mj}的值\end{CJK}.

\begin{enumerate}
  \setcounter{enumi}{2}
  \item \begin{CJK}{UTF8}{mj}设实向量\end{CJK} $X=(a, b, c)$ \begin{CJK}{UTF8}{mj}的三个分量满足\end{CJK} $\left(\begin{array}{ll}a & 0 \\ b & c\end{array}\right)^{2022}=I$, \begin{CJK}{UTF8}{mj}求\end{CJK} $X$.
\end{enumerate}
\begin{CJK}{UTF8}{mj}二\end{CJK}、\begin{CJK}{UTF8}{mj}设\end{CJK} $A$ \begin{CJK}{UTF8}{mj}为\end{CJK} $n$ \begin{CJK}{UTF8}{mj}阶复方阵\end{CJK}, $\bar{A}$ \begin{CJK}{UTF8}{mj}为\end{CJK} $A$ \begin{CJK}{UTF8}{mj}的共轭矩阵\end{CJK}. \begin{CJK}{UTF8}{mj}证明\end{CJK}: \begin{CJK}{UTF8}{mj}对任意\end{CJK} $u, u I-\bar{A} A$ \begin{CJK}{UTF8}{mj}的特征多项式必为实系数多项式\end{CJK}.

\begin{CJK}{UTF8}{mj}三\end{CJK}、\begin{CJK}{UTF8}{mj}设\end{CJK} $H_{1}=\left(\begin{array}{ll}0 & 1 \\ 1 & 0\end{array}\right), H_{n+1}=\left(\begin{array}{cc}H_{n} & I \\ I & H_{n}\end{array}\right), n \geq 1$. \begin{CJK}{UTF8}{mj}求\end{CJK} $\mathrm{r}\left(H_{4}\right)$.

\begin{CJK}{UTF8}{mj}四\end{CJK}、\begin{CJK}{UTF8}{mj}设\end{CJK} $n$ \begin{CJK}{UTF8}{mj}阶方阵\end{CJK} $A, B$ \begin{CJK}{UTF8}{mj}满足\end{CJK} $A B=A+B$. \begin{CJK}{UTF8}{mj}证明\end{CJK}:

(1) $A B=B A$.

(2) \begin{CJK}{UTF8}{mj}若存在正整数\end{CJK} $k$, \begin{CJK}{UTF8}{mj}使得\end{CJK} $A^{k}=O$. \begin{CJK}{UTF8}{mj}则\end{CJK} $|B+2022 A|=|B|$.

\begin{CJK}{UTF8}{mj}五\end{CJK}、\begin{CJK}{UTF8}{mj}设向量组\end{CJK} $\alpha_{1}, \cdots, \alpha_{m}$ \begin{CJK}{UTF8}{mj}线性无关\end{CJK}, \begin{CJK}{UTF8}{mj}对非零向量\end{CJK} $\beta$ \begin{CJK}{UTF8}{mj}有\end{CJK} $\beta, \alpha_{1}, \cdots, \alpha_{m}$ \begin{CJK}{UTF8}{mj}线性相关\end{CJK}. \begin{CJK}{UTF8}{mj}证明\end{CJK}: \begin{CJK}{UTF8}{mj}在向量组\end{CJK} $\beta, \alpha_{1}, \cdots, \alpha_{m}$ \begin{CJK}{UTF8}{mj}中\end{CJK}, \begin{CJK}{UTF8}{mj}存在唯一一个向量\end{CJK} $\alpha_{j}(1 \leq j \leq m)$, \begin{CJK}{UTF8}{mj}使\end{CJK} $\alpha_{j}$ \begin{CJK}{UTF8}{mj}可由其前面的向量\end{CJK} $\beta, \alpha_{1}, \cdots, \alpha_{j-1}$ \begin{CJK}{UTF8}{mj}线性表出\end{CJK}.

\begin{CJK}{UTF8}{mj}六\end{CJK}、\begin{CJK}{UTF8}{mj}设\end{CJK} $n$ \begin{CJK}{UTF8}{mj}元实二次型\end{CJK} $f\left(x_{1}, x_{2}, \cdots, x_{n}\right)=\sum_{i=1}^{n} x_{i}^{2}-\sum_{i=1}^{n-1} x_{i} x_{i+1}, n \geq 2$.

(1) \begin{CJK}{UTF8}{mj}证明\end{CJK} $f\left(x_{1}, x_{2}, \cdots, x_{n}\right)$ \begin{CJK}{UTF8}{mj}正定\end{CJK}.

(2) \begin{CJK}{UTF8}{mj}求\end{CJK} $f\left(x_{1}, x_{2}, \cdots, x_{n}\right)$ \begin{CJK}{UTF8}{mj}在条件\end{CJK} $x_{n}=1$ \begin{CJK}{UTF8}{mj}下的最小值\end{CJK}.

\begin{CJK}{UTF8}{mj}七\end{CJK}、\begin{CJK}{UTF8}{mj}已知\end{CJK} $B$ \begin{CJK}{UTF8}{mj}为一个\end{CJK} $n$ \begin{CJK}{UTF8}{mj}阶实对称正定矩阵\end{CJK}.

(1) \begin{CJK}{UTF8}{mj}证明\end{CJK}: \begin{CJK}{UTF8}{mj}对任意\end{CJK} $n$ \begin{CJK}{UTF8}{mj}阶实对称矩阵\end{CJK} $A$, \begin{CJK}{UTF8}{mj}存在实可逆矩阵\end{CJK} $P$, \begin{CJK}{UTF8}{mj}使得\end{CJK} $P^{\prime} A P, P^{\prime} B P$ \begin{CJK}{UTF8}{mj}都为对角矩阵\end{CJK}.

(2) \begin{CJK}{UTF8}{mj}证明\end{CJK}: $V=\{A \mid A$ \begin{CJK}{UTF8}{mj}为\end{CJK} $n$ \begin{CJK}{UTF8}{mj}阶实对称矩阵\end{CJK}, \begin{CJK}{UTF8}{mj}且对任何实数\end{CJK} $k, k A+B$ \begin{CJK}{UTF8}{mj}为恒定矩阵\end{CJK} $\}$. \begin{CJK}{UTF8}{mj}按通常的矩阵加法和数乘运算构\end{CJK} \begin{CJK}{UTF8}{mj}成一个实数域上的线性空间\end{CJK}, \begin{CJK}{UTF8}{mj}其中恒定矩阵为零矩阵\end{CJK}、\begin{CJK}{UTF8}{mj}正定矩阵和负定矩阵的统称\end{CJK}.

(3) \begin{CJK}{UTF8}{mj}求\end{CJK} $V$ \begin{CJK}{UTF8}{mj}的维数\end{CJK}.

\begin{CJK}{UTF8}{mj}八\end{CJK}、\begin{CJK}{UTF8}{mj}设\end{CJK} $n$ \begin{CJK}{UTF8}{mj}阶复方阵\end{CJK} $A_{1}, A_{2}$ \begin{CJK}{UTF8}{mj}均相似对角阵\end{CJK}, $\mathbb{C}^{n}$ \begin{CJK}{UTF8}{mj}表示复\end{CJK} $n$ \begin{CJK}{UTF8}{mj}维列向量空间\end{CJK}. \begin{CJK}{UTF8}{mj}记\end{CJK}
$$
V_{1}^{(k)}=\left\{\alpha \mid A_{k} \alpha=0, \alpha \in \mathbb{C}^{n}\right\}, V_{2}^{(k)}=\left\{A_{k} \beta \mid \beta \in \mathbb{C}^{n}\right\}, k=1,2
$$
\begin{CJK}{UTF8}{mj}证明\end{CJK} $\mathbb{C}^{n}=V_{1}^{(k)} \oplus V_{2}^{(k)}, k=1,2$. \begin{CJK}{UTF8}{mj}进一步\end{CJK}, \begin{CJK}{UTF8}{mj}若\end{CJK} $A_{1} A_{2}=O$, \begin{CJK}{UTF8}{mj}则\end{CJK} $\mathbb{C}^{n}=\left(V_{1}^{(1} \cap V_{2}^{(2)}\right) \oplus V_{2}^{(1)} \oplus V_{2}^{(2)}$.

\begin{CJK}{UTF8}{mj}九\end{CJK}、\begin{CJK}{UTF8}{mj}设\end{CJK} $V$ \begin{CJK}{UTF8}{mj}是一个\end{CJK} $n$ \begin{CJK}{UTF8}{mj}维欧式空间\end{CJK}, $V_{1}, V_{2}$ \begin{CJK}{UTF8}{mj}是\end{CJK} $V$ \begin{CJK}{UTF8}{mj}的两个非平凡子空间且\end{CJK} $V=V_{1} \oplus V_{2}$. \begin{CJK}{UTF8}{mj}设\end{CJK} $\sigma_{i}$ \begin{CJK}{UTF8}{mj}是\end{CJK} $V$ \begin{CJK}{UTF8}{mj}到\end{CJK} $V$ \begin{CJK}{UTF8}{mj}的正交投影\end{CJK}. \begin{CJK}{UTF8}{mj}即\end{CJK} $\forall \alpha=\beta+\gamma \in V_{I} \oplus V_{i}^{\top}$, \begin{CJK}{UTF8}{mj}其中\end{CJK} $\beta \in V_{i}, \gamma \in V_{i}^{\top}$, \begin{CJK}{UTF8}{mj}有\end{CJK} $\sigma_{i}(\alpha)=\beta, i=1$, 2. \begin{CJK}{UTF8}{mj}证明\end{CJK}:

(1) \begin{CJK}{UTF8}{mj}任意\end{CJK} $v_{1} \in V_{1}, v_{2} \in V_{2}$, \begin{CJK}{UTF8}{mj}有\end{CJK} $V_{1}-\sigma_{2}\left(v_{1}\right) \in V_{2}^{\top}, V_{2}-\sigma_{1}\left(v_{2}\right) \in V_{1}^{\top}$.

(2) $\left.\sigma_{2} \sigma_{1}\right|_{V_{2}}$ \begin{CJK}{UTF8}{mj}的任何一个特征值\end{CJK} $\lambda$ \begin{CJK}{UTF8}{mj}都满足\end{CJK} $0 \leq \lambda<1$.

\section{$19.2$ 数学分析}
\begin{CJK}{UTF8}{mj}一\end{CJK}、\begin{CJK}{UTF8}{mj}计算题\end{CJK} (\begin{CJK}{UTF8}{mj}每小题\end{CJK} 10 \begin{CJK}{UTF8}{mj}分\end{CJK}, \begin{CJK}{UTF8}{mj}共\end{CJK} 40 \begin{CJK}{UTF8}{mj}分\end{CJK})

\begin{enumerate}
  \item \begin{CJK}{UTF8}{mj}计算极限\end{CJK}
\end{enumerate}
$$
\lim _{x \rightarrow 0} \frac{(x-\sin x) \mathrm{e}^{-x^{2}}}{\sqrt{1-x^{2}}-1}
$$

\begin{enumerate}
  \setcounter{enumi}{2}
  \item $f(x)$ \begin{CJK}{UTF8}{mj}在\end{CJK} $[0,1]$ \begin{CJK}{UTF8}{mj}上连续可微\end{CJK}, \begin{CJK}{UTF8}{mj}且\end{CJK} $\int_{0}^{1} f(x) \mathrm{d} x=\frac{5}{2}, \int_{0}^{1} x f(x) \mathrm{d} x=\frac{3}{2}$. \begin{CJK}{UTF8}{mj}求\end{CJK} $\int_{0}^{1} x(1-x)\left(3-f^{\prime}(x)\right) \mathrm{d} x$. 3. \begin{CJK}{UTF8}{mj}计算极限\end{CJK}
\end{enumerate}
$$
\lim _{\substack{x \rightarrow 2 \\ y \rightarrow+\infty}}\left(\cos \frac{x^{2}}{y}\right)^{\frac{y^{2}+x}{x^{2}}}
$$

\begin{enumerate}
  \setcounter{enumi}{4}
  \item \begin{CJK}{UTF8}{mj}计算第二型曲面积分\end{CJK}
\end{enumerate}
$$
\iint_{\Sigma} x^{3} \mathrm{~d} y \mathrm{~d} z
$$
\begin{CJK}{UTF8}{mj}其中\end{CJK} $\Sigma: x^{2}+y^{2}+z^{2}=1$, \begin{CJK}{UTF8}{mj}取外侧\end{CJK}.

\begin{CJK}{UTF8}{mj}二\end{CJK}、(15 \begin{CJK}{UTF8}{mj}分\end{CJK}) $f(x)$ \begin{CJK}{UTF8}{mj}在\end{CJK} $[0,1]$ \begin{CJK}{UTF8}{mj}上连续\end{CJK}, \begin{CJK}{UTF8}{mj}在\end{CJK} $(0,1)$ \begin{CJK}{UTF8}{mj}上可导\end{CJK}, \begin{CJK}{UTF8}{mj}且\end{CJK} $f(0)=0, f(1)=1$. \begin{CJK}{UTF8}{mj}证明\end{CJK}:\\
(1) \begin{CJK}{UTF8}{mj}存在\end{CJK} $x_{0} \in(0,1)$, \begin{CJK}{UTF8}{mj}使得\end{CJK} $f\left(x_{0}\right)=2-3 x_{0}$.\\
(2) \begin{CJK}{UTF8}{mj}存在\end{CJK} $\xi, \eta \in(0,1)$, \begin{CJK}{UTF8}{mj}使得\end{CJK} $\left(1+f^{\prime}(\xi)\right)\left(1+f^{\prime}(\eta)\right)=4 .(\xi \neq \eta)$

\begin{CJK}{UTF8}{mj}三\end{CJK}、(15 \begin{CJK}{UTF8}{mj}分\end{CJK}) $f(x)$ \begin{CJK}{UTF8}{mj}为\end{CJK} $[a, b]$ \begin{CJK}{UTF8}{mj}上的非常值连续函数\end{CJK}, $m, M$ \begin{CJK}{UTF8}{mj}为\end{CJK} $f(x)$ \begin{CJK}{UTF8}{mj}在\end{CJK} $[a, b]$ \begin{CJK}{UTF8}{mj}上的最小\end{CJK}、\begin{CJK}{UTF8}{mj}最大值\end{CJK}. \begin{CJK}{UTF8}{mj}证明存在区间\end{CJK} $[\alpha, \beta] \subseteq$ $[a, b]$ \begin{CJK}{UTF8}{mj}满足\end{CJK}:\\
(1) $m<f(x)<M, x \in(\alpha, \beta)$.\\
(2) $f(\alpha), f(\beta)$ \begin{CJK}{UTF8}{mj}分别为\end{CJK} $f(x)$ \begin{CJK}{UTF8}{mj}在\end{CJK} $[\alpha, \beta]$ \begin{CJK}{UTF8}{mj}上的最值\end{CJK}.

\begin{CJK}{UTF8}{mj}四\end{CJK}、(15 \begin{CJK}{UTF8}{mj}分\end{CJK}) \begin{CJK}{UTF8}{mj}求幂级数\end{CJK} $\sum_{n=1}^{\infty}\left[1-n \ln \left(1+\frac{1}{n}\right)\right] x^{n}$ \begin{CJK}{UTF8}{mj}得到收敛半径和收敛域\end{CJK}.

\begin{CJK}{UTF8}{mj}五\end{CJK}、(15 \begin{CJK}{UTF8}{mj}分\end{CJK}) $f(x)$ \begin{CJK}{UTF8}{mj}在\end{CJK} $(0,+\infty)$ \begin{CJK}{UTF8}{mj}可导\end{CJK}, \begin{CJK}{UTF8}{mj}且\end{CJK} $\lim _{x \rightarrow+\infty} f^{\prime}(x)=0$, \begin{CJK}{UTF8}{mj}证明\end{CJK}: $\lim _{x \rightarrow+\infty} \frac{f(x)}{x}=0$.

\begin{CJK}{UTF8}{mj}六\end{CJK}、(20 \begin{CJK}{UTF8}{mj}分\end{CJK}) \begin{CJK}{UTF8}{mj}讨论\end{CJK}
$$
f(x, y)=\left\{\begin{array}{l}
\frac{x y(x-y)}{x^{2}+y^{2}}, x^{2}+y^{2} \neq 0 \\
0, x^{2}+y^{2}=0
\end{array}\right.
$$
\begin{CJK}{UTF8}{mj}在原点处的连续性\end{CJK}、\begin{CJK}{UTF8}{mj}偏导数的存在性\end{CJK}、\begin{CJK}{UTF8}{mj}可微性以及二阶混合偏导数的存在性\end{CJK}.

\begin{CJK}{UTF8}{mj}七\end{CJK}、(15 \begin{CJK}{UTF8}{mj}分\end{CJK}) \begin{CJK}{UTF8}{mj}讨论广义积分\end{CJK} $\int_{0}^{\infty} x \mathrm{e}^{-x y} \mathrm{~d} y$ \begin{CJK}{UTF8}{mj}在\end{CJK} $(1,+\infty)$ \begin{CJK}{UTF8}{mj}和\end{CJK} $(0,+\infty)$ \begin{CJK}{UTF8}{mj}上的一致收敛性\end{CJK}.

\begin{CJK}{UTF8}{mj}八\end{CJK}、(15 \begin{CJK}{UTF8}{mj}分\end{CJK}) $L$ \begin{CJK}{UTF8}{mj}为球面\end{CJK} $x^{2}+y^{2}+z^{2}=3$ \begin{CJK}{UTF8}{mj}与平面\end{CJK} $x+y+z=1$ \begin{CJK}{UTF8}{mj}的交线\end{CJK}.\\
(1) \begin{CJK}{UTF8}{mj}求\end{CJK} $L$ \begin{CJK}{UTF8}{mj}在点\end{CJK} $P(1,1,-1)$ \begin{CJK}{UTF8}{mj}处的\end{CJK}\\
(2) \begin{CJK}{UTF8}{mj}求\end{CJK} $\int_{L}\left[(x+1)^{2}+(y-2)^{2}\right] \mathrm{d} S$

\section{第 20 章 吉林大学}
\section{$20.1$ 高等代数与空间解析几何}
\begin{enumerate}
  \item $f, g$ \begin{CJK}{UTF8}{mj}都是数域\end{CJK} $\Omega[x]$ \begin{CJK}{UTF8}{mj}上的首一多项式\end{CJK}, $[f, g]=f g$ \begin{CJK}{UTF8}{mj}的充要条件是\end{CJK} $\left[f^{3}, g^{4}\right]=f^{3} g^{4} \cdot[f, g]$ \begin{CJK}{UTF8}{mj}表示\end{CJK} $f$ \begin{CJK}{UTF8}{mj}与\end{CJK} $g$ \begin{CJK}{UTF8}{mj}的最小\end{CJK} \begin{CJK}{UTF8}{mj}公倍式\end{CJK}.
\end{enumerate}
$2 . V$ \begin{CJK}{UTF8}{mj}是\end{CJK} $n$ \begin{CJK}{UTF8}{mj}阶矩阵构成加法\end{CJK}、\begin{CJK}{UTF8}{mj}数乘封闭的矩阵空间\end{CJK}.
$$
\mathscr{T}=\{x \in V \mid A X B=0\}
$$
(1) \begin{CJK}{UTF8}{mj}证明\end{CJK} $\mathscr{T}$ \begin{CJK}{UTF8}{mj}是\end{CJK} $V$ \begin{CJK}{UTF8}{mj}的子空间\end{CJK}.

(2) $r(A)=r(B)=1$. \begin{CJK}{UTF8}{mj}求\end{CJK} $\operatorname{dim} \mathscr{T}$.

\begin{enumerate}
  \setcounter{enumi}{3}
  \item $A$ \begin{CJK}{UTF8}{mj}是非零的\end{CJK} $n(n \geq 2)$ \begin{CJK}{UTF8}{mj}阶矩阵\end{CJK}. \begin{CJK}{UTF8}{mj}证明\end{CJK}: \begin{CJK}{UTF8}{mj}存在非零矩阵\end{CJK} $B$, \begin{CJK}{UTF8}{mj}使得\end{CJK} $(A B)^{n}=(B A)^{n}=O$.

  \item $A, B$ \begin{CJK}{UTF8}{mj}是\end{CJK} $n$ \begin{CJK}{UTF8}{mj}阶实对称矩阵\end{CJK}, $A^{2} B+B A^{2}$ \begin{CJK}{UTF8}{mj}为正定矩阵\end{CJK}. \begin{CJK}{UTF8}{mj}证明\end{CJK}: $B$ \begin{CJK}{UTF8}{mj}为正定矩阵\end{CJK}.

\end{enumerate}
$5 . \sigma$ \begin{CJK}{UTF8}{mj}是\end{CJK} $\Omega$ \begin{CJK}{UTF8}{mj}上的反对称变换\end{CJK}, $\sigma^{2}$ \begin{CJK}{UTF8}{mj}是\end{CJK} $\Omega$ \begin{CJK}{UTF8}{mj}上的正交变换\end{CJK}. \begin{CJK}{UTF8}{mj}证明\end{CJK}: $\sigma^{-1}=\sigma^{-3}$.

\begin{enumerate}
  \setcounter{enumi}{6}
  \item \begin{CJK}{UTF8}{mj}求过点\end{CJK} $M(1,1,1)$ \begin{CJK}{UTF8}{mj}平行于\end{CJK} $l_{1}: \frac{x-1}{1}=\frac{y}{2}=\frac{z}{1}$ \begin{CJK}{UTF8}{mj}和\end{CJK} $l_{2}:\left\{\begin{array}{l}x+2 y=0 \\ y+z+2=0\end{array}\right.$

  \item \begin{CJK}{UTF8}{mj}求到直线\end{CJK} $l:\left\{\begin{array}{l}x+y=1 \\ z=0\end{array}\right.$ \begin{CJK}{UTF8}{mj}与平面\end{CJK} $\pi: x+y+2 z=3$ \begin{CJK}{UTF8}{mj}距离相等的点的轨迹\end{CJK}

  \item \begin{CJK}{UTF8}{mj}证明\end{CJK}: \begin{CJK}{UTF8}{mj}单叶双曲面\end{CJK} $\frac{x^{2}}{4}+\frac{y^{2}}{4}-\frac{z^{2}}{4}=0$ \begin{CJK}{UTF8}{mj}的正交直母线的交线的轨迹为圆\end{CJK}.

\end{enumerate}
\section{第 21 章东北大学}
\section{$21.1$ 高等代数}
\section{一、计算题}
\begin{enumerate}
  \item $f(x), g(x)$ \begin{CJK}{UTF8}{mj}已给出\end{CJK}, \begin{CJK}{UTF8}{mj}求\end{CJK} $c(x), u(x), v(x)$.

  \item \begin{CJK}{UTF8}{mj}计算行列式\end{CJK}

\end{enumerate}
$$
\left|\begin{array}{ccccc}
a b & a & a & \cdots & a \\
b & a^{2} b^{2} & a & \cdots & a \\
b & b & a^{3} b^{3} & \cdots & a \\
\vdots & \vdots & \vdots & & \vdots \\
b & b & b & \cdots & a^{b} b^{n}
\end{array}\right|
$$

\begin{enumerate}
  \setcounter{enumi}{3}
  \item $f(x)=-7 x_{1}^{2}+2 x_{2}^{2}+8 x_{3}^{2}-2 a x_{1} x_{2}+20 x_{2} x_{3}+8 a x_{1} x_{3}$.
\end{enumerate}
(1) \begin{CJK}{UTF8}{mj}求\end{CJK} $a$.

(2) \begin{CJK}{UTF8}{mj}给出一正交变换\end{CJK}, \begin{CJK}{UTF8}{mj}写出矩阵\end{CJK}.

\begin{enumerate}
  \setcounter{enumi}{4}
  \item $\varepsilon_{1}, \varepsilon_{2}, \varepsilon_{3}$ \begin{CJK}{UTF8}{mj}为\end{CJK} $V$ \begin{CJK}{UTF8}{mj}中的一组基\end{CJK}, $\mathscr{A}\left(\varepsilon_{1}\right), \mathscr{A}\left(\varepsilon_{2}\right), \mathscr{A}\left(\varepsilon_{3}\right)$.
\end{enumerate}
(1) \begin{CJK}{UTF8}{mj}求\end{CJK} $k$.

(2) \begin{CJK}{UTF8}{mj}求一基在\end{CJK} $\varepsilon_{1}, \varepsilon_{2}, \varepsilon_{3}$ \begin{CJK}{UTF8}{mj}下为对角阵\end{CJK},\begin{CJK}{UTF8}{mj}求出这组基\end{CJK}, \begin{CJK}{UTF8}{mj}并写出対角阵\end{CJK}.

\begin{enumerate}
  \setcounter{enumi}{5}
  \item \begin{CJK}{UTF8}{mj}求\end{CJK} Jordan \begin{CJK}{UTF8}{mj}标准型\end{CJK}, \begin{CJK}{UTF8}{mj}并写出\end{CJK} $P^{-1} A P=J$.
\end{enumerate}
\section{二、证明题}
$1 . C$ \begin{CJK}{UTF8}{mj}的全部列向量都为\end{CJK} $B x=0$ \begin{CJK}{UTF8}{mj}的解\end{CJK}, \begin{CJK}{UTF8}{mj}且\end{CJK} $A$ \begin{CJK}{UTF8}{mj}的列向量都可被\end{CJK} $C$ \begin{CJK}{UTF8}{mj}表示出来\end{CJK}. $A$ \begin{CJK}{UTF8}{mj}与\end{CJK} $C$ \begin{CJK}{UTF8}{mj}不等价\end{CJK}. \begin{CJK}{UTF8}{mj}求证\end{CJK}: $B, A$ \begin{CJK}{UTF8}{mj}有公共的特\end{CJK} \begin{CJK}{UTF8}{mj}征向旺\end{CJK}.

\begin{enumerate}
  \setcounter{enumi}{2}
  \item $A, B$ \begin{CJK}{UTF8}{mj}分别为\end{CJK} $m \times n, m \times n$ \begin{CJK}{UTF8}{mj}矩阵\end{CJK}, $P, Q$ \begin{CJK}{UTF8}{mj}为\end{CJK} $n \times n$ \begin{CJK}{UTF8}{mj}矩阵\end{CJK}, $A P=B, B P=A$.
\end{enumerate}
(1) \begin{CJK}{UTF8}{mj}证明存在矩阵\end{CJK} $K$, \begin{CJK}{UTF8}{mj}使得\end{CJK} $P+(P Q+I) K$ \begin{CJK}{UTF8}{mj}为可逆矩阵\end{CJK}.

(2) \begin{CJK}{UTF8}{mj}证明存在可逆矩阵\end{CJK} $M$, \begin{CJK}{UTF8}{mj}有\end{CJK} $A M=B$.

\begin{enumerate}
  \setcounter{enumi}{3}
  \item \begin{CJK}{UTF8}{mj}证明\end{CJK}: $\mathscr{A}$ \begin{CJK}{UTF8}{mj}的\end{CJK} $V_{\lambda_{0}}$ \begin{CJK}{UTF8}{mj}特征子空间是\end{CJK} $\mathscr{A}$ \begin{CJK}{UTF8}{mj}的子空间\end{CJK}.

  \item \begin{CJK}{UTF8}{mj}证明\end{CJK}: $\mathscr{A}$ \begin{CJK}{UTF8}{mj}的\end{CJK} $V_{\lambda_{0}}$ \begin{CJK}{UTF8}{mj}是不变子空间\end{CJK}, \begin{CJK}{UTF8}{mj}则\end{CJK} A \begin{CJK}{UTF8}{mj}的任意一特征向量生成的空间都为不变子空间\end{CJK}.

  \item \begin{CJK}{UTF8}{mj}镜像变换\end{CJK} $\mathscr{A} \alpha=\alpha-k\left(\alpha_{1} \varepsilon\right) \varepsilon, \varepsilon$ \begin{CJK}{UTF8}{mj}为单位基向量\end{CJK}.\\
(1) \begin{CJK}{UTF8}{mj}证明\end{CJK} $\mathscr{A}$ \begin{CJK}{UTF8}{mj}的特征值为\end{CJK} 1 \begin{CJK}{UTF8}{mj}的空昍的维数为\end{CJK} $n-1 .(k \neq 0)$\\
(2) \begin{CJK}{UTF8}{mj}证明\end{CJK} $\mathscr{A}$ \begin{CJK}{UTF8}{mj}为对称变换\end{CJK}.\\
(3) \begin{CJK}{UTF8}{mj}证明\end{CJK} $\mathscr{A}$ \begin{CJK}{UTF8}{mj}为正交变换且行列式的值为\end{CJK} $-1$.

\end{enumerate}
\section{第 22 章复日大学}
\section{$22.1$ 数学分析}
\begin{enumerate}
  \item \begin{CJK}{UTF8}{mj}设\end{CJK} $f(x)=\sqrt{x-\ln (1+x)}, x \in[0,+\infty)$, \begin{CJK}{UTF8}{mj}则\end{CJK} $f^{\prime}(0)=(\quad), f^{\prime \prime}(0)=(\quad)$.

  \item \begin{CJK}{UTF8}{mj}极限\end{CJK} $\lim _{n \rightarrow \infty} \int_{0}^{10 n}\left(1-\left|\sin \frac{x}{n}\right|\right)^{n} \mathrm{~d} x=(\quad)$.

  \item \begin{CJK}{UTF8}{mj}设\end{CJK} $\alpha$ \begin{CJK}{UTF8}{mj}为给定实数\end{CJK}, \begin{CJK}{UTF8}{mj}若函数项级数\end{CJK} $\sum_{n=1}^{\infty} \frac{1}{n^{\alpha}} \sin \left(n x+\frac{1}{n x}\right)$ \begin{CJK}{UTF8}{mj}关于\end{CJK} $x \in(0,2 \pi)$ \begin{CJK}{UTF8}{mj}内闭一致收敛\end{CJK}, \begin{CJK}{UTF8}{mj}则\end{CJK} $\alpha$ \begin{CJK}{UTF8}{mj}的取值范围是\end{CJK} ( ).

  \item $\int_{0}^{\pi}(\sin x)^{\frac{4}{3}}(\cos x)^{\frac{2}{3}} d x=(\quad)$.

  \item \begin{CJK}{UTF8}{mj}积分\end{CJK} $\iiint_{2 x^{2}+y^{2} \leqslant z \leqslant x^{2}+4 x+8 y}(x+y) \mathrm{d} x \mathrm{~d} y \mathrm{~d} z=()$.

\end{enumerate}
\includegraphics[max width=\textwidth]{2022_04_18_7db0708508f26638f054g-211}

\begin{CJK}{UTF8}{mj}二\end{CJK}. \begin{CJK}{UTF8}{mj}设\end{CJK} $f$ \begin{CJK}{UTF8}{mj}为\end{CJK} $[-1,1]$ \begin{CJK}{UTF8}{mj}上的实函数\end{CJK}, $M>0$ \begin{CJK}{UTF8}{mj}使得对任何\end{CJK} $x, y \in[-1,1]$ \begin{CJK}{UTF8}{mj}成立\end{CJK} $|f(x)-f(y)| \leqslant M|x-y|$, \begin{CJK}{UTF8}{mj}若对任何\end{CJK} \begin{CJK}{UTF8}{mj}固定的\end{CJK} $x$, \begin{CJK}{UTF8}{mj}成立\end{CJK} $\lim _{n \rightarrow \infty} n f\left(\frac{x}{n}\right)=0$. \begin{CJK}{UTF8}{mj}证明\end{CJK}: $f$ \begin{CJK}{UTF8}{mj}在\end{CJK} $x=0$ \begin{CJK}{UTF8}{mj}处可导且导数为\end{CJK} 0 .

\begin{CJK}{UTF8}{mj}三\end{CJK}. \begin{CJK}{UTF8}{mj}计算级数\end{CJK} $\sum_{n=1}^{\infty} \frac{H_{n}}{2^{n}(n+1)}$ \begin{CJK}{UTF8}{mj}并说明理由\end{CJK}, \begin{CJK}{UTF8}{mj}其中\end{CJK} $H_{n}=\sum_{k=1}^{n} \frac{1}{k}(n \geqslant 1)$.

\begin{CJK}{UTF8}{mj}四\end{CJK}. \begin{CJK}{UTF8}{mj}证明\end{CJK}:

\begin{enumerate}
  \item \begin{CJK}{UTF8}{mj}曲线积分\end{CJK}
\end{enumerate}
$$
\int_{C} \mathrm{e}^{-2 x y} \cos \left(x^{2}-y^{2}\right) \mathrm{d} x-\mathrm{e}^{-2 x y} \sin \left(x^{2}-y^{2}\right) \mathrm{d} y
$$
\begin{CJK}{UTF8}{mj}与路径无关\end{CJK}.

\begin{enumerate}
  \setcounter{enumi}{2}
  \item $\lim _{R \rightarrow+\infty}\left(\int_{0}^{R} \cos x^{2} \mathrm{~d} x-\int_{0}^{R} \mathrm{e}^{-2 x^{2}} \mathrm{~d} x\right)=0$.

  \item $\lim _{R \rightarrow+\infty}\left(\int_{0}^{R} \sin x^{2} \mathrm{~d} x-\int_{0}^{R} \mathrm{e}^{-2 x^{2}} \mathrm{~d} x\right)=0$.

\end{enumerate}
\section{$22.2$ 常微分方程}
$$
\mathrm{d} x
$$
\begin{CJK}{UTF8}{mj}五\end{CJK}. (15 \begin{CJK}{UTF8}{mj}分\end{CJK}) \begin{CJK}{UTF8}{mj}设实平面上如下初值问题\end{CJK} $x \overline{\mathrm{d} t}=t^{2}+x^{2}, x(0)=0$ \begin{CJK}{UTF8}{mj}的解为\end{CJK} $\phi(t)$, \begin{CJK}{UTF8}{mj}最大存在区间为\end{CJK} $(\alpha, \beta)$. \begin{CJK}{UTF8}{mj}试证明\end{CJK}: \begin{CJK}{UTF8}{mj}且\end{CJK} $\left[-\frac{\sqrt{2}}{2}, \frac{\sqrt{2}}{2}\right] \subseteq(\alpha, \beta)$ \begin{CJK}{UTF8}{mj}且\end{CJK} $)$
$$
\max _{-\frac{\sqrt{2}}{2} \leqslant \leqslant \frac{\sqrt{2}}{2}}\left|\phi(t)-\frac{1}{3} t^{3}\right|<\frac{\sqrt{2}}{11} .
$$
\begin{CJK}{UTF8}{mj}六\end{CJK}. (15 \begin{CJK}{UTF8}{mj}分\end{CJK}) \begin{CJK}{UTF8}{mj}给定\end{CJK} $\mu, \beta \in \mathbb{R}$, \begin{CJK}{UTF8}{mj}试求出实平面上如下系统的所有奇点\end{CJK}, \begin{CJK}{UTF8}{mj}分析系统在这些奇点处的类型\end{CJK}, \begin{CJK}{UTF8}{mj}并绘出这些\end{CJK} \begin{CJK}{UTF8}{mj}奇点附近系统的大致相轨线图\end{CJK}
$$
\left\{\begin{array}{l}
\frac{\mathrm{d} x}{\mathrm{~d} t}=\mu x-y-\beta x\left(x^{2}+y^{2}\right) \\
\frac{\mathrm{d} y}{\mathrm{~d} t}=x+\mu y-\beta y\left(x^{2}+y^{2}\right)
\end{array}\right.
$$

\section{$22.3$ 复变函数}
\begin{CJK}{UTF8}{mj}七\end{CJK}. (15 \begin{CJK}{UTF8}{mj}分\end{CJK}) \begin{CJK}{UTF8}{mj}设\end{CJK} $\Omega$ \begin{CJK}{UTF8}{mj}是由分段光滑\end{CJK} Jordan \begin{CJK}{UTF8}{mj}曲线所围成的有界区域\end{CJK}, \begin{CJK}{UTF8}{mj}函数\end{CJK} $f(z)$ \begin{CJK}{UTF8}{mj}在\end{CJK} $\bar{\omega}$ \begin{CJK}{UTF8}{mj}上连续\end{CJK}, \begin{CJK}{UTF8}{mj}在\end{CJK} $\Omega$ \begin{CJK}{UTF8}{mj}上解析\end{CJK}. \begin{CJK}{UTF8}{mj}如果\end{CJK} $|f(z)|$ \begin{CJK}{UTF8}{mj}在\end{CJK} $\partial \Omega$ \begin{CJK}{UTF8}{mj}上是常数\end{CJK}, \begin{CJK}{UTF8}{mj}证明\end{CJK} $f(z)$ \begin{CJK}{UTF8}{mj}在\end{CJK} $\Omega$ \begin{CJK}{UTF8}{mj}上是常数或在\end{CJK} $\Omega$ \begin{CJK}{UTF8}{mj}内有一个零点\end{CJK}. \begin{CJK}{UTF8}{mj}八\end{CJK}. (15 \begin{CJK}{UTF8}{mj}分\end{CJK}) \begin{CJK}{UTF8}{mj}设\end{CJK} $D$ \begin{CJK}{UTF8}{mj}是单位圆\end{CJK}, $\mathcal{F}=\left\{f \mid f: D \rightarrow D\right.$ \begin{CJK}{UTF8}{mj}是解析映射\end{CJK}, \begin{CJK}{UTF8}{mj}且\end{CJK} $\left.f\left(\frac{1}{2}\right)=0\right\}$. \begin{CJK}{UTF8}{mj}证明\end{CJK}: \begin{CJK}{UTF8}{mj}对任意\end{CJK} $f \in \mathcal{F},|f(0)| \leqslant \frac{1}{2}$, \begin{CJK}{UTF8}{mj}并且存在\end{CJK} $f \in \mathcal{F}$ \begin{CJK}{UTF8}{mj}使得\end{CJK} $f(0)=\frac{1}{2}$.

\section{$22.4$ 实变函数}
\begin{CJK}{UTF8}{mj}九\end{CJK}. (15 \begin{CJK}{UTF8}{mj}分\end{CJK}) \begin{CJK}{UTF8}{mj}在\end{CJK} $\mathbb{R}^{2}$ \begin{CJK}{UTF8}{mj}上定义函数\end{CJK} $P(x, y)=\sqrt{x^{2}+y^{2}}, A$ \begin{CJK}{UTF8}{mj}是\end{CJK} $\mathbb{R}$ \begin{CJK}{UTF8}{mj}中的\end{CJK} Lebesgue \begin{CJK}{UTF8}{mj}零测集\end{CJK}. \begin{CJK}{UTF8}{mj}证明\end{CJK}: $P^{-1}(A)$ \begin{CJK}{UTF8}{mj}是\end{CJK} $\mathbb{R}^{2}$ \begin{CJK}{UTF8}{mj}中的\end{CJK} Lebesgue \begin{CJK}{UTF8}{mj}零测集\end{CJK}.

\begin{CJK}{UTF8}{mj}十\end{CJK}. (15 \begin{CJK}{UTF8}{mj}分\end{CJK}) \begin{CJK}{UTF8}{mj}证明\end{CJK}: \begin{CJK}{UTF8}{mj}存在一个常数\end{CJK} $C$, \begin{CJK}{UTF8}{mj}使得对于任意的非负可测函数\end{CJK} $f$, \begin{CJK}{UTF8}{mj}都成立\end{CJK}
$$
\int_{\mathbb{R}^{2}} f\left(\sqrt{x^{2}+y^{2}}\right) \mathrm{d} x=C \int_{0}^{+\infty} t f(t) \mathrm{d} t .
$$

\section{$22.5$ 高等代数}
\begin{enumerate}
  \item \begin{CJK}{UTF8}{mj}设二次型\end{CJK} $a x_{1}^{2}-2 x_{1} x_{2}+2 x_{1} x_{3}+b x_{2}^{2}-4 x_{2} x_{3}+2 x_{3}^{2}$ \begin{CJK}{UTF8}{mj}在正交变换\end{CJK} $x=P y$ \begin{CJK}{UTF8}{mj}下化成标准型\end{CJK} $3 y_{1}^{2}+6 y_{2}^{2}$, \begin{CJK}{UTF8}{mj}求\end{CJK} $a, b$ \begin{CJK}{UTF8}{mj}的值\end{CJK} \begin{CJK}{UTF8}{mj}并试求出满足条件的\end{CJK} $P$.

  \item \begin{CJK}{UTF8}{mj}设\end{CJK} $\mathbb{K}$ \begin{CJK}{UTF8}{mj}是数域\end{CJK}, $A_{1}, A_{2}$ \begin{CJK}{UTF8}{mj}为\end{CJK} $\mathbb{K}$ \begin{CJK}{UTF8}{mj}上的\end{CJK} $n$ \begin{CJK}{UTF8}{mj}阶方阵\end{CJK}, $b_{1}, b_{2}$ \begin{CJK}{UTF8}{mj}为\end{CJK} $\mathbb{K}$ \begin{CJK}{UTF8}{mj}上的\end{CJK} $n$ \begin{CJK}{UTF8}{mj}维列向量\end{CJK}. \begin{CJK}{UTF8}{mj}假定线性方程组\end{CJK} $A_{1} x=b_{1}$ \begin{CJK}{UTF8}{mj}和线性方\end{CJK} \begin{CJK}{UTF8}{mj}程组\end{CJK} $A_{2} x=b_{2}$ \begin{CJK}{UTF8}{mj}的解集相同\end{CJK}. \begin{CJK}{UTF8}{mj}证明\end{CJK}: \begin{CJK}{UTF8}{mj}存在可逆的\end{CJK} $n$ \begin{CJK}{UTF8}{mj}阶方阵\end{CJK} $P$, \begin{CJK}{UTF8}{mj}使得\end{CJK} $P A_{1}=A_{2}$ \begin{CJK}{UTF8}{mj}且\end{CJK} $P b_{1}=b_{2}$.

  \item \begin{CJK}{UTF8}{mj}设方阵\end{CJK}

\end{enumerate}
$$
A=\left(\begin{array}{ccc}
14 t-11 & 10 t-7 & 5-6 t \\
8-7 t & 4-5 t & 3 t-4 \\
21 t-10 & 15 t-8 & 4-9 t
\end{array}\right)
$$
$A$ \begin{CJK}{UTF8}{mj}有一个不变子空间\end{CJK} $U$ ,

\includegraphics[max width=\textwidth]{2022_04_18_7db0708508f26638f054g-212}

\begin{CJK}{UTF8}{mj}求参数\end{CJK} $t$ \begin{CJK}{UTF8}{mj}的值\end{CJK}, \begin{CJK}{UTF8}{mj}并在此时求可逆矩阵\end{CJK} $P$, \begin{CJK}{UTF8}{mj}使得\end{CJK} $P^{-1} A P$ \begin{CJK}{UTF8}{mj}为\end{CJK} Jordan \begin{CJK}{UTF8}{mj}标准型\end{CJK}.

\begin{enumerate}
  \setcounter{enumi}{4}
  \item \begin{CJK}{UTF8}{mj}设\end{CJK} $A$ \begin{CJK}{UTF8}{mj}是\end{CJK} $n$ \begin{CJK}{UTF8}{mj}阶实对称阵\end{CJK}. \begin{CJK}{UTF8}{mj}如果\end{CJK} $A$ \begin{CJK}{UTF8}{mj}同时满足\end{CJK}: (1) \begin{CJK}{UTF8}{mj}主子式的值均非负\end{CJK}; (2) \begin{CJK}{UTF8}{mj}当\end{CJK} $A$ \begin{CJK}{UTF8}{mj}的某个主子式的值为零时\end{CJK}, \begin{CJK}{UTF8}{mj}这个\end{CJK} \begin{CJK}{UTF8}{mj}主式所在的行构成的向量组线性相关\end{CJK}; \begin{CJK}{UTF8}{mj}证明\end{CJK}: $A$ \begin{CJK}{UTF8}{mj}是半正定矩阵\end{CJK}.

  \item \begin{CJK}{UTF8}{mj}以\end{CJK} $V$ \begin{CJK}{UTF8}{mj}记所有迹为零的\end{CJK} $n$ \begin{CJK}{UTF8}{mj}阶复方阵组成的复线性空间\end{CJK}.(\begin{CJK}{UTF8}{mj}注\end{CJK}: \begin{CJK}{UTF8}{mj}方阵的主对角元之和称为迹\end{CJK}). \begin{CJK}{UTF8}{mj}设\end{CJK} $H$ \begin{CJK}{UTF8}{mj}是\end{CJK} $V$ \begin{CJK}{UTF8}{mj}的线性\end{CJK} \begin{CJK}{UTF8}{mj}子空间\end{CJK}, \begin{CJK}{UTF8}{mj}同时满足\end{CJK}

\end{enumerate}
\begin{CJK}{UTF8}{mj}性质\end{CJK} (I): $H$ \begin{CJK}{UTF8}{mj}中的矩阵都可対角化\end{CJK};

\begin{CJK}{UTF8}{mj}性质\end{CJK} (II): $H$ \begin{CJK}{UTF8}{mj}中的任两个矩阵乘法可交换\end{CJK}.

(1) \begin{CJK}{UTF8}{mj}在子空间的包含关系下设\end{CJK} $H$ \begin{CJK}{UTF8}{mj}是满足性质\end{CJK} (I) \begin{CJK}{UTF8}{mj}与性质\end{CJK} (II) \begin{CJK}{UTF8}{mj}的极大子空间\end{CJK}, \begin{CJK}{UTF8}{mj}求\end{CJK} $\operatorname{dim} H$;

(2) \begin{CJK}{UTF8}{mj}如果\end{CJK} $H_{1}$ \begin{CJK}{UTF8}{mj}和\end{CJK} $H_{2}$ \begin{CJK}{UTF8}{mj}都是在包含关系下满足性质\end{CJK} (I) \begin{CJK}{UTF8}{mj}与性质\end{CJK} (II) \begin{CJK}{UTF8}{mj}的极大线性子空间\end{CJK}, \begin{CJK}{UTF8}{mj}证明存在\end{CJK} $n$ \begin{CJK}{UTF8}{mj}阶可逆复方阵\end{CJK} $P$, \begin{CJK}{UTF8}{mj}使得\end{CJK} $H_{2}=\left\{P^{-1} A P \mid A \in H_{1}\right\}$.

\begin{enumerate}
  \setcounter{enumi}{6}
  \item \begin{CJK}{UTF8}{mj}设\end{CJK} $V$ \begin{CJK}{UTF8}{mj}和\end{CJK} $U$ \begin{CJK}{UTF8}{mj}分别为\end{CJK} $n$ \begin{CJK}{UTF8}{mj}维\end{CJK}、 $m$ \begin{CJK}{UTF8}{mj}维复线性空间\end{CJK}, $\varphi: V \rightarrow V, \psi: U \rightarrow U$ \begin{CJK}{UTF8}{mj}都是线性变换\end{CJK}. \begin{CJK}{UTF8}{mj}设\end{CJK} $\gamma: V \rightarrow U$ \begin{CJK}{UTF8}{mj}是线性变换\end{CJK}, \begin{CJK}{UTF8}{mj}满足\end{CJK} $\gamma \varphi-\psi \gamma=0$. \begin{CJK}{UTF8}{mj}如果\end{CJK} $f(\lambda)$ \begin{CJK}{UTF8}{mj}是\end{CJK} $\varphi$ \begin{CJK}{UTF8}{mj}的特征多项式\end{CJK}, $g(\lambda)$ \begin{CJK}{UTF8}{mj}是\end{CJK} $\psi$ \begin{CJK}{UTF8}{mj}的特征多项式\end{CJK}, $d(\lambda)=(f(\lambda), g(\lambda))$ \begin{CJK}{UTF8}{mj}是\end{CJK} $f(\lambda)$ \begin{CJK}{UTF8}{mj}与\end{CJK} $g(\lambda)$ \begin{CJK}{UTF8}{mj}的\end{CJK} \begin{CJK}{UTF8}{mj}最大公因式\end{CJK}. \begin{CJK}{UTF8}{mj}证明\end{CJK}: $\operatorname{dim} \operatorname{Ker} \gamma \geqslant n-\operatorname{deg} d(\lambda)$. \begin{CJK}{UTF8}{mj}这里\end{CJK} $\operatorname{Im}$ \begin{CJK}{UTF8}{mj}表示像\end{CJK}, Ker \begin{CJK}{UTF8}{mj}表示核\end{CJK}, $\operatorname{deg}$ \begin{CJK}{UTF8}{mj}表示次数\end{CJK}.

  \item \begin{CJK}{UTF8}{mj}设\end{CJK} $A$ \begin{CJK}{UTF8}{mj}是\end{CJK} 2 \begin{CJK}{UTF8}{mj}阶非异实阵\end{CJK}. \begin{CJK}{UTF8}{mj}证明存在\end{CJK} 2 \begin{CJK}{UTF8}{mj}阶非异实阵\end{CJK} $B$, \begin{CJK}{UTF8}{mj}使得\end{CJK} $A B=B A^{2}$ \begin{CJK}{UTF8}{mj}的充要条件是\end{CJK} $A$ \begin{CJK}{UTF8}{mj}至少满足下列条件中的\end{CJK} \begin{CJK}{UTF8}{mj}一个\end{CJK}:

\end{enumerate}
(1) $A^{3}=I_{2}$; \begin{CJK}{UTF8}{mj}或\end{CJK} (2) $A-I_{2}$ \begin{CJK}{UTF8}{mj}为幂零矩阵\end{CJK}.

\section{6 抽象代数}
\begin{enumerate}
  \item \begin{CJK}{UTF8}{mj}证明\end{CJK} 91 \begin{CJK}{UTF8}{mj}阶群都是循环群\end{CJK}.

  \item \begin{CJK}{UTF8}{mj}考虑多项式环\end{CJK} $R=\mathbb{Q}[x, y]$. \begin{CJK}{UTF8}{mj}设\end{CJK} $I=\left(x^{2}+1, y^{2}+1\right)$ \begin{CJK}{UTF8}{mj}是\end{CJK} $R$ \begin{CJK}{UTF8}{mj}的一个理想\end{CJK}. \begin{CJK}{UTF8}{mj}证明\end{CJK} $I$ \begin{CJK}{UTF8}{mj}不是一个素理想\end{CJK}, \begin{CJK}{UTF8}{mj}并找到\end{CJK} $R$ \begin{CJK}{UTF8}{mj}的一\end{CJK} \begin{CJK}{UTF8}{mj}个真素理想包含\end{CJK} $I$.

  \item \begin{CJK}{UTF8}{mj}设\end{CJK} $F$ \begin{CJK}{UTF8}{mj}是一个特征为\end{CJK} 2 \begin{CJK}{UTF8}{mj}的域\end{CJK}. \begin{CJK}{UTF8}{mj}考虑\end{CJK} $F$ \begin{CJK}{UTF8}{mj}的一个二次扩张\end{CJK} $E=F[t] /\left(t^{2}+b t+c\right), \quad$ \begin{CJK}{UTF8}{mj}这里\end{CJK} $b, c \in F$ \begin{CJK}{UTF8}{mj}求解\end{CJK} $b, c$ \begin{CJK}{UTF8}{mj}所满\end{CJK} \begin{CJK}{UTF8}{mj}足的条件\end{CJK}, \begin{CJK}{UTF8}{mj}使得\end{CJK} $E / F$ \begin{CJK}{UTF8}{mj}是一个二次伽罗瓦扩张\end{CJK}.

\end{enumerate}
\section{$22.7$ 微分几何}
\begin{enumerate}
  \item \begin{CJK}{UTF8}{mj}如果曲线的密切平面处处平行\end{CJK}, \begin{CJK}{UTF8}{mj}证明该曲线是平面曲线\end{CJK}.

  \item \begin{CJK}{UTF8}{mj}在直角坐标系下\end{CJK}, \begin{CJK}{UTF8}{mj}如果曲面的方程是\end{CJK} $z=f(x, y)$, \begin{CJK}{UTF8}{mj}这里\end{CJK} $f(x, y)$ \begin{CJK}{UTF8}{mj}是\end{CJK} $C^{2}$ \begin{CJK}{UTF8}{mj}函数\end{CJK}, $x, y$ \begin{CJK}{UTF8}{mj}作为曲面的参数\end{CJK}. \begin{CJK}{UTF8}{mj}请给出\end{CJK} \begin{CJK}{UTF8}{mj}该曲面的\end{CJK} Gauss \begin{CJK}{UTF8}{mj}曲率和平均曲率的表达式\end{CJK}.

  \item \begin{CJK}{UTF8}{mj}如果紧致无边定向曲面\end{CJK} $S$ \begin{CJK}{UTF8}{mj}的\end{CJK} Gauss \begin{CJK}{UTF8}{mj}曲率\end{CJK} $K \geqslant 0$, \begin{CJK}{UTF8}{mj}但不恒为零\end{CJK}. \begin{CJK}{UTF8}{mj}证明该曲面必皿球面同肧\end{CJK}.

\end{enumerate}
\section{第 23 章四川大学}
\section{$23.1$ 高等代数}
1.(1). \begin{CJK}{UTF8}{mj}已知一矩阵\end{CJK} $A_{5 \times 4}, A X=0$ \begin{CJK}{UTF8}{mj}的基础解系为\end{CJK} $\alpha_{1}, \alpha_{2}, \alpha_{3}$, \begin{CJK}{UTF8}{mj}求一向量组表示\end{CJK} $A$ \begin{CJK}{UTF8}{mj}的极大线性无关组\end{CJK}. $\alpha_{1}, \alpha_{2}, \alpha_{3}$, \begin{CJK}{UTF8}{mj}告\end{CJK} \begin{CJK}{UTF8}{mj}诉了其体数值但不记得\end{CJK}.

(2)
$$
\left\{\begin{array}{c}
x_{1}+x_{2}+x_{3}=0 \\
2 x_{1}+a x_{2}+2 x_{3}=0 \\
2 x_{1}+2 x_{2}+a x_{3}=0
\end{array}\right.
$$
\begin{CJK}{UTF8}{mj}此方程组存在非零解\end{CJK}. \begin{CJK}{UTF8}{mj}求\end{CJK} $a$ \begin{CJK}{UTF8}{mj}及基础解系\end{CJK}.

(3) $A X=0$ \begin{CJK}{UTF8}{mj}的解为\end{CJK} $k\left(\begin{array}{c}1 \\ -1 \\ -1\end{array}\right), A^{*}$ \begin{CJK}{UTF8}{mj}为其伴随矩阵\end{CJK}, \begin{CJK}{UTF8}{mj}求\end{CJK} $A^{*} X=0$ \begin{CJK}{UTF8}{mj}的基础解系\end{CJK}.

(4) $A=\left(\begin{array}{ll}B & \gamma \\ \gamma^{\prime} & \alpha\end{array}\right)$ \begin{CJK}{UTF8}{mj}为可逆矩阵\end{CJK}, \begin{CJK}{UTF8}{mj}求\end{CJK} $r(B)$ \begin{CJK}{UTF8}{mj}的值\end{CJK} (\begin{CJK}{UTF8}{mj}可能满足的值\end{CJK}), $A$ \begin{CJK}{UTF8}{mj}为\end{CJK} $n$ \begin{CJK}{UTF8}{mj}阶矩阵\end{CJK}.

2.(1) \begin{CJK}{UTF8}{mj}只记得最后一问\end{CJK}.

(2) \begin{CJK}{UTF8}{mj}已知\end{CJK} $A, A^{*} B A=3 B A+E$, \begin{CJK}{UTF8}{mj}求\end{CJK} $B$.

(4) $f, g$ \begin{CJK}{UTF8}{mj}为\end{CJK} $V$ \begin{CJK}{UTF8}{mj}上的线性函数\end{CJK}, \begin{CJK}{UTF8}{mj}则\end{CJK} $\operatorname{Ker} f \cap \operatorname{Ker} g$ \begin{CJK}{UTF8}{mj}的维数为\end{CJK} $n-1$, \begin{CJK}{UTF8}{mj}当且仅当\end{CJK} $f, g$ \begin{CJK}{UTF8}{mj}线性相关\end{CJK}.

\begin{enumerate}
  \setcounter{enumi}{3}
  \item $M_{2}(F)$ \begin{CJK}{UTF8}{mj}是二阶矩阵空间\end{CJK}, $A\left(\begin{array}{cc}1 & 2022 \\ 0 & 1\end{array}\right)$, \begin{CJK}{UTF8}{mj}其上的线性变换\end{CJK} $\phi: \phi(X)=A X-X A$.
\end{enumerate}
(1) $\phi$ \begin{CJK}{UTF8}{mj}的特征子空间\end{CJK}, \begin{CJK}{UTF8}{mj}并求出\end{CJK} Jordan \begin{CJK}{UTF8}{mj}标准型\end{CJK};

(2) $B$ \begin{CJK}{UTF8}{mj}为一\end{CJK} 4 \begin{CJK}{UTF8}{mj}阶矩阵\end{CJK}, $B$ \begin{CJK}{UTF8}{mj}是具体的\end{CJK},\begin{CJK}{UTF8}{mj}但无法回忆\end{CJK}), \begin{CJK}{UTF8}{mj}问是否存在一组基\end{CJK}, \begin{CJK}{UTF8}{mj}使得\end{CJK} $\phi$ \begin{CJK}{UTF8}{mj}在这组基下的矩阵是\end{CJK} $B$.

(3) $\varphi$ \begin{CJK}{UTF8}{mj}是\end{CJK} $M_{2}(F)$ \begin{CJK}{UTF8}{mj}上的线性变换\end{CJK}, \begin{CJK}{UTF8}{mj}问是否存在\end{CJK} $\phi=\varphi^{3}$ ?

(4) $\phi$ \begin{CJK}{UTF8}{mj}是否可作为内积空间\end{CJK} $(-,-)$ P\begin{CJK}{UTF8}{mj}的对称变换\end{CJK}?

$4 . V$ \begin{CJK}{UTF8}{mj}是\end{CJK} 3 \begin{CJK}{UTF8}{mj}维空间\end{CJK}, $\varepsilon_{1}, \varepsilon_{2}, \varepsilon_{3}$ \begin{CJK}{UTF8}{mj}是其组其\end{CJK}. \begin{CJK}{UTF8}{mj}其下的度量矩阵\end{CJK} $G$ \begin{CJK}{UTF8}{mj}为\end{CJK}: $\left(\begin{array}{ccc}3 & 1 & 1 \\ 1 & 2 & 1 \\ 1 & 1 & 2\end{array}\right)$ 。

(1) \begin{CJK}{UTF8}{mj}请说出两种求标准正交基的方法\end{CJK} (\begin{CJK}{UTF8}{mj}不用证明\end{CJK}, \begin{CJK}{UTF8}{mj}并求出其中一组正交基\end{CJK}).

(2) $G$ \begin{CJK}{UTF8}{mj}是\end{CJK} $\phi$ \begin{CJK}{UTF8}{mj}的表示矩阵\end{CJK}, \begin{CJK}{UTF8}{mj}问是否存在内积空间使得\end{CJK} $G$ \begin{CJK}{UTF8}{mj}为正交变换\end{CJK}.

(3) \begin{CJK}{UTF8}{mj}忘了\end{CJK}.

(4) \begin{CJK}{UTF8}{mj}求\end{CJK} $t E+G$ \begin{CJK}{UTF8}{mj}不是正定矩阵时\end{CJK}, \begin{CJK}{UTF8}{mj}求\end{CJK} $t$ \begin{CJK}{UTF8}{mj}的最小值\end{CJK} (\begin{CJK}{UTF8}{mj}最大值\end{CJK})?

\begin{enumerate}
  \setcounter{enumi}{5}
  \item $f(x)=x^{p-1}+x^{p-2} \cdots+x+1, A$ \begin{CJK}{UTF8}{mj}为\end{CJK} $n$ \begin{CJK}{UTF8}{mj}阶矩阵\end{CJK}, \begin{CJK}{UTF8}{mj}且\end{CJK} $f(A)=0$.
\end{enumerate}
(1) $f(x)$ \begin{CJK}{UTF8}{mj}在\end{CJK} $\mathbb{Q}$ \begin{CJK}{UTF8}{mj}上是否可约\end{CJK}? \begin{CJK}{UTF8}{mj}说明理由\end{CJK}.

(2) $A$ \begin{CJK}{UTF8}{mj}是否可以对角化\end{CJK}?

(3) $p-1 \mid n$.

(4) \begin{CJK}{UTF8}{mj}忘了\end{CJK}.

\begin{enumerate}
  \setcounter{enumi}{6}
  \item $f(x), g(x)$ \begin{CJK}{UTF8}{mj}是整系数多项式\end{CJK}, \begin{CJK}{UTF8}{mj}且存在无穷多个\end{CJK} $c_{i}, c_{i}$ \begin{CJK}{UTF8}{mj}是整数\end{CJK}, \begin{CJK}{UTF8}{mj}使得\end{CJK} $f\left(c_{i}\right) \mid g\left(c_{i}\right)$, \begin{CJK}{UTF8}{mj}证明\end{CJK}:
\end{enumerate}
(1) \begin{CJK}{UTF8}{mj}存在有理系数多项式\end{CJK} $h(x)$, \begin{CJK}{UTF8}{mj}使得\end{CJK} $g(x)=f(x) h(x)$.

(2) \begin{CJK}{UTF8}{mj}问\end{CJK} $h(x)$ \begin{CJK}{UTF8}{mj}是否一定为整系数多项式\end{CJK}?

\section{第 24 章厦门大学}
\section{$24.1$ 数学分析}
\begin{CJK}{UTF8}{mj}一\end{CJK}、\begin{CJK}{UTF8}{mj}设\end{CJK} $f(x)$ \begin{CJK}{UTF8}{mj}在\end{CJK} $\mathbf{R}$ \begin{CJK}{UTF8}{mj}上有连续的二阶导数\end{CJK}, \begin{CJK}{UTF8}{mj}且函数\end{CJK} $f(x)$ \begin{CJK}{UTF8}{mj}与二阶导函数\end{CJK} $f^{\prime \prime}(x)$ \begin{CJK}{UTF8}{mj}有界\end{CJK}, \begin{CJK}{UTF8}{mj}设\end{CJK}
$$
M_{0}=\sup _{x \in R}|f(x)|, M_{2}=\sup _{x \in R}\left|f^{\prime \prime}(x)\right|,
$$
\begin{CJK}{UTF8}{mj}证明\end{CJK}: $\sup _{x \in R}\left(f^{\prime}(x)\right)^{2} \leqslant 4 M_{0} M_{2}$.

\begin{CJK}{UTF8}{mj}二\end{CJK}、\begin{CJK}{UTF8}{mj}证明级数\end{CJK} $\sum_{n=0}^{\infty} \int_{0}^{x} t^{n} \sin (\pi t) \mathrm{d} t$ \begin{CJK}{UTF8}{mj}在\end{CJK} $x \in[0,1]$ \begin{CJK}{UTF8}{mj}上一致收敛\end{CJK}.

\begin{CJK}{UTF8}{mj}三\end{CJK}、\begin{CJK}{UTF8}{mj}设数列\end{CJK} $\left\{x_{n}\right\}$ \begin{CJK}{UTF8}{mj}由以下迭代\end{CJK} $x_{1}=\frac{1}{2}, x_{n+1}=x_{n}^{2}+x_{n}$ \begin{CJK}{UTF8}{mj}确定\end{CJK}, \begin{CJK}{UTF8}{mj}证明\end{CJK}: $\lim _{n \rightarrow \infty}\left(\frac{1}{1+x_{1}}+\frac{1}{1+x_{2}}+\cdots+\frac{1}{1+x_{n}}\right)=2$.

\begin{CJK}{UTF8}{mj}四\end{CJK}、\begin{CJK}{UTF8}{mj}设\end{CJK} $f(x)$ \begin{CJK}{UTF8}{mj}在\end{CJK} $(a, b)$ \begin{CJK}{UTF8}{mj}上可导\end{CJK}, \begin{CJK}{UTF8}{mj}对任意\end{CJK} $x_{0} \in(a, b)$, \begin{CJK}{UTF8}{mj}证明\end{CJK}:\begin{CJK}{UTF8}{mj}存在\end{CJK} $x_{n} \in(a, b)(n=1,2, \ldots)$, \begin{CJK}{UTF8}{mj}使得\end{CJK} $\lim _{n \rightarrow \infty} x_{n}=x_{0}$, \begin{CJK}{UTF8}{mj}且\end{CJK} $\lim _{n \rightarrow \infty} f^{\prime}\left(x_{n}\right)=f\left(x_{0}\right)$.

\begin{CJK}{UTF8}{mj}五\end{CJK}、\begin{CJK}{UTF8}{mj}设由曲面\end{CJK} $F\left(\frac{x-a}{z-c}, \frac{y-b}{z-c}\right)=0$ \begin{CJK}{UTF8}{mj}确定的函数\end{CJK} $z=z(x, y)$ \begin{CJK}{UTF8}{mj}具有二阶连续连续偏导数\end{CJK}, \begin{CJK}{UTF8}{mj}其中\end{CJK} $a, b, c \in \mathrm{R}$, \begin{CJK}{UTF8}{mj}证明\end{CJK}:

(1) \begin{CJK}{UTF8}{mj}曲面定过一定点\end{CJK};

(2) $z=z(x, y)$ \begin{CJK}{UTF8}{mj}满足\end{CJK} $z_{x x} z_{y y}-z_{x y}^{2}=0$.

\begin{CJK}{UTF8}{mj}六\end{CJK}、\begin{CJK}{UTF8}{mj}计算第二型曲面积分\end{CJK}
$$
\iint_{S} x y \mathrm{~d} y \mathrm{~d} z+\left(x^{2}+y^{2}\right) y \mathrm{~d} z \mathrm{~d} x+x y \mathrm{~d} x \mathrm{~d} y
$$
\begin{CJK}{UTF8}{mj}其中\end{CJK} $S$ \begin{CJK}{UTF8}{mj}为\end{CJK} $4-y=x^{2}+z^{2}$ \begin{CJK}{UTF8}{mj}的外侧\end{CJK}.

\begin{CJK}{UTF8}{mj}七\end{CJK}、\begin{CJK}{UTF8}{mj}证明函数\end{CJK} $f(x, y)=\frac{1}{1-x y}$ \begin{CJK}{UTF8}{mj}在\end{CJK} $[0,1] \times[0,1]$ \begin{CJK}{UTF8}{mj}上不一致连续\end{CJK}.

\begin{CJK}{UTF8}{mj}八\end{CJK}、\begin{CJK}{UTF8}{mj}证明当\end{CJK} $p \geq 0$ \begin{CJK}{UTF8}{mj}时\end{CJK}, \begin{CJK}{UTF8}{mj}反常积分\end{CJK} $\int_{0}^{+\infty} \frac{\sin \left(x^{2}\right)}{1+x^{p}} \mathrm{~d} x$ \begin{CJK}{UTF8}{mj}收敛\end{CJK}.

\section{$24.2$ 高等代数}
\section{一、填空题}
1、\begin{CJK}{UTF8}{mj}设\end{CJK} $A, B, C$ \begin{CJK}{UTF8}{mj}都是三阶矩阵\end{CJK}, \begin{CJK}{UTF8}{mj}且行列式均为\end{CJK} 3 , \begin{CJK}{UTF8}{mj}则\end{CJK}
$$
\operatorname{det}\left(\begin{array}{cc}
0 & -3 A \\
B^{-1} & C
\end{array}\right)=
$$
2、\begin{CJK}{UTF8}{mj}设\end{CJK} $A=\left(a_{i j}\right)_{n \times n}$ \begin{CJK}{UTF8}{mj}不可逆\end{CJK}, $A_{11} \neq 0$, \begin{CJK}{UTF8}{mj}则\end{CJK} $A$ \begin{CJK}{UTF8}{mj}的伴随矩阵的行向量的极大线性无关组为\end{CJK}

3、\begin{CJK}{UTF8}{mj}设\end{CJK} $V_{1}, V_{2}$ \begin{CJK}{UTF8}{mj}是\end{CJK} $n$ \begin{CJK}{UTF8}{mj}维线性空间\end{CJK} $V$ \begin{CJK}{UTF8}{mj}的子空间\end{CJK}, \begin{CJK}{UTF8}{mj}且\end{CJK} $\operatorname{dim}\left(V_{1}+V_{2}\right)=\operatorname{dim} V_{1}+1$, \begin{CJK}{UTF8}{mj}则\end{CJK} $\operatorname{dim} V_{2}-\operatorname{dim}\left(V_{1} \cap V_{2}\right)=$

4、\begin{CJK}{UTF8}{mj}设\end{CJK} $n$ \begin{CJK}{UTF8}{mj}维线性空间\end{CJK} $V$ \begin{CJK}{UTF8}{mj}上的线性变换\end{CJK} $\varphi, \psi$ \begin{CJK}{UTF8}{mj}在基\end{CJK} $\xi_{1}, \xi_{2}, \ldots, \xi_{n}$ \begin{CJK}{UTF8}{mj}下的矩阵分别为\end{CJK} $A, B$, \begin{CJK}{UTF8}{mj}又\end{CJK} $\xi_{1}, \xi_{2}, \ldots, \xi_{n}$ \begin{CJK}{UTF8}{mj}到\end{CJK} $\eta_{1}, \eta_{2}, \ldots, \eta_{n}$ \begin{CJK}{UTF8}{mj}的过渡矩阵为\end{CJK} $P$, \begin{CJK}{UTF8}{mj}则\end{CJK} $\varphi \psi+2 \varphi^{3}-\mathrm{id}_{V}$ \begin{CJK}{UTF8}{mj}在\end{CJK} $\eta_{1}, \eta_{2}, \ldots, \eta_{n}$ \begin{CJK}{UTF8}{mj}下的矩阵为\end{CJK}

5、\begin{CJK}{UTF8}{mj}设\end{CJK} $f(x)=x^{5}-2020 x^{4}-2019 x^{3}-4041 x^{2}-2020 x-100$, \begin{CJK}{UTF8}{mj}则\end{CJK} $f(2021)=$

6、\begin{CJK}{UTF8}{mj}设\end{CJK} $A=\left(\begin{array}{ccc}1 & -1 & a \\ 1 & 3 & 5 \\ 0 & 0 & 2\end{array}\right)$ \begin{CJK}{UTF8}{mj}只有一个线性无关的特征向量\end{CJK}, \begin{CJK}{UTF8}{mj}则\end{CJK} $A$ \begin{CJK}{UTF8}{mj}的特征值为\end{CJK}

7、\begin{CJK}{UTF8}{mj}设\end{CJK} $A$ \begin{CJK}{UTF8}{mj}的不变因子为\end{CJK} $1,1,1,1, \lambda, \lambda^{2}(\lambda+1)^{3}$, \begin{CJK}{UTF8}{mj}则\end{CJK} $A$ \begin{CJK}{UTF8}{mj}的\end{CJK} Jordan \begin{CJK}{UTF8}{mj}标准型\end{CJK}, \begin{CJK}{UTF8}{mj}特征多项式\end{CJK}, \begin{CJK}{UTF8}{mj}极小多项式分别为\end{CJK}

8、\begin{CJK}{UTF8}{mj}设\end{CJK} $f\left(x_{1}, x_{2}, x_{3}\right)=x_{1}^{2}+x_{2}^{2}+x_{3}^{2}-2 x_{1} x_{2}-2 x_{1} x_{3}-2 a x_{2} x_{3}$ \begin{CJK}{UTF8}{mj}经正交变换\end{CJK} $X=P Y$ \begin{CJK}{UTF8}{mj}后化为\end{CJK}
$$
f=2 y_{1}^{2}+2 y_{2}^{2}+b y_{3}^{2} \text {, }
$$
\begin{CJK}{UTF8}{mj}则\end{CJK} $a, b$ \begin{CJK}{UTF8}{mj}的值为\end{CJK} \begin{CJK}{UTF8}{mj}二\end{CJK}、\begin{CJK}{UTF8}{mj}设\end{CJK} $A=\left(\begin{array}{lll}2 & 1 & 1 \\ 1 & 2 & 1 \\ 1 & 1 & a\end{array}\right), X=(1, b, 1)^{\prime}$ \begin{CJK}{UTF8}{mj}是\end{CJK} $A$ \begin{CJK}{UTF8}{mj}的伴随\end{CJK} $A^{*}$ \begin{CJK}{UTF8}{mj}属于特征值\end{CJK} $\lambda$ \begin{CJK}{UTF8}{mj}的特征向量\end{CJK}, \begin{CJK}{UTF8}{mj}求\end{CJK} $a, b, \lambda$ \begin{CJK}{UTF8}{mj}的值\end{CJK}, \begin{CJK}{UTF8}{mj}并讨论\end{CJK} $A$ \begin{CJK}{UTF8}{mj}是\end{CJK} \begin{CJK}{UTF8}{mj}否可以相似对角化\end{CJK}.

\begin{CJK}{UTF8}{mj}三\end{CJK}、\begin{CJK}{UTF8}{mj}设\end{CJK} $n$ \begin{CJK}{UTF8}{mj}维实列向量空间中\end{CJK},
$$
\alpha_{i}=\left(a_{i 1}, a_{i 2}, \ldots, a_{i n}\right)^{\prime}(i=1,2, \ldots, r, r \leq n)
$$
\begin{CJK}{UTF8}{mj}且\end{CJK} $\alpha_{1}, \alpha_{2}, \alpha_{r}$ \begin{CJK}{UTF8}{mj}线性无关\end{CJK}, $\beta$ \begin{CJK}{UTF8}{mj}为齐次线性方程组\end{CJK}
$$
\left(\begin{array}{cccc}
a_{11} & a_{12} & \cdots & a_{1 n} \\
a_{21} & a_{22} & \cdots & a_{2 n} \\
\vdots & \vdots & \ddots & \vdots \\
a_{r 1} & a_{r 2} & \cdots & a_{r n}
\end{array}\right)\left(\begin{array}{c}
x_{1} \\
x_{2} \\
\vdots \\
x_{n}
\end{array}\right)=0
$$
\begin{CJK}{UTF8}{mj}的非零解\end{CJK}, \begin{CJK}{UTF8}{mj}证明\end{CJK}: $\alpha_{1}, \alpha_{2}, \ldots, \alpha_{r}, \beta$ \begin{CJK}{UTF8}{mj}线性无关\end{CJK}.

\begin{CJK}{UTF8}{mj}四\end{CJK}、\begin{CJK}{UTF8}{mj}设\end{CJK} $A$ \begin{CJK}{UTF8}{mj}是\end{CJK} $n$ \begin{CJK}{UTF8}{mj}阶方阵\end{CJK}, \begin{CJK}{UTF8}{mj}证明\end{CJK}
$$
\operatorname{rank}(A)+\operatorname{rank}(E-A)=n
$$
\begin{CJK}{UTF8}{mj}的充分必要条件是\end{CJK} $A^{2}=A$.

\begin{CJK}{UTF8}{mj}五\end{CJK}、\begin{CJK}{UTF8}{mj}设\end{CJK} $f(x)=x^{4}+6 x^{3}+4 x+2 \in \mathbb{Q}[x], c$ \begin{CJK}{UTF8}{mj}为\end{CJK} $f(x)$ \begin{CJK}{UTF8}{mj}的一个复根\end{CJK}, \begin{CJK}{UTF8}{mj}记\end{CJK} $Q[c]=\left\{a_{0}+a_{1} c+a_{2} c^{2}+a_{3} c^{3}: a_{0}, a_{1}, a_{2}, a_{3} \in \mathbb{Q}\right\}$, \begin{CJK}{UTF8}{mj}证明\end{CJK}:

(1) $f(x)$ \begin{CJK}{UTF8}{mj}在有理数域\end{CJK} $Q$ \begin{CJK}{UTF8}{mj}上不可约\end{CJK};

(2) \begin{CJK}{UTF8}{mj}对任意\end{CJK} $g(x) \in \mathrm{Q}[x]$, \begin{CJK}{UTF8}{mj}有\end{CJK} $g(c) \in \mathrm{Q}[c]$;

(3) \begin{CJK}{UTF8}{mj}对任意\end{CJK} $g(x \in \mathbb{Q}[x])$, \begin{CJK}{UTF8}{mj}若\end{CJK} $f(x) \nmid g(x)$, \begin{CJK}{UTF8}{mj}则存在\end{CJK} $h(x) \in \mathbb{Q}[x]$ \begin{CJK}{UTF8}{mj}使得\end{CJK} $g(c) h(c)=1$.

\begin{CJK}{UTF8}{mj}六\end{CJK}、\begin{CJK}{UTF8}{mj}设\end{CJK} $Q$ \begin{CJK}{UTF8}{mj}是\end{CJK} $n$ \begin{CJK}{UTF8}{mj}阶正定矩阵\end{CJK}, $x$ \begin{CJK}{UTF8}{mj}为\end{CJK} $n$ \begin{CJK}{UTF8}{mj}维实列向量\end{CJK}, \begin{CJK}{UTF8}{mj}证明\end{CJK} $0 \leq x^{\prime}\left(Q+x x^{\prime}\right)^{-1} x<1$

\begin{CJK}{UTF8}{mj}七\end{CJK}、\begin{CJK}{UTF8}{mj}设\end{CJK} $U, V$ \begin{CJK}{UTF8}{mj}为数域\end{CJK} $F$ \begin{CJK}{UTF8}{mj}上的有限为线性空间\end{CJK},
$$
\varphi: V \rightarrow U, \psi: U \rightarrow V, \text { 且 } \psi \varphi=\mathrm{id}_{V}
$$
\begin{CJK}{UTF8}{mj}证明\end{CJK}: $U=\operatorname{Im} \varphi \oplus \operatorname{Ker} \psi$.

\begin{CJK}{UTF8}{mj}八\end{CJK}、\begin{CJK}{UTF8}{mj}设\end{CJK} $\varphi$ \begin{CJK}{UTF8}{mj}为\end{CJK} $\mathbb{F}$ \begin{CJK}{UTF8}{mj}上\end{CJK} $n$ \begin{CJK}{UTF8}{mj}维线性空间\end{CJK} $V$ \begin{CJK}{UTF8}{mj}上的线性变换\end{CJK}, \begin{CJK}{UTF8}{mj}设\end{CJK} $0 \neq \alpha \in V$, \begin{CJK}{UTF8}{mj}记\end{CJK}
$$
F[\varphi] \alpha=\{f(\varphi) \alpha: f(\lambda) \in \mathbb{F}[\lambda]\}
$$
\begin{CJK}{UTF8}{mj}设\end{CJK} $\varphi$ \begin{CJK}{UTF8}{mj}限制在\end{CJK} $F[\varphi] \alpha$ \begin{CJK}{UTF8}{mj}上的线性变换为\end{CJK} $\varphi_{1}$, \begin{CJK}{UTF8}{mj}若\end{CJK} $\varphi_{1}$ \begin{CJK}{UTF8}{mj}的极小多项式\end{CJK} $m_{\varphi}(\lambda)$ \begin{CJK}{UTF8}{mj}的标准分解为\end{CJK} $m_{\varphi_{i}}(\lambda)=p_{1}^{e_{1}}(\lambda) \cdots p_{t}^{e_{t}}(\lambda)$, \begin{CJK}{UTF8}{mj}其\end{CJK} \begin{CJK}{UTF8}{mj}中\end{CJK} $p_{i}(\lambda)$ \begin{CJK}{UTF8}{mj}两两互素\end{CJK}, \begin{CJK}{UTF8}{mj}且对每个\end{CJK} $1 \leq i \leq t$, \begin{CJK}{UTF8}{mj}都有\end{CJK} $p_{i}(\lambda)$ \begin{CJK}{UTF8}{mj}首一且不可约\end{CJK}, $e_{i} \geq 1$, \begin{CJK}{UTF8}{mj}则存在\end{CJK} $\alpha_{1}, \alpha_{2}, \ldots, \alpha_{t}$, \begin{CJK}{UTF8}{mj}使得\end{CJK}

(1) $\boldsymbol{F}[\varphi] \alpha_{i}=\operatorname{Ker} p_{i}^{\varphi_{i}}\left(\varphi_{i}\right)$;

(2) $\boldsymbol{F}[\varphi] \alpha=\boldsymbol{F}[\varphi] \alpha_{1} \oplus \boldsymbol{F}[\varphi] \alpha_{2} \oplus \cdots \oplus \boldsymbol{F}[\varphi] \alpha_{1}$.


\end{document}