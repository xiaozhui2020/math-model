\documentclass[10pt]{article}
\usepackage[utf8]{inputenc}
\usepackage[T1]{fontenc}
\usepackage{ctex}
\usepackage{amsmath}
\usepackage{amsfonts}
\usepackage{amssymb}
\usepackage{mhchem}
\usepackage{stmaryrd}
\usepackage{bbold}
\usepackage{mathrsfs}
\usepackage{graphicx}
\usepackage[export]{adjustbox}
\graphicspath{ {./images/} }
\usepackage{esint}
\begin{document}
 北京大学  1996  年全国硕士研究生招生考试高代解几试题及解答 

   

2019.05.25

 一 . (15  分 )  在仿射坐标系中 ,  求过点  $M_{0}(0,0,-2)$,  与平面  $\pi_{1}: 3 x-y+2 z-1=0$  平行 ,  且与直线 
$$
\ell_{1}: \quad \frac{x-1}{4}=\frac{y-3}{-2}=\frac{z}{-1}
$$
 相交的直线  $\ell$  的方程 .

 二 . (25  分 )  作直角坐标变换 ,  把下述二次曲面方程化成标准方程 ,  并且指出它是什么曲面 :
$$
x^{2}+4 y^{2}+z^{2}-4 x y-8 x z-4 y z+2 x+y+2 z-\frac{25}{16}=0 .
$$
 三 . (16  分 )  设线性空间  $V$  中的向量组  $\alpha_{1}, \alpha_{2}, \alpha_{3}, \alpha_{4}$  线性无关 .

(1)  试问 :  向量组  $\alpha_{1}+\alpha_{2}, \alpha_{2}+\alpha_{3}, \alpha_{3}+\alpha_{4}, \alpha_{4}+\alpha_{1}$  是否线性无关 ?  要求说明理由 .

(2)  求向量组  $\alpha_{1}+\alpha_{2}, \alpha_{2}+\alpha_{3}, \alpha_{3}+\alpha_{4}, \alpha_{4}+\alpha_{1}$  生成的线性子空间  $W$  的一个基以及  $W$  的维数 .

 四 . (16  分 )  设  $V$  是数域  $\mathbb{K}$  上的  $n$  维线性空间 ,  并且  $V=U \oplus W$.  任给  $\alpha \in V$,  设  $\alpha=\alpha_{1}+\alpha_{2}$,  其中  $\alpha_{1} \in U, \alpha_{2} \in W$.  令  $\mathscr{P}(\alpha)=\alpha_{1}$.  证明 :

(1) $\mathscr{P}$  是  $V$  上的线性变换 ,  并且  $\mathscr{P}^{2}=\mathscr{P}$;

(2) $\mathscr{P}$  的核  $\operatorname{Ker} \mathscr{P}=W, \mathscr{P}$  的象  ( 值域 ) $\operatorname{Im} \mathscr{P}=U$;

(3) $V$  中存在一个基 ,  使得  $\mathscr{P}$  在这个基下的矩阵是 
$$
\left(\begin{array}{ll}
I_{r} & O \\
O & O
\end{array}\right),
$$
 其中  $I_{r}$  表示  $r$  级单位矩阵 ,  请指出  $r$  等于什么 .

 五 . (12  分 ) $n$  阶矩阵  $A$  称为周期矩阵 ,  如果存在正整数  $m$,  使  $A^{m}=I$,  其中  $I$  是单位矩阵 .  证明 :  复数域  $\mathbb{C}$  上   的周期矩阵一定可以对角化 .

 六 . (16  分 )  用  $\mathbb{R}[x]_{4}$  表示实数域  $\mathbb{R}$  上次数小于  4  的一元多项式组成的集合 ,  它是一个欧几里得空间 ,  其上的内   积为 
$$
(f, g)=\int_{0}^{1} f(x) g(x) \mathrm{d} x
$$
 设  $W$  是由零次多项式组成的子空间 ,  求  $W^{\perp}$  以及它的一个基 .  一 .  过  $M_{0}$  与  $\pi_{1}$  平行的平面为  $\pi_{2}: 3 x-y+2 z+4=0, \ell_{1}$  与  $\pi_{2}$  的交点为  $(0,7 / 2,-1 / 4)$,  从而直线  $\ell$  的方程为 
$$
\frac{x-0}{0-0}=\frac{y-0}{\frac{7}{2}-0}=\frac{z+2}{-\frac{1}{4}+2}
$$
 化简得 
$$
\frac{x}{0}=\frac{y}{2}=\frac{z+2}{1}
$$
 二 .  二次型部分对应的矩阵为 
$$
A=\left[\begin{array}{ccc}
1 & -2 & -4 \\
-2 & 4 & -2 \\
-4 & -2 & 1
\end{array}\right]
$$
$|\lambda E-A|=(\lambda-5)^{2}(\lambda+4), \lambda=-4$  对应的一个特征向量为  $(2,1,1)^{\mathrm{T}}, \lambda=5$  对应的两个特征向量为  $(1,0,-1)^{\mathrm{T}},(0,2,-1)^{\mathrm{T}}$.  令 
$$
\left[\begin{array}{l}
x \\
y \\
z
\end{array}\right]=\left[\begin{array}{ccc}
\frac{2}{3} & \frac{1}{\sqrt{2}} & 0 \\
\frac{1}{3} & 0 & \frac{2}{\sqrt{5}} \\
\frac{2}{3} & \frac{-1}{\sqrt{2}} & \frac{-1}{\sqrt{5}}
\end{array}\right]\left[\begin{array}{l}
x_{1} \\
y_{1} \\
z_{1}
\end{array}\right],
$$
 带入原方程化简配方可得 
$$
-4\left(x_{1}-\frac{3}{8}\right)^{2}+5 y_{1}^{2}+5 z_{1}^{2}=1
$$
 再令 
$$
\left\{\begin{aligned}
x_{1}-\frac{3}{8} &=u \\
y_{1} &=v \\
z_{1} &=w
\end{aligned}\right.
$$
 就可以化成标准方程 ,  从而知道曲面为单叶双曲面 .

 注   此题为丘维声  《 高等代数 》 创新教材上册第  321  页例  3 .

 三 . (1)
$$
\left(\alpha_{1}+\alpha_{2}, \alpha_{2}+\alpha_{3}, \alpha_{3}+\alpha_{4}, \alpha_{4}+\alpha_{1}\right)=\left(\alpha_{1}, \alpha_{2}, \alpha_{3}, \alpha_{4}\right)\left[\begin{array}{llll}
1 & 0 & 0 & 1 \\
1 & 1 & 0 & 0 \\
0 & 1 & 1 & 0 \\
0 & 0 & 1 & 1
\end{array}\right],
$$
 又因为  $|A|=0$,  故  $\alpha_{1}+\alpha_{2}, \alpha_{2}+\alpha_{3}, \alpha_{3}+\alpha_{4}, \alpha_{4}+\alpha_{1}$  线性无关 .

(2) $\alpha_{1}+\alpha_{2}, \alpha_{2}+\alpha_{3}, \alpha_{3}+\alpha_{4}$  是线性无关的 ,  而  $\alpha_{4}+\alpha_{1}$  可以由前面三个线性表出 ,  从而  $W$  的一个基为  $\alpha_{1}+\alpha_{2}, \alpha_{2}+\alpha_{3}, \alpha_{3}+\alpha_{4}$,  并且  $\operatorname{dim} W=3 .$

 四 . (1) $\forall \alpha=\alpha_{1}+\alpha_{2}, \beta=\beta_{1}+\beta_{2} \in V$,  其中  $\alpha_{1}, \beta_{1} \in U, \alpha_{2}, \beta_{2} \in W$,  则 
$$
\begin{gathered}
\mathscr{P}\left(k_{1} \alpha+k_{2} \beta\right)=k_{1} \alpha_{1}+k_{2} \beta_{1}=k_{1} \mathscr{P}(\alpha)+k_{2} \mathscr{P}(\beta), \\
\mathscr{P}^{2}(\alpha)=\mathscr{P}(\mathscr{P}(\alpha))=\mathscr{P}\left(\alpha_{1}\right)=\alpha_{1}=\mathscr{P}(\alpha),
\end{gathered}
$$
 故  $\mathscr{P}$  是  $V$  上的线性变换 ,  由  $\alpha$  的任意性知  $\mathscr{P}^{2}=\mathscr{P}$.

(2) $\forall \alpha \in W$,  均有  $\mathscr{P}(\alpha)=0$,  从而  $W \subset \operatorname{Ker} \mathscr{P} . \forall \alpha \in V, \alpha=\alpha_{1}+\alpha_{2}$,  若  $\mathscr{P}(\alpha)=0$,  则  $\alpha_{1}=0$,  从而  $\alpha=\alpha_{2} \in W$,  于是  $\operatorname{Ker} \mathscr{P}=W$.

 易知  $\operatorname{Im} \mathscr{P} \subset U$,  又由于  $\forall \alpha \in U, \mathscr{P}(\alpha)=\alpha$,  从而  $U \subset \operatorname{Im} \mathscr{P}$,  于是  $U=\operatorname{Im} \mathscr{P}$. (3)  取  $U$  的一组基  $\xi_{1}, \xi_{2}, \ldots, \xi_{s}, W$  的一组基  $\xi_{s+1}, \ldots, \xi_{n}$,  则  $\xi_{1}, \xi_{2}, \ldots, \xi_{n}$  是  $V$  的一组基 ,  并且 
$$
\mathscr{P}\left(\xi_{1}, \xi_{2}, \ldots, \xi_{n}\right)=\left(\xi_{1}, \xi_{2}, \ldots, \xi_{n}\right)\left[\begin{array}{cc}
I_{s} & O \\
O & O
\end{array}\right]
$$
 从而看出  $r=\operatorname{dim} U$.

 平 .  令 
$$
g(x)=x^{n}-1=\prod_{k=0}^{n-1}\left(x-\mathrm{e}^{\frac{2 \pi k \mathrm{i}}{n}}\right)
$$
 则  $g(A)=0$,  于是  $A$  的最小多项式  $m_{A}(x)$  将整除  $g(x)$,  从而  $m_{A}(x)$  为  $\mathbb{C}$  上互素一次因式的乘积 ,  从而一   定可以相似对角化 .

 六 . $W$  的标准正交基是  $1,1, x, x^{2}, x^{3}$  是  $\mathbb{R}[x]_{4}$  的一组基 ,  从而  $\forall f(x)=a_{0}+a_{1} x+a_{2} x^{2}+a_{3} x^{3} \in W^{\perp}$,
$$
\int_{0}^{1} 1 \cdot f(x) \mathrm{d} x=a_{0}+\frac{a_{1}}{2}+\frac{a_{2}}{3}+\frac{a_{3}}{4}=0
$$
 于是 
$$
f(x)=a_{1}\left(x-\frac{1}{2}\right)+a_{2}\left(x^{2}-\frac{1}{3}\right)+a_{3}\left(x^{3}-\frac{1}{4}\right) \text {, }
$$
 由此可以看出  $x-1 / 2, x^{2}-1 / 3, x^{3}-1 / 4$  为  $W^{\perp}$  的一组基 .  北京大学  1997  年全国硕士研究生招生考试高代解几试题及解答 

   

2019.05.25

 一 . (12  分 )  判断下列二次曲线类型 

\begin{enumerate}
  \item $x^{2}-3 x y+y^{2}+10 x-10 y+21=0$
  \item $x^{2}+4 x y+4 y^{2}-20 x+10 y-50=0$.
\end{enumerate}
 二 . (18  分 )  过  $x$  轴和  $y$  分别做动平面 ,  交角  $\alpha$  是常数 ,  求交线轨迹的方程 ,  并且证明它是一个雉面 .

 三 . (20  分 )  设  $A, B$  是数域  $\mathbb{K}$  上的  $n$  阶方阵 , $X$  是末知量  $x_{1}, \cdots, x_{n}$  所成的  $n \times 1$  矩阵 .  已知齐次线性方程组  $A X=0$  和  $B X=0$  分别有  $l, m$  个线性无关解向量 ,  这里  $l \geqslant 0, m \geqslant 0$.

\begin{enumerate}
  \item  证明  $(A B) X=0$  至少有  $\max (l, m)$  个线性无关的解向量 .

  \item  如果  $l+m>n$,  证明  $(A+B) X=0$  必有非零解 .

  \item  如果  $A X=0$  和  $B X=0$  无公共非零解向量 ,  且  $l+m=n$;  证明  $\mathbb{K}^{n}$  中任一向量  $\alpha$  可唯一表成  $\alpha=\beta+\gamma$,  这里  $\beta, \gamma$  分别是  $A X=0$  和  $B X=0$  的解向量 .

\end{enumerate}
 四 . (20  分 ) $\mathscr{A}$  是实数域  $\mathbb{R}$  上的  3  维线性空间  $V$  上的一个线性变换 ,  对  $V$  的一组基  $\varepsilon_{1}, \varepsilon_{2}, \varepsilon_{3}$,  有 
$$
\mathscr{A} \varepsilon_{1}=3 \varepsilon_{1}+6 \varepsilon_{2}+6 \varepsilon_{3}, \quad \mathscr{A} \varepsilon_{2}=4 \varepsilon_{1}+3 \varepsilon_{2}+4 \varepsilon_{3}, \quad \mathscr{A} \varepsilon_{3}=-5 \varepsilon_{1}-4 \varepsilon_{2}-6 \varepsilon_{3}
$$

\begin{enumerate}
  \item  求  $\mathscr{A}$  的全部特征值和特征向量 .

  \item  设  $\mathscr{B}=\mathscr{A}^{3}-5 \mathscr{A}$,  求  $\mathscr{B}$  的一个非平凡的不变子空间 .

\end{enumerate}
 五 . (10  分 )  设  $f(x)$  是有理数域  $\mathbb{Q}$  上的一个  $m$  次多项式  $(m \geqslant 0), n$  是大于  $m$  的正整数 .  证明 : $\sqrt[n]{2}$  不是  $f(x)$  的实根 .

 六 . (20  分 )  设  $\mathscr{A}$  是  $n$  维欧式空间  $V$  上的一个线性变换 ,  满足 
$$
(\mathscr{A} \alpha, \beta)=-(\alpha, \mathscr{A} \beta), \quad(\forall \alpha, \beta \in V)
$$

\begin{enumerate}
  \item  若  $\lambda$  是  $\mathscr{A}$  的一个特征值 ,  证明  $\lambda=0$.

  \item  证明  $V$  内存在一组标准正交基 ,  使  $\mathscr{A}^{2}$  在此组基下的矩阵为对角矩阵 .

  \item  设  $\mathscr{A}$  在  $V$  的某组标准正交基下的矩阵为  $A$.  证明 :  把  $A$  看做复数域  $\mathbb{C}$  上的  $n$  阶方阵 ,  其特征值必为  0  或纯虚数 .  一 . 1.  对应的矩阵为 

\end{enumerate}
$$
\left[\begin{array}{ccc}
1 & -\frac{3}{2} & 5 \\
-\frac{3}{2} & 1 & -5 \\
5 & -5 & 21
\end{array}\right]
$$
$I_{1}=2, I_{2}=-5 / 4, I_{3}=-5 / 4$,  曲线为双曲线 .

\begin{enumerate}
  \setcounter{enumi}{2}
  \item  对应的矩阵为 
\end{enumerate}
$$
\left[\begin{array}{ccc}
1 & 2 & -10 \\
2 & 4 & 5 \\
-10 & 5 & -50
\end{array}\right]
$$
$I_{1}=5, I_{2}=0, I_{3}=-625$,  曲线为抛物线 .

 注   丘维声的 《 解析几何 》 第三版第  166  页例  $2.1$.

 二 .  设  $P(x, y, z)$  是交线上一点 ,  且不是原点 ,  又因为原点也是交线上一点 ,  于是过  $P$  点且  $x$  轴的平面的法向量   为  $(1,0,0) \times(x, y, z)=(0, z,-y)$,  同样可得另一张平面的法向量为  $(0,1,0) \times(x, y, z)=(z, 0,-x)$,  由于平   面大角为  $\alpha$,  故 
$$
|\cos \alpha|=\frac{|y x|}{\sqrt{z^{2}+y^{2}} \sqrt{z^{2}+x^{2}}}
$$
 上述等式可以转化为 
$$
\left(z^{2}+y^{2}\right)\left(z^{2}+x^{2}\right) \cos ^{2} \alpha=y^{2} x^{2}
$$
 令  $F(x, y, z)=\left(z^{2}+y^{2}\right)\left(z^{2}+x^{2}\right) \cos ^{2} \alpha-y^{2} x^{2}$,  则  $F(x, y, z)=0$  即为交线的轨迹方程 .  因为  $F(t x, t y, t z)=$ $t^{4} F(x, y, z)$,  从而曲面为雉面 .

 三 . 1.  因为  $\operatorname{rank}(A B) \leqslant \operatorname{rank}(A), \operatorname{rank}(A B) \leqslant \operatorname{rank}(B)$,  故  $n-\operatorname{rank}(A B) \geqslant n-\operatorname{rank}(A) \geqslant l, n-\operatorname{rank}(A B) \geqslant$ $n-\operatorname{rank}(B) \geqslant m$,  从而  $n-\operatorname{rank}(A B) \geqslant \max (l, m)$,  因此  $(A B) X=0$  至少有  $\max (l, m)$  个线性无关   的解向量 .

\begin{enumerate}
  \setcounter{enumi}{2}
  \item $\operatorname{rank}(A+B) \leqslant \operatorname{rank}(A)+\operatorname{rank}(B) \leqslant n-l+n-m<2 n-n=n$,  于是  $(A+B) X=0$  必有非零解 .

  \item  唯一性 :  若  $\alpha=\beta_{1}+\gamma_{1}=\beta+\gamma$,  其中  $\beta_{1}, \beta$  是  $A X=0$  的解 , $\gamma_{1}, \gamma$  是  $B X=0$  的解 ,  则  $\beta_{1}-\beta=\gamma_{1}-\gamma$  既是  $A X=0$  的解 ,  又是  $B X=0$  的解 ,  从而  $\beta_{1}=\beta, \gamma_{1}=\gamma$.

\end{enumerate}
 存在性 :  若  $\xi_{1}, \ldots, \xi_{l}$  是  $A X=0$  的  $l$  个线性无关解 , $\eta_{1}, \ldots, \eta_{m}$  是  $B X=0$  的  $m$  个线性无关解 .  若能   证明  $\xi_{1}, \ldots, \xi_{l}, \eta_{1}, \ldots, \eta_{m}$  是线性无关的 ,  则就得到了  $K^{n}$  的一组基 ,  自然就保证了分解的存在性 .  设 
$$
a_{1} \xi_{1}+\cdots+a_{l} \xi_{l}+b_{1} \eta_{1}+\cdots+b_{m} \eta_{m}=0
$$
 则 
$$
\left.\begin{array}{l}
A\left(b_{1} \eta_{1}+\cdots+b_{m} \eta_{m}\right)=0 \\
B\left(b_{1} \eta_{1}+\cdots+b_{m} \eta_{m}\right)=0
\end{array}\right\} \Longrightarrow b_{1} \eta_{1}+\cdots+b_{m} \eta_{m}=0
$$
 从而  $b_{1}=\cdots=b_{m}=0, a_{1}=\cdots=a_{l}=0$.

 四 . $1 .$
$$
\mathscr{A}\left(\varepsilon_{1}, \varepsilon_{2}, \varepsilon_{3}\right)=\left(\varepsilon_{1}, \varepsilon_{2}, \varepsilon_{3}\right)\left[\begin{array}{ccc}
3 & 4 & -5 \\
6 & 3 & -4 \\
6 & 4 & -6
\end{array}\right] .
$$
 记  $\mathscr{A}$  在基  $\varepsilon_{1}, \varepsilon_{2}, \varepsilon_{3}$  下的矩阵为  $A$,  则  $|\lambda E-A|=\left(\lambda^{2}+3 \lambda+4\right)(\lambda-3) . \lambda=3$  对应的  $A$  的一个特征   向量为  $(8,15,12)^{\mathrm{T}}$,  从而  $\mathscr{A}$  的特征值为  3 ,  对应的特征向量为  $k\left(\varepsilon_{1}, \varepsilon_{2}, \varepsilon_{3}\right)(8,15,12)^{\mathrm{T}}, k \neq 0, k \in \mathbb{R}$. 2. $\mathscr{A}$  的特征值为  3  的特征子空间为  $\mathscr{B}$  的一个非平凡的不变子空间 .

 五 . ( 法一 )  若  $\sqrt[n]{2}$  是  $f(x)$  的实根 ,  则在  $\mathbb{R}[x]$  中  $\left(x^{n}-2, f(x)\right) \neq 1$,  于是在  $\mathbb{Q}[x]$  中  $\left(x^{n}-2, f(x)\right) \neq 1$,  又因为  $x^{n}-2$  是  $\mathbb{Q}[x]$  中的不可约多项式 ,  故  $x^{n}-2 \mid f(x)$,  矛盾 .

( 法二 )  若  $\sqrt[n]{2}$  是  $f(x)$  的实根 ,  则  $1, \sqrt[n]{2}, \sqrt[n]{2^{2}}, \ldots, \sqrt[n]{2^{m}}$  在  $\mathbb{Q}$  上是线性相关的 ,  存在最小的  $k, 0<k \leqslant m$,  使得 
$$
\sqrt[n]{2^{k}}=a_{0}+a_{1} \sqrt[n]{2}+a_{2} \sqrt[n]{2^{2}}+\cdots+a_{k-1} \sqrt[n]{2^{k-1}}
$$
 令  $f(x)=x^{k}-a_{0}-a_{1} x-\cdots-a_{k-1} x^{k-1}$,  用  $f(x)$  与  $x^{n}-2$  做带余除法 ,  可设  $x^{n}-2=f(x) q(x)+r(x)$,  由  $k$  的最小性得  $r(x)=0, f(x) \mid x^{n}-2$,  矛盾 .

 注   法二的想法源于蓝以中的 《 高等代数 》 第二版下册第  173  页的例  $2.4$.

 六 . 1.  设  $\lambda$  对应的一个特征向量为  $\xi$,  则  $\mathscr{A} \xi=\lambda \xi$,
$$
\lambda(\xi, \xi)=(\mathscr{A} \xi, \xi)=-(\xi, \mathscr{A} \xi)=-\lambda(\xi, \xi)
$$
 故  $2 \lambda(\xi, \xi)=0$,  由  $\xi \neq 0$  知  $(\xi, \xi) \neq 0$,  故  $\lambda=0$.

$2 .$
$$
\left(\mathscr{A}^{2} \alpha, \beta\right)=-(\mathscr{A} \alpha, \mathscr{A} \beta)=\left(\alpha, \mathscr{A}^{2} \beta\right),
$$
 于是  $\mathscr{A}^{2}$  是对称变换 ,  故在  $V$  内存在一组标准正交基 ,  使  $\mathscr{A}^{2}$  在此组基下的矩阵为对角矩阵 .

\begin{enumerate}
  \setcounter{enumi}{3}
  \item  设  $\mu$  为  $A$  的特征值 , $\eta$  为对应的特征向量 ,  则  $A \eta=\mu \eta, \eta \neq 0$,
\end{enumerate}
$$
A \bar{\eta}=\bar{\mu} \bar{\eta}, \quad \eta^{\mathrm{T}} A^{\mathrm{T}}=\mu \eta^{\mathrm{T}}
$$
 故 
$$
\eta^{\mathrm{T}} A \bar{\eta}=\bar{\mu} \eta^{\mathrm{T}} \bar{\eta}, \quad-\eta^{\mathrm{T}} A \bar{\eta}=\bar{\mu} \eta^{\mathrm{T}} \bar{\eta},
$$
 因此  $(\bar{\mu}+\mu) \eta^{\mathrm{T}} \bar{\eta}=0$,  由于  $\eta^{\mathrm{T}} \bar{\eta} \neq 0$,  故  $\bar{\mu}+\mu=0$,  故  $\mu$  必为  0  或纯虚数 .  北京大学  1998  年全国硕士研究生招生考试高代解几试题及解答 

   

2019.05.25

 一 . (15  分 )  设在直角坐标系中给出了两条互相异面的直线  $\ell_{1}, \ell_{2}$  的普通方程 :
$$
\ell_{1}:\left\{\begin{array}{r}
x+y+z-1=0 \\
x+y+2 z+1=0
\end{array}, \quad \ell_{2}:\left\{\begin{array}{l}
3 x+y+1=0 \\
y+3 z+2=0
\end{array}\right.\right.
$$
(1)  过  $\ell_{1}$  作平面  $\pi$,  使  $\pi$  与  $\ell_{2}$  平行 ;

(2)  求  $\ell_{1}$  与  $\ell_{2}$  间的距离 ;

(3)  求  $\ell_{1}$  与  $\ell_{2}$  的公垂线的方程 .

 二 . (15  分 )  在直角坐标系中 ,  球面的方程为 :
$$
(x-1)^{2}+y^{2}+(z+1)^{2}=4 .
$$
 求所有与向量  $u(1,1,1)$  平行的球面的切线所构成的曲面的方程 .

 三 . ( 本大题满分  16  分 )  讨论  $a, b$  满足什么条件时 ,  数域  $\mathbb{K}$  上的下述线性方程组有唯一解 ,  有无穷多个解 ,  无解 ?  当有解时 ,  求出该方程组的全部解 .
$$
\left\{\begin{array}{r}
a x_{1}+3 x_{2}+3 x_{3}=3 \\
x_{1}+4 x_{2}+x_{3}=1 \\
2 x_{1}+2 x_{2}+b x_{3}=2
\end{array}\right.
$$
 四 . ( 本大题满分  12  分 )  设  $V$  是定义域为实数集  $\mathbb{R}$  的所有实值函数组成的集合 ,  对于  $f, g \in V, \alpha \in \mathbb{R}$,  分别用   下列式子定义  $f+g$  与  $a f:$
$$
(f+g)(x)=f(x)+g(x), \quad(\alpha f)(x)=a(f(x)), \quad \forall x \in \mathbb{R}
$$
 则  $V$  成为实数域上的一个线性空间 .  设  $f_{0}(x)=1, f_{1}(x)=\cos x, f_{2}(x)=\cos 2 x, f_{3}(x)=\cos 3 x$.

(1)  判断  $f_{0}, f_{1}, f_{2}, f_{3}$  是否线性相关 ,  写出理由 ;

(2)  用  $\langle f, g\rangle$  表示  $f, g$  生成的线性子空间 ,  判断  $\left\langle f_{0}, f_{1}\right\rangle+\left\langle f_{2}, f_{3}\right\rangle$  是否为直和 ,  写出理由 .

 五 . ( 本大题满分  20  分 )  用  $J$  表示元素全为  1  的  $n$  级方阵 , $n \geqslant 2$.  设  $f(x)=a+b x$  是有理数域上的一元多项   式 ,  令  $A=f(J)$.

(1)  求  $J$  的全部特征值和全部特征向量 ;

(2)  求  $A$  的所有特征子空间 ;

(3) $A$  是否可以对角化 ?  如果可对角化 ,  求出有理数域上的一个可逆矩阵  $P$,  使得  $P^{-1} A P$  为对角矩阵 ,  并   且写出这个对角矩阵 .  六 . ( 本大题满分  22  分 )  用  $M_{2}(\mathbb{C})$  表示复数域  $\mathbb{C}$  上所有  2  级矩阵组成的集合 .  令 
$$
V=\left\{A \in M_{2}(\mathbb{C}) \mid \operatorname{Tr}(A)=0, \text { 且 } A^{*}=A\right\} \text {. }
$$
 其中  $\operatorname{Tr}(A)$  表示  $A$  的迹 , $A^{*}$  表示  $A$  的转置共轭矩阵 .

(1) 证明  $V$  对于矩阵的加法 ,  以及实数与矩阵的数量乘法成为实数域上的线性空间 ,  并且说明  $V$  中元素形   如 
$$
\left(\begin{array}{cc}
a_{1} & a_{2}+\mathrm{i} a_{3} \\
a_{2}-\mathrm{i} a_{3} & -a_{1}
\end{array}\right)
$$
 其中  $a_{1}, a_{2}, a_{3}$  都是实数 , $\mathrm{i}=\sqrt{-1}$.

(2)  设 
$$
A=\left(\begin{array}{cc}
a_{1} & a_{2}+\mathrm{i} a_{3} \\
a_{2}-\mathrm{i} a_{3} & -a_{1}
\end{array}\right), \quad B=\left(\begin{array}{cc}
b_{1} & b_{2}+\mathrm{i} b_{3} \\
b_{2}-\mathrm{i} b_{3} & -b_{1}
\end{array}\right)
$$
 考虑  $V$  上的一个二元函数 :
$$
(A, B)=a_{1} b_{1}+a_{2} b_{2}+a_{3} b_{3} .
$$
 证明这个二元函数是  $V$  上的一个内积 ,  从而  $V$  成为欧几里得空间 ;  并且求出  $V$  的一个标准正交基 ,  要   求写出理由 .

(3)  设  $T$  是一个西矩阵  ( 即 , $T$  满足  $T^{*} T=I$,  其中  $I$  是单位矩阵 ),  对任意  $A \in V$,  规定  $\Psi_{T}(A)=T A T-1$,  证明  $\Psi_{T}$  是  $V$  上的正交变换 .

(4) $\Psi_{T}$  的意义同第  $(3)$  小题 ,  求下述集合 
$$
S=\left\{T \mid \operatorname{det} T=1 \text { 且 } \Psi_{T}=1_{V}\right\} .
$$
 其中  $\operatorname{det} T$  表示  $T$  的行列式 , $1_{V}$  表示  $V$  上的恒等变换 .  一 . (1)  设  $\pi: \mu(x+y+z-1)+\lambda(x+y+2 z+1)=0$,  则  $\pi$  的一个法向量为  $(\mu+\lambda, \mu+\lambda, \mu+2 \lambda)$,  因为  $\pi$  与  $\ell_{2}$  平行 ,  从而 
$$
\left|\begin{array}{ccc}
3 & 1 & 0 \\
0 & 1 & 3 \\
\mu+\lambda & \mu+\lambda & \mu+2 \lambda
\end{array}\right|=0
$$
 解得  $\mu=0$,  于是  $\pi$  的方程为  $x+y+2 z+1=0$.

(2) $\ell_{1}$  的一个方向向量为  $(1,-1,0)$,  过点  $(3,0,-2), \ell_{2}$  的一个方向向量为  $(1,-3,1)$,  过点  $(-1 / 3,0,-2 / 3)$,  两直线的距离为 
$$
d=\frac{\left|\frac{10}{3}-\frac{8}{3}\right|}{\sqrt{1+1+4}}=\frac{\sqrt{6}}{9}
$$
(3)  直线  $\ell_{1}$  与公垂线决定的平面为 
$$
\left|\begin{array}{ccc}
1 & 1 & 2 \\
1 & -1 & 0 \\
x-3 & y & z+2
\end{array}\right|=0,
$$
 即  $x+y-z-5=0$.  直线  $\ell_{2}$  与公垂线决定的平面为 
$$
\left|\begin{array}{ccc}
1 & 1 & 2 \\
1 & -3 & 1 \\
x+\frac{1}{3} & y & z+\frac{2}{3}
\end{array}\right|=0,
$$
 即  $21 x+3 y-12 z-1=0$.  因此公垂线的方程为 
$$
\left\{\begin{array}{r}
x+y-z-5=0 \\
21 x+3 y-12 z-1=0
\end{array}\right.
$$
 二 .  曲面为圆柱面 ,  以过  $(1,0,-1)$  方向为  $(1,1,1)$  的直线为中心轴 ,  半径为  2 .  设  $(x, y, z)$  是曲面上一点 ,  则 
$$
\frac{|(x-1, y, z+1) \times(1,1,1)|}{|(1,1,1)|}=2 \text {, }
$$
 即  $(y-z-1)^{2}+(z-x+2)^{2}+(x-1-y)^{2}=12$.

 三 .
$$
\left[\begin{array}{llll}
a & 3 & 3 & 3 \\
1 & 4 & 1 & 1 \\
2 & 2 & b & 2
\end{array}\right] \rightarrow\left[\begin{array}{ccccc}
1 & 4 & 1 & 1 \\
0 & -6 & b-2 & 0 \\
0 & 3-4 a & 3-a & 3-a
\end{array}\right] \rightarrow\left[\begin{array}{cccc}
1 & 4 & 1 & 1 \\
0 & -6 & b-2 & 0 \\
0 & 0 & 3-a+\frac{(3-4 a)(b-2)}{6} & 3-a
\end{array}\right]
$$
 当  $3-a+\frac{(3-4 a)(b-2)}{6} \neq 0$  时 ,  有唯一解 .

 当  $3-a+\frac{(3-4 a)(b-2)}{6}=0$  且  $3-a=0$  时 ,  即  $a=3, b=2$  时 ,  有无穷个解 ,  解为  $k_{1}(0,0,1)^{\mathrm{T}}+$ $k_{2}(1,0,0)^{\mathrm{T}}, k_{1}, k_{2} \in K .$

 当  $3-a+\frac{(3-4 a)(b-2)}{6}=0$  且  $3-a \neq 0$  时 ,  无解 

 四 . (1)  设  $k_{0} f_{0}+k_{1} f_{1}+k_{2} f_{2}+k_{3} f_{3}=0$,  分别取  $x=\pi / 2,0, \pi, \pi / 4$  得 
$$
\left\{\begin{array}{r}
k_{0}-k_{2}=0 \\
k_{0}+k_{1}+k_{2}+k_{3}=0 \\
k_{0}-k_{1}+k_{2}-k_{3}=0 \\
k_{0}+\frac{\sqrt{2}}{2} k_{1}-\frac{\sqrt{2}}{2} k_{3}=0
\end{array}\right.
$$
 于是  $k_{0}=k_{1}=k_{2}=k_{3}=0$,  从而  $f_{0}, f_{1}, f_{2}, f_{3}$  线性无关 . (2)  任取  $f \in\left\langle f_{0}, f_{1}\right\rangle \cap\left\langle f_{2}, f_{3}\right\rangle$,  则  $f=k_{0} f_{0}+k_{1} f_{1}=k_{2} f_{2}+k_{3} f_{3}$,  于是  $k_{0}=k_{1}=0$,  从而  $f=0$,  故  $\left\langle f_{0}, f_{1}\right\rangle+\left\langle f_{2}, f_{3}\right\rangle$  是直和 .

 五 . (1) $|\lambda E-J|=(\lambda-n) \lambda^{n-1}, 0$  是  $J$  的特征值 ,  对应的特征向量为 
$$
k_{1}\left[\begin{array}{c}
1 \\
-1 \\
0 \\
\vdots \\
0
\end{array}\right]+k_{2}\left[\begin{array}{c}
1 \\
0 \\
-1 \\
\vdots \\
0
\end{array}\right]+\cdots+k_{n-1}\left[\begin{array}{c}
1 \\
0 \\
0 \\
\vdots \\
-1
\end{array}\right]
$$
 其中  $k_{1}, k_{2}, \ldots, k_{n-1}$  不全为  $0 . n$  是  $J$  的特征值 ,  对应的特征向量为  $k(1,1, \ldots, 1)^{\mathrm{T}}, k \neq 0$.

(2) $A$  的特征值  $a$  对应的特征子空间为 
$$
\left\{k_{1}\left[\begin{array}{c}
1 \\
-1 \\
0 \\
\vdots \\
0
\end{array}\right]+k_{2}\left[\begin{array}{c}
1 \\
0 \\
-1 \\
\vdots \\
0
\end{array}\right]+\cdots+k_{n-1}\left[\begin{array}{c}
1 \\
0 \\
0 \\
\vdots \\
-1
\end{array}\right] k_{1}, k_{2}, \ldots, k_{n-1} \text { 不全为 } 0\right\}
$$
$A$  的特征值  $a+b n$  对应的特征子空间为  $\left\{k(1,1, \ldots, 1)^{\mathrm{T}} \mid k \neq 0\right\}$.

(3) $A$  有  $n$  个线性无关的特征向量 ,  从而可对角化 ,  取 
$$
P=\left[\begin{array}{ccccc}
1 & 1 & 1 & \cdots & 1 \\
1 & -1 & 0 & \cdots & 0 \\
1 & 0 & -1 & \cdots & 0 \\
\vdots & \vdots & \vdots & & \vdots \\
1 & 0 & 0 & \cdots & -1
\end{array}\right] \text {, }
$$
 则  $P^{-1} A P=\operatorname{diag}\{a+b n, a, \ldots, a\}$.

 六 . (1) $\forall A, B \in V, \operatorname{Tr}(A+B)=\operatorname{Tr}(A)+\operatorname{Tr}(B)=0,(A+B)^{*}=A^{*}+B^{*}=A+B$,  于是  $A+B \in V$. $\forall k \in \mathbb{R}, A \in V, \operatorname{Tr}(k A)=k \operatorname{Tr}(A)=0,(k A)^{*}=k A^{*}=k A$,  于是  $k A \in V$.  从而  $V$  构成了一个子空间 . $\forall A \in V$,  由  $A^{*}=\overline{\left(A^{\mathrm{T}}\right)}=A$  知 
$$
\left[\begin{array}{ll}
\overline{a_{11}} & \overline{a_{21}} \\
\overline{a_{12}} & \overline{a_{22}}
\end{array}\right]=\left[\begin{array}{ll}
a_{11} & a_{12} \\
a_{21} & a_{22}
\end{array}\right]
$$
 敇 
$$
\left\{\begin{array}{l}
\overline{a_{11}}=a_{11} \\
\overline{a_{22}}=a_{22} \\
\overline{a_{12}}=a_{21} \\
\overline{a_{21}}=a_{12}
\end{array}\right.
$$
 再结合  $\operatorname{Tr}(A)=0$  知  $a_{11}+a_{22}=0$,  整理后就得到  $V$  中元素的一般形式 .

(2) $(A, A)=a_{1}^{2}+a_{2}^{2}+a_{3}^{2} \geqslant 0$  且  $(A, A)=0 \Longleftrightarrow A=0$.

$(A, B)=(B, A)$.

$(k A, B)=k(A, B), \forall k \in \mathbb{R} .$ $(A+B, C)=(A, C)+(B, C) .$

 从而此二元函数是  $V$  上的一个内积 .
$$
A=a_{1}\left[\begin{array}{cc}
1 & 0 \\
0 & -1
\end{array}\right]+a_{2}\left[\begin{array}{ll}
0 & 1 \\
1 & 0
\end{array}\right]+a_{3}\left[\begin{array}{cc}
0 & \mathrm{i} \\
-\mathrm{i} & 0
\end{array}\right]
$$
 由此可知 
$$
\left[\begin{array}{cc}
1 & 0 \\
0 & -1
\end{array}\right]\left[\begin{array}{ll}
0 & 1 \\
1 & 0
\end{array}\right]\left[\begin{array}{cc}
0 & \mathrm{i} \\
-\mathrm{i} & 0
\end{array}\right]
$$
 是  $V$  的一组基 ,  并且恰好为标准正交基 .

(3) $\forall A \in V, \operatorname{Tr}\left(T A T^{-1}\right)=\operatorname{Tr}\left(T^{-1} T A\right)=\operatorname{Tr}(A),\left(T A T^{-1}\right)^{*}=T A^{*} T^{*}=T A T^{-1}$,  故  $\Psi_{T}(A) \in V$,  从而  $\Psi_{T}$  是  $V$  到  $V$  的映射 .
$$
\begin{gathered}
\Psi_{T}(k A)=T(k A) T^{-1}=k T A T^{-1}=k \Psi_{T}(A) \\
\Psi_{T}(A+B)=T(A+B) T^{-1}=T A T^{-1}+T B T^{-1}=\Psi_{T}(A)+\Psi_{T}(B),
\end{gathered}
$$
 从而  $\Psi_{T}$  是  $V$  上的线性变换 .

$(A, A)=-|A|$,
$$
(A, B)=\frac{1}{2}((A+B, A+B)-(A, A)-(B, B))=\frac{1}{2}(-|A+B|+|A|+|B|)
$$
 于是 
$$
\left(T A T^{-1}, T B T^{-1}\right)=\frac{1}{2}\left(-\left|T(A+B) T^{-1}\right|+\left|T A T^{-1}\right|+\left|T B T^{-1}\right|\right)=(A, B),
$$
 故  $\Psi_{T}$  是  $V$  上的正交变换 .

(4)  对于任意取定的  $S$  中元素  $T=\left(t_{i j}\right)_{2 \times 2}$, $\operatorname{det} T=1$  并且  $\forall A \in V, T A T T^{-1}=A \Longleftrightarrow T A$.  由 
$$
\left[\begin{array}{ll}
t_{11} & t_{12} \\
t_{21} & t_{22}
\end{array}\right]\left[\begin{array}{cc}
1 & 0 \\
0 & -1
\end{array}\right]=\left[\begin{array}{cc}
1 & 0 \\
0 & -1
\end{array}\right]\left[\begin{array}{ll}
t_{11} & t_{12} \\
t_{21} & t_{22}
\end{array}\right] \Longrightarrow\left\{\begin{array}{l}
t_{12}=0 \\
t_{21}=0
\end{array} .\right.
$$
H
$$
\left[\begin{array}{ll}
t_{11} & t_{12} \\
t_{21} & t_{22}
\end{array}\right]\left[\begin{array}{ll}
0 & 1 \\
1 & 0
\end{array}\right]=\left[\begin{array}{ll}
0 & 1 \\
1 & 0
\end{array}\right]\left[\begin{array}{ll}
t_{11} & t_{12} \\
t_{21} & t_{22}
\end{array}\right] \Longrightarrow t_{11}=t_{22}
$$
 于是  $T=t_{11} E$,  由  $\operatorname{det} T=t_{11}^{2}=1$  可得  $T=E$  或  $T=-E$,  因此  $S=\{E,-E\}$.  北京大学  1999  年全国硕士研究生招生考试高代解几试题及解答 

   

2019.05.26

 一 . (20  分 )  在仿射坐标系中 ,  已知直线  $\ell_{1}, \ell_{2}$  的方程分别是 :
$$
\frac{x+13}{2}=\frac{y-5}{3}=\frac{z}{1}, \quad \frac{x-10}{5}=\frac{y+7}{4}=\frac{z}{1} .
$$
(1)  判断  $\ell_{1}$  与  $\ell_{2}$  的位置关系 ,  并说明理由 .

(2)  设直线  $\ell$  的一个方向向量  $\vec{v}(8,7,1)$,  并且  $\ell$  与  $\ell_{1}$  和  $\ell_{2}$  都相交 ,  求直线  $\ell$  的方程 .

 二 . (10  分 )  在直角坐标系  $O x y z$  中 ,  设顶点在原点的二次雉面  $S$  的方程为 :
$$
a_{11} x^{2}+a_{22} y^{2}+a_{33} z^{2}+2 a_{12} x y+2 a_{13} x z+2 a_{23} y z=0 .
$$
(1)  如果三条坐标轴都是  $S$  的母线 ,  求  $a_{11}, a_{22}, a_{33}$;

(2)  证明 :  如果  $S$  有三条互相垂直的直母线 ,  则 
$$
a_{11}+a_{22}+a_{33}=0 .
$$
 三 . (16  分 )  设实数域上的矩阵  $A$  为 
$$
A=\left(\begin{array}{ccc}
1 & 1 & 0 \\
-1 & 0 & 1 \\
-3 & 0 & 0
\end{array}\right)
$$
(1)  求  $A$  的特征多项式  $f(\lambda)$;

(2) $f(\lambda)$  是否为实数域上的不可约多项式 ;

(3)  求  $A$  的最小多项式 ,  要求写出理由 ;

(4)  实数域上的矩阵  $A$  是否可对角化 ,  要求写出理由 .

 四 . (16  分 )  设实数域上的矩阵  $A$  为 
$$
A=\left(\begin{array}{ccc}
1 & 0 & 1 \\
0 & 6 & -2 \\
1 & -2 & 2
\end{array}\right) \text {. }
$$
(1)  判断  $A$  是否为正定矩阵 ,  要求写出理由 ;

(2) 设  $V$  是实数域上的  3  维线性空间 , $V$  上的一个双线性函数  $f(\alpha, \beta)$  在  $V$  的一个基  $\alpha_{1}, \alpha_{2}, \alpha_{3}$  下的度量   矩阵为  $A$.  证明 : $f(\alpha, \beta)$  是  $V$  的一个内积 ,  并且求出  $V$  对于这个内积所成的欧式空间的一个标准正交   基 .

 五 . (16  分 )  设  $V$  是数域  $\mathbb{K}$  上的一个  $n$  维线性空间 , $\alpha_{1}, \alpha_{2}, \cdots, \alpha_{n}$  是  $V$  的一个基 .  用  $V_{1}$  表示由  $\alpha_{1}+\alpha_{2}+\cdots+\alpha_{n}$  生成的线性子空间 ,  令 
$$
V_{2}=\left\{\sum_{i=1}^{n} k_{i} \alpha_{i} \mid \sum_{i=1}^{n} k_{i}=0, k_{i} \in \mathbb{K}\right\} .
$$
(1)  证明  $V_{2}$  是  $V$  的子空间 ;

(2)  证明  $V=V_{1} \oplus V_{2}$;

(3)  设  $V$  上的一个线性变换  $\mathscr{A}$  在基  $\alpha_{1}, \alpha_{2}, \cdots, \alpha_{n}$  下的矩阵  $A$  是置换矩阵  ( 即 : $A$  的每一行与每一列都   只有一个元素是  1 ,  其余元素全为  0$)$,  证明  $V_{1}$  与  $V_{2}$  都是  $\mathscr{A}$  的不变子空间 .

 六 . (12  分 )  设  $V$  和  $U$  分别是数域  $\mathbb{K}$  上的  $n$  维 , $m$  维线性空间 , $\mathscr{A}$  是  $V$  到  $U$  的一个线性映射 ,  即  $\mathscr{A}$  是  $V$  到  $U$  的映射 ,  且满足 
$$
\begin{aligned}
&\mathscr{A}(\alpha+\beta)=\mathscr{A} \alpha+\mathscr{A} \beta, \quad \forall \alpha, \beta \in V \\
&\mathscr{A}(k \alpha)=k \mathscr{A} \alpha, \quad \forall \alpha \in V, k \in \mathbb{K} .
\end{aligned}
$$
 今 
$$
\operatorname{Ker} \mathscr{A}:=\{\alpha \in V \mid \mathscr{A} \alpha=0\},
$$
 称  $\operatorname{Ker} \mathscr{A}$  是  $\mathscr{A}$  的核 ,  它是  $V$  的一个子空间 ,  用  $\operatorname{Im} \mathscr{A}$  表示  $\mathscr{A}$  的象  ( 即 ,  值域 ).

(1)  证明 : $\operatorname{dim}(\operatorname{Ker} \mathscr{A})+\operatorname{dim}(\operatorname{Im} \mathscr{A})=\operatorname{dim} V$;

(2)  证明 :  如果  $\operatorname{dim} V=\operatorname{dim} U$,  则  $\mathscr{A}$  是单射当且仅当  $\mathscr{A}$  是满射 .

 七 . (10  分 )  设  $V$  是实数域  $\mathbb{R}$  上的  $n$  维线性空间 . $V$  上的所有复值函数组成的集合 ,  对于函数的加法以及复数   与函数的数量乘法 ,  形成复数域  $\mathbb{C}$  上的一个线性空间 ,  记作  $C^{V}$.

 证明 :  如果  $f_{1}, f_{2}, \cdots, f_{n+1}$  是  $C^{V}$  中  $n+1$  个不同的函数 ,  并且它们满足 
$$
\begin{gathered}
f_{i}(\alpha+\beta)=f_{i}(\alpha)+f_{i}(\beta), \quad \forall \alpha, \beta \in V, \\
f_{i}(k \alpha)=k f_{i}(\alpha), \quad \forall k \in \mathbb{R}, \alpha \in V
\end{gathered}
$$
 则  $f_{1}, f_{2}, \cdots, f_{n+1}$  是  $C^{V}$  中线性相关的向量组 .  一 . (1) $\ell_{1}$  过点  $(-13,5,0)$,  记为  $P_{1}, \ell_{2}$  过点  $(10,-7,0)$,  记为  $P_{2}$,  则  $\overrightarrow{P_{1} P_{2}}=(23,-12,0)$.  因为 
$$
\left|\begin{array}{ccc}
23 & -12 & 0 \\
2 & 3 & 1 \\
5 & 4 & 1
\end{array}\right|=\left|\begin{array}{ccc}
23 & -12 & 0 \\
-3 & -1 & 0 \\
5 & 4 & 1
\end{array}\right| \neq 0,
$$
 故  $\ell_{1}$  与  $\ell_{2}$  异面 .

(2)  设  $\ell$  与  $\ell_{1}$  和  $\ell_{2}$  的交点分别为  $Q_{1}(-13+2 t, 5+3 t, t), Q_{2}(10+5 s,-7+4 s, s)$,  则 
$$
\overrightarrow{Q_{1} Q_{2}}=(23+5 s-2 t,-12+4 s-3 t, s-t)
$$
 由  $\overrightarrow{Q_{1} Q_{2}} / / \vec{v}$  得 
$$
\frac{23+5 s-2 t}{8}=\frac{-12+4 s-3 t}{7}=\frac{s-t}{1} \Longrightarrow\left\{\begin{array}{l}
s=-\frac{82}{3} \\
t=-\frac{35}{2}
\end{array} \Longrightarrow Q_{1}\left(-48,-\frac{95}{2},-\frac{35}{2}\right)\right.
$$
 因此直线  $\ell$  的方程为 
$$
\frac{x+48}{8}=\frac{y+\frac{95}{2}}{7}=z+\frac{35}{2} .
$$
 注   丘维声的 《 解析几何 》 第三版第  69  页习题  $2.3$  第  10  题的  (3).

 二 . (1)  因为点  $(1,0,0),(0,1,0),(0,0,1)$  在  $S$  上 ,  故  $a_{11}=a_{22}=a_{33}=0$.

(2)  作正交坐标变换 
$$
\left[\begin{array}{l}
x \\
y \\
z
\end{array}\right]=T\left[\begin{array}{l}
x^{\prime} \\
y^{\prime} \\
z^{\prime}
\end{array}\right]
$$
 其中  $T^{\prime} T=E$,  使得  $S$  的三条相互垂直的直母线为新坐标系的坐标轴 .  设  $T^{\prime} A T=\left(b_{i j}\right)_{3 \times 3}$,  由  $(1)$  知  $b_{11}=b_{22}=b_{33}=0$,  从而 
$$
a_{11}+a_{22}+a_{33}=\operatorname{tr}(A)=\operatorname{tr}\left(T^{\prime} A T\right)=\operatorname{tr}(B)=b_{11}+b_{22}+b_{33}=0
$$
 注   丘维声的 《 解析几何 》 第三版第  144  页习题  $4.4$  第  10  题的必要性部分 .

 三 . (1) $f(\lambda)=\lambda^{3}-\lambda^{2}+\lambda+3=(\lambda+1)\left(\lambda^{2}-2 \lambda+3\right)$.

(2)  否 .

(3) $f(\lambda)$  即为  $A$  的最小多项式 ,  因为  $f(\lambda)$  为  $\mathbb{C}$  上互素一次因式的乘积 .

(4) $A$  不能在  $\mathbb{R}$  上对角化 ,  因为最小多项式  $f(\lambda)$  不是  $\mathbb{R}$  上互素一次因式的乘积 .

 四 . (1)  一阶主子式 : $1>0$; 二阶主子式 : $\left|\begin{array}{ll}1 & 0 \\ 0 & 6\end{array}\right|=6>0 ;$  三阶主子式 : $|A|=2>0$,  故  $A$  是正定矩阵 .

(2) $\forall \alpha, \beta \in V$,  可设  $\alpha=x_{1} \alpha_{1}+x_{2} \alpha+x_{3} \alpha_{3}, \beta=y_{1} \beta_{1}+y_{2} \beta_{2}+y_{3} \beta_{3}$,  则 
$$
f(\alpha, \beta)=\left[\begin{array}{lll}
x_{1} & x_{2} & x_{3}
\end{array}\right] A\left[\begin{array}{l}
y_{1} \\
y_{2} \\
y_{3}
\end{array}\right]
$$
 由于  $A^{\prime}=A$,  故  $f(\alpha, \beta)=f(\beta, \alpha), \forall \alpha, \beta \in V$.  由矩阵乘法的性质知  $f\left(k_{1} \alpha_{1}+k_{2} \alpha_{2}, \beta\right)=k_{1} f\left(\alpha_{1}, \beta\right)+$ $k_{2} f\left(\alpha_{2}, \beta\right), \forall \alpha_{1}, \alpha_{2}, \beta \in V$.  因为  $A$  正定 ,  故  $f(\alpha, \alpha) \geqslant 0$,  并且  $f(\alpha, \alpha)=0$  当且仅当  $\alpha=0$.
$$
\xi_{1}=\alpha_{1}, \quad \xi_{2}=\alpha_{2}-\frac{f\left(\alpha_{2}, \xi_{1}\right)}{f\left(\xi_{1}, \xi_{1}\right)} \xi_{1}=\alpha_{2}, \quad \xi_{3}=\alpha_{3}-\frac{f\left(\alpha_{3}, \xi_{1}\right)}{f\left(\xi_{1}, \xi_{1}\right)} \xi_{1}-\frac{f\left(\alpha_{3}, \xi_{2}\right)}{f\left(\xi_{2}, \xi_{2}\right)} \xi_{2}=\alpha_{3}-\alpha_{1}+\frac{1}{3} \alpha_{2}
$$
 再单位化得 
$$
\eta_{1}=\frac{\xi_{1}}{\sqrt{f\left(\xi_{1}, \xi_{1}\right)}}=\alpha_{1}, \quad \eta_{2}=\frac{\xi_{2}}{\sqrt{f\left(\xi_{2}, \xi_{2}\right)}}=\frac{\alpha_{2}}{\sqrt{6}}, \quad \eta_{3}=\frac{\xi_{3}}{\sqrt{f\left(\xi_{3}, \xi_{3}\right)}}=\sqrt{3} \alpha_{3}-\sqrt{3} \alpha_{1}+\frac{\sqrt{3}}{3} \alpha_{2} .
$$
 五 . (1) $\mathbf{0} \in V_{2}$,  从而  $V_{2} \neq \varnothing . \forall \beta, \gamma \in V_{2}$,  可设 
$$
\beta=\sum_{i=1}^{n} k_{1 i} \alpha_{i}, \quad \text { 其中 } \sum_{i=1}^{n} k_{1 i}=0 ; \quad \gamma=\sum_{i=1}^{n} k_{2 i} \alpha_{i}, \quad \text { 其中 } \sum_{i=1}^{n} k_{2 i}=0 .
$$
 于是 
$$
\beta+\gamma=\sum_{i=1}^{n}\left(k_{1 i}+k_{2 i}\right) \alpha_{i}, \quad \sum_{i=1}^{n}\left(k_{1 i}+k_{2 i}\right)=0
$$
 故  $\beta+\gamma \in V_{2}$.  类似可知  $\forall \lambda \in K, \lambda \beta \in V_{2}$,  故  $V_{2}$  是  $V$  的子空间 .

(2)  若  $\alpha \in V_{1} \cap V_{2}$,  则存在  $\lambda \in K, \alpha=\lambda\left(\alpha_{1}+\alpha_{2}+\cdots+\alpha_{n}\right)$,  同时有  $\sum_{i=1}^{n} \lambda=n \lambda=0$,  故  $\lambda=0, \alpha=0$,  因此  $V_{1} \cap V_{2}=\{0\}$.

$\forall \alpha \in V$,  可设  $\alpha=x_{1} \alpha_{1}+x_{2} \alpha_{2}+\cdots+x_{n} \alpha_{n}$,  那么 
$$
\alpha=\left(\sum_{i=1}^{n} x_{i}\right)\left(\alpha_{1}+\alpha_{2}+\cdots+\alpha_{n}\right)+\sum_{i=1}^{n}\left(x_{i}-\sum_{j=1}^{n} x_{j}\right) \alpha_{i} \in V_{1}+V_{2}
$$
 综合上面两点得  $V=V_{1} \oplus V_{2}$.

(3)  设  $\mathscr{A} \alpha_{i}=\alpha_{p_{i}}, 1 \leqslant i \leqslant n$.  其中  $p_{1}, p_{2}, \ldots, p_{n}$  为  $1,2, \ldots, n$  的一个重排 ,  则 
$$
\mathscr{A}\left(\alpha_{1}+\alpha_{2}+\cdots+\alpha_{n}\right)=\alpha_{p_{1}}+\cdots+\alpha_{p_{n}}=\alpha_{1}+\alpha_{2}+\cdots+\alpha_{n},
$$
 因此  $V_{1}$  是  $\mathscr{A}$  的不变子空间 .
$$
\mathscr{A}\left(\sum_{i=1}^{n} k_{i} \alpha_{i}\right)=\sum_{i=1}^{n} k_{i} \mathscr{A} \alpha_{i}=\sum_{i=1}^{n} k_{i} \alpha_{p_{i}}, \quad \sum_{i=1}^{n} k_{i}=0
$$
 因此  $V_{2}$  是  $\mathscr{A}$  的不变子空间 .

 注   相关题目见维声  《 高等代数 》 创新教材下册第  201  页习题  $8.2$  第  22  题 .

 六 . (1)  设  $\alpha_{1}, \ldots, \alpha_{s}$  是  $\operatorname{Ker} \mathscr{A}$  的一组基 ,  因为  $\operatorname{Ker} \mathscr{A}$  是  $V$  的子空间 ,  可将它扩充为  $V$  的一组基  $\alpha_{1}, \ldots, \alpha_{s}$, $\alpha_{s+1}, \ldots, \alpha_{n}$,  然后验证  $\mathscr{A} \alpha_{s+1}, \ldots, \mathscr{A} \alpha_{n}$  为  $\operatorname{Im} \mathscr{A}$  的一组基即可 .

(2)  利用  $(1)$.  注意到  $\mathscr{A}$  为单射当且仅当  $\operatorname{Ker} \mathscr{A}=\{0\}$. $\mathscr{A}$  为满射当且仅当  $\operatorname{Im} \mathscr{A}=U$.

 注   高等代数教材上线性映射部分的定理 ,  此处只给出提示 ,  如果还是不明白 ,  请翻书 .

 七 .  设  $\alpha_{1}, \alpha_{2}, \ldots, \alpha_{n}$  是  $V$  的一组基 ,  则  $f_{i}$  完全由  $f\left(\alpha_{1}\right), f\left(\alpha_{2}\right), \ldots, f\left(\alpha_{n}\right)$  决定 .  线性方程组 
$$
\left[\begin{array}{ccc}
f_{1}\left(\alpha_{1}\right) & \cdots & f_{n+1}\left(\alpha_{1}\right) \\
\vdots & \vdots & \vdots \\
f_{1}\left(\alpha_{n}\right) & \cdots & f_{n+1}\left(\alpha_{n}\right)
\end{array}\right]\left[\begin{array}{c}
k_{1} \\
\vdots \\
k_{n+1}
\end{array}\right]=\left[\begin{array}{c}
0 \\
\vdots \\
0
\end{array}\right]
$$
 必有非零解  $\left[\begin{array}{lll}k_{1}^{*} & \ldots & k_{n+1}^{*}\end{array}\right]^{\mathrm{T}}$,  其中  $k_{i}^{*} \in \mathbb{C}$.  下面说明  $g=k_{1}^{*} f_{1}+\cdots+k_{n+1}^{*} f_{n+1}=0$,  事实上由于  $g\left(\alpha_{1}\right)=\cdots=g\left(\alpha_{n}\right)=0$,  故  $g=0$.  北京大学  2000  年全国硕士研究生招生考试高代解几试题及解答 

   

2019.05.26

 一 . (20  分 )

(1)  在直角坐标系中 ,  一个柱面的准线方程为 
$$
\left\{\begin{array}{r}
x y=4 \\
z=0
\end{array}\right.
$$
 母线方向为  $(1,-1,1)$,  求这个柱面的方程 ;

(2)  在平面直角坐标系  $O x y$  中 ,  二次曲线的方程为 :
$$
x^{2}-3 x y+y^{2}+10 x-10 y+21=0,
$$
 求  $I_{1}, I_{2}, I_{3}$;  指出这是什么二次曲线 ,  并且确定其形状 .

 二 . $(22$  分 )

(1)  设实数域上的矩阵 
$$
A=\left(\begin{array}{lll}
2 & 0 & 4 \\
0 & 6 & 0 \\
4 & 0 & 2
\end{array}\right),
$$
 求正交矩阵  $T$,  使得  $T^{-1} A T$  为对角矩阵 ,  并且写出这个对角矩阵 .

(2)  在直角坐标系  $O x y z$  中 ,  二次曲面  $S$  的方程为 :
$$
2 x^{2}+6 y^{2}+2 z^{2}+8 x z=1,
$$
 作直角坐标变换 ,  把  $S$  的方程化成标准方程 ,  并且指出它是什么二次曲面 .

 三 . (12  分 )  设实数域上的  $s \times n$  矩阵  $A$  的元素只有  0  和  1 ,  并且  $A$  的每一行的元素的和是常数  $r, A$  的每两个   行向量的内积为常数  $m$,  其中  $m<r$.

(1)  求  $\left|A A^{\prime}\right|$;

(2)  证明  $s \leqslant n$;

(3)  证明  $A A^{\prime}$  的特征值全为正实数 .

 四 . (8  分 )  设  $V$  是数域  $\mathbb{K}$  上的  $n$  维线性空间 , $\mathscr{A}$  是  $V$  上的线性变换 ,  且满足  $\mathscr{A}^{3}-7 \mathscr{A}=-6 \mathscr{I}$,  其中  $\mathscr{I}$  表   示  $V$  上的恒等变换 ,  判断  $\mathscr{A}$  是否可对角化 ,  写出理由 .

 五 . (12  分 )  设  $V$  和  $V^{\prime}$  都是数域  $\mathbb{K}$  上的有限维线性空间 , $\mathscr{A}$  是  $V$  到  $V^{\prime}$  的一个线性映射 .  证明 :  存在直和分解 
$$
V=U \oplus W, \quad V^{\prime}=M \oplus N,
$$
 使得  $\operatorname{Ker} \mathscr{A}=U$,  并且  $W \cong M$.  六 . (10  分 )  设  $f(x)$  和  $p(x)$  都是首相系数为  1  的整系数多项式 ,  且  $p(x)$  在有理数域  $\mathbb{Q}$  上不可约 ,  如果  $p(x)$  与  $f(x)$  有公共复根  $\alpha$,  证明 :

(1)  在  $\mathbb{Q}[x]$  中 , $p(x)$  整除  $f(x)$;

(2)  存在首相系数为  1  的整系数多项式  $g(x)$,  使得 
$$
f(x)=p(x) g(x)
$$
 七 . (16  分 )

(1) 设  $V$  是实数域上的线性空间 , $f$  是  $V$  上的正定的对称双线性函数 , $U$  是  $V$  的有限维子空间 ,  证明 :
$$
V=U \oplus U^{\perp}
$$
 其中  $U^{\perp}=\{\alpha \in V \mid f(\alpha, \beta)=0, \forall \beta \in U\}$.

(2) 设  $V$  是数域  $\mathbb{K}$  上的  $n$  维线性空间 , $g$  是  $V$  上的非退化的对称双线性函数 , $W$  是  $V$  的子空间 .  令 
$$
W^{\perp}=\{\alpha \in V \mid g(\alpha, \beta)=0, \forall \beta \in W\}
$$
 证明 :

(a) $\operatorname{dim} V=\operatorname{dim} W+\operatorname{dim} W^{\perp}$;

(b) $\left(W^{\perp}\right)^{\perp}=W$.  一 . (1)  设  $(x, y, z)$  为柱面上一点 , $\left(x_{0}, y_{0}, z_{0}\right)$  为同一条母线上与准线相交的那个点 ,  则 
$$
\left\{\begin{array}{r}
\frac{x-x_{0}}{1}=\frac{y-y_{0}}{-1}=\frac{z-z_{0}}{1} \\
x_{0} y_{0}=4 \\
z_{0}=0
\end{array}\right.
$$
 化简得柱面方程为  $(x-z)(y+z)=4$.

 注   丘维声的 《 解析几何 》 第三版第  91  页习题  $3.2$  第  4  题的  (2).

(2) $I_{1}=2, I_{2}=-5 / 4, I_{3}=-5 / 4$.  因为  $I_{2}<0$,  故是双曲线型曲线 .  又因为  $I_{3} \neq 0$,  故曲线为双曲线 .  注   丘维声的 《 解析几何 》 第三版第  166  页例  $2.1$  的  (1).

 二 . (1) $|\lambda E-A|=(\lambda+2)(\lambda-6)^{2} . \lambda=-2$  对应的一个特征向量  $\xi_{1}=(1,0,-1)^{\mathrm{T}} . \lambda=6$  对应的两个特征向量  $\xi_{2}=(1,0,1)^{\mathrm{T}}, \xi_{3}=(0,1,0)^{\mathrm{T}}$.  令 
$$
T=\left[\begin{array}{ccc}
\frac{1}{\sqrt{2}} & \frac{1}{\sqrt{2}} & 0 \\
0 & 0 & 1 \\
-\frac{1}{\sqrt{2}} & \frac{1}{\sqrt{2}} & 0
\end{array}\right]
$$
 则  $T^{-1} A T=\operatorname{diag}\{-2,6,6\}$.

(2)  今今 
$$
\left[\begin{array}{l}
x \\
y \\
z
\end{array}\right]=T\left[\begin{array}{l}
x^{\prime} \\
y^{\prime} \\
z^{\prime}
\end{array}\right]
$$
 带入原方程得 
$$
-2\left(x^{\prime}\right)^{2}+6\left(y^{\prime}\right)^{2}+6\left(z^{\prime}\right)^{2}=1
$$
 曲面为单叶双曲面 .

 三 . (1)  将  $A$  按行向量分块 ,  即设 
$$
A=\left[\begin{array}{c}
\alpha_{1} \\
\vdots \\
\alpha_{s}
\end{array}\right]
$$
 冊 
$$
A A^{\prime}=\left[\begin{array}{c}
\alpha_{1} \\
\vdots \\
\alpha_{s}
\end{array}\right]\left[\begin{array}{ccc}
\alpha_{1}^{\mathrm{T}} & \ldots & \alpha_{s}^{\mathrm{T}}
\end{array}\right]=\left[\begin{array}{cccc}
r & m & \ldots & m \\
m & r & \ldots & m \\
\vdots & \vdots & & \vdots \\
m & m & \ldots & r
\end{array}\right]
$$
 故  $\left|A A^{\prime}\right|=(r+(s-1) m)(r-m)^{s-1}$.

(2) $\operatorname{rank}\left(A A^{\prime}\right)=s$.  若  $s>n$,  则  $\operatorname{rank}\left(A A^{\prime}\right) \leqslant \operatorname{rank}(A) \leqslant n<s$,  矛盾 .

(3)
$$
|\lambda E-A|=\left|\begin{array}{cccc}
\lambda-r & -m & \cdots & -m \\
-m & \lambda-r & \ldots & -m \\
\vdots & \vdots & & \vdots \\
-m & -m & \ldots & \lambda-r
\end{array}\right|=(\lambda-r-(s-1) m)(\lambda-r+m)^{s-1},
$$
 故特征值为  $r+(s-1) m$  或  $r-m$,  均是正数 .  四 .  令  $g(x)=x^{3}-7 x+6=(x-1)(x-2)(x+3)$,  则  $g(\mathscr{A})=0$. $\mathscr{A}$  的最小多项式必为  $g(x)$  的因式 ,  从而必为  $\mathbb{K}$  上互素一次因式的乘积 ,  从而  $\mathscr{A}$  可对角化 .

 五 .  令  $U=\mathrm{Ker} \mathscr{A}$,  则  $U$  为  $V$  的子空间 ,  从而必存在子空间  $W$,  使得  $V=U \oplus W$.  下面只需证明  $V^{\prime}$  有与  $W$  同构的子空间  $M$.  设  $U$  的一组基为  $\alpha_{1}, \ldots, \alpha_{s}$,  将其扩充为  $V$  的一组基  $\alpha_{1}, \ldots, \alpha_{s}, \alpha_{s+1}, \ldots, \alpha_{n}$,  令  $M=L\left(\mathscr{A} \alpha_{s+1}, \ldots, \mathscr{A} \alpha_{n}\right)$,  则  $M$  是  $V^{\prime}$  的子空间 ,  若能说明  $\operatorname{dim} M=n-s=\operatorname{dim} W$,  则命题得证 .

 事实上 ,  若  $k_{s+1} \mathscr{A} \alpha_{s+1}+\cdots+k_{n} \mathscr{A} \alpha_{n}=0$,  则  $\mathscr{A}\left(k_{s+1} \alpha_{s+1}+\cdots+k_{n} \alpha_{n}\right)=0$,  因此  $k_{s+1} \alpha_{s+1}+\cdots+k_{n} \alpha_{n}=$ $k_{1} \alpha_{1}+\cdots+k_{s} \alpha_{s}$,  故  $k_{1}=\cdots=k_{s}=k_{s+1}=\cdots=k_{n}=0$.  这就说明  $\mathscr{A} \alpha_{s+1}, \ldots, \mathscr{A} \alpha_{n}$  线性无关 , $\operatorname{dim} M=n-s$.

 注   相关题目见庍维声 《 高等代数 》 创新教材下册第  244  页例  6 .

 六 . (1) $p(x)$  在  $\mathbb{Q}[x]$  中不可约 ,  故在  $\mathbb{Q}[x]$  中 ,  要么  $(p(x), f(x))=1$,  要么  $p(x) \mid f(x)$.  如果在  $\mathbb{Q}[x]$  中  $(p(x), f(x))=1$,  则在  $\mathbb{C}[x]$  中同样有  $(p(x), f(x))=1$,  这与  $p(x)$  和  $f(x)$  有公共复根  $\alpha$  矛盾 ,  从   而在  $\mathbb{Q}[x]$  中只能是  $p(x) \mid f(x)$.

(2)  由  (1)  知 , $\exists g(x) \in \mathbb{Q}[x], f(x)=g(x) p(x)$.  因为  $f(x), g(x)$  均为首一的整系数多项式 ,  从而  $f(x), g(x)$  均为本原多项式 .  设  $g(x)=r g_{1}(x), r \in \mathbb{Q}, g_{1}(x)$  本原并且首项系数为正 .  由  $f(x)=r g_{1}(x) p(x)$  得  $r=1$,  从而  $g(x)$  本原 ,  在  $\mathbb{Z}[x]$  比较系数得  $g(x)$  的首项系数为  1 .

 注   相关题目见维声 《 高等代数 》 创新教材下册第  56  页例  11 .

 七 . (1)  若  $\alpha \in U \cap U^{\perp}$,  则  $f(\alpha, \alpha)=0$,  因为  $f$  正定 ,  得到  $\alpha=0$,  从而  $U \cap U^{\perp}=\{0\}$.

$f$  诱导了  $V$  上的一个内积  $f(\alpha, \beta)$,  可设  $\eta_{1}, \ldots, \eta_{s}$  为  $U$  的一组标准正交基 . $\forall \alpha \in V$,
$$
\alpha=\sum_{i=1}^{s} f\left(\alpha, \eta_{i}\right) \eta_{i}+\left(\alpha-\sum_{i=1}^{s} f\left(\alpha, \eta_{i}\right) \eta_{i}\right) \in U+U^{\perp}
$$
 故  $V=U \oplus U^{\perp}$.

(2) (a)  设  $\alpha_{1}, \ldots, \alpha_{s}$  为  $W$  的一组基 ,  将其扩充为  $V$  的一组基  $\alpha_{1}, \ldots, \alpha_{s}, \alpha_{s+1}, \ldots, \alpha_{n}$,  下面只需说明  $\operatorname{dim} W^{\perp}=n-s$.  设  $\alpha=x_{1} \alpha_{1}+\cdots+x_{n} \alpha_{n} \in V$,  则  $\alpha \in W^{\perp}$  等价于 
$$
\left[\begin{array}{cccc}
g\left(\alpha_{1}, \alpha_{1}\right) & g\left(\alpha_{2}, \alpha_{1}\right) & \cdots & g\left(\alpha_{n}, \alpha_{1}\right) \\
g\left(\alpha_{1}, \alpha_{2}\right) & g\left(\alpha_{2}, \alpha_{2}\right) & \cdots & g\left(\alpha_{n}, \alpha_{2}\right) \\
\vdots & \vdots & & \vdots \\
g\left(\alpha_{1}, \alpha_{s}\right) & g\left(\alpha_{2}, \alpha_{s}\right) & \cdots & g\left(\alpha_{n}, \alpha_{s}\right)
\end{array}\right]\left[\begin{array}{c}
x_{1} \\
x_{2} \\
\vdots \\
x_{n}
\end{array}\right]=\left[\begin{array}{c}
0 \\
0 \\
\vdots \\
0
\end{array}\right]
$$
 其中系数矩阵的秩为  $s$,  这是因为  $g$  非退化 ,  故 
$$
\left[\begin{array}{cccc}
g\left(\alpha_{1}, \alpha_{1}\right) & g\left(\alpha_{2}, \alpha_{1}\right) & \cdots & g\left(\alpha_{n}, \alpha_{1}\right) \\
g\left(\alpha_{1}, \alpha_{2}\right) & g\left(\alpha_{2}, \alpha_{2}\right) & \cdots & g\left(\alpha_{n}, \alpha_{2}\right) \\
\vdots & \vdots & & \vdots \\
g\left(\alpha_{1}, \alpha_{n}\right) & g\left(\alpha_{2}, \alpha_{n}\right) & \cdots & g\left(\alpha_{n}, \alpha_{n}\right)
\end{array}\right]
$$
 可逆 ,  从而矩阵的前  $s$  行线性无关 .  由此看出  $\operatorname{dim} W^{\perp}=n-s$.

(b) $W^{\perp}$  也是  $V$  的子空间 ,  从而由  $\operatorname{dim} V=\operatorname{dim} W+\operatorname{dim} W^{\perp}$  得到  $\operatorname{dim} V=\operatorname{dim} W^{\perp}+\operatorname{dim}\left(W^{\perp}\right)^{\perp}$,  因此  $\operatorname{dim} W^{\perp}=\operatorname{dim}\left(W^{\perp}\right)^{\perp}$.

 因为  $\forall \alpha \in W, g(\alpha, \beta)=g(\beta, \alpha)=0$  对任意  $\beta \in W^{\perp}$  成立 ,  从而  $\alpha \in\left(W^{\perp}\right)^{\perp}$,  故  $W^{\perp}\left(W^{\perp}\right)^{\perp}$.  把前面两个结论结合起来就得到  $W^{\perp}=\left(W^{\perp}\right)^{\perp}$.

 注   相关题目见丘维声  《 高等代数 》 创新教材下册第  437  页例  8 .  北京大学  2001  年全国硕士研究生招生考试高代解几试题及解答 

   

2019.05.26

\begin{enumerate}
  \item (15  分 )  在空间直角坐标系中 ,  点  $A, B, C$  的坐标依次为 :
\end{enumerate}
$$
(-2,1,4),(-2,-3,-4),(-1,3,3) \text {. }
$$
(1)  求四面体  $O A B C$  的体积 ;

(2)  求三角形  $A B C$  的面积 .

\begin{enumerate}
  \setcounter{enumi}{2}
  \item (15  分 )  在空间直角坐标系中 ,
\end{enumerate}
$$
\ell_{1}: \frac{x-a}{1}=\frac{y}{-2}=\frac{z}{3}
$$
 与 
$$
\ell_{2}: \frac{x}{2}=\frac{y-1}{1}=\frac{z}{-2}
$$
 是一对相交直线 .

(1)  求  $a$.

(2)  求  $\ell_{2}$  绕  $\ell_{1}$  旋转出的曲面的方程 .

\begin{enumerate}
  \setcounter{enumi}{3}
  \item (12  分 )  设  $\omega$  是复数域  $\mathbb{C}$  上的本原  $n$  次单位根  ( 即 , $\omega^{n}=1$,  而当  $0<l<n$  时 , $\left.\omega^{l} \neq 1\right), s, b$  都是正整数 ,  而且  $s<n$.  令 
\end{enumerate}
$$
A=\left(\begin{array}{ccccc}
1 & \omega^{b} & \omega^{2 b} & \ldots & \omega^{(n-1) b} \\
1 & \omega^{b+1} & \omega^{2(b+1)} & \ldots & \omega^{(n-1)(b+1)} \\
\vdots & \vdots & \vdots & \ddots & \vdots \\
1 & \omega^{b+s-1} & \omega^{2(b+s-1)} & \ldots & \omega^{(n-1)(b+s-1)}
\end{array}\right)
$$
 任取  $\beta \in \mathbb{C}^{s}$,  判断线性方程组  $A X=\beta$  有无解 ?  有多少解 ?  写出理由 .

\begin{enumerate}
  \setcounter{enumi}{4}
  \item (18  分 )
\end{enumerate}
(1)  设 
$$
A=\left(\begin{array}{ccc}
0 & 1 & 0 \\
0 & 0 & 1 \\
-2 & 3 & -1
\end{array}\right)
$$
a.  若把  $A$  看成有理数域上的矩阵 ,  判断  $A$  是否可对角化 ,  写出理由 ;

b.  若把  $A$  看成复数域上的矩阵 ,  判断  $A$  是否可对角化 ,  写出理由 .

(2)  设  $A$  是有理数域上的  $n$  级对称矩阵 ,  并且在有理数域上  $A$  合同于单位矩阵  $I$.  用  $\delta$  表示元素全为  1  的列向量 , $b$  是有理数 .  证明 :  在有理数域上 
$$
\left(\begin{array}{cc}
A & b \delta \\
b \delta^{\prime} & b
\end{array}\right) \cong\left(\begin{array}{cc}
I & 0 \\
0 & b-b^{2} \delta^{\prime} A^{-1} \delta
\end{array}\right)
$$

\begin{enumerate}
  \setcounter{enumi}{5}
  \item (14  分 )  在实数域上的  $n$  维列向量空间  $\mathbb{R}^{n}$  中 ,  定义内积为  $(\alpha, \beta)=\alpha^{\prime} \beta$,  从而  $\mathbb{R}^{n}$  成为欧几里得空间 .
\end{enumerate}
(1) 设实数域上的矩阵 
$$
A=\left(\begin{array}{cccc}
1 & -3 & 5 & -2 \\
-2 & 1 & -3 & 1 \\
-1 & -7 & 9 & -4
\end{array}\right)
$$
 求齐次线性方程组  $A X=0$  的解空间的一个正交基 .

(2) 设  $A$  是实数域  $\mathbb{R}$  上的  $s \times n$  矩阵 ,  用  $W$  表示齐次线性方程组  $A X=0$  的解空间 ,  用  $U$  表示  $A^{\prime}$  的列   空间  ( 即 , $A^{\prime}$  的列向量组生成的子空间 ).  证明  $U=W^{\perp}$.

\begin{enumerate}
  \setcounter{enumi}{6}
  \item (12  分 )  设  $\mathscr{A}$  是数域  $\mathbb{K}$  上的  $n$  维线性空间  $V$  上的一个线性变换 .  在  $\mathbb{K}[x]$  中 , $f(x)=f_{1}(x) f_{2}(x)$,  且  $f_{1}(x)$  与  $f_{2}(x)$  互素 .  用  $\operatorname{Ker} \mathscr{A}$  表示线性变换  $\mathscr{A}$  的核 .  证明 :
\end{enumerate}
$$
\operatorname{Ker} f(\mathscr{A})=\operatorname{Ker} f_{1}(\mathscr{A}) \oplus \operatorname{Ker} f_{2}(\mathscr{A})
$$

\begin{enumerate}
  \setcounter{enumi}{7}
  \item (14  分 )  设  $\mathscr{A}$  是数域  $\mathbb{K}$  上  $n$  维线性空间  $V$  上的一个线性变换 , $\mathscr{I}$  是恒等变换 .  证明 : $\mathscr{A}^{2}=\mathscr{A}$  的充分必   要条件是 
\end{enumerate}
$$
\operatorname{rank}(\mathscr{A})+\operatorname{rank}(\mathscr{A}-\mathscr{I})=n .
$$

\begin{enumerate}
  \item (1)  因为 
\end{enumerate}
$$
\frac{1}{6}\left|\begin{array}{ccc}
-2 & 1 & 4 \\
-2 & -3 & -4 \\
-1 & 3 & 3
\end{array}\right|=-\frac{16}{3}
$$
 取绝对值得体积为  $16 / 3$.

(2) $\overrightarrow{A B}=(0,-4,-8), \overrightarrow{A C}=(1,2,-1)$,  故面积为 
$$
\frac{1}{2}|\overrightarrow{A B} \times \overrightarrow{A C}|=\frac{1}{2}|(20,-8,4)|=2 \sqrt{30}
$$

\begin{enumerate}
  \setcounter{enumi}{2}
  \item (1)
\end{enumerate}
$$
\left|\begin{array}{ccc}
a & -1 & 0 \\
1 & -2 & 3 \\
2 & 1 & -2
\end{array}\right|=0 \Longrightarrow a=8
$$
(2)  设  $(x, y, z)$  是曲面上任意一点 ,  而  $\left(x_{0}, y_{0}, z_{0}\right)$  是过  $(x, y, z)$  且与  $\ell_{1}$  垂直的平面与  $\ell_{2}$  的交点 ,  则 
$$
\left\{\begin{array}{c}
\left(x-x_{0}, y-y_{0}, z-z_{0}\right) \cdot(1,-2,3)=0 \\
\frac{x_{0}}{2}=\frac{y_{0}-1}{1}=\frac{z_{0}}{-2} \\
|(x-8, y, z) \times(1,-2,3)|=\left|\left(x_{0}-8, y_{0}, z_{0}\right) \times(1,-2,3)\right|
\end{array}\right.
$$
 整理前两式可得 
$$
x_{0}=\frac{x-2 y+3 z+2}{3}
$$
 代入第三式可得 
$$
(3 y+2 z)^{2}+(3 x-24-z)^{2}+(2 x-16+y)^{2}=\frac{5}{2}(x+y+3 z-16)^{2}
$$

\begin{enumerate}
  \setcounter{enumi}{3}
  \item  考虑  $A$  的前  $s$  行 ,  前  $s$  列组成的矩阵  $B$,  则由于  $\omega^{b}, \omega^{b+1}, \ldots, \omega^{b+s-1}$  是互不相同的 ,  根据  Vandermonde  行   列式就只知道  $|B| \neq 0$,  从而  $\operatorname{rank}(A)=s, \operatorname{rank}(A \beta)=s$,  方程  $A X=\beta$  必有解 ,  再根据  $s<n$  知解的个数   为无穷个 .

  \item (1)  特征多项式  $f(\lambda)=|\lambda E-A|=\lambda^{3}+\lambda^{2}-3 \lambda+2$.

\end{enumerate}
a.  由于  $f(\pm 1) \neq 0, f(\pm 2) \neq 0$,  从而  $f(\lambda)$  没有有理根 ,  故  $A$  没有有理特征值 ,  从而不能在有理数域   上对角化 .

b. $\left(f(\lambda), f^{\prime}(\lambda)=1\right.$,  从而  $f(\lambda)=0$  没有重根 ,  即  $f(\lambda)$  在  $\mathbb{C}[\lambda]$  中可分解为三个互素一次因式的乘积 ,  于是  $A$  在复数域上可对角化 .

(2) 设  $C^{\mathrm{T}} A C=I$,  令 
$$
B=\left[\begin{array}{cc}
C & -b A^{-1} \delta \\
0 & 1
\end{array}\right]
$$
 贝䋆 
$$
B^{\mathrm{T}}\left[\begin{array}{cc}
A & b \delta \\
b \delta^{\prime} & b
\end{array}\right] B=\left[\begin{array}{cc}
I & 0 \\
0 & b-b^{2} \delta^{\prime} A^{-1} \delta
\end{array}\right]
$$
 然后只需注意到  $B \in G L_{n}(\mathbb{Q})$. 5. (1)  做初等变换 
$$
\left[\begin{array}{cccc}
1 & -3 & 5 & -2 \\
-2 & 1 & -3 & 1 \\
-1 & -7 & 9 & -4
\end{array}\right] \rightarrow\left[\begin{array}{cccc}
1 & -3 & 5 & -2 \\
0 & -5 & 7 & -3 \\
0 & -10 & 14 & -6
\end{array}\right] \rightarrow\left[\begin{array}{cccc}
1 & -3 & 5 & -2 \\
0 & -5 & 7 & -3 \\
0 & 0 & 0 & 0
\end{array}\right]
$$
 可以看出 
$$
\eta_{1}=\left[\begin{array}{c}
-4 \\
7 \\
5 \\
0
\end{array}\right], \quad \eta_{2}=\left[\begin{array}{c}
1 \\
-3 \\
0 \\
5
\end{array}\right]
$$
 是  $A X=0$  的解空间中的线性无关向量 ,  注意到解空间的维数是  2 ,  从而  $\eta_{1}, \eta_{2}$  是解空间的一组基 .  进   行如下正交化 
$$
\xi_{1}=\eta_{1}, \quad \xi_{2}=\eta_{2}-\frac{\left(\eta_{2}, \xi_{1}\right)}{\left(\xi_{1}, \xi_{1}\right)} \xi_{1}=\frac{1}{18}\left[\begin{array}{c}
-2 \\
-19 \\
25 \\
90
\end{array}\right]
$$
$\xi_{1}, \xi_{2}$  即是解空间的正交基 .

(2)  设  $A^{\mathrm{T}}=\left(\alpha_{1}, \ldots, \alpha_{s}\right)$,  则  $U=L\left(\alpha_{1}, \ldots, \alpha_{s}\right) . \forall X_{0} \in W, A X_{0}=0, X_{0}^{\mathrm{T}} A^{\mathrm{T}}=0$,  即 
$$
\left(X_{0}^{\mathrm{T}} \alpha_{1}, \ldots, X_{0}^{\mathrm{T}} \alpha_{s}\right)=(0, \ldots, 0)
$$
 从而  $\left(X_{0}, \alpha_{i}\right)=0, i=1,2, \ldots, s$,  于是  $\alpha_{i} \in W^{\perp}, i=1,2, \ldots, s$.  故  $U \subset W^{\perp}$,  再注意到 
$$
\operatorname{dim} W=n-\operatorname{rank}(A), \quad \operatorname{dim} U=\operatorname{rank}(A), \quad \operatorname{dim} W+\operatorname{dim} W^{\perp}=n,
$$
 于是  $\operatorname{dim} U=\operatorname{dim} W^{\perp}$,  从而  $U=W^{\perp}$.

\begin{enumerate}
  \setcounter{enumi}{6}
  \item  设  $V_{0}=\operatorname{ker} f(\mathscr{A}), V_{1}=\operatorname{ker} f_{1}(\mathscr{A}), V_{2}=\operatorname{ker} f_{2}(\mathscr{A})$.  容易看出  $V_{1} \subset V_{0}, V_{2} \subset V_{0}$.  由于  $f_{1}(x)$  与  $f_{2}(x)$  互   素 ,  于是存在  $u(x), v(x) \in K[x]$,  使得 
\end{enumerate}
$$
u(x) f_{1}(x)+v(x) f_{2}(x)=1, \quad u(\mathscr{A}) f_{1}(\mathscr{A})+v(\mathscr{A}) f_{2}(\mathscr{A})=\mathscr{E},
$$
$\forall \alpha \in V_{0}$,
$$
\alpha=u(\mathscr{A}) f_{1}(\mathscr{A}) \alpha+v(\mathscr{A}) f_{2}(\mathscr{A}) \alpha \in V_{2}+V_{1}
$$
 从而  $V_{0}=V_{1}+V_{2}$.

$\forall \gamma \in V_{1} \cap V_{2}$,  则 
$$
\gamma=u(\mathscr{A}) f_{1}(\mathscr{A}) \gamma+v(\mathscr{A}) f_{2}(\mathscr{A}) \gamma=0
$$
 从而  $V_{1} \cap V_{2}=\{0\}, V_{0}=V_{1} \oplus V_{2}$.

\begin{enumerate}
  \setcounter{enumi}{7}
  \item  令  $f_{1}(x)=x, f_{2}(x)=x-1, f(x)=f_{1}(x) f_{2}(x)=x^{2}-x$,  则  $\left(f_{1}(x), f_{2}(x)\right)=1, \operatorname{ker} f(\mathscr{A})=\operatorname{ker} f_{1}(\mathscr{A}) \oplus$ $\operatorname{ker} f_{2}(\mathscr{A})$.  从而 
\end{enumerate}
$$
\begin{aligned}
\mathscr{A}^{2}=\mathscr{A} & \Longleftrightarrow f(\mathscr{A})=0 \Longleftrightarrow \operatorname{ker} f(\mathscr{A})=V \\
& \Longleftrightarrow \operatorname{ker} f_{1}(\mathscr{A}) \oplus \operatorname{ker} f_{2}(\mathscr{A})=V \\
& \Longleftrightarrow \operatorname{dim} \operatorname{ker} f_{1}(\mathscr{A})+\operatorname{dim} \operatorname{ker} f_{2}(\mathscr{A})=n \\
& \Longleftrightarrow\left[n-\operatorname{rank}\left(f_{1}(\mathscr{A})\right)\right]+\left[n-\operatorname{rank}\left(f_{2}(\mathscr{A})\right)\right]=n \\
& \Longleftrightarrow \operatorname{rank}(\mathscr{A})+\operatorname{rank}(\mathscr{A}-\mathscr{E})=n .
\end{aligned}
$$
 注   也可参考丘维声的 《 高等代数 》 创新教材上册第  199  页例  3 .  北京大学  2002  年全国硕士研究生招生考试高代解几试题及解答 

   

2019.05.26

\begin{enumerate}
  \item (18  分 )  在空间直角坐标系中 ,  直线  $\ell_{1}$  和  $\ell_{2}$  分别有方程 
\end{enumerate}
$$
\left\{\begin{array}{r}
x+y+z-1=0 \\
x+y+2 z+1=0
\end{array},\left\{\begin{array}{l}
3 x+y+1=0 \\
x+3 z+2=0
\end{array}\right. \text {. }\right.
$$
(1)  求过  $\ell_{1}$  平行于  $\ell_{2}$  的平面的方程 .

(2)  求  $\ell_{1}$  和  $\ell_{2}$  的距离 .

(3)  求  $\ell_{1}$  和  $\ell_{2}$  的公垂线的方程 .

\begin{enumerate}
  \setcounter{enumi}{2}
  \item (12  分 )  在空间直角坐标系中 ,  求直线 
\end{enumerate}
$$
\left\{\begin{array}{l}
z=3 x+2 \\
z=2 y-1
\end{array}\right.
$$
 绕  $z$  轴旋转所得旋转曲面的方程 .

\begin{enumerate}
  \setcounter{enumi}{3}
  \item (15  分 )  用正交变换化下面的二次型为标准形 
\end{enumerate}
$$
f\left(x_{1}, x_{2}, x_{3}\right)=x_{1}^{2}+x_{2}^{2}+x_{3}^{2}-4 x_{1} x_{2}-4 x_{1} x_{3}-4 x_{2} x_{3} .
$$
( 要求写出正交变换的矩阵和相应的标准形 ).

\begin{enumerate}
  \setcounter{enumi}{4}
  \item (12  分 )  对于任意非负整数  $n$,  令  $f_{n}(x)=x^{n+2}-(x+1)^{2 n+1}$,  证明 :
\end{enumerate}
$$
\left(x^{2}+x+1, f_{n}(x)\right)=1 \text {. }
$$

\begin{enumerate}
  \setcounter{enumi}{5}
  \item (18  分 )  设正整数  $n \geqslant 2$,  用  $M_{n}(\mathbb{K})$  表示数域  $\mathbb{K}$  上全体  $n \times n$  矩阵关于矩阵加法和数乘所构成的  $\mathbb{K}$  上的线   性空间 . 在  $M_{n}(\mathbb{K})$  中定义变换  $\sigma$  如下 :
\end{enumerate}
$$
\sigma\left(\left(a_{i j}\right)_{n \times n}\right)=\left(a_{i j}^{\prime}\right)_{n \times n}, \quad \forall\left(a_{i j}\right)_{n \times n} \in M_{n}(\mathbb{K}),
$$
 其中 
$$
a_{i j}^{\prime}= \begin{cases}a_{i j}, & i \neq j \\ i \cdot \operatorname{tr}(A), & i=j\end{cases}
$$
(1)  证明  $\sigma$  是  $M_{n}(\mathbb{K})$  上的线性变换 .

(2)  求出  $\operatorname{ker}(\sigma)$  的维数与一组基 .

(3) 求出  $\sigma$  的全部特征子空间 . 6. (12  分 )  用  $\mathbb{R}$  表示实数域 ,  定义  $\mathbb{R}^{n}$  到  $\mathbb{R}$  的映射  $f$  如下 :
$$
f(X)=\left|x_{1}\right|+\cdots+\left|x_{r}\right|-\left|x_{r+1}\right|-\cdots-\left|x_{r+s}\right|, \quad \forall X=\left(x_{1}, x_{2}, \cdots, x_{n}\right)^{\mathrm{T}} \in \mathbb{R}^{n}
$$
 其中  $r \geqslant s \geqslant 0$.  证明 :

(1)  存在  $\mathbb{R}^{n}$  的一个  $n-r$  维子空间  $W$,  使得  $f(X)=0 \quad \forall X \in W$.

(2)  若  $W_{1}, W_{2}$  是  $\mathbb{R}^{n}$  的两个  $n-r$  维子空间 ,  且满足 
$$
f(X)=0, \quad \forall X \in W_{1} \cup W_{2}
$$
 则一定有  $\operatorname{dim}\left(W_{1} \cap W_{2}\right) \geqslant n-(r+s)$.

\begin{enumerate}
  \setcounter{enumi}{7}
  \item (13  分 )  设  $V$  是数域  $\mathbb{K}$  上  $n$  维线性空间 , $V_{1}, \ldots, V_{s}$  是  $V$  的  $s$  个真子空间 ,  证明 :
\end{enumerate}
(1)  存在  $\alpha \in V$,  使得  $\alpha \notin V_{1} \cup V_{1} \cup V_{2} \cup \ldots \cup V_{s}$.

(2)  存在  $V$  中的一组基  $\varepsilon_{1}, \varepsilon_{2}, \cdots, \varepsilon_{n}$,  使得 
$$
\left\{\varepsilon_{1}, \varepsilon_{2}, \cdots, \varepsilon_{n}\right\} \cap\left(V_{1} \cup V_{1} \cup V_{2} \cup \cdots \cup V_{s}\right)=\varnothing
$$

\begin{enumerate}
  \item (1)  设过  $\ell_{1}$  平行于  $\ell_{2}$  的平面的方程为  $\lambda(x+y+z-1)+\mu(x+y+2 z+1)=0$,  则  $(\lambda+\mu, \lambda+\mu, \lambda+2 \mu)$  为平面法向量 . $\ell_{2}$  的一个方向向量为  $(3,1,0) \times(1,0,3)=(3,-9,-1)$.  由于  $3(\lambda+\mu)-9(\lambda+\mu)-$ $(\lambda+2 \mu)=0$,  故  $7 \lambda+8 \mu=0$.  因此方程为  $x+y-6 z-15=0$.
\end{enumerate}
(2) $\ell_{1}$  的一个方向向量为  $(1,1,1) \times(1,1,2)=(1,-1,0) . \ell_{1}$  与  $\ell_{2}$  的公垂线的一个方向向量为  $\vec{n}=$ $(1,-1,0) \times(3,-9,-1)=(-1,-1,6) . \ell_{1}$  过点  $P_{1}(1,2,-2), \ell_{2}$  过点  $P_{2}(-2,5,0), \overrightarrow{P_{1} P_{2}}=(-3,3,2)$. $\ell_{1}$  和  $\ell_{2}$  的距离 
$$
d=\frac{\left|\vec{n} \cdot \overrightarrow{P_{1} P_{2}}\right|}{\left|\overrightarrow{P_{1} P_{2}}\right|}=\frac{|3-3+12|}{\sqrt{1+1+36}}=\frac{12}{\sqrt{38}}
$$
$$
\left|\begin{array}{ccc}
x-1 & y-2 & z+2 \\
1 & -1 & 0 \\
-1 & -1 & 6
\end{array}\right|=0, \quad\left|\begin{array}{ccc}
x+2 & y-5 & z \\
3 & -9 & -1 \\
-1 & -1 & 6
\end{array}\right|=0
$$
 故  $\ell_{1}$  和  $\ell_{2}$  的公垂线的方程为 
$$
\left\{\begin{array}{r}
3 x+3 y+z-7=0 \\
55 x+17 y-12 z+25=0
\end{array}\right.
$$

\begin{enumerate}
  \setcounter{enumi}{2}
  \item  设  $P(x, y, z)$  为曲面上一点而  $\left(x_{0}, y_{0}, z_{0}\right)$  为过  $P$  点且垂直于  $z$  轴的平面与原直线的交点 ,  则 
\end{enumerate}
$$
\left\{\begin{aligned}
x^{2}+y^{2} &=x_{0}^{2}+y_{0}^{2} \\
z &=z_{0} \\
z_{0} &=3 x_{0}+2 \\
z_{0} &=2 y_{0}-1
\end{aligned}\right.
$$
 消去其中的  $x_{0}, y_{0}, z_{0}$  得方程 
$$
x^{2}+y^{2}=\left(\frac{z-2}{3}\right)^{2}+\left(\frac{z+1}{2}\right)^{2}
$$

\begin{enumerate}
  \setcounter{enumi}{3}
  \item  对应的矩阵为 
\end{enumerate}
$$
A=\left[\begin{array}{rrr}
1 & -2 & -2 \\
-2 & 1 & -2 \\
-2 & -2 & 1
\end{array}\right]
$$
$|\lambda E-A|=(\lambda+3)(\lambda-3)^{2} . \lambda=-3$  的一个特征向量为  $[1,1,1]^{\mathrm{T}}, \lambda=3$  的两个特征向量为  $[1,-1,0]^{\mathrm{T}},[1,1,-2]^{\mathrm{T}}$. 숫
$$
\left[\begin{array}{l}
y_{1} \\
y_{2} \\
y_{3}
\end{array}\right]=\left[\begin{array}{ccc}
\frac{1}{\sqrt{3}} & \frac{1}{\sqrt{2}} & \frac{1}{\sqrt{6}} \\
\frac{1}{\sqrt{3}} & \frac{-1}{\sqrt{2}} & \frac{1}{\sqrt{6}} \\
\frac{1}{\sqrt{3}} & 0 & \frac{-2}{\sqrt{6}}
\end{array}\right]
$$
 于是标准型为  $f\left(x_{1}, x_{2}, x_{3}\right)=-3 y_{2}^{2}+3 y_{2}^{2}+3 y_{3}^{2}$.

\begin{enumerate}
  \setcounter{enumi}{4}
  \item  设  $\alpha$  为  $x^{2}+x+1=0$  的一个复根 ,  则  $f_{n}(\alpha)=\alpha^{n+2}-\left(-\alpha^{2}\right)^{2 n+1}=\alpha^{n+2}+\alpha^{4 n+2}=\alpha^{n+2}\left(1+\alpha^{3 n}\right)$.  显然  $\alpha \neq 0$,  且由  $\alpha^{3}-1=(\alpha-1)\left(\alpha^{2}+\alpha+1\right)=0$  知  $\alpha^{3}=1$,  故  $f_{n}(\alpha)=2 \alpha^{n+2} \neq 0$,  从而  $\left(x^{2}+x+1, f_{n}(x)\right)=1$.

  \item (1)  因为  $\sigma(A)=A+\operatorname{tr}(A) \operatorname{diag}\{1,2, \ldots, n\}-E_{11} A E_{11}-E_{22} A E_{22}-\cdots-E_{n n} A E_{n n}$,  由矩阵乘法的性质   知  $\left.\sigma\left(k_{1} A+k_{2} B\right)=k_{1} \sigma(A)+k_{2} \sigma(B), \forall A, B \in M_{(} K\right), k_{1}, k_{2} \in K$. (2) $A \in \operatorname{ker} \sigma \Longleftrightarrow \operatorname{tr}(A)=0$  且  $a_{i j}=0, \forall i \neq j$.  于是  $\operatorname{ker} \sigma$  的一组基为  $E_{11}-E_{22}, E_{11}-E_{33}, \ldots, E_{11}-$ $E_{n n}, \operatorname{dim}$ ker $\sigma=n-1$.

\end{enumerate}
(3) $\operatorname{ker} \sigma$  为特征值为  0  对应的特征子空间 .

 若  $\sigma M=\lambda M$  且  $\lambda \neq 0$,  则 
$$
\begin{cases}a_{i j}=\lambda a_{i j}, & i \neq j \\ i \cdot \operatorname{tr}(A)=\lambda a_{i i}, & i=1,2, \ldots, n\end{cases}
$$
 于是  $n(n+1) / 2 \operatorname{tr}(A)=\lambda \operatorname{tr}(A)$.

\begin{itemize}
  \item  若  $\lambda=n(n+1) / 2$,  则  $\forall i \neq j, a_{i j}=0$,  于是  $A=k \operatorname{diag}\{1,2, \ldots, n\}, k \in \mathbb{K}$,  因此  $\operatorname{diag}\{1,2, \ldots, n\}$  生成  $\sigma$  的特征值  $n(n+1) / 2$  对应的特征子空间 .

  \item  若  $\lambda \neq n(n+1) / 2$,  则  $\operatorname{tr}(A)=0$,  于是  $a_{i i}=0, i=1,2, \ldots, n$.  再由  $M \neq 0$  可知  $\lambda=1$.  于是  $\sigma$  的   特征值  $n(n+1) / 2$  对应的特征子空间为 

\end{itemize}
$$
\left\{\left[\begin{array}{cccc}
0 & a_{12} & \ldots & a_{n n} \\
a_{21} & 0 & \ldots & a_{2 n} \\
\vdots & \vdots & & \vdots \\
a_{n 1} & a_{n 2} & \cdots & 0
\end{array}\right] \mid a_{i j} \in K, i \neq j, i, j \in\{1,2, \ldots, n\}\right\}
$$

\begin{enumerate}
  \setcounter{enumi}{6}
  \item (1)  由题意 , $r+s \leqslant n$.  令  $\xi_{1}=e_{1}+e_{r+1}, \xi_{2}=e_{2}+e_{r+2}, \ldots, \xi_{s}=e_{s}+e_{r+s}$.
\end{enumerate}
\begin{itemize}
  \item  若  $r+s=n$,  取  $W=L\left(\xi_{1}, \xi_{2}, \ldots, \xi_{s}\right)$  即可 .

  \item  若  $r+s<n$,  取  $W=L\left(\xi_{1}, \xi_{2}, \ldots, \xi_{s}, e_{r+s+1}, e_{r+s+2}, \ldots, e_{n}\right)$  即可 .

\end{itemize}
 总之 ,  前  $r+s$  个分量贡献  $s$  维 ,  后  $n-(r+s)$  个分量贡献  $n-(r+s)$  维 ,  加起来就是  $n-r$  维 .

(2)  设  $W$  是  $\mathbb{R}$  的任意一个  $n-r$  维子空间 ,  若能证明此时必有  $W \supset L\left(e_{r+s+1}, e_{r+s+2}, \ldots, e_{n}\right)$,  则  $W_{1} \cap$ $W_{2} \supset L\left(e_{r+s+1}, e_{r+s+2}, \ldots, e_{n}\right)$,  从而  $\operatorname{dim}\left(W_{1} \cap W_{2}\right) \geqslant \operatorname{dim} L\left(e_{r+s+1}, e_{r+s+2}, \ldots, e_{n}\right)=n-(r+s)$.

 下面用反证法来证明上面假设的结论的为真 .  假设存在  $W \nsupseteq L\left(e_{r+s+1}, e_{r+s+2}, \ldots, e_{n}\right)$,  并且 
$$
X_{i}=\left[a_{1}^{(i)}, a_{2}^{(i)}, \ldots, a_{n}^{(i)}\right]^{\mathrm{T}}, i=1,2, \ldots, m
$$
 为  $W$  的一组基 .  考虑对如下矩阵 
$$
A=\left[\begin{array}{cccccccccc}
a_{r+1}^{(1)} & \ldots & a_{r+s}^{(1)} & a_{1}^{(1)} & a_{2}^{(1)} & \ldots & a_{r}^{(1)} & a_{r+s+1}^{(1)} & \ldots & a_{n}^{(1)} \\
a_{r+1}^{(2)} & \ldots & a_{r+s}^{(2)} & a_{1}^{(2)} & a_{2}^{(2)} & \ldots & a_{r}^{(2)} & a_{r+s+1}^{(2)} & \ldots & a_{n}^{(2)} \\
\vdots & & \vdots & \vdots & \vdots & & \vdots & \vdots & & \vdots \\
a_{r+1}^{(m)} & \ldots & a_{r+s}^{(m)} & a_{1}^{(m)} & a_{2}^{(m)} & \ldots & a_{r}^{(m)} & a_{r+s+1}^{(m)} & \ldots & a_{n}^{(m)}
\end{array}\right]
$$
 做初等行变换化为阶梯形矩阵  $\tilde{A}$,  然后再把所得矩阵的做初等列交换把  $a_{j}^{(i)}$  换到第  $j$  列 ,  那么最后所   得矩阵的行向量取转置后仍为  $W$  的一组基 .  注意到若  $X \in W$, 则  $X$  的前  $r$  个分量均为零当且仅当  $X$  的第  $r+1$  到  $r+s$  个分量均为零 .  根据前面的假定有  $\operatorname{dim} W=m$,  这个  $m$  是  $\tilde{A}$  的行数 .

 观察阶梯形矩阵  $\tilde{A}$  的形状来估计  $W$  的维数 .  对于  $\tilde{A}$  的一个行向量 ,  满足前  $s$  个分量不同时为零   的至多只会有  $s$  个 ,  这里最多为总共的  $m$  行贡献  $s$  行 .  对于  $\tilde{A}$  的一个行向量 ,  若它的前  $s$  个分   量同时为零 ,  则前  $r+s$  个分量均为零 ,  这样的行向量不会超过  $n-(r+s)$  个 ;  更进一步地 ,  由于  $W \nsupseteq L\left(e_{r+s+1}, e_{r+s+2}, \ldots, e_{n}\right)$,  这样的行向量不会超过  $n-(r+s)-1$  个 .  综合上面的讨论就有  $m \leqslant s+(n-(r+s)-1)=n-r-1<n-r$,  矛盾 . 7.  直接证明  (2),  然后  (1)  自然成立 .

 设  $\eta_{1}, \eta_{2}, \ldots, \eta_{n}$  是  $V$  的一组基 , $\lambda_{1}, \lambda_{2}, \ldots, \lambda_{n}$  是  $K$  中互不相同的  $n$  个数 ,  令 
$$
\left(\xi_{1}, \xi_{2}, \ldots, \xi_{n}\right)=\left(\eta_{1}, \eta_{2}, \ldots, \eta_{n}\right)\left[\begin{array}{cccc}
1 & 1 & \ldots & 1 \\
\lambda_{1} & \lambda_{2} & \ldots & \lambda_{n} \\
\vdots & \vdots & & \vdots \\
\lambda_{1}^{n-1} & \lambda_{2}^{n-1} & \ldots & \lambda_{n}^{n-1}
\end{array}\right],
$$
 则  $\xi_{1}, \xi_{2}, \ldots, \xi_{n}$  是线性无关的 ,  由于  $V_{i}$  是真子空间 ,  故  $V_{i}$  中至多包含  $\operatorname{dim} V_{i}$  个这样的  $\xi_{1}, \xi_{2}, \ldots, \xi_{n}$.  由于  $K$  是数域 ,  其中元素有无穷个 ,  从而我们可以选出  $K$  中  $n$  个不同的数 ,  使得由这  $n$  个数对应的  $V$  中元素   线性无关 ,  并且均不在  $V_{1} \cup V_{1} \cup V_{2} \cup \cdots \cup V_{s}$  中 .  北京大学  2005  年全国硕士研究生招生考试高代解几试题及解答 

   

2019.05.26

\begin{enumerate}
  \item  在直角坐标系中 ,  求直线  $\ell:\left\{\begin{array}{l}2 x+y-z=0 \\ x+y+2 z=1\end{array}\right.$  到平面  $\pi: 3 x+B y+z=0$  的正交投影轨迹的方程 ,  其中  $B$  是常数 .

  \item  在直角坐标系中对于参数  $\lambda$  的不同取值 ,  判断平面二次曲线  $x^{2}+y^{2}+2 \lambda x y+\lambda=0$  的形状 ,  并且对于中心   型曲线 ,  写出对称中心的坐标 ;  对于线心型曲线 ,  写出对称直线的方程 .

  \item  设数域  $\mathbb{K}$  上的  $n$  阶矩阵  $A$  的  $(i, j)$  元为  $a_{i}-b_{j}$.

\end{enumerate}
(1)  求  $|A|$.

(2)  当  $n \geqslant 2$  时 , $a_{1} \neq a_{2}, b_{1} \neq b_{2}$.  求齐次线性方程组  $A X=0$  的解空间的维数和一组基 .

\begin{enumerate}
  \setcounter{enumi}{4}
  \item (1)  设  $C=E_{12}+E_{23}+\cdots+E_{(n-1) n}+E_{n 1}$  是数域  $\mathbb{K}$  上的  $n$  阶矩阵 ,  对任意正整数  $m$,  求  $C^{m}$;
\end{enumerate}
(2)  用  $M_{n}(\mathbb{K})$  表示数域  $\mathbb{K}$  上所有  $n$  阶矩阵组成的集合 ,  它对于矩阵的加法和数量乘法成为  $\mathbb{K}$  上的线性   空间 .  称数域  $\mathbb{K}$  上  $n$  阶矩阵 
$$
A=\left(\begin{array}{ccccc}
a_{1} & a_{2} & a_{3} & \cdots & a_{n} \\
a_{n} & a_{1} & a_{2} & \cdots & a_{n-1} \\
\vdots & \vdots & \vdots & \ddots & \vdots \\
a_{2} & a_{3} & a_{4} & \cdots & a_{1}
\end{array}\right)
$$
 为循环矩阵 .  用  $U$  表示数域  $\mathbb{K}$  上所有  $n$  阶循环矩阵组成的集合 .  证明  $U$  是  $M_{n}(\mathbb{K})$  的一个子空间 ,  并   求  $U$  的一组基和维数 .

\begin{enumerate}
  \setcounter{enumi}{5}
  \item (1)  设实数域  $\mathbb{R}$  上  $n$  阶矩阵  $H$  的  $(i, j)$  元为  $\frac{1}{i+j-1}$.  任取  $\alpha, \beta \in \mathbb{R}^{n}$,  令  $f(\alpha, \beta)=\alpha^{\prime} H \beta$.  试问  $f$  是不   是  $\mathbb{R}^{n}$  上的一个内积 ,  写出理由 .
\end{enumerate}
(2)  设  $A$  是  $n$  阶正定矩阵 , $\alpha \in \mathbb{R}^{n}$,  且  $\alpha$  是非零列向量 .  令  $B=A \alpha \alpha^{\prime}$,  求  $B$  的最大特征值以及  $B$  的属于   这个特征值的特征子空间的维数和一个基 .

\begin{enumerate}
  \setcounter{enumi}{6}
  \item  设  $\mathscr{A}$  是数域  $\mathbb{R}$  上  $n$  维线性空间  $V$  上的一个线性变换 ,  用  $\mathscr{E}$  表示  $V$  上的恒等变换 ,  证明 :
\end{enumerate}
$$
\mathscr{A}^{3}=\mathscr{E} \Longleftrightarrow \operatorname{rank}(\mathscr{E}-\mathscr{A})+\operatorname{rank}\left(\mathscr{E}+\mathscr{A}+\mathscr{A}^{2}\right)=n .
$$

\begin{enumerate}
  \item  设直线  $\ell$  与其投影所决定的平面方程为  $\lambda(2 x+y-z)+\mu(x+y+2 z-1)=0$,  则  $(2 \lambda+\mu, \lambda+\mu, 2 \mu-\lambda)$. $(3, B, 1)=0$,  整理后得到  $(5+B)(\lambda+\mu)=0$.
\end{enumerate}
\begin{itemize}
  \item  若  $B \neq-5$,  则  $\lambda=-\mu$,  投影直线的方程为 
\end{itemize}
$$
\left\{\begin{array}{l}
x-3 z+1=0 \\
3 x+B y+z=0
\end{array}\right.
$$

\begin{itemize}
  \item  若  $B=-5$,  则直线  $\ell$  垂直于平面  $\pi$,  投影为一个点  $(4 / 35,1 / 7,13 / 35)$.
\end{itemize}
\begin{enumerate}
  \setcounter{enumi}{2}
  \item $I_{1}=2, I_{2}=1-\lambda^{2}, I_{3}=\lambda\left(1-\lambda^{2}\right)$.
\end{enumerate}
\begin{itemize}
  \item  当  $\lambda<-1$  时 ,  形状为双曲线 , $(0,0)$  为对称中心 .

  \item  当  $\lambda=-1$  时 ,  方程可变为  $2 y^{* 2}-1=0$,  形状为两条平行直线 , $y^{*}=0$,  即  $y=x$  为对称直线 .

  \item  当  $-1<\lambda<0$  时 ,  形状为椭圆 , $(0,0)$  为对称中心 .

  \item  当  $\lambda=0$  时 ,  形状为一个点  $(0,0)$.

  \item  当  $0<\lambda<1$  时 ,  形状为虚椭圆 , $(0,0)$  为对称中心 .

  \item  当  $\lambda=1$  时 ,  形状为一对虚平行直线 , $y=-x$  为对称直线 .

  \item  当  $\lambda>1$  时 ,  形状为双曲线 , $(0,0)$  为对称中心 .

\end{itemize}
\begin{enumerate}
  \setcounter{enumi}{3}
  \item (1)  当  $n=1$  时 , $|A|=a_{1}-b_{1}$.  当  $n \geqslant 2$  时 ,  因为 
\end{enumerate}
$$
A=\left[\begin{array}{cc}
a_{1} & 1 \\
a_{2} & 1 \\
\vdots & \vdots \\
a_{n} & 1
\end{array}\right]\left[\begin{array}{cccc}
1 & 1 & \cdots & 1 \\
-b_{1} & -b_{2} & \ldots & -b_{n}
\end{array}\right]
$$
 故当  $n=2$  时 , $|A|=\left(a_{1}-a_{2}\right)\left(b_{1}-b_{2}\right)$;  当  $n>2$  时 , $\operatorname{rank}(A) \leqslant 2<n$,  从而  $|A|=0$.

(2)  二阶主子式不为  0 ,  从而可以知道  $\operatorname{rank}(A)=2$,  解空间的维数为  $n-2$.
$$
\left[\begin{array}{cccc}
1 & 1 & \ldots & 1 \\
-b_{1} & -b_{2} & \ldots & -b_{n}
\end{array}\right] \rightarrow\left[\begin{array}{ccccc}
1 & 1 & \ldots & 1 \\
0 & b_{1}-b_{2} & \ldots & b_{1}-b_{n}
\end{array}\right] \rightarrow\left[\begin{array}{ccccc}
b_{1}-b_{2} & 0 & b_{3}-b_{2} & \ldots & b_{n}-b_{2} \\
0 & b_{1}-b_{2} & b_{1}-b_{3} & \cdots & b_{1}-b_{n}
\end{array}\right]
$$
 因此 
$$
\left[\begin{array}{c}
b_{2}-b_{n} \\
b_{n}-b_{1} \\
0 \\
\vdots \\
0 \\
b_{1}-b_{2}
\end{array}\right],\left[\begin{array}{c}
b_{2}-b_{n-1} \\
b_{n-1}-b_{1} \\
0 \\
\vdots \\
b_{1}-b_{2} \\
0
\end{array}\right], \cdots,\left[\begin{array}{c}
b_{2}-b_{3} \\
b_{3}-b_{1} \\
b_{1}-b_{2} \\
0 \\
\vdots \\
0
\end{array}\right]
$$
 为  $A X=0$  的  $n-2$  个解 ,  又由于它们是线性无关的 ,  故它们就是解空间的一组基 .

 注   相关题目见丘维声 《 高等代数 》 创新教材上册第  176  页例  12 .  将第一列看作两个列向量相减 ,  把  $|A|$  拆成   两个行列式的差也可以算出行列式  $|A|$.  通过行变换的方法也能求出解空间 .

\begin{enumerate}
  \setcounter{enumi}{4}
  \item (1)  回忆版的试题没有给出  $C$  具体代表什么 ,  这里的  $C$  是我根据第二问自己编的 ,  它其实就是把下面那个   矩阵  $A$  中的  $a_{2}$  换成  1 ,  其余元素均换成  0  而得的矩阵 . (2)  因为  $U \neq 0$,  并且其中元素关于矩阵的加法与数乘封闭 ,  故  $U$  是  $\left.M_{(} K\right)$  的子空间 . $A=a_{1} E+a_{2} C+$ $a_{3} C^{2}+\cdots+a_{n} C^{n-1}$,  由此易得  $E, C, C^{2}, \ldots, C^{n-1}$  为  $U$  的一组基 , $\operatorname{dim} U=n$.
\end{enumerate}
 注   计算循环矩阵的行列式是很常见的一个题目 ,  具体的计算方法可以参考蓝以中的 《 高等代数学习指南 》 第  115  页例  $2.3$  或者丘维声  《 高等代数 》 创新教材上册第  173  页例  7 .

\begin{enumerate}
  \setcounter{enumi}{5}
  \item (1) $\forall \alpha=\left[a_{1}, \ldots, a_{n}\right]^{\mathrm{T}} \in \mathbb{R}^{n}$,
\end{enumerate}
$$
\begin{aligned}
\alpha^{\mathrm{T}} H \alpha &=\sum_{i=1}^{n} \sum_{j=1}^{n} \frac{a_{i} a_{j}}{i+j-1}=\sum_{i=1}^{n} \sum_{j=1}^{n} \int_{0}^{1} a_{i} a_{j} x^{i+j-2} \mathrm{~d} x=\int_{0}^{1} \sum_{i=1}^{n} \sum_{j=1}^{n} a_{i} a_{j} x^{i+j-2} \mathrm{~d} x \\
&=\int_{0}^{1}\left(\sum_{i=1}^{n} a_{i} x^{i-1}\right)^{2} \mathrm{~d} x \geqslant 0 .
\end{aligned}
$$
 并且  $\alpha^{\mathrm{T}} H \alpha=0 \Longleftrightarrow a_{1}+a_{2} x+\cdots+a_{n} x^{n-1}=0 \Longleftrightarrow a_{1}=a_{2}=\cdots=a_{n}=0 \Longleftrightarrow \alpha=0$.  再结合  $H^{\mathrm{T}}=H$,  就得到  $H$  为正定矩阵 ,  从而  $f$  是内积 .

(2)  由于  $\operatorname{rank}(B) \leqslant 1, B \neq 0$,  故  $\operatorname{rank}(B)=1$,  从而  $B$  只有一个非零特征值 ,  并且他对应的特征子空间的   维数为  1.  注意到  $B(A \alpha)=A \alpha \alpha^{\prime}(A \alpha)=A \alpha\left(\alpha^{\prime} A \alpha\right)=\alpha^{\prime} A \alpha(A \alpha), A \alpha \neq 0, \alpha^{\prime} A \alpha>0$,  从而  $\alpha^{\prime} A \alpha$  为  $B$  的最大特征值 , $A \alpha$  为对应的特征子空间的一组基 .

 注   第一问其实考察的  Hilbert  矩阵 ,  可以参考左维声的 《 高等代数 》 创新教材下册第  459  页例  $5 .$  第二问中的   非零特征值也可借助于矩阵的迹算出来 .

\begin{enumerate}
  \setcounter{enumi}{6}
  \item ( 法一 )  取  $V$  的一组基 ,  设  $\mathscr{A}$  在这组基下的矩阵为  $A$,  对分块矩阵进行初等变换 
\end{enumerate}
 因此  $\operatorname{rank}\left(E+A+A^{2}\right)+\operatorname{rank}(E-A)=\operatorname{rank}(E)+\operatorname{rank}\left(A^{3}-E\right)=n+\operatorname{rank}\left(A^{3}-E\right)$.  从而  $A^{3}=E \Longleftrightarrow$ $\operatorname{rank}\left(A^{3}-E\right)=0 \Longleftrightarrow \operatorname{rank}\left(E+A+A^{2}\right)+\operatorname{rank}(E-A)=n$.  由矩阵与线性变换的对应关系知原命题成  ㅍ.

( 法二 )  令  $g(x)=1-x, h(x)=1+x+x^{2}, f(x)=g(x) h(x)=1-x^{3}$,  则  $(g(x), h(x))=1, \operatorname{ker} f(\mathscr{A})=$ $\operatorname{ker} g(\mathscr{A}) \oplus \operatorname{ker} h(\mathscr{A})$.  从而 
$$
\begin{aligned}
\mathscr{A}^{3}=\mathscr{E} & \Longleftrightarrow f(\mathscr{A})=0 \Longleftrightarrow \operatorname{ker} f(\mathscr{A})=V \\
& \Longleftrightarrow \operatorname{ker} g(\mathscr{A}) \oplus \operatorname{ker} h(\mathscr{A})=V \\
& \Longleftrightarrow \operatorname{dim} \operatorname{ker} g(\mathscr{A})+\operatorname{dim} \operatorname{ker} h(\mathscr{A})=n \\
& \Longleftrightarrow[n-\operatorname{rank}(g(\mathscr{A}))]+[n-\operatorname{rank}(h(\mathscr{A}))]=n \\
& \Longleftrightarrow \operatorname{rank}(\mathscr{E}-\mathscr{A})+\operatorname{rank}\left(\mathscr{E}+\mathscr{A}+\mathscr{A}^{2}\right)=n
\end{aligned}
$$
 注   类似的题目可以参考丘维声的 《 高等代数 》 创新教材上册第  199  页例  3 .  法二中用到的结论可以参考蓝以中   老师的 《 高等代数简明教程 》 第二片下册第  150  至  152  页 .
$$
\begin{aligned}
& \left[\begin{array}{cc}E+A+A^{2} & \\& E-A\end{array}\right] \rightarrow\left[\begin{array}{cc}E+A+A^{2} & E+A+A^{2} \\E-A\end{array}\right]=\left[\begin{array}{cc}E+A+A^{2} & (A-E)(A+2 E)+3 E \\E-A\end{array}\right] \\
& \rightarrow\left[\begin{array}{cc}E+A+A^{2} & 3 E \\& E-A\end{array}\right] \rightarrow\left[\begin{array}{cc}3 E \\\frac{1}{3}\left(A^{3}-E\right) & E-A\end{array}\right] \rightarrow\left[\begin{array}{cc}3 E \\\frac{1}{3}\left(A^{3}-E\right)\end{array}\right] \rightarrow\left[\begin{array}{ll}3 E \\A^{3}-E\end{array}\right] 
\end{aligned}
$$
 北京大学  2006  年全国硕士研究生招生考试高代解几试题及解答 

   

2019.05.15

\begin{enumerate}
  \item (16  分 )
\end{enumerate}
(1)  设  $A, B$  分别是数域  $\mathbb{K}$  上的  $s \times n$  和  $s \times m$  矩阵 ,  叙述矩阵方程  $A X=B$  有解的充要条件 ,  并且给予   证明 .

(2)  设  $A$  是数域  $\mathbb{K}$  上的  $s \times n$  列满秩矩阵 ,  试问 :  方程  $X A=E_{n}$  是否有解 ?  有解 ,  写出它的解集 ;  无解 ,  说明理由 .

(3)  设  $A$  是数域  $\mathbb{K}$  上的  $s \times n$  列满秩矩阵 ,  对于数域  $\mathbb{K}$  上任意  $s \times m$  矩阵  $B$,  矩阵  $A X=B$  是否一定有   解 ?  当有解时 ,  它有多少个解 ?  求出它的解集 .  要求说明理由 .

\begin{enumerate}
  \setcounter{enumi}{2}
  \item (16  分 )
\end{enumerate}
(1)  设  $A$  和  $B$  分别是数域  $\mathbb{K}$  上的  $s \times n$  和  $n \times s$  矩阵 .  证明 :
$$
\operatorname{rank}(A-A B A)=\operatorname{rank}(A)+\operatorname{rank}\left(E_{n}-B A\right)-n .
$$
(2)  设  $A, B$  是实数域上的  $n$  阶矩阵 .  证明 :  矩阵  $A$  与矩阵  $B$  的相似关系不随数域扩大而改变 .

\begin{enumerate}
  \setcounter{enumi}{3}
  \item (16  分 )
\end{enumerate}
(1)  设  $A$  是数域  $\mathbb{K}$  上的  $n$  阶矩阵 .  证明 :  如果矩阵  $A$  的各阶顺序主子式都不为  0 ,  那么  $A$  可以唯一地分   解成  $A=B C$,  其中  $B$  是主对角元都为  1  的下三角矩阵 , $C$  是上三角矩阵 .

(2) 设  $A$  是数域  $\mathbb{K}$  上的  $n$  阶可逆矩阵 .  试问 : $A$  是否可以分解成  $A=B C$,  其中  $B$  是主对角元都为  1  的   下三角矩阵 , $C$  是上三角矩阵 ,  说明理由 .

\begin{enumerate}
  \setcounter{enumi}{4}
  \item (10  分 )
\end{enumerate}
(1) 设  $A$  是实数域  $\mathbb{R}$  上的  $n$  阶对称矩阵 ,  它的特征多项式  $f(\lambda)$  的所有不同复根为实数 : $\lambda_{1}, \lambda_{2}, \cdots, \lambda_{s}$.  把  $A$  的最小多项式  $m(\lambda)$  分解成  $\mathbb{R}$  上不可约多项式的乘积 .

(2)  设  $A$  是  $n$  阶实对称矩阵 , $\mathscr{A}$  是  $\mathbb{R}^{n}$  上的一个线性变换 ,  满足对任意  $\alpha \in \mathbb{R}^{n}$,  有  $\mathscr{A}(\alpha)=A \alpha$.  根据上一   问中  $m(\lambda)$  的分解 ,  把  $\mathbb{R}^{n}$  分解成线性变换  $\mathscr{A}$  的不变子空间的直和 .  说明理由 .
$$
\langle f(x), g(x)\rangle=\sum_{j=1}^{n} f(j) \overline{g(j)} .
$$
 这个二元函数是复线性空间  $\mathbb{C}^{X}$  上的一个内积 ,  从而  $\mathbb{C}^{X}$  成为一个西空间 .  设  $p_{1}(x), p_{2}(x), \cdots, p_{n}(x) \in \mathbb{C}^{X}$,  且满足 
$$
p_{k}(j)=\frac{\omega^{k j}}{\sqrt{n}}, \forall j \in X \text {, 其中 } \omega=\mathrm{e}^{\frac{2 \pi i}{n}} .
$$
(1) 求复线性空间  $\mathbb{C}^{X}$  的维数 ;

(2)  证明  $p_{1}(x), p_{2}(x), \cdots, p_{n}(x)$  是西空间  $\mathbb{C}^{X}$  的一个标准正交基 ;

(3)  令  $\sigma(f(x))=\hat{f}(x), \forall f(x) \in \mathbb{C}^{X}$,  其中  $\hat{f}(x)$  在  $x=k$  处的函数值  $\hat{f}(k)$  是  $f(x)$  在标准正交基  $p_{1}(x), p_{2}(x), \cdots, p_{n}(x)$  下的坐标的第  $k$  个分量 .  证明  $\sigma$  是西空间  $\mathbb{C}^{X}$  上的一个线性变换 ,  并且求  $\sigma$  在   标准正交基  $p_{1}(x), p_{2}(x), \cdots, p_{n}(x)$  下的矩阵 ;

(4)  证明第  $(3)$  小问中的  $\sigma$  是西空间  $\mathbb{C}^{X}$  上的一个西变换 .

\begin{enumerate}
  \setcounter{enumi}{6}
  \item (20  分 )  设  $V$  是数域  $\mathbb{K}$  上的  $n$  维线性空间 , $\mathscr{A}_{1}, \mathscr{A}_{2}, \cdots, \mathscr{A}_{s}$  为  $V$  上的线性变换 ,  令  $\mathscr{A}=\mathscr{A}_{1}+\mathscr{A}_{2}+\cdots+\mathscr{A}_{s}$.  求证 : $\mathscr{A}$  为幂等变换且  $\operatorname{rank}(\mathscr{A})=\operatorname{rank}\left(\mathscr{A}_{1}\right)+\cdots+\operatorname{rank}\left(\mathscr{A}_{s}\right)$  的充分必要条件是 :  各  $\mathscr{A}$  均为幂等变换 ,  且  $\mathscr{A}_{i} \mathscr{A}_{j}=0, i \neq j .$

  \item (15  分 )  求一个过  $x$  轴的平面  $\pi$,  使得其与单叶双曲面 

\end{enumerate}
$$
\frac{x^{2}}{4}+y^{2}-z^{2}=1
$$
 的交线为一个圆 .

\begin{enumerate}
  \setcounter{enumi}{8}
  \item (15  分 )  证明四面体的每一个顶点到对面重心的线段共点 ,  且这点到顶点的距离是它到对面重心距离的  3  倍 .

  \item (10  分 )  一条直线与坐标平面  $y O z$  面 , $x O z$  面 , $x O y$  面的交点分别是  $A, B$, $C$.  当直线变动时 ,  直线上的三个   定点  $A, B, C$  也分别在坐标平面上变动 .  此外 ,  直线上有第四点  $P$,  点  $P$  到三点的距离分别是  $a, b, c$.  求该   直线按照保持  $A, B, C$  分别在坐标平面上的规则移动时 ,  点  $P$  的轨迹 .

  \item (10  分 )  在一个仿射坐标系中 ,  已知直线  $l_{1}$  的方程为 

\end{enumerate}
$$
\left\{\begin{array}{r}
x-y+z+7=0 \\
2 x+y-6=0
\end{array}\right.
$$
 直线  $l_{2}$  过点  $M(-1,1,2)$,  并且平行于向量  $\vec{u}(1,2,3)$  判别这两条直线的位置关系 ,  并说明理由 . 1. (1)  充要条件为  $\operatorname{rank}(A)=\operatorname{rank}(A B)$.  详细解答见丘维声的 《 高等代数 》 创新教材上册第  202  页例  10 .

(2)  因为  $\operatorname{rank}(A)=n$,  故  $s \geq n$.  当  $s=n$  时 ,  解唯一  $X=A^{-1}$.  当  $s>n$  时 ,  存在  $s$  级 , $n$  级可逆矩阵  $P, Q$  使得 
$$
P A Q=\left[\begin{array}{c}
E_{n} \\
0
\end{array}\right]
$$
 设  $Q^{-1} X P^{-1}=\left[\begin{array}{ll}X_{1} & X_{2}\end{array}\right]$,  则 
$$
\left[\begin{array}{ll}
X_{1} & X_{2}
\end{array}\right]\left[\begin{array}{c}
E_{n} \\
0
\end{array}\right]=E_{n}
$$
 由此看出  $X A=E_{n}$  必有解 ,  解集为 
$$
\left\{Q\left[\begin{array}{ll}
E_{n} & X_{2}
\end{array}\right] P \mid X_{2} \text { 是 } K \text { 上任意 } n \times(s-n) \text { 矩阵 }\right\} .
$$
(3)  可以参考乒维声的 《 高等代数 》 创新教材上册第  259  页例  2 .

\begin{enumerate}
  \setcounter{enumi}{2}
  \item (1)  做初等变换 
\end{enumerate}
 因为初等变换不改变矩阵的秩 ,  故  $\operatorname{rank}(A-A B A)=\operatorname{rank}(A)+\operatorname{rank}\left(E_{n}-B A\right)-n$.

 注   本题为丘维声的 《 高等代数 》 创新教材上册第  260  页例  5 .

(2)  在实数域相似 ,  则自然在更大的数域上相似 . 下面考虑相反的情形 :  设  $n$  阶实方阵  $A, B$  在复数域相   似 ,  则存在复可逆矩阵  $P$,  使得  $P^{-1} A P=B$,  故  $A P=B P$.  将  $P$  的实部与虚部分开 ,  即设  $P=$ $P_{1}+i P_{2}, P_{1}, P_{2} \in M_{n}(\mathrm{R})$,  则 
$$
A P_{1}+i A P_{2}=P_{1} B+i P_{2} B \Longrightarrow A P_{1}=P_{1} B, A P_{2}=P_{2} B
$$
 因为  $|P|=\left|P_{1}+i P_{2}\right| \neq 0$,  故多项式  $\left|P_{1}+t P_{2}\right|$  为非零多项式 ,  存在  $t_{0} \in \mathbb{R}$  使得  $\left|P_{1}+t_{0} P_{2}\right| \neq 0$,  此时  $A\left(P_{1}+t_{0} P_{2}\right)=\left(P_{1}+t_{0} P_{2}\right) B$, 从而它们在实数域也相似 .  充要条件可以参考丘维声的 《 高等代数 》 创新教材上册第  214  页习题  $4.5$  中第  22  题 .

(2)  方程 
$$
\left[\begin{array}{ll}
1 & 0 \\
x & 1
\end{array}\right]\left[\begin{array}{ll}
a & b \\
0 & c
\end{array}\right]=\left[\begin{array}{ll}
0 & 1 \\
1 & 0
\end{array}\right]
$$
 无解 ,  由此可以看出  (1)  中的分解不是对所有矩阵都存在的 .

(2)  实对称矩阵的不同特征值对应的线性无关的特征向量个数之和为  $n$,  且不同特征值对应的特征向量线   性无关 .
$$
\mathbb{R}^{n}=\operatorname{ker}\left(\mathscr{A}-\lambda_{1} \mathscr{E}\right) \oplus \cdots \oplus \operatorname{ker}\left(\mathscr{A}-\lambda_{s} \mathscr{E}\right)
$$
 注   丘维声的 《 高等代数 》 创新教材下册第  313  页例  6 .

\begin{enumerate}
  \setcounter{enumi}{5}
  \item (1) $\forall f \in \mathbb{C}^{X}, f$  完全由  $f(1), f(2), \ldots, f(n)$  的值决定 ,  定义 
\end{enumerate}
$$
f_{i}(j)=\left\{\begin{array}{ll}
1, & i=j \\
0, & i \neq j
\end{array}, \quad 1 \leq i, j \leq n\right.
$$
 容易验证  $f_{1}, f_{2}, \ldots, f_{n}$  是线性无关的且是  $\mathbb{C}^{X}$  的一组基 .  因此  $\operatorname{dim} \mathbb{C}^{X}=n$.

(2)
$$
\begin{gathered}
\left\langle p_{k}(x), p_{k}(x)\right\rangle=\sum_{j=1}^{n} p_{k}(j) \overline{p_{k}(j)}=1 \\
\left\langle p_{k}(x), p_{l}(x)\right\rangle=\sum_{j=1}^{n} p_{k}(j) \overline{p_{l}(j)}=\sum_{j=1}^{n} \mathrm{e}^{\frac{2 \pi \mathrm{i}}{n}(k-l) j}=0
\end{gathered}
$$
 故  $p_{1}(x), p_{2}(x), \cdots, p_{n}(x)$  是西空间  $\mathbb{C}^{X}$  的标准正交基 

(3) $\forall f, g \in \mathbb{C}^{X}$,  设  $f=k_{1} p_{1}+\cdots+k_{n} p_{n}, g=l_{1} p_{1}+\cdots+l_{n} p_{n}$,  则  $f+g=\left(k_{1}+l_{1}\right) p_{1}+\cdots+\left(k_{n}+l_{n}\right) p_{n}$, $\sigma(f(x)+g(x))=(\widehat{f+g})(x)$  在  $x=q$  处的函数值是  $k_{q}+l_{q}$,  而  $\hat{f}(x)+\hat{g}(x)$  在  $x=q$  处的函数值是  $k_{q}+l_{q}$,  从而  $\sigma(f+g)=\sigma(f)+\sigma(g)$.  类似地可以证明  $\forall \lambda \in \mathbb{C}, f \in \mathbb{C}^{X}, \sigma(\lambda f)=\lambda \sigma(f)$.  因此  $\sigma$  是西   空间  $\mathbb{C}^{X}$  上的一个线性变换 .

 设  $\sigma\left(p_{1}, \ldots, p_{n}\right)=\left(p_{1}, \ldots, p_{n}\right) P$,  我们需要求  $P$.  因为  $\forall f \in \mathbb{C}^{X}, f=f(1) f_{1}+f(2) f_{2}+\cdots+f(n) f_{n}$,  故 
$$
\left(p_{1}, \ldots, p_{n}\right)=\left(f_{1}, \ldots, f_{n}\right)\left[\begin{array}{cccc}
\frac{\omega}{\sqrt{n}} & \frac{\omega^{2}}{\sqrt{n}} & \cdots & \frac{\omega^{n}}{\sqrt{n}} \\
\frac{\omega^{2}}{\sqrt{n}} & \frac{\omega^{4}}{\sqrt{n}} & \cdots & \frac{\omega^{2 n}}{\sqrt{n}} \\
\vdots & \vdots & & \vdots \\
\frac{\omega^{n}}{\sqrt{n}} & \frac{\omega^{2 n}}{\sqrt{n}} & \cdots & \frac{\omega^{n^{2}}}{\sqrt{n}}
\end{array}\right]
$$
 记上面的过渡矩阵为  $M$,  再结合  $\left(\widehat{p_{1}}, \ldots, \widehat{p_{n}}\right)=\left(f_{1}, \ldots, f_{n}\right) E$  就能得到  $M P=E, P=M^{-1}$.

(4)  因为  $(\bar{P})^{\mathrm{T}} P=E$,  故  $\sigma$  是西空间  $\mathbb{C}^{X}$  上的一个西变换 .

 注   丘维声的  《 高等代数 》 创新教材下册第  536  页例  60 .

\begin{enumerate}
  \setcounter{enumi}{6}
  \item  一个线性变换  $\mathscr{A}$  如果满足  $\mathscr{A}^{2}=\mathscr{A}$,  则称该线性变换为幂等变换 .  充分性是容易证明的 ,  直接计算就有  $\mathscr{A}$  是幂等变换 ,  关于秩的等式可以借助于矩阵的秩来说明 ,  还需要注意到的一点是对于幂等矩阵有  $\operatorname{tr}(A)=$ $\operatorname{rank}(A)$.  必要性比较复杂 ,  可以参考丘维声的 《 高等代数 》 创新教材下册第  260  页例  13 .

  \item  平面为  $z=0$  时交线不是圆 .  设平面方程为  $y=k z$,  则交线为 

\end{enumerate}
$$
\left\{\begin{array}{r}
y=k z \\
\frac{x^{2}}{4}+y^{2}-z^{2}=1
\end{array}\right.
$$
 根据对称性 ,  若交线为圆 ,  则原点为圆心 .  因为 
$$
x^{2}+y^{2}+z^{2}=4\left(1-y^{2}+z^{2}\right)+z^{2}+k^{2} z^{2}=4+\left(5-3 k^{2}\right) z^{2}
$$
 故当  $k=\pm \sqrt{\frac{5}{3}}$  时交线为圆 .

 注   相关题目见庍维声的 《 解析几何 》 第三版第  102  页习题  $3.3$  中的第  9  题 ,  只可惜书上的参考答案是错的 . 8.  设  $B, C$  的中点为  $E, \Delta B C D$  与  $\triangle A B C$  的重心分别为  $G, H$,  则  $A G$  与  $D H$  均在平面  $A E D$  中 ,  而且是相   交的 ,  设交点为  $Q$.
$$
\overrightarrow{A G}=\frac{1}{3} \overrightarrow{A D}+\frac{2}{3} \overrightarrow{A E}, \quad \overrightarrow{D H}=\frac{1}{3} \overrightarrow{D A}+\frac{2}{3} \overrightarrow{D E}
$$
 叒 
$$
\overrightarrow{A Q}=\lambda \overrightarrow{A G}=\frac{\lambda}{3} \overrightarrow{A D}+\frac{2 \lambda}{3} \overrightarrow{A E}
$$
 则 
$$
\overrightarrow{D Q}=\overrightarrow{D A}+\overrightarrow{A Q}=\left(\frac{\lambda}{3}-1\right) \overrightarrow{A D}+\frac{2 \lambda}{3} \overrightarrow{A E}, \quad \overrightarrow{D H}=\frac{2}{3} \overrightarrow{A E}-\overrightarrow{A D}
$$
 由  $\overrightarrow{D Q}$  与  $\overrightarrow{D H}$  共线知 ,
$$
\frac{\frac{\lambda}{3}-1}{-1}=\frac{\frac{2 \lambda}{3}}{\frac{2}{3}} \Longrightarrow \lambda=\frac{3}{4}
$$
 故 
$$
\overrightarrow{A Q}=\frac{3}{4} \overrightarrow{A G}=\frac{1}{4}(\overrightarrow{A B}+\overrightarrow{A C}+\overrightarrow{A D})
$$
 类似可证其他两两顶点到对面重心的交点  $Q^{\prime}$  也满足 
$$
\overrightarrow{A Q^{\prime}}=\frac{1}{4}(\overrightarrow{A B}+\overrightarrow{A C}+\overrightarrow{A D})
$$
 从而这些线段共点 ,  距离比可由  $\lambda=3 / 4$  得到 .

\begin{enumerate}
  \setcounter{enumi}{9}
  \item  设直线方程为 
\end{enumerate}
$$
\left\{\begin{array}{l}
x=x_{0}+u t \\
y=y_{0}+v t \\
z=z_{0}+w t
\end{array}\right.
$$
 其中  $u, v, w$  均非零 ,  否则不会与  $y O z, x O z, x O y$  均相交 ,  可解得 
$$
A=\left(0, y_{0}-\frac{x_{0} v}{u}, z_{0}-\frac{x_{0} w}{u}\right), \quad B=\left(x_{0}-\frac{u y_{0}}{v}, 0, z_{0}-\frac{w y_{0}}{v}\right), \quad C=\left(x_{0}-\frac{u z_{0}}{w}, y_{0}-\frac{v z_{0}}{w}, 0\right) .
$$
 由于 
$$
\begin{aligned}
&\sqrt{u^{2}+v^{2}+w^{2}}\left|t+\frac{x_{0}}{u}\right|=\sqrt{u^{2}+v^{2}+w^{2}}\left|\frac{x}{u}\right|=a \\
&\sqrt{u^{2}+v^{2}+w^{2}}\left|t+\frac{y_{0}}{v}\right|=\sqrt{u^{2}+v^{2}+w^{2}}\left|\frac{y}{v}\right|=b \\
&\sqrt{u^{2}+v^{2}+w^{2}}\left|t+\frac{z_{0}}{w}\right|=\sqrt{u^{2}+v^{2}+w^{2}}\left|\frac{z}{w}\right|=c
\end{aligned}
$$
 故  $P$  的轨迹为 
$$
\frac{x^{2}}{a^{2}}+\frac{y^{2}}{b^{2}}+\frac{z^{2}}{c^{2}}=1
$$

\begin{enumerate}
  \setcounter{enumi}{10}
  \item  直线  $\ell_{1}$  过点  $N(0,6,-1), \overrightarrow{M N}=(1,5,-3)$,  直线  $\ell_{1}$  的一个方向向量为  $\vec{v}=(-1,2,3)$.  因为 
\end{enumerate}
$$
\left|\begin{array}{ccc}
1 & 5 & -3 \\
-1 & 2 & 3 \\
1 & 2 & -3
\end{array}\right|=0
$$
 且  $\vec{v}$  与  $\vec{u}$  不平行 ,  故  $\ell_{1}$  与  $\ell_{2}$  共面相交 .  北京大学  2007  年全国硕士研究生招生考试高代解几试题及解答 

   

2019.05.12

\begin{enumerate}
  \item  回答下列问题并说明理由 :
\end{enumerate}
(1)  是否存在  $n$  阶方阵  $A, B$,  满足  $A B-B A=E$ ( 单位矩阵 )?  是否存在  $n$  维线性空间上的线性变换  $\mathscr{A}, \mathscr{B}$,  满足  $\mathscr{A} \mathscr{B}-\mathscr{B} \mathscr{A}=\mathscr{E}$ ( 恒等变换 )?

(2)  设  $n$  阶矩阵  $A$  的各行元素之和均为常数  $c$,  问  $A^{3}$  的各行元素之和是否均为常数 ?

(3)  设  $m \times n$  矩阵  $A$  的秩为  $r$,  任取  $r$  个线性无关的行向量 ,  再任取  $r$  个线性无关的列向量 ,  组成的  $r$  阶   子式是否一定不为  0 ?

(4)  设  $A, B$  都是  $m \times n$  矩阵 ,  线性方程组  $A X=0$  与  $B X=0$  同解 ,  则  $A$  与  $B$  的列向量组是否等价 ?  行   向量组是否等价 ?

(5)  把实数域  $\mathbb{R}$  看成有理数域  $\mathbb{Q}$  上的线性空间 , $b=p^{3} q^{2} r$,  其中  $p, q, r$  是互不相同的素数 ,  向量组  $1, \sqrt[n]{b}, \sqrt[n]{b^{2}}, \cdots, \sqrt[n]{b^{n-1}}$  是否线性相关 ?

\begin{enumerate}
  \setcounter{enumi}{2}
  \item  矩阵  $n$  阶矩阵  $A, B$  可交换 ,  证明  $\operatorname{rank}(A+B) \leqslant \operatorname{rank}(A)+\operatorname{rank}(B)-\operatorname{rank}(A B)$.

  \item  设  $f$  为双线性函数 ,  且对任意的  $\alpha, \beta, \gamma$  都有  $f(\alpha, \beta) f(\gamma, \alpha)=f(\beta, \alpha) f(\alpha, \gamma)$.  证明  $f$  为对称的或反对称的 .

  \item  设  $V$  是  Euclid  空间 , $U$  是  $V$  的子空间 , $\beta \in V$.  求证 : $\beta$  是  $\alpha \in V$  在  $U$  上的正交投影的充要条件为 :  对任   意的  $\gamma \in U$,  都有  $|\alpha-\beta| \leqslant|\alpha-\gamma|$.

  \item  设  $n$  阶复矩阵  $A$  满足 :  对任意  $k \in \mathbb{N}_{+}$,  都有  $\operatorname{Tr}\left(A^{k}\right)=0$.  求  $A$  的特征值 .

  \item  设  $n$  维线性空间  $V$  上的线性变换  $\mathscr{A}$  的最小多项式与特征多项式相同 .  求证 :  存在  $\alpha \in V$,  使得  $\alpha, \mathscr{A} \alpha$, $\cdots, \mathscr{A}^{n-1} \alpha$  为  $V$  的一组基 .

  \item $P$  是球内一定点 , $A, B, C$  是球面上三动点 , $\angle A P B=\angle B P C=\angle C P A=\pi / 2$.  以  $P A, P B, P C$  为棱作平   行六面体 ,  记与  $P$  相对的顶点为  $Q$,  求  $Q$  点的轨迹 .

  \item  直线  $\ell$  的方程为  $\left\{\begin{array}{l}A_{1} x+B_{1} y+C_{1} z+D_{1}=0 \\ A_{2} x+B_{2} y+C_{2} z+D_{2}=0\end{array}\right.$.  问系数满足什么条件时 ,  直线  $\ell$

\end{enumerate}
(1)  过原点 ;

(2)  平行于  $x$  轴 ,  但不与  $x$  轴重合 ;

(3)  与  $y$  轴相交 ;

(4) 与  $z$  轴重合 .

\begin{enumerate}
  \setcounter{enumi}{9}
  \item  证明双曲抛物面  $\frac{x^{2}}{a^{2}}-\frac{y^{2}}{b^{2}}=2 z$  的相互垂直的直母线的交点在双曲线上 .

  \item  求椭球面  $\frac{x^{2}}{25}+\frac{y^{2}}{16}+\frac{z^{2}}{9}=2 z$  被点  $(2,1,-1)$  平分的弦 . 1. (1)  假设存在这样的矩阵 ,  则计算矩阵方程两端的迹就有  $0=\operatorname{tr}(A B-B A)=\operatorname{tr}(A B)-\operatorname{tr}(B A)=\operatorname{tr}(E)=$ $n$,  矛盾 .  有限维线性空间上的线性变换可以通过取定一组基然后和  $n$  阶方阵对应起来 ,  从而这样的线   性变换也不存在 .

\end{enumerate}
 注   在无穷维线性空间中这样的线性变换是存在的 ,  比如在  $\mathbb{R}[x]$  中定义  $\mathscr{A} f(x)=f^{\prime}(x), \mathscr{B} f(x)=x f(x)$.

(2)  设  $\xi=(1,1, \ldots, 1)^{\mathrm{T}}$  是  $n$  维列向量 ,  则  $A \xi=c \xi$,  由此得  $A^{3} \xi=c^{3} \xi$,  因此  $A^{3}$  的各行元素之和均为  $c^{3}$.

(3)  按题中方法选出的  $r$  阶子式一定不为  0 .  可以参考丘维声的 《 高等代数 》 创新教材第  162  页例  6 .

(4)  列向量组不一定等价 ,  例如考虑 
$$
A=\left[\begin{array}{ll}
1 & 0 \\
1 & 0
\end{array}\right], \quad B=\left[\begin{array}{ll}
0 & 0 \\
1 & 0
\end{array}\right]
$$
 行向量组一定等价 .  由题意可得 
$$
A X=0, \quad\left[\begin{array}{c}
A \\
B
\end{array}\right] X=0
$$
 是同解的 ,  从而  $B$  的行向量组一定可以由  $A$  的行向量组线性表示 ,  否则将导致 
$$
\operatorname{rank}\left[\begin{array}{c}
A \\
B
\end{array}\right]>\operatorname{rank}(A),
$$
 矛盾 .  同理可得  $A$  的行向量组一定可以由  $B$  的行向量组线性表示 ,  故两个行向量组等价 .

(5)  若线性相关则 ,  则存在不全为零的  $k_{0}, k_{1}, \ldots, k_{n-1}$,  使得 
$$
k_{0}+k_{1} \sqrt[n]{b}+k_{2} \sqrt[n]{b^{2}}+\cdots+k_{n-1} \sqrt[n]{b^{n-1}}=0
$$
 令  $f(x)=k_{0}+k_{1} x+\cdots+k_{n-1} x^{n-1}$,  则  $f(x) \neq 0$,  并且在  $\mathbb{R}[x]$  中  $\left(x^{n}-b, f(x)\right) \neq 1$,  于是在  $\mathbb{Q}[x]$  中  $\left(x^{n}-b, f(x)\right) \neq 1$,  又因为  $x^{n}-b$  是  $\mathbb{Q}[x]$  中的不可约多项式 ,  故  $x^{n}-b \mid f(x)$,  矛盾 .

\begin{enumerate}
  \setcounter{enumi}{2}
  \item ( 法一 )  令 
\end{enumerate}
$$
W_{1}=\left\{X \in \mathbb{R}^{n} \mid A X=0\right\}, \quad W_{2}=\left\{X \in \mathbb{R}^{n} \mid B X=0\right\},
$$
 则  $\operatorname{dim} W_{1}=n-\operatorname{rank}(A), \operatorname{dim} W_{2}=n-\operatorname{rank}(B)$.  注意到 
$$
W_{1} \cap W_{2} \subset\left\{X \in \mathbb{R}^{n} \mid(A+B) X=0\right\}, \quad W_{1} \cup W_{2} \subset\left\{X \in \mathbb{R}^{n} \mid(A B) X=0\right\},
$$
 结合结合维数公式就有 
$$
n-\operatorname{rank}(A)+n-\operatorname{rank}(B) \leqslant n-\operatorname{rank}(A+B)+n-\operatorname{rank}(A B)
$$
 移项即得结果 .

( 法二 )  做初等变换 
$$
\begin{gathered}
{\left[\begin{array}{cc}
A & \\
& B
\end{array}\right] \rightarrow\left[\begin{array}{ll}
A & B \\
& B
\end{array}\right] \rightarrow\left[\begin{array}{cc}
A+B & B \\
B & B
\end{array}\right] \rightarrow\left[\begin{array}{cc}
A+B & B \\
-A &
\end{array}\right]} \\
{\left[\begin{array}{cc}
A+B & (A+B) B \\
-A &
\end{array}\right] \rightarrow\left[\begin{array}{cc}
A+B & \\
-A & A B
\end{array}\right]}
\end{gathered}
$$
 然后可知 
$$
\begin{aligned}
\operatorname{rank}(A)+\operatorname{rank}(B) &=\operatorname{rank}\left[\begin{array}{ll}
A \\
B
\end{array}\right]=\operatorname{rank}\left[\begin{array}{cc}
A+B & B \\
-A &
\end{array}\right]=\operatorname{rank}\left[\begin{array}{cc}
A+B & \\
-A & A B
\end{array}\right] \\
& \geqslant \operatorname{rank}\left[\begin{array}{cc}
A+B & (A+B) B \\
-A
\end{array}\right.\\
& \geqslant \operatorname{rank}(A+B)+\operatorname{rank}(A B)
\end{aligned}
$$
 注   同样是做初等变换 ,  网上有个稍微有些不一样的做法 ,  利用 
$$
\left(\begin{array}{cc}
E & 0 \\
B & -A-B
\end{array}\right)\left(\begin{array}{cc}
A+B & B \\
B & B
\end{array}\right)=\left(\begin{array}{cc}
A+B & B \\
0 & -A B
\end{array}\right)
$$

\begin{enumerate}
  \setcounter{enumi}{3}
  \item  取题设方程中的  $\gamma=\alpha$  得 
\end{enumerate}
$$
\begin{array}{cc}
f(\alpha, \beta) f(\alpha, \alpha)=f(\beta, \alpha) f(\alpha, \alpha), & \forall \alpha, \beta \\
(f(\alpha, \beta)-f(\beta, \alpha)) f(\alpha, \alpha)=0, & \forall \alpha, \beta
\end{array}
$$
 取题设方程中的  $\alpha=\beta+\gamma$,  则  $f(\beta+\gamma, \beta) f(\gamma, \beta+\gamma)=f(\beta, \beta+\gamma) f(\beta+\gamma, \gamma)$,  展开后利用  $(1)$  化简得 
$$
\begin{gathered}
f(\gamma, \beta) f(\gamma, \beta)=f(\beta, \gamma) f(\beta, \gamma), \quad \forall \beta, \gamma \\
(f(\gamma, \beta)-f(\beta, \gamma))(f(\gamma, \beta)+f(\beta, \gamma))=0, \quad \forall \beta, \gamma
\end{gathered}
$$

\begin{itemize}
  \item  若  $\forall \alpha, f(\alpha, \alpha)=0$,  则  $\forall \alpha, \beta$
\end{itemize}
$$
0=f(\alpha+\beta, \alpha+\beta)=f(\alpha, \alpha)+f(\beta, \alpha)+f(\alpha, \beta)+f(\beta, \beta)=f(\beta, \alpha)+f(\alpha, \beta)
$$
 从而  $f(\beta, \alpha)=-f(\alpha, \beta)$,  即  $f$  是反对称的 .

\begin{itemize}
  \item  若  $\exists \alpha_{0}, f\left(\alpha_{0}, \alpha_{0}\right) \neq 0$,  则根据  (1)  可知 
\end{itemize}
$$
f\left(\alpha_{0}, \beta\right)=f\left(\beta, \alpha_{0}\right), \quad \forall \beta
$$
 如果  $f$  不是对称的 ,  则  $\exists \alpha_{1}, \beta_{1}$,  使得  $f\left(\alpha_{1}, \beta_{1}\right) \neq f\left(\beta_{1}, \alpha_{1}\right)$,  由  (2)  知 
$$
f\left(\alpha_{1}, \beta_{1}\right)=-f\left(\beta_{1}, \alpha_{1}\right) \text { 且 } f\left(\beta_{1}, \alpha_{1}\right) \neq 0 .
$$
 由题设条件知 
$$
f\left(\beta_{1}, \alpha_{1}\right) f\left(\alpha_{0}, \beta_{1}\right)=f\left(\alpha_{1}, \beta_{1}\right) f\left(\beta_{1}, \alpha_{0}\right)
$$
 消去  $f\left(\beta_{1}, \alpha_{1}\right)$  得 
$$
f\left(\alpha_{0}, \beta_{1}\right)=-f\left(\beta_{1}, \alpha_{0}\right),
$$
 而由  (3)  得  $f\left(\alpha_{0}, \beta_{1}\right)=f\left(\beta_{1}, \alpha_{0}\right)$, 于是 
$$
f\left(\beta_{1}, \alpha_{0}\right)=-f\left(\beta_{1}, \alpha_{0}\right),
$$
 故  $f\left(\beta_{1}, \alpha_{0}\right)=f\left(\alpha_{0}, \beta_{1}\right)=0$.  同样可以得到  $f\left(\alpha_{0}, \alpha_{1}\right)=f\left(\alpha_{1}, \alpha_{0}\right)=0$.

 此时 
$$
\begin{aligned}
&f\left(\alpha_{0}+\alpha_{1}, \alpha_{0}+\beta_{1}\right)=f\left(\alpha_{0}, \alpha_{0}\right)+f\left(\alpha_{1}, \alpha_{0}\right)+f\left(\alpha_{0}, \beta_{1}\right)+f\left(\alpha_{1}, \beta_{1}\right)=f\left(\alpha_{0}, \alpha_{0}\right)+f\left(\alpha_{1}, \beta_{1}\right) \\
&f\left(\alpha_{0}+\beta_{1}, \alpha_{0}+\alpha_{1}\right)=f\left(\alpha_{0}, \alpha_{0}\right)+f\left(\beta_{1}, \alpha_{0}\right)+f\left(\alpha_{0}, \alpha_{1}\right)+f\left(\beta_{1}, \alpha_{1}\right)=f\left(\alpha_{0}, \alpha_{0}\right)+f\left(\beta_{1}, \alpha_{1}\right)
\end{aligned}
$$
 从而  $f\left(\alpha_{0}+\alpha_{1}, \alpha_{0}+\beta_{1}\right) \neq f\left(\alpha_{0}+\beta_{1}, \alpha_{0}+\alpha_{1}\right)$,  由  (2)  可得 
$$
f\left(\alpha_{0}+\alpha_{1}, \alpha_{0}+\beta_{1}\right)=-f\left(\alpha_{0}+\beta_{1}, \alpha_{0}+\alpha_{1}\right)
$$
 于是 
$$
f\left(\alpha_{0}, \alpha_{0}\right)+f\left(\alpha_{1}, \beta_{1}\right)=-f\left(\alpha_{0}, \alpha_{0}\right)-f\left(\beta_{1}, \alpha_{1}\right)
$$
 从而 
$$
f\left(\alpha_{0}, \alpha_{0}\right)=-f\left(\alpha_{0}, \alpha_{0}\right)
$$
 故  $f\left(\alpha_{0}, \alpha_{0}\right)=0$,  矛盾 ! 4.  详细证明可参考丘维声的 《 高等代数 》 创新教材上册第  225  页例  18 .

5 .  设  $f(x)$  的特征多项式为 
$$
f(\lambda)=\lambda^{n_{0}} \prod_{i=1}^{s}\left(\lambda-\lambda_{i}\right)^{n_{i}}
$$
 其中  $\lambda_{1}, \lambda_{2}, \ldots, \lambda_{s}$  互不相同且不为  $0, n_{0}+n_{1}+\cdots+n_{s}=n$.  则由  $\operatorname{tr}\left(A^{k}\right)=0$  可得方程组 
$$
\left\{\begin{array}{c}
n_{1} \lambda_{1}+n_{2} \lambda_{2}+\cdots+n_{s} \lambda_{s}=0 \\
n_{1} \lambda_{1}^{2}+n_{2} \lambda_{2}^{2}+\cdots+n_{s} \lambda_{s}^{2}=0 \\
\vdots \\
n_{1} \lambda_{1}^{s}+n_{2} \lambda_{2}^{s}+\cdots+n_{s} \lambda_{s}^{s}=0
\end{array}\right.
$$
 由上述方程组可得  $n_{1}=n_{2}=\cdots=n_{s}=0$,  于是  $n_{0}=n, f(\lambda)=\lambda^{n}$,  从而  $A$  的特征值均为  0 .

 注   常见的做法是利用  Newton  公式 ,  具体方法可参考丘维声的 《 高等代数 》 创新教材下册第  332  页例  4  或者   蓝以中的 《 高等代数简明教程 》 第二版下册第  217  页例  2.1.

\begin{enumerate}
  \setcounter{enumi}{6}
  \item  设  $\mathcal{A}$  的最小多项式和特征多项式为  $m(x)=f(x)=x^{n}+a_{n-1} x^{n-1}+\cdots+a_{0}$,  则线性变换  $\mathcal{A}$  的不变因子   为  $1,1, \ldots, 1, m(x)$.  易知矩阵 
\end{enumerate}
$$
A=\left[\begin{array}{ccccc}
0 & 0 & \cdots & 0 & -a_{0} \\
1 & 0 & \cdots & 0 & -a_{1} \\
0 & 1 & \cdots & 0 & -a_{2} \\
\vdots & \vdots & & \vdots & \vdots \\
0 & 0 & \cdots & 1 & -a_{n-1}
\end{array}\right]
$$
 的不变因子也为  $1,1, \ldots, 1, m(x)$.  于是存在一组基  $\alpha_{1}, \alpha_{2}, \ldots, \alpha_{n}$,  使得 
$$
\mathcal{A}\left(\alpha_{1}, \alpha_{2}, \ldots, \alpha_{n}\right)=\left(\alpha_{1}, \alpha_{2}, \ldots, \alpha_{n}\right) A
$$
 从而  $\mathcal{A} \alpha_{1}=\alpha_{2}, \mathcal{A} \alpha_{2}=\alpha_{3}=\mathcal{A}^{2} \alpha_{1}, \ldots, \mathcal{A} \alpha_{n-1}=\alpha_{n}=\mathcal{A}^{n-1} \alpha_{1}$,  故  $\alpha_{1}, \mathcal{A} \alpha_{1}, \ldots, \mathcal{A}^{n-1} \alpha_{1}$  为  $V$  的一组基 .

\begin{enumerate}
  \setcounter{enumi}{7}
  \item ( 法一 )  设球的中心为  $O$,  半径为  $r,|\overrightarrow{O P}|=d$,  则由  $\overrightarrow{O A}=\overrightarrow{O P}+\overrightarrow{P A}$  得 
\end{enumerate}
$$
r^{2}=d^{2}+2 \overrightarrow{O P} \cdot \overrightarrow{P A}+|\overrightarrow{P A}|^{2}
$$
 同理由  $\overrightarrow{O B}=\overrightarrow{O P}+\overrightarrow{P B}, \overrightarrow{O C}=\overrightarrow{O P}+\overrightarrow{P C}$  可得 
$$
r^{2}=d^{2}+2 \overrightarrow{O P} \cdot \overrightarrow{P B}+|\overrightarrow{P B}|^{2}, \quad r^{2}=d^{2}+2 \overrightarrow{O P} \cdot \overrightarrow{P C}+|\overrightarrow{P C}|^{2}
$$
 将三个式子相加得 
$$
2 \overrightarrow{O P}(\overrightarrow{P A}+\overrightarrow{P B}+\overrightarrow{P C})+|\overrightarrow{P A}|^{2}+|\overrightarrow{P B}|^{2}+|\overrightarrow{P C}|^{2}=3\left(r^{2}-d^{2}\right),
$$
 注意到  $\overrightarrow{O Q}=\overrightarrow{O P}+\overrightarrow{P A}+\overrightarrow{P B}+\overrightarrow{P C}$,  根据题设条件  $\overrightarrow{P A} \cdot \overrightarrow{P B}=0, \overrightarrow{P A \cdot P C}=0, \overrightarrow{P C} \cdot \overrightarrow{P B}=0$,  再利用内积   的性质得 
$$
\begin{aligned}
|\overrightarrow{O Q}|^{2} &=|\overrightarrow{O P}|^{2}+2 \overrightarrow{O P}(\overrightarrow{P A}+\overrightarrow{P B}+\overrightarrow{P C})+|\overrightarrow{P A}|^{2}+|\overrightarrow{P B}|^{2}+|\overrightarrow{P C}|^{2} \\
&=d^{2}+3\left(r^{2}-d^{2}\right)=3 r^{2}-2 d^{2}
\end{aligned}
$$
 这表明动点  $Q$  到球心  $O$ ( 定点 )  的距离是常数  $R=\sqrt{3 r^{2}-2 d^{2}}$,  故  $Q$  点的轨迹为以  $O$  圆心 , $R$  为半径的球   面 . ( 法二 )  以  $P$  为坐标原点 , $\overrightarrow{P A}, \overrightarrow{P B}, \overrightarrow{P C}$  所指方向为  $x, y, z$  轴正方向建立空间直角坐标系 .  可设球面的方程  ข
$$
x^{2}+y^{2}+z^{2}+2 a_{1} x+2 a_{2} y+2 a_{3} z+a_{0}=0
$$
 并且  $A, B, C$  三点的坐标为  $(a, 0,0),(0, b, 0),(0,0, c)$,  则  $Q$  的坐标为  $(a, b, c)$.  因为  $A, B, C$  三点在球面上 ,  吅 
$$
\left\{\begin{array}{l}
a^{2}+2 a_{1} a+a_{0}=0 \\
b^{2}+2 a_{2} b+a_{0}=0 \\
c^{2}+2 a_{3} c+a_{0}=0
\end{array}\right.
$$
 将三式相加就得 :  点  $Q$  在下述曲面上 
$$
x^{2}+y^{2}+z^{2}+2 a_{1} x+2 a_{2} y+2 a_{3} z+3 a_{0}=0
$$
 注   法一源自某不知名网友流传下来的解答 ,  法二为我的想法 ,  可以验证结果是一样的 ,  不过他那种做法得到的   结果的几何意义更清楚 .

\begin{enumerate}
  \setcounter{enumi}{8}
  \item (1) $D_{1}=D_{2}=0$.
\end{enumerate}
(2) $D_{1}^{2}+D_{2}^{2} \neq 0, A_{1}=0, A_{2}=0$.

(3)
$$
\left|\begin{array}{cc}
B_{1} & D_{1} \\
B_{2} & D_{2}
\end{array}\right|=0
$$
(4) $D_{1}=D_{2}=C_{1}=C_{2}=0$.

\begin{enumerate}
  \setcounter{enumi}{9}
  \item ( 法一 )  双曲抛物面上相互垂直的直母线 
\end{enumerate}
$$
\ell_{1}:\left\{\begin{array}{l}
\frac{x}{a}+\frac{y}{b}=2 \lambda \\
z=\lambda\left(\frac{x}{a}-\frac{y}{b}\right)
\end{array} \quad, \quad \ell_{2}:\left\{\begin{array}{l}
\frac{x}{a}-\frac{y}{b}=2 \mu \\
z=\mu\left(\frac{x}{a}+\frac{y}{b}\right)
\end{array}\right.\right.
$$
 的交点为  $(a(\lambda+\mu), b(\lambda-\mu), 2 \lambda \mu)$,  直线  $\ell_{1}$  的一个方向向量为 
$$
\left(\frac{1}{a}, \frac{1}{b}, 0\right) \times\left(\frac{\lambda}{a},-\frac{\lambda}{b},-1\right)=\left(-\frac{1}{b}, \frac{1}{a},-\frac{2 \lambda}{a b}\right)
$$
 直线  $\ell_{2}$  的一个方向向量为 
$$
\left(\frac{1}{a},-\frac{1}{b}, 0\right) \times\left(\frac{\mu}{a}, \frac{\mu}{b},-1\right)=\left(\frac{1}{b}, \frac{1}{a}, \frac{2 \mu}{a b}\right) .
$$
 由于  $\ell_{1}, \ell_{2}$  垂直 ,  故 
$$
-\frac{1}{b^{2}}+\frac{1}{a^{2}}-\frac{4 \lambda \mu}{a^{2} b^{2}}=0, \Longrightarrow \lambda \mu=\frac{b^{2}-a^{2}}{4} .
$$
 因此交点满足方程 
$$
\left\{\begin{aligned}
\frac{x^{2}}{a^{2}}-\frac{y^{2}}{b^{2}} &=b^{2}-a^{2} \\
z &=\frac{b^{2}-a^{2}}{2}
\end{aligned}\right.
$$
( 法二 )  设  $P\left(x_{0}, y_{0}, z_{0}\right)$  是任意一个交点 ,  并且它所在的一条直母线为 
$$
\ell:\left\{\begin{array}{l}
x=x_{0}+u t \\
y=y_{0}+v t, \quad \text { 其中 } u^{2}+v^{2}+w^{2} \neq 0 . \\
z=z_{0}+w t
\end{array}\right.
$$
 因为  $\ell$  在曲面上 ,  故 
$$
\frac{\left(x_{0}+u t\right)^{2}}{a^{2}}-\frac{\left(y_{0}+v t\right)^{2}}{b^{2}}=2\left(z_{0}+w t\right), \quad \forall t \in \mathbb{R}
$$
 服 
$$
\left(\frac{u^{2}}{a^{2}}-\frac{v^{2}}{b^{2}}\right) t^{2}+2\left(\frac{u x_{0}}{a^{2}}-\frac{v y_{0}}{b^{2}}-w\right) t+\left(\frac{x_{0}^{2}}{a^{2}}-\frac{y_{0}^{2}}{b^{2}}-2 z_{0}\right)=0, \quad \forall t \in \mathbb{R},
$$
 上面的方程对于  $t \in \mathbb{R}$  均成立 ,  故 
$$
\left\{\begin{array}{c}
\frac{u^{2}}{a^{2}}-\frac{v^{2}}{b^{2}}=0 \\
\frac{u x_{0}}{a^{2}}-\frac{v y_{0}}{b^{2}}=w \\
\frac{x_{0}^{2}}{a^{2}}-\frac{y_{0}^{2}}{b^{2}}=2 z_{0}
\end{array}\right.
$$
 于是可设两条直母线的方向向量为  $\left(a, b, w_{1}\right),\left(a,-b, w_{2}\right)$,  其中  $w_{1}=\frac{x_{0}}{a}-\frac{y_{0}}{b}, w_{2}=\frac{x_{0}}{a}+\frac{y_{0}}{b}$  由向量垂直知  $a^{2}-b^{2}+w_{1} w_{2}=0$,  于是  $\frac{x_{0}^{2}}{a^{2}}-\frac{y_{0}^{2}}{b^{2}}=b^{2}-a^{2}$.

 综上得交点满足方程 
$$
\left\{\begin{array}{l}
\frac{x^{2}}{a^{2}}-\frac{y^{2}}{b^{2}}=b^{2}-a^{2} \\
\frac{x^{2}}{a^{2}}-\frac{y^{2}}{b^{2}}=2 z
\end{array}\right.
$$

\begin{enumerate}
  \setcounter{enumi}{10}
  \item  设弦的一个方向向量为  $(a, b, c)$,  设弦与椭球面的一个交点为  $(2+a t, 1+b t,-1+c t)$,  则另一个交点为  $(2-a t, 1-b t,-1-c t)$.  两个交点均在椭球面上 ,  故 
\end{enumerate}
$$
\left\{\begin{array}{l}
\frac{(2+a t)^{2}}{25}+\frac{(1+b t)^{2}}{16}+\frac{(-1+c t)^{2}}{9}=1 \\
\frac{(2-a t)^{2}}{25}+\frac{(1-b t)^{2}}{16}+\frac{(-1-c t)^{2}}{9}=1
\end{array}\right.
$$
 由此可得 
$$
\frac{2 a}{25}+\frac{b}{16}-\frac{c}{9}=0
$$
 故弦所在直线的方程为 
$$
\left\{\begin{array}{l}
x=2+a t \\
y=1+b t \\
z=-1+\left(\frac{18 a}{25}+\frac{9 b}{16}\right) t
\end{array}\right.
$$
 由此看出弦在平面 
$$
\frac{2}{25}(x-2)+\frac{1}{16}(y-1)-\frac{1}{9}(z+1)=0
$$
 上 ,  将以上推导过程反过来就看出过点  $(2,1,-1)$  且位于上述平面的直线均是椭球面的平分弦所在直线 .  北京大学  2008  年全国硕士研究生招生考试高代解几试题及解答 

   

2019.05.09

\begin{enumerate}
  \item  回答下列问题 :
\end{enumerate}
(1) $A$  是  $s \times n$  矩阵 .  非齐次线性方程组  $A X=\beta$  有解且  $\operatorname{rank}(A)=r$,  则  $A X=\beta$  的解向量中线性无关   的最多有多少个 ?  并找出一组个数最多的线性无关解向量 .

(2) $A X=\beta$  对于所有的  $s$  维非零向量  $\beta$  都有解 ,  求  $\operatorname{rank}(A)$.

\begin{enumerate}
  \setcounter{enumi}{2}
  \item (1) $A$  是  $s \times n$  矩阵 , $B$  是  $n \times m$  矩阵 , $\operatorname{rank}(A B)=\operatorname{rank}(B)$,  则对于所有  $m \times l$  矩阵  $C$,  是否有  $\operatorname{rank}(A B C)=\operatorname{rank}(B C)$ ?  给出理由 .
\end{enumerate}
(2) $A$  是  $n$  阶实矩阵 , $A$  的每一元素的代数余子式都等于此元素 ,  求  $\operatorname{rank}(A)$.

\begin{enumerate}
  \setcounter{enumi}{3}
  \item (1) $A, C$  分别是  $n, m$  实对称矩阵 , $B$  是  $n \times m$  实矩阵 ,  并且  $\left(\begin{array}{cc}A & B \\ B^{\mathrm{T}} & C\end{array}\right)$  是正定矩阵 .  证明 
\end{enumerate}
$$
\left|\begin{array}{cc}
A & B \\
B^{\mathrm{T}} & C
\end{array}\right| \leqslant|A| \cdot|C|,
$$
 并且当且仅当  $B=0$  时等号成立 .

(2) $A=\left(a_{i j}\right)_{n \times n}$  是  $n$  阶实矩阵 , $\left|a_{i j}\right| \leqslant 1$,  证明  $|A|^{2} \leqslant n^{n}$.

\begin{enumerate}
  \setcounter{enumi}{4}
  \item $f(x)$  为一整系数多项式 , $n$  不能整除  $f(0), f(1), \cdots, f(n-1)$.  证明  $f(x)$  无整数根 .

  \item $A$  是数域  $\mathbb{K}$  上的  $n$  阶矩阵 , $A$  的特征多项式的复根都属于  $\mathbb{K}$,  证明  $A$  相似于上三角矩阵 .

  \item $V$  是数域  $\mathbb{K}$  上的线性空间 , $\mathscr{A}, \mathscr{B}$  是  $V$  上的线性变换 ,  且  $\mathscr{A}, \mathscr{B}$  的最小多项式互素 ,  求满足  $\mathscr{A} \mathscr{C}=\mathscr{C} \mathscr{B}$  的所有线性变换  $\mathscr{C}$.

  \item $\mathscr{A}$  是  $n$  维欧式空间  $V$  上的线性变换 ,  证明  $\mathscr{A}$  是  $V$  上的第一类正交变换当且仅当存在  $V$  上的正交变换  $\mathscr{B}$,  使得  $\mathscr{A}=\mathscr{B}^{2}$.

  \item  求过直线  $\ell$ : $\left\{\begin{array}{c}x-y+z+4=0 \\ x+y-3 z=0\end{array}\right.$,  且与平面  $\pi_{1}: x+y+2 z=0$  垂直的平面  $\pi_{2}$.

  \item  平面  $A x+B y+C z+D=0$  与单叶双曲面  $x^{2}+y^{2}-z^{2}=1$  的交线是两条直线 ,  证明  $A^{2}+B^{2}=C^{2}+D^{2}$.

  \item  直线  $\ell_{1}$  过点  $(1,1,1)$,  并且与直线  $\ell_{2}:\left\{\begin{array}{c}x+y+z=0 \\ x-y-3 z=0\end{array}\right.$  相交 ,  交角为  $\frac{\pi}{3}$.  求  $\ell_{1}$  的方程 .

  \item  证明球面  $\sigma_{1}: x^{2}+y^{2}+z^{2}-2 x-2 y-4 z+2=0$  与球面  $\sigma_{2}: x^{2}+y^{2}+z^{2}+2 x-6 y+1=0$  有交点 ,  并求   出交圆的圆心坐标 . 1. (1)  设  $X_{1}, X_{2}, \ldots, X_{n-r}$  为  $A X=0$  的解空间的一组基 , $\gamma$  为  $A X=\beta$  的一个特解 .  下面证明  $\gamma, \gamma+X_{1}, \gamma+$ $X_{2}, \ldots, \gamma+X_{n-r}$  是线性无关的 .  设 

\end{enumerate}
$$
k_{0} \gamma+k_{1}\left(\gamma+X_{1}\right)+k_{2}\left(\gamma+X_{2}\right)+\cdots+k_{n-r}\left(\gamma+X_{n-r}\right)=0
$$
 贝羽 
$$
\left(k_{0}+k_{1}+\cdots+k_{n-r}\right) \gamma+k_{1} X_{1}+k_{2} X_{2}+\cdots+k_{n-r} X_{n-r}=0,
$$
 方程两端同时左乘  $A$  得 
$$
\left(k_{0}+k_{1}+\cdots+k_{n-r}\right) \beta=0
$$
 由于  $\beta \neq 0$,  故  $k_{0}+k_{1}+\cdots+k_{n-r}=0$,  故 
$$
k_{1} X_{1}+k_{2} X_{2}+\cdots+k_{n-r} X_{n-r}=0
$$
 由此就得  $k_{1}=k_{2}=\cdots=k_{n-r}=0, k_{0}=0$.  然后只需说明元素个数大于等于  $n-r+2$  的解向量必定   线性相关 ,  就知道线性无关的解向量最多只有  $n-r+1$  个 ,  具体的例子可取上面找到线性无关组 .

 假设存在  $n-r+2$  个解向量  $\xi_{1}, \ldots, \xi_{n-r+2}$  线性无关 ,  则  $\xi_{1}-\xi_{n-r+2}, \xi_{2}-\xi_{n-r+2}, \ldots, \xi_{n-r+1}-\xi_{n-r+2}$  是线性无关的 ,  并且是  $A X=0$  的解向量 ,  这与  $A X=0$  的解空间的维数为  $n-r$  矛盾 .

(2)  将  $\beta$  依次取成  $E_{s}$  的列向量 ,  设相应的解为  $X_{1}, X_{2}, \ldots, X_{s}$,  则  $A\left(X_{1}, X_{2}, \ldots, X_{s}\right)=E_{s}$,  由此知  $s=$ $\operatorname{rank}\left(E_{s}\right) \leqslant \operatorname{rank}(A)$,  又由于  $A$  的行数为  $s$,  故  $\operatorname{rank}(A) \leqslant s$,  于是  $\operatorname{rank}(A)=s$.

\begin{enumerate}
  \setcounter{enumi}{2}
  \item (1)  设  $W_{1}=\{X \mid A B X=0\}, W_{2}=\{X \mid B X=0\}$,  则  $W_{2} \subset W_{1}, \operatorname{dim} W_{1}=m-\operatorname{rank}(A B)=m-$ $\operatorname{rank}(B)=\operatorname{dim} W_{2}$,  于是  $W_{1}=W_{2}$.  令  $U_{1}=\{X \mid A B C X=0\}, U_{2}=\{X \mid B C X=0\}$,  则  $U_{2} \subset U_{1}$, $\forall \alpha \in U_{1}, A B C \alpha=0$,  故  $C \alpha \in W_{1}=W_{2}$,  于是  $B C \alpha=0$,  从而  $\alpha \in U_{2}$,  故  $U_{1} \subset U_{2}, U_{1}=U_{2}$,  因此  $l-\operatorname{rank}(A B C)=l-\operatorname{rank}(B C)$,  故  $\operatorname{rank}(A B C)=\operatorname{rank}(B C)$.
\end{enumerate}
 注   丘维声的 《 高等代数 》 创新教材上册第  179  页习题  $4.3$  第  15  题 .

(2)  由题意有  $A^{\mathrm{T}}=A^{*}$,  其中  $A^{*} A$  的伴随矩阵 ,  故  $A^{\mathrm{T}} A=A^{*} A=|A| E$.  若  $|A|=0$,  则  $A=0, \operatorname{rank}(A)=0$.  若  $|A| \neq 0$,  则  $\operatorname{rank}(A)=n$.

\begin{enumerate}
  \setcounter{enumi}{3}
  \item (1)  详细解答见庍维声的 《 高等代数 》 创新教材上册第  350  页例  16 .

  \item  由题意知  $f(0) \not \equiv 0(\bmod n), f(1) \not \equiv 0(\bmod n), \ldots, f(n-1) \not \equiv 0(\bmod n)$.  对于任意  $k \in \mathbb{Z}$,  存在唯一的 

  \item  由  Jordan  标准型就知道结论成立 ,  不过不用那么深刻的定理 ,  用数学归纳法也能做出来 .  具体方法可以参   考丘维声的 《 高等代数 》 创新教材上册第  298  页例  6  或者蓝以中的 《 高等代数简明教程 》 第二版上册第  324  页命题  $4.8$  或者蓝以中的 《 高等代数学习指南 》 第  217  页例  $4.17$.

  \item  线性变换  $\mathscr{C}$  一定是零变换 .  设  $\mathscr{A}, \mathscr{B}$  的最小多项式分别为  $m_{A}(x), m_{B}(x)$,  则由两者互素知存在  $u(x), v(x) \in$

\end{enumerate}
$$
u(x) m_{A}(x)+v(x) m_{B}(x)=1
$$
 于是 
$$
v(\mathscr{A}) m_{B}(\mathscr{A})=\mathscr{E}
$$
 从而 
$$
\mathscr{C}=v(\mathscr{A}) m_{B}(\mathscr{A}) \mathscr{C}=v(\mathscr{A}) \mathscr{C} m_{B}(\mathscr{B})=0
$$
 注   相关的题目参考丘维声的 《 高等代数 》 创新教材下册第  301  页例  27 .  上面证明的关键是从  $A C=C B$  推出  $f(A) C=C f(B), \forall f(x) \in \mathbb{K}[x] .$

\begin{enumerate}
  \setcounter{enumi}{7}
  \item  充分性容易证明 ,  取定  $V$  的一组标准正交基 ,  设  $\mathscr{A}, \mathscr{B}$  在此组基下的矩阵为  $A, B$,  则由  $\mathscr{A}=\mathscr{B}{ }^{2}$  得  $A=B^{2},|A|=|B|^{2}=1$,  而且直接按定义就能验证  $\mathscr{A}$  是正交变换 ,  从而是第一类正交变换 .
\end{enumerate}
 证明必要性可以利用蓝以中的 《 高等代数简明教程 》 第二版下册第  115  页的 

 命题  $4.3$  设  $A$  是实数域上的  $n$  阶方阵 ,  则  $A$  是正交矩阵且行列式为  1  的充分必要条件是存在实数域上  $n$  阶反对称矩阵  $S$,  使  $A=\mathrm{e}^{S}$.

 我们已知  $\mathscr{A}$  是正交变换 ,  取定  $V$  的一组标准正交基 ,  设  $\mathscr{A}$  在这组基下的矩阵为  $A$,  则  $A$  是正交矩阵 ,  根   据上面的定理知存在实数域上  $n$  阶反对称矩阵  $S$,  使  $A=\mathrm{e}^{S}$,  令  $B=\mathrm{e}^{S / 2}$,  并且定义  $\mathscr{B}$  为  $B$  对应的线性   变换 ,  则  $\mathscr{B}$  是所要找的正交变换 .

 必要性的另一种证明方法是利用蓝以中的 《 高等代数简明教程 》 第二版下册第  24  页的 

 定理  $2.1$  设  $\mathscr{A}$  是  $n$  维欧氏空间  $V$  内的正交变换 ,  则在  $V$  内存在一组标准正交基 ,  使  $\mathscr{A}$  在该组基下的矩   阵成如下准对角形 :

%\includegraphics[max width=\textwidth]{2022_04_18_33b622a7abd81c227674g-045}

 其中  $\lambda_{i}=\pm 1(i=1,2, \ldots, k)$,  而 
$$
S_{j}=\left[\begin{array}{cc}
\cos \varphi_{j} & -\sin \varphi_{j} \\
\sin \varphi_{j} & \cos \varphi_{j}
\end{array}\right] \quad\left(\varphi_{j} \neq k \pi, j=1,2, \ldots, l\right)
$$
 由于  $\mathscr{A}$  是第一类的 ,  则  $|J|=1$,  因此  $\lambda_{i}$  中  $-1$  的个数是偶数个 ,  可设为  $2 s$.  通过交换基的次序可以把  $J$  变成对角线上前  $k-2 s$  个  $\lambda_{i}$  为  1 ,  紧跟的  $2 s$  个  $\lambda_{i}$  为  $-1$,  记所得矩阵为  $A$.  为了证明原命题 ,  我们只需要 
$$
\left[\begin{array}{cc}
\cos \frac{\varphi_{j}}{2} & -\sin \frac{\varphi_{j}}{2} \\
\sin \frac{\varphi_{j}}{2} & \cos \frac{\varphi_{j}}{2}
\end{array}\right]^{2}=S_{j}, \quad\left[\begin{array}{cc}
-1 & 0 \\
0 & -1
\end{array}\right]=\left[\begin{array}{cc}
\cos \pi & -\sin \pi \\
\sin \pi & \cos \pi
\end{array}\right]=\left[\begin{array}{cc}
\cos \frac{\pi}{2} & -\sin \frac{\pi}{2} \\
\sin \frac{\pi}{2} & \cos \frac{\pi}{2}
\end{array}\right]^{2}\left[\begin{array}{cc}
0 & -1 \\
1 & 0
\end{array}\right]^{2}
$$
$$
\left[\begin{array}{cc}
0 & -1 \\
1 & 0
\end{array}\right]
$$
$$
\left[\begin{array}{cc}
\cos \frac{\varphi_{j}}{2} & -\sin \frac{\varphi_{j}}{2} \\
\sin \frac{\varphi_{j}}{2} & \cos \frac{\varphi_{j}}{2}
\end{array}\right]
$$
 这样得到的  $B$  就满足  $A=B^{2}$.

 注   必要性的第二种证法中找  $B$  的时候其实是利用了  $S_{j}$  的几何意义 ,  它说的是绕原点旋转  $\varphi_{j}$,  于是  $B$  中的对   应矩阵只需取为绕原点旋转  $\varphi_{j} / 2$  的 . 8.  设  $\pi_{2}: \mu(x-y+z+4)+\nu(x+y-3 z)=0$,  则  $\pi_{2}$  的一个法向量为  $(\mu+\nu, \nu-\mu, \mu-3 \nu)$.  由  $\pi_{1} \perp \pi_{2}$  得  $\mu+\nu+\nu-\mu+2(\mu-3 \nu)=0$,  故  $\mu=2 \nu$,  于是  $\pi_{2}$  的方程为  $3 x-y-z+8=0$.

\begin{enumerate}
  \setcounter{enumi}{9}
  \item ( 法一 )  单叶双曲面同族的任意两条不同的直母线异面 ,  故题设中的平面与单叶双曲面的交线应为不同族的   直母线 ,  可设直母线的方程分别为 
\end{enumerate}
$$
\left\{\begin{array}{l}
u_{1}(x+z)+v_{1}(1+y)=0 \\
u_{1}(1-y)+v_{1}(x-z)=0
\end{array},\left\{\begin{array}{l}
u_{2}(x+z)+v_{2}(1-y)=0 \\
u_{2}(1+y)+v_{2}(x-z)=0
\end{array}\right.\right.
$$
 它们的一个方向向量分别为 
$$
\left(u_{1}^{2}-v_{1}^{2}, 2 u_{1} v_{1},-u_{1}^{2}-v_{1}^{2}\right), \quad\left(v_{2}^{2}-u_{2}^{2}, 2 u_{2} v_{2}, u_{2}^{2}+v_{2}^{2}\right),
$$
 平面的法向量与上面两个方向向量均垂直 ,  从而  $(A, B, C)$  平行于 
$$
\begin{aligned}
&\left(u_{1}^{2}-v_{1}^{2}, 2 u_{1} v_{1},-u_{1}^{2}-v_{1}^{2}\right) \times\left(v_{2}^{2}-u_{2}^{2}, 2 u_{2} v_{2}, u_{2}^{2}+v_{2}^{2}\right) \\
=& 2\left(u_{1} v_{2}+u_{2} v_{1}\right)\left[\left(u_{1} u_{2}+v_{1} v_{2}\right),-\left(u_{1} v_{2}-u_{2} v_{1}\right),\left(u_{1} u_{2}-v_{1} v_{2}\right)\right],
\end{aligned}
$$
 因为  $(A, B, C) \neq 0$,  故  $u_{1} v_{2}+u_{2} v_{1} \neq 0$,  并且可设 
$$
\left\{\begin{array}{l}
A=\left(u_{1} v_{2}+u_{2} v_{1}\right)\left(u_{1} u_{2}+v_{1} v_{2}\right) t \\
B=-\left(u_{1} v_{2}+u_{2} v_{1}\right)\left(u_{1} v_{2}-u_{2} v_{1}\right) t \\
C=\left(u_{1} v_{2}+u_{2} v_{1}\right)\left(u_{1} u_{2}-v_{1} v_{2}\right) t
\end{array}\right.
$$
 下面来求出两条直线的交点 ,  然后就能根据交点  $\left(x_{0}, y_{0}, z_{0}\right)$  在平面上求出  $D$.  第一条直线的方程中的第一   个式子乘以  $u_{2}$  瑊去第二条直线的方程中的第一个式子乘以  $u_{1}$  得 
$$
u_{2} v_{1}(1+y)-u_{1} v_{2}(1-y)=0
$$
 类似地可以得到关于  $(x+z)$  与  $(x-z)$  的式子 ,  然后解出 
$$
x_{0}=-\frac{u_{1} u_{2}+v_{1} v_{2}}{u_{1} v_{2}+u_{2} v_{1}}, \quad y_{0}=\frac{u_{1} v_{2}-u_{2} v_{1}}{u_{1} v_{2}+u_{2} v_{1}}, \quad z_{0}=\frac{u_{1} u_{2}-v_{1} v_{2}}{u_{1} v_{2}+u_{2} v_{1}},
$$
 从而  $D=-A x_{0}-B y_{0}-C z_{0}=\left(u_{1} v_{2}+u_{2} v_{1}\right)^{2} t$,  故  $A^{2}+B^{2}=C^{2}+D^{2}$.

( 法二 )  设  $P\left(x_{0}, y_{0}, z_{0}\right)$  是平面与单叶双曲面的一个交点 ,  则 
$$
\left\{\begin{array}{r}
A x_{0}+B y_{0}+C z_{0}+D=0 \\
x_{0}^{2}+y_{0}^{2}-z_{0}^{2}=1
\end{array}\right.
$$
 可设过点  $P$  且在平面与曲面相交集合中的直线为 
$$
\ell:\left\{\begin{array}{l}
x=x_{0}+u t \\
y=y_{0}+v t, \quad \text { 其中 } u^{2}+v^{2}+w^{2} \neq 0 . \\
z=z_{0}+w t
\end{array}\right.
$$
 那么由于  $\ell$  在平面上 ,  故  $A u+B v+C w=0$;  又因为  $\ell$  在曲面上 ,  故 
$$
\left(x_{0}+u t\right)^{2}+\left(y_{0}+v t\right)^{2}-\left(z_{0}+w t\right)^{2}=1, \quad \forall t \in \mathbb{R}
$$
 即 
$$
\left(u^{2}+v^{2}-w^{2}\right) t^{2}+2\left(u x_{0}+v y_{0}-w z_{0}\right) t=0, \quad \forall t \in \mathbb{R}
$$
 上面的方程对于  $t \in \mathbb{R}$  均成立 ,  故 
$$
\left\{\begin{aligned}
u^{2}+v^{2}-w^{2} &=0 \\
u x_{0}+v y_{0}-w z_{0} &=0
\end{aligned}\right.
$$
 田 
$$
\left\{\begin{array}{r}
A u+B v+C w=0 \\
u x_{0}+v y_{0}-w z_{0}=0
\end{array}\right.
$$
 得到 
$$
(u, v, w)=k\left(-B z_{0}-C y_{0}, A z_{0}+C x_{0}, A y_{0}-B x_{0}\right) \text {, 其中 } k \neq 0,
$$
 将上面等式右端代入到  $u^{2}+v^{2}-w^{2}=0$  中并结合  $(1)$  可得  $D^{2}+C^{2}-B^{2}-A^{2}=0$.

\begin{enumerate}
  \setcounter{enumi}{10}
  \item  直线  $\ell_{2}$  的一个方向向量为  $(1,1,1) \times(1,-1,-3)=-2(1,-2,1)$,  设  $\ell_{1}$  的一个方向向量为  $(u, v, w)$,  则 
\end{enumerate}
$$
\frac{|u-2 v+w|}{\sqrt{u^{2}+v^{2}+w^{2}} \sqrt{1^{2}+(-2)^{2}}+1^{2}}=\frac{1}{2} .
$$
$\ell_{2}$  过点  $(0,0,0)$,  由  $\ell_{1}$  与  $\ell_{2}$  共面得 
$$
\left|\begin{array}{ccc}
1 & 1 & 1 \\
u & v & w \\
1 & -2 & 1
\end{array}\right|=0 \Longrightarrow u=w
$$
 将  $u=w$  代入前面那个式子得 
$$
\frac{2|u-v|}{\sqrt{2 u^{2}+v^{2}} \sqrt{6}}=\frac{1}{2}
$$
 平方移项得 
$$
2 u^{2}-16 u v+5 v^{2}=0
$$
 故 
$$
\frac{v}{u}=\frac{8 \pm 3 \sqrt{6}}{5}
$$
 从而  $\ell_{1}$  的方向向量为 
$$
\left(1, \frac{8+3 \sqrt{6}}{5}, 1\right) \text { 或 }\left(1, \frac{8-3 \sqrt{6}}{5}, 1\right) \text {, }
$$
 故  $\ell_{1}$  的方程为 
$$
\frac{x-1}{1}=\frac{y-1}{\frac{8+3 \sqrt{6}}{5}}=\frac{z-1}{1} \text { 或 } \frac{x-1}{1}=\frac{y-1}{\frac{8-3 \sqrt{6}}{5}}=\frac{z-1}{1} .
$$

\begin{enumerate}
  \setcounter{enumi}{11}
  \item $\sigma_{1}$  的球心为  $O_{1}(1,1,2)$,  半径为  $r_{1}=2, \sigma_{2}$  的球心为  $O_{2}(-1,3,0)$,  半径为  $r_{2}=3$,  球心距为  $d=\sqrt{4+4+4}=$ $2 \sqrt{3}$,  因为  $r_{2}-r_{1}<d<r_{1}+r_{2}$,  从而两个球面有交点 .  将两个方程相减得到交点所在平面为  $4 x-4 y+4 z-1=$ 0 .  过  $O_{1}, O_{2}$  的直线的参数方程为 
\end{enumerate}
$$
\left\{\begin{array}{l}
x=-1+t \\
y=3-t \\
z=t
\end{array}\right.
$$
 平面与直线的交点即为交圆的圆心  $\left(\frac{5}{12}, \frac{19}{12}, \frac{17}{12}\right)$.  北京大学  2009  年全国硕士研究生招生考试高代解几试题及解答 

   

2019.06.25

 注   本试题中  $\mathrm{r}(A)$  表示  $A$  的秩 , $|A|$  表示矩阵  $A$  的行列式 , $A^{\prime}$  表示矩阵  $A$  的转置矩阵 .

\begin{enumerate}
  \item ((10  分 )  一般说来一个向量组的极大线性无关部分组是不唯一的 ,  那么什么向量组的极大线性无关部分组   是唯一的 ?  证明你的结论 .

  \item (10  分 )  设多项式  $f(x)$  的所有复根都是实数 ,  证明 :  如果  $a$  是  $f(x)$  的导数  $f^{\prime}(x)$  的重根 ,  则  $a$  也是  $f(x)$  的   根 .

  \item (10  分 )  设  $S$  为  $n$  阶实对称矩阵 , $S_{1}, S_{2}$  都是  $m$  阶实对称矩阵 ,  证明 :  若准对角矩阵 

\end{enumerate}
$$
\left(\begin{array}{cc}
S & 0 \\
0 & S_{1}
\end{array}\right) \text { 与 }\left(\begin{array}{cc}
S & 0 \\
0 & S_{2}
\end{array}\right)
$$
 合同 ,  则  $S_{1}$  与  $S_{2}$  合同 .

\begin{enumerate}
  \setcounter{enumi}{4}
  \item (15  分 )  解方程组 
\end{enumerate}
$$
\left\{\begin{aligned}
x+y+z &=2 \\
(x-y)^{2}+(y-z)^{2}+(z-x)^{2} &=14 . \\
x^{2} y^{2} z+x^{2} y z^{2}+x y^{2} z^{2} &=2
\end{aligned}\right.
$$

\begin{enumerate}
  \setcounter{enumi}{5}
  \item $\left(15\right.$  分 )  设  $A$  为  $n$  阶实方阵且有  $A A^{\prime}=A^{2}$,  证明 : $A$  是对称矩阵 

  \item (15  分 )  设  $n \geqslant 2, M_{n}(K)$  为  $K$  上所有  $n$  阶方阵所成集合 , $M_{n}(K)$  上的一个函数  $f$  即为映射  $f: M_{n}(K) \rightarrow$ $K, M_{n}(K)$  上的所有函数组成的集合记为  $F(K)$,  在  $F(K)$  中定义加法和数乘运算如下 :  对任意  $f, g \in F(K)$,  对任意  $k \in K$  和任意  $A \in M_{n}(K)$,

\end{enumerate}
$$
(f+g)(A)=f(A)+g(A), \quad(k f)(A)=k f(A)
$$
 则  $F(K)$  关于此运算成为数域  $K$  上的一个线性空间 .  对于  $f \in F(K), f$  称为是列线性函数如果  $f$  对于矩   阵的每一列都是线性的 ,  即对  $K^{n}$  中任意列向量  $\beta_{1}, \beta_{2}, \ldots, \beta_{n}, \beta$,  任意  $1 \leqslant j \leqslant n$  以及任意  $k \in K$,  都有 

$f\left(\beta_{1}, \cdots, \beta_{j-1}, \beta_{j}+\beta, \beta_{j+1}, \cdots, \beta_{n}\right)=f\left(\beta_{1}, \cdots, \beta_{j-1}, \beta_{j}, \beta_{j+1}, \cdots, \beta_{n}\right)+f\left(\beta_{1}, \cdots, \beta_{j-1}, \beta, \beta_{j+1}, \cdots, \beta_{n}\right)$

 和 
$$
f\left(\beta_{1}, \cdots, \beta_{j-1}, k \beta_{j}, \beta_{j+1}, \cdots, \beta_{n}\right)=k f\left(\beta_{1}, \cdots, \beta_{j-1}, \beta_{j}, \beta_{j+1}, \cdots, \beta_{n}\right)
$$
( 其中的矩阵用它们的列向量组表示出 ).  若对于任意  $A \in M_{n}(K)$,  当  $A$  有两个列向量相同时必有  $f(A)=0$,  则称  $f$  是反对称的 .  用  $\mathrm{SP}(K)$  表示  $F(K)$  中所有反对称列线性函数所成的集合 .

 证明 : $\mathrm{SP}(K)$  是  $F(K)$  的一个子空间 ,  并求  $\mathrm{SP}(K)$  的维数和一组基 . 7. (15  分 )  设  $U$  为齐次线性方程组  $A B X=0$  的解空间 ,  其中  $A$  为  $n \times m$  矩阵 , $B$  为  $m \times p$  矩阵 , $X$  为  $p \times 1$  矩阵 ,  证明 : $m$  维向量空间  $K^{m}$  中子集合 
$$
W=\{Y \mid Y=B X, X \in U\}
$$
 是子空间 ,  它的维数等于  $\mathrm{r}(B)-\mathrm{r}(A B)$,  并利用此结论证明对任意三个矩阵  $A, B, C$  有 
$$
\mathrm{r}(A B)+\mathrm{r}(B C) \leqslant \mathrm{r}(B)+\mathrm{r}(A B C)
$$

\begin{enumerate}
  \setcounter{enumi}{8}
  \item (10  分 )  设  $\mathbb{R}$  为实数域 , $\alpha_{1}, \alpha_{2}, \ldots, \alpha_{s}$  是  $n$  维欧氏空间  $\mathbb{R}^{n}$  中的一线性无关向量组 ,  其中  $\mathbb{R}^{n}$  中的内积为标   准内积  $(\alpha, \beta)=\alpha \cdot \beta^{\prime}$,  这里的向量  $\alpha$  和  $\beta$  都看成是  $1 \times n$  矩阵 ,  用  $B$  表示  $(i, j)$  元为  $\left(\alpha_{i}, \alpha_{j}\right), 1 \leqslant i, j \leqslant s$  的  $s \times s$  矩阵 ,  对向量组  $\alpha_{1}, \alpha_{2}, \ldots, \alpha_{s}$  施行施密特  (Schmidt)  正交化过程后得到向量组  $\beta_{1}, \beta_{2}, \ldots, \beta_{s}$,  证   明 :
\end{enumerate}
$$
|B|=\prod_{i=1}^{s}\left\|\beta_{i}\right\|^{2}
$$
 其中  $\left\|\beta_{i}\right\|$  表示向量  $\beta_{i}$  的长度 .

\begin{enumerate}
  \setcounter{enumi}{9}
  \item (12  分 )  请问直线 
\end{enumerate}
$$
\ell:\left\{\begin{array}{l}
A_{1} x+B_{1} y+C_{1} z+D_{1}=0 \\
A_{2} x+B_{2} y+C_{2} z+D_{2}=0
\end{array}\right.
$$
 的系数满足什么条件时才具有以下性质 ?\\
(1)  经过原点 ;\\
(2) 与  $x$  轴平行但不重合 ;\\
(3)  与  $y$  轴相交 ;\\
(4)  与  $z$  轴垂直  ( 不必相交 ).

\begin{enumerate}
  \setcounter{enumi}{10}
  \item (15  分 )  设平面  $A x+B y+C z+D=0$  与双曲抛物面  $2 z=x^{2}-y^{2}$  的交线为两条直线 ,  证明 :
\end{enumerate}
$$
A^{2}-B^{2}-2 C D=0
$$

\begin{enumerate}
  \setcounter{enumi}{11}
  \item (10  分 )  设空间直角坐标系中的曲面  $Q$  的方程为  $x^{2}+y^{2}-z^{2}=1$,  取一个过  $z$  轴的平面  $\Sigma$  并考虑全体与之   平行的平面族 .  问 :  这些平行平面与  $Q$  的截线是什么类型的曲线 ?  当它们与  $\Sigma$  的距离变动时 ,  截线的形状如   何变化 ?  请给出清楚的描述并说明判断理由 .

  \item (13  分 )  给出空间中半径为  1  的球面  $S$  和到球心距离为  2  的一点  $P$,  考虑过  $P$  点且与  $S$  相交的任一条直   线 ,  取两个交点的中点 ,  用解析几何的方法证明这些中点的轨迹在一个球面上 ,  并求出球心和半径 . 1.  如果一个向量组除去零向量后所剩的向量组线性无关 ,  则该向量组的极大线性无关部分组是唯一的 .

\end{enumerate}
 证明 :  一个向量组除去零向量后所剩的向量组是唯一的且与原向量组等价 ,  又由于此时它线性无关 ,  故它是   极大线性无关组 .

\begin{enumerate}
  \setcounter{enumi}{2}
  \item ( 法一 )  设 
\end{enumerate}
$$
f(x)=\prod_{i=1}^{s}\left(x-x_{i}\right)^{n_{i}}
$$
 其中  $x_{1}, x_{2}, \ldots, x_{s}$  为互不相同的实数 ,  则 
$$
\frac{f^{\prime}(x)}{f(x)}=\sum_{i=1}^{s} \frac{n_{i}}{x-x_{i}}
$$
 故 
$$
f^{\prime}(x)=f(x) \sum_{i=1}^{s} \frac{n_{i}}{x-x_{i}}
$$
 求导得 
$$
f^{\prime \prime}(x)=f^{\prime}(x) \sum_{i=1}^{s} \frac{n_{i}}{x-x_{i}}+f(x) \sum_{i=1}^{s} \frac{-n_{i}}{\left(x-x_{i}\right)^{2}},
$$
 若  $a$  是  $f^{\prime}(x)$  的重根 ,  则  $f^{\prime}(a)=f^{\prime \prime}(a)=0$,  如果  $a$  不是  $f(x)$  的根 ,  则  $a$  必不与  $x_{1}, x_{2}, \ldots, x_{s}$  中的任意一   个数相等 ,  从而 
$$
f(a) \sum_{i=1}^{s} \frac{-n_{i}}{\left(a-x_{i}\right)^{2}}=0
$$
 于是  $f(a)=0$,  矛盾 !

( 法二 )  设 
$$
f(x)=\prod_{i=1}^{s}\left(x-x_{i}\right)^{n_{i}}
$$
 其中  $x_{1}, x_{2}, \ldots, x_{s}$  为互不相同的实数 , $n_{1}+n_{2}+\cdots+n_{s}=n$,  则  $x_{i}$  为  $f^{\prime}(x)$  的  $n_{i}-1$  重根 , $i=1,2, \ldots, s$.  根据  Rolle  定理知在  $x_{i}, x_{i+1}$  之间必有  $f^{\prime}(x)$  的一个根 , $i=1,2, \ldots, s-1$,  这样新得到的  $f^{\prime}(x)$  的根共  $s-1$  个 ,  加上原先  $x_{1}, x_{2}, \ldots, x_{s}$  中还剩下的  $f^{\prime}(x)$  的根就得到  $f^{\prime}(x)$  的  $n-1$  个根 .  因为新得到的  $s-1$  个根是  $f^{\prime}(x)$  的单根 ,  所以从  $a$  为  $f^{\prime}(x)$  的重根知  $a$  为  $f(x)$  的根 .

 注   此题为胡适耕与刘先忠编著的 《 高等代数定理问题方法 》 第  16  页第  71  题 .  裴礼文的 《 数学分析中的典型   问题与方法 》 第二版第  207  页例  $3.2 .2$  与上面法二相关 .  胡适耕的那本书上一个比较有趣的题目见第  14  页   的第  61  题 :  设复多项式  $f(x)$  的根全位于上半平面 ,  则  $f^{\prime}(x)$  亦必如此 .

\begin{enumerate}
  \setcounter{enumi}{3}
  \item Witt  消去定理的推广 .  详细证明见丘维声的 《 高等代数 》 创新教材下册第  426  页定理  7  或者蓝以中的 《 高   等代数学习指南 》 第  250  页例  $2.7$.

  \item  由  $x+y+z=2$  得  $(x+y+z)^{2}=4$,  即  $x^{2}+y^{2}+z^{2}+2(x y+x z+y z)=4$,  而原方程的第二个式子   展开化简可得  $x^{2}+y^{2}+z^{2}-(x y+x z+y z)=7$,  故  $x^{2}+y^{2}+z^{2}=6, x y+x z+y z=-1$,  再由原方   程的第三个式子可得  $x y z=-2$.  这样  $x, y, z$  可以看作三次方程  $t^{3}-2 t^{2}-t+2=0$  的三个不同的根 ,  而  $t^{3}-2 t^{2}-t+2=(t-2)(t-1)(t+1)=0$  的三个根为  $2,1,-1$,  故  $(x, y, z)$  有  6  种取法 .

  \item ( 法一 )  因为 

\end{enumerate}
$$
\operatorname{tr}\left(A^{\prime}-A\right)\left(A^{\prime}-A\right)^{\prime}=\operatorname{tr}\left(A^{\prime}-A\right)\left(A-A^{\prime}\right)=\operatorname{tr}\left(A^{\prime} A-A^{2}-\left(A^{\prime}\right)^{2}+A A^{\prime}\right)=0
$$
 并且  $A$  是实矩阵 ,  从而  $A^{\prime}-A=0, A^{\prime}=A$. ( 法二 )  对实方阵的阶数做数学归纳法 .  当阶数为  1  时命题成立 .  假设当阶数小于  $n$  时命题均成立 ,  下面考虑   阶数为  $n$  的情形 .  若  $A$  可逆自然命题成立 .  若  $A$  不可逆 ,  则  $A$  必有特征值为  0 ,  设  $\alpha$  为对应的一个特征向   量 ,  将  $\alpha$  单位化并且在其基础上扩充为  $\mathbb{R}^{n}$  的一组基 ,  记为  $\xi_{1}, \xi_{2}, \ldots, \xi_{n}$,  则 
$$
\left(\begin{array}{c}
\xi_{1}^{\mathrm{T}} \\
\xi_{2}^{\mathrm{T}} \\
\vdots \\
\xi_{n}^{\mathrm{T}}
\end{array}\right) A\left(\begin{array}{llll}
\xi_{1} & \xi_{2} & \cdots & \xi_{n}
\end{array}\right)=\left(\begin{array}{c}
\xi_{1}^{\mathrm{T}} \\
\xi_{2}^{\mathrm{T}} \\
\vdots \\
\xi_{n}^{\mathrm{T}}
\end{array}\right)\left(\begin{array}{llll}
0 & A \xi_{2} & \cdots & A \xi_{n}
\end{array}\right)=\left(\begin{array}{cc}
0 & \beta^{\mathrm{T}} \\
0 & A_{1}
\end{array}\right) .
$$
 令  $P=\left(\xi_{1}, \xi_{2}, \ldots, \xi_{n}\right)$,  则由题设得  $\left(P^{\mathrm{T}} A P\right)\left(P^{\mathrm{T}} A^{\mathrm{T}} P\right)=\left(P^{\mathrm{T}} A P\right)^{2}$,  因此 
$$
\left(\begin{array}{ll}
0 & \beta^{\mathrm{T}} \\
0 & A_{1}
\end{array}\right)\left(\begin{array}{cc}
0 & 0 \\
\beta & A_{1}^{\mathrm{T}}
\end{array}\right)=\left(\begin{array}{cc}
0 & \beta^{\mathrm{T}} \\
0 & A_{1}
\end{array}\right)\left(\begin{array}{ll}
0 & \beta^{\mathrm{T}} \\
0 & A_{1}
\end{array}\right)
$$
 展开即为 
$$
\left(\begin{array}{cc}
\beta^{\mathrm{T}} \beta & \beta^{\mathrm{T}} A_{1}^{\mathrm{T}} \\
A_{1} \beta & A_{1} A_{1}^{\mathrm{T}}
\end{array}\right)=\left(\begin{array}{cc}
0 & \beta^{\mathrm{T}} A_{1} \\
0 & A_{1}^{2}
\end{array}\right)
$$
 因此  $\beta^{\mathrm{T}} \beta=0$,  而  $\beta \in \mathbb{R}^{n}$,  故  $\beta=0$;  同时  $A_{1} A_{1}^{\mathrm{T}}=A_{1}^{2}$,  由归纳假设知  $A_{1}$  为实对称矩阵 .  综合前面的结果   就有  $P^{\mathrm{T}} A P$  是实对称矩阵 ,  从而  $A$  是实对称矩阵 .  由数学归纳法原理知原命题成立 .

 注   与法二中解法类似的题目见斤维声的 《 高等代数 》 创新教材上册第  297  页例  4 .

\begin{enumerate}
  \setcounter{enumi}{6}
  \item  因为  det $\in \operatorname{SP}(K)$,  从而  $\mathrm{SP}(K)$  非空 .  任取  $f_{1}, f_{2} \in \mathrm{SP}(K), k \in K$,  则  $f_{1}+f_{2}, k f_{1}$  仍为列线性函数且反对   称 ,  故  $f_{1}+f_{2} \in \mathrm{SP}(K), k f_{1} \in \mathrm{SP}(K)$,  这就说明  $\mathrm{SP}(K)$  为  $F(K)$  的一个子空间 .
\end{enumerate}
 任取  $f \in \mathrm{SP}(K)$,  下面证明  $f=f(E)$ det.  对于任意  $n$  阶方阵  $A$,  若  $\mathrm{r}(A)<n$,  则  $f(A)=0=(f(E)$ det $)(A)$.  若  $\mathrm{r}(A)=n$,  则经过有限次初等列变换可将  $E$  变为  $A$,  由于  $f, f(E) \operatorname{det}$  是  $M_{n}(K)$  上两个反对称列线性函   数 ,  且  $f(E)=(f(E) \operatorname{det})(E)$,  故  $f(A)=(f(E) \operatorname{det})(A)$.  综合前面两点就有  $f=f(E)$ det.  又由于  $\operatorname{det}$  不   是零函数 ,  故  det  为  $\mathrm{SP}(K)$  的一组基 , $\operatorname{dim} \mathrm{SP}(K)=1$.

 注   上面证明中用到了蓝以中的 《 高等代数简明教程 》 第二版上册第  164  页的推论  1,2 .  相关的题目见  《 高等代   数简明教程 》 第二版上册第  191  页的第  13  题或者 《 高等代数学习指南 》 第  128  页的例  $1.2$,  第  146  页的例  $2.3$,  第  151  页例  $2.6$.

\begin{enumerate}
  \setcounter{enumi}{7}
  \item  因为  $0 \in U$,  从而  $0 \in W, W \neq \varnothing . \forall y_{1}, y_{2} \in W, k \in K, \exists x_{1}, x_{2} \in U$,  使得  $y_{1}=B x_{1}, y_{2}=B x_{2}$,  于是  $y_{1}+y_{2}=B\left(x_{1}+x_{2}\right) \in W, k y_{1}=B\left(k x_{1}\right) \in W$, 从而  $W$  是  $K^{m}$  的子空间 .  定义  $U$  到  $W$  的线性映射为  $\mathscr{B}(X)=B X, \forall X \in U$,  则 
\end{enumerate}
$$
\operatorname{dim} W=\operatorname{dimim} \mathscr{B}=\operatorname{dim} U-\operatorname{dim} \operatorname{ker} \mathscr{B}=p-\mathrm{r}(A B)-(p-r(B))=\mathrm{r}(B)-\mathrm{r}(A B)
$$
 设  $S$  为齐次线性方程组  $A B C X=0$  的解空间 , $T=\{Y \mid Y=B C X, X \in S\}$,  则由前面证明的结果得  ( 相   当于把  $B$  换成  $B C$ )
$$
\operatorname{dim} T=r(B C)-r(A B C)
$$
 又由于  $T \subset W$,  故  $\operatorname{dim} T \leqslant \operatorname{dim} W$,  把前面得到结果代入就完成了证明 .

\begin{enumerate}
  \setcounter{enumi}{8}
  \item  乒维声的 《 高等代数 》 创新教材下册第  462  页例  11 .

  \item (1) $D_{1}=D_{2}=0$. $(3)$

\end{enumerate}
$$
\left|\begin{array}{cc}
B_{1} & D_{1} \\
B_{2} & D_{2}
\end{array}\right|=0
$$
(4)
$$
\left|\begin{array}{ccc}
A_{1} & B_{1} & C_{1} \\
A_{2} & B_{2} & C_{2} \\
0 & 0 & 1
\end{array}\right|=0 \Longrightarrow A_{1} B_{2}-A_{2} B_{1}=0 .
$$

\begin{enumerate}
  \setcounter{enumi}{10}
  \item  两条相交直线必属于异族 ,  设它们的方程分别为 
\end{enumerate}
$$
\left\{\begin{array}{r}
x+y=2 u \\
u(x-y)=z
\end{array},\left\{\begin{array}{r}
x-y=2 v \\
v(x+y)=z
\end{array}\right.\right.
$$
 则两条直线的一个方向向量分别可取为  $(-1,1,-2 u),(1,1,2 v)$,  两直线的交点为  $(\mathrm{u}+\mathrm{v}, \mathrm{u}-\mathrm{v}, 2 \mathrm{uv})$.  应当有 
$$
\left\{\begin{array}{c}
A(u+v)+B(u-v)+2 C u v+D=0 \\
(-1,1,-2 u) \times(1,1,2 v) / /(A, B, C)
\end{array}\right.
$$
 于是存在  $t \neq 0$,  使得  $(A, B, C)=2 t(v+u, v-u,-1)$,  这时  $D=-A x-B y-C z=-4 u v t$,  直接代入计   算就知  $A^{2}-B^{2}-2 C D=0$.

\begin{enumerate}
  \setcounter{enumi}{11}
  \item $\Sigma$  过  $z$  轴 ,  不妨设为  $x=0$,  因为其他情况可以通过坐标变换把平面的方程变为  $x=0$.  于是与  $\Sigma$  平行的平   面为  $x=a$.  截线的方程为 
\end{enumerate}
$$
\left\{\begin{aligned}
x &=a \\
y^{2}-z^{2} &=1-a^{2}
\end{aligned}\right.
$$
 当  $-1<a<1$  时 ,  截线为双曲线 ,  中心都在一条直线  $x$  轴上 .

 当  $|a|=1$  时 ,  截线为两个点 .

 当  $|a|>1$  时 ,  截线为双曲线 ,  但是实轴与虚轴变了 ,  中心仍都在一条直线  $x$  轴上 .

\begin{enumerate}
  \setcounter{enumi}{12}
  \item  通过建立适当的方程使  $S$  的方程为  $x^{2}+y^{2}+z^{2}=1$,  点  $P$  的坐标为  $(0,0,2)$.  任取一条过  $P$  的直线 ,  设它   的的方程为 
\end{enumerate}
$$
\left\{\begin{array}{l}
x=a t \\
y=b t \\
z=c t+2
\end{array}\right.
$$
 设它与球面相交所得两个点的参数分别为  $t_{1}, t_{2}$,  则把直线方程代入球面方程可得 
$$
\left(a^{2}+b^{2}+c^{2}\right) t^{2}+4 c t+3=0,
$$
 由根与系数的关系知交点的中点对应的参数为 
$$
\frac{t_{1}+t_{2}}{2}=\frac{-2 c}{a^{2}+b^{2}+c^{2}},
$$
 故交点的中点的坐标为 
$$
\left(x_{0}, y_{0}, z_{0}\right)=\left(\frac{-2 a c}{a^{2}+b^{2}+c^{2}}, \frac{-2 b c}{a^{2}+b^{2}+c^{2}}, \frac{2\left(a^{2}+b^{2}\right)}{a^{2}+b^{2}+c^{2}}\right),
$$
 直接计算就知  $x_{0}^{2}+y_{0}^{2}+\left(z_{0}-1\right)^{2}=1$,  故交点的中点都在圆心为  $(0,0,1)$,  半径为  1  的球面上 .  北京大学 2010 年全国硕士研究生招生考试高代解几试题及解答 

   

2019.04.09

\begin{enumerate}
  \item (13 分 )  整系数多项式  $f(x)=\sum_{k=0}^{n} a_{k} x^{k}(n \geq 2010)$.  若存在素数  $p$  满足 :
\end{enumerate}
(1) $p \nmid a_{n}$;

(2) $p \mid a_{i}, i=0,1,2, \cdots, 2008$;

(3) $p^{2} \nmid a_{0}$.

 证明  $f(x)$  必有次数不低于  2009  的不可约整系数因式 .

\begin{enumerate}
  \setcounter{enumi}{2}
  \item (13 分 )  向量组  $\alpha_{1}, \alpha_{2}, \cdots, \alpha_{s}$  线性无关 ,  且可以由向量组  $\beta_{1}, \beta_{2}, \cdots, \beta_{t}$  线性表出 ,  证明必存在某个向量  $\beta_{j}(j=$ $1,2, \cdots, t$ )  使得向量组  $\beta_{j}, \alpha_{2}, \cdots, \alpha_{s}$  线性无关 .

  \item (12 分 )  设  $A$  是非零矩阵 ,  证明  $A$  可以写成某个列满秩矩阵与某个行满秩矩阵的乘积 .

  \item (13 分 ) $A, B$  是  $n$  阶实矩阵 ,  且满足  $A=\left(B-\frac{1}{110} E\right)^{\prime}\left(B+\frac{1}{110} E\right)$,  证明 : 对任意的  $n$  维非零实列向量  $\xi$,  方程   组  $A^{\prime}\left(A^{2}+A\right) X=A^{\prime} \xi$  必有非零解 .

  \item (13 分 )  设  $A$  是  $n$  阶正定矩阵 ,  向量组  $\beta_{1}, \beta_{2}, \cdots, \beta_{n}$  满足  $\beta_{i}^{\mathrm{T}} A \beta_{j}=0(1 \leq i<j \leq n)$.  试问  $\beta_{1}, \beta_{2}, \cdots, \beta_{n}$  的秩   可能是多少并予以证明 .

  \item (12 分 )  线性变换  $\mathscr{A}$  是对称变换 ,  且  $\mathscr{A}$  是正交变换 ,  证明  $\mathscr{A}$  是某个对合变换  ( 即满足  $\mathscr{A}^{2}=\mathscr{E}, \mathscr{E}$  是单位变换 ).

  \item (12 分 ) $V$  是内积空间 , $\xi, \eta$  是  $V$  中两个长度相等的向量 .  证明必存在某个正交变换将  $\xi$  变到  $\eta$.

  \item (12 分 ) $A$  是复矩阵 , $B$  是幂零矩阵 ,  且  $A B=B A$,  证明 : $|A+2010 B|=|A|$.

  \item (12 分 )  求过  $z$  轴且与平面  $x+2 y+3 z=1$  大角为  $60^{\circ}$  的平面的方程 .

  \item (12 分 )  求直线  $\left\{\begin{array}{l}x-y+z=1 \\ x+y-z=1\end{array}\right.$  绕  $z$  轴旋转所成旋转曲面的方程 ,  并指出这是什么曲面 .

  \item (14  分 )  设仿射坐标系 XOY 中的一个仿射变换  $f$  的变换公式为  $\left\{\begin{array}{l}x^{\prime}=x+2 y \\ y^{\prime}=4 x+3 y\end{array}\right.$.

\end{enumerate}
(1)  求  $f$  的不变直线 .

(2)  以  $f$  的两条不变直线为坐标轴建立仿射坐标系  $\mathrm{X}^{\prime} \mathrm{O}^{\prime} \mathrm{Y}^{\prime}$,  求在此坐标系中  $f$  的变换公式 .

\begin{enumerate}
  \setcounter{enumi}{12}
  \item (12 分 )  用不过圆雉顶点的平面切害圆雉 ,  证明所截的曲线只可能为椭圆 ,  双曲线和抛物线 ,  并说明曲线类型   随切害角度的变换规律 . 1.  首先不妨假设  $f(x)$  为本原多项式 ,  事实上如果  $f(x)$  不是本原多项式 ,  则提出  $f(x)$  的系数的最大公因数后所   得的多项式同样满足上述三个条件 ,  于是下面假设  $f(x)$  为本原多项式 .
\end{enumerate}
 若  $f(x)$  在  $\mathbb{Z}[x]$  中不可约 ,  则命题成立 .

 若  $f(x)$  在  $\mathbb{Z}[x]$  中可约 ,  则可以将  $f(x)$  表示为至少两个  $\mathbb{Z}[x]$  中不可约多项式的乘积 ,  其中必有一个不可约因   式  $g(x)$  满足  $p$  整除  $g(x)$  的常数项 .  假设 
$$
a_{n} x^{n}+\cdots+a_{0}=f(x)=g(x) h(x)=\left(b_{m} x^{m}+\cdots+b_{0}\right)\left(c_{s} x^{s}+\cdots+c_{0}\right), p \mid b_{0}
$$
 先约定  $b_{k}=c_{j}=0, k>m, j>s$.  由  $a_{0}=b_{0} c_{0}, p\left|a_{0}, p^{2} \nmid a_{0}, p\right| b_{0}$,  得  $p \nmid c_{0}, p \mid b_{0}$.  又由  $b_{0} c_{1}+b_{1} c_{0}=$ $a_{1}, p \mid a_{1}$,  得  $p \mid b_{1}$.  又由  $b_{0} c_{2}+b_{1} c_{1}+b_{2} c_{0}=a_{2}, p \mid a_{2}$,  得  $p \mid b_{2}$.  以此类推可得  $p \mid b_{k}, 0 \leq k \leq 2008$.  而   由  $a_{n}=b_{m} c_{s}, p \nmid a_{n}$,  得  $p \nmid b_{m}$,  于是  $m \geq 2009$,  从而  $g(x)$  为次数不低于  2009  的不可约整系数因式 .

 注   更一般性的题目见蓝以中老师的 《 高等代数学习指南 》 第  396  页例  $2.8$.

\begin{enumerate}
  \setcounter{enumi}{2}
  \item $\alpha_{1}, \alpha_{2}, \cdots, \alpha_{s}$  线性无关 ,  从而  $\alpha_{2}, \cdots, \alpha_{s}$  是线性无关的 .  如果  $\forall 1 \leq j \leq t$  均有  $\beta_{j}, \alpha_{2}, \cdots, \alpha_{s}$  线性相关 ,  则每   个  $\beta_{j}$  均由  $\alpha_{2}, \cdots, \alpha_{s}$  线性表出 ,  从而  $\alpha_{1}$  能被  $\alpha_{2}, \cdots, \alpha_{s}$  线性表出 ,  这与  $\alpha_{1}, \alpha_{2}, \cdots, \alpha_{s}$  线性无关矛盾 .

  \item  设  $A$  为  $n \times m$  矩阵  $\alpha_{1}, \cdots, \alpha_{m}$  为其列向量 ,  则由  $A \neq 0$  知存在极大线性无关组设为  $\alpha_{i_{1}}, \cdots, \alpha_{i_{s}}$.  将其他列向量   用极大组中的向量线性表示出来 ,  写成矩阵形式即为  $A=\left(\alpha_{i_{1}}, \cdots, \alpha_{i_{s}}\right) B=C B . C$  为  $n \times s$  矩阵 , $B$  为  $s \times m$  矩   阵 ,  且  $C$  列满秩 .  由前面的约定知  $r(A)=s, s \leq n, s \leq m, r(B) \leq s$.  若  $r(B)<s$,  则  $s=r(A)=r(C B) \leq$ $r(B)<s$,  予盾 .  从而  $r(B)=s$,  即  $B$  行满秩 .

  \item  对于任意取定的非零  $\xi$,  我们需要说明两件事情 :  首先方程组有解 ,  其次解集中有非零解 .

\end{enumerate}
 若方程组的解集中只有零解 ,  则将有  $A^{\mathrm{T}} \xi=0$,  并且  $\operatorname{det} A \neq 0$,  从而  $\xi=0$,  矛盾 .  从而若能证明方程组有 
$$
r\left(A^{\mathrm{T}} A(A+E)\right) \leq r\left(A^{\mathrm{T}} A(A+E) A^{\mathrm{T}} \xi\right) \leq r\left(A^{\mathrm{T}}\right)=r\left(A^{\mathrm{T}} A\right)
$$
 因此只要能说明  $r(A+E)=n$,  则命题成立 .  若  $r(A+E)<n$,  则  $-1$  是  $A$  的特征值 ,  从而存在非零  $X_{0} \in$
$$
\begin{aligned}
-X_{0}^{\mathrm{T}} X_{0}=& X_{0}^{\mathrm{T}}\left(B-\frac{1}{110} E\right)^{\mathrm{T}}\left(B+\frac{1}{110} E\right) X_{0}=\left(B X_{0}\right)^{\mathrm{T}} B X_{0}-\frac{X_{0}^{\mathrm{T}} X_{0}}{110^{2}} \\
& \Longrightarrow 0 \leq\left(B X_{0}\right)^{\mathrm{T}} B X_{0}=\left(-1+\frac{1}{110^{2}}\right) X_{0}^{\mathrm{T}} X_{0}<0,
\end{aligned}
$$

\begin{enumerate}
  \setcounter{enumi}{5}
  \item $0,1, \cdots, n$  均可能 .  因为  $A$  为正定矩阵 ,  合同标准型为  $E$,  从而不奻设  $A=E$.  若  $\beta_{i}=0,1 \leq i \leq n$,  则秩为  0 .

  \item  若  $\xi=\eta$,  则直接取恒等变换即可 .

\end{enumerate}
 若  $\xi \neq \eta$,  则只需找某个合适的镜面反射 .  事实上 ,  对于给定的  $V$  中单位向量  $\beta$,  可以定义镜面反射 : $\mathscr{A} \alpha=$ $2(\xi, \beta) \beta=\eta$  中解得  $\beta$,  从而也就得到  $\mathscr{A}$. 8.  存在可逆矩阵  $P$  使得  $A, B$  同时上三角化 ,  再注意到  $B$  的特征值均为  0 ,  从而 
$$
|A+2010 B|=\left|P^{-1}(A+2010 B) P\right|=|A| .
$$
 注   证明中用到的结论可以在丘维声老师的 《 高等代数 》 创新教材下册第  290  页例  5  找到 ,  在该书的第  414  页有对   可交换线性变换的总结 .

\begin{enumerate}
  \setcounter{enumi}{9}
  \item  设平面的方程为  $a x+b y=0$.  则  $\frac{|a+2 b|}{\sqrt{a^{2}+b^{2}} \sqrt{14}}=1 / 2,5 a^{2}-8 a b-b^{2}=0, a / b=\frac{4 \pm \sqrt{21}}{5}$.  故平面的方程为 : $y=-\frac{4 \pm \sqrt{21}}{5} x .$

  \item  设  $(x, y, z)$  是曲面上一点 ,  而  $\left(x_{0}, y_{0}, z_{0}\right)$  是原直线上与其  $z$  轴坐标相同的一点 ,  则有 

\end{enumerate}
$$
\left\{\begin{aligned}
x^{2}+y^{2}-x_{0}^{2}-y_{0}^{2} &=0 \\
z &=z_{0} \\
x_{0}-y_{0}+z_{0} &=1 \\
x_{0}+y_{0}-z_{0} &=1
\end{aligned}\right.
$$
 这是单叶双曲面 .

\begin{enumerate}
  \setcounter{enumi}{11}
  \item (1)  设  $f$  的不变直线的方程为  $a x+b y+c=0 .$  由  $\left\{\begin{array}{l}x^{\prime}=x+2 y \\ y^{\prime}=4 x+3 y\end{array} \Longrightarrow\left\{\begin{array}{l}x=\frac{3 x^{\prime}-2 y^{\prime}}{-5}, f \text { 的不变直线经过 } \\ y=\frac{4 x^{\prime}-y^{\prime}}{5}\end{array}\right.\right.$  变换后的方程为  $-a \frac{3 x^{\prime}-2 y^{\prime}}{5}+b \frac{4 x^{\prime}-y^{\prime}}{5}+c=0$,  即  $(4 b-3 a) x^{\prime}+(2 a-b) y^{\prime}+5 c=0$.  若  $c \neq 0$,  则  $\left\{\begin{array}{l}5 a=4 b-3 a \\ 5 b=2 a-b\end{array} \Longrightarrow\left\{\begin{array}{l}a=0 \\ b=0\end{array}\right.\right.$,  矛盾 .  从而  $c=0, b(4 b-3 a)-a(2 a-b)=0 \Longrightarrow a^{2}+a b-2 b^{2}=0, \Longrightarrow$ $a=-2 b$  或者  $a=b$,  故  $f$  的不变直线的方程为 : $2 x-y=0$  和  $x+y=0$.
\end{enumerate}
(2)  可设新仿射坐标系到原仿射坐标系的坐标变换公式为  $\left(\begin{array}{l}x \\ y\end{array}\right)=\left(\begin{array}{cc}\lambda & \mu \\ 2 \lambda & -\mu\end{array}\right)\left(\begin{array}{l}x_{1} \\ y_{1}\end{array}\right), \lambda \mu \neq 0$.  于是  $f$  在新   仿射坐标系中的方程为  $\left(\begin{array}{cc}\lambda & \mu \\ 2 \lambda & -\mu\end{array}\right)\left(\begin{array}{l}x_{1}^{\prime} \\ y_{1}^{\prime}\end{array}\right)=\left(\begin{array}{cc}1 & 2 \\ 4 & 3\end{array}\right)\left(\begin{array}{cc}\lambda & \mu \\ 2 \lambda & -\mu\end{array}\right)\left(\begin{array}{l}x_{1} \\ y_{1}\end{array}\right)$,  即  $\left(\begin{array}{l}x_{1}^{\prime} \\ y_{1}^{\prime}\end{array}\right)=\left(\begin{array}{c}5 \\ -1\end{array}\right)\left(\begin{array}{l}x_{1} \\ y_{1}\end{array}\right)$.  注   此题取自丘维声老师的 《 解析几何 》 第三版第  228  页第  7  题 .

\begin{enumerate}
  \setcounter{enumi}{12}
  \item  把平面固定 ,  让圆雉变动位置 ,  观察截线的形状 .  建立空间直角坐标系 ,  使固定平面为  $x O y$  平面 ,  设圆雉的顶   点坐标为  $(0,0,-a)$.  假设圆雉的中心直线始终在平面  $y O z$  上 ,  正方向为圆雉顶点到底面中心的方向 ,  母线与   中心直线的夹角为  $\alpha, 0<\alpha<\pi / 2$.  让圆锥的中心直线在平面  $y O z$  上从  $z$  轴负方向逆时针向  $z$  轴正方向旋转 .  先设法求出圆雉与平面  $z=0$  的交线方程 ,  进而可确定截线的类型 .  设中心直线与  $z$  轴夹角为  $\pi / 2-\beta$,  则中心   直线的方程为 : $\frac{x}{0}=\frac{y}{\cos \beta}=\frac{z+a}{\sin \beta}$.  设  $(x, y, z)$  为圆雉曲面上一点 ,  则  $\frac{|y \cos \beta+(z+a) \sin \beta|}{\sqrt{x^{2}+y^{2}+(z+a)^{2}}}=|\cos \alpha|$.  于 
\end{enumerate}
$$
\left\{\begin{array}{r}
x^{2} \cos ^{2} \alpha+\left(\cos ^{2} \alpha-\cos ^{2} \beta\right) y^{2}-(a \sin 2 \beta) y+\left(\cos ^{2} \alpha-\sin ^{2} \beta\right) a^{2}=0 \\
z=0
\end{array}\right.
$$
 当  $\cos ^{2} \alpha>\cos ^{2} \beta$  时为椭圆 , $\cos ^{2} \alpha=\cos ^{2} \beta$  时为抛物线 , $\cos ^{2} \alpha<\cos ^{2} \beta$  时为双曲线 .  北京大学  2011  年全国硕士研究生招生考试高代几何试题及解答     

2019.04.04

\begin{enumerate}
  \item (40  分 ,  每题各  4  分 )  判断是非 ,  并陈述理由 
\end{enumerate}
(1) $A$  是一个秩为  5  的矩阵 , $A$  的  3,4  行线性无关 , 1,3  列也线性无关 ,  那么  $A$  的行列式的一个  2  阶子式  $A(3,4 ; 1,3)$  不等于  0 .

(2)  齐次线性方程组  $A X=0$  的解唯一 ,  则  $A X=b$  的解也唯一 .

(3) $F$  是一个数域 , $W$  是  $F^{5}$  的子空间 ,  那么存在  $F^{6}$  到  $F^{5}$  的线性映射  $\phi$,  使  $\phi\left(F^{6}\right)=W$.

(4)  非零线性变换  $A$,  在某组基上的矩阵的对角线上元素均不为  0 ,  则  $A$  必有非  0  的特征根 .

(5)  对于线性变换  $\sigma$  及其共轭转置  $\sigma^{*}, \operatorname{ker} \sigma^{*} \sigma=\operatorname{ker} \sigma$.

(6) $V^{*}$  是线性空间  $V$  的对偶空间 , $W$  是  $V$  的真子空间 ,  则存在  $f \in V^{*}, f \neq 0$,  使得  $f(W)=0$.

(7) 复数域上  13  维线性空间必有  10  维不变子空间 .

(8)  对任意的  $n$,  存在  $n$  次多项式  $p(x)$  在有理数域上不可约 .

(9)  对角线上元素均不相等的上三角矩阵必可相似对角化 .

(10) $A$  是域  $\mathbb{F}$  上的矩阵 ,  且  $A$  可逆 ,  则必存在  $\mathbb{F}$  中的数  $a_{0}, a_{1}, \cdots, a_{n-1}$,  使得  $A^{-1}=a_{0} I+a_{1} A+a_{2} A^{2}+\cdots+$ $a_{n-1} A^{n-1}$.

\begin{enumerate}
  \setcounter{enumi}{2}
  \item (15  分 ) $A=\left(\begin{array}{cccc}1 & 1 & 0 & 0 \\ 0 & 1 & 0 & 2 \\ 0 & 0 & 1 & 2 \\ 0 & 0 & 0 & -1\end{array}\right)$.
\end{enumerate}
(1)  求  $A$  的最小多项式 .

(2) 求  $A^{15}$.

(3)  求  $A$  的  Jordan  标准型 .

(4)  定义  $\mathbb{Q}[A]=\left\{\sum_{i=1}^{n} a_{i} A^{i}, a_{i} \in \mathbb{Q}\right\}$,  求这个线性空间的维数 .

\begin{enumerate}
  \setcounter{enumi}{3}
  \item (15  分 ) $f\left(x_{1}, x_{2}, x_{3}\right)=x_{1}^{2}+x_{2}^{2}+x_{3}^{2}+4 x_{1} x_{2}+4 x_{2} x_{3}+4 x_{3} x_{1}$.
\end{enumerate}
(1)  求  $f\left(x_{1}, x_{2}, x_{3}\right)=X^{\mathrm{T}} A X$  的矩阵  $A$  的特征值和特征向量 .

(2)  若  $A=C D C^{\mathrm{T}}$,  其中  $C$  为正交矩阵 , $D$  为对角矩阵 ,  求  $C, D$.

(3)  在单位球  $x_{1}^{2}+x_{2}^{2}+x_{3}^{2}=1$  上求  $f\left(x_{1}, x_{2}, x_{3}\right)$  的最大值和最小值 .

(1) $\operatorname{dim} \operatorname{Hom}(W, U)=r s$. (2)  末知 .

(3)  末知 .

\begin{enumerate}
  \setcounter{enumi}{5}
  \item (15  分 )  空间中从同一点出发的四个向量共面的充要条件是  $[a, b, c]-[b, c, d]+[c, d, a]-[d, a, b]=0$,  其中  $[x, y, z]$  表示混合积 .

  \item (15  分 )  到两条异面直线的距离相等的点的集合是什么图形 ?

  \item (20  分 )  设椭球面的中心是  $O$.  证明 

\end{enumerate}
(1)  平行的平面束害椭球面得到的割线也是椭圆 ,  并且这些平行的椭圆的中心点在同一直线上 .

(2)  做固定点到椭球面的切线 ,  其切点的集合在同一个平面上 .

(3)  设  $P, Q$  是椭球面外  2  个点 , $Q$  在  $O P$  上 ,  过  $P$  和  $Q$  分别作椭球面的切线 ,  则切点所在的两个平面平行 ,  并且这两个平面截椭球面所得的椭圆的中心与  $O$  共线 . 1. (1)  错 .  如  $A=\left(\begin{array}{lllll}0 & 0 & 1 & 0 & 0 \\ 0 & 1 & 0 & 0 & 0 \\ 0 & 0 & 0 & 0 & 1 \\ 0 & 0 & 0 & 1 & 0 \\ 1 & 0 & 0 & 0 & 0\end{array}\right)$.

(2)  当  $A$  是方阵时 , $A X=0$  解唯一  $\Longrightarrow|A| \neq 0 \Longrightarrow A x=b$  解唯一 ;  当  $A$  不是方阵时 ,  结论不对 .

(3)  对 . $W=0$  时 ,  取  $\phi=0 . W \neq 0$  时 ,  取  $W$  的一组基  $\beta_{1}, \ldots, \beta_{r}, r \leq 5, F^{6}$  的一组基  $\alpha_{1}, \ldots, \alpha_{6}$,  取  $\phi\left(x_{1} \alpha_{1}+\cdots+x_{6} \alpha_{6}\right)=x_{1} \beta_{1}+\cdots+x_{r} \beta_{r}$  即可 .

(4)  错 .  考虑  $A$  在某组基下的矩阵为  $\left(\begin{array}{rr}1 & 1 \\ -1 & -1\end{array}\right)$.

(5) $r\left(A^{\mathrm{H}} A\right)=r(A)$,  其中  $A^{\mathrm{H}}$  表示矩阵  $A$  的转置共轭矩阵 .  再注意到  $\mathrm{Ker} \sigma \subseteq \mathrm{Ker} \sigma^{*} \sigma$,  于是结论正确 .

(6) 对 .  取  $W$  的一组基 ,  然后将其扩充为  $V$  的一组基  $\beta_{1}, \ldots, \beta_{r}, \beta_{r+1}, \ldots, \beta_{n}$.  假设  $\beta_{r+1}, \ldots, \beta_{n}$  不在  $W$  中 ,  取  $f\left(x_{1} \beta_{1}+\cdots+x_{n} \beta_{n}\right)=x_{r+1} \beta_{r+1}+\cdots+x_{n} \beta_{n}$  即可 .

(7)  对 .  对于任意线性变换  $\mathcal{A}$,  存在一组基使  $\mathcal{A}$  在这组基下的矩阵为上三角矩阵 .

(8)  对 . $f(x)=x^{n}+2$.

(9)  对 . 最小多项式为互素一次因式的乘积 .

(10)  对 .  设  $f(x)$  为  $A$  的特征多项式 ,  则  $f(A)=A^{n}-\operatorname{tr}(A) A^{n-1}+\cdots+(-1)^{n}|A| E=0$,  稍加变形得结论 .

\begin{enumerate}
  \setcounter{enumi}{2}
  \item (1) $|\lambda E-A|=(\lambda-1)^{2}(\lambda-1)(\lambda+1)=(\lambda-1)^{3}(\lambda+1)$.  又因为  $\operatorname{rank}(E-A)=2$,  因此  $A$  的  Jordan  标准型   为  $\left(\begin{array}{cccc}1 & & & \\ & -1 & & \\ & & 1 & 1 \\ & & & 1\end{array}\right) . A$  的最小多项式为  $m(x)=(x-1)^{2}(x+1)$.
\end{enumerate}
(2)  由题可设  $x^{15}=m(x) q(x)+a x^{2}+b x+c$,  在上式及其求导后的式子代入  $m(x)$  的根可得 :
$$
\left\{\begin{array}{l}
a+b+c=1 \\
a-b+c=-1 \\
2 a+b=15
\end{array} \Longrightarrow\left\{\begin{array}{l}
a=7 \\
b=1 \\
c=-7
\end{array}\right\}\right.
$$
(3)  见  (1).

(4)  首先有  $A^{3}-A^{2}-A+E=0$,  又因为对任意  $q(A) \in \mathbb{Q}[A]$  做带余除法可知  $q(A)$  能用  $E, A, A^{2}$  线性表示 ,  从而维数小于等于  3 ,  而  $E, A, A^{2}$  是线性无关的 ,  于是  $\mathbb{Q}[A]$  的维数为  3 . 3. (1) $A=\left(\begin{array}{rrr}1 & 2 & 2 \\ 2 & 1 & 2 \\ 2 & 2 & 1\end{array}\right), A$  的特征值为  $-1$ ( 二重根  $), 5($  单根  $),-1$  对应的两个特征向量 : $\left(\begin{array}{r}1 \\ -1 \\ 0\end{array}\right),\left(\begin{array}{r}1 \\ -2\end{array}\right)$. 5  对应的一个特征向量 : $\left(\begin{array}{l}1 \\ 1 \\ 1\end{array}\right)$.

(2) $C=\left(\begin{array}{rrr}\frac{1}{\sqrt{2}} & \frac{1}{\sqrt{6}} & \frac{1}{\sqrt{3}} \\ -\frac{1}{\sqrt{2}} & \frac{1}{\sqrt{6}} & \frac{1}{\sqrt{3}} \\ 0 & -\frac{2}{\sqrt{6}} & \frac{1}{\sqrt{3}}\end{array}\right) \cdot C^{\mathrm{T}} A C=\left(\begin{array}{rrr}-1 & 0 & 0 \\ 0 & -1 & 0 \\ 0 & 0 & 5\end{array}\right)=D$.

(3)  令  $X=C Y$,  则原问题等价于求  $-y_{1}^{2}-y_{2}^{2}+5 y_{3}^{2}$  在  $y_{1}^{2}+y_{2}^{2}+y_{3}^{2}=1$  上的最大最小值 .  显然 ,  最大值为  5 ,  最小值为  $-1$.

\begin{enumerate}
  \setcounter{enumi}{4}
  \item (1)  设  $\alpha_{1}, \alpha_{2}, \cdots, \alpha_{r}$  为  $W$  的一组基 . $\beta_{1}, \beta_{2}, \cdots, \beta_{s}$  为  $U$  的一组基 .
\end{enumerate}
$$
\text { 设 } f\left(\alpha_{1}, \alpha_{2}, \cdots, \alpha_{r}\right)=\left(\beta_{1}, \beta_{2}, \cdots, \beta_{s}\right)\left(\begin{array}{rrrr}
a_{11} & a_{12} & \cdots & a_{1 r} \\
a_{21} & a_{22} & \cdots & a_{2 r} \\
\vdots & \vdots & & \vdots \\
a_{s 1} & a_{s 2} & \cdots & a_{s r}
\end{array}\right) \text {, 则 } W \text { 到 } U \text { 的一个线性映射 }
$$
$f$  完全由  $f\left(\alpha_{1}\right), f\left(\alpha_{2}\right), \cdots, f\left(\alpha_{r}\right)$  唯一确定 ,  从而又由  $A$  唯一确定 .  可以证明  $\operatorname{Hom}(\mathrm{W}, \mathrm{U}) \cong M_{s \times r}(F)$,  从   而  $\operatorname{dim} \operatorname{Hom}(\mathrm{W}, \mathrm{U})=\operatorname{dim} M_{s \times r}(F)=r s$.

\begin{enumerate}
  \setcounter{enumi}{5}
  \item  必要性 :  若  $a, b, c, d$  共面 ,  则混合积  $[a, b, c]=[b, c, d]=[c, d, a]=[d, a, b]=0$.  充分性 :  设  $a=\left(a_{1}, a_{2}, a_{3}\right), b=\left(b_{1}, b_{2}, b_{3}\right), c=\left(c_{1}, c_{2}, c_{3}\right), d=\left(d_{1}, d_{2}, d_{3}\right)$,  则 
\end{enumerate}
$$
\begin{aligned}
\left|\begin{array}{lll}
a_{1} & b_{1} & c_{1} \\
a_{2} & b_{2} & c_{2} \\
a_{3} & b_{3} & c_{3}
\end{array}\right|-\left|\begin{array}{lll}
b_{1} & c_{1} & d_{1} \\
b_{2} & c_{2} & d_{2} \\
b_{3} & c_{3} & d_{3}
\end{array}\right|+\left|\begin{array}{ccc}
c_{1} & d_{1} & a_{1} \\
c_{2} & d_{2} & a_{2} \\
c_{3} & d_{3} & a_{3}
\end{array}\right|-\left|\begin{array}{ccc}
d_{1} & a_{1} & b_{1} \\
d_{2} & a_{2} & b_{2} \\
d_{3} & a_{3} & b_{3}
\end{array}\right|=\left|\begin{array}{cccc}
-1 & -1 & -1 & -1 \\
a_{1} & b_{1} & c_{1} & d_{1} \\
a_{2} & b_{2} & c_{2} & d_{2} \\
a_{3} & b_{3} & c_{3} & d_{3}
\end{array}\right|=0 \\
& \Longrightarrow\left|\begin{array}{cccc}
1 & 0 & 0 \\
a_{1} & b_{1}-a_{1} & c_{1}-a_{1} & d_{1}-a_{1} \\
a_{2} & b_{2}-a_{2} & c_{2}-a_{2} & d_{2}-a_{2} \\
a_{3} & b_{3}-a_{3} & c_{3}-a_{3} & d_{3}-a_{3}
\end{array}\right|=0
\end{aligned}
$$
 于是  $a, b, c, d$  共面 .

\begin{enumerate}
  \setcounter{enumi}{6}
  \item  以异面直线中的一条为  $z$  轴 ,  以它们的公垂线为  $x$  轴 ,  设另一条直线与  $x$  轴正半轴交于点  $(d, 0,0)$.  设其方程为 
\end{enumerate}
$$
\left\{\begin{aligned}
y \cos \theta-z \sin \theta &=0 \quad, 0<\theta<\pi \\
x &=d
\end{aligned}\right.
$$
 设  $\left(x_{1}, y_{1}, z_{1}\right)$  到两异面直线距离相等 ,  则 
$$
\begin{aligned}
\sqrt{x_{1}^{2}+y_{1}^{2}} &=\sqrt{\left(x_{1}-d\right)^{2}+\left(y_{1} \cos \theta-z_{1} \sin \theta\right)^{2}} \\
x_{1}^{2}+y_{1}^{2} &=x_{1}^{2}-2 x_{1} d+d^{2}+y_{1}^{2} \cos ^{2} \theta+z_{1}^{2} \sin ^{2} \theta-2 y_{1} z_{1} \cos \theta \sin \theta \\
0 &=-2 x_{1} d+d^{2}-y_{1}^{2} \sin ^{2} \theta+z_{1}^{2} \sin ^{2} \theta-2 y_{1} z_{1} \cos \theta \sin \theta
\end{aligned}
$$
 又因为 
$$
\left|\begin{array}{cc}
-\sin ^{2} \theta & -\cos \theta \sin \theta \\
-\cos \theta \sin \theta & \sin ^{2} \theta
\end{array}\right|=-\sin ^{4} \theta-\sin ^{2} \cos ^{2}=-\sin ^{2} \theta<0
$$
 故  $(*)$  对应矩阵的特征值为一正一负一零 ,  从而图形为马鞍面 .

\begin{enumerate}
  \setcounter{enumi}{7}
  \item (1)  以平行平面束中的一个平面为  $x O y$  平面建立空间直角坐标系 ,  则平行平面束中的任一直线的方程可表示为  $z=c, c \in \mathbb{R}$.  设椭球面的方程为  $\left(\begin{array}{lll}x & y & z\end{array}\right) A\left(\begin{array}{l}x \\ y \\ z\end{array}\right)+2 \delta^{\mathrm{T}}\left(\begin{array}{l}x \\ y \\ z\end{array}\right)+d=0$.  其中  $A$  为对称实矩阵 ,  由于是椭圆方程  $A$  有三个正的实特征值 ,  椭球面与  $z=c$  的交线为 
\end{enumerate}
$$
\left\{\begin{array}{c}
z=c \\
\left(\begin{array}{lll}
x & y & z
\end{array}\right) A\left(\begin{array}{l}
x \\
y \\
z
\end{array}\right)+2 \delta^{\mathrm{T}}\left(\begin{array}{l}
x \\
y \\
z
\end{array}\right)+d=0
\end{array}\right.
$$
 即 
$$
\left\{\begin{array}{r}
z=c \\
a_{11} x^{2}+2 a_{12} x y+a_{22} y^{2}+2 a_{13} c x+2 a_{23} c y+2 a_{1} x+2 b_{1} y+2 c_{1} z+d=0
\end{array}\right.
$$
 实对称矩阵  $A$  的二阶顺序主子式  $\left|\begin{array}{ll}a_{11} & a_{12} \\ a_{21} & a_{22}\end{array}\right|>0$,  从而割线均是椭圆型曲线 ,  结合图像知为椭圆 .

 椭圆的中心点满足方程  $\left\{\begin{aligned} z &=c \\ a_{11} x+a_{12} y+a_{13} c+a_{1} &=0 \\ a_{12} x+a_{22} y+a_{23} c+b_{1} &=0 \end{aligned}\right.$,

 从而椭圆的中心点均在直线  $\left\{\begin{array}{l}a_{11} x+a_{12} y+a_{13} z+a_{1}=0 \\ a_{12} x+a_{22} y+a_{23} z+b_{1}=0\end{array}\right.$  上 .

(2)  建立合适的直角坐标系使得椭球面的方程为  $\frac{x^{2}}{a^{2}}+\frac{y^{2}}{b^{2}}+\frac{z^{2}}{c^{2}}=1$,  设  $\left(x_{1}, y_{1}, z_{1}\right)$  为其上一点 ,  则过该点的切   平面方程为  $\frac{x_{1} x}{a^{2}}+\frac{y_{1} y}{b^{2}}+\frac{z_{1} z}{c^{2}}=1$.  设  $\left(x_{1}, y_{1}, z_{1}\right)$  同时也为过椭球外一点  $\left(p_{1}, p_{2}, p_{3}\right)$  作切线而得的切点 ,  则  $\frac{x_{1} p_{1}}{a^{2}}+\frac{y_{1} p_{2}}{b^{2}}+\frac{z_{1} p_{3}}{c^{2}}=1$.  于是过点  $\left(p_{1}, p_{2}, p_{3}\right)$  作切线所得切点均在平面  $\frac{p_{1} x}{a^{2}}+\frac{p_{2} y}{b^{2}}+\frac{p_{3} z}{c^{2}}=1$  上 .

(3)  椭球面方程的设法同  (2).  已求得过  $\left(p_{1}, p_{2}, p_{3}\right)$  作椭球面切线所得切点均在平面  $\frac{p_{1} x}{a^{2}}+\frac{p_{2} y}{b^{2}}+\frac{p_{3} z}{c^{2}}=1$  上 .  同理过  $\left(q_{1}, q_{2}, q_{3}\right)$  作椭球面切线所得切点均在平面  $\frac{q_{1} x}{a^{2}}+\frac{q_{2} y}{b^{2}}+\frac{q_{3} z}{c^{2}}=1$  上 .  由于点  $Q$  在  $O P$  上 ,  故  $\left(q_{1}, q_{2}, q_{3}\right)=k\left(p_{1}, p_{2}, p_{3}\right), k \neq 0$.  由于两个平面的法向量平行故两平面平行 ,  又因为平面  $\frac{p_{1} x}{a^{2}}+\frac{p_{2} y}{b^{2}}+\frac{p_{3} z}{c^{2}}=$ 0  与平面  $\frac{p_{1} x}{a^{2}}+\frac{p_{2} y}{b^{2}}+\frac{p_{3} z}{c^{2}}=1$  平行 .  于是得到了三个平行面 .  由  (1)  知它们与椭球的害线均为椭圆 ,  而且   中心共线 ,  而  $\left\{\begin{array}{c}\frac{p_{1} x}{a^{2}}+\frac{p_{2} y}{b^{2}}+\frac{p_{3} z}{c^{2}}=0 \\ \frac{x^{2}}{a^{2}}+\frac{y^{2}}{b^{2}}+\frac{z^{2}}{c^{2}}=1\end{array}\right.$  的中心为  $O$,  从而得证 .  北京大学  2012  年全国硕士研究生招生考试高代解几试题及解答     

2019.04.22

\begin{enumerate}
  \item  设  $\lambda_{1}, \cdots, \lambda_{n}$  是有理系数多项式  $g(x)$  在  $\mathbb{C}$  内所有的根 ,  对任意的  $f(x) \in \mathbb{Q}[x], \prod_{i=1}^{n} f\left(\lambda_{i}\right)$  是否一定为有理   数 ?

  \item  证明 

\end{enumerate}
$$
\operatorname{det}\left(\begin{array}{ccccccccc}
1 & 2 & 3 & 4 & 5 & \cdots & 2009 & 2010 & 2011 \\
2^{2} & 3^{2} & 4^{2} & 5^{2} & 6^{2} & \cdots & 2010^{2} & 2011^{2} & 2012^{2} \\
3^{3} & 4^{3} & 5^{3} & 6^{3} & 7^{3} & \cdots & 2011^{3} & 2012^{3} & 2012^{3} \\
4^{4} & 5^{4} & 6^{4} & 7^{4} & 8^{4} & \cdots & 2012^{4} & 2012^{4} & 2012^{4} \\
\vdots & \vdots & \vdots & \vdots & \vdots & & \vdots & \vdots & \vdots \\
2011^{2011} & 2012^{2011} & 2012^{2011} & 2012^{2011} & 2012^{2011} & \cdots & 2012^{2011} & 2012^{2011} & 2012^{2011}
\end{array}\right) \neq
$$

\begin{enumerate}
  \setcounter{enumi}{3}
  \item $n$  阶方阵  $A$  的每行每列恰有一个元素为  1  或  $-1$,  其余元素均为零 .  证明存在  $k \in \mathbb{N}^{*}$  使得  $A^{k}=E$.

  \item  对  $n$  阶矩阵  $A, B$  定义矩阵的拟乘法为  $A \circ B=\left(a_{i j} b_{i j}\right)_{n \times n}$.  证明 

\end{enumerate}
$$
\operatorname{rank}(A \circ B) \leqslant \operatorname{rank}(A) \cdot \operatorname{rank}(B)
$$

\begin{enumerate}
  \setcounter{enumi}{5}
  \item  设  $\mathcal{A}_{1}, \cdots, \mathcal{A}_{2012}$  为线性空间  $V$  上的  2012  个互不相同的线性变换 ,  问是否存在向量  $\alpha \in V$  使得 
\end{enumerate}
$$
\mathcal{A}_{1} \alpha, \cdots, \mathcal{A}_{2012} \alpha
$$
 都不相等 .

\begin{enumerate}
  \setcounter{enumi}{6}
  \item  设  $A, B \in M_{n}(\mathbb{R}), A$  正定 , $B$  对称 ,  证明存在可逆矩阵  $T$  使得  $A, B$  可同时合同对角化 .

  \item  设  $f(\alpha, \beta)=g_{1}(\alpha) g_{2}(\beta)$  是数域  $\mathbb{P}$  上的欧式空间  $V$  上的对称双线性函数 ,  且其中  $g_{1}, g_{2}$  是线性函数 .  证明   存在线性函数  $h(x)$,  以及  $k \neq 0 \in \mathbb{P}$  使得  $f(\alpha, \beta)=k h(\alpha) h(\beta)$.

  \item  证明在  $n(n \geqslant 2)$  维欧式空间中两两夹角为钝角的向量个数的最大值为  $n+1$.

\end{enumerate}
$9 .$

\begin{enumerate}
  \setcounter{enumi}{10}
  \item (20  分 )  已知  $a^{2}+b^{2}+c^{2} \neq 0$,  讨论  $2 x^{2}+3 y^{2}=a z^{2}+2 b z+c$  的形状 .

  \item  已知  $\mathbb{R}^{3}$  中的正交变换 

\end{enumerate}
$$
\left[\begin{array}{l}
x^{\prime} \\
y^{\prime} \\
z^{\prime}
\end{array}\right]=\left[\begin{array}{rrr}
\frac{2}{3} & -\frac{1}{3} & \frac{2}{3} \\
\frac{2}{3} & \frac{2}{3} & -\frac{1}{3} \\
-\frac{1}{3} & \frac{2}{3} & \frac{2}{3}
\end{array}\right]\left[\begin{array}{l}
x \\
y \\
z
\end{array}\right]
$$
 是旋转变换 ,  试求旋转轴的方向向量及旋转角度 . 1.  因为  $\prod_{i=1}^{n} f\left(x_{i}\right)$  是  $\mathbb{Q}$  上的  $n$  元对称多项式 ,  根据对称多项式基本定理 ,  存在  $\mathbb{Q}$  上的  $n$  元多项式  $h$  使得  $h\left(\sigma_{1}, \cdots, \sigma_{n}\right)=\prod_{i=1}^{n} f\left(x_{i}\right)$,  其中  $\sigma_{i}$  分别为  $x_{1}, x_{2}, \ldots, x_{n}$  的初等对称多项式 .  当  $x_{i}=\lambda_{i}, i=1,2, \ldots, n$  时 ,  由根与系数的关系知  $\sigma_{i} \in \mathbb{Q}, i=1,2, \ldots, n$,  于是  $\prod_{i=1}^{n} f\left(\lambda_{i}\right) \in \mathbb{Q}$.

\begin{enumerate}
  \setcounter{enumi}{2}
  \item  矩阵的行列式展开式告诉我们 :  矩阵的行列式为矩阵中元素进行有限次加减乘法后所得的结果 ,  因此先将矩   阵中的元素都模  2  取余再计算行列式得到的数与先按原矩阵元素计算行列式再模  2  取余所得的数在模  2  的   意义下相等 .  将题中矩阵的每个元素都模  2  取余后所得矩阵的次对角线下方元素均变为了  0 ,  对角线上的元   素全是  1 ,  不用关心对角线上方的元素是什么就知道这个时候最后的结果一定是模  2  余  1 ,  从而原矩阵的行   列式不为  0 .
\end{enumerate}
 注   上述证明中用到了初等数论的知识 ,  文字虽多 ,  但想法很简单 .  如果不能理解就考虑行列式的奇偶性 ,  同上   面的解法本质上是一样的 .  有兴趣的读者可以思考下怎么证明 
$$
\operatorname{det}\left(\begin{array}{cccc}
1 & 2 & \cdots & 2011 \\
2^{2} & 3^{2} & \cdots & 2012^{2} \\
\vdots & \vdots & \ddots & \vdots \\
2011^{2011} & 2012^{2011} & \cdots & 4021^{2011}
\end{array}\right) \neq 0 .
$$

\begin{enumerate}
  \setcounter{enumi}{3}
  \item  因为  $|\operatorname{det} A|=1$,  故  $A$  为可逆矩阵 .  设每行每列恰有一个元素为  1  或  $-1$,  其余元素均为  0  的  $n$  阶方阵组   成的集合为  $S$,  则  $A \in S$.  任取  $B \in S, A B$  可以看作对  $B$  的行做一个置换 ,  并且对某些行倍乘  $-1$  而得的矩   阵 ,  从而  $A B \in S$,  于是  $A, A^{2}, A^{3}, \ldots, A^{n}, \ldots$  均属于  $S$.  而  $S$  是有限集 ,  故存在  $p>q \in \mathbb{N}$  使得  $A^{p}=A^{q}$,  所以  $A^{p-q}=E$.

  \item  证法一 :  利用一点点张量的知识 ,  先来考虑  $A, B$  的张量积 

\end{enumerate}
$$
A \otimes B=\left(\begin{array}{cccc}
a_{11} B & a_{12} B & \cdots & a_{1 n} B \\
a_{21} B & a_{22} B & \cdots & a_{2 n} B \\
\vdots & \vdots & & \vdots \\
a_{n 1} B & a_{n 2} B & \cdots & a_{n n} B
\end{array}\right)
$$
 那么有  $\operatorname{rank}(A \otimes B)=\operatorname{rank}(A) \cdot \operatorname{rank}(B)$,  由于  $A \circ B$  正好是  $A \otimes B$  的主子式 ,  故 
$$
\operatorname{rank}(A \circ B) \leqslant \operatorname{rank}(A \otimes B)=\operatorname{rank}(A) \cdot \operatorname{rank}(B)
$$
 证法二 :  当  $\operatorname{rank}(A)=0$  时结论成立 .  当  $\operatorname{rank}(A)=1$  时 ,  可设  $A=\left(c_{1}, c_{2}, \ldots, c_{n}\right)^{\mathrm{T}}\left(d_{1}, d_{2}, \ldots, d_{n}\right)$,  则  $A \circ B=\operatorname{diag}\left(c_{1}, c_{2}, \ldots, c_{n}\right) B \operatorname{diag}\left(d_{1}, d_{2}, \ldots, d_{n}\right)$,  则  $\operatorname{rank}(A \circ B) \leqslant \operatorname{rank}(B)=\operatorname{rank}(A) \cdot \operatorname{rank}(B)$.  当  $\operatorname{rank}(A)=m>0$  时 , $A$  可以表示为  $m$  个秩为  1  的矩阵的和 ,  故可设  $A=A_{1}+A_{2}+\cdots+A_{m}, \operatorname{rank}\left(A_{i}\right)=$ $1, i=1,2, \ldots, m$,  则 
$$
\operatorname{rank}(A \circ B) \leqslant \sum_{i=1}^{m} \operatorname{rank}\left(A_{i} \circ B\right) \leqslant m \cdot \operatorname{rank}(B)=\operatorname{rank}(A) \cdot \operatorname{rank}(B)
$$

\begin{enumerate}
  \setcounter{enumi}{5}
  \item  存在向量  $\alpha \in V$  使得  $\mathcal{A}_{1} \alpha, \cdots, \mathcal{A}_{2012} \alpha$  互不相等 .  首先注意到  $\mathcal{A}_{i} \alpha=\mathcal{A}_{j} \alpha \Longleftrightarrow \alpha \in \operatorname{ker}\left(\mathcal{A}_{i}-\mathcal{A}_{j}\right)$.  而   注   上述证明中用到了一个结论 :  线性空间  $V$  的任意有限个真子空间的并集不能填满  $V$.  很多 《 高等代数 》 教   材上有这个课后习题 ,  推荐阅读蓝以中老师的 《 高等代数学习指南 》 第  150  页例  $2.6$  中提供的证明方法 .

  \item  因为  $A$  是正定矩阵 ,  于是存在实可逆矩阵  $P$,  使得  $P^{\mathrm{T}} A P=E$.  此时  $P^{\mathrm{T}} B P$  是实对称矩阵 ,  故存在正交矩阵  $Q$,  使得  $Q^{\mathrm{T}} P^{\mathrm{T}} B P Q=\operatorname{diag}\left\{\lambda_{1}, \lambda_{2}, \ldots, \lambda_{n}\right\}$,  其中  $\lambda_{i}, i=1,2, \ldots, n$  为  $P^{\mathrm{T}} B P$  的特征值 .  又  $(P Q)^{\mathrm{T}} A(P Q)=$ $Q^{\mathrm{T}} E Q=E$,  故取  $T=P Q$  就能将  $A, B$  同时合同对角化 .

  \item  当  $g_{1}=0$  或者  $g_{2}=0$,  取  $h=0, k=1$  即可 .

\end{enumerate}
 当  $g_{1} \neq 0$  且  $g_{2} \neq 0$  时 ,  存在  $\alpha_{0} \in V$  使得  $g_{1}\left(\alpha_{0}\right) \neq 0$.  由于  $f$  是对称双线性函数 ,  故对于  $\forall \beta \in V$,
$$
\begin{aligned}
g_{1}\left(\alpha_{0}\right) g_{2}(\beta) &=f\left(\alpha_{0}, \beta\right)=f\left(\beta, \alpha_{0}\right)=g_{1}(\beta) g_{2}\left(\alpha_{0}\right) \\
\Longrightarrow g_{2}(\beta) &=\frac{g_{2}\left(\alpha_{0}\right)}{g_{1}\left(\alpha_{0}\right)} g_{1}(\beta) \\
\Longrightarrow g_{2} &=\frac{g_{2}\left(\alpha_{0}\right)}{g_{1}\left(\alpha_{0}\right)} g_{1}
\end{aligned}
$$
 此时取  $k=\frac{g_{2}\left(\alpha_{0}\right)}{g_{1}\left(\alpha_{0}\right)} \neq 0, h=g_{1}$.

\begin{enumerate}
  \setcounter{enumi}{8}
  \item  我们需要证明两件事情 :  第一 ,  在  $n$  维欧式空间中存在  $n+1$  个两两夹角为钝角的向量 ;  第二 ,  在  $n$  维欧式   空间中不存在  $n+2$  个两两夹角为钝角的向量 .
\end{enumerate}
 先用数学归纳法证明第一件事情 ,  因为  $n$  维欧式空间同构于  $\mathbb{R}^{n}$,  于是只需要证明  $\mathbb{R}^{n}$  的情形 .

 当  $n=2$  时 ,  存在向量  $(1,0)^{\mathrm{T}},(-1 / 2, \sqrt{3} / 2)^{\mathrm{T}},(-1 / 2,-\sqrt{3} / 2)^{\mathrm{T}}$,  它们两两夹角为钝角 .  假设结论对维数不   超过  $n$  的欧式空间均成立 ,  下面考虑  $\mathbb{R}^{n+1}$.  设  $\alpha_{1}, \alpha_{2}, \ldots, \alpha_{n+1}$  是  $\mathbb{R}^{n}$  中  $n+1$  个两两夹角为钝角的向量 ,  构造  $\mathbb{R}^{n+1}$  中向量  $\tilde{\alpha}_{i}$,  它的前  $n$  个分量与  $\alpha_{i}$  相同 ,  最后一个分量取为  $-\varepsilon,(\varepsilon>0$  待定  $), i=1,2, \ldots, n+1$.  此时将有  $\tilde{\alpha}_{i}^{\mathrm{T}} \tilde{\alpha}_{j}=\alpha_{i}^{\mathrm{T}} \alpha_{j}+\varepsilon^{2}, 1 \leqslant i, j \leqslant n+1$,  可以把  $\varepsilon$  取得充分小 ,  使前面式子的值始终小于  0 ,  然后就   有  $\tilde{\alpha}_{i}, \tilde{\alpha}_{j}$  的夹角大于  $\pi / 2$  小于等于  $\pi$,  而实际上应该是严格小于  $\pi$,  否则将有  $\tilde{\alpha}_{i}$  与  $\tilde{\alpha}_{j}$  线性相关 ,  那么  $\alpha_{i}$  与  $\alpha_{j}$  线性相关 ,  那么它们的夹角将不是针角 ,  矛盾 .  最后我们再取  $\tilde{\alpha}_{n+2}=(0,0, \ldots, 0,1)$,  直接计算就会发 

 最后用反证法来说明第二件事情 .  若  $n$  维欧式空间  $V$  中存在  $n+2$  个两两夹角为钝角的向量 ,  设为 

 注   这题其实是王莒芳与石生明修订的北大第四版 《 高等代数 》 第  421  页总习题中的第七题第二问 .  第一部分  i.  若  $a c-b^{2}>0$,  则为虚椭球面 .

ii.  若  $a c-b^{2}=0$,  则为单个点 .

iii.  若  $a c-b^{2}<0$,  则为椭球面 .

\begin{enumerate}
  \setcounter{enumi}{11}
  \item  记  $A=\left[\begin{array}{rrr}\frac{2}{3} & -\frac{1}{3} & \frac{2}{3} \\ \frac{2}{3} & \frac{2}{3} & -\frac{1}{3} \\ -\frac{1}{3} & \frac{2}{3} & \frac{2}{3}\end{array}\right]$,  则  $|\lambda E-A|=(\lambda-1)\left(\lambda^{2}-\lambda+1\right)$.  因为特征值  $\lambda=1$  对应的一个特征向   量为  $n=\left[\begin{array}{c}\frac{1}{\sqrt{3}} \\ \frac{1}{\sqrt{3}} \\ \frac{1}{\sqrt{3}}\end{array}\right]$,  故不动直线的方向向量为  $k\left[\begin{array}{l}1 \\ 1 \\ 1\end{array}\right], k \neq 0$.  因为该变换将  $a=\left[\begin{array}{l}1 \\ 0 \\ 0\end{array}\right]$  变为  $b=\left[\begin{array}{r}\frac{2}{3} \\ \frac{2}{3} \\ -\frac{1}{3}\end{array}\right]$,  故  $|\cos \theta|=\left|\frac{(a \times n) \cdot(b \times n)}{|(a \times n)||(b \times n)|}\right|=\frac{1}{2}$,  因此  $\theta=\frac{\pi}{3}$.
\end{enumerate}
 注   回忆版中的那个矩阵不是正交矩阵 ,  原题中的矩阵应该不是那样 ,  因此我自行选择了一个与回忆版中矩阵较   为相似的矩阵作为题中矩阵 .  相反的问题见斤维声老师的 《 解析几何 》 第三版第  143  页第  7  题 .  北京大学  2013  年全国硕士研究生招生考试高代解几试题及解答 

   

2019.04.07

\begin{enumerate}
  \item $\alpha=2013+2013 \frac{1}{10 \pi}$  是有理多项式  $g(x)$  的一个根 ,  证明  $\beta=2013+2013 \frac{1}{\frac{1}{100}} \mathrm{e}^{\frac{2 \pi i}{s 3}}$  是  $g(x)$  的一个复根 .

  \item  矩阵  $A$  的特征多项式为  $f(x)=(x-1)^{2}(x+3)^{2}$,  极小多项式  $m(x)=(x-1)^{2}(x+3)$,  求  $A$  的  Jordan  标   准型 .

  \item (1)  若  $A$  可逆 ,  对于任意  $B, A B=B A$,  证明  $\operatorname{det}\left(\begin{array}{ll}A & B \\ B & A\end{array}\right)=\operatorname{det}\left(A^{2}-B^{2}\right)$.

\end{enumerate}
(2)  如果  $A$  为数域  $K$  上不可逆矩阵 ,  对于任意  $K$  上矩阵  $B, A B=B A,(1)$  中结论是否成立 ?

(3)  如果  $A$  可逆 , $A B=B A$  不成立 , (1)  中结论是否成立 ?

\begin{enumerate}
  \setcounter{enumi}{4}
  \item  形如  $\left(\begin{array}{ll}a & b \\ b & c\end{array}\right)$  的实矩阵形成一个线性空间  $V$,  定义  $V \times V \rightarrow \mathbb{R}$  的映射为 
\end{enumerate}
$$
(A, B)=\frac{1}{2}(\operatorname{det}(A+B)-\operatorname{det}(A)-\operatorname{det}(B)) .
$$
(1)  证明这是一个对称双线性映射 .

%\includegraphics[max width=\textwidth]{2022_04_18_33b622a7abd81c227674g-065}

(3)  对于上述  $M$,  求  $f(X)=\frac{X^{\prime} M X}{X^{\prime} X}, \quad(X \neq 0)$  的最大值与最小值  ( 注 :  用数学分析方法不给分 ).

\begin{enumerate}
  \setcounter{enumi}{5}
  \item  一个关于投影映射的问题 ,  题目定义了到子空间  $\mathrm{W}$  上的投影算子 : $T_{\mathrm{w}}(\alpha)=\sum_{i=1}^{r}\left(\alpha, \alpha_{i}\right) \alpha_{i}, \quad \alpha_{1}, \ldots, \alpha_{r}$  是  $\mathrm{W}$  上的标准正交基 .  证明 
\end{enumerate}
(1)  对于任意  $\beta \in W,\left|\alpha-T_{\mathrm{w}}(\alpha)\right| \leq|\alpha-\beta|$.

(2) $A$  是投影映射的充要条件是  $A^{2}=A$  且  $A$  为对称映射 .

(3)  任意对称映射可以表示为  $f(\alpha)=\sum_{i=1}^{r} \lambda_{i} T_{\mathrm{w}_{i}}(\alpha), \mathrm{W}_{i}$  为子空间 .

\begin{enumerate}
  \setcounter{enumi}{6}
  \item  定义矩阵  $A$  的双曲余弦  $\cosh (A)=E+\frac{A^{2}}{2 !}+\frac{A^{4}}{4 !}+\cdots$,  是否存在二阶复矩阵  $A$,  使得  $\cosh (A)=\left(\begin{array}{c}1 \\ 0\end{array}\right) 13$.

  \item  末知 .

  \item $x O y$  平面上有一个过原点的圆 ,  它的圆心在  $x$  轴正半轴 , $A$  是圆上一动点 , $B$  是  $z$  轴上的一动点 ,  且  $|O A|=k|O B|, k$  为定值 ,  求直线  $A B$  确定的曲面方程 .

  \item  根据二次曲线方程的参数讨论其曲线类型 ,  具体方程已忘 .

  \item (20  分 )  给定了一个雉面 ,  还有一个带两个参数的平面 ,  讨论截口形状 ,  并画图 ,  具体内容已忘 . 1.  令  $f(x)=(x-2013)^{106}-2013$,  则  $\alpha$  是  $f(x)$  的一个根 .  注意到  $2013=3 \times 671$,  取  $p=3$  可以由  Eisenstein  判别法知  $y^{106}-2013$  在  $\mathbb{Q}[y]$  中不可约 ,  于是  $f(x)$  在  $\mathbb{Q}[x]$  中不可约 ,  于是  $f(x) \mid g(x)$,  结合  $f(\beta)=0$  得  $g(\beta)=0 .$

  \item $A$  的  Jordan  标准型为  $\left(\begin{array}{llll}-3 & & & \\ & -3 & & \\ & 1 & \\ & 1 & 1\end{array}\right)$.

  \item (1) $\left|\begin{array}{cc}A & B \\ B & A\end{array}\right|=\left|\begin{array}{cc}A & B \\ 0 & A-B A^{-1} B\end{array}\right|=|A|\left|A-B A^{-1} B\right|=\left|A^{2}-A B A^{-1} B\right|=\left|A^{2}-B^{2}\right|$.

\end{enumerate}
(2)  若  $A$  不可逆 , $A B=B A$,  则结论仍成立 .

$\forall t \in K,(A+t E) B=B(A+t E)$,  使  $A+t E$  不可逆的  $t$  只有有限个 .  对于其余的无限个  $t$  而言 , $A+t E$  可逆 ,  由  (1)  知  $\left|\begin{array}{cc}A+t E & B \\ B & A+t E\end{array}\right|=\left|(A+t E)^{2}-B^{2}\right|$,  等号两端均是  $t$  的多项式 ,  从而上式对  $t \in K$  成立 ,  取  $t=0$  得结论 .

(3)  结论不成立 .  例如取  $A=\left(\begin{array}{ll}0 & 1 \\ 1 & 0\end{array}\right), B=E_{21}$.

\begin{enumerate}
  \setcounter{enumi}{4}
  \item (1)  设  $A=\left(\begin{array}{ll}a_{1} & a_{2} \\ a_{3} & a_{4}\end{array}\right), B=\left(\begin{array}{ll}b_{1} & b_{2} \\ b_{3} & b_{4}\end{array}\right)$,  则 
\end{enumerate}
$$
(A, B)=\frac{1}{2}\left(\left|\begin{array}{ll}
a_{1}+b_{1} & a_{2}+b_{2} \\
a_{3}+b_{3} & a_{4}+b_{4}
\end{array}\right|-\left|\begin{array}{ll}
a_{1} & a_{2} \\
a_{3} & a_{4}
\end{array}\right|-\left|\begin{array}{cc}
b_{1} & b_{2} \\
b_{3} & b_{4}
\end{array}\right|\right)=\frac{1}{2}\left(a_{1} b_{4}-a_{2} b_{3}-a_{3} b_{2}+a_{4} b_{1}\right)=(B, A)
$$
 在此基础上易验算  $\left(k_{1} A_{1}+k_{2} A_{2}, B\right)=k_{1}\left(A_{1}, B\right)+k_{2}\left(A_{2}, B\right)$, 从而上述映射为对称双线性函数 .

(2) $M=\frac{1}{2}\left(\begin{array}{ccc}0 & 1 & 0 \\ 1 & 0 & 0 \\ 0 & 0 & -2\end{array}\right)$.

(3) $|\lambda E-M|=\left|\begin{array}{ccc}\lambda & -1 / 2 & 0 \\ -1 / 2 & \lambda & 0 \\ 0 & 0 & \lambda+1\end{array}\right|=(\lambda+1)\left(\lambda^{2}-1 / 4\right) . \lambda=-1$  对应的特征向量为  $\left(\begin{array}{l}0 \\ 0 \\ 1\end{array}\right), \lambda=-1 / 2$  对应的特征向量为  $\left(\begin{array}{c}1 / \sqrt{2} \\ -1 / \sqrt{2} \\ 0\end{array}\right), \lambda=1 / 2$  对应的特征向量  $\left(\begin{array}{c}1 / \sqrt{2} \\ 1 / \sqrt{2} \\ 0\end{array}\right)$.  取  $U=\left(\begin{array}{ccc}0 & 1 / \sqrt{2} & 1 / \sqrt{2} \\ 0 & -1 / \sqrt{2} & 1 / \sqrt{2} \\ 1 & 0\end{array}\right)$,  则  $U^{\mathrm{T}} M U=\left(\begin{array}{ccc}-1 & & \\ & -1 / 2 & \\ & & 1 / 2\end{array}\right)$.  令  $X=U Y$.  则原问题转化为求 
$$
\frac{1}{2} \times \frac{-2 y_{1}^{2}-y_{2}^{2}+y_{3}^{2}}{y_{1}^{2}+y_{2}^{2}+y_{3}^{2}}
$$
 的最大最小值 ,  显然最大值为  $1 / 2$,  最小值为  $-1$.

\begin{enumerate}
  \setcounter{enumi}{5}
  \item (1)  首先注意到  $\left(\alpha-T_{\mathrm{w}}(\alpha), \alpha_{i}\right)=0, i=1,2, \ldots, n$,  于是  $\alpha-T_{\mathrm{w}}(\alpha) \in W^{\perp} . T_{\mathrm{w}}(\alpha) \in W, \beta \in W \Rightarrow$ $T_{\mathrm{w}}(\alpha)-\beta \in W$.  故 
\end{enumerate}
$$
\begin{aligned}
|\alpha-\beta|^{2} &=\left|\alpha-T_{\mathrm{w}}(\alpha)+T_{\mathrm{w}}(\alpha)-\beta\right|^{2} \\
&=\left|\alpha-T_{\mathrm{w}}(\alpha)\right|^{2}+\left|T_{\mathrm{w}}(\alpha)-\beta\right|^{2} \\
& \geq\left|\alpha-T_{\mathrm{w}}(\alpha)\right|^{2}
\end{aligned}
$$
 于是 
$$
\left|\alpha-T_{\mathrm{w}}(\alpha)\right| \leq|\alpha-\beta|
$$
(2)  必要性 : $T_{\mathrm{w}}^{2}(\alpha)=T_{\mathrm{w}}\left(T_{\mathrm{w}}(\alpha)\right)=T_{\mathrm{w}}(\alpha) \Longrightarrow T_{\mathrm{w}}^{2}=T_{\mathrm{w}} . \quad\left(\beta, T_{\mathrm{w}}(\alpha)\right)=\left(T_{\mathrm{w}}(\beta), \alpha\right)$.

 充分性 : $A^{2}=A$  表明特征值为  0  或  $1, A$  对称表明其可正交相似对角化 ,  于是  $A$  在某一组标准正交基   下的矩阵为  $\left(\begin{array}{ll}E_{s} & \\ & 0\end{array}\right)$,  从而为投影映射 .

(3) $V=V_{1} \oplus \cdots \oplus V_{s}$,  其中  $V_{i}$  为特征值  $\lambda_{i}$  对应的特征子空间 .

\begin{enumerate}
  \setcounter{enumi}{6}
  \item  不存在这样的矩阵  $A$.  假若存在 ,  则由于  $\left(\begin{array}{cc}1 & 2013 \\ 0 & 1\end{array}\right)$  的  Jordan  标准型为  $\left(\begin{array}{ll}1 & 1 \\ 0 & 1\end{array}\right)$,  从而  $A$  不能相似对   角化 .  可设  $P^{-1} A P=\left(\begin{array}{ll}\lambda & 1 \\ 0 & \lambda\end{array}\right), \operatorname{det}(P) \neq 0$.  此时  $\cosh (A)$  的特征值为  $\frac{\mathrm{e}^{\lambda}+\mathrm{e}^{-\lambda}}{2}=1 \Longrightarrow \lambda$. $\Longrightarrow \cosh (A)=P E P^{-1}=E$,  矛盾 .
\end{enumerate}
 注   矩阵函数的相关的知识在蓝以中老师的 《 高等代数简明教程 》 第二版下册第七章第四节有较详细的讲解 .

$7 .$

\begin{enumerate}
  \setcounter{enumi}{8}
  \item  设圆的方程为  $(x-a)^{2}+y^{2}=a^{2}, a>0 . A\left(x_{0}, y_{0}, 0\right), B\left(0,0, z_{0}\right)$.  直线  $A B$  上的点  $(x, y, z)$  可以表示为 
\end{enumerate}
$$
\left\{\begin{array}{l}
x=x_{0} t \\
y=y_{0} t \quad t \in \mathbb{R} \text {. 同时 } x_{0}, y_{0}, z_{0} \text { 应满足 }\left\{\begin{array}{l}
\left(x_{0}-a\right)^{2}+y_{0}^{2}=a^{2} \\
z=z_{0}-z_{0} t
\end{array} \quad \text { 当 } t \neq 0 y_{0}^{2}-k^{2} z_{0}^{2}=0\right. \\
2 a x_{0}=k^{2} z_{0}^{2} \Longrightarrow 2 a\left(\frac{x}{t}\right)=k^{2}\left(\frac{z}{1-t}\right)^{2} \Longrightarrow t=\frac{4 a x+k^{2} z^{2} \pm k|z| \sqrt{k^{2} z^{2}+8 a x}}{4 a x}
\end{array}\right.
$$

\begin{enumerate}
  \setcounter{enumi}{9}
  \item  可以试着做下后维声老师的 《 解析几何 》 第三版第  168  页的例  $2.2$
\end{enumerate}
 按参数  $\lambda$  的值讨论曲线  $\lambda x^{2}-2 x y+\lambda y^{2}-2 x+2 y+5=0$  的类型 .

\begin{enumerate}
  \setcounter{enumi}{10}
  \item  按照前人的回忆我自己编了一道题供大家参考 ,  如下 
\end{enumerate}
 设雉面的方程为  $x^{2}+y^{2}=z^{2}$,  按照参数  $a, b$  的值讨论平面  $a x+b y+z=1$  与雉面的截线的形状 .  北京大学  2014  年全国硕士研究生招生考试高代解几试题及解答     

2019.04.19

\begin{enumerate}
  \item  设  $f(x)=\prod_{i=1}^{2013}(x-i)^{2}+2014, f(x)$  在  $\mathbb{Q}[x]$  内可约吗 ?  说明理由 .

  \item  设  $M, N$  都是  $n$  阶矩阵 , $n \geqslant 2$.  如果  $M N M N$  为零矩阵 ,  那么  $N M N M$  是否也一定是零矩阵 ?  说明理由 .

  \item  除了单位矩阵 ,  是否有别的  $n$  阶  Hermit  矩阵  $M$  满足  $4 M^{5}+2 M^{3}+M=7 E_{n}$ ?  说明理由 .

  \item $V$  是有理数域  $\mathbb{Q}$  上  $n$  维线性空间 .  线性变换  $\mathcal{A}$  的最小多项式的次数是  $n$.  证明 

\end{enumerate}
(1)  存在向量  $\alpha$,  使得  $\alpha, \mathcal{A} \alpha, \ldots, \mathcal{A}^{n-1} \alpha$  是  $V$  的一组基 ;

(2)  任何与  $\mathcal{A}$  可交换的线性变换可表示为  $\mathcal{A}$  的多项式 .

\begin{enumerate}
  \setcounter{enumi}{5}
  \item $V$  是所有  $n$  阶复矩阵组成的线性空间 .  求证  $\{M N-N M \mid M, N \in V\}$  为  $V$  的子空间 ,  并求这个子空间的   维数 .

  \item  对于欧式空间  $V$  中对称线性变换  $\mathcal{A}$,  若  $\forall \alpha \in V$,  都有  $(\alpha, \mathcal{A}(\alpha)) \geqslant 0$  成立 ,  并且当且仅当  $\alpha=0$  时等号成   立 ,  则称  $\mathcal{A}$  为  “ 正的 ”.  证明 

\end{enumerate}
(1)  若线性变换  $\mathcal{A}$  是正的 ,  则  $\mathcal{A}$  可逆 ;

(2)  若线性变换  $\mathcal{B}$  是正的 , $\mathcal{A}-\mathcal{B}$  是正的 ,  则  $\mathcal{B}^{-1}-\mathcal{A}^{-1}$  是正的 ;

(3)  对于正的线性变换  $\mathcal{A}$,  总存在正的线性变换  $\mathcal{B}$,  使得  $\mathcal{A}=\mathcal{B}^{2}$.

\begin{enumerate}
  \setcounter{enumi}{7}
  \item  求单叶双曲面  $\frac{x^{2}}{a^{2}}+\frac{y^{2}}{b^{2}}-\frac{z^{2}}{c^{2}}=1$  的相互垂直的直母线的交点的轨迹 .

  \item  保距变换  $\left[\begin{array}{l}x^{\prime} \\ y^{\prime} \\ z^{\prime}\end{array}\right]=\left[\begin{array}{rrr}\frac{2}{3} & -\frac{1}{3} & \frac{2}{3} \\ \frac{2}{3} & \frac{2}{3} & -\frac{1}{3} \\ -\frac{1}{3} & \frac{2}{3} & \frac{2}{3}\end{array}\right]\left[\begin{array}{l}x \\ y \\ z\end{array}\right]$  可以看作绕不动直线旋转一个角度  $\theta$  而得到 .

\end{enumerate}
(1)  求不动直线的方向向量 ;

(2)  求旋转角  $\theta$.

\begin{enumerate}
  \setcounter{enumi}{9}
  \item  点  $A\left(a_{1}, a_{2}, a_{3}\right), B\left(b_{1}, b_{2}, b_{3}\right)$  在直线  $\frac{x+a}{2}=\frac{y+b}{2}=\frac{z}{3}$  上的投影点为  $A_{1}, B_{1}$,  求  $A_{1}, B_{1}$  的坐标以及两点   间的距离 . 1. $f(x)$  在  $\mathbb{Q}[x]$  内不可约 .  若  $f(x)$  在  $\mathbb{Q}[x]$  中可约 ,  则它能分解为两个次数都大于  1  的整系数多项式的乘积 ,  设为  $f(x)=g(x) h(x)$,  于是 
\end{enumerate}
$$
g(k) h(k)=2014, k=1,2, \ldots, 2013 .
$$
$g(k), h(k)$  应当全是正整数或者全是负整数 ,  否则与  $f(x)$  无实根矛盾 .  我们可以假设  $g(k), h(k)$  全是正整数 ,  因为如果不是这样 ,  可以通过把  $g(x), h(x)$  都乘以  $-1$  得到 .  由于  $2014=2 \times 19 \times 53$,  因此将  2014  表示为   两个正整数的乘积只有  8  种不同的表示方法 .  由抽屉原理知 ,  在  $g(k)$  的  8  个可能取值中至少有一个出现的   次数大于等于  $\frac{2013}{8}>251$,  设这个数为  $l$,  则有 
$$
\left(x-a_{1}\right)\left(x-a_{2}\right) \ldots\left(x-a_{252}\right) \mid g(x)-l,
$$
 其中  $a_{1}, a_{2}, \ldots, a_{252}$  为  $\{1,2, \ldots, 2013\}$  中互不相同的数 .  因为  $g(x)-l \neq 0$,  于是  $\exists b \in\{1,2, \ldots, 2013\}$,  使   得  $g(b)-l \neq 0$.  此时有 
$$
\left(b-a_{1}\right)\left(b-a_{2}\right) \ldots\left(b-a_{252}\right) \mid g(b)-l
$$
 而 
$$
|g(b)-l|<2014, \quad\left|\left(b-a_{1}\right)\left(b-a_{2}\right) \ldots\left(b-a_{252}\right)\right| \geqslant(126 !)^{2}>2014
$$
 矛盾 .

 注   上述证明的灵感来自一道经典的题目 :  设  $a_{1}, a_{2}, \ldots, a_{n}$  是  $n$  个互不相同的整数 .  证明  $f(x)=\prod_{i=1}^{n}\left(x-a_{i}\right)^{2}+1$  在  $\mathbb{Q}[x]$  中不可约 .

\begin{enumerate}
  \setcounter{enumi}{2}
  \item $N M N M$  不一定为  0 .  当  $M N M N=0$  时 ,  一定有  $N M N M N M=0$.  要使  $N M N M \neq 0$,  则可先假设 
\end{enumerate}
$$
N M=\left[\begin{array}{lll}
0 & 1 & 0 \\
0 & 0 & 1 \\
0 & 0 & 0
\end{array}\right]
$$
 为了使  $N M$  为前面那个矩阵 ,  可取 
$$
N=\left[\begin{array}{lll}
1 & 0 & 0 \\
0 & 1 & 0 \\
0 & 0 & 0
\end{array}\right], \quad M=\left[\begin{array}{lll}
0 & 1 & 0 \\
0 & 0 & 1 \\
0 & 0 & 0
\end{array}\right]
$$
 此时将有 
$$
M N=\left[\begin{array}{lll}
0 & 1 & 0 \\
0 & 0 & 0 \\
0 & 0 & 0
\end{array}\right], \quad M N M N=0
$$
 注   命题不成立的根本原因是矩阵乘法不满足交换律 .  按照上述解题思路 ,  我们可以举出  $M N M N M N=$ $0, N M N M N M \neq 0$  的例子 ,  对于矩阵个数更多的情况也可以举出相应的例子 .

\begin{enumerate}
  \setcounter{enumi}{3}
  \item  不存在不是单位矩阵的  Hermit  矩阵满足题中条件 .
\end{enumerate}
$$
\begin{aligned}
\phi(x) &=4 x^{5}+2 x^{3}+x-7 \\
&=(x-1)\left(4 x^{4}+4 x^{3}+6 x^{2}+6 x+7\right) \\
&=(x-1)\left(x^{2}(2 x+1)^{2}+5 x^{2}+6 x+7\right)
\end{aligned}
$$
 故  $\phi(x)$  只有一个实根  $x=1$.  而  Hermit  矩阵的特征值都为实数 ,  从而特征值均为  1 ,  又因为  Hermit  矩阵   一定可以相似对角化 ,  从而必定为单位矩阵 . 4. (1)  设  $\mathcal{A}$  的最小多项式为  $m(x)=x^{n}+a_{n-1} x^{n-1}+\cdots+a_{0}$,  因为  $\mathcal{A}$  是  $n$  维线性空间中的线性变换 ,  故  $\mathcal{A}$  的特征多项式  $f(x)$  的次数为  $n$  次 ,  又因为  $m(x) \mid f(x)$,  并且两者的首项系数相同 ,  故  $f(x)=m(x)$,  从而线性变换  $\mathcal{A}$  的不变因子为  $1,1, \ldots, 1, m(x)$.  易知矩阵 
$$
A=\left[\begin{array}{ccccc}
0 & 0 & \cdots & 0 & -a_{0} \\
1 & 0 & \cdots & 0 & -a_{1} \\
0 & 1 & \cdots & 0 & -a_{2} \\
\vdots & \vdots & & \vdots & \vdots \\
0 & 0 & \cdots & 1 & -a_{n-1}
\end{array}\right]
$$
 的不变因子也为  $1,1, \ldots, 1, m(x)$.  于是存在一组基  $\alpha_{1}, \alpha_{2}, \ldots, \alpha_{n}$,  使得 
$$
\mathcal{A}\left(\alpha_{1}, \alpha_{2}, \ldots, \alpha_{n}\right)=\left(\alpha_{1}, \alpha_{2}, \ldots, \alpha_{n}\right) A
$$
 从而  $\mathcal{A} \alpha_{1}=\alpha_{2}, \mathcal{A} \alpha_{2}=\alpha_{3}=\mathcal{A}^{2} \alpha_{1}, \ldots, \mathcal{A} \alpha_{n-1}=\alpha_{n}=\mathcal{A}^{n-1} \alpha_{1}$,  故  $\alpha_{1}, \mathcal{A} \alpha_{1}, \ldots, \mathcal{A}^{n-1} \alpha_{1}$  为  $V$  的一   组基 .

(2)  为了简化记号 ,  把  $(1)$  中的  $\alpha_{1}$  记为  $\alpha$.  设  $\mathcal{A B}=\mathcal{B} \mathcal{A}, \mathcal{B} \alpha=k_{0} \alpha+k_{1} \mathcal{A} \alpha+\cdots+k_{n-1} \mathcal{A}^{n-1} \alpha$.  令  $g(x)=k_{0}+k_{1} x+\cdots+k_{n-1} x^{n-1}$,  则 
$$
\begin{aligned}
\mathcal{B}\left(\alpha, \mathcal{A} \alpha, \ldots, \mathcal{A}^{n-1} \alpha\right) &=\left(\mathcal{B} \alpha, \mathcal{B} \mathcal{A} \alpha, \ldots, \mathcal{B} \mathcal{A}^{n-1} \alpha\right) \\
&=\left(\mathcal{B} \alpha, \mathcal{A B} \alpha, \ldots, \mathcal{A}^{n-1} \mathcal{B} \alpha\right) \\
&=\left(g(\mathcal{A}) \alpha, \mathcal{A} g(\mathcal{A}) \alpha, \ldots, \mathcal{A}^{n-1} g(\mathcal{A}) \alpha\right) \\
&=\left(g(\mathcal{A}) \alpha, g(\mathcal{A}) \mathcal{A} \alpha, \ldots, g(\mathcal{A}) \mathcal{A}^{n-1} \alpha\right) \\
&=g(\mathcal{A})\left(\alpha, \mathcal{A} \alpha, \ldots, \mathcal{A}^{n-1} \alpha\right)
\end{aligned}
$$
 因此  $\mathcal{B}=g(\mathcal{A})$.

 注   此题与王茴芳和石生明修订的北大第四版 《 高等代数 》 总习题中的第  35,37  题类似 ,  页码为  425 .  关于第一   小问多说一点 ,  上面的做法用了  $\lambda$  矩阵相关的知识 ,  但用  Jordan  标准型理论也能做出来 .  大致过程如下 :

 由于两个相似的矩阵可以看成一个线性变换在不同基下对应的矩阵 ,  因此如果我们能证明线性变换  $\mathcal{A}$  在一   个基下的矩阵  $M$  与上面的矩阵  $A$  情形 ,  就能知道基的存在性 .

 考虑  $\mathbb{Q}$  上的矩阵  $M$,  假设  $M$  的最小多项式和特征多项式相同等于  $f(x)=x^{n}+a_{n-1} x^{n-1}+\cdots+a_{0}$,  把  $M$  视作  $\mathbb{C}$  上的矩阵 ,  它在  $\mathbb{C}$  上的最小多项式和特征多项式不变  ( 原因可以参考蓝以中的 《 高等代数简明教   程 》 第二版下册最小多项式一节 ),  考虑  $M$  的  Jordan  标准型 ,  我们会发现它的  Jordan  标准型由  $M$  的特征   多项式完全刻画清楚了  ( 每个特征值对应一个阶为特征值重数的  Jordan  块 ),  而我们给的矩阵  $A$  的最小多   项式和特征多项式相同等于  $f(x)$,  因此  $M$  和  $A$  在  $\mathbb{C}$  上相似 .

 于是存在  $P \in G L_{n}(\mathbb{C})$,  使得  $P A=M P$,  若  $P \in G L_{n}(\mathbb{Q})$  则  $M$  和  $A$  在  $\mathbb{Q}$  上相似 ;  否则把  $P$  表示为  $P=P_{0}+x_{1} P_{1}+x_{2} P_{2}+\cdots+x_{r} P_{r}$,  其中  $P_{i}, 0 \leqslant i \leqslant r$  为  $\mathbb{Q}$  上矩阵 ,  无理数  $x_{1}, x_{2}, \ldots, x_{r}$  在  $\mathbb{Q}$  上线性无   关 ,  然后就有  $P_{i} A=M P_{i}, 0 \leqslant i \leqslant r$.

 设  $P(t)=P_{0}+t_{1} P_{1}+t_{2} P_{2}+\cdots+t_{r} P_{r}$,  则由  $\operatorname{det}(P) \neq 0$  知关于  $t_{1}, t_{2}, \ldots, t_{r}$  的多元多项式  $\operatorname{det}(P(t))$  不   为零 ,  因此存在  $t^{*} \in \mathbb{Q}^{r}$  使  $\operatorname{det}\left(P\left(t^{*}\right)\right) \neq 0$,  由  $P\left(t^{*}\right) A=M P\left(t^{*}\right)$  同样可得  $M$  和  $A$  在  $\mathbb{Q}$  上相似 .

 更进一步地 ,  我们用上述想法可以从矩阵的  Jordan  标准型理论导出矩阵的有理标准型理论 ,  也即是 :  对于   任意一个有理数域上的矩阵 ,  它一定相似于一个分块对角矩阵 ,  其中每一个分块矩阵形如上面的矩阵  $A$. 5.  令  $U=\{M N-N M \mid M, N \in V\}, W=\{A \in V \mid \operatorname{tr}(A)=0\}$  则  $U \subset W$.  再注意到 
$$
E_{i j}=E_{i i} E_{i j}-E_{i j} E_{i i}, \quad E_{i i}-E_{11}=E_{i 1} E_{1 i}-E_{1 i} E_{i 1}, \quad 1 \leqslant i, j \leqslant n,
$$
 这就说明  $V$  的子空间  $W$  的一组基属于  $U$,  因此若  $U$  为  $V$  的子空间 ,  将有  $n^{2}-1 \leqslant \operatorname{dim} U \leqslant \operatorname{dim} W=n^{2}-1$,  从而  $\operatorname{dim} U=n^{2}-1$.  下面说明  $U$  是  $V$  的子空间 .

 由于  $U$  是非空的集合并且对数乘封闭 , 只需再说明  $U$  中元素对加法封闭就证明了  $U$  是一个子空间 .  又   因为  $U$  中任意两个元素相加的结果为一个迹为  0  的矩阵 ,  若能说明迹为  0  的矩阵能能表示为  $M N-$ $N M, M, N \in V$  的形式 ,  则命题得证 .  最后需要说明的命题可以通过对阶数做数学归纳法来证明 ,  详细证明   可以参考王茴芳与石生明编的北大第四版 《 高等代数辅导与习题解答 》 第  412  页题  34 .

 注   此题包含了王䓫芳与石生明修订的北大第四版 《 高等代数 》 总习题中的第  $32,33,34$  题 ,  页码为  424,425 .

\begin{enumerate}
  \setcounter{enumi}{6}
  \item  设  $\varepsilon_{1}, \varepsilon_{2}, \ldots, \varepsilon_{n}$  为  $V$  的一组标准正交基 ,  并且  $\mathcal{A}\left(\varepsilon_{1}, \varepsilon_{2}, \ldots, \varepsilon_{n}\right)=\left(\varepsilon_{1}, \varepsilon_{2}, \ldots, \varepsilon_{n}\right) A$.  由于  $\mathcal{A}$  是对称的 ,  从而  $A$  是对称矩阵 .  对于  $\forall \alpha \in V$,  可设  $\alpha=x_{1} \varepsilon_{1}+x_{2} \varepsilon_{2}+\cdots+x_{n} \varepsilon_{n}=\left(\varepsilon_{1}, \varepsilon_{2}, \ldots, \varepsilon_{n}\right) x$,  从而  $(\alpha, \mathcal{A} \alpha)=x^{\mathrm{T}} A x$.  由  $\mathcal{A}$  是正的可以得到  $A$  是正定矩阵 .  反之 ,  若一个线性变换在任意标准正交基下的矩阵为正定矩阵 ,  则该   线性变换为正的 .
\end{enumerate}
(1)  正定矩阵  $A$  是可逆矩阵 ,  从而线性变换  $\mathcal{A}$  可逆 .

(2)  设  $\mathcal{B}$  在标准正交基  $\varepsilon_{1}, \varepsilon_{2}, \ldots, \varepsilon_{n}$  下的矩阵为  $B$.  因为  $A$  是正定矩阵 ,  于是存在实可逆矩阵  $P$,  使得  $P^{\mathrm{T}} A P=E$.  此时  $P^{\mathrm{T}} B P$  是正定矩阵 ,  故存在正交矩阵  $Q$,  使得  $Q^{\mathrm{T}} P^{\mathrm{T}} B P Q=\operatorname{diag}\left\{\lambda_{1}, \lambda_{2}, \ldots, \lambda_{n}\right\}, \lambda_{i}>$ $0, i=1,2, \ldots, n$.  又  $(P Q)^{\mathrm{T}} A(P Q)=Q^{\mathrm{T}} E Q=E$,  故  $A-B$  合同于  $\operatorname{diag}\left\{1-\lambda_{1}, 1-\lambda_{2}, \ldots, 1-\lambda_{n}\right\}$.  因为  $A-B$  正定 ,  故  $0<\lambda_{i}<1, i=1,2, \ldots, n$.  注意到  $(P Q)^{-1} B^{-1}\left((P Q)^{-1}\right)^{\mathrm{T}}=\operatorname{diag}\left\{1 / \lambda_{1}, 1 / \lambda_{2}, \ldots, 1 / \lambda_{n}\right\}$, $(P Q)^{-1} A^{-1}\left((P Q)^{-1}\right)^{\mathrm{T}}=E$,  故  $B^{-1}-A^{-1}$  合同于  $\operatorname{diag}\left\{1 / \lambda_{1}-1,1 / \lambda_{2}-1, \ldots, 1 / \lambda_{n}-1\right\}$,  结合  $1 / \lambda_{i}-1>0, i=1,2, \ldots, n$.  知  $B^{-1}-A^{-1}$  正定 ,  从而  $\mathcal{B}^{-1}-\mathcal{A}^{-1}$  是正的 .

(3)  存在正交矩阵  $U$  使得  $U^{\mathrm{T}} A U=\operatorname{diag}\left\{\lambda_{1}, \lambda_{2}, \ldots, \lambda_{n}\right\}=\operatorname{diag}\left\{\sqrt{\lambda_{1}}, \sqrt{\lambda_{2}}, \ldots, \sqrt{\lambda_{n}}\right\}^{2}=D^{2}$,  于是  $A=$

\begin{enumerate}
  \setcounter{enumi}{7}
  \item ( 法一 )  因为单叶双曲面同族的直母线异面 ,  所以两条直线要垂直相交只能是异族直母线 .  设两条直线的方程 
\end{enumerate}
$$
\left\{\begin{array}{l}
u_{1}\left(\frac{x}{a}-\frac{z}{c}\right)=v_{1}\left(1-\frac{y}{b}\right) \\
v_{1}\left(\frac{x}{a}+\frac{z}{c}\right)=u_{1}\left(1+\frac{y}{b}\right)
\end{array}, \quad\left\{\begin{array}{l}
u_{2}\left(\frac{x}{a}+\frac{z}{c}\right)=v_{2}\left(1-\frac{y}{b}\right) \\
v_{2}\left(\frac{x}{a}-\frac{z}{c}\right)=u_{2}\left(1+\frac{y}{b}\right)
\end{array} .\right.\right.
$$
$$
\left(\frac{v_{1}^{2}-u_{1}^{2}}{b c},-\frac{2 u_{1} v_{1}}{a c},-\frac{u_{1}^{2}+v_{1}^{2}}{a b}\right), \quad\left(\frac{u_{2}^{2}-v_{2}^{2}}{b c}, \frac{2 u_{2} v_{2}}{a c},-\frac{u_{2}^{2}+v_{2}^{2}}{a b}\right) \text {. }
$$
$$
\left(a \frac{u_{1} v_{2}+v_{1} u_{2}}{u_{1} u_{2}+v_{1} v_{2}}, b \frac{v_{1} v_{2}-u_{1} u_{2}}{u_{1} u_{2}+v_{1} v_{2}}, c \frac{u_{1} v_{2}-v_{1} u_{2}}{u_{1} u_{2}+v_{1} v_{2}}\right)
$$
$$
a^{2}\left(v_{1}^{2}-u_{1}^{2}\right)\left(u_{2}^{2}-v_{2}^{2}\right)-4 b^{2} u_{1} u_{2} v_{1} v_{2}+c^{2}\left(u_{1}^{2}+v_{1}^{2}\right)\left(u_{2}^{2}+v_{2}^{2}\right)=0
$$
$$
\left\{\begin{aligned}
x^{2}-a^{2}+y^{2}-b^{2}+z^{2}+c^{2} &=0 \\
\frac{x^{2}}{a^{2}}+\frac{y^{2}}{b^{2}}-\frac{z^{2}}{c^{2}} &=1
\end{aligned}\right.
$$
( 法二 )  设  $P\left(x_{0}, y_{0}, z_{0}\right)$  是任意一个交点 ,  并且它所在的一条直母线为 
$$
\ell:\left\{\begin{array}{l}
x=x_{0}+u t \\
y=y_{0}+v t, \quad \text { 其中 } u^{2}+v^{2}+w^{2} \neq 0 . \\
z=z_{0}+w t
\end{array}\right.
$$
 因为  $\ell$  在曲面上 ,  故 
$$
\frac{\left(x_{0}+u t\right)^{2}}{a^{2}}+\frac{\left(y_{0}+v t\right)^{2}}{b^{2}}-\frac{\left(z_{0}+w t\right)^{2}}{c^{2}}=1, \quad \forall t \in \mathbb{R}
$$
 即 
$$
\left(\frac{u^{2}}{a^{2}}+\frac{v^{2}}{b^{2}}-\frac{w^{2}}{c^{2}}\right) t^{2}+2\left(\frac{u x_{0}}{a^{2}}+\frac{v y_{0}}{b^{2}}-\frac{w z_{0}}{c^{2}}\right) t+\left(\frac{x_{0}^{2}}{a^{2}}+\frac{y_{0}^{2}}{b^{2}}-\frac{z_{0}^{2}}{c^{2}}-1\right)=0, \quad \forall t \in \mathbb{R},
$$
 上面的方程对于  $t \in \mathbb{R}$  均成立 ,  故 
$$
\left\{\begin{array}{c}
\frac{u^{2}}{a^{2}}+\frac{v^{2}}{b^{2}}-\frac{w^{2}}{c^{2}}=0 \\
\frac{u x_{0}}{a^{2}}+\frac{v y_{0}}{b^{2}}-\frac{w z_{0}}{c^{2}}=0 \\
\frac{x_{0}^{2}}{a^{2}}+\frac{y_{0}^{2}}{b^{2}}-\frac{z_{0}^{2}}{c^{2}}=1
\end{array}\right.
$$
 我们可设两条直母线的方向向量分别为  $\left(a \cos \theta_{i}, b \sin \theta_{i}, c\right), i=1,2$,  并且  $2 \pi>\theta_{1}>\theta_{2} \geqslant 0$,  则 
$$
\begin{aligned}
\left(\frac{x_{0}}{a^{2}}, \frac{y_{0}}{b^{2}}, \frac{z_{0}}{c^{2}}\right) &=k\left(a \cos \theta_{1}, b \sin \theta_{1}, c\right) \times\left(a \cos \theta_{2}, b \sin \theta_{2}, c\right) \\
&=k\left(b c\left(\sin \theta_{1}-\sin \theta_{2}\right), a c\left(\cos \theta_{2}-\cos \theta_{1}\right),-a b \sin \left(\theta_{1}-\theta_{2}\right)\right)
\end{aligned}
$$
 由于  $\frac{x_{0}^{2}}{a^{2}}+\frac{y_{0}^{2}}{b^{2}}-\frac{z_{0}^{2}}{c^{2}}=1$,  于是代入计算可得  $k^{2}(a b c)^{2}\left(1-\cos \left(\theta_{1}-\theta_{2}\right)\right)^{2}=1$.  后面只考虑  $k>0$  的情形  $(k<0$  的情况是类似的  $)$,  这时 
$$
\begin{aligned}
\left(x_{0}, y_{0}, z_{0}\right) &=\frac{1}{1-\cos \left(\theta_{1}-\theta_{2}\right)}\left(a\left(\sin \theta_{1}-\sin \theta_{2}\right), b\left(\cos \theta_{2}-\cos \theta_{1}\right), c \sin \left(\theta_{1}-\theta_{2}\right)\right) \\
&=\left(a \frac{\cos \frac{\theta_{1}+\theta_{2}}{2}}{\sin \frac{\theta_{1}-\theta_{2}}{2}}, b \frac{\sin \frac{\theta_{1}+\theta_{2}}{2}}{\sin \frac{\theta_{1}-\theta_{2}}{2}}, c \frac{\cos \frac{\theta_{1}-\theta_{2}}{2}}{\sin \frac{\theta_{1}-\theta_{2}}{2}}\right) .
\end{aligned}
$$
 再由两直母线垂直可得  $a^{2} \cos \theta_{1} \cos \theta_{2}+b^{2} \sin \theta_{1} \theta_{2}+c^{2}=0$,  因此计算知道  $x_{0}^{2}-a^{2}+y_{0}^{2}-b^{2}+z_{0}^{2}+c^{2}=0$.  综合上面的结果可以得到 :  交点满足方程 
$$
\left\{\begin{array}{r}
x^{2}-a^{2}+y^{2}-b^{2}+z^{2}+c^{2}=0 \\
\frac{x^{2}}{a^{2}}+\frac{y^{2}}{b^{2}}-\frac{z^{2}}{c^{2}}=1
\end{array}\right.
$$
 注   两种方法计算量都比较大 ,  不过法二中的想法更自然一些 ,  法二中如果最后的处理方法是把三角函数表示为  $x_{0}, y_{0}, z_{0}$  的式子 ,  再代入由向量相互垂直的得到的那个式子 ,  会得到不一样的关于  $x_{0}, y_{0}, z_{0}$  的关系式 ,  但是   为了和法一结果看起来一样 ,  并且也是为了最终结果的美观 ,  还是选取了  $x^{2}+y^{2}+z^{2}=a^{2}+b^{2}-c^{2}$  为最终   的结果 .  用法二的想法去处理双曲抛物面的情形可以做得很简洁 ,  具体见北京大学  2007  年第  9  题 .

\begin{enumerate}
  \setcounter{enumi}{8}
  \item  记 
\end{enumerate}
$$
A=\left[\begin{array}{rrr}
\frac{2}{3} & -\frac{1}{3} & \frac{2}{3} \\
\frac{2}{3} & \frac{2}{3} & -\frac{1}{3} \\
-\frac{1}{3} & \frac{2}{3} & \frac{2}{3}
\end{array}\right]
$$
 则  $|\lambda E-A|=(\lambda-1)\left(\lambda^{2}-\lambda+1\right)$.  因为特征值  $\lambda=1$  对应的一个特征向量为  $\vec{n}=\left[\begin{array}{c}\frac{1}{\sqrt{3}} \\ \frac{1}{\sqrt{3}} \\ \frac{1}{\sqrt{3}}\end{array}\right]$,  故不动直线的方   向向量为  $k\left[\begin{array}{l}1 \\ 1 \\ 1\end{array}\right], k \neq 0$.  因为该变换将  $\vec{a}=\left[\begin{array}{l}1 \\ 0 \\ 0\end{array}\right]$  变为  $\vec{b}=\left[\begin{array}{r}\frac{2}{3} \\ \frac{2}{3} \\ -\frac{1}{3}\end{array}\right]$,  故  $|\cos \theta|=\left|\frac{(\vec{a} \times \vec{n}) \cdot(\vec{b} \times \vec{n})}{|(\vec{a} \times \vec{n})||(\vec{b} \times \vec{n})|}\right|=\frac{1}{2}$,  因   此  $\theta=\frac{\pi}{3}$.

 注   原来试卷题中矩阵的具体数字不清楚 ,  这里题中的矩阵是我选取的 .  相反的问题见斦维声老师的 《 解析几   何 》 第三版第  143  页第  7  题 .

\begin{enumerate}
  \setcounter{enumi}{9}
  \item  设  $A_{1}$  的坐标为  $(2 t-a, 2 t-b, 3 t)$,  则  $\left(2 t-a-a_{1}, 2 t-b-a_{2}, 3 t-a_{3}\right) \cdot(2,2,3)=0 \quad t=$ $\frac{2 a+2 b+2 a_{1}+2 a_{2}+3 a_{3}}{17 .}$  由此可以得到  $A_{1}$  的坐标 ,  类似地得到  $B_{1}$  的坐标 ,  最后就可算出  $A_{1} B_{1}$.  北京大学  2015  年全国硕士研究生招生考试高代解几试题及解答     
\end{enumerate}
2019.04.23

\begin{enumerate}
  \item  欧氏空间  $\mathbb{R}^{3}$  中线性变换  $\sigma$  自然基底  $e_{1}, e_{2}, e_{3}$  下的矩阵是  $A=\left(\begin{array}{rrr}-\frac{2}{7} & \frac{3}{7} & \frac{6}{7} \\ \frac{3}{7} & \frac{6}{7} & -\frac{2}{7} \\ \frac{6}{7} & -\frac{2}{7} & \frac{3}{7}\end{array}\right)$,  试证明  $\sigma$  为镜像变换并   求反射面的基底 .

  \item  矩阵  $A=\left(\begin{array}{rrr}2 & -1 & -1 \\ 1 & 4 & 2 \\ -1 & -1 & 1\end{array}\right)$,  求矩阵  $T$  使  $T^{-1} A T$  为  Jordan  标准型 .

  \item $\mathbb{R}^{3}$  中二次型  $Q=-2 x^{2}+4 x y-y^{2}-2 x z$,  证明不存在二维子空间  $V \subset \mathbb{R}^{3}$  使  $Q$  限定在  $V$  上为正定 .

  \item (1)  举例说明  $A B$  与  $B A$  不一定相似 .

\end{enumerate}
(2)  设  $n$  阶实方阵  $A, B$  在复数域相似 ,  证明它们在实数域也相似 .

\begin{enumerate}
  \setcounter{enumi}{5}
  \item  试求  $t$  的值使  $\left(\begin{array}{ccc}2 & 1 & t-1 \\ 0 & 1 & 0 \\ -1-t & -2 & 1\end{array}\right)$  的最小多项式的次数为  2 ,  并且求出这个最小多项式 .

  \item $V=\{$  所有次数  $\leq 2$  的复系数多项式  $\}$  定义  $V$  上内积  $(f, g)=\int_{0}^{1} f(x) \overline{g(x)} x^{2} \mathrm{~d} x$.

\end{enumerate}
(1)  求  $V$  上一组标准正交基 .

(2)  求微分算子  $D: f(x) \rightarrow f^{\prime}(x)$  关于内积的共轭变换  $D^{*}$  在自然基底  $\left\{1, x, x^{2}\right\}$  下的矩阵 .

\begin{enumerate}
  \setcounter{enumi}{7}
  \item  分别过  $\mathrm{x}$  轴和  $\mathrm{y}$  轴做平面 ,  使两个平面的夹角为定值  $\alpha$.  求两个平面的交线的轨迹方程 .

  \item (14  分 )  设二次曲线与直线  $l_{1}: a_{1} x+b_{1} y+c_{1}=0$  相切于点  $\left(x_{1}, y_{1}\right)$,  与  $l_{2}: a_{2} x+b_{2} y+c_{2}=0$  相切于点  $\left(x_{2}, y_{2}\right)$,  求曲线的方程 .

\end{enumerate}
 注 :  原题中系数不清楚 ,  可以思考与  $l_{1}: x-2 y=0$  相切于  $(0,0)$;  与  $l_{2}: 10 x+13 y-9=0$  相切于  $(-3,3)$.

\begin{enumerate}
  \setcounter{enumi}{9}
  \item  单叶双曲面  $\frac{x^{2}}{a^{2}}+\frac{y^{2}}{b^{2}}-\frac{z^{2}}{c^{2}}=1$  相互垂直的直母线交点的轨迹的方程 .

  \item (18  分 )  按参数  $\lambda$  的值讨论二次曲线  $x^{2}-4(1+\lambda) x y+4 y^{2}-2 \lambda x+8 y-2 \lambda+3=0$  的类型 . 1.  首先注意到  $A$  为正交矩阵 . $|\lambda E-A|=(\lambda-1)^{2}(\lambda+1)$.  而  $(A-E) X=0$  的解为 

\end{enumerate}
$$
X=k_{1}\left[\begin{array}{l}
1 \\
1 \\
1
\end{array}\right]+k_{2}\left[\begin{array}{c}
1 \\
-5 \\
4
\end{array}\right]
$$
 故特征值  1  对应的特征子空间  $V_{1}$  的维数为  $2 .(A+E) X=0$  的解为 
$$
X=k\left[\begin{array}{c}
3 \\
-1 \\
-2
\end{array}\right]
$$
 故特征值  $-1$  对应的特征子空间  $V_{-1}$  的维数为  1 .  又由于  $V_{1} \perp V_{-1}$,  故  $\sigma$  为镜像变换 .  反射面的基底取  $V_{1}$  的基即可 .

\begin{enumerate}
  \setcounter{enumi}{2}
  \item  特征多项式 
\end{enumerate}
$$
|\lambda E-A|=\left|\begin{array}{ccc}
\lambda-2 & 1 & 1 \\
-1 & \lambda-4 & -2 \\
1 & 1 & \lambda-1
\end{array}\right|=(\lambda-2)^{2}(\lambda-3) .
$$
 又因为  $(A-2 E)(A-3 E) \neq 0$,  故  $A$  的极小多项式为  $m(x)=(x-2)^{2}(x-3), A$  的  Jordan  标准型为 
$$
\left[\begin{array}{lll}
2 & 1 & 0 \\
0 & 2 & 0 \\
0 & 0 & 3
\end{array}\right]
$$
 设 
$$
A\left(\xi_{1}, \xi_{2}, \xi_{3}\right)=\left(\xi_{1}, \xi_{2}, \xi_{3}\right)\left[\begin{array}{ccc}
2 & 1 & 0 \\
0 & 2 & 0 \\
0 & 0 & 3
\end{array}\right]
$$
 则  $A \xi_{1}=2 \xi_{1}, A \xi_{2}=\xi_{1}+2 \xi_{2}, A \xi_{3}=3 \xi_{3}$,  于是  $(A-2 E) \xi_{2}=\xi_{1},(A-2 E)^{2} \xi_{2}=0,(A-3 E) \xi_{3}=0$.  因为 
$$
(A-2 E)^{2}=\left[\begin{array}{rrr}
0 & -1 & -1 \\
0 & 1 & 1 \\
0 & 0 & 0
\end{array}\right]
$$
 可取  $\xi_{2}=\left[\begin{array}{l}1 \\ 0 \\ 0\end{array}\right], \xi_{1}=\left[\begin{array}{r}0 \\ 1 \\ -1\end{array}\right]$,  再取  $\xi_{3}=\left[\begin{array}{r}1 \\ -1 \\ 0\end{array}\right]$,  则令 
$$
T=\left[\begin{array}{rrr}
0 & 1 & 1 \\
1 & 0 & -1 \\
-1 & 0 & 0
\end{array}\right]
$$
 即可 .

\begin{enumerate}
  \setcounter{enumi}{3}
  \item ( 法一  $)$  若  $V=\left\{\left(\begin{array}{lll}x & y & z\end{array}\right)^{\mathrm{T}} \mid z=a x+b y\right\}$,  则 
\end{enumerate}
$$
\begin{aligned}
\left.Q\right|_{V} &=-2 x^{2}+4 x y-y^{2}-2 x(a x+b y) \\
&=(-2-2 a) x^{2}+(4-2 b) x y-y^{2}
\end{aligned}
$$
 此时  $\left.Q\right|_{V}$  不是正定的 .

 若  $V=\left\{\left(\begin{array}{lll}x & y & z\end{array}\right)^{\mathrm{T}} \mid y=a x+c z\right\}$,  则 
$$
\begin{aligned}
\left.Q\right|_{V} &=-2 x^{2}+4 x(a x+c z)-(a x+c z)^{2}-2 x z \\
&=\left(-2+4 a-a^{2}\right) x^{2}+(4 c-2 a c-2) x z-c^{2} z^{2}
\end{aligned}
$$
 此时  $\left.Q\right|_{V}$  不是正定的 .

 若  $V=\left\{\left(\begin{array}{lll}x & y & z\end{array}\right)^{\mathrm{T}} \mid x=b y+c z\right\}$,  则 
$$
\begin{aligned}
\left.Q\right|_{V} &=-2(b y+c z)^{2}+4(b y+c z) y-y^{2}-2(b y+c z) z \\
&=\left(-2 b^{2}+4 b-1\right) y^{2}+(-4 b c+4 c-2 b) y z+\left(-2 c^{2}-2 c\right) z^{2},
\end{aligned}
$$
 如果此时  $\left.Q\right|_{V}$  是正定的 ,  则 

 矛盾 .

( 法二 ) $Q$  对应的实对称矩阵为 
$$
A=\left[\begin{array}{rrr}
-2 & 2 & -1 \\
2 & -1 & 0 \\
-1 & 0 & 0
\end{array}\right]
$$
$|\lambda E-A|=(\lambda-1)(\lambda+2-\sqrt{3})(\lambda+2+\sqrt{3})$.  于是  $A$  只有一个正的特征值 , $A$  的正惯性指数为  1 .  下面用反   证法来证明原结论 ,  假设存在一个这样的二维子空间  $V=\{(x, y, z) \mid a x+b y+c z=0\}$,  设  $V$  的一组标准正   交基为  $\alpha_{1}=\left(p_{11}, p_{12}, p_{13}\right), \alpha_{2}=\left(p_{21}, p_{22}, p_{23}\right)$,  下面做正交变换 
$$
\left(\begin{array}{l}
u \\
v \\
w
\end{array}\right)=\left(\begin{array}{ccc}
p_{11} & p_{12} & \left.p_{13}\right) \\
p_{21} & p_{22} & \left.p_{23}\right) \\
\frac{a}{\sqrt{a^{2}+b^{2}+c^{2}}} & \frac{b}{\sqrt{a^{2}+b^{2}+c^{2}}} & \frac{c}{\sqrt{a^{2}+b^{2}+c^{2}}}
\end{array}\right)\left(\begin{array}{l}
x \\
y \\
z
\end{array}\right)=P\left(\begin{array}{l}
x \\
y \\
z
\end{array}\right)
$$
 根据前面的设定可知  $P$  是正交矩阵 ,  做变量替换后可得 
$$
Q=(u, v, w) P A P^{\mathrm{T}}\left(\begin{array}{c}
u \\
v \\
w
\end{array}\right)=(u, v, w)\left(\begin{array}{cc}
B & \xi \\
\xi^{\mathrm{T}} & d
\end{array}\right)\left(\begin{array}{l}
u \\
v \\
w
\end{array}\right)
$$
 由于  $Q$  在  $V$  上正定 ,  代  $w=0$  可得矩阵  $B$  正定 ,  于是 
$$
\left(\begin{array}{cc}
E_{2} & 0 \\
-\xi^{\mathrm{T}} B^{-1} & 1
\end{array}\right) P A P^{\mathrm{T}}\left(\begin{array}{cc}
E_{2} & -B^{-1} \xi \\
0 & 1
\end{array}\right)=\left(\begin{array}{cc}
B & 0 \\
0 & d-\xi^{\mathrm{T}} B^{-1} \xi
\end{array}\right)
$$
 从而  $A$  的正惯性指数大于等于  2 ,  矛盾 !

 注   法二的想法来自  27 Lines On a Cubic Surface,  用这里的证明方法可以证明更一般性的结论 :  在  $n$  维欧式   空间中 ,  一个对称双线性型如果在一个  $r$  维子空间中正定 ,  那么它的正惯性指数不小于  $r$.
$$
\begin{aligned}
& \left\{\begin{array}{l}-2 b^{2}+4 b-1>0, \\-2 c^{2}-2 c>0\end{array} \Longrightarrow\left|\begin{array}{cc}-2 b^{2}+4 b-1 & -2 b c+2 c-b \\-2 b c+2 c-b & -2 c^{2}-2 c\end{array}\right|=-2 c^{2}-b^{2}-4 b c+2 c\right. \\
& =-(b+2 c)^{2}+2 c^{2}+2 c \\
& <0 \text {, } 
\end{aligned}
$$

\begin{enumerate}
  \setcounter{enumi}{4}
  \item (1)  取 
\end{enumerate}
$$
A=\left[\begin{array}{ll}
1 & 0 \\
0 & 0
\end{array}\right], \quad B=\left[\begin{array}{ll}
0 & 1 \\
0 & 0
\end{array}\right]
$$
 则 
$$
A B=\left[\begin{array}{ll}
0 & 1 \\
0 & 0
\end{array}\right], \quad B A=\left[\begin{array}{ll}
0 & 0 \\
0 & 0
\end{array}\right]
$$
 与  $B A$  相似的矩阵只有二阶零矩阵 ,  故  $B A$  不与  $A B$  相似 .

(2)  因为  $n$  阶实方阵  $A, B$  在复数域相似 ,  故存在复可逆矩阵  $P$,  使得  $P^{-1} A P=B$,  故  $A P=B P$.  将  $P$  的实部与虚部分开 ,  即设  $P=P_{1}+\mathrm{i} P_{2}, P_{1}, P_{2} \in M_{n}(\mathrm{R})$,  则  $A P_{1}+\mathrm{i} A P_{2}=P_{1} B+\mathrm{i} P_{2} B \Longrightarrow A P_{1}=$ $P_{1} B, A P_{2}=P_{2} B$.  因为  $|P|=\left|P_{1}+\mathrm{i} P_{2}\right| \neq 0$,  故多项式  $\left|P_{1}+t P_{2}\right|$  为非零多项式 ,  存在  $t_{0} \in \mathbb{R}$  使得  $\left|P_{1}+t_{0} P_{2}\right| \neq 0$,  此时  $A\left(P_{1}+t_{0} P_{2}\right)=\left(P_{1}+t_{0} P_{2}\right) B$,  从而它们在实数域也相似 .

\begin{enumerate}
  \setcounter{enumi}{5}
  \item  由于最小多项式的次数小于矩阵的阶数 ,  故矩阵的特征多项式有重根 .  而  $f(\lambda)=|\lambda E-A|=(\lambda-1)\left(\lambda^{2}-\right.$ $\left.3 \lambda+1+t^{2}\right)$.  若  1  是重根 ,  则  $t^{2}=1, \Longrightarrow f(\lambda)=(\lambda-1)^{2}(\lambda-2) \Longrightarrow$  最小多项式  $m(\lambda)$  为  $(\lambda-1)(\lambda-2)$,  眕 
\end{enumerate}
$$
(A-E)(A-2 E)=\left[\begin{array}{ccc}
1-t^{2} & -2(t-1) & 0 \\
0 & 0 & \\
0 & 1-t & 1-t^{2}
\end{array}\right]=0 \Rightarrow t=1
$$
 若  1  不是重根 ,  则  $t^{2}+1=\frac{9}{4}, \Longrightarrow f(\lambda)=(\lambda-1)\left(\lambda-\frac{3}{2}\right)^{2} \Longrightarrow$  最小多项式  $m(\lambda)$  为  $(\lambda-1)\left(\lambda-\frac{3}{2}\right)$,  故  $(A-E)\left(A-\frac{3}{2} E\right)=0$,  此时无解 .

\begin{enumerate}
  \setcounter{enumi}{6}
  \item (1) $V$  的一组基为  $1, x, x^{2}$.  先进行正交化得 
\end{enumerate}
$$
\xi_{1}=1, \quad \xi_{2}=x-\frac{\left(x, \xi_{1}\right)}{\left(\xi_{1}, \xi_{1}\right)} \xi_{1}=x-\frac{3}{4}, \quad \xi_{3}=x^{2}-\frac{\left(x^{2}, \xi_{1}\right)}{\left(\xi_{1}, \xi_{1}\right)} \xi_{1}-\frac{\left(x^{2}, \xi_{2}\right)}{\left(\xi_{2}, \xi_{2}\right)} \xi_{2}=x^{2}-\frac{4}{3} x+\frac{2}{5} .
$$
 再单位化得 
$$
\eta_{1}=\frac{\xi_{1}}{\sqrt{\left(\xi_{1}, \xi_{1}\right)}}=\sqrt{3}, \quad \eta_{2}=\frac{\xi_{2}}{\sqrt{\left(\xi_{2}, \xi_{2}\right)}}=\sqrt{5}(4 x-3), \quad \eta_{3}=\frac{\xi_{3}}{\sqrt{\left(\xi_{3}, \xi_{3}\right)}}=\sqrt{7}\left(15 x^{2}-20 x+6\right) .
$$
(2)  设  $D^{*}\left(1, x, x^{2}\right)=\left(1, x, x^{2}\right)\left(a_{i j}\right)_{3 \times 3}$,  则  $D^{*} 1=a_{11}+a_{21} x+a_{31} x^{2}$.
$$
\left\{\begin{array}{l}
(D 1,1)=\left(1, D^{*} 1\right)=\overline{a_{11}}(1,1)+\overline{a_{21}}(1, x)+\overline{a_{31}}\left(1, x^{2}\right) \\
(D x, 1)=\left(x, D^{*} 1\right)=\overline{a_{11}}(x, 1)+\overline{a_{21}}(x, x)+\overline{a_{31}}\left(x, x^{2}\right) \\
\left(D x^{2}, 1\right)=\left(x^{2}, D^{*} 1\right)=\overline{a_{11}}\left(x^{2}, 1\right)+\overline{a_{21}}\left(x^{2}, x\right)+\overline{a_{31}}\left(x^{2}, x^{2}\right)
\end{array} \quad \Longrightarrow\left[\begin{array}{l}
0 \\
\frac{1}{3} \\
\frac{1}{2}
\end{array}\right]=\left[\begin{array}{ccc}
\frac{1}{3} & \frac{1}{4} & \frac{1}{5} \\
\frac{1}{4} & \frac{1}{5} & \frac{1}{6} \\
\frac{1}{5} & \frac{1}{6} & \frac{1}{7}
\end{array}\right]\left[\begin{array}{l}
\overline{a_{11}} \\
\overline{a_{21}} \\
\overline{a_{31}}
\end{array}\right]\right.
$$
 类似地可以求出其他  $a_{i j}$,  最终结果为 
$$
\bar{A}=\left[\begin{array}{ccc}
\frac{1}{3} & \frac{1}{4} & \frac{1}{5} \\
\frac{1}{4} & \frac{1}{5} & \frac{1}{6} \\
\frac{1}{5} & \frac{1}{6} & \frac{1}{7}
\end{array}\right]^{-1}\left[\begin{array}{ccc}
0 & 0 & 0 \\
\frac{1}{3} & \frac{1}{4} & \frac{1}{5} \\
\frac{1}{2} & \frac{2}{5} & \frac{1}{3}
\end{array}\right]=A
$$

\begin{enumerate}
  \setcounter{enumi}{7}
  \item  设过  $x$  轴 , $y$  轴的平面的方程分别  $a y+b z=0, c x+d z=0$.  平面夹角为  $\alpha$,  故 
\end{enumerate}
$$
|\cos \alpha|=\frac{|b d|}{\sqrt{a^{2}+b^{2}} \sqrt{c^{2}+d^{2}}} \Longrightarrow\left(z^{2}+y^{2}\right)\left(z^{2}+x^{2}\right) \cos ^{2} \alpha=y^{2} x^{2} .
$$

\begin{enumerate}
  \setcounter{enumi}{8}
  \item  先把二次曲线的方程设出来 ,  再按条件找一些等量关系 ,  就可得到结果 .  按照我编的那个系数最后算出来应   该是  $5 x^{2}+7 x y+y^{2}-x+2 y=0$. 9.  因为单叶双曲面同族的直母线异面 ,  所以两条直线要垂直相交只能是异族直母线 .  设两条直线的方程分别为 
\end{enumerate}
$$
\left\{\begin{array}{l}
u_{1}\left(\frac{x}{a}-\frac{z}{c}\right)=v_{1}\left(1-\frac{y}{b}\right) \\
v_{1}\left(\frac{x}{a}+\frac{z}{c}\right)=u_{1}\left(1+\frac{y}{b}\right)
\end{array},\left\{\begin{array}{l}
u_{2}\left(\frac{x}{a}+\frac{z}{c}\right)=v_{2}\left(1-\frac{y}{b}\right) \\
v_{2}\left(\frac{x}{a}-\frac{z}{c}\right)=u_{2}\left(1+\frac{y}{b}\right)
\end{array}\right.\right.
$$
 于是两条直线的方向向量分别为  $\left(\frac{v_{1}^{2}-u_{1}^{2}}{b c},-\frac{2 u_{1} v_{1}}{a c},-\frac{u_{1}^{2}+v_{1}^{2}}{a b}\right),\left(\frac{u_{2}^{2}-v_{2}^{2}}{b c}, \frac{2 u_{2} v_{2}}{a c},-\frac{u_{2}^{2}+v_{2}^{2}}{a b}\right)$.  两条直   线的交点为  $\left(a \frac{u_{1} v_{2}+v_{1} u_{2}}{u_{1} u_{2}+v_{1} v_{2}}, b \frac{v_{1} v_{2}-u_{1} u_{2}}{u_{1} u_{2}+v_{1} v_{2}}, c \frac{u_{1} v_{2}-v_{1} u_{2}}{u_{1} u_{2}+v_{1} v_{2}}\right)$.  由于两直线垂直 ,  故 
$$
a^{2}\left(v_{1}^{2}-u_{1}^{2}\right)\left(u_{2}^{2}-v_{2}^{2}\right)-4 b^{2} u_{1} u_{2} v_{1} v_{2}+c^{2}\left(u_{1}^{2}+v_{1}^{2}\right)\left(u_{2}^{2}+v_{2}^{2}\right)=0
$$
 综合上面的结果可以得到 :  交点满足方程 
$$
\left\{\begin{array}{c}
x^{2}+y^{2}+z^{2}=a^{2}+b^{2}-c^{2} \\
\frac{x^{2}}{a^{2}}+\frac{y^{2}}{b^{2}}-\frac{z^{2}}{c^{2}}=1
\end{array}\right.
$$
 注   与北京大学  2014  年第  7  题一样 .

\begin{enumerate}
  \setcounter{enumi}{10}
  \item  二次曲线对应的实对称矩阵为 
\end{enumerate}
$$
\left[\begin{array}{ccc}
1 & -2(1+\lambda) & -\lambda \\
-2(1+\lambda) & 4 & 4 \\
-\lambda & 4 & 3-2 \lambda
\end{array}\right]
$$
 计算不变量 :
$$
\begin{aligned}
I_{1} &=1+4=5, \\
I_{2} &=\left|\begin{array}{ccc}
1 & -2(1+\lambda) \\
-2(1+\lambda) & 4
\end{array}\right|=-4 \lambda(\lambda+2), \\
I_{3} &=\left|\begin{array}{ccc}
1 & -2(1+\lambda) & -\lambda \\
-2(1+\lambda) & 4 & 4 \\
-\lambda & 4 & 3-2 \lambda
\end{array}\right|=8(\lambda+2)(\lambda-1)(\lambda+1) .
\end{aligned}
$$
 当  $I_{2}=0$  时为抛物线型 .  更具体地 ,  当  $\lambda=0$  时为抛物线 .  当  $\lambda=-2$  时为一对虚平行直线 .

 当  $I_{2}<0$  时为双曲线型 .  更具体地 ,  当  $\lambda=1$  时为一对相交直线 .  当  $\lambda \in(-\infty,-2) \cup(0,1) \cup(1,+\infty)$  时为   双曲线 .

 当  $I_{2}>0$  时为椭圆型 .  更具体地 ,  当  $\lambda \in(-2,-1)$  时为虚椭圆 .  当  $\lambda=-1$  时为一个点 .  当  $\lambda \in(-1,0)$  时   为椭圆 .

 注   用不变量来判断二次曲线的类型和形状可以参考丘维声的 《 解析几何 》 第三版第  164  页 .

 后记 :  北京大学  2015  年的高代解几试题我以前没有见到过 ,  几天前无意中发现了两个回忆版 ,  分别见   龙呈祥的   发言与   时穴之翼的发言 ,  但是有些地方是明显回忆错了 ,  修正试题花了我不少时间 .  北京大学  2016  年全国硕士研究生招生考试高代解几试题及解答 

   

2019.04.13

\begin{enumerate}
  \item (10  分 )  设  $\mathbb{R}^{3}$  上线性变换  $\mathscr{A}$  在自然基  $\varepsilon_{1}=\left(\begin{array}{l}1 \\ 0 \\ 0\end{array}\right), \varepsilon_{2}=\left(\begin{array}{l}0 \\ 1 \\ 0\end{array}\right), \varepsilon_{3}=\left(\begin{array}{c}0 \\ 0 \\ 1\end{array}\right)$  的一组基 ,  使得  $\mathscr{A}$  在伩组基下的矩阵为   阵为  $\left(\begin{array}{ccc}0 & 1 & -1 \\ 0 & 0 & 1 \\ 0 & 0 & 0\end{array}\right)$.  求  $\mathbb{R}^{3}$  的一组基 ,  使得  $\mathscr{A}$  在这组基下的矩阵为  Jordan  矩阵 .

  \item (10  分 ) 3  阶实矩阵  $A$  的特征多项式为  $x^{3}-3 x^{2}+4 x-2$.  证明  $A$  既不是对称阵也不是正交阵 .

  \item (15  分 )  在所有  2  阶实方阵上定义二次型  $f: X \rightarrow \operatorname{Tr}\left(X^{2}\right)$.  求  $f$  的秩和符号差 .

  \item (15  分 )  设  $V$  是有限维线性空间 , $\mathscr{A}, \mathscr{B}$  是  $V$  上线性变换满足下面条件 

\end{enumerate}
(1) $\mathscr{A} \mathscr{B}=\mathscr{O}$.  这里  $\mathscr{O}$  是  0  变换 .

(2) $\mathscr{A}$  的任意不变子空间也是  $\mathscr{B}$  的不变子空间 .

(3) $\mathscr{A}^{5}+\mathscr{A}^{4}+\mathscr{A}^{3}+\mathscr{A}^{2}+\mathscr{A}=\mathscr{O}$.

 证明  $\mathscr{B} \mathscr{A}=\mathscr{O}$.

\begin{enumerate}
  \setcounter{enumi}{5}
  \item (15  分 )  设  $V$  是全体次数不超过  $n$  的实系数多项式组成的线性空间 .  定义线性变换  $\mathscr{A}: f(x) \rightarrow f(1-x)$.  求  $\mathscr{A}$  的特征值和对应的特征子空间 .

  \item (15  分 )  计算行列式 

\end{enumerate}
$$
\left|\begin{array}{ccccc}
1^{50} & 2^{50} & 3^{50} & \cdots & 100^{50} \\
2^{50} & 3^{50} & 4^{50} & \cdots & 101^{50} \\
\vdots & \vdots & \vdots & \vdots & \vdots \\
100^{50} & 101^{50} & 102^{5^{50}} & \cdots & 199^{50}
\end{array}\right| .
$$
 其中各行元素的底数为等差数列 ,  各列元素的底数也为等差数列 ,  所有元素的指数都是  50 .

\begin{enumerate}
  \setcounter{enumi}{7}
  \item (20  分 )  设  $V$  是复数域上有限维线性空间 , $\mathscr{A}$  是  $V$  上线性变换 , $\mathscr{A}$  在一组基下矩阵为  $F$.
\end{enumerate}
(1) 若  $\mathscr{A}$  可对角化 ,  则对任意  $\mathscr{A}$  的不变子空间  $U$,  存在  $U$  的一个补空间  $W$  是  $\mathscr{A}$  的不变子空间 .

(2)  若对任意  $\mathscr{A}$  的不变子空间  $U$,  存在  $U$  的一个补空间  $W$  是  $\mathscr{A}$  的不变子空间 ,  证明  $F$  可对角化 .

\begin{enumerate}
  \setcounter{enumi}{8}
  \item (20  分 )  平面上一个可逆仿射变换将一个圆映为椭圆或圆 ,  详细论证这一点 .

  \item (15  分 )  平面  $A x+B y+C z+D=0$  与双曲抛物面  $2 z=x^{2}-y^{2}$  交于两条直线 ,  证明  $A^{2}-B^{2}-2 C D=0$.

  \item (15  分 )  正十二面体有  12  个面 ,  每个面为正五边形 ,  每个顶点连接  3  条棱 ,  求它的内切球与外接球半径之比 . 1. $\lambda E-A=\left(\begin{array}{ccc}\lambda & -1 & 1 \\ 0 & \lambda & -1 \\ 0 & 0 & \lambda\end{array}\right) \rightarrow\left(\begin{array}{ccc}-1 & \lambda & 1 \\ \lambda & 0 & -1 \\ 0 & 0 & \lambda\end{array}\right) \rightarrow\left(\begin{array}{ccc}-1 & 0 & 0 \\ 0 & \lambda^{2} & -1 \\ 0 & 0 & \lambda\end{array}\right) \rightarrow\left(\begin{array}{c}1 \\ 1\end{array}\right)$,  故  $A$  的  Jordan  标准型   为  $J=\left(\begin{array}{ccc}0 & 1 & 0 \\ & 0 & 1 \\ & & 0\end{array}\right) . P^{-1} A P=J, A P=P J$.  设  $P=\left(\xi_{1}, \xi_{2}, \xi_{3}\right)$,  则  $A\left(\xi_{1}, \xi_{2}, \xi_{3}\right)=\left(\xi_{1}, \xi_{2} \xi_{3}\right) J \Longrightarrow A \xi_{1}=$ $0, A \xi_{2}=\xi_{1}, A \xi_{3}=\xi_{2}, A^{3} \xi_{3}=0$,  于是可以取  $\xi_{1}=\left(\begin{array}{l}1 \\ 0 \\ 0\end{array}\right), \xi_{2}=\left(\begin{array}{l}0 \\ 1 \\ 0\end{array}\right), \xi_{3}=\left(\begin{array}{l}1 \\ 1 \\ 1\end{array}\right)$.

\end{enumerate}
 注   这里是通过  $\lambda$  矩阵理论得出的标准型 .  另一方法是先得出特征多项式为  $\lambda^{3}$,  再注意到  $A^{2} \neq 0$,  从而得到  Jordan  标准型为  $J(0 ; 3)$.

\begin{enumerate}
  \setcounter{enumi}{2}
  \item $f(x)=x^{3}-3 x^{2}+4 x-2=(x-1)\left(x^{2}-2 x+2\right)$,  实对称矩阵的特征值均是实数 ,  而  $f(x)=0$  只有  1  个实   根 ,  从而  $A$  不是实对称矩阵 .  正交阵的行列式为  1  或  $-1$,  而  $f(0)=-2$,  从而  $A$  不是正交矩阵 .

  \item  设  $X=\left(\begin{array}{ll}a & b \\ c & d\end{array}\right)$,  则  $X^{2}=\left(\begin{array}{cc}a^{2}+b c & a b+b d \\ a c+c d & b c+d^{2}\end{array}\right) \cdot \operatorname{tr}\left(X^{2}\right)=a^{2}+2 b c+d^{2}=\left(\begin{array}{llll}a & b & c & d\end{array}\right)\left(\begin{array}{ccc}1 & 0 & 1 \\ b & 0 & 1\end{array}\right.$ $f$  的秩为  4 ,  特征值为  $1,1,1,-1$,  符号差为  2 .

  \item $f(x)=x+x^{2}+\cdots+x^{5}=x \frac{x^{5}-1}{x-1}=x \prod_{k=1}^{4}\left(x-\mathrm{e}^{\frac{2 \pi k \mathrm{i}}{5}}\right)$,  于是  $f(x)$  为互素一次因式的乘积 ,  从而  $\mathscr{A}$  在任意   一组基下的矩阵在复数域上可相似对角化 ,  于是  $V=V_{1} \oplus \cdots \oplus V_{s}, V_{i}$  为  $\mathscr{A}$  的特征值  $\lambda_{i}$  对应的特征子空   间 , $1 \leq i \leq s$.  对于  $\forall \alpha \in V$,

\end{enumerate}
$$
\mathscr{B} \mathscr{A} \alpha=\mathscr{B} \mathscr{A}\left(\alpha_{1}+\cdots+\alpha_{s}\right)=\mathscr{B} \lambda_{1} \alpha_{1}+\cdots+\mathscr{B} \lambda_{s} \alpha_{s}=\mathscr{A} \mathscr{B} \alpha=\mathscr{O} \alpha=\mathbf{0} \Longrightarrow \mathscr{B} \mathscr{A}=\mathscr{O}
$$

\begin{enumerate}
  \setcounter{enumi}{5}
  \item $\mathscr{A}^{2} f(x)=\mathscr{A} f(1-x)=f(x) \Longrightarrow \mathscr{A}^{2}=\mathscr{E}$,  故  $\mathscr{A}$  的特征值为  $1,-1$.
\end{enumerate}
$$
\begin{gathered}
V_{1}=\{f(x) \mid f(1-x)=f(x)\}=\operatorname{span}\left(1,\left(x-\frac{1}{2}\right)^{2},\left(x-\frac{1}{2}\right)^{4}, \cdots\right) \\
V_{-1}=\{f(x) \mid f(1-x)=-f(x)\}=\operatorname{span}\left(\left(x-\frac{1}{2}\right),\left(x-\frac{1}{2}\right)^{3},\left(x-\frac{1}{2}\right)^{5} \cdots\right)
\end{gathered}
$$

\begin{enumerate}
  \setcounter{enumi}{6}
  \item  设  $f(x)=\left|\begin{array}{ccccc}1^{50} & 2^{50} & 3^{50} & \cdots & 100^{50} \\ 2^{50} & 3^{50} & 4^{50} & \cdots & 101^{50} \\ \vdots & \vdots & \vdots & \vdots & \vdots \\ x^{50} & (x+1)^{50} & (x+2)^{50} & \cdots & (x+99)^{50}\end{array}\right|$,  则  $f(x)$  为次数不超过  50  的多项式 ,  而  $1,2,3, \ldots, 99$  为  $f(x)$  的根 ,  从而  $f(x) \equiv 0$,  故所求行列式为  $f(100)=0$.

  \item  详细解答见蓝以中老师的 《 高等代数学习指南 》 第  212  页例  $4.12$  与例  $4.13$,  或者丘维声老师的 《 高等代数 》  仓新教材下册第  310  页命题  10 . 8.  设  $\left(\begin{array}{l}x_{1} \\ y_{1}\end{array}\right)=\mathscr{A}\left(\begin{array}{l}x \\ x\end{array}\right)=A\left(\begin{array}{l}x \\ x\end{array}\right)$,  其中  $A=\left(\begin{array}{ll}a & b \\ c & d\end{array}\right), a d-b c \neq 0$.  设圆的方程为  $x^{2}+y^{2}=R^{2}$,  则由 

\end{enumerate}
 因  $\left(\begin{array}{cc}d^{2}+c^{2} & -(a c+b d) \\ -(a c+b d) & a^{2}+b^{2}\end{array}\right)$  的行列式大于  0,  迹大于  $0, R^{2}(a d-b c)^{2}>0$,  故变换后的图形是圆或椭圆 .

\begin{enumerate}
  \setcounter{enumi}{9}
  \item  两条相交直线必属于异族 ,  设它们的方程分别为  $\left\{\begin{array}{c}x+y=2 u \\ u(x-y)=z\end{array},\left\{\begin{array}{c}x-y=2 v \\ v(x+y)=z\end{array}\right.\right.$,  则分别可取方向向量  $(-1,1,-2 u),(1,1,2 v)$,  两直线的交点为  $(\mathrm{u}+\mathrm{v}, \mathrm{u}-\mathrm{v}, 2 \mathrm{uv}) .$  应当有  $\left\{\begin{array}{c}A(u+v)+B(u-v)+2 C u v+D=0 \\ (-1,1,-2 u) \times(1,1,2 v) / /(A, B, C)\end{array}\right.$,  而  $C \neq 0 \Longrightarrow v+u=\frac{-A}{C}, v-u=\frac{-B}{C} \Longrightarrow A^{2}-B^{2}-2 C D=0$.
\end{enumerate}
 注   此题与北京大学  2009  年高代解几试题第  10  题相同 .

\begin{enumerate}
  \setcounter{enumi}{10}
  \item  不考虑正十二面体的大小 ,  决定正十二面体的形状的只是从一个顶点  $O$  引出的三条棱  $O A, O B, O C$  之间的   位置关系 .  设  $\vec{a}=\overrightarrow{O A}, \vec{b}=\overrightarrow{O B}, \vec{c}=\overrightarrow{O C}$,  则  $\vec{a}, \vec{b}, \vec{c}$  两两之间夹角为  $\theta=\frac{3 \pi}{5}$.  平面  $O A B$  与正十二   面体的外接球相截得到一个圆 ,  该圆的内接正五边形为正十二面体的一个面 ,  该圆的圆心  $P$  与正十二面体   的外接球的球心  $Q$  的连线应当与平面  $O A B$  垂直 , $P$  点也是一个正五边形的中心 , $Q$  点也是正十二面体的   内切球的球心 . $O, Q$  与  $\triangle A B C$  的重心共线 .  由于  $\frac{\text { 内切球半径 }}{\text { 外切球半径 }}=\frac{|P Q|}{|O Q|}=\cos \angle P Q O$,  不妨  $|O Q|=1$,  则  $\overrightarrow{O Q}=\frac{\vec{a}+\vec{b}+\vec{c}}{|\vec{a}+\vec{b}+\vec{c}|}$,  设  $\overrightarrow{O P}=t \frac{\vec{a}+\vec{b}}{|\vec{a}+\vec{b}|}$,  由于  $\overrightarrow{P Q} \cdot \vec{a}=0$,  则  $\left(\frac{\vec{a}+\vec{b}+\vec{c}}{|\vec{a}+\vec{b}+\vec{c}|}-t \frac{\vec{a}+\vec{b}}{|\vec{a}+\vec{b}|}\right) \cdot \vec{a}=0$,  解   得  $t=\sqrt{\frac{2(1+2 \cos \theta)}{3(1+\cos \theta)}}$.  因此  $\frac{\text { 内切球半径 }}{\text { 外切球半径 }}=\frac{|P Q|}{|O Q|}=\sqrt{1-t^{2}}=\sqrt{\frac{1-\cos \theta}{3(1+\cos \theta)}}=\frac{1}{\sqrt{3}} \tan \frac{\theta}{2}, \theta=\frac{3 \pi}{5}$.
\end{enumerate}
 注   此解法的灵感来于考虑正四面体的两个球的半径之比 .
$$
\begin{aligned}
& \left\{\begin{array}{l}\frac{1}{a d-b c}\left(\begin{array}{cc}d & -b \\-c & a\end{array}\right)\left(\begin{array}{l}x_{1} \\y_{1}\end{array}\right)=\left(\begin{array}{l}x \\y\end{array}\right) \\x^{2}+y^{2}=R^{2}\end{array}\right. \\
& \Longrightarrow\left(\frac{d x_{1}-b y_{1}}{a d-b c}\right)^{2}+\left(\frac{-c x_{1}+a y_{1}}{a d-b c}\right)^{2}=R^{2} \\
& \Longrightarrow d^{2} x_{1}^{2}-2 d b x_{1} y_{1}+b^{2} y_{1}^{2}+c^{2} x_{1}^{2}+a^{2} y_{1}^{2}-2 a c x_{1} y_{1}=R^{2}(a d-b c)^{2} \\
& \Longrightarrow\left(d^{2}+c^{2}\right) x_{1}^{2}+\left(a^{2}+b^{2}\right) y_{1}^{2}-2(a c+b d) x_{1} y_{1}=R^{2}(a d-b c)^{2} 
\end{aligned}
$$
 北京大学  2017  年全国硕士研究生招生考试高代解几试题及解答 

   

2019.04.14

\begin{enumerate}
  \item (15  分 ) $x_{1}=x_{2}=1, x_{n}=x_{n-1}+x_{n-2}$.  试用矩阵论方法求出  $x_{n}$  的通项 .

  \item (15  分 ) $\xi, \eta$  为欧氏空间  $V$  中两个长度相等的向量 .  证明存在正交变换  $A$  使得  $A \xi=\eta$

  \item (10  分 )  证明  $n$  阶  Hermite  矩阵  $A$  有  $n$  个实特征值  ( 考虑重数 ).

  \item (20  分 )  设  $F$  为数域 , $\alpha_{1}, \alpha_{2} \cdots, \alpha_{n}, \beta_{1}, \beta_{2}, \cdots, \beta_{n}$  是  $F^{n}$  中  $2 n$  个列向量 .  用  $\left|\alpha_{1}, \cdots, \alpha_{n}\right|$  表示以  $\alpha_{1}, \alpha_{2}, \cdots, \alpha_{n}$  为列向量的矩阵的行列式 .  证明下面的行列式等式 

\end{enumerate}
$$
\left|\alpha_{1}, \cdots, \alpha_{n}\right| \cdot\left|\beta_{1}, \cdots, \beta_{n}\right|=\sum_{i=1}^{n}\left|\alpha_{1}, \cdots, \alpha_{i-1}, \beta_{1}, \alpha_{i+1}, \cdots, \alpha_{n}\right| \cdot\left|\alpha_{i}, \beta_{2}, \cdots, \beta_{n}\right| .
$$

\begin{enumerate}
  \setcounter{enumi}{5}
  \item (20  分 )  设  $V$  是数域  $F$  上  $n$  维线性空间 , $A$  是  $V$  上线性变换 , $A$  的  $n$  个特征值均属于  $F$.  证明存在唯一可   对角化线性变换  $A_{1}$,  幂零线性变换  $A_{2}$,  使得  $A=A_{1}+A_{2}, A_{1} A_{2}=A_{2} A_{1}$.

  \item (20  分 )  设  $F$  为数域 , $A, B, P \in M_{n}(F), P$  幂零且  $(A-B) P=P(A-B), B P-P B=2(A-B)$.  求一个可   逆矩阵  $Q$  使得  $A Q=Q B$.

  \item (15  分 ).  证明  $\vec{a}, \vec{b}, \vec{c}$  共面的充要条件为  $\vec{a} \times \vec{b}, \vec{b} \times \vec{c}, \vec{c} \times \vec{a}$  共面 

  \item (20  分 )  空间中有四点  $O, A, B, C$  满足 

\end{enumerate}
$$
\angle A O B=\frac{\pi}{2}, \angle B O C=\frac{\pi}{3}, \angle C O A=\frac{\pi}{4} .
$$
 设  $A O B$  决定的平面为  $\pi_{1}, B O C$  决定的平面为  $\pi_{2}$,  求  $\pi_{1}, \pi_{2}$  二面角  ( 求出二面角的余弦值即可 ).

\begin{enumerate}
  \setcounter{enumi}{9}
  \item (15  分 )  设  $F$  为单叶双曲面 , $\vec{n}$  为给定非零向量 .  证明空间中所有与  $\vec{n}$  垂直的平面与  $F$  的交线的对称中心   在一条直线上 . 1.  由递推关系得  $\left(\begin{array}{c}x_{n} \\ x_{n-1}\end{array}\right)=\left(\begin{array}{ll}1 & 1 \\ 1 & 0\end{array}\right)\left(\begin{array}{l}x_{n-1} \\ x_{n-2}\end{array}\right)=\left(\begin{array}{ll}1 & 1 \\ 1 & 0\end{array}\right)^{n-2}\left(\begin{array}{l}x_{2} \\ x_{1}\end{array}\right)=\left(\begin{array}{ll}1 & 1 \\ 1 & 0\end{array}\right)^{n-2}\left(\begin{array}{l}1 \\ 1\end{array}\right), n \geq 2$.  而  $A=$ $\left(\begin{array}{ll}1 & 1 \\ 1 & 0\end{array}\right)$  的特征多项式为  $f(\lambda)=|\lambda E-A|=\lambda^{2}-\lambda-1=\left(\lambda-\frac{1+\sqrt{5}}{2}\right)\left(\lambda-\frac{1-\sqrt{5}}{2}\right)$,  根据多项式的   带余除法可设  $x^{n-2}=f(x) q(x)+a x+b$,  代入特征多项式的两个根 ,  可以算出 
\end{enumerate}
$a=\frac{1}{\sqrt{5}}\left[\left(\frac{1+\sqrt{5}}{2}\right)^{n-2}-\left(\frac{1-\sqrt{5}}{2}\right)^{n-2}\right], b=\frac{1}{\sqrt{5}}\left[\frac{1+\sqrt{5}}{2}\left(\frac{1-\sqrt{5}}{2}\right)^{n-2}-\frac{1-\sqrt{5}}{2}\left(\frac{1+\sqrt{5}}{2}\right)^{n-2}\right]$

 于是  $A^{n-2}=a A+b E=\left(\begin{array}{cc}a+b & a \\ a & b\end{array}\right), n \geq 2$.  从而  $x_{n}=2 a+b=\frac{1}{\sqrt{5}}\left(\frac{1+\sqrt{5}}{2}\right)^{n}-\frac{1}{\sqrt{5}}\left(\frac{1-\sqrt{5}}{2}\right)^{n}, n \geq 2$,  而  $n=1$  时也是成立的 ,  从而得到了通项公式 .

 注   用矩阵来求  $x_{n}$  的通项的关键是要算出  $A^{n-2}$, 一般常见的做法是利用相似对角化来求 ,  但是这种方法对于   不能对角化的矩阵而言就不好用了 .  这里给出的用矩阵的极小多项式来求矩阵方幂的方法一般而言计算量   小 ,  而且对于不可对角化的矩阵也能用 .

\begin{enumerate}
  \setcounter{enumi}{2}
  \item  若  $\xi=\eta$,  则直接取恒等变换即可 .
\end{enumerate}
 若  $\xi \neq \eta$,  则只需找某个合适的镜面反射 .  事实上 ,  对于给定的  $V$  中单位向量  $\beta$,  可以定义镜面反射 : $\mathscr{A} \alpha=\alpha-2(\alpha, \beta) \beta, \forall \alpha \in V$.  容易验证  $\mathscr{A}$  是正交变换 .  设  $\mathscr{A} \xi=\eta$,  如果解出了决定  $\mathscr{A}$  的  $\beta$,  则证明了原   命题 .

 由  $\mathscr{A} \xi=\xi-2(\xi, \beta) \beta=\eta$,  得  $(\xi, \xi)-2(\xi, \beta)^{2}=(\xi, \eta)$,  再结合  $(\xi, \xi)=(\eta, \eta) \Longrightarrow(\xi, \xi)+(\eta, \eta)-2(\xi, \eta)=$ $4(\xi, \beta)^{2} \Longrightarrow|\xi-\eta|^{2}=4(\xi, \beta)^{2} \Longrightarrow(\xi, \beta)=\frac{|\xi-\eta|}{2}$  或  $(\xi, \beta)=-\frac{|\xi-\eta|}{2}$,  任取其一可从  $\mathscr{A} \xi=$ $\xi-2(\xi, \beta) \beta=\eta$  中解得  $\beta$,  从而也就得到  $\mathscr{A}$.

 注   此题与北京大学  2010  年第  7  题相同 ,  在各种高等代数的教科书上也很常见 .

\begin{enumerate}
  \setcounter{enumi}{3}
  \item  由题意  $\bar{A}^{\mathrm{T}}=A$.  设  $\lambda \in \mathbb{C}$  是  $A$  的特征值 , $\xi$  是其对应的一个特征向量 ,  则  $A \xi=\lambda \xi$,  对等式两边做共轭转   置得  $\bar{\xi}^{\mathrm{T}} \bar{A}^{\mathrm{T}}=\bar{\lambda} \bar{\xi}^{\mathrm{T}}$,  于是  $\bar{\xi}^{\mathrm{T}} A \xi=\bar{\lambda} \bar{\xi}^{\mathrm{T}} \xi, \bar{\xi}^{\mathrm{T}} A \xi=\lambda \bar{\xi}^{\mathrm{T}} \xi \Longrightarrow(\lambda-\bar{\lambda}) \bar{\xi}^{\mathrm{T}} \xi=0$,  而  $\bar{\xi}^{\mathrm{T}} \xi \neq 0 \Longrightarrow \lambda=\bar{\lambda}^{\prime}$  于是 

  \item  若  $\beta_{2}, \cdots, \beta_{n}$  线性相关 ,  则等式两边都为  0 ,  原等式自然成立 .  下设  $\beta_{2}, \cdots, \beta_{n}$  线性无关 ,  将  $\beta_{2}, \cdots, \beta_{n}$  扩   充为  $F^{n}$  的一组基  $\gamma, \beta_{2}, \cdots, \beta_{n}$.  设 

\end{enumerate}
$$
\beta_{1}=\left(\begin{array}{llll}
\gamma & \beta_{2} & \cdots & \beta_{n}
\end{array}\right)\left(\begin{array}{c}
x_{1} \\
x_{2} \\
\vdots \\
x_{n}
\end{array}\right), \alpha_{i}=\left(\begin{array}{llll}
\gamma & \beta_{2} & \cdots & \beta_{n}
\end{array}\right)\left(\begin{array}{c}
p_{1 i} \\
p_{2 i} \\
\vdots \\
p_{n i}
\end{array}\right), 1 \leq i \leq n .
$$
 令  $P=\left(p_{i j}\right)_{n \times n}$,  此时原来等式的   注   曾经还想到过另外一个证明方法 :  设  $A=\left(\begin{array}{llll}\alpha_{1} & \alpha_{2} & \ldots & \alpha_{n}\end{array}\right), B=\left(\begin{array}{lll}\beta_{1} & \beta_{2} & \cdots\end{array} \beta_{n}\right)$,  记  $\alpha_{i}$  的  $j$  个元   素为  $\alpha_{i j}, \beta_{i}$  的  $j$  个元素为  $\beta_{i j}, 1 \leq i, j \leq n$,  则 
$$
\begin{aligned}
\text { 右边 } &=\sum_{i=1}^{n} \sum_{k=1}^{n} \beta_{1 k} A_{k i} \sum_{j=1}^{n} \alpha_{i j} B_{1 j}=\sum_{k=1}^{n} \sum_{j=1}^{n} \beta_{1 k} B_{1 j}\left(\sum_{i=1}^{n} \alpha_{i j} A_{k i}\right) \\
&=\sum_{k=1}^{n} \sum_{j=1}^{n} \beta_{1 k} B_{1 j}\left(\delta_{j k}|A|\right)=|A| \sum_{j=1}^{n} \sum_{j=1}^{n} \beta_{1 j} B_{1 j}=|A||B|=\text { 左边. }
\end{aligned}
$$

\begin{enumerate}
  \setcounter{enumi}{5}
  \item  详细证明请翻阅蓝以中老师的 《 高等代数简明教程 》 第二版下册第  161  页中的  Jordan-Chevalley  分解定理 . TangSong  回忆的试题中应该是漏了条件 ,  例如考虑矩阵  $\left(\begin{array}{cc}0 & 1 \\ -1 & 0\end{array}\right)$,  在  $M_{2}(\mathbb{R})$  中就不能做出满足条件的分   解 .

  \item  设  $A-B=H$,  则  $H P=P H, B P-P B=2 H \Longrightarrow B P^{2}-P^{2} B=(B P-P B) P+P(B P-P B)=4 P H$,  假设  $B P^{n-1}-P^{n-1} B=2(n-1) P^{n-2} H$,  则  $B P^{n}-P^{n} B=(B P-P B 2) P^{n-1}+P\left(B P^{n-1}-P^{n-1} B\right)=$ $2 H P^{n-1}+P\left(2(n-1) P^{n-2} H\right)=2 n P^{n-1} H$,  由数学归纳法得  $B P^{n}-P^{n} B=2 n P^{n-1} H, n \in \mathbb{N}_{+}$.  而  $A Q=Q B \Longleftrightarrow(B+H) Q=Q B \Longleftrightarrow B Q-Q B=-H Q$.  我们需要找到一个  $Q$  使前面的等式成立 ,  设  $Q=a_{0} E+a_{1} P+\cdots+a_{n-1} P^{n-1}+a_{n} P^{n}$,  然后试着确定系数 .  此时 

\end{enumerate}
$$
\begin{aligned}
B Q-Q B &=2 H\left(a_{1}+2 a_{2} P+\cdots+(n-1) a_{n-1} P^{n-2}+n a_{n} P^{n-1}\right) \\
-H Q &=-H\left(a_{0}+a_{1} P+\cdots+a_{n-1} P^{n-1}+a_{n} P^{n}\right) \\
&=-H\left(a_{0}+a_{1} P+\cdots+a_{n-1} P^{n-1}\right)
\end{aligned}
$$
$$
\left\{\begin{array} { r l } 
{ - a _ { 0 } } & { = a _ { 1 } \times 2 } \\
{ - a _ { 1 } } & { = 2 a _ { 2 } \times 2 } \\
{ - a _ { 2 } } & { = 3 a _ { 3 } \times 2 } \\
{ \cdots } & { \text { 取 } a _ { 0 } = 1 } \\
{ - a _ { n - 1 } } & { = n a _ { n } \times 2 }
\end{array} \Longrightarrow \left\{\begin{array}{l}
a_{1}=-\frac{1}{2} \\
a_{2}=-\left(-\frac{1}{2}\right)^{2} \frac{1}{2 !} \\
a_{3}=-\left(-\frac{1}{2}\right)^{3} \frac{1}{3 !} \\
\cdots \\
a_{n}=-\left(-\frac{1}{2}\right)^{n} \frac{1}{n !}
\end{array}\right.\right.
$$
 于是  $Q=E+\left(-\frac{1}{2}\right) P+\left(-\frac{1}{2}\right)^{n} P^{n}=\mathrm{e}^{-\frac{P}{2}}$,  此时  $Q$  是可逆的 ,  从而就是满足题意的一个  $Q$.

 注   此题较难 ,  这里给出的证明方法改编自  succeme  的解法 ,  引入  $A-B=H$  简化了计算 ,  设  $Q$  为  $P$  的多项式   这一想法比较自然 , $P$  幂零这一性质巧妙地与矩阵函数联系着 ,  很不错的一道题 ,  不知道有没有更好的解法 .

\begin{enumerate}
  \setcounter{enumi}{7}
  \item  因为 
\end{enumerate}
$$
\begin{aligned}
& {[(\vec{a} \times \vec{b}) \times(\vec{b} \times \vec{c})] \cdot(\vec{c} \times \vec{a}) } \\
=&\{[(\vec{a} \times \vec{b}) \cdot \vec{c}] \vec{b}-[(\vec{a} \times \vec{b}) \cdot \vec{b}] \vec{c}\} \cdot(\vec{c} \times \vec{a}) \\
=& {[(\vec{a} \times \vec{b}) \cdot \vec{c}]^{2} }
\end{aligned}
$$
 故 
$$
\begin{aligned}
& \vec{a}, \vec{b}, \vec{c} \text { 共面 } \\
\Longleftrightarrow &(\vec{a} \times \vec{b}) \cdot \vec{c}=0 \\
\Longleftrightarrow & {[(\vec{a} \times \vec{b}) \times(\vec{b} \times \vec{c})] \cdot(\vec{c} \times \vec{a})=0 } \\
\Longleftrightarrow & \vec{a} \times \vec{b}, \vec{b} \times \vec{c}, \vec{c} \times \vec{a} \text { 共面. }
\end{aligned}
$$
 注   这里的证明主要是利用了二重外积公式  $\vec{a} \times(\vec{b} \times \vec{c})=(\vec{a} \cdot \vec{c}) \vec{b}-(\vec{a} \cdot \vec{b}) \vec{c}$,  不用这个公式 ,  直接用   线性相关也能证明 .

\begin{enumerate}
  \setcounter{enumi}{8}
  \item  设  $\overrightarrow{O A}, \overrightarrow{O B}, \overrightarrow{O C}$  的同向单位向量为  $\vec{a}, \vec{b}, \vec{c}$,  则  $\langle\vec{a}, \vec{b}\rangle=\frac{\pi}{2}$, $\langle\vec{a}, \vec{c}\rangle=\frac{\pi}{3},\langle\vec{b}, \vec{c}\rangle=\frac{\pi}{4}$.  利用  Lagrange  恒等式 
\end{enumerate}
$$
(\vec{a} \times \vec{b}) \cdot(\vec{c} \times \vec{d})=\left|\begin{array}{ll}
\vec{a} \cdot \vec{c} & \vec{a} \cdot \vec{d} \\
\vec{b} \cdot \vec{c} & \vec{b} \cdot \vec{d}
\end{array}\right|,
$$
 我们有 

%\includegraphics[max width=\textwidth]{2022_04_18_33b622a7abd81c227674g-085}

 故所求二面角的余弦值为  $\frac{\sqrt{6}}{3}$.

 注   通过建立空间直角坐标系 ,  高中生应该也能把这道题做出来 .  这里用  Lagrange  恒等式的解法很巧妙 ,  源   自陶哲轩小弟的回帖 .

\begin{enumerate}
  \setcounter{enumi}{9}
  \item  以  $\vec{n}$  的方向为  $z$  轴正方向建立空间直角坐标系 ,  设  $F$  的方程为  $\left(\begin{array}{lll}x & y & z\end{array}\right)\left(\begin{array}{lll}a_{11} & a_{12} & a_{13} \\ a_{21} & a_{22} & a_{23} \\ a_{31} & a_{32} & a_{33}\end{array}\right)\left(\begin{array}{l}x \\ y \\ z\end{array}\right)+$ $2\left(\begin{array}{lll}x & y & z\end{array}\right)\left(\begin{array}{l}a_{1} \\ a_{2} \\ a_{3}\end{array}\right)+a_{0}=0$.  与  $\vec{n}$  垂直的平面的方程为  $z=c, c \in \mathbb{R}$,  交线的方程为 
\end{enumerate}
$$
\left\{\left(\begin{array}{lll}
x & y & z
\end{array}\right)\left(\begin{array}{lll}
a_{11} & a_{12} & a_{13} \\
a_{21} & a_{22} & a_{23} \\
a_{31} & a_{32} & a_{33}
\end{array}\right)\left(\begin{array}{l}
x \\
y \\
z
\end{array}\right)+2\left(\begin{array}{lll}
x & y & z
\end{array}\right)\left(\begin{array}{l}
a_{1} \\
a_{2} \\
a_{3}
\end{array}\right)+a_{0}=0\right.
$$
 从而对称中心点满足方程  $\left\{\begin{array}{r}z=c \\ a_{11} x+a_{12} y+a_{13} c+a_{1}=0 \\ a_{12} x+a_{22} y+a_{23} c+a_{2}=0\end{array}\right.$

 从而  $F$  的对称中心点均在直线  $\left\{\begin{array}{l}a_{11} x+a_{12} y+a_{13} z+a_{1}=0 \\ a_{12} x+a_{22} y+a_{23} z+a_{2}=0\end{array}\right.$  上 .

 注   此题与北京大学  2011  年第  7  题第一问没有本质的区别 ,  其实为了完整应该说明下交线为中心对称图形 ,  不   然就没有对称中心点在一条直线上了 ,  但是这里没有要求证明 ,  我就不多写了 .  北京大学  2018  年全国硕士研究生招生考试高代解几试题及解答 

   

2019.04.03

\begin{enumerate}
  \item (15  分 )  试确定实数域上所有的  3  阶  $(0,1)$  行列式  ( 即所有元素只能是  0,1  的行列式 )  的最大值 ,  给出证明   并举一个取到最大值的例子 .

  \item (20  分 )

\end{enumerate}
(1)  试证明 :  任给一个  $\mathbb{K}$  上的  $k$  次多项式  $f(x)$,  一定能找到一个不超过  $k+1$  次的多项式  $S_{f}(x)$,  使得对   每个正整数  $n$,  都有 
$$
S_{f}(n)=\sum_{j=0}^{n-1} f(j) .
$$
(2)  构造一个多项式  $g(x)$  满足对每个正整数  $n$  都有 
$$
g(n)=0^{2}+1^{2}+\cdots+(n-1)^{2} .
$$

\begin{enumerate}
  \setcounter{enumi}{3}
  \item (15  分 ) $V=\mathbb{K}^{4}$  是一个向量空间 , $V_{1}=\mathrm{L}\left(\alpha_{1}, \alpha_{2}, \alpha_{3}, \alpha_{4}\right), V_{2}=\mathrm{L}\left(\beta_{1}, \beta_{2}, \beta_{3}, \beta_{4}, \beta_{5}\right)$,  求  $V_{1} \cap V_{2}$  的一组基 ,  其   中  $\left(\alpha_{1}, \alpha_{2}, \alpha_{3}, \alpha_{4}, \beta_{1}, \beta_{2}, \beta_{3}, \beta_{4}, \beta_{5}\right)=(\ldots)_{4 \times 9}$,  具体数字失察不详 .

  \item (15  分 )  设  $V$  是一个  $n$  维  $\mathbb{K}$- 线性空间 , $\left(\alpha_{j}\right)_{j=1}^{n}$  是它的一组基 .  若对  $V$  上的线性变换  $\mathscr{A}$

\end{enumerate}
$$
\mathscr{A}\left(\begin{array}{c}
\alpha_{1} \\
\alpha_{2} \\
\vdots \\
\alpha_{n}
\end{array}\right)=\left(\begin{array}{c}
\mathscr{A} \alpha_{1} \\
\mathscr{A} \alpha_{2} \\
\vdots \\
\mathscr{A} \alpha_{n}
\end{array}\right)=A\left(\begin{array}{c}
\alpha_{1} \\
\alpha_{2} \\
\vdots \\
\alpha_{n}
\end{array}\right),
$$
 则记  $A=\mathscr{A}_{f}$.  若对任一个线性变换  $\mathscr{B}$  都有  $(\mathscr{A} \mathscr{B})_{f}=\mathscr{A}_{f} \mathscr{B}_{f}$,  试确定  $\mathscr{A}$  在基底  $\left(\alpha_{j}\right)_{j=1}^{n}$  下的矩阵 .

\begin{enumerate}
  \setcounter{enumi}{5}
  \item (15  分 )  证明每个  $n$  阶可逆矩阵  $A$  都可以只通过第三类初等行变换  ( 即把一行乘上一定的倍数加到另一行 )  变成满足以下性质的矩阵  $A:$  对每个  $j$  都只存在一行 ,  该行的前  $j-1$  个元素全是  0 ,  而第  $j$  个元素不是  $0, j=1, \cdots, n$.

  \item (15  分 )  考虑线性空间  $M_{n}(\mathbb{K})$.  称  $V \subset M_{n}(\mathbb{K})$  是一个公共子空间 ,  如果对每个  $A \subset M_{n}(\mathbb{K})$  及每个  $B \in V$  使得  $A B \in V$.

\end{enumerate}
(1)  构造  $(n+1)$  个不同的  $n$  维公共子空间 .

(2)  证明每个  $n$  维公共子空间都是极小的 ,  即若有另外的子空间  $V^{\prime} \subset V$,  则要么  $V=V^{\prime}$  要么  $V^{\prime}=0$.

\begin{enumerate}
  \setcounter{enumi}{7}
  \item (5  分 )  用  Euclidean  空间向量的夹角给出  $n$  阶正交矩阵的一般形式 ,  试给出证明 .

  \item (20  分 )  取定一个平面坐标系 .  方程  $a x^{2}+4 x y+a y^{2}-10 x+20 y-1=0(a>0)$  表示一条抛物线 .

\end{enumerate}
(1)  求  $a$. (2)  求抛物线的顶点 .

\begin{enumerate}
  \setcounter{enumi}{9}
  \item (15  分 )  记  $A$  是与下面三条直线都相交的直线的并集 : $\left\{\begin{array}{l}y=0 \\ z=0\end{array},\left\{\begin{array}{l}x=1 \\ z=1\end{array},\left\{\begin{array}{l}x=-1 \\ y=-1\end{array}\right.\right.\right.$.  给出  $A$  的一个一   般表达式  $f(x, y, z)=0$,  其中  $f$  是一个三元多项式 .

  \item (15  分 )  证明几何空间中任意一个旋转变换  $f$,  只要转轴通过原点 ,  就一定可以写成  $f=g_{z} \circ g_{y} \circ g_{x}$  的形式 ,  其中  $g_{x}, g_{y}, g_{z}$  分别表示绕  $x, y, z$  轴的旋转变换 . 1.  由行列式展开式知 :  最大值不会超过  3.  因为若取到  3 ,  则矩阵为  $\left(\begin{array}{ccc}1 & 1 & 1 \\ 1 & 1 & 1 \\ 1 & 1 & 1\end{array}\right)$,  矛盾 .  因此行列式最大值   小于等于  2 ,  注意到  $\left|\begin{array}{ccc}0 & 1 & 1 \\ 1 & 0 & 1 \\ 1 & 1 & 0\end{array}\right|=2$,  因此是大值为  2 .

\end{enumerate}
 注   丘维声编 《 高等代数 》 创新教材第  73  页第  3  题原题 .

\begin{enumerate}
  \setcounter{enumi}{2}
  \item (1)  设  $f(x)=a_{k} x^{k}+\cdots+a_{0}$.  由  Lagrange  揷值公式知 ,  存在一个次数不超过  $k+1$  次的多项式  $S_{f}(x)$  使   得 
\end{enumerate}
$$
\begin{aligned}
S_{f}(1) &=f(0) \\
S_{f}(2) &=f(0)+f(1) \\
& \cdots \\
S_{f}(k+1) &=f(0)+\cdots+f(k) \\
S_{f}(k+2) &=f(0)+\cdots+f(k)+f(k+1)
\end{aligned}
$$
 于是  $S_{f}(x)-S_{f}(x-1)-f(x-1)$  为次数不超过  $k$  次的多项式 ,  且有  $2,3, \cdots, k+2$  是  $k+1$  个两两   不同的根 .  于是 
$$
S_{f}(x)-S_{f}(x-1)-f(x-1) \equiv 0
$$
 因此  $S_{f}(n)=\sum_{j=0}^{n-1} f(j)$,  对  $\forall n \in \mathbb{N}^{*}$  成立 .

(2) $g(n)=\frac{(n-1) n(2 n-1)}{6}$,  于是令  $g(x)=\frac{(x-1) x(2 x-1)}{6}$  即可 .

 注   与北京大学数学系前代数小组编 《 高等代数 》 第四版第  424  页第  27  题类似 .

\begin{enumerate}
  \setcounter{enumi}{3}
  \item  做行变换得到极大组 ,  同时得到  $\operatorname{dim}\left(V_{1}+V_{2}\right), \operatorname{dim} V_{1}, \operatorname{dim} V_{2}$,  再由维数公式计算出  $\operatorname{dim} V_{1} \cap V_{2}$.
\end{enumerate}
 注   高等代数书上子空间的和与交那部分的常规题目 ,  送分题 .

\begin{enumerate}
  \setcounter{enumi}{4}
  \item  由题意知 :
\end{enumerate}
$$
\mathscr{A} \alpha_{i}=\sum_{j=1}^{n} a_{i j} \alpha_{j}, \quad i=1,2, \ldots, n
$$
 设 
$$
\mathscr{B} \alpha_{i}=\sum_{j=1}^{n} b_{i j} \alpha_{j}, \quad i=1,2, \ldots, n
$$
 则 
$$
\mathscr{A} \mathscr{B} \alpha_{i}=\mathscr{A}\left(\sum_{k=1}^{n} b_{i k} \alpha_{k}\right)=\sum_{k=1}^{n} b_{i k} \mathscr{A} \alpha_{k}=\sum_{k=1}^{n} b_{i k} \sum_{j=1}^{n} a_{k j} \alpha_{j}=\sum_{j=1}^{n}\left(\sum_{k=1}^{n} b_{i k} a_{k j}\right) \alpha_{j}, \quad i=1,2, \ldots, n
$$
 因此 
$$
A B=\mathscr{A}_{f} \mathscr{B}_{f}=(\mathscr{A} \mathscr{B})_{f}=B A, \quad \forall B \in M_{n}(K) \Longrightarrow A=k E, \quad k \in K
$$

\begin{enumerate}
  \setcounter{enumi}{5}
  \item  用数学归纳法来证明 :  可以只通过第三类行变换把一个可逆矩阵变成主对角线上元素均不为零的上三角矩   阵 . $n=1$  时显然成立 .  设结论对阶数小于  $n-1$  阶的可逆矩阵均成立 ,  下面考虑阶数为  $n$  的情况 .  此时第   一列不会全为  0 ,  否则不可逆 .  于是可使用第三类初等行变换将  $a_{11}$  变为非零  ( 若本身非零则不变 ).  通过第   三类初等行变换可以把第一列下方元素均变为  0 ,  此时  $A$  变为  $\left(\begin{array}{cc}a_{11}^{\prime} & * \\ \mathbf{0} & A_{22}^{\prime}\end{array}\right)$  的形式 .  初等行变换不改变   矩阵的可逆性 ,  于是  $A_{22}^{\prime}$  可逆且阶数为  $n-1$,  由归纳假设可由第三类行变换变为上三角矩阵 ,  此时的行变   换只针对第  $2,3, \ldots, n$  行进行 ,  同时第  1  列下方的元素仍然全为  0 ,  由此知最终的矩阵为上三角矩阵 .  由数   学归纳法原理知原命题成立 .

  \item (1)  显然  $V=\mathbf{0}$  及  $V=M_{n}(K)$  为两个平凡的公共子空间 ,  但不是  $n$  维的 .  设  $V_{i}=\operatorname{span}\left\{E_{1 i}, E_{2 i}, \cdots, E_{n i}\right\}$, $i=1,2, \ldots, n$.  则  $V_{i}$  是  $n$  维公共子空间 .  另外 , $V=\left\{(\alpha, \alpha, \mathbf{0}, \ldots, \mathbf{0}) \mid \alpha \in K^{n}\right\}$  也是  $n$  维公共子空间 .

\end{enumerate}
(2)  若  $V^{\prime} \subset V$,  但是  $V^{\prime} \neq 0$,  则存在  $B \in V^{\prime}$  设  $b_{i j} \neq 0$,  则 
$$
E_{i i} B=\left(\begin{array}{c}
\mathbf{0} \\
b_{i 1}, \ldots, b_{i j}, \ldots, b_{i n} \\
\mathbf{0}
\end{array}\right) \in V^{\prime}
$$
 对  $E_{i i} B$  作初等行变换相当于用某个矩阵左乘  $E_{i i} B$,  按照公共子空间的定义知最后得到的矩阵仍然属   于  $V^{\prime}$,  因此第  $k$  行元素为  $b_{i 1}, b_{i 2}, b_{i 3}, \ldots, b_{i n}$,  其余元素为  0  的矩阵属于  $V^{\prime}, k=1,2, \ldots, n$.  这  $n$  个矩   阵是线性无关的 ,  因此  $\operatorname{dim} V^{\prime} \geq n$,  从而  $\operatorname{dim} V^{\prime}=n=\operatorname{dim} V^{\prime}$,  因此  $V=V^{\prime}$.

 注   与蓝以中编 《 高等代数学习指南 》 第  161  页例  $2.11$  类似 .

\begin{enumerate}
  \setcounter{enumi}{7}
  \item  设  $A$  为正交矩阵并且  $\left(\alpha_{1}, \alpha_{2}, \cdots, \alpha_{n}\right)=\left(\beta_{1}, \beta_{2}, \cdots, \beta_{n}\right) A$,  其中  $\alpha_{1}, \alpha_{2}, \cdots, \alpha_{n}$  和  $\beta_{1}, \beta_{2}, \cdots, \beta_{n}$  均为  $n$  维  Euclidean  空间的标准正交基 ,  则 
\end{enumerate}
$$
A=\left(\begin{array}{cccc}
\cos \left(\beta_{1}, \alpha_{1}\right) & \cos \left(\beta_{1}, \alpha_{2}\right) & \cdots & \cos \left(\beta_{1}, \alpha_{n}\right) \\
\cos \left(\beta_{2}, \alpha_{1}\right) & \cos \left(\beta_{2}, \alpha_{2}\right) & \cdots & \cos \left(\beta_{2}, \alpha_{n}\right) \\
\vdots & \vdots & & \vdots \\
\cos \left(\beta_{n}, \alpha_{1}\right) & \cos \left(\beta_{n}, \alpha_{2}\right) & \cdots & \cos \left(\beta_{n}, \alpha_{n}\right)
\end{array}\right)
$$
 注   灵感来源于丘维声编 《 解析几何 》 第三版第  132  页 .

\begin{enumerate}
  \setcounter{enumi}{8}
  \item (1)
\end{enumerate}
$$
\left.\begin{array}{rl}
\left|\begin{array}{ll}
a & 2 \\
2 & a
\end{array}\right| & =0 \\
& a>0
\end{array}\right\} \Longrightarrow a=2
$$
$(2)$
$$
\begin{gathered}
2 x^{2}+4 x y+2 y^{2}-10 x+20 y-1=0 \\
\left(\begin{array}{l}
x_{1} \\
y_{1}
\end{array}\right)=\left(\begin{array}{rr}
\frac{1}{\sqrt{2}} & \frac{1}{\sqrt{2}} \\
-\frac{1}{\sqrt{2}} & \frac{1}{\sqrt{2}}
\end{array}\right)\left(\begin{array}{l}
x \\
y
\end{array}\right)
\end{gathered}
$$
 原方程变为 
$$
4 x_{1}^{2}-10 \frac{x_{1}-y_{1}}{\sqrt{2}}+20 \frac{x_{1}+y_{1}}{\sqrt{2}}-1=0
$$
 化简方程得 
$$
\begin{gathered}
4 x_{1}^{2}+5 \sqrt{2} x_{1}+15 \sqrt{2} y_{1}=1 \\
\left(x_{1}+\frac{5 \sqrt{2}}{8}\right)^{2}=\frac{33}{32}-\frac{15 \sqrt{2}}{4} y_{1}
\end{gathered}
$$
 于是顶点坐标为 
$$
\left(\begin{array}{cc}
\frac{1}{\sqrt{2}} & \frac{1}{\sqrt{2}} \\
-\frac{1}{\sqrt{2}} & \frac{1}{\sqrt{2}}
\end{array}\right)^{-1}\left(\begin{array}{r}
\frac{-5 \sqrt{2}}{8} \\
\frac{11 \sqrt{2}}{80}
\end{array}\right)=\left(\begin{array}{r}
-\frac{61}{80} \\
-\frac{39}{80}
\end{array}\right)
$$

\begin{enumerate}
  \setcounter{enumi}{9}
  \item  设  $(x, y, z)$  为与三条直线都相交的直线上的点且该直线的方向向量为  $(u, v, w)$,  则由 
\end{enumerate}
$$
\left|\begin{array}{ccc}
x & y & z \\
1 & 0 & 0 \\
u & v & w
\end{array}\right|=0,\left|\begin{array}{ccc}
x-1 & y & z-1 \\
0 & 1 & 0 \\
u & v & w
\end{array}\right|=0,\left|\begin{array}{ccc}
x+1 & y+1 & z \\
0 & 0 & 1 \\
u & v & w
\end{array}\right|=0
$$
 可得 
$$
\left\{\begin{array}{r}
y w-v z=0 \\
(x-1) w-(z-1) u=0 \\
(x+1) v-(y+1) u=0
\end{array}\right.
$$
 因为  $(u, v, w) \neq \mathbf{0}$,  因此上述线性方程组有非零解 ,  从而 
$$
\left|\begin{array}{ccc}
0 & -z & y \\
1-z & 0 & x-1 \\
-y-1 & x+1 & w
\end{array}\right|=0
$$
 化简得 : $-2 y z+x z+x y-z+y=0$.

\begin{enumerate}
  \setcounter{enumi}{10}
  \item  可设  $f$  的方程为  $\left(\begin{array}{l}x_{1} \\ y_{1} \\ z_{1}\end{array}\right)=A\left(\begin{array}{l}x \\ y \\ z\end{array}\right)$,  其中  $A^{\mathrm{T}} A=E$,  且  $|A|>0$.  设 
\end{enumerate}
$$
\begin{aligned}
B &=\left(\begin{array}{ccc}
\cos \theta_{3} & -\sin \theta_{3} & \\
\sin \theta_{3} & \cos \theta_{3} & \\
& 1
\end{array}\right)\left(\begin{array}{ccc}
\cos \theta_{2} & -\sin \theta_{2} \\
1 & \\
\sin \theta_{2} & & \cos \theta_{2}
\end{array}\right)\left(\begin{array}{ccc}
1 & \\
\cos \theta_{1} & -\sin \theta_{1} \\
\sin \theta_{1} & \cos \theta_{1}
\end{array}\right) \\
&=\left(\begin{array}{ccc}
\cos \theta_{2} \cos \theta_{3} & -\sin \theta_{1} \sin \theta_{2} \cos \theta_{3}-\cos \theta_{1} \sin \theta_{3} & -\cos \theta_{1} \sin \theta_{2} \cos \theta_{3}+\sin \theta_{1} \sin \theta_{3} \\
\cos \theta_{2} \sin \theta_{3} & -\sin \theta_{1} \sin \theta_{2} \sin \theta_{3}+\cos \theta_{1} \cos \theta_{3} & -\cos \theta_{1} \sin \theta_{2} \sin \theta_{3}-\sin \theta_{1} \cos \theta_{3} \\
\sin \theta_{2} & \sin \theta_{1} \cos \theta_{2} & \cos \theta_{1} \cos \theta_{2}
\end{array}\right) .
\end{aligned}
$$
 要证明原命题成立 ,  只需说明  $\exists \theta_{1}, \theta_{2}, \theta_{3}$  使得  $A=B$.  设  $A=\left(a_{i j}\right)_{3 \times 3}, B=\left(b_{i j}\right)_{3 \times 3}$.

(1)  当  $\left|a_{31}\right| \neq 1$  时 ,  取  $\theta_{1}, \theta_{2}, \theta_{3}$  使得 
$$
\left\{\begin{array}{l}
\sin \theta_{1}=\frac{a_{32}}{\sqrt{1-a_{31}^{2}}} \\
\cos \theta_{1}=\frac{a_{33}}{\sqrt{1-a_{31}^{2}}}
\end{array},\left\{\begin{array}{l}
\sin \theta_{2}=a_{31} \\
\cos \theta_{2}=\sqrt{1-a_{31}^{2}}
\end{array},\left\{\begin{array}{l}
\sin \theta_{3}=\frac{a_{11}}{\sqrt{1-a_{31}^{2}}} \\
\cos \theta_{3}=\frac{a_{21}}{\sqrt{1-a_{31}^{2}}}
\end{array}\right.\right.\right.
$$
 这个时候将有  $b_{11}=a_{11}, b_{21}=a_{21}, b_{31}=a_{31}, b_{32}=a_{32}, b_{33}=a_{33}$.  注意到此时  $B$  为行列式为  1  的三阶   正交矩阵 ,  要说明这个时候就有  $A=B$,  那么只需说明 :

 对于任意一个行列式为  1  的三阶正交矩阵  $A$,  当  $a_{11}, a_{21}, a_{31}, a_{32}, a_{33}$  确定且  $\left|a_{31}\right| \neq 1$  时 , $a_{12}, a_{22}, a_{13}, a_{23}$  就唯一确定了 .

 事实上 ,  记  $a_{12}=x, a_{22}=y$  则  $x, y$  满足方程 
$$
\left\{\begin{aligned}
a_{11} x+a_{21} y+a_{31} a_{32} &=0 \\
a_{11} y-a_{21} x &=a_{33} \\
x^{2}+y^{2}+a_{32}^{2} &=1
\end{aligned}\right.
$$
 直接计算知上述方程组的解是存在唯一的 ,  从而  $a_{12}, a_{22}$  被唯一确定了 ,  同理可证  $a_{13}, a_{23}$  也是存在唯   一的 . (2)  当  $\left|a_{31}\right|=1$  时 ,  若  $a_{31}=1$,  此时  $A=\left(\begin{array}{cc}\sin \theta & -\cos \theta \\ \cos \theta & \sin \theta \\ 1\end{array}\right)$.  取  $\theta_{2}=\pi / 2,-\left(\theta_{1}+\theta_{3}\right)=\theta$,  则此时 

%\includegraphics[max width=\textwidth]{2022_04_18_33b622a7abd81c227674g-091}

 当  $a_{31}=-1$  时是类似的 .

 注   这道题其实考的  Euler angles,  在丘维声编的 《 高等代数 》 创新教材第  565  页有提到 ,  在丘维声编的 《 解析   几何 》 第三版第  144  页第  12  题也有提及 , 但是都没有给出旋转对应的矩阵能表示成三个绕坐标轴的旋转对   应的矩阵的乘积的证明 ,  此处给出的证明为原创的 .  北京大学  2019  年全国硕士研究生招生考试高代解几试题及解答 

   

2019.03.23

 注 :  本试题中  $\mathrm{r}(A)$  表示矩阵  $A$  的秩 ; $E_{i j}$  表示第  $i$  行第  $j$  列元素为  1 ,  其余元素全为  0  的矩阵 ; $A^{\mathrm{T}}$  表示矩阵  $A$  的转置 ; $|A|$  表示矩阵  $A$  的行列式 .

\begin{enumerate}
  \item (20  分 )  设  $\alpha_{1}, \alpha_{2}, \cdots, \alpha_{r}$  是  $\mathbb{R}^{m}$  中线性无关的列向量组 , $\beta_{1}, \beta_{2}, \cdots, \beta_{s}$  是  $\mathbb{R}^{n}$  中线性无关的列向量组 .  求证 :  若有实数  $c_{i j}$  使得  $\sum_{i=1}^{r} \sum_{j=1}^{s} c_{i j} \alpha_{i} \beta_{j}^{\mathrm{T}}=\mathbf{0}$,  则  $c_{i j}=0, i=1,2, \cdots, r, j=1,2, \cdots, s$.

  \item (20  分 )  设  $A$  是  3  阶实方阵 , $A A^{\mathrm{T}}=A^{\mathrm{T}} A$,  且  $A \neq A^{\mathrm{T}}$.

\end{enumerate}
(1)  证明存在正交矩阵  $P$,  使得  $P^{\mathrm{T}} A P=\left(\begin{array}{ccc}a & 0 & 0 \\ 0 & b & c \\ 0 & -c & b\end{array}\right)$,  其中  $a, b, c$  都是实数 .

(2)  若  $A=\sum_{i=1}^{3} \sum_{j=1}^{3} a_{i j} E_{i j}, A A^{\mathrm{T}}=A^{\mathrm{T}} A=I_{3}$,  且  $|A|=1$.  证明  1  是  $A$  的一个特征值 ,  并求特征值  1  对应   的特征向量 .

\begin{enumerate}
  \setcounter{enumi}{3}
  \item (20  分 ) $A \in M_{n}(\mathbb{C}), A$  的特征值为  $\lambda_{1}, \lambda_{2}, \cdots, \lambda_{n}$.  定义  $M_{n}(\mathbb{C})$  上的线性变换  $T$  为 
\end{enumerate}
$$
\begin{aligned}
T: M_{n}(\mathbb{C}) & \longrightarrow M_{n}(\mathbb{C}) \\
B & \mapsto A B-B A
\end{aligned}
$$
(1)  求变换  $T$  的特征值 ;

(2) 若  $A$  可对角化 ,  证明  $T$  也可对角化 .

\begin{enumerate}
  \setcounter{enumi}{4}
  \item (20  分 )  设  $A \in M_{n}(\mathbb{R}), A^{\mathrm{T}}=A, S=\left\{X \in \mathbb{R}^{n} \mid X^{\mathrm{T}} A X=0\right\}$.
\end{enumerate}
(1)  给出使  $S$  为  $\mathbb{R}^{n}$  中的一个子空间的充要条件并证明 ;

(2)  若  $\mathrm{r}(A)=r<n$  且  $S$  为  $\mathbb{R}^{n}$  中的一个子空间 ,  求  $\operatorname{dim} S$.

\begin{enumerate}
  \setcounter{enumi}{5}
  \item (20  分 )  设  $\varepsilon$  是事先给定的正数 ,  证明对任意的  $n$  阶实方阵  $A$,  存在一个  $n$  阶对角矩阵  $D, D$  的每个对角元   为  $\varepsilon$  或  $-\varepsilon$  中的一个 ,  使得  $|A+D| \neq 0$.

  \item (15  分 )  设  $l_{1}:\left\{\begin{array}{l}x+y+z+1=0 \\ x+2 y+3 z+3=0\end{array}, l_{2}: \frac{x-1}{1}=\frac{y-1}{2}=\frac{z-1}{3}\right.$,  求这两条直线的距离和公垂线的方程 .

  \item (20  分 )  在空间中有三条直线  $l_{1}, l_{2}, l_{3}$  两两异面 ,  且不平行于同一个平面 ,  证明空间中与这三条直线都共面的   直线的并集是一个单叶双曲面 .

  \item (15  分 )  证明平面与双曲抛物面的交线不可能是一个椭圆 . 1.  令  $A=\left(\alpha_{1}, \alpha_{2}, \cdots, \alpha_{r}\right), B=\left(\beta_{1}, \beta_{2}, \cdots, \beta_{s}\right), C=\left(c_{i j}\right)_{r \times s}$,  则  $\mathbf{0}=\sum_{i=1}^{r} \sum_{j=1}^{s} c_{i j} \alpha_{i} \beta_{j}^{\mathrm{T}}=A C B^{\mathrm{T}}$,  因为  $\mathrm{r}(A)=r$,  因此  $C B^{\mathrm{T}}=\mathbf{0}$,  于是  $B C^{\mathrm{T}}=\mathbf{0}$,  又因为  $\mathrm{r}(B)=s$,  因此  $C^{\mathrm{T}}=\mathbf{0}$,  因此  $c_{i j}=0, i=1,2, \cdots, r, j=$ $1,2, \cdots, s$.

\end{enumerate}
 注   此题与  Michael Artin  的  Algebra  英文版第二版第  100  页题  $3.7$  类似 .

\begin{enumerate}
  \setcounter{enumi}{2}
  \item (1)  由于  $A \in M_{3}(\mathbb{R})$,  故必有一个实的特征值 ,  设为  $a$,  设它对应的一个单位特征向量为  $\xi_{1}$,  将  $\xi_{1}$  扩充为一组   标准正交基  $\xi_{1}, \xi_{2}, \xi_{3}$,  则  $A \xi_{1}=a \xi_{1}$,  设  $P=\left(\xi_{1}, \xi_{2}, \xi_{3}\right)$,  则  $P^{\mathrm{T}} A P=P^{\mathrm{T}}\left(a \xi_{1}, A \xi_{2}, A \xi_{3}\right)=\left(\begin{array}{cc}a & \alpha^{\mathrm{T}} \\ \mathbf{0}\end{array}\right)$,  因为  $A A^{\mathrm{T}}=A^{\mathrm{T}} A$,  所以  $\left(P^{\mathrm{T}} A P\right)\left(P^{\mathrm{T}} A P\right)^{\mathrm{T}}=\left(P^{\mathrm{T}} A P\right)^{\mathrm{T}}\left(P^{\mathrm{T}} A P\right)$,  即  $\left(\begin{array}{cc}a & \alpha^{\mathrm{T}} \\ \mathbf{0} & B\end{array}\right)\left(\begin{array}{ll}a & \mathbf{0} \\ \alpha & B^{\mathrm{T}}\end{array}\right)=$ $\left(\begin{array}{ll}a & \mathbf{0} \\ \alpha & B^{\mathrm{T}}\end{array}\right)\left(\begin{array}{cc}a & \alpha^{\mathrm{T}} \\ \mathbf{0} & B\end{array}\right)$  于是  $\left(\begin{array}{cc}a^{2}+\alpha^{\mathrm{T}} \alpha & \alpha^{\mathrm{T}} B^{\mathrm{T}} \\ B \alpha & B B^{\mathrm{T}}\end{array}\right)=\left(\begin{array}{cc}a^{2} & a \alpha^{\mathrm{T}} \\ a \alpha & B^{\mathrm{T}} B+\alpha \alpha^{\mathrm{T}}\end{array}\right)$,  故  $\left\{\begin{array}{c}\alpha=0 \\ B B^{\mathrm{T}}=B^{\mathrm{T}} B\end{array}\right.$,  于是  $P^{\mathrm{T}} A P=\left(\begin{array}{ll}a & \mathbf{0} \\ \mathbf{0} & B\end{array}\right)$.  设  $B=\left(\begin{array}{cc}e & f \\ g & h\end{array}\right)$,  则  $\left(\begin{array}{cc}e & f \\ g & h\end{array}\right)\left(\begin{array}{cc}e & g \\ f & h\end{array}\right)=\left(\begin{array}{ll}e & g \\ f & h\end{array}\right)\left(\begin{array}{ll}e & f \\ g & h\end{array}\right)$, $\Longrightarrow\left(\begin{array}{cc}e^{2}+f^{2} & e g+f h \\ g e+f h & g^{2}+h^{2}\end{array}\right)=\left(\begin{array}{cc}e^{2}+g^{2} & e f+g h \\ f e+g h & f^{2}+h^{2}\end{array}\right), \Longrightarrow f^{2}=g^{2}$,  必定有  $f=-g \neq 0$,  否则与  $A^{\mathrm{T}} \neq A$  矛盾 ,  令  $f=c$,  则  $c(e-h)=-c(e-h), \Longrightarrow e=h$,  记  $e=b$,  则  $P^{\mathrm{T}} A P=\left(\begin{array}{cc}a & 0\end{array}\right.$  其中  $a, b, c$  都是实数 .
\end{enumerate}
(2)  因为  $A A^{\mathrm{T}}=A^{\mathrm{T}} A=I_{3}$  且  $|A|=1$,  于是  $\left|I_{3}-A\right|=\left|A^{\mathrm{T}} A-A\right|=\left|A^{\mathrm{T}}-I_{3}\right||A|=|A|\left|A-I_{3}\right|=$ $-\left|I_{3}-A\right|, \Longrightarrow\left|I_{3}-A\right|=0$,  于是  1  是  $A$  的一个特征值 , $\left(I_{3}-A\right) X=0$  的非零解即为特征值  1  对   应的特征向量 .  由  $A X=X$,  得  $A^{\mathrm{T}} A X=A^{\mathrm{T}} X$,  即  $X=A^{\mathrm{T}} X$,  因此  $A X=A^{\mathrm{T}} X$.  因为  $A-A^{\mathrm{T}}=$ $\left(\begin{array}{ccc}0 & a_{12}-a_{21} & a_{13}-a_{31} \\ a_{21}-a_{12} & 0 & a_{23}-a_{32} \\ a_{31}-a_{13} & a_{32}-a_{23} & 0\end{array}\right)$,  结合  $A \neq A^{\mathrm{T}}$  知  $X_{0}=\left(\begin{array}{l}a_{32}-a_{23} \\ a_{13}-a_{31} \\ a_{21}-a_{12}\end{array}\right)$  为  $\left(A-A^{\mathrm{T}}\right) X=0$  的一个   非零解 .  此时  $A\left(A-A^{\mathrm{T}}\right) X_{0}=0$,  即  $\left(A^{2}-I\right) X_{0}=0$,  于是  $(A+I)(A-I) X_{0}=0$,  由  $A \neq A^{\mathrm{T}}$  可知  $\operatorname{det}(A+I) \neq 0$,  从而  $(A-I) X_{0}=0$,  因此  $k X_{0}(k \neq 0)$  为特征值  1  对应的特征向量 .

 注   实际上考的正规变换矩阵的正交相似标准型 ,  在蓝以中老师的 《 高等代数学习指南 》 第  306  页有更一般性   的结论 ,  在丘维声老师的 《 高等代数 》 创新教材下册也能找到类似的题目 .  此题第二问为  Michael Artin  的  Algebra  英文版第二版第  151  页题  $1.5$  的  (b).

\begin{enumerate}
  \setcounter{enumi}{3}
  \item (1)  设  $\lambda$  是  $T$  的特征值 , $B(B \neq \mathbf{0})$  是其对应的一个特征向量 ,  则  $T(B)=\lambda B$,  即为  $A B-B A=\lambda B$, $\Longleftrightarrow$ $A B=B(\lambda E+A), \Longleftrightarrow A$  与  $\lambda E+A$  有相同的特征值 , $\Longleftrightarrow \exists i, j \in\{1,2, \cdots, n\}, \lambda_{i}=\lambda+\lambda_{j}, \Longleftrightarrow$ $\exists i, j \in\{1,2, \cdots, n\}, \lambda=\lambda_{i}-\lambda_{j}$.  于是  $T$  的所有特征值构成的集合为  $\left\{\lambda_{i}-\lambda_{j} \mid i, j=1,2, \cdots, n .\right\}$.  特别地 , 0  始终是  $T$  的特征值 .  前面的论述告诉我们  $T$  的特征值的所有可能取值 ,  但是并没有告   诉我们那些特征值的代数重数 .  下面用一种更直接的方法来证明一个更精确的结果 : $T$  的特征值为  $\lambda_{i}-\lambda_{j}, i, j=1,2, \cdots, n$.  也即是说  $T$  的  $n^{2}$  个特征值为让  $i, j$  分别取遍  1  到  $n$  计算得到的  $\lambda_{i}-\lambda_{j}$.  特别地 , 0  为  $T$  的特征值且代数重数至少为  $n$.  三角矩阵 ,  并且主对角线上元素分别为它们的特征值 .  由于 
\end{enumerate}
$$
\left(P_{1}^{-1} \otimes P_{2}^{-1}\right)\left(P_{1} \otimes P_{2}\right)=\left(P_{1}^{-1} P_{1}\right) \otimes\left(P_{2}^{-1} P_{2}\right)=E_{n} \otimes E_{n}=E_{n^{2}}
$$
 于是  $A \otimes E-E \otimes A^{\mathrm{T}}$  相似于 
$$
\left(P_{1}^{-1} \otimes P_{2}^{-1}\right)\left(A \otimes E-E \otimes A^{\mathrm{T}}\right)\left(P_{1} \otimes P_{2}\right)=\left(P_{1}^{-1} A P_{1}\right) \otimes E-E \otimes\left(P_{2}^{-1} A^{\mathrm{T}} P_{2}\right)
$$
 这个矩阵为上三角矩阵 ,  主对角线上元素即为  $T$  的特征值 ,  由此可得  $T$  的特征值为  $\lambda_{i}-\lambda_{j}, i, j=$ $1,2, \cdots, n$.

(2)  因  $A$  可对角化 ,  于是  $A$  有  $n$  个线性无关的特征向量 ,  设  $A \xi_{i}=\lambda_{i} \xi_{i}, i=1,2, \cdots, n$,  令  $P=$ $\left(\xi_{1}, \xi_{2}, \cdots, \xi_{n}\right)$,  则  $P^{-1} A P=\operatorname{diag}\left\{\lambda_{1}, \lambda_{2}, \cdots, \lambda_{n}\right\}$,  于是  $T\left(P E_{i j} P^{-1}\right)=\left(\lambda_{i}-\lambda_{j}\right) P E_{i j} P^{-1}$,  这说明  $P E_{i j} P^{-1}$  是  $T$  的特征向量 ,  于是  $T$  有  $n^{2}$  个线性无关的特征向量 ,  从而  $T$  也可对角化 .

 注   上面证明中用到下维声老师的 《 高等代数 》 创新教材下册  301  页例  27  给出的一个结论  ( 容易证明 ):

 设  $A, B$  分别是  $n$  阶 , $m$  阶复方阵 ,  则矩阵方程  $A X-X B=\mathbf{0}$  只有零解的充分必要条件是  $A$  与  $B$  没有公共特征值 .

 与此题相关的题目 :  中国科学技术大学线性代数考研试题  2013  年第  10  题 , 2009  年第  7  题 .  矩阵张量积的   知识请自行查阅教材 ,  这里只说一点 ,  定义  $A \otimes B=\left(a_{i j} B\right)$,  即运算所得矩阵是一个更大的矩阵 ,  若看做分   成相同大小的分块矩阵 ,  则第  $i$  行第  $j$  列对应的小矩阵为  $a_{i j} B$.  此题第一问与  Michael Artin  的  Algebra  英   文版第二版第  151  页题  $2.3$  几乎完全一样 ,  利用该书第五章第二节中的内容可以给出此题的另一个做法 .  此   题第一小问相当于是要证明 :

 当  $A$  的特征值为  $\lambda_{1}, \lambda_{2}, \cdots, \lambda_{n}$  时 , $T$  的特征值为  $\lambda_{i}-\lambda_{j}, i, j=1,2, \cdots, n$.

 约定矩阵序列收敛是指矩阵序列中相同行相同列的元素构成的数列均收敛 ,  多项式序列收敛是指多项式的   同次幂的系数构成的数列均收敛 .

 先考虑  $A$  是可对角化的矩阵的情形 ,  上面证明原问题的第二小问的过程中就包含了这一结果 .

 然后考虑一般的矩阵  $A$,  我们可以取一列可对角化的矩阵  $A_{k} \rightarrow A,(k \rightarrow \infty)$,  并且  $A_{k}$  的特征值  $\lambda_{i}^{(k)}$  趋于  $A$  的特征值  $\lambda_{i}$,  即  $\lambda_{i}^{(k)} \rightarrow \lambda_{i}(k \rightarrow \infty)$.  设  $T$  在基  $E_{11}, E_{12}, \ldots, E_{1 n}, E_{21}, E_{22}, \ldots, E_{2 n}, \ldots, E_{n 1}, E_{n 2}, \ldots, E_{n n}$
$$
p_{k}(x)=\prod_{i=1}^{n} \prod_{j=1}^{n}\left(x-\left(\lambda_{i}^{(k)}-\lambda_{j}^{(k)}\right)\right)
$$
$$
p_{k}(x) \rightarrow \prod_{i=1}^{n} \prod_{j=1}^{n}\left(x-\left(\lambda_{i}-\lambda_{j}\right)\right), \quad(k \rightarrow \infty)
$$
$$
p(x)=\prod_{i=1}^{n} \prod_{j=1}^{n}\left(x-\left(\lambda_{i}-\lambda_{j}\right)\right)
$$
 点可以这样 :  设可逆矩阵  $P$  使得  $P^{-1} A P$  为上三角矩阵 ,  考虑矩阵  $A(t)=A+P \operatorname{diag}\left\{t, t^{2}, \ldots, t^{n}\right\} P^{-1}$,  则  $A(t)$  的特征   值为  $\lambda_{i}+t^{i}, 1 \leq i \leq n$.  使  $A(t)$  有相同特征值的复数  $t$  只有有限个 ,  因此存在无穷个  $t \in \mathbb{C}$  使  $A(t)$  的  $n$  个   特征值均不相同 ,  这样的  $A(t)$  为可对角化矩阵 ,  利用已经知道的  $A$  为可对角化矩阵时的结果 ,  这时线性变   换  $T_{A(t)}$  在基  $E_{11}, E_{12}, \ldots, E_{1 n}, E_{21}, E_{22}, \ldots, E_{2 n}, \ldots, E_{n 1}, E_{n 2}, \ldots, E_{n n}$  下的矩阵  $M(t)$  的特征多项式为 
$$
|\lambda E-M(t)|=\prod_{i=1}^{n} \prod_{j=1}^{n}\left(\lambda-\left(\lambda_{i}-\lambda_{j}+t^{i}-t^{j}\right)\right)
$$
 先将两端看作  $\lambda$  的多项式 ,  同次项的系数应该相等 ,  系数为  $t$  的多项式 ,  而使等式成立的  $t$  的个数是无限   的 ,  因此原先对除去有限个  $t$  成立的式子实际上对任意  $t \in \mathbb{C}$  均成立 .  特别地 ,  原线性变换  $T=T_{A(0)}$  在基  $E_{11}, E_{12}, \ldots, E_{1 n}, E_{21}, E_{22}, \ldots, E_{2 n}, \ldots, E_{n 1}, E_{n 2}, \ldots, E_{n n}$  下的矩阵为  $M(0)$,  它的特征多项式为 
$$
|\lambda E-M(0)|=\prod_{i=1}^{n} \prod_{j=1}^{n}\left(\lambda-\left(\lambda_{i}-\lambda_{j}\right)\right)
$$
 于是  $T$  的特征值为  $\lambda_{i}-\lambda_{j}, i, j=1,2, \cdots, n$.

\begin{enumerate}
  \setcounter{enumi}{4}
  \item (1) $A$  半正定或者半负定 .
\end{enumerate}
(2) $\operatorname{dim} S=n-r$.

 注   详细解答见丘维声老师的 《 高等代数 》 创新教材下册第  440,441  页例  13  及例  14 .

\begin{enumerate}
  \setcounter{enumi}{5}
  \item  当矩阵阶数为  1  时 ,  原命题成立 .
\end{enumerate}
 假设原命题对阶数小于  $n$  的矩阵均成立 .  现考虑矩阵阶数为  $n$  的情况 .  设  $A=\left(\begin{array}{cc}a_{11} & \alpha^{\mathrm{T}} \\ \beta & A_{1}\end{array}\right)$.  由归纳假设 ,  存在对角线上元素为  $\varepsilon$  或者  $-\varepsilon$  的对角矩阵  $D_{1}$,  使得  $\left|A_{1}+D_{1}\right| \neq 0$.  设  $D=(d)$,  其中  $d$  待定 .  注   意到 
$$
\begin{aligned}
|A+D| &=\left|\begin{array}{cc}
a_{11}+d & \alpha^{\mathrm{T}} \\
\beta & A_{1}+D_{1}
\end{array}\right| \\
&=\left|\begin{array}{cc}
a_{11}+d-\alpha^{\mathrm{T}}\left(A_{1}+D_{1}\right)^{-1} \beta & 0 \\
\beta & A_{1}+D_{1}
\end{array}\right| \\
&=\left|A_{1}+D_{1}\right|\left(a_{11}+d-\alpha^{\mathrm{T}}\left(A_{1}+D_{1}\right)^{-1} \beta\right)
\end{aligned}
$$
 可以选择  $d=\varepsilon$  或者  $d=-\varepsilon$  使得  $\left(a_{11}+d-\alpha^{\mathrm{T}}\left(A_{1}+D_{1}\right)^{-1} \beta\right) \neq 0$,  于是就得到满足条件的对角矩阵  $D$.  由数学归纳法原理知原命题成立 .

 注  TangSong  指出了我原来写的那个证明中的错误 ,  此处的证明是在他指出我的错误后 ,  我自己想的 .  现在来   看这道题不是太难 ,  然而考试的时候却做错了 .  数学归纳法是个好东西 ,  很多涉及矩阵的命题都可以用它来   证明 .

\begin{enumerate}
  \setcounter{enumi}{7}
  \item  设  $l_{i}$  的方向向量为  $\left(a_{i}, b_{i}, c_{i}\right)$,  过点  $\left(x_{i}, y_{i}, z_{i}\right), i=1,2,3$.  设  $(x, y, z)$  为与三条直线  $l_{1}, l_{2}, l_{3}$  共面的直线  $a, b, c$  的线性方程组 .  因为  $(a, b, c) \neq \mathbf{0}$,  于是系数矩阵的行列式为  0 ,  即 
\end{enumerate}
$$
\left|\begin{array}{lll}
b_{1}\left(z-z_{1}\right)-c_{1}\left(y-y_{1}\right) & c_{1}\left(x-x_{1}\right)-a_{1}\left(z-z_{1}\right) & a_{1}\left(y-y_{1}\right)-b_{1}\left(x-x_{1}\right) \\
b_{2}\left(z-z_{2}\right)-c_{2}\left(y-y_{2}\right) & c_{2}\left(x-x_{2}\right)-a_{2}\left(z-z_{2}\right) & a_{2}\left(y-y_{2}\right)-b_{2}\left(x-x_{2}\right) \\
b_{3}\left(z-z_{3}\right)-c_{3}\left(y-y_{3}\right) & c_{3}\left(x-x_{3}\right)-a_{3}\left(z-z_{3}\right) & a_{3}\left(y-y_{3}\right)-b_{3}\left(x-x_{3}\right)
\end{array}\right|=0,
$$
 化简得到一个关于  $x, y, z$  的二次方程 ,  从而与直线  $l_{1}, l_{2}, l_{3}$  都共面的直线的并集是个二次曲面 , 注意到  $l_{1}, l_{2}, l_{3}$  在上述曲面中 ,  从而是直纹面且不能是柱面和锥面 ,  又因为  $l_{1}, l_{2}, l_{3}$  不平行于同一个平面 ,  从而是一   个单叶双曲面 .

 注   只是把北京大学  2018  年解析几何的第二个计算题变成了证明题 ,  在丘维声老师的 《 解析几何 》 第三版第  106  页有类似题 .

\begin{enumerate}
  \setcounter{enumi}{8}
  \item  以给定平面为  $x 0 y$  平面建立空间直角坐标系 ,  设双曲抛物面的方程为 
\end{enumerate}
$$
\left(\begin{array}{lll}
x & y & z
\end{array}\right)\left(\begin{array}{lll}
a_{11} & a_{12} & a_{13} \\
a_{21} & a_{22} & a_{23} \\
a_{31} & a_{32} & a_{33}
\end{array}\right)\left(\begin{array}{l}
x \\
y \\
z
\end{array}\right)+2\left(\begin{array}{lll}
x & y & z
\end{array}\right)\left(\begin{array}{l}
a_{1} \\
a_{2} \\
a_{3}
\end{array}\right)+a_{0}=0, a_{i j}=a_{j i}, i, j=1,2,3 .
$$
 平面与曲面的交线的方程为  $\left\{\begin{array}{l}\left(\begin{array}{ll}x & y\end{array}\right)\left(\begin{array}{ll}a_{11} & a_{12} \\ a_{21} & a_{22}\end{array}\right)\left(\begin{array}{l}x \\ y\end{array}\right)+2\left(\begin{array}{ll}x & y\end{array}\right)\left(\begin{array}{l}a_{1} \\ a_{2}\end{array}\right)+a_{0}=0 \\ z=0\end{array}\right.$,

 若交线是椭圆 ,  则  $\left(\begin{array}{ll}a_{11} & a_{12} \\ a_{21} & a_{22}\end{array}\right)$  为正定矩阵 ,  又因为曲面是双曲抛物面 ,  因此  $\left(\begin{array}{lll}a_{11} & a_{12} & a_{13} \\ a_{21} & a_{22} & a_{23} \\ a_{31} & a_{32} & a_{33}\end{array}\right)$  的特征值为   一正一负一个为  0 ,  于是存在  $x, y \in \mathbb{R}$,  使得  $\left\{\begin{array}{l}x a_{11}+y a_{21}=a_{31} \\ x a_{12}+y a_{22}=a_{32} \text {, } \\ x a_{13}+y a_{23}=a_{33}\end{array}\right.$
$$
\begin{aligned}
\left|\begin{array}{ll}
a_{11} & a_{12} \\
a_{21} & a_{22}
\end{array}\right|+\left|\begin{array}{ll}
a_{11} & a_{13} \\
a_{31} & a_{33}
\end{array}\right|+\left|\begin{array}{ll}
a_{22} & a_{23} \\
a_{32} & a_{33}
\end{array}\right| &=\left|\begin{array}{ll}
a_{11} & a_{12} \\
a_{21} & a_{22}
\end{array}\right|+y\left|\begin{array}{ll}
a_{11} & a_{13} \\
a_{21} & a_{23}
\end{array}\right|+x\left|\begin{array}{ll}
a_{22} & a_{23} \\
a_{12} & a_{13}
\end{array}\right| \\
&=\left|\begin{array}{ll}
a_{11} & a_{12} \\
a_{21} & a_{22}
\end{array}\right|+y^{2}\left|\begin{array}{ll}
a_{11} & a_{21} \\
a_{21} & a_{22}
\end{array}\right|+x^{2}\left|\begin{array}{ll}
a_{22} & a_{12} \\
a_{12} & a_{11}
\end{array}\right| \\
&=\left(1+x^{2}+y^{2}\right)\left|\begin{array}{ll}
a_{11} & a_{12} \\
a_{21} & a_{22}
\end{array}\right|>0,
\end{aligned}
$$
 这与特征值为一正一负一个为  0  矛盾 ,  因此平面与双曲抛物面的交线不可能是一个椭圆 .  北京大学  2020  年全国硕士研究生招生考试高代解几试题及解答 

   

$2020.01 .30$

 记号 :  设  $\varphi: V \rightarrow W$  是线性映射 ,  则  $\operatorname{ker} \varphi=\{v \in V \mid \varphi(v)=0\}, \operatorname{im} \varphi=\{\varphi(v) \mid v \in V\}$.

\begin{enumerate}
  \item (10  分 )  给定数域  $\mathbb{F}$  上的有限维空间  $V_{0}=\{0\}, V_{1}, \ldots, V_{n+1}=\{0\}$.  设线性映射  $\varphi_{i}: V_{i} \rightarrow V_{i+1}, i=$ $0,1, \ldots, n$,  满足  $\operatorname{ker} \varphi_{i+1}=\operatorname{im} \varphi_{i}, i=0,1, \ldots, n-1$.  证明 : $\sum_{i=1}^{n}(-1)^{i} \operatorname{dim} V_{i}=0$.

  \item (15  分 )  设正整数  $k \geqslant 2$,  给定  $\mathbb{C}$  上的  $k+1$  个数  $c_{0}, \ldots, c_{k}$.  证明 :  存在唯一一个不超过  $k$  次的复系数多项   式  $p(x)$  满足 : $p(0)=c_{0}, \ldots, p(k)=c_{k}$.

  \item (20  分 )  设  $n \geqslant 2, A \in M_{n}(\mathbb{R}), A=A^{\mathrm{T}}, \operatorname{rank}(A)=r$.  证明 : $A$  存在  $r$  阶非零主子式且所有不为零的  $r$  阶   主子式符号相同 .

  \item (20  分 )  设  $\varphi$  是  $\mathbb{C}$  上  $n$  阶线性空间  $V$  上的线性变换 .  证明 : $\varphi$  可对角化  $\Longleftrightarrow$  对于  $\varphi$  的每一个特征值  $\lambda$, $\operatorname{dim}(\operatorname{im}(\lambda \operatorname{id}-\varphi))=\operatorname{dim}\left(\operatorname{im}(\lambda \mathrm{id}-\varphi)^{2}\right)$,  这里  id  表示恒等变换 .

  \item (20  分 )  设  $n \geqslant 2, A \in M_{n}(\mathbb{C}), A$  的所有特征值为  $\lambda_{1}, \ldots, \lambda_{k}$,  每个特征值  $\lambda_{i}$  的特征子空间有一组以特征向   量  $\alpha_{i 1}, \ldots, \alpha_{i j_{i}}$  构成的基底 , $A^{*}$  为  $A$  的伴随矩阵 .  求  $A^{*}$  的所有特征值和特征向量 .

  \item (15  分 )  设  $\eta$  是欧氏空间  $V($  内积为  $(\cdot, \cdot))$  上的一个单位向量 ,  定义映射  $\sigma: V \rightarrow V, \sigma(\alpha)=\alpha-$ $2(\eta, \alpha) \eta, \forall \alpha \in V$.

\end{enumerate}
(1)  证明 : $\sigma$  是正交变换  ( 这样的正交变换被称为镜面反射 ).

(2)  证明 :  欧氏空间  $V$  上的任一正交变换都可以表示为一系列镜面反射的乘积 .

\begin{enumerate}
  \setcounter{enumi}{7}
  \item (15  分 )  已知向量  $\vec{u}, \vec{v}, \vec{w}$  满足  $|\vec{u}|=|\vec{v}|=|\vec{w}|>0$,  并且  $\vec{u} \cdot \vec{v}=\vec{v} \cdot \vec{w}=\vec{w} \cdot \vec{u}$.  证明任取向量  $\vec{x}$,  如果存在系   数  $a, b, c$  使得  $\vec{x} \times \vec{u}=a \vec{u}+b \vec{v}+c \vec{w}, \vec{x} \times \vec{v}=a \vec{v}+b \vec{w}+c \vec{u}$,  则  $\vec{x} \times \vec{w}=a \vec{w}+b \vec{u}+c \vec{v}$.

  \item (20  分 )  在平面  $\pi$  上取定平面直角坐标系 ,  设该平面里的一条二次曲线  $\gamma$  的方程为  $x^{2}+2 y^{2}+6 x y+8 x+$ $10 y+6=0$.

\end{enumerate}
(1)  证明 : $\gamma$  是双曲线 .

(2)  写出  $\gamma$  的长短轴方程和长短轴长 ,  并指出长短轴中哪一个与  $\gamma$  有交点 .

\begin{enumerate}
  \setcounter{enumi}{9}
  \item (15  分 )  在平面  $\pi$  上取定平面直角坐标系 ,  已知该平面里的一个椭圆  $\gamma$  的方程为  $x^{2}+8 y^{2}+4 x y+6 x+20 y+4=$ 0 .  求  $\gamma$  的内接三角形  ( 即三个顶点都在  $\gamma$  上的三角形 )  的面积的最大值 . 1.  因为  $\operatorname{im} \varphi_{n}$  为  $V_{n+1}$  的线性子空间 ,  从而  $\operatorname{im} \varphi_{n}=\{0\}, \operatorname{ker} \varphi_{0}$  为  $V_{0}$  的线性子空间 ,  从而  $\operatorname{ker} \varphi_{0}=\{0\}$.  再注   意到 
\end{enumerate}
$\operatorname{dim} \operatorname{ker} \varphi_{i}+\operatorname{dim} \operatorname{im} \varphi_{i}=\operatorname{dim} V_{i}, i=0,1, \ldots, n, \quad \operatorname{dim} \operatorname{ker} \varphi_{i+1}=\operatorname{dim} \operatorname{im} \varphi_{i}, i=0,1, \ldots, n-1$

 故有 
$$
\begin{aligned}
\sum_{i=1}^{n}(-1)^{i} \operatorname{dim} V_{i} &=\sum_{i=0}^{n}(-1)^{i} \operatorname{dim} V_{i}=\sum_{i=0}^{n}(-1)^{i}\left(\operatorname{dim} \operatorname{ker} \varphi_{i}+\operatorname{dim} \operatorname{im} \varphi_{i}\right) \\
&=-\operatorname{dim} \operatorname{ker} \varphi_{0}+(-1)^{n} \operatorname{dim} \operatorname{im} \varphi_{n}=0
\end{aligned}
$$
2 .  设  $p(x)=a_{0}+a_{1} x+\cdots+a_{k} x^{k}$,  由  $p(0)=c_{0}, \ldots, p(k)=c_{k}$  得 

%\includegraphics[max width=\textwidth]{2022_04_18_33b622a7abd81c227674g-098}

 由线性方程组的解的存在唯一性知  $p(x)$  的存在唯一性 .

\begin{enumerate}
  \setcounter{enumi}{3}
  \item  设第  $i_{1}, i_{2}, \ldots, i_{r}$  行是行向量中的一个极大线性无关组 ,  先做行变换将其余行变为零 ,  再做对应的列变换 ,  由   于  $A=A^{\mathrm{T}}$,  于是矩阵中对应的列将变为零 ,  此时得到的矩阵中只有唯一一个非零  $r$  阶主子式 ,  而且这个主   子式也是  $A$  的一个主子式 ,  至此就说明了非零  $r$  阶主子式的存在性 .
\end{enumerate}
 下面说明非零  $r$  阶主子式均符号相同 .  设  $\alpha_{i_{1}}, \alpha_{i_{2}}, \ldots, \alpha_{i_{r}}$  与  $\beta_{j_{1}}, \beta_{j_{2}}, \ldots, \beta_{j_{r}}$  分别是两组不同的极大线性   无关行向量组 .  由前部分的论述知第  $i_{1}, i_{2}, \ldots, i_{r}$  行 ,  第  $i_{1}, i_{2}, \ldots, i_{r}$  列对应的主子式非零 ,  做行对换将第  $i_{1}$  行换到第  1  行 ,  第  $i_{2}$  行换到第  2  行 , ...  第  $i_{r}$  行换到第  $r$  行 ,  再做对应的列对换 ,  此时得到的矩阵  $A_{1}$  的第  $1,2, \ldots, r$  行 ,  第  $1,2, \ldots, r$  列对应的主子式非零且与原矩阵  $A$  的第  $i_{1}, i_{2}, \ldots, i_{r}$  行 ,  第  $i_{1}, i_{2}, \ldots, i_{r}$  列对应   的主子式的值相同 .  同样可以将第  $j_{1}, j_{2}, \ldots, j_{r}$  行 ,  第  $j_{1}, j_{2}, \ldots, j_{r}$  列换到第  $1,2, \ldots, r$  行 ,  第  $1,2, \ldots, r$  列   得到矩阵  $A_{2}$,  它的  $r$  阶顺序主子式非零 .

 设 
$$
\left(\beta_{j_{1}}, \beta_{j_{2}}, \ldots, \beta_{j_{r}}\right)^{\mathrm{T}}=P\left(\alpha_{i_{1}}, \alpha_{i_{2}}, \ldots, \alpha_{i_{r}}\right)^{\mathrm{T}},
$$
 由于矩阵  $P$  可以表示为有限个初等行变换对应的初等矩阵的乘积 ,  因此若只考虑  $A_{1}, A_{2}$  的前  $r$  行前  $r$  列   对应的矩阵  $\tilde{A}_{1}, \tilde{A}_{2}$,  则  $\tilde{A}_{2}=P \tilde{A}_{1} P^{\mathrm{T}}$,  于是  $\operatorname{det} \tilde{A}_{2}=\operatorname{det} \tilde{A}_{1}(\operatorname{det} P)^{2}$,  又由于  $\operatorname{det} P \neq 0$,  综上可得  $A$  的非   零主子式均同号 .

 注   在证明  $A$  的非零主子式均同号时 ,  我们做得更彻底一点 ,  在进行互换位置的操作后 ,  再做一些初等变换 ,
$$
\begin{aligned}
(\lambda \mathrm{id}-\varphi)\left(\alpha_{1}, \alpha_{2}, \ldots, \alpha_{n}\right) &=\left(\alpha_{1}, \alpha_{2}, \ldots, \alpha_{n}\right) \operatorname{diag}\left\{\lambda-\lambda_{1}, \lambda-\lambda_{2}, \ldots, \lambda-\lambda_{n}\right\} \\
(\lambda \mathrm{id}-\varphi)^{2}\left(\alpha_{1}, \alpha_{2}, \ldots, \alpha_{n}\right) &=\left(\alpha_{1}, \alpha_{2}, \ldots, \alpha_{n}\right) \operatorname{diag}\left\{\left(\lambda-\lambda_{1}\right)^{2},\left(\lambda-\lambda_{2}\right)^{2}, \ldots,\left(\lambda-\lambda_{n}\right)^{2}\right\}
\end{aligned}
$$
 因此 
$$
\begin{aligned}
\operatorname{dim}(\operatorname{im}(\lambda \operatorname{id}-\varphi)) &=\operatorname{rank}\left(\operatorname{diag}\left\{\lambda-\lambda_{1}, \lambda-\lambda_{2}, \ldots, \lambda-\lambda_{n}\right\}\right) \\
&=\operatorname{rank}\left(\operatorname{diag}\left\{\left(\lambda-\lambda_{1}\right)^{2},\left(\lambda-\lambda_{2}\right)^{2}, \ldots,\left(\lambda-\lambda_{n}\right)^{2}\right\}\right) \\
&=\operatorname{dim}\left(\operatorname{im}(\lambda \mathrm{id}-\varphi)^{2}\right)
\end{aligned}
$$
 充分性 : 存在一组基使  $\varphi$  在此组基下的矩阵为  Jordan  标准型 ,  可设  $\varphi\left(\beta_{1}, \beta_{2}, \ldots, \beta_{n}\right)=\left(\beta_{1}, \beta_{2}, \ldots, \beta_{n}\right) J$, $J$  中每一个  Jordan  块都只能是一阶的 ,  否则将有  $\operatorname{dim}(\operatorname{im}(\lambda \mathrm{id}-\varphi))>\operatorname{dim}\left(\operatorname{im}(\lambda \mathrm{id}-\varphi)^{2}\right)$,  矛盾 ,  于是  $J$  是对角矩阵 ,  故  $\varphi$  可对角化 .

\begin{enumerate}
  \setcounter{enumi}{5}
  \item  当  $\operatorname{rank}(A)<n-1$  时 , $A^{*}=0$,  于是  $A^{*}$  的特征值为  0 ,  特征向量为  $\mathbb{C}^{n}$  中任意非零向量 .
\end{enumerate}
 当  $\operatorname{rank}(A)=n-1$  时 , $\operatorname{rank}\left(A^{*}\right)=1$,  于是  $A^{*}$  的特征值为  $0(n-1$  重  $), \operatorname{tr}\left(A^{*}\right)(1$  重  $)$,  设  $A^{*}=\alpha \beta^{\mathrm{T}}$,  则  $\operatorname{tr}\left(A^{*}\right)$  对应的特征向量为  $k \alpha, k \neq 0 ; 0$  对应的特征向量为由  $A$  的列向量线性生成的非零向量 .

 当  $\operatorname{rank}(A)=n$  时 , $A^{*}=|A| A^{-1}$,  于是  $A^{*}$  的特征值为  $\frac{|A|}{\lambda_{1}}, \ldots, \frac{|A|}{\lambda_{k}}$,  并且  $\frac{|A|}{\lambda_{i}}$  的特征子空间有一组以特征   向量  $\alpha_{i 1}, \ldots, \alpha_{i j_{i}}$  构成的基底 .

\begin{enumerate}
  \setcounter{enumi}{6}
  \item (1)  因为  $(\sigma(\alpha), \sigma(\alpha))=(\alpha-2(\eta, \alpha) \eta, \alpha-2(\eta, \alpha) \eta)=(\alpha, \alpha), \forall \alpha \in V$,  于是  $\sigma$  是正交变换 .
\end{enumerate}
(2)  对于任一正交变换 ,  存在  $V$  中的标准正交基 ,  使得正交变换在该组基下的矩阵为如下形式 
$$
J=\left[\begin{array}{ccccccc}
\lambda_{1} & & & & & & \\
& \lambda_{2} & & & & & \\
& & \ddots & & & & \\
& & & \lambda_{k} & & & \\
& & & S_{1} & & \\
& & & & \ddots & \\
& & & & & S_{l}
\end{array}\right]
$$
 其中  $\lambda_{i}=\pm 1(i=1,2, \ldots, k)$,  而 
$$
S_{j}=\left[\begin{array}{cc}
\cos \varphi_{j} & -\sin \varphi_{j} \\
\sin \varphi_{j} & \cos \varphi_{j}
\end{array}\right]=\left[\begin{array}{cc}
-\sin \varphi_{j} & \cos \varphi_{j} \\
\cos \varphi_{j} & \sin \varphi_{j}
\end{array}\right]\left[\begin{array}{ll}
0 & 1 \\
1 & 0
\end{array}\right], \quad\left(\varphi_{j} \neq k \pi, j=1,2, \ldots, l\right) .
$$
 注意到若  $\sigma$  是正交变换 ,  则  $\sigma$  是镜面反射当且仅当  $\sigma$  在  $V$  中的标准正交基下的矩阵的特征值为  1 ( $n-1$  重 ), $-1$ (1  重 ),  而把  $J$  分解成有限个那样的正交矩阵的乘积的分解是存在的 ,  这里的有限个更   精确一点可改为不超过  $n$  个 ,  于是  $\sigma$  可以表示为一系列镜面反射的乘积 .

 注   利用了蓝以中的  《 高等代数简明教程 》 第二版下册第  24  页的定理  $2.1$.

\begin{enumerate}
  \setcounter{enumi}{7}
  \item ( 法一 )  由题设知  $\vec{u}, \vec{v}, \vec{w}$  长度相等 ,  并且两两之间夹角大小相同 .
\end{enumerate}
 若夹角为  0 ,  则  $\vec{u}=\vec{v}=\vec{w}$,  自然有  $\vec{x} \times \vec{w}=a \vec{w}+b \vec{u}+c \vec{v}$.

 若夹角大于  0 ,  则几何上可考虑正三棱雉 ,  可以建立适当的直角坐标系 ,  使得  $\vec{u}=(1,0, t), \vec{v}=\left(-\frac{1}{2}, \frac{\sqrt{3}}{2}, t\right)$, $\vec{w}=\left(-\frac{1}{2},-\frac{\sqrt{3}}{2}, t\right)$,  设  $\vec{x}=\left(x_{1}, x_{2}, x_{3}\right)$,  则由题设中的两个外积关系式得 
$$
\begin{aligned}
\left(t x_{2}, x_{3}-t x_{1},-x_{2}\right) &=\left(a-\frac{b}{2}-\frac{c}{2}, \frac{\sqrt{3}}{2} b-\frac{\sqrt{3}}{2} c,(a+b+c) t\right) \\
\left(t x_{2}-\frac{\sqrt{3}}{2} x_{3},-\frac{1}{2} x_{3}-t x_{1}, \frac{\sqrt{3}}{2} x_{1}+\frac{x_{2}}{2}\right) &=\left(-\frac{a}{2}-\frac{b}{2}+c, \frac{\sqrt{3}}{2} a-\frac{\sqrt{3}}{2} b,(a+b+c) t\right),
\end{aligned}
$$
 两式相减得 
$$
\left(-\frac{\sqrt{3}}{2} x_{3},-\frac{3}{2} x_{3}, \frac{\sqrt{3}}{2} x_{1}+\frac{3}{2} x_{2}\right)=\left(-\frac{3}{2} a+\frac{3}{2} c, \frac{\sqrt{3}}{2} a-\sqrt{3} b+\frac{\sqrt{3}}{2} c, 0\right),
$$
 由最后一个式子的前两个分量可得  $2 a-b-c=0$,  从第三个分量得  $x_{1}=-\sqrt{3} x_{2}$,  再从第一个式子的   前后两个分量得  $t x_{2}=0,0=(a+b+c) t^{2}$.  若  $t=0$,  则  $\vec{u}+\vec{v}+\vec{w}=0$,  于是  $\vec{x} \times(\vec{u}+\vec{v}+\vec{w})=$ $(a+b+c)(\vec{u}+\vec{v}+\vec{w})$,  从而  $\vec{x} \times \vec{w}=a \vec{w}+b \vec{u}+c \vec{v}$.  若  $t \neq 0$,  则  $x_{1}=x_{2}=0, a+b+c=0$,  直接计算得  $\vec{x} \times(\vec{u}+\vec{v}+\vec{w})=(a+b+c)(\vec{u}+\vec{v}+\vec{w})$,  从而  $\vec{x} \times \vec{w}=a \vec{w}+b \vec{u}+c \vec{v} .$

( 法二 )  不妨设  $|\vec{u}|=|\vec{v}|=|\vec{w}|=1, \vec{u} \cdot \vec{v}=\vec{v} \cdot \vec{w}=\vec{w} \cdot \vec{u}=\theta$.  题目条件为 
$$
\begin{aligned}
&\vec{x} \times \vec{u}=a \vec{u}+b \vec{v}+c \vec{w} \\
&\vec{x} \times \vec{v}=a \vec{v}+b \vec{w}+c \vec{u}
\end{aligned}
$$
 要证 
$$
\vec{x} \times \vec{w}=a \vec{w}+b \vec{u}+c \vec{v}
$$
 如果  $\vec{u}=\vec{v}=\vec{w}$,  显然结论成立 .

 如果  $\vec{u}, \vec{v}, \vec{w}$  共面 ,  容易看出它们的关系位置为正三角形中心指向各顶点 ,  即两两夹角为  $\frac{2}{3} \pi$.  此时  $\vec{u}+\vec{v}+\vec{w}=$ 0 ,  因此有  $\vec{x} \times(\vec{u}+\vec{v}+\vec{w})=(a+b+c)(\vec{u}+\vec{v}+\vec{w})$,  就可得  (3).

 如果  $\vec{u}, \vec{v}, \vec{w}$  不共面 ,  则它们构成空间一组基 .  为证  $(3)$,  只要验证两边分别与  $\vec{u}, \vec{v}, \vec{w}$  作内积相等 .  即只要证   明 
$$
\begin{aligned}
&(\vec{x} \times \vec{w}, \vec{u})=(a \vec{w}+b \vec{u}+c \vec{v}, \vec{u}) \\
&(\vec{x} \times \vec{w}, \vec{v})=(a \vec{w}+b \vec{u}+c \vec{v}, \vec{v}) \\
&(\vec{x} \times \vec{w}, \vec{w})=(a \vec{w}+b \vec{u}+c \vec{v}, \vec{w}) .
\end{aligned}
$$
(1)  两边与  $\vec{u}$  作内积得 
$$
(\vec{x} \times \vec{u}, \vec{u})=(a \vec{u}+b \vec{v}+c \vec{w}, \vec{u}),
$$
 注意上式左边为  0 ,  得 
$$
a+(b+c) \theta=0 .
$$
(1)  两边与  $\vec{v}$  作内积 , (2)  两边与  $\vec{u}$  作内积 ,  两式相加得 
$$
(\vec{x} \times \vec{u}, \vec{v})+(\vec{x} \times \vec{v}, \vec{u})=(a \vec{u}+b \vec{v}+c \vec{w}, \vec{v})+(a \vec{v}+b \vec{w}+c \vec{u}, \vec{u}),
$$
 由混合积的性质 ,  上式左边为  0 ,  得 
$$
(a+c) \theta+b+(a+b) \theta+c=0 .
$$
 现在  (4)  式左边 
$$
=(\vec{u} \times \vec{x}, \vec{w})=-(a \vec{u}+b \vec{v}+c \vec{w}, \vec{w})=-(a+b) \theta-c
$$
(4)  式右边 
$$
=(a+c) \theta+b,
$$
 由  (8)  得到  (4)  成立 .  类似地来验证  (5).  现在  (5)  式左边 
$$
=(\vec{v} \times \vec{x}, \vec{w})=-(a \vec{v}+b \vec{w}+c \vec{u}, \vec{w})=-(a+c) \theta-b
$$
(5)  式右边 
$$
=(a+b) \theta+c,
$$
 仍然由  (8)  得到  (5)  成立 . (6)  式左边  $=0$,  右边 
$$
=a+(b+c) \theta
$$
 由  (7)  得到上式为  0 .

 注   观察到式子的对称性 ,  我的想法是证明  $\vec{x} \times(\vec{u}+\vec{v}+\vec{w})=(a+b+c)(\vec{u}+\vec{v}+\vec{w})$,  而后的分类讨论的出发   点就是根据这个式子来的 ,  不过在想处理  $\vec{u}, \vec{v}, \vec{w}$  不共面这一情形时 ,  忘记利用它们现在构成空间的一组基 ,  最终的处理方式不太自然 .  法二中的想法就很好地处理了不共面的情形 ,  它由  TangSong  提供 ,  我对排版和   文字进行了微调 .

\begin{enumerate}
  \setcounter{enumi}{8}
  \item (1)  记 
\end{enumerate}
$$
A=\left[\begin{array}{ll}
1 & 3 \\
3 & 2
\end{array}\right]
$$
 由于  $\operatorname{det} A=-7<0$,  故  $\gamma$  为双曲线 .

(2) $A$  的特征多项式为  $\lambda^{2}-3 \lambda-7$,  特征值为  $\lambda_{1}=\frac{3+\sqrt{37}}{2}, \lambda_{2}=\frac{3-\sqrt{37}}{2}$,  由于 
$$
\operatorname{det}\left[\begin{array}{ccc}
1 & 3 & 4 \\
3 & 2 & 5 \\
4 & 5 & 6
\end{array}\right]=21
$$
 因此  $\gamma$  的标准方程为  $\lambda_{1} \tilde{x}^{2}+\lambda_{2} \tilde{y}^{2}+21 /\left(\lambda_{1} \lambda_{2}\right)=0$,  即  $\lambda_{1} \tilde{x}^{2}+\lambda_{2} \tilde{y}^{2}=3$.  于是短轴长为  $\sqrt{\frac{2 \times 3}{3+\sqrt{37}}}$,  长轴   长为  $\sqrt{\frac{2 \times 3}{\sqrt{37}-3}} \cdot \gamma$  的对称中心为  $2 x+6 y+8=0$  与  $4 y+6 x+10=0$  的交点 ,  计算得  $(-1,-1), \lambda_{1}$  对   应的  $A$  的一个特征向量为  $\left(3, \lambda_{1}-1\right)^{\mathrm{T}}$,  于是短轴方程为  $3(x+1)+\left(\lambda_{1}-1\right)(y+1)=0$,  同样可得长   轴方程为  $3(x+1)+\left(\lambda_{2}-1\right)(y+1)=0$,  短轴与  $\gamma$  相交 .

\begin{enumerate}
  \setcounter{enumi}{9}
  \item  考虑椭圆的方程为  $\frac{x^{2}}{a^{2}}+\frac{y^{2}}{b^{2}}=1$  的情形 ,  椭圆的内接三角形的三边中必有不与横轴垂直的边 ,  设三角形的两   个交点为直线  $y=k x+t$  与粗圆的交点 ,  则要使三角形的面积最大 ,  另一个顶点为与直线  $y=k x+t$  平行   的直线  $y=k x+t_{0}$  与椭圆的切点 .  直线与䢶圆联立方程化简得  $\left(b^{2}+a^{2} k^{2}\right) x^{2}+2 k t a^{2} x+a^{2}\left(t^{2}-b^{2}\right)=0$,  判别式  $\Delta=4 a^{2} b^{2}\left(a^{2} k^{2}+b^{2}-t^{2}\right)$,  记  $\alpha=\sqrt{a^{2} k^{2}+b^{2}}$,  则  $-\alpha<t<\alpha$  截线长为  $\sqrt{1+k^{2}} \frac{\sqrt{\Delta}}{b^{2}+a^{2} k^{2}}$,  切线对   应的  $t_{0}$  满足  $\Delta\left(t_{0}\right)=4 a^{2} b^{2}\left(a^{2} k^{2}+b^{2}-t_{0}^{2}\right)=0$,  解得  $t_{0}=\alpha$  或  $t_{0}=-\alpha$.  下面设  $t_{0}=\alpha$,  另一种情况的处   理方法类似并且结果一样 ,  这时切点到截线的距离为  $\frac{\alpha-t}{\sqrt{1+k^{2}}}$,  于是三角形面积为 
\end{enumerate}
$$
S=a b \frac{\sqrt{(\alpha-t)^{3}(\alpha+t)}}{\alpha^{2}} \leqslant \frac{a b}{\alpha^{2}} \sqrt{\frac{1}{3}\left(\frac{(\alpha-t)+(\alpha-t)+(\alpha-t)+3(\alpha+t)}{4}\right)^{4}}=\frac{3 \sqrt{3}}{4} a b,
$$
 当且仅当  $(\alpha-t)=3(\alpha+t)$  时取等号 ,  利用导数也可以得到  $S$  的最大值为  $\frac{3 \sqrt{3}}{4} a b$.  对于一般的椭圆方程只   需要先把方程化为标准方程就能套公式计算出结果 .

 注   利用仿射变换把椭圆变成圆可以做得更简单一些 .  北京大学  1987  年全国硕士研究生招生考试数学分析试题及解答 

   

2019.04.03

\begin{enumerate}
  \item (18  分 )  证明  $f(x)=\frac{\int_{x}^{+\infty} \mathrm{e}^{-\frac{t^{2}}{2}} \mathrm{~d} t}{\mathrm{e}^{-\frac{x^{2}}{2}}}$  在  $[0,+\infty)$  有界 ,  但在  $(-\infty,+\infty)$  无界 .

  \item (18  分 )  计算下列积分 :\\
(1) $\int_{0}^{\frac{\pi}{2}}(\sqrt{\tan x}+\sqrt{\cot x}) d x$.\\
(2) $\iiint_{V} y \sqrt{16-z^{2}} \mathrm{~d} x \mathrm{~d} y \mathrm{~d} z$,  其中  $V$  是由  $z=y^{2}, z=2 y^{2}(y>0), z=x, z=2 x$  和  $z=4$  围成的区域 .

  \item (18  分 )  设  $f(x)=\frac{1}{1-x-x^{2}}$,  求证 : $\sum_{n=0}^{\infty} \frac{n !}{f^{(n)}(0)}$  收玫 .

  \item (15  分 )  证明 :\\
(1) $\sum_{n=1}^{\infty} x^{n} \ln x$  在  $(0,1]$  上不一致收敛 .\\
(2) $\int_{0}^{1}\left(\sum_{n=1}^{\infty} x^{n} \ln x\right) \mathrm{d} x=1-\frac{\pi^{2}}{6}$.

  \item (15  分 )  设  $f(x)$  在  $(-\infty,+\infty)$  连续 ,

\end{enumerate}
$$
\int_{-\infty}^{+\infty}|f(x)| \mathrm{d} x<+\infty, \int_{-\infty}^{+\infty}|f(x)|^{2} \mathrm{~d} x<+\infty .
$$
 定义 :
$$
\psi(x)=\int_{-\infty}^{+\infty} \mathrm{e}^{-(|x-\xi|+|x-\eta|)}|f(\xi)||f(\eta)| \mathrm{d} \xi \mathrm{d} \eta .
$$
 证明 :
$$
\int_{-\infty}^{+\infty} \psi(x) \mathrm{d} x \leq 4 \int_{-\infty}^{+\infty}|f(x)|^{2} \mathrm{~d} x .
$$

\begin{enumerate}
  \setcounter{enumi}{6}
  \item (16  分 )  设  $f_{n}(x)=\cos x+\cos ^{2} x+\cdots+\cos ^{n} x$.  求证 :
\end{enumerate}
(1)  对任意自然数  $n$,  方程  $f_{n}(x)=1$  在  $\left[0, \frac{\pi}{3}\right)$  内有且仅有一个根 .

(2)  设  $x_{n} \in\left[0, \frac{\pi}{3}\right)$  是  $f_{n}(x)=1$  的根 ,  则  $\lim _{n \rightarrow \infty} x_{n}=\frac{\pi}{3}$. 1.  因为  $\lim _{x \rightarrow+\infty} f(x)=0, f(x)$  在  $[0,+\infty)$  连续 ,  故  $f(x)$  在  $[0,+\infty)$  有界 .  而  $\lim _{x \rightarrow-\infty} f(x)=+\infty$,  故  $f(x)$  在  $(-\infty,+\infty)$  无界 .

\begin{enumerate}
  \setcounter{enumi}{2}
  \item (1) $\sqrt{2} \pi$.  详细解答见林源渠与方企勤编的 《 数学分析解题指南 》 第  206  页例  6 .
\end{enumerate}
(2)
$$
\begin{aligned}
\text { 原式 } &=2 \int_{2}^{4} \mathrm{~d} z \int_{\frac{z}{2}}^{z} \mathrm{~d} x \int_{\sqrt{\frac{z}{2}}}^{\sqrt{z}} y \sqrt{16-z^{2}} \mathrm{~d} y \\
&=\frac{1}{4} \int_{0}^{4} z^{2} \sqrt{16-z^{2}} \mathrm{~d} z \\
&=\frac{1}{4} \int_{0}^{\frac{\pi}{2}} 16 \sin ^{2} \theta \times 4 \cos \theta \times 4 \cos \theta \mathrm{d} \theta \\
&=4 \pi .
\end{aligned}
$$

\begin{enumerate}
  \setcounter{enumi}{3}
  \item  见 《 数学分析解题指南 》 第  381  页例  6 .

  \item (1)

\end{enumerate}
$$
\sum_{k=1}^{n} x^{k} \ln x=\left\{\begin{array}{ll}
\frac{x\left(1-x^{n}\right)}{1-x} \ln x & 0<x<1 \\
0 & x=1
\end{array} \Longrightarrow \sum_{n=1}^{\infty} x^{n} \ln x= \begin{cases}\frac{x \ln x}{1-x} & 0<x<1 \\
0 & x=1\end{cases}\right.
$$
 又因为  $\lim _{x \rightarrow 1^{-}} \frac{x \ln x}{1-x}=-1$,  故极限函数在  $(0,1]$  不连续 ,  从而原函数项级数在  $(0,1]$  上不一致收敛 .

(2)  对于  $\forall N \in \mathbb{N}_{+}$
$$
\begin{aligned}
\int_{0}^{1} f(x) \mathrm{d} x &=\int_{0}^{1} \sum_{k=1}^{N} x^{k} \ln x \mathrm{~d} x+\int_{0}^{1} \sum_{k=N+1}^{\infty} x^{k} \ln x \mathrm{~d} x \\
&=\sum_{k=1}^{N} \int_{0}^{1} x^{k} \ln x \mathrm{~d} x+\int_{0}^{1} \frac{x^{N+1}}{1-x} \ln x \mathrm{~d} x \\
&=-\sum_{k=1}^{N} \frac{1}{(k+1)^{2}}+\int_{0}^{1} x^{N} \frac{x \ln x}{1-x} \mathrm{~d} x
\end{aligned}
$$
 又因为  $\frac{x \ln x}{1-x}$  在  $(0,1)$  上连续 , $\lim _{x \rightarrow 0^{+}} \frac{x \ln x}{1-x}=0, \lim _{x \rightarrow 1^{-}} \frac{x \ln x}{1-x}=-1$,  因此  $\left|\frac{x \ln x}{1-x}\right|$  在  $[0,1]$  上有上界  $M$,  故  $\left|\int_{0}^{1} x^{N} \frac{x \ln x}{1-x} \mathrm{~d} x\right| \leq \frac{M}{N+1}$,  令  $N \rightarrow \infty$  得  $\int_{0}^{1} f(x) \mathrm{d} x=-\sum_{k=1}^{\infty} \frac{1}{(k+1)^{2}}=-\left(\frac{\pi^{2}}{6}-1\right)=1-\frac{\pi^{2}}{6}$.

\begin{enumerate}
  \setcounter{enumi}{5}
  \item  见 《 数学分析解题指南 》 第  383  页例  7 .

  \item  见  《 数学分析解题指南 》 第  378  页例  5 .  北京大学  1996  年全国硕士研究生招生考试数学分析试题及解答 

\end{enumerate}
   

2019.05.21

 一 .  判断下列命题的真  $(\checkmark)$  伪  $(X)$,  不必说明理由 . ( 共  25  分 )

\begin{enumerate}
  \item  对数列  $\left\{a_{n}\right\}$  作和  $S_{n}=\sum_{k=1}^{n} a_{k}$,  若  $\left\{S_{n}\right\}$  是有界数列 ,  则  $\left\{a_{n}\right\}$  是有界数列 .

  \item  数列  $\left\{a_{n}\right\}$  存在极限  $\lim _{n \rightarrow \infty} a_{n}$  的充分必要条件是 :  对任一自然数  $p$,  都有  $\lim _{n \rightarrow \infty}\left|a_{n+p}-a_{n}\right|=0 . \quad(\quad)$

  \item  设  $f(x)$  是  $[a,+\infty)$  上的递增连续函数 ,  若  $f(x)$  在  $[a,+\infty)$  上有界 ,  则  $f(x)$  在  $[a,+\infty)$  上一致连   续 .

  \item  设  $f(x)$  在  $[a, b]$  上连续 ,  且在  $(a, b)$  上可微 ,  若存在极限  $\lim _{x \rightarrow a+0} f^{\prime}(x)=l$,  则右导数  $f_{+}^{\prime}(a)$  存在且等于  $l$.

  \item  若  $f(x)$  是  $[a,+\infty)$  上的非负连续函数 ,  且积分  $\int_{a}^{+\infty} f(x) \mathrm{d} x$  收玫 ,  则  $\lim _{x \rightarrow+\infty} f(x)=0$.

\end{enumerate}
 二 . (13  分 )  设  $f(x)$  在  $x=a$  处可微 , $f(a) \neq 0$,  求极限  $\lim _{n \rightarrow \infty}\left(\frac{f\left(a+\frac{1}{n}\right)}{f(a)}\right)^{n}$.

 三 . $(20$  分  $)$

\begin{enumerate}
  \item  求幂级数  $\sum_{n=1}^{\infty} n x^{n-1},(|x|<1)$  的和 .

  \item  求级数  $\sum_{n=1}^{\infty} \frac{2 n}{3^{n}}$  的和 .

\end{enumerate}
 四 . (12  分 )  求积分  $I=\iiint_{D}(x+y+z) \mathrm{d} x \mathrm{~d} y \mathrm{~d} z$  的值 .  其中  $D$  是由平面  $x+y+z=1$  以及三个坐标平面所   围成的区域 .

 五 . (20  分 )  设  $a_{n} \neq 0,(n=1,2, \ldots)$,  且  $\lim _{n \rightarrow \infty} a_{n}=0$,  若存在极限  $\lim _{n \rightarrow \infty} \frac{a_{n+1}}{a_{n}}=l$,  证明 : $|l| \leqslant 1$.

 六 . (10  分 )  设在  $[a, b]$  上 , $f_{n}(x)$  一致收玫于  $f(x), g_{n}(x)$  一致收玫于  $g(x)$,  若存在正数列  $\left\{M_{n}\right\}$  使得  $\left|f_{n}(x)\right| \leqslant$ $M_{n},\left|g_{n}(x)\right| \leqslant M_{n},(x \in[a, b], n=1,2, \ldots)$,  证明 : $f_{n}(x) g_{n}(x)$  在  $[a, b]$  上一致收玫于  $f(x) g(x)$.  一 . 1. $\sqrt{.}$  只需注意到  $a_{n}=S_{n}-S_{n-1},\left|a_{n}\right| \leqslant\left|S_{n}\right|+\left|S_{n-1}\right|$.

\begin{enumerate}
  \setcounter{enumi}{2}
  \item X.  取  $a_{n}=\sum_{k=1}^{n} \frac{1}{k}$.
\end{enumerate}
%\includegraphics[max width=\textwidth]{2022_04_18_33b622a7abd81c227674g-105}

%\includegraphics[max width=\textwidth]{2022_04_18_33b622a7abd81c227674g-105(1)}

\begin{enumerate}
  \setcounter{enumi}{5}
  \item X.  反例可以参考谢惠民等人的 《 数学分析习题课讲义 》 上册第  386  页例  $12.2 .5$.  也可参考裴礼文的  《 数学分析中的典型问题与方法 》 第二版第  413  页的总结 .
\end{enumerate}
$$
\begin{aligned}
\text { 原式 } &=\mathrm{e}^{\lim _{n \rightarrow \infty} n \ln \frac{f\left(a+\frac{1}{n}\right)}{f(a)}} \\
&=\mathrm{e}^{\lim _{n \rightarrow \infty} n \frac{f\left(a+\frac{1}{n}\right)-f(a)}{f(a)}} \\
&=\mathrm{e}^{\frac{f^{\prime}(a)}{f(a)}} .
\end{aligned}
$$
 注   也可参考裴礼文的 《 数学分析中的典型问题与方法 》 第二版第  46  页例  $1.1 .13$.

 三 . 1 .  因为 
$$
\sum_{n=1}^{\infty} x^{n}=\frac{x}{1-x}, \quad|x|<1
$$
 逐项求导得 
$$
\sum_{n=1}^{\infty} n x^{n-1}=\frac{1}{(1-x)^{2}}, \quad|x|<1
$$
$2 .$
$$
\sum_{n=1}^{\infty} \frac{2 n}{3^{n}}=\frac{2}{3} \sum_{n=1}^{\infty} n\left(\frac{1}{3}\right)^{n-1}=\frac{3}{2}
$$
 四 .  转化为累次积分可以算出结果为  $\frac{1}{8}$.  参考裴礼文的 《 数学分析中的典型问题与方法 》 第二版第  867  页练习   题  3 .

 五 .  由题设可知  $\lim _{n \rightarrow \infty}\left|\frac{a_{n+1}}{a_{n}}\right|=|l|$.  若  $|l|>1$,  则  $\exists q>1, N>0$,  当  $n \geqslant N$  时 ,
$$
\left|\frac{a_{n+1}}{a_{n}}\right|>q \Longrightarrow\left|\frac{a_{n+1}}{a_{N}}\right| \geqslant q^{n+1-N} \Longrightarrow\left|a_{n+1}\right| \geqslant q^{n+1-N} a_{N} \Longrightarrow \lim _{n \rightarrow \infty} a_{n}=+\infty,
$$
 矛盾 .

 六 .  详细过程请参考裴礼文的 《 数学分析中的典型问题与方法 》 第二版第  512  页例  $5.2 .37$.  北京大学  1997  年全国硕士研究生招生考试数学分析试题及解答 

   

2019.05.21

 一 、(10  分 )  将函数  $f(x)=\arctan \frac{2 x}{1-x^{2}}$  在  $x=0$  点展开为幂级数 ,  并指出收玫区间 .

 二 、(10  分 )  判断广义积分的收玫性 : $\int_{0}^{+\infty} \frac{\ln (1+x)}{x^{p}} \mathrm{~d} x$.

 三 、(15  分 )  设  $f(x)$  在  $(-\infty,+\infty)$  上有任意阶导数  $f^{(n)}(x)$,  且对任意有限闭区间  $[a, b], f^{(n)}(x)$  在  $[a, b]$  上一致   收玫于  $\phi(x)(n \rightarrow+\infty)$,  求证 : $\phi(x)=c \mathrm{e}^{x}, c$  为常数 .

 四 、(15  分 )  设  $x_{n}>0(n=1,2, \ldots)$  及  $\lim _{n \rightarrow+\infty} x_{n}=a$,  用  $\varepsilon-N$  语言证明 : $\lim _{n \rightarrow+\infty} \sqrt{x_{n}}=\sqrt{a}$.

 五 、 (15  分 )  求第二型曲面积分  $\oiint_{S}(x \mathrm{~d} y \mathrm{~d} z+\cos y \mathrm{~d} z \mathrm{~d} x+\mathrm{d} x \mathrm{~d} y)$,  其中  $S$  为  $x^{2}+y^{2}+z^{2}=1$  的外侧 .

 六 、(20  分 )  设  $x=f(u, v), y=g(u, v), w=w(x, y)$  有二阶连续偏导数 ,  满足  $\frac{\partial f}{\partial u}=\frac{\partial g}{\partial v}, \frac{\partial f}{\partial v}=-\frac{\partial g}{\partial u}, \frac{\partial^{2} w}{\partial x^{2}}+$ $\frac{\partial^{2} w}{\partial y^{2}}=0$,  证明 :

(1) $\frac{\partial^{2}(f g)}{\partial u^{2}}+\frac{\partial^{2}(f g)}{\partial v^{2}}=0$,

(2) $w(u, v)=w(f(u, v), g(u, v))$  满足  $\frac{\partial^{2} w}{\partial u^{2}}+\frac{\partial^{2} w}{\partial v^{2}}=0$.

 七 、 (15  分 )  计算三重积分  $\iiint_{\Omega: x^{2}+y^{2}+z^{2} \leq 2 z}\left(x^{2}+y^{2}+z^{2}\right)^{5 / 2} \mathrm{~d} x \mathrm{~d} y \mathrm{~d} z$.  一 、 $f(x)=2 \sum_{n=0}^{\infty} \frac{(-1)^{n}}{2 n+1} x^{2 n+1} \quad(|x|<1)$.  详细过程见林源渠 、 方企勤编的 《 数学分析解题指南 》 第  241  页例  5 .

 二 、 当  $1<p<2$  时 ,  原广义积分收敛 .  详细过程见林源渠 、 方企勤编的 《 数学分析解题指南 》 第  203  页例  3  的  (1).

 三 、 此题为林源渠 、 方企勤编的 《 数学分析解题指南 》 第  235  页练习题  4.2.16.  证明过程可参考裴礼文的 《 数学   分析中的典型问题与方法 》 第二版第  538  页练习题  $5.2 .28$.

 四 、 因为  $\lim _{n \rightarrow+\infty} x_{n}=a$,  故有  $\forall \varepsilon>0, \exists N>0$,  当  $n>N$  时 , $\left|x_{n}-a\right|<\sqrt{a} \varepsilon$.  于是 
$$
\left|\sqrt{x_{n}}-\sqrt{a}\right|=\frac{\left|x_{n}-a\right|}{\sqrt{x_{n}}+\sqrt{a}}<\frac{\left|x_{n}-a\right|}{\sqrt{a}}<\varepsilon .
$$
 五 、 先由对称性知 :  所求的积分  $I=\oiint_{S} x \mathrm{~d} y \mathrm{~d} z$,  再用  Gauss  公式得  $I=\iiint_{V} \mathrm{~d} x \mathrm{~d} y \mathrm{~d} z=\frac{4 \pi}{3}$.

 六 、 此题为林源渠 、 方企勤编的 《 数学分析解题指南 》 第  283  页练习题  $5.2 .23$.  证明过程可参考裴礼文的 《 数学   分析中的典型问题与方法 》 第二版第  670  页练习题  $6.2 .12$.

 七 、 通过做极坐标变换可以算出结果为  $\frac{64 \pi}{9}$.  此题为林源渠 、 方企勤编的 《 数学分析解题指南 》 第  336  页练习   题  $5.2 .9$  的  (2).

\section{北京大学 1998 年全国硕士研究生招生考试数学分析试题解答}
   

2019.05.25

 一 . (26  分 )  选一个最确切的答案 ,  填入括号中 :

\begin{enumerate}
  \item  设  $f(x)$  定义在区间  $[\mathrm{a}, \mathrm{b}]$  上 .  若对任意的  $g \in \mathcal{R}([a, b])$,  有  $f \cdot g \in \mathcal{R}([a, b])$,  则 \\
(A) $f \in \mathcal{R}([a, b])$\\
(B) $g \in \mathcal{C}([a, b])$\\
(C) $f$  可微 \\
(D) $f$  可导 

  \item $f \in \mathcal{C}((a, b))$.  若存在  $\lim _{x \rightarrow a^{+}} f(x)=1, \lim _{x \rightarrow b^{-}} f(x)=2$,  则 \\
(A) $f(x)$  在  $[a, b]$  一致连续 \\
(B) $f(x)$  在  $[a, b]$  连续 \\
(C) $f(x)$  在  $(a, b)$  一致连续 \\
(D) $f(x)$  在  $(a, b)$  可微 

  \item  若反常  ( 广义 )  积分  $\int_{0}^{1} f(x) \mathrm{d} x$  和  $\int_{0}^{1} g(x) \mathrm{d} x$  都存在 ,  则反常积分  $\int_{0}^{1} f(x) g(x) \mathrm{d} x$\\
(A)  收敛 \\
(B)  发散 \\
(C)  不一定收敛 \\
(D)  一定不收敛 

  \item  若  $\lim _{n \rightarrow \infty} n a_{n}=1$,  则  $\sum_{n=1}^{\infty} a_{n}$\\
(A)  发散 \\
(B)  收敛 \\
(C)  不一定收敛 \\
(D)  绝对收敛 

  \item  设  $f(x, y)$  在区域  $\left\{(x, y) \mid x^{2}+y^{2}<1\right\}$  上有定义 .  若存在偏导数  $f_{x}^{\prime}(0,0)=0=f_{y}^{\prime}(0,0)$,  则  $f(x, y)$\\
(A)  在点  $(0,0)$  处连续 \\
(B)  在点  $(0,0)$  处可微 \\
(C)  在点  $(0,0)$  处不一定连续 \\
(D)  在点  $(0,0)$  处不可微 

\end{enumerate}
 二 . (24  分 )  计算下列极限  ( 写出演算过程  $)$ :

\begin{enumerate}
  \item $\lim _{n \rightarrow \infty} \sqrt[n]{1+a^{n}}(a>0)$

  \item $\lim _{x \rightarrow 0}\left(\frac{1}{x^{2}}-\frac{\cot x}{x}\right)$

  \item $\lim _{x \rightarrow 0^{+}} \sum_{n=1}^{\infty} \frac{1}{2^{n} n^{x}}$

\end{enumerate}
 三 . (10  分 )  求下列积分值 :

\begin{enumerate}
  \item $\iint_{S} x^{3} \mathrm{~d} y \mathrm{~d} z+x^{2} y \mathrm{~d} z \mathrm{~d} x+x^{2} z \mathrm{~d} x \mathrm{~d} y, \quad S: z=0, z=b, x^{2}+y^{2}=a^{2} .$

  \item $\int_{C} \frac{1}{y} \mathrm{~d} x+\frac{1}{x} \mathrm{~d} y, \quad C: y=1, x=4, y=\sqrt{x} \quad$  逆时针一周 .  四 . (16  分 )  解答下列问题 :

  \item  求幂级数  $\sum_{n=1}^{\infty} \frac{(-1)^{n}}{n !}\left(\frac{n}{\mathrm{e}}\right)^{n} x^{n}$  的收敛半径 .

\end{enumerate}
2 .  求级数  $\sum_{n=0}^{\infty} \frac{2^{n}(n+1)}{n !}$  的和 .

 五 . (24  分 )  试证明下列命题 :

\begin{enumerate}
  \item  广义积分  $\int_{0}^{+\infty} \frac{\sin x^{2}}{1+x^{p}} \mathrm{~d} x(p \geqslant 0)$  是收敛的 ;

  \item  设  $f(x, y)$  在  $G=\left\{(x, y) \mid x^{2}+y^{2}<1\right\}$  上有定义 ,  若  $f(x, 0)$  在  $x=0$  处连续 ,  且  $f_{y}^{\prime}$  在  $G$  上有界 ,  则  $f(x, y)$  在  $(0,0)$  处连续 .  一 . 1. A.  考虑  $g(x)=1, x \in[a, b]$  即可 .

  \item C.  相关的题目见林源渠 、 方企勤编的 《 数学分析解题指南 》 第  44  页例  11 ,  裴礼文的 《 数学分析中的   典型问题与方法 》 第二版第  151  页例  $2.2 .6$,  谢惠民等人的 《 数学分析习题课讲义 》 上册第  140  页例题  5.4.5.  考虑到  $f(x)$  的定义域 ,  不选  A.

  \item C.  考虑  $f(x)=g(x)=1 / \sqrt{x}$.

  \item A. $\exists N>0$,  当  $n>N$  时 , $a_{n}>\frac{1}{2 n}$.

  \item C.  考虑 

\end{enumerate}
$$
f(x, y)= \begin{cases}\frac{x y}{x^{2}+y^{2}}, & x^{2}+y^{2} \neq 0 \\ 0, & x^{2}+y^{2}=0\end{cases}
$$
 二 . 1.  记极限为  $L$.  当  $a=1$  时 , $L=1$.  当  $0<a<1$  时 ,
$$
L=\lim _{n \rightarrow \infty} \exp \left(\frac{\ln \left(1+a^{n}\right)}{n}\right)=\exp \left(\lim _{n \rightarrow \infty} \frac{a^{n}}{n}\right)=1
$$
 当  $a>1$  时 ,
$$
L=\lim _{n \rightarrow \infty} \exp \left(\frac{\ln \left(1+a^{n}\right)}{n}\right)=\exp \left(\lim _{n \rightarrow \infty} \frac{n \ln a+\ln \left(\left(\frac{1}{a}\right)^{2}+1\right)}{n}\right)=a
$$
$2 .$
$$
\text { 原式 }=\lim _{x \rightarrow 0}\left(\frac{\sin x-x \cos x}{x^{2} \sin x}\right)=\lim _{x \rightarrow 0} \frac{x-\frac{x^{3}}{3 !}+o\left(x^{4}\right)-x\left(1-\frac{x^{2}}{2 !}+o\left(x^{3}\right)\right)}{x^{3}}=\frac{1}{3} \text {. }
$$

\begin{enumerate}
  \setcounter{enumi}{3}
  \item  当  $x>0$  时 ,
\end{enumerate}
$$
\sum_{n=1}^{\infty} \frac{1}{2^{n} n^{x}} \leqslant \sum_{n=1}^{\infty} \frac{1}{2^{n}}
$$
 从而关于  $x$  的级数 
$$
\sum_{n=1}^{\infty} \frac{1}{2^{n} n^{x}}
$$
 在  $(0,+\infty)$  上一致收敛 ,  于是 
$$
\lim _{x \rightarrow 0^{+}} \sum_{n=1}^{\infty} \frac{1}{2^{n} n^{x}}=\sum_{n=1}^{\infty} \lim _{x \rightarrow 0^{+}} \frac{1}{2^{n} n^{x}}=\sum_{n=1}^{\infty} \frac{1}{2^{n}}=1
$$
 三 . 1.  只考虑  $b>0$  的情况 ,  设  $V$  表示  $S$  的内部 .  利用  Gauss  公式 ,  再做极坐标变换得 :
$$
\begin{aligned}
\text { 原式 } &=\iiint_{V}\left(3 x^{2}+x^{2}+x^{2}\right) \mathrm{d} x \mathrm{~d} y \mathrm{~d} z \\
&=5 \int_{0}^{b} \mathrm{~d} z \int_{0}^{2 \pi} \mathrm{d} \theta \int_{0}^{a} r(r \cos \theta)^{2} \mathrm{~d} r \\
&=\frac{5 \pi a^{4} b}{4} .
\end{aligned}
$$

\begin{enumerate}
  \setcounter{enumi}{2}
  \item  直接用第二型曲线积分的定义 ,  把曲线参数化来算 ,
\end{enumerate}
$$
\begin{aligned}
\text { 原式 } &=\int_{1}^{2} \frac{1}{4} \mathrm{~d} y+\int_{4}^{1}\left(\frac{\mathrm{d} x}{\sqrt{x}}+\frac{1}{x} \mathrm{~d} \sqrt{x}\right)+\int_{1}^{4} \mathrm{~d} x \\
&=\frac{3}{4} .
\end{aligned}
$$
 四 . 1.  令 
$$
a_{n}=\frac{(-1)^{n}}{n !}\left(\frac{n}{\mathrm{e}}\right)^{n}
$$
 因为 
$$
\lim _{n \rightarrow \infty}\left|\frac{a_{n}}{a_{n+1}}\right|=1
$$
 故原幂级数的收敛半径为  1 .

$2 .$
$$
\sum_{n=0}^{\infty} \frac{2^{n}(n+1)}{n !}=\sum_{n=1}^{\infty} \frac{2^{n}}{(n-1) !}+\sum_{n=0}^{\infty} \frac{2^{n}}{n !}=3 e^{2}
$$
 五 . 1.  只需证明当  $p \geqslant 0$  时 , $\int_{1}^{+\infty} \frac{\sin x^{2}}{1+x^{p}} \mathrm{~d} x$  是收敛的 .  因为 
$$
\left|\int_{1}^{A} x \sin x^{2} \mathrm{~d} x\right| \leqslant 1
$$
$\frac{1}{x\left(1+x^{p}\right)}(p \geqslant 0)$  在  $[1,+\infty)$  上单调递减趋于  0 ,  故由  Dirichlet  判别法知广义积分  $\int_{1}^{+\infty} \frac{\sin x^{2}}{1+x^{p}} \mathrm{~d} x$  收   敛 

2 .
$$
\begin{aligned}
|f(x, y)-f(0,0)| & \leqslant|f(x, y)-f(x, 0)|+|f(x, 0)-f(0,0)| \\
&=\left|f_{y}^{\prime}(x, \theta y)\right||y|+|f(x, 0)-f(0,0)|, \quad(\theta \in(0,1))
\end{aligned}
$$
 当  $\sqrt{x^{2}+y^{2}} \rightarrow 0$  时 , $x \rightarrow 0$  且  $y \rightarrow 0$.  利用  $f(x, 0)$  在  $x=0$  处连续 ,  且  $f_{y}^{\prime}$  在  $G$  上有界知 
$$
\lim _{\sqrt{x^{2}+y^{2}} \rightarrow 0}|f(x, y)-f(0,0)|=0
$$
 这就说明  $f(x, y)$  在  $(0,0)$  处连续 .  北京大学  1999  年全国硕士研究生招生考试数学分析试题及解答 

   

2019.05.26

 一 . (15  分 )  判断下列命题的真  $(\sqrt{ })$  伪  $(X)$ :

\begin{enumerate}
  \item  设  $\left\{a_{n}\right\}$  是一个数列 ,  若在任意子列  $\left\{a_{n_{k}}\right\}$  中均存在收玫子列  $\left\{a_{n_{k_{i}}}\right\}$,  则  $\left\{a_{n}\right\}$  必为收玫列 .

  \item  设  $f \in \mathcal{C}((a, b))$,  若存在 

\end{enumerate}
$$
\lim _{x \rightarrow a^{+}} f(x)=A<0, \lim _{x \rightarrow b^{-}} f(x)=B>0,
$$
 则必存在  $\xi \in(a, b)$,  使得  $f(\xi)=0$.

\begin{enumerate}
  \setcounter{enumi}{3}
  \item  设  $f(x)$  在  $[a, b]$  上有界 .  若对任意  $\delta>0, f(x)$  在  $[a+\delta, b]$  上可积 ,  则  $f(x)$  在  $[a, b]$  上可积 .

  \item  设  $f(x), g(x)$  在  $[0,1]$  上的瑕积分均存在 ,  则乘积  $f(x) \cdot g(x)$  在  $[0,1]$  上的瑕积分必存在 .

  \item  设级数  $\sum_{n=1}^{\infty} b_{n}$  收玫 ,  若有  $a_{n} \leqslant b_{n},(n=1,2, \cdots)$,  则级数  $\sum_{n=1}^{\infty} a_{n}$  必收玫 .

\end{enumerate}
 二 . (40  分 )  求下列极限值  ( 写出计算过程 ):

\begin{enumerate}
  \item $\lim _{x \rightarrow 0} \frac{a \tan x+b(1-\cos x)}{\alpha \log (1-x)+\beta\left(1-\mathrm{e}^{-x^{2}}\right)}, \quad\left(a^{2}+\alpha^{2} \neq 0\right)$

  \item $\lim _{n \rightarrow \infty}\left(\frac{\sin \frac{\pi}{n}}{n+1}+\frac{\sin \frac{2 \pi}{n}}{n+\frac{1}{2}}+\cdots+\frac{\sin \pi}{n+\frac{1}{n}}\right)$

  \item $\lim _{n \rightarrow \infty} \int_{0}^{1}\left(1-x^{2}\right)^{n} \mathrm{~d} x$

  \item $\lim _{n \rightarrow \infty} \sqrt[n]{1+a^{n}}, \quad(a>0)$

\end{enumerate}
 三 . (45  分 )  求解下列命题 :

\begin{enumerate}
  \item  求级数  $\sum_{n=0}^{\infty} \frac{n}{3^{n}} 2^{n}$  之和 .

  \item  设  $f \in \mathcal{C}([0,1])$,  且在  $(0,1)$  上可微 .  若有  $8 \int_{7 / 8}^{1} f(x) \mathrm{d} x=f(0)$,  证明 :  存在  $\xi \in(0,1)$,  使得  $f^{\prime}(\xi)=0$.

  \item  证明级数  $\sum_{n=1}^{\infty}(-1)^{n} \frac{\arctan n}{\sqrt{n}}$  收敛 .

  \item  证明 :  积分  $\int_{0}^{+\infty} x \mathrm{e}^{-x y} \mathrm{~d} y$  在  $(0,+\infty)$  不一致收玫 .

  \item  设  $u=f(x, y, z), g\left(x^{2}, \mathrm{e}^{y}, z\right)=0, y=\sin x$,  且已知  $f$  与  $g$  都有一阶连续偏导数 , $\frac{\partial g}{\partial z} \neq 0$.  求  $\frac{\mathrm{d} u}{\mathrm{~d} x}$.

  \item  设  $f(x)$  在  $[-1,1]$  上二次连续可微 ,  且有  $\lim _{x \rightarrow 0} \frac{f(x)}{x}=0$.  证明 :  级数  $\sum_{n=1}^{\infty} f\left(\frac{1}{n}\right)$  绝对收玫 .  一 . 1. X.  考虑  $a_{n}=(-1)^{n}$.

  \item $\sqrt{.}$  用极限的局部保号性以及连续函数的介值定理 .

  \item $\sqrt{\text{. 设 }|f(x)|<M, \forall \varepsilon>0 \text{, 由于 } f(x) \text{ 在 }\left[a+\frac{\varepsilon}{4 M}, b\right] \text { 上可积, 故存在 }\left[a+\frac{\varepsilon}{4 M}, b\right] \text{ 的分割 } P_{1} \text{, 使得 } \sigma\left(P_{1}, f\right)<$ $\frac{\varepsilon}{2}$,  在分割  $P_{1}$  的基础上添加一个点  $a$,  得到  $[a, b]$  的一个分割  $P$,  此时  $\sigma(P, f)<2 M \frac{\varepsilon}{4 M}+\sigma\left(P_{1}, f\right)<\varepsilon$,  从而  $f(x)$  在  $[a, b]$  上  Riemann  可积 .

  \item X.  令  $f(x)=g(x)=\frac{1}{\sqrt{x}}$.

  \item X.  考虑  $b_{n}=\frac{1}{n^{2}} a_{n}=-\frac{1}{n}$.

\end{enumerate}
 二 . $1 .$
$$
\begin{aligned}
\text { 原式 } &=\lim _{x \rightarrow 0} \frac{a\left(x+\frac{x^{3}}{3}+o\left(x^{4}\right)\right)+b\left(\frac{x^{2}}{2}+o\left(x^{3}\right)\right)}{\alpha\left(-x+\frac{x^{2}}{2}-\frac{x^{3}}{3}+o\left(x^{3}\right)\right)+\beta\left(-x^{2}+o\left(x^{3}\right)\right)} \\
&=\lim _{x \rightarrow 0} \frac{a x+\frac{b}{2} x^{2}+\frac{a}{3} x^{3}+o\left(x^{3}\right)}{-\alpha x+\left(\frac{\alpha}{2}-\beta\right) x^{2}-\frac{\alpha}{3} x^{3}+o\left(x^{3}\right)} .
\end{aligned}
$$
 当  $\alpha \neq 0$  时 ,  原极限为  $-a / \alpha$.  当  $\alpha=0$  时 ,  必有  $\beta \neq 0$,  此时极限不存在 .

\begin{enumerate}
  \setcounter{enumi}{2}
  \item  用夹逼定理及  Riemann  积分 ,  结果为  $2 / \pi$.  详细过程参考裴礼文的 《 数学分析中的典型问题与方法 》 第   二版第  47  页例  1.3.14.

  \item  可以做三角变换令  $x=\cos t$,  然后用  Wallis  公式得极限为  0 .  也可如下分段估计 :

\end{enumerate}
$\forall \varepsilon>0$,  不妨设  $\varepsilon<1$,  因为 
$$
\left|\int_{1-\frac{\varepsilon}{2}}^{1}\left(1-x^{2}\right)^{n} \mathrm{~d} x\right|<\frac{\varepsilon}{2}
$$
 又因为 
$$
\left|\int_{0}^{1-\frac{\varepsilon}{2}}\left(1-x^{2}\right)^{n} \mathrm{~d} x\right|<\left(1-\frac{\varepsilon}{2}\right)\left(1-\left(1-\frac{\varepsilon}{2}\right)^{2}\right)^{n} \rightarrow 0, \quad(n \rightarrow \infty),
$$
 故存在  $N>0$,  当  $n>N$  时 ,
$$
\left|\int_{0}^{1-\frac{\varepsilon}{2}}\left(1-x^{2}\right)^{n} \mathrm{~d} x\right|<\frac{\varepsilon}{2}
$$
 从而 
$$
\left|\int_{0}^{1}\left(1-x^{2}\right)^{n} \mathrm{~d} x\right| \leq\left|\int_{0}^{1-\frac{\varepsilon}{2}}\left(1-x^{2}\right)^{n} \mathrm{~d} x\right|+\left|\int_{1-\frac{\varepsilon}{2}}^{1}\left(1-x^{2}\right)^{n} \mathrm{~d} x\right|<\varepsilon
$$

\begin{enumerate}
  \setcounter{enumi}{4}
  \item  记极限为  $L$.  当  $a=1$  时 , $L=1$.  当  $0<a<1$  时 ,
\end{enumerate}
$$
L=\lim _{n \rightarrow \infty} \exp \left(\frac{\ln \left(1+a^{n}\right)}{n}\right)=\exp \left(\lim _{n \rightarrow \infty} \frac{a^{n}}{n}\right)=1
$$
 当  $a>1$  时 ,
$$
L=\lim _{n \rightarrow \infty} \exp \left(\frac{\ln \left(1+a^{n}\right)}{n}\right)=\exp \left(\lim _{n \rightarrow \infty} \frac{n \ln a+\ln \left(\left(\frac{1}{a}\right)^{2}+1\right)}{n}\right)=a
$$
 三 . 1 .  因为 
$$
\sum_{n=1}^{\infty} x^{n}=\frac{x}{1-x}, \quad|x|<1
$$
 逐项求导得 
$$
\sum_{n=1}^{\infty} n x^{n-1}=\frac{1}{(1-x)^{2}}, \quad|x|<1
$$
 从而 
$$
\sum_{n=0}^{\infty} \frac{n}{3^{n}} 2^{n}=\frac{\frac{2}{3}}{\left(1-\frac{2}{3}\right)^{2}}=6
$$

\begin{enumerate}
  \setcounter{enumi}{2}
  \item  由积分中值定理知  $\exists \eta \in[7 / 8,1]$,  使得  $f(\eta)=f(0)$,  然后用  Rolle  定理知  $\exists \xi \in(0, \eta) \subset(0,1), f^{\prime}(\xi)=0$.

  \item  因为  $\sum_{n=0}^{\infty} \frac{(-1)^{n}}{\sqrt{n}}$  收敛 , $\arctan n$  单调递增趋于  $\pi / 2$,  由  Abel  判别法知级数  $\sum_{n=1}^{\infty}(-1)^{n} \frac{\arctan n}{\sqrt{n}}$  收敛 .

  \item  要证明  $\exists \varepsilon_{0}>0, \forall \Delta>0, \exists A>\Delta, \exists x \in(0,+\infty)$,  使得 

\end{enumerate}
$$
\left|\int_{A}^{+\infty} x \mathrm{e}^{-x y} \mathrm{~d} y\right|=\mathrm{e}^{-A x} \geq \varepsilon_{0}
$$
 令  $\varepsilon_{0}=\mathrm{e}^{-1}, \forall \Delta>0$,  令  $A=\Delta+1, x=1 / A$  即可 .

$5 .$
$$
\begin{gathered}
\frac{\mathrm{d} u}{\mathrm{~d} x}=\frac{\partial f}{\partial x}+\frac{\partial f}{\partial y} \cos x+\frac{\partial f}{\partial x} \frac{\partial z}{\partial x} \\
2 x g_{1}^{\prime}+\mathrm{e}^{\sin x} \cos x g_{2}^{\prime}+\frac{\partial z}{\partial x} g_{3}^{\prime}=0
\end{gathered}
$$
 由 (2) 可以得到  $\frac{\partial z}{\partial x}$,  再带入 (1) 得到结果 .

\begin{enumerate}
  \setcounter{enumi}{6}
  \item  由  $\lim _{x \rightarrow 0} \frac{f(x)}{x}=0$  及  $f(x)$  在  $[-1,1]$  上二次连续可微知 
\end{enumerate}
$$
f(0)=\lim _{x \rightarrow 0} f(x)=0, \quad f^{\prime}(0)=\lim _{x \rightarrow 0} \frac{f(x)-f(0)}{x-0}=0
$$
 从而当  $x>0$  时 
$$
f(x)=f(0)+f^{\prime}(0) x+\frac{f^{\prime \prime}(\xi)}{2 !} x^{2}=\frac{f^{\prime \prime}(\xi)}{2 !} x^{2}, \quad(\xi \in(0, x))
$$
 今 
$$
M=\max _{x \in[-1,1]}\left|f^{\prime \prime}(x)\right|
$$
 贝则 
$$
\sum_{n=1}^{\infty}\left|f\left(\frac{1}{n}\right)\right| \leq M \sum_{n=1}^{\infty} \frac{1}{n^{2}}
$$
 从而级数  $\sum_{n=1}^{\infty} f\left(\frac{1}{n}\right)$  绝对收敛 .  北京大学  2000  年全国硕士研究生招生考试数学分析试题及解答 

   

2019.05.26

 一 .  计算 : $(8$  分  $\times 5=40$  分 )

\begin{enumerate}
  \item  求极限  $\lim _{x \rightarrow 0} \frac{(a+x)^{x}-a^{x}}{x^{2}}, a>0$.

  \item  求  $\mathrm{e}^{2 x-x^{2}}$  到含  $x^{5}$  项的  Taylor  展开式 .

\end{enumerate}
3 .  求积分  $\int_{0}^{1} \frac{x^{b}-x^{a}}{\ln x} \mathrm{~d} x$,  其中  $a>b>0$.

\begin{enumerate}
  \setcounter{enumi}{4}
  \item  求积分  $\iiint_{V}\left(x^{2}+y^{2}+z^{2}\right)^{\alpha} \mathrm{d} x \mathrm{~d} y \mathrm{~d} z, V$  是实心球  $x^{2}+y^{2}+z^{2} \leqslant R^{2}, \alpha>0$.

  \item  求积分  $\iint_{S} x^{3} \mathrm{~d} y \mathrm{~d} z+y^{3} \mathrm{~d} x \mathrm{~d} z+z^{3} \mathrm{~d} x \mathrm{~d} y, S$  是  $x^{2}+y^{2}+z^{2}=a^{2}$  的外表面 .

\end{enumerate}
 二 . (5  分  $+5$  分  $=10$  分 )  叙述定义 .

\begin{enumerate}
  \item $\lim _{x \rightarrow-\infty} f(x)=+\infty$.

  \item  当  $x \rightarrow a-0$  时 , $f(x)$  不以  $A$  为极限 .

\end{enumerate}
 三 . (13  分 )  函数  $f(x)$  在  $[a, b]$  上一致连续 ,  又在  $[b, c]$  上一致连续 , $a<b<c$.  用定义证明  $f(x)$  在  $[a, c]$  上一致   连续 .

 四 . (10  分 )  构造一个二元函数  $f(x, y)$,  使得它在原点  $(0,0)$  两个偏导数都存在 ,  但在原点不可微 .

 五 . (12  分 )  函数  $f(x)$  在  $[a, b]$  连续 .  证明不等式 : $\left(\int_{a}^{b} f(x) \mathrm{d} x\right)^{2} \leqslant(b-a) \int_{a}^{b} f^{2}(x) \mathrm{d} x$.

 六 . $(7$  分  $+8$  分  $=15$  分 )

\begin{enumerate}
  \item  在区间  $(0,2 \pi)$  内展开  $f(x)$  的  Fourier  级数 ,  其中  $f(x)=\frac{\pi-x}{2}$.

  \item  证明它的  Fourier  级数在  $(0,2 \pi)$  内每一点上收玫于  $f(x)$.

\end{enumerate}
$$
\begin{aligned}
\text { 原式 } &=\lim _{x \rightarrow 0} \frac{\mathrm{e}^{x \ln (a+x)}-\mathrm{e}^{x \ln }}{x^{2}} \\
&=\lim _{x \rightarrow 0} \frac{\mathrm{e}^{x \ln \frac{a+x}{a}}-1}{x^{2}} \\
&=\lim _{x \rightarrow 0} \frac{\ln \frac{a+x}{a}}{x} \\
&=\frac{1}{a} .
\end{aligned}
$$
$2 .$
$$
\begin{aligned}
\mathrm{e}^{2 x-x^{2}} &=\left(1+2 x+\frac{(2 x)^{2}}{2 !}+\frac{(2 x)^{3}}{3 !}+\frac{(2 x)^{4}}{4 !}+\frac{(2 x)^{5}}{5 !}+o\left(x^{5}\right)\right)\left(1-x^{2}++\frac{x^{4}}{2 !}+o\left(x^{5}\right)\right) \\
&=1+2 x+\frac{(2 x)^{2}}{2 !}+\frac{(2 x)^{3}}{3 !}+\frac{(2 x)^{4}}{4 !}+\frac{(2 x)^{5}}{5 !}-x^{2}-2 x^{3}-2 x^{4}-\frac{4}{3} x^{5}+\frac{x^{4}}{2}+x^{5}+o\left(x^{5}\right) \\
&=1+2 x+x^{2}-\frac{2}{3} x^{2}-\frac{5}{6} x^{4}-\frac{1}{15} x^{5}+o\left(x^{5}\right)
\end{aligned}
$$

\begin{enumerate}
  \setcounter{enumi}{3}
  \item  设这积分为  $I$,  则它可以写成 
\end{enumerate}
$$
I=\int_{0}^{1} \mathrm{~d} x \int_{a}^{b} x^{y} \mathrm{~d} y
$$
 因为 
$$
f(x, y)=x^{y}
$$
 在  $[0,1] \times[a, b]$  连续 ,  所以两个积分号可以交换次序 .  我们得到 
$$
I=\int_{0}^{1} \mathrm{~d} x \int_{a}^{b} x^{y} \mathrm{~d} y=\int_{a}^{b} \mathrm{~d} y \int_{0}^{1} x^{y} \mathrm{~d} x=\int_{a}^{b} \frac{1}{y+1} \mathrm{~d} y=\ln \frac{b+1}{a+1} .
$$
 注   此题为数学分析教材上含参变量积分部分常见的一个例子 ,  比如上面的证明取自张筑生的  《 数学分析   新讲 》 第三册第  337  页例  $1 .$

\begin{enumerate}
  \setcounter{enumi}{4}
  \item  用一下极坐标变换即可 ,  当然也可以用如下的柱坐标变换 :
\end{enumerate}
$$
\begin{aligned}
\text { 原式 } &=\int_{-R}^{R} \mathrm{~d} z \iint_{x^{2}+y^{2} \leqslant R^{2}-z^{2}}\left(x^{2}+y^{2}+z^{2}\right)^{\alpha} \mathrm{d} x \mathrm{~d} y \\
&=\int_{-R}^{R} \mathrm{~d} z \int_{0}^{2 \pi} \mathrm{d} \theta \int_{0}^{\sqrt{R^{2}-z^{2}}}\left(r^{2}+z^{2}\right)^{\alpha} r \mathrm{~d} r \\
&=4 \pi \frac{R^{2 \alpha+3}}{2 \alpha+3} .
\end{aligned}
$$

\begin{enumerate}
  \setcounter{enumi}{5}
  \item  用  Gauss  公式并利用上题的结果得 
\end{enumerate}
$$
\begin{aligned}
\text { 原式 } &=\iiint_{V} 3\left(x^{2}+y^{2}+z^{2}\right) \mathrm{d} x \mathrm{~d} y \mathrm{~d} z \\
&=\frac{12 \pi a^{5}}{5} .
\end{aligned}
$$
 二 . $1 . \forall M>0, \exists \Delta>0$,  当  $x<-M$  时 ,  均有  $f(x)>M$.

\begin{enumerate}
  \setcounter{enumi}{2}
  \item $\exists \varepsilon_{0}>0, \forall \delta>0, \exists x_{\delta} \in(a-\delta, a)$,  使得  $\left|f\left(x_{\delta}\right)-A\right| \geqslant \varepsilon_{0}$.
\end{enumerate}
 三 .  由题意知 , $\forall \varepsilon>0, \exists \delta>0$,  当  $x, y \in[a, b]$  且  $|x-y|<\delta$  时 , $|f(x)-f(y)|<\varepsilon / 2$.  当  $x, y \in[b, c]$  且  $|x-y|<\delta$  时 , $|f(x)-f(y)|<\varepsilon / 2 . \forall x, y \in[a, c]$,  不妨设  $x<y$,  当  $|x-y|<\delta$  时 ,  若  $x, y$  同时在  $[a, b]$  里或者同时在  $[b, c]$  里 ,  则自然  $|f(x)-f(y)|<\varepsilon / 2<\varepsilon$.  若  $x<b<y$,  则  $|f(x)-f(y)| \leqslant|f(x)-f(b)|+|f(b)-f(y)|<\varepsilon$.  注   谢惠民等人的 《 数学分析习题课讲义 》 上册第  140  页例题  $5.4 .4$  与此题几乎完全一样 .

 四 .  考虑 
$$
f(x, y)= \begin{cases}\frac{x y}{x^{2}+y^{2}}, & x^{2}+y^{2} \neq 0 \\ 0, & x^{2}+y^{2}=0\end{cases}
$$
 五 .  关于  $t$  的二次函数 
$$
\int_{a}^{b} f^{2}(x) \mathrm{d} x+2 t \int_{a}^{b} f(x) \mathrm{d} x+t^{2} \int_{a}^{b} \mathrm{~d} x=\int_{a}^{b}(f(x)+t)^{2} \mathrm{~d} x \geqslant 0,
$$
 从而它对应的二次方程的判别式 
$$
\Delta=4\left(\int_{a}^{b} f(x) \mathrm{d} x\right)^{2}-4 \int_{a}^{b} \mathrm{~d} x \int_{a}^{b} f^{2}(x) \mathrm{d} x \leqslant 0
$$
 故 
$$
\left(\int_{a}^{b} f(x) \mathrm{d} x\right)^{2} \leqslant(b-a) \int_{a}^{b} f^{2}(x) \mathrm{d} x
$$
․ $1 .$
$$
a_{0}=\frac{1}{\pi} \int_{0}^{2 \pi} \frac{\pi-x}{2} \mathrm{~d} x=0, a_{n}=\frac{1}{\pi} \int_{0}^{2 \pi} \frac{\pi-x}{2} \cos n x \mathrm{~d} x=0, b_{n}=\frac{1}{\pi} \int_{0}^{2 \pi} \frac{\pi-x}{2} \sin n x \mathrm{~d} x=\frac{1}{n}
$$
 从而 
$$
f(x) \sim \sum_{n=1}^{\infty} \frac{\sin n x}{n}
$$
2 .  设  $0<x<2 \pi$,  令 
$$
S_{N}(x)=\sum_{n=1}^{N} \frac{\sin (n x)}{n}
$$
 则 
$$
S_{N}^{\prime}(x)=\sum_{n=1}^{N} \cos (n x)=\frac{\sin \left(\left(N+\frac{1}{2}\right) x\right)-\sin \frac{x}{2}}{2 \sin \frac{x}{2}}
$$
 故 
$$
S_{N}(x)=\int_{\pi}^{x} S_{N}^{\prime}(t) \mathrm{d} t+S_{N}(\pi)=\int_{\pi}^{x} \frac{\sin \left(\left(N+\frac{1}{2}\right) t\right)-\sin \frac{t}{2}}{2 \sin \frac{t}{2}} \mathrm{~d} t
$$
 结合  Riemann-Lebesgue  引理知 
$$
\sum_{n=1}^{\infty} \frac{\sin (n x)}{n}=\lim _{N \rightarrow \infty} S_{N}(x)=\frac{\pi-x}{2}=f(x)
$$
 注   此题第二问的证明源自裴礼文的 《 数学分析中典型问题与方法 》 第二版第  442  页例  5.1.10.  北京大学  2017  年与  2018  年的数学分析试题中有关  Fourier  级数的题目再次考了第二问证明中用到的技巧 .  北京大学  2001  年全国硕士研究生招生考试数学分析试题及解答 

   

2019.05.26

\begin{enumerate}
  \item (10  分 )  求极限 
\end{enumerate}
$$
\lim _{n \rightarrow \infty} \frac{a^{2 n}}{1+a^{2 n}} .
$$

\begin{enumerate}
  \setcounter{enumi}{2}
  \item (10  分 )  设  $f(x)$  在点  $a$  可导 , $f(a) \neq 0$.  求极限 
\end{enumerate}
$$
\lim _{n \rightarrow \infty}\left(\frac{f\left(a+\frac{1}{n}\right)}{f(a)}\right)^{n} .
$$

\begin{enumerate}
  \setcounter{enumi}{3}
  \item (10  分 )  证明函数  $f(x)=\sqrt{x} \ln x$  在  $[1,+\infty)$  上一致连续 .

  \item (10  分 )  设  $D$  是包含原点的平面凸区域 , $f(x, y)$  在  $D$  上可微 ,  且 

\end{enumerate}
$$
x \frac{\partial f}{\partial x}+y \frac{\partial f}{\partial y}=0 .
$$
 证明 : $f(x, y)$  在  $D$  上恒为常数 .

\begin{enumerate}
  \setcounter{enumi}{5}
  \item (10  分 )  计算第一型曲面积分 
\end{enumerate}
$$
\iint_{\Sigma} x \mathrm{~d} S,
$$
 其中  $\Sigma$  是雉面  $z=\sqrt{x^{2}+y^{2}}$  被柱面  $x^{2}+y^{2}=a x(a>0)$  割下的部分 .

\begin{enumerate}
  \setcounter{enumi}{6}
  \item (10  分 )  求极限 
\end{enumerate}
$$
\lim _{t \rightarrow 0^{+}} \frac{1}{t^{4}} \iiint_{x^{2}+y^{2} \leqslant t^{2}} f\left(\sqrt{x^{2}+y^{2}+z^{2}}\right) \mathrm{d} x \mathrm{~d} y \mathrm{~d} z,
$$
 其中  $f$  在  $[0,1]$  上连续 , $f(0)=0, f^{\prime}(0)=1$.

\begin{enumerate}
  \setcounter{enumi}{7}
  \item (10  分 )  求常数  $\lambda$,  使得曲线积分 
\end{enumerate}
$$
\int_{L} \frac{x}{y} r^{\lambda} \mathrm{d} x-\frac{x^{2}}{y^{2}} r^{\lambda} \mathrm{d} y=0 \quad\left(r=\sqrt{x^{2}+y^{2}}\right)
$$
 对上半平面的任何光滑闭曲线  $L$  成立 .

\begin{enumerate}
  \setcounter{enumi}{8}
  \item (10  分 )  证明函数  $f(x)=\sum_{n=1}^{\infty} \frac{1}{n^{x}}$  在  $(1,+\infty)$  上无穷次可微 .

  \item (10  分 )  求广义积分 

\end{enumerate}
$$
\int_{0}^{+\infty} \frac{\arctan \left(b x^{2}\right)-\arctan \left(a x^{2}\right)}{x} \mathrm{~d} x, \quad b>a>0
$$

\begin{enumerate}
  \setcounter{enumi}{10}
  \item (10  分 )  设  $f(x)$  是以  $2 \pi$  为周期的周期函数 ,  且  $f(x)=x,-\pi \leqslant x<\pi$.  求  $f(x)$  与  $|f(x)|$  的  Fourier  级数 ,  它们的  Fourier  级数是否一致收敛  ( 给出证明 )? 1.  当  $|a|=1$  时 ,  极限为  $1 / 2$.  当  $|a|<1$  时 ,  极限为  0 .  当  $|a|>1$  时 ,  极限为  1 .
\end{enumerate}
$2 .$
$$
\begin{aligned}
\text { 原式 } &=\mathrm{e}^{\lim _{n \rightarrow \infty} n \ln \frac{f\left(a+\frac{1}{n}\right)}{f(a)}} \\
&=\mathrm{e}^{\lim _{n \rightarrow \infty} n \frac{f\left(a+\frac{1}{n}\right)-f(a)}{f(a)}} \\
&=\mathrm{e}^{\frac{f^{\prime}(a)}{f(a)}} .
\end{aligned}
$$
 注   此题为北京大学  1996  年第二大题 .  也可参考裴礼文的 《 数学分析中的典型问题与方法 》 第二版第  46  页例  1.1.13.

\begin{enumerate}
  \setcounter{enumi}{3}
  \item  因为 
\end{enumerate}
$$
f^{\prime}(x)=\frac{\ln x}{2 \sqrt{x}}+\frac{1}{\sqrt{x}}, \quad \lim _{x \rightarrow+\infty} f^{\prime}(x)=0
$$
 从而存在  $M>1$,  当  $x>M$  时 , $\left|f^{\prime}(x)\right|<1$.  而  $f^{\prime}(x)$  为  $[1, M]$  上的连续函数 ,  必定有界 .  从而  $f^{\prime}(x)$  在  $[1,+\infty)$  上有界 .  设  $\left|f^{\prime}(x)\right|<K, \forall x \in[1,+\infty)$.

 那么  $\forall \varepsilon>0, \exists \delta=\varepsilon /(2 K)>0$,  当  $x, y \in[1,+\infty)$  且  $|x-y|<\delta$  时 , $|f(x)-f(y)| \leqslant K|x-y|<\varepsilon$.

\begin{enumerate}
  \setcounter{enumi}{4}
  \item  对应取定的  $(x, y) \in D$,  令  $g(t)=f(t x, t y), t \in[0,1]$,  则 
\end{enumerate}
$$
g^{\prime}(t)=x \frac{\partial f}{\partial x}(t x, t y)+y \frac{\partial f}{\partial y}(t x, t y)=\frac{1}{t}\left(t x \frac{\partial f}{\partial x}(t x, t y)+t y \frac{\partial f}{\partial y}(t x, t y)\right)=0
$$
 从而  $f(x, y)-f(0,0)=g(1)-g(0)=0$.

 注   林源渠 、 方企勤编的 《 数学分析解题指南 》 第  283  页练习题  $5.2 .27$  的前半部分 ,  原题后半部分说了当  $D$  不   包含原点时  $f(x, y)$  可以不是常数函数 .  与此题相关的知识点为齐次函数 ,  可以参考裴礼文的  《 数学分析中   的典型问题与方法 》 第二版第  650  页例  $6.2 .6$.

$5 .$
$$
\text { 原式 }=\sqrt{2} \iint_{x^{2}+y^{2} \leqslant a x} x \mathrm{~d} x \mathrm{~d} y=\sqrt{2} \int_{0}^{2 \pi} \mathrm{d} \theta \int_{0}^{a / 2}\left(\frac{a}{2}+r \cos \theta\right) r \mathrm{~d} r=\frac{\sqrt{2} \pi a^{3}}{8} \text {. }
$$

\begin{enumerate}
  \setcounter{enumi}{6}
  \item  做极坐标变换得 
\end{enumerate}
$$
\iiint_{x^{2}+y^{2} \leqslant t^{2}} f\left(\sqrt{x^{2}+y^{2}+z^{2}}\right) \mathrm{d} x \mathrm{~d} y \mathrm{~d} z=\int_{0}^{2 \pi} \mathrm{d} \theta \int_{-\pi / 2}^{\pi / 2} \mathrm{~d} \varphi \int_{0}^{t} f(r) r^{2} \cos \varphi \mathrm{d} r=4 \pi \int_{0}^{t} f(r) r^{2} \mathrm{~d} r .
$$
 因此 
$$
\text { 原式 }=\lim _{t \rightarrow 0^{+}} \frac{4 \pi}{t^{4}} \int_{0}^{t} f(r) r^{2} \mathrm{~d} r=\lim _{t \rightarrow 0^{+}} \frac{4 \pi f(t) t^{2}}{4 t^{3}}=\pi \lim _{t \rightarrow 0^{+}} \frac{f(t)-f(0)}{t-0}=\pi f^{\prime}(0)=\pi .
$$
 注   基本相同的题目见裴礼文的  《 数学分析中的典型问题与方法 》 第二版第  915  页练习题  $7.2 .14$.

$7 .$
$$
\begin{aligned}
\frac{\partial}{\partial y}\left(\frac{x}{y} r^{\lambda}\right) &=-\frac{x}{y^{2}}\left(x^{2}+y^{2}\right)^{\frac{\lambda}{2}}+\lambda x\left(x^{2}+y^{2}\right)^{\frac{\lambda}{2}-1} \\
\frac{\partial}{\partial y}\left(\frac{x^{2}}{y^{2}} r^{\lambda}\right) &=\frac{2 x}{y^{2}}\left(x^{2}+y^{2}\right)^{\frac{\lambda}{2}}+\lambda \frac{x^{3}}{y^{2}}\left(x^{2}+y^{2}\right)^{\frac{\lambda}{2}-1}
\end{aligned}
$$
 当  $\lambda=-1$  时 ,  上面两式的和为  0 ,  由  Green  公式知满足题意 . 8.  对于任意取定的  $s>1, \forall x \in\left[\frac{s+1}{2}, s+1\right]$,
$$
\sum_{n=1}^{\infty} \frac{\ln n}{n^{x}} \leqslant \sum_{n=1}^{\infty} \frac{\ln n}{n^{\frac{s+1}{2}}}
$$
 从而  $\sum_{n=1}^{\infty} \frac{-\ln n}{n^{x}}$  在  $\left[\frac{s+1}{2}, s+1\right]$  上一致收敛 ,  故 
$$
f^{\prime}(x)=\left(\sum_{n=1}^{\infty} \frac{1}{n^{x}}\right)^{\prime}=\sum_{n=1}^{\infty} \frac{-\ln n}{n^{x}}, \quad x \in\left[\frac{s+1}{2}, s+1\right]
$$
 从而  $f(x)$  在  $x=s$  处可微 ,  由  $s$  的任意性知  $f(x)$  在  $(1,+\infty)$  上可微 ,  且  $f^{\prime}(x)=\sum_{n=1}^{\infty} \frac{-\ln n}{n^{x}}$.

 假若 
$$
f^{(k)}(x)=\sum_{n=1}^{\infty} \frac{(-\ln n)^{k}}{n^{x}}, \quad x \in(1,+\infty)
$$
 同样可以证得  $f^{(k)}(x)$  在  $(1,+\infty)$  上可微 ,  且  $f^{(k+1)}(x)=\sum_{n=1}^{\infty} \frac{(-\ln n)^{k+1}}{n^{x}}$.  由数学归纳法原理知原命题成立 .  注   也可以参考林源渠 、 方企勤编的 《 数学分析解题指南 》 第  230  页例  9 .

$9 .$
$$
\begin{aligned}
\text { 原式 } &=\int_{0}^{+\infty}\left(\frac{1}{x} \int_{a}^{b} \frac{x^{2} \mathrm{~d} y}{1+x^{4} y^{2}}\right) \mathrm{d} x=\int_{0}^{+\infty} \mathrm{d} x \int_{a}^{b} \frac{x}{1+x^{4} y^{2}} \mathrm{~d} y \\
& \stackrel{*}{=} \int_{a}^{b} \mathrm{~d} y \int_{0}^{+\infty} \frac{x}{1+x^{4} y^{2}} \mathrm{~d} x=\frac{\pi}{4} \int_{a}^{b} \frac{1}{y} \mathrm{~d} y=\frac{\pi}{4} \ln \frac{b}{a} .
\end{aligned}
$$
 其中带  $*$  的那个等号能成立是因为  $\int_{0}^{+\infty} \frac{x}{1+x^{4} y^{2}} \mathrm{~d} x$  在  $y \in[a, b]$  上一致收敛 ,  从而能交换积分的顺序 .

\begin{enumerate}
  \setcounter{enumi}{10}
  \item  因为  $f(x)$  为奇函数 ,  故  $a_{n}=0, n=0,1,2, \ldots$,
\end{enumerate}
$$
b_{n}=\frac{2}{\pi} \int_{0}^{\pi} x \sin n x \mathrm{~d} x=\frac{2(-1)^{n}}{n}, \quad n=1,2, \ldots
$$
 因此  $f(x)$  的  Fourier  级数为 
$$
\sum_{n=1}^{\infty} \frac{2(-1)^{n}}{n} \sin n x
$$
 由  Fourier  级数的收敛性判别法知  $f(x)$  的  Fourier  级数在  $[-\pi, \pi]$  上逐点收敛到 
$$
s(x)=\frac{f(x+0)+f(x-0)}{2}= \begin{cases}x, & -\pi<x<\pi \\ 0, & |x|=\pi\end{cases}
$$
 若  Fourier  级数在  $[-\pi, \pi]$  上一致收敛到  $s(x)$,  则  $s(x)$  在  $[-\pi, \pi]$  上连续 ,  矛盾 ,  从而不是一致收敛 .  因为  $|f(x)|$  为偶函数 ,  故  $b_{n}=0, n=1,2, \ldots$,
$$
a_{0}=\frac{2}{\pi} \int_{0}^{\pi} x \mathrm{~d} x=\pi, a_{n}=\frac{2}{\pi} \int_{0}^{\pi} x \cos n x \mathrm{~d} x=\frac{2}{\pi} \frac{(-1)^{n}-1}{n^{2}}, \quad n=1,2, \ldots
$$
 因此  $|f(x)|$  的  Fourier  级数为 
$$
\frac{\pi}{2}+\frac{2}{\pi} \sum_{n=1}^{\infty} \frac{(-1)^{n}-1}{n^{2}} \cos n x .
$$
 由  Weierstrass  判别法知  $|f(x)|$  的  Fourier  级数在  $[-\pi, \pi]$  上一致收敛 .  北京大学  2002  年全国硕士研究生招生考试数学分析试题及解答 

   

2019.05.26

\begin{enumerate}
  \item (10  分 )  求极限 
\end{enumerate}
$$
\lim _{x \rightarrow 0}\left(\frac{\sin x}{x}\right)^{\frac{1}{1-\cos x}} .
$$

\begin{enumerate}
  \setcounter{enumi}{2}
  \item (10  分 )  设  $a \geqslant 0, x_{1}=\sqrt{2+a}, x_{n+1}=\sqrt{2+x_{n}}, n=1,2, \cdots$,  证明极限  $\lim _{n \rightarrow \infty} x_{n}$  存在并求极限值 .

  \item (10  分 )  设  $f(x)$  在  $[a, a+2 \alpha]$  上连续 ,  证明存在  $x \in[a, a+\alpha]$,  使得 

\end{enumerate}
$$
f(x+\alpha)-f(x)=\frac{f(a+2 \alpha)-f(a)}{2} .
$$

\begin{enumerate}
  \setcounter{enumi}{4}
  \item (10  分 )  设  $f(x)=x \sqrt{1-x^{2}}+\arctan x$,  求  $f^{\prime}(x)$.

  \item (10  分 )  设  $u(x, y)$  有二阶连续偏导数 .  证明  $u$  满足偏微分方程 

\end{enumerate}
$$
\frac{\partial^{2} u}{\partial x^{2}}-2 \frac{\partial^{2} u}{\partial x \partial y}+\frac{\partial^{2} u}{\partial y^{2}}=0
$$
 当且仅当 :  存在二阶连续可微函数  $\varphi(t), \psi(t)$,  使得  $u(x, y)=x \varphi(x+y)+y \psi(x+y)$.

\begin{enumerate}
  \setcounter{enumi}{6}
  \item (10  分 )  计算三重积分 
\end{enumerate}
$$
\iiint_{\Omega} x^{2} \sqrt{x^{2}+y^{2}} \mathrm{~d} x \mathrm{~d} y \mathrm{~d} z,
$$
 其中  $\Omega$  是曲面  $z=\sqrt{x^{2}+y^{2}}$  与  $z=x^{2}+y^{2}$  围成的有界区域 .

\begin{enumerate}
  \setcounter{enumi}{7}
  \item (10  分 )  计算第二型曲面积分 
\end{enumerate}
$$
I=\iint_{\Sigma} x^{2} \mathrm{~d} y \mathrm{~d} z+y^{2} \mathrm{~d} z \mathrm{~d} x+z^{2} \mathrm{~d} x \mathrm{~d} y,
$$
 其中  $\Sigma$  是球面  $x^{2}+y^{2}+z^{2}=a z(a>0)$  的外侧 .

\begin{enumerate}
  \setcounter{enumi}{8}
  \item (10  分 )  判断级数  $\sum_{n=1}^{\infty} \ln \cos \frac{1}{n}$  的玫散性并给出证明 .

  \item (10  分 )  证明 :

\end{enumerate}
(1)  函数项级数  $\sum_{n=1}^{\infty} n x \mathrm{e}^{-n x}$  在区间  $(0,+\infty)$  上不一致收敛 ;

(2)  函数项级数  $\sum_{n=1}^{\infty} n x \mathrm{e}^{-n x}$  在区间  $(0,+\infty)$  上可逐项求导 .

\begin{enumerate}
  \setcounter{enumi}{10}
  \item (10  分 )  设  $f(x)$  连续 ,
\end{enumerate}
$$
g(x)=\int_{0}^{x} y f(x-y) \mathrm{d} y
$$
 求  $g^{\prime \prime}(x)$. $1 .$
$$
\text { 原式 }=\exp \left(\lim _{x \rightarrow 0} \frac{\ln \frac{\sin x}{x}}{1-\cos x}\right)=\exp \left(\lim _{x \rightarrow 0} \frac{\sin x-x}{(1-\cos x) x}\right)=\exp \left(\frac{-\frac{1}{3 !}}{\frac{1}{2 !}}\right)=\mathrm{e}^{-1 / 3} \text {. }
$$
$2 .$
$$
\left|x_{n+2}-x_{n+1}\right|=\left|\sqrt{2+x_{n+1}}-\sqrt{2+x_{n}}\right|=\frac{\left|x_{n+1}-x_{n}\right|}{\sqrt{2+x_{n+1}}+\sqrt{2+x_{n}}} \leqslant \frac{1}{2 \sqrt{2}}\left|x_{n+1}-x_{n}\right| .
$$
 令  $q=\frac{1}{2 \sqrt{2}}$,  则  $0<q<1 . \forall p \in \mathbb{N}$,
$$
\left|x_{n+p}-x_{n}\right| \leqslant \sum_{k=n}^{n+p-1}\left|x_{k+1}-x_{k}\right| \leqslant\left(q^{n-1}+q^{n}+\cdots+q^{n+p-1}\right)\left|x_{2}-x_{1}\right|<\frac{q^{n-1}}{1-q}\left|x_{2}-x_{1}\right|,
$$
 因为  $q^{n-1} \rightarrow 0, n \rightarrow \infty$,  由  Cauchy  收敛原理知  $\left\{x_{n}\right\}$  收敛 ,  设极限为  $b$,  则  $b>\sqrt{2}, b=\sqrt{2+b}$,  故  $b=2$.  注   压缩映射原理 .

\begin{enumerate}
  \setcounter{enumi}{3}
  \item  令  $g(x)=f(x+\alpha)-f(x), x \in[a, a+\alpha]$,  则  $g(x)$  在  $[a, a+\alpha]$  上连续 ,  由介值定理知  $\exists \xi \in[a, a+\alpha]$,  使得 
\end{enumerate}
$$
f(\xi+\alpha)-f(\xi)=g(\xi)=\frac{g(a)+g(a+\alpha)}{2}=\frac{f(a+2 \alpha)-f(a)}{2} .
$$

\begin{enumerate}
  \setcounter{enumi}{4}
  \item $f^{\prime}(x)=2 \sqrt{1-x^{2}}, x \in(-1,1)$.

  \item  充分性 : 因为 

\end{enumerate}
$$
\begin{aligned}
\frac{\partial^{2} u}{\partial x^{2}} &=2 \varphi^{\prime}(x+y)+x \varphi^{\prime \prime}(x+y)+y \psi^{\prime \prime}(x+y) \\
\frac{\partial^{2} u}{\partial x \partial y} &=\varphi^{\prime}(x+y)+x \varphi^{\prime \prime}(x+y)+\psi^{\prime}(x+y)+y \psi^{\prime \prime}(x+y) \\
\frac{\partial^{2} u}{\partial y^{2}} &=2 \psi^{\prime}(x+y)+y \psi^{\prime \prime}(x+y)+x \varphi^{\prime \prime}(x+y)
\end{aligned}
$$
 故 
$$
\frac{\partial^{2} u}{\partial x^{2}}-2 \frac{\partial^{2} u}{\partial x \partial y}+\frac{\partial^{2} u}{\partial y^{2}}=0
$$
 必要性 : 做变量替换 
$$
\left\{\begin{array}{l}
\xi=x+y \\
\eta=x-y
\end{array}\right.
$$
 则 
$$
\begin{aligned}
\frac{\partial^{2} u}{\partial x^{2}} &=\frac{\partial^{2} u}{\partial \xi^{2}}+2 \frac{\partial^{2} u}{\partial \xi \partial \eta}+\frac{\partial^{2} u}{\partial \eta^{2}} \\
\frac{\partial^{2} u}{\partial x \partial y} &=\frac{\partial^{2} u}{\partial \xi^{2}}-\frac{\partial^{2} u}{\partial \eta^{2}} \\
\frac{\partial^{2} u}{\partial y^{2}} &=\frac{\partial^{2} u}{\partial \xi^{2}}-2 \frac{\partial^{2} u}{\partial \xi \partial \eta}+\frac{\partial^{2} u}{\partial \eta^{2}}
\end{aligned}
$$
 故 
$$
\frac{\partial^{2} u}{\partial x^{2}}-2 \frac{\partial^{2} u}{\partial x \partial y}+\frac{\partial^{2} u}{\partial y^{2}}=0 \Longrightarrow 4 \frac{\partial^{2} u}{\partial \eta^{2}}=0 \Longrightarrow \frac{\partial u}{\partial \eta}=f(\xi) \Longrightarrow u=f(\xi) \eta+g(\xi)
$$
 因此 
$$
\begin{aligned}
u &=f(x+y)(x-y)+g(x+y)=f(x+y)(x-y)+(x+y) \frac{g(x+y)}{x+y} \\
&=x\left(f(x+y)+\frac{g(x+y)}{x+y}\right)+y\left(\frac{g(x+y)}{x+y}-f(x+y)\right) .
\end{aligned}
$$
 注   在证明必要性的时候之所以要做那么一个变量替换 ,  是为了简化偏微分方程 ,  方便求解 .  相关的题目可以参   考谢惠民等人的  《 数学分析习题课讲义 》 下册第  205  页例题  $20.5 .2$,  那里的解法更好 .  中国科学技术大学的   数学分析考研试题中也有不少这种类型的题目 .

\begin{enumerate}
  \setcounter{enumi}{6}
  \item  原式  $=\int_{0}^{1} \mathrm{~d} z \iint_{z^{2} \leqslant x^{2}+y^{2} \leqslant z} x^{2} \sqrt{x^{2}+y^{2}} \mathrm{~d} x \mathrm{~d} y=\int_{0}^{1} \mathrm{~d} z \int_{0}^{2 \pi} \mathrm{d} \theta \int_{z}^{\sqrt{z}} r r^{2} \cos ^{2} \theta r \mathrm{~d} r=\frac{\pi}{42}$.

  \item  先利用  Gauss  公式 ,  再利用积分区域的对称性 ,  最后化为累次积分计算 ,

\end{enumerate}
$$
\begin{aligned}
I &=\iiint_{x^{2}+y^{2}+z^{2} \leqslant a z} 2(x+y+z) \mathrm{d} x \mathrm{~d} y \mathrm{~d} z=2 \iiint_{x^{2}+y^{2}+z^{2} \leqslant a z} z \mathrm{~d} x \mathrm{~d} y \mathrm{~d} z \\
&=2 \int_{0}^{a} z \mathrm{~d} z \iint_{x^{2}+y^{2} \leqslant z^{2}-a z} \mathrm{~d} x \mathrm{~d} y=2 \pi \int_{0}^{a} z\left(z^{2}-a z\right) \mathrm{d} z=\frac{\pi a^{4}}{6} .
\end{aligned}
$$

\begin{enumerate}
  \setcounter{enumi}{8}
  \item  因为 
\end{enumerate}
$$
\lim _{n \rightarrow \infty} \frac{\ln \cos \frac{1}{n}}{\frac{1}{n^{2}}}=-\frac{1}{2},
$$
 由比值判别法知  $\sum_{n=1}^{\infty} \ln \cos \frac{1}{n}$  绝对收敛 .

 注   与它相比较的级数是通过计算  $\ln \cos x$  的  Taylor  级数找到的 ,  此题与北京大学  2014  年数学分析第  5  题基   本一样 .

\begin{enumerate}
  \setcounter{enumi}{9}
  \item (1)  若函数项级数  $\sum_{n=1}^{\infty} n x \mathrm{e}^{-n x}$  在区间  $(0,+\infty)$  上一致收敛 ,  则  $\forall \varepsilon>0, \exists N>0$,  当  $n>m>N$  时 ,
\end{enumerate}
$$
\left|\sum_{k=n}^{m} k x \mathrm{e}^{-k x}\right|<\varepsilon, \quad \forall x \in(0,+\infty) .
$$
 特别地取  $m=n+1$,  则上式变为  $\left|n x \mathrm{e}^{-n x}\right|<\varepsilon$,  也即是说  $n x \mathrm{e}^{-n x}$  在  $(0,+\infty)$  上一致收敛于  0 .

 不过对于  $\varepsilon_{0}=\mathrm{e}^{-1}, \forall N>0$,  令  $n=N+1, m=N+2$,  取  $x=1 / n$,  则  $\left|n x \mathrm{e}^{-n x}\right|=\mathrm{e}^{-1} \geqslant \varepsilon_{0}$,  予盾 .

(2)  我们需要证明 
$$
\sum_{n=1}^{\infty}\left(n \mathrm{e}^{-n x}-n^{2} x \mathrm{e}^{-n x}\right)=\left(\sum_{n=1}^{\infty} n x \mathrm{e}^{-n x}\right)^{\prime}
$$
 注意到当  $x \in(-1,1)$  时有如下恒等式 
$$
\sum_{n=1}^{\infty} x^{n}=\frac{x}{1-x} \sum_{n=1}^{\infty} n x^{n-1}=\frac{1}{(1-x)^{2}} \quad \sum_{n=1}^{\infty} n x^{n}=\frac{x}{(1-x)^{2}} \quad \sum_{n=1}^{\infty} n^{2} x^{n-1}=\frac{1+x}{(1-x)^{3}} .
$$
 因此要证明  (1),  只需验证如下等式是否成立 
$$
\frac{\mathrm{e}^{-x}}{\left(1-\mathrm{e}^{-x}\right)^{2}}-x \frac{\mathrm{e}^{-x}\left(1+\mathrm{e}^{-x}\right)}{\left(1-\mathrm{e}^{-x}\right)^{3}}=\left(x \frac{\mathrm{e}^{-x}}{\left(1-\mathrm{e}^{-x}\right)^{2}}\right)^{\prime}
$$
 而直接求导就发现上面等式确实成立 .

 注   裴礼文的 《 数学分析中的典型问题与方法 》 第二版第  520  页例  $5.2 .45$  与本题类似 ,  那里有另外一种形式的   解法 ,  简单地说是利用函数在一点可微是一个局部的性质 ,  找一个内闭区间用逐项微分定理 .

$10 .$
$$
\begin{gathered}
g(x)=\int_{0}^{x} y f(x-y) \mathrm{d} y=-\int_{x}^{0}(x-t) f(t) \mathrm{d} t=\int_{0}^{x}(x-t) f(t) \mathrm{d} t=x \int_{0}^{x} f(t) \mathrm{d} t-\int_{0}^{x} t f(t) \mathrm{d} t \\
g^{\prime}(x)=\int_{0}^{x} f(t) \mathrm{d} t+x f(x)-x f(x)=\int_{0}^{x} f(t) \mathrm{d} t \\
g^{\prime \prime}(x)=f(x) .
\end{gathered}
$$
 北京大学  2005  年全国硕士研究生招生考试数学分析试题及解答 

   

2019.05.26

\begin{enumerate}
  \item  设  $f(x)=\frac{x^{2} \sin x-1}{x^{2}-\sin x} \sin x$,  试求  $\limsup _{x \rightarrow+\infty} f(x)$  和  $\liminf _{x \rightarrow+\infty} f(x)$.

  \item  证明下列各题 :

\end{enumerate}
(1)  设  $f(x)$  在开区间  $(a, b)$  上可微 ,  且  $f^{\prime}(x)$  在  $(a, b)$  上有界 ,  证明  $f^{\prime}(x)$  在  $(a, b)$  上一致连续 .

(2)  设  $f(x)$  在开区间  $(a, b)(-\infty<a<b<+\infty)$  上可微且一致连续 ,  试问  $f^{\prime}(x)$  在  $(a, b)$  上是否一定有   界 . ( 若肯定回答 ,  请证明 ;  若否定回答 ,  举例说明 )

\begin{enumerate}
  \setcounter{enumi}{3}
  \item  设  $f(x)=\sin ^{2}\left(x^{2}+1\right)$,
\end{enumerate}
(1)  求  $f(x)$  的麦克劳林展开式 .

(2)  求  $f^{(n)}(0), n=1,2,3, \cdots$.

\begin{enumerate}
  \setcounter{enumi}{4}
  \item  试作出定义在  $\mathbb{R}^{2}$  中的一个函数  $f(x, y)$,  使得它在原点处同时满足以下三个条件 :
\end{enumerate}
(1) $f(x, y)$  的两个偏导数都存在 ;

(2)  任何方向极限都存在 ;

(3)  在原点处不连续 .

\begin{enumerate}
  \setcounter{enumi}{5}
  \item  计算  $\int_{L} x^{2} \mathrm{~d} s$,  其中  $L$  是球面  $x^{2}+y^{2}+z^{2}=1$  与平面  $x+y+z=0$  的交线 .

  \item  设函数列  $\left\{f_{n}(x)\right\}$  满足下列条件 :

\end{enumerate}
(1)  对  $\forall n, f_{n}(x)$  在区间  $[a, b]$  上连续且有  $f_{n}(x) \leqslant f_{n+1}(x), x \in[a, b]$;

(2) $\left\{f_{n}(x)\right\}$  点点收敛于  $[a, b]$  上的连续函数  $s(x)$.

 证明 : $\left\{f_{n}(x)\right\}$  在  $[a, b]$  上一致收敛于  $s(x)$. 1.  当  $x \neq 0$  时 ,
$$
f(x)=\frac{\sin ^{2} x-\frac{\sin x}{x^{2}}}{1-\frac{\sin x}{x^{2}}}
$$
 从而 
$$
\limsup _{x \rightarrow+\infty} f(x)=1, \quad \liminf _{x \rightarrow+\infty} f(x)=0
$$

\begin{enumerate}
  \setcounter{enumi}{2}
  \item (1)  因为  $f^{\prime}(x)$  在  $(a, b)$  上有界 ,  可设  $\left|f^{\prime}(x)\right| \leqslant L . \forall x, y \in(a, b)$,
\end{enumerate}
$$
|f(x)-f(y)|=\left|f^{\prime}(\xi)(x-y)\right| \leqslant L|x-y|
$$
 故  $f(x)$  在  $(a, b)$  上满足  Lipschitz  条件 ,  从而  $f(x)$  在  $(a, b)$  上一致连续 .

(2)  要使一个函数的导函数无界 ,  函数应该震荡得非常厉害 , 一个自然想到的例子是 :
$$
f(x)=x \sin \frac{1}{x}, x \in(0,1) .
$$

\begin{enumerate}
  \setcounter{enumi}{3}
  \item (1)
\end{enumerate}
$$
\begin{aligned}
f(x) &=\frac{1-\cos \left(2 x^{2}+2\right)}{2}=\frac{1}{2}-\frac{1}{2}\left(\cos 2 x^{2} \cos 2-\sin 2 x^{2} \sin 2\right) \\
&=\frac{1}{2}-\frac{\cos 2}{2} \cos 2 x^{2}+\frac{\sin 2}{2} \sin 2 x^{2} \\
&=\frac{1}{2}-\frac{\cos 2}{2} \sum_{k=0}^{\infty} \frac{(-1)^{k}}{(2 k) !}\left(2 x^{2}\right)^{2 k}+\frac{\sin 2}{2} \sum_{k=0}^{\infty} \frac{(-1)^{k}}{(2 k+1) !}\left(2 x^{2}\right)^{2 k+1} \\
&=\frac{1}{2}-\frac{\cos 2}{2} \sum_{k=0}^{\infty} \frac{(-1)^{k} 2^{2 k}}{(2 k) !} x^{4 k}+\frac{\sin 2}{2} \sum_{k=0}^{\infty} \frac{(-1)^{k} 2^{2 k+1}}{(2 k+1) !} x^{4 k+2}
\end{aligned}
$$
(2)
$$
f^{(n)}(0)=\left\{\begin{array}{ll}
0, & n=2 k+1, k \in \mathbb{N} \\
-\frac{\cos 2}{2} \frac{(-1)^{k} 2^{2 k}}{(2 k) !}(4 k) !, & n=4 k, k \in \mathbb{N} \\
\frac{\sin 2}{2} \frac{(-1)^{k} 2^{2 k+1}}{(2 k+1) !}(4 k+1) !, & n=4 k+1, k \in \mathbb{N}
\end{array} .\right.
$$

\begin{enumerate}
  \setcounter{enumi}{4}
  \item  考虑 
\end{enumerate}
$$
f(x, y)= \begin{cases}\frac{x y}{x^{2}+y^{2}}, & x^{2}+y^{2} \neq 0 \\ 0, & x^{2}+y^{2}=0\end{cases}
$$

\begin{enumerate}
  \setcounter{enumi}{5}
  \item  由对称性 
\end{enumerate}
$$
\int_{L} x^{2} \mathrm{~d} s=\int_{L} y^{2} \mathrm{~d} s=\int_{L} z^{2} \mathrm{~d} s
$$
 故 
$$
\int_{L} x^{2} \mathrm{~d} s=\frac{1}{3} \int_{L} x^{2}+y^{2}+z^{2} \mathrm{~d} s=\frac{1}{3} \int_{L} \mathrm{~d} s=\frac{2 \pi}{3}
$$

\begin{enumerate}
  \setcounter{enumi}{6}
  \item Dini  定理 ,  在常见的数学分析教科书上都能找到证明方法 ,  推荐翻阅  Walter Rudin  牛上的  Theorem $7.13$.  北京大学  2006  年全国硕士研究生招生考试数学分析试题及解答 
\end{enumerate}
   

2019.05.15

\begin{enumerate}
  \item (15  分 )  确界原理是关于实数域完备性的一种描述 ,  试给出一个描述实数域完备性的其他定理 ,  并证明其与   确界原理的等价性 .

  \item (15  分 )  设函数  $f(x, y)=x^{3}+3 x y-y^{2}-6 x+2 y+1$,  求  $f(x, y)$  在  $(-2,2)$  处带二阶  Peano  余项的  Taylor  展式 .  问  $f(x, y)$  在  $\mathbb{R}^{2}$  上有哪些关于极值的判别点 ,  这些判别点是否为极值点 ,  说明理由 .

  \item $\left(15\right.$  分 )  设  $F(x, y)=x^{2} y^{3}+|x| y+y-5 .$

\end{enumerate}
(1)  证明方程  $F(x, y)=0$  在  $(-\infty,+\infty)$  上确定唯一的隐函数  $y=f(x)$;

(2)  求  $f(x)$  的极值点 .

\begin{enumerate}
  \setcounter{enumi}{4}
  \item (15  分 )  计算第二型曲面积分  $I=\iint_{\Sigma} x^{2} \mathrm{~d} y \mathrm{~d} z+y^{2} \mathrm{~d} z \mathrm{~d} x+z^{2} \mathrm{~d} x \mathrm{~d} y$,  其中曲面  $\Sigma$  为椭球面  $\frac{x^{2}}{a^{2}}+\frac{y^{2}}{b^{2}}+\frac{z^{2}}{c^{2}}=1$,  方向取外侧 .

  \item (15  分 )  证明 :  广义积分  $\int_{0}^{+\infty} \frac{\sin x}{x} \mathrm{~d} x$  收敛 ,  并计算此积分 .

  \item (15  分 )  设  $f(x, y)$  定义在  $D=(a, b) \times[c, d]$  上 , $x$  固定时 ,  对  $y$  连续 .  设  $x_{0} \in(a, b)$  取定 ,  对于任意  $y \in[c, d]$,  极限  $\lim _{x \rightarrow x_{0}} f(x, y)=g(y)$  收敛 .  证明 :  重极限  $\lim _{\substack{x \rightarrow x_{0} \\ y \rightarrow y_{0}}} f(x, y)=g\left(y_{0}\right)$  对任意  $y_{0} \in[c, d]$  成立的充分必要条件   是 :  极限  $\lim _{x \rightarrow x_{0}} f(x, y)=g(y)$  在  $[c, d]$  上一致收敛 .

  \item (15  分 )  设  $f(x)$  在  $[a, b]$  上有界 ,  给出并证明  $f(x)$  在  $[a, b]$  上的  Riemann  和的极限  $\lim _{\lambda(\Delta) \rightarrow 0} \sum_{i=1}^{n} f\left(\xi_{i}\right)\left(x_{i}-x_{i-1}\right)$  收敛的  Cauchy  准则 .

  \item (15  分 )  设  $\left\{f_{n}(x)\right\}$  是  $(-\infty,+\infty)$  上的一连续函数列 ,  并且一致有界  ( 即存在常数  $M$,  使得对于任意的  $f_{n}(x)$  和  $x \in(-\infty,+\infty)$  恒有  $\left.\left|f_{n}(x)\right| \leqslant M\right)$.  假定对  $(-\infty,+\infty)$  中的任意区间  $[a, b]$  都有  $\lim _{n \rightarrow \infty} \int_{a}^{b} f_{n}(x) \mathrm{d} x=0$.  证明 :  对于任意区间  $[c, d] \subset(-\infty,+\infty)$  以及  $[c, d]$  上绝对可积函数  $h(x)$,  恒有  $\lim _{n \rightarrow \infty} \int_{a}^{b} f_{n}(x) h(x) \mathrm{d} x=0$.

\end{enumerate}
$$
\frac{a_{0}}{2}+\sum_{n=1}^{\infty} a_{n} \cos n x+b_{n} \sin n x, \quad \frac{\alpha_{0}}{2}+\sum_{n=1}^{\infty} \alpha_{n} \cos n x+\beta_{n} \sin n x
$$
 都在  $[a, b]$  上收敛 ,  并且其和函数在  $[a, b]$  上连续且相等 .  试问 :  对于任意自然数  $n, a_{n}=\alpha_{n}, b_{n}=\beta_{n}$  是否 

\begin{enumerate}
  \setcounter{enumi}{10}
  \item (15  分 )  设  $f(x)$  在  $[0,+\infty)$  上内闭  Riemann  可积 .  证明 :  广义积分  $\int_{0}^{+\infty} f(x) \mathrm{d} x$  绝对可积的充分必要条  1.  实数列的单调有界收敛定理 :  若  $\left\{x_{n}\right\}$  是单调实数列且  $\exists M>0,\left|x_{n}\right| \leqslant M$,  则极限  $\lim _{n \rightarrow \infty} x_{n}$  存在 .
\end{enumerate}
 确界存在原理证明单调收敛定理   不妨假设  $\left\{x_{n}\right\}$  是单调递增的 ,  因为  $M$  是集合  $\left\{x_{n} \mid n \in \mathbb{N}\right\}$  的一个上界 ,  令  $\alpha=\sup \left\{x_{n} \mid n \in \mathbb{N}\right\}$,  则  $\alpha \in \mathbb{R}$,  并且  $x_{n} \leqslant \alpha, n \in \mathbb{N}$. $\forall \varepsilon>0, \exists N \in \mathbb{N}$,  使得  $\alpha-\varepsilon<x_{N} \leqslant \alpha$,  于是   当  $n>N$  时 ,
$$
\alpha-\varepsilon<x_{N} \leqslant x_{n} \leqslant \alpha<\alpha+\varepsilon
$$
 故  $\lim _{n \rightarrow \infty} x_{n}=\alpha$.

 单调收敛定理证明确界存在原理   只考虑上确界的情形 ,  设  $S \subset \mathbb{R}$,  并且  $S$  有上界 ,  要证明  $\sup S$  存在 .

 若在  $S$  中存在一个元素为  $S$  的上界 ,  则该元素就是上确界 .

 若  $S$  中的元素均不是  $S$  的上界 ,  取  $a \in S$  以及  $S$  的一个上界  $b$.  若  $\frac{a+b}{2} \in S$,  则令  $a_{1}=\frac{a+b}{2}, b_{1}=b$;  若  $\frac{a+b}{2} \notin S$,  则令  $a_{1}=a, b_{1}=\frac{a+b}{2}$.  无论哪种情况均有  $a_{1} \in S$  且  $b_{1}$  是  $S$  的上界 .  按上面的方法一直   二分区间 ,  可以得到数列  $\left\{a_{n}\right\},\left\{b_{n}\right\}, a_{n} \in S$  且  $b_{n}$  是  $S$  的上界 .  由于  $\left\{a_{n}\right\}$  是单调递增有界的 ,  可设  $\alpha=\lim _{n \rightarrow \infty} a_{n}$,  再结合  $b_{n}-a_{n}=(b-a) / 2^{n}$  可知  $\lim _{n \rightarrow \infty} b_{n}=\lim _{n \rightarrow \infty} a_{n}=\alpha$.  下面来说明  $\sup S=\alpha$.

 对于任意取定的  $x \in S, x \leqslant b_{n}$,  令  $n \rightarrow \infty$  得  $x \leqslant \alpha$,  由  $x$  的任意性 ,  这就说明了  $\alpha$  为  $S$  的一个上界 .  最后只需说明  $\alpha$  为  $S$  的最小上界 . $\forall \varepsilon>0, \alpha-\varepsilon<\alpha=\lim _{n \rightarrow \infty} a_{n}$,  于是存在  $N \in \mathbb{N}$,  使得  $\alpha-\varepsilon<a_{N}$,  这就说明  $\alpha-\varepsilon$  不是  $S$  的上界 ,  根据  $\varepsilon$  的任意性知  $\alpha$  为  $S \rightarrow$  的最小上界 .

\begin{enumerate}
  \setcounter{enumi}{2}
  \item $f_{x}^{\prime}(x, y)=3 x^{2}+3 y-6, f_{y}^{\prime}(x, y)=3 x-2 y+2, f_{x x}^{\prime \prime}(x, y)=6 x, f_{x y}^{\prime \prime}(x, y)=3$, $f_{y y}^{\prime \prime}(x, y)=-2$,  将  $(x, y)=(-2,2)$  带入计算套用公式可得 
\end{enumerate}
$f(x, y)=-7+12(x+2)-8(y-2)+\frac{1}{2}\left[-12(x+2)^{2}+6(x+2)(y-2)-2(y-2)^{2}\right]+o\left((x+2)^{2}+(y-2)^{2}\right) .$

H
$$
\left\{\begin{array} { l } 
{ f _ { x } ^ { \prime } ( x , y ) = 0 } \\
{ f _ { y } ^ { \prime } ( x , y ) = 0 }
\end{array} \Longrightarrow \left\{\begin{array} { l } 
{ x = \frac { 1 } { 2 } } \\
{ y = \frac { 7 } { 4 } }
\end{array} \text { 或 } \left\{\begin{array}{l}
x=-2 \\
y=-2
\end{array}\right.\right.\right. \text {. }
$$

\begin{enumerate}
  \setcounter{enumi}{3}
  \item (1)  我们要证明 :  对于任意取定的  $x \in \mathbb{R}$,  关于  $y$  的方程  $F(x, y)=0$  都有唯一解  $y=f(x)$.  当  $x=0$  时 ,  容 
\end{enumerate}
$$
\frac{\partial F}{\partial y}(x, y)=3 x^{2} y^{2}+|x|+1>0
$$
 于是只有一个实根 .  综合上面两种情况就有 :  方程  $F(x, y)=0$  在  $(-\infty,+\infty)$  上确定唯一的隐函数  $y=f(x) .$

(2)
$$
F(x, y)=\left\{\begin{aligned}
y^{3} x^{2}+x y+y-5, & x>0 \\
y-5, & x=0 \\
y^{3} x^{2}-x y+y-5, & x<0
\end{aligned}\right.
$$
 当  $x>0$  时 ,  对  $F(x, f(x))=0$  求导得 
$$
f^{\prime}(x)=-\frac{F_{x}^{\prime}(x, f(x))}{F_{y}^{\prime}(x, f(x))}=-\frac{f(x)\left(2 x f(x)^{2}+1\right)}{3 x^{2} f(x)^{2}+x+1}<0
$$
 从而  $f(x)$  在  $(0,+\infty)$  上单调递减 .

 当  $x<0$  时 ,  对  $F(x, f(x))=0$  求导得 
$$
f^{\prime}(x)=-\frac{F_{x}^{\prime}(x, f(x))}{F_{y}^{\prime}(x, f(x))}=-\frac{f(x)\left(2 x f(x)^{2}-1\right)}{3 x^{2} f(x)^{2}-x+1}>0
$$
 从而  $f(x)$  在  $(-\infty, 0)$  上单调递增 .

 因此  $x=0$  为  $f(x)$  的唯一极大值点 ,  无极小值点 .

\begin{enumerate}
  \setcounter{enumi}{4}
  \item  先用  Gauss  公式 ,  再利用对称性得 
\end{enumerate}
$$
\text { 原式 }=2 \iiint_{\Sigma}(x+y+z) \mathrm{d} x \mathrm{~d} y \mathrm{~d} z=0 .
$$

\begin{enumerate}
  \setcounter{enumi}{5}
  \item  由  Dirichlet  判别法知  $\int_{0}^{+\infty} \frac{\sin x}{x} \mathrm{~d} x$  收敛 .  因为 
\end{enumerate}
$$
\frac{1}{2}+\cos x+\cos 2 x+\cdots+\cos n x=\frac{\sin \left(n+\frac{1}{2}\right) x}{2 \sin \frac{x}{2}}
$$
 两边同时对  $x$  在  $[0, \pi]$  积分得 
$$
\frac{\pi}{2}=\int_{0}^{\pi} \frac{\sin \left(n+\frac{1}{2}\right) x}{2 \sin \frac{x}{2}} \mathrm{~d} x=\int_{0}^{\pi} \frac{\sin \left(n+\frac{1}{2}\right) x}{x} \mathrm{~d} x+\int_{0}^{\pi} \frac{x-2 \sin \frac{x}{2}}{2 x \sin \frac{x}{2}} \sin \left(n+\frac{1}{2}\right) x \mathrm{~d} x .
$$
 因此 
$$
\lim _{n \rightarrow \infty} \int_{0}^{\pi} \frac{\sin \left(n+\frac{1}{2}\right) x}{x} \mathrm{~d} x=\lim _{n \rightarrow \infty} \int_{0}^{\left(n+\frac{1}{2}\right) \pi} \frac{\sin x}{x} \mathrm{~d} x=\int_{0}^{+\infty} \frac{\sin x}{x} \mathrm{~d} x .
$$
 再由  Riemann-Lebesgue  引理知 
$$
\lim _{n \rightarrow \infty} \int_{0}^{\pi} \frac{x-2 \sin \frac{x}{2}}{2 x \sin \frac{x}{2}} \sin \left(n+\frac{1}{2}\right) x \mathrm{~d} x=0 .
$$
 因此 
$$
\int_{0}^{+\infty} \frac{\sin x}{x} \mathrm{~d} x=\frac{\pi}{2}
$$

\begin{enumerate}
  \setcounter{enumi}{6}
  \item  充分性 :  因为极限  $\lim _{x \rightarrow x_{0}} f(x, y)=g(y)$  在  $[c, d]$  上一致收敛 ,  故对于任意给定的  $\varepsilon>0, \exists \delta_{1}>0$,  当  $0<$ $\left|x-x_{0}\right|<\delta_{1}$  时 ,
\end{enumerate}
$$
|f(x, y)-g(y)|<\frac{\varepsilon}{6}, \quad y \in[c, d] .
$$
 取定一个满足  $0<\left|x_{1}-x_{0}\right|<\delta_{1}$  的  $x_{1}$,  对于任意取定的  $y_{0} \in[c, d]$,  由于  $f\left(x_{1}, y\right)$  关于  $y$  在  $[c, d]$  上连续 ,  故对于上面的  $\varepsilon>0$,  存在  $\delta_{2}$,  当  $0<\left|y-y_{0}\right|<\delta_{2}$  时 ,
$$
\left|f\left(x_{1}, y\right)-f\left(x_{1}, y_{0}\right)\right|<\frac{\varepsilon}{6},
$$
 于是 
$$
\left|g(y)-g\left(y_{0}\right)\right| \leqslant\left|g(y)-f\left(x_{1}, y\right)\right|+\left|f\left(x_{1}, y\right)-f\left(x_{1}, y_{0}\right)\right|+\left|f\left(x_{1}, y_{0}\right)-g\left(y_{0}\right)\right|<\frac{\varepsilon}{6}+\frac{\varepsilon}{6}+\frac{\varepsilon}{6}=\frac{\varepsilon}{2} .
$$
 令  $\delta=\min \left\{\delta_{1}, \delta_{2}\right\}$,  则当  $0<\left|x-x_{0}\right|<\delta, 0<\left|y-y_{0}\right|<\delta$  时 
$$
\left|f(x, y)-g\left(y_{0}\right)\right| \leqslant|f(x, y)-g(y)|+\left|g(y)-g\left(y_{0}\right)\right|<\frac{\varepsilon}{6}+\frac{\varepsilon}{2}<\varepsilon .
$$
 因此重极限  $\lim _{\substack{x \rightarrow x_{0} \\ y \rightarrow y_{0}}} f(x, y)=g\left(y_{0}\right)$  对任意  $y_{0} \in[c, d]$  成立 .

 必要性 :  对任意  $y_{0} \in[c, d], \forall \varepsilon>0, \exists \delta_{y_{0}}>0$,  当  $0<\left|x-x_{0}\right|<\delta_{y_{0}},\left|y-y_{0}\right|<\delta_{y_{0}}$  时 ,
$$
\left|f(x, y)-g\left(y_{0}\right)\right|<\frac{\varepsilon}{2},
$$
 令  $x \rightarrow x_{0}$  得 
$$
\left|g(y)-g\left(y_{0}\right)\right| \leqslant \frac{\varepsilon}{2},
$$
 于是对于任意  $y \in O\left(y_{0}, \delta_{y_{0}}\right)$,  当  $0<\left|x-x_{0}\right|<\delta_{y_{0}}$  时 ,
$$
|f(x, y)-g(y)| \leqslant\left|f(x, y)-g\left(y_{0}\right)\right|+\left|g\left(y_{0}\right)-g(y)\right|<\varepsilon
$$
 因为 
$$
\bigcup_{y_{0} \in[c, d]} O\left(y_{0}, \delta_{y_{0}}\right) \supset[c, d]
$$
 从而存在有限开覆盖 ,  设为  $O\left(y_{i}, \delta_{i}\right), i=1,2, \ldots, m$.  令  $\delta=\min \left\{\delta_{i} \mid 1 \leqslant i \leqslant m\right\}$,  那么对于任意  $y \in[c, d]$,  当  $0<\left|x-x_{0}\right|<\delta$  时 ,
$$
|f(x, y)-g(y)|<\varepsilon .
$$
 从而极限  $\lim _{x \rightarrow x_{0}} f(x, y)=g(y)$  在  $[c, d]$  上一致收敛 .

 注   这个证明整合了我自己的想法和一位网友的解答 .

\begin{enumerate}
  \setcounter{enumi}{7}
  \item  关于  Riemann  和的  Cauchy  收敛准则 :  有界函数  $f(x)$  在  $[a, b]$  上的  Riemann  和收敛的充分必要条件是  $\forall \varepsilon>0, \exists \delta>0$,  使得对于  $[a, b]$  的任意分割  $P_{1}: a=x_{0}<x_{1}<\cdots<x_{n}=b$  和  $P_{2}: a=y_{0}<y_{1}<\cdots<$ $y_{m}=b$,  以及任意介点  $\xi_{i} \in\left[x_{i-1}, x_{i}\right], \eta_{j} \in\left[y_{j-1}, y_{j}\right]$,  只要其相应的最大模  $\left\|P_{1}\right\|<\delta,\left\|P_{2}\right\|<\delta$,  就有 
\end{enumerate}
$$
\left|\sum_{i=1}^{n} f\left(\xi_{i}\right)\left(x_{i}-x_{i-1}\right)-\sum_{j=1}^{m} f\left(\eta_{j}\right)\left(y_{j}-y_{j-1}\right)\right|<\varepsilon
$$
 若  Riemann  和的极限收敛 ,  用三角不等式就能证明  Riemann  和满足  Cauchy  准则 .

 若  Riemann  和满足  Cauchy  准则 ,  下面来证  Riemann  和的极限收敛 ,  也即是要证明  $f(x)$  在  $[a, b]$  上可积 . $\forall \varepsilon>0, \exists \delta>0$,  使得对于  $[a, b]$  的任意分割  $P: a=x_{0}<x_{1}<\cdots<x_{n}=b$,  当  $0<\|P\|<\delta$  时 ,  对于分割  $P$  的任意两个介点集  $\left\{\xi_{i}\right\},\left\{\eta_{i}\right\}$,  均有 
$$
\left|\sum_{i=1}^{n} f\left(\xi_{i}\right)\left(x_{i}-x_{i-1}\right)-\sum_{i=1}^{n} f\left(\eta_{i}\right)\left(y_{i}-y_{i-1}\right)\right|<\frac{\varepsilon}{2}
$$
 取上下确界得到 
$$
\left|\sum_{i=1}^{n}\left(M_{i}-m_{i}\right) \Delta x_{i}\right| \leqslant \frac{\varepsilon}{2}<\varepsilon
$$
 故  $f(x)$  在  $[a, b]$  上可积 .

\begin{enumerate}
  \setcounter{enumi}{8}
  \item  因为  $h(x)$  为  $[a, b]$  上绝对可积函数 , $\forall \varepsilon>0$,  存在分割  $P: a=x_{0}<x_{1}<\cdots<x_{N}=b$,  使得 
\end{enumerate}
$$
\left|\sum_{i=1}^{N}\left(M_{i}-m_{i}\right) \Delta x_{i}\right|<\frac{\varepsilon}{2 M},
$$
 其中  $M_{i}=\sup \left\{h(x) \mid x \in\left[x_{i-1}, x_{i}\right]\right\}, m_{i}=\inf \left\{h(x) \mid x \in\left[x_{i-1}, x_{i}\right]\right\}, \Delta x_{i}=x_{i}-x_{i-1}, i=1,2, \ldots, N$.  于   是 
$$
\left|\int_{a}^{b} h(x) \mathrm{d} x-\sum_{i=1}^{N} m_{i} \Delta x_{i}\right|=\left|\sum_{i=1}^{N} \int_{x_{i-1}}^{x_{i}}\left(h(x)-m_{i}\right) \mathrm{d} x\right| \leqslant\left|\sum_{i=1}^{N}\left(M_{i}-m_{i}\right) \Delta x_{i}\right|<\frac{\varepsilon}{2 M}
$$
 取定上面的分割 ,  这时就有 
$$
\begin{aligned}
\left|\int_{a}^{b} f_{n}(x) h(x) \mathrm{d} x\right| &=\left|\sum_{i=1}^{N} \int_{x_{i-1}}^{x_{i}} f_{n}(x) h(x) \mathrm{d} x\right| \\
&=\left|\sum_{i=1}^{N} \int_{x_{i-1}}^{x_{i}} f_{n}(x)\left(h(x)-m_{i}\right) \mathrm{d} x+\sum_{i=1}^{N} \int_{x_{i-1}}^{x_{i}} f_{n}(x) m_{i} \mathrm{~d} x\right| \\
& \leqslant \frac{M \varepsilon}{2 M}+\left|\sum_{i=1}^{N} \int_{x_{i-1}}^{x_{i}} f_{n}(x) m_{i} \mathrm{~d} x\right|
\end{aligned}
$$
 又因为  $\lim _{n \rightarrow \infty} \int_{a}^{b} f_{n}(x) \mathrm{d} x=0$  对于任意区间  $[a, b]$  均成立 ,  故  $\exists N_{0}>0$,  当  $n>N_{0}$  时 ,
$$
\left|\sum_{i=1}^{N} \int_{x_{i-1}}^{x_{i}} f_{n}(x) m_{i} \mathrm{~d} x\right|<\frac{\varepsilon}{2},
$$
 从而 
$$
\left|\int_{a}^{b} f_{n}(x) h(x) \mathrm{d} x\right|<\varepsilon .
$$
 此即说明  $\lim _{n \rightarrow \infty} \int_{a}^{b} f_{n}(x) h(x) \mathrm{d} x=0$.

 注   之所以想到这么证明是因为  $f_{n}(x)$  的一种可能的情形是  $f_{n}(x)=\sin n x$,  这个时候原题其实就是说的  RiemannLebesgue  引理 ,  证明这个引理是要用阶梯函数来逼近 ,  然后把这个方法用到原题也刚好能做出来 .

\begin{enumerate}
  \setcounter{enumi}{9}
  \item  原题中说的结论不成立 ,  比如考虑在  $[-\pi, \pi]$  上将  $f(x)=x, g(x)=|x|$  展开为  Fourier  级数 ,  则这两个级数   在  $[0, \pi / 2]$  上均逐点收敛于  $x$,  但是这两个  Fourier  级数的系数却是完全不同的 .
\end{enumerate}
 加上条件  $b-a \geqslant 2 \pi$,  并且这两个级数都在  $[a, b]$  上一致收敛就能保证结论成立 .  事实上 ,  设和函数分别为  $s_{1}(x), s_{2}(x)$,  则这两个函数均是以  $2 \pi$  为周期的函数 ,  因为它们在一个长度大于等于  $2 \pi$  的区间上相等 ,  故  $s_{1}(x)=s_{2}(x)$.  因为是在长度大于  $2 \pi$  的一个区间上一致收敛 ,  可以逐项积分 ,  就得到和函数的  Fourier  级数   就是原先收敛到它的那个  Fourier  级数 .

 注   北京大学  2007  年数学分析最后一题与此题相关 .

\begin{enumerate}
  \setcounter{enumi}{10}
  \item  必要性 : 对于任意满足  $x_{0}=0, x_{n} \rightarrow+\infty,(n \rightarrow \infty)$  的单调增数列  $\left\{x_{n}\right\}$,  由  $\int_{0}^{+\infty} f(x) \mathrm{d} x$  绝对可积知 : $\forall \varepsilon>0, \exists \Delta>0$,  当  $A, B>\Delta$  时 ,
\end{enumerate}
$$
\left|\int_{A}^{B}\right| f(x)|\mathrm{d} x|<\varepsilon .
$$
 因为  $\lim _{n \rightarrow \infty} x_{n}=+\infty$,  故存在  $N>0$,  当  $n>N$  时 , $x_{n}>\Delta$,  于是当  $n \geqslant m>N$  时 ,
$$
\sum_{k=m}^{n}\left|\int_{x_{k}}^{x_{k+1}} f(x) \mathrm{d} x\right| \leqslant \sum_{k=m}^{n} \int_{x_{k}}^{x_{k+1}}|f(x)| \mathrm{d} x=\int_{x_{m}}^{x_{n+1}}|f(x)| \mathrm{d} x<\varepsilon
$$
 由  Cauchy  收敛原理知级数  $\sum_{n=0}^{\infty} \int_{x_{n}}^{x_{n+1}} f(x) \mathrm{d} x$  绝对收敛 .  充分性 :  若  $\int_{0}^{+\infty} f(x) \mathrm{d} x$  不是绝对可积 ,  则  $\exists \varepsilon_{0}>0, \forall A>0, \exists A_{2}>A_{1}>A_{0}$,  使得  $\int_{A_{1}}^{A_{2}}|f(x)| \mathrm{d} x \geqslant \varepsilon_{0}$.  对于任意的  $[a, b] \subset[0,+\infty)$,  若  $\int_{a}^{b}|f(x)| \mathrm{d} x \geqslant \varepsilon_{0}$,  令 
$$
f_{+}(x)=\frac{|f(x)|+f(x)}{2}, \quad f_{-}(x)=\frac{|f(x)|-f(x)}{2}
$$
 则  $\int_{a}^{b} f_{+}(x) \mathrm{d} x \geqslant \varepsilon_{0} / 2$  或者  $\int_{a}^{b} f_{-}(x) \mathrm{d} x \geqslant \varepsilon_{0} / 2$  至少有一个成立 .  不妨设  $\int_{a}^{b} f_{+}(x) \mathrm{d} x \geqslant \varepsilon_{0} / 2$,  则存在  $[a, b]$  的分害  $P: a=x_{0}<x_{1}<\cdots<x_{n}=b$,  使得 
$$
\left|\sum_{i=1}^{n} m_{i} \Delta x_{i}\right|>\frac{\varepsilon_{0}}{4},
$$
 其中  $m_{i}=\inf \left\{f_{+}(x) \mid x \in\left[x_{i-1}, x_{i}\right]\right\}, \Delta x_{i}=x_{i}-x_{i-1}$.  把  $m_{i}>0$  对应的区间从左至右记为  $\left[x_{n_{k}-1}, x_{n_{k}}\right], k=$ $1,2, \ldots, s$.  则  $f(x)$  在这些区间上取值恒为正 ,  从而 
$$
\left|\int_{x_{n_{k}-1}}^{x_{n_{k}}} f(x) \mathrm{d} x\right|=\int_{x_{n_{k}-1}}^{x_{n_{k}}}|f(x)| \mathrm{d} x, \quad k=1,2, \ldots, s .
$$
 因此 
$$
\sum_{i=1}^{n}\left|\int_{x_{i-1}}^{x_{i}} f(x) \mathrm{d} x\right| \geqslant \sum_{k=1}^{s}\left|\int_{x_{n_{k}-1}}^{x_{n_{k}}} f(x) \mathrm{d} x\right|=\sum_{k=1}^{s} \int_{x_{n_{k}-1}}^{x_{n_{k}}}|f(x)| \mathrm{d} x \geqslant \sum_{k=1}^{s} m_{n_{k}} \Delta x_{n_{k}}=\sum_{i=1}^{n} m_{i} \Delta x_{i}>\frac{\varepsilon_{0}}{4} .
$$
 上面的结论说明 :  取定  $\varepsilon_{0}>0$,  对于任意的  $[a, b] \subset[0,+\infty)$,  若  $\int_{a}^{b}|f(x)| \mathrm{d} x \geqslant \varepsilon_{0}$,  则能找到  $[a, b]$  的某个分   割  $P$,  使得  $f(x)$  在分割得到的区间上的定积分的绝对值的和大于  $\varepsilon_{0} / 4$.

 当  $A=1$  时 ,  找到  $A_{2}>A_{1}>1$,  对  $\left[A_{1}, A_{2}\right]$  进行分害 ;

 当  $A=\max \left\{2, A_{2}\right\}$  时 ,  找到  $A_{4}>A_{3}>2$,  对  $\left[A_{3}, A_{4}\right]$  进行分割 ;

 最终得到一列  $A_{1}<A_{2}<A_{3}<A_{4}<\ldots$,  再把  $A_{0}=0$  和分割中的点加进去就得到首项为  0 ,  严格单调递   增的序列 ,  记为  $\left\{a_{n}\right\}$.  由前面的论述知 
$$
\sum_{i=0}^{\infty}\left|\int_{a_{i}}^{a_{i+1}} f(x) \mathrm{d} x\right| \geqslant \sum_{n=1}^{\infty} \frac{\varepsilon_{0}}{4}=+\infty
$$
 矛盾 .

 注   证明充分性的时候我的直觉是用反证法 ,  为了能让绝对值符号和积分号能互换位置需要  $f(x)$  不变号 ,  由此   想到将  $f(x)$  分解为  $f_{+}(x)+f_{-}(x)$,  这里的想法与解决北京大学  2008  年的数学分析最后一题是相通的 .  北京大学  2007  年全国硕士研究生招生考试数学分析试题及解答 

   

2019.05.12

\begin{enumerate}
  \item  用有限覆盖定理证明连续函数的介值性定理 .

  \item $f(x)$  和  $g(x)$  在有界区间上一致连续 ,  证明  $f(x) g(x)$  在此区间上也一致连续 .

  \item  已知  $f(x)$  在  $[a, b]$  上有  4  阶导数 ,  且有  $f^{(4)}(\beta) \neq 0, f^{\prime \prime \prime}(\beta)=0, \beta \in(a, b)$.  证明 :  存在  $x_{1}, x_{2} \in(a, b)$,  使得  $f\left(x_{1}\right)-f\left(x_{2}\right)=f^{\prime}(\beta)\left(x_{1}-x_{2}\right)$  成立 .

  \item  构造一函数在  $\mathbb{R}$  上无穷次可微 ,  且  $f^{(2 n+1)}(0)=n, f^{(2 n)}(0)=0,(n=0,1,2, \ldots)$,  并说明满足条件的函数   有任意多个 .

  \item  设  $D=[0,1] \times[0,1], f(x, y)$  是  $D$  上的连续函数 .  证明 :  满足  $\iint_{D} f(x, y) \mathrm{d} x \mathrm{~d} y=f(\xi, \eta)$  的点  $(\xi, \eta)$  有无   穷多个 .

  \item  求  $\iint_{\Sigma} \sin ^{4} x \mathrm{~d} y \mathrm{~d} z+\mathrm{e}^{-|y|} \mathrm{d} z \mathrm{~d} x+z^{2} \mathrm{~d} x \mathrm{~d} y$,  其中  $\Sigma$  是  $x^{2}+y^{2}+z^{2}=1, z>0$,  方向向上 .

  \item $f(x, y)$  是  $\mathbb{R}^{2}$  上的连续函数 ,  试作一无界区域  $D$,  使  $f(x)$  在  $D$  上的广义积分收玫 .

  \item  已知  $f(x)=\ln \left(1+\frac{\sin x}{x^{p}}\right)$,  讨论当  $p$  取不同值时 , $f(x)$  在  $[1,+\infty)$  上的广义积分的敛散性 .

  \item  已知  $F(x, y)=\sum_{n=1}^{+\infty} n y \mathrm{e}^{-n(x+y)}$,  是否存在  $a>0$  以及函数  $h(x), h(x)$  在  $(1-a, 1+a)$  上可导 ,  且  $h(1)=0$,  使得  $F(x, h(x))=0$.

  \item  设  $f(x)$  和  $g(x)$  在  $[a, b]$  上  Riemann  可积 .  证明 : $f(x)$  和  $g(x)$  的  Fourier  展开式有相同系数的充要条件是  $\int_{a}^{b}|f(x)-g(x)| \mathrm{d} x=0 .$ 1.  不妨设  $f(x)$  在  $[a, b]$  上连续 , $f(a)<f(b)$,  只需要证 : $\forall c \in(f(a), f(b)), \exists \xi \in(a, b)$,  使得  $f(\xi)=c$. $\forall x \in[a, b]$,  若  $f(x) \neq c$,  则由  $f(x)$  的连续性 ,  存在  $x$  的一个邻域  $\left(x-\delta_{x}, x+\delta_{x}\right),\left(\delta_{x}>0\right)$,  在该邻   域内  $f(x)<c$  或者  $f(x)>c$.  对于所有的  $x \in[a, b]$,  邻域  $\left(x-\delta_{x} / 2, x+\delta_{x} / 2\right)$  构成  $[a, b]$  的一个开覆   盖 ,  由有限覆盖定理知存在有限开覆盖 ,  设对应的点为  $x_{i}, i=1,2, \ldots, n$,  为了简化符号 ,  记对应的  $\delta_{x_{i}}$  为  $\delta_{i}, i=1,2, \ldots, n$.  令  $\delta=\min \left\{\delta_{1}, \delta_{2}, \ldots, \delta_{n}\right\}$,  下面说明当  $|x-y|<\delta / 2$  时 ,  将有  $x, y$  属于同一个邻域 .  事实   上 , $x$  必定属于某个邻域 ,  可设  $x \in\left(x_{i}-\delta_{i} / 2, x_{i}+\delta_{i} / 2\right)$,  则  $\left|y-x_{i}\right| \leqslant|y-x|+\left|x-x_{i}\right|<\delta / 2+\delta_{i} / 2 \leqslant \delta_{i}$,  从而  $y \in\left(x_{i}-\delta_{i}, x_{i}+\delta_{i}\right)$.

\end{enumerate}
 取  $a=y_{0}<y_{1}<\ldots,<y_{m}=b, y_{j}-y_{j-1}=(b-a) / m<\delta / 2, j=1,2, \ldots, m$.  则  $\left[y_{0}, y_{1}\right]$  在上面有限开   覆盖的某个邻域中 ,  从而在  $\left[y_{0}, y_{1}\right]$  上 ,  要么均是  $f(x)<c$,  要么均是  $f(x)>c ;$  同样在  $\left[y_{1}, y_{2}\right]$  上 ,  要么均是  $f(x)<c$,  要么均是  $f(x)>c$.  从而在  $\left[y_{0}, y_{2}\right]$  上 ,  要么均是  $f(x)<c$,  要么均是  $f(x)>c$.  从左至右一直做下   去 ,  则有  $f(x)$  在  $[a, b]$  上均是  $f(x)<c$,  或者均是  $f(x)>c$,  这与  $f(a)<c<f(b)$  矛盾 .

 注   此处的证明手法与我写的北京大学  2016  年数学分析第一题相同 .  其中找到的那个数  $\delta / 2$  叫做  Lebesgue  数 ,  与之相关的定理在谢惠民等人的  《 数学分析习题课讲义 》 上册  82  页例题  $3.5 .3$ ( 加强形式的覆盖定理 ),  书   上那个证明借助了几何直观 ,  但是我更喜欢上面这种证明方式 .

\begin{enumerate}
  \setcounter{enumi}{2}
  \item  设有界区间为  $I$.  若  $I$  不是闭区间 ,  由于  $f(x)$  在  $I$  上一致连续 ,  利用  Cauchy  收敛原理可知  $f(x)$  在  $I$  的   两个端点处的单侧极限均存在 ,  从而我们可以把  $f(x)$  连续延拓到  $I$  的闭包  $\bar{I}$  上 ,  对  $g(x)$  也是一样的 ,  这时  $f(x) g(x)$  在有界闭集  $\bar{I}$  上连续 ,  从而  $f(x) g(x)$  在有界闭集  $\bar{I}$  上一致连续 ,  故  $f(x) g(x)$  在  $I$  上一致连续 .  若  $I$  是闭区间 ,  则不延拓 ,  然后用前面一样的方法就能证明原命题 .
\end{enumerate}
 注   其实只是对一道经典的题目进行了一下灳装 ,  相关的题目见林源渠 、 方企勤编的 《 数学分析解题指南 》 第  44  页例  11 ,  裴礼文的  《 数学分析中的典型问题与方法 》 第二版第  151  页例  $2.2 .6$,  谢惠民等人的  《 数学   分析习题课讲义 》 上册第  140  页例题  5.4.5.  另外一种做法是先证明  $f(x)$  和  $g(x)$  有界 ,  然后用不等式 

\begin{enumerate}
  \setcounter{enumi}{3}
  \item  令  $g(x)=f(x)-f(\beta)-f^{\prime}(\beta)(x-\beta)$,  则要证明原命题 ,  只需证明存在  $x_{1}, x_{2} \in(a, b)$,  使得  $g\left(x_{1}\right)=g\left(x_{2}\right)$.
\end{enumerate}
$$
\lim _{x \rightarrow \beta} \frac{g(x)}{(x-\beta)^{2}}=\frac{f^{\prime \prime}(\beta)}{2} \neq 0,
$$
$$
\lim _{x \rightarrow \beta} \frac{g(x)}{(x-\beta)^{4}}=\frac{f^{(4)}(\beta)}{4 !} \neq 0,
$$
$$
f(x)=\sum_{n=0}^{\infty} \frac{n}{(2 n+1) !} x^{2 n+1}, x \in \mathbb{R}, \quad g(x)= \begin{cases}\mathrm{e}^{-\frac{1}{x^{2}}}, & x \neq 0 \\ 0, & x=0\end{cases}
$$

\begin{enumerate}
  \setcounter{enumi}{5}
  \item  设  $f(x, y)$  在  $D$  上的最大值为  $M$,  最小值为  $m$.  若  $M=m$,  则  $f(x, y)$  为  $D$  上的常数函数 ,  自然结论   成立 .  若  $M \neq m$,  则在  $D$  中 ,  取一条从使  $f(x, y)=m$  的一个点出发到使  $f(x, y)=M$  的一个点的连   续参数曲线  $\gamma(t)=(x(t), y(t)), t \in[0,1]$,  由  $f(x(t), y(t))$  在  $[0,1]$  上的连续性知存在  $t_{0} \in(0,1)$,  使得   注   参考了赵斌写的那份解答中的想法 .

  \item  设所求积分为  $I$,  则由对称性得  $I=\iint_{\Sigma} z^{2} \mathrm{~d} x \mathrm{~d} y$,  补上球  $x^{2}+y^{2}+z^{2} \leqslant 1$  与  $z=0$  相截所得的面 ,  再利用  Gauss  公式得 

\end{enumerate}
$$
\begin{aligned}
I &=\iint_{x^{2}+y^{2}+z^{2} \leqslant 1, z \geqslant 0} 2 z \mathrm{~d} x \mathrm{~d} y \mathrm{~d} z \\
&=2 \pi \int_{0}^{1} z\left(1-z^{2}\right) \mathrm{d} z \\
&=\frac{\pi}{2}
\end{aligned}
$$

\begin{enumerate}
  \setcounter{enumi}{7}
  \item  设  $|f(x, y)|$  在  $P_{n}=\{(x, y) \mid n-1 \leqslant x \leqslant n, 0 \leqslant y \leqslant 1\}$  上的最大值为  $M_{n}$,  取 
\end{enumerate}
$$
D_{n}=\left\{(x, y) \mid n-1 \leqslant x \leqslant n, 0 \leqslant y \leqslant \frac{1}{n^{2}\left(M_{n}+1\right)}\right\}, D=\bigcup_{n=1}^{\infty} D_{n}
$$
 则  $f(x, y)$  在  $D$  上的广义积分收敛 .

 注   参考了赵斌写的那份解答中的想法 .

\begin{enumerate}
  \setcounter{enumi}{8}
  \item  要使  $f(x)$  再  $[1,+\infty)$  上有意义 , $p$  需要满足  $p>0$.  当  $p>0$  时 ,
\end{enumerate}
$$
\begin{gathered}
f(x)=\frac{\sin x}{x^{p}}-\frac{1}{2} \frac{\sin ^{2} x}{x^{2 p}}+o\left(\frac{1}{x^{2 p}}\right)=\frac{\sin x}{x^{p}}+\frac{\cos 2 x}{4 x^{2 p}}-\frac{1}{4 x^{2 p}}+o\left(\frac{1}{x^{2 p}}\right),(x \rightarrow+\infty) \\
-\frac{1}{4 x^{2 p}}+o\left(\frac{1}{x^{2 p}}\right) \sim-\frac{1}{4 x^{2 p}},(x \rightarrow+\infty) .
\end{gathered}
$$
 令 
$$
g(x)=\frac{\sin x}{x^{p}}, \quad h(x)=\frac{\cos 2 x}{4 x^{2 p}}, \quad i(x)=f(x)-g(x)-h(x) .
$$
 当  $p>1$  时 , $g(x), h(x), i(x)$  在  $[1,+\infty)$  上的广义积分均绝对收敛 ,  从而  $f(x)$  在  $[1,+\infty)$  上的广义积分绝   对收敛 .

 当  $\frac{1}{2}<p \leqslant 1$  时 , $g(x)$  在  $[1,+\infty)$  上的广义积分条件收敛 , $h(x), i(x)$  在  $[1,+\infty)$  上的广义积分绝对收敛 ,  从而  $f(x)$  在  $[1,+\infty)$  上的广义积分收敛 ,  但不是绝对收敛 ,  否则就由  $|g(x)| \leqslant|f(x)|+|h(x)|+|i(x)|$  得到  $g(x)$  在  $[1,+\infty)$  上的广义积分绝对收敛 ,  矛盾 .  因此 , $f(x)$  在  $[1,+\infty)$  上的广义积分条件收敛 .

 当  $0<p \leqslant \frac{1}{2}$  时 , $g(x), h(x)$  在  $[1,+\infty)$  上的广义积分收敛 , $i(x)$  在  $[1,+\infty)$  上的广义积分发散 ,  从而  $f(x)$  在  $[1,+\infty)$  上的广义积分发散 .

 注   林源渠 、 方企勤编的 《 数学分析解题指南 》 第  217  页例  9  与此题相仿 .

\begin{enumerate}
  \setcounter{enumi}{9}
  \item $F(x, y)$  在  $D=\{(x, y) \mid x+y>0\}$  上有意义 .  当  $x+y>0$  时 ,
\end{enumerate}
$$
F(x, y)=\sum_{n=1}^{+\infty} n y \mathrm{e}^{-n(x+y)}, \quad \mathrm{e}^{-(x+y)} F(x, y)=\sum_{n=1}^{+\infty} n y \mathrm{e}^{-(n+1)(x+y)}
$$
 两式相减得 
$$
F(x, y)\left(1-\mathrm{e}^{-(x+y)}\right)=\sum_{n=1}^{+\infty} y \mathrm{e}^{-n(x+y)}=y \frac{\mathrm{e}^{-(x+y)}}{1-\mathrm{e}^{-(x+y)}}
$$
 故 
$$
F(x, y)=\frac{y \mathrm{e}^{-(x+y)}}{\left(1-\mathrm{e}^{-(x+y)}\right)^{2}}, \quad(x, y) \in D
$$
 因此  $F(x, y)$  在  $D$  上连续 ,  并且有连续偏导数 , $F(1,0)=0$.  由于 
$$
F_{y}(x, y)=\frac{\mathrm{e}^{-(x+y)}}{\left(1-\mathrm{e}^{-(x+y)}\right)^{2}}+y \frac{\partial}{\partial y}\left(\frac{\mathrm{e}^{-(x+y)}}{\left(1-\mathrm{e}^{-(x+y)}\right)^{2}}\right), \quad(x, y) \in D
$$
 姑 
$$
F_{y}(1,0)=\frac{\mathrm{e}^{-1}}{\left(1-\mathrm{e}^{-1}\right)^{2}} \neq 0
$$
 由隐函数定理知满足条件的  $a$  及  $h(x)$  是存在的 .

 注   此题中函数项级数的和函数能通过乘公比错位相减法求出来 ,  说理就相对简单一些 ,  若是没有办法求出和函   数 ,  就需要借助一致收敛去说明  $F(x, y)$  的连续可微性 .

\begin{enumerate}
  \setcounter{enumi}{10}
  \item  只证明  $a=-\pi, b=\pi$  的情形 .  设  $f(x)$  的  Fourier  展开式的系数为 
\end{enumerate}
$$
a_{n}=\frac{1}{\pi} \int_{-\pi}^{\pi} f(x) \cos n x \mathrm{~d} x, \quad b_{n}=\frac{1}{\pi} \int_{-\pi}^{\pi} f(x) \sin n x \mathrm{~d} x,
$$
$g(x)$  的  Fourier  展开式的系数为 
$$
a_{n}^{\prime}=\frac{1}{\pi} \int_{-\pi}^{\pi} g(x) \cos n x \mathrm{~d} x, \quad b_{n}^{\prime}=\frac{1}{\pi} \int_{-\pi}^{\pi} g(x) \sin n x \mathrm{~d} x .
$$
 充分性 :
$$
\left|a_{n}-a_{n}^{\prime}\right|=\left|\frac{1}{\pi} \int_{-\pi}^{\pi}(f(x)-g(x)) \cos n x \mathrm{~d} x\right| \leqslant \frac{1}{\pi} \int_{-\pi}^{\pi}|f(x)-g(x)| \mathrm{d} x=0
$$
 必要性 : $f(x)$  和  $g(x)$  的  Fourier  展开式有相同系数 ,  故  $f(x)-g(x)$  的  Fourier  展开式的系数为  0 .  又由于  $f(x)$  和  $g(x)$  在  $[-\pi, \pi]$  上  Riemann  可积 ,  从而  $f(x)-g(x)$  在  $[-\pi, \pi]$  上  Riemann  可积 .  根据  Parseval  等   式得 ,
$$
\frac{1}{\pi} \int_{-\pi}^{\pi}(f(x)-g(x))^{2} \mathrm{~d} x=0
$$
 由  Cauchy-Schwarz  不等式得 
$$
\int_{-\pi}^{\pi}|f(x)-g(x)| \mathrm{d} x \leqslant \sqrt{\int_{-\pi}^{\pi} \mathrm{d} x} \sqrt{\int_{-\pi}^{\pi}|f(x)-g(x)|^{2} \mathrm{~d} x}=0 .
$$
 北京大学  2008  年全国硕士研究生招生考试数学分析试题及解答 

   

2019.05.10

\begin{enumerate}
  \item  证明有界闭区间上的连续函数一致连续 .

  \item  是否存在  $(-\infty,+\infty)$  上的连续函数  $f(x)$,  满足  $f(f(x))=\mathrm{e}^{-x}$ ?  说明理由 

  \item  数列  $\left\{x_{n}\right\}(n \geqslant 1)$,  满足任意  $n<m$,  有  $\left|x_{n}-x_{m}\right|>\frac{1}{n}$,  求证  $\left\{x_{n}\right\}$  无界 .

  \item $f(x)$  是  $(-1,1)$  上的无穷次可微函数 , $f(0)=1,\left|f^{\prime}(0)\right| \leqslant 2$.  令  $g(x)=\frac{f^{\prime}(x)}{f(x)}$,  若  $\left|g^{(n)}(0)\right| \leqslant 2 n !$.  证明对所   有的正整数  $n,\left|f^{(n)}(0)\right| \leqslant(n+1) !$  均成立 .

  \item  计算第二类曲面积分  $\iint_{\Sigma}(y-z) \mathrm{d} y \mathrm{~d} z+(z-x) \mathrm{d} z \mathrm{~d} x+(x-y) \mathrm{d} x \mathrm{~d} y$,  其中曲面  $\Sigma$  是球面  $x^{2}+y^{2}+z^{2}=2 R x$  被圆柱面  $x^{2}+y^{2}=2 r x,(0<r<R)$  所截下的位于  $z \geqslant 0$  的部分 ,  定向取外侧 .

  \item  已知函数  $F(x, y)=2-\sin x+y^{3} \mathrm{e}^{-y}$  定义在全平面上 .  证明  $F(x, y)=0$  唯一确定了全平面上连续可微的   隐函数  $y=y(x)$.

  \item  设函数  $f(x)$  在  $[0,+\infty)$  上内闭  Riemann  可积 ,  且广义积分  $\int_{0}^{+\infty} f(x) \mathrm{d} x$  收玫 ,  证明 

\end{enumerate}
$$
\lim _{a \rightarrow 0^{+}} \int_{0}^{+\infty} \mathrm{e}^{-a x} f(x) \mathrm{d} x=\int_{0}^{+\infty} f(x) \mathrm{d} x .
$$

\begin{enumerate}
  \setcounter{enumi}{8}
  \item  已知函数  $f(x)$  是  $(-\infty,+\infty)$  上  2  阶连续可微函数 ,  满足  $\lim _{|x| \rightarrow+\infty}(f(x)-|x|)=0$,  且存在一点  $x_{0} \in \mathbb{R}$,  使得  $f\left(x_{0}\right) \leqslant 0$.  证明  $f^{\prime \prime}(x)$  在  $(-\infty,+\infty)$  上变号 .

  \item  设函数  $f(x)$  在区间  $[0,1]$  上有一阶连续导函数 ,  且  $f(0)=f(1), g(x)$  是周期为  1  的连续函数 ,  并且满足  $\int_{0}^{1} g(x) \mathrm{d} x=0$.  记  $a_{n}=\int_{0}^{1} f(x) g(n x) \mathrm{d} x$,  证明  $\lim _{n \rightarrow \infty} n a_{n}=0$.

  \item  若  $f(x)$  在区间  $[0,1]$  上  Riemann  可积 ,  并且对  $[0,1]$  中任意有限个两两不相交的闭区间  $\left[a_{i}, b_{i}\right], 1 \leqslant i \leqslant n$,  都有  $\left|\sum_{i=1}^{n} \int_{a_{i}}^{b_{i}} f(x) \mathrm{d} x\right| \leqslant 1$.  证明  $\int_{0}^{1}|f(x)| \mathrm{d} x \leqslant 2$. 1.  请参考数学分析教材 .  也可参考我写的北京大学  2016  年数学分析解答 .

  \item  不存在 .  若  $f\left(x_{1}\right)=f\left(x_{2}\right)$,  则  $\mathrm{e}^{-x_{1}}=f\left(f\left(x_{1}\right)\right)=f\left(f\left(x_{2}\right)\right)=\mathrm{e}^{-x_{2}}$,  于是  $x_{1}=x_{2}$,  从而  $f$  是单射 .  再结合  $f(x)$  在  $\mathbb{R}$  上的连续性 ,  就有  $f(x)$  是  $\mathbb{R}$  上的单调函数 ,  从而  $f(f(x))$  在  $\mathbb{R}$  上单调增 ,  矛盾 .

\end{enumerate}
 注   裴礼文的 《 数学分析中的典型问题与方法 》 第二版第  131  页例  $2.1 .5$  与本题相关 .

\begin{enumerate}
  \setcounter{enumi}{3}
  \item $\forall n \in \mathbb{N}$,  确定了一个区间  $\left[x_{n}-\frac{1}{2 n}, x_{n}+\frac{1}{2 n}\right]$.  下面证明  $\forall n<m$,
\end{enumerate}
$$
\left[x_{n}-\frac{1}{2 n}, x_{n}+\frac{1}{2 n}\right] \bigcap\left[x_{m}-\frac{1}{2 m}, x_{m}+\frac{1}{2 m}\right]=\varnothing
$$
 假若  $\exists y_{0} \in \mathbb{R}$,  满足 
$$
y_{0} \in\left[x_{n}-\frac{1}{2 n}, x_{n}+\frac{1}{2 n}\right], \quad y_{0} \in\left[x_{m}-\frac{1}{2 m}, x_{m}+\frac{1}{2 m}\right]
$$
 则 
$$
\left|x_{n}-x_{m}\right| \leqslant\left|x_{n}-y_{0}\right|+\left|y_{0}-x_{m}\right| \leqslant \frac{1}{2 n}+\frac{1}{2 m}<\frac{1}{n},
$$
 矛盾 .

 若  $\left\{x_{n}\right\}$  有界 ,  则  $\left\{x_{n}\right\}$  在某个有界闭集  $[-M, M]$  中 ,  从而 
$$
2 M+3>\sum_{n=1}^{\infty} \frac{1}{n},
$$
 矛盾 .

\begin{enumerate}
  \setcounter{enumi}{4}
  \item  因为  $f(x) g(x)=f^{\prime}(x)$,  于是  $(f(x) g(x))^{(n)}=f^{(n+1)}(x)$.  用数学归纳法来证明  $\left|f^{(n)}(0)\right| \leqslant(n+1)$ !, $n \in \mathbb{N}$. $n=1$  时 ,  命题成立 .  假设原命题对小于等于  $n$  的数均成立 ,  下面考虑  $n+1$  的情形 .  由  Leibniz  公式得 
\end{enumerate}
$$
\begin{aligned}
\left|f^{(n+1)}(0)\right| &=\left|(f(x) g(x))^{(n)}\right|_{x=0} \mid \\
& \leqslant\left|f(0) g^{(n)}(0)\right|+C_{n}^{1}\left|f^{\prime}(0) g^{(n-1)}(0)\right|+\cdots+\mid f^{(n)}(0) \\
& \leqslant 2 n !+n \cdot 2 \cdot(2(n-1) !)+\cdots+(n+1) ! \cdot 2 \\
&=\sum_{k=0}^{n} C_{n}^{k} \cdot(k+1) ! \cdot 2(n-k) ! \\
&=\sum_{k=0}^{n} 2(k+1) n ! \\
&=(n+2) !
\end{aligned}
$$
 由数学归纳法原理知原命题成立 .

 注   试题回忆者说 :

 本试卷打印有误 ,  最后的  $n$  阶导数没有加括号  $(n)$.  这里已更正 .

\begin{enumerate}
  \setcounter{enumi}{5}
  \item $\pi R r^{2}$.  详细过程见谢惠民等人的 《 数学分析习题课讲义 》 下册第  345  页例  25.2.4.

  \item $F_{y}(x, y)=y^{2}(3-y) \mathrm{e}^{-y}$.  因为当  $y=3$  时 , $F(x, y)=0 \Longleftrightarrow 2-\sin x+9 \mathrm{e}-3=0$,  此时无解 ,  故  $y(x)$  的值   域不能含有  3.  类似地可以看出 ,  其实当  $y \geqslant 0$  时 , $F(x, y)=0$  均无解 ,  因此使  $F(x, y)=0$  有解的  $y$  必须   满足  $y<0$.  当  $y<0$  时 , $F_{y}(x, y)>0$.  此即说明 ,  对于取定的  $x \in \mathbb{R}$,  若  $F(x, y)=0$  有解 ,  则解唯一 .  又因为 

\end{enumerate}
$$
\lim _{y \rightarrow-\infty} F(x, y)=-\infty, \quad F(x, 0)=2-\sin x>0
$$
 故  $F(x, y)=0$  有解  $y=y(x)$.

$\forall x_{0} \in \mathbb{R}$,  由隐函数定理知  $y=y(x)$  在  $x_{0}$  的某个邻域内是唯一确定的 ,  并且连续可微 .  从而  $F(x, y)=0$  确   定了唯一的函数  $y=y(x), y(x)$  在  $\mathbb{R}$  上连续可微 .

\begin{enumerate}
  \setcounter{enumi}{7}
  \item $\forall \varepsilon>0, \exists A_{0}>0$,  当  $A_{2}>A_{1} \geqslant A_{0}$  时 , $\left|\int_{A_{1}}^{A_{2}} f(x) \mathrm{d} x\right|<\frac{\varepsilon}{8}$.  于是 
\end{enumerate}
$$
\left|\int_{A_{1}}^{A_{2}}\left(\mathrm{e}^{-a x}-1\right) f(x) \mathrm{d} x\right| \leqslant\left|\left(\mathrm{e}^{-a A_{1}}-1\right) \int_{A_{1}}^{\xi} f(x) \mathrm{d} x\right|+\left|\left(\mathrm{e}^{-a A_{2}}-1\right) \int_{\xi}^{A_{2}} f(x) \mathrm{d} x\right| \leqslant \frac{\varepsilon}{2},
$$
 因此 
$$
\left|\int_{A_{0}}^{+\infty}\left(\mathrm{e}^{-a x}-1\right) f(x) \mathrm{d} x\right| \leqslant \frac{\varepsilon}{2}
$$
 又因为 
$$
\begin{gathered}
\left|\int_{0}^{A_{0}}\left(\mathrm{e}^{-a x}-1\right) f(x) \mathrm{d} x\right| \leqslant\left(1-\mathrm{e}^{-a A_{0}}\right) \int_{0}^{A_{0}}|f(x)| \mathrm{d} x \\
\lim _{a \rightarrow 0^{+}}\left(1-\mathrm{e}^{-a A_{0}}\right) \int_{0}^{A_{0}}|f(x)| \mathrm{d} x=0,
\end{gathered}
$$
 故  $\exists \delta>0$,  当  $0<a<\delta$  时 ,
$$
\left(1-\mathrm{e}^{-a A_{0}}\right) \int_{0}^{A_{0}}|f(x)| \mathrm{d} x<\frac{\varepsilon}{2}
$$
 于是 
$$
\left|\int_{0}^{+\infty}\left(\mathrm{e}^{-a x}-1\right) f(x) \mathrm{d} x\right| \leqslant\left|\int_{0}^{A_{0}}\left(\mathrm{e}^{-a x}-1\right) f(x) \mathrm{d} x\right|+\left|\int_{A_{0}}^{+\infty}\left(\mathrm{e}^{-a x}-1\right) f(x) \mathrm{d} x\right|<\varepsilon
$$

\begin{enumerate}
  \setcounter{enumi}{8}
  \item  因为  $\lim _{x \rightarrow+\infty}(f(x)-x)=0$,  故  $\exists x_{1}>x_{0}$,  使得  $f\left(x_{1}\right)>0$.  又  $f\left(x_{0}\right) \leqslant 0$,  从而由  Lagrange  中值定理知  $\exists \xi_{1} \in\left(x_{0}, x_{1}\right)$,  使得  $f^{\prime}\left(\xi_{1}\right)>0$.  同样可以证得  $\exists \xi_{2} \in\left(-\infty, x_{0}\right)$,  使得  $f^{\prime}\left(\xi_{2}\right)<0$.  再次利用  Lagrange  定理得  $\exists \xi \in\left(\xi_{2}, \xi_{1}\right)$,  使得  $f^{\prime \prime}(\xi)>0$.  因此若  $f^{\prime \prime}(x)$  在  $\mathbb{R}$  上不变号 ,  则  $f^{\prime \prime}(x)>0, \forall x \in \mathbb{R}$.
\end{enumerate}
 若  $\exists y_{0} \in \mathbb{R}$,  使得  $f^{\prime}\left(y_{0}\right)>1$,  则  $f(x)>f^{\prime}\left(y_{0}\right)\left(x-y_{0}\right) f\left(y_{0}\right)$,  这将与  $\lim _{x \rightarrow+\infty}(f(x)-x)=0$  矛盾 .  从而  $f^{\prime}(x) \leqslant 1, \forall x \in \mathbb{R}$.  同样可证  $f^{\prime}(x) \geqslant-1, \forall x \in \mathbb{R}$.

 若  $\exists x^{*}>0$,  使得  $f\left(x^{*}\right) \neq x^{*}$,  则由  Lagrange  定理知 , $\exists \eta>0$,  使得  $f^{\prime}(\eta)>1$,  矛盾 .  从而  $f(x)=x, \forall x>0$.  类似可证  $f(x)=-x, \forall x<0$.  于是  $f(x)=|x|$,  这与  $f(x)$  在  $\mathbb{R}$  上可微矛盾 .

$9 .$
$$
\begin{aligned}
n \int_{0}^{1} f(x) g(n x) \mathrm{d} x &=\int_{0}^{1} f(x) \mathrm{d} x \int_{0}^{n x} g(t) \mathrm{d} t \\
&=\left.\left(f(x) \int_{0}^{n x} g(t) \mathrm{d} t\right)\right|_{0} ^{1}-\int_{0}^{1}\left(\int_{0}^{n x} g(t) \mathrm{d} t\right) f^{\prime}(x) \mathrm{d} x \\
&=-\int_{0}^{1}\left(\int_{0}^{n x} g(t) \mathrm{d} t\right) f^{\prime}(x) \mathrm{d} x .
\end{aligned}
$$
 令  $G(x)=\int_{0}^{x} g(t) \mathrm{d} t$,  则 
$$
G(x+1)=\int_{0}^{x+1} g(t) \mathrm{d} t=\int_{1}^{x+1} g(t) \mathrm{d} t=\int_{0}^{x} g(t) \mathrm{d} t=G(x)
$$
 由更一般的化的  Riemann-Lebesgue  引理知 
$$
\lim _{n \rightarrow \infty} \int_{0}^{1} f^{\prime}(x) G(n x) \mathrm{d} x=\frac{1}{1} \int_{0}^{1} f^{\prime}(x) \mathrm{d} x \int_{0}^{1} G(x) \mathrm{d} x=0,
$$
 从而 
$$
\lim _{n \rightarrow \infty} n a_{n}=\lim _{n \rightarrow \infty}-\int_{0}^{1}\left(\int_{0}^{n x} g(t) \mathrm{d} t\right) f^{\prime}(x) \mathrm{d} x=0
$$
 注  Riemann  引理的证明可以参考裴礼文的 《 数学分析中的典型问题与方法 》 第二版第  315  页例  $4.1 .10$.

\begin{enumerate}
  \setcounter{enumi}{10}
  \item  令 
\end{enumerate}
$$
f_{+}(x)=\frac{|f(x)|+f(x)}{2}, \quad f_{-}(x)=\frac{|f(x)|-f(x)}{2}
$$
 若  $\int_{0}^{1} f_{+}(x) \mathrm{d} x>1$,  则存在  $[0,1]$  的分割  $P: a=x_{0}<x_{1}<\cdots<x_{n}=1$,  使得 
$$
\sum_{i=1}^{n} m_{i} \Delta x_{i}>1
$$
 其中  $m_{i}=\inf \left\{f_{+}(x) \mid x \in\left[x_{i-1}, x_{i}\right]\right\}, \Delta x_{i}=x_{i}-x_{i-1}, 1 \leqslant i \leqslant n$.  把  $m_{i}>0$  的那些区间  $\left[a_{i}, b_{i}\right]$  取出来 , $f(x)$  在其上均大于  0 ,  于是 
$$
\left|\sum_{i=1}^{n} \int_{a_{i}}^{b_{i}} f(x) \mathrm{d} x\right|>1
$$
 矛盾 .  从而  $\int_{0}^{1} f_{+}(x) \mathrm{d} x \leqslant 1$.  类似地可以证明  $\int_{0}^{1} f_{-}(x) \mathrm{d} x \leqslant 1$.  因此 
$$
\int_{0}^{1}|f(x)| \mathrm{d} x=\int_{0}^{1} f_{+}(x) \mathrm{d} x+\int_{0}^{1} f_{-}(x) \mathrm{d} x \leqslant 2
$$
 北京大学  2009  年全国硕士研究生招生考试数学分析试题及解答 

   

2019.05.09

\begin{enumerate}
  \item  证明闭区间上的连续函数能取到最大值和最小值 .

  \item  设  $f(x)$  和  $g(x)$  是实数集  $\mathbb{R}$  上的有界一致连续函数 ,  证明  $f(x) g(x)$  在  $\mathbb{R}$  上一致连续 .

  \item  设  $f(x)$  是周期为  $2 \pi$  的连续函数 ,  且其  Fourier  级数  $\frac{a_{0}}{2}+\sum_{n=1}^{\infty} a_{n} \cos n x+b_{n} \sin n x$  处处收玫 .  证明这个  Fourier  级数处处收玫到  $f(x)$.

  \item  设  $\left\{a_{n}\right\}_{n=1}^{\infty}$  和  $\left\{b_{n}\right\}_{n=1}^{\infty}$  都是有界数列 ,  且  $a_{n+1}+2 a_{n}=b_{n}$.  若  $\lim _{n \rightarrow \infty} b_{n}$  存在 ,  证明  $\lim _{n \rightarrow \infty} a_{n}$  也存在 .

  \item  是否存在  $\mathbb{R} \rightarrow \mathbb{R}$  的连续可微函数  $f(x)$,  满足 : $f(x)>0$  且  $f^{\prime}(x)=f(f(x))$.

  \item  已知  $f(x)$  是  $[0,+\infty)$  上的单调连续函数且  $\lim _{x \rightarrow+\infty} f(x)=0$.  证明  $\lim _{n \rightarrow \infty} \int_{0}^{+\infty} f(x) \sin n x \mathrm{~d} x=0$.

  \item  计算曲线积分  $\int_{L}(y-z) \mathrm{d} x+(z-x) \mathrm{d} y+(x-y) \mathrm{d} z$,  这里  $L$  是球面  $x^{2}+y^{2}+z^{2}=1$  与  $(x-1)^{2}+(y-$ $1)^{2}+(z-1)^{2}=4$  交成的曲线 .  从  $x$  轴正向看去 ,  定向是逆时针方向 .

  \item  设  $x, y, z \geqslant 0, x+y+z=\pi$.  求  $2 \cos x+3 \cos y+4 \cos z$  的最大值和最小值 .

  \item  设  $f(x) \in C(a, b)$,  且对任何  $x \in(a, b)$  都有  $\lim _{h \rightarrow 0^{+}} \frac{f(x+h)-f(x-h)}{h} \geqslant 0$.  证明 : $f(x)$  在  $(a, b)$  上单调不减 .

  \item  已知  $f(x)$  是  $[0,+\infty)$  上的正连续函数 ,  且  $\int_{0}^{+\infty} \frac{1}{f(x)} \mathrm{d} x<+\infty$.  证明  $\lim _{A \rightarrow+\infty} \frac{1}{A^{2}} \int_{0}^{A} f(x) \mathrm{d} x=+\infty$. 1.  请参考数学分析的教材 .

  \item  设  $|f(x)| \leqslant M,|g(x)| \leqslant M, \forall x \in \mathbb{R}$.  因为  $f(x)$  与  $g(x)$  在  $\mathbb{R}$  上一致连续 ,  故  $\forall \varepsilon>0, \exists \delta>0$,  当  $x, y \in$ $\mathbb{R},|x-y|<\delta$  时 ,

\end{enumerate}
$$
|f(x)-f(y)|<\frac{\varepsilon}{2 M}, \quad|g(x)-g(y)|<\frac{\varepsilon}{2 M},
$$
 此时 
$$
|f(x) g(x)-f(y) g(y)| \leqslant|f(x)||g(x)-g(y)|+|f(x)-f(y)||g(y)| \leqslant \varepsilon
$$
3 .  设 
$$
S_{0}(x)=\frac{a_{0}}{2}, \quad S_{N}(x)=\frac{a_{0}}{2}+\sum_{n=1}^{N} a_{n} \cos n x+b_{n} \sin n x, \quad \lim _{N \rightarrow+\infty} S_{N}(x)=g(x)
$$
 则 
$$
\lim _{n \rightarrow+\infty} \sigma_{n}(x)=\frac{1}{n} \sum_{k=0}^{n-1} S_{k}(x)=g(x),
$$
 由  Fejér  定理 , $\left\{\sigma_{n}(x)\right\}$  一致收敛于  $f(x)$,  故 
$$
\lim _{n \rightarrow+\infty} \sigma_{n}(x)=f(x),
$$
 因此 
$$
\lim _{N \rightarrow+\infty} S_{N}(x)=g(x)=f(x)
$$

\begin{enumerate}
  \setcounter{enumi}{4}
  \item  设 
\end{enumerate}
$$
\varlimsup_{n \rightarrow \infty} a_{n}=A, \quad \underline{\lim _{n \rightarrow \infty}} a_{n}=a, \quad \lim _{n \rightarrow \infty} b_{n}=b .
$$
 对  $a_{n+1}=b_{n}-2 a_{n}$  两端同时取上极限或者下极限得 
$$
\left\{\begin{array}{l}
A=b-2 a \\
a=b-2 A
\end{array} \Longrightarrow a=A=\frac{b}{3} \Longrightarrow \lim _{n \rightarrow \infty} a_{n}=\frac{b}{3} .\right.
$$
 注   更难一点的题目见谢惠民老师的 《 数学分析习题课讲义 》 第  90  页例题  3.6.3.  一道相关的题目见林源渠 、 方   企勤编的 《 数学分析解题指南 》 第  221  页例  16 .

\begin{enumerate}
  \setcounter{enumi}{5}
  \item  不存在 .  因为  $f^{\prime}(x)=f(f(x))>0$,  故  $f(x)$  在  $\mathbb{R}$  上单调递增 .  又  $f(x)>0$,  说明  $f(x)$  有下界 ,  从而极限  $\lim _{x \rightarrow-\infty} f(x)$  存在 ,  设极限值为  $a, a \geqslant 0$.  当  $x<1$  时 ,
\end{enumerate}
$$
\int_{x}^{1} \frac{f^{\prime}(t)}{f(f(t))} \mathrm{d} t=1-x \Longrightarrow \int_{f(x)}^{f(1)} \frac{1}{f(u)} \mathrm{d} u=1-x
$$
 令  $x \rightarrow-\infty$  得 
$$
\int_{a}^{f(1)} \frac{1}{f(u)} \mathrm{d} u=+\infty
$$
 矛盾 .

\begin{enumerate}
  \setcounter{enumi}{6}
  \item $\forall \varepsilon>0, \exists \Delta>0$,  当  $x>\Delta$  时 , $|f(x)|<\frac{\varepsilon}{8}$.  对于上述的  $\Delta>0$,  当  $A>\Delta+1$  时 ,
\end{enumerate}
$$
\begin{aligned}
\left|\int_{\Delta+1}^{A} f(x) \sin n x \mathrm{~d} x\right| & \leqslant\left|f(\Delta+1) \int_{\Delta+1}^{\xi} \sin n x \mathrm{~d} x\right|+\mid f(A) \int_{\xi}^{A} \sin n x \mathrm{~d} x \\
& \leqslant \frac{2(f(\Delta+1)+f(A))}{n} \\
& \leqslant \frac{\varepsilon}{2}
\end{aligned}
$$
 因此 
$$
\left|\int_{\Delta+1}^{+\infty} f(x) \sin n x \mathrm{~d} x\right| \leqslant \frac{\varepsilon}{2},
$$
 又因为  $f(x)$  在  $[0, \Delta+1]$  上连续 ,  由  Riemann-Lebesgue  引理知 : $\exists N>0$,  当  $n>N$  时 ,
$$
\left|\int_{0}^{\Delta+1} f(x) \sin n x \mathrm{~d} x\right|<\frac{\varepsilon}{2},
$$
 于是 
$$
\left|\int_{0}^{+\infty} f(x) \sin n x \mathrm{~d} x\right| \leqslant\left|\int_{0}^{\Delta+1} f(x) \sin n x \mathrm{~d} x\right|+\left|\int_{\Delta+1}^{+\infty} f(x) \sin n x \mathrm{~d} x\right|<\varepsilon
$$

\begin{enumerate}
  \setcounter{enumi}{7}
  \item  球面  $x^{2}+y^{2}+z^{2}=1$  与  $(x-1)^{2}+(y-1)^{2}+(z-1)^{2}=4$  交成的曲线在平面  $x+y+z=0$  上 ,  于是  $L$  也   可以看作是球面  $x^{2}+y^{2}+z^{2}=1$  与平面  $x+y+z=0$  相交所得的曲线 .  由  Stokes  公式得 ,
\end{enumerate}
$$
\text { 原式 }=\iint_{\Sigma}\left|\begin{array}{ccc}
\cos \alpha & \cos \beta & \cos \gamma \\
\frac{\partial}{\partial x} & \frac{\partial}{\partial y} & \frac{\partial}{\partial z} \\
y-z & z-x & x-y
\end{array}\right| \mathrm{d} S=\iint_{\Sigma} \frac{1}{\sqrt{3}} \times(-6) \mathrm{d} S=-2 \sqrt{3} \pi \text {. }
$$
 注   此题与北京大学  2010  年数学分析第  8  题类似 .  不用  Stokes  公式 ,  通过将  $L$  参数化再计算比较麻烦 .

\begin{enumerate}
  \setcounter{enumi}{8}
  \item  原问题等价于求  $F(x, y)=2 \cos x+3 \cos y-4 \cos (x+y)$  在  $D=\{(x, y) \mid x \geqslant 0, y \geqslant 0, x+y \leqslant \pi\}$  上的最   大最小值 .  因为  $F$  在有界闭集  $D$  上连续 ,  因此能在  $D$  上取得最大最小值 .
\end{enumerate}
 先考虑  $F$  在  $D$  的边界上的最值 .  容易演算得最大值为  5 ,  最小值为  1 .

 再考虑  $F$  在  $D$  的内部的最值 ,  而内部的最值一定是极值 .  设方程组 
$$
\left\{\begin{array}{l}
F_{x}(x, y)=-2 \sin x+4 \sin (x+y)=0 \\
F_{y}(x, y)=-3 \sin y+4 \sin (x+y)=0
\end{array}\right.
$$
 的解为  $\left(x_{0}, y_{0}\right)$,  则角度为  $x_{0}, y_{0}, \pi-\left(x_{0}+y_{0}\right)$  的三个角可以看作一个三角形的三个内角 ,  由正弦定理知它   们所对应的边的长度之比为  $6: 4: 3$,  由余弦定理知 
$$
\left\{\begin{array}{l}
\cos x_{0}=\frac{-11}{24} \\
\cos y_{0}=\frac{29}{36} \\
\cos \left(\pi-\left(x_{0}+y_{0}\right)\right)=\frac{43}{48}
\end{array} \quad \Longrightarrow F\left(x_{0}, y_{0}\right)=\frac{61}{12}>5 .\right.
$$
 综上知  $F$  在  $D$  上的最大值为  $\frac{61}{12}$,  最小值为  1 .

 注   类似的题目见张筑生的 《 数学分析新讲 》 第二册第  281  页例  2 .

\begin{enumerate}
  \setcounter{enumi}{9}
  \item  定义 
\end{enumerate}
$$
D_{+} F(x)=\underline{\lim _{h \rightarrow 0^{+}}} \frac{F(x+h)-F(x-h)}{2 h} .
$$
 先证明一个引理 :

 设  $F(x)$  在  $[a, b]$  上连续且  $F(a)>F(b)$,  则存在  $c \in[a, b]$,  使得  $D_{+} F(c) \leqslant 0$.

 取  $m \in(F(a), F(b))$,  令  $S=\{x \in[a, b] \mid F(x)>m\}$,  则  $a \in S$,  从而  $S \neq \varnothing$,  又因为  $b$  是  $S$  的一个上   界 ,  若令  $c=\sup S$,  则  $a<c<b$.  当  $x \in(c, b]$  时 , $F(x) \leqslant m$.  同时  $\exists\left\{a_{n}\right\} \subset S$,  满足  $\lim _{n \rightarrow \infty} a_{n}=c$.  于是  $h_{n}=c-a_{n}>0, F\left(c-h_{n}\right)>m$,  同时  $F\left(c+h_{n}\right) \leqslant m$,  于是  $F\left(c+h_{n}\right)-F\left(c-h_{n}\right)<0$,  故  $D_{+} F(c) \leqslant 0$.  下面来证明原命题 .

$\forall \varepsilon>0, F(x)=f(x)+\varepsilon x$  在  $(a, b)$  上应该单调不减 .  事实上 ,  若前一句话不成立 ,  则  $\exists x_{1}, x_{2} \in(a, b), x_{1}<x_{2}$,  但  $F\left(x_{1}\right)>F\left(x_{2}\right)$.  由上面证得的引理 , $\exists \xi \in\left(x_{1}, x_{2}\right)$,  使得  $D_{+} F(\xi) \leqslant 0$.  而  $D_{+} F(\xi)=D_{+} f(\xi)+\varepsilon>0$,  矛   自 .

 因为  $F(x)=f(x)+\varepsilon x$  在  $(a, b)$  上应该单调不减 ,  故  $\forall x_{1}<x_{2}, f\left(x_{1}\right)+\varepsilon x_{1} \leqslant f\left(x_{2}\right)+\varepsilon x_{2}$.  由  $\varepsilon$  的任意性 ,  令  $\varepsilon \rightarrow 0^{+}$ 得 , $f\left(x_{1}\right) \leqslant f\left(x_{2}\right)$.

 注   此题参考了  SCIbird  整理的一份解答 .

\begin{enumerate}
  \setcounter{enumi}{10}
  \item  由  Cauchy-Schwarz  不等式知 
\end{enumerate}
$$
\int_{\frac{A}{2}}^{A} f(x) \mathrm{d} x \int_{\frac{A}{2}}^{A} \frac{1}{f(x)} \mathrm{d} x \geqslant \frac{A^{2}}{4}
$$
 因此 
$$
\frac{1}{A^{2}} \int_{0}^{A} f(x) \mathrm{d} x \geqslant \frac{1}{A^{2}} \int_{\frac{A}{2}}^{A} f(x) \mathrm{d} x \geqslant \frac{1}{4 \int_{\frac{A}{2}}^{A} f(x) \mathrm{d} x}
$$
 又因为 
$$
\lim _{A \rightarrow+\infty} \int_{\frac{A}{2}}^{A} f(x) \mathrm{d} x=0
$$
 故 
$$
\lim _{A \rightarrow+\infty} \frac{1}{A^{2}} \int_{0}^{A} f(x) \mathrm{d} x=+\infty
$$
 注   此题与北京大学  2011  年数学分析第  9  题本质上相同 .  北京大学  2010  年全国硕士研究生招生考试数学分析试题及解答 

   

2019.05.05

\begin{enumerate}
  \item (15  分 )  用有限覆盖定理证明聚点定理 .

  \item (15  分 )  是否存在数列  $\left\{x_{n}\right\}$,  其极限点构成的集合为  $M=\left\{1, \frac{1}{2}, \frac{1}{3}, \cdots\right\}$,  说明理由 .

  \item (15  分 )  设  $I$  为无穷区间 , $f(x)$  为  $I$  上的非多项式连续函数 .  证明 :  不存在  $I$  上一致收敛的多项式序列  $\left\{P_{n}(x)\right\}$,  其极限函数为  $f(x)$.

  \item (15  分 )  设  $f(x)$  在  $[0,1]$  上连续 ,  在  $(0,1)$  内可导 ,  且满足  $f(1)=2 \int_{0}^{\frac{1}{2}} e^{1-x^{2}} f(x) \mathrm{d} x$.  试证明 :  存在  $\xi \in(0,1)$,  使得  $f^{\prime}(\xi)=2 \xi f(\xi)$.

  \item (15  分 )  设  $f \in C^{1}(I), I$  是有界闭区间 , $F_{n}(x)=n\left[f\left(x+\frac{1}{n}\right)-f(x)\right]$.  证明函数序列  $\left\{F_{n}(x)\right\}$  在  $I$  上一   致收敛 .  若  $I$  是有界开区间 , $\left\{F_{n}(x)\right\}$  在  $I$  上是否仍一致收敛并说明理由 .

  \item (15  分 )  构造  $\mathbb{R}$  上的函数  $f(x)$,  使其在  $\mathbb{Q}$  上间断 ,  其他点连续 .

  \item (15  分 )  设广义积分  $\int_{0}^{\infty} x f(x) \mathrm{d} x$  与  $\int_{0}^{\infty} \frac{f(x)}{x} \mathrm{~d} x$  均收敛 ,  且  $\forall 0<a<A<+\infty, f(x) \in R[a, A]$.  证明  $I(t)=\int_{0}^{\infty} x^{t} f(x) \mathrm{d} x$  在  $(-1,1)$  有定义 ,  并且有连续导函数 .

  \item (15  分 )  计算曲线积分  $I=\oint_{\Gamma} y \mathrm{~d} x+z \mathrm{~d} y+x \mathrm{~d} z$,  其中  $\Gamma$  为  $x^{2}+y^{2}+z^{2}=1$  与  $x+y+z=0$  的交线 .  从  $x$  轴正向看去 ,  定向是逆时针方向 .

  \item (15  分 )  证明方程  $x+\frac{1}{2} y^{2}+\frac{1}{2} z+\sin z=0$  在点  $(0,0,0)$  附近唯一确定了隐函数  $z=f(x, y)$,  并将  $f(x, y)$  在点  $(0,0)$  处展开为带二阶  Peano  型余项的  Taylor  公式 .

  \item (15  分 )  设  $f(x), g(x)$  是  $[0,+\infty)$  上单调下降的连续正函数 ,  并且  $\int_{0}^{+\infty} f(x) \mathrm{d} x$  与  $\int_{0}^{+\infty} g(x) \mathrm{d} x$  发散 ,  记  $h(x)=\min \{f(x), g(x)\},(x \in[0,+\infty)) .$  试问  $\int_{0}^{+\infty} h(x) \mathrm{d} x$  是否必定发散  ( 说明理由 ). 1.  要证明闭区间  $[a, b]$  中的无限子集  $S$  必有极限点 .  假若没有极限点 ,  则  $\forall x_{0} \in[a, b], \exists x_{0}$  的一个邻域只包含  $S$  中至多一点  $x_{0}$,  所有这些邻域构成  $[a, b]$  的一个开覆盖 ,  存在  $[a, b]$  的有限开覆盖 ,  这个有限开覆盖也能盖   住  $S$,  从而  $S$  中只有有限个点 ,  与  $S$  为无限子集矛盾 .

  \item  不存在这样的数列 .  因为数列的极限点构成的集合为闭集 ,  从而  $0 \in M$,  矛盾 .

\end{enumerate}
 注   利用了  Walter Rudin  的  Principles of mathematical analysis  英文版第三版第  52  页定理  $3.7$.

\begin{enumerate}
  \setcounter{enumi}{3}
  \item  假设存在  $\left\{P_{n}(x)\right\}$  在  $I$  上一致收敛到  $f(x)$,  则  $\forall \varepsilon>0, \exists N_{\varepsilon}>0$,  当  $n>m \geqslant N_{\varepsilon}$  时 ,
\end{enumerate}
$$
\left|P_{n}(x)-P_{m}(x)\right|<\varepsilon .
$$
 因为  $I$  为无穷区间 ,  因此当  $n>m \geqslant N_{\varepsilon}$  时 , $\left|P_{n}(x)-P_{m}(x)\right|$  为常数 .  设 
$$
\left|P_{N_{\varepsilon}}(x)-P_{n}(x)\right|=c_{n}, \quad n>N,
$$
 于是  $\left\{c_{n}\right\}$  为有界数列 ,  必有收敛子列  $\left\{c_{n_{k}}\right\}_{k=1}^{\infty}$,  设 
$$
\lim _{k \rightarrow \infty} c_{n_{k}}=c .
$$
 结合  $\left|P_{N_{\varepsilon}}(x)-P_{n_{k}}(x)\right|=c_{n_{k}}$,  令  $k \rightarrow \infty$  得 
$$
\left|P_{N_{\varepsilon}}(x)-f(x)\right|=c,
$$
 于是  $f(x)$  为多项式 ,  矛盾 .

 注   类似的题目见谢惠民等人的 《 数学分析习题课讲义 》 下册第  76  页第  10  题 .

\begin{enumerate}
  \setcounter{enumi}{4}
  \item  令  $g(x)=\mathrm{e}^{1-x^{2}} f(x)$,  要证明原命题 ,  只需证明  $\exists \xi \in(0,1)$,  使得  $g^{\prime}(\xi)=0$.  因为 
\end{enumerate}
$$
g(1)=f(1)=2 \int_{0}^{\frac{1}{2}} g(x) \mathrm{d} x=g(\eta), \quad \eta \in\left(0, \frac{1}{2}\right)
$$
 再利用  Rolle  定理知存在  $\xi \in(\eta, 1)$,  使得  $g^{\prime}(\xi)=0$.

\begin{enumerate}
  \setcounter{enumi}{5}
  \item  对于固定的  $x \in I, \lim _{n \rightarrow \infty} F_{n}(x)=f^{\prime}(x)$.  下面证明  $F_{n}(x)$  在有界闭区间  $I$  上一致收敛到  $f^{\prime}(x)$.  因为  $f \in$ $C^{1}(I)$,  故  $f^{\prime}(x)$  在  $I$  上连续 ,  从而  $f^{\prime}(x)$  在  $I$  上一致连续 . $\forall \varepsilon>0, \exists \delta>0$,  当  $x, y \in I,|x-y|<\delta$  时 , $\left|f^{\prime}(x)-f^{\prime}(y)\right|<\varepsilon$.  于是当  $n>\frac{1}{\delta}$  时 ,
\end{enumerate}
$$
\left|F_{n}(x)-f^{\prime}(x)\right|=\left|f^{\prime}\left(x+\xi_{n}\right)-f^{\prime}(x)\right|<\varepsilon, \quad 0<\xi_{n}<\frac{1}{n}<\delta .
$$
 当  $I$  为有界开区间时结论不一定对 .  考虑  $f(x)=\ln x, x \in(0,1)$.  此时  $F_{n}(x)$  在  $(0,1)$  上逐点收敛到  $\frac{1}{x}$.  我们要证明  $\exists \varepsilon_{0}>0, \forall N>0, \exists n>N, x \in(0,1)$  使得  $\left|F_{n}(x)-\frac{1}{x}\right| \geqslant \varepsilon_{0}$.

 取  $\varepsilon_{0}=1-\ln 2$,  令  $x=\frac{1}{n}, n \in \mathbb{N}_{+}$,  则  $\left|F_{n}(x)-\frac{1}{x}\right|=(1-\ln 2) n \geqslant 1-\ln 2$.

 注   此题前半部分与林源渠 、 方企勤编的 《 数学分析解题指南 》 第  234  页练习题  $4.2 .2$  或者陈纪修等人编的 《 数   学分析 》 第二版下册第  69  页第  6  题几乎完全一样 .

\begin{enumerate}
  \setcounter{enumi}{6}
  \item  取 
\end{enumerate}
$$
f(x)=R(x)= \begin{cases}\frac{1}{p}, & x=\frac{q}{p},(p, q)=1 \\ 1, & x=0 \\ 0, & x \in \mathbb{R} \backslash \mathbb{Q}\end{cases}
$$
 即可 .  考察的  Riemann  函数 . 7.  因为 
$$
I(t)=\int_{0}^{1} x^{t+1} \cdot \frac{f(x)}{x} \mathrm{~d} x+\int_{1}^{+\infty} \frac{1}{x^{1-t}} \cdot x f(x) \mathrm{d} x, \quad t \in(-1,1) .
$$
 由  Abel  判别法知  $I(t)$  在  $(-1,1)$  有意义 .  令 
$$
J(t)=\int_{0}^{+\infty} x^{t} \ln x \cdot f(x) \mathrm{d} x, \quad t \in(-1,1) .
$$
 我先证明  $J(t)$  在  $(-1,1)$  上连续 .

$\forall[a, b] \subset(-1,1)$,
$$
J(t)=\int_{0}^{1} x^{t-a} \cdot x^{1+a} \ln x \cdot \frac{f(x)}{x} \mathrm{~d} x+\int_{1}^{+\infty} x^{t-b} \cdot x^{b-1} \ln x \cdot x f(x) \mathrm{d} x
$$
 当  $\delta$  足够小时 , $x^{1+a} \ln x$  在  $(0, \delta)$  上单调递减 ,  再注意到  $\lim _{x \rightarrow 0^{+}} x^{1+a} \ln x=0$,  由广义积分的  Abel  判别法知  $\int_{0}^{1} x^{1+a} \ln x \cdot \frac{f(x)}{x} \mathrm{~d} x$  是收敛的 ,  因此  $\int_{0}^{1} x^{1+a} \ln x \cdot \frac{f(x)}{x} \mathrm{~d} x$  关于  $t$  在  $[a, b]$  上一致收敛 .  又因为对于取定   的  $t \in[a, b], x^{t-a}$  为关于  $x$  的单调函数 ,  并且 
$$
\left|x^{t-a}\right| \leqslant 1, \forall x \in[0,1], t \in[a, b]
$$
 由含参变量广义积分的  Abel  判别法知  $J_{1}(t)=\int_{0}^{1} x^{t-a} \cdot x^{1+a} \ln x \cdot \frac{f(x)}{x} \mathrm{~d} x$  关于  $t$  在  $[a, b]$  上一致收敛 .  从   而  $\forall \varepsilon>0, \exists \eta_{0}>0$,  对于任意  $\eta \in\left(0, \eta_{0}\right]$,  都有 
$$
\left|\int_{0}^{\eta} x^{t-a} \cdot x^{1+a} \ln x \cdot \frac{f(x)}{x} \mathrm{~d} x\right|<\frac{\varepsilon}{3}, \quad \forall t \in[a, b] .
$$
 于是  $\forall t, t_{0} \in[a, b]$
$$
\begin{aligned}
\left|J_{1}(t)-J_{1}\left(t_{0}\right)\right| &=\left|\int_{0}^{1} x^{t-a} \cdot x^{1+a} \ln x \cdot \frac{f(x)}{x} \mathrm{~d} x-\int_{0}^{1} x^{t_{0}-a} \cdot x^{1+a} \ln x \cdot \frac{f(x)}{x} \mathrm{~d} x\right| \\
& \leqslant\left|\int_{0}^{\eta_{0}} x^{t} \ln x \cdot f(x) \mathrm{d} x\right|+\left|\int_{0}^{\eta_{0}} x_{0}^{t} \ln x \cdot f(x) \mathrm{d} x\right|+\mid \int_{\eta_{0}}^{1}\left(x^{t}-x^{t_{0}}\right) \ln x \cdot f(x) \mathrm{d} x \\
&<\frac{\varepsilon}{3}+\frac{\varepsilon}{3}+\left|t-t_{0}\right|\left|\int_{\eta_{0}}^{1} x^{t_{0}+\theta\left(t-t_{0}\right)}(\ln x)^{2} \cdot f(x) \mathrm{d} x\right| \quad \theta \in(0,1) \\
& \leqslant \frac{2 \varepsilon}{3}+\left|t-t_{0}\right| \int_{\eta_{0}}^{1}\left|x^{a}(\ln x)^{2} \cdot f(x)\right| \mathrm{d} x
\end{aligned}
$$
 因此存在正数  $\delta<\frac{\varepsilon}{3 M}$,  其中  $M=\max \left\{\int_{\eta_{0}}^{1}\left|x^{a}(\ln x)^{2} \cdot f(x)\right| \mathrm{d} x, 1\right\}$.  当  $\left|t-t_{0}\right|<\delta$  时 ,  就有 
$$
\left|J_{1}(t)-J_{1}\left(t_{0}\right)\right|<\varepsilon .
$$
 这说明  $J_{1}(t)$  在  $[a, b]$  上连续 ,  由  $[a, b]$  的任意性知  $J_{1}(t)$  在  $(-1,1)$  上连续 .

 当  $X_{0}$  充分大时 , $x^{b-1} \ln x$  在  $\left(X_{0},+\infty\right)$  上单调递减趋于零 ,  而  $\int_{1}^{+\infty} x f(x) \mathrm{d} x$  收敛 ,  由广义积分的  Abel  判   别法知  $\int_{1}^{+\infty} x^{b-1} \ln x \cdot x f(x) \mathrm{d} x$  收敛 .  从而  $\int_{1}^{+\infty} x^{b} \ln x \cdot f(x) \mathrm{d} x$  关于  $t$  在  $[a, b]$  上一致收敛 ,  又因为对于   取定的  $t \in[a, b], x^{t-b}$  为关于  $x$  的单调函数 ,  并且 
$$
\left|x^{t-b}\right| \leqslant 1, \forall x \in[1,+\infty), t \in[a, b],
$$
 由含参变量广义积分的  Abel  判别法知  $J_{2}(t)=\int_{1}^{+\infty} x^{t-b} \cdot x^{b-1} \ln x \cdot x f(x) \mathrm{d} x$  关于  $t$  在  $[a, b]$  上一致收敛 .  从而  $\forall \varepsilon>0, \exists A_{0}>0$,  对于任意  $A \geqslant A_{0}$,  都有 
$$
\left|\int_{A}^{+\infty} x^{t-b} \cdot x^{b-1} \ln x \cdot x f(x) \mathrm{d} x\right|<\frac{\varepsilon}{3}, \quad \forall t \in[a, b] .
$$
 于是  $\forall t, t_{0} \in[a, b]$
$$
\begin{aligned}
\left|J_{2}(t)-J_{2}\left(t_{0}\right)\right| &=\left|\int_{1}^{+\infty} x^{t-b} \cdot x^{b-1} \ln x \cdot x f(x) \mathrm{d} x-\int_{1}^{+\infty} x^{t_{0}-b} \cdot x^{b-1} \ln x \cdot x f(x) \mathrm{d} x\right| \\
& \leqslant\left|\int_{0}^{A_{0}}\left(x^{t}-x^{t_{0}}\right) \ln x \cdot f(x) \mathrm{d} x\right|+\left|\int_{A_{0}}^{+\infty} x^{t} \ln x \cdot f(x) \mathrm{d} x\right|+\mid \int_{A_{0}}^{+\infty} x_{0}^{t} \ln x \cdot f(x) \mathrm{d} x \\
&<\left|t-t_{0}\right|\left|\int_{0}^{A_{0}} x^{t_{0}+\theta\left(t-t_{0}\right)}(\ln x)^{2} \cdot f(x) \mathrm{d} x\right|+\frac{\varepsilon}{3}+\frac{\varepsilon}{3} \quad \theta \in(0,1) \\
& \leqslant\left|t-t_{0}\right| \int_{0}^{A_{0}}\left|x^{b}(\ln x)^{2} \cdot f(x)\right| \mathrm{d} x+\frac{2 \varepsilon}{3}
\end{aligned}
$$
 因此存在正数  $\delta<\frac{\varepsilon}{3 M}$,  其中  $M=\max \left\{\int_{0}^{A_{0}}\left|x^{b}(\ln x)^{2} \cdot f(x)\right| \mathrm{d} x, 1\right\}$.  当  $\left|t-t_{0}\right|<\delta$  时 ,  就有 
$$
\left|J_{2}(t)-J_{2}\left(t_{0}\right)\right|<\varepsilon
$$
 这说明  $J_{2}(t)$  在  $[a, b]$  上连续 ,  由  $[a, b]$  的任意性知  $J_{2}(t)$  在  $(-1,1)$  上连续 .  因此  $J(t)=J_{1}(t)+J_{2}(t)$  在  $(-1,1)$  上连续 .

 最后只需证明  $I^{\prime}(t)=J(t), t \in(-1,1)$.  或者只需证明  $\forall[a, b] \subset(-1,1)$,
$$
\int_{a}^{u} J(t) \mathrm{d} t=I(u)-I(a), \quad \forall u \in[a, b] .
$$
 今 
$$
I_{1}(t)=\int_{0}^{1} x^{t} f(x) \mathrm{d} x, \quad I_{2}(t)=\int_{1}^{+\infty} x^{t} f(x) \mathrm{d} x, \quad t \in(-1,1)
$$
 下面证明 
$$
\int_{a}^{u} J_{2}(t) \mathrm{d} t=I_{2}(u)-I_{2}(a), \quad \forall u \in[a, b]
$$
 由于  $J_{2}(t)$  在  $[a, b]$  上关于  $t$  一致收敛 ,  由  Cauchy  收敛原理知  $\forall \varepsilon>0, \exists A_{0}>1, \forall A_{2}>A_{1}>A_{0}$,
$$
\left|\int_{A_{1}}^{A_{2}} x^{t} \ln x \cdot f(x) \mathrm{d} x\right|<\varepsilon, \quad t \in[a, b] .
$$
 选取一列单调递增趋于  $+\infty$  的数列  $\left\{a_{n}\right\}$,  并令  $a_{0}=1$,  此时存在  $N>0$,  当  $n>m>N$  时 , $a_{n}>a_{m}>A_{0}$,  于是 
$$
\left|\sum_{k=m+1}^{n} \int_{a_{k-1}}^{a_{k}} x^{t} \ln x \cdot f(x) \mathrm{d} x\right|=\left|\int_{a_{m}}^{a_{n}} x^{t} \ln x \cdot f(x) \mathrm{d} x\right|<\varepsilon, \quad t \in[a, b] .
$$
 因此关于  $t$  的函数项级数  $\sum_{k=1}^{\infty} \int_{a_{k-1}}^{a_{k}} x^{t} \ln x \cdot f(x) \mathrm{d} x$  在  $[a, b]$  上一致收敛 .  因此  $\forall u \in[a, b]$,
$$
\int_{a}^{u}\left(\sum_{k=1}^{\infty} \int_{a_{k-1}}^{a_{k}} x^{t} \ln x \cdot f(x) \mathrm{d} x\right) \mathrm{d} t=\sum_{k=1}^{\infty} \int_{a}^{u}\left(\int_{a_{k-1}}^{a_{k}} x^{t} \ln x \cdot f(x) \mathrm{d} x\right) \mathrm{d} t
$$
 若能证明 
$$
\int_{a}^{u} \mathrm{~d} t \int_{a_{k-1}}^{a_{k}} x^{t} \ln x \cdot f(x) \mathrm{d} x=\int_{a_{k-1}}^{a_{k}} \mathrm{~d} x \int_{a}^{u} x^{t} \ln x \cdot f(x) \mathrm{d} t
$$
 那么 
$$
\sum_{k=1}^{\infty} \int_{a}^{u}\left(\int_{a_{k-1}}^{a_{k}} x^{t} \ln x \cdot f(x) \mathrm{d} x\right) \mathrm{d} t=\sum_{k=1}^{\infty} \int_{a_{k-1}}^{a_{k}}\left(\int_{a}^{u} x^{t} \ln x \cdot f(x) \mathrm{d} t\right) \mathrm{d} x
$$
 于是 
$$
\int_{a}^{u} J_{2}(t) \mathrm{d} t=\int_{1}^{+\infty}\left(\int_{a}^{u} x^{t} \ln x \cdot f(x) \mathrm{d} t\right) \mathrm{d} x=\int_{1}^{+\infty}\left(x^{u}-x^{a}\right) f(x) \mathrm{d} x=I_{2}(u)-I_{2}(a), \quad \forall u \in[a, b]
$$
 类似地又可以证明 
$$
\int_{a}^{u} J_{1}(t) \mathrm{d} t=I_{1}(u)-I_{1}(a), \quad \forall u \in[a, b]
$$
 于是只要我们最后再说明了上面那个二重积分确实能交换顺序 ,  我们就完成了此题的证明 .  而上面二重积分   能交换顺序可以由下面的命题得到 ,  具体证明请参考北京大学  2013  年数学分析第  7  题 .

$f(x, y)$  定义在  $[a, b] \times[c, d]$  上 ,  且对于任意固定的  $x \in[a, b], f(x, y)$  在  $[c, d]$  上可积 .  而且  $\forall \varepsilon>$ $0, \exists \delta>0$,  当  $x_{1}, x_{2} \in[a, b],\left|x_{1}-x_{2}\right|<\delta$  时 ,  有 
$$
\left|f\left(x_{1}, y\right)-f\left(x_{2}, y\right)\right|<\varepsilon, \quad \forall y \in[c, d]
$$
 那么  $p(x)=\int_{c}^{d} f(x, y) \mathrm{d} y$  在  $[\mathrm{a}, \mathrm{b}]$  上可积 , $q(y)=\int_{a}^{b} f(x, y) \mathrm{d} x$  在  $[c, d]$  上可积 ,  且积分相等 .

 注   相关的题目陈纪修等人编的  《 数学分析 》 第二版下册第  392  页第  3  题和第  7  题 .  如果  $f(x)$  在  $[0,+\infty)$  上   连续则比较好证明 ,  可以利用数学分析教材上的定理完成证明 .  证明  $J(t)$  在  $(-1,1)$  上的连续性时 ,  分成了   证明  $J_{1}(t)$  与  $J_{2}(t)$  在  $(-1,1)$  上的连续性 , $J_{1}(t)$  与  $J_{2}(t)$  其实是类似的 ,  这里我为了完整起见就都写了下 ,  因此这个证明就显得很长 .  为了证明  $I^{\prime}(t)=J(t), t \in(-1,1)$,  我最初的想法是通过定义来直接证明 ,  虽然   炮制了一个证明 ,  但是后来发现是错的 ,  现在才想到这样一个证明 ,  不知道是否还是有问题 ,  只能留给读者   去评判了 ,  如果有误 ,  还望告知我 ,  我也比较期待一个更简洁的证明 .

\begin{enumerate}
  \setcounter{enumi}{8}
  \item  由  Stokes  公式得 
\end{enumerate}
$$
I=\iint_{\Sigma}\left|\begin{array}{ccc}
\cos \alpha & \cos \beta & \cos \gamma \\
\frac{\partial}{\partial x} & \frac{\partial}{\partial y} & \frac{\partial}{\partial z} \\
y & z & x
\end{array}\right| \mathrm{d} S=\iint_{\Sigma}\left(-\frac{1}{\sqrt{3}}-\frac{1}{\sqrt{3}}-\frac{1}{\sqrt{3}}\right) \mathrm{d} S=-\sqrt{3} \pi
$$
 注   此题与林源渠 、 方企勤编的 《 数学分析解题指南 》 第  360  页例  3  或者裴礼文的 《 数学分析中的典型问题与   方法 》 第二片第  1014  页练习题  $7.4 .21$  或者陈纪修等人编的 《 数学分析 》 第二版下册第  347  页第  12  题的  (1)  几平完全一样 .

9 .  令  $F(x, y, z)=x+\frac{1}{2} y^{2}+\frac{1}{2} z+\sin z$,  则 
$$
F(0,0,0)=0,\left.\frac{\partial F}{\partial z}\right|_{(0,0,0)}=\frac{3}{2} \neq 0 .
$$
 由隐函数定理知  $F(x, y, z)=0$  在  $(0,0,0)$  附近唯一确定了一个隐函数  $z=f(x, y)$.  经过运算可以得到 
$$
f(0,0)=0, f_{x}^{\prime}(0,0)=-\frac{2}{3}, f_{y}^{\prime}(0,0)=0, f_{x x}^{\prime \prime}(0,0)=0, f_{x y}^{\prime \prime}(0,0)=0, f_{y y}^{\prime \prime}(0,0)=-\frac{2}{3}
$$
 于是 
$$
f(x, y)=-\frac{2}{3} x-\frac{1}{3} y^{2}+o\left(x^{2}+y^{2}\right), \quad\left(\sqrt{x^{2}+y^{2}} \rightarrow 0\right)
$$

\begin{enumerate}
  \setcounter{enumi}{10}
  \item  不一定发散 .  令 
\end{enumerate}
 则 
$$
h(x)= \begin{cases}1, & x \in[0,1] \\ \frac{1}{x^{2}}, & x \in[1,+\infty)\end{cases}
$$
 此时  $\int_{0}^{+\infty} h(x) \mathrm{d} x$  收敛 .

 注   此题与谢惠民等人编写的 《 数学分析习题课讲义 》 下册第  38  页第  4  题 

 试构造两个发散的正项级数  $\sum_{n=1}^{\infty} a_{n}$  与  $\sum_{n=1}^{\infty} a_{n}$,  它们的通项都单调减少 ,  但级数  $\sum_{n=1}^{\infty} \min \left\{a_{n}, b_{n}\right\}$  收   剑 .

 没有本质区别 .  上面那道题我们可以类似地构造 

 则  $\sum_{n=1}^{\infty} \min \left\{a_{n}, b_{n}\right\}=\sum_{n=1}^{\infty} \frac{1}{n^{2}}$  收敛 .  我当初是通过画图想到这个例子的 ,  写出具体的式子后 ,  看起来比较复   杂 ,  但是如果大家把函数图像画出来 ,  或者用列举法列举一些通项就能明白其中的奥妙 .  测地者曾指出 

 这道题见于北大伍胜健教授编著的 《 数学分析 》( 第二册 )( 北京大学数学教学系列丛书 , 2010  年  2  月出版 )  第八章习题第  7  题 ,  原书有提示 .
$$
\begin{aligned}
& f(x)= \begin{cases}1, & x \in[0,1] \\ \frac{1}{(2 k-1)^{2}}, & x \in\left[(2 k-1)^{2},(2 k)^{2}-1\right], k \in \mathbb{N} \\ \left(\frac{1}{(2 k)^{2}}-\frac{1}{(2 k-1)^{2}}\right)\left(x-(2 k)^{2}\right)+\frac{1}{(2 k)^{2}}, & x \in\left((2 k)^{2}-1,(2 k)^{2}\right), k \in \mathbb{N}_{+} \\ \frac{1}{x^{2}}, & x \in\left[(2 k)^{2},(2 k+1)^{2}\right), k \in \mathbb{N}_{+}\end{cases} \\
& g(x)= \begin{cases}1, & x \in[0,1] \\ \frac{1}{(2 k)^{2}}, & x \in\left[(2 k)^{2},(2 k+1)^{2}-1\right], k \in \mathbb{N}_{+} \\ \left(\frac{1}{(2 k+1)^{2}}-\frac{1}{(2 k)^{2}}\right)\left(x-(2 k+1)^{2}\right)+\frac{1}{(2 k+1)^{2}}, & x \in\left((2 k+1)^{2}-1,(2 k+1)^{2}\right), k \in \mathbb{N}_{+} \\ \frac{1}{x^{2}}, & x \in\left[(2 k-1)^{2},(2 k)^{2}\right), k \in \mathbb{N}_{+}\end{cases} \\
& a_{n}=\left\{\begin{array}{cl}\frac{1}{(2 k-1)^{2}}, & (2 k-1)^{2} \leqslant n<(2 k)^{2}, k \in \mathbb{N}_{+} \\\frac{1}{n^{2}}, & (2 k)^{2} \leqslant n<(2 k+1)^{2}, k \in \mathbb{N}_{+}\end{array}\right. \\
& b_{n}=\left\{\begin{array}{cl}\frac{1}{(2 k)^{2}}, & (2 k)^{2} \leqslant n<(2 k+1)^{2}, k \in \mathbb{N}_{+} \\\frac{1}{n^{2}}, & (2 k-1)^{2} \leqslant n<(2 k)^{2}, k \in \mathbb{N}_{+}\end{array} \quad n \in \mathbb{N}_{+},\right. 
\end{aligned}
$$
 北京大学  2011  年全国硕士研究生招生考试数学分析试题及解答     

2019.04.19

\begin{enumerate}
  \item  用确界存在定理证明 :  如果  $f(x)$  是区间  $I$  上的连续函数 ,  则  $f(I)$  是一个区间 .

  \item  可微函数  $f$  在  $(0,1)$  上有界 , $\lim _{x \rightarrow 0^{+}} f(x)$  不存在 .  证明 :  存在数列  $\left\{x_{n}\right\}$  满足  $\lim _{n \rightarrow \infty} x_{n}=0$,  使得  $f^{\prime}\left(x_{n}\right)=0$.

  \item  证明如果  $f(x)$  在  $I$  上连续 , $|f(x)|$  可导 ,  则  $f(x)$  也可导 .

  \item  构造两个以  $2 \pi$  为周期的函数 ,  它们的  Fourier  级数在  $[0, \pi]$  上一致收玫于  0 .

  \item  证明  $f(x)$  在  $[0,1]$  上可积的充要条件是  $F(x, y)=f(x)$  在  $[0,1] \times[0,1]$  上可积 .

  \item $f(x, y)$  在其定义域中的某个点上存在非零方向导数 ,  且在三个方向上的方向向量均相等 .  证明  $f(x, y)$  不可   微 .

  \item  设  $D$  为  $\mathbb{R}^{2}$  中由一条光滑曲线所围成的无界闭区域 ,  试构造一个函数  $f(x, y)$,  使它在  $D$  上的二重积分  $\iint_{D} f(x, y) \mathrm{d} x \mathrm{~d} y$  发散 .

  \item  设  $D \subset \mathbb{R}^{n}, D$  是一个凸区域 , $T(x)$  在  $D$  上有连续二阶偏导数 ,  其  Jacobi  行列式正定 .  证明  $T(x)$  是单射 .

  \item  设正项级数  $\sum_{n=1}^{\infty} a_{n}$  收敛 .  证明极限  $\lim _{n \rightarrow \infty} \frac{n^{2}}{\frac{1}{a_{1}}+\frac{1}{a_{2}}+\ldots+\frac{1}{a_{n}}}$  存在 .

  \item  设  $f_{n}(x)$  在  $[a, b]$  上连续 , $f_{n}^{\prime}(x)$  在  $[a, b]$  上一致有界 ,  并且  $f_{n}(x)$  逐点收玫于极限函数  $f(x)$.  证明  $f(x)$  在  $[a, b]$  上连续 . 1.  若  $f(x)$  为常函数 ,  则  $f(I)$  为一个孤立点集 .  若  $f(x)$  不是常函数 ,  要证  $f(I)$  为一个区间 ,  即要说明  $\forall x_{1}, x_{2} \in$ $I, f\left(x_{1}\right) \neq f\left(x_{2}\right)$. (1) 若  $f\left(x_{1}\right)<f\left(x_{2}\right)$,  则  $\forall c \in\left(f\left(x_{1}\right), f\left(x_{2}\right)\right), \exists \xi \in I$  使得  $f(\xi)=c$. (2)  若  $f\left(x_{2}\right)<f\left(x_{1}\right)$,  则  $\forall c \in\left(f\left(x_{2}\right), f\left(x_{1}\right)\right), \exists \eta \in I$  使得  $f(\eta)=c$.  只证明 (1) 的情形 .  下面证明  $\exists \xi \in\left(x_{1}, x_{2}\right) \subset I$  使得  $f(\xi)=c$.  于是要证明原命题 ,  则只需证明连续函数的介值定理 ,  而为证明介值定理又只需证明连续函数的零点存在定   理 .

\end{enumerate}
 于是只需证明 :  若  $f\left(x_{1}\right)<0, f\left(x_{2}\right)>0, x_{1}<x_{2}$,  则存在  $\xi \in\left(x_{1}, x_{2}\right)$,  使得  $f(\xi)=0$.

 令  $S=\left\{x \in\left[x_{1}, x_{2}\right] \mid f(x)<0\right\}$,  则  $x_{1} \in S, x_{2} \notin S$  且  $x_{2}$  为  $S$  的上界 . $S$  为非空有上界集合 ,  必有上确   界  $\alpha=\sup S$.  若  $f(\alpha)>0$,  则在  $\alpha$  的一个邻域内  $f(x)$  仍大于  0 ,  从而不是  $S$  中的点 ,  从而  $\alpha$  不是上确   界 .  若  $f(\alpha)<0$,  则  $f(x)$  在  $\alpha$  的一个邻域内小于  0 .  于是存在大于  $\alpha$  的数使得  $f(x)<0$,  同样矛盾 .  于是  $f(\alpha)=0, \alpha \in\left(x_{1}, x_{2}\right) .$

\begin{enumerate}
  \setcounter{enumi}{2}
  \item ( 证法一 ) $\forall n \in \mathbb{N}_{+}, \exists x_{n} \in\left(0, \frac{1}{2 n}\right)$,  使得  $f^{\prime}\left(x_{n}\right)=0$.  事实上 ,  若上述结论不成立 ,  则由导函数介值定理知  $f^{\prime}(x)$  在  $\left(0, \frac{1}{2 n}\right)$  上不变号 ,  从而  $f(x)$  在  $\left(0, \frac{1}{2 n}\right)$  单调 ,  故  $\lim _{x \rightarrow 0^{+}} f(x)$  存在 ,  矛盾 .
\end{enumerate}
( 证法二 )  令  $x_{n}=\frac{1}{2 n}$,  则  $\left\{f\left(x_{n}\right)\right\}$  为有界数列 ,  必有收敛子列  $f\left(x_{n_{1 k}}\right) \rightarrow y_{1},(k \rightarrow \infty)$.  因为  $\lim _{x \rightarrow 0^{+}} f(x)$  不存   在 ,  故必有子列  $f\left(x_{n_{2 k}}\right) \rightarrow y_{2},(k \rightarrow+\infty)$  并且  $y_{1} \neq y_{2}$.  不妨设  $y_{1}<y_{2}$,  于是对于  $y_{0}=\frac{y_{1}+y_{2}}{2}, \exists k_{0}>0$,  当  $k>k_{0}$  时 , $f\left(x_{n_{1 k}}\right)<\frac{y_{1}+y_{2}}{2}, f\left(x_{n_{2 k}}\right)>\frac{y_{1}+y_{2}}{2}$.  于是存在  $\xi_{k}$  介于  $x_{n_{1 k}}$  与  $x_{n_{2 k}}$  之间使得  $f\left(\xi_{k}\right)=y_{0}$.  由于  $x_{n_{1 k}} \rightarrow 0, x_{n_{2 k}} \rightarrow 0$  可以选取数列  $\xi_{k} \rightarrow 0$  使得  $f\left(\xi_{k}\right)=y_{0}$,  由  Rolle  定理知存在一数列  $\eta_{k} \rightarrow 0, f^{\prime}\left(\eta_{k}\right)=0$.

 注   证法一由  SCIbird  提供 .  证法二源自对函数图像的观察 .

\begin{enumerate}
  \setcounter{enumi}{3}
  \item  由于  $|f(x)|$  在  $x_{0}$  处可导 ,  故极限  $\lim _{h \rightarrow 0} \frac{\left|f\left(x_{0}+h\right)\right|-\left|f\left(x_{0}\right)\right|}{h}$  存在 .  若  $f\left(x_{0}\right)=0$,  则必有  $\lim _{h \rightarrow 0} \frac{\left|f\left(x_{0}+h\right)\right|}{h}=0$,  于是  $\lim _{h \rightarrow 0} \frac{f\left(x_{0}+h\right)-f\left(x_{0}\right)}{h}=\lim _{h \rightarrow 0} \frac{f\left(x_{0}+h\right)}{h}=0$.  若  $f\left(x_{0}\right) \neq 0$,  则  $f(x)$  在  $x_{0}$  的某个邻域内不为  0 ,  从而当  $f\left(x_{0}\right)>0$  时 , $\lim _{h \rightarrow 0} \frac{\left|f\left(x_{0}+h\right)\right|-\left|f\left(x_{0}\right)\right|}{h}=\lim _{h \rightarrow 0} \frac{f\left(x_{0}+h\right)-f\left(x_{0}\right)}{h}$;  当  $f\left(x_{0}\right)<0$  时 , $\lim _{h \rightarrow 0} \frac{\left|f\left(x_{0}+h\right)\right|-\left|f\left(x_{0}\right)\right|}{h}=$ $-\lim _{h \rightarrow 0} \frac{f\left(x_{0}+h\right)-f\left(x_{0}\right)}{h}$.  综上 ,  若  $|f(x)|$  在  $x_{0}$  处可导则  $f(x)$  在  $x_{0}$  处可导 ,  由  $x_{0}$  的任意性知原命题成立 .

  \item (1)  取  $f(x) \equiv 0$,  显然满足题意 .

\end{enumerate}
(2)  取  $f(x)=\left\{\begin{array}{ll}\sin x, & x \in[-\pi, 0) \\ 0, & x \in[0, \pi]\end{array}, f(x+2 \pi)=f(x)\right.$.  此时 

\begin{enumerate}
  \setcounter{enumi}{5}
  \item  必要性 .  若  $f(x)$  在  $[0,1]$  可积 ,  则  $\forall \varepsilon>0$,  存在分割  $\mathrm{P}$  使得  $\sum_{i=1}^{n}\left(M_{i}-m_{i}\right) \Delta x_{i}<\varepsilon$.  对  $x \in[0,1]$  按  $\mathrm{P}$  进   行分割 ,  对  $y \in[0,1]$  不分割 ,  得到  $[0,1] \times[0,1]$  的一个分割 .  此时  $\sum_{i=1}^{n}\left(M_{i}-m_{i}\right) \Delta x_{i}<\varepsilon$,  于是  $F(x, y)$  在 
\end{enumerate}
 充分性 . 若  $F(x, y)$  在  $[0,1] \times[0,1]$  上可积 ,  则存在分割  $\mathrm{P}$  使得  $\sum_{i=1}^{n} \sum_{j=1}^{k}\left(M_{i j}-m_{i j}\right) \Delta x_{i} \Delta y_{j}<\varepsilon$.  按照上述分 
$$
\lim _{t \rightarrow 0^{+}} \frac{F\left(x_{0}+t \cos \theta_{1}, y_{0}+t \sin \theta_{1}\right)-F\left(x_{0}, y_{0}\right)}{t} \neq 0 .
$$
 若  $F(x, y)$  可微 ,  则由上式得到  $F_{x}^{\prime}\left(x_{0}, y_{0}\right) \cos \theta_{1}+F_{y}^{\prime}\left(x_{0}, y_{0}\right) \sin \theta_{1} \neq 0$.  由题设知存在  $\theta_{2}, \theta_{3}, \theta_{4}, 0 \leqslant \theta_{2}<$ $\theta_{3}<\theta_{4}<2 \pi$  使得 

$F_{x}^{\prime}\left(x_{0}, y_{0}\right) \cos \theta_{2}+F_{y}^{\prime}\left(x_{0}, y_{0}\right) \sin \theta_{2}=F_{x}^{\prime}\left(x_{0}, y_{0}\right) \cos \theta_{3}+F_{y}^{\prime}\left(x_{0}, y_{0}\right) \sin \theta_{3}=F_{x}^{\prime}\left(x_{0}, y_{0}\right) \cos \theta_{4}+F_{y}^{\prime}\left(x_{0}, y_{0}\right) \sin \theta_{4} .$  记上述等式的值为  $k$,  并将上面的等式转化为矩阵的形式 
$$
\left(\begin{array}{cc}
\cos \theta_{2} & \sin \theta_{2} \\
\cos \theta_{3} & \sin \theta_{3} \\
\cos \theta_{4} & \sin \theta_{4}
\end{array}\right)\left(\begin{array}{l}
F_{x}^{\prime}\left(x_{0}, y_{0}\right) \\
F_{y}^{\prime}\left(x_{0}, y_{0}\right)
\end{array}\right)=\left(\begin{array}{l}
k \\
k \\
k
\end{array}\right) \text {. }
$$
 若  $k=0$,  则解得  $F_{x}^{\prime}\left(x_{0}, y_{0}\right)=F_{y}^{\prime}\left(x_{0}, y_{0}\right)=0$,  矛盾 .

 若  $k \neq 0$,  则增广矩阵的秩为  3 ,  系数矩阵的秩为  2 ,  线性方程组无解 ,  矛盾 .

\begin{enumerate}
  \setcounter{enumi}{7}
  \item  记  $D_{n}=\left\{(x, y) \mid(n-1)^{2} \leqslant x^{2}+y^{2} \leqslant n^{2}\right\}$,  则  $\bigcup_{n=1}^{\infty} D_{n}=\mathbb{R}^{2}$.  设  $D_{n}$  与由光滑曲线围成的无界区域相   交部分的面积为  $S_{n}$,  因为是无界区域 ,  故当  $n$  充分大时必然  $S_{n}>0$.  设  $f(x, y)$  在  $S_{n}$  上的值为  $\frac{1}{n S_{n}}$,  则  $\iint_{D} f(x, y) \mathrm{d} x \mathrm{~d} y$  发散 .

  \item  若  $T(x)$  不是单射 ,  则存在  $x_{1}, x_{2} \in D, x_{1} \neq x_{2}, T\left(x_{1}\right)=T\left(x_{2}\right)$.

\end{enumerate}
 令  $\varphi(t)=\left(x_{2}-x_{1}\right)^{\mathrm{T}}\left(T\left(x_{1}+t\left(x_{2}-x_{1}\right)\right)-T\left(x_{1}\right)\right)$,  则 
$$
0=\varphi(1)-\varphi(0)=\varphi(\xi)=\left(x_{2}-x_{1}\right)^{\mathrm{T}} T^{\prime}\left(x_{1}+\xi\left(x_{2}-x_{1}\right)\right)\left(x_{2}-x_{1}\right)>0, \quad \xi \in(0,1) .
$$

\begin{enumerate}
  \setcounter{enumi}{9}
  \item $\forall \varepsilon>0, \exists N>0$,  当  $n>m>N$  时 , $a_{m}+a_{m+1}+\cdots+a_{n}<\frac{\varepsilon}{2}$.
\end{enumerate}
$$
\begin{aligned}
0<\frac{n^{2}}{\frac{1}{a_{1}}+\frac{1}{a_{2}}+\ldots+\frac{1}{a_{n}}} & \leqslant \frac{n^{2}}{\frac{1}{a_{m}}+\frac{1}{a_{m+1}}+\ldots+\frac{1}{a_{n}}} \\
&=\frac{n^{2}\left(a_{m}+a_{m+1}+\cdots+a_{n}\right)}{\left(\frac{1}{a_{m}}+\frac{1}{a_{m+1}}+\ldots+\frac{1}{a_{n}}\right)\left(a_{m}+a_{m+1}+\cdots\right.} \\
& \leqslant \frac{n^{2}\left(a_{m}+a_{m+1}+\cdots+a_{n}\right)}{(n-m+1)^{2}} \\
&<\frac{n^{2} \varepsilon}{2(n-m+1)^{2}} .
\end{aligned}
$$
 取定  $m=N+1$,  令  $n \rightarrow \infty$  得 
$$
0 \leqslant \liminf _{n \rightarrow \infty} \frac{n^{2}}{\frac{1}{a_{1}}+\frac{1}{a_{2}}+\ldots+\frac{1}{a_{n}}} \leqslant \limsup _{n \rightarrow \infty} \frac{n^{2}}{\frac{1}{a_{1}}+\frac{1}{a_{2}}+\ldots+\frac{1}{a_{n}}} \leqslant \frac{\varepsilon}{2} .
$$
 由  $\varepsilon$  的任意性知 
$$
\lim _{n \rightarrow \infty} \frac{n^{2}}{\frac{1}{a_{1}}+\frac{1}{a_{2}}+\ldots+\frac{1}{a_{n}}}=0 .
$$
 注   此题与北京大学  2009  年数学分析第  10  题类似 .

\begin{enumerate}
  \setcounter{enumi}{10}
  \item  设  $M$  为  $\left|f_{n}^{\prime}(x)\right|$  的一个上界 ,  则对于  $\forall s, t \in[a, b]$,
\end{enumerate}
$$
\left|f_{n}(s)-f_{n}(t)\right| \leqslant M|s-t|,
$$
 令  $n \rightarrow \infty$,  则 
$$
|f(s)-f(t)| \leqslant M|s-t|,
$$
 于是  $f(x)$  在  $[a, b]$  上一致连续 ,  从而  $f(x)$  在  $[a, b]$  上连续 .

 注   此证明由  SCIbird  提出 .  根据裴礼文的 《 数学分析中的典型问题与方法 》 第二版中例  $5.2 .35$  知  $f_{n}(x)$  在  $[a, b]$  上一致收敛到  $f(x)$,  也就能证明原问题 ,  但是不如这个证明直接 .  北京大学  2012  年全国硕士研究生招生考试数学分析试题及解答 

   

2019.05.05

\begin{enumerate}
  \item  叙述函数在区间  $[a, b]$  上  Riemann  可积的定义 ,  问 :  定义中的任意分割是否可以改为等距分割并说明理由 .

  \item  由  $2 x+y^{2}+\sin x y+1=\mathrm{e}^{x}$  确定  $y$  为  $x$  的函数  $y=f(x)$,  求在  $x=0$  处  $f(x)$  带两阶  Peano  余项的  Taylor  展开式 . $(\times)$

  \item  设  $f(x)=\frac{\mathrm{e}^{x}}{1-\sin x}$,  求  $f^{(4)}(0)$.

  \item  求  $\iiint_{V}\left(x^{2}+y^{2}+z^{2}\right) \mathrm{d} x \mathrm{~d} y \mathrm{~d} z$,  其中  $V=\left\{(x, y, z) \mid \frac{x^{2}}{a^{2}}+\frac{y^{2}}{b^{2}}+\frac{z^{2}}{c^{2}} \leqslant 1\right\}$.

  \item  由实数域的确界原理直接证明连续函数的介值定理 ,  再应用连续函数介值定理证明确界定理 .

  \item  设  $D=\left\{(x, y) \mid x^{2}+y^{2}<1\right\}, u(x, y)$  是定义在  $D$  上的二阶连续可微的二元函数并且在  $\bar{D}$  上连续 ,  在  $D$  的边界  $\partial D$  上有  $u(x, y) \geqslant 0$,  又  $u(x, y)$  在  $D$  上满足  $\frac{\partial^{2} u}{\partial x^{2}}+\frac{\partial u}{\partial x}+\frac{\partial u}{\partial y}=2 u$,  求证在  $D$  上恒有  $u(x, y) \geqslant 0$.

  \item $f(x)$  在  $(a, b)$  上可导 ,  且  $x_{0} \in(a, b)$  是  $f^{\prime}(x)$  的唯一间断点 ,  记 

\end{enumerate}
$$
S=\left\{t \in \mathbb{R} \cup\{-\infty\} \cup\{+\infty\} \mid \text { 存在数列 }\left\{x_{n}\right\} \subset(a, b) \text { 满足 } x_{n} \rightarrow x_{0}, f^{\prime}\left(x_{n}\right) \rightarrow t,(n \rightarrow \infty)\right\} .
$$
 问 : $S$  是什么样的集合 ?  若  $f(x)$  在  $(a, b) \backslash\left\{x_{0}\right\}$  上  $n(n \geqslant 2)$  阶可导 ,  证明  $f^{(n)}(x)$  在  $(a, b) \backslash\left\{x_{0}\right\}$  上有无   穷多个零点 .

\begin{enumerate}
  \setcounter{enumi}{8}
  \item  设  $f(x), g(x) \in C^{\infty}(-\infty,+\infty)$,  构造一个在  $\mathbb{R}$  上无穷次可微的函数  $h(x)$,  使得  $h(x)$  在  $[-1,1]$  上等于  $f(x)$,  在  $\mathbb{R} \backslash(-2,2)$  上等于  $g(x)$.

  \item  叙述并证明  $\frac{\infty}{\infty}$  型的  L'Hospital  法则 .

  \item  当年参加考试的人说  :

\end{enumerate}
 最后一题我考试时侯都没时间看  。 好像是什么可数个函数之类的没读完就打了下课軨  1.  函数在  $[a, b]$  上  Riemann  可积的定义 :  设  $f(x)$  是  $[a, b]$  上的有界函数 ,  如果  $\exists I \in \mathbb{R}, \forall \varepsilon>0, \exists \delta>0$,  对   于  $[a, b]$  的任意分割  $P: a=x_{0}<x_{1}<x_{2}<\cdots<x_{n}=b$  及任意  $\xi_{i} \in\left[x_{i-1}, x_{i}\right]$,  记  $\Delta x_{i}=x_{i}-x_{i-1}$,  当  $\max _{1 \leqslant i \leqslant n}\left\{\Delta x_{i}\right\}<\delta$  时 ,  有  $\left|\sum_{i=1}^{n} f\left(\xi_{i}\right) \Delta x_{i}-I\right|<\varepsilon$,  则称  $f(x)$  在  $[a, b]$  上  Riemann  可积 ,  记作  $\int_{a}^{b} f(x) \mathrm{d} x=I$.  定义中的任意分割可以改为等距分割 .

 首先等距分割是任意分割中的一种 ,  因此如果按任意分割  Riemann  可积 ,  则按等距分割也可积 .  下面考虑   另一个方向 .  假设按等距分割  Riemann  可积 ,  则  $\forall \varepsilon>0, \exists \delta>0$,  对于任意等距分害  $P: a=x_{0}<x_{1}<\cdots<$ $x_{n}=b, \Delta x_{i}=x_{i}-x_{i-1}=\frac{b-a}{n}, i=1,2, \cdots, n, \forall \xi_{i} \in\left[x_{i-1}, x_{i}\right]$,  只要  $\frac{b-a}{n}<\delta$  则  $\left|\sum_{i=1}^{n} f\left(\xi_{i}\right) \Delta x_{i}-I\right|<$ $\varepsilon$,  取定一个  $P$,  由  $f\left(\xi_{i}\right)$  的任意性 ,  取上下确界 ,  则  $\left|\sum_{i=1}^{n} M_{i} \Delta x_{i}-I\right| \leqslant \varepsilon,\left|\sum_{i=1}^{n} m_{i} \Delta x_{i}-I\right| \leqslant \varepsilon$,  于是  $\left|\sum_{i=1}^{n}\left(M_{i}-m_{i}\right) \Delta x_{i}\right| \leqslant 2 \varepsilon$,  从而  $f(x)$  在  $[a, b]$  上按任意分割  Riemann  可积 .

 注   相关的题目见谢惠民等人的 《 数学分析习题课讲义 》 上册第  335  页题  2 .

\begin{enumerate}
  \setcounter{enumi}{2}
  \item  令  $F(x, y)=2 x+y^{2}+\sin x y+1-\mathrm{e}^{x}$,  则  $F(0,0)=0, \frac{\partial F}{\partial y}(0,0)=\left.(2 y+x \cos x y)\right|_{(0,0)}=0$.  此时我们发现不能   由隐函数定理得出原来方程确定了一个函数 .  但是当我们假设原来方程确定了一个连续可微函数  $y=y(x)$,  并且代入原方程求导并令  $x=0$,  则得到  $2=1$,  矛盾 ,  于是我们最后发现这道题是错题 .
\end{enumerate}
$3 .$
$$
\begin{aligned}
&f(x)(1-\sin x)=\mathrm{e}^{x}, \quad f(0)=1 \\
&f^{\prime}(x)(1-\sin x)+f(x)(-\cos x)=\mathrm{e}^{x}, \quad f^{\prime}(0)=2 \\
&f^{\prime \prime}(x)(1-\sin x)+2 f^{\prime}(x)(-\cos x)+f(x) \sin x=\mathrm{e}^{x}, \quad f^{\prime \prime}(0)=5 \\
&f^{\prime \prime \prime}(x)(1-\sin x)+3 f^{\prime \prime}(x)(-\cos x)+3 f^{\prime}(x) \sin x+f(x) \cos x=\mathrm{e}^{x}, \quad f^{\prime \prime \prime}(0)=15 \\
&f^{(4)}(x)(1-\sin x)+4 f^{\prime \prime \prime}(x)(-\cos x)+6 f^{\prime \prime}(x) \sin x+4 f^{\prime}(x) \cos x+f(x)(-\sin x)=\mathrm{e}^{x}, \quad f^{(4)}(0)=-67
\end{aligned}
$$

\begin{enumerate}
  \setcounter{enumi}{4}
  \item  令  $\left\{\begin{array}{l}x=a r \cos \theta \sin \varphi \\ y=b r \sin \theta \sin \varphi \\ z=c r \cos \varphi\end{array}\right.$,  其中  $0 \leqslant \theta \leqslant 2 \pi, 0 \leqslant \varphi \leqslant \pi, 0 \leqslant r \leqslant 1$.  于是 
\end{enumerate}
$$
\begin{aligned}
\text { 原式 } &=\int_{0}^{2 \pi} \mathrm{d} \theta \int_{0}^{\pi} \mathrm{d} \varphi \int_{0}^{1}\left(a^{2} r^{2} \cos ^{2} \theta \sin ^{2} \varphi+b^{2} r^{2} \sin ^{2} \theta \sin ^{2} \varphi+c^{2} r^{2} \cos ^{2} \varphi\right) a b c r^{2} \sin \varphi \mathrm{d} r \\
&=\frac{a b c}{5} \int_{0}^{2 \pi} \mathrm{d} \theta \int_{0}^{\pi}\left(a^{2} \cos ^{2} \theta \sin ^{2} \varphi+b^{2} \sin ^{2} \theta \sin ^{2} \varphi+c^{2} \cos ^{2} \varphi\right) \sin \varphi \mathrm{d} \varphi \\
&=\frac{4 a b c \pi}{15}\left(a^{2}+b^{2}+c^{2}\right) .
\end{aligned}
$$

\begin{enumerate}
  \setcounter{enumi}{5}
  \item  用实数域的确界原理证明连续函数的介值定理 :  设  $f(x)$  是  $[a, b]$  上的连续函数 ,  不妨设  $f(a)<f(b)$,  要证   对于任意  $c \in(f(a), f(b)), \exists \xi \in(a, b)$,  使得  $f(\xi)=c$.  考虑  $S=\{x \in[a, b] \mid f(x)<c\}$,  取  $\xi=\sup S$  即可 .  由连续函数介值定理证明确界定理 : 我觉得这个题出得有问题 .  只摘录  SCIbird  的如下发言 
\end{enumerate}
 关于介值定理证明确界原理那题 ,  印象中是北大谭小江  \&  彭立中两位老师写的  3  小本数学分析 

 至于连续性的理解不能单靠思维定式 ,  函数连续性依赖定义域 ,  比如函数  $y=1 / x$  在  $(-\infty, 0) \cup$ $(0,+\infty)$  上是连续的 。( 定义域中没有  $0 )$  理解了上面的想法后 ,  回到原题 : 若假设  $\mathbb{R}$  上确界存在原理不成立后 ,  则定义域上必然出现间断 ,  连续性就得按定义域范围理解 。 设  $S$  为一有界集合 ,  取一个闭区间  $[a, b]$  真包含  $S$,  在  $[a, b]$  上定   义函数  $f(x)$,  是上界点定义为  1 ,  非上界点定义为  $0, f(x)$  是一个连续函数 ,  但不满足介值定理 。

\begin{enumerate}
  \setcounter{enumi}{6}
  \item  假设存在  $\left(x_{0}, y_{0}\right) \in D$  使  $u\left(x_{0}, y_{0}\right)<0$,  则  $u(x, y)$  在  $\bar{D}$  上的最小值在  $D$  内取得 .  设最小值点为  $M_{1}\left(x_{1}, y_{1}\right)$,  于是 
\end{enumerate}
$$
\left.\frac{\partial u}{\partial x}\right|_{M_{1}}=0,\left.\quad \frac{\partial u}{\partial y}\right|_{M_{1}}=0,\left.\quad \frac{\partial^{2} u}{\partial x^{2}}\right|_{M_{1}} \geqslant\left. 0 \Longrightarrow(2 u)\right|_{M_{1}}=\left.\left(\frac{\partial^{2} u}{\partial x^{2}}+\frac{\partial u}{\partial x}+\frac{\partial u}{\partial y}\right)\right|_{M_{1}} \geqslant 0
$$
 这与  $\left.u\right|_{M_{1}}<0$  矛盾 .

 注   相关的题目见裴礼文的 《 数学分析中的典型问题与方法 》 第二版  $713-714$  页练习题  $6.3 .19$  与  $6.3 .20$.

\begin{enumerate}
  \setcounter{enumi}{7}
  \item $S$  是  $\mathbb{R}$  中的连通闭集 .
\end{enumerate}
 先证明  $S$  是闭集 :  设  $\left\{t_{n}\right\}$  是  $S$  中的一个收敛数列 ,  收敛于  $t_{0}$,  我们要证  $t_{0} \in S$.  对于取定的  $n \in \mathbb{N}$,  假设 
$$
\lim _{k \rightarrow \infty} x_{n k}=x_{0}, \quad \lim _{k \rightarrow \infty} f^{\prime}\left(x_{n k}\right)=t_{n}
$$
 于是  $\exists k_{n}$,
$$
\left|x_{n k_{n}}-x_{0}\right|<\frac{1}{n}, \quad\left|f^{\prime}\left(x_{n k_{n}}\right)-t_{n}\right|<\frac{1}{n} .
$$
 因此 
$$
\left|f^{\prime}\left(x_{n k_{n}}\right)-t_{0}\right| \leqslant\left|f^{\prime}\left(x_{n k_{n}}\right)-t_{n}\right|+\left|t_{n}-t_{0}\right|<\frac{1}{n}+\left|t_{n}-t_{0}\right|
$$
 眕 
$$
x_{n k_{n}} \rightarrow x_{0}, \quad f^{\prime}\left(x_{n k_{n}}\right) \rightarrow t_{0},(n \rightarrow \infty) \Longrightarrow t_{0} \in S
$$
 再证明它是连通集 :  设  $t_{1}, t_{2} \in S, t_{1}<t_{2}$,  任取  $c \in\left(t_{1}, t_{2}\right)$,  我们要证  $c \in S$.  假设 
$$
\lim _{k \rightarrow \infty} x_{n k}=x_{0}, \quad \lim _{k \rightarrow \infty} f^{\prime}\left(x_{n k}\right)=t_{n}, n=1,2 .
$$
 于是  $\exists K_{1} \in \mathbb{N}_{+}$,  当  $k>K_{1}$  时 ,
$$
f^{\prime}\left(x_{1 k}\right)<\frac{t_{1}+c}{2}, \quad f^{\prime}\left(x_{2 k}\right)>\frac{t_{2}+c}{2} .
$$
 对于任意的  $k>K_{1}$,  由导函数介值定理知  $\exists x_{k}$  介于  $x_{1 k}$  与  $x_{2 k}$  之间 ,  使得 
$$
c-\frac{c-t_{1}}{2^{k}}<f^{\prime}\left(x_{k}\right)<c+\frac{t_{2}-c}{2^{k}} .
$$
 任意选择  $[a, b]$  中的数作为  $x_{1}, x_{2}, \ldots, x_{K_{1}}$,  于是得到了数列  $\left\{x_{k}\right\}$,  它满足  $\lim _{k \rightarrow \infty} x_{k}=x_{0}, \lim _{k \rightarrow \infty} f^{\prime}\left(x_{k}\right)=c$,  于   是  $c \in S$.

 因为  $f^{\prime}(x)$  在  $x_{0}$  处间断 ,  而导函数没有第一类间断点 ,  故不妨设极限  $\lim _{x \rightarrow x_{0}^{+}} f^{\prime}(x)$  不存在 .  于是可以假设存   在  $\left(x_{0}, b\right)$  中的数列  $\left\{x_{1 k}\right\},\left\{x_{2 k}\right\}$,
$$
\lim _{k \rightarrow \infty} x_{1 k}=\lim _{k \rightarrow \infty} x_{2 k}=x_{0}, \quad \lim _{k \rightarrow \infty} f^{\prime}\left(x_{1 k}\right)=A, \quad \lim _{k \rightarrow \infty} f^{\prime}\left(x_{2 k}\right)=B, \quad(A<B, A, B \in \overline{\mathbb{R}}) .
$$
 对于  $A<C<B$,  能找到一列互不相同的点  $\left\{\xi_{n}\right\}$,  使得  $f^{\prime}\left(\xi_{n}\right)=C, n \in \mathbb{N}$.  然后用  Rolle  定理知题中后半   部分正确 .

 注   一个具体的例子是  $f(x)=x^{2} \sin \frac{1}{x}, x \in(-1,1)$. 8.  设  $\varphi(x)=\left\{\begin{array}{ll}e^{-\frac{1}{x^{2}}}, & x>0 \\ 0, & x \leqslant 0\end{array}\right.$,  则  $\varphi(x)$  在  $\mathbb{R}$  上无穷次可微 ,  令 
$$
h(x)=g(x)+(f(x)-g(x)) \frac{\varphi\left((1+\delta)^{2}-x^{2}\right)}{\varphi\left((1+\delta)^{2}-x^{2}\right)+\varphi\left(x^{2}-1\right)}, \quad 0<\delta<1
$$
 注   相关的知识可以参考张筑生的 《 数学分析新讲 》 第二册第  13  页的注记 .  也可参考谢惠民等人的 《 数学分析   习题课讲义 》 上册第  171  页例题  $6.2 .5$.

\begin{enumerate}
  \setcounter{enumi}{9}
  \item $\frac{\infty}{\infty}$  型的  L'Hospital  法则 :  设  $f(x), g(x)$,  在  $(a, b)$  上可微 ,  若  $\lim _{x \rightarrow a^{+}} g(x)=\infty, \lim _{x \rightarrow a^{+}} f(x)=\infty$  且  $\lim _{x \rightarrow a^{+}} \frac{f^{\prime}(x)}{g^{\prime}(x)}=$ $A$,  则  $\lim _{x \rightarrow a^{+}} \frac{f(x)}{g(x)}=A$.
\end{enumerate}
 证明 :
$$
\begin{gathered}
\frac{f(x)}{g(x)}=\frac{f(x)-f(y)}{g(x)-g(y)} \cdot \frac{g(x)-g(y)}{g(x)}+\frac{f(y)}{g(x)} \\
\frac{f(x)}{g(x)}-A=\left(\frac{f(x)-f(y)}{g(x)-g(y)}-A\right)\left(1-\frac{g(y)}{g(x)}\right)+\frac{f(y)}{g(x)}-A \frac{g(y)}{g(x)}
\end{gathered}
$$
 因为  $\lim _{x \rightarrow a^{+}} \frac{f^{\prime}(x)}{g^{\prime}(x)}=A$,  则  $\forall \varepsilon>0, \exists \delta>0$,  当  $a<x<a+\delta$  时 , $\left|\frac{f^{\prime}(x)}{g^{\prime}(x)}-A\right|<\frac{\varepsilon}{4}$.  令  $a<x<y<a+\delta$,  则 
$$
\left|\frac{f(x)}{g(x)}-A\right| \leqslant\left|\frac{f^{\prime}(\xi)}{g^{\prime}(\xi)}-A\right|\left|1-\frac{g(y)}{g(x)}\right|+\frac{|f(y)-A g(y)|}{|g(x)|}, \quad \xi \in(x, y)
$$
 从而  $\left|\frac{f^{\prime}(\xi)}{g^{\prime}(\xi)}-A\right|<\frac{\varepsilon}{4}$,  固定  $y$,  令  $x \rightarrow a^{+}$,  则由  $1-\frac{g(y)}{g(x)} \rightarrow 1, \frac{f(y)-A g(y)}{|g(x)|} \rightarrow 0$, 于是  $\exists \delta_{1} \in(0, \delta)$,  当  $0<x<\delta_{1}$  时 , $\left|\frac{f(x)}{g(x)}-A\right|<\frac{\varepsilon}{4} \times 2+\frac{\varepsilon}{2}=\varepsilon$.

 注   其实应该是  $\frac{*}{\infty}$  型的  L'Hospital  法则 .

$10 .$  北京大学  2013  年全国硕士研究生招生考试数学分析试题及解答     

2019.05.04

\begin{enumerate}
  \item  用柯西收玫准则证明  $\mathbb{R}^{n}$  上的有限覆盖定理 .

  \item $f(x)=\sin x+x^{2}+1$,  在  0  附近有反函数 ,  求  $\left(f^{-1}\right)^{(4)}(1)$.

  \item  类比第二型曲线积分  $\int p(x, y) \mathrm{d} x+q(x, y) \mathrm{d} y$,  给出积分  $\int p(x, y) \mathrm{d} q(x, y)$  的定义 ,  并给出合理的可积判别   准则和计算方法 ,  证明你的结论 .

  \item $f(x)$  是  $[-\pi, \pi]$  上的单调函数 ,  定义  $F(x)=\int_{-\pi}^{x} f(t) \mathrm{d} t$,  证明任意  $x \in(-\pi, \pi)$,  极限 

\end{enumerate}
$$
\lim _{\Delta \rightarrow 0^{+}} \frac{F(x+\Delta x)-F(x-\Delta x)}{2 \Delta x}
$$
 存在 ,  并等于  $f(x)$  的傅里叶级数的和 .

\begin{enumerate}
  \setcounter{enumi}{5}
  \item  证明拉格朗日中值定理 ,  并给出它的一个应用 .

  \item $f_{n}(x)$  是  $[0,+\infty)$  上一致有界的函数列 ,  并且在任意的闭区间上一致收玫于  $f(x)$.  对于任意固定的  $n, f_{n}(x)$  是单调递增或者递减的函数 .  又  $\int_{0}^{+\infty} g(x) \mathrm{d} x$  收玫 ,  证明  $f(x) g(x)$  与  $f_{n}(x) g(x)$  都在  $[0,+\infty)$  上可积 ,  并   且有  $\lim _{n \rightarrow+\infty} \int_{0}^{+\infty} f_{n}(x) g(x) \mathrm{d} x=\int_{0}^{+\infty} f(x) g(x) \mathrm{d} x$.

  \item $f(x, y)$  定义在  $[a, b] \times[c, d]$  上 ,  且对于任意固定的  $x \in[a, b], f(x, y)$  在  $[c, d]$  上可积 .  而且  $\forall \varepsilon>0, \exists \delta>0$,  当  $x_{1}, x_{2} \in[a, b],\left|x_{1}-x_{2}\right|<\delta$  时 ,  有 

\end{enumerate}
$$
\left|f\left(x_{1}, y\right)-f\left(x_{2}, y\right)\right|<\varepsilon, \quad \forall y \in[c, d] .
$$
 证明  $p(x)=\int_{c}^{d} f(x, y) \mathrm{d} y$  在  $[\mathrm{a}, \mathrm{b}]$  上可积 , $q(y)=\int_{a}^{b} f(x, y) \mathrm{d} x$  在  $[c, d]$  上可积 ,  且积分相等 .

\begin{enumerate}
  \setcounter{enumi}{8}
  \item  正项级数  $\sum_{n=1}^{+\infty} a_{n}$  收玫 , $b_{n} \rightarrow 0,(n \rightarrow+\infty)$,  证明  $a_{1} b_{n}+a_{2} b_{n-1}+\ldots+a_{n} b_{1} \rightarrow 0,(n \rightarrow+\infty)$.

  \item  用两种方法将积分  $\iint_{\Omega} x y \mathrm{~d} x \mathrm{~d} y$  化成累次积分 ,  其中  $\Omega$  是  $y=x^{2}, x+y=2, x$  轴围成的区域 .

  \item  当年考试的人说 :

\end{enumerate}
 是证明一个类似重积分的中值定理 ,  然后用到格林公式 ,  具体式子记不清了 . 1.  设  $K$  为  $\mathbb{R}^{n}$  中的有界闭集 .  若  $\left\{O_{\lambda}\right\}_{\lambda \in \Lambda}$  为  $K$  的任意一个开覆盖 ,  需证明一定有有限个开集覆盖  $K$. $K$  一   定能被一个闭方体覆盖 ,  对方体的各边取中点进行分割 ,  最后得到  $2^{n}$  个小的闭方体 .  假设开覆盖定理不真 ,  则存在一个开覆盖 ,  存在一个与  $K$  交非空的小方体不能被有限覆盖 .  对该小方体再对分中点 ,  然后又有一   个与  $K$  交非空的小方体不能被有限覆盖住 , 一直进行下去可得到一列小方体 ,  从上述每个与  $K$  交非空的小   方体中取一个点构成一个点列  $\left\{x_{n}\right\}$,  该点列为  Cauchy  列 ,  必有极限 ,  设  $x_{n} \rightarrow x$,  则  $x \in K . x$  能被某个开   集盖住 ,  于是  $x$  的某个邻域能被此开集盖住而这个邻域又必定包含上述某个小方体 ,  矛盾 .

\begin{enumerate}
  \setcounter{enumi}{2}
  \item  设反函数为  $x=x(y)$,  则  $x(1)=0$.  对方程  $y=\sin x(y)+x^{2}(y)+1$  两边同时对  $y$  求导得 
\end{enumerate}
$$
(\cos x(y)) x^{\prime}(y)+2 x(y) x^{\prime}(y)=1
$$
 令  $y=1$  解得  $x^{\prime}(1)=1$.  再对  $y$  求导得 
$$
-\sin x(y)\left(x^{\prime}(y)\right)^{2}+(\cos x(y)) x^{\prime \prime}(y)+2\left(x^{\prime}(y)\right)^{2}+2 x(y) x^{\prime \prime}(y)=0
$$
 令  $y=1$  解得  $x^{\prime \prime}(1)=-2$.  再对  $y$  求导得 
$$
\begin{aligned}
&-\cos x(y)\left(x^{\prime \prime}(y)\right)^{3}-2 \sin x(y) x^{\prime}(y) x^{\prime \prime}(y)-\sin x(y) x^{\prime}(y) x^{\prime \prime}(y) \\
&+\cos x(y) x^{\prime \prime \prime}(y)+4 x^{\prime}(y) x^{\prime \prime}(y)+2 x^{\prime}(y) x^{\prime \prime}(y)+2 x(y) x^{\prime \prime \prime}(y)=0 .
\end{aligned}
$$
 令  $y=1$  解得  $x^{(3)}(1)=13$.  再对  $y$  求导得 
$$
\begin{gathered}
\sin x(y)\left(x^{\prime}(y)\right)^{4}-3 \cos x(y)\left(x^{\prime \prime}(y)\right)^{2} x^{\prime \prime}(y)-3\left(\cos x(y)\left(x^{\prime}(y)\right)^{2} x^{\prime \prime}(y)+\sin x(y)\left(x^{\prime}(y) x^{\prime \prime}(y)\right)^{\prime}\right) \\
-\sin x(y) x^{\prime}(y) x^{\prime \prime \prime}(y)+\cos x(y) x^{(4)}(y)+6\left(x^{\prime \prime}(y)\right)^{2}+6 x^{\prime \prime}(y) x^{(3)}(y)+2^{\prime}(y) x^{(3)}(y)+2 x(y) x^{(4)}(y)=0 .
\end{gathered}
$$
 令  $y=1$  解得  $x^{(4)}(1)=-140$.

\begin{enumerate}
  \setcounter{enumi}{3}
  \item  如果  $q(x, y)$  可微 ,  则  $\mathrm{d} q(x, y)=q_{x}(x, y) \mathrm{d} x+q_{y}(x, y) \mathrm{d} y$,  此时可以用第二型曲线积分来定义  $\int p(x, y) \mathrm{d} q(x, y)$.
\end{enumerate}
$4 .$
$$
\begin{aligned}
& \lim _{\Delta x \rightarrow 0^{+}} \frac{F(x+\Delta x)-F(x-\Delta x)}{2 \Delta x} \\
=& \lim _{\Delta x \rightarrow 0^{+}} \frac{1}{2 \Delta x} \int_{x-\Delta}^{x+\Delta} f(t) \mathrm{d} t \\
=& \lim _{\Delta x \rightarrow 0^{+}}\left(\frac{\int_{x}^{x+\Delta x} f(t) \mathrm{d} t}{2 \Delta x}+\frac{\int_{x-\Delta x}^{x} f(t) \mathrm{d} t}{2 \Delta x}\right) \\
=& \frac{f\left(x^{+}\right)+f\left(x^{-}\right)}{2}(\text { 利用积分第一中值定理 }) .
\end{aligned}
$$

\begin{enumerate}
  \setcounter{enumi}{5}
  \item Lagrange  中值定理  :  设  $f(x)$  在  $[a, b]$  上连续 ,  在  $(a, b)$  上可微 ,  则存在  $\xi \in(a, b)$  使得  $\frac{f(b)-f(a)}{b-a}=f^{\prime}(\xi)$.  令  $F(x)=f(x)-f(a)-\frac{f(b)-f(a)}{b-a}(x-a)$,  则  $F(x)$  在  $[a, b]$  上连续 ,  在  $(a, b)$  上可微 , $F(a)=F(b)=0$.
\end{enumerate}
 由  Rolle  定理知  $\exists \xi \in(a, b)$  使得  $f(\xi)=0$,  于是  $\frac{f(b)-f(a)}{b-a}=f^{\prime}(\xi)$.

 应用 :  设  $f(x)$  在  $\mathbb{R}$  上连续可微且  $f^{\prime}(x)$  一致有界 ,  则  $f(x)$  在  $\mathbb{R}$  上一致连续 .

\begin{enumerate}
  \setcounter{enumi}{6}
  \item $f_{n}(x)$  单调有界 , $\int_{0}^{+\infty} g(x) \mathrm{d} x$  收敛 ,  由  Abel  判别法知  $\int_{0}^{+\infty} f_{n}(x) g(x) \mathrm{d} x$  收敛 ,  即  $f_{n}(x) g(x)$  在  $[0,+\infty)$  上   可积 .  因为  $\lim _{n \rightarrow \infty} f_{n}(x)=f(x)$  而  $f_{n}(x)$  在  $[0,+\infty)$  上一致有界 ,  于是  $f(x)$  在  $[a,+\infty)$  上一致有界 .  假设  $f(x)$  在  $[0,+\infty)$  上不单调 ,  则  $\exists x_{1}<x_{2}<x_{3}, f\left(x_{1}\right)<f\left(x_{2}\right), f\left(x_{2}\right)>f\left(x_{3}\right)$  或  $f\left(x_{1}\right)>f\left(x_{2}\right), f\left(x_{2}\right)<f\left(x_{3}\right)$,  若  $f\left(x_{1}\right)<f\left(x_{2}\right), f\left(x_{2}\right)>f\left(x_{3}\right)$,  则  $\exists N$,  当  $n>N$  时 , $f_{n}\left(x_{1}\right)<f_{n}\left(x_{2}\right), f_{n}\left(x_{2}\right)>f_{n}\left(x_{3}\right)$  与  $f_{n}(x)$  的单调性矛   盾 ,  另一种情况是类似的 ,  从而  $f_{n}(x)$  必为单调函数 ,  从而  $f(x) g(x)$  在  $[0,+\infty)$  上可积 .  设  $\left|f_{n}(x)\right| \leqslant M, \forall x \in[0,+\infty), \forall n \in \mathbb{N}$.  由于  $f(x) g(x)$  与  $g(x)$  在  $[0,+\infty)$  上可积 ,  于是  $\forall \varepsilon>0, \exists A_{0}>0$,  当  $A_{2}>A_{1} \geqslant A_{0}$  时 ,
\end{enumerate}
$$
\left|\int_{A_{1}}^{A_{2}} f(x) g(x) \mathrm{d} x\right|<\frac{\varepsilon}{4}, \quad\left|\int_{A_{1}}^{A_{2}} g(x) \mathrm{d} x\right|<\frac{\varepsilon}{4 M}
$$
 对于上述  $A_{0}, \exists N>0$,  当  $n>N$  时 ,
$$
\left|\int_{0}^{A_{0}}\left(f_{n}(x)-f(x)\right) g(x) \mathrm{d} x\right|<\frac{\varepsilon}{4}
$$
 于是 
$$
\begin{aligned}
&\left|\int_{0}^{+\infty} f_{n}(x) g(x) \mathrm{d} x-\int_{0}^{+\infty} f(x) g(x) \mathrm{d} x\right| \\
=&\left|\int_{0}^{+\infty}\left(f_{n}(x)-f(x)\right) g(x) \mathrm{d} x\right| \\
\leqslant &\left|\int_{0}^{A_{0}}\left(f_{n}(x)-f(x)\right) g(x) \mathrm{d} x\right|+\left|\int_{A_{0}}^{+\infty} f_{n}(x) g(x) \mathrm{d} x\right|+\mid \int_{A_{0}}^{+\infty} f(x) g(x) \mathrm{d} x \\
&<\frac{\varepsilon}{4}+\left|f_{n}\left(A_{0}\right) \int_{A_{0}}^{\xi} g(x) \mathrm{d} x\right|+\left|f_{n}(+\infty) \int_{\xi}^{+\infty} g(x) \mathrm{d} x\right|+\frac{\varepsilon}{4} \\
\leqslant & \varepsilon
\end{aligned}
$$
 注   上述证明中用到了广义积分第二中值定理 ,  见于谢惠民等人编的 《 数学分析习题课讲义 》 上姗第  400  页第   二组参考题  13 .

\begin{enumerate}
  \setcounter{enumi}{7}
  \item  由题设  $\forall \varepsilon>0, \exists \delta>0$,  当  $x_{1}, x_{2} \in[a, b],\left|x_{1}-x_{2}\right|<\delta$  时 ,  有 
\end{enumerate}
$$
\left|f\left(x_{1}, y\right)-f\left(x_{2}, y\right)\right|<\varepsilon, \quad \forall y \in[c, d]
$$
 对  $[a, b]$  进行等距划分  $P: a=x_{0}<x_{1}<\cdots<x_{n}=b$,  并且让间距  $\Delta x_{i}=x_{i}-x_{i-1}=\frac{b-a}{n}<\delta$,  则  $\forall \xi_{i}, \eta_{i} \in\left[x_{i-1}, x_{i}\right]$,
$$
\left|\sum_{i=1}^{n} p\left(\xi_{i}\right) \Delta x_{i}-\sum_{i=1}^{n} p\left(\eta_{i}\right) \Delta x_{i}\right|=\left|\sum_{i=1}^{n} \Delta x_{i} \int_{c}^{d}\left(f\left(\xi_{i}, y\right)-f\left(\eta_{i}, y\right)\right) \mathrm{d} y\right| \leqslant \varepsilon(d-c)(b-a),
$$
 取上下确界 ,  得  $\left|\sum_{i=1}^{n} M_{i} \Delta x_{i}-\sum_{i=1}^{n} m_{i} \Delta x_{i}\right| \leqslant \varepsilon(d-c)(b-a)$,  由  Darboux  定理知  $p(x)$  在  $[a, b]$  上可积 .

 由于  $p(x)$  在  $[a, b]$  上可积 , $\forall \varepsilon>0, \exists \delta_{1}>0$,  对于任意满足最大间距  $\left\|P_{x}\right\|<\delta_{1}$  的每个  $[a, b]$  的分划  $P_{x}$  以   及从属于  $P_{x}$  的每个介点集  $\left\{\xi_{i}\right\}$,
$$
\left|\sum_{i=1}^{n} \int_{c}^{d} f\left(\xi_{i}, y\right) \mathrm{d} y \Delta x_{i}-\int_{a}^{b} \mathrm{~d} x \int_{c}^{d} f(x, y) \mathrm{d} y\right|<\varepsilon
$$
 取定一个满足上述条件的分划  $P_{x}$  及介点集  $\left\{\xi_{i}\right\}$,  对于上述的  $\varepsilon>0$,  由于  $f\left(\xi_{i}, y\right)$  在  $[c, d]$  上可积 , $\exists \delta_{2}>0$,  对于任意满足最大间距  $\left\|P_{y}\right\|<\delta_{2}$  的每个  $[c, d]$  的分划  $P_{y}$  以及从属于  $P_{y}$  的每个介点集  $\left\{\eta_{j}\right\}$,
$$
\left|\sum_{j=1}^{m} f\left(\xi_{i}, \eta_{j}\right) \Delta y_{j}-\int_{c}^{d} f\left(\xi_{i}, y\right) \mathrm{d} y\right|<\varepsilon, \quad i=1,2, \ldots, n .
$$
 此时 
$$
\begin{aligned}
\sum_{j=1}^{m} q\left(\eta_{j}\right) \Delta y_{j} &=\sum_{j=1}^{m} \sum_{i=1}^{n} \int_{x_{i-1}}^{x_{i}} f\left(x, \eta_{j}\right) \mathrm{d} x \Delta y_{j} \\
&=\sum_{j=1}^{m} \sum_{i=1}^{n} f\left(\zeta_{i}, \eta_{j}\right) \Delta x_{i} \Delta y_{j} \\
&=\sum_{i=1}^{n}\left(\sum_{j=1}^{m} f\left(\zeta_{i}, \eta_{j}\right) \Delta y_{j}\right) \Delta x_{i}
\end{aligned}
$$
 于是 
$$
\begin{aligned}
&\left|\sum_{j=1}^{m} q\left(\eta_{j}\right) \Delta y_{j}-\int_{a}^{b} \mathrm{~d} x \int_{c}^{d} f(x, y) \mathrm{d} y\right| \\
\leqslant &\left.\mid \sum_{i=1}^{n}\left(\sum_{j=1}^{m} f\left(\zeta_{i}, \eta_{j}\right) \Delta y_{j}\right) \Delta x_{i}-\sum_{i=1}^{n}\left(\sum_{j=1}^{m} f\left(\xi_{i}, \eta_{j}\right) \Delta y_{j}\right) \Delta x_{i}\right) \Delta x_{i}-\sum_{i=1}^{n}\left(\int_{c}^{d} f\left(\xi_{i}, y\right) \mathrm{d} y\right) \Delta x_{i} \\
&+\left|\sum_{i=1}^{n}\left(\sum_{j=1}^{m} f\left(\xi_{i}, \eta_{j}\right) \Delta y_{j}\right) \Delta \int_{a}^{b} \mathrm{~d} x \int_{c}^{d} f(x, y) \mathrm{d} y\right| \\
&+\left|\sum_{i=1}^{n}\left(\int_{c}^{d} f\left(\xi_{i}, y\right) \mathrm{d} y\right) \Delta x_{i}-\int^{\prime}\right| \\
<& \varepsilon(b-a)(d-c)+\varepsilon(b-a)+\varepsilon \\
=&((b-a)(d-c)+(b-a)+1) \varepsilon
\end{aligned}
$$
 因此 
$$
\int_{c}^{d} q(y) \mathrm{d} y=\int_{a}^{b} \mathrm{~d} x \int_{c}^{d} f(x, y) \mathrm{d} y=\int_{a}^{b} p(x) \mathrm{d} x
$$

\begin{enumerate}
  \setcounter{enumi}{8}
  \item  因为  $\lim _{n \rightarrow \infty} b_{n}=0$  故  $\left\{b_{n}\right\}$  有界 ,  设  $\left|b_{n}\right| \leqslant M$. $\forall \varepsilon>0, \exists N_{1}$,  当  $n>m>N_{1}$  时 , $\left|\sum_{k=m}^{n} a_{k}\right|<\frac{\varepsilon}{2 M}$,
\end{enumerate}
$$
\left|a_{1} b_{n}+\cdots+a_{n} b_{1}\right| \leqslant\left|a_{1} b_{n}+\cdots+a_{N_{1}} b_{n-N_{1}+1}\right|+\left|a_{N_{1}+1} b_{n-N_{1}}+\cdots+a_{n} b_{1}\right|
$$
 固定上述  $N_{1}$,  因为  $\lim _{n \rightarrow \infty}\left(a_{1} b_{n}+\cdots+a_{N_{1}} b_{n-N_{1}+1}\right)=0$,  于是  $\exists N>N_{1}>0$,  当  $n>N$  时 ,
$$
\begin{gathered}
\left|a_{1} b_{n}+\cdots+a_{N_{1}} b_{n-N_{1}+1}\right|<\frac{\varepsilon}{2} \\
\left|a_{1} b_{n}+\cdots+a_{n} b_{1}\right|<\frac{\varepsilon}{2}+\frac{\varepsilon}{2 M} M=\varepsilon
\end{gathered}
$$
$9 .$
$$
\begin{aligned}
I &=\int_{0}^{1} \mathrm{~d} x \int_{0}^{x^{2}} x y \mathrm{~d} y+\int_{1}^{2} \mathrm{~d} x \int_{0}^{2-x} x y \mathrm{~d} y \\
&=\int_{0}^{1} \mathrm{~d} y \int_{\sqrt{y}}^{2-y} x y \mathrm{~d} x .
\end{aligned}
$$
$10 .$  北京大学  2014  年全国硕士研究生招生考试数学分析试题及解答     

2019.04.06

\begin{enumerate}
  \item  叙述实数序列  $\left\{x_{n}\right\}$  的  Cauchy  收玫原理 ,  并且使用  Bolzano-Weierstrass  定理证明 .

  \item  序列  $\left\{x_{n}\right\}$  满足  $x_{1}=1, x_{n+1}=\sqrt{4+3 x_{n}}, \mathrm{n}=1,2, \ldots$.  证明此序列收玫并求极限 .

  \item  计算 

\end{enumerate}
$$
\iiint_{\Omega} f(x, y, z) \mathrm{d} x \mathrm{~d} y \mathrm{~d} z
$$
 其中  $\Omega$  是曲面  $z=\sqrt{x^{2}+y^{2}}$  与  $z=1$  围成的有界区域 .

\begin{enumerate}
  \setcounter{enumi}{4}
  \item  证明函数项级数  $\sum_{n=1}^{+\infty} x^{3} \mathrm{e}^{-n x^{2}}$  在  $[0,+\infty)$  一致收玫 .

  \item  讨论级数  $\sum_{n=3}^{+\infty} \ln \cos \left(\frac{\pi}{n}\right)$  的玫散性 .

  \item  设函数  $f: \mathbb{R}^{n} \rightarrow \mathbb{R}$  在  $\mathbb{R}^{n} \backslash\{\mathbf{0}\}$  可微 ,  在  $\mathbf{0}$  点连续 ,  且  $\lim _{\mathbf{p} \rightarrow \mathbf{0}} \frac{\partial f(\mathbf{p})}{\partial x_{i}}=0, i=1,2, \cdots, n$,  证明  $f$  在  $\mathbf{0}$  处可微 .

  \item  设  $f(x), g(x)$  是  $[0,1]$  上的连续函数 ,  且  $\sup _{x \in[0,1]} f(x)=\sup _{x \in[0,1]} g(x)$.  证明存在  $x_{0} \in[0,1]$,  使得  $\mathrm{e}^{f\left(x_{0}\right)}+3 f\left(x_{0}\right)=$ $\mathrm{e}^{g\left(x_{0}\right)}+3 g\left(x_{0}\right) .$

  \item  记  $\Omega=\left\{\mathbf{p} \in \mathbb{R}^{3}|| \mathbf{p} \mid \leq 1\right\}$,  设  $V: \mathbb{R}^{3} \rightarrow \mathbb{R}^{3}, V=\left(V_{1}, V_{2}, V_{3}\right)$  是  $C^{1}$  向量场 , $V$  在  $\mathbb{R}^{3} \backslash \Omega$  上恒为  0 , $\frac{\partial V_{1}}{\partial x}+\frac{\partial V_{2}}{\partial y}+\frac{\partial V_{3}}{\partial z}$  在  $\mathbb{R}^{3}$  恒为  $0 .$

\end{enumerate}
(1)  设  $f: \mathbb{R}^{3} \rightarrow \mathbb{R}$  是  $C^{1}$  函数 ,  求  $\iiint_{\Omega} \nabla f \cdot V \mathrm{~d} x \mathrm{~d} y \mathrm{~d} z$;

(2)  求  $\iiint_{\Omega} V_{1} \mathrm{~d} x \mathrm{~d} y \mathrm{~d} z$.

\begin{enumerate}
  \setcounter{enumi}{9}
  \item  设  $f: \mathbb{R} \rightarrow \mathbb{R}$  是有界连续函数 ,  求 
\end{enumerate}
$$
\lim _{t \rightarrow 0^{+}} \int_{-\infty}^{+\infty} f(x) \frac{t}{t^{2}+x^{2}} \mathrm{~d} x .
$$

\begin{enumerate}
  \setcounter{enumi}{10}
  \item $f:[0,1] \rightarrow[0,1]$  是  $C^{2}$  函数 , $f(0)=f(1)=0$,  且  $f^{\prime \prime}(x)<0, \forall x \in[0,1]$.  记曲线  $\{(x, f(x)) \mid x \in[0,1]\}$  的   弧长是  $L$.  证明  $L<3$. 1.  实数列的  Cauchy  收敛原理 : $\left\{x_{n}\right\}$  为  $\mathbb{R}$  上的数列 ,  若  $\forall \varepsilon>0, \exists N>0$,  当  $n, m>N$  时均有  $\left|x_{n}-x_{m}\right|<\varepsilon$,  则极限  $\lim _{n \rightarrow+\infty} x_{n}$  存在 .
\end{enumerate}
 证明 :  固定  $\varepsilon=1, \exists N_{1}>0$,  当  $n>N_{1}$  时有  $\left|x_{n}\right| \leq\left|x_{n}-x_{N_{1}+1}\right|+\left|x_{N_{1}+1}\right|<1+\left|x_{N_{1}+1}\right|$.  令  $M=$ $\max \left\{\left|x_{1}\right|, \cdots,\left|x_{N 1}\right|,\left|x_{N_{1}+1}\right|+1\right\}$,  则  $M$  为  $\left|x_{n}\right|$  的上界 ,  于是  $\left\{x_{n}\right\}$  为  $\mathbb{R}$  上有界数列 .  由  Weierstrass  定   理知必有收敛子列  $\left\{x_{n_{k}}\right\}$.  设  $\lim _{k \rightarrow+\infty} x_{n_{k}}=a$,  于是对  $\forall \varepsilon>0, \exists k_{1}>0$,  当  $k>k_{1}$  时 , $\left|x_{n_{k}}-a\right|<\varepsilon$  又由于对于上述取定的  $\varepsilon>0, \exists N_{1}$,  当  $n, m>N_{1}$  时 , $\left|x_{n}-x_{m}\right|<\varepsilon$.  从而当  $n>\max \left\{K_{1}, N_{1}\right\}$  时 , $\left|x_{n}-a\right| \leq\left|x_{n}-x_{n_{k}}\right|+\left|x_{n_{k}}-a\right|<2 \varepsilon$,  于是  $\lim _{n \rightarrow+\infty} x_{n}=a$.

\begin{enumerate}
  \setcounter{enumi}{2}
  \item $x_{2}=\sqrt{7}>x_{1}$.  设  $x_{n}>x_{n-1}$,  则  $x_{n+1}=\sqrt{4+3 x_{n}}>\sqrt{4+3 x_{n-1}}=x_{n}$.  由数学归纳法知  $\left\{x_{n}\right\}$  单调递   增 .  又  $1 \leq x_{1}<4$.  设  $1 \leq x_{n}<4$,  则  $1 \leq x_{n+1}<4$,  从而  $\left\{x_{n}\right\}$  有界 ,  从而  $\lim _{n \rightarrow \infty} x_{n}$  存在 ,  设其为  $a$,  则  $a=\sqrt{4+3 a} \Longrightarrow a=4$.

  \item  原式  $=\int_{0}^{1} \mathrm{~d} z \int_{x^{2}+y^{2} \leq z^{2}} \sqrt{x^{2}+y^{2}} \mathrm{~d} x \mathrm{~d} y=\int_{0}^{1} \mathrm{~d} z \int_{0}^{2 \pi} \mathrm{d} \theta \int_{0}^{z} r^{2} \mathrm{~d} r=\frac{\pi}{6}$.

  \item  令  $h(x)=x^{3} \mathrm{e}^{-n x^{2}}$,  则  $h^{\prime}(x)=3 x^{2} \mathrm{e}^{-n x^{2}}+x^{3} \mathrm{e}^{-n x^{2}}(-2 n x)=x^{2} \mathrm{e}^{-n x^{2}}\left(3-2 n x^{2}\right)$.  当  $x \in\left(0\right.$, $\left.\sqrt{\frac{3}{2 n}}\right)$  时 , $h^{\prime}(x)>0, h(x)$  单调递增 .  当  $x \in\left(\sqrt{\frac{3}{2 n}},+\infty\right)$  时 , $h^{\prime}(x)<0, h(x)$  单调递减 .  故  $0 \leq h(x) \leq\left(\frac{3}{2 n}\right)^{\frac{3}{2}} \mathrm{e}^{-\frac{3}{2}}$.  从而  $\sum_{n=1}^{\infty} x^{3} \mathrm{e}^{-n x^{2}} \leq \sum_{n=1}^{+\infty} \mathrm{e}^{-\frac{3}{2}}\left(\frac{3}{2 n}\right)^{\frac{3}{2}}$.  后面一个级数是收敛的从而原级数在  $[0,+\infty)$  一致收敛 .

  \item  首先注意到 

\end{enumerate}
$$
0<-\ln \cos \left(\frac{\pi}{n}\right)=-\ln \left(1-2 \sin ^{2}\left(\frac{\pi}{2 n}\right)\right) \sim 2 \sin ^{2}\left(\frac{\pi}{2 n}\right) \sim \frac{\pi^{2}}{2 n^{2}} \quad(n \rightarrow+\infty)
$$
 又由于  $\sum_{n=3}^{+\infty} \frac{\pi^{2}}{2 n^{2}}$  收敛 ,  故  $\sum_{n=3}^{+\infty} \ln \cos \left(\frac{\pi}{n}\right)$  绝对收敛 .

\begin{enumerate}
  \setcounter{enumi}{6}
  \item  由导函数极限定理知  $\left.\frac{\partial f}{\partial x_{i}}\right|_{x=0}=0, i=1,2, \cdots, n$.  要证  $f(x)$  在  $x=0$  处可微 ,  只需证 
\end{enumerate}
$$
\lim _{\Delta x \rightarrow 0} \frac{f(\Delta x)-f(0)}{|\Delta x|}=0 .
$$
Пा

 当  $\Delta x \rightarrow 0$  时 , $\Delta x^{i} \rightarrow 0, i=1,2, \cdots, n$
$$
\Longrightarrow \lim _{\Delta x \rightarrow 0} \frac{f(\Delta x)-f(0)}{|\Delta x|}=0 .
$$
$$
\begin{aligned}
& \frac{f\left(\Delta x^{1}, \Delta x^{2}, \cdots, \Delta x^{n}\right)-f(0,0, \cdots, 0)}{\sqrt{\left(\Delta x^{1}\right)^{2}+\left(\Delta x^{2}\right)^{2}+\cdots+\left(\Delta x^{n}\right)^{2}}}=\frac{f\left(\Delta x^{1}, \Delta x^{2}, \cdots, \Delta x^{n}\right)-f\left(\Delta x^{1}, \Delta x^{2}, \cdots, \Delta x^{n-1}, 0\right)}{|\Delta x|} \\
& +\frac{f\left(\Delta x^{1}, \Delta x^{2}, \cdots, \Delta x^{n-1}, 0\right)-f\left(\Delta x^{1}, \Delta x^{2}, \cdots, \Delta x^{n-2}, 0,0\right)}{|\Delta x|} \\
& +\cdots \\
& +\frac{f\left(\Delta x^{1}, 0, \cdots, 0\right)-f(0, \cdots, 0)}{|\Delta x|} \\
& =\frac{\frac{\partial f}{\partial x_{n}}\left(\Delta x^{1}, \cdots, \Delta x^{n-1}, \theta_{n} \Delta x^{n}\right)}{|\Delta x|} \Delta x^{n} \\
& +\cdots \\
& +\frac{\frac{\partial f}{\partial x_{1}}\left(\theta_{1} \Delta x^{1}, 0, \cdots, 0\right)}{|\Delta x|} \Delta x^{1} . 
\end{aligned}
$$

\begin{enumerate}
  \setcounter{enumi}{7}
  \item  令  $\varphi(x)=\mathrm{e}^{f(x)}+3 f(x)-\mathrm{e}^{g(x)}-3 g(x)$.  设  $f\left(x_{1}\right)=\sup _{x \in[0,1]} f(x)=\sup _{x \in[0,1]} g(x)=g\left(x_{2}\right)$.  若  $x_{1}=x_{2}$.  则取  $x_{0}=x_{1}$  即可 .  若  $x_{1}<x_{2}$.  则 
\end{enumerate}
$$
\begin{gathered}
\mathrm{e}^{f\left(x_{1}\right)}+3 f\left(x_{1}\right)=\mathrm{e}^{g\left(x_{2}\right)}+3 g\left(x_{2}\right) \geq \mathrm{e}^{g\left(x_{1}\right)}+3 g\left(x_{1}\right) \geq \mathrm{e}^{f\left(x_{2}\right)}+3 f\left(x_{2}\right) . \\
\mathrm{e}\left(x_{1}\right) \geq 0, \mathrm{e}\left(x_{2}\right) \leq 0, \exists x_{0} \in\left[x_{1}, x_{2}\right] \text { 使得 } \varphi\left(x_{0}\right)=0 .
\end{gathered}
$$
 当  $x_{1}>x_{2}$  时是类似的 .

\begin{enumerate}
  \setcounter{enumi}{8}
  \item (1)  由连续性知  $V$  在  $\partial \Omega$  上为  0 ,  故 
\end{enumerate}
$$
\begin{aligned}
\iint_{\Omega} \nabla f \cdot V \mathrm{~d} x \mathrm{~d} y \mathrm{~d} z & \equiv \iint_{\Omega}\left(\frac{\partial f}{\partial x} V_{1}+\frac{\partial f}{\partial y} V_{2}+\frac{\partial f}{\partial z} V_{3}\right) \mathrm{d} x \mathrm{~d} y \mathrm{~d} z \\
0 &=\iint_{\partial \Omega} f V_{1} \mathrm{~d} y \mathrm{~d} z+f V_{2} \mathrm{~d} z \mathrm{~d} x+f V_{3} \mathrm{~d} x \mathrm{~d} y \\
&=\iint_{\Omega}\left(\frac{\partial f}{\partial x} V_{1}+f \frac{\partial V_{1}}{\partial x}+\frac{\partial f}{\partial y} V_{2}+f \frac{\partial V_{2}}{\partial y}+\frac{\partial f}{\partial z} V_{3}+f \frac{\partial V_{3}}{\partial z}\right) \mathrm{d} x \mathrm{~d} y \mathrm{~d} z \\
&=\iint_{\Omega}\left(\frac{\partial f}{\partial x} V_{1}+\frac{\partial f}{\partial y} V_{2}+\frac{\partial f}{\partial z} V_{3}\right) \mathrm{d} x \mathrm{~d} y \mathrm{~d} z
\end{aligned}
$$
(2)  取  $f(x, y, z)=x$,  由  (1)  知结果为  0 .

\begin{enumerate}
  \setcounter{enumi}{9}
  \item  设  $|f(x) \leq M<+\infty|$.  因  $f(x)$  在  $\mathbb{R}$  上连续 ,  故  $\forall \varepsilon>0, \exists \delta>0$.  当  $|x|<\delta$  时  $|f(x)-f(0)|<\varepsilon$.  于是 
\end{enumerate}
$$
\begin{aligned}
\left|\int_{-\infty}^{+\infty}(f(x)-f(0)) \frac{t}{t^{2}+x^{2}} \mathrm{~d} x\right| & \leq\left|\int_{-\delta}^{\delta} \frac{t}{t^{2}+x^{2}} \mathrm{~d} x\right| \varepsilon+2 M\left|\int_{-\infty}^{-\delta} \frac{t}{t^{2}+x^{2}} \mathrm{~d} x+\int_{\delta}^{+\infty} \frac{t}{t^{2}+x^{2}} \mathrm{~d} x\right| \\
&=2 \varepsilon \arctan \frac{\delta}{t}+2 M\left|\left(\left(\frac{\pi}{2}-\arctan \frac{\delta}{t}\right)+\left(\arctan \left(-\frac{\delta}{t}\right)+\frac{\pi}{2}\right)\right)\right|
\end{aligned}
$$
 对于取定的  $\delta>0$,  令  $t \rightarrow 0^{+}$,  则上式右端趋于  $\pi \varepsilon$.  故 
$$
\lim _{t \rightarrow 0^{+}} \int_{-\infty}^{+\infty}(f(x)-f(0)) \frac{t}{t^{2}+x^{2}} \mathrm{~d} x=0 \Longrightarrow \lim _{t \rightarrow 0^{+}} \int_{-\infty}^{+\infty} f(x) \frac{t}{t^{2}+x^{2}} \mathrm{~d} x=\lim _{t \rightarrow 0^{+}}^{+\infty} \int_{-\infty}^{+\infty} f(0) \frac{t}{t^{2}+x^{2}} \mathrm{~d} x=f(0) \pi
$$
$10 .$
$$
L=\int_{0}^{1} \sqrt{1+\left(f^{\prime}(x)\right)^{2}} \mathrm{~d} x, f^{\prime \prime}(x)<0 .
$$
 于是  $f^{\prime}(x)$  在  $[0,1]$  严格单调递减 ,  由  $f(0)=f(1)=0$  知存在唯一  $x_{0} \in(0,1)$  使得  $f^{\prime}\left(x_{0}\right)=0$.  于是  $f(x)$  在  $\left(0, x_{0}\right)$  单调递增 ,  在  $\left(x_{0}, 1\right)$  单调递减 .  从而  $f\left(x_{0}\right)$  为最大值 ,  应有  $f\left(x_{0}\right) \leq 1$.
$$
\begin{aligned}
\int_{0}^{x_{0}} \sqrt{1+\left(f^{\prime}(x)\right)^{2}} \mathrm{~d} x &=\int_{0}^{x_{0}} f^{\prime}(x) \mathrm{d} x+\int_{0}^{x_{0}} \sqrt{1+\left(f^{\prime}(x)\right)^{2}}-f^{\prime}(x) \mathrm{d} x \\
&=f\left(x_{0}\right)+\int_{0}^{x_{0}} \frac{\mathrm{d} x}{\sqrt{1+\left(f^{\prime}(x)\right)^{2}}+f^{\prime}(x)} \\
\int_{x_{0}}^{1} \sqrt{1+\left(f^{\prime}(x)\right)^{2}} \mathrm{~d} x &=\int_{x_{0}}^{1}-f^{\prime}(x) \mathrm{d} x+\int_{x_{0}}^{1} \frac{\sqrt{1+\left(f^{\prime}(x)\right)^{2}}-f^{\prime}(x)}{1} \mathrm{~d} x \\
\int_{0}^{1} \sqrt{1+\left(f^{\prime}(x)\right)^{2}} \mathrm{~d} x &<2 f\left(x_{0}\right)+1 \leq 3
\end{aligned}
$$
 北京大学  2015  年全国硕士研究生招生考试数学分析试题及解答     

2019.04.18

\begin{enumerate}
  \item (15  分 )  计算  $\lim _{x \rightarrow 0^{+}} \frac{\int_{0}^{x} \mathrm{e}^{-t^{2}} \mathrm{~d} t-x}{\sin x-x}$.

  \item (15  分 )  讨论积分  $\int_{1}^{+\infty}\left[\ln \left(1+\frac{1}{x}\right)-\sin \frac{1}{x}\right] \mathrm{d} x$  的玫散性 .

  \item (15  分 )  函数  $f(x, y)=\left\{\begin{array}{ll}\left(1-\cos \frac{x^{2}}{y}\right) \sqrt{x^{2}+y^{2}}, & y \neq 0 ; \\ 0, & y=0 .\end{array} f(x, y)\right.$  在  $(0,0)$  是否可微 ?  说明理由 .

  \item (15  分 )  计算  $\int_{L} \mathrm{e}^{x}[(1-\cos y) \mathrm{d} x-(y-\sin y) \mathrm{d} y]$,  这里  $L$  是曲线  $y=\sin x$  上从  $(0,0)$  到  $(\pi, 0)$  那一段 .

  \item (15  分 )  证明函数项级数  $\sum_{n=0}^{\infty} \frac{\cos n x}{n^{2}+1}$  在  $(0,2 \pi)$  一致收玫 ,  并且在  $(0,2 \pi)$  有连续导数 .

  \item (15  分 ) $x_{0}=1, x_{n+1}=\frac{3+2 x_{n}}{3+x_{n}}, n \geqslant 0$.  证明序列  $\left\{x_{n}\right\}$  收玫并求其极限 .

  \item (15  分 )  函数  $f \in C^{2}\left(\mathbb{R}^{2}\right)$,  且对于任意  $(x, y) \in \mathbb{R}^{2}, \frac{\partial^{2} f}{\partial x^{2}}(x, y)+\frac{\partial^{2} f}{\partial y^{2}}(x, y)>0$.  证明 : $f$  没有极大值点 .

  \item (15  分 ) $f$  在  $[a, b]$  连续 ,  在  $(a, b)$  可导 ,  且  $f(b)>f(a), c=\frac{f(b)-f(a)}{b-a}$.  证明 : $f$  必具备下述两条性质中的   一个 :

\end{enumerate}
(1)  任意  $x \in[a, b]$,  有  $f(x)-f(a)=c(x-a)$;

(2)  存在  $\xi \in(a, b)$,  使得  $f^{\prime}(\xi)>c$.

\begin{enumerate}
  \setcounter{enumi}{9}
  \item (15  分 ) $\mathbf{F}: \mathbb{R}^{3} \rightarrow \mathbb{R}^{2}$  是  $C^{1}$  映射 , $\mathbf{F}\left(x_{0}\right)=y_{0}, x_{0} \in \mathbb{R}^{3}, y_{0} \in \mathbb{R}^{2}$,  且  $\mathbf{F}$  在  $x_{0}$  处的  Jacobi  矩阵  $\mathbf{D F}\left(x_{0}\right)$  的秩   为  2.  证明 :  存在  $\varepsilon>0$,  以及  $C^{1}$  映射  $\gamma(t):(-\varepsilon, \varepsilon) \rightarrow \mathbb{R}^{3}$,  使得  $\gamma^{\prime}(0)$  是非零向量 ,  且  $\mathbf{F}(\gamma(t))=y_{0}$.

  \item (15  分 ) $U \subseteq \mathbb{R}^{n}$  为开集 , $f: U \rightarrow \mathbb{R}^{n}$  是同肧映射 ,  且  $f$  在  $U$  上一致连续 .  证明 : $U=\mathbb{R}^{n}$. 1.  原式  $=\lim _{x \rightarrow 0^{+}} \frac{e^{-x^{2}}-1}{\cos x-1}=\lim _{x \rightarrow 0^{+}} \frac{-x^{2}}{-\frac{1}{2} x^{2}}=2$.

  \item  因为  $\lim _{x \rightarrow+\infty} \frac{\ln \left(1+\frac{1}{x}\right)-\sin \frac{1}{x}}{-\frac{1}{2 x^{2}}}=1$,  故  $\int_{1}^{+\infty}\left|\ln \left(1+\frac{1}{x}\right)-\sin \frac{1}{x}\right| \mathrm{d} x$  与  $\int_{1}^{+\infty} \frac{1}{2 x^{2}} \mathrm{~d} x$  同敛散 .  从而原广义积   分绝对收敛 .

\end{enumerate}
$3 .$
$$
\begin{gathered}
f_{x}(0,0)=\lim _{\Delta x \rightarrow 0} \frac{f(\Delta x, 0)-f(0,0)}{\Delta x}=0, \lim _{\Delta y \rightarrow 0} \frac{f(\Delta y, 0)-f(0,0)}{\Delta y}=0 \\
\frac{f(\Delta x, \Delta y)-f(0,0)-f_{x}(0,0) \Delta x-f_{y}(0,0) \Delta y}{\sqrt{\Delta x^{2}+\Delta y^{2}}}=1-\cos \frac{(\Delta x)^{2}}{\Delta y}
\end{gathered}
$$
 因为  $\lim _{(\Delta x, \Delta y) \rightarrow(0,0)}\left(1-\cos \frac{(\Delta x)^{2}}{\Delta y}\right)$  不存在 ,  故  $f(x, y)$  在点  $(0,0)$  处不可微 .  事实上  $\left(0, \frac{1}{n}\right) \rightarrow(0,0)$,  上式  $\rightarrow 0$. $\left(\frac{1}{\sqrt{n}}, \frac{1}{n}\right) \rightarrow(0,0)$,  上式  $\rightarrow 1-\cos 1 \neq 0 .$

\begin{enumerate}
  \setcounter{enumi}{4}
  \item  设  $L_{1}$  为从  $(\pi, 0)$  到  $(0,0)$  的直线段 ,  则 

  \item  因为  $\sum_{n=0}^{\infty}\left|\frac{\cos n x}{n^{2}+1}\right| \leqslant \sum_{n=1}^{\infty} \frac{1}{n^{2}+1}$,  从而  $\sum_{n=0}^{\infty} \frac{\cos n x}{n^{2}+1}$  在  $(0,2 \pi)$  一致收敛 .  考虑级数  $\sum_{n=0}^{\infty} \frac{-n \sin n x}{n^{2}+1}$.  对于固定   的  $x \in(0,2 \pi),\left|\sum_{n=0}^{N} \sin n x\right|=\left|\frac{\cos \left(N+\frac{1}{2}\right) x-\cos \frac{x}{2}}{-2 \sin \frac{x}{2}}\right| \leqslant \frac{2}{2 \sin \frac{x}{2}}$,  而  $\frac{n}{n^{2}+1}$  关于  $n$  单调减小趋于  0 .  从而  $\sum_{n=0}^{\infty} \frac{-n \sin n x}{n^{2}+1}$  在  $(0,2 \pi)$  上收敛 .  对于  $x_{0} \in(0,2 \pi), \exists \delta>0$  使得  $\left[x_{0}-\delta, x_{0}+\delta\right] \subseteq(0,2 \pi), \sum_{n=0}^{N}-\sin n x$  在  $x \in\left[x_{0}-\delta, x_{0}+\delta\right]$  上一致有界 ,  而  $\frac{n}{n^{2}+1}$  一致单调递减趋于  0 ,  从而  $\sum_{n=0}^{\infty} \frac{-n \sin n x}{n^{2}+1}$  在  $\left[x_{0}-\delta, x_{0}+\delta\right]$  上   一致收敛并且收敛到连续函数 ,  因此  $\left(\sum_{n=0}^{\infty} \frac{\cos n x}{n^{2}+1}\right)^{\prime}=\sum_{n=0}^{\infty} \frac{-n \sin n x}{n^{2}+1}, x \in\left[x_{0}-\delta, x_{0}+\delta\right]$.  由  $x_{0}$  的任意性知   原级数的和函数在  $(0,2 \pi)$  有连续导数 .

  \item $x_{1}=\frac{5}{4}>x_{0}>0$.  由数学归纳法易知  $\left\{x_{n}\right\}$  单调递增 .  再注意到  $x_{n+1}=\frac{3+2 x_{n}}{3+x_{n}}=2-\frac{3}{x_{n}+3}<2$,  故  $\left\{x_{n}\right\}$  有上界 ,  从而收敛 ,  设极限为  $a$,  则  $a=\frac{3+2 a}{3+a} \Longrightarrow a=\frac{-1+\sqrt{13}}{2}$.

  \item  设  $\left(x_{0}, y_{0}\right)$  为  $f(x, y)$  的极大值点 ,  则  $\frac{\partial f}{\partial x}\left(x_{0}, y_{0}\right)=0, \frac{\partial f}{\partial y}\left(x_{0}, y_{0}\right)=0$,

\end{enumerate}
$$
0 \geqslant f\left(x_{0}+\Delta x, y_{0}\right)-f\left(x_{0}, y_{0}\right)=\frac{\partial^{2} f}{\partial x^{2}}\left(x_{0}, y_{0}\right) \frac{(\Delta x)^{2}}{2}+o\left((\Delta x)^{2}\right)
$$
$$
\begin{aligned}
& \text { 原式 }=\int_{L+L_{1}}-\int_{L_{1}}=\int_{L+L_{1}}=-\iint-\mathrm{e}^{x}(y-\sin y)-\mathrm{e}^{x} \sin y \mathrm{~d} x \mathrm{~d} y=\iint \mathrm{e}^{x} y \mathrm{~d} x \mathrm{~d} y \\
& =\int_{0}^{\pi} \mathrm{d} x \int_{0}^{\sin x} \mathrm{e}^{x} y \mathrm{~d} y=\int_{0}^{\pi} \mathrm{e}^{x} \frac{1}{2} \sin ^{2} x \mathrm{~d} x=\int_{0}^{\pi} \frac{\mathrm{e}^{x}}{2} \frac{1-\cos 2 x}{2} \mathrm{~d} x=\frac{1}{4}\left(\mathrm{e}^{\pi}-1\right)-\frac{1}{4} \int_{0}^{\pi} \mathrm{e}^{x} \cos 2 x \mathrm{~d} x \\
& \int_{0}^{\pi} \mathrm{e}^{x} \cos 2 x \mathrm{~d} x=\int_{0}^{\pi} \cos 2 x \mathrm{de}^{x}=\left.\mathrm{e}^{x} \cos 2 x\right|_{0} ^{\pi}+2 \int_{0}^{\pi} \mathrm{e}^{x} \sin 2 x \mathrm{~d} x=-4 \int_{0}^{\pi} \mathrm{e}^{x} \cos 2 x \mathrm{~d} x+\mathrm{e}^{\pi}-1 \\
& \Longrightarrow \int_{0}^{\pi} \mathrm{e}^{x} \cos 2 x \mathrm{~d} x=\frac{\mathrm{e}^{\pi}-1}{5} \text {, 原式 }=\frac{\mathrm{e}^{\pi}-1}{5} \text {. } 
\end{aligned}
$$
 上式两边同除以  $(\Delta x)^{2}$,  令  $\Delta x \rightarrow 0$  则得  $\frac{\partial^{2} f}{\partial x^{2}}\left(x_{0}, y_{0}\right) \leqslant 0$,  同理有  $\frac{\partial^{2} f}{\partial y^{2}}\left(x_{0}, y_{0}\right) \leqslant 0$.  于是 
$$
\frac{\partial^{2} f}{\partial x^{2}}\left(x_{0}, y_{0}\right)+\frac{\partial^{2} f}{\partial y^{2}}\left(x_{0}, y_{0}\right) \leqslant 0
$$
 矛盾 .

\begin{enumerate}
  \setcounter{enumi}{8}
  \item (1)  若  $\exists \xi \in(a, b)$,  使得  $f^{\prime}(\xi)>c$,  则命题为真 .
\end{enumerate}
(2)  若  $\forall \xi \in(a, b)$  有  $f^{\prime}(\xi) \leqslant c$,  则要证明原命题只需证明  $f(x)=f(a)+\frac{f(b)-f(a)}{b-a}(x-a)=f(a)+c(x-a)$.  令  $\varphi(x)=f(x)-f(a)-c(x-a)$,  则  $\varphi^{\prime}(x) \leqslant 0$,  而  $\varphi(a)=\varphi(b)=0$,  若  $\exists \xi \in(a, b)$  使得  $\varphi(\xi)>0$  或  $\varphi(\xi)<0$.  则由  Lagrange  中值定理将找到  $\eta \in(a, b), \varphi(\eta)>0$,  矛盾 .  从而  $\varphi(x) \equiv 0, x \in[a, b]$.

\begin{enumerate}
  \setcounter{enumi}{9}
  \item  设  $F\left(x^{1}, x^{2}, x^{3}\right)=\left(\begin{array}{l}F^{1}\left(x^{1}, x^{2}, x^{3}\right) \\ F^{2}\left(x^{1}, x^{2}, x^{3}\right)\end{array}\right)$.  考虑方程组  $\left\{\begin{array}{l}F^{1}\left(x^{1}, x^{2}, x^{3}\right)-y_{0}^{1}=0 \\ F^{2}\left(x^{1}, x^{2}, x^{3}\right)-y_{0}^{2}=0\end{array}\right.$.  由题设知  $x_{0}$  满足上述方程   组 ,  而且  $F^{1}\left(x^{1}, x^{2}, x^{3}\right)-y_{0}^{1}$  与  $F^{2}\left(x^{1}, x^{2}, x^{3}\right)-y_{0}^{2}$  均在  $x_{0}$  附近连续并且偏导数连续 .  因为  $D F\left(x_{0}\right)$  的秩   为  2 ,  于是不妨设  $\frac{\partial\left(F^{1}, F^{2}\right)}{\partial\left(x^{2}, x^{3}\right)} \neq 0$.  则上述方程组在  $x_{0}$  附近唯一确定了一个隐函数  $\left\{\begin{array}{l}x^{2}=x^{2}\left(x^{1}\right) \\ x^{3}=x^{3}\left(x^{1}\right)\end{array}\right.$.  令  $\gamma(t)=\left(\begin{array}{c}x_{0}^{1}+t \\ x^{2}\left(x_{0}^{1}+t\right) \\ x^{3}\left(x_{0}^{1}+t\right)\end{array}\right)$  即可 .

  \item  若  $U$  是  $\mathbb{R}^{n}$  中的闭集 ,  则  $\mathbb{R}^{n}=U \cup\left(\mathbb{R}^{n} \backslash U\right)$.  按题意  $U \neq \varnothing$,  注意到  $\mathbb{R}^{n}$  是连通的 ,  故  $\mathbb{R}^{n} \backslash U=\varnothing \Rightarrow$ $\mathbb{R}^{n}=U$.  于是要证明原命题 ,  只需证明  $U$  是  $\mathbb{R}^{n}$  中的闭集 .  设  $\left\{x_{n}\right\}$  是  $U$  中任一数列且收敛到  $\mathbb{R}^{n}$  中一点  $x_{0}$,  下面只需证明  $x_{0} \in U$.  由一致连续的定义得 : $\forall \varepsilon>0, \exists \delta>0$,  当  $|x-y|<\delta$  时 , $|f(x)-f(y)|<\varepsilon$.  因为  $\lim _{n \rightarrow \infty} x_{n}=x_{0}$,  故  $\exists N>0$,  当  $m, n>N$  时 , $\left|x_{n}-x_{m}\right|<\delta \Longrightarrow\left|f\left(x_{n}\right)-f\left(x_{m}\right)\right|<\varepsilon$.  故  $\left\{f\left(x_{n}\right)\right\}$  为  $\mathbb{R}^{n}$  中  Cauchy  列 ,  可设  $\lim _{n \rightarrow \infty} f\left(x_{n}\right)=y_{0}$,  则 

\end{enumerate}
$$
\lim _{n \rightarrow \infty} x_{n}=\lim _{n \rightarrow \infty} f^{-1}\left(f\left(x_{n}\right)\right)=f^{-1}\left(\lim _{n \rightarrow \infty} f\left(x_{n}\right)\right)=f^{-1}\left(y_{0}\right) \Longrightarrow x_{0}=f^{-1}\left(y_{0}\right) \in U
$$
 北京大学  2016  年全国硕士研究生招生考试数学分析试题及解答     

2019.04.16

\begin{enumerate}
  \item (15  分 )  用开覆盖定理证明有界闭区间上的连续函数必一致连续 .

  \item (15  分 ) $f(x)$  是  $[a, b]$  上的实函数 .  叙述关于  Riemann  和  $\sum_{i=1}^{n} f\left(t_{i}\right)\left(x_{i}-x_{i-1}\right)$  的  Cauchy  准则  ( 不用证明 )  并   用你叙述的  Cauchy  准则证明闭区间上的单调函数可积 .

  \item (15  分 ) $(a, b)$  上的连续函数  $f(x)$  有反函数 .  证明反函数连续 .

  \item (15  分 ) $f\left(x_{1}, x_{2}, x_{3}\right)$  是  $C^{2}$  映射 , $\frac{\partial f}{\partial x_{1}}\left(x_{1}^{0}, x_{2}^{0}, x_{3}^{0}\right) \neq 0$.  证明  $f\left(x_{1}, x_{2}, x_{3}\right)=0$  在  $\left(x_{1}^{0}, x_{2}^{0}, x_{3}^{0}\right)$  附近确定了一   个隐函数  $x_{1}=x_{1}\left(x_{2}, x_{3}\right)$.  证明  $x_{1}=x_{1}\left(x_{2}, x_{3}\right)$  二次可微并求出  $\frac{\partial^{2} x_{1}}{\partial x_{2} \partial x_{3}}$  的表达式 .

  \item (15  分 ) $n \geqslant m, f: U \subseteq R^{n} \rightarrow R^{m}$  是  $C^{1}$  映射 , $U$  为开集且  $f$  的  Jacobi  矩阵的秩处处为  $m$.  证明  $f$  将  $U$  中的开集映为开集 .

  \item (15  分 ) $x_{1}=\sqrt{2}, x_{n+1}=\sqrt{2+x_{n}}$.  证明  $\left\{x_{n}\right\}$  收玫并求极限值 .

  \item (15  分 )  证明  $\int_{0}^{+\infty} \frac{\sin x}{x} \mathrm{~d} x$  收玫并求值 .  写出计算过程 .

  \item (15  分 )

\end{enumerate}
(1)  证明存在  $[a, b]$  上的多项式序列  $\left\{p_{n}(x)\right\}$  使得  $\int_{a}^{b} p_{i}(x) p_{j}(x) \mathrm{d} x=\delta_{i j}$  并使得对于  $[a, b]$  上的连续函数  $f(x)$,  若  $\int_{a}^{b} f(x) p_{n}(x) \mathrm{d} x=0, \forall n \in \mathbb{N}$,  则必有  $f \equiv 0$.

(2)  设  $g(x)$  在  $[a, b]$  平方可积 , $g(x)$  关于  (1)  中  $\left\{p_{n}(x)\right\}$  的展式为  $g(x) \sim \sum_{n=1}^{+\infty}\left(\int_{a}^{b} g(x) p_{n}(x) \mathrm{d} x\right) p_{n}(x)$.  问 
$$
\int_{a}^{b} g^{2}(x) \mathrm{d} x=\sum_{n=1}^{+\infty}\left[\int_{a}^{b} g(x) p_{n}(x) \mathrm{d} x\right]^{2}
$$
 是否成立 .

\begin{enumerate}
  \setcounter{enumi}{9}
  \item (15  分 )  正项级数  $\sum_{n=1}^{+\infty} a_{n}$  收玫 , $\lim _{n \rightarrow+\infty} b_{n}=0$,  令  $c_{n}=a_{1} b_{n}+a_{2} b_{n-1}+\cdots+a_{n} b_{1}$.  证明  $\left\{c_{n}\right\}$  收玫并求  $\lim _{n \rightarrow+\infty} c_{n}$.

  \item (15  分 )  幂级数  $\sum_{n=1}^{+\infty} a_{n} x^{n}$  的收敛半径为  $R, 0<R<+\infty$.  证明  $\sum_{n=1}^{+\infty} a_{n} R^{n}$  收敛的充分必要条件为  $\sum_{n=1}^{+\infty} a_{n} x^{n}$  在  $[0, R)$  上一致收敛 . 1.  设  $f(x)$  在  $[a, b]$  上连续 ,  则  $\forall x \in[a, b], \forall \varepsilon>0, \exists \delta_{x}>0$,  当  $|y-x|<\delta_{x}$  时 , $|f(y)-f(x)|<\frac{\varepsilon}{2}$.  因为  $\left\{\left(x-\frac{\delta_{x}}{2}, x+\frac{\delta_{x}}{2}\right) \mid x \in[a, b]\right\}$  构成  $[a, b]$  的一个开覆盖 ,  存在有限开覆盖 ,  设对应的点为  $x_{1}, x_{2}, \ldots, x_{n}$.  取  $\delta=\frac{1}{2} \min \left\{\delta_{1}, \delta_{2}, \ldots, \delta_{n}\right\}$,  则  $\forall x, y \in[a, b]$,  当  $|y-x|<\delta$  时 , $\exists i \in\{1,2, \ldots, n\}$,  使得  $x \in\left(x_{i}-\frac{\delta_{i}}{2}, x_{i}+\frac{\delta_{i}}{2}\right)$,  此时就有  $\left|y-x_{i}\right| \leqslant|y-x|+\left|x-x_{i}\right|<\delta_{i}$,  因此  $|f(x)-f(y)| \leqslant\left|f(x)-f\left(x_{i}\right)\right|+\left|f\left(x_{i}\right)-f(y)\right|<\varepsilon$.

\end{enumerate}
 注   证明有界闭区间上连续函数必定一致连续 ,  比较常见的做法是反证法用聚点定理 .  直接用紧集的定义来证这   个结论比较少见 ,  我是在  Walter Rudin  的  Principles of mathematical analysis  英文版第三版  90  页第一次   见到类似上述证明方法 .

\begin{enumerate}
  \setcounter{enumi}{2}
  \item  关于  Riemann  和的  Cauchy  收敛准则 :  有界函数  $f(x)$  在  $[a, b]$  上的  Riemann  和收敛的充分必要条件是  $\forall \varepsilon>0, \exists \delta>0$,  使得对于  $[a, b]$  的任意分割  $P_{1}: a=x_{0}<x_{1}<\cdots<x_{n}=b$  和  $P_{2}: a=y_{0}<y_{1}<\cdots<$ $y_{m}=b$,  以及任意介点  $\xi_{i} \in\left[x_{i-1}, x_{i}\right], \eta_{j} \in\left[y_{j-1}, y_{j}\right]$,  只要其相应的最大模  $\left\|P_{1}\right\|<\delta,\left\|P_{2}\right\|<\delta$,  就有 
\end{enumerate}
$$
\left|\sum_{i=1}^{n} f\left(\xi_{i}\right)\left(x_{i}-x_{i-1}\right)-\sum_{j=1}^{m} f\left(\eta_{j}\right)\left(y_{j}-y_{j-1}\right)\right|<\varepsilon .
$$
 不妨设  $f(x)$  是  $[a, b]$  的单调增函数 ,  下面证明  $f(x)$  在  $[a, b]$  上  Riemann  可积 .

 若  $P_{1}=P_{2}$,  则 
$$
\begin{aligned}
\left|\sum_{i=1}^{n} f\left(\xi_{i}\right)\left(x_{i}-x_{i-1}\right)-\sum_{j=1}^{m} f\left(\eta_{j}\right)\left(y_{j}-y_{j-1}\right)\right| & \leqslant\left|\sum_{i=1}^{n}\left(f\left(x_{i}\right)-f\left(x_{i-1}\right)\right)\left(x_{i}-x_{i-1}\right)\right| \\
& \leqslant(f(b)-f(a))\|P\|
\end{aligned}
$$
 于是  $\forall \varepsilon>0$,  令  $0<\delta<\frac{\varepsilon}{f(b)-f(a)}$  即可 .

 若  $P_{1} \neq P_{2}$,  则令  $P^{*}$  为把  $P_{1}, P_{2}$  的分点合在一起所得的分割 ,  此时 
$$
\begin{aligned}
\left|\sigma\left(P_{1}, f, \xi\right)-\sigma\left(P_{2}, f, \eta\right)\right| & \leqslant\left|\sigma\left(P_{1}, f, \xi\right)-\sigma\left(P^{*}, f, \xi^{*}\right)\right|+\left|\sigma\left(P^{*}, f, \xi^{*}\right)-\sigma\left(P_{2}, f, \eta\right)\right| \\
& \leqslant 2(f(b)-f(a))\|P\|
\end{aligned}
$$
 于是  $\forall \varepsilon>0$,  令  $0<\delta<\frac{\varepsilon}{2(f(b)-f(a))}$  即可 .

 注   前一部分是北京大学  2006  年数学分析试题中的第  7  题 .

\begin{enumerate}
  \setcounter{enumi}{3}
  \item  设  $f(x)$  的反函数为  $g(y)$,  要证明  $g(y)$  在  $f$  的值域  $R(f)$  上连续 .  因为  $f(x)$  在  $(a, b)$  上有反函数且连续 ,  故  $f(x)$  在  $(a, b)$  上是单调的 ,  不妨设为单调增 .  固定  $y_{0} \in R(f), \exists x_{0} \in(a, b)$,  使得  $y_{0}=f\left(x_{0}\right)$. $\forall \varepsilon>0$,  令  $y_{1}=f\left(x_{0}+\varepsilon\right), y_{2}=f\left(x_{0}-\varepsilon\right)$,  取  $\delta=\min \left\{\left|y_{0}-y_{1}\right|,\left|y_{0}-y_{2}\right|\right\}$,  则当  $\left|y-y_{0}\right|<\delta$  时 , $\left|g(y)-g\left(y_{0}\right)\right|<\varepsilon$.

  \item  前半部分的证明请参考数学分析教材上的证明 .

\end{enumerate}
 把隐函数代入原方程求两次偏导再整理可得 
$$
\frac{\partial^{2} x_{1}}{\partial x_{2} \partial x_{3}}=\frac{-f_{31} f_{2} f_{1}+f_{1}^{2} f_{32}+f_{11} f_{2} f_{3}-f_{1} f_{3} f_{12}}{-f_{1}^{3}}
$$
 设  $G$  是  $\mathbb{R}^{2}$  中的开集 ,  要证  $P_{x}(G)$  是  $\mathbb{R}$  中开集 .  任取  $x_{0} \in P_{x}(G)$,  则存在  $\left(x_{0}, y_{0}\right) \in G$  使得  $P_{x}\left(x_{0}, y_{0}\right)=x_{0}$.  又由于  $G$  是开集 ,  故存在邻域  $\left\{(x, y) \mid\left(x-x_{0}\right)^{2}+\left(y-y_{0}\right)^{2}<\delta^{2}\right\} \subset G$,  由此知  $\left(x_{0}-\delta, x_{0}+\delta\right) \subset P_{x}(G)$,  从而  $P_{x}(G)$  是开集 ,  故  $P_{x}$  是开映射 .  更一般的投影映射是开映射的证明是类似的 .

 对于任意  $n>m, x_{0} \in U$,  因为  $\operatorname{rank}\left(\left.J(f)\right|_{x=x_{0}}\right)=m$,  不妨设  $\left.J(f)\right|_{x=x_{0}}$  的前  $m$  列是线性无关的 .  定义 
$$
\begin{aligned}
F: \quad \mathbb{R}^{n} & \longrightarrow \\
x & \longmapsto\left(f(x), x_{m+1}, \ldots, x_{n}\right)
\end{aligned}
$$
 此时  $F$  在  $x_{0}$  的一个邻域内为开映射 ,  将  $F$  与一个投影映射复合后得到  $f$.  开映射的复合还是开映射 ,  从而  $f$  是  $x_{0}$  的一个邻域内的开映射 ,  由  $x_{0}$  的任意性以及任意个数个开集的并集仍为开集知  $f$  是开映射 .

\begin{enumerate}
  \setcounter{enumi}{6}
  \item  用数学归纳法可以证明  $0<x_{n}<x_{n+1}<2, n \in \mathbb{N}$,  从而  $\left\{x_{n}\right\}$  单调有界 ,  必有极限 ,  等式两边令  $n \rightarrow \infty$  算   出  $\lim _{n \rightarrow \infty} x_{n}=2$.

  \item  因为 

\end{enumerate}
$$
\frac{1}{2}+\cos x+\cos 2 x+\cdots+\cos n x=\frac{\sin \left(n+\frac{1}{2}\right) x}{2 \sin \frac{x}{2}}
$$
 两边同时对  $x$  在  $[0, \pi]$  积分得 
$$
\frac{\pi}{2}=\int_{0}^{\pi} \frac{\sin \left(n+\frac{1}{2}\right) x}{2 \sin \frac{x}{2}} \mathrm{~d} x=\int_{0}^{\pi} \frac{\sin \left(n+\frac{1}{2}\right) x}{x} \mathrm{~d} x+\int_{0}^{\pi} \frac{x-2 \sin \frac{x}{2}}{2 x \sin \frac{x}{2}} \sin \left(n+\frac{1}{2}\right) x \mathrm{~d} x
$$
 由  Dirichlet  判别法知  $\int_{0}^{+\infty} \frac{\sin x}{x} \mathrm{~d} x$  存在 ,  因此 
$$
\lim _{n \rightarrow \infty} \int_{0}^{\pi} \frac{\sin \left(n+\frac{1}{2}\right) x}{x} \mathrm{~d} x=\lim _{n \rightarrow \infty} \int_{0}^{\left(n+\frac{1}{2}\right) \pi} \frac{\sin x}{x} \mathrm{~d} x=\int_{0}^{+\infty} \frac{\sin x}{x} \mathrm{~d} x .
$$
 再由  Riemann-Lebesgue  引理知 
$$
\lim _{n \rightarrow \infty} \int_{0}^{\pi} \frac{x-2 \sin \frac{x}{2}}{2 x \sin \frac{x}{2}} \sin \left(n+\frac{1}{2}\right) x \mathrm{~d} x=0 .
$$
 因此 
$$
\int_{0}^{+\infty} \frac{\sin x}{x} \mathrm{~d} x=\frac{\pi}{2}
$$
 注   北京大学  2006  年数学分析试题中的第  5  题 .

\begin{enumerate}
  \setcounter{enumi}{8}
  \item (1)  令 
\end{enumerate}
$$
L_{n}(x)=\frac{1}{2^{n} n !} \sqrt{\frac{2 n+1}{2}}\left(\left(x^{2}-1\right)^{n}\right)^{(n)}, n \in \mathbb{N}
$$
 则 
$$
\int_{-1}^{1} L_{i}(x) L_{j}(x) \mathrm{d} x=\delta_{i j}
$$
 做变量替换 
$$
t=\frac{b-a}{2} x+\frac{a+b}{2}
$$
 倡 
$$
\int_{a}^{b} L_{i}\left(\frac{t-\frac{a+b}{2}}{\frac{b-a}{2}}\right) L_{j}\left(\frac{t-\frac{a+b}{2}}{\frac{b-a}{2}}\right) \frac{2}{b-a} \mathrm{~d} t=\delta_{i j}
$$
 令 
$$
p_{n}(x)=\sqrt{\frac{2}{b-a}} L_{n}\left(\frac{x-\frac{a+b}{2}}{\frac{b-a}{2}}\right), n \in \mathbb{N}
$$
 即可 .

$\forall \varepsilon>0$,  存在实多项式  $p(x)$,  使得  $|p(x)-f(x)|<\varepsilon, \forall x \in[a, b]$.  设  $|f(x)|$  在  $[a, b]$  上的最大值为  $M$,  则 
$$
\int_{a}^{b} f^{2}(x) \mathrm{d} x \leqslant\left|\int_{a}^{b}(f(x)-p(x)) f(x) \mathrm{d} x\right|+\left|\int_{a}^{b} p(x) f(x) \mathrm{d} x\right| \leqslant M(b-a) \varepsilon
$$
 由  $\varepsilon$  的任意性知  $\int_{a}^{b} f^{2}(x) \mathrm{d} x=0$,  于是  $f \equiv 0$.

(2)  摘抄  songlei1994  的话 :

 平方可积的那道题出错了 。 如果在黎曼积分下 ,  平方可积并不保证可积 ,  之后的计算是无法进   行的 。 但是如果是勒贝格框架下是对的 ,  这样这道题就像是泛函分析了 。 我认为数分下用的应   该是黎曼积分吧 。

 注   解答第一问时利用了  Legendre  多项式 ,  更具体的内容可以参考陈纪修等人的 《 数学分析 》 第二版上册第  301  页例  $7.3 .7$,  蓝以中的 《 高等代数学习指南 》 第  285  页例  $1.9$  也是关于  Legendre  多项式的 .  第二问其实   是想考察广义  Fourier  级数的  Parseval  等式 ,  如果把题目中的平方可积改成  Riemann  可积则等式成立 ,  证   明方法是先证明  Fourier  级数部分和的最佳均方逼近性 ,  再利用连续函数可以用多项式一致逼近以及连续   函数可以去逼近  Riemann  可积函数 ,  更具体地可以参考教材相关内容 .

\begin{enumerate}
  \setcounter{enumi}{9}
  \item  因为  $\lim _{n \rightarrow \infty} b_{n}=0$  故  $\left\{b_{n}\right\}$  有界 ,  设  $\left|b_{n}\right| \leqslant M$. $\forall \varepsilon>0, \exists N_{1}$,  当  $n>m>N_{1}$  时 , $\left|\sum_{k=m}^{n} a_{k}\right|<\frac{\varepsilon}{2 M}$,
\end{enumerate}
$$
\left|a_{1} b_{n}+\cdots+a_{n} b_{1}\right| \leqslant\left|a_{1} b_{n}+\cdots+a_{N_{1}} b_{n-N_{1}+1}\right|+\left|a_{N_{1}+1} b_{n-N_{1}}+\cdots+a_{n} b_{1}\right| .
$$
 固定上述  $N_{1}$,  因为  $\lim _{n \rightarrow \infty}\left(a_{1} b_{n}+\cdots+a_{N_{1}} b_{n-N_{1}+1}\right)=0$,  于是  $\exists N>N_{1}>0$,  当  $n>N$  时 ,
$$
\begin{gathered}
\left|a_{1} b_{n}+\cdots+a_{N_{1}} b_{n-N_{1}+1}\right|<\frac{\varepsilon}{2} \\
\left|a_{1} b_{n}+\cdots+a_{n} b_{1}\right|<\frac{\varepsilon}{2}+\frac{\varepsilon}{2 M} M=\varepsilon
\end{gathered}
$$
 因此  $\lim _{n \rightarrow+\infty} c_{n}=0$.

 注   北京大学  2013  年数学分析试题中的第  8  题 .

\begin{enumerate}
  \setcounter{enumi}{10}
  \item  充分性 : $\forall \varepsilon>0, \exists N>0$,  当  $n>m>N$  时 ,
\end{enumerate}
$$
\left|\sum_{k=m}^{n} a_{k} x^{k}\right|<\varepsilon, \quad \forall x \in[0, R)
$$
 令  $x \rightarrow R^{-}$ 得 
$$
\left|\sum_{k=m}^{n} a_{k} R^{k}\right|<\varepsilon \Longrightarrow \sum_{n=1}^{\infty} a_{n} R^{n} \text { 收敛. }
$$
 必要性 : 首先注意到 
$$
\sum_{n=1}^{\infty} a_{n} x^{n}=\sum_{n=1}^{\infty} a_{n} R^{n}\left(\frac{x}{R}\right)^{n}
$$
 又因为  $\sum_{n=1}^{\infty} a_{n} R^{n}$  关于  $x$  在  $[0, R)$  上一致收敛 , $\left(\frac{x}{R}\right)^{n}$  一致单调有界 ,  由  Abel  判别法知  $\sum_{n=1}^{\infty} a_{n} x^{n}$  在  $[0, R)$  上一致收敛 .  北京大学  2017  年全国硕士研究生招生考试数学分析试题及解答     

2019.04.16

\begin{enumerate}
  \item (10  分 )  证明 :
\end{enumerate}
$$
\lim _{n \rightarrow+\infty} \int_{0}^{\frac{\pi}{2}} \frac{\sin ^{n} x}{\sqrt{\pi-2 x}} \mathrm{~d} x .
$$

\begin{enumerate}
  \setcounter{enumi}{2}
  \item (10  分 )  证明 : $\sum_{n=1}^{\infty} \frac{1}{1+n x^{2}} \sin \frac{x}{n^{\alpha}}$  在任何有限区间上一致收玫的充要条件是 : $\alpha>\frac{1}{2}$.

  \item (10  分 )  设  $\sum_{n=1}^{\infty} a_{n}$  收玫 .  证明 

\end{enumerate}
$$
\lim _{s \rightarrow 0^{+}} \sum_{n=1}^{\infty} a_{n} n^{-s}=\sum_{n=1}^{\infty} a_{n} .
$$

\begin{enumerate}
  \setcounter{enumi}{4}
  \item $\left(10\right.$  分 )  设  $\gamma(t)=(x(t), y(t)),(t$  属于某个区间  $I)$  是  $\mathbb{R}^{1}$  上  $C^{1}$  向量场  $(P(x, y), Q(x, y))$  的积分曲线 ,  若  $x^{\prime}(t)=P(\gamma(t)), y^{\prime}(t)=Q(\gamma(t)), \forall t \in I$.  设  $P_{x}+Q_{y}$  在  $\mathbb{R}^{1}$  上处处非零 ,  证明向量场  $(P, Q)$  的积分曲线不可   能封闭  ( 单点情形除外 ).

  \item (20  分 )  假设  $x_{0}=1, x_{n}=x_{n-1}+\cos x_{n-1},(n=1,2, \cdots,$, .  证明 :  当  $n \rightarrow \infty$  时 , $x_{n}-\frac{\pi}{2}=o\left(\frac{1}{n^{n}}\right)$.

  \item (20  分 )  假如  $f \in C[0,1], \lim _{x \rightarrow 0^{+}} \frac{f(x)-f(0)}{x}=\alpha<\beta=\lim _{x \rightarrow 1^{-}} \frac{f(x)-f(1)}{x-1}$.  证明 : $\forall \lambda \in(\alpha, \beta), \exists x_{1}, x_{2} \in[0,1]$  使得  $\lambda=\frac{f\left(x_{2}\right)-f\left(x_{1}\right)}{x_{2}-x_{1}}$.

  \item $\left(20\right.$  分 )  设  $f$  是  $(0,+\infty)$  上的凹  ( 或凸 )  函数且  $\lim _{x \rightarrow+\infty} f(x)$  存在有限 ,  证明  $\lim _{x \rightarrow+\infty} x f^{\prime}(x)=0($  仅在  $f$  可导   的点考虑极限过程 ).

  \item (20  分 )  设  $\phi \in C^{3}\left(\mathbb{R}^{3}\right), \phi$  及其各个偏导数  $\partial_{i} \phi(i=1,2,3)$  在点  $X_{0} \in \mathbb{R}^{3}$  处取值都是  0 . $X_{0}$  点的  $\delta$  邻域记   为  $U_{\delta}(\delta>0)$.  如果  $\left(\partial_{i j}^{2} \phi\left(X_{0}\right)\right)_{3 \times 3}$  是严格正定的 ,  则当  $\delta$  充分小时 ,  证明如下极限存在并求之 :

\end{enumerate}
$$
\lim _{t \rightarrow+\infty} t^{\frac{3}{2}} \iiint_{U_{\delta}} \mathrm{e}^{-t \phi\left(x_{1}, x_{2}, x_{3}\right)} \mathrm{d} x_{1} \mathrm{~d} x_{2} \mathrm{~d} x_{3} .
$$

\begin{enumerate}
  \setcounter{enumi}{9}
  \item (30  分 )  将  $(0, \pi)$  上常值函数  $f(x)=1$  进行周期  $2 \pi$  奇延拓并展为正弦级数 :
\end{enumerate}
$$
f(x) \sim \frac{4}{\pi} \sum_{n=1}^{\infty} \frac{1}{2 n-1} \sin (2 n-1) x .
$$
 记该  Fourier  级数的前  $n$  项和为  $S_{n}(x)$,  则  $\forall x \in(0, \pi), S_{n}(x)=\frac{2}{\pi} \int_{0}^{x} \frac{\sin 2 n t}{\sin t} \mathrm{~d} t$,  且  $\lim _{n \rightarrow \infty} S_{n}(x)=1$.  证明  $S_{n}(x)$  的最大值点是  $\frac{\pi}{2 n}$  且  $\lim _{n \rightarrow \infty} S_{n}\left(\frac{\pi}{2 n}\right)=\frac{2}{\pi} \int_{0}^{\pi} \frac{\sin t}{t} \mathrm{~d} t$. 1.  对  $\forall \delta \in\left(0, \frac{\pi}{2}\right)$,
$$
\begin{aligned}
\int_{0}^{\frac{\pi}{2}} \frac{\sin ^{n} x}{\sqrt{\pi-2 x}} \mathrm{~d} x & \leqslant \int_{0}^{\delta} \frac{\sin ^{n} \delta}{\sqrt{\pi-2 \delta}} \mathrm{d} x+\int_{\delta}^{\frac{\pi}{2}} \frac{\mathrm{d} x}{\sqrt{\pi-2 x}} \\
&=\frac{\delta \sin ^{n} \delta}{\sqrt{\pi-2 \delta}}+\left.(-\sqrt{\pi-2 x})\right|_{\delta} ^{\frac{\pi}{2}} \\
&=\frac{\delta \sin ^{n} \delta}{\sqrt{\pi-2 \delta}}+\sqrt{\pi-2 \delta}
\end{aligned}
$$
$\forall \varepsilon \in(0,1)$,  取  $\delta=\frac{\pi-\left(\frac{\varepsilon}{2}\right)^{2}}{2}$,  则  $0<\sin \delta<1$.  于是  $\sin ^{n} \delta \rightarrow 0, n \rightarrow+\infty$,  从而  $\exists N$,  当  $n>N$  时 ,
$$
\left|\int_{0}^{\frac{\pi}{2}} \frac{\sin ^{n} x}{\sqrt{\pi-2 x}} \mathrm{~d} x\right|<\frac{\varepsilon}{2}+\frac{\varepsilon}{2}=\varepsilon
$$

\begin{enumerate}
  \setcounter{enumi}{2}
  \item  充分性 : 当  $\alpha>\frac{1}{2}$  时 ,  对任意属于有限区间  $[a, b]$  的  $x$  有 
\end{enumerate}
$$
\begin{aligned}
\left|\sum_{n=1}^{N} \frac{1}{1+n x^{2}} \sin \frac{x}{n^{\alpha}}\right| & \leqslant \sum_{n=1}^{N} \frac{1}{1+n x^{2}} \frac{|x|}{n^{\alpha}} \\
& \leqslant \sum_{n=1}^{N} \frac{1}{2 \sqrt{n} n^{\alpha}}
\end{aligned}
$$
 由  Weierstrass  判别法知原级数在  $[a, b]$  上一致收敛 .

 必要性 :  若  $\alpha=0, x=\frac{\pi}{2}$,  则  $\sum_{n=1}^{+\infty} \frac{1}{1+n x^{2}} \sin \frac{x}{n^{\alpha}}=\sum_{n=1}^{+\infty} \frac{1}{1+\frac{\pi^{2}}{4} n}$,  等号右边的级数发散 ,  从而  $\alpha \neq 0$.

 若  $0<\alpha \leqslant \frac{1}{2}, x=\frac{1}{\sqrt{2 n}} \in[0,1]$,  则 
$$
\begin{aligned}
\sum_{k=n}^{2 n-1} \frac{1}{1+k x^{2}} \sin \frac{x}{k^{\alpha}} &>\frac{2}{\pi} \sum_{k=n}^{2 n-1} \frac{1}{1+k x^{2}} \frac{x}{k^{\alpha}} \\
&=\frac{2}{\pi} \sum_{k=n}^{2 n-1} \frac{\frac{1}{\sqrt{2 n}}}{1+\frac{k}{2 n}} \frac{1}{k^{\alpha}} \\
&>\frac{1}{\pi} \frac{n}{(2 n)^{\alpha+\frac{1}{2}}} \\
& \geqslant \frac{1}{\pi} \frac{1}{2^{\alpha+\frac{1}{2}}}
\end{aligned}
$$
 由  Cauchy  收敛准则知级数不收敛 ,  从而不会在  $[0,1]$  一致收敛 .

 若  $\alpha<0, x=(2 n)^{\alpha}$,  则当  $n \leqslant k \leqslant 2 n$  时 , $\frac{x}{k^{\alpha}}=\left(\frac{2 n}{k}\right)^{\alpha} \in\left[2^{\alpha}, 1\right]$,
$$
\begin{aligned}
\sum_{k=n}^{2 n} \frac{1}{1+k x^{2}} \sin \frac{x}{k^{\alpha}} &>\sum_{k=n}^{2 n-1} \frac{\sin 2^{\alpha}}{1+k(2 n)^{\alpha}} \\
&>\frac{n \sin 2^{\alpha}}{1+(2 n)^{1+\alpha}} \rightarrow+\infty,(n \rightarrow+\infty)
\end{aligned}
$$
 由  Cauchy  收敛准则知级数不收敛 ,  从而不会在  $[0,1]$  一致收敛 .

 综合上述三点就说明了必要性 .

 注   充分性的证明比较简单 ,  但是必要性比较难想到 ,  这里必要性的证明借用了  Hansschwarzkopf  的想法 .  证明   必要性时用到了  Jordan  不等式 
$$
\sin x \geqslant \frac{2}{\pi} x, x \in\left[0, \frac{\pi}{2}\right] .
$$

\begin{enumerate}
  \setcounter{enumi}{3}
  \item  关于  $s$  的级数  $\sum_{n=1}^{\infty} a_{n}$  在  $[0, \delta]$  一致收敛 ,  而  $\frac{1}{n^{s}}$  一致单调有界 ,  由  Abel  判别法知  $\sum_{n=1}^{\infty} \frac{a_{n}}{n^{s}}$  在  $[0, \delta]$  一致收   敛 ,  从而  $\lim _{s \rightarrow 0^{+}} \sum_{n=1}^{\infty} \frac{a_{n}}{n^{s}}=\sum_{n=1}^{\infty} a_{n}$.

  \item  假设向量场的积分曲线封闭 ,  则 

\end{enumerate}
$$
\begin{aligned}
&\oint P \mathrm{~d} y-Q \mathrm{~d} x=\int_{t \in I} P(\gamma(t)) Q(\gamma(t))-Q(\gamma(t)) P(\gamma(t))=0 \\
&\oint P \mathrm{~d} y-Q \mathrm{~d} x=\iint\left(P_{x}+Q_{y}\right) \mathrm{d} x \mathrm{~d} y=\left.\left(P_{x}+Q_{y}\right)\right|_{(\xi, \eta)} \cdot S \neq 0
\end{aligned}
$$
 其中  $S$  表示曲线  $\gamma$  所围图形的面积 ,  矛盾 .

\begin{enumerate}
  \setcounter{enumi}{5}
  \item  令  $y_{n}=\frac{\pi}{2}-x_{n}, n=0,1,2, \ldots$,  则  $y_{n}=y_{n-1}-\sin y_{n-1}$.  显然  $0<y_{1}<y_{0}<1$.  假设  $0<y_{n}<y_{n-1}<1$,  则 
\end{enumerate}
$$
0<y_{n+1}=y_{n}-\sin y_{n}<y_{n}<1
$$
 由数学归纳法原理知  $\left\{y_{n}\right\}$  单调递减有下界 , $y_{n} \in(0,1)$.  根据不等式 
$$
x-\frac{x^{3}}{6}<\sin x<x, x \in(0,+\infty)
$$
 得 
$$
0<y_{n}=y_{n-1}-\sin y_{n-1}<\frac{y_{n-1}^{3}}{6}<y_{n-1}^{3}, n \in \mathbb{N}_{+}
$$
 因此 
$$
\ln y_{n}<3 \ln y_{n-1}<3^{n} \ln y_{0} \Longrightarrow y_{n}<y_{0}^{3^{n}},
$$
 故  $\exists N>0$,  当  $n>N$  时 ,
$$
0<y_{n} n^{n}<y_{0}^{3^{n}} n^{n}<n^{n} y_{0}^{n^{2}}=\left(\frac{n}{\left(\frac{1}{y_{0}}\right)^{n}}\right)^{n}<\left(\frac{1}{2}\right)^{n} \Longrightarrow y_{n}=o\left(\frac{1}{n^{n}}\right)
$$
 注   上述证明参考了  shuxue1985  的想法 .  关于  $\sin x$  的不等式比较重要 ,  更一般的结论见谢惠民等人的 《 数学   分析习题课讲义 》 第八章第五节练习题  15 , 一个比较好的证明见该书的例题  11.2.4.

\begin{enumerate}
  \setcounter{enumi}{6}
  \item  令  $g(x)=f(x)-\lambda x$,  则因为 
\end{enumerate}
$$
\lim _{x \rightarrow 0^{+}} \frac{g(x)-g(0)}{x}=\alpha-\lambda<0,
$$
 于是  $\exists \delta_{1}>0$,  当  $0<x<\delta_{1}$  时 , $g(x)-g(0)<0, g(x)<g(0)$.  因为 
$$
\lim _{x \rightarrow 1^{-}} \frac{g(x)-g(1)}{x-1}=\beta-\lambda>0,
$$
 于是  $\exists \delta_{2}$,  当  $1-\delta_{2}<x<1$  时 , $g(x)<g(1)$,  于是  $g(x)$  的最小值在  $(0,1)$  内取得 .  设  $g\left(x_{m}\right)=\min _{x \in(0,1)} g(x)$,  则  $g\left(x_{m}\right)<g(0), g\left(x_{m}\right)<g(1)$.  若  $\exists x_{1} \neq x_{2}, x_{1}, x_{2} \in[0,1]$  使得  $g\left(x_{1}\right)=g\left(x_{2}\right)$,  则 
$$
f\left(x_{1}\right)-\lambda x_{1}=f\left(x_{2}\right)-\lambda x_{2} \quad \Longrightarrow \quad \frac{f\left(x_{2}\right)-f\left(x_{1}\right)}{x_{2}-x_{1}}=\lambda .
$$
 当  $g(1)=g(0)$  时 ,  显然命题成立 .  当  $g(1) \neq g(0)$  时 ,  不妨设  $g(0)<g(1)$,  则由于  $g(x)$  在  $\left[x_{m}, 1\right]$  上连续 ,  由介值定理知  $\exists \xi \in\left(x_{m}, 1\right)$  使得  $g(\xi)=g(1)$,  从而命题也成立 . 7.  不妨设  $f(x)$  为  $(0,+\infty)$  上的上凸函数 ,  则 
$$
\frac{f(2 x)-f(x)}{2 x-x} \leqslant f^{\prime}(x) \leqslant \frac{f(x)-f\left(\frac{x}{2}\right)}{x-\frac{x}{2}}
$$
 㔹 
$$
f(2 x)-f(x) \leqslant x f^{\prime}(x) \leqslant 2\left(f(x)-f\left(\frac{x}{2}\right)\right)
$$
 结合 
$$
\lim _{x \rightarrow+\infty} f(2 x)-f(x)=0,
$$
 由夹逼定理知 
$$
\lim _{x \rightarrow+\infty} x f^{\prime}(x)=0
$$

\begin{enumerate}
  \setcounter{enumi}{8}
  \item  设  $A=\frac{1}{2}\left(\partial_{i j}^{2}\right)_{3 \times 3}$,  则  $A$  为正定矩阵 , $\phi\left(X_{0}\right)$  为严格极小值 .  当  $\delta$  足够小时 , $\phi(X)$  在  $U_{\delta}$  内非负 ,  并且对  $\forall \delta^{\prime} \in(0, \delta), \phi(X)$  在  $\overline{U_{\delta} \backslash U_{\delta^{\prime}}}$  上恒为正数 ,  从而  $\phi(X)$  在  $\overline{U_{\delta} \backslash U_{\delta^{\prime}}}$  上的最小值为正数 ,  故有 
\end{enumerate}
$$
\lim _{t \rightarrow+\infty} t^{\frac{3}{2}} \iiint_{\frac{U_{\delta} \backslash U_{\delta^{\prime}}}{}} \mathrm{e}^{-t \phi(X)} \mathrm{d} X=0 .
$$
 设  $A$  的特征值为  $\lambda_{1}, \lambda_{2}, \lambda_{3}$,  并且  $\lambda_{1} \geqslant \lambda_{2} \geqslant \lambda_{3}>0$.  对于任意事先给定的  $\varepsilon \in\left(0, \lambda_{3}\right), \exists \delta_{\varepsilon}^{\prime}$  使得对于任意属   于球形邻域  $U_{\delta_{\varepsilon}}^{\prime}$  的  $X$  有 

$\left(X-X_{0}\right)^{\mathrm{T}} A\left(X-X_{0}\right)-\varepsilon\left(X-X_{0}\right)^{\mathrm{T}}\left(X-X_{0}\right)<\phi(X)<\left(X-X_{0}\right)^{\mathrm{T}} A\left(X-X_{0}\right)+\varepsilon\left(X-X_{0}\right)^{\mathrm{T}}\left(X-X_{0}\right)$  敇 
$$
\mathrm{e}^{-t\left(\left(X-X_{0}\right)^{\mathrm{T}} A\left(X-X_{0}\right)+\varepsilon\left|X-X_{0}\right|^{2}\right)}<\mathrm{e}^{-t \phi(X)}<\mathrm{e}^{-t\left(\left(X-X_{0}\right)^{\mathrm{T}} A\left(X-X_{0}\right)-\varepsilon\left|X-X_{0}\right|^{2}\right)}
$$
 从而 

$\varlimsup_{t \rightarrow+\infty} t^{\frac{3}{2}} \iiint_{U_{\delta}} \mathrm{e}^{-t \phi(X)} \mathrm{d} X=\varlimsup_{t \rightarrow+\infty} t^{\frac{3}{2}} \iiint_{U_{\delta_{\varepsilon}^{\prime}}} \mathrm{e}^{-t \phi(X)} \mathrm{d} X \leqslant \varlimsup_{t \rightarrow \infty} t^{\frac{3}{2}} \iiint_{U_{\delta_{\varepsilon}^{\prime}}} \mathrm{e}^{-t\left(\left(X-X_{0}\right)^{\mathrm{T}} A\left(X-X_{0}\right)-\varepsilon\left|X-X_{0}\right|^{2}\right)} \mathrm{d} X$

 做适当的平移及正交变换得 
$$
\begin{aligned}
& t^{\frac{3}{2}} \iiint_{U_{\delta_{\varepsilon}^{\prime}}} \mathrm{e}^{-t\left(\left(X-X_{0}\right)^{\mathrm{T}} A\left(X-X_{0}\right)-\varepsilon\left|X-X_{0}\right|^{2}\right)} \mathrm{d} X \\
=& t^{\frac{3}{2}} \iiint_{U_{\delta_{\varepsilon}^{\prime}}(0)} \mathrm{e}^{-t\left(\left(\lambda_{1}-\varepsilon\right) x_{1}^{2}+\left(\lambda_{2}-\varepsilon\right) x_{2}^{2}+\left(\lambda_{3}-\varepsilon\right) x_{3}^{2}\right)} \mathrm{d} X \\
<&\left(\sqrt{t} \int_{-\delta_{\varepsilon}^{\prime}}^{\delta_{\varepsilon}^{\prime}} \mathrm{e}^{-t\left(\lambda_{1}-\varepsilon\right) x_{1}^{2}} \mathrm{~d} x_{1}\right)\left(\sqrt{t} \int_{-\delta_{\varepsilon}^{\prime}}^{\delta_{\varepsilon}^{\prime}} \mathrm{e}^{-t\left(\lambda_{2}-\varepsilon\right) x_{2}^{2}} \mathrm{~d} x_{2}\right)\left(\sqrt{t} \int_{-\delta_{\varepsilon}^{\prime}}^{\delta_{\varepsilon}^{\prime}} \mathrm{e}^{-t\left(\lambda_{3}-\varepsilon\right) x_{3}^{2}} \mathrm{~d} x_{3}\right) \\
=& \frac{1}{\sqrt{\left(\lambda_{1}-\varepsilon\right)\left(\lambda_{2}-\varepsilon\right)\left(\lambda_{3}-\varepsilon\right)}}\left(\int_{-\sqrt{t\left(\lambda_{1}-\varepsilon\right)} \delta_{\varepsilon}^{\prime}}^{\sqrt{t\left(\lambda_{1}-\varepsilon\right)} \delta_{\varepsilon}^{\prime}} \mathrm{e}^{-x_{1}^{2}} \mathrm{~d} x_{1}\right)\left(\int_{-\sqrt{t\left(\lambda_{2}-\varepsilon\right) \delta_{\varepsilon}^{\prime}}}^{\sqrt{t\left(\lambda_{2}-\varepsilon\right) \delta_{\varepsilon}^{\prime}}} \mathrm{e}^{-x_{2}^{\prime}} \mathrm{d} x_{2}\right)\left(\int_{-\sqrt{t\left(\lambda_{3}-\varepsilon\right)} \delta_{\varepsilon}^{\prime}}^{\sqrt{t\left(\lambda_{3}-\varepsilon\right)} \delta_{\varepsilon}^{\prime}} \mathrm{e}^{-x_{3}^{2}} \mathrm{~d} x_{3}\right)
\end{aligned}
$$
 从而 
$$
\varlimsup_{t \rightarrow+\infty} t^{\frac{3}{2}} \iiint_{U_{\delta}} \mathrm{e}^{-t \phi(X)} \mathrm{d} X \leqslant \frac{\pi^{\frac{3}{2}}}{\sqrt{\left(\lambda_{1}-\varepsilon\right)\left(\lambda_{2}-\varepsilon\right)\left(\lambda_{3}-\varepsilon\right)}}
$$
 类似地可以得到 
$$
\underline{t \rightarrow+\infty} \underline{\lim }^{\frac{3}{2}} \iiint_{U_{\delta}} \mathrm{e}^{-t \phi(X)} \mathrm{d} X \geqslant \frac{\pi^{\frac{3}{2}}}{\sqrt{\left(\lambda_{1}+\varepsilon\right)\left(\lambda_{2}+\varepsilon\right)\left(\lambda_{3}+\varepsilon\right)}} .
$$
 令  $\varepsilon \rightarrow 0^{+}$ 得 
$$
\lim _{t \rightarrow+\infty} t^{\frac{3}{2}} \iiint_{U_{\delta}} \mathrm{e}^{-t \phi\left(x_{1}, x_{2}, x_{3}\right)} \mathrm{d} x_{1} \mathrm{~d} x_{2} \mathrm{~d} x_{3}=\frac{\pi^{\frac{3}{2}}}{\sqrt{\operatorname{det}(A)}}=\frac{\pi^{\frac{3}{2}}}{\sqrt{\frac{1}{8} \operatorname{det}\left(\left(\partial_{i j}^{2} \phi\left(X_{0}\right)\right)_{3 \times 3}\right)}}=\frac{(2 \pi)^{\frac{3}{2}}}{\sqrt{\operatorname{det}\left(\left(\partial_{i j}^{2} \phi\left(X_{0}\right)\right)_{3 \times 3}\right)}} .
$$
 注   此处的证明改写自  Xionger  整理的解答 .

\begin{enumerate}
  \setcounter{enumi}{9}
  \item  经计算得到 
\end{enumerate}
$$
S_{n}^{\prime}(x)=\frac{2}{\pi} \frac{\sin 2 n x}{\sin x}, S_{n}^{\prime \prime}(x)=\frac{2}{\pi} \frac{2 n \cos 2 n x \sin x-\sin 2 n x \cos x}{(\sin x)^{2}}, x \in(0, \pi)
$$
 因为  $S_{n}^{\prime}\left(\frac{(2 k-1) \pi}{2 n}\right)=0, S_{n}^{\prime \prime}\left(\frac{(2 k-1) \pi}{2 n}\right)<0$,  故  $\frac{(2 k-1) \pi}{2 n}, k=1,2, \ldots, n$  为极大值点 ,  并且不难看出这就是  $S_{n}(x)$  在  $(0, \pi)$  上所有的极大值点 .  当  $k \leqslant \frac{n-1}{2}$  时 ,

 从而 
$$
S_{n}\left(\frac{(2 k+1) \pi}{2 n}\right)<S_{n}\left(\frac{(2 k-1) \pi}{2 n}\right)<\cdots<S_{n}\left(\frac{3 \pi}{2 n}\right)<S_{n}\left(\frac{\pi}{2 n}\right), k \leqslant \frac{n-1}{2}
$$
 再注意到 
$$
\begin{aligned}
\left.S_{n}(\pi-x)-S_{(} x\right) &=\int_{x}^{\pi-x} \frac{\sin 2 n t}{\sin t} \mathrm{~d} t \\
&=\int_{x}^{\frac{\pi}{2}} \frac{\sin 2 n t}{\sin t} \mathrm{~d} t+\int_{\frac{\pi}{2}}^{\pi-x} \frac{\sin 2 n t}{\sin t} \mathrm{~d} t \\
&=0
\end{aligned}
$$
 故  $S_{n}(x)$  在  $(0, \pi)$  上的最大值在  $\frac{\pi}{2 n}, \frac{(2 n-1) \pi}{2 n}$  处取得 .

 要证 
$$
\lim _{n \rightarrow+\infty} S_{n}\left(\frac{\pi}{2 n}\right)=\lim _{n \rightarrow+\infty} \frac{2}{\pi} \int_{0}^{\frac{\pi}{2 n}} \frac{\sin (2 n t)}{\sin t} \mathrm{~d} t=\lim _{n \rightarrow+\infty} \frac{2}{\pi} \int_{0}^{\pi} \frac{\sin x}{2 n \sin \frac{x}{2 n}} \mathrm{~d} x=\frac{2}{\pi} \int_{0}^{\pi} \frac{\sin t}{t} \mathrm{~d} t
$$
 只需证 
$$
\lim _{n \rightarrow \infty}\left(\int_{0}^{\pi} \frac{\sin x}{2 n \sin \frac{x}{2 n}} \mathrm{~d} x-\int_{0}^{\pi} \frac{\sin t}{t} \mathrm{~d} t\right)=\lim _{n \rightarrow \infty} \int_{0}^{\pi} \sin x \frac{x-2 n \sin \frac{x}{2 n}}{\left(2 n \sin \frac{x}{2 n}\right) x} \mathrm{~d} x=0
$$
 注意至 
$$
0 \leqslant \frac{x-2 n \cdot \sin \frac{x}{2 n}}{x \cdot 2 n \cdot \sin \frac{x}{2 n}}<\frac{\frac{2 n}{3 !}\left(\frac{x}{2 n}\right)^{3}}{\frac{2}{\pi} \cdot 2 n x \cdot \frac{x}{2 n}}=\frac{\pi}{2} \frac{2 n}{6} \frac{x}{8 n^{3}}<\frac{\pi}{48} \frac{\pi}{n^{2}}, x \in(0, \pi)
$$
 故 
$$
\begin{gathered}
0 \leqslant \int_{0}^{\pi} \sin x \frac{x-2 n \sin \frac{x}{2 n}}{\left(2 n \sin \frac{x}{2 n}\right) x} \mathrm{~d} x \leqslant \frac{\pi^{3}}{48 n^{2}}, \\
\lim _{n \rightarrow \infty} \int_{0}^{\pi} \sin x \frac{x-2 n \sin \frac{x}{2 n}}{\left(2 n \sin \frac{x}{2 n}\right) x} \mathrm{~d} x=0 .
\end{gathered}
$$
$$
\begin{aligned}
& S_{n}\left(\frac{(2 k+1) \pi}{2 n}\right)-S_{n}\left(\frac{(2 k-1) \pi}{2 n}\right)=\int_{\frac{(2 k-1) \pi}{2 n}}^{\frac{(2 k+1) \pi}{2 n}} \frac{\sin 2 n t}{\sin t} \mathrm{~d} t \\
& =\int_{\frac{(2 k-1) \pi}{2 n}}^{\frac{2 k \pi}{2 n}} \frac{\sin 2 n t}{\sin t} \mathrm{~d} t+\int_{\frac{2 k \pi}{2 n}}^{\frac{(2 k+1) \pi}{2 n}} \frac{\sin 2 n t}{\sin t} \mathrm{~d} t \\
& =-\frac{1}{\sin \xi_{1}}+\frac{1}{\sin \xi_{2}} \quad\left(0<\xi_{1}<\xi_{2}<\frac{\pi}{2}\right) \\
& <0 \text {. } 
\end{aligned}
$$
 北京大学  2018  年全国硕士研究生招生考试数学分析试题及解答     

2019.04.18

\begin{enumerate}
  \item ( 每小题  10  分 ,  共  30  分 )  证明如下极限 :\\
(1) $\lim _{n \rightarrow \infty}\left(1+\int_{0}^{1} \frac{\sin ^{n} x}{x^{n}} \mathrm{~d} x\right)^{n}=+\infty$\\
(2) $\lim _{n \rightarrow \infty}\left(\int_{0}^{1} \frac{\sin \left(x^{n}\right)}{x^{n}} \mathrm{~d} x\right)^{n}=\prod_{k=1}^{+\infty} \mathrm{e}^{\frac{(-1)^{k}}{2 k(2 k+1) !} \text {; }}$\\
(3) $\lim _{n \rightarrow \infty} \frac{1}{n} \sum_{k=1}^{n} \ln \left(1+\frac{k^{2}-k}{n^{2}}\right)=\ln 2-2+\frac{\pi}{2}$.

  \item (10  分 ) $f \in C(0,1), \frac{f\left(x_{2}\right)-f\left(x_{1}\right)}{x_{2}-x_{1}}=\alpha<\beta=\frac{f\left(x_{4}\right)-f\left(x_{3}\right)}{x_{4}-x_{3}}$,  这里  $x_{1}, x_{2}, x_{3}, x_{4} \in(0,1)$.  证明 :  对任意  $\lambda \in(\alpha, \beta)$,  存在  $x_{5}, x_{6} \in(0,1)$,  使得  $\lambda=\frac{f\left(x_{6}\right)-f\left(x_{5}\right)}{x_{6}-x_{5}}$.

  \item (10  分 )  设  $\gamma$  是联结  $\mathbb{R}^{3}$  中两点  $A, B$  且长度为  $L$  的光滑曲线 , $U$  是  $\mathbb{R}^{3}$  中包含  $\gamma$  的开集 , $f$  在  $U$  上连续可   微 ,  梯度  $\nabla f$  的长度在  $\gamma$  上的上界为  $M$.  证明 :

\end{enumerate}
$$
|f(A)-f(B)| \leqslant M L .
$$

\begin{enumerate}
  \setcounter{enumi}{4}
  \item (20  分 ) $f$  在  $(0,0)$  点局部三阶连续可微 , $D_{R}$  表示圆盘 : $x^{2}+y^{2} \leqslant R^{2}$.  计算 :
\end{enumerate}
$$
\lim _{R \rightarrow 0^{+}} \frac{1}{R^{4}} \iint_{D_{R}}(f(x, y)-f(0,0)) \mathrm{d} x \mathrm{~d} y .
$$

\begin{enumerate}
  \setcounter{enumi}{5}
  \item (20  分 ) $\varphi(x)$  在  0  处可导 , $\varphi(0)=0, f(x, y)$  在  $(0,0)$  点局部  2  阶连续可微 , $\nabla f(x, \varphi(x))=0,\left(\partial_{i j} f(0,0)\right)$  衣   为半正定非  0  阵 .  证明  $f$  在  $(0,0)$  点取得极小值 .

  \item (20  分 )  证明 : $\mathrm{e}^{-x}+\cos (2 x)+x \sin x=0$  在区间  $((2 n-1) \pi,(2 n+1) \pi)$  恰有两个根  $x_{2 n-1}<x_{2 n}, \forall n=1$, $2,3, \ldots$  证明如下极限存在并求之 : $\lim _{n \rightarrow \infty}(-1)^{n} n\left(x_{n}-n \pi\right)$.

  \item (20  分 )  证明 : $\lim _{x \rightarrow 0} \sum_{n=1}^{\infty} \frac{\cos (n x)}{n}=+\infty$.

  \item (20  分 ) $\forall x \in[1,+\infty), f(x)>0, f^{\prime \prime}(x) \leqslant 0$,  且  $\lim _{x \rightarrow+\infty} f(x)=+\infty$.  证明如下极限存在并求之 :

\end{enumerate}
$$
\lim _{s \rightarrow 0^{+}} \sum_{n=1}^{\infty} \frac{(-1)^{n}}{f^{s}(n)}
$$

\begin{enumerate}
  \item (1)  注意到  $\sin x \geqslant x-\frac{x^{3}}{6}, \forall x \geqslant 0$,  得到 
\end{enumerate}
$$
\begin{aligned}
\left(1+\int_{0}^{1} \frac{\sin ^{n} x}{x^{n}} \mathrm{~d} x\right)^{n} & \geqslant\left(1+\int_{0}^{1}\left(1-\frac{x^{2}}{6}\right)^{n} \mathrm{~d} x\right)^{n} \\
& \geqslant\left(1+\int_{0}^{\frac{1}{\sqrt{n}}}\left(1-\frac{x^{2}}{6}\right)^{n} \mathrm{~d} x\right)^{n} \\
& \geqslant\left(1+\frac{5}{6 \sqrt{n}}\right)^{n} \\
& \geqslant \frac{5 \sqrt{n}}{6}
\end{aligned}
$$
 因此有 
$$
\lim _{n \rightarrow \infty}\left(1+\int_{0}^{1} \frac{\sin ^{n} x}{x^{n}} \mathrm{~d} x\right)^{n}=+\infty
$$
 注   证明的关键是用关于  $\sin x$  的不等式进行放缩 .  此处的证明由  Hansschwarzkopf  提供 .

(2)

 于是 

$\lim _{n \rightarrow \infty}\left(\int_{0}^{1} \frac{\sin \left(x^{n}\right)}{x^{n}} \mathrm{~d} x\right)^{n}=\exp \left(\lim _{n \rightarrow \infty} n \sum_{k=1}^{\infty} \frac{(-1)^{k}}{(2 k+1) !(2 n k+1)}\right)=\exp \left(\lim _{n \rightarrow \infty} \sum_{k=1}^{\infty} \frac{(-1)^{k}}{(2 k+1) !\left(2 k+\frac{1}{n}\right)}\right) .$

 因为 
$$
\sum_{k=1}^{\infty} \frac{(-1)^{k}}{(2 k+1) !} \frac{1}{2 k+x}
$$
 在  $[0,1]$  上一致收敛 ,  从而 
$$
\lim _{n \rightarrow \infty} \sum_{k=1}^{\infty} \frac{(-1)^{k}}{(2 k+1) !} \frac{1}{2 k+\frac{1}{n}}=\lim _{x \rightarrow 0+} \sum_{k=1}^{\infty} \frac{(-1)^{k}}{(2 k+1) !} \frac{1}{2 k+x}=\sum_{k=1}^{\infty} \frac{(-1)^{k}}{2 k(2 k+1) !}
$$
 注   用  $\sin x$  的幂级数展开式进行变形及积分就容易得到结果 ,  唯一需要注意的是说明极限能交换运算次序   的理由 .

(3)  因为 
$$
\frac{1}{n} \sum_{k=1}^{n} \ln \left(1+\left(\frac{k-1}{n}\right)^{2}\right) \leqslant \frac{1}{n} \sum_{k=1}^{n} \ln \left(1+\frac{k^{2}-k}{n^{2}}\right) \leqslant \frac{1}{n} \sum_{k=1}^{n} \ln \left(1+\left(\frac{k}{n}\right)^{2}\right)
$$
$$
\begin{aligned}
& \left(\int_{0}^{1} \frac{\sin \left(x^{n}\right)}{x^{n}} \mathrm{~d} x\right)^{n}=\exp \left(n \ln \left(\int_{0}^{1} \frac{\sin \left(x^{n}\right)-x^{n}}{x^{n}} \mathrm{~d} x+1\right)\right) \\
& \frac{\sin \left(x^{n}\right)-x^{n}}{x^{n}}=\sum_{k=2}^{\infty} \frac{(-1)^{k-1}\left(x^{n}\right)^{2 k-2}}{(2 k-1) !}=\sum_{k=1}^{\infty} \frac{(-1)^{k}\left(x^{n}\right)^{2 k}}{(2 k+1) !} \\
& \left|\int_{0}^{1} \frac{\sin \left(x^{n}\right)-x^{n}}{x^{n}} \mathrm{~d} x\right|=\left|\sum_{k=1}^{\infty} \frac{(-1)^{k}}{(2 k+1) !(2 n k+1)}\right| \\
& \leqslant \sum_{k=1}^{\infty} \frac{1}{(2 k+1) !(2 n+1)} \\
& <\frac{\mathrm{e}}{2 n+1} \rightarrow 0,(n \rightarrow \infty) 
\end{aligned}
$$
$$
\lim _{n \rightarrow \infty} \frac{1}{n} \sum_{k=1}^{n} \ln \left(1+\left(\frac{k-1}{n}\right)^{2}\right)=\lim _{n \rightarrow \infty} \frac{1}{n} \sum_{k=1}^{n} \ln \left(1+\left(\frac{k}{n}\right)^{2}\right)=\int_{0}^{1} \ln \left(1+x^{2}\right) \mathrm{d} x=\ln 2-2+\frac{\pi}{2}
$$
 故 
$$
\lim _{n \rightarrow \infty} \frac{1}{n} \sum_{k=1}^{n} \ln \left(1+\frac{k^{2}-k}{n^{2}}\right)=\ln 2-2+\frac{\pi}{2} .
$$
 注   想到把题中算式与  Riemann  和联系起来就不难得出结果 ,  夹逼定理是个好东西 .

\begin{enumerate}
  \setcounter{enumi}{2}
  \item ( 证法一 )  定义 
\end{enumerate}
$$
F(x, y)=\frac{f(y)-f(x)}{y-x}, 0<x<y<1 .
$$
$D$,  且 
$$
\alpha=F\left(x_{1}, x_{2}\right)<\lambda<F\left(x_{3}, x_{4}\right)=\beta .
$$
$$
\lambda=\frac{f\left(x_{6}\right)-f\left(x_{5}\right)}{x_{6}-x_{5}} .
$$
$$
G(t)=\frac{f\left((1-t) x_{2}+t x_{4}\right)-f\left((1-t) x_{1}+t x_{3}\right)}{(1-t)\left(x_{2}-x_{1}\right)+t\left(x_{4}-x_{3}\right)},
$$
$$
x_{5}=\left(1-t_{0}\right) x_{1}+t_{0} x_{3}, x_{6}=\left(1-t_{0}\right) x_{2}+t_{0} x_{4}
$$
$$
\lambda=G\left(t_{0}\right)=\frac{f\left(x_{6}\right)-f\left(x_{5}\right)}{x_{6}-x_{5}}
$$
$3 .$
$$
|f(A)-f(B)|=\left|f\left(\gamma\left(t_{A}\right)\right)-f\left(\gamma\left(t_{B}\right)\right)\right|=\left|\int_{t_{B}}^{t_{A}} f^{\prime}(\gamma(t)) \gamma^{\prime}(t) \mathrm{d} t\right| \leqslant M\left|\int_{t_{B}}^{t_{A}}\right| \gamma^{\prime}(t)|\mathrm{d} t|=M L
$$
$$
\begin{aligned}
& \lim _{R \rightarrow 0^{+}} \frac{1}{R^{4}} \iint_{x^{2}+y^{2} \leqslant R^{2}}(f(x, y)-f(0,0)) \mathrm{d} x \mathrm{~d} y \\
=& \lim _{R \rightarrow 0^{+}} \frac{1}{R^{4}} \int_{0}^{2 \pi} \mathrm{d} \theta \int_{0}^{R}(f(r \cos \theta, r \sin \theta)-f(0,0)) r \mathrm{~d} r \\
=& \lim _{R \rightarrow 0^{+}} \frac{\int_{0}^{2 \pi}(f(R \cos \theta, R \sin \theta)-f(0,0)) R \mathrm{~d} \theta}{4 R^{3}} \\
=& \lim _{R \rightarrow 0^{+}} \frac{\int_{0}^{2 \pi} f(R \cos \theta, R \sin \theta)-f(0,0) \mathrm{d} \theta}{4 R^{2}} \\
=& \lim _{R \rightarrow 0^{+}} \frac{\int_{0}^{2 \pi} f_{x}(R \cos \theta, R \sin \theta) \cos \theta+f_{y}(R \cos \theta, R \sin \theta) \sin \theta \mathrm{d} \theta}{8 R} \\
=& \lim _{R \rightarrow 0^{+}} \frac{\int_{0}^{2 \pi} f_{x x}(R \cos \theta, R \sin \theta) \cos 2 \theta+2 f_{x y}(R \cos \theta, R \sin \theta) \sin \theta \cos \theta+f_{y y}(R \cos \theta, R \sin \theta) \sin { }^{2} \theta \mathrm{d} \theta}{8} \\
=& \frac{\pi}{8} f_{x x}(0,0)+\frac{\pi}{8} f_{y y}(0,0) .
\end{aligned}
$$

\begin{enumerate}
  \setcounter{enumi}{5}
  \item  由题意知  $\nabla f(x, \varphi(x))=\left(f_{x}(x, \varphi(x)), f_{y}(x, \varphi(x))\right)=(0,0)$,
\end{enumerate}
$$
A=\left(\begin{array}{ll}
f_{x x}(0,0) & f_{x y}(0,0) \\
f_{y x}(0,0) & f_{y y}(0,0)
\end{array}\right)
$$
 半正定非零 .  由 
$$
\left\{\begin{array}{l}
f_{x}(x, \varphi(x))=0 \\
f_{y}(x, \varphi(x))=0
\end{array} \quad, \quad \text { 令 } x=0 \Longrightarrow\left\{\begin{array}{l}
f_{x}(0,0)=0 \\
f_{y}(0,0)=0
\end{array}\right. \text {. }\right.
$$
 根据题中条件可得 
$$
\begin{aligned}
0 &=\lim _{h \rightarrow 0} \frac{f_{x}(h, \varphi(h))-f_{x}(0, \varphi(0))}{h} \\
&=\lim _{h \rightarrow 0} \frac{f_{x}(h, \varphi(h))-f_{x}(0, \varphi(h))}{h}+\lim _{h \rightarrow 0} \frac{f_{x}(0, \varphi(h))-f_{x}(0,0)}{h} \\
&=f_{x x}(0,0)+f_{x y}(0,0) \varphi^{\prime}(0)=0
\end{aligned}
$$
 类似地可以得到  $f_{y x}(0,0)+f_{y y}(0,0) \varphi^{\prime}(0)=0$.  因此  $\left(1, \varphi^{\prime}(0)\right)^{\mathrm{T}}$  为  $A$  的属于特征值  0  的一个特征向量 .  实   对称矩阵  $A$  的不同特征值对应的特征向量相互正交 ,  故可设  $\left(-\varphi^{\prime}(0), 1\right)^{\mathrm{T}}$  为  $A$  的属于正特征值  $\lambda$  的一个   特征向量 .  于是 
$$
\begin{aligned}
A &=\frac{1}{1+\left(\varphi^{\prime}(0)\right)^{2}}\left(\begin{array}{cc}
1 & -\varphi^{\prime}(0) \\
\varphi^{\prime}(0) & 1
\end{array}\right)\left(\begin{array}{ll}
0 & 0 \\
0 & \lambda
\end{array}\right)\left(\begin{array}{cc}
1 & \varphi^{\prime}(0) \\
-\varphi^{\prime}(0) & 1
\end{array}\right) \\
&=\frac{\lambda}{1+\left(\varphi^{\prime}(0)\right)^{2}}\left(\begin{array}{cc}
\left(\varphi^{\prime}(0)\right)^{2} & -\varphi^{\prime}(0) \\
-\varphi^{\prime}(0) & 1
\end{array}\right)
\end{aligned}
$$
 由此我们得到  $f_{y y}(0,0)>0$,  于是   由连续性知 :  存在  $(0,0)$  的一个小邻域 ,  在那上面有  $f_{y y}(x, y)>0$.  下面的   推演都在前面这个小邻域上进行 .  对于固定的  $x$  而言 , $f_{y}(x, y)$  关于  $y$  单调递增 ,  注意到  $f_{y}(x, \varphi(x))=0$,  故   当  $y>\varphi(x)$  时 , $f_{y}(x, y)>0$,  从而  $f(x, y)$  关于  $y$  在  $y>\varphi(x)$  上单增 ,  从而  $f(x, y)>f(x, \varphi(x)), \forall y>\varphi(x)$;  当  $y<\varphi(x)$  时 , $f_{y}(x, y)<0$,  从而  $f(x, y)$  关于  $y$  在  $y<\varphi(x)$  上单减 ,  从而  $f(x, y)>f(x, \varphi(x)), \forall y<\varphi(x)$.  综合这两方面的结果得到  $f(x, y) \geqslant f(x, \varphi(x))$.

 由于  $f_{y y}(x, y) \neq 0$,  对方程  $f_{y}(x, \varphi(x))=0$  用隐函数定理可得  $\varphi(x)$  连续可微 .  从而 
$$
f\left(x_{1}, \varphi\left(x_{1}\right)\right)-f\left(x_{2}, \varphi\left(x_{2}\right)\right)=\left(f_{x}(\xi, \varphi(\xi))+f_{y}(\xi, \varphi(\xi)) \varphi^{\prime}(\xi)\right)\left(x_{1}-x_{2}\right)=0
$$
 故  $f(x, \varphi(x))=f(0, \varphi(0))=f(0,0)$.  因此  $f(x, y) \geqslant f(x, \varphi(x))=f(0,0)$.  何上看图像差不多是山沟那种形状 ,  具体地可以考虑函数  $f(x, y)=(x-y)^{2}$,  直线  $y=x$  上的点都是极小值 

\begin{enumerate}
  \setcounter{enumi}{6}
  \item  令  $\varphi(x)=\mathrm{e}^{-x}+\cos (2 x)+x \sin x$,  则  $\varphi^{\prime}(x)=-\mathrm{e}^{-x}-2 \sin (2 x)+\sin x+x \cos x$.
\end{enumerate}
$$
\varphi((2 n-1) \pi)=\mathrm{e}^{-(2 n-1) \pi}+1>0, \quad \varphi\left((2 n-1) \pi+\frac{\pi}{2}\right)=\mathrm{e}^{-\left((2 n-1) \pi+\frac{\pi}{2}\right)}-1-\left((2 n-1) \pi+\frac{\pi}{2}\right)>0
$$
$$
\varphi(x)=\mathrm{e}^{-x}+\cos ^{2} x+(x-\sin x) \sin x>0
$$
 于是  $\exists x_{2 n-1} \in\left((2 n-1) \pi,(2 n-1) \pi+\frac{\pi}{2}\right), x_{2 n} \in\left((2 n-1) \pi+\frac{\pi}{2}, 2 n \pi\right)$  使得  $\varphi\left(x_{2 n-1}\right)=\varphi\left(x_{2 n}\right)=0$.  下面   证明  $\varphi(x)=0$  在  $((2 n-1) \pi,(2 n+1) \pi)$  上只有上述两个根 .

 当  $x \in\left((2 n-1) \pi+\frac{\pi}{4},(2 n-1) \pi+\frac{3 \pi}{4}\right)$  时 ,
$$
\varphi(x)<-\frac{\sqrt{2}}{2}\left((2 n-1) \pi+\frac{\pi}{4}\right)+1+1<0,
$$
 故  $\varphi(x)=0$  的两个根在  $\left((2 n-1) \pi,(2 n-1) \pi+\frac{\pi}{4}\right)$  与  $\left((2 n-1) \pi+\frac{3 \pi}{4}, 2 n \pi\right)$  上 .

 当  $x \in\left((2 n-1) \pi,(2 n-1) \pi+\frac{\pi}{4}\right)$  时 ,
$$
\varphi^{\prime}(x)<-\frac{\sqrt{2}}{2}(2 n-1) \pi<0
$$
 当  $x \in\left((2 n-1) \pi+\frac{3 \pi}{4}, 2 n \pi\right)$  时 ,
$$
\varphi^{\prime}(x)>\frac{\sqrt{2}}{2}\left((2 n-1) \pi+\frac{3 \pi}{4}\right)-1>0 .
$$
 由函数的单调性确定了根的个数 .

 因为  $\lim _{n \rightarrow \infty} x_{n}=+\infty$,  故 
$$
1=\lim _{n \rightarrow \infty} \mathrm{e}^{-x_{n}}+1=\lim _{n \rightarrow \infty} \sin x_{n}\left(2 \sin x_{n}-x_{n}\right)=\lim _{n \rightarrow \infty}(-1)^{n} \sin \left(x_{n}-n \pi\right)\left(2 \sin x_{n}-x_{n}\right) .
$$
 因此  $\lim _{n \rightarrow \infty}\left(x_{n}-n \pi\right)=0$,
$$
\begin{aligned}
\lim _{n \rightarrow \infty}(-1)^{n} n\left(x_{n}-n \pi\right) &=\lim _{n \rightarrow \infty} \frac{(-1)^{n} n\left(x_{n}-n \pi\right)}{(-1)^{n} \sin \left(x_{n}-n \pi\right)\left(2 \sin x_{n}-x_{n}\right)} \\
&=\lim _{n \rightarrow \infty} \frac{n}{2 \sin x_{n}-x_{n}} \\
&=-\frac{1}{\pi} .
\end{aligned}
$$
 注   计算出题中式子的极限不难 ,  说明根的个数是本题的难点 .

7 .  设  $0<x \leqslant \pi$,  令 
$$
S_{N}(x)=\sum_{n=1}^{N} \frac{\cos (n x)}{n}
$$
 则 
$$
S_{N}^{\prime}(x)=-\sum_{n=1}^{N} \sin (n x)=\frac{\cos \left(\left(N+\frac{1}{2}\right) x\right)-\cos \frac{x}{2}}{2 \sin \frac{x}{2}}
$$
 于是 
$$
S_{N}(x)=\int_{\pi}^{x} S_{N}^{\prime}(t) \mathrm{d} t+S_{N}(\pi)=\int_{\pi}^{x} \frac{\cos \left(\left(N+\frac{1}{2}\right) t\right)-\cos \frac{t}{2}}{2 \sin \frac{t}{2}} \mathrm{~d} t+\sum_{n=1}^{N} \frac{(-1)^{n}}{n}
$$
 因此结合  Riemann-Lebesgue  引理知原级数的和函数 
$$
S(x)=\sum_{n=1}^{\infty} \frac{\cos (n x)}{n}=\lim _{N \rightarrow \infty} S_{N}(x)=-\ln \left|\sin \frac{x}{2}\right|-\ln 2=-\ln \left|2 \sin \frac{x}{2}\right|
$$
 故  $\lim _{x \rightarrow 0^{+}} S(x)=+\infty$,  又由于  $S(x)=S(-x)$,  故  $\lim _{x \rightarrow 0} S(x)=+\infty$.  京大学  2017  年数学分析第  9  题中的  $S_{n}(x)$  其实也是这样求出来的 .  谢惠民等人编写的 《 数学分析习题课   讲义 》 下册  117  页例题  $16.2 .5$  用幂级数也求出了这个和式 .  林源渠与方企勤的 《 数学分析解题指南 》 在第  247  页练习题中给出了和函数 . 8. ( 法一 )  由二阶导数非正 , $f^{\prime}(x)$  在  $[1,+\infty)$  单减 ,  容易看出  $f^{\prime}$  恒正 .  事实上若有某个  $x_{0}, f^{\prime}\left(x_{0}\right) \leqslant 0$  则由单调   性 
$$
\forall x \geqslant x_{0}, f^{\prime}(x) \leqslant f^{\prime}\left(x_{0}\right) \leqslant 0, f(x) \leqslant f\left(x_{0}\right)
$$
 与  $f(+\infty)=+\infty$  矛盾 .  因此  $f$  在  $[1,+\infty)$  严增 .  我们将收敛性的证明与求值放在一起进行 .
$$
S_{2 n}(s)=\sum_{k=1}^{n}\left(\frac{1}{f^{s}(2 k)}-\frac{1}{f^{s}(2 k-1)}\right)
$$
 注意和式中每个括号都是负的且级数通项趋于  0 ,  只需要证明对固定的  $s>0, S_{2 n}(s)$  有下界 ,  则 
$$
\lim _{n \rightarrow+\infty} S_{2 n}(s)
$$
 存在且等于 
$$
\sum_{n=1}^{+\infty} \frac{(-1)^{n}}{f^{s}(n)}
$$
 由  Lagrange  中值定理 ,
$$
\frac{1}{f^{s}(2 k)}-\frac{1}{f^{s}(2 k-1)}=\frac{-s f^{\prime}(\xi)}{f^{s+1}(\xi)}, \quad \xi \in(2 k-1,2 k)
$$
 注意  $f$  单增而  $f^{\prime}$  单减我们有 
$$
\begin{gathered}
\frac{-s f^{\prime}(2 k-1)}{f^{s+1}(2 k-1)} \leqslant \frac{1}{f^{s}(2 k)}-\frac{1}{f^{s}(2 k-1)} \leqslant \frac{-s f^{\prime}(2 k)}{f^{s+1}(2 k)} \\
\sum_{k=1}^{n} \frac{-s f^{\prime}(2 k-1)}{f^{s+1}(2 k-1)} \leqslant S_{2 n}(s) \leqslant \sum_{k=1}^{n} \frac{-s f^{\prime}(2 k)}{f^{s+1}(2 k)}
\end{gathered}
$$
 利用面积原理的思想来估计左右两端 .  由单调性  $k \geqslant 2$  时 
$$
\begin{gathered}
\frac{f^{\prime}(2 k-1)}{f^{s+1}(2 k-1)} \leqslant \frac{1}{2} \int_{2 k-3}^{2 k-1} \frac{f^{\prime}(t)}{f^{s+1}(t)} \mathrm{d} t \\
\sum_{k=2}^{+\infty} \frac{f^{\prime}(2 k-1)}{f^{s+1}(2 k-1)} \leqslant \frac{1}{2} \int_{1}^{+\infty} \frac{f^{\prime}(t)}{f^{s+1}(t)} \mathrm{d} t=\left.\frac{1}{2} \frac{-1}{s f^{s}(t)}\right|_{t=1} ^{t=+\infty}=\frac{1}{2 s f^{s}(1)}
\end{gathered}
$$
$S_{2 n}(s)$  有下界故极限存在 .  再次利用面积原理  $k \geqslant 1$  时 
$$
\begin{gathered}
\frac{f^{\prime}(2 k)}{f^{s+1}(2 k)} \geqslant \frac{1}{2} \int_{2 k}^{2 k+2} \frac{f^{\prime}(t)}{f^{s+1}(t)} \mathrm{d} t \\
\sum_{k=1}^{+\infty} \frac{f^{\prime}(2 k)}{f^{s+1}(2 k)} \geqslant \frac{1}{2} \int_{2}^{+\infty} \frac{f^{\prime}(t)}{f^{s+1}(t)} \mathrm{d} t=\left.\frac{1}{2} \frac{-1}{s f^{s}(t)}\right|_{t=2} ^{t=+\infty}=\frac{1}{2 s f^{s}(2)} \\
-s\left(\frac{f^{\prime}(1)}{f^{s+1}(1)}+\frac{1}{2 s f^{s}(1)}\right) \leqslant \lim _{n \rightarrow+\infty} S_{2 n}(s) \leqslant-s \frac{1}{2 s f^{s}(2)}
\end{gathered}
$$
$$
-s\left(\frac{f^{\prime}(1)}{f^{s+1}(1)}+\frac{1}{2 s f^{s}(1)}\right) \leqslant \sum_{n=1}^{+\infty} \frac{(-1)^{n}}{f^{s}(n)} \leqslant-s \frac{1}{2 s f^{s}(2)}
$$
$$
\lim _{s \rightarrow 0^{+}} \sum_{n=1}^{+\infty} \frac{(-1)^{n}}{f^{s}(n)}=-\frac{1}{2}
$$
( 法二 )  利用解析数论教材中可以找到的  Euler-Maclaurin  求和法得到 :
$$
\begin{aligned}
\sum_{n=1}^{2 N+1} \frac{(-1)^{n}}{f^{s}(n)} &=\frac{-1}{f^{s}(1)}+\sum_{1<n \leqslant N}\left(\frac{1}{f^{s}(2 n)}-\frac{1}{f^{s}(2 n+1)}\right) \\
&=\frac{-1}{f^{s}(1)}+\int_{1}^{N}\left(\frac{1}{f^{s}(2 u)}-\frac{1}{f^{s}(2 u+1)}\right) \mathrm{d}([u]-u+u) \\
&=\frac{-1}{f^{s}(1)}+\int_{1}^{N}\left(\frac{1}{f^{s}(2 u)}-\frac{1}{f^{s}(2 u+1)}\right) \mathrm{d} u+\int_{1}^{N}(u-[u])\left(\frac{-2 s f^{\prime}(2 u)}{f^{s+1}(2 u)}+\frac{2 s f^{\prime}(2 u+1)}{f^{s+1}(2 u+1)}\right) \mathrm{d} u \\
&=\frac{-1}{f^{s}(1)}+\frac{1}{2}\left(\int_{2}^{2 N} \frac{\mathrm{d} t}{f^{s}(t)}-\int_{3}^{2 N+1} \frac{\mathrm{d} t}{f^{s}(t)}\right)+2 s \int_{1}^{N}(u-[u])\left(\frac{f^{\prime}(2 u+1)}{f^{s+1}(2 u+1)}-\frac{f^{\prime}(2 u)}{f^{s+1}(2 u)}\right) \mathrm{d} u \\
&=\frac{-1}{f^{s}(1)}+\frac{1}{2}\left(\int_{2}^{3} \frac{\mathrm{d} t}{f^{s}(t)}-\int_{2 N}^{2 N+1} \frac{\mathrm{d} t}{f^{s}(t)}\right)+2 s \int_{1}^{N}(u-[u])\left(\frac{f^{\prime}(2 u+1)}{f^{s+1}(2 u+1)}-\frac{f^{\prime}(2 u)}{f^{s+1}(2 u)}\right) \mathrm{d} u
\end{aligned}
$$
 当  $s>0$  时 ,  令  $N$  趋于正无穷可得 
$$
\sum_{n=1}^{+\infty} \frac{(-1)^{n}}{f^{s}(n)}=\frac{-1}{f^{s}(1)}+\frac{1}{2} \int_{2}^{3} \frac{\mathrm{d} t}{f^{s}(t)}+2 s \int_{1}^{+\infty}(u-[u])\left(\frac{f^{\prime}(2 u+1)}{f^{s+1}(2 u+1)}-\frac{f^{\prime}(2 u)}{f^{s+1}(2 u)}\right) \mathrm{d} u
$$
 由 
$$
\left(\frac{f^{\prime}(x)}{f^{s+1}(x)}\right)^{\prime}=\frac{f^{\prime \prime}(x) f^{s+1}(x)-(s+1)\left(f^{\prime}(x)\right)^{2} f^{s}(x)}{f^{2(s+1)}(x)} \leqslant 0,
$$
 因此  $\frac{f^{\prime}(x)}{f^{s+1}(x)}$  在  $[1,+\infty)$  上单调递减 ,  于是有 
$$
\begin{aligned}
\left|\int_{1}^{+\infty}(u-[u])\left(\frac{f^{\prime}(2 u+1)}{f^{s+1}(2 u+1)}-\frac{f^{\prime}(2 u)}{f^{s+1}(2 u)}\right) \mathrm{d} u\right| & \leqslant \int_{1}^{+\infty}\left(\frac{f^{\prime}(2 u)}{f^{s+1}(2 u)}-\frac{f^{\prime}(2 u+1)}{f^{s+1}(2 u+1)}\right) \mathrm{d} u \\
&=\frac{1}{2} \int_{2}^{3} \frac{f^{\prime}(t)}{f^{s+1}(t)} \mathrm{d} t \leqslant \frac{1}{2} \int_{2}^{3} \frac{f^{\prime}(t)}{f(t)} \mathrm{d} t
\end{aligned}
$$
 敇 
$$
\lim _{s \rightarrow 0^{+}} \sum_{n=1}^{+\infty} \frac{(-1)^{n}}{f^{s}(n)}=\lim _{s \rightarrow 0^{+}}\left(\frac{-1}{f^{s}(1)}+\frac{1}{2} \int_{2}^{3} \frac{\mathrm{d} t}{f^{s}(t)}\right)=-\frac{1}{2}
$$
 注   证法一由  TangSong  提供 ,  证明的关键点是把级数与广义积分相联系并利用夹逼定理 .  法二为我原创的 ,  关   键在于知道那个解析数论教材中常见的把有限和变为积分的公式 ,  这么来看就是先求出和函数再求极限 .  为   了更好地理解证明 ,  读者可以考虑  $f(x)=x$  的情形 .  北京大学  2019  年全国硕士研究生招生考试数学分析试题及解答     

2019.03.23

\begin{enumerate}
  \item (10  分 )  讨论数列  $a_{n}=\sqrt[n]{1+\sqrt[n]{2+\cdots+\sqrt[n]{n}}}$  的玫散性 .

  \item (10  分 ) $f(x) \in C[a, b]$  且  $f(a)=f(b)$.  证明存在  $x_{n}, y_{n} \in[a, b]$,  使得  $\lim _{n \rightarrow+\infty}\left(x_{n}-y_{n}\right)=0$  且  $f\left(x_{n}\right)=$ $f\left(y_{n}\right), \forall n \in \mathbb{N}^{*}$.

  \item (10  分 )  证明恒等式  $\sum_{k=0}^{n}(-1)^{k} C_{n}^{k} \frac{1}{k+m+1}=\sum_{k=0}^{m}(-1)^{k} C_{m}^{k} \frac{1}{k+n+1}$.

  \item (10  分 )  已知无穷乘积  $\prod_{n=1}^{+\infty}\left(1+a_{n}\right)$  收玫 ,  是否有  $\sum_{n=1}^{+\infty} a_{n}$  收玫 ?  证明或者举出反例 .

  \item (10  分 )  设  $f(x)=\sum_{n=1}^{+\infty} x^{n} \ln x$,  求  $\int_{0}^{1} f(x) \mathrm{d} x$.

  \item (20  分 ) $f(x)$  为  $(0,+\infty)$  上二次可微函数 ,  若  $\lim _{x \rightarrow+\infty} f(x)$  存在 , $f^{\prime \prime}(x)$  有界 .  证明  $\lim _{x \rightarrow+\infty} f^{\prime}(x)=0$.

  \item (20  分 )  数列  $\left\{x_{n}\right\}$  有界 ,  且  $\lim _{n \rightarrow+\infty}\left(x_{n+1}-x_{n}\right)=0, \underline{\lim }_{n \rightarrow+\infty} x_{n}=l, \varlimsup_{n \rightarrow+\infty} x_{n}=L,-\infty<l<L<+\infty$.  证明  $\forall c \in[l, L]$,  都有  $\left\{x_{n}\right\}$  的子列收敛于  $c$.

  \item ( 20  分 ) $p>0$,  讨论级数 

\end{enumerate}
$$
\sum_{n=1}^{+\infty} \frac{\sin \frac{n \pi}{4}}{n^{p}+\sin \frac{n \pi}{4}}
$$
 的绝对敛散性和条件敛散性 .

\begin{enumerate}
  \setcounter{enumi}{9}
  \item (20  分 )  求  $f(x)=\frac{2 x \sin \theta}{1-2 x \cos \theta+x^{2}}$  在  $x=0$  处的  Taylor  展开式 ,  并求  $\int_{0}^{\pi} \ln \left(1-2 x \cos \theta+x^{2}\right) \mathrm{d} \theta$.

  \item (20  分 )  证明  $\int_{0}^{+\infty} \frac{\sin x}{x} \mathrm{~d} x=\frac{\pi}{2}$,  求  $\int_{0}^{+\infty} \frac{\sin ^{2}(y x)}{x^{2}} \mathrm{~d} x$. 1.  因为 

\end{enumerate}
$$
1 \leqslant a_{n} \leqslant \sqrt[n]{\frac{n(n+1)}{2}} \leqslant \sqrt[n]{n^{2}}=(\sqrt[n]{n})^{2}
$$
 又因为  $\lim _{n \rightarrow \infty}(\sqrt[n]{n})^{2}=1$,  由夹逼定理知  $\lim _{n \rightarrow \infty} a_{n}=1$.

 注   求复杂结构式子的极限时 ,  夹逼定理是王道 .  此题与谢惠民等人的 《 数学分析习题课讲义 》 上册第二章第二   组参考题第一题有些类似 . 这题你只要想到用夹逼定理基本上就能自己写出来 ,  上述证明中的放缩方式是   小小提出的 .  另外的放缩方式可以是用几何 - 算术平均值不等式 ,  如果连续用  $n$  次也可估计出来 ,  不过用一   次就行了 ;  或者从内到外放缩 ,  先注意到  $1 \leqslant \sqrt[n]{n}<2$,  就有  $\sqrt[n]{(n-1)+\sqrt[n]{n}<\sqrt[n]{n+1}<3 \text {, 以此类推, 最 }$  后可以得到  $1 \leqslant a_{n} \leqslant \sqrt[n]{1+2}$.

2 .  取  $x_{n}=y_{n}=a, n \in \mathbb{N}^{*}$  即可 .

 注   出题老师大意了 ,  我仔细看了题干几次 ,  原题并没有添加其他要求 ,  说原题有条件  $x_{n} \neq y_{n}$  的人纯属胡扯 ,  这题是送分题 .

\begin{enumerate}
  \setcounter{enumi}{3}
  \item  设  $B(m, n, x)=\sum_{k=0}^{n}(-1)^{k} C_{n}^{k} \frac{x^{m+k+1}}{m+k+1}$,  则有  $\frac{\mathrm{d} B(m, n, x)}{\mathrm{d} x}=\sum_{k=0}^{n}(-1)^{k} C_{n}^{k} x^{m+k}=x^{m}(1-x)^{n}$.  于是 
\end{enumerate}
$$
\begin{aligned}
\sum_{k=0}^{n}(-1)^{k} C_{n}^{k} \frac{1}{k+m+1}=B(m, n, 1) &=\int_{0}^{1} x^{m}(1-x)^{n} \mathrm{~d} x \\
&=\int_{0}^{1} x^{n}(1-x)^{m} \mathrm{~d} x=B(n, m, 1)=\sum_{k=0}^{m}(-1)^{k} C_{m}^{k} \frac{1}{k+n+1}
\end{aligned}
$$
 注   此题可以看作是对 《 数学分析习题课讲义 》 下册第  28,29  页所述内容的一个补充 .  陈纪修等人编的 《 数学   分析 》 第二版下册第  54  页第  7  题给了另外一个例子 .

\begin{enumerate}
  \setcounter{enumi}{5}
  \item  因为 
\end{enumerate}
$$
\begin{aligned}
\int_{0}^{1} f(x) \mathrm{d} x &=\int_{0}^{1} \sum_{k=1}^{N} x^{k} \ln x \mathrm{~d} x+\int_{0}^{1} \sum_{k=N+1}^{\infty} x^{k} \ln x \mathrm{~d} x \\
&=\sum_{k=1}^{N} \int_{0}^{1} x^{k} \ln x \mathrm{~d} x+\int_{0}^{1} \frac{x^{N+1}}{1-x} \ln x \mathrm{~d} x \\
&=-\sum_{k=1}^{N} \frac{1}{(k+1)^{2}}+\int_{0}^{1} x^{N} \frac{x \ln x}{1-x} \mathrm{~d} x
\end{aligned}
$$
 又因为  $\frac{x \ln x}{1-x}$  在  $(0,1)$  上连续 , $\lim _{x \rightarrow 0^{+}} \frac{x \ln x}{1-x}=0, \lim _{x \rightarrow 1^{-}} \frac{x \ln x}{1-x}=-1$,  因此  $\left|\frac{x \ln x}{1-x}\right|$  在  $[0,1]$  上有上界  $M$,  故 

 注   此题为北京大学  1987  年数学分析考研真题第四题第二问 ,  它与裴礼文的 《 数学分析中的典型问题与方法 》  第二版第  529  页例  $5.2 .52$  类似 . 6.  因  $f^{\prime \prime}(x)$  在  $(0,+\infty)$  有界 ,  故  $f^{\prime}(x)$  在  $(0,+\infty)$  上一致连续 . $\lim _{x \rightarrow+\infty} f(x)$  存在 ,  说明  $\int_{1}^{+\infty} f^{\prime}(x) \mathrm{d} x$  有意义 ,  下面证明  $\lim _{x \rightarrow+\infty} f^{\prime}(x)=0$.  若  $\lim _{x \rightarrow+\infty} f^{\prime}(x) \neq 0$,  则存在  $\varepsilon_{0}>0$,  对于任意的  $x>0, \exists t>x$,  使得  $\left|f^{\prime}(t)\right| \geqslant \varepsilon_{0}$,  于是存在  $x_{n}$  单调递增趋于  $+\infty$,  且  $\left|f^{\prime}\left(x_{n}\right)\right| \geqslant \varepsilon_{0}$.  由一致连续 ,  对于上述  $\varepsilon_{0}, \exists \delta_{0}>0$.  当  $|x-y|<\delta_{0}$  时 , $\left|f^{\prime}(x)-f^{\prime}(y)\right|<\varepsilon_{0} / 2$.  从而有 
$$
\begin{aligned}
\left|\int_{x_{n}}^{x_{n}+\delta_{0}} f^{\prime}(x) \mathrm{d} x\right| &=\left|\int_{x_{n}}^{x_{n}+\delta_{0}} f^{\prime}(x)-f^{\prime}\left(x_{n}\right)+f^{\prime}\left(x_{n}\right) \mathrm{d} x\right|-\mid \int_{x_{n}}^{x_{n}+\delta_{0}} f^{\prime}(x)-f^{\prime}\left(x_{n}\right) \mathrm{d} x \\
& \geqslant\left|\int_{x_{n}}^{x_{n}+\delta_{0}} f^{\prime}\left(x_{n}\right) \mathrm{d} x\right|-\varepsilon_{0} \delta_{0} \\
& \geqslant \varepsilon_{0} \delta_{0}-\frac{\varepsilon_{0}}{2} \\
&=\frac{\varepsilon_{0} \delta_{0}}{2}
\end{aligned}
$$
 这与  $\int_{1}^{+\infty} f^{\prime}(x) \mathrm{d} x$  有意义的  Cauchy  收敛原理予盾 .

 注   裴礼文的 《 数学分析中的典型问题与方法 》 第二版第  249  页例  $3.3 .11$  与本题几乎完全相同 ,  那里有另外一   种证明方法 .  我写的这个解法是源于一个很经典的题目 ,  可以见 《 数学分析习题课讲义 》 上册第  396  页命题  $12.4 .1$,  陈纪修等人编著的 《 数学分析 》 第二片上册第  379  页例  $8.2 .9$,  林源渠 、 方企勤编的 《 数学分析解题   指南 》 第  417  页题  $7.11$,  裴礼文 《 数学分析中的典型问题与方法 》 第二版第  415  页例  $4.5 .24$.

\begin{enumerate}
  \setcounter{enumi}{7}
  \item  首先  $l$  与  $L$  均是  $\left\{x_{n}\right\}$  的某个子列的极限 .  下面任取  $l<c<L$,  则  $\exists x_{n_{1}}$,  使得  $\frac{l+c}{2}<x_{n_{1}} \leqslant c$,  否则  $\forall n \in \mathbb{N}$,  要么  $x_{n} \leqslant \frac{l+c}{2}$,  要么  $x_{n}>c$,  结合  $\varliminf_{n \rightarrow+\infty} x_{n}=l, \varlimsup_{n \rightarrow+\infty} x_{n}=L$,  知  $\left\{x_{n}\right\}$  中有无穷项小于等于  $\frac{l+c}{2}$,  有无穷项   大于  $c$.  从而  $\left|x_{n+1}-x_{n}\right|$  有无穷多项大于等于  $\frac{c-l}{2}$,  矛盾 .  类似地 ,  存在  $n_{2}>n_{1}$  使得  $\frac{x_{n_{1}}+c}{2}<x_{n_{2}} \leqslant c$.  以 
\end{enumerate}
 注   几乎完全一样的题目见 《 数学分析习题课讲义 》 上册  96  页第三章第二组参考题第  10  题或者裴礼文的 《 数 
$$
\sum_{n=1}^{+\infty} \frac{\sin \frac{n \pi}{4}}{n^{p}+\sin \frac{n \pi}{4}}-\sum_{n=1}^{+\infty} \frac{\sin \frac{n \pi}{4}}{n^{p}}=-\sum_{n=1}^{+\infty} \frac{\sin ^{2} \frac{n \pi}{4}}{\left(n^{p}+\sin \frac{n \pi}{4}\right) n^{p}},
$$
$$
\frac{\sin ^{2} \frac{n \pi}{4}}{\left(n^{p}+\sin \frac{n \pi}{4}\right) n^{p}} \sim \frac{\sin ^{2} \frac{n \pi}{4}}{n^{2 p}}=\frac{1-\cos \frac{n \pi}{2}}{n^{2 p}},(n \rightarrow+\infty)
$$
 注   解题思路与 《 数学分析习题课讲义 》 上册  284  页第十二章例题  $12.2 .4$  一样 ,  可以算作改编题 .
$$
f(x)=2 \sum_{n=1}^{\infty} \sin (n \theta) x^{n}, x \in(-1,1) .
$$
 记  $I(x)=\int_{0}^{\pi} \ln \left(1-2 x \cos \theta+x^{2}\right) \mathrm{d} \theta$,  则  $I(0)=0$.

 当  $|x|<1$  时 ,
$$
I^{\prime}(x)=\int_{0}^{\pi} \frac{-2 \cos \theta+2 x}{1-2 x \cos \theta+x^{2}} \mathrm{~d} \theta=\frac{1}{2} \int_{0}^{2 \pi} \frac{-2 \cos \theta+2 x}{1-2 x \cos \theta+x^{2}} \mathrm{~d} \theta=\frac{1}{2} \int_{|z|=1} \frac{1}{\mathrm{i} z} \frac{-\left(z+\frac{1}{z}\right)+2 x}{1-\left(z+\frac{1}{z}\right) x+x^{2}} \mathrm{~d} z=0
$$
 因此  $I(x)=0, x \in(-1,1)$.

 当  $|x|>1$  时 ,
$$
I(x)=\int_{0}^{\pi} \ln \left(1-2 x \cos \theta+x^{2}\right) \mathrm{d} \theta=\int_{0}^{\pi} \ln \left(x^{2}\right)+\ln \left(1-2 \frac{1}{x} \cos \theta+\frac{1}{x^{2}}\right) \mathrm{d} \theta=\int_{0}^{\pi} \ln \left(x^{2}\right) \mathrm{d} \theta=2 \pi \ln |x|
$$
 又因为 
$$
\begin{aligned}
I(1) &=\int_{0}^{\pi} \ln (2-2 \cos \theta) \mathrm{d} \theta=2 \int_{0}^{\pi} \ln \left(2 \sin \frac{\theta}{2}\right) \mathrm{d} \theta=2 \pi \ln 2+4 \int_{0}^{\frac{\pi}{2}} \ln (\sin \theta) \mathrm{d} \theta \\
I(-1) &=\int_{0}^{\pi} \ln (2+2 \cos \theta) \mathrm{d} \theta=2 \int_{0}^{\pi} \ln \left(2 \cos \frac{\theta}{2}\right) \mathrm{d} \theta=2 \pi \ln 2+4 \int_{0}^{\frac{\pi}{2}} \ln (\cos \theta) \mathrm{d} \theta
\end{aligned}
$$
 今 
$$
I_{1}=\int_{0}^{\frac{\pi}{2}} \ln (\sin \theta) \mathrm{d} \theta=\int_{0}^{\frac{\pi}{2}} \ln (\cos \theta) \mathrm{d} \theta
$$
 则 
$$
2 I_{1}=\int_{0}^{\frac{\pi}{2}} \ln \left(\frac{1}{2} \sin 2 \theta\right) \mathrm{d} \theta=-\frac{\pi \ln 2}{2}+\frac{1}{2} \int_{0}^{\pi} \ln (\sin \theta) \mathrm{d} \theta=-\frac{\pi \ln 2}{2}+I_{1}
$$
 于是 
$$
I_{1}=-\frac{\pi \ln 2}{2}, \quad I(1)=I(-1)=0
$$
 综上得 
$$
I(x)= \begin{cases}0, & |x|<1 \\ 2 \pi \ln |x|, & |x| \geqslant 1\end{cases}
$$
 计算当  $|x|<1$  时 , $I(x)$  的表达式的另一种方法 :

 对于固定的  $x \in(-1,1)$,  由  Weierstrass  判别法知关于  $\theta$  的级数  $2 \sum_{n=1}^{\infty} x^{n} \sin (n \theta)$  在  $\mathbb{R}$  上一致收敛到 
$$
\begin{gathered}
\int_{0}^{\alpha} \frac{2 x \sin \theta}{1-2 x \cos \theta+x^{2}} \mathrm{~d} \theta=2 \sum_{n=1}^{\infty} \int_{0}^{\alpha} x^{n} \sin (n \theta) \mathrm{d} \theta \\
\ln \left(1-2 x \cos \alpha+x^{2}\right)-2 \ln (1-x)=-2 \sum_{n=1}^{\infty} \frac{\cos (n \alpha)}{n} x^{n}+2 \sum_{n=1}^{\infty} \frac{x^{n}}{n} \\
\ln \left(1-2 x \cos \alpha+x^{2}\right)=-2 \sum_{n=1}^{\infty} \frac{\cos (n \alpha)}{n} x^{n}, \quad \alpha \in \mathbb{R}
\end{gathered}
$$
 对于固定的  $x \in(-1,1)$,  同样由  Weierstrass  判别法知关于  $\theta$  的级数  $-2 \sum_{n=1}^{\infty} \frac{\cos (n \theta)}{n} x^{n}$  在  $\mathbb{R}$  上一致收敛到  $\ln \left(1-2 x \cos \theta+x^{2}\right)$,  因此 
$$
I(x)=\int_{0}^{\pi} \ln \left(1-2 x \cos \theta+x^{2}\right) \mathrm{d} \theta=\int_{0}^{\pi}-2 \sum_{n=1}^{\infty} \frac{\cos (n \theta)}{n} x^{n} \mathrm{~d} \theta=-2 \sum_{n=1}^{\infty} \int_{0}^{\pi} \frac{\cos (n \theta)}{n} x^{n} \mathrm{~d} \theta=0, x \in(-1,1)
$$
 注   此题前半部分与 《 数学分析习题课讲义 》 下册  73  页例题  $14.4 .5$  几乎一模一样 ,  差别在于多了一个常数  2 ,  也   可参考 《 数学分析中的典型问题与方法 》 第二版第  549  页例  $5.3 .14$  或者第  600  页例  $5.4 .16$.  后半部分是一   个很经典的积分 ,  可在各种数学分析教材和习题书的含参变量积分部分找到 ,  比如 《 数学分析中的典型问题   与方法 》 第二版第  600  页例  $5.4 .17$  或者第  816  页练习题  $7.1 .13$  中的  $b)$,  张筑生的 《 数学分析新讲 》 第三册第  337  页例  2 ,  林源渠 、 方企勤编的 《 数学分析解题指南 》 第  304  页例  2 .  利用这里算出的  $\ln \left(1-2 x \cos \theta+x^{2}\right)$  的幂级数展开和  Abel  第二定理 ,  我们还可以完成北京大学  2018  年数学分析倒数第二题的证明 .

\begin{enumerate}
  \setcounter{enumi}{10}
  \item  因为 
\end{enumerate}
$$
\frac{1}{2}+\cos x+\cos 2 x+\cdots+\cos n x=\frac{\sin \left(n+\frac{1}{2}\right) x}{2 \sin \frac{x}{2}}
$$
 两边同时对  $x$  在  $[0, \pi]$  积分得 
$$
\frac{\pi}{2}=\int_{0}^{\pi} \frac{\sin \left(n+\frac{1}{2}\right) x}{2 \sin \frac{x}{2}} \mathrm{~d} x=\int_{0}^{\pi} \frac{\sin \left(n+\frac{1}{2}\right) x}{x} \mathrm{~d} x+\int_{0}^{\pi} \frac{x-2 \sin \frac{x}{2}}{2 x \sin \frac{x}{2}} \sin \left(n+\frac{1}{2}\right) x \mathrm{~d} x
$$
$$
\lim _{n \rightarrow \infty} \int_{0}^{\pi} \frac{\sin \left(n+\frac{1}{2}\right) x}{x} \mathrm{~d} x=\lim _{n \rightarrow \infty} \int_{0}^{\left(n+\frac{1}{2}\right) \pi} \frac{\sin x}{x} \mathrm{~d} x=\int_{0}^{+\infty} \frac{\sin x}{x} \mathrm{~d} x .
$$
 再由  Riemann-Lebesgue  引理知 
$$
\lim _{n \rightarrow \infty} \int_{0}^{\pi} \frac{x-2 \sin \frac{x}{2}}{2 x \sin \frac{x}{2}} \sin \left(n+\frac{1}{2}\right) x \mathrm{~d} x=0
$$
$$
\begin{aligned}
\int_{0}^{+\infty} \frac{\sin x}{x} \mathrm{~d} x=\frac{\pi}{2} \\
\int_{0}^{+\infty} \frac{\sin ^{2}(x y)}{x^{2}} \mathrm{~d} x &=-\int_{0}^{+\infty} \sin ^{2}(x y) \mathrm{d} \frac{1}{x} \\
&=-\left.\frac{1}{x} \sin ^{2}(x y)\right|_{0} ^{+\infty}+\int_{0}^{+\infty} \frac{2 y \sin (x y) \cos (x y)}{x} \\
&=y \int_{0}^{+\infty} \frac{\sin (2 x y)}{x} \mathrm{~d} x \\
&=\frac{\pi|y|}{2} .
\end{aligned}
$$
 注   前半部分是  Dirichlet  积分 ,  与北京大学  2006  年数学分析第  5  题 , 2016  年数学分析第  7  题一样 ,  在各种数   学分析教材和习题书上也很常见 .  这里给出的证明方法见于张筑生老师  《 数学分析新讲 》 第三册第  285  页   引理  3 ;  更常见的证明方法在数学分析教材含参变量积分部分 ,  是通过引入收敛因子来做 ;  学了复变函数后   也可以用留数定理来证明这个结论 .  解决后半部分只需用下分部积分 ,  与 《 数学分析习题课讲义 》 下册  295  北京大学  2020  年全国硕士研究生招生考试数学分析试题及解答 

   

2020.01.10

\begin{enumerate}
  \item (15  分 )  设  $f(x)$  在  $[a, b]$  上上半连续 ,  即  $\forall x_{0} \in[a, b]$  皆有上极限  $\limsup _{x \rightarrow x_{0}} f(x) \leqslant f\left(x_{0}\right.$ ) ( 端点处只考虑   单侧极限 ).  问  $f(x)$  在  $[a, b]$  上必有最大值 ?  给出证明或反例 .

  \item (15  分 ) $f(x)=\frac{x}{1+x \cos ^{2} x}$  在  $[0,+\infty)$  上是否一致连续 ?  证明你的结论 .

  \item $\left(15\right.$  分 )  设  $f(x)$  在  $[1,+\infty)$  上连续 , $f(x) \geqslant 0$  及  $f(x+y) \leqslant f(x)+f(y), \forall x, y \in[0,+\infty)$.  问  $\lim _{x \rightarrow+\infty} \frac{f(x)}{x}$  是否存在 ?  证明你的结论或估出反例 .

  \item (15  分 )  设  $f(x) \in C[0,1]$,  单增且  $f(x)>0, x \in[0,1]$.  定义  $s=\frac{\int_{0}^{1} x f(x) \mathrm{d} x}{\int_{0}^{1} f(x) \mathrm{d} x}$.

\end{enumerate}
(1) (7  分 )  证明  $s \geqslant \frac{1}{2}$.

(2) (8  分 )  指出  $\int_{0}^{s} f(x) \mathrm{d} x$  与  $\int_{s}^{1} f(x) \mathrm{d} x$  的大小关系并证明你的结论 . ( 可应用物理或几何直观 )

\begin{enumerate}
  \setcounter{enumi}{6}
  \item (15  分 )  设曲面  $S$  由  $C^{2}$  函数  $z=f(x, y),(x, y) \in D$  给出 ,  此处  $D$  为  $X O Y$  平面上的单连通区域 ,  其边界   形  Stokes  公式的证明 
\end{enumerate}
$$
\int_{\vec{L}} R(x, y, z) \mathrm{d} z=\int_{\vec{S}} \frac{\partial R}{\partial y} \mathrm{~d} y \mathrm{~d} z-\frac{\partial R}{\partial x} \mathrm{~d} z \mathrm{~d} x
$$

\begin{enumerate}
  \setcounter{enumi}{7}
  \item (15  分 )  设  $f(x, y)$  在  $\mathbb{R}$  上有连续二阶偏导数 ,  满足  $f(0,0)=0$  及  $f_{x x}+f_{y y}=x^{2}+y^{2}$.  用  $C_{r}$  表示中心在原 
\end{enumerate}
$$
\frac{\cos (q-p) x}{p}+\frac{\cos (q-p+1) x}{p-1}+\cdots+\frac{\cos (q-1) x}{1}-\frac{\cos (q+1) x}{1}-\frac{\cos (q+2) x}{2}-\cdots-\frac{\cos (q+p) x}{p} .
$$
(1) (10  分 )  证明  $f(x)=\sum_{k=1}^{\infty} a_{k} T_{p_{k}, q_{k}}(x)$  为以  $2 \pi$  为周期的连续函数 . 1.  上半连续函数在有界闭区间上必有最大值 .  由  $\lim _{\sup }^{x \rightarrow x_{0}}$  时 , $f(x)<f\left(x_{0}\right)+\varepsilon$,  于是  $f(x)$  在  $x_{0}$  的一个小邻域内有上界 ,  由  $x_{0}$  的任意性 ,  所有的小邻域构成  $[a, b]$  的一   个开覆盖 ,  因为  $[a, b]$  是紧集 ,  于是存在有限开覆盖 ,  从而可得  $f(x)$  在  $[a, b]$  上有上界 .  设  $M=\sup _{x \in[a, b]} f(x)$,  则  $M \in \mathbb{R} . \forall n \in \mathbb{N}^{+}, \exists x_{n}$,  使得  $M-\frac{1}{n}<f\left(x_{n}\right) \leqslant M$,  如果  $\left\{x_{n}\right\}$  不收敛 ,  我们可以取收敛子列 ,  从而我们 

 注   这里的结论为裴礼文的 《 数学分析中的典型问题与方法 》 第二片第  168  页定理  4 ,  若想更为熟悉这方面的内 
$$
\begin{aligned}
\Delta(f, n, m)=f\left(\left(n+\frac{1}{2}\right) \pi\right)-f\left(\left(n+\frac{1}{2}\right) \pi+\frac{1}{m}\right) &=\left(n+\frac{1}{2}\right) \pi-\frac{\left(n+\frac{1}{2}\right) \pi+\frac{1}{m}}{1+\left(\left(n+\frac{1}{2}\right) \pi+\frac{1}{m}\right) \sin ^{2} \frac{1}{m}} \\
&=\frac{\left(n+\frac{1}{2}\right) \pi\left(\left(n+\frac{1}{2}\right) \pi+\frac{1}{m}\right) \sin ^{2} \frac{1}{m}-\frac{1}{m}}{1+\left(\left(n+\frac{1}{2}\right) \pi+\frac{1}{m}\right) \sin ^{2} \frac{1}{m}}
\end{aligned}
$$
$$
\Delta(f, n, m) \geqslant \frac{\left(n+\frac{1}{2}\right)^{2} \pi^{2}\left(\frac{2}{\pi}\right)^{2} \frac{1}{n}-1}{\left(n+\frac{1}{2}\right)+2} \rightarrow 4 \text {, 当 } n \rightarrow+\infty,
$$
 注   判断这种在无穷区间上的连续可微函数是否一致收敛 ,  首先是看函数在无穷处的极限是否存在 ,  若存在则一   致连续 ,  若此法不行则看导函数是否一致有界 ,  有界则一致连续 ,  前面两种方法都失效后自然的想法就是想   办法寻找距离趋于无穷小但是对应的函数值数列的差大于一个正的常数的两个数列 ,  找到了就不一致连续 .

\begin{enumerate}
  \setcounter{enumi}{3}
  \item $\lim _{x \rightarrow+\infty} \frac{f(x)}{x}$  存在且等于  $\inf \left\{\frac{f(x)}{x}: x \geqslant 1\right\}$,  将前面那个下确界记为  $\alpha$,  则  $0 \leqslant \alpha \leqslant f(1)$,  故  $\alpha \in \mathbb{R}$.  用数 
\end{enumerate}
$$
\begin{aligned}
f(x)=f\left(x_{0}\left(\left[\frac{x}{x_{0}}\right]+\left\{\frac{x}{x_{0}}\right\}\right)\right) & \leqslant f\left(x_{0}\left(\left\{\frac{x}{x_{0}}\right\}+1\right)\right)+f\left(x_{0}\left(\left[\frac{x}{x_{0}}\right]-1\right)\right) \\
& \leqslant M+\left(\left[\frac{x}{x_{0}}\right]-1\right) f\left(x_{0}\right) \leqslant M+\frac{x-x_{0}}{x_{0}} f\left(x_{0}\right)
\end{aligned}
$$
$$
\alpha-\varepsilon<\alpha \leqslant \frac{f(x)}{x} \leqslant \frac{M}{x}+\frac{x-x_{0}}{x} \frac{f\left(x_{0}\right)}{x_{0}}<\frac{\varepsilon}{2}+\alpha+\frac{\varepsilon}{2}=\alpha+\varepsilon,
$$
 注几乎与林源渠 、 方企勤编的 《 数学分析解题指南 》 第  45  页例  13  一模一样的题目 .  类似的题目见谢惠民等   人编写的 《 数学分析习题课讲义 》 上册第  90  页例题  3.6.4  以及裴礼文编写的 《 数学分析中的典型问题与方 
$$
I=\int_{0}^{\frac{1}{2}}\left(x-\frac{1}{2}\right) f(x) \mathrm{d} x+\int_{\frac{1}{2}}^{1}\left(x-\frac{1}{2}\right) f(x) \mathrm{d} x=\int_{0}^{\frac{1}{2}}\left(x-\frac{1}{2}\right)(f(x)-f(1-x)) \mathrm{d} x
$$
(2) ( 法一 )  设  $F(x)=\int_{0}^{x} f(t) \mathrm{d} t, x \in[0,1]$,  则  $F(x)$  为在  $[0,1]$  下凸单增连续函数 .  只需证明  $F(s) \leqslant$ $F(1)-F(s)$,  也即只需证明  $F(s) \leqslant \frac{1}{2} F(1)$.  令  $x_{i}=i / n, i=0,1, \ldots, n$.  则由  Jensen  不等式得 
$$
F(s)=F\left(\frac{\sum_{i=1}^{n} \int_{x_{i-1}}^{x_{i}} t f(t) \mathrm{d} t}{\sum_{i=1}^{n} \int_{x_{i-1}}^{x_{i}} f(t) \mathrm{d} t}\right) \leqslant F\left(\frac{\sum_{i=1}^{n} x_{i} \int_{x_{i-1}}^{x_{i}} f(t) \mathrm{d} t}{\sum_{i=1}^{n} \int_{x_{i-1}}^{x_{i}} f(t) \mathrm{d} t}\right) \leqslant \sum_{i=1}^{n} \frac{F\left(x_{i}\right)}{F(1)} \int_{x_{i-1}}^{x_{i}} f(t) \mathrm{d} t .
$$
 又因为  $F(x)$  在  $[0,1]$  上一致连续 ,  故  $\forall \varepsilon>0, \exists N>0$,  当  $n>N$  时 , $\left|F\left(x_{i}\right)-F\left(x_{i-1}\right)\right|<\varepsilon$,  此时 
$$
0 \leqslant \sum_{i=1}^{n} F\left(x_{i}\right) \int_{x_{i-1}}^{x_{i}} f(t) \mathrm{d} t-\sum_{i=1}^{n} \int_{x_{i-1}}^{x_{i}} F(t) f(t) \mathrm{d} t=\sum_{i=1}^{n} \int_{x_{i-1}}^{x_{i}}\left(F\left(x_{i}\right)-F(t)\right) f(t) \mathrm{d} t \leqslant \int_{0}^{1} f(t) \mathrm{d} t \varepsilon
$$
 于是 
$$
\lim _{n \rightarrow+\infty} \sum_{i=1}^{n} F\left(x_{i}\right) \int_{x_{i-1}}^{x_{i}} f(t) \mathrm{d} t=\sum_{i=1}^{n} \int_{x_{i-1}}^{x_{i}} F(t) f(t) \mathrm{d} t=\int_{0}^{1} F(t) f(t) \mathrm{d} t=\int_{0}^{1} F(t) \mathrm{d} F(t)=\frac{F^{2}(1)}{2},
$$
 综上即得  $F(s) \leqslant \frac{1}{2} F(1)$.

( 法二 )  设  $F(x)=\int_{0}^{x} f(t) \mathrm{d} t, x \in[0,1]$,  则  $F(x)$  为在  $[0,1]$  下凸单增连续函数 .  只需证明  $F(s) \leqslant F(1)-$ $F(s)$,  也即只需证明  $F(s) \leqslant \frac{1}{2} F(1) . F(x)$  在  $x=s$  处的切线在  $F(x)$  下方 ,  即  $F(x) \geqslant f(s)(x-s)+F(s)$,  不等式两边同时乘以  $f(x)$,  然后从  0  到  1  积分 ,  得到  $\frac{F^{2}(1)}{2} \geqslant F(s) F(1)$,  故  $F(s) \leqslant \frac{1}{2} F(1)$.

 注   第一小问用重心的观点来看的话 , $s \geqslant \frac{1}{2}$  是自然成立的 .  用物理学中的杜杆平衡条件容易猜出第二小问中的   大小关系 ,  或者取定一个增函数也能猜到大小关系 ,  此处对这一关系的严格证明法一参考了 zhangzujin 361  的   解答 ,  法二源自张辰  LMY.

\begin{enumerate}
  \setcounter{enumi}{5}
  \item  利用分部积分及变量替换有 
\end{enumerate}
$$
I=-\int_{0}^{+\infty} \sin ^{2} x \mathrm{~d} \frac{1}{x}=-\left.\frac{1}{x} \sin ^{2} x\right|_{0} ^{+\infty}+\int_{0}^{+\infty} \frac{2 \sin x \cos x}{x} \mathrm{~d} x=\int_{0}^{+\infty} \frac{\sin (2 x)}{x} \mathrm{~d} x=\int_{0}^{+\infty} \frac{\sin x}{x} \mathrm{~d} x=\frac{\pi}{2}
$$
 注   比北京大学  2019  年数学分析最后一题更简单的一题 .

\begin{enumerate}
  \setcounter{enumi}{6}
  \item  先把三维空间中的第二型曲线积分变为平面上的第二型曲线积分 ,  接着用  Green  公式 ,  再把第一型曲面积   分变为第二型曲面积分就得结果 .
\end{enumerate}
$$
\begin{aligned}
\int_{\vec{L}} R(x, y, z) \mathrm{d} z &=\int_{\partial D} R(x, y, f(x, y)) \frac{\partial f}{\partial x}(x, y) \mathrm{d} x+R(x, y, f(x, y)) \frac{\partial f}{\partial y}(x, y) \mathrm{d} y \\
&=\iint_{D} \frac{\partial R}{\partial x}(x, y, f(x, y)) \frac{\partial f}{\partial y}(x, y)-\frac{\partial R}{\partial y}(x, y, f(x, y)) \frac{\partial f}{\partial x}(x, y) \mathrm{d} x \mathrm{~d} y \\
&=\int_{\vec{S}} \frac{\partial R}{\partial y} \mathrm{~d} y \mathrm{~d} z-\frac{\partial R}{\partial x} \mathrm{~d} z \mathrm{~d} x
\end{aligned}
$$
$7 .$
$$
\begin{aligned}
A(r) &=\frac{1}{2 \pi} \int_{0}^{2 \pi} f(r \cos \theta, r \sin \theta) \mathrm{d} \theta=\frac{1}{2 \pi} \int_{0}^{2 \pi} \mathrm{d} \theta \int_{0}^{r} f_{x}(u \cos \theta, u \sin \theta) \cos \theta+f_{y}(u \cos \theta, u \sin \theta) \sin \theta \mathrm{d} u \\
&=\frac{1}{2 \pi} \int_{0}^{r} \mathrm{~d} u \int_{0}^{2 \pi} f_{x}(u \cos \theta, u \sin \theta) \cos \theta+f_{y}(u \cos \theta, u \sin \theta) \sin \theta \mathrm{d} \theta \\
&=\frac{1}{2 \pi} \int_{0}^{r} \frac{1}{u} \mathrm{~d} u \int_{x^{2}+y^{2}=u^{2}} f_{x} \mathrm{~d} y-f_{y} \mathrm{~d} x \quad(\text { 第二型曲线积分) }\\
&=\frac{1}{2 \pi} \int_{0}^{r} \frac{1}{u} \mathrm{~d} u \int_{x^{2}+y^{2} \leqslant u^{2}} f_{x x}+f_{y y} \mathrm{~d} x \mathrm{~d} y=\frac{r^{4}}{16} \text {. (先用 Green 公式, 然后极坐标变换) }
\end{aligned}
$$
 注   类似题目见谢惠民等人编写的 《 数学分析习题课讲义 》 下册第  384  页第  26  章第一组参考题第  7  题 ,  也即是  8. (1)  直接套公式可计算出  $f(x)$  的  Fourier  级数为 
$$
\frac{\sin \pi p}{\pi p}+\sum_{n=1}^{\infty} \frac{(-1)^{n} \sin \pi p}{\pi}\left(\frac{1}{p+n}+\frac{1}{p-n}\right) \cos n x
$$
 由于  $f(x)=\cos p x$  是分段单调有界的 ,  故上述级数收敛于  $\cos p x$.

(2)  取  $x=0$,  由  $(1)$  知 :
$$
\frac{\pi}{\sin \pi p}=\frac{1}{p}+\sum_{n=1}^{\infty}(-1)^{n}\left(\frac{1}{p+n}+\frac{1}{p-n}\right)
$$
 注意至 
$$
B(p, 1-p)=\int_{0}^{1} t^{p-1}(1-t)^{-p} \mathrm{~d} t=\int_{0}^{+\infty} \frac{y^{p-1}}{y+1} \mathrm{~d} y=\int_{0}^{1} \frac{y^{p-1}+y^{-p}}{y+1} \mathrm{~d} y
$$
 于是对于任意  $N \in \mathbb{N}^{+}$,
$$
\begin{aligned}
B(p, 1-p) &=\int_{0}^{1}\left(y^{p-1}+y^{-p}\right) \frac{1-(-y)^{N+1}}{1+y} \mathrm{~d} y+\int_{0}^{1}\left(y^{p-1}+y^{-p}\right) \frac{(-y)^{N+1}}{1+y} \mathrm{~d} y \\
&=\frac{1}{p}+\sum_{n=1}^{N}(-1)^{n}\left(\frac{1}{p+n}+\frac{1}{p-n}\right)+\frac{(-1)^{N}}{N-p+1}+\int_{0}^{1}\left(y^{p-1}+y^{-p}\right) \frac{(-y)^{N+1}}{1+y} \mathrm{~d} y .
\end{aligned}
$$
 又因为 
$$
\left|\int_{0}^{1}\left(y^{p-1}+y^{-p}\right) \frac{(-y)^{N+1}}{1+y} \mathrm{~d} y\right| \leqslant \int_{0}^{1}\left(y^{p-1}+y^{-p}\right) \frac{y^{N+1}}{1+y} \mathrm{~d} y \leqslant \frac{1}{N+p+1}+\frac{1}{N-p+2},
$$
 令  $N \rightarrow+\infty$  就得 
$$
B(p, 1-p)=\frac{1}{p}+\sum_{n=1}^{\infty}(-1)^{n}\left(\frac{1}{p+n}+\frac{1}{p-n}\right)=\frac{\pi}{\sin \pi p}, \quad 0<p<1 .
$$
 注   第二问中为实现积分与求和交换次序所用手法与解北京大学  2019  年第  5  题一样 ,  有限和部分逐项积分 ,  再   估计剩下的式子的大小 .

\begin{enumerate}
  \setcounter{enumi}{9}
  \item (1)  经过简单变形将得到 
\end{enumerate}
$$
T_{p, q}(x)=\sum_{t=1}^{p} \frac{\cos (q-t) x-\cos (q+t) x}{t}=2 \sum_{t=1}^{p} \frac{\sin (t x)}{t} \sin q x
$$
 下面先证明  $S_{n}(x)=\sum_{k=1}^{n} \frac{\sin k x}{k}$  在  $\mathbb{R}$  上一致有界 ,  说得更具体一点 ,  下面将证明 
$$
\left|\sum_{k=1}^{n} \frac{\sin k x}{k}\right| \leqslant 10, \quad \forall n \in \mathbb{N}^{+}, x \in \mathbb{R} .
$$
 因为  $S_{n}(x)$  是以  $2 \pi$  为周期的奇函数 ,  故  $\left|S_{n}(x)\right|$  在  $[0, \pi]$  上的最大值即为其在  $\mathbb{R}$  上的最大值 .
$$
S_{n}^{\prime}(x)=\sum_{k=1}^{n} \cos k x=\frac{\sin \left(n+\frac{1}{2}\right) x-\sin \frac{x}{2}}{2 \sin \frac{x}{2}}, \quad x \neq 0 .
$$
 因为使  $S_{n}^{\prime}(x)=0$  的点比较复杂 ,  为了方便起见考虑  $T_{n}(x)=S_{n}(x)+\frac{x}{2}, x \in(0, \pi)$,  由  $T_{n}^{\prime}(x)=0$  解   得  $x_{m}=\frac{m \pi}{n+\frac{1}{2}}, 1 \leqslant m \leqslant n, m \in \mathbb{Z}$.
$$
\begin{aligned}
T_{n}\left(x_{m}\right) &=\int_{0}^{x_{m}} T_{n}(u) \mathrm{d} u+T_{n}(0)=\int_{0}^{x_{m}} \frac{\sin \left(n+\frac{1}{2}\right) u}{2 \sin \frac{u}{2}} \mathrm{~d} u=\int_{0}^{m \pi} \frac{\sin t}{(2 n+1) \sin \frac{t}{2 n+1}} \mathrm{~d} t \\
&=\sum_{k=1}^{m} \int_{(k-1) \pi}^{k \pi} \frac{\sin t}{(2 n+1) \sin \frac{t}{2 n+1}} \mathrm{~d} t=\sum_{k=1}^{m} \int_{0}^{\pi}(-1)^{k-1} \frac{\sin u}{(2 n+1) \sin \frac{(k-1) \pi+u}{2 n+1}} \mathrm{~d} u .
\end{aligned}
$$
 令 
$$
b_{k}=\int_{0}^{\pi} \frac{\sin u}{(2 n+1) \sin \frac{(k-1) \pi+u}{2 n+1}} \mathrm{~d} u, k \in\{1,2, \ldots, m\},
$$
 则  $b_{1}>b_{2}>\cdots>b_{m}>0, T_{n}\left(x_{m}\right)=\sum_{k=1}^{m}(-1)^{k-1} b_{k}$,  于是有 
$$
0<T_{n}\left(x_{m}\right)<b_{1}=\int_{0}^{\pi} \frac{\sin u}{(2 n+1) \sin \frac{u}{2 n+1}} \mathrm{~d} u \leqslant \int_{0}^{\pi} \frac{\sin u}{(2 n+1) \frac{2}{\pi} \frac{u}{2 n+1}} \mathrm{~d} u=\frac{\pi}{2} \int_{0}^{\pi} \frac{\sin u}{u} \mathrm{~d} u, 1 \leqslant m \leqslant n .
$$
 再注意到 
$$
\int_{0}^{\pi} \frac{\sin u}{u} \mathrm{~d} u>\int_{0}^{\pi} \frac{u-\frac{u^{3}}{3 !}}{u} \mathrm{~d} u=\pi-\frac{\pi^{3}}{18}>1
$$
 于是 
$$
\left|T_{n}(x)\right| \leqslant \frac{\pi}{2} \int_{0}^{\pi} \frac{\sin u}{u} \mathrm{~d} u, \quad \forall n \in \mathbb{N}^{+}, x \in[0, \pi]
$$
 从而有 
$$
\left|S_{n}(x)\right| \leqslant \max _{x \in[0, \pi]}\left|T_{n}(x)\right|+\frac{\pi}{2} \leqslant \frac{\pi}{2}\left(\int_{0}^{\pi} \frac{\sin u}{u} \mathrm{~d} u+1\right)<\frac{\pi}{2}\left(\int_{0}^{\pi} \frac{u}{u} \mathrm{~d} u+1\right)<10 .
$$
 做到这里就说明了  $S_{n}(x)$  是一致有界的 ,  于是 
$$
\left|\sum_{k=1}^{\infty} a_{k} T_{p_{k}, q_{k}}(x)\right| \leqslant \sum_{k=1}^{\infty} a_{k}\left|T_{p_{k}, q_{k}}(x)\right|<10 \sum_{k=1}^{\infty} a_{k}<+\infty,
$$
 从而级数在  $\mathbb{R}$  上一致收敛于  $f(x)$,  又因为  $T_{p_{k}, q_{k}}(x)$  是以  $2 \pi$  为周期的连续函数 ,  故  $f(x)$  是以  $2 \pi$  为周   期的连续函数 .

(2)
$$
\frac{1}{\pi} \int_{0}^{2 \pi} f(x) \cos (n x) \mathrm{d} x=\frac{1}{\pi} \int_{0}^{2 \pi}\left(\sum_{k=1}^{\infty} a_{k} T_{p_{k}, q_{k}}(x)\right) \cos (n x) \mathrm{d} x=\frac{a_{k}}{\pi} \sum_{k=1}^{\infty} \int_{0}^{2 \pi} T_{p_{k}, q_{k}}(x) \cos (n x) \mathrm{d} x,
$$
 能交换求和与积分次序是因为级数在  $\mathbb{R}$  上一致收敛于  $f(x)$.  由于  $q_{k}+p_{k}<q_{k+1}-p_{k+1}$,  最后会计算   发现  $f(x)$  的  Fourier  级数为 
$$
\sum_{n=1}^{\infty} \alpha_{n} \cos n x, \quad \alpha_{n}= \begin{cases}\frac{a_{k}}{q_{k}-n}, & q_{k}-p_{k} \leqslant n \leqslant q_{k}+p_{k}, n \neq q_{k}, k \in \mathbb{N}^{+} \\ 0, & \text { 其他 }\end{cases}
$$
$f(x)$  的  Fourier  级数在  $x=0$  处不收敛 .  考虑它的第  $q_{k}+1$  到第  $q_{k}+p_{k}$  项部分和的绝对值为 
$$
\sum_{n=q_{k}+1}^{q_{k}+p_{k}} \frac{a_{k}}{n-q_{k}}=a_{k} \sum_{k=1}^{p_{k}} \frac{1}{k}=a_{k} \ln p_{k}+a_{k}\left(\sum_{k=1}^{p_{k}} \frac{1}{k}-\ln p_{k}\right)
$$
 因为 
$$
\liminf _{n \rightarrow \infty} \sum_{n=q_{k}+1}^{q_{k}+p_{k}} \frac{a_{k}}{n-q_{k}}=\liminf _{n \rightarrow \infty} a_{k} \ln p_{k}>0,
$$
 由  Cauchy  收敛准则知  $f(x)$  的  Fourier  级数在  $x=0$  处不收敛 .

 注   相关的题目见谢惠民等人编写的 《 数学分析习题课讲义 》 下册第  77  页第  14  章第二组参考题第  7  题 、 第  90  页例题  15.2.1、 第  46  页例题  $14.1 .8$ 、 第  105  页参考题第  20  题 ,  北京大学  2017  年数学分析最后一题 、2018  年   第  7  题 .  解决这题的关键在于知道  $S_{n}(x)$  是一致有界的 ,  谢老师等人的书上给的题目是证明  $\left|S_{n}(x)\right| \leqslant 2 \sqrt{\pi}$,  提示中说了两种做法 ,  一种是模仿前面例题中的做法 ,  另外一种方法就是按我写的这种思路来做 ,  仿照例题   的做法能直接证明那个上界 .  为了完整起见 ,  下面将做点补充 ,  用另一种方法以得出书上的上界估计 .  首先   有平凡估计 
$$
\left|S_{n}(x)\right| \leqslant \sum_{k=1}^{n} \frac{1}{k}<1+\int_{1}^{n} \frac{1}{u} \mathrm{~d} u=1+\ln n
$$
$$
\begin{aligned}
b_{1}-\int_{0}^{\pi} \frac{\sin u}{u} \mathrm{~d} u &=\int_{0}^{\pi} \frac{\sin u}{(2 n+1) \sin \frac{u}{2 n+1}}-\frac{\sin u}{u} \mathrm{~d} u=\int_{0}^{\pi} \frac{\sin u\left(u-(2 n+1) \sin \frac{u}{2 n+1}\right)}{(2 n+1) u \sin \frac{u}{2 n+1}} \mathrm{~d} u \\
& \leqslant \int_{0}^{\pi} \frac{\frac{u^{3}}{6(2 n+1)^{2}}}{\frac{2}{\pi} u} \mathrm{~d} u=\frac{\pi^{4}}{36(2 n+1)^{2}},
\end{aligned}
$$
$$
\left|S_{n}(x)\right|<\int_{0}^{\pi} \frac{\sin u}{u} \mathrm{~d} u+\frac{\pi^{4}}{36 \times(2 \times 7+1)^{2}}+\frac{\pi}{2}<2 \sqrt{\pi} .
$$
 其实这道题是说了这样一件事情 : 连续函数的  Fourier  级数不一定是逐点收敛的 .  在  Elias M.Stein  等人编   写的  Fourier analysis  的第三章第二节的第二小节就是在举例说明这样一件事情 ,  也是先证明了  $S_{n}(x)$  的一   致有界性 ,  再构造了一个级数 .  值得指出的是 , Stein  书上证明  $S_{n}(x)$  的一致有界性的方法不同于前面说的   两种方法 ,  并且在第  94  页的练习题  19  中还提示了另一种能更简单地证明  $S_{n}(x)$  的一致有界性的方法 ,  具   体方法如下 :

 令  $f(x)=\int_{0}^{x} \frac{\sin t}{t} \mathrm{~d} t, x \in[0,+\infty)$,  则由于  $f(x)$  在  $[0,+\infty)$  上的连续性以及极限  $\lim _{x \rightarrow+\infty} f(x)$  存在 ,  故  $f(x)$  为在  $[0,+\infty)$  上有界函数 ,  设  $|f(x)| \leqslant M$.  因为  $S_{n}(x)$  是以  $2 \pi$  为周期的奇函数 ,  故  $\left|S_{n}(x)\right|$  在  $[0, \pi]$
$$
S_{n}^{\prime}(x)=\sum_{k=1}^{n} \cos k x=\frac{\sin \left(n+\frac{1}{2}\right) x-\sin \frac{x}{2}}{2 \sin \frac{x}{2}}, \quad x \neq 0
$$
$$
S_{n}(x)=\int_{0}^{x} \frac{\sin \left(n+\frac{1}{2}\right) t-\sin \frac{t}{2}}{2 \sin \frac{t}{2}} \mathrm{~d} t=\int_{0}^{x} \frac{\sin \left(n+\frac{1}{2}\right) t}{2 \sin \frac{t}{2}}-\frac{\sin \left(n+\frac{1}{2}\right) t}{t} \mathrm{~d} t+\int_{0}^{x} \frac{\sin \left(n+\frac{1}{2}\right) t}{t} \mathrm{~d} t-\frac{x}{2},
$$
$$
\left|\int_{0}^{x} \frac{\sin \left(n+\frac{1}{2}\right) t}{2 \sin \frac{t}{2}}-\frac{\sin \left(n+\frac{1}{2}\right) t}{t} \mathrm{~d} t\right| \leqslant \int_{0}^{\pi} \frac{t-2 \sin \frac{t}{2}}{2 t \sin \frac{t}{2}} \mathrm{~d} t, \quad\left|\frac{x}{2}\right| \leqslant \frac{\pi}{2}
$$
$$
\left|\int_{0}^{x} \frac{\sin \left(n+\frac{1}{2}\right) t}{t} \mathrm{~d} t\right|=\left|\int_{0}^{\left(n+\frac{1}{2}\right) x} \frac{\sin u}{u} \mathrm{~d} u\right| \leqslant M
$$
$$
\left|S_{n}(x)\right| \leqslant \int_{0}^{\pi} \frac{t-2 \sin \frac{t}{2}}{2 t \sin \frac{t}{2}} \mathrm{~d} t+M+\frac{\pi}{2}, \quad x \in[0,+\infty)
$$

\section{1. 北京工业大学 2009 年硕士研究生入学考试试题(高等代数865) 
 李扬 
 微信公众号: sxkyliyang}
 一 、 填空题 ( 本题共  50  分 ,  每题  5  分 )

\begin{enumerate}
  \item $A$  是  3  阶实矩阵 , $B=-\left(A^{*}\right)^{\prime}$.  如果  $A^{2009}=E$,  则  $B^{2009}=$

  \item  记矩阵  $A=\left(\begin{array}{ccc}1 & 1 & 10 \\ 1 & -1 & 19 \\ 1 & 5 & 25\end{array}\right)$  第三列三个位置的代数余子式依次是  $A_{13}, A_{23}, A_{33}$,  则表达式  $A_{13}+A_{23}+$ $25 A_{33}=\underline{\text {. }}$ ( 要求写出计算结果 )

  \item  如果  4  维列向量组  $\left\{\alpha_{1}, \alpha_{2}, \alpha_{3}, \alpha_{4}\right\}$  和  $\left\{\beta_{1}, \beta_{2}, \beta_{3}\right\}$  满足  $\left\{\begin{array}{c}\alpha_{1}=3 \beta_{1}+\beta_{2}-\beta_{3} \\ \alpha_{2}=2 \beta_{1}-\beta_{2}+\beta_{3} \\ \alpha_{3}=\beta_{1}+2 \beta_{2}+\beta_{3}\end{array}\right.$,  则行列式  $\mid \alpha_{1}, 2 \alpha_{2}, \alpha_{2}+$ 4.  矩阵乘积  $\left(\begin{array}{ccc}1 & 0 & 0 \\ 0 & 1 & 0 \\ -2 & 0 & 1\end{array}\right)\left(\begin{array}{ccccc}2 & 4 & 6 & 5 & 1 \\ 1 & 1 & 1 & 0 & -1 \\ 3 & 9 & 10 & 11 & 2\end{array}\right)=$

  \item  矩阵方程  $\left(\begin{array}{ccc}1 & -2 & 1 \\ 0 & 1 & 0 \\ 0 & -1 & 1\end{array}\right) X=\left(\begin{array}{ccc}1 & 1 & 1 \\ 1 & 0 & 1 \\ 1 & -1 & 0\end{array}\right)$  的解  $X=$

  \item  如果  $\alpha=(1,1,2,0)^{\prime}$  是  4  阶方阵  $A$  的属于特征值  2  的一个特征向量 ,  则它一定也是方阵  $A^{3}-2 A^{2}-3 E$

  \item  如果把  $n$  阶实对称矩阵按合同关系分类 ( 即两个  $n$  阶实对称矩阵属于同一类 ,  当且仅当它们合同 ),  则不同 

  \item  设  $\mathscr{A}$  是线性空间  $V$  的线性变换 ,  对自然数  $k$,  如果向量  $\xi$  满足  $\mathscr{A}^{k-1} \xi \neq 0, \mathscr{A}^{k} \xi=0$.  则  $\xi, \mathscr{A} \xi, \cdots, \mathscr{A}^{k-1} \xi$

  \item  矩阵  $A=\left(\begin{array}{ccc}1 & -1 & 0 \\ 2 & 1 & 3 \\ 1 & 2 & 3\end{array}\right)$  定义了  3  维向生量空间  $R^{3}(R$  是实数域  $)$  的一个线性变换  $\mathscr{A}: \xi \rightarrow A \xi\left(\xi \in R^{3}\right)$.  其值域  $\mathscr{A} R^{3}$  的维数是 

\end{enumerate}
 二 、 选择题 ( 本大题共  30  分 ,  每小题  3  分 )

\begin{enumerate}
  \item  如果  2009  阶实方阵  $A=\left(a_{i j}\right)$  中的元素满足 : $i+j=2010$  时 , $a_{i j}$  是奇数 ; $i+j>2010$  时 , $a_{i j}$  是偶数 ,  则行列式  $|A|$  的值 ( )\\
(C) $|A B|=|A||B|$

  \item  如果  $n$  维空间中的向量组  $\alpha_{1}, \alpha_{2}, \cdots, \alpha_{m}$  线性无关 ,  向量组  $\beta$  与  $\alpha_{k}(k=1,2, \cdots, m)$  都正交 ,  则向量组 

  \item  如果  $n$  阶实方阵  $A$  的秩  $R(A)=s$,  列向量  $X=\left(x_{1}, x_{2}, \cdots, x_{n}\right)^{\prime}, \beta=\left(b_{1}, b_{2}, \cdots, b_{n}\right)^{\prime}$.  齐次线性方程组  $A X=0$  与  $\beta^{\prime} X=1$  有公共解 ,  则 ( )\\
(A) $s>n$\\
(B) $s<n$\\
(C) $s=n$\\
(D)  无确定结论 

  \item $A$  是  $n$  阶实方阵 , $E$  是同阶单位矩阵 ,  如果  $A^{-1}=A$,  则必然  ( )\\
(A) $A=E$\\
(B) $A$  是初等矩阵 \\
(C) $|A|=1$\\
(D)  前三个选项都不正确 

  \item  如果  $n$  阶实方阵  $A=-A^{\prime}$,  则其特征值  $\lambda$  的平方 ( $)$\\
(A) $\lambda^{2}=0$\\
(B) $\lambda^{2}>0$\\
(C) $\lambda^{2} \leq 0$\\
(D)  与  0  不可比较 

  \item  记  $A^{*}$  为  $n$  阶实方阵  $A$  的伴随矩阵 .  如果齐次线性方程组  $A^{*} X=0$  解空间的维数是  $n-1$,  则  $A X=0$  解   空间的维数必然  $($ )\\
(A)  等于  1\\
(B)  等于  $n-1$\\
(C)  不确定 \\
(D)  前三个选项都不正确 

  \item  如果实对称矩阵  $A$  满足  $A^{5}-A^{4}+2 A^{3}-2 A^{2}+A=E$,  则  ( )\\
(A) $A$  是正交矩阵 \\
(B) $A$  是正定矩阵 \\
(C) $A$  是可逆矩阵 \\
(D) $A=E$

  \item  二次型  $x^{2}+2 y^{2}-z^{2}+4 x y-2 y z+2 z x$  的矩阵是  ( )\\
(D)  在目前条件下不确定 \\
(A)  相似 \\
(B)  合同 \\
(C)  既相似 ,  又合同 \\
(D)  既不相似 ,  也不合同 

\end{enumerate}
1.( 10  分 )  设  $n$  阶方阵  $A=\left(a_{i j}\right)_{n \times n}$  的每一行只有一个元素是  1 ,  其余元素是  0 ;  而且每一列的元素之和是  1 .

2.( 12  分 )  设  $A=\left(a_{i j}\right)_{m \times n}$  是实  $m \times n$  矩阵 , $\beta$  是实  $m$  维列向量 .  证明 :  线性方程组  $\left(A^{\prime} A\right) X=A^{\prime} \beta$  总是有   实数解的 .

3.( 13  分 )  设实矩阵  $A, B$  可以作乘法  $A B . R(B), R(A B)$  分别表示  $B, A B$  的秩 .  证明 : 满足条件  $A B X=0$  的所有  $B X$  构成一个维数是  $R(B)-R(A B)$  的向量空间 .
$$
\alpha_{1}=(1,1,-1,0,2)^{\prime}, \alpha_{2}=(0,1,1,1,-1)^{\prime}, \alpha_{3}=(-1,0,2,1,-3)^{\prime}, \alpha_{4}=(1,-1,-3,-2,4)^{\prime}
$$

\section{2. 北京工业大学 2010 年硕士研究生入学考试试题(高等代数 865) 
 李扬 
 微信公众号: sxkyliyang}
 一 、 填空题 ( 本题共  50  分 , 每题  5  分 )

\begin{enumerate}
  \item  如果  $n$  阶方阵的每一行 ,  每一列都只有一个元素  1 ,  其余元素是  0 ,  则称其为置换矩阵 .  则所有  $n$  阶置换矩   阵的行列式的和为 

  \item  记矩阵  $\left(\begin{array}{ccc}0 & 2 & -5 \\ -2 & 0 & 1 \\ 3 & -8 & 21\end{array}\right)$  第三列三个位置的代数余子式依次是  $A_{13}, A_{23}, A_{33}$,  则表达式  $-3 A_{13}+8 A_{23}=$ ( 要求写出计算结果 )

  \item  如果四维列向量组  $\left\{\alpha_{1}, \alpha_{2}, \alpha_{3}, \alpha_{4}\right\}$  和  $\left\{\beta_{1}, \beta_{2}, \beta_{3}\right\}$  满足  $\left\{\begin{array}{c}\alpha_{1}=3 \beta_{1}+\beta_{2}-\beta_{3} \\ \alpha_{2}=2 \beta_{1}-5 \beta_{2}+\beta_{3} \\ \alpha_{3}=\beta_{1}+2 \beta_{2}+\beta_{3} \\ \alpha_{4}=\beta_{1}-\beta_{2}-7 \beta_{3}\end{array}\right.$,  则矩阵  $\left\{\alpha_{1}-3 \alpha_{4}, 2 \alpha_{2}, \alpha_{3}, \alpha_{4}\right\}$  的秩  $R\left\{\alpha_{1}-3 \alpha_{4}, 2 \alpha_{2}, \alpha_{3}, \alpha_{4}\right\}$ \_\_.( 填比较关系 )

  \item  矩阵乘积  $\left(\begin{array}{ccc}1 & 0 & 0 \\ 0 & 1 & 0 \\ -2 & 0 & 1\end{array}\right)\left(\begin{array}{cccc}1 & 1 & 0 & -2 \\ 2 & 3 & 0 & 1 \\ 2 & 2 & 0 & -4\end{array}\right)\left(\begin{array}{llll}1 & 0 & 0 & 0 \\ 0 & 1 & 0 & 0 \\ 0 & 0 & 0 & 1 \\ 0 & 0 & 1 & 0\end{array}\right)=$

  \item  矩阵方程  $X\left(\begin{array}{ccc}1 & -2 & 1 \\ 0 & 1 & 0 \\ 0 & -1 & 1\end{array}\right)=\left(\begin{array}{ccc}1 & 1 & 1 \\ 1 & 0 & 1 \\ 1 & -1 & 0\end{array}\right)$  的解  $X=$

  \item  如果  $\left(x_{1}, x_{2}, x_{3}\right)\left(\begin{array}{ccc}1 & 0 & 3 \\ 2 & a & -1 \\ -3 & 1 & 3-a\end{array}\right)\left(\begin{array}{c}x_{1} \\ x_{2} \\ x_{3}\end{array}\right)$  是正定二次型 , $a$  是整数 ,  则  $a=$

  \item  设  $V$  是实数域  $R$  上的  $n$  维线性空间 .  记  $L_{V}$  为  $V$  的全体线性变换构成的集合 .  若定义  $(\mathscr{A}+\mathscr{B})(v)=$ $\mathscr{A}(v)+\mathscr{B}(v),(r \mathscr{A})(v)=r \mathscr{A}(v)$,  其中  $\mathscr{A}, \mathscr{B} \in L_{V}, r \in R, v \in V$,  则装配上这两种运算的  $L_{V}$  形成一 

  \item  如果  2009  阶实整数方阵  $A=\left(a_{i j}\right)$  中的元素满足 : $a_{i i}$  是奇数  $(i=1,2, \cdots, 2009) ; i<j$  时 , $a_{i j}$  是偶数 ,  则行列式  $|A|$  的值 ( )

  \item  如果  $A, B$  都是  $n$  阶实方阵 ,  则下列选项中不正确的是  ( )

  \item $\lambda_{1}, \lambda_{2}\left(\lambda_{1} \neq \lambda_{2}\right)$  是实对称矩阵  $A$  的特征值 , $\left\{\alpha_{1}, \alpha_{2}, \alpha_{3}\right\}$  是属于  $\lambda_{1}$  的一个线性无关特征向量组 , $\left\{\beta_{1}, \beta_{2}\right\}$\\
(A) $\alpha_{1}, \alpha_{2}, \beta_{2}$  一定线性无关 \\
(B) $\left\{\alpha_{1}, \alpha_{3}, \beta_{1}\right\}$  一定是正交向量组 \\
(C) $\alpha_{1}, \beta_{1}, \beta_{2}$  一定线性无关 \\
(D) $\alpha_{1}, \beta_{1}, \beta_{2}$  一定线性相关 

  \item  如果  $n$  阶实矩阵  $A$  的列向量组是  $n$  维向量空间  $R^{n}$  的一组标准正交基 ,  则 ( )\\
(A) $|A|=1$  一定成立 .\\
(B) $A$  的行向量组也构成  $n$  维向量空间的一组标准正交基 .\\
(C) $A$  的秩不一定等于  $n$.\\
(D)  以上三个选项都不正确 .

  \item  若  $A$  是  $n$  阶初等矩阵 ,  则  ( )\\
(A) $|A| \neq 0$  一定成立 \\
(B) $|A|=1$  一定成立 \\
(C) $|A|=1$  一定不成立 \\
(D) $|A|<0$  一定成立 

  \item  如果  $n$  阶实矩阵  $A=r A^{\prime}(r \neq 0)$,  则  $A$  的特征值  $\lambda$  及其共轭  $\bar{\lambda}$  之间的关系是  $($ )\\
(A) $\bar{\lambda}=r \lambda$\\
(B) $\lambda=r \bar{\lambda}$\\
(C) $\bar{\lambda}=-r \lambda$\\
(D)  没有确定的关系   空间的维数必然  ( )\\
(A)  等于  1\\
(C)  不确定 \\
(D)  前三个选项都不正确 \\
(A) $A$  的特征值一定是正数 \\
(B)  二次型  $X^{\prime} A X$  一定是正定二次型 \\
(C) $A$  一定与单位矩阵相似 \\
(D) $A$  一定与单位矩阵合同 

\end{enumerate}
$$
\left\{\begin{array}{l}
a_{11} x_{1}+\cdots+a_{1 n} x_{n}=b_{1} \\
\cdots \cdots \cdots \\
a_{n 1} x_{1}+\cdots+a_{n n} x_{n}=b_{n} \\
a_{n+1,1} x_{1}+\cdots+a_{n+1,1} x_{n}=b_{n+1}
\end{array}\right.
$$
 有解的 ( )\\
(A)  必要条件 \\
(B)  充分条件 \\
(D)  以上三个选项都不正确 \\
(C)  既相似 ,  又合同 \\
(D)  既不相似 ,  也不合同 

 三 、 证明题 ( 本大题共  35  分 )

\begin{enumerate}
  \setcounter{enumi}{3}
  \item ( 13  分 )  对实矩阵  $\left(a_{i j}\right)_{m \times n},\left(b_{i j}\right)_{n \times p}$,  定义运算  $\otimes$  如下 :
\end{enumerate}
$$
\left(a_{i j}\right)_{m \times n} \otimes\left(b_{i j}\right)_{n \times p}=\left(c_{i j}\right)_{m \times p}
$$
 其中  $c_{i j}=\sum_{s+l=n+1,1 \leq s, l \leq n} a_{i s} b_{l j}$.

 证明 :  如果  $A, B$  是  $n$  阶实方阵 ,  则行列式  $|A \otimes B|=(-1)^{\left(\begin{array}{c}n \\ 2\end{array}\right)}|A \| B|$,  其中  $\left(\begin{array}{c}n \\ 2\end{array}\right)=\frac{n(n-1)}{2}$. ( 提示 :  若  $M_{1}, M_{2}$  是实  $m_{1}, m_{2}$  阶方阵 , $C$  是  $m_{2} \times m_{1}$  型实矩阵 ,  则行列式  $\left.\left|\begin{array}{cc}M_{1} & 0 \\ C & M_{2}\end{array}\right|=\left|M_{1}\right|\left|M_{2}\right| .\right)$

 四 、 计算题 ( 本大题共  35  分 .  要求写出解题过程 )

\begin{enumerate}
  \item ( 10  分 )  将矩阵  $A=\left(\begin{array}{ccc}3 & 0 & 2 \\ 3 & -1 & 2 \\ 1 & 0 & 1\end{array}\right)$  分解成初等矩阵乘积的形式 .
\end{enumerate}
2.( 12  分 )  向量组 
$$
\alpha_{1}=(1,-1,0,2)^{\prime}, \alpha_{2}=(1,1,1,-1)^{\prime}, \alpha_{3}=(1,3,2,-4)^{\prime}, \alpha_{4}=(5,1,3,1)
$$
 求出它的一个极大线性无关组 ,  并把其余向量用该极大线性无关组线性表出 .

3.( 13  分 )  计算  $\left(\begin{array}{lll}0 & 1 & 1 \\ 1 & 0 & 1 \\ 1 & 1 & 0\end{array}\right)^{101}\left(\begin{array}{c}-1 \\ 2 \\ 2\end{array}\right)$.

\section{3. 北京工业大学 2011 年硕士研究生入学考试试题(高等代数 865) 
 李扬 
 微信公众号: sxkyliyang}
 一 、 填空题 ( 本题共  50  分 , 每题  5  分 )

\begin{enumerate}
  \item  如果  $\left|\begin{array}{lll}x_{1} & x_{2} & x_{3} \\ y_{1} & y_{2} & y_{3} \\ z_{1} & z_{2} & z_{3}\end{array}\right|=1$,  则  $\left|\begin{array}{ccc}x_{2}+x_{3} & x_{1}+x_{3} & x_{1}+x_{2} \\ y_{2}+y_{3} & y_{1}+y_{3} & y_{1}+y_{2} \\ z_{2}+z_{3} & z_{1}+z_{3} & z_{1}+z_{2}\end{array}\right|=$

  \item  记矩阵  $\left(\begin{array}{ccc}1 & 1 & 1 \\ -2 & 3 & -1 \\ 4 & 9 & 1\end{array}\right)$  第三列三个位置的代数余子式依次是  $A_{13}, A_{23}, A_{33}$,  则表达式  $-A_{13}+5 A_{23}-$ $25 A_{31}=\underline{(\text { (要求写出计算结果) } .}$

  \item  如果  $\left|\begin{array}{cccc}x-1 & 2 & 3 & 0 \\ 2 & x+1 & -1 & -2 \\ 3 & -1 & x-3 & 0 \\ 0 & -2 & 0 & x+5\end{array}\right|=0$  的四个根是  $x_{1}, x_{2}, x_{3}, x_{4}$,  则  $\sum_{k=1}^{4} x_{k}=\underline{\text { (填写具体数 }}$  值 ).

\end{enumerate}
\includegraphics[max width=\textwidth]{2022_04_18_33b622a7abd81c227674g-199}

\begin{enumerate}
  \setcounter{enumi}{5}
  \item  矩阵方程  $X\left(\begin{array}{ccc}1 & 2 & 0 \\ 0 & 1 & 0 \\ 1 & 0 & 1\end{array}\right)=\left(\begin{array}{ccc}1 & -1 & 0 \\ 2 & 3 & -1\end{array}\right)$  的解  $X=$

  \item  若  $A$  是  3  阶实矩阵 , $A+E, A-E, A+2 E$  都不可逆 ,  则行列式  $\left|A^{*}+A^{-1}\right|=$ ( 填写具体数   值 ).

  \item  如果  $A=\left(\begin{array}{cccc}1 & 2 & -1 & 0 \\ -1 & 1 & -2 & -3 \\ 2 & 1 & 1 & a\end{array}\right), B$  是  3  阶非零矩阵 ,  且  $B A=0$,  则  $a=$ ( 填具体数值 ).

  \item  二次型  $\left(x_{1}, x_{2}, x_{3}\right)\left(\begin{array}{ccc}1 & 0 & 5 \\ 2 & -1 & -1 \\ -5 & 1 & 0\end{array}\right)\left(\begin{array}{l}x_{1} \\ x_{2} \\ x_{3}\end{array}\right)$  的正负惯性指数之和  $=$ ( 写出具体数字 ).

  \item  记  $R^{5}$  为  5  维列向量空间 , $A$  是  5  阶实方阵 .  若齐次线性方程组  $A X=0$  解空间的维数是  2 ,  则线性变换  $\mathscr{A}: \alpha \rightarrow A \alpha\left(\alpha \in R^{5}\right)$  像空间  $\mathscr{A}\left(R^{5}\right)=\left\{A \alpha \mid \alpha \in R^{5}\right\}$  的维数是 

  \item  记  $A=\left(\begin{array}{cccc}68 & -29 & 41 & -37 \\ -17 & 31 & 79 & 32 \\ 59 & 28 & -23 & 61 \\ -11 & -77 & 8 & 49\end{array}\right)$,  则  $A$  的逆矩阵  $A^{-1}=$

\end{enumerate}
 二 、  选择题 ( 本大题共  30  分 , 每题  3  分 )

\begin{enumerate}
  \item  如果  2011  阶实整数方阵  $A=\left(a_{i j}\right)$  中的元素满足 : $i+j=2012$  时 , $a_{i j}$  是奇数  $(i, j=1,2, \cdots, 2011)$, $i+j>2012$  时 , $a_{i j}$  是偶数 ,  则行列式  $|A|$  的值 ( $($ )\\
(A)  不等于零 \\
(B)  等于零 \\
(C)  不确定 \\
(D)  前三个选项都不正确 

  \item  如果  $n$  阶实方阵  $A$  是对单位矩阵  $E$  连续施行两次行初等变换得到的矩阵 ,  则  ( )\\
(A) $A$  必然是初等矩阵 \\
(B) $A$  必然不是初等矩阵 \\
(C) $A$  可能是初等矩阵 \\
(D)  前三个选项都不正确 

  \item  若  0,1  是实对称矩阵  $A$  的特征值 , $\left\{\alpha_{1}, \alpha_{2}, \alpha_{3}\right\}$  是属于  0  的一个线性无关特征向量组 , $\left\{\beta_{1}, \beta_{2}\right\}$  是属于  1  的一个特征向量构成的正交向量组 ,  则  ( )\\
(A) $\alpha_{1}, \alpha_{2}, \beta_{2}$  一定线性相关 \\
(B) $\left\{\alpha_{1}, \alpha_{3}, \beta_{1}\right\}$  一定是正交向量组 \\
(C) $\alpha_{1}, \beta_{1}, \beta_{2}$  一定线性无关 \\
(D) $\left\{\alpha_{3}, \beta_{1}, \beta_{2}\right\}$  一定是正交向量组 

  \item $x_{i}, y_{i}, z_{i}(i=1,2)$  是实数 ,  记  $a=x_{1} x_{2}+y_{1} y_{2}+z_{1} z_{2}, b=x_{1}^{2}+y_{1}^{2}+z_{1}^{2}, c=x_{2}^{2}+y_{2}^{2}+z_{2}^{2}$,  则  ( )\\
(A) $a^{2} \leq b c$.\\
(B) $a^{2} \geq b c$.\\
(C) $a, b, c$  之间没有正确的逻辑关系 .\\
(D)  以上三个选项都不正确 .

  \item  若  $S_{1}=\left\{\alpha_{1}, \alpha_{2}, \cdots, \alpha_{r}\right\}, S_{2}=\left\{\beta_{1}, \beta_{2}, \cdots, \beta_{t}\right\}$  是有限维向量空间  $V$  的两个线性无关向量组 ,  且  $r<t$,  则  ( )\\
(A)  一定存在  $\beta_{h} \in S_{2}$,  使得  $S_{1} \cup\left\{\beta_{h}\right\}$  仍是线性无关的 \\
(B)  一定不存在  $\beta_{h} \in S_{2}$,  使得  $S_{1} \cup\left\{\beta_{h}\right\}$  仍是线性无关的 \\
(C)  可能存在  $S_{3}=\left\{\beta_{j_{1}}, \beta_{j_{2}}, \cdots, \beta_{j_{t-r}}\right\} \subseteq S_{2}$,  使得  $S_{1} \cup S_{3}$  是线性无关的 ( 其中  $\left.1 \leq j_{1}<\cdots<j_{t-r} \leq t\right)$\\
(D)  以上三个选项都不正确 

  \item  若  $\alpha$  是  $n$  阶矩阵  $A$  的一个属于特征值  $\theta$  的特征向量 ,  则 ( )\\
(A) $\alpha$  仍是  $A^{2}+A$  的一个特征向量 \\
(B)  对实数  $\lambda, \lambda \alpha$  仍是  $A^{2}+A$  的一个特征向量 \\
(A) $(A B)^{*}=A^{*} B^{*}$  一定成立 \\
(B) $(A B)^{*}=B^{*} A^{*}$  一定成立 \\
(C) $A^{*}, B^{*},(A B)^{*}$  三者间没有确定的逻辑关系 \\
(A) $A$  的特征值一定是正数 \\
(B)  二次型  $X^{\prime} A X$  一定是正定二次型 \\
(C) $A$  一定与单位矩阵合同 \\
(D) $|A|>0$

\end{enumerate}
(C) $\quad D=0$\\
(A) $D>0$\\
(B) $D<0$

 的关系是 ( )

 三 、 证明题 ( 本大题共  35  分 )

1.( 10  分 )  证明 :  若  $n(n \geq 2)$  阶实方阵  $A$  满足  $(A+E)(A+2 E)=0$,  则  $A$  必可相似对角化 ( 即存在可逆矩 

2.(12  分 )  用行列式 ,  秩 ,  线性相关等知识证明 :  若由数字  0,1  构成的  $n(n \geq 2)$  阶方阵  $A$  的任意两行都不相同 ,

3.( 13  分 )  对给定的元素全不为零的实数列  $r_{1}, r_{2}, \cdots, r_{n}, \cdots$,  定义实矩阵  $\left(a_{i j}\right)_{m \times n},\left(b_{i j}\right)_{n \times p}$  之间的一种加 
$$
\left(a_{i j}\right)_{m \times n} \otimes\left(b_{i j}\right)_{n \times p}=\left(c_{i j}\right)_{m \times p}
$$
 四 、 计算题 ( 本大题共  35  分 .  要求写出解题过程 ) 1. ( 10  分 )  将矩阵  $A=\left(\begin{array}{ccc}3 & 0 & 2 \\ -1 & 1 & 2 \\ -2 & 1 & 1\end{array}\right)$  分解成初等矩阵乘积的形式 .

2.( 12  分 )  向量组  $\alpha_{1}=(0,1,-1,1)^{\prime}, \alpha_{2}=(1,-1,0,-1)^{\prime}, \alpha_{3}=(2,-1,-1,-1)^{\prime}, \alpha_{4}=(7,-4,-3,-4)$.

(1)  求此向量组的秩 .

(2)  求出它的一个极大线性无关组 .

(3)  在  (2)  的基础上 ,  把其余向量用该极大线性无关组线性表出 .

3.( 13  分 )  计算  $\left(\begin{array}{lll}2 & 1 & 1 \\ 1 & 2 & 1 \\ 1 & 1 & 2\end{array}\right)^{100}\left(\begin{array}{l}2 \\ 1 \\ 3\end{array}\right)$.

\section{4. 北京工业大学 2012 年硕士研究生入学考试试题(高等代数 865) 
 李扬 
 微信公众号: sxkyliyang}
 一 、 填空题 ( 本题共  50  分 , 每题  5  分 . $a=a$  型答案无效 )

\begin{enumerate}
  \item  已知  $n$ ( 自然数  $n \geq 1$ )  阶方阵  $J=(1)_{n \times n}$  的所有元素都是  $1 ; A=\left(a_{i j}\right)_{n \times n}$  中除了  $a_{11}$  外 ,  其余元素  $a_{i j}=0$.  如果  $J$  和  $A$  相似 ,  则  $a_{11}=$

  \item  如果  $A$  是  $n$ ( 自然数  $n \geq 1$ )  阶方阵 , $E$  是同阶单位矩阵 ,  且  $E+A$  可逆 , $B=(E+A)^{-1}(E-A)$.  则  $(E+B)$  是可逆的 ,  其逆矩阵  $(E+B)^{-1}=$ ( 写出最简表达式 )

  \item  如果  $\left|\begin{array}{cccc}x-1 & -1 & -1 & -1 \\ -1 & x-2 & 1 & 3 \\ -1 & -4 & x-1 & -9 \\ -1 & -8 & 1 & x+27\end{array}\right|=0$  的四个根是  $x_{1}, x_{2}, x_{3}, x_{4}$,  则乘积  $x_{1} x_{2} x_{3} x_{4}=$

\end{enumerate}
 写具体数值 )

\begin{enumerate}
  \setcounter{enumi}{5}
  \item $A^{*}$  是  $n$ ( 自然数  $n \geq 2$ )  阶实方阵  $A$  的伴随矩阵 , $A^{\prime}$  是  $A$  的转置矩阵 .  如果  $A A^{\prime}$  的秩是  $n-1$,  则齐次线   性方程组  $A^{*} X=0$  的解空间的维数是  $=$

  \item 3  阶实对称矩阵  $A=\left(\alpha_{1}, \alpha_{2}, \alpha_{3}\right)$  的列向量组  $\alpha_{1}, \alpha_{2}, \alpha_{3}$  满足  $\lambda_{1} \alpha_{1}+\lambda_{2} \alpha_{2}+\lambda_{3} \alpha_{3}=0\left(\right.$  其中 , $\lambda_{1}, \lambda_{2}, \lambda_{3}$

\end{enumerate}
.( 写出具体数  ( 写出具体数字 ).

(C)  一定大于零 

(D)  不确定 \\
(B)  等于零 

\begin{enumerate}
  \setcounter{enumi}{2}
  \item $A$  是  3  阶实方阵 , $E$  是同阶单位矩阵 .  如果  $A+E, A-E, 2 A-E$  都不可逆 ,  则下列选项中不正确的是  (\\
(A) $A$  一定是可逆的 \\
(D) $A$  一定可以写成初等矩阵的乘积 

  \item  已知  0,1  是  3  阶实对称矩阵  $A$  的特征值 , $\alpha$  是属于  0  的一个特征向量 , $\beta_{1}, \beta_{2}$  是属于  1  的 ,  由特征向量组 \\
(A) $\alpha, \beta_{1}, \beta_{2}$  一定线性相关 \\
(B) $\left\{\alpha, \beta_{1}, \beta_{1}\right\}$  一定是正交向量组 \\
(C) $B$  一定是正交矩阵 \\
(D) $B^{\prime} B$  一定是对角矩阵 

  \item  若  $\lambda$  是实正交矩阵  $A$  的特征值 , $\alpha$  是  $A$  的特征向量 ,  则 ( )\\
(A) $\lambda$  是  1  或  $-1$.\\
(B)  任意给定实系数多项式  $f(x), f(\lambda)$  总是  $f(A)$  的特征值 .\\
(C) $\alpha$  是  $A^{-1}$  的特征向量 .\\
(D)  前三个选项都不正确 .

  \item  若实数  $a>b>0$,  则矩阵  $\left(\begin{array}{ccc}a & b & b \\ b & a & b \\ b & b & a\end{array}\right)$  与  $\left(\begin{array}{ccc}a+2 b & 0 & 0 \\ 0 & a-b & 0 \\ 0 & 0 & a+2 b\end{array}\right)$  的关系是  $($ )\\
(A)  相似而且等价 \\
(B)  合同但不相似 \\
(C)  相似但不合同 \\
(D)  合同但不等价 

  \item  设  $E$  是  $n$ ( 自然数  $n \geq 2)$  阶单位矩阵 ;  同阶方阵  $A=\left(a_{i j}\right)$  满足  $a_{i i}=2(i=1,2, \cdots, n), a_{k, k+1}, a_{k+1, k}=$ $1(k=1,2, \cdots, n-1)$,  其余元素皆为  0 .  下列选项中不正确的是  $($ )\\
(A) $A^{3}+E$  是正定矩阵 \\
(B) $A^{3}+E$  的特征值皆为正数 \\
(C) $A^{3}+E$  的特征值皆为负数 \\
(D)  行列式  $\left|A^{3}+E\right|>0$

  \item  设  $A, B$  是  $n$ ( 自然数  $n \geq 1$ )  阶实方阵 , $E$  是单位矩阵 .  下述选项中不正确的是  ( )\\
(A)  若  $A B=E$,  则  $B A=E$  一定成立 \\
(B) $(A B)^{\prime}=B^{\prime} A^{\prime}$  一定成立 \\
(C) $(A B)^{-1}=B^{-1} A^{-1}$  一定成立 \\
(D)  行列式  $|A B|=|A||B|$  一定成立 

  \item  若  3  维列向量组  $\left\{\alpha_{1}, \alpha_{2}, \alpha_{3}\right\},\left\{\beta_{1}, \beta_{2}\right\}$  作为列向量形成的矩阵  $A=\left(\alpha_{1}, \alpha_{2}, \alpha_{3}\right), B=\left(\beta_{1}, \beta_{2}\right)$  满足  $A=B\left(\begin{array}{ccc}1 & -5 & 7 \\ 2 & 6 & 8\end{array}\right)$,  则齐次线性方程组  $A X=0$  的解的情况是  $($ )\\
(C)  有无穷多解 \\
(D)  不确定 :  依赖于  $\alpha_{1}, \alpha_{2}, \alpha_{3}$  的具体情况 

  \item  设自然数  $m>n>1, R$  表示实数域 .  记  $m \times n$  型实矩阵  $\left(a_{i j}\right)_{m \times n}$  的行向量组为  $\left\{\alpha_{1}, \alpha_{2}, \cdots, \alpha_{m}\right\}$,  列向   量组为  $\left\{\beta_{1}, \beta_{2}, \cdots, \beta_{n}\right\}$.  若它们线性组合成的向量空间分别记为 

\end{enumerate}
$$
\begin{gathered}
S_{1}=\left\{\lambda_{1} \alpha_{1}+\lambda_{2} \alpha_{2}+\cdots+\lambda_{m} \alpha_{m} \mid \lambda_{i} \in R, i=1,2, \cdots, m\right\} \\
S_{2}=\left\{\gamma_{1} \beta_{1}+\gamma_{2} \beta_{2}+\cdots+\gamma_{n} \beta_{n} \mid \gamma_{i} \in R, i=1,2, \cdots, n\right\}
\end{gathered}
$$
 则维数  $\operatorname{dim} S_{1}, \operatorname{dim} S_{2}$  之间的关系是  $($ )\\
(D)  没有确定的大小比较关系 

\begin{enumerate}
  \setcounter{enumi}{10}
  \item $A$  是  $m \times n$  实矩阵 ( 自然数  $m, n>1) . A^{\prime} A$  定义了  $n$  维实数列向量空间  $R^{n}$  到自身的一个线性变换  $\mathscr{A}: \alpha \rightarrow\left(A^{\prime} A\right) \alpha$.  若  $A$  的秩  $r(A)=k$,  则像空间  $W=\left\{\mathscr{A}(\alpha) \mid \alpha \in R^{n}\right\}$  的维数  ( )\\
(A) $\operatorname{dim} W=k$
\end{enumerate}
 三 、 证明题 ( 本大题共  36  分 )

3.( 12  分 )  将  $n$ ( 自然数  $n \geq 2$ )  阶实矩阵  $A$  的第一行的  $-1$  倍加到其余所有行上 ,  得到矩阵  $A_{1}$;  将  $A_{1}$  的第一列的  $-1$  倍加到其余所有列上 ,  得到矩阵  $A_{2}$;  将  $A_{2}$  的第一行 ,  第一列删掉 ,  得到矩阵  $A_{3}$.  记  $f\left(x_{1}, x_{2} \cdots, x_{n}\right)=X^{\prime} A^{*} X$ ( 其中 ,  行向量  $X^{\prime}=\left(x_{1}, x_{2}, \cdots, x_{n}\right), A^{*}$  是  $A$  的伴随矩阵 ).  证明 :  当  $x_{1}=1(i=1,2, \cdots, n)$  时 , $f(1,1, \cdots, 1)=\left|A_{3}\right|$ ( 提示 :  可考虑  $A+J$  及其行列式  $|A+J|$.  其中 , $J$  表示所有   元素都是  1  的  $n$  阶矩阵 ).  四 、 计算题 ( 本大题共  34  分 .  要求写出解题过程 )

\begin{enumerate}
  \item ( 12  分 )  求两对  $3 \times 2$  型 , $2 \times 3$  型矩阵  $\left(A_{3 \times 2}, B_{2 \times 3}\right),\left(C_{3 \times 2}, D_{2 \times 3}\right)$  使得它们具有相同的乘积  $A B=C D=$ $\left(\begin{array}{ccc}8 & 2 & -2 \\ 2 & 5 & 4 \\ -2 & 4 & 5\end{array}\right)$

  \item ( 10  分 )  参数  $t$  取何整数时 ,  线性方程组  $\left\{\begin{array}{c}2 x_{1}+x_{2}+4 x_{3}+3 x_{4}=1 \\ x_{1}+3 x_{2}+2 x_{3}-x_{4}=3 t \\ x_{1}+x_{2}+2 x_{3}+x_{4}=t^{2}\end{array}\right.$  有解 ?  写出相应情况下方程组的一   般解 .

\end{enumerate}
3.( 12  分 )  已知数列  $a_{0}, a_{1}, \cdots, a_{n}, a_{n+1}, a_{n+2}, \cdots$  满足  $a_{n+2}-5 a_{n+1}+6 a_{n}=0$.  通过考虑  $\left(\begin{array}{c}a_{n+2} \\ a_{n+1}\end{array}\right)$,  利用   相似对角化的知识 ,  求通项  $a_{n}$  关于初始项  $a_{0}, a_{1}$  的表达式 .

\section{5. 北京工业大学 2013 年硕士研究生入学考试试题(高等代数865) 
 李扬 
 微信公众号: sxkyliyang}
 记号说明 :

$A^{\prime}$  表示矩阵  $A$  的转置 ; $A^{*}$  表示矩阵  $A$  的伴随矩阵 ;

$|A|$  表示矩阵  $A$  的行列式 ; $\max \{a, b\}$  表示实数  $a, b$  中较大的数 ;

$R$  表示实数域 ; $E$  表示单位矩阵 .

 一 、 填空题 ( 写出正确答案 ,  本题共  25  分 , 每小题  5  分 )

\begin{enumerate}
  \item  如果实方阵  $A=\left(\begin{array}{ccc}1 & -1 & 0 \\ 0 & 1 & -1 \\ 0 & 0 & 1\end{array}\right)$,  则  $A^{n}=$

  \item  已知  $n$ ( 自然数  $n \geq 1$ )  阶方阵  $J$  的所有元素都是  $-1, A=\left(a_{i j}\right)_{n \times n}$  中除了  $a_{n n}$  外 ,  其余元素  $a_{i j}=0$.  如   果  $J$  和  $A$  相似 ,  则  $a_{n \times n}=$

  \item  一个  $n$  阶行列式  $D$  的元素由  $a_{i j}=\max \{i, j\}$  给定 ,  则  $D=$

  \item  设  $\alpha$  为  3  维列向量 , $\alpha^{\prime}$  是  $\alpha$  的转置 ,  如果  $\alpha \alpha^{\prime}=\left(\begin{array}{ccc}1 & -2 & 1 \\ -2 & 4 & -2 \\ 1 & -2 & 1\end{array}\right)$,  则  $\alpha^{\prime} \alpha=$

  \item  设  $R$  为实数域 ,  集合  $T=\left\{\left(\begin{array}{ccc}u & v & u \\ v & x+y & x \\ u & x & u\end{array}\right) \mid u, v, x, y \in R\right\}$  关于矩阵的加法和数乘构成  $R-$  线性空   间 ,  则  $T$  的一组基为   维数是 

\end{enumerate}
 二 、 选择题 ( 将正确答案的选项填入括号中 ,  本题共  25  分 ,  每小题  5  分 )

\begin{enumerate}
  \item  设  $A, B$  均为  $n$  阶矩阵 ,  若  $|A|=2,|B|=3$,  则分块矩阵  $\left(\begin{array}{cc}0 & A \\ B & 0\end{array}\right)$  的伴随矩阵为  $($ )\\
(A) $(-1)^{n}\left(\begin{array}{cc}0 & 3 B^{*} \\ 2 A^{*} & 0\end{array}\right)$\\
(B) $(-1)^{n}\left(\begin{array}{cc}0 & 2 B^{*} \\ 3 A^{*} & 0\end{array}\right)$\\
(C) $\left(\begin{array}{cc}0 & 2 B^{*} \\ 3 A^{*} & 0\end{array}\right)$\\
(D) $\left(\begin{array}{cc}0 & 3 B^{*} \\ 2 A^{*} & 0\end{array}\right)$

  \item  设  $A, P$  均为  3  阶矩阵 ,  且  $P^{-1} A P=\left(\begin{array}{lll}1 & 0 & 0 \\ 0 & 1 & 0 \\ 0 & 0 & 2\end{array}\right)$,  若  $P=\left(\alpha_{1}, \alpha_{2}, \alpha_{3}\right), Q=\left(\alpha_{1}, \alpha_{1}+\alpha_{2}, \alpha_{3}\right)$,  则  $Q^{-1} A Q=($ )\\
(A) $\left(\begin{array}{lll}2 & 1 & 0 \\ 1 & 1 & 0 \\ 0 & 0 & 2\end{array}\right)$\\
(B) $\left(\begin{array}{lll}1 & 1 & 0 \\ 1 & 2 & 0 \\ 0 & 0 & 2\end{array}\right)$\\
(C) $\left(\begin{array}{lll}1 & 0 & 0 \\ 0 & 1 & 0 \\ 0 & 0 & 2\end{array}\right)$\\
(D) $\left(\begin{array}{lll}1 & 0 & 0 \\ 0 & 2 & 0 \\ 0 & 0 & 2\end{array}\right)$

  \item  设向量组  $I: \alpha_{1}, \alpha_{2}, \cdots, \alpha_{r}$  可由向量组  $I I: \beta_{1}, \beta_{2}, \cdots, \beta_{s}$  线性表示 ,  则 ( )\\
(A)  当  $r<s$  时 ,  向量组  $I I$  必线性相关 ;\\
(B)  当  $r>s$  时 ,  向量组  $I I$  必线性相关 ;\\
(C)  当  $r<s$  时 ,  向量组  $I$  必线性相关 ;\\
(D)  当  $r>s$  时 ,  向量组  $I$  必线性相关 ;

  \item $n$  阶方阵  $A$  满足  $A^{2}=3 A$,  且  $A$  的秩为  $r$,  则行列式  $|A-E|=($ )\\
(A) $(-1)^{n-r} 3^{r}$\\
(B) $3^{r}$\\
(C) $(-1)^{n-r} 2^{r}$\\
(D) $2^{r}$

  \item  设  $A$  是  $n$  阶实矩阵 ,  令  $B=A^{\prime} A$,  则  ( )\\
(A) $B$  一定既相似又合同于一个对角矩阵 \\
(B) $B$  一定相似但不合同于一个对角矩阵 \\
(C) $B$  一定合同但不相似于一个对角矩阵 \\
(D) $B$  一定不相似也不合同于一个对角矩阵 

\end{enumerate}
 三 、 ( 本题  18  分 )  设  $n$  元线性方程组  $A X=b$,  其中 
$$
A=\left(\begin{array}{cccc}
2 a & a^{2} & & \\
1 & 2 a & \ddots & \\
& & \ddots & a^{2} \\
& & 1 & 2 a
\end{array}\right)_{n \times n}, \quad X=\left(\begin{array}{c}
x_{1} \\
x_{2} \\
\vdots \\
x_{n}
\end{array}\right), \quad b=\left(\begin{array}{c}
1 \\
0 \\
\vdots \\
0
\end{array}\right)
$$
(1)  证明 : $|A|=(n+1) a^{n}$;

(2)  根据  $a$  取值讨论方程组解的情况 ;  若有解 ,  求出所有解  $X$;  若无解 ,  请说明理由 .

 四 、 ( 本题  20  分 )  设二次型 
$$
f\left(x_{1}, x_{2}, x_{3}\right)=X^{\prime} A X=a x_{1}^{2}+2 x_{2}^{2}-2 x_{3}^{2}+2 b x_{1} x_{3}(b>0)
$$
 中二次型的矩阵  $A$  的特征值之和为  1 ,  特征值之积为  $-12$.\\
(1)  求  $a, b$  的值 ;\\
(2)  用正交替换将二次型  $f\left(x_{1}, x_{2}, x_{3}\right)$  化为标准型 ,  并写出所用的正交替换 .

 五 、( 本题  12  分 )  设  $A$  是  $n$  阶实对称矩阵 ,  证明 :\\
(1) $A$  的特征值都是实数 ;\\
(2) $A$  的属于不同特征值的两个特征向量的对应分量乘积的和为  0 .

 六 、( 本题  12  分 )  设  $A$  是  $n$  阶实矩阵 , $A^{*}$  是  $A$  的伴随矩阵 .

(1)  证明 : $\left|A^{*}\right|=|A|^{n-1}$;\\
(2)  如果  $A^{*}=\left(\begin{array}{cccc}1 & 0 & 0 & 0 \\ 0 & 1 & 0 & 0 \\ 1 & 0 & 4 & 0 \\ 0 & -3 & 0 & 4\end{array}\right)$\\
 且  $A B A^{-1}=B A^{-1}+3 E$,  求矩阵  $B$.

 七 、 ( 本题  20  分 )  设  $A$  是  2  阶实矩阵 ,  若存在正整数  $k$,  使得  $A^{k}=0$,  则称  $A$  是幂零矩阵 ; $V$  是全体  2  阶实矩阵   组成的  $R-$  线性空间 , $E_{i j}$  表示  $(i, j)(i, j=1,2)$  位置元素为  1 ,  其余位置上为  0  的  2  阶矩阵 .  定义映射  $\mathscr{A}$ :
$$
\mathscr{A}(X)=A X-X A, X \in V
$$
(1)  证明 : $\left\{E_{11}, E_{12}, E_{21}, E_{22}\right\}$  是  $V$  的一组基 , $\mathscr{A}$  是线性空间  $V$  上的线性变换 ;

(2)  若  $A=\left(\begin{array}{cc}1 & 2 \\ 0 & 1\end{array}\right)$,  求  $\mathscr{A}$  在基  $\left\{E_{11}, E_{12}, E_{21}, E_{22}\right\}$  下的矩阵 ;\\
(1) $r(A)+r(E+A)=n$;

(3) $A$  可以表示成两个秩均为  $r$  的对称矩阵的乘积 ,  并说明表法是否唯一 .

\section{6. 北京工业大学 2014 年硕士研究生入学考试试题(高等代数 865) 
 李扬 
 微信公众号: sxkyliyang}
 记号说明 :

$A^{\prime}$  表示矩阵  $A$  的转置 ; $A^{*}$  表示矩阵  $A$  的伴随矩阵 ;

$|A|$  表示矩阵  $A$  的行列式 ; $\max \{a, b\}$  表示实数  $a, b$  中较大的数 ;

$R$  表示实数域 ; $E$  表示单位矩阵 .

 一 、 填空题 ( 写出正确答案 ,  本题共  25  分 , 每小题  5  分 )

\begin{enumerate}
  \item  如果实方阵  $A=\left(\begin{array}{lll}1 & 1 & 1 \\ 1 & 1 & 1 \\ 1 & 1 & 1\end{array}\right)$,  则  $A^{*}=$

  \item  已知  3  阶方阵  $A$  的特征值是方程  $x^{3}=1$  的三个不同根 ,  则  $|A+E|=$

  \item  二次型  $\left(x_{1} x_{2} x_{3}\right)\left(\begin{array}{ccc}1 & 2 & 1 \\ -4 & 1 & 2 \\ -1 & 2 a & 0\end{array}\right)\left(\begin{array}{l}x_{1} \\ x_{2} \\ x_{3}\end{array}\right)$  的秩为  2 ,  则  $a=$

  \item  设  $A=\left(\begin{array}{ccc}1 & -1 & 0 \\ 0 & 1 & -1 \\ 0 & 0 & 1\end{array}\right), T=\{B \mid A B=B A\}$,  其中  $B$  为  3  阶实方阵 , $T$  关于矩阵加法和数乘构成  $R-$  线性空间 ,  则  $T$  的一组基为  ,  维数是 

  \item  设  $D_{n}=\left|a_{i j}\right|_{n \times n}$  是  $n$  阶行列式 ,  其中  $a_{i i}=2, a_{i, i+1}=a_{i+1, i}=-1, i=1,2, \cdots, n-1$,  则  $D_{n}=$ ( 写出具体表达式 ).

\end{enumerate}
 二 、 选择题 ( 将正确答案的选项填入括号中 ,  本题共  25  分 ,  每小题  5  分 )

\begin{enumerate}
  \item  设  $A, P$  均为  3  阶矩阵 ,  且  $P^{\prime} A P=\left(\begin{array}{lll}1 & 0 & 0 \\ 0 & 1 & 0 \\ 0 & 0 & 2\end{array}\right)$,  若  $P=\left(\alpha_{1}, \alpha_{2}, \alpha_{3}\right), Q=\left(\alpha_{1}, \alpha_{1}+\alpha_{2}, \alpha_{3}\right)$.  则  $Q^{\prime} Q=($ )\\
(A) $\left(\begin{array}{lll}1 & 1 & 0 \\ 1 & 2 & 0 \\ 0 & 0 & 2\end{array}\right)$\\
(B) $\left(\begin{array}{lll}1 & 1 & 0 \\ 1 & 1 & 0 \\ 0 & 0 & 2\end{array}\right)$\\
(C) $\left(\begin{array}{lll}1 & 0 & 0 \\ 0 & 1 & 0 \\ 0 & 0 & 2\end{array}\right)$\\
(D) $\left(\begin{array}{lll}1 & 0 & 0 \\ 0 & 2 & 0 \\ 0 & 0 & 2\end{array}\right)$

  \item  秩为  $r$  的  $n$  阶方阵  $A$  满足  $A^{2}=2 A$,  则行列式  $|A+E|=($ )\\
(A) $(-1)^{r} 3^{r}$\\
(B) $3^{r}$\\
(C) $(-1)^{r} 2^{r}$\\
(D) $2^{r}$

  \item  设  $A=\left(\begin{array}{lll}a & b & b \\ b & a & b \\ b & b & a\end{array}\right)$,  且  $A$  的伴随矩阵  $A^{*}$  的值是  1,  则  $a$  和  $b$  的关系是  $($ )

\end{enumerate}
(D) $a+2 b=0$.\\
(A) $a=b$.\\
(B) $a \neq b$  且  $a \neq 2 b$.\\
(C) $a \neq b$  且  $a+2 b=0$.

\begin{enumerate}
  \setcounter{enumi}{4}
  \item  向量组  $\alpha_{1}, \alpha_{2}, \alpha_{3}$  线性无关 ,  而  $\alpha_{2}, \alpha_{3}, \alpha_{4}$  线性相关 ,  则下面论断正确的是 ( )\\
(A) $\alpha_{1}$  能被  $\alpha_{2}, \alpha_{3}, \alpha_{4}$  线性表出 .\\
(B) $\alpha_{1}$  不能被  $\alpha_{2}, \alpha_{3}, \alpha_{4}$  线性表出 .\\
(C) $\alpha_{1}$  能被  $\alpha_{3}, \alpha_{4}$  线性表出 .\\
(D) $\alpha_{4}$  不能被  $\alpha_{2}, \alpha_{3}$  线性表出 .

  \item  设  $V, U$  分别是  $n$  维 , $m$  维线性空间  $(m \neq n), \varphi: V \rightarrow U$  的线性映射 ,  则 ( )\\
(A) $\operatorname{dim} \operatorname{ker} \varphi+\operatorname{dim} \operatorname{Im} \varphi=n$.\\
(B) $\operatorname{dim} \operatorname{ker} \varphi+\operatorname{dim} \operatorname{Im} \varphi=m$.\\
(C) $\operatorname{dim} \operatorname{ker} \varphi+\operatorname{dim} \operatorname{Im} \varphi=|m-n|$.\\
(D) $\operatorname{dim} \operatorname{ker} \varphi+\operatorname{dim} \operatorname{Im} \varphi=m+n$.

\end{enumerate}
 三 、 ( 本题  18  分 )  已知线性方程组 
$$
\left\{\begin{array}{l}
x_{1}+x_{2}+x_{3}=1 \\
2 x_{1}+(a+2) x_{2}+(a+1) x_{3}=a+3 \\
x_{1}+2 x_{2}+a x_{3}=3
\end{array}\right.
$$
 有无穷多解 ;  设  $A$  是三阶矩阵 , $\alpha_{1}=(1, a, 0)^{\prime}, \alpha_{2}=(-a, 1,0)^{\prime}, \alpha_{3}=(0,0, a)^{\prime}$  分别为  $A$  的属于特征值  $\lambda_{1}=1, \lambda_{2}=-2, \lambda_{3}=-1$  的特征向量 .\\
(1)  求所给线性方程组的通解 ;\\
(2)  求矩阵  $A$;\\
(3)  求行列式  $\left|A^{*}+3 E\right|$  的值 .

 四 、 ( 本题  20  分 )  设  $\eta$  是欧氏空间中的一单位向量 ,  定义  $\mathscr{A}(\alpha)=\alpha-2(\eta, \alpha) \eta$.  证明 :\\
(1) $\mathscr{A}$  是正交变换 .  这样的正交变换称为镜面反射 ;\\
(2)  如果  $n$  维欧氏空间中正交变换  $\mathscr{A}$  以  1  作为一个特征值 ,  且属于特征值  1  的特征子空间  $V_{1}$  的维数为  $n-1$,  则  $\mathscr{A}$  一定是镜面反射 .

 五 、( 本题  20  分 )  设  $V$  是数域  $P$  上的  $n$  维线性空间 , $\mathscr{A}$  是  $V$  的线性变换 , $\mathscr{A}\left(\alpha_{1}\right)=2 \alpha_{1}, I$  为  $V$  上的恒等变   换 .  向量组  $\alpha_{1}, \alpha_{2}, \cdots, \alpha_{n}$  满足  $(\mathscr{A}-2 I) \alpha_{i+1}=\alpha_{i}(i=1,2, \cdots, n-1)$\\
(1)  证明  $\alpha_{1}, \alpha_{2}, \cdots, \alpha_{n}$  为  $V$  的一组基 ;\\
(2)  求  $\mathscr{A}$  在  $\alpha_{1}, \alpha_{2}, \cdots, \alpha_{n}$  下的矩阵 .

 六 、 ( 本题  18  分 )  设  $A$  是实数域上的  $n$  阶矩阵 ,  且  $A^{2}+2 A=3 E$.\\
(1)  证明 :  矩阵  $A$  可逆 , 并用矩阵  $A$  的多项式表示  $A^{-1}$;\\
(2)  证明 : $r(A-E)+r(A+3 E)=n$;\\
(3)  证明 : $A$  是可对角化矩阵并且可以表示成两个可逆的实对称矩阵的乘积 .

 七 、 ( 本题  24  分 )  设  $A, B$  是实数域上的  $n$  阶矩阵 , $f(x)$  是矩阵  $B$  的特征多项式 ,  令  $f^{(k)}(x)$  表示  $f(x)$  的  $k$  阶导   数 . $C=A B-B A$.  假定  $C$  与  $A, B$  可交换 .  证明 :\\
(1)  对任意正整数  $k$,  有  $A B^{k}-B^{k} A=k B^{k-1} C$;\\
(2)  对每个正整数  $k \leq n$,  有  $f^{(k)}(B) C^{k}=0$,  特别地 ,  有  $C^{n}=0_{n \times n}$;\\
(3)  若  $A, B$  均为实对称矩阵 ,  则  $A B=B A$.

\section{7. 北京工业大学 2015 年硕士研究生入学考试试题(高等代数 865) 
 李扬 
 微信公众号: sxkyliyang}
 记号说明 :

$\mathbb{R}$  表示实数域 , $\mathbb{R}^{n \times n}$  表示实数域上所有  $n$  阶方阵组成的线性空间 ,

$|A|$  表示方阵  $A$  的行列式 , $A^{\prime}$  表示矩阵  $A$  的转置 ,

$E_{n}$  表示  $n$  阶单位矩阵 .

 一 、 填空题  ( 25  分 , 每题 5 分 )

\begin{enumerate}
  \item  设  $A$  是  $n$  阶方阵 , $\alpha$  为  $n \times 1$  矩阵 , $\beta$  为  $1 \times n$  矩阵 ,  且  $|A|=2,\left|\begin{array}{cc}A & \beta \\ \alpha & 1\end{array}\right|=0$,  则  $\left|\begin{array}{cc}A & \beta \\ \alpha & 4\end{array}\right|=$

  \item  若实对称矩阵  $A$  与矩阵  $B=\left(\begin{array}{lll}0 & 1 & 0 \\ 1 & 0 & 0 \\ 0 & 0 & 1\end{array}\right)$  合同且  $X=\left(\begin{array}{l}x_{1} \\ x_{2} \\ x_{3}\end{array}\right)$,  则实二次型  $X^{\prime} A X$  的规范型为 

  \item  设矩阵  $A=\left(\begin{array}{llll}1 & 2 & 1 & 2 \\ 0 & 1 & t & t \\ 1 & t & 0 & 1\end{array}\right)$,  齐次线性方程组  $A X=0$  解空间的维数为  2 ,  则  $t=$

  \item  设  $A$  与  $B$  均为  $n$  阶方阵 , $A^{*}, B^{*}$  分别为它们的伴随矩阵 , $|A|=2,|B|=-4$,  则  $\left|A^{*} B^{-1}-A^{-1} B^{*}\right|=$

  \item  如果  $\left|\begin{array}{cccc}x-1 & -1 & -1 & -1 \\ -1 & x-3 & 1 & 4 \\ -1 & -9 & x-1 & -16 \\ -1 & -27 & 1 & x+64\end{array}\right|=0$  的四个根是  $\lambda_{1}, \lambda_{2}, \lambda_{3}, \lambda_{4}$,  则乘积  $\lambda_{1} \lambda_{2} \lambda_{3} \lambda_{4}=$

\end{enumerate}
( 请给出具体数值 )

 二 、  选择题 ( 25  分 , 每题  5  分 )

\begin{enumerate}
  \setcounter{enumi}{3}
  \item  设  $\lambda_{1}, \lambda_{2}$  分别是方阵  $A$  的两个不同特征值 , $\alpha_{1}, \alpha_{2}$  分别是它们对应的特征向量 ,  则向量组  $\alpha_{1}, A\left(\alpha_{1}+\alpha_{2}\right)$\\
A. $B$  是正定矩阵 .\\
B. $B$  是半正定矩阵 .\\
D.  无法确定  $B$  的正 ,  负定性 .

  \item  实二次型  $f\left(x_{1}, x_{2}, x_{3}\right)=a\left(x_{1}^{2}+x_{2}^{2}+x_{3}^{2}\right)+4 x_{1} x_{2}+4 x_{1} x_{3}+4 x_{2} x_{3}$  经过非退化线性替换  $X=C Y$  可退化   量 ,  且  $\alpha_{1}+\alpha_{2}=(1,2,0,4)^{\prime}, \alpha_{3}+\alpha_{2}=(1,0,0,1)^{\prime}$

\end{enumerate}
 四 、 ( 22  分 )  设  $A$  是  $n$  阶方阵 .  令  $T(A)=\left\{B \mid A B=B A, B \in \mathbb{R}^{n \times n}\right\}$. (2)  若  $A$  是主对角线元素两两不等的对角矩阵 ,  求  $T(A)$  的维数和一组基 .

(3)  设  $A=\left[\begin{array}{cccc}1 & 0 & \cdots & 0 \\ \vdots & \ddots & \ddots & \vdots \\ \vdots & & \ddots & 0 \\ 1 & \cdots & \cdots & 1\end{array}\right]_{n \times n}$,  求  $T(A)$  的维数和一组基 .

 五 、(28  分 )  设  $V$  是实数域  $\mathbb{R}$  上的  $n$  维线性空间 , $\alpha_{1}, \alpha_{2}, \cdots, \alpha_{n}$  为  $V$  的一组基 ,  于是由 
$$
\mathscr{A}\left(\alpha_{i}\right)=\alpha_{i+1}, \quad(i=1,2, \cdots, n), \quad \mathscr{A}\left(\alpha_{n}\right)=0
$$
 定义了  $V$  的一个线性变换  $\mathscr{A}$.  回答下列问题 :

(1)  试求  $\mathscr{A}$  在  $\alpha_{1}, \alpha_{2}, \cdots, \alpha_{n}$  下的矩阵 .

(2)  证明  $\mathscr{A}^{n}=0, \mathscr{A}^{n-1} \neq 0$.

(3)  若  $V$  有一个线性变换  $\mathscr{B}$  满足  $\mathscr{B}^{n}=0, \mathscr{B}^{n-1} \neq 0$,  则存在  $V$  的一组基 ,  使得  $\mathscr{B}$  在这组基下的矩阵与  (1)  中得到的矩阵相同 .

(4)  若  $\mathbb{R}$  上的  $n$  阶方阵  $M, N$  满足  $M^{n}=N^{n}=0, M^{n-1} \neq 0, N^{n-1} \neq 0$.  证明  $M$  与  $N$  相似 .

 六 、(22  分 )  设  $A$  是  $n$  阶实对称矩阵 ,  秩  $(A)=n, A_{i j}$  是  $|A|$  中元素  $a_{i j}$  的代数余子式 ,  记  $x=\left(x_{1}, x_{2}, \cdots, x_{n}\right)^{\prime}$,  设二次型  $f(x)=\frac{1}{|A|} \sum_{i=1}^{n} \sum_{j=1}^{n} A_{i j} x_{i} x_{j}$.

(1)  写出二次型  $f(x)$  的矩阵形式 ,  并求该二次型的矩阵 .

(2)  二次型  $g(x)=x^{\prime} A x$  与  $f(x)$  的规范型是否相同 ?  说明理由 .

 七 、 (16  分 )  设  $A=\left(a_{i j}\right)_{n \times n}$  是  $n$  阶非零实方阵 , $|A|, A^{\prime}, A^{*}$  分别表示矩阵  $A$  的行列式 ,  转置矩阵和伴随矩阵 .

(b)  若  $n>2$,  证明  $|A|=1$,  并且  $A$  是正交矩阵 .

\section{8. 北京工业大学 2016 年硕士研究生入学考试试题(高等代数 865) 
 李扬 
 微信公众号: sxkyliyang}
 记号说明 :

$A^{\prime}$  表示矩阵  $A$  的转置 , $A^{*}$  表示矩阵  $A$  的伴随矩阵 ,

$|A|$  表示方阵  $A$  的行列式 , $\operatorname{dim} V$  表示线性空间  $V$  的维数 ,

$R(A)$  表示矩阵  $A$  的秩 .

 一 、 填空题 ( 25  分 , 每题 5 分 )

\begin{enumerate}
  \item  设  $A=\left(\begin{array}{cccc}2 & 5 & 3 & 1 \\ 1 & 1 & 3 & 1 \\ 2 & 3 & -1 & 2 \\ 1 & 1 & 5 & 5\end{array}\right)$.  其中  $A_{i, j}$  是  $A$  中元素  $a_{i, j}$  的代数余子式 ,  则  $A_{1,1}+A_{1,2}+A_{1,3}+A_{1,4}=$

  \item  若  $n$  阶矩阵  $A$  的各行元素之和均为零 ,  且  $R(A)=n-1$,  则齐次线性方程组  $A x=0$  的一个基础解系为 

  \item  设  $\mathbb{R}$  为实数域 ,  集合  $V=\left\{A \in \mathbb{R}^{n \times n}: A^{\prime}=A\right\}$  关于矩阵的加法和数乘构成一个实线性空间 ,  则  $\operatorname{dim} V=$

  \item  设矩阵  $A=\left(\begin{array}{cccc}1 & 1 & 1 & 1 \\ 1 & 1 & -1 & -1 \\ 1 & -1 & 1 & -1 \\ 1 & -1 & -1 & 1\end{array}\right)$  的特征值为  $\lambda_{1}, \lambda_{2}, \lambda_{3}, \lambda_{4}$,  则  $\lambda_{1} \lambda_{2} \lambda_{3} \lambda_{4}=$

  \item  已知线性方程组 

\end{enumerate}
$$
\left\{\begin{array}{l}
-x_{1}-x_{2}+3 x_{3}=1+\lambda \\
-2 x_{1}+x_{2}+2 x_{3}=1 \\
x_{1}+x_{2}+\lambda x_{3}=\lambda
\end{array}\right.
$$
 二 、  选择题 ( 25  分 , 每题  5  分 )

D. $A, B$  的特征值为正实数 .

\begin{enumerate}
  \setcounter{enumi}{5}
  \item  下列说法正确的是  ( ).
\end{enumerate}
$A$.  数域  $\mathbb{P}$  上两线性空间同构的充要条件是它们的维数相等 .

$B$.  设矩阵  $A$  满足  $A^{2}=E$  则  1  与  $-1$  一定是  $A$  的特征值 . $C$.  正交变换在任意基下的矩阵都是正交矩阵 .

$D$  任意对称矩阵的特征值都是实数 .

 三 、 ( 22  分 )  计算行列式 
$$
\left|\begin{array}{cccccc}
a+b & a b & 0 & \cdots & 0 & 0 \\
1 & a+b & a b & \cdots & 0 & 0 \\
0 & 1 & a+b & \cdots & 0 & 0 \\
\vdots & \vdots & \vdots & & \vdots & \vdots \\
0 & 0 & 0 & \cdots & a+b & a b \\
0 & 0 & 0 & \cdots & 1 & a+b
\end{array}\right|_{n \times n}
$$
 四 、 ( 22  分 )  设  $m \times n$  矩阵  $A$  的秩为  $r$,  考虑线性方程组  $A x=b(b \neq 0)$.

(1)  设  $A x=b$  有特解  $\alpha_{0}$,  它的导出组  $A x=b$  的一组基础解系为  $\eta_{1}, \eta_{2}, \cdots, \eta_{n-r}$,  证明  $\alpha_{0}, \alpha_{0}+\eta_{1}, \alpha_{0}+$ $\eta_{2}, \cdots, \alpha_{0}+\eta_{n-r}$  线性无关 ;

(2)  设  $\alpha_{0}, \alpha_{1}, \alpha_{2}, \cdots, \alpha_{n-r}$  是  $A x=b$  的  $n-r+1$  个线性无关的解向量 ,  证明 :

$\alpha_{1}-\alpha_{0}, \alpha_{2}-\alpha_{0}, \cdots, \alpha_{n-r}-\alpha_{0}$  是导出组  $A x=b$  的一组基础解系 .

 五 、(18  分 )  设  $\mathscr{A}$  是  $n$  维线性空间  $V$  上的一个线性变换 .

(1)  设  $\lambda_{1}, \lambda_{2}$  是  $\mathscr{A}$  的两个不同特征值 , $\alpha_{1}, \alpha_{2}$  是分别属于  $\lambda_{1}, \lambda_{2}$  的特征向量 ,  证明  $\alpha_{1}+\alpha_{2}$  不是  $\mathscr{A}$  的特征   向量 .

(2)  若  $\mathscr{A}$  以  $V$  中每一个非零向量为它的特征向量 ,  证明  $\mathscr{A}$  是数乘变换 ,  即 :  存在  $k$  使得  $\mathscr{A} \alpha=k \alpha$.
$$
\left(\begin{array}{cccc}
1 & 0 & 2 & 1 \\
-1 & 1 & 1 & 3 \\
1 & 1 & 5 & 5 \\
3 & -1 & 3 & -1
\end{array}\right)
$$
(2)  求  $\mathscr{A}(V)$  的一组基 .

(2)  设  $A, B$  为  $n$  阶实矩阵 , $A$  有  $n$  个两两互不相同的实特征值 .  证明 : $A B=B A$  的充要条件是存在实数  $a_{0}, a_{1}, \cdots, a_{n-1}$  使得  $B=a_{0} E+a_{1} A+\cdots+a_{n-1} A^{n-1}$.

\section{9. 北京工业大学 2017 年硕士研究生入学考试试题(高等代数 865) 
 李扬 
 微信公众号: sxkyliyang}
 记号说明 :

$A^{\prime}$  表示矩阵  $A$  的转置 , $A^{*}$  表示矩阵  $A$  的伴随矩阵 ,

$|A|$  表示方阵  $A$  的行列式 , $\operatorname{dim} V$  表示线性空间  $V$  的维数 ,

$\theta$  表示零向量 , $V_{1} \oplus V_{2}$  表示子空间的直和 , $R(A)$  表示矩阵  $A$  的秩 .

 一 、 填空题  ( 25  分 , 每题 5 分 )

\begin{enumerate}
  \item  设  $A=\left|\begin{array}{cccc}4 & 6 & 7 & 3 \\ 2 & 5 & 0 & 0 \\ 1 & 3 & -1 & -1 \\ 7 & -4 & 1 & 2\end{array}\right|, A_{i, j}$  是  $A$  中元素  $a_{i, j}$  的代数余子式 ,  则  $A_{1,1}+2 A_{1,2}-A_{1,3}-A_{1,4}=$

  \item  把复数域看成它自身上的线性空间 ,  它的维数是 

  \item  设向量  $\alpha=(1,2,-1,1), \beta=(2,1,0,4), \gamma=(4,5,-2, t)$  线性相关 ,  则  $t=$

  \item  设矩阵  $\left(\begin{array}{cccc}1 & 1 & -1 & -1 \\ 1 & 2 & 1 & -1 \\ -1 & 1 & 3 & 4 \\ -1 & -1 & 4 & 0\end{array}\right)$  的特征值为  $\lambda_{1}, \lambda_{2}, \lambda_{3}, \lambda_{4}$,  则  $\lambda_{1}+\lambda_{2}+\lambda_{3}+\lambda_{4}=$

  \item  设  $A$  与  $B$  分别是  $3 \times 2$  与  $2 \times 3$  矩阵 ,  且满足  $A B=\left(\begin{array}{ccc}1 & 0 & 1 \\ 0 & 1 & 2 \\ 2 & -1 & 0\end{array}\right)$,  则  $R(A)=$

\end{enumerate}
 二 、 选择题 ( 25  分 , 每题 5 分 )

\begin{enumerate}
  \item  设  $A$  为  2  阶方阵 ,  满足  $|A-E|=0,|A+2 E|=0$,  则  $\left|A^{*}+E\right|=(~)$.\\
A. 0 .\\
B. 2 .\\
C. $-2$.\\
D. 1 .\\
C. $|A B|=|B||A|$.\\
D. $A^{k} B^{k}=(A B)^{k}$.

  \item  设  $A$  是  $n$  阶方阵 ,  则下列选项中正确的是  ( ).\\
A.  当齐次线性方程组  $A x=0$  有唯一解时 ,  行列式  $|A|=0$.\\
C.  当行列式  $|A|=0$  时 ,  线性方程组  $A x=b$  无解 .\\
D.  当行列式  $|A|=0$  时 ,  线性方程组  $A x=b$  有无穷多解 .

\end{enumerate}
\includegraphics[max width=\textwidth]{2022_04_18_33b622a7abd81c227674g-214}

 四 、 ( 22  分 )  设  $A=\left(\begin{array}{lll}0 & 1 & 1 \\ 1 & 0 & 1 \\ 1 & 1 & 0\end{array}\right)$,  求正交矩阵  $P$  使得  $P^{-1} A P$  为对角矩阵 .

 五 、 (22  分 )  在  $\mathbb{R}^{4}$  中设  $\alpha_{1}=(1,2,1,0), \alpha_{2}=(-1,1,1,1), \beta_{1}=(2,-1,0,1), \beta_{2}=(1,-1,3,7)$. W=L( $\left.\alpha_{1}, \alpha_{2}\right)$  为  $\alpha_{1}, \alpha_{2}$  生成的子空间 , $V=L\left(\beta_{1}, \beta_{2}\right)$  为  $\beta_{1}, \beta_{2}$  生成的子空间 .

(1)  求  $W+V$  的维数与一组基 .

(2)  求  $W \cap V$  的维数与一组基 .

 六 、 $(20$  分  $)$  设  $A$  与  $B$  分别是  $n$  阶实方阵 .

(1)  证明 : $R(B)=R(A B)$  的充要条件是齐次线性方程组  $B x=0$  与  $A B x=0$  的解相同 .

(2)  若存在列向量  $\alpha \in R^{n}$  满足  $A^{k-1} \alpha \neq 0, A^{k} \alpha=0$.  证明 : $\alpha, A \alpha, \cdots, A^{k-1} \alpha$  线性无关 .

(3)  证明 : $R\left(A^{n}\right)=R\left(A^{n+1}\right)$.

 七 、 (16  分 )  设  $A, B$  为实对称矩阵 , $A$  正定 .

(1)  证明 :  存在可逆矩阵  $P$  使得  $P^{\prime} A P=E$  且  $P^{\prime} B P$  为对角矩阵 .

(2)  若  $B$  也正定 ,  证明 : $|A+B|>|A|$. 10.  北京工业大学  2009  年硕士研究生入学考试试题 ( 数学分析 663)

 李扬 

 微信公众号 : sxkyliyang

 一 、 ( 15  分 )  证明 :  若  $a_{1}=\sqrt{6}, a_{n+1}=\sqrt{6+a_{n}}, n=1,2, \cdots$,  则数列  $\left\{a_{n}\right\}$  收玫 ,  并求其极限 .

 二 、 (15  分 )  证明 :  若函数  $f(x)$  在区间  $[a, b]$  上连续 , $x_{1}, x_{2}, \cdots, x_{n} \in[a, b]$,  且  $t_{1}+t_{2}+\cdots+t_{n}=1, t_{1}, f_{2}, \cdots, t_{n}>$ 0 ,  则存在一点  $q \in[a, b]$  使得  $f(q)=t_{1} f\left(x_{1}\right)+t_{2} f\left(x_{2}\right)+\cdots+t_{n} f\left(x_{n}\right)$.

 三 、 $(15$  分  $)$  证明 :  若函数  $f(x)$  在区间  $(a, b)$  内二阶可导 ,  且对  $\forall x \in(a, b)$,  有  $f^{\prime \prime}(x)>0$.  则对  $\forall x_{1}, x_{2} \in(a, b)$, $\lambda_{1}+\lambda_{2}=1, \lambda_{1}, \lambda_{2}>0$,  有  $f\left(\lambda_{1} x_{1}+\lambda_{2} x_{2}\right) \leq \lambda_{1} f\left(x_{1}\right)+\lambda_{2} f\left(x_{2}\right)$.

 四 、 (15  分 )  设曲线  $\left\{\begin{array}{l}x=x(t) \\ y=y(t)\end{array}\right.$  由方程组 
$$
\left\{\begin{array}{l}
t e^{y}+2 x-y=2 \\
x+y+2 t(1-t)=1
\end{array}\right.
$$
 确定 ,  求曲线在  $t=0$  处的切线方程与法线方程 .

 五 、 $\left(15\right.$  分 )  求旋轮线  $\left\{\begin{array}{l}x=a(t-\sin t) \\ y=a(1-\cos t)\end{array}\right.$  的长度与旋轮线绕  $x$  轴旋转一周后所围的旋转体体积 .

 六 、 (15  分 )  设  $f(x)$  在  $[0, \pi]$  上有连续二阶导数 ,  且  $f(0)=f(\pi)=0$,  令 
$$
a_{n}=\frac{2}{\pi} \int_{0}^{\pi} f(x) \sin n x \mathrm{~d} x, n=1,2, \cdots
$$
 证明 : $\sum_{n=1}^{\infty} n^{2} a_{n}^{2}$  收敛 .

 七 、 ( 15  分 )  设  $f(x, y)=\left\{\begin{array}{cc}\frac{x^{2} y}{x^{2}+y^{2}}, & x^{2}+y^{2} \neq 0 \\ 0, & x^{2}+y^{2}=0\end{array}\right.$  试讨论  $f(x, y)$  在原点处的连续性 , 偏导数存在性及可微性 .

 八 、(15  分 )  将函数  $f(x)=x^{2}$  在  $[-\pi, \pi]$  上展成傅里叶级数 ,  并求级数  $\sum_{n=1}^{\infty} \frac{1}{n^{2}}$  的和 .

 九 、 (15  分 )  计算积分  $\int_{0}^{1} \frac{x^{3}-x}{\ln x} \mathrm{~d} x$.

 十 、(15  分 )  计算  $\iint_{S}\left(x^{2} \mathrm{~d} y \mathrm{~d} z+y^{2} \mathrm{~d} x \mathrm{~d} z+z \mathrm{~d} x \mathrm{~d} y\right)$,  其中  $S$  为曲面  $z=\sqrt{x^{2}+y^{2}}$  被平面  $z=1$  所截部分的外侧 . 11.  北京工业大学  2010  年硕士研究生入学考试试题 ( 数学分析 663)

 李扬 

 微信公众号 : sxkyliyang

 一 、 (15  分 )  证明 :  若  $a_{0}=a>0, a_{n}=\frac{1}{2}\left(a_{n-1}+\frac{2}{a_{n-1}}\right), n=1,2, \cdots$,  则数列  $\left\{a_{n}\right\}$  收玫 ,  求其极限 .

 二 、 ( 15  分 )  证明 :  若函数  $f(x)$  在  $[a,+\infty)$  上连续 ,  且  $\lim _{x \rightarrow+\infty}[b x-f(x)]=0$,  其中  $b$  为非零常数 ,  则  $f(x)$  在  $[a,+\infty)$  上一致收敛 .

 三 、 ( 15  分 )  证明 :  若  $\forall x \in(a, b)$,  有  $f^{\prime \prime}(x) \geq 0$,  则  $\forall x_{1}, x_{2}, \cdots, x_{n} \in(a, b)$,  下列不等式成立 
$$
f\left(\frac{x_{1}+x_{2}+\cdots+x_{n}}{n}\right) \leq \frac{1}{n}\left[f\left(x_{1}\right)+f\left(x_{2}\right)+\cdots+f\left(x_{n}\right)\right]
$$
 四 、 $\left(15\right.$  分 )  设  $f(x)=\sum_{n=1}^{\infty} \frac{x^{n}}{n^{2}}, 0 \leq x \leq 1$.

(1)  证明 : $\forall x \in(0,1)$,  有  $f(x)+f(1-x)+(\ln x) \ln (1-x)=c($  常数 )

(2)  求常数  $c$.

 五 、(15  分 )  求极限  $\lim _{n \rightarrow \infty} \sum_{k=1}^{n} \frac{k}{n^{3}} \sqrt{n^{2}-k^{2}}$.

 六 、 (15  分 )  设  $f(x, y)=\left\{\begin{array}{cc}\frac{x^{2} y}{x^{2}+y^{2}}, & x^{2}+y^{2} \neq 0 \\ 0, & x^{2}+y^{2}=0\end{array}\right.$,  讨论  $f(x, y)$  在原点处的连续性 ,  偏导数存在性及可微性 .

 七 、(15  分 )  求  $x^{2}+(y-2)^{2}=1$  绕  $x$  轴旋转一周后所得旋转体体积与表面积 .

 八 、(15  分 )  求曲线 
$$
\left\{\begin{array}{l}
x^{2}+y^{2}+z^{2}=6 \\
x+y+z=0
\end{array}\right.
$$
 在点  $P(1,-2,1)$  处切线方程与法平面方程 .

 九 、 ( 15  分 )  计算积分  $\int_{0}^{+\infty} \frac{e^{-x}-e^{-3 x}}{x} \mathrm{~d} x$.

 十 、( 15  分 )  计算  $\iint_{S}\left(x^{2} \mathrm{~d} y \mathrm{~d} z-2 y \mathrm{~d} x \mathrm{~d} z\right)$,  其中  $S$  为曲面  $z=\sqrt{x^{2}+y^{2}}$  被平面  $z=1$  所截部分的外侧 . 12.  北京工业大学  2011  年硕士研究生入学考试试题 ( 数学分析 663)

 李扬 

 微信公众号 : sxkyliyang

 一 、 (15  分 )  设  $x_{1}>a>0, x_{n+1}=\sqrt{x_{n}^{2}-2 a x_{n}+2 a^{2}}, n=1,2, \cdots$,  证明 : $\lim _{n \rightarrow+\infty} x_{n}$  收玫 ,  并求其值 .

 二 、 (15  分 ) $f(x)$  在  $[a, b]$  上可导 ,  且  $f^{\prime}(a) f^{\prime}(b)<0$.  证明 :  在  $(a, b)$  内至少存在一点  $c$,  使得  $f^{\prime}(c)=0$.

 三 、 (15  分 )  计算定积分  $\int_{0}^{\pi} e^{x} \sin ^{2} x \mathrm{~d} x$.

 四 、(15  分 )  设函数  $f(x)$  在区间  $[0,1]$  上连续 ,  且  $\int_{0}^{1} f(x) \mathrm{d} x=0.625$,  计算  $\lim _{n \rightarrow+\infty} \sum_{k=1}^{n} 4 \ln \left[1+\frac{1}{n} f\left(\frac{k}{n}\right)\right]$.

 五 、 $\left(15\right.$  分 )  求圆  $x^{2}+(y-3)^{2}=1$  绕  $x$  轴旋转一周后所得旋转体表面积 .

 六 、(15  分 )  求幂级数  $\sum_{n=1}^{\infty} \frac{x^{n}}{n(n+1)}$  的和函数 .

 七 、 (15  分 )  设  $f(x)$  在  $[0, \pi]$  上有连续二阶导函数 ,  且  $f(0)=f(\pi)=0$,  令  $a_{n}=\frac{2}{\pi} \int_{0}^{\pi} f(x) \sin n x \mathrm{~d} x, n=1,2, \cdots$.  证明 : $\sum_{n=1}^{\infty} n^{2} a_{n}^{2}$  收敛 .

 八 、 (15  分 )  求曲面  $x y+\sin z+y=z$  在点  $P(-1,1,0)$  处的切平面方程 .

 九 、 (15  分 )  证明 : $\int_{0}^{+\infty} \frac{e^{-x}-e^{-2 x}}{x} \mathrm{~d} x=\ln 2$.

 十 、(15  分 )  计算曲面积分  $\oint_{C} \frac{x \mathrm{~d} y-y \mathrm{~d} x}{x^{2}+y^{2}}$,  其中  $C$  为曲线  $x^{2}+4 y^{2}=1$  的正向 ( 逆时针方向 )

\section{3. 北京工业大学 2012 年硕士研究生入学考试试题(数学分析663) 
 李扬 
 微信公众号: sxkyliyang}
 一 、 (15  分 )  证明 :  若对每个正整数  $n$  都有  $a_{n}>0$  且  $\lim _{n \rightarrow+\infty} \sqrt[n]{a_{n}}=r<1$,  则  $\lim _{n \rightarrow+\infty} a_{n}=0$.

 二 、 $(15$  分  $)$  证明 :  若函数  $f(x)$  与  $g(x)$  分别满足  $\lim _{x \rightarrow+\infty} f(x)=+\infty$  与  $\lim _{x \rightarrow+\infty} g(x)=a>o$,  则  $\lim _{x \rightarrow+\infty} f(x) g(x)=$ $+\infty$.

 三 、 ( 15  分 )  证明 :  若函数  $f(x)$  在  $[a,+\infty)$  连续且  $\lim _{x \rightarrow+\infty} f(x)=5$,  则函数  $f(x)$  在  $[a,+\infty)$  一致连续 .

 四 、( 15  分 )  证明 :  若函数  $f(x)$  在区间  $[0, a)$  可导且  $f(0)=0$,  同时导函数  $f^{\prime}(x)$  在  $[0, a)$  单调增加 ,  则函数  $\frac{f(x)}{x}$  在  $(0, a)$  也单调增加 .

 五 、 ( 15  分 )  求幂级数  $\sum_{n=0}^{\infty}(2 n+1) x^{n}$  的收敛域 ,  并求该幂级数的和函数 .

 六 、 ( 15  分 )  已知函数  $f(x)=\left\{\begin{array}{cc}x^{\alpha} \sin \left(\frac{1}{x}\right), & x \neq 0 \\ 0, & x=0\end{array}\right.$,  请讨论并回答下列问题 :

(1) $\alpha$  满足什么条件时函数  $f(x)$  在  $x=0$  处连续 ?

(2) $\alpha$  满足什么条件时函数  $f(x)$  在  $x=0$  处可导 ?

(3) $\alpha$  满足什么条件时导函数  $f^{\prime}(x)$  在  $x=0$  处连续 ?

 七 、 (15  分 )  计算空间立体  $\Omega$  的体积 ,  其中该立体  $\Omega$  是由曲面  $z=\left|x^{2}+y^{2}-4\right|$,  平面  $z=0$  及圆柱面  $x^{2}+y^{2}=16$  所围成的空间立体 .

 八 、 (15  分 )  计算曲面积分  $\oiint_{S} \frac{x \mathrm{~d} y \mathrm{~d} z+y \mathrm{~d} z \mathrm{~d} x+z \mathrm{~d} x \mathrm{~d} y}{\sqrt{x^{2}+y^{2}+z^{2}}}$,  其中  $S$  为球面  $x^{2}+y^{2}+z^{2}=9$  的外侧 .

 九 、 (15  分 )  证明 :  若函数  $f(x)$  在  $[a,+\infty)$  有连续的导函数  $f^{\prime}(x)$,  且无穷积分  $\int_{0}^{+\infty} f(x) \mathrm{d} x$  与  $\int_{0}^{+\infty} f^{\prime}(x) \mathrm{d} x$  都   收敛 ,  则  $\lim _{x \rightarrow+\infty} f(x)=0$.

 十 、 ( 15  分 )  设函数  $f(x)$  在区间  $[0,1]$  连续 ,  在区间  $(0,1)$  内二阶可导且  $M$  为  $f(x)$  在  $[0,1]$  的最大值 ,  同时  $f(0)=f(1)=0$  且对每个  $x \in(0,1)$  都有  $f^{\prime \prime}(x)<0$,  证明下述结论 :

\section{4. 北京工业大学 2013 年硕士研究生入学考试试题(数学分析663) 
 李扬 
 微信公众号: sxkyliyang}
 一 、 (15  分 )  若函数  $f(x)$  在  $(a,+\infty)$  连续 ,  且  $\lim _{x \rightarrow a^{+}} f(x)=c$  与  $\lim _{x \rightarrow+\infty} f(x)=d$.  证明  $f(x)$  在  $(a,+\infty)$  有界 .

 二 、 (15  分 )  证明 :  当  $x>0$  时 ,  有  $\ln (1+x)<\frac{x}{\sqrt{1+x}}$.

 三 、 (15  分 )  证明函数  $f(x)=\frac{x^{2}}{e^{x}}+1$  在  $[1,+\infty)$  一致连续 .

 四 、 $\left(15\right.$  分 )  已知  $f(x)$  在  $[0,1]$  连续 ,  在  $(0,1)$  内可导 ,  且  $f(0)=f(1)=0, f\left(\frac{1}{2}\right)=1$.  证明存在  $\xi \in(0,1)$,  使得  $f^{\prime}(\xi)=1 .$

 五 、 ( 15  分 )  已知  $p>0$,  当  $p$  与  $q$  满足什么关系时方程  $x^{3}=3 p x+q$  恰有三个实根 .

 六 、( 15  分 )  利用有限覆盖定理证明下述结论 :  如果  $D$  是平面  $R^{2}$  上的有界闭区域且函数  $f(x, y)$  在  $D$  连续 ,  则函   数  $f(x, y)$  在区域  $D$  有界 .

 七 、 ( 15  分 )  若级数  $\sum_{n=1}^{\infty} a_{n}$  与级数  $\sum_{n=1}^{\infty} b_{n}$  都收敛 ,  且  $a_{n} \leq c_{n} \leq b_{n}, n \in N$,  则级数  $\sum_{n=1}^{\infty} c_{n}$  也收敛 .

 八 、(15  分 )  若函数级数  $\sum_{n=1}^{\infty} u_{n}(x)$  在区间  $I$  一致收敛于和函数  $f(x)$,  且对任意  $n \in N, u_{n}(x)$  在区间  $I$  连续 ,  证   明和函数  $f(x)$  在区间  $I$  连续 .

 九 、(15  分 )  计算下列二重积分  $\iint_{D}|\cos (x+y)| \mathrm{d} x \mathrm{~d} y$,  其中  $D$  是由直线  $x+y=\pi$  与两坐标轴所成的三角形闭区   域 .

 十 、(15  分 )  设空间区域  $\Omega$  由曲面  $z=a^{2}-x^{2}-y^{2}$  与平面  $z=0$  围成 ,  其中  $\Omega$  的表面外侧为  $S, \Omega$  的体积为  $V$,  证明  $\oiint_{S} x^{2} y z^{2} \mathrm{~d} y \mathrm{~d} z-x y^{2} z^{2} \mathrm{~d} z \mathrm{~d} x+z(1+x y z) \mathrm{d} x \mathrm{~d} y=V$. 15.  北京工业大学  2014  年硕士研究生入学考试试题 ( 数学分析 663)

 李扬 

 微信公众号 : sxkyliyang

 一 、 (15  分 )  证明 :  若  $y_{n}=1+\frac{1}{2}+\frac{1}{3}+\cdots+\frac{1}{n}$,  其中  $n$  为正整数 ,  则数列  $\left\{y_{n}\right\}$  发散 .

 二 、 (15  分 )  证明 :  如果函数  $f(x)$  在区间  $(a, b)$  连续 ,  且对任意有理数  $r \in(a, b)$  有  $f(r)=0$,  则对任意  $x \in(a, b)$,  有  $f(x)=0$.

 三 、 ( 15  分 )  证明 :  如果函数  $f(x)$  在区间  $[a, b]$  上连续 ,  则函数  $f(x)$  在  $[a, b]$  上取得最小值 .

 四 、 (15  分 )  半径为  $a$  的球内有一球内接直圆柱 ,  问直圆柱的底半径与高为多大时使直圆柱的体积最大 ?

 五 、(15  分 )  证明 :  若函数  $f(x)$  在区间  $[a, b]$  上连续 ,  则积分上限函数  $\phi(x)=\int_{a}^{x} f(t) d t$  在区间  $[a, b]$  可导且  $\phi^{\prime}(x)=f(x) .$

 六 、 (15  分 )  设函数  $f(x)$  与  $g(x)$  在  $[0,+\infty)$  都连续且  $\lim _{x \rightarrow+\infty} g(x)=6$,  若无穷积分  $\int_{0}^{+\infty} f(x) \mathrm{d} x$  绝对收敛 ,  则无   穷积分  $\int_{0}^{+\infty} f(x) g(x) \mathrm{d} x$  绝对收敛 .

 七 、 ( 15  分 )  证明 :  若幂级数  $\sum_{n=0}^{\infty} a_{n} x^{a}$  的收敛半径为正数  $r$  且该幂级数  $\sum_{n=0}^{\infty} a_{n} x^{n}$  在区间  $(-r, r)$  上一致收敛 ,  则   莫级数  $\sum_{n=0}^{\infty} a_{n} x^{n}$  在区间  $[-r, r]$  上一致收敛 .

 八 、 (15  分 )  计算三重积分  $\iiint_{V} z^{2} \mathrm{~d} x \mathrm{~d} y \mathrm{~d} z$,  其中  $V$  是  $x^{2}+y^{2}+z^{2} \leq 4$  和  $x^{2}+y^{2}+(2-z)^{2} \leq 4$  的公共部分 .

 九 、( 15  分 ,  其中第一题  10  分 ,  第二题  5  分 )  设曲线  $y=\frac{1}{x^{3}}$  与直线  $y=\frac{x}{n^{4}}$  及直线  $y=\frac{x}{(n+1)^{4}}$  在第一象限围城   的面积为  $I(n)$,  其中  $n$  为正整数 .

\begin{enumerate}
  \item  求证 : $I(n)=\frac{2 n+1}{n^{2}(n+1)^{2}}$;

  \item  求级数  $\sum_{n=1}^{\infty} I(n)$  的和 .

\end{enumerate}
 十 、 (15  分 ,  其中第一题  10  分 ,  第二题  5  分 )

\begin{enumerate}
  \item  证明函数级数  $\sum_{n=1}^{\infty} \frac{1}{n^{x}}$  的和函数  $f(x)$  在  $(1,+\infty)$  上连续 .

  \item  证明函数级数  $\sum_{n=1}^{\infty} \frac{1}{n^{x}}$  在  $(1,+\infty)$  上非一致收敛 .

\end{enumerate}
\section{6. 北京工业大学 2015 年硕士研究生入学考试试题(数学分析663) 
 李扬 
 微信公众号: sxkyliyang}
 一 、 (15  分 )  设数列  $\left\{a_{n}\right\}$  满足  $a_{1}=\sqrt{2}, a_{n+1}=\sqrt{2+a_{n}}$,  证明  $\left\{a_{n}\right\}$  收玫 ,  并求其极限 .

 二 、 (15  分 )  求数列极限 : $\lim _{n \rightarrow \infty} \sum_{k=1}^{n} \frac{k}{n^{3}} \sqrt{n^{2}-k^{2}}$.

 三 、 (15  分 )  证明奇次多项式 
$$
P(x)=a_{0} x^{2 n+1}+a_{1} x^{2 n}+\cdots+a_{2 n+1}
$$
 至少存在一个实根 ,  其中  $a_{0}, a_{1}, \cdots, a_{2 n+1}$  都是常数 ,  且  $a_{0} \neq 0$.

 四 、( 15  分 )  证明 :  若函数  $f(x)$  在  $[a,+\infty)$  连续 ,  且存在  $b \in \mathbb{R}$  使得  $\lim _{x \rightarrow+\infty} f(x)=b$,  则函数  $f(x)$  在  $[a,+\infty)$  一   致连续 .

 五 、(15  分 )  求函数  $f(x)=-2 x^{3}+3 x^{2}+6 x-1$  在区间  $[-2,2]$  的最小值和最大值 .

 六 、(15  分 )  求  $\sum_{n=1}^{\infty} \frac{x^{n}}{n}$  的和函数 .

 七 、 $\left(15\right.$  分 )  求函数  $u=\left(2 a x-x^{2}\right)\left(2 b y-y^{2}\right),(a b \neq 0)$  的极值 

 八 、 $\left(15\right.$  分 )  求曲线  $\left(a_{1} x+b_{1} y+c_{1}\right)^{2}+\left(a_{2} x+b_{2} y+c_{2}\right)^{2}=1\left(a_{1} b_{2}-a_{2} b_{1} \neq 0\right)$  所围区域的面积 .

 九 、 $(15$  分  $)$  设  $F(t)=\iiint_{V} f\left(x^{2}+y^{2}+z^{2}\right) \mathrm{d} x \mathrm{~d} y \mathrm{~d} z$,  其中  $V: x^{2}+y^{2}+z^{2} \leq t^{2}, f$  是可微函数 ,  求  $F^{\prime}(t)$.

 十 、 (15  分 )  计算  $\oint_{C} x y^{2} \mathrm{~d} y-x^{2} y \mathrm{~d} x$,  其中  $C$  是圆周  $x^{2}+y^{2}=a^{2}$,  取正向 . 17.  北京工业大学  2016  年硕士研究生入学考试试题 ( 数学分析 663)

 李扬 

 微信公众号 : sxkyliyang

 一 、 (15  分 )  用函数极限的  $\epsilon-\delta$  定义证明  $\lim _{x \rightarrow c} \frac{1}{x}=\frac{1}{c}(c \neq 0)$.

 二 、 (15  分 )  计算数列极限  $\lim _{n \rightarrow \infty}\left(\frac{1}{n+1}+\frac{1}{n+2}+\cdots+\frac{1}{2 n}\right)$

 三 、 $\left(15\right.$  分 )  证明 :  若函数  $f(x)$  在  $(a,+\infty)$  连续 ,  且存在  $A, B \in \mathbb{R}$,  使得  $\lim _{x \rightarrow a^{+}} f(x)=A$  与  $\lim _{x \rightarrow+\infty} f(x)=B$,  则  $f(x)$  在  $(a,+\infty)$  有界 .

 四 、 (15  分 )  证明方程  $\frac{5}{x-1}+\frac{7}{x-2}+\frac{6}{x-3}=0$  在  $(1,2)$  与  $(2,3)$  内各有一个实根 .

 五 、( 15  分 )  求函数  $f(x)=-2 x^{3}+3 x^{2}+6 x-1$  在  $[-2,2]$  上的最小值与最大值 .

 六 、(15  分 )  证明函数列  $\left\{n x(1-x)^{n}\right\}$  在  $[0,1]$  不一致收敛 .

 七 、 (15  分 )  求  $\sum_{n=1}^{\infty} \frac{x^{2 n-1}}{2 n-1}$  的和函数 .

 八 、 $(15$  分  $)$  设  $n$  个正数  $x_{1}, x_{2}, \cdots, x_{n}$  之和为  $a$,  求函数  $u=\left(x_{1} x_{2} \cdots x_{n}\right)^{\frac{1}{n}}$  的最大值 .

 九 、 (15  分 )  证明  $\int_{0}^{+\infty} \frac{e^{-a x}-e^{-b x}}{x} \mathrm{~d} x=\ln \frac{b}{a}, \quad 0<a<b$.

 十 、 (15  分 )  计算曲面积分 
$$
\oiint_{S}\left(x^{3}-y z\right) \mathrm{d} y \mathrm{~d} z-2 x^{2} y \mathrm{~d} z \mathrm{~d} x+z \mathrm{~d} x \mathrm{~d} y
$$
 其中  $S$  是平面  $x=a, y=a, z=a(a>0)$  及三个坐标平面围成的立方体  $V$  的表面 . 18.  北京工业大学  2017  年硕士研究生入学考试试题 ( 数学分析 663)

 李扬 

 微信公众号 : sxkyliyang

 一 、(15  分 )  证明 :  若  $\lim _{n \rightarrow \infty} x_{n}=a$,  则  $\lim _{n \rightarrow \infty} \frac{x_{1}+x_{2}+\cdots+x_{n}}{n}=a$.

 二 、 ( 15  分 )  求级数  $\sum_{n=1}^{\infty} \frac{x^{n-1}}{n(n+1)}$  的收玫域及和函数 .

 三 、 已知  $\lim _{x \rightarrow a} f(x)=b$,  证明对任意数列  $\left\{a_{n}\right\}, a_{n} \neq a$,  若  $\lim _{n \rightarrow \infty} a_{n}=a$,  都有  $\lim _{n \rightarrow \infty} f\left(a_{n}\right)=b$.

 四 、(15  分 )  设  $V$  是由椭球面  $\frac{x^{2}}{a^{2}}+\frac{y^{2}}{b^{2}}+\frac{z^{2}}{c^{2}}=1$  的切平面和三个坐标平面所围成的四面体的体积 ,  求  $V$  的最小   值 .

 五 、 ( 15  分 )  证明 :  若  $f(x)$  在  $[a,+\infty)$  连续 ,  函数  $g(x)$  在  $[a,+\infty)$  一致连续 ,  且  $\lim _{x \rightarrow \infty}[g(x)-f(x)]=0$,  则  $f(x)$  在  $[a,+\infty)$  一致连续 .

 六 、 $(15$  分 )  已知 
$$
f(x, y)=\left\{\begin{array}{l}
\frac{x^{3}}{x^{2}+y^{2}},(x, y) \neq(0,0) \\
0,(x, y)=(0,0)
\end{array}\right.
$$
(1)  求出  $f_{x}^{\prime}(x, y)$  及  $f_{y}^{\prime}(x, y)$.

(2)  证明 : $f(x, y)$  在原点连续 

(3)  证明 : $f(x, y)$  在原点不可微 

 七 、 (15  分 )  证明 :  若函数  $f(x)$  在  $(a,+\infty)$  内可导且  $\lim _{x \rightarrow+\infty} f^{\prime}(x)=0$,  则  $\lim _{x \rightarrow+\infty} \frac{f(x)}{x}=0$.

 八 、 ( 15  分 )  已知三次方程  $x^{3}-3 a^{2} x-6 a^{2}+3 a=0$  只有一个正的实根 ,  求  $a$  范围 .

 九 、 (15  分 )  若函数列  $\left\{f_{n}(x)\right\}$  在闭区间  $[a, b]$  一致收敛于  $f(x)$,  而每个  $f_{n}(x)$  在闭区间  $[a, b]$  连续  $(n=1,2, \cdots)$, $x_{n} \in[a, b](n=1,2, \cdots)$  且数列  $\left\{x_{n}\right\}$  收玫于  $x_{0}$,  证明  $\lim _{n \rightarrow \infty} f_{n}\left(x_{n}\right)=f\left(x_{0}\right)$.

 十 、(15  分 )  求极限  $\lim _{t \rightarrow 0^{+}} \frac{1}{t^{4}} \iiint_{x^{2}+y^{2}+z^{2} \leq t^{2}} f\left(\sqrt{x^{2}+y^{2}+z^{2}}\right) \mathrm{d} x \mathrm{~d} y \mathrm{~d} z$,  其中函数  $f(t)$  在区间  $[0,1]$  上连续 ,  在  0  点   左右导数存在 ,  且  $f(0)=0, f_{+}^{\prime}(0)=1$.

\section{9. 北京工业大学 2018 年硕士研究生入学考试试题(数学分析663) 
 李扬 
 微信公众号: sxkyliyang}
 一 、 (30  分 )  下面各题每题  6  分 ,  在答题纸上写明题号并直接写出最后结果 ( 不需要写明过程 )

\begin{enumerate}
  \item  计算  $\iiint_{V}\left(x^{2}+y^{2}+z^{2}\right)^{\frac{5}{2}} \mathrm{~d} x \mathrm{~d} y \mathrm{~d} z$,  其中  $V$  是空间域  $x^{2}+y^{2}+z^{2} \leq 2 z$.

  \item  计算  $\oiint_{\Sigma} x\left(y^{2}+z^{2}\right) \mathrm{d} y \mathrm{~d} z$,  其中  $\Sigma$  为球面  $x^{2}+y^{2}+z^{2}=4$  且外侧为正侧 .

  \item  已知方程  $\ln \sqrt{x^{2}+y^{2}}=\arctan \frac{y}{x}$  确定了隐函数  $y=f(x)$,  求  $\frac{\mathrm{d} y}{\mathrm{~d} x}$.

  \item  设函数  $f(x)$  在点  $a$  可导且  $f(a) \neq 0$,  求极限  $\lim _{n \rightarrow+\infty}\left[\frac{f\left(a+\frac{1}{n}\right)}{f(a)}\right]^{n}$.

  \item  求曲面  $\sqrt{x}+\sqrt{y}+\sqrt{z}=$  在  $(1,1,1)$  点的切平面与坐标轴的交点到原点的距离的和 .

\end{enumerate}
 二 、 (15  分 )  已知  $f(x)$  在区间  $[a, b]$  连续且  $f(a)<0, f(b)>0$.  证明 :

(1)  集合  $A=\{x \in[a, b]: f(x)<0\}$  有上确界 .

(2)  如果  $\sup A=l$,  则  $f(l)=0$.

 三 、 $\left(15\right.$  分 )  求函数  $f(x)=\frac{x^{3}-3 x^{2}+3 x+1}{x-1}$  的极值点与拐点 .

 四 、 (15  分 )  证明 :  若函数  $f(x)$  满足如下条件 :

(1)  在闭区间  $[a, b]$  连续 ,

(2)  在开区间  $(a, b)$  可导 .

 则在开区间  $(\mathrm{a}, \mathrm{b})$  内至少存在一点  $\xi$,  使得  $f^{\prime}(\xi)=\frac{f(b)-f(a)}{b-a}$.

 五 、 $\left(20\right.$  分 )  求圆  $(x-b)^{2}+y^{2}=a^{2}(0<a<b)$  绕  $y$  轴旋转一周的旋转体的体积 .

 六 、( 20  分 )  讨论函数  $\xi(x)=\sum_{n=1}^{\infty} \frac{1}{n^{x}}$  的定义域以及它在定义域内的可微性 . $x \in[a, b]$  有  $f(x)=g(x)$.

\section{1. 北京交通大学 2009 年研究生入学考试试题高等代数 
 李扬 
 微信公众号: sxkyliyang}
 一 、 ( 本题  60  分 ,  每小题  4  分 )  填空题 .

\begin{enumerate}
  \item  设  $A$  是  $n$  阶矩阵 , $|A| \neq 0, A^{*}$  是  $A$  的伴随矩阵 .  若  $A$  有特征值  $\lambda$,  则  $\left(2 A^{*}\right)^{-1}$  必有一个特征值是 

  \item  已知向量  $\alpha=(0,1,0,1)$.  若矩阵  $E+b \alpha^{T} \alpha$  是矩阵  $E+2 \alpha^{T} \alpha$  的逆矩阵  ( 其中  $E$  是  4  阶单位矩阵 , $b$  是   实数 ),  则  $b=$

  \item  排列  $13 \cdots(2 n-1) 24 \cdots(2 n)$  的逆序数为 

  \item  设  $A^{*}$  是  $n$  阶矩阵  $A$  的伴随矩阵 , $|A|=\frac{1}{2}$,  则  $\left(2 A^{*}\right)^{*}=$

  \item  已知齐次线性方程组 

\end{enumerate}
$$
\left(\begin{array}{ccc}
1 & 2 & 1 \\
2 & 3 & a+2 \\
1 & a & -2
\end{array}\right)\left(\begin{array}{l}
x_{1} \\
x_{2} \\
x_{3}
\end{array}\right)=\left(\begin{array}{l}
1 \\
3 \\
0
\end{array}\right)
$$
 无解 ,  则  $a=$

\begin{enumerate}
  \setcounter{enumi}{6}
  \item  若二次型  $f\left(x_{1}, x_{2}, x_{3}\right)=2 x_{1}^{2}+x_{2}^{2}+x_{3}^{2}+2 x_{1} x_{2}+a x_{2} x_{3}$  是正定二次型 ,  则  $a$  的取值范围是 

  \item  设向量  $\alpha=(1, k, 1)^{T}$  是矩阵  $A=\left(\begin{array}{lll}2 & 1 & 1 \\ 1 & 2 & 1 \\ 1 & 1 & 2\end{array}\right)$  的逆矩阵  $A^{-1}$  的特征向量 ,  则常数  $k$  需要满足的条件 

  \item  实二次型  $f\left(x_{1}, x_{2}, x_{3}\right)=x_{1}^{2}+3 x_{2}^{2}+4 x_{1} x_{2}-4 x_{2} x_{3}$  的正惯性指数是 

  \item  实二次型  $f\left(x_{1}, x_{2}, x_{3}\right)=X^{T} A X$  经正交变换化为  $y_{1}^{2}+5 y_{2}^{2}$,  则  $A$  的最小的特征值是 

\end{enumerate}
10 .  设  $X$  是  $3 \times 1$  矩阵 ,  已知  $X X^{T}=\left(\begin{array}{lll}1 & 2 & 3 \\ 2 & 4 & 6 \\ 3 & 6 & 9\end{array}\right)$,  则  $X^{T} X=$

\begin{enumerate}
  \setcounter{enumi}{11}
  \item  设  $\alpha_{1}=(1,1,1,2)^{T}, \alpha_{2}=(4,6,2 a+7,10)^{T}, \alpha_{3}=(3, a+4,2 a+5, a+7)^{T}, \beta=(2,3,2 a+3,5)^{T}$,  若  $\beta$  不能用  $\alpha_{1}, \alpha_{2}, \alpha_{3}$  线性表示 ,  则  $a=$

  \item  若  $n(n>3)$  阶矩阵 

\end{enumerate}
$$
A=\left(\begin{array}{ccccc}
1 & a & a & \cdots & a \\
a & 1 & a & \cdots & a \\
a & a & 1 & \cdots & a \\
\vdots & \vdots & \vdots & & \vdots \\
a & a & a & \cdots & 1
\end{array}\right)
$$
 的秩  $\mathrm{r}(A)=n-1$,  则  $a$  必为 

\begin{enumerate}
  \setcounter{enumi}{13}
  \item  两个多项式  $f(x)=x^{4}-2 x^{3}-4 x^{2}+4 x-3, g(x)=2 x^{3}-5 x^{2}-4 x+3$  的最大公因式  $(f(x), g(x))=$

  \item  多项式  $f(x)=3 x^{4}+5 x^{3}+x^{2}+5 x-2$  的有理根是 

  \item  设  $A=\left(\begin{array}{ccc}-1 & 0 & 0 \\ 0 & -1 & 1 \\ 0 & 0 & -1\end{array}\right)$,  则  $A$  的最小多项式为 

\end{enumerate}
 二 、 ( 本题满分  15  分 )  计算极限 : $\lim _{n \rightarrow \infty}\left(\begin{array}{ccc}\frac{1}{2} & -1 & 2 \\ 0 & \frac{1}{3} & 1 \\ 0 & 0 & \frac{1}{4}\end{array}\right)^{n}$.  三 、 $($  本题满分  15  分  $)$  二次型  $f(X)=f\left(x_{1}, x_{2}, x_{3}\right)=x_{1}^{2}+x_{2}^{2}+x_{3}^{2}-4 x_{1} x_{2}-4 x_{1} x_{3}+2 a x_{2} x_{3}$  经正交变换化为标   准形  $f=3 y_{1}^{2}+3 y_{2}^{2}+b y_{3}^{2}$,

(1)  求  $a, b$  及所用的正交变换矩阵 ;

(2)  若  $X^{T} X=2$,  求  $f$  的最大值 .

 四 、( 本题满分  15  分 )  设  $t_{1}, t_{2}, \cdots, t_{n+1}$  是区间  $[0,1]$  中  $n+1$  个不同的点 ,  函数  $\varphi(t)$  满足  $\varphi\left(t_{1}\right), \varphi\left(t_{2}\right), \cdots$, $\varphi\left(t_{n+1}\right)$  不全为零 ,  问是否可以找到唯一的一个  $n$  次多项式  $f(t)=a_{0}+a_{1} t+a_{2} t^{2}+\cdots+a_{n} t^{n}$  使得  $f\left(t_{i}\right)=\varphi\left(t_{i}\right) \quad(i=1,2, \cdots, n+1) .$

 五 、( 本题满分  15  分 )  设  $A=\left(a_{i j}\right)_{3 \times 3}, A_{i j}$  的行列式  $|A|$  中元素  $a_{i j}$  的代数余子式 ,  且  $A_{i j}=a_{i j}$,  又  $a_{11} \neq 0$,  求  $|A|$.

 六 、 ( 本题满分  15  分 )  设  $\left(2 E-C^{-1} B\right) A^{T}=C^{-1}$,  其中  $E$  是  4  阶单位阵 , $A^{T}$  是  4  阶矩阵  $A$  的转置矩阵 ,
$$
B=\left(\begin{array}{cccc}
1 & 2 & -3 & -2 \\
0 & 1 & 2 & -3 \\
0 & 0 & 1 & 2 \\
0 & 0 & 0 & 1
\end{array}\right), C=\left(\begin{array}{cccc}
1 & 2 & 0 & 1 \\
0 & 1 & 2 & 0 \\
0 & 0 & 1 & 2 \\
0 & 0 & 0 & 1
\end{array}\right)
$$
 求  $A$.

 七 、 ( 本题满分  15  分 )  令  $\mathbb{R}[x]_{3}$  的内积为  $(f, g)=\int_{-1}^{1} f(x) g(x) \mathrm{d} x, \mathbb{R}[x]_{3}$  的基为  $f_{0}=1, f_{1}=x, f_{2}=x^{2}$,  试应用   施密特正交化方法求  $\mathbb{R}[x]_{3}$  的一组标准正交基 .

\section{2. 北京交通大学 2010 年研究生入学考试试题高等代数}
 李扬 

 微信公众号 : sxkyliyang

 一 、 ( 本题  60  分 ,  每小题  4  分 )  填空题 .

\begin{enumerate}
  \item  设  $n$  阶行列式 
\end{enumerate}
$$
D_{n}=\left|\begin{array}{cccccc}
1 & 2 & 3 & \cdots & n-1 & n \\
-1 & 0 & 3 & \cdots & n-1 & n \\
-1 & -2 & 0 & \cdots & n-1 & n \\
\vdots & \vdots & \vdots & & \vdots & \vdots \\
-1 & -2 & -3 & \cdots & 0 & n \\
-1 & -2 & -3 & \cdots & -(n-1) & 0
\end{array}\right|,
$$
 则其值为 

\begin{enumerate}
  \setcounter{enumi}{2}
  \item  已知  $f\left(x_{1}, x_{2}, x_{3}\right)=x_{1}^{2}+x_{2}^{2}+k x_{3}^{2}+6 x_{1} x_{2}+4 x_{1} x_{3}+2 x_{2} x_{3}$  的秩为  2 ,  则参数  $k=$

  \item  设向量组  $\alpha_{1}=(1,2,1,3), \alpha_{2}=(1,1,-1,1), \alpha_{3}=(1,3,3,5), \alpha_{4}=(4,5,-2,6), \alpha_{5}=(-3,-5,-1,-7)$,  则其秩为 

  \item  设二次型  $f\left(x_{1}, x_{2}, x_{3}\right)=x_{1}^{2}+x_{2}^{2}-2 x_{3}^{2}+2 x_{1} x_{2}$  的正惯性指数为  $p$,  负惯性指数为  $q$,  则  $p-q=$

  \item  设  $A$  为  3  阶方阵且行列式  $|E-A|=|2 E-A|=|3 E-A|=0$ ( 其中  $E$  为  3  阶单位阵 ),  则伴随矩阵  $A^{*}$  的行列式  $\left|A^{*}\right|=$

  \item  设  $\alpha=\left(\begin{array}{c}1 \\ 1 \\ -1\end{array}\right)$  是矩阵  $A=\left(\begin{array}{ccc}a & -1 & 2 \\ 5 & -3 & 3 \\ -1 & 0 & -2\end{array}\right)$  的特征向量 ,  则  $a=$

  \item  当  $t=$ -  时 ,  多项式  $f(x)=x^{3}-3 x^{2}+t x-1$  有重根 .

  \item  多项式  $f(x)=x^{3}-10 x+5$  的实根个数为 

  \item  已知矩阵  $A=\left(\begin{array}{rr}4 & -5 \\ 2 & -3\end{array}\right)$,  则  $A^{100}=$

\end{enumerate}
10 .  设  $n$  阶矩阵  $A$  及  $m$  阶矩阵  $B$  都可逆 ,  则矩阵  $\left(\begin{array}{cc}A & 0 \\ C & B\end{array}\right)$  的逆矩阵为 

\begin{enumerate}
  \setcounter{enumi}{11}
  \item  设线性空间  $\mathbb{Q}(\sqrt{2})=\{a+b \sqrt{2} \mid a, b$  为任意有理数  $\}$,  则其基和维数分别是 

  \item $\lambda$- 矩阵 

\end{enumerate}
$$
\left(\begin{array}{ccc}
1-\lambda & 2 \lambda-1 & \lambda \\
\lambda & \lambda^{2} & -\lambda \\
1+\lambda^{2} & \lambda^{3}+\lambda-1 & -\lambda^{2}
\end{array}\right)
$$
 的不变因子为 

\begin{enumerate}
  \setcounter{enumi}{13}
  \item  设  $\mathbb{R}[x]_{3}$  是次数小于  3  的所有实系数多项式组成的线性空间 , $\mathbb{R}[x]_{3}$  的线性变换  $T$  满足 :  对任意  $f(x)=a_{0}+a_{1} x+a_{2} x^{2} \in \mathbb{R}[x]_{3}, T f(x)=\left(a_{1}+a_{2}\right)+\left(a_{0}+a_{2}\right) x+\left(a_{0}+a_{1}+2 a_{2}\right) x^{2}$,  则线性变换的特   征值为 
\end{enumerate}
$$
\mathrm{I}: \alpha_{1}=1, \alpha_{2}=1+x, \alpha_{3}=1+x+x^{2} ; \mathrm{II}: \beta_{1}=1+x^{2}, \beta_{2}=x+x^{2}, \beta_{3}=1+x+x^{2}
$$

\begin{enumerate}
  \setcounter{enumi}{15}
  \item  矩阵 
\end{enumerate}
$$
\left(\begin{array}{cccccc}
-1 & 1 & & & & \\
& -1 & & & & \\
& & 0 & 1 & & \\
& & & 0 & & \\
& & & & 0 & \\
& & & & & -1
\end{array}\right)
$$
 的最小多项式为 

 二 、( 本题满分  15  分 )  问常数  $a, b$  各取何值时 ,  方程组 
$$
\left\{\begin{array}{l}
x_{1}+x_{2}+x_{3}+x_{4}=1 \\
x_{2}-x_{3}+2 x_{4}=1 \\
2 x_{1}+3 x_{2}+(a+2) x_{3}+4 x_{4}=b+3 \\
3 x_{1}+5 x_{2}+x_{3}+(a+8) x_{4}=5
\end{array}\right.
$$
 无解 ,  有唯一解 ,  或有无穷多解 ,  并在无穷多解时写出其一般解 .

 三 、( 本题  15  分 ,  第  1  小题  10  分 ,  第  2  小题  5  分 )  已知三阶矩阵  $A$  与三维向量  $x$,  使得向量组  $x, A x, A^{2} x$  线性无   关 ,  且满足  $A^{3} x=3 A x-2 A^{2} x$

(1)  记  $P=\left[x, A X, A^{2} x\right]$,  求三阶矩阵  $B$,  使得  $A=P B P^{-1}$;

(2)  计算行列式  $|A+E|$,  其中  $E$  为  3  阶单位阵 .

 四 、( 本题满分  15  分 )  设  $T$  是  $n$  维线性空间  $V$  的线性变换 ,  证明 : $T$  的秩  $+T$  的零度  $=n$.

 五 、( 本题满分  15  分 )  设  $\varphi$  是欧式空间  $V$  的线性变换 ,  证明 : $\varphi$  是正交变换的充分必要条件是对于任意  $\alpha \in V,|\varphi(\alpha)|=|\alpha|$.

 六 、( 本题  15  分 ,  第  1  小题  7  分 ,  第  2  小题  8  分 )  设  $A, B$  都是实对称矩阵 ,  且  $A$  为正定矩阵 , $B$  是非零半正定矩   阵 ,  证明 :

(1) $A+B$  是正定矩阵 ;

(2) $|A+B|>|A|$.

 七 、 ( 本题满分  15  分 )  证明 : $\lambda$  一矩阵  $A(\lambda)$  可逆的充要条件是其行列式  $|A(\lambda)|$  是一个非零常数 .

\section{3. 北京交通大学 2011 年研究生入学考试试题高等代数 
 李扬 
 微信公众号: sxkyliyang}
 一 、 ( 本题  48  分 ,  每小题  4  分 )  填空题 .

\begin{enumerate}
  \item  设  $W=\left\{\left(a_{i j}\right) \in P^{4 \times 4} \mid a_{31}+a_{32}+a_{33}+a_{34}=0, a_{41}+a_{42}+a_{43}+a_{44}=0\right\}$,  则  $W$  是  $P^{4 \times 4}$  的子空间 , $\operatorname{dim}(W)=$

  \item  设  $n$  阶行列式  $D=\left|a_{i j}\right|=d$.  则行列式 

\end{enumerate}
$$
D_{1}=\left|\begin{array}{cccc}
2 a_{21} & 2 a_{22} & \cdots & 2 a_{2 n} \\
3 a_{31} & 3 a_{32} & \cdots & 3 a_{3 n} \\
\vdots & \vdots & & \vdots \\
n a_{n 1} & n a_{n 2} & \cdots & n a_{n n} \\
a_{11} & a_{12} & \cdots & a_{1 n}
\end{array}\right|=
$$

\begin{enumerate}
  \setcounter{enumi}{3}
  \item  设 
\end{enumerate}
$$
A=\left(\begin{array}{lll}
1 & 1 & 0 \\
0 & 1 & 1 \\
0 & 0 & 1
\end{array}\right)
$$
 则  $A^{n}=$

\begin{enumerate}
  \setcounter{enumi}{4}
  \item  设  $A$  为主对角线上元素为  $1,-2,1$  的三阶对角方阵 , $B$  为三阶方阵且  $A^{*} B A=2 B A-8 E\left(A^{*}\right.$  为  $A$  的   伴随矩阵 , $E$  为单位矩阵 ),  则  $B=$

  \item  设  $f(x)=2 x^{2}-3, g(x)=8 x^{4}-6 x^{2}+4 x-7$,  则  $f^{3}(x) g(x)$  的所有系数之和为 

  \item  设多项式  $f(x)$  被  $x-1, x-2, x-3$  除所得余数依次为  4,8 , 16,  则  $f(x)$  被  $(x-1)(x-2)(x-3)$  除所   得余式为 

  \item  实二次型  $f\left(x_{1}, x_{2}, x_{3}\right)=x_{2}^{2}+2 x_{1} x_{2}+4 x_{1} x_{3}+2 x_{2} x_{3}$  的符号差为 

  \item $t$  取何值时  4  元实二次型 

\end{enumerate}
$$
f=x_{1}^{2}+x_{2}^{2}+x_{3}^{2}+9 x_{4}^{2}+2 t\left(x_{1} x_{2}+x_{1} x_{3}+x_{2} x_{3}\right)
$$
 为正定的 ?

\begin{enumerate}
  \setcounter{enumi}{9}
  \item  设  $-2,3,-1$  是三阶方阵  $A$  的特征值 ,  则  $\left|A^{3}-6 A+11 E\right|=$ ( $E$  为单位矩阵 )

  \item $\lambda$- 矩阵 

\end{enumerate}
$$
\left(\begin{array}{ccc}
\lambda^{2}-1 & 0 & 0 \\
0 & \lambda & 0 \\
0 & 0 & \lambda(\lambda+1)^{2}
\end{array}\right)
$$
 的标准形为 

\begin{enumerate}
  \setcounter{enumi}{11}
  \item  在欧式空间  $\mathbb{R}^{4}$  中 ,  向量  $\alpha=(1,2,2,3), \beta=(3,1,5,1)$  的夹角  $\langle\alpha, \beta\rangle=$
\end{enumerate}
12 .  设  $A^{*}, B^{*}$  是  $n$  阶可逆矩阵  $A, B$  的伴随矩阵 ,  则  $(A B)^{-1}=B^{*} A^{*}$.

 二 、 ( 本题满分  12  分 )  设  $f_{i j}(x)$  为实连续函数  ( 简记为  $\left.f_{i j}\right), D(x)=\left|f_{i j}\right|$  为  $n$  阶行列式 ,  证明 : $D(x)$  的导数 
$$
D^{\prime}(x)=\sum_{j=1}^{n}\left|\begin{array}{ccccc}
f_{11} & \cdots & f_{1 j}^{\prime} & \cdots & f_{1 n} \\
\vdots & & \vdots & & \vdots \\
f_{n 1} & \cdots & f_{n j}^{\prime} & \cdots & f_{n n}
\end{array}\right|,
$$
 这里  $f_{i j}^{\prime}$  是  $f_{i j}$  的导数 .  三 、 ( 本题满分  15  分 )  求以下向量组的秩 :
$$
\alpha_{1}=(1,-1,2,1,0)^{\prime}, \alpha_{2}=(2,-2,4,-2,0)^{\prime}, \alpha_{3}=(3,0,6,-1,1)^{\prime}, \alpha_{4}=(0,3,0,0,1)^{\prime}
$$
 再求一个极大无关组 ,  并把其余向量用极大无关组表示出来 .

 四 、( 本题满分  15  分 ) $a, b$  取何值时以下方程组有解 ?  并求其解 .
$$
\left\{\begin{array}{l}
x_{1}+x_{2}+x_{3}+x_{4}+x_{5}=1 \\
3 x_{1}+2 x_{2}+x_{3}+x_{4}-3 x_{5}=a \\
x_{2}+2 x_{3}+2 x_{4}+6 x_{5}=3 \\
5 x_{1}+4 x_{2}+3 x_{3}+3 x_{4}-x_{5}=b
\end{array}\right.
$$
 五 、( 本题满分  15  分 )  求以下二次型的矩阵 :
$$
f\left(x_{1}, \cdots, x_{n}\right)=\sum_{i=1}^{m}\left(a_{i 1} x_{1}+a_{i 2} x_{2}+\cdots+a_{i n} x_{n}\right)^{2}
$$
 六 、 ( 本题满分  15  分 )  设 
$$
A=\left(\begin{array}{ccc}
a & -2 & 0 \\
b & 1 & -2 \\
c & -2 & 0
\end{array}\right), B=\left(\begin{array}{lll}
2 & 1 & 1 \\
1 & 2 & 1 \\
1 & 1 & 2
\end{array}\right)
$$
(1)  若  $A$  有特征值  $4,1,-2$,  求  $a, b, c$.

(2)  设  $\alpha=(1, k, 1)^{\prime}$  是  $B^{-1}$  的一个特征向量 ,  求  $k$.

 七 、 ( 本题满分  15  分 )  设  $A, B$  为同阶正交方阵 ,  证明 :  若  $|A|+|B|=0$,  则  $|A+B|=0$.

 八 、( 本题满分  15  分 )  设  $A$  为  $n$  阶实方阵 ,  证明 : $A$  为半正定的充要条件是 ,  对任意正实数  $c$,  方阵  $c E+A$  都是正   定的  ( $E$  为单位矩阵 ).

\section{4. 北京交通大学 2012 年研究生入学考试试题高等代数 
 李扬 
 微信公众号: sxkyliyang}
 一 、 ( 本题  40  分 ,  每小题  4  分 )  填空题 .

\begin{enumerate}
  \item  已知  $\alpha=(1,2,1)^{T}, \beta=(1,1,0)^{T}, A=\alpha \beta^{T}$.  若  $A X+X=A^{T}+A^{*} X$, ( 其中  $A^{*}$  是  $A$  的伴随矩阵 , $A^{T}$  是  $A$  的转置矩阵 ).  则  $X=$

  \item  已知实二次型 

\end{enumerate}
$$
f\left(x_{1}, x_{2}, x_{3}\right)=x_{1}^{2}+t x_{2}^{2}+t x_{3}^{2}+2 x_{1} x_{2}+2 x_{1} x_{3}-2 x_{2} x_{3}
$$
 是正定的 ,  则常数  $t$  的取值范围是 

\begin{enumerate}
  \setcounter{enumi}{3}
  \item  若向量组  $\alpha_{1}=(1,0,1), \alpha_{2}=(a,-1,0), \alpha_{3}=(-1, a,-1)$  的秩是  2 ,  则  $a=$

  \item  设  $n$  阶矩阵 

\end{enumerate}
$$
X=\left(\begin{array}{cccccc}
0 & a_{1} & 0 & \cdots & 0 & 0 \\
0 & 0 & a_{2} & \cdots & 0 & 0 \\
\vdots & \vdots & \vdots & & \vdots & \vdots \\
0 & 0 & 0 & \cdots & 0 & a_{n-1} \\
a_{n} & 0 & 0 & \cdots & 0 & 0
\end{array}\right),\left(a_{i} \neq 0, i=1,2, \cdots, n\right)
$$
 则  $X^{-1}=$

\begin{enumerate}
  \setcounter{enumi}{5}
  \item  设  3  阶方阵  $A$  的特征值为  $-1,1,2, A^{*}$  是  $A$  的伴随矩阵 ,  则矩阵  $\left(3 A^{*}\right)^{-1}$  的特征值为 

  \item  设  $f(x)=x^{4}+x^{3}-3 x^{2}-4 x-1, g(x)=x^{3}+x^{2}-2-1$,  则它们的最大公因式为 

  \item  多项式  $x^{3}-6 x^{2}+15 x-14$  的有理根为 

  \item 5  阶行列式 

\end{enumerate}
$$
D_{5}=\left|\begin{array}{ccccc}
2 & -1 & -1 & -1 & -1 \\
1 & 2 & -1 & -1 & -1 \\
1 & 1 & 2 & -1 & -1 \\
1 & 1 & 1 & 2 & -1 \\
1 & 1 & 1 & 1 & 2
\end{array}\right|
$$
 的值为 

\begin{enumerate}
  \setcounter{enumi}{9}
  \item $n$  阶行列式 
\end{enumerate}
$$
\left|\begin{array}{cccccc}
x+y & x y & 0 & \cdots & 0 & 0 \\
1 & x+y & x y & \cdots & 0 & 0 \\
0 & 1 & x+y & \cdots & 0 & 0 \\
\vdots & \vdots & \vdots & & \vdots & \vdots \\
0 & 0 & 0 & \cdots & 1 & x+y
\end{array}\right|
$$
 的值为 

10 .  设  $\lambda$- 矩阵 
$$
A(\lambda)=\left(\begin{array}{ccc}
\lambda(\lambda+1) & 0 & 0 \\
0 & \lambda & 0 \\
0 & 0 & (\lambda+1)^{2}
\end{array}\right)
$$
 则  $A(\lambda)$  的标准形为   二 、 ( 本题满分  15  分 )  已知 
$$
\alpha_{1}=\left(\begin{array}{l}
1 \\
0 \\
3
\end{array}\right), \alpha_{2}=\left(\begin{array}{c}
1 \\
-1 \\
a
\end{array}\right), \alpha_{3}=\left(\begin{array}{c}
2 \\
a+1 \\
1
\end{array}\right) \beta=\left(\begin{array}{c}
1 \\
1 \\
b+2
\end{array}\right)
$$
(1) $a, b$  为何值时 , $\beta$  不能被  $\alpha_{1}, \alpha_{2}, \alpha_{3}$  线性表出 ?

(2) $a, b$  为何值时 , $\beta$  可由  $\alpha_{1}, \alpha_{2}, \alpha_{3}$  线性表出且表法不唯一 ;  此时写出其一般表达式 .

 三 、 ( 本题满分  15  分 )  已知三阶实对称矩阵  $A$  有特征值  0 ( 二重 )  和  2 .

 若  $\alpha_{1}=\left(\begin{array}{c}1 \\ 2 \\ 1\end{array}\right), \alpha_{2}=\left(\begin{array}{l}2 \\ 1 \\ 2\end{array}\right)$  是  $A$  的属于特征值  0  的特征向量 .

(1)  求正交矩阵  $P$,  使  $P^{-1} A P$  为对角形 ;

(2)  求矩阵  $A$.

 四 、( 本题满分  15  分 )  设  $A$  是  $n$  阶实对称矩阵 ,  对任意非零的  $n$  维列向量  $X \in \mathbb{R}^{n}$,  定义实数 
$$
\mathbb{R}(X)=\frac{X^{T} A X}{X^{T} X}
$$
 其中  $X^{T}$  是  $X$  的转置 .

 证明 : $\min \left\{\lambda_{1}, \lambda_{2}, \cdots, \lambda_{n}\right\} \leqslant \mathbb{R}(X) \leqslant \max \left\{\lambda_{1}, \lambda_{2}, \cdots, \lambda_{n}\right\}$,  其中  $\lambda_{1}, \lambda_{2}, \cdots, \lambda_{n}$  为矩阵  $A$  的  $n$  个特征值 .

 五 、( 本题满分  15  分 )  用正交线性替换化下面二次型为标准型 :
$$
x_{1}^{2}+2 x_{2}^{2}+3 x_{3}^{2}-4 x_{1} x_{2}-4 x_{2} x_{3}
$$
 六 、 ( 本题满分  15  分 )  设向量组  $\beta_{1}, \beta_{2}, \cdots, \beta_{m}$  线性无关 ,  且 
$$
\xi_{i}=a_{1 i} \beta_{1}+a_{2 i} \beta_{2}+\cdots+a_{m i} \beta_{m},(i=1,2, \cdots, s)
$$
 证明 :  向量组  $\xi_{1}, \xi_{2}, \cdots, \xi_{s}$  的秩 = 矩阵  $\left(a_{i j}\right)_{m \times s}$  的秩 .

 七 、 ( 本题满分  15  分 )  设  $\varepsilon_{1}, \varepsilon_{2}, \varepsilon_{3}, \varepsilon_{4}, \varepsilon_{5}$  是欧式空间  $V$  的一组标准正交基 , $V_{1}=L\left(\alpha_{1}, \alpha_{2}, \alpha_{3}\right)$,  其中  $\alpha_{1}=\varepsilon_{1}+\varepsilon_{5}$, $\alpha_{2}=\varepsilon_{1}-\varepsilon_{2}+\varepsilon_{4}, \alpha_{3}=2 \varepsilon_{1}+\varepsilon_{2}+\varepsilon_{3}$,  求  $V_{1}$  的一组标准正交基 .

 八 、 ( 本题满分  20  分 )  设  $T$  是欧式空间  $V$  的一个线性变换 ,  如果对任意的  $\alpha, \beta \in V$  都有  $(T(\alpha), \beta)=(\alpha, T(\beta))$,  则称  $T$  为  $V$  的一个对称变换 .  证明 : 欧式空间  $V$  的一个线性变换  $T$  是对称变换的充要条件是  $T$  在  $V$  的任   意一组标准正交基下的矩阵是对称矩阵 .

\section{5. 北京交通大学 2013 年研究生入学考试试题高等代数 
 李扬 
 微信公众号: sxkyliyang}
 一 、  填空题 ( 本大题共  10  小题 ,  每小题  4  分 ,  共  40  分 )

\begin{enumerate}
  \item  在欧式空间  $\mathbb{R}^{4}$  中 ,  向量  $\alpha=(1,1,1,2), \beta=(3,1,-1,0)$  的夹角  $\langle\alpha, \beta\rangle$  为 

  \item  设  $A$  为  $n$  阶方阵且  $|A|=a$,  又  $a b \neq 0$.  则  $\left|(b A)^{-1}-c A^{*}\right|=$

  \item  设  $A$  为主对角线上元素为  $1,-2,1$  的三阶对角方阵 , $B$  为三阶方阵且  $A^{*} B A=2 B A-8 E$,  则  $B=$

  \item  设  $f\left(x_{1}, x_{2}, x_{3}\right)=10 x_{1}^{2}+2 x_{2}^{2}+2 a x_{3}^{2}+8 x_{1} x_{2}-4 x_{1} x_{3}-4 x_{2} x_{3}$  正定 ,  则  $a$  满足的条件是 

  \item  设  $\lambda_{1}, \lambda_{2}, \lambda_{3}$  为三阶方阵  $A$  的全部特征值 ,  且有相应的特征向量依次为  $(1,1,1)^{\prime},(0,1,1)^{\prime},(0,0,1)^{\prime}$,  则  $A^{n}=$

  \item $\lambda$  一矩阵 

\end{enumerate}
$$
\left(\begin{array}{ccc}
\lambda^{2}+1 & \lambda^{2} & -\lambda^{2} \\
1 & \lambda^{2}+\lambda & 0 \\
\lambda & \lambda & -\lambda
\end{array}\right)
$$
 的标准形为 

\begin{enumerate}
  \setcounter{enumi}{7}
  \item  设数域  $P$  上次数小于  $n$  的所有多项式构成的线性空间为  $P[x]_{n}$,  若  $f(x) \in P[x]_{n}$.  则  $f(x)$  在  $P[x]_{n}$  的一   组基为  $1, x-a,(x-a)^{2}, \cdots,(x-a)^{n-1}$  下的坐标为 

  \item  计算  $n$  阶行列式 

\end{enumerate}
$$
D_{n}=\left|\begin{array}{cccccc}
2 & 1 & 0 & \cdots & 0 & 0 \\
1 & 2 & 1 & \cdots & 0 & 0 \\
\vdots & \vdots & \vdots & & \vdots & \vdots \\
0 & 0 & 0 & \cdots & 2 & 1 \\
0 & 0 & 0 & \cdots & 1 & 2
\end{array}\right|=
$$

\begin{enumerate}
  \setcounter{enumi}{9}
  \item  设  $A=\left(\begin{array}{lll}0 & 0 & 1 \\ x & 1 & y \\ 1 & 0 & 0\end{array}\right)$  有三个线性无关的特征向量 ,  则  $x$  和  $y$  应满足的条件为 

  \item  多项式  $x^{3}+p x+q$  有重根的条件是 

\end{enumerate}
 二 、 ( 本题满分  10  分 )  设  $A, B$  均为  $n$  阶可逆方阵 .  证明 :  若  $A+B$  可逆 ,  则  $A^{-1}+B^{-1}$  也可逆 ,  并求其逆   方阵 .

 三 、 ( 本题满分  10  分 ) $\lambda$  取何值时 
$$
\left\{\begin{array}{l}
\lambda x_{1}+x_{2}+x_{3}=\lambda-3 \\
x_{1}+\lambda x_{2}+x_{3}=-2 \\
x_{1}+x_{2}+\lambda x_{3}=-2
\end{array}\right.
$$
 有解 ?  并求其解 .

 四 、 ( 本题满分  10  分 )  设  $A$  为  $m \times n$  实矩阵 .  证明 : $A^{\prime} A$  的秩  $=A$  的秩 .

 五 、( 本题满分  10  分 )  证明 : $n$  阶方阵  $A$  为数量矩阵 ,  当且仅当  $\lambda E-A$  的  $n-1$  阶行列式因子的次数为  $n-1$.

 六 、( 本题满分  10  分 )  证明 :  若  $\lambda_{0}$  是正交方阵  $A$  的特征根 ,  则  $\lambda_{0}^{-1}$  也是  $A$  的特征根 .  七 、 ( 本题满分  15  分 )  在欧式空间  $\mathbb{R}^{4}$  中 ,  设  $W$  为 
$$
\left\{\begin{array}{l}
2 x_{1}+x_{2}+3 x_{3}-x_{4}=0 \\
3 x_{1}+2 x_{2}-2 x_{4}=0 \\
3 x_{1}+x_{2}+9 x_{3}-x_{4}=0
\end{array}\right.
$$
 的解空间 ,  求  $W^{\perp}=$ ?

 八 、 ( 本题满分  15  分 )  设  $V$  是有理数域  $\mathbb{Q}$  上的三维空间 , $V$  的线性变换  $T$  在基  $\alpha_{1}, \alpha_{2}, \alpha_{3}$  下的矩阵为 
$$
A=\left(\begin{array}{ccc}
5 & 6 & -3 \\
-1 & 0 & 1 \\
1 & 2 & -1
\end{array}\right)
$$
 求  $T$  的特征值和相应的特征向量 ;  又问 : $A$  可否对角化 ?

 九 、 ( 本题满分  15  分 )  用正交线性替换化  $f=2 x_{1} x_{2}+2 x_{3} x_{4}$  为标准形 .

 十 、 ( 本题满分  15  分 )  设  $A, B$  分别为  $n$  阶实对称和正定矩阵 .  证明 :  乘积  $A B$  的特征根全为实数 .

\section{6. 北京交通大学 2014 年研究生入学考试试题高等代数 
 李扬 
 微信公众号: sxkyliyang}
 一 、( 本题满分  10  分 )  求  $n$  阶行列式 
$$
D_{n}=\left|\begin{array}{cccccc}
3 & 2 & 0 & \cdots & 0 & 0 \\
1 & 3 & 2 & \cdots & 0 & 0 \\
\vdots & \vdots & \vdots & & \vdots & \vdots \\
0 & 0 & 0 & \cdots & 3 & 2 \\
0 & 0 & 0 & \cdots & 1 & 3
\end{array}\right| .
$$
 的值 .

 二 、 $($  本题满分  10  分  $)$  求两个多项式  $f(x)=2 x^{4}-5 x^{3}+6 x^{2}-5 x+2, g(x)=3 x^{3}-8 x^{2}+7 x-2$  的最大公因式 .

 三 、 ( 本题满分  10  分 )  设  $A$  是  $n$  阶方阵 ,  证明 : 存在矩阵  $B, C$  使得  $A=B C$,  其中  $B$  可逆且  $C=C^{2}$.

 四 、( 本题满分  15  分 )  设  $V_{1}$  与  $V_{2}$  是有限维线性空间  $V$  的两个子空间 ,  证明 :
$$
\operatorname{dim} V_{1}+\operatorname{dim} V_{2}=\operatorname{dim}\left(V_{1}+V_{2}\right)+\operatorname{dim}\left(V_{1} \cap V_{2}\right)
$$
 五 、( 本题满分  15  分 )  当  $a, b$  为何值时 ,  线性方程组 
$$
\left\{\begin{array}{l}
x_{1}+x_{2}+x_{3}+x_{4}=0 \\
x_{2}+2 x_{3}+2 x_{4}=1 \\
-x_{2}+(a-3) x_{3}-2 x_{4}=b \\
3 x_{1}+2 x_{2}+x_{3}+a x_{4}=-1
\end{array}\right.
$$
 有唯一解 ,  无解 ,  有无穷多组解 ,  并求出有无穷多组解时的通解 .

 六 、( 本题满分  15  分 )  设  $V$  为欧式空间 ,  记  $\|\star\|$  为向量的长度 ,  证明 :  对任意向量  $\alpha, \beta \in V$,
$$
|(\alpha, \beta)| \leqslant\|\alpha\| \cdot\|\beta\|,
$$
 而且 ,  当且仅当  $\alpha, \beta$  线性相关时 ,  等号才成立 .

 七 、 ( 本题满分  15  分 )  若二次型 
$$
f=x_{1}^{2}+x_{2}^{2}+x_{3}^{2}+2 x_{1} x_{2}+2 \alpha x_{1} x_{3}+2 \beta x_{2} x_{3}
$$
 八 、( 本题满分  20  分 )  设  $T$  是实向量空间  $V$  上的线性变换 ,  且满足  $T^{2}=I$,  这里  $I$  表示  $V$  上的恒等变换 .  定义两 
$$
V_{1}=\{v \in V: T(v)=v\} ; V_{2}=\{v \in V: T(v)=-v\},
$$
$$
A=\left(\begin{array}{ccc}
-1 & -2 & 6 \\
-1 & 0 & 3 \\
-1 & -1 & 4
\end{array}\right)
$$
 十 、 ( 本题满分  20  分 )  设向量空间  $\mathbb{R}^{2}$  按照某种  ( 不一定是通常的 )  内积方式构成欧式空间 ,  记为  $V^{2}$.  已知  $V^{2}$  的   两组基为 :
$$
\text { (I) } \alpha_{1}=(1,1), \alpha_{2}=(1,-1) ; \text { (II) } \beta_{1}=(0,2), \beta_{2}=(6,12) \text {. }
$$
 且  $\alpha_{i}$  和  $\beta_{i}$  的内积为 
$$
\left(\alpha_{1}, \beta_{1}\right)=1,\left(\alpha_{1}, \beta_{2}\right)=15,\left(\alpha_{2}, \beta_{1}\right)=-1,\left(\alpha_{2}, \beta_{2}\right)=3
$$
(1)  求基  (I)  的度量矩阵  $A$;

(2)  求基  (II)  的度量矩阵  $B$;

(3) 求欧式空间  $V^{2}$  的一个标准正交基 .

\section{7. 北京交通大学 2015 年研究生入学考试试题高等代数 
 李扬 
 微信公众号: sxkyliyang}
 一 、 选择题 ( 本大题共  10  小题 ,  每小题  3  分 ,  共  30  分 )

\begin{enumerate}
  \item  设  $f(x), g(x), h(x) \in P[x]$.  若  $(f(x), g(x))=1,(f(x), h(x))=1$,  则下列叙述正确的有   个 .\\
(1) $(f(x), g(x)+h(x))=1$;\\
(2) $\left(f^{2}(x), g(x) h(x)\right)=1$;\\
(3) $(f(x),(g(x), h(x)))=1$;\\
(4) $\left(f(x) g(x), h^{2}(x)\right)=1$.\\
(A) 1\\
(B) 2\\
(C) 3\\
(D) 4

  \item  已知  $n$  阶矩阵  $A$  合同于  $B=\left(\begin{array}{cccc}\lambda_{1} & & & \\ & \lambda_{2} & & \\ & & \ddots & \\ & & & \lambda_{n}\end{array}\right)$,  则必有  .\\
(A) $\lambda_{1}, \lambda_{2}, \cdots, \lambda_{n}$  是  $A$  的特征值 ;\\
(B) $\lambda_{1} \lambda_{2} \cdots \lambda_{n}=|A|$;\\
(C) $A$  为正定阵 ;\\
(D) $A$  为对称阵 .

  \item  下列结论中正确的是 \\
(A)  特征矩阵  $\lambda E_{n}-A$  的秩一定等于  $n$;\\
$(\mathrm{B})$  若  $A(\lambda)=\left(\begin{array}{ccc}1 & & \\ & \lambda & \\ & & \lambda^{2}\end{array}\right)$,  则  $A(\lambda)$  的不变因子为  $\lambda, \lambda^{2} ;$\\
(C)  设  $A, B \in P^{n \times n}$,  若  $A$  与  $B$  等价 ,  则它们有相同的行列式因子组 ;\\
(D)  若两个同阶的  $\lambda$- 矩阵有相同的秩 ,  则它们一定等价 .\\
(B)  必为对角阵 ;\\
(C)  必为上三角阵 ,  但末必是对角阵 ;\\
(D)  必为下三角阵 ,  但末必是对角阵 .  件是 \\
(C)  向量组  $\alpha_{1}, \alpha_{2}, \cdots, \alpha_{m}$  与向量组  $\beta_{1}, \beta_{2}, \cdots, \beta_{m}$  等价 ;\\
(D)  矩阵  $A=\left(\alpha_{1}, \alpha_{2}, \cdots, \alpha_{m}\right)$  与矩阵  $B=\left(\beta_{1}, \beta_{2}, \cdots, \beta_{m}\right)$  的秩相等 .

  \item  下列子集能构成  $\mathbb{R}^{2 \times 2}$  的子空间的是 \\
(B) $V_{2}=\left\{A \mid \operatorname{tr}(A)=0, A \in \mathbb{R}^{2 \times 2}\right\}$;\\
(D) $V_{4}=\left\{A \mid A^{\prime}=A\right.$  或  $\left.-A, A \in \mathbb{R}^{2 \times 2}\right\}$.\\
(A)  若  $A^{2}=A, B^{2}=B$,  且  $A, B$  的秩相同 ,  则  $A$  与  $B$  相似 \\
(B)  若  $A, B$  为对称阵 ,  且  $A$  与  $B$  合同 ,  则  $A$  与  $B$  相似 ; 8.  设  $A, B$  均为实对称矩阵 ,  则  $A, B$  在  $\mathbb{R}$  上合同的充要条件是 \\
(A) $A, B$  的秩相等 ;\\
(B) $A, B$  都合同于对角阵 ;\\
(C) $A, B$  的特征值相同 ;\\
(D) $A, B$  的正负惯性指数相同 .

  \item  设  $\alpha_{1}, \alpha_{2}, \cdots, \alpha_{n}, \beta, \gamma$  是数域  $P$  上线性空间  $V$  中的向量 ,

\end{enumerate}
 秩  $\left(\alpha_{1}, \alpha_{2}, \cdots, \alpha_{n}, \beta\right)=$  秩  $\left(\alpha_{1}, \alpha_{2}, \cdots, \alpha_{n}\right)=r$  且秩  $\left(\alpha_{1}, \alpha_{2}, \cdots, \alpha_{n}, \gamma\right)=r+1$,  则对任意  $k \in P$,  秩  $\left(\alpha_{1}, \alpha_{2}, \cdots, \alpha_{n}, \beta, \gamma+k \beta\right)=$\\
(A) $r$;\\
(B) $r+1$;\\
(C) $r+2$;\\
(D)  无法确定 .

\begin{enumerate}
  \setcounter{enumi}{10}
  \item  在以下的变换  $T$  中 ,  有 \\
(1)  设  $\alpha \neq 0$  为线性空间  $V$  中某固定向量 , $T x=x+\alpha$ ( 对任意  $x \in V$ )\\
(2)  在线性空间  $P[x]$  中 , $T f(x)=f(x+1)$ ( 对任意  $f(x) \in P[x])$;\\
(3)  设  $A, B$  为  $n$  阶固定方阵 , $T X=A X B$ ( 对任意  $\left.X \in P^{n \times n}\right)$;\\
(4)  设  $A$  为  $n$  阶固定方阵 , $T X=A X-X A$ ( 对任意  $X \in P^{n \times n}$ ).\\
(A) 1\\
(B) 2\\
(C) 3\\
(D) 4
\end{enumerate}
 二 、 填空题 ( 本大题共  10  小题 ,  每小题  3  分 ,  共  30  分 )

\begin{enumerate}
  \item  若  $A$  是十阶非零矩阵且  $A^{2}=0$,  则  $A$  的  Jordan  标准型中  Jordan  块的最大阶数为 
\end{enumerate}
$$
f\left(x_{1}, x_{2}, x_{3}\right)=a\left(x_{1}^{2}+x_{2}^{2}+x_{3}^{2}\right)+4 x_{1} x_{2}+4 x_{1} x_{3}+4 x_{2} x_{3},
$$
 经正交变换  $X=P Y$  可化为标准型  $f=6 y_{1}^{2}$,  则  $a=$

\begin{enumerate}
  \setcounter{enumi}{3}
  \item  设 
\end{enumerate}
$$
f=\sum_{i=1}^{m}\left(a_{i 1} x_{1}+a_{i 2} x_{2}+\cdots+a_{i n} x_{n}\right)^{2},
$$
 则  $f$  的正惯性指数为   标准型是  $\|u\|=$  类 .

9 .  设 
$$
A=\left(\begin{array}{ll}
1 & 0 \\
3 & 1
\end{array}\right), W=\left\{B \mid B A=A B, B \in \mathbb{R}^{2 \times 2}\right\}
$$
 则  \_  是  $W$  的一组基 . 1.  设  $\alpha_{1}, \alpha_{2}, \alpha_{3}$  是  3  维欧式空间  $V$  的一组基 ,  这组基的度量矩阵是  $A=\left(\begin{array}{ccc}1 & -1 & 1 \\ -1 & 2 & 0 \\ 1 & 0\end{array}\right)$,  求  $V$  的一组   标准正交基 .

\begin{enumerate}
  \setcounter{enumi}{2}
  \item  求  3  阶实对称矩阵  $A$,  使得  $A$  的特征值为  $3,1,1$,  且  $(1,1,0)^{\prime}$  是  $A$  属于  3  的特征向量 .

  \item  设  3  阶矩阵  $A=\left(a_{i j}\right)_{3 \times 3}$  满足  $A^{*}=A^{\prime}$,  其中  $A^{*}, A^{\prime}$  分别表示  $A$  的伴随矩阵和转置 ,  且  $a_{11}=a_{12}=$ $a_{13}>0$.\\
(1)  求  $A$  的行列式 .\\
(2)  求  $a_{11}$  的值 .

  \item  计算  $n-1$  阶行列式 

\end{enumerate}
$$
\left|\begin{array}{ccccc}
2^{2}-2 & 2^{3}-2 & \cdots & 2^{n-1}-2 & 2^{n}-2 \\
3^{2}-3 & 3^{3}-3 & \cdots & 3^{n-1}-3 & 3^{n}-3 \\
\vdots & \vdots & & \vdots & \vdots \\
n^{2}-n & n^{3}-n & \cdots & n^{n-1}-n & n^{n}-n
\end{array}\right|
$$

\begin{enumerate}
  \setcounter{enumi}{5}
  \item  设列向量 
\end{enumerate}
$$
\begin{aligned}
&\alpha_{1}=(1,3,0,5)^{\prime}, \alpha_{2}=(1,2,1,4)^{\prime}, \alpha_{3}=(1,1,2,3)^{\prime}, \alpha_{4}=(1,1,2,3)^{\prime}, \alpha_{5}=(1,-3,6,-1)^{\prime}, \beta=(1, a, 3, b)^{\prime} . \\
&\text { 令 } A=\left(\alpha_{1}, \alpha_{2}, \alpha_{3}, \alpha_{4}, \alpha_{5}\right) \text {, 问 } a, b \text { 为何值时, 线性方程组 } A x=\beta \text { 有解? 在有解的情形时, 求其全部解. }
\end{aligned}
$$
 四 、 证明题 ( 本大题共  3  题 ,  每小题  10  分 ,  共  30  分 )

\begin{enumerate}
  \item  设  $A=\left(a_{i j}\right)_{n \times n}$  是  $n$  阶实对称方阵 ,  秩  $(A)=n$.  作实二次型 
\end{enumerate}
$$
f\left(x_{1}, x_{2}, \cdots, x_{n}\right)=\sum_{i=1}^{n} \sum_{j=1}^{n} a_{i j} x_{i} x_{j}, g\left(x_{1}, x_{2}, \cdots, x_{n}\right)=\sum_{i=1}^{n} \sum_{j=1}^{n} \frac{A_{i j}}{|A|} x_{i} x_{j},
$$
 其中  $A_{i j}$  是  $a_{i j}$  的代数余子式 .  证明 : $f$  与  $g$  具有相同的正负惯性指数 .

\begin{enumerate}
  \setcounter{enumi}{2}
  \item  设  $A \in P^{m \times n}$.  证明 :  秩  $(A)=r$  的充分必要条件是存在  $B \in P^{m \times r}, C \in P^{r \times n}$,  使得秩  $(B)=$  秩  $(C)=r$  且  $A=B C$.

  \item  设  $A, C$  为  $n$  阶正定阵 ,  已知  $B$  是矩阵方程  $A X+X A=C$  的唯一解 ,  证明  $B$  是正定矩阵 .

\end{enumerate}
\section{8. 北京交通大学 2016 年研究生入学考试试题高等代数 
 李扬 
 微信公众号: sxkyliyang}
 一 、  选择题 ( 本大题共  8  小题 ,  每小题  3  分 ,  共  24  分 )

\begin{enumerate}
  \item  下列说法正确的有   个 .
\end{enumerate}
(1)  数域  $P$  上  $n$  阶矩阵  $A$  相似于对角阵的充要条件是  $A$  的最小多项式是  $P$  上互素的一次因式之积 .

(2)  两个矩阵有相同的最小多项式 ,  则它们是相似矩阵 .

(3) $n$  阶矩阵  $A$  的任一特征根都是最小多项式的根 .

(4) $n$  阶矩阵  $A$  的最小多项式的根都是  $A$  的特征根 .\\
(A) 1\\
(B) 2\\
(C) 3\\
(D) 4

\begin{enumerate}
  \setcounter{enumi}{2}
  \item  已知  $Q=\left(\begin{array}{lll}1 & 2 & 3 \\ 2 & 4 & t \\ 3 & 6 & 9\end{array}\right), P$  为三阶非零矩阵 ,  且满足  $P Q=O$,  这里  $O$  为零矩阵 ,  则 \\
$(A) t=6$  时 , $P$  的秩必为  1\\
(B) $t=6$  时 , $P$  的秩必为  2\\
$(C) t \neq 6$  时 , $P$  的秩必为  1\\
$(D) t \neq 6$  时 , $P$  的秩必为  2

  \item  设 

\end{enumerate}
$$
A=\left(\begin{array}{ccc}
2 & -1 & -1 \\
-1 & 2 & -1 \\
-1 & -1 & 2
\end{array}\right), B=\left(\begin{array}{ccc}
1 & & \\
& 1 & \\
& & 0
\end{array}\right)
$$
 则  $A$  与  $B$\\
(A)  合同且相似 \\
(B)  合同但不相似 \\
$(C)$  不合同但相似 \\
(D)  既不合同也不相似 

\begin{enumerate}
  \setcounter{enumi}{4}
  \item  设矩阵  $A=\left(\alpha_{1}, \alpha_{2}, \alpha_{3}, \alpha_{4}\right)$  经行的初等变换变为矩阵  $B=\left(\beta_{1}, \beta_{2}, \beta_{3}, \beta_{4}\right)$,  且  $\alpha_{1}, \alpha_{2}, \alpha_{3}$  线性无关 , $\alpha_{1}, \alpha_{2}, \alpha_{3}, \alpha_{4}$  线性相关 ,  则 \\
(A) $\beta_{4}$  不能由  $\beta_{1}, \beta_{2}, \beta_{3}$  线性表示 .\\
(B) $\beta_{4}$  可由  $\beta_{1}, \beta_{2}, \beta_{3}$  线性表示 ,  但表示法不唯一 .\\
$(C) \beta_{4}$  可由  $\beta_{1}, \beta_{2}, \beta_{3}$  线性表示 ,  且表示法唯一 .\\
(D) $\beta_{4}$  能否由  $\beta_{1}, \beta_{2}, \beta_{3}$  线性表示不能确定 .\\
$(B)$  存在正交矩阵  $Q$,  使  $Q^{-1}\left(A^{-1}+B^{-1}\right) Q=\Lambda$.\\
$(C)$  存在正交矩阵  $Q$,  使  $Q^{T}\left(A^{*}+B^{*}\right) Q=\Lambda$.

  \item  下列所定义的变换 ,  有  \_\_  个是线性变换 .\\
(1)  在  $P^{3}$  中 , $\sigma\left(x_{1}, x_{2}, x_{3}\right)=\left(x_{1}^{2}, x_{2}+x_{3}, x_{3}^{2}\right)$\\
(2)  在  $P^{3}$  中 , $\sigma\left(x_{1}, x_{2}, x_{3}\right)=\left(2 x_{1}-x_{2}, x_{2}+x_{3}, x_{1}\right)$\\
(3)  在  $P[x]$  中 , $\sigma f(x)=f(x+1)$\\
(4)  在  $P[x]$  中 , $\sigma f(x)=f\left(x_{0}\right), x_{0} \in P$,  是一固定的数 .\\
(A) 1\\
(B) 2\\
(C) 3\\
(D) 4 7.  设  $A$  是  $n$  阶矩阵 , $P$  是  $n$  阶可逆矩阵 , $n$  维列向量  $\alpha$  是矩阵  $A$  的属于特征值  $\lambda$  的特征向量 ,  那么在下   列矩阵中 :\\
(1) $A^{2}$\\
(2) $P^{-1} A P$\\
(3) $A^{T}$\\
(4) $E-\frac{1}{2} A$

\end{enumerate}
$\alpha$  一定是其特征向量的矩阵共有 \\
 个 .

(A) 1

(B) 2

(C) 3

(D) 4

\begin{enumerate}
  \setcounter{enumi}{8}
  \item $a=1$  是齐次方程组 
\end{enumerate}
$$
\left\{\begin{array}{l}
x_{1}+x_{2}+x_{3}=0 \\
x_{1}+2 x_{2}+a x_{3}=0 \\
x_{1}+4 x_{2}+a^{2} x_{3}=0
\end{array}\right.
$$
 有非零解的 \\
(A)  充分必要条件 \\
(B)  充分而非必要条件 \\
(C)  必要而非充分条件 \\
(D)  既非充分又非必要条件 

 二 、 填空题 ( 本大题共  8  小题 ,  每小题  3  分 ,  共  24  分 )

\begin{enumerate}
  \item  两个多项式  $f(x)=x^{4}+x^{3}-3 x^{2}-4 x-1, g(x)=x^{3}+x^{2}-x-1$  的最大公因式为 

  \item  已知三阶实矩阵 

\end{enumerate}
$$
A=\left(\begin{array}{ccc}
-1 & a & a x-y \\
a & 1 & x+a y \\
c & b & b x+c y
\end{array}\right)
$$
 则  $A$  的秩  $\mathrm{r}(A)=$

\begin{enumerate}
  \setcounter{enumi}{3}
  \item  当  $t$  取值满足   时 , $f\left(x_{1}, x_{2}, x_{3}\right)=x_{1}^{2}+x_{2}^{2}+5 x_{3}^{2}+2 t x_{1} x_{2}-2 x_{1} x_{3}+4 x_{2} x_{3}$  是正定的 .

  \item  当  $x$  和  $y$  满足  $—$ \_ 时 , $A=\left(\begin{array}{lll}0 & 0 & 1 \\ x & 1 & y \\ 1 & 0 & 0\end{array}\right)$  有三个线性无关的特征向量 .

  \item  设  $\operatorname{dim} V=4, \sigma \in L(V), \sigma$  在基  $\varepsilon_{1}, \varepsilon_{2}, \varepsilon_{3}, \varepsilon_{4}$  下的矩阵为 

\end{enumerate}
$$
A=\left(\begin{array}{llll}
1 & 2 & 1 & 0 \\
0 & 1 & 0 & 0 \\
1 & 3 & 1 & 0 \\
0 & 4 & 2 & 1
\end{array}\right)
$$
 则  $\sigma$  的含  $\varepsilon_{1}$  的最小不变子空间  $W=$

\begin{enumerate}
  \setcounter{enumi}{6}
  \item  设半正定二次型  $f\left(x_{1}, \cdots, x_{n}\right)=X^{T} A X$  的秩为  $r$,  则  $f\left(x_{1}, \cdots, x_{n}\right)=0$  的实数解构成  $\mathbb{R}^{n}$  的一个   维子空间 .

  \item  设  $A=\left(a_{i j}\right)$  是一个  $n$  阶非零方阵 ,  且  $a_{i j}$  全为实数 .  如果  $A$  的每一个元素  $a_{i j}$  都等于它的代数余子式 ,  则  $A$  的秩  $\mathrm{r}(A)=$

  \item  设  $A$  是  5  阶矩阵 , $A^{*}$  是  $A$  的伴随矩阵 ,  若  $\eta_{1}, \eta_{2}$  是方程组  $A X=0$  的两个线性无关的解 ,  那么秩  $\mathrm{r}\left(A^{*}\right)=$

\end{enumerate}
 三 、 计算题 ( 本大题共  6  小题 ,  每小题  12  分 ,  共  72  分 )

\begin{enumerate}
  \item  已知线性方程组 
\end{enumerate}
$$
\left\{\begin{array}{l}
x_{1}+x_{2}+x_{3}+x_{4}=-1 \\
4 x_{1}+3 x_{2}+5 x_{3}-x_{4}=-1 \\
a x_{1}+x_{2}+3 x_{3}+b x_{4}=1
\end{array}\right.
$$
 有  3  个线性无关的解 ,  求  $a, b$  的值及方程组的通解 .

\begin{enumerate}
  \setcounter{enumi}{2}
  \item  计算 
\end{enumerate}
$$
D_{n}=\left|\begin{array}{cccccc}
1+a_{1} & 1 & 1 & \cdots & 1 & 1 \\
2 & 2+a_{2} & 2 & \cdots & 2 & 2 \\
3 & 3 & 3+a_{3} & \cdots & 3 & 3 \\
\vdots & \vdots & \vdots & & \vdots & \vdots \\
n-1 & n-1 & n-1 & \cdots & n-1+a_{n-1} & n-1 \\
n & n & n & \cdots & n & n+a_{n}
\end{array}\right|
$$
 其中  $a_{i}(i=1, \cdots, n)$  均不为  0 .

\begin{enumerate}
  \setcounter{enumi}{3}
  \item  设矩阵  $A=\left(\begin{array}{ccc}a & -1 & c \\ 5 & b & 3 \\ 1-c & 0 & -a\end{array}\right),|A|=-1, A$  的伴随矩阵  $A^{*}$  有一个特征值  $\lambda_{0}$,  属于  $\lambda_{0}$  的一个特征   向量为  $\alpha=(-1,-1,1)^{T}$,  求  $a, b, c$  和  $\lambda_{0}$  的值 .

  \item  设  $V=\left\{a_{m} x^{m}+a_{m-1} x^{m-1}+\cdots+a_{1} x+a_{0} \mid a_{i} \in\right.$  数域  $\left.P\right\}, T \in L(V)$,

\end{enumerate}
$$
T: V \rightarrow V, T(f(x))=x f^{\prime}(x)-f(x)
$$
(1)  求  $T$  的核空间和像空间 : $\operatorname{ker} T$  及  $\operatorname{Im} T$;

(2)  求证 : $V=\operatorname{ker} T \oplus \operatorname{Im} T$.

5 .  设  $2 n$  阶方阵  $A=\left(\begin{array}{cc}-E & E \\ E & E\end{array}\right)$,  其中  $E$  是  $n$  阶单位矩阵 .

(1)  求  $A$  的特征多项式 ;

(2)  求  $A$  的最小多项式 ;

(3)  求  $A$  的若当标准形 .

\begin{enumerate}
  \setcounter{enumi}{6}
  \item  已知二次型  $f\left(x_{1}, x_{2}, x_{3}\right)=(1-a) x_{1}^{2}+(1-a) x_{2}^{2}+2 x_{3}^{2}+2(1+a) x_{1} x_{2}$  的秩为  2 .
\end{enumerate}
(1)  求  $a$  的值 .

(2)  求正交变换  $X=Q Y$  把  $f\left(x_{1}, x_{2}, x_{3}\right)$  化成标准型 .

 四 、 证明题 ( 本大题共  3  题 ,  每小题  10  分 ,  共  30  分 )

\begin{enumerate}
  \item  设线性变换  $\sigma$  与  $\tau$  满足  $\sigma^{2}=\sigma, \tau^{2}=\tau$,  证明 : $\sigma$  与  $\tau$  有相同的核的充分必要条件是  $\sigma \tau=\sigma, \tau \sigma=\tau$.

  \item  设  $n$  阶方阵  $A, B$  满足  $A B=A-B$,  证明 :

\end{enumerate}
(1) $\lambda=1$  不是  $B$  的特征值 ;

(2)  若  $B$  相似于对角矩阵 ,  则有可逆矩阵  $T$,  使  $T^{-1} A T$  与  $T^{-1} B T$  均为对角矩阵 .

\begin{enumerate}
  \setcounter{enumi}{3}
  \item  证明 : $n$  阶可逆对称矩阵  $A$  是正定矩阵的充要条件是对任意  $n$  阶正定矩阵  $B, A B$  的迹  $\operatorname{tr}(A B)$  均大于  0 .
\end{enumerate}
\section{9. 北京交通大学 2017 年研究生入学考试试题高等代数 
 李扬 
 微信公众号: sxkyliyang}
 一 、 选择题 ( 本大题共  6  小题 ,  每小题  3  分 ,  共  18  分 )

\begin{enumerate}
  \item $A$  是  $n$  阶方阵 ,  下列说法中错误的有   个 .\\
(1) $A$  与对角矩阵相似的充要条件是  $A$  的最小多项式无重根 ;\\
(2) $A$  与对角矩阵相似的充要条件是  $A$  的不变因子都没有重根 ;\\
(3) $A$  与对角矩阵相似的充要条件是  $A$  有  $n$  个不同的特征值 ;\\
(4) $A$  与对角矩阵相似的充要条件是  $A$  的初等因子全为一次的 .\\
(A) 1\\
(B) 2\\
(C) 3\\
(D) 4

  \item  设  $A$  为  $n$  阶矩阵 , $\alpha$  为  $n$  维列向量 ,  若秩  $\left(\begin{array}{cc}A & \alpha \\ \alpha^{T} & 0\end{array}\right)=$  秩  $A$,  则线性方程组 \\
(A) $A X=\alpha$  必有无穷多解 .\\
(B) $A X=\alpha$  必有唯一解 .\\
$(C)\left(\begin{array}{cc}A & \alpha \\ \alpha^{T} & 0\end{array}\right)\left(\begin{array}{c}X \\ Y\end{array}\right)=0$  仅有零解 .\\
(D) $\left(\begin{array}{cc}A & \alpha \\ \alpha^{T} & 0\end{array}\right)\left(\begin{array}{l}X \\ Y\end{array}\right)=0$  必有非零解 .

  \item  设  $n$  阶方阵  $\left(a_{i j}\right)_{n \times n}$  的特征值为  $\lambda_{i}(i=1,2, \cdots, n)$,  则  $\sum_{i=1}^{n} \lambda_{i}^{2}$  的值为 \\
(A) $\sum_{i=1}^{n} a_{i i}^{2}$\\
(B) $\left(\sum_{i=1}^{n} a_{i i}\right)^{2}$\\
(C) $\sum_{i=1}^{n} \sum_{j=1}^{n} a_{i j} a_{j i}$\\
(D) $\sum_{i=1}^{n} a_{i j}^{2}$

  \item  设  $A$  为正交矩阵 ,  则下列不一定为正交矩阵的是 \\
(A) $A^{T}$\\
(B) $A^{2}$\\
$(C) A^{*}$ ( $A^{*}$  为  $A$  的伴随矩阵 )\\
(D) $k A(k \neq 0)$

  \item  设 

\end{enumerate}
$$
A=\left(\begin{array}{llll}
2 & 2 & 2 & 2 \\
2 & 2 & 2 & 2 \\
2 & 2 & 2 & 2 \\
2 & 2 & 2 & 2
\end{array}\right), B=\left(\begin{array}{llll}
1 & 0 & 0 & 0 \\
0 & 0 & 0 & 0 \\
0 & 0 & 0 & 0 \\
0 & 0 & 0 & 0
\end{array}\right)
$$
 则  $A 与 B$\\
(A)  合同且相似 \\
(B)  合同但不相似 \\
(C)  不合同但相似 \\
(D)  不合同且不相似 

\begin{enumerate}
  \setcounter{enumi}{6}
  \item  下列所定义的变换 ,\\
$(A)$  在  $\mathbb{R}^{2}$  中 , $\sigma\left(x_{1}, x_{2}, x_{3}\right)=\left(2 x_{1}+x_{2}, x_{2}-x_{3}, 2 x_{2}+4 x_{3}\right)$;\\
$(B)$  在  $\mathbb{R}^{2}$  中 ,
\end{enumerate}
$$
\sigma(a, b)= \begin{cases}(a, b), & \text { 若 } a b \geqslant 0 \\ (a,-b), & \text { 若 } a b<0 ;\end{cases}
$$
\includegraphics[max width=\textwidth]{2022_04_18_33b622a7abd81c227674g-243}

$(D)$  设  $V$  是数域  $F$  上的  1  维线性空间 , $\sigma(\alpha)=a \alpha$,  其中  $a$  是  $F$  中一固定数 .

 二 、 填空题 ( 本大题共  10  小题 ,  每小题  3  分 ,  共  30  分 )

\begin{enumerate}
  \item  设  4  阶方阵  $A$  有特征值  $1,2,2,3$,  则  $\left|A^{-1}+3 E\right|=$ ,$\left|A^{*}\right|=$ 2.  设向量组  $\alpha_{1}, \alpha_{2}, \alpha_{3}$  线性无关 ,  当  $t$  满足  \_\_ 时 , $\alpha_{1}+t \alpha_{2}, \alpha_{2}+t \alpha_{3}, \alpha_{3}+t \alpha_{1}$  也线性无关 .

  \item  已知  $A=\left(\begin{array}{ccc}1 & 0 & 0 \\ 0 & 0 & 1 \\ 0 & 1 & x\end{array}\right)$  与  $B=\left(\begin{array}{ccc}1 & 0 & 0 \\ 0 & y & 0 \\ 0 & 0 & -1\end{array}\right)$  相似 ,  则  $x=$

\end{enumerate}
$y=$

\begin{enumerate}
  \setcounter{enumi}{4}
  \item  设  $V$  为数域  $F$  上的一切  $n$  阶对称矩阵所构成的向量空间 ,  则  $\operatorname{dim} V=$

  \item  设 

\end{enumerate}
$$
\alpha_{1}=\left(\frac{1}{\sqrt{2}}, 0, \frac{1}{\sqrt{2}}\right), \alpha_{2}=(0,1,0), \alpha_{3}=\left(\frac{1}{\sqrt{2}}, 0,-\frac{1}{\sqrt{2}}\right)
$$
 为欧几里得空间  $\mathbb{R}^{3}$  的一个标准正交基 ,  则  $\mathbb{R}^{3}$  中向量  $\xi=(2,1,-2)$  在  $\alpha_{1}, \alpha_{2}, \alpha_{3}$  下的坐标为 

\begin{enumerate}
  \setcounter{enumi}{6}
  \item  已知实二次型 
\end{enumerate}
$$
f\left(x_{1}, x_{2}, x_{3}\right)=a\left(x_{1}^{2}+x_{2}^{2}+x_{3}^{2}\right)+4 x_{1} x_{2}+4 x_{1} x_{3}+4 x_{2} x_{3}
$$
 经正交变换  $X=P Y$  化成标准形  $f=6 y_{1}^{2}$,  则  $a=$

\begin{enumerate}
  \setcounter{enumi}{7}
  \item  设矩阵  $A=\left(\begin{array}{ccc}1 & 0 & 0 \\ 0 & \omega & 0 \\ 0 & 0 & \omega^{2}\end{array}\right), \omega=\frac{-1+\sqrt{3} \mathrm{i}}{2}$,  设  $V=\{f(A) \mid f(x) \in \mathbb{R}[x]\}$,  则  $\operatorname{dim} V=\underline{C}, V$  的一组   基为 

  \item  设  $A$  是元素都是  1  的  $n$  阶方阵 ,  则  $A$  的最小多项式为 

  \item  若  $n$  维线性空间  $V$  的线性变换  $\sigma$  有  $n$  个不同的特征值 ,  则  $\sigma$  有 

  \item  设  $\alpha=(1,0,1)^{T}, A=\alpha \alpha^{T}$,  若  $B=(k E+A)^{2}$  是正定矩阵 ,  则  $k$  满足 

\end{enumerate}
 三 、 计算题 ( 本大题共  6  小题 ,  每小题  12  分 ,  共  72  分 )

\begin{enumerate}
  \item  计算  $n$  阶行列式 
\end{enumerate}
$$
D_{n}=\left|\begin{array}{ccccc}
x & y & y & \cdots & y \\
z & x & y & \cdots & y \\
z & z & x & \cdots & y \\
\vdots & \vdots & \vdots & & \vdots \\
z & z & z & \cdots & x
\end{array}\right|, n \geqslant 2 .
$$

\begin{enumerate}
  \setcounter{enumi}{2}
  \item  设  $A$  为三阶矩阵 , $\alpha_{1}, \alpha_{2}, \alpha_{3}$  为线性无关的三维列向量 ,  且 
\end{enumerate}
$$
A \alpha_{1}=\alpha_{1}+\alpha_{2}+\alpha_{3}, A \alpha_{2}=2 \alpha_{2}+\alpha_{3}, A \alpha_{3}=2 \alpha_{2}+3 \alpha_{3}
$$
(1)  求矩阵  $B$,  使  $A\left(\alpha_{1}, \alpha_{2}, \alpha_{3}\right)=\left(\alpha_{1}, \alpha_{2}, \alpha_{3}\right) B$;

(2)  求矩阵  $A$  的特征值 ;

(3)  求可逆矩阵  $P$,  使得  $P^{-1} A P$  为对角矩阵 .

\begin{enumerate}
  \setcounter{enumi}{3}
  \item  请给出方程组 
\end{enumerate}
$$
\left\{\begin{array}{l}
x_{1}+a_{1} x_{2}+a_{1}^{2} x_{3}=a_{1}^{3} \\
x_{1}+a_{2} x_{2}+a_{2}^{2} x_{3}=a_{2}^{3} \\
x_{1}+a_{3} x_{2}+a_{3}^{2} x_{3}=a_{3}^{3} \\
x_{1}+a_{4} x_{2}+a_{4}^{2} x_{3}=a_{4}^{3}
\end{array}\right.
$$
 无解的一个充分必要条件 ,  并且当 : $\beta_{1}=(-1,1,1)^{T}, \beta_{2}=(1,1,-1)^{T}$  为解时 ,  求全部解 .

\begin{enumerate}
  \setcounter{enumi}{4}
  \item (1)  证明 :  在  $P[x]_{n}$  中 ,  多项式 
\end{enumerate}
$$
f_{i}=\left(x-a_{1}\right) \cdots\left(x-a_{i-1}\right)\left(x-a_{i+1}\right) \cdots\left(x-a_{n}\right), i=1,2, \cdots, n,
$$
 是一组基 ,  其中  $a_{1}, a_{2}, \cdots, a_{n}$  是互不相同的数 .

(2)  在  (1)  中 ,  取  $a_{1}, a_{2}, \cdots, a_{n}$  为全体  $n$  次单位根 ,  求由基  $1, x, x^{2}, \cdots, x^{n-1}$  到基  $f_{1}, f_{2}, \cdots, f_{n}$  的   过渡矩阵 . 5.  求矩阵 
$$
A(\lambda)=\left(\begin{array}{cccccc}
\lambda & 0 & 0 & \cdots & 0 & a_{n} \\
-1 & \lambda & 0 & \cdots & 0 & a_{n-1} \\
0 & -1 & \lambda & \cdots & 0 & a_{n-2} \\
\vdots & \vdots & \vdots & & \vdots & \vdots \\
0 & 0 & 0 & \cdots & \lambda & a_{2} \\
0 & 0 & 0 & \cdots & -1 & \lambda+a_{1}
\end{array}\right)
$$
 的不变因子与行列式因子 .

\begin{enumerate}
  \setcounter{enumi}{6}
  \item  已知二次型  $X^{T} A X$  经正交变换化为  $2 y_{1}^{2}-y_{2}^{2}-y_{3}^{2}$,  又已知  $A^{*} \alpha=\alpha$,  其中  $\alpha=(1,1,-1)^{T}, A^{*}$  为  $A$  的   伴随矩阵 ,  求此二次型的表达式 .
\end{enumerate}
 四 、 证明题 ( 本大题共  3  小题 ,  每小题  10  分 ,  共  30  分 )

\begin{enumerate}
  \item  设  $a_{1}, a_{2}, \cdots, a_{n}$  是互异的整数 ,  求证 : $f(x)=\left(x-a_{1}\right)\left(x-a_{2}\right) \cdots\left(x-a_{n}\right)-1$  在有理数域上不可约 .

  \item  设  $V$  是  $n$  维欧式空间 , $\sigma$  是  $V$  的正交变换 ,

\end{enumerate}
$$
V_{1}=\{\alpha \mid \sigma(\alpha)=\alpha, \alpha \in V\}, V_{2}=\{\beta \mid \beta=\sigma(\gamma)-\gamma, \gamma \in V\}
$$
 求证 : $V_{2}=V_{1}^{\perp}$.  其中 , $V_{1}^{\perp}$  表示  $V_{1}$  的正交补 .

\begin{enumerate}
  \setcounter{enumi}{3}
  \item  设 
\end{enumerate}
$$
A=\left(\begin{array}{cccc}
a_{11} & a_{12} & \cdots & a_{1 n} \\
a_{21} & a_{22} & \cdots & a_{2 n} \\
\vdots & \vdots & & \vdots \\
a_{n-1,1} & a_{n-1,2} & \cdots & a_{n-1, n}
\end{array}\right)
$$
 为线性方程组 
$$
\left\{\begin{array}{l}
a_{11} x_{1}+a_{12} x_{2}+\cdots+a_{1 n} x_{n}=0 \\
a_{21} x_{1}+a_{22} x_{2}+\cdots+a_{2 n} x_{n}=0 \\
\cdots \cdots \\
a_{n-1,1} x_{1}+a_{n-1,2} x_{2}+\cdots+a_{n-1, n} x_{n}=0
\end{array}\right.
$$
 的系数矩阵 .  设  $M_{j}(j=1,2, \cdots, n)$  是将  $A$  去掉第  $j$  列所得的  $n-1$  阶子式 .

 求证 :

(1) $\left(M_{1},-M_{2}, \cdots,(-1)^{n-1} M_{n}\right)$  是线性方程组的一个解 ;

(2)  若  $A$  的秩为  $n-1$,  那么方程组的解都是  $\left(M_{1},-M_{2}, \cdots,(-1)^{n-1} M_{n}\right)$  的倍数 .

\section{0. 北京交通大学 2009 年研究生入学考试试题数学分析 
 李扬 
 微信公众号: sxkyliyang}
 一 、 ( 本题满分  25  分 )

 用一致连续函数的定义证明 :  函数  $f(x)=\frac{1}{x} \sin \frac{1}{x}$  在区间  $(a,+\infty),(a>0)$  上一致连续 ,  而在区间  $(0, a)$  上   非一致连续 .

 二 、 ( 本题满分  25  分 )

 设  $f(x)$  是区间  $[0, \pi]$  上的连续函数 ,  证明 :
$$
\lim _{n \rightarrow \infty} \int_{0}^{\pi} f(x)|\sin n x| \mathrm{d} x=\frac{2}{\pi} \int_{0}^{\pi} f(x) \mathrm{d} x .
$$
 三 、 ( 本题满分  25  分 )

 对任意的  $n=1,2,3, \cdots$,  求能使不等式 
$$
\left(1+\frac{1}{n}\right)^{n+\alpha} \leqslant e \leqslant\left(1+\frac{1}{n}\right)^{n+\beta}
$$
 成立的  $\alpha$  的最大值与  $\beta$  最小值 .

 四 、( 本题满分  25  分 )

 设正数列  $\left\{a_{n}\right\}$  单调递增 ,  证明 :  级数  $\sum_{n=1}^{\infty}\left(1-\frac{a_{n}}{a_{n+1}}\right)$  收敛的充分必要条件是数列  $\left\{a_{n}\right\}$  有界 .

 五 、( 本题满分  25  分 )

 设  $z=f\left(x y^{2}, x^{2} y\right)$,  求  $\frac{\partial^{2} z}{\partial x^{2}}, \frac{\partial^{2} z}{\partial y^{2}}, \frac{\partial^{2} z}{\partial x \partial y}$,  其中函数  $f$  具有连续的二阶偏导数 .

 六 、( 本题满分  25  分 )

 设函数  $f(x)=\sum_{n=1}^{\infty}\left(x+\frac{1}{n}\right)^{n}$.

\begin{enumerate}
  \item  确定函数  $f(x)$  的定义域  $D$;

  \item  证明 :  级数  $\sum_{n=1}^{\infty}\left(x+\frac{1}{n}\right)^{n}$  在  $D$  上非一致收敛 ;

  \item  证明 :  函数  $f(x)$  的定义域  $D$  上连续 .

\end{enumerate}
\section{1. 北京交通大学 2010 年研究生入学考试试题数学分析}
 李扬 

 微信公众号 : sxkyliyang

 一 、 ( 本题满分  15  分 )

 证明 :  极限  $\lim _{n \rightarrow \infty} \sin n$  不存在 ,  其中  $n$  为正整数 .

 二 、 ( 本题满分  15  分 )

 求极限 : $\lim _{x \rightarrow 0} \frac{\tan (\tan x)-\sin (\sin x)}{\tan x-\sin x}$.

 三 、 ( 本题满分  15  分 )

 设函数  $f(x)$  与  $g(x)$  满足 : 在闭区间  $[a, b]$  上连续 ,  在开区间  $(a, b)$  内可导 ,  而且  $f(a)=f(b)=0$.  证明 :  至少   存在一点  $c \in(a, b)$,  使得  $f^{\prime}(c)+f(c) g^{\prime}(c)=0$.

 四 、 ( 本题满分  15  分 )

 设当  $x>0$  时 ,  方程  $k x+\frac{1}{x^{2}}=1$  有且仅有一个解 ,  求  $k$  的取值范围 .

 五 、( 本题满分  15  分 )

 设  $f\left(x^{2}-1\right)=\ln \frac{x^{2}}{x^{2}-2}$,  且  $f[\varphi(x)]=\ln x$,  求积分  $\int \varphi(x) \mathrm{d} x$.

 六 、 ( 本题满分  15  分 )

 设 
$$
f(x, y)= \begin{cases}x y \frac{x^{2}-y^{2}}{x^{2}+y^{2}}, & x^{2}+y^{2} \neq 0 \\ 0, & x^{2}+y^{2}=0\end{cases}
$$
 求  $f_{x y}^{\prime \prime}(0,0)$  与  $f_{y x}^{\prime \prime}(0,0)$.

 七 、 ( 本题满分  15  分 )

 讨论级数 
$$
\sqrt{2}+\sqrt{2-\sqrt{2}}+\sqrt{2-\sqrt{2+\sqrt{2}}}+\sqrt{2-\sqrt{2+\sqrt{2+\sqrt{2}}}}+\cdots
$$
 的玫散性 .

 八 、 ( 本题满分  15  分 )

 确定  Riemann $\zeta$  函数  $\zeta(x)=\sum_{n=1}^{\infty} \frac{1}{n^{x}}$  的定义域 .  并讨论  Riemann $\zeta$  函数  $\zeta(x)=\sum_{n=1}^{\infty} \frac{1}{n^{x}}$  在其定义域上的连续   性 .

 九 、 ( 車题满分  15  分 )

 计算由曲面 
$$
\left(x^{2}+y^{2}+z^{2}\right)^{2}=a^{3} z,(a>0)
$$
 所围立体的体积  $V$.

 十 、 ( 本题满分  15  分 )

 计算曲线积分 
$$
I=\oint_{C}\left(y^{2}-z^{2}\right) \mathrm{d} x+\left(z^{2}-x^{2}\right) \mathrm{d} y+\left(x^{2}-y^{2}\right) \mathrm{d} z
$$
 其中曲面  $C$  是用平面  $x+y+z=\frac{3}{2}$  截立方体  $\Omega: 0 \leqslant x \leqslant 1,0 \leqslant y \leqslant 1,0 \leqslant z \leqslant 1$  表面所得的截痕 ,  从  $x$  轴   正向看去 ,  取逆时针方向 .

\section{2. 北京交通大学 2011 年研究生入学考试试题数学分析}
 李扬 

 微信公众号 : sxkyliyang

 一 、( 本题满分  15  分 )

\begin{enumerate}
  \item  叙述极限  $\lim _{x \rightarrow-\infty} f(x)$  存在的柯西  (Cauchy)  收敛准则 ;

  \item  利用柯西  (Cauchy)  收敛准则证明极限  $\lim _{x \rightarrow-\infty} \sin x$  不存在 .

\end{enumerate}
 二 、 ( 本题满分  15  分 )

 如果函数  $f(x)$  在区间  $[a, b]$  上连续 ,  而且  $f(a)=f(b)=k, f_{+}^{\prime}(a) \cdot f_{-}^{\prime}(b)>0$,  则在开区间  $(a, b)$  内至少存在   一点  $\xi$,  使得  $f(\xi)=k$.

 三 、 ( 本题满分  15  分 )

\begin{enumerate}
  \item  设函数  $f(x)$  在闭区间  $[a, b]$  上连续 ,  证明 : $\int_{a}^{b} f(x) \mathrm{d} x=\int_{a}^{b} f(a+b-x) \mathrm{d} x$.

  \item  计算定积分  $\int_{\frac{\pi}{5}}^{\frac{3}{10} \pi} \frac{\sin ^{2} x}{x(\pi-2 x)} \mathrm{d} x$.

\end{enumerate}
 四 、 ( 本题满分  15  分 )

 设函数  $f(x)$  在开区间  $(a, b)$  内连续 .  证明 : $f(x)$  在开区间  $(a, b)$  内一致连续的充分必要条件是  $f(a+0)$  与  $f(b-0)$  都存在 .

 五 、( 本题满分  15  分 )

 设  $\sum_{n=1}^{\infty} a_{n}$  是正项级数 ,  极限  $\lim _{n \rightarrow \infty}\left(\frac{a_{n}}{a_{n-1}}\right)^{n}=k$  存在 .

 证明 :  当  $k<e^{-1}$  时 ,  级数  $\sum_{n=1}^{\infty} a_{n}$  收敛 ;  当  $k>e^{-1}$  时 ,  级数  $\sum_{n=1}^{\infty} a_{n}$  发散 .

 六 、 ( 本题满分  15  分 )

 设数列  $\left\{x_{n}\right\} \subset(0,1)$,  而数列  $\left\{x_{n}\right\}$  中各项两两不相等 ,  讨论函数  $f(x)=\sum_{n=1}^{\infty} \frac{\operatorname{sgn}\left(x-x_{n}\right)}{2^{n}}$  的连续性 .

 七 、 ( 本题满分  15  分 )

 设  $u=f(x, z)$,  而  $z(x, y)$  是由方程  $z=x+y \varphi(z)$  所确定的函数 ,  求微分  $\mathrm{d} u$.

 八 、 ( 本题满分  15  分 )

 设  $x=e^{v}+u^{3}, y=e^{u}-v^{3}$,  求反函数的偏导数  $\frac{\partial v}{\partial x}$.

 九 、 ( 本题满分  15  分 )

 求极限 

\includegraphics[max width=\textwidth]{2022_04_18_33b622a7abd81c227674g-248}

 十 、( 本题满分  15  分 )

 计算曲线积分 :
$$
I=\oint_{L}(y-z) \mathrm{d} x+(z-x) \mathrm{d} y+(x-y) \mathrm{d} z,
$$
 其中  $L$  为椭圆  $x^{2}+y^{2}=a^{2}, \frac{x}{a}+\frac{y}{b}=1,(a>0, b>0)$,  从  $x$  轴正向看去 ,  取逆时针方向 .

\section{3. 北京交通大学 2012 年研究生入学考试试题数学分析 
 李扬 
 微信公众号: sxkyliyang}
 一 、 ( 本题满分  15  分 )

 设函数  $f(x)=a_{1} \sin x+a_{2} \sin (2 x)+\cdots+a_{n} \sin (n x)$,  其中  $a_{1}, a_{2}, \cdots, a_{n}$  是常数 ,  如果对任意的实数  $x$,  有  $|f(x)| \leqslant|\sin x|$.  证明 : $\left|a_{1}+2 a_{2}+\cdots+n a_{n}\right| \leqslant 1 .$

 二 、 ( 本题满分  15  分 )

 设函数  $f(x)$  在区间  $[a,+\infty)$  上连续 ,  而且极限  $\lim _{x \rightarrow+\infty} f(x)$  存在 ,  证明 :  函数  $f(x)$  在区间  $[a,+\infty)$  上一致连   续 .

 三 、 ( 本题满分  15  分 )

 设函数  $f(x)$  在区间  $[0,1]$  上连续 ,  开区间  $(0,1)$  内可导 ,  而且  $\lim _{x \rightarrow 0} \frac{f(x)}{x}=0, f(1)=0$.  证明 :  在区间  $(0,1)$  内   至少存在一点  $\xi$,  使得  $f^{\prime \prime}(\xi)=0$.

 四 、 ( 車题满分  15  分 )

 求积分 
$$
I=\int_{0}^{1} x(\ln x)^{2012} \mathrm{~d} x
$$
 五 、 ( 本题满分  15  分 )

 已知数列  $\left\{x_{n}\right\}$  满足条件 
$$
\left|x_{n+1}-x_{n}\right| \leqslant \frac{1}{3}\left|x_{n}-x_{n-1}\right|,(n \geqslant 2) .
$$
 证明 :  数列  $\left\{x_{n}\right\}$  收敛 .

 公 、( 本题满分  15  分 )

 证明 :  函数项级数  $\sum_{n=1}^{\infty} \frac{1}{n^{3}} \ln \left(1+n^{2} x^{2}\right)$  在区间  $[0,1]$  上一致收敛 ,  并讨论其和函数在区间  $[0,1]$  上的连续性 ,  可   积性和可导性 .

 七 、 ( 本题满分  15  分 )

 设二元函数 
$$
f(x, y)= \begin{cases}\frac{x y}{x^{2}+y^{2}}, & x^{2}+y^{2}>0 \\ 0, & x^{2}+y^{2}=0 .\end{cases}
$$

\begin{enumerate}
  \item  试判断函数  $f(x, y)$  的两个偏导数在平面各点处是否存在 ?

  \item  试判断函数  $f(x, y)$  在原点  $(0,0)$  沿任何方向的极限是否存在 ?

  \item  试判断函数  $f(x, y)$  在原点  $(0,0)$  是否连续 ?

\end{enumerate}
 八 、 ( 本题满分  15  分 )

 设二元函数  $f(x, y)$  在平面  $\mathbb{R}^{2}$  上二阶连续可微 ,  而且 
$$
f(x, 2 x)=x, f_{x}(x, 2 x)=x^{2}, f_{x x}(x, y)=f_{y y}(x, y), \forall(x, y) \in \mathbb{R}^{2}
$$
 求  $f_{y}(x, 2 x), f_{y y}(x, 2 x)$  及  $f_{x y}(x, 2 x)$.

 九 、 ( 本题满分  15  分 )

 设平面区域  $D$  是由  4  条抛物线 
$$
y^{2}=p x, y^{2}=q x, x^{2}=a y, x^{2}=b y
$$
 所围成 ,  其中  $0<p<q, 0<a<b$.  计算二重积分  $A=\iint_{D} \frac{1}{x y} \mathrm{~d} x \mathrm{~d} y$.  十 、 ( 本题满分  15  分 )

 计算曲面积分 
$$
I=\iint_{S} \sin ^{4} x \mathrm{~d} y \mathrm{~d} z+e^{-|y|} \mathrm{d} z \mathrm{~d} x+z^{2} \mathrm{~d} x \mathrm{~d} y,
$$
 其中  $S$  为曲面  $x^{2}+y^{2}+z^{2}=1, z \geqslant 0$,  定向为上侧 .

\section{4. 北京交通大学 2013 年研究生入学考试试题数学分析}
 李扬 

 微信公众号 : sxkyliyang

 一 、 ( 本题满分  20  分 )

 设函数  $f(x)$  在区间  $[0,+\infty)$  上一致连续 ,  而且有 
$$
\lim _{n \rightarrow \infty} f(x+n)=0,(x \in[0,+\infty)) .
$$
 证明 : $\lim _{x \rightarrow+\infty} f(x)=0$.

 二 、 ( 本题满分  20  分 )

 设函数  $f(x)$  在区间  $[a, b]$  上连续 ,  在  $(a, b)$  内有二阶导数 ,  则对  $x \in(a, b)$,  存在  $\xi \in(a, b)$,  使得 
$$
\frac{f(x)-f(a)}{x-a}-\frac{f(b)-f(a)}{b-a}=\frac{1}{2}(x-a) f^{\prime \prime}(\xi) .
$$
 三 、 ( 本题满分  20  分 )

 求  $A$  的最小值 ,  使得当  $x>0$  时 ,  函数  $f(x)=5 x^{2}+A x^{-5}$  的值不小于  28 .

 四 、 ( 本题满分  20  分 )

 设  $f(x)$  的原函数  $F(x)>0$,  且  $F(0)=1$;  当  $x \geqslant 0$  时 ,  有  $f(x) F(x)=\sin ^{2} 2 x$.  试求  $f(x)$.

 五 、( 本题满分  20  分 )

 设函数  $f(x)=\sum_{n=1}^{\infty}\left(x+\frac{1}{n}\right)^{n}$.

\begin{enumerate}
  \item  求该函数的定义域 ;

  \item  讨论该函数在其定义域内的连续性 .

\end{enumerate}
 六 、 ( 本题满分  20  分 )

 求函数 
$$
f(x, y)= \begin{cases}\frac{x^{2} y}{x^{4}+y^{2}}, & x^{2}+y^{2} \neq 0 \\ 0, & x^{2}+y^{2}=0 .\end{cases}
$$
 的全微分 ,  并研究在点  $(0,0)$  处该函数的全微分是否存在 ?

 七 、 ( 本题满分  15  分 )

 设函数  $f(x)$  在区间  $[0,+\infty)$  上连续 , $0<a<b$.

\begin{enumerate}
  \item  证明 :  如果  $\lim _{x \rightarrow+\infty} f(x)=k$,  则  $\int_{0}^{+\infty} \frac{f(a x)-f(b x)}{x} \mathrm{~d} x=(f(0)-k) \ln \frac{b}{a}$.

  \item  证明 :  如果积分  $\int_{0}^{+\infty} \frac{f(x)}{x} \mathrm{~d} x$  收敛 ,  则  $\int_{0}^{+\infty} \frac{f(a x)-f(b x)}{x} \mathrm{~d} x=f(0) \ln \frac{b}{a}$.

\end{enumerate}
 八 、( 本题满分  15  分 )

 计算曲面积分 
$$
I=\iint_{\Sigma} \frac{\mathrm{d} y \mathrm{~d} z}{x}+\frac{\mathrm{d} z \mathrm{~d} x}{y}+\frac{\mathrm{d} x \mathrm{~d} y}{z}
$$
 其中  $\Sigma$  是  $\frac{x^{2}}{a^{2}}+\frac{y^{2}}{b^{2}}+\frac{z^{2}}{c^{2}}=1$  的外侧 .

\section{5. 北京交通大学 2014 年研究生入学考试试题数学分析}
 李扬 

 微信公众号 : sxkyliyang

 一 、( 本题满分  15  分 )

 用一致连续的定义证明 :

\begin{enumerate}
  \item  函数  $f(x)=x^{2}$  在有限区间  $(a, b)(a<b$  都是实数  $)$  内一致连续 ;

  \item  函数  $f(x)=x^{2}$  在无穷区间  $(-\infty,+\infty)$  内非一致连续 .

\end{enumerate}
 二 、 ( 本题满分  15  分 )

 设函数  $f(x)$  在区间  $[0,1]$  上连续 ,  在区间  $(0,1)$  内可导 ,  而且  $f(1)=0$,  则存在点  $\xi \in(0,1)$,  使得 
$$
f^{\prime}(\xi)=\left(1-\xi^{-1}\right) f(\xi)
$$
 三 、 ( 本题满分  15  分 )

 证明不等式  $\ln \left(e^{2 x}+x\right)>3 x-\frac{5}{2} x^{2}$  当  $x>0$  时成立 .

 四 、( 本题满分  15  分 )

 计算无穷积分 
$$
\int_{1}^{+\infty} \frac{\arctan x}{x^{2}\left(1+x^{2}\right)} \mathrm{d} x
$$
 五 、 ( 本题满分  15  分 )

 设曲线  $y=a x^{2},(a>0, x \geqslant 0)$  与  $y=1-x^{2}$  交于点  $A$,  过原点坐标  $O$  和点  $A$  的直线与曲线  $y=a x^{2}$  围成   一个平面图形 ,  问  $a$  为何值时 ,  该图形绕  $x$  轴旋转一周所得的旋转体体积最大 ?  其最大值是多少 ?

 六 、 ( 本题满分  15  分 )

 讨论级数 
$$
\sum_{n=1}^{\infty} \frac{1}{n}\left[e-\left(1+\frac{1}{n}\right)^{n}\right]^{p},(-\infty<p<+\infty)
$$
 的敛散性 .

 七 、 ( 本题满分  15  分 )

 设函数  $u_{n}(x)$  在闭区间  $[a, b]$  上连续  $(n=1,2,3, \cdots)$,  级数  $\sum_{n=1}^{\infty} u_{n}(x)$  在开区间  $(a, b)$  内一致收敛 .  证明 :  函   数  $f(x)=\sum_{n=1}^{\infty} u_{n}(x)$  在闭区间  $[a, b]$  上一致连续 .

 八 、 ( 本题满分  15  分 )

 设二元函数  $u=u(x, y)$  由方程  $u+e^{u}=x y$  所确定 ,  求  $\frac{\partial^{2} u}{\partial x \partial y}$.

 九 、( 本题满分  15  分 )

 设  $S$  为上半球  $z=\sqrt{R^{2}-x^{2}-y^{2}}$  的内侧  $(R>0)$,  计算曲面积分 
$$
I=\iint_{S}(x-y) \mathrm{d} x \mathrm{~d} y+x(y-z) \mathrm{d} y \mathrm{~d} z .
$$
 十 、 ( 本题满分  15  分 )

 计算积分 
$$
I=\int_{-1}^{1} \mathrm{~d} x \int_{0}^{\sqrt{1-x^{2}}} \mathrm{~d} y \int_{1}^{1+\sqrt{1-x^{2}-y^{2}}} \frac{1}{\sqrt{x^{2}+y^{2}+z^{2}}} \mathrm{~d} z
$$

\section{6. 北京交通大学 2015 年研究生入学考试试题数学分析 
 李扬 
 微信公众号: sxkyliyang}
 一 、 ( 本题满分  15  分 )

 设数列  $\left\{a_{n}\right\}$  满足条件 :  存在正数  $M$,  使得对一切  $n$,  有 
$$
A_{n}=\left|a_{2}-a_{1}\right|+\left|a_{3}-a_{2}\right|+\left|a_{4}-a_{3}\right|+\cdots+\left|a_{n}-a_{n-1}\right| \leqslant M .
$$
 证明 :

\begin{enumerate}
  \item  数列  $\left\{A_{n}\right\}$  收敛 ;

  \item  数列  $\left\{a_{n}\right\}$  收敛 .

\end{enumerate}
 二 、 ( 本题满分  15  分 )

 设函数  $f(x)$  是区间  $[0,+\infty)$  上非负的可导函数 ,  且满足  $f(0)=0$,
$$
f(x)-2 x f^{\prime}(x) \geqslant 0 \quad x \in[0,+\infty),
$$
 证明 : $f(x) \equiv 0$.

 三 、 ( 本题满分  15  分 )

 设 
$$
f^{\prime}(\cos x+2)=\sin ^{2} x+\tan ^{2} x
$$
 试求  $f(x)$.

 四 、 ( 車题满分  15  分 )

 设函数  $f(x)$  在点  $x=0$  处的某邻域内有二阶导数 ,  且 
$$
\lim _{x \rightarrow 0} \frac{\sin x+x f(x)}{x^{3}}=0,
$$
 试求  $f(0), f^{\prime}(0)$  及  $f^{\prime \prime}(0)$  的值 .

 五 、( 本题满分  15  分 )

 求积分 
$$
\int_{0}^{\frac{\pi}{2}} \frac{\sin ^{4} x-\cos ^{4} x}{4+\sin x+\cos x} \mathrm{~d} x
$$
 六 、 ( 本题满分  15  分 )

 设正项级数  $\sum_{n=1}^{\infty} a_{n}$  发散 , $s_{n}=\sum_{k=1}^{n} a_{k}$,  记  $f(x)=\sum_{n=1}^{\infty} a_{n} e^{-s_{n} x}$.

\begin{enumerate}
  \item  确定函数  $f(x)$  的定义域  $D$;

  \item  证明 :  级数  $\sum_{n=1}^{\infty} a_{n} e^{-s_{n} x}$  在  $D$  上非一致收敛 ;

  \item  证明 :  函数  $f(x)=\sum_{n=1}^{\infty} a_{n} e^{-s_{n} x}$  在  $D$  上连续 .

\end{enumerate}
 七 、( 本题满分  15  分 )

 求函数 
$$
f(x, y)= \begin{cases}\frac{x^{2} y}{x^{4}+y^{2}}, & x^{2}+y^{2} \neq 0 \\ 0, & x^{2}+y^{2}=0 .\end{cases}
$$
 的全微分 ,  并研究在点  $(0,0)$  处该函数的全微分是否存在 ?  八 、 ( 题满分  15  分 )

 求经过点  $\left(2,1, \frac{1}{3}\right)$  的所有平面中 ,  哪一个平面与坐标平面所围成的立体的体积为最小 ?  并求其最小值 .

 九 、 ( 本题满分  15  分 )

 计算 
$$
\iint_{D} \sqrt{\frac{1-x^{2}-y^{2}}{1+x^{2}+y^{2}}} \mathrm{~d} x \mathrm{~d} y
$$
 其中  $D$  是由圆周  $x^{2}+y^{2}=1$  及坐标轴所围成的在第一象限内的闭区域 .

 十 、 ( 本题满分  15  分 )

 计算曲面积分 
$$
I=\iint_{S}\left(\frac{x}{y} f\left(\frac{x}{y}\right)+x^{3}\right) \mathrm{d} y \mathrm{~d} z+\left(f\left(\frac{x}{y}\right)+y^{3}\right) \mathrm{d} z \mathrm{~d} x+\left(-\frac{z}{y} f\left(\frac{x}{y}\right)+z^{2}\right) \mathrm{d} x \mathrm{~d} y,
$$
 其中  $f(u)$  具有连续的导函数 , $S$  为  $x^{2}+y^{2}+z^{2}=2 R z$  的内侧 . 17.  北京交通大学  2016  年研究生入学考试试题数学分析 

 李扬 

 微信公众号 : sxkyliyang

 一 、 ( 本题满分  15  分 )

 试用数列极限的  “ $\varepsilon-N$ ”  语言证明极限 : $\lim _{n \rightarrow \infty} \sqrt[n]{n}=1$.

 二 、 ( 本题满分  15  分 )

 设函数  $f(x)$  在区间  $[0,+\infty)$  上一致连续 ,  而且对任意  $x \in[0,+\infty)$,  有 
$$
\lim _{n \rightarrow \infty} f(x+n)=0 .
$$
 证明 : $\lim _{x \rightarrow+\infty} f(x)=0$.

 三 、 ( 本题满分  15  分 )

 设函数  $f(x)$  在区间  $[a, c]$  上有二阶导数 , $a<b<c$,  则存在  $\xi \in(a, c)$,  使得 
$$
\frac{f(a)}{(a-b)(a-c)}+\frac{f(b)}{(b-c)(b-a)}+\frac{f(c)}{(c-a)(c-b)}=\frac{1}{2} f^{\prime \prime}(\xi) .
$$
 四 、( 本题满分  15  分 )

\begin{enumerate}
  \item  求  $I_{n}=\int_{0}^{\frac{\pi}{2}} \sin ^{n} x \mathrm{~d} x$,  其中  $n$  为正整数 .

  \item  通过计算积分  $\int_{0}^{1}\left(1-x^{2}\right)^{n} \mathrm{~d} x$,  证明等式  $\sum_{k=0}^{n} \frac{(-1)^{k} C_{n}^{k}}{2 k+1}=\frac{(2 n) ! !}{(2 n+1) ! !}$.

\end{enumerate}
 五 、( 本题满分  15  分 )

 设  $\left\{a_{n}\right\}$  是单调减少的正数数列 .  证明 :  极限  $\lim _{n \rightarrow \infty} a_{n}=a=0$  的充分必要条件是级数  $\sum_{n=1}^{\infty}\left(1-\frac{a_{n+1}}{a_{n}}\right)$  发散 .

 六 、( 本题满分  15  分 )

 设  $f_{n}(x),(n=1,2, \cdots)$  是定义在区间  $(-\infty,+\infty)$  上函数序列 ,  满足下列条件 :

(1)  在任意有限区间  $[a, b]$  上 , $f_{n}(x)(n=1,2, \cdots)$  一致收敛于函数  $f(x)$;

(2)  在任意有限区间  $[a, b]$  上 , $f_{n}(x)(n=1,2, \cdots)$  可积 ;

(3)  存在区间  $(-\infty,+\infty)$  上的可积函数  $F(x)$,  使得  $\left|f_{n}(x)\right| \leqslant F(x),(x \in(-\infty,+\infty)),(n=1,2, \cdots)$.  证明 :
$$
\lim _{n \rightarrow \infty} \int_{-\infty}^{+\infty} f_{n}(x) \mathrm{d} x=\int_{-\infty}^{+\infty} f(x) \mathrm{d} x
$$
 七 、 ( 本题满分  15  分 )

 设  $u=f(x, z)$,  而  $z(x, y)$  是由方程  $z=x+y \varphi(z)$  所确定的函数 ,  求微分  $\mathrm{d} u$.

 八 、 ( 本题满分  15  分 )

 求原点到曲面  $z^{2}=x y+x-y+4$  的最短距离 .

 九 、 ( 本题满分  15  分 )

 十 、 ( 本题满分  15  分 )
$$
I=\oint_{L}\left(y^{2}+z^{2}\right) \mathrm{d} x+\left(z^{2}+x^{2}\right) \mathrm{d} y+\left(x^{2}+y^{2}\right) \mathrm{d} z,
$$
 其中  $L$  是球面  $x^{2}+y^{2}+z^{2}=2 b x$  与柱面  $x^{2}+y^{2}=2 a x(b>a>0), L$  的方向从  $z$  轴正向看去为逆时针方向 .

\section{8. 北京交通大学 2017 年研究生入学考试试题数学分析}
 李扬 

 微信公众号 : sxkyliyang

 一 、( 本题满分  15  分 )

\begin{enumerate}
  \item  叙述数列  $\left\{a_{n}\right\}$  收敛的柯西  (Cauchy)  收敛准则  ( 满分  5  分 ).

  \item  设 

\end{enumerate}
$$
a_{n}=\frac{\sin 1}{2}+\frac{\sin 2}{2^{2}}+\frac{\sin 3}{2^{3}}+\cdots+\frac{\sin n}{2^{n}},(n=1,2,3, \cdots) .
$$
 应用柯西  (Cauchy)  收敛准则 ,  判断数列  $\left\{a_{n}\right\}$  是否收敛 ? ( 10  分 )

 一 、 ( 本题满分  15  分 )

 证明 :  函数  $f(x)=\sin x^{2}$  在区间  $(-\infty,+\infty)$  上是连续函数 ,  但是在此区间上并非一致连续 .

 三 、 ( 本题满分  15  分 )

 设函数  $f(x)$  在闭区间  $[a, b]$  上可导 ,  且  $f^{\prime}(a) f^{\prime}(b)<0$.  证明 :  存在  $\xi \in(a, b)$,  使得  $f^{\prime}(\xi)=0$.  又如果设函数  $f(x)$  在  $(a, b)$  二阶可导 ,  且  $f^{\prime \prime}(x) \neq 0$,  则上述的  $\xi$  是唯一的 .

 四 、 ( 本题满分  15  分 )

 设曲线  $C$  的极坐标方程为  $\rho=a(1-\cos \theta)$,  试求该曲线在  $\theta=\frac{\pi}{2}$  处的切线方程 .

 五 、( 本题满分  15  分 )

 设曲线  $y=\cos x\left(0 \leqslant x \leqslant \frac{\pi}{2}\right)$  与  $x$  轴 , $y$  轴所围图形的面积被曲线  $y=a \sin x(a>0)$  等分 ,  试确定  $a$  之值 .

 六 、 ( 本题满分  15  分 )

 设数列  $\left\{x_{n}\right\} \subset(0,1)$,  而数列  $\left\{x_{n}\right\}$  中各项两两不相等 ,  讨论函数  $f(x)=\sum_{n=1}^{\infty} \frac{\operatorname{sgn}\left(x-x_{n}\right)}{2^{n}}$  的连续性 .

 七 、 ( 本题满分  15  分 )

 设 
$$
f(x, y)= \begin{cases}x y \frac{x^{2}-y^{2}}{x^{2}+y^{2}}, & x^{2}+y^{2} \neq 0 \\ 0, & x^{2}+y^{2}=0\end{cases}
$$
 求  $f_{x y}^{\prime \prime}(0,0)$  与  $f_{y x}^{\prime \prime}(0,0)$.

 八 、 ( 本题满分  15  分 )

 设在力  $\vec{F}=(y z, z x, x y)$  作用下 ,  一质点从原点沿直线移动到第  I  卦限内的椭球面  $\frac{x^{2}}{a^{2}}+\frac{y^{2}}{b^{2}}+\frac{z^{2}}{c^{2}}=1$  上点  $M$,  试问点  $M$  在什么位置时 ,  力  $\vec{F}$  所做的功最大 ?  并求出最大的功 .

 九 、 ( 本题满分  15  分 )

 证明 : 曲线积分 
$$
I=\int_{L} \frac{x y(x \mathrm{~d} y-y \mathrm{~d} x)}{x^{4}+y^{4}}
$$
 在开区域  $x^{2}+y^{2}>0$  内的积分与路径无关 .

 十 、 ( 本题满分  15  分 )

 计算曲面积分 
$$
I=\iint_{\Sigma}\left(x^{2} \cos \alpha+y^{2} \cos \beta+z^{2} \cos \gamma\right) \mathrm{d} S,
$$
 其中  $\Sigma$  为曲面  $x^{2}+y^{2}=z^{2}(0 \leqslant z \leqslant h), \cos \alpha, \cos \beta, \cos \gamma$  为曲面的外法线的方向余弦 .

\section{1. 北京科技大学 2009 年硕士研究生入学考试试题(高等代数 825) 
 李扬 
 微信公众号: sxkyliyang}
 一 、 (20  分 )  计算下列问题 .

$\left|\begin{array}{llll}1 & 2 & 3 & 4 \\ 2 & 3 & 4 & 1 \\ 3 & 4 & 1 & 2 \\ 4 & 1 & 2 & 3\end{array}\right|$

(1)

(2)  设  $D=\left|a_{i j}\right|=\left(\begin{array}{ccccc}2 & 2 & 2 & \cdots & 2 \\ 0 & 1 & 1 & \cdots & 1 \\ 0 & 0 & 1 & \cdots & 1 \\ \vdots & \vdots & \vdots & \vdots & \vdots \\ 0 & 0 & 0 & \vdots & 1\end{array}\right), A_{i j}$  为元素  $a_{i j}$  的代数余子式 ,  计算  $\sum_{i, j=1}^{n} A_{i j}$.

 二 、 (20  分 )  已知方程组 
$$
\left\{\begin{array}{l}
x_{1}+x_{2}-x_{3}=1 \\
2 x_{1}+(a+2) x_{2}-(b+2) x_{3}=3 \\
-3 a x_{2}+(a+2 b) x_{3}=-3
\end{array}\right.
$$
 问 :  当  $a, b$  取什么值时 ,  方程组无解 ?  有唯一解 ?  有无穷多解 ?  并在有无穷多解时 ,  给出这个方程组的通解 .

 三 、 $(20$  分  $)$  设  $A$  为  $n$  阶矩阵 ,  证明 : $A^{2}=A$  的充要条件为  $r(A)+r(E-A)=n$,  其中  $r(A)$  表示矩阵  $A$  的秩 .

 四 、( 20  分 )  设  $A, B$  是  $n$  阶非零矩阵 ,  且有  $A^{2}=A, B^{2}=B, A B=B A=0$.  证明 :

(1) 0,1  必是  $A, B$  的特征值 .

(2)  若  $\xi$  是  $A$  的属于特征值  1  的特征向量 ,  则  $\xi$  也是  $B$  的属于特征值  0  的特征向量 .

 五 、(20  分 )  设  $V_{1}$  与  $V_{2}$  是线性空间  $V$  的子空间 ,  证明 : $V_{1} \cup V_{2}$  是子空间的充要条件为 
$$
V_{1} \cup V_{2}=V_{1}+V_{2}
$$
 六 、(20  分 )  设  $\mathscr{A}$  是  $n$  维线性空间上的线性变换 ,  求证 :
$$
\operatorname{dim} \operatorname{Im} \mathscr{A}^{2}=\operatorname{dim} \operatorname{Im} \mathscr{A}
$$
 的充要条件是  $V=V_{1} \oplus V_{2}$.

 七 、 ( 20  分 )  若  $B$  为正定阵 , $A$  为半正定阵 ,  求证 :
$$
|A+B| \geq|B|
$$
 且等号成立的充要条件是  $A=0$.

 八 、 (10  分 )  设  $a_{i}(1 \leq i \leq n)$  是  $n$  个非负整数 ,  试求多项式 
$$
\sum_{i=1}^{n} x^{a_{i}}
$$
 被  $x^{2}+x+1$  整除的充要条件 .

\section{2. 北京科技大学 2010 年硕士研究生入学考试试题(高等代数 825) 
 李扬 
 微信公众号: sxkyliyang}
 一 、 选择题  ( 15  分 ,  每小题  3  分 )

\begin{enumerate}
  \item  设向量组  $\alpha_{1}, \alpha_{2}, \alpha_{3}$  线性无关 ,  则下列向量组中线性无关的是 \\
(A) $\alpha_{1}+\alpha_{2}, \alpha_{2}+\alpha_{3}, \alpha_{3}+\alpha_{1}$\\
(B) $\alpha_{1}, \alpha_{1}+\alpha_{2}, \alpha_{1}+\alpha_{2}+\alpha_{3}$\\
(C) $\alpha_{1}-\alpha_{2}, \alpha_{2}-\alpha_{3}, \alpha_{3}-\alpha_{1}$\\
(D) $\alpha_{1}+\alpha_{2}, 2 \alpha_{1}+\alpha_{3}, 3 \alpha_{3}+\alpha_{1}$

  \item  设  $A$  为二阶矩阵 , $\alpha_{1}, \alpha_{2}$  均为非零向量且满足  $A \alpha_{1}=0, A \alpha_{2}=\alpha_{2}$,  则线性方程组  $A x=\frac{1}{2} \alpha_{2}$  的通解为  ( $k$  为任意常数 )\\
(A) $k \alpha_{1}+\frac{\alpha_{1}+\alpha_{2}}{2}$\\
(B) $k \alpha_{1}+\frac{\alpha_{1}-\alpha_{2}}{2}$\\
(C) $k \alpha_{2}+\frac{\alpha_{1}+\alpha_{2}}{2}$\\
(D) $k \alpha_{2}+\frac{\alpha_{1} \stackrel{2}{-} \alpha_{2}}{2}$

  \item  设五阶矩阵  $A$  的秩为  3 ,  那么其伴随矩阵  $A^{*}$  的秩为 \\
(A) 0\\
(B) 1\\
(C) 3\\
(D) 5

  \item  设  $A=\left(\begin{array}{lll}a_{11} & a_{12} & a_{13} \\ a_{21} & a_{22} & a_{23} \\ a_{31} & a_{32} & a_{33}\end{array}\right), B\left(\begin{array}{ccc}a_{21} & a_{22} & a_{23} \\ a_{11} & a_{12} & a_{13} \\ a_{31}+a_{11} & a_{32}+a_{12} & a_{33}+a_{13}\end{array}\right), P_{1}=\left(\begin{array}{ccc}0 & 1 & 0 \\ 1 & 0 & 0 \\ 0 & 1\end{array}\right), \quad P_{2}=$ $\left(\begin{array}{lll}1 & 0 & 0 \\ 0 & 1 & 0 \\ 1 & 0 & 1\end{array}\right)$,  则必有 \\
(A) $A P_{1} P_{2}=B$\\
(B) $A P_{1} P_{2}=B$\\
(C) $P_{1} P_{2} A=B$\\
(D) $P_{2} P_{1} A=B$

  \item  设矩阵  $A=\left(\begin{array}{ll}3 & 2 \\ 2 & 0\end{array}\right), B=\left(\begin{array}{cc}-1 & 0 \\ 0 & 4\end{array}\right)$,  则矩阵  $A$  与矩阵  $B$\\
(A)  相似但不合同 \\
(B)  合同但不相似 \\
(C)  相似且合同 \\
(D)  不合同且不相似 

\end{enumerate}
 二 、 填空题 ( 15  分 ,  每小题三分 )

\begin{enumerate}
  \item  在  6  阶行列式中的项  $a_{32} a_{43} a_{14} a_{51} a_{66} a_{25}$  应带有的符号为 

  \item  方程组  $x_{1}+x_{2}+\cdots+x_{n}=0$  的基础解系中解向量的个数为 

  \item  设  3  阶矩阵  $A, B$  相似 ,  矩阵  $A$  的特征值为  $1,2,3$,  则  $|B|=$ $(1,1, b+3,5)^{\prime}$. (1) $a, b$  为何值时 , $\beta$  不能表示成  $\alpha_{1}, \alpha_{2}, \alpha_{3}, \alpha_{4}$  的线性组合 ?

\end{enumerate}
(2) $a, b$  为何值时 , $\beta$  可由  $\alpha_{1}, \alpha_{2}, \alpha_{3}, \alpha_{4}$  唯一线性表示 ?  并写出该表示式 .

 五 、 (15  分 )  若整系数多项式  $p(x)$  与  $f(x)$  有一个公共根 ,  且  $p(x)$  为不可约多项式 ,  那么  $p(x) \mid f(x)$.

 六 、(15  分 )  设  $A$  为  $n$  阶矩阵 ,  则  $\operatorname{rank}(A)=1$  的充分必要条件是存在  $n$  维非零列向量  $\alpha, \beta$,  使得  $A=\alpha \beta^{\prime}$.

 七 、 ( 20  分 )  设  $A=\left(\begin{array}{ccc}-4 & -10 & 0 \\ 1 & 3 & 0 \\ 3 & 6 & 1\end{array}\right)$,  求  $A^{100}$.

 八 、 ( 20  分 )  设线性空间  $V$  的线性变换  $\mathscr{A}$  满足  $\mathscr{A}^{2}=\mathscr{A}$,  称之为幂等变换 ,  证明 :

(1) $V$  中向量  $\beta$  属于  $\mathscr{A}$  的象集  $\operatorname{Im} \mathscr{A}$  当且仅当  $\mathscr{A}(\beta)=\beta$.

(2) $V=\operatorname{Im} \mathscr{A} \oplus \operatorname{ker} \mathscr{A}$,  且  $V$  的任一向量的直和分解为  $\alpha=\mathscr{A}(\alpha)+(\alpha-\mathscr{A}(\alpha))$.

(3)  对任一直和分解  $V=V_{1} \oplus V_{0}$,  存在唯一的幂等变换  $\mathscr{A}$  使得  $V_{1}=\operatorname{Im} \mathscr{A}, V_{0}=$ ker $\mathscr{A}$.

(4)  每个幂等变换都有方阵表示  $\left(\begin{array}{cc}E & 0 \\ 0 & 0\end{array}\right)$.

 九 、 (10  分 )  设  $f\left(x_{1}, x_{2}, \cdots, x_{n}\right)$  是秩为  $n$  的二次型 .  证明 :  存在  $R^{n}$  的一个  $\frac{1}{2}(n-|s|)$  维子空间  $V_{1}(s$  是符号差  $)$  使对任一  $\left(x_{1}, x_{2}, \cdots, x_{n}\right) \in V_{1}$  都有  $f\left(x_{1}, x_{2}, \cdots, x_{n}\right)=0$.

\section{3. 北京科技大学 2011 年硕士研究生入学考试试题(高等代数 825) 
 李扬 
 微信公众号: sxkyliyang}
 一 、 填空题 ( 本题  20  分 ,  每小题  4  分 )

\begin{enumerate}
  \item  已知  $A$  为  $n$  阶方阵且  $|A|=3$,  则  $\left|A^{-1}+2 A^{*}\right|=$

  \item  设  $A$  是  3  阶可逆矩阵 , $A$  的第  1  行与第  2  行交换后得到矩阵  $B$,  则  $A B^{-1}=$

  \item  已知方程组  $\left(\begin{array}{lll}a & 1 & 1 \\ 1 & a & 1 \\ 1 & 1 & a\end{array}\right)\left(\begin{array}{l}x_{1} \\ x_{2} \\ x_{3}\end{array}\right)=\left(\begin{array}{c}1 \\ 1 \\ -2\end{array}\right)$  有无穷解 ,  则  $a=$

  \item  设  $A=\left(\begin{array}{ccc}1 & -1 & 1 \\ a & 4 & -3 \\ -3 & -3 & 5\end{array}\right)$,  且  $A$  有一特征值  $\lambda=6$,  则  $a=$

  \item  从  $R^{3}$  的基  $\alpha_{1}=(1,0,1)^{\prime}, \quad \alpha_{2}=(1,1,-1)^{\prime}, \quad \alpha_{3}=(0,1,0)^{\prime}$  到基  $\beta_{1}=(1,-2,1)^{\prime}, \beta_{2}=(1,2,-1)^{\prime}$, $\beta_{3}=(0,1,-2)^{\prime}$  的过渡矩阵  $P=$

\end{enumerate}
\section{二、选择题(本题 20 分, 每小题 4 分)}
\begin{enumerate}
  \item  设  5  阶矩阵  $A$  的秩为  3 ,  那么其伴随矩阵  $A^{*}$  的秩为 \\
(A) 0\\
(B) 1\\
(C) 3\\
(D) 5

  \item  设向量组  $\alpha_{1}, \alpha_{2}, \alpha_{3}$  线性无关 ,  则下列向量组线性相关的是 \\
(A) $\alpha_{1}+\alpha_{2}, \alpha_{2}+\alpha_{3}, \alpha_{3}+\alpha_{1}$\\
(B) $\alpha_{1}, \alpha_{1}+\alpha_{2}, \alpha_{1}+\alpha_{2}+\alpha_{3}$\\
(C) $\alpha_{1}+\alpha_{2}, \alpha_{2}+\alpha_{3}, \alpha_{3}-\alpha_{1}$\\
(D) $\alpha_{1}+\alpha_{2}, 2 \alpha_{1}+\alpha_{3}, 3 \alpha_{3}+\alpha_{1}$

  \item  设三元非齐次线性方程组  $A x=b$  的两个解为  $\alpha=(1,0,2)^{\prime}, \beta=(1,-1,3)^{\prime}$,  且系数矩阵  $A$  的秩为  2 ,  则对   于任意常数  $k, k_{1}, k_{2}$,  方程组的通解为 \\
(A) $k_{1}(1,0,2)^{\prime}+k_{2}(1,-1,3)^{\prime}$\\
(B) $(1,0,2)^{\prime}+k(1,-1,3)^{\prime}$\\
(C) $(1,0,2)^{\prime}+k(2,-1,5)^{\prime}$\\
(D) $(1,0,2)^{\prime}+k(0,1,-1)^{\prime}$\\
(A) 4\\
(B) 3\\
(C) 2\\
(D) 1\\
(A) 0

\end{enumerate}
 五 、( 本题  15  分 )  证明 :  实对称的正交矩阵的特征值必为  1  或  $-1$

 六 、( 本题  15  分 )  求以三次方程  $x^{3}+x+1=0$  的三个根的平方为根的三次方程   七 、 ( 本题  15  分 )  设  $n>1$  阶实方阵  $A_{n}=\left(\begin{array}{ccccc}x & a & a & \cdots & a \\ a & x & a & \cdots & a \\ a & a & x & \cdots & a \\ \vdots & \vdots & \vdots & \vdots & \vdots \\ a & a & a & \cdots & x\end{array}\right)$.

(1)  求矩阵  $A_{n}$  的秩 ;

(2)  矩阵  $A_{n}$  何时是正定的 ?

 八 、 ( 本题  15  分 )  设  $\alpha_{1}, \alpha_{2}, \cdots, \alpha_{n}$  是  $n$  维线性空间  $V$  的一组基 ,  向量  $\beta \in V$  可以由这组基中的任意  $n-1$  个线   性表示 ,  证明  $\beta=0$

 九 、 ( 本题  20  分 )  设线性空间  $V=W_{1} \oplus W_{2} \oplus \cdots \oplus W_{s}$.  证明 :  存在  $V$  的线性变换  $\mathscr{A}_{1}, \mathscr{A}_{2}, \cdots, \mathscr{A}_{s}$.  使得 

(1) $\mathscr{A}_{i}^{2}=\mathscr{A}_{i}, 1 \leq i \leq s$;

(2) $\mathscr{A}_{i} \mathscr{A}_{j}=0, i \neq j$;

(3) $\mathscr{A}_{1}+\mathscr{A}_{2}+\cdots+\mathscr{A}_{s}=I$  为恒等变换 ;

(4) $\operatorname{Im} \mathscr{A}_{i}=W_{i}, 1 \leq i \leq s$.

\section{4. 北京科技大学 2012 年硕士研究生入学考试试题(高等代数 825) 
 李扬 
 微信公众号: sxkyliyang}
 一 、 (15  分 )  判断 
$$
f(x)=x^{5}-3 x^{4}+5 x^{3}-7 x^{2}+6 x-2
$$
 有无重因式 ,  若有 ,  请求出  $f(x)$  的所有重因式并指出其重数 .

 二 、 (20  分 )  设矩阵  $A=\left(\begin{array}{ccc}1 & 1 & -1 \\ 2 & 1 & 0 \\ 1 & -1 & 1\end{array}\right), B=\left(\begin{array}{ccc}2 & 1 & -2 \\ 1 & -1 & -1\end{array}\right)$

(1)  计算矩阵  $A B^{\prime}$  以及行列式  $\left|A B^{\prime} B A^{\prime}\right|$;

(2)  求矩阵  $C$,  使得  $C A=B$.

 三 、 $(20$  分  $)$  研究  $k$  取何值时 ,  线性方程组 
$$
\left\{\begin{array}{l}
k x_{1}+x_{2}+x_{3}=5 \\
3 x_{1}+2 x_{2}+k x_{3}=18-5 k \\
x_{2}+2 x_{3}=2
\end{array}\right.
$$
(1)  有唯一解 ;

(2)  有无穷多解 ;

(3)  无解 .

 四 、 $\left(20\right.$  分 )  设二次型  $f\left(x_{1}, x_{2}, x_{3}\right)=5 x_{1}^{2}+5 x_{2}^{2}+c x_{3}^{2}-2 x_{1} x_{2}+6 x_{1} x_{3}-6 x_{2} x_{3}$  的秩为  2

(1)  求参数  $c$  使得该二次型的秩等于  2 ;

(2)  写出该二次型的矩阵  $A$;

(3)  求一个正交矩阵  $P$  和一个对角矩阵  $\Lambda$  使得  $P^{-1} A P=\Lambda$;

(4)  求一个非退化线性替换  $x=C y$  把该二次型化为标准形 .

 五 、 $\left(20\right.$  分 )  设  $V=R^{4}, V_{1}=L\left(\alpha_{1}, \alpha_{2}, \alpha_{3}\right), V_{2}=L\left(\beta_{1}, \beta_{2}\right)$.  其中  $\alpha_{1}=(1,2,-1,-3), \alpha_{2}=(-1,-1,2,1)$, $\alpha_{3}=(-1,-3,0,5), \beta_{1}=(-1,0,4,-2), \beta_{2}=(0,5,9,-14)$.  求 

(1) $V_{1}$  的维数与一组基 ;

(2) $V_{2}$  的维数与一组基 ;

(3) $V_{1}+V_{2}$  的维数与一组基 ;

(4) $V_{1} \cap V_{2}$  的维数与一组基 ;

 六 、 (15  分 )  设  $A=\left(\begin{array}{ccc}4 & -1 & 2 \\ -9 & 4 & -6 \\ -9 & 3 & -5\end{array}\right)$

(1)  求  $A$  的初等因子 

(2)  求出  $A$  的  Jordan  标准形 

 七 、 (10  分 )  设  $\mathscr{A}$  是线性空间  $V$  上的线性变换 ,  而  $\xi \in V$,  设  $\xi, \mathscr{A}(\xi), \cdots, \mathscr{A}^{k-1}(\xi)$  都不等于零 ,  但  $\mathscr{A}^{k}(\xi)=0$.  证明 : $\xi, \mathscr{A}(\xi), \cdots, \mathscr{A}^{k-1}(\xi)$  线性无关 .

 八 、(15  分 )  设  $\mathscr{A}$  是线性空间  $V$  上的线性变换 ,  满足  $\mathscr{A}^{2}-3 \mathscr{A}+2 I=0$,  其中  $I$  是恒等变换 .  证明  $\mathscr{A}$  的特征值   只能是  1  或  $2 .$

 九 、 ( 15  分 )  设  $A, B$  都是  $n$  阶方阵 ,  证明 :  如果  $A^{2}=E$,  则  $\operatorname{rank}(A+E)+\operatorname{rank}(A-E)=n$.  其中  $E$  为  $n$  阶单   位矩阵 , $\operatorname{rank}(\cdot)$  表示矩阵的秩 .

\section{5. 北京科技大学 2013 年硕士研究生入学考试试题(高等代数 825) 
 李扬 
 微信公众号: sxkyliyang}
 一 、 填空题 ( 每小题  4  分 ,  共  20  分 )

\begin{enumerate}
  \item  设  $f(x)=\left|\begin{array}{cccc}x+1 & x+10 & 111 & 7 \\ 0 & x+2 & 0 & 0 \\ 0 & 78 & x-7 & 6 \\ 0 & 99 & 10 & x+8\end{array}\right|$,  则  $f(x)$  中  $x^{3}$  的系数是 \_\_ 常数项等于 

  \item  设四阶矩阵  $A$  的初等因子为  $(\lambda-1)^{2},(\lambda-2)^{2}$,  则  $A$  的  Jordan  标准形是 

  \item  设  $\mathscr{A}$  为线性空间  $V$  的一个线性变换 ,  且  $\mathscr{A}^{2}=\mathscr{A}$,  则  $\mathscr{A}$  的特征值只能是 

  \item  设  $u$  是  $n$  维列向量 , $(u, u)=1, H=E-2 u u^{\prime}$,  则  $\lambda=1$  是  $H$  的 \_ 重特征值 .

  \item  设  $A, B$  分别是  $k$  阶和  $r$  阶可逆矩阵 , $D=\left(\begin{array}{cc}A & 0 \\ C & B\end{array}\right)$,  则  $D^{*}=$

\end{enumerate}
 二 、 $\left(10\right.$  分 )  设  $f(x)=x^{4}+x^{3}-3 x^{2}-4 x-a, g(x)=x^{3}+x^{2}-x-1$.  求  $(f(x), g(x))$.

 三 、( 20  分 )  设线性变换  $\mathscr{A}$  在三维线性空间  $V$  的一组基  $\varepsilon_{1}, \varepsilon_{3}, \varepsilon_{3}$  下的矩阵是  $A=\left(\begin{array}{cccc}1 & 2 & -1 \\ 2 & 1 & 0 \\ 3 & 0 & 1\end{array}\right)$

(1)  求  $\mathscr{A}$  在基  $\eta_{1}, \eta_{2}, \eta_{3}$  下的矩阵 ,  其中  $\left\{\begin{array}{c}\eta_{1}=2 \varepsilon_{1}+\varepsilon_{2}+3 \varepsilon_{3} \\ \eta_{2}=\varepsilon_{1}+\varepsilon_{2}+2 \varepsilon_{3} \\ \eta_{3}=-\varepsilon_{1}+\varepsilon_{2}+\varepsilon_{3}\end{array}\right.$

(2)  求  $\mathscr{A}$  的值域  $\mathscr{A}(V)$  和核  $\mathscr{A}^{-1}(0)$;

(3)  把  $\mathscr{A}^{-1}(0)$  的基扩充为  $V$  的基 ,  并求  $\mathscr{A}$  在这组基下的矩阵 .

 四 、 ( 20  分 )  设  $S, A$  分别是  $P^{n \times n}$  中的对称矩阵和反对称矩阵构成的子空间 .  证明 
$$
P^{n \times n}=S \oplus A
$$
 五 、 $(20$  分  $)$  设实对称矩阵  $A=\left(\begin{array}{cccc}-1 & -3 & 3 & -3 \\ -3 & -1 & -3 & 3 \\ 3 & -3 & -1 & -3 \\ -3 & 3 & -3 & -1\end{array}\right)$.

(1)  求可逆矩阵  $T$,  使得  $T^{\prime} A T$  成对角阵 ,  并写出该对角矩阵 .

(2)  求一个非退化线性替换把二次型  $f(x)=x^{\prime} A x$  化为标准形 .

 六 、 $(20$  分  $)$  设  $A$  是一个  $n$  阶方阵 ,  证明 : $\operatorname{rank}\left(A^{*}\right)=\left\{\begin{array}{cc}n & \operatorname{rank}(A)=n \\ 1 & \operatorname{rank}(A)=n-1 \\ 0 & \operatorname{rank}(A)<n-1\end{array}\right.$

 七 、 ( 20  分 )  如果齐次线性方程组  $\left\{\begin{array}{c}a_{11} x_{1}+a_{12} x_{2}+\cdots+a_{1 n} x_{n}=0 \\ a_{21} x_{1}+a_{22} x_{2}+\cdots+a_{2 n} x_{n}=0 \\ \cdots \cdots \cdots \cdots \\ a_{n 1} x_{1}+a_{n 2} x_{2}+\cdots+a_{n n} x_{n}=0\end{array}\right.$  的系数矩阵为  $A, M_{i}$  是矩阵  $A$  中划去 

 第  $i$  列所得到的  $(n-1) \times(n-1)$  矩阵的行列式 ,  证明 : $\left(M_{1},-M_{2}, \cdots,(-1)^{n-1} M_{n}\right)^{\prime}$  是该方程组的一个解 .

 八 、 (20  分 )  设  $P[x]_{3}$  是次数不超过  3  的多项式全体连同  0  多项式构成的线性空间 , $f(x)=a_{0}+a_{1} x+a_{2} x^{2}+a_{3} x^{3} \in$ $P[x]_{3}$,  现有  $P[x]_{3}$  的线性变换  $\mathscr{A}$ :
$$
\mathscr{A}(f(x))=\left(a_{0}-2 a_{1}\right)+\left(-3 a_{0}+2 a_{1}\right) x+\left(2 a_{2}-3 a_{3}\right) x^{2}+\left(-4 a_{2}+3 a_{3}\right) x^{3}
$$

\section{6. 北京科技大学 2014 年硕士研究生入学考试试题(高等代数 825) 
 李扬 
 微信公众号: sxkyliyang}
 一 、 (15  分 )  设  $f(x)=a_{n} x^{n}+a_{n-1} x^{n-1}+\cdots+a_{1} x+a_{0}$  是一个整系数多项式 ,  证明 :  如果  $a_{n}+a_{n-1}+\cdots+a_{1}+a_{0}$  是奇数 ,  则  $f(x)$  既不能被  $x-1$  整除 ,  又不能被  $x+1$  整除 .

 二 、 $(20$  分  $)$  计算行列式  $D_{n}=\left|\begin{array}{cccc}1+a_{1} & 1 & \cdots & 1 \\ 1 & 1+a_{2} & \cdots & 1 \\ \cdots & \cdots & \cdots & \cdots \\ 1 & 1 & \cdots & 1+a_{n}\end{array}\right|$,  其中  $a_{1} a_{2} \cdots a_{n} \neq 0$.

 三 、 (15  分 )  设  $A$  为  $n$  阶方阵 ,  证明 : $\operatorname{rank}(A)=1$  的充分必要条件是存在  $n$  维非零列向量  $\alpha, \beta$,  使得  $A=\alpha \beta^{\prime}$,  其中  $\operatorname{rank}(A)$  表示  $A$  的秩 .

 四 、 (20  分 )

(1)  若  $A, B$  都是  $n$  阶方阵 ,  证明 : $\operatorname{rank}(A)+\operatorname{rank}(B) \leq n+\operatorname{rank}(A B)$.

(2)  若  $A_{1}, A_{2}, \cdots, A_{s}(s \geq 2)$  都是  $n$  阶方阵 ,  证明 :
$$
\sum_{i=1}^{s} \operatorname{rank}\left(A_{i}\right) \leq n(s-1)+\operatorname{rank}\left(A_{1} A_{2} \cdots A_{s}\right)
$$
 五 、 后面题目暂无 .

\section{7. 北京科技大学 2009 年硕士研究生入学考试试题(数学分析613) 
 李扬 
 微信公众号: sxkyliyang}
 一 、 (15  分 )  设  $f(x)$  在区间  $[a,+\infty)$  上二阶可导 ,  且  $f(a)>0, f^{\prime}(a)<0$,  而当  $x>a$  时 , $f^{\prime \prime}(x) \leq 0$.  证明 :  在  $(a,+\infty)$  内 ,  方程  $f(x)=0$  有且仅有一个实根 .

 二 、(15  分 )  设  $f(x)$  有连续的二阶导数 , $f(0)=f^{\prime}(0)=0$,  且  $f^{\prime \prime}(x)>0$,  求  $\lim _{x \rightarrow 0^{+}} \frac{\int_{0}^{u(x)} f(t) \mathrm{d} t}{\int_{0}^{x} f(t) \mathrm{d} t}$,  其中  $u(x)$  是曲线  $y=f(x)$  在点  $(x, f(x))$  处的切线在  $x$  轴上的截距 .

 三 、 $\left(15\right.$  分 )  函数  $f(x)$  在区间  $[a, b]$  上连续 , $F(x)=\int_{a}^{x} f(t) \mathrm{d} t$,  证明  $F(x)$  可导 ,  且  $F^{\prime}(x)=f(x)$.

 四 、(15  分 )  设函数  $f(x)$  在区间  $[0,1]$  上连续 ,  证明 :
$$
\left(\int_{0}^{1} \frac{f(x)}{t^{2}+x^{2}} \mathrm{~d} x\right)^{2} \leq \frac{\pi}{2 t} \int_{0}^{1} \frac{f^{2}(x)}{t^{2}+x^{2}} \mathrm{~d} x \quad(t>0)
$$
 五 、 (15  分 )  用有限覆盖定理证明根的存在性定理 .

 六 、( 15  分 )  计算二重积分  $\iint_{D} e^{\frac{x-y}{x+y}} \mathrm{~d} x \mathrm{~d} y$,  其中  $D$  是由  $x=0, y=0, x+y=1$  所围成的区域 .

 七 、 ( 15  分 )  讨论函数序列  $\left\{f_{n}(x)=n^{2} x e^{-n^{2} x^{2}}\right\}, x \in[0,1]$  的一致收敛性 .

 八 、 ( 15  分 )  求幂级数  $\sum_{n=1}^{\infty} \frac{x^{2 n}}{n-3^{2 n}}$  的收敛半径和收敛区域 .

 九 、 (15  分 )  求  $\int_{L} \frac{(x-y) \mathrm{d} x+(x+y) \mathrm{d} y}{x^{2}+y^{2}}$,  其中  $L$  为  $y=1-2 x^{2}$  自点  $A(-1,-1)$  至点  $B(1,-1)$  的弧段 .

 十 、 (15  分 )  证明 : $S e^{\frac{-9}{2}} \leq \int_{c} e^{-\sqrt{x^{2} y}} \mathrm{~d} s \leq S$,  其中  $c$  是直线  $3 x+4 y-12=0$  介于两坐标轴之间的线段 .

\section{8. 北京科技大学 2010 年硕士研究生入学考试试题(数学分析613) 
 李扬 
 微信公众号: sxkyliyang}
 一 、(15  分 ) (1)  求极限  $\lim _{x \rightarrow 0} \frac{\int_{0}^{x} e^{-t^{2}} \mathrm{~d} t-x}{\sin x-x}$; (2) $\lim _{x \rightarrow 0} \frac{\int_{0}^{\sin x} \ln (1+t) \mathrm{d} t}{\sqrt{1+x^{2}}-1}$.

 二 、 ( 15  分 )  设  $S$  为有界数集 ,  证明 :  若  $\sup S=a \notin S$,  则存在严格增数列  $\left\{x_{n}\right\} \subset S$,  使得  $\lim _{x \rightarrow \infty} x_{n}=a$.

 三 、 (15  分 )  若函数  $f(x)$  在点  $a$  处有连续的二阶导数 ,  证明 :
$$
\lim _{h \rightarrow 0} \frac{f(a-h)-2 f(a)+f(a+h)}{h^{2}}=f^{\prime \prime}(a)
$$
 四 、 (15  分 )  设函数  $f(x)$  在  $[0,1]$  上可微 ,  且 
$$
f(1)-2 \int_{0}^{\frac{1}{2}} x f(x) \mathrm{d} x=0
$$
 存在 ,  则存在  $\xi \in(0,1)$,  使  $f^{\prime}(\xi)=\frac{f(\xi)}{\xi}$.

 五 、 (15  分 )  设  $z=f(x, y)$  满足  $\frac{\partial^{2} f}{\partial y^{2}}=2 x, f(x, 1)=0, \frac{\partial f(x, 0)}{\partial y}=\sin x$,  求  $f(x, y)$.

 六 、(15  分 )  设级数  $\sum_{n=1}^{\infty} a_{n}^{2}$  和  $\sum_{n=1}^{\infty} b_{n}^{2}$  收敛 ,  求证 : (1)  级数  $\sum_{n=1}^{\infty}\left|a_{n} b_{n}\right|$  收敛 ; (2)  级数  $\sum_{n=1}^{\infty} \frac{a_{n}}{n}$  收敛 .

 七 、 (15  分 )  计算  $\iint_{\Sigma} x^{2} \mathrm{~d} y \mathrm{~d} z+y^{2} \mathrm{~d} z \mathrm{~d} x+z^{2} \mathrm{~d} x \mathrm{~d} y$,  其中  $\sum: x^{2}+y^{2}+z^{2}=1(z \geq 1)$  取外侧 .

 八 、( 15  分 )  设函数  $f(x)$  在实数轴  $R$  上一致连续 ,  证明存在正数  $A, B$,  使得对任意  $x \in R$,  都有 
$$
|f(x)| \leq A|x|+B
$$
 九 、 (15  分 )  设函数  $f(x)$  在闭区间  $[a, b]$  上有定义 ,  且满足 

\begin{enumerate}
  \item  任意  $x_{0} \in[a, b], f\left(x_{0}\right) \in[a, b]$;

  \item  存在  $\eta \in(0,1)$,  使得任意  $x_{1}, x_{2} \in[a, b]$,  有  $\left|f\left(x_{1}\right)-f\left(x_{2}\right)\right| \leq \eta\left|x_{1}-x_{2}\right|$.

\end{enumerate}
 证明对任意数列  $\left\{a_{n}\right\}: a_{1} \in[a, b]$  且  $a_{n+1}=f\left(a_{n}\right)$,  极限  $\lim _{n \rightarrow \infty} a_{n}$  收敛 .

 十 、(15  分 )  设函数  $f(x)$  在  $[0,+\infty)$  上单调 ,  且  $\int_{0}^{+\infty} f(x) \mathrm{d} x$  存在 ,  则  $\lim _{h \rightarrow 0^{+}} h \sum_{k=1}^{\infty} f(k h)=\int_{0}^{+\infty} f(x) \mathrm{d} x$.

\section{9. 北京科技大学 2011 年硕士研究生入学考试试题(数学分析613) 
 李扬 
 微信公众号: sxkyliyang}
 一 、(15  分 )  设函数  $f(x)$  在闭区间  $[0,1]$  上连续 ,  在开区间  $(0,1)$  内可微 ,  且  $f(0)=f(1)=0, f\left(\frac{1}{2}\right)=1$,  证明 :

(1)  存在  $\xi \in\left(\frac{1}{2}, 1\right)$,  使得  $f(\xi)=\xi$;

(2)  存在  $\eta \in(0, \xi)$,  使得  $f^{\prime}(\eta)=f(\eta)-\eta+1$.

 二 、 $(15$  分  $)$  求函数  $f(x)=\int_{1}^{x^{2}}\left(x^{2}-t\right) e^{-t^{2}} \mathrm{~d} t$  的单调区间与极值 .

 三 、 (15  分 )  设  $f(x)$  在  $[a, b]$  上二次连续可微 ,  且  $f\left(\frac{a+b}{2}\right)=0$,  证明 :
$$
\left|\int_{a}^{b} f(x) \mathrm{d} x\right| \leq \frac{M(b-a)^{3}}{24}
$$
 其中  $M=\sup _{a \leq x \leq b}\left|f^{\prime \prime}(x)\right|$.

 四 、(15  分 )  求积分  $\int \frac{5 \sin x+2 \cos x}{\sin x+3 \cos x} \mathrm{~d} x$.

 五 、 ( 15  分 )  已知  $f(x, y)=\left\{\begin{array}{cc}\frac{\sin x y}{\sqrt{x^{2}+y^{2}}}, & x^{2}+y^{2} \neq 0 \\ 0, & x^{2}+y^{2}=0\end{array}\right.$,  试讨论函数  $f(x, y)$  在原点  $(0,0)$  处是否连续 ..

 六 、 $\left(15\right.$  分 )  求空间一点  $\left(x_{0}, y_{0}, z_{0}\right)$  到平面  $A x+B y+C z+D=0$  的最短距离 .

 七 、 (15  分 )  证明 :  反常积分  $\int_{0}^{+\infty} e^{-x^{2} y} \mathrm{~d} y$  在  $[a, b](a>0)$  上一致收敛 .

 八 、 $(15$  分  $)$  计算 
$$
\iint_{S} x^{2} \mathrm{~d} y \mathrm{~d} z+y^{2} \mathrm{~d} z \mathrm{~d} x+z^{2} \mathrm{~d} x \mathrm{~d} y
$$
 其中  $S$  是球面  $(x-a)^{2}+(y-b)^{2}+(z-c)^{2}=R^{2}$,  并取外侧为正向 .

 九 、 $\left(15\right.$  分 )  设  $f(x)$  为区间  $[a, b]$  上的连续函数 ,  且  $x_{1}, x_{2}, \cdots, x_{n} \in(a, b)$.  证明 :  存在  $\xi \in(a, b)$,  使得 
$$
f(\xi)=\frac{1}{n^{2}} \sum_{k=1}^{n}(2 k-1) f\left(x_{k}\right)
$$
 十 、 (15  分 )  判断下列级数是绝对收敛 ,  条件收敛还是发散 .\\
(1) $\sum_{n=1}^{\infty} 3^{n} \sin \frac{\pi}{5^{n}}$\\
(2) $\sum_{n=1}^{\infty} \frac{(-1)^{n}}{2^{\ln n}}$

\section{0. 北京科技大学 2012 年硕士研究生入学考试试题(数学分析613) 
 李扬 
 微信公众号: sxkyliyang}
 一 、 (20  分 )

(1)  求极限  $\lim _{n \rightarrow \infty} \frac{1}{n} \sqrt[n]{(n+1)(n+2) \cdots(2 n)}$.

(2)  证明积分  $\int_{0}^{\frac{\pi}{2}} \ln (\sin x) \mathrm{d} x$  收敛且求其值 .

 二 、 $(20$  分 )

(1)  证明 :  对于  $\lambda>0$,  级数  $\sum_{n=1}^{\infty}(-1)^{n} \tan \left(\sqrt{n^{2}+\lambda} \pi\right)$  都收玫 .

(2)  设  $f(x)$  连续 ,  求极限  $\lim _{x \rightarrow a} \frac{x}{x-a} \int_{a}^{x} f(t) \mathrm{d} t$.

 三 、 ( 15  分 )  已知给定函数 
$$
f(x)= \begin{cases}(x-a)^{m} \sin \frac{1}{x-a}, & x \neq a \\ 0, & x=a\end{cases}
$$
 其中  $m$  为正整数 ,  试讨论  $f(x)$  在  $x=a$  的连续性与可导性以及导函数  $f^{\prime}(x)$  在  $x=a$  的连续性 .

 四 、 ( 15  分 )  设函数  $f(x)$  在  $[0, b]$  上连续 ,  且  $\int_{0}^{x} f(t) \mathrm{d} t \geq b f(x) \geq 0, \forall x \in[0, b]$,  证明 : $f(x) \equiv 0$.

 五 、 $\left(15\right.$  分 )  设  $f(x)$  在  $[a, b]$  连续 , $x_{1}, x_{2}, \cdots, x_{n} \in[a, b]$.  证明 :  存在  $\xi \in[a, b]$,  使  $f(\xi)=\frac{1}{n} \sum_{i=1}^{n} f\left(x_{i}\right)$.

 六 、 ( 15  分 )  已知曲线 
$$
C:\left\{\begin{array}{l}
x^{2}+y^{2}-2 z^{2}=0 \\
x+y+3 z=5
\end{array}\right.
$$
 求曲线  $C$  距离  $X O Y$  面最远的点和最近的点 .

 七 、 $\left(15\right.$  分 )  设  $f(x)$  在  $[a, b]$  连续 ,  在  $(a, b)$  可导 ,  且  $f^{\prime}(x) \neq 0$.  试证明 :  存在  $\xi, \eta \in(a, b)$,  使 
$$
\frac{f^{\prime}(\xi)}{f^{\prime}(\eta)}=\frac{e^{b}-e^{a}}{b-a} e^{-\eta}
$$
 八 、 $\left(15\right.$  分 )  设  $f(x)$  在区间  $[-1,1]$  上连续且为奇函数 ,  区域  $D$  由曲线  $y=4-x^{2}$  与  $y=-3 x, x=1$  所围成 ,  求 
$$
I=\iint_{D}\left(1+f(x) \ln \left(y+\sqrt{1+y^{2}}\right)\right) \mathrm{d} x \mathrm{~d} y .
$$
 九 、 (10  分 )  试利用闭区间套定理证明数列  $\left\{a_{n}\right\}$  收敛的充要条件是 :  对任意的  $\varepsilon>0$,  存在  $N>0$,  使得当  $m, n>N$  时 , $\left|a_{m}-a_{n}\right|<\varepsilon$.

 十 、 (10  分 )

(1)  设  $a$  为不是整数的实参数 ,  计算函数  $\cos a x$  在  $[-\pi, \pi]$  的三角级数展开式 ;

(2)  证明 :
$$
\frac{1}{\sin t}=\frac{1}{t}+\sum_{n=1}^{+\infty}(-1)^{n}\left(\frac{1}{t-n \pi}+\frac{1}{t+n \pi}\right)
$$
 其中  $t$  不是  $\pi$  的整数倍 ;

(3)  利用上面结果计算广义积分 :
$$
\int_{0}^{+\infty} \frac{\sin x}{x} \mathrm{~d} x
$$

\section{1. 北京科技大学 2013 年硕士研究生入学考试试题(数学分析613) 
 李扬 
 微信公众号: sxkyliyang}
 一 、 (20  分 )

(1)  设  $z=f(x, y, u)=x y+x F(u)$,  其中  $F$  为可微函数 ,  且  $u=\frac{y}{x}$,  证明 : $x \frac{\partial z}{\partial x}+y \frac{\partial z}{\partial y}=z+x y$

(2)  设  $u=x^{y^{2}}$,  求  $\frac{\partial u}{\partial z}, \frac{\partial^{2} u}{\partial z^{2}}$

 二 、 $(20$  分 )

(1)  设  $f(x)$  在  $[a, b]$  上连续 , $\int_{a}^{b} f(x) \mathrm{d} x=\frac{1}{4}+\int_{a}^{b} f\left(x^{2}\right) \mathrm{d} x$,  则存在  $\xi \in(a, b)$  使得  $f(\xi)-f\left(\xi^{2}\right)=\frac{1}{4(b-a)}$.

(2)  求极限  $\lim _{x \rightarrow \infty}\left(\int_{0}^{x} e^{t} \mathrm{~d} t\right)^{\frac{1}{x}}$

 三 、 $\left(20\right.$  分  )  设  $f(x)=\left\{\begin{array}{cc}\frac{g(x)-e^{-x}}{x}, & x \neq 0 \\ 0, & x=0\end{array}\right.$,  其中  $g(x)$  有二阶连续的导数 ,  且  $g(0)=1, g^{\prime}(0)=-1$,  求  $f^{\prime}(x)$,  并讨论  $f^{\prime}(x)$  在  $(-\infty,+\infty)$  上的连续性 .

 四 、( 15  分 )  设  $f(x)$  在  $[0,1]$  上连续可微 ,  且  $f(0)=0, f(1)=1$,  求证 :

(1) $\forall x \in[0,1],\left|f(x)-f^{\prime}(x)\right| \geq\left(e^{-x} f(x)\right)^{\prime}$

(2) $\int_{0}^{1}\left|f(x)-f^{\prime}(x)\right| \mathrm{d} x \geq e^{-1}$

 五 、 ( 15  分 )  若  $\left\{\left[a_{n}, b_{n}\right]\right\}$  是一个闭区间套 ,  即  $\left[a_{n+1}, b_{n+1}\right] \subset\left[a_{n}, b_{n}\right], n=1,2 \cdots$,  且  $\lim _{n \rightarrow \infty}\left(b_{n}-a_{n}\right)=0$,  证明 :  存在唯一点  $\xi$,  使得  $\xi \in\left[a_{n}, b_{n}\right], n=1,2, \cdots .$.

 六 、 ( 15  分 )  计算二重积分 
$$
\iint_{D} \frac{\sin y}{y} \mathrm{~d} x \mathrm{~d} y
$$
 其中  $D$  是由曲线  $y=x$  以及  $x=y^{2}$  所围成的闭区域 .

 七 、 (15  分 )  计算 
$$
\iiint_{\Omega} \frac{1}{1+x^{2}+y^{2}} \mathrm{~d} x \mathrm{~d} y \mathrm{~d} z
$$
 其中  $\Omega$  是由抛物面  $x^{2}+y^{2}=4 z$  与平面  $z=h>0$  围成的空间区域 .

 八 、 $\left(10\right.$  分 )  设  $f_{0}(x)$  在  $[0,1]$  上连续 ,  定义函数序列 :
$$
f_{n+1}(x)=\int_{0}^{x} f_{n}(x) \mathrm{d} t, \quad n=0,1,2, \cdots
$$
 证明 :  函数项级数  $\sum_{n=1}^{\infty} f_{n}(x)$  在  $[0,1]$  上一致收敛 .

 九 、 $\left(10\right.$  分 )  设函数  $y=f(x)$  的二阶可导 ,  且  $f^{\prime \prime}(x)>0, f(0)=0, f^{\prime}(0)=0$,  求 
$$
\lim _{x \rightarrow 0} \frac{x^{3} f(u)}{f(x) \sin ^{3} u},
$$
 其中  $u$  是曲线  $y=f(x)$  在点  $P(x, f(x))$  处的切线在  $x$  轴上的截距 .

 十 、 (10  分 )  计算曲面积分 
$$
I=\iint_{\Sigma}\left(x+z^{2}\right) \mathrm{d} y \mathrm{~d} z-z \mathrm{~d} x \mathrm{~d} y,
$$
 其中  $\sum$  是旋转抛物面  $z=\frac{1}{2}\left(x^{2}+y^{2}\right)$  介于平面  $z=0$  和  $z=2$  之间的部分的下侧 .

\section{2. 北京科技大学 2014 年硕士研究生入学考试试题(数学分析613) 
 李扬 
 微信公众号: sxkyliyang}
 一 、 (15  分 )

(1)  计算极限  $\lim _{x \rightarrow 0} \frac{\int_{0}^{x^{2}} \cos x \mathrm{~d} x}{\ln \left(1+x^{2}\right)}$;

(2)  设  $a_{1}>0, a_{n+1}=\frac{2\left(1+a_{n}\right)}{2+a_{n}},(n=1,2, \cdots)$,  证明 : $\lim _{n \rightarrow \infty} a_{n}$  存在 ,  并求该极限 .

 二 、 $(15$  分 )

(1)  设  $u=x^{2}+y^{2}+z^{2}$,  其中  $z=f(x, y)$  是由方程  $x^{3}+y^{3}+z^{3}=3 x y z$  所确定的隐函数 ,  求  $u_{x}$.

(2)  设 
$$
\left\{\begin{array}{l}
x=u+v \\
y=u^{2}+v^{2} \\
z=u^{3}+v^{3}
\end{array}\right.
$$
 求  $\frac{\partial z}{\partial x}$

 三 、 $\left(15\right.$  分 )  设  $f(x)$  在  $[0,2]$  上连续 ,  且  $f(0)=f(2)$,  证明  $\exists x_{0} \in[0,1]$,  使  $f\left(x_{0}\right)=f\left(x_{0}+1\right)$.

 四 、(15  分 )  设  $f(x)$  为偶函数 ,  试证明 :
$$
\iint_{D} f(x-y) \mathrm{d} x \mathrm{~d} y=2 \int_{0}^{2 a}(2 a-u) f(u) d u
$$
 其中  $D:|x| \leq a,|y| \leq a(a>0)$.

 五 、 ( 15  分 )  设  $f(x)$  在区间  $[0,1]$  上具有二阶连续导数 ,  且对一切  $x \in[0,1]$,  均有  $|f(x)|<M,\left|f^{\prime \prime}(x)\right|<M$.  证   明 :  对一切  $x \in[0,1]$,  成立  $\left|f^{\prime}(x)\right|<3 M$.

 六 、 (15  分 )  设  $a>0, f(x)$  是定义在区间  $[-a, a]$  上的连续偶函数 .

(1)  证明 : $\int_{-a}^{a} \frac{f(x)}{1+e^{x}} \mathrm{~d} x=\int_{0}^{a} f(x) \mathrm{d} x$;

(2)  计算积分  $\int_{-\frac{\pi}{2}}^{\frac{\pi}{2}} \frac{\cos ^{3} x}{1+e^{x}} \mathrm{~d} x$.

 七 、 (15  分 )

(1)  证明 :  级数  $\sum_{n=1}^{+\infty} \frac{x}{1+n^{4} x^{2}}$  在  $[0,+\infty)$  上一致收敛 ;

(2)  求级数  $\sum_{n=1}^{+\infty} \frac{(-1)^{n} 8^{n}}{n \ln \left(n^{3}+n\right)} x^{3 n-2}$  的收敛域 .

 八 、(15  分 )  证明 :  若  $f(x, y)$  在矩形区域  $D$  满足 :
$$
\left|f\left(x_{1}, y\right)-f\left(x_{2}, y\right)\right| \leq L_{1}\left|x_{1}-x_{2}\right|,\left|f\left(x, y_{1}\right)-f\left(x, y_{2}\right)\right| \leq L_{2}\left|y_{1}-y_{2}\right|,
$$
 其中  $L_{1}, L_{2}$  是正的常数 ,  则函数  $f(x, y)$  在  $D$  一致连续 .

 九 、 (15  分 )  设对于半空间  $x>0$  内任意的分片光滑的有向封闭曲面  $\sum$,  都有 
$$
\oiint_{\Sigma} f(x) \mathrm{d} y \mathrm{~d} z-\frac{x y}{1+x^{2}} \mathrm{~d} z \mathrm{~d} x-\mathrm{d} x \mathrm{~d} y=0
$$
 其中函数  $f(x)$  在  $[0,+\infty)$  上具有一阶连续导数 ,  且  $f(0)=1$,  求  $f(x)$.

 十 、 (15  分 )  设  $|f(x)| \leq \pi, f^{\prime}(x) \geq m>0(a \leq x \leq b)$,  证明 : $\left|\int_{a}^{b} \sin f(x) \mathrm{d} x\right| \leq \frac{2}{m}$.

\section{3. 北京科技大学 2016 年硕士研究生入学考试试题(数学分析613) 
 李扬 
 微信公众号: sxkyliyang}
 一 、 (15  分 )

(1)  求极限  $\lim _{n \rightarrow \infty}\left(1-\frac{1}{2^{2}}\right)\left(1-\frac{1}{3^{2}}\right) \cdots\left(1-\frac{1}{n^{2}}\right)$

(2)  求极限  $\lim _{x \rightarrow \infty, y \rightarrow a}\left(1+\frac{1}{x}\right)^{\frac{x}{x^{2}+y}}$.

 一 、 $(15$  分  $)$  设 
$$
f(x, y)=\left\{\begin{array}{cc}
\left(x^{2}+y^{2}\right) \sin \frac{1}{\sqrt{x^{2}+y^{2}}}, & (x, y) \neq(0,0) \\
0, & (x, y)=(0,0)
\end{array}\right.
$$
 判断  $f(x, y)$  在原点处的偏导数是否连续及可微 .

 三 、 $(15$  分 )  已知 
$$
f(x)=\int_{0}^{x}\left[\int_{t}^{x} e^{-s^{2}} \mathrm{~d} s\right] \mathrm{d} t
$$
 求  $f(x)$  及  $f^{\prime}(x)$.

 四 、 (15  分 )\\
(1)  证明 : $\int_{-\pi}^{\pi} \frac{x \sin x \arctan e^{x}}{1+\cos ^{2} x} \mathrm{~d} x=\frac{\pi}{2} \int_{0}^{\pi} \frac{x \sin x}{1+\cos ^{2} x} \mathrm{~d} x$\\
(2)  求积分  $\int_{0}^{\pi} \frac{x \sin x}{1+\cos ^{2} x} \mathrm{~d} x$

 五 、( 15  分 ) $f(x)$  在  $[0,1]$  上连续 ,  在  $(0,1)$  上可导 , $f(0)=0, f(1)=1$,  在  $[0,1]$  上存在三个不同的常数  $k_{1}, k_{2}, k_{3}$,  且  $k_{1}+k_{2}+k_{3}=1$,  证明 :  存在三个不同的点  $t_{1}, t_{2}, t_{3}$,  使得 
$$
\frac{k_{1}}{f^{\prime}\left(t_{1}\right)}+\frac{k_{2}}{f^{\prime}\left(t_{2}\right)}+\frac{k_{3}}{f^{\prime}\left(t_{3}\right)}=1 .
$$
 六 、 (15  分 )  已知二重积分 
$$
I=\iint_{\Sigma}\left(x^{3}-x\right) \mathrm{d} y \mathrm{~d} z+\left(y^{3}-y\right) \mathrm{d} z \mathrm{~d} x+\left(z^{3}-z\right) \mathrm{d} x \mathrm{~d} y
$$
 要使  $I$  最小 ,  试确定曲面  $\sum$,  并求相应曲面  $\sum$  下  $I$  的最小值 .

 七 、 $(15$  分 )  求 
$$
I=\iint_{D}\left|x^{2}-x+y^{2}\right| \mathrm{d} x \mathrm{~d} y
$$
 其中  $D$  为  $x^{2}+y^{2} \leq 1$  的曲面 .

 八 、 (15  分 ) $f(x)$  是非负函数且在  $(-\infty,+\infty)$  上一致连续 , $\int_{0}^{+\infty} f(x) \mathrm{d} x$  收敛 ,  求证  $\lim _{x \rightarrow+\infty} f(x)=0$.

 九 、 ( 15  分 ) $f(x)$  在  $(-\infty,+\infty)$  上具有导函数  $f^{\prime}(x)$,  且  $f^{\prime}(x)$  在  $(-\infty,+\infty)$  上一致连续 ,  求证 :
$$
\left\{n\left[f\left(x+\frac{1}{n}\right)-f(x)\right]\right\} \quad(n=1,2 \cdots, \infty)
$$
 在  $(-\infty,+\infty)$  一致收敛于  $f^{\prime}(x)$.

 十 、 (15  分 )  证明 :  反常积分 
$$
\int_{0}^{+\infty} \frac{\sin x y}{y} \mathrm{~d} y
$$
 在  $(0,+\infty)$  上一致收敛 . 14.  北京科技大学  2017  年硕士研究生入学考试试题 ( 数学分析 613)

 李扬 

 微信公众号 : sxkyliyang

 一 、 (1)  求极限 
$$
\lim _{x \rightarrow \infty}(\cos 2 x+2 \sin x)^{\frac{1}{x^{4}}} .
$$
(2) 求极限 
$$
\lim _{n \rightarrow \infty} \frac{1}{n^{2}}\left(\sin \frac{1}{n}+2 \sin \frac{2}{n}+\cdots+n \sin \frac{n}{n}\right) .
$$
 二 、 $(1) f(x)$  在  $[0,1]$  连续 , $f(x)<0$,  证明 : $2 x-\int_{0}^{x} f(t) \mathrm{d} t$  在  $(0,1)$  上有且只有一个根 .

(2) $f(x)$  在  $[0,1]$  连续 , $I=\int_{0}^{1} f(x) \mathrm{d} x \neq 0$,  证存在两点  $x_{1}, x_{2} \in(0,1)$  使得 
$$
\frac{1}{f\left(x_{1}\right)}+\frac{1}{f\left(x_{2}\right)}=\frac{2}{I}
$$
 三 、 $f$  在  $[a, b]$  上二阶可导 , $f^{\prime}(x)=2 f(x)=\beta f^{\prime \prime}(x)$,  证明当  $\beta=0, \beta \neq 0$  时  $f$  在  $[a, b]$  上无穷阶可导 .

 四 、 $I_{1}, \cdots, I_{n}$  是  $n$  个向量 ,  且相邻两个向量的夹角为  $\cos \frac{2 \pi}{n}$,  证明 
$$
\sum_{i=1}^{n} \frac{\partial f\left(x_{0}, y_{0}\right)}{\partial l_{i}}=0
$$
 五 、 设 
$$
f(x, y)= \begin{cases}(x+y)^{p} \sin \frac{1}{\sqrt{x^{2}+y^{2}}}, & x^{2}+y^{2} \neq 0 \\ 0, & x^{2}+y^{2}=0\end{cases}
$$
 判断  $p$  取什么值  $f(x, y)$  的偏导数在  $(0,0)$  连续 .

 六 、 已知  $f^{\prime \prime}(x)>0$,  证明 
$$
\int_{0}^{a} f[g(t)] \mathrm{d} t \geq a f\left(\frac{1}{a} \int_{0}^{a} g(t) \mathrm{d} t\right)
$$
 七 、 设 
$$
I=\frac{1}{2 \pi} \oint_{C} \frac{X \mathrm{~d} Y-Y \mathrm{~d} X}{X^{2}+Y^{2}}(X=a x+b y, Y=c x+d y)
$$
 其中  $C$  是简单的封闭曲线 ,  求  $I$.

 八 、 设 
$$
f(x, y, z)= \begin{cases}x^{2}+y^{2}, & z \geq \sqrt{x^{2}+y^{2}} \\ 0, & z<\sqrt{x^{2}+y^{2}}\end{cases}
$$
 试求曲面积分  $F(t)=\iint_{x^{2}+y^{2}+z^{2}=t^{2}} f(x, y, z) \mathrm{d} S(-\infty<t<+\infty)$  之值 .

 九 、 (1)  证明 
$$
\sum_{n=1}^{\infty} \frac{(-1)^{n} x^{n} \ln ^{n} x}{n !}
$$
 在  $[0,1]$  上一致收敛 .

(2)  证明 
$$
\int_{0}^{1} x^{-x} \mathrm{~d} x=\sum_{n=1}^{\infty} n^{-n}
$$

\section{1. 北京师范大学 2009 年研究生入学考试试题高等代数 
 李扬 
 微信公众号: sxkyliyang}
\begin{enumerate}
  \item  设  $A$  为  $m \times n$  矩阵 ,  则存在  $n \times m$  矩阵  $B$  使得  $A B=E_{m}$  的充要条件是  $r(A)=m$.

  \item  设  $A$  是三阶矩阵 , $r(A)=2$,  其二重特征值  $\lambda_{1}=\lambda_{2}=6$,  且属于  $\lambda_{1}=\lambda_{2}=6$  的线性无关的特征向量有  $\alpha_{1}=(1,1,0)^{t}, \alpha_{2}=(2,1,1)^{t}$.  求矩阵  $A$.

  \item  已知  $\sigma$  为对称变换 , $V$  是一个空间 , $W$  是  $V$  的一个子空间 ,  试证 : $W$  是  $\sigma$  的不变子空间 .

  \item  已知  $A, B$  为复矩阵 , $A^{n-2}=B^{n-2} \neq 0, A^{n-1}=B^{n-1}=0$.  求证 : $A, B$  相似 .

\end{enumerate}
\section{2. 北京师范大学 2010 年研究生入学考试试题高等代数 
 李扬 
 微信公众号: sxkyliyang}
\begin{enumerate}
  \item $A^{2}=A$,  则  $r(A)+r(A-I)=n$,  其中  $I$  为  $n$  阶单位矩阵 .

  \item $V_{1}, V_{2}$  为有限维向量空间 ,  证明 : $V_{1}+V_{2}$  为有限维 ,  且  $\operatorname{dim} V_{1}+\operatorname{dim} V_{2}=\operatorname{dim}\left(V_{1}+V_{2}\right)+\operatorname{dim}\left(V_{1} \cap V_{2}\right)$.

  \item $V_{1}, V_{2}$  为  $V$  的两个子空间 , $\operatorname{dim} V_{1}+\operatorname{dim} V_{2}=n$,  证明 :  存在  $V$  的线性变换  $\sigma: \operatorname{Ker} \sigma=V_{1}, \operatorname{Im} \sigma=V_{2}$.

  \item  求正交矩阵  $Q$,  使得  $Q^{\prime} A Q=\Lambda, \Lambda$  为对角矩阵 ,  其中  $A=\left(\begin{array}{ccc}* & * & * \\ * & * & * \\ * & * & *\end{array}\right)$ ( 具体数据忘了 )

\end{enumerate}
\section{3. 北京师范大学 2011 年研究生入学考试试题高等代数 
 李扬 
 微信公众号: sxkyliyang}
\begin{enumerate}
  \item  解一个方程组 .

  \item $\operatorname{dim} V=n$,  证明  $V$  的任一真子空间都可以表示为若干个  $n-1$  维子空间的交 .

  \item  如果  $A^{*}=A$,  证明  $A$  的特征根为实数 .

  \item $f(x)=X^{\prime} A X$  正定的充要条件是  $A$  的各阶主子式大于  0 . ( 说明  3,4  题都和课本原命题几乎相同 )

\end{enumerate}
\section{4. 北京师范大学 2012 年研究生入学考试试题高等代数 
 李扬 
 微信公众号: sxkyliyang}
\begin{enumerate}
  \item  给出了一个含参数  $a$  的线性方程组 ,
\end{enumerate}
(1)  当方程组有非零解时 ,  求参数  $a$  的值 .

(2)  求线性方程组的秩 .

\begin{enumerate}
  \setcounter{enumi}{2}
  \item  计算行列式 
\end{enumerate}
$$
D_{n}=\left|\begin{array}{ccccccc}
1+a_{1} & 1 & 1 & 1 & \cdots & 1 & 1 \\
1 & 1+a_{2} & 1 & 1 & \cdots & 1 & 1 \\
1 & 1 & 1+a_{3} & 1 & \cdots & 1 & 1 \\
\vdots & \vdots & \vdots & \vdots & & \vdots & \vdots \\
1 & 1 & 1 & 1 & \cdots & 1 & 1+a_{n}
\end{array}\right| .
$$

\begin{enumerate}
  \setcounter{enumi}{3}
  \item  存在非零向量  $\alpha$,  使得  $A^{m} \alpha \neq 0, A^{m+1} \alpha=0$,  证明 :
\end{enumerate}
(1) $\alpha, A \alpha, A^{2} \alpha, \cdots, A^{m} \alpha$  线性无关 .

(2) $r\left(A^{n}\right)=r\left(A^{n+1}\right)$.

\begin{enumerate}
  \setcounter{enumi}{4}
  \item  给出了一个  $3 \times 3$  阶实数矩阵  $A$
\end{enumerate}
(1)  求矩阵  $A$  的特征值和特征向量 .

(2)  求正交矩阵  $Q$  和对角矩阵  $D$,  使得  $Q^{\prime} A Q=D$.

\section{5. 北京师范大学 2013 年研究生入学考试试题高等代数 
 李扬 
 微信公众号: sxkyliyang}
\begin{enumerate}
  \item  叙述并证明克拉默法则 .

  \item  证明  $(f(x), g(x))=1$  时 , $(a f(x)+b g(x), c f(x)+d g(x))=1$,  其中  $a d-b c \neq 0$.

  \item $n$  阶矩阵  $A$,  证明  $A$  可以分解为  $A=B C$  的形式 ,  其中  $B$  为可逆矩阵 , $C$  有  $C=C^{2}$  成立 .

  \item $V_{1}, V_{2}$  为欧式空间  $V$  的子空间 ,  证明 :

\end{enumerate}
$$
\operatorname{dim} V_{1}+\operatorname{dim} V_{2}=\operatorname{dim}\left(V_{1}+V_{2}\right)+\operatorname{dim}\left(V_{1} \cap V_{2}\right)
$$

\section{6. 北京师范大学 2014 年研究生入学考试试题高等代数 
 李扬 
 微信公众号: sxkyliyang}
\begin{enumerate}
  \item  数域  $F$  上的多项式  $f(x)$  不可约 ,  证明 : $f(x)$  在复数域内没有重根 .

  \item  子空间  $V_{1}$  由  $\alpha_{1}=(\quad)^{\prime}, \alpha_{2}=(\quad)^{\prime}$  生成 , $V_{2}$  由  $\beta_{1}=(\quad)^{\prime}, \beta_{2}=(\quad)^{\prime}$  生成 .  求  $V_{1}+V_{2}, V_{1} \cap V_{2}$  的基和维数 .

  \item (1)  验证双非线性退化函数 ;

\end{enumerate}
(2) $\operatorname{Hom}(V, V) \rightarrow \operatorname{Hom}(V, V)^{*}$  同构 ;

(3) $l_{V}$  在同构下的像 .

\begin{enumerate}
  \setcounter{enumi}{4}
  \item $S=\left(a_{i j}\right)_{n \times n}$  是一个随机矩阵 ,  且满足 : $a_{i 1}+a_{i 2}+\cdots+a_{i n}=1,(i=1,2, \cdots, n)$.  证明 :
\end{enumerate}
( 1) $S$  有特征值  1 .

(2) $S$  的特征值  $|\lambda| \leqslant 1$.

\section{7. 北京师范大学 2015 年研究生入学考试试题高等代数与解析几何 
 李扬 
 微信公众号: sxkyliyang}
\begin{enumerate}
  \item (15  分 )  已知直线  $l_{1}: \quad$  和  $l_{2}:$,  说明  $l_{1}, l_{2}$  异面 ,  并求出其距离及其公垂线方程 .

  \item (15  分 ) $l: \frac{x}{\alpha}=\frac{y-\beta}{0}=\frac{z}{1}$  绕  $z$  轴旋转 ,  求出旋转曲面 ,  并根据  $\alpha, \beta$  的值讨论曲面类型 .

  \item (15  分 )  求过原点与球面  ()$^{2}+()^{2}+()^{2}=16$  相切的直线组成的直纹面 . ( 记不清了 )

  \item (20  分 )  把一个曲面方程  ( 记不清了 )  化成标准型并判断曲面类型 .

  \item (10  分 )  已知  $f(x)=a_{n} x^{n}+a_{n-1} x^{n-1}+\cdots+a_{0}, x=\frac{u}{v}$  是  $f(x)$  的一个根 ,  证明 : $u\left|a_{0}, v\right| a_{n}$.

  \item ( 15  分 ) $A$  是  $n$  阶矩阵 ,  证明  $A$  的秩是  2  的充要条件是  $A=B C$,  其中  $B$  是秩为  2  的  $n \times 2$  矩阵 , $C$  是秩为  2  的  $2 \times n$  矩阵 .

  \item ( 15  分 )  已知  $\alpha_{1}=(-3,1,-2), \alpha_{2}=(1,-1,1), \alpha_{3}=(2,3,-1), \beta_{1}=(1,1,1), \beta_{2}=(1,2,3), \beta_{3}=(2,0,1)$,  证明 : $\alpha_{1}, \alpha_{2}, \alpha_{3}$  及  $\beta_{1}, \beta_{2}, \beta_{3}$  均是三维行空间的基 ,  并求出  $\alpha_{1}, \alpha_{2}, \alpha_{3}$  到  $\beta_{1}, \beta_{2}, \beta_{3}$  的过渡矩阵 .

  \item (15  分 )  已知  $W$  是所有迹为  0  的  3  阶实对称矩阵构成的集合 ,  证明 : $W$  是由加法和数乘构成的向量空间并   给出一组基 .

  \item (15  分 )  已知  $\operatorname{dim}\left(V_{1}+V_{2}\right)=\operatorname{dim}\left(V_{1} \cap V_{2}\right)+1$.  证明 : $V_{1} \subseteq V_{2}$  或  $V_{2} \subseteq V_{1}$.

  \item (15  分 )  已知在复数域上  $A=\left(\begin{array}{ccc}a & b & c \\ c & a & b \\ b & c & a\end{array}\right)$,  证明 :

\end{enumerate}
(1) $A$  在复数域上相似于对角阵 ;

(2) $A$  与  $A^{\prime}$  相似 .

\section{8. 北京师范大学 2016 年研究生入学考试试题高等代数与解析几何 
 李扬 
 微信公众号: sxkyliyang}
\begin{enumerate}
  \item (10  分 )  证明 :  正弦函数  $\sin x$  在实数域内不能表示为  $x$  的多项式 .

  \item (10  分 )  设  $f(x)=a_{3} x^{3}+a_{2} x^{2}+a_{1} x+a_{0}$  的三个根为  $x_{1}, x_{2}, x_{3}$,  且  $x_{i} \neq 0(i=1,2,3)$,  求以  $\frac{1}{x_{1}}, \frac{1}{x_{2}}, \frac{1}{x_{3}}$  为根的多项式 .

  \item ( 15  分 )  设  $A$  是  $n$  阶方阵 , $A^{*}$  是  $A$  的伴随矩阵 ,  证明 

\end{enumerate}
$$
r(A)=\left\{\begin{array}{l}
n, r(A)=n \\
1, r(A)=n-1 \\
0, r(A)<n-1
\end{array}\right.
$$

\begin{enumerate}
  \setcounter{enumi}{4}
  \item (10  分 )  设  $V$  是域  $F$  上的一个  $n$  维向量空间 ,  域  $F$  包含域  $E, F$  可以看作域  $E$  上的向量空间  ( 其加法是域  $F$  的加法 ,  数乘是  $E$  中元素与  $F$  中元素在域  $F$  中做乘法 ),  设  $\operatorname{dim}_{E} F=m$.
\end{enumerate}
(1) 证明 : $V$  可以成为域  $E$  上的一个向量空间 .

(2) 求  $V$  作为域  $E$  上向量空间的维数 .

\begin{enumerate}
  \setcounter{enumi}{5}
  \item ( 10  分 )  设  $\sigma$  是域  $F$  上向量空间  $V$  的幂等变换 ,  即  $\sigma$  是满足条件  $\sigma^{2}=\sigma$  的线性变换 .
\end{enumerate}
(1)  证明 : $V=\operatorname{Im}(\sigma) \oplus \operatorname{Ker}(\sigma)$.

(2)  证明 : $\sigma$  可对角化 .

\begin{enumerate}
  \setcounter{enumi}{6}
  \item (30  分 )
\end{enumerate}
(1)  设  $g(x)=a_{0}+a_{1} x+\cdots+a_{n} x^{m}$  是数域  $F$  上一个多项式 .  证明 ,  如果  $\lambda_{0}$  是  $F$  上  $n$  阶方阵  $A$  的一个   特征值 ,  且  $\alpha$  是  $A$  的属于  $\lambda_{0}$  的一个特征向量 ,  那么  $g\left(\lambda_{0}\right)$  是  $g(A)$  的一个特征值 ,  且  $\alpha$  是  $g(A)$  的属于  $g\left(\lambda_{0}\right)$  的一个特征向量 .

(2)  令 
$$
A=\left|\begin{array}{ccccc}
0 & 1 & 0 & \cdots & 0 \\
0 & 0 & 1 & \cdots & 0 \\
\vdots & \vdots & \vdots & & \vdots \\
0 & 0 & 0 & \cdots & 1 \\
1 & 0 & 0 & \cdots & 0
\end{array}\right|,
$$
 计算  $A^{2}, A^{3}, \cdots, A^{n-1}, A^{n}$,  并求  $A$  在复数域  $C$  中的全部特征值 .

(3)  设  $a_{1}, a_{2}, \cdots, a_{n} \in \mathbb{C}$,  行列式 
$$
D=\left|\begin{array}{ccccc}
a_{1} & a_{2} & a_{3} & \cdots & a_{n} \\
a_{n} & a_{1} & a_{2} & \cdots & a_{n-1} \\
a_{n-1} & a_{n} & a_{1} & \cdots & a_{n-2} \\
\vdots & \vdots & \vdots & & \vdots \\
a_{2} & a_{3} & a_{4} & \cdots & a_{1}
\end{array}\right|
$$
 叫做循环行列式 .  证明  $D=f\left(w_{1}\right) f\left(w_{2}\right) \cdots f\left(w_{n}\right)$,  其中  $w_{1}, w_{2}, \cdots, w_{n}$  是全部  $n$  次单位  ( 好遗㨔 ,  后面   的内容没有拍到 ) 7. ( 8  分  $)$  在  $E^{3}$  的  Descartes  直角坐标系  $O x y z$  下 ,  已知直线  $l_{1}$  过点  $P_{1}(0,-5,0)$  并具有方向向量  $l_{1}=(3,2,-1)$,  直线  $l_{2}$  过点  $P_{2}(5,0,0)$  并具有方向向量  $l_{2}=(-1,2,3)$,  平面  $\Pi$  过点  $P_{0}(0,0,4)$  且平行于  $l_{1}$  和  $l_{2}$.  试求平面  $\Pi$  的一般方程 ,  并确定直线  $l_{2}$  到平面  $\Pi$  的距离 .

\begin{enumerate}
  \setcounter{enumi}{8}
  \item (6  分 )  在  $E^{3}$  的  Descartes  直角坐标系  $O x y z$  下 ,  将抛物线  $C_{1}:\left\{\begin{array}{l}y^{2}=5 z ; \\ x=0 .\end{array}\right.$  平行移动 ,  且使其顶点总落在抛   物线  $C_{2}:\left\{\begin{array}{l}x^{2}=-3 z ; \\ y=0 .\end{array}\right.$  上 ,  试求其轨迹方程 ,  并指出其规范名称 .

  \item ( 9  分 )  在  $E^{3}$  中的仿射标架  $\left\{O ; e_{1}, e_{2}, e_{3}\right\}$  所对应的坐标系  $O x y z$  之下 ,  已知度量系数矩阵 

\end{enumerate}
$$
g=\left(e_{i} \cdot e_{j}\right)_{3 \times 3}=\left|\begin{array}{ccc}
4 & -3 & 0 \\
-3 & 4 & 0 \\
0 & 0 & 2
\end{array}\right|
$$
 且  $\triangle A B C$  的顶点坐标分别为  $A(0,-1,0), B(-1,0,3), C(-1,0,-3)$,  试求  $\triangle A B C$  的面积  $S_{\triangle A B C}$.

\begin{enumerate}
  \setcounter{enumi}{10}
  \item ( 11  分 )  在  $E^{3}$  的单位正交右手标架  $\{O ; i, j, k\}$  即右手直角坐标系  $I: O x y z$  之下 ,  已知三个互相垂直的平面  $\Pi_{1}: x-y+z+1=0, \Pi_{2}: x-z-3=0, \Pi_{1} 3: x+2 y+z-5=0$.  试确定新的单位正交标架  $\left\{O^{\prime} ; i, j, k\right\}$  即直角坐标系  $I I: O^{\prime} x^{\prime} y^{\prime} z^{\prime}$,  使得  $\Pi_{1}, \Pi_{2}, \Pi_{3}$  分别对应称为坐标面  $y^{\prime} O^{\prime} z^{\prime}, z^{\prime} O^{\prime} x^{\prime}, x^{\prime} O^{\prime} y^{\prime}$,  并使点  $O$  落在  $I I$  的第一卦限内 .

  \item ( 11  分 )  已知  $E^{3}$  的点变换  $\sigma$  在  Descartes  直角坐标系  $O x y z$  之下的表示为 

\end{enumerate}
$$
\sigma(x, y, z)=(x, y, z)\left|\begin{array}{ccc}
\frac{1}{\sqrt{2}} & \frac{1}{\sqrt{2}} & 0 \\
-\frac{1}{2} & \frac{1}{2} & \frac{1}{\sqrt{2}} \\
\frac{1}{2} & -\frac{1}{2} & \frac{1}{\sqrt{2}}
\end{array}\right|
$$
 试证  $\sigma$  为保持原点不动的旋转变换 ,  并确定  $\sigma$  所对应的旋转轴的点向式方程 .

\section{9. 北京师范大学 2009 年研究生入学考试试题数学分析 
 李扬 
 微信公众号: sxkyliyang}
\begin{enumerate}
  \item  求  $\iint_{D}|x| \mathrm{d} x \mathrm{~d} y$,  其中  $D$  为三角形  $\triangle A B C: A(-2,0), B(1,1), C(2,3)$.

  \item  把  $\iiint_{V} f(x, y, z) \mathrm{d} x \mathrm{~d} y \mathrm{~d} z$  化为累次积分 ,  其中  $f(x, y, z)$  为连续函数 , $V$  为四面体 : $P(2,2,0), A(-2,0,0), B(0,0,2)$, $C(1,1,3) .$

  \item  求  $\lim _{n \rightarrow \infty} \frac{1^{p}+2^{p}+\cdots+n^{p}}{n^{p}}-\frac{n}{p+1}$.

  \item  设函数  $f$  处处可导 ,  证明若  $f^{\prime}$  有间断点 ,  则一定为第二类间断点 .

  \item  设  $\left\{a_{n}\right\}$  为实数列 ,

\end{enumerate}
$$
S_{n}=a_{1}+a_{2}+\cdots+a_{n}, \sigma_{n}=\frac{1}{n+1}\left(S_{1}+S_{2}+\cdots+S_{n}\right)
$$
 已知级数  $\sum_{n=1}^{\infty}\left|S_{n}-\sigma_{n}\right|^{2}$  收敛 ,  求证 : $\sum_{n=1}^{\infty} a_{n}$  收敛 .

\begin{enumerate}
  \setcounter{enumi}{6}
  \item  求证 :
\end{enumerate}
$$
\sum_{n=1}^{\infty} \lim _{x \rightarrow \infty} \cos \frac{x}{2^{n}}=\ln \frac{\sin x}{x}, \sum_{n=1}^{\infty} \frac{1}{2^{n}} \tan \frac{x}{2^{n}}=\frac{1}{x}-\cot x
$$

\begin{enumerate}
  \setcounter{enumi}{7}
  \item  已知  $f$  连续 , $\lim _{t \rightarrow x} f(t)=f(x)$.  证明 : $f$  黎曼可积 .
\end{enumerate}
\section{0. 北京师范大学 2010 年研究生入学考试试题数学分析 
 李扬 
 微信公众号: sxkyliyang}
\begin{enumerate}
  \item $\left\{x_{n}\right\}$  为  $R^{n}$  上的点列 ,  证明 : $\left\{x_{n}\right\}$  收玫当且仅当  $\left\{x_{n}\right\}$  为基本列 .

  \item $x_{n+1}=x_{n}-\frac{f\left(x_{n}\right)}{f^{\prime}\left(x_{n}\right)}$,  其中  $f(x)$  有二阶连续导数 ,  且  $f(0)>0, f(1)<0$,  证明 :

\end{enumerate}
(1) $\left\{x_{n}\right\}$  收敛于  $f(x)$  的零点  $a$.

(2) $x_{n+1}-a=O\left(\left(x_{n}-a\right)^{2}\right),(n \rightarrow \infty)$.

\begin{enumerate}
  \setcounter{enumi}{3}
  \item $\lim _{n \rightarrow \infty} \sup a_{n}=\lambda$,  当且仅当 :
\end{enumerate}
(1) $\exists k, \forall n>k$,  有  $a_{n}>\lambda+\varepsilon$

(2) $\forall n, \exists k>n$,  有  $a_{n}<\lambda-\varepsilon($  记不太清了 )

\begin{enumerate}
  \setcounter{enumi}{4}
  \item  证明 : $\left(\int_{0}^{\infty} e^{-x^{2}} \mathrm{~d} x\right)^{2}=\int_{0}^{\infty} e^{-x^{2}} \mathrm{~d} x \int_{0}^{\infty} x e^{-x^{2} y^{2}} \mathrm{~d} y$,  并由此求  $\int_{0}^{\infty} e^{-x^{2}} \mathrm{~d} x$  的值 .

  \item $\left\{X_{i}\right\}$  非负递减 , $\sum_{i=1}^{\infty} X_{i}$  收敛 ,  当且仅当  $\sum_{i=1}^{\infty} 2^{i} X_{2^{i}}$  收敛 .

  \item  求  $\oint_{c} \frac{-y \mathrm{~d} x+x \mathrm{~d} y}{x^{2}+y^{2}}$,  其中  $c$  为包含原点的闭合曲线 .

\end{enumerate}
\section{1. 北京师范大学 2011 年研究生入学考试试题数学分析 
 李扬 
 微信公众号: sxkyliyang}
\begin{enumerate}
  \item  给出函数列一致收玫的定义 ,  给出充要条件 ,  并证明你给出的充要条件 .

  \item $f(x)$  二阶连续可导 , ( 等条件我忘记了 )  证明 :

\end{enumerate}
$$
f(x)=f(a)+(x-a) f^{\prime}(a)+(x-a)^{2} \int_{0}^{1} f^{\prime \prime}[a+t(x-a)](1-t) \mathrm{d} t .
$$

\begin{enumerate}
  \setcounter{enumi}{3}
  \item $f(x, t)=\int_{-\infty}^{+\infty} \varphi(x+t y) e^{-y^{2}} \mathrm{~d} y$,  证明  $f$  在  $\mathbb{R} \times[0,+\infty)$  上一致收敛 .

  \item $\sum_{k=1}^{\infty}(-1)^{k} \frac{\sin k x}{k}$,  判断其敛散性  ( 条件 ,  绝对 )  及一致收玫性 .

  \item $S:\left(x_{1}-a_{1}\right)^{2}+\left(x_{2}-a_{2}\right)^{2}+\left(x_{3}-a_{3}\right)^{2}=r^{2}$,

\end{enumerate}
(1) 求其参数表示 ;

(2)  面积元  $\mathrm{d} \sigma$;

(3)  求连续函数  $f$  表示其第一曲面积分 ;  如果  $S$  与  $x_{3}$  无关 ,  则  $f$  也与  $x_{3}$  无关 .

( 说明 : 3  题还有  2  问但很简单 ,  我记不清了 . 5  题  4  问有点长但大致内容我记下了 )

\section{2. 北京师范大学 2012 年研究生入学考试试题数学分析 
 李扬 
 微信公众号: sxkyliyang}
\begin{enumerate}
  \item  曲线  $C$  是由  $y=\sqrt{x^{2}-1}$  和  $y=x^{2}-1$  围成的封闭曲线 .
\end{enumerate}
(1)  求曲线  $C$  的外法向量  $\vec{n}$.

(2)  已知  $f(x, y)=($  不记得了  $)$,  求  $\oint_{C} \frac{\partial f}{\partial \vec{n}} \mathrm{~d} s$,  其中  $\mathrm{d} s$  为弧长微分 , $\vec{n}$  为外法向量 .

\begin{enumerate}
  \setcounter{enumi}{2}
  \item  求  $\sum_{k=1}^{+\infty} \frac{(-1)^{k}}{2 k+1} \cdot \frac{1}{3^{k}}$.

  \item  求  $\iiint_{\Omega} z \mathrm{~d} x \mathrm{~d} y \mathrm{~d} z$,  其中 , $\Omega$  由  $x+y+z=3, y-z=0, x-z=0, z=0$  围成 .

  \item  已知  $f(x)$  单调不减 ,  连续 , $0 \leqslant g(x) \leqslant 1$,  连续 .  利用这些条件证明一个不等式 .

  \item  判断  $F(\alpha)=\int \frac{1}{1+x^{\alpha}(\ln x)^{\alpha}} \mathrm{d} x$ ( 大概是这样的 )  的定义域 ,  连续性 ,  可微性 .

\end{enumerate}
\section{3. 北京师范大学 2013 年研究生入学考试试题数学分析}
 李扬 

 微信公众号 : sxkyliyang

\begin{enumerate}
  \item  求二元函数  $f(x, y)=(x+y) e^{-\left(x^{2}+y^{2}\right)}$  的极值点 .

  \item  求  $\iiint_{V}(x+y) \mathrm{d} x \mathrm{~d} y \mathrm{~d} z$,  其中  $V$  是  $z=x^{2}-y^{2}, z=0, x=1$  所围的体积 .

  \item  求  $\arctan x$  的泰勒级数 ,  并且求出  $\sum_{k=1}^{\infty} \frac{(-1)^{k}}{2 k+1}$.

  \item  已知  $f(x), g(x)$  是  $[a, b]$  上的黎曼可积 , $f(x)$  的导函数在  $[a, b]$  上黎曼可积 ,

\end{enumerate}
$$
T_{n}: a=x_{0}<x_{1}<\cdots<x_{m(n)}=b
$$
 是  $[a, b]$  的一个分害 , $\left\|T_{n}\right\|=\max _{j=1}^{m(n)}\left(x_{j}-x_{j-1}\right) \rightarrow 0, n \rightarrow \infty$.  求证 :
$$
\lim \sum_{j=1}^{m(n)} g\left(x_{j}\right)\left(f\left(x_{j}\right)-f\left(x_{j-1}\right)\right)=\int_{a}^{b} g(x) f^{\prime}(x) \mathrm{d} x
$$

\begin{enumerate}
  \setcounter{enumi}{5}
  \item $f(x)$  在  $\mathbb{R}$  上连续 ,  且  $\lim _{|x| \rightarrow \infty} f(x)=A$ ( 有限数 ),  求证 :
\end{enumerate}
(1) $f(x)$  在  $\mathbb{R}$  上一致连续 .

(2) $\eta$  为  $(0, \pi)$  上一固定数 , $F_{n}(x)=\int_{\eta}^{\pi} f(x) \sin n x \mathrm{~d} x$,  证明  $F_{n}(x)$  等度连续 .

(3) $F_{n}(x)$  在  $\mathbb{R}$  上一致收敛于  0 .

\section{4. 北京师范大学 2014 年研究生入学考试试题数学分析 
 李扬 
 微信公众号: sxkyliyang}
\begin{enumerate}
  \item  求  $\iiint_{\Omega}\left|\sin \left(x^{2}+y^{2}\right)\right| \mathrm{d} x \mathrm{~d} y \mathrm{~d} z, \Omega$  由  $x=0, y=0, x=\pi, y=\frac{\pi}{2}, y=z$  围成 .

  \item $F(y)=\int_{0}^{+\infty} e^{-x y} \frac{\sin x}{x} \mathrm{~d} x, 0 \leqslant y<\infty$.

\end{enumerate}
(1)  证明  $F(y)$  在  $[0,+\infty)$  内连续 ,  在  $(0,+\infty)$  内可微 .

$(2)$  求  $F(y)$  及  $F(0)$.

\begin{enumerate}
  \setcounter{enumi}{3}
  \item  对于函数  $f(x, y)=\left\{\begin{array}{ll}\frac{x^{2} y}{\sqrt{x^{2}+y^{2}}}, & x^{+} y^{2} \neq 0 ; \\ 0 . & x^{2}+y^{2}=0 .\end{array}\right.$,
\end{enumerate}
(1)  求  $\frac{\partial f}{\partial x}, \frac{\partial f}{\partial y}$  在  $(0,0)$  点的值 ;

( 2 )  证明  $f(x, y)$  在  $(0,0)$  处连续 ;

(3)  证明  $\frac{\partial f}{\partial x}, \frac{\partial f}{\partial y}$  在  $(0,0)$  处不连续 ;

(4)  证明  $f(x, y)$  在  $(0,0)$  不可微 .

\begin{enumerate}
  \setcounter{enumi}{4}
  \item $f \in C[0,1]$  且  $f$  在  $(0,1)$  上二阶可导 , $f^{\prime \prime}(x) \neq 0, \int_{0}^{1} f(t) \mathrm{d} t=0, f(0)=f(1)>0$.  证明 :
\end{enumerate}
(1) $f^{\prime \prime}(x)>0$;

(2) $f(x)=0$  在  $(0,1)$  上恰有两根 .

(3) $\exists \xi \in(0,1)$, s.t. $f^{\prime}(\xi)=\int_{\xi}^{1} f(t) \mathrm{d} t$.

\section{5. 北京师范大学 2015 年研究生入学考试试题数学分析 
 李扬 
 微信公众号: sxkyliyang}
\begin{enumerate}
  \item  对于每个  $x \in[0,+\infty)$,  定义函数 
\end{enumerate}
$$
f(x)=\left(\int_{0}^{x} e^{-t^{2}} \mathrm{~d} t\right)^{2}, g(x)=\int_{0}^{1} \frac{e^{-x^{2}\left(1+t^{2}\right)}}{1+t^{2}} \mathrm{~d} t
$$
 证明 :

(1) $f(x)+g(x)=\frac{\pi}{4}$

(2) $\lim _{x \rightarrow+\infty} \int_{0}^{x} e^{-t^{2}} \mathrm{~d} t=\frac{\sqrt{\pi}}{2}$.

\begin{enumerate}
  \setcounter{enumi}{2}
  \item  计算三重积分 
\end{enumerate}
$$
I=\iiint_{\Omega}|x y| z \mathrm{~d} x \mathrm{~d} y \mathrm{~d} z
$$
 其中区域  $\Omega$  由平面  $z+x=1, z-x=1, z+y=1, z-y=1, z=0$  所围成 .

\begin{enumerate}
  \setcounter{enumi}{3}
  \item  计算曲面积分 
\end{enumerate}
$$
I=\iint_{S} x^{2} \mathrm{~d} y \mathrm{~d} z+\left(2 y+y^{2}\right) \mathrm{d} z \mathrm{~d} x+z \mathrm{~d} x \mathrm{~d} y
$$
 其中  $S$  为  $z=x^{2}+y^{2}$  含于柱体  $x^{2}+y^{2} \leqslant 1$  内部的那部分曲面的下侧 .

\begin{enumerate}
  \setcounter{enumi}{4}
  \item  对于每个正整数  $n$,  定义函数  $f_{n}(x)=x\left(1+\frac{1}{n}\right), x \in \mathbb{R}$,  及 
\end{enumerate}
$$
g_{n}(x)= \begin{cases}\frac{1}{n}, & x=0 \text { 或 } x \text { 为无理数; } \\ b+\frac{1}{n}, & x \text { 为非零有理数 } \frac{a}{b} .\end{cases}
$$
 其中  $a, b$  表示非零整数且  $b>0$.

(1)  证明 :  函数列  $\left\{f_{n}(x)\right\}$  和  $\left\{g_{n}(x)\right\}$  关于  $x$  在任何  ( 长度不为零的 )  有界区间上一致收敛 ;

(1)  证明 :  函数列  $\left\{f_{n}(x) g_{n}(x)\right\}$  关于  $x$  在任何  ( 长度不为零的 )  有界区间上不一致收敛 .

\begin{enumerate}
  \setcounter{enumi}{5}
  \item  求函数项级数  $\sum_{n=0}^{+\infty}\left(\frac{x^{2 n+1}}{2 n+1}-\frac{x^{n+1}}{2 n+2}\right)$  的收敛域 ,  和函数  $s(x)$  及  $s^{(n)}(0), n \in N$,  其中  $s^{(0)}$  表示  $s(0)$.

  \item  设  $F(x, y, z)=x^{3}+y^{2}-z^{2}-1, G(x, y, z)=x^{2}+y^{2}+z^{2}-3$.

\end{enumerate}
(1) 证明 :  方程组  $F(x, y, z)=0, G(x, y, z)=0$,  在  $(1,1,1)$  点的某个邻域  $U$  内确定一条过  $(1,1,1)$  的光滑曲   线  $C$,  即 ,  存在  $r>0$,  使得 
$$
C: \gamma(t)=(x(t), y(t), z(t)), t \in(-r, r), \gamma(0)=(1,1,1)
$$
 满足方程组 
$$
\left\{\begin{array}{l}
F(x(t), y(t), z(t))=0, \\
G(x(t), y(t), z(t))=0 .
\end{array},\right.
$$
 其中  $x(t), y(t)$  及  $z(t)$  是  $(-r, r)$  上连续可微的函数 .

( 2 )  在空间  $\mathbb{R}^{3}$  中 ,  求过该曲线  $C$  上点  $(1,1,1)$  处的切线方程和法平面方程 . 7.  利用等式 
$$
\frac{1}{2}+\sum_{k=1}^{n} \cos k x=\frac{\sin \left(n+\frac{1}{2}\right) x}{2 \sin \frac{x}{2}},-\pi \leqslant x \leqslant \pi, x \neq 0
$$
 证明 :\\
(1) $\lim _{n \rightarrow \infty} \int_{0}^{\pi}\left(\frac{1}{2 \sin \frac{x}{2}}-\frac{1}{x}\right) \sin \left(n+\frac{1}{2}\right) \mathrm{d} x=0$;\\
(2) $\int_{0}^{+\infty} \frac{\sin x}{x} \mathrm{~d} x=\frac{\pi}{2}$.

\section{6. 北京师范大学 2016 年研究生入学考试试题数学分析 
 李扬 
 微信公众号: sxkyliyang}
\begin{enumerate}
  \item  设  $f$  和  $g$  是定义在闭区间  $[0,1]$  上的两个有界实值函数 ,  证明 :
\end{enumerate}
$$
\begin{aligned}
&\text { (1) } \quad \sup _{x \in[0,1]} f(x)-\sup _{x \in[0,1]} g(x)\left|\leqslant \sup _{x \in[0,1]}\right| f(x)-g(x) \mid \\
&\text { ( 2 ) } \quad \sup _{x, y \in[0,1]}[f(x)-f(y)]=\sup _{x, y \in[0,1]}|f(x)-f(y)| .
\end{aligned}
$$

\begin{enumerate}
  \setcounter{enumi}{2}
  \item  证明 :  数列  $\left\{a_{n}\right\}$  有极限的充分必要条件是  $\left\{a_{n}\right\}$  的各个有极限子列的极限是相同的 .

  \item  设  $f$  在有界闭区间  $[a, b]$  上有  $m+1$  阶连续导数 .  证明 

\end{enumerate}
$$
f(b)=f(a)+\sum_{k=1}^{m} \frac{f^{(k)}(a)}{k !}(b-a)^{k}+\frac{1}{m !} \int_{a}^{b} f^{(m+1)}(t)(b-t)^{m} \mathrm{~d} t
$$

\begin{enumerate}
  \setcounter{enumi}{4}
  \item  给定实数列  $\left\{a_{n}\right\}$,  定义函数列  $\left\{f_{n}\right\}$ :  当  $x>a_{n}$  时 , $f_{n}(x)=1$;  当  $x \leqslant a_{n}$  时 , $f_{n}(x)=0$.  记 
\end{enumerate}
$$
f(x)=\sum_{n=1}^{+\infty} 2^{-n} f_{n}(x)
$$
 给出的间断点的集合 ,  并证明你的结论 .

\begin{enumerate}
  \setcounter{enumi}{5}
  \item  计算 \\
(1) $\lim _{x \rightarrow 0}\left(\frac{1}{x} \int_{0}^{x} \cos t^{2} \mathrm{~d} t\right)^{\frac{1}{x^{4}}}$;\\
(2) $\int_{0}^{+\infty} e^{-x^{2}} \mathrm{~d} x$;\\
(3) $\int_{0}^{+\infty} \frac{\sqrt[4]{x}}{(1+x)^{2}} \mathrm{~d} x .$

  \item  给出证明平面光滑曲线的弧长公式 .

  \item  叙述散度定理 ,  并对区域为球体的情形证明该定理 .

  \item  设  $f(x)=\int_{0}^{1} \ln \left(x^{2}+t^{2}\right) \mathrm{d} t$.\\
(1)  给出  $f$  的定义域 ;\\
(2)  讨论  $f$  在定义域上的连续性 ;\\
(3)  讨论  $f$  在定义域上的可微性 .

  \item  设  $f$  是  $(-\infty,+\infty)$  上的凸函数 , $y=k x+b$  是  $y=f(x)$  在  $x \rightarrow-\infty$  时的渐近线 .  证明 :

\end{enumerate}
$$
f(x) \geqslant k x+b, \forall x \in(-\infty,+\infty) .
$$

\begin{enumerate}
  \setcounter{enumi}{10}
  \item  设  $f$  是  $n$  维实数空间  $\mathbb{R}^{n}$  上的二阶连续偏导数的实值函数 , $\hat{x} \in \mathbb{R}^{n}$.  给出  $\hat{x}$  是  $f$  的局部极大点的充分条件 ,  并给出证明 . 17.  北京师范大学  2017  年研究生入学考试试题数学分析 
\end{enumerate}
 李扬 

 微信公众号 : sxkyliyang

\begin{enumerate}
  \item ( 15  分 )  设函数  $f(x)=e^{\sin x} \cos (\sin x)$.  求  $f^{\prime}(0), f^{\prime \prime}(0), f^{\prime \prime \prime}(0)$.

  \item ( 15  分 )  求函数  $z=x^{2}-x y+2 y^{2}$  在区域  $D: x^{2}+y^{2} \leqslant 1$  上的极值和最值 .

  \item ( 20  分 )  计算三重积分 

\end{enumerate}
$$
\iiint_{\Omega} x \mathrm{~d} x \mathrm{~d} y \mathrm{~d} z .
$$
 其中区域  $\Omega$  由  $x+y=2, y=x^{2}-x, z=0, z=x+y$  所围成 .

\begin{enumerate}
  \setcounter{enumi}{4}
  \item ( 20  分 )  设 
\end{enumerate}
$$
a_{n}=\sum_{k=1}^{n} \frac{1}{\sqrt{k}}-2 \sqrt{n}, n \in N^{*}
$$
 研究数值级数  $\sum_{n=1}^{\infty}\left|a_{n+1}-a_{n}\right|^{p}$  的敛散性 ,  其中  $p>0$.

\begin{enumerate}
  \setcounter{enumi}{5}
  \item ( 20  分 )  证明 
\end{enumerate}
(1)  函数  $f(x)=x+\sin ^{2} x$  在  $\mathbb{R}$  上一致连续 .

(2)  函数  $f(x)=x \sin ^{2} x$  在  $\mathbb{R}$  上不一致连续 .

\begin{enumerate}
  \setcounter{enumi}{6}
  \item ( 20  分 )  设函数  $f$  在  $[a,+\infty)$  上连续 ,  在  $(a,+\infty)$  上可导且存在正常数  $c$  使得 
\end{enumerate}
$$
f^{\prime}(x) \leqslant-c, x \in(a,+\infty) .
$$
 证明 :  若  $f(a)>0$,  则在区间  $\left(a, a+\frac{f(a)}{c}\right]$  内存在唯一一个  $\eta$,  满足  $f(\eta)=0$.

\begin{enumerate}
  \setcounter{enumi}{7}
  \item ( 20  分 )  设函数  $f \in C([0, a])(a>0)$  满足  $f(x)=f(a-x), x \in[0, a]$.  证明 
\end{enumerate}
$$
\int_{0}^{a} x f(x) \mathrm{d} x=\frac{a}{2} \int_{0}^{a} f(x) \mathrm{d} x .
$$

\begin{enumerate}
  \setcounter{enumi}{8}
  \item ( 20  分 )  设函数列 
\end{enumerate}
$$
f_{n}(x)=n^{2}\left(\frac{x}{n}-\ln \left(1+\frac{x}{n}\right)\right), x \in[0,+\infty), n \in N^{*} .
$$
(1)  证明 : 函数列  $f_{n}(x)$  在  $[0,+\infty)$  的任意有界闭子区间上一致收敛 ;

( 2 )  证明 : $e^{f_{n}(x)} \geqslant e^{\frac{x^{2}}{8}}, x \in[0, n)$;

(3)  计算  $\lim _{n \rightarrow+\infty} \int_{0}^{n} e^{-f_{n}(x)} \mathrm{d} x$.

\section{1. 大连理工大学 2009 年研究生入学考试试题高等代数 
 李扬 
 微信公众号: sxkyliyang}
\begin{enumerate}
  \item  设多项式  $f(x)=x^{2 s+1}+x^{2 t+1}+a$,  讨论  $f(x)$  的实根个数 .  其中  $s, t$  均为正整数 .

  \item  实系数多项式  $f(x)=x^{4}-6 x^{3}+a x^{2}+b x+2$  有  4  个实根 ,  证明 :  至少有一个根小于  1 .

  \item  证明 : $f^{\prime}(x) \mid f(x)$  的充要条件是  $f(x)$  可以表示成  $f(x)=k(x-a)^{n}$.

  \item  证明下面的行列式 :

\end{enumerate}
$$
\left|\begin{array}{ccccc}
1 & 1 & \cdots & 1 & 1 \\
1 & c_{2}^{1} & \cdots & c_{n-1}^{1} & c_{n}^{1} \\
1 & c_{3}^{2} & \cdots & c_{n}^{2} & c_{n+1}^{2} \\
\vdots & \vdots & & \vdots & \vdots \\
1 & c_{n}^{n-1} & \cdots & c_{2 n-3}^{n-1} & c_{2 n-1}^{n-1}
\end{array}\right|=1
$$
 其中  $c_{i}^{k}$  表示组合数 .

\begin{enumerate}
  \setcounter{enumi}{5}
  \item  已知  $A^{2}=A$,  且  $0 \leqslant r(A) \leqslant n$,  证明  $A$  相似于  $\left(\begin{array}{cc}E_{r} & 0 \\ 0 & 0\end{array}\right)$,  并证明  $\left|A+I_{n}\right|=2^{r}$.

  \item  已知  $\alpha_{1}, \alpha_{2}, \alpha_{3}$  为  $\alpha_{1}, \alpha_{2}, \cdots, \alpha_{n}$  的极大线性无关组 ,  且  $\alpha_{1}=\beta_{1}+\beta_{2}+\beta_{3}, \alpha_{2}=\beta_{1}+2 \beta_{2}+\beta_{3}$, $\alpha_{3}=\beta_{1}+2 \beta_{2}+3 \beta_{3}$,  求证  $\beta_{1}, \beta_{2}, \beta_{3}$  也为  $\alpha_{1}, \alpha_{2}, \cdots, \alpha_{n}$  的极大线性无关组 .

  \item  已知  $r(A)=1$,  证明 :  存在数  $k$,  使得  $A^{2}=k A$.

  \item  设  $A=\left(a_{i j}\right)_{n \times n}$,  且对于  $\forall i, j$,  有  $\left|a_{i j}\right| \leqslant k$,  设  $A^{m}=\left(b_{i j}\right)_{n \times n}$,  证明对于  $\forall i, j$,  有  $\left|b_{i j}\right| \leqslant n^{m-1} k^{m}$.

  \item  设  $A, B$  为复数域上的  $n$  阶矩阵 ,  且  $A B=B A$.  求证 : $A$  与  $B$  具有相同的特征向量 .

  \item  设  $\mathscr{A}, \mathscr{B}, \mathscr{C}$  是线性空间  $V$  上两两不同的线性变换 ,  求证 :  存在  $\alpha \in V$,  使得  $\mathscr{A} \alpha, \mathscr{B} \alpha, \mathscr{C} \alpha$  两两不同 .

  \item  证明 :  已知  $A=\left(a_{i j}\right)_{m \times n}, A x=\beta$  有解的充分必要条件是对于方程组  $A^{\prime} y=0$  的每一组解  $c$  都有  $\beta^{\prime} c=0$.

  \item  证明  $\sum_{i=1}^{n} a_{i} x_{i}=0$  与  $\sum_{i=1}^{n} b_{i} x_{i}=0$  同解的充分必要条件是系数对应成比例 .

\end{enumerate}
\section{2. 大连理工大学 2011 年研究生入学考试试题高等代数 
 李扬 
 微信公众号: sxkyliyang}
\begin{enumerate}
  \item (10  分 )  设  $g_{i}(x)$  是数域  $P$  上的  $i$  次多项式 ,  其首项系数为  $i+1(i=0,1,2, \cdots, n-1)$,  试计算  $n$  级行列式 :
\end{enumerate}
$$
\left|\begin{array}{cccc}
g_{0}(1) & g_{0}(2) & \cdots & g_{0}(n) \\
g_{1}(1) & g_{1}(2) & \cdots & g_{1}(n) \\
\vdots & \vdots & & \vdots \\
g_{n-1}(1) & g_{n-1}(2) & \cdots & g_{n-1}(n)
\end{array}\right|
$$

\begin{enumerate}
  \setcounter{enumi}{2}
  \item ( 20  分 )  设  $A=\left(\begin{array}{cccc}1 & 1 & 0 & 0 \\ 0 & 1 & 1 & s \\ 1 & 2 & 1 & -1\end{array}\right), \beta=\left(\begin{array}{c}1 \\ 0 \\ t\end{array}\right), \gamma_{1}=\left(\begin{array}{l}0 \\ 0 \\ 1 \\ 1\end{array}\right), \gamma_{2}=\left(\begin{array}{l}2 \\ 1 \\ 0 \\ 1\end{array}\right)$.
\end{enumerate}
(1)  已知数域  $P$  上的线性方程组  $A x=\beta$  有解 ,  求  $s$  和  $t$  需要满足的条件 ;

(2)  当  $s=0$  时 ,  求  $P$  上的齐次线性方程组  $A X=\left(\begin{array}{l}0 \\ 0 \\ 0\end{array}\right)$  的基础解系 ;

(3)  当  $s=0, t=1$  时 ,  给出  $A x=\beta$  的两个线性无关的解 ;

(4)  已知某齐次线性方程组的通解为  $k_{1} \gamma_{1}+k_{2} \gamma_{2}, k_{1}, k_{2} \in P$,  求这个齐次线性方程组与 (2) 中方程组的所   有公共解 .

\begin{enumerate}
  \setcounter{enumi}{3}
  \item ( 20  分 )  设  4  级实矩阵  $A=\left(\begin{array}{llll}1 & s & s & s \\ s & 1 & s & s \\ s & s & 1 & s \\ s & s & s & 1\end{array}\right)$.
\end{enumerate}
(1)  求  $A$  的特征值及所有特征向量 ;

(2) $A$  可逆的充要条件是什么 ?  当  $s=-1$  时 , $A$  是否可逆 ?  若  $A$  可逆 ,  求  $A$  的逆矩阵  $A^{-1}$;

(3) $s$  为何值时 , $A$  是正定矩阵 ?

\begin{enumerate}
  \setcounter{enumi}{4}
  \item ( 15  分 )  设  $V=\left\{\left(\begin{array}{ccc}a & b-a & -b \\ -a-b & 0 & a+b \\ b & a-b & -a\end{array}\right) \mid a, b \in P\right\} \subseteq P^{3 \times 3}$  是数域  $P$  上  3  阶实方阵的子集合 .
\end{enumerate}
(1)  证明  $V$  是  $P^{3 \times 3}$  的一个子空间 ;

(2)  证明  $V=V_{1} \oplus V_{2}$,  其中  $V_{1}$  是由  $V$  中对称矩阵构成的集合 , $V_{2}$  是由  $V$  中反对称矩阵构成的集合 ;

(3)  求  $V$  的一组基底 ,  并求  $\left(\begin{array}{ccc}2 & 3 & -5 \\ -7 & 0 & 7 \\ 5 & -3 & -2\end{array}\right)$  在该基底下的坐标 .

\begin{enumerate}
  \setcounter{enumi}{5}
  \item (25  分 )
\end{enumerate}
(1) 设  $A=\left(\begin{array}{ccc}2 & -5 & k \\ 1 & -4 & 1 \\ 0 & 0 & 1\end{array}\right)$,  问  $k$  为何值时 ,  存在可逆矩阵  $T$  使得  $T^{-1} A T$  是对角阵 ? (2)  设  $A=\left(\begin{array}{ccc}0 & 1 & 1 \\ 1 & 0 & 1 \\ 1 & 1 & 0\end{array}\right)$,  求正交矩阵  $Q$,  使得  $Q^{-1} A Q=\left(\begin{array}{ccc}2 & 0 & 0 \\ 0 & -1 & 0 \\ 0 & 0 & -1\end{array}\right)$.

(3) $A$  是  4  级复矩阵且有特征值  1 ,  又  $A$  只有一个线性无关的特征向量 ,  求  $A$  的  Jordan  标准形 .

\begin{enumerate}
  \setcounter{enumi}{6}
  \item (10  分 )  设  $f(x)=a_{n} x^{n}+a_{n-1} x^{n-1}+\cdots+a_{1} x+a_{0}\left(a_{0} \neq 0\right)$  是数域  $P$  上的  $n$  次多项式 .
\end{enumerate}
(1)  若  $f(x)$  有  $n$  个根  $x_{1}, x_{2}, \cdots, x_{n}$,  求以  $\frac{1}{x_{1}}, \frac{1}{x_{2}}, \cdots, \frac{1}{x_{n}}$  为根的  $n$  次多项式 .

(2)  若  $f(x)$  可约 ,  证明多项式  $g(x)=a_{0} x^{n}+a_{1} x^{n-1}+\cdots+a_{n-1} x+a_{n}$  在  $P$  上也可约 .

\begin{enumerate}
  \setcounter{enumi}{7}
  \item ( 20  分 )  设实数域  $\mathbb{R}$  中所有  2  阶对称阵构成子空间  $V$,  在  $V$  中定义内积  $(A, B)=\operatorname{tr}(A B)$.
\end{enumerate}
(1) 证明  $V$  关于内积  $(A, B)=\operatorname{tr}(A B)$  是一个欧式空间 ;

( 2 )  求  $V$  的一组标准正交基 ;

(3)  在  $V$  中求向量  $A=\left(\begin{array}{cc}1 & 1 \\ 1 & 1\end{array}\right)$  和  $B=\left(\begin{array}{cc}-1 & 0 \\ 0 & -1\end{array}\right)$  的夹角  $\varphi$;

(4)  设  $C=\left(\begin{array}{ll}1 & 0 \\ 0 & 0\end{array}\right)$,  求子空间  $M=\{B \in V \mid(B, C)=0\}$  的维数 .

\begin{enumerate}
  \setcounter{enumi}{8}
  \item ( 15  分 )  设  $V$  是数域  $P$  上的  $n$  维线性空间 , $\mathscr{A}$  是  $V$  上的线性变换 ,  且存在  $\alpha \in V$,  使得  $V=$ $L\left(\alpha, \mathscr{A} \alpha, \mathscr{A}^{2} \alpha, \cdots\right)$,  其中  $L\left(\alpha, \mathscr{A} \alpha, \mathscr{A}^{2} \alpha, \cdots\right)$  表示  $\alpha, \mathscr{A} \alpha, \mathscr{A}^{2} \alpha, \cdots$  生成的  $V$  的子空间 .
\end{enumerate}
(1)  证明  $\alpha, \mathscr{A} \alpha, \mathscr{A}^{2} \alpha, \cdots, \mathscr{A}^{n-1} \alpha$  是  $V$  的一组基底 .

(2)  求  $\mathscr{A}$  在这组基下的矩阵 ,  及  $\mathscr{A}$  的特征多项式与最小多项式 .

\begin{enumerate}
  \setcounter{enumi}{9}
  \item (15  分 )  已知数域  $P$  上的非齐次线性方程组  $A x=\beta$  有解 ,  其中  $A, \beta$  分别是  $m \times n$  和  $m \times 1$  型矩阵 ,  求证 :  每个解向量的第  $k$  个分量都等于零的充分必要条件是将增广矩阵  $(A, \beta)$  的第  $k$  列划去之后得到的矩阵的秩   比  $(A, \beta)$  的秩小 .
\end{enumerate}
\section{3. 大连理工大学 2012 年研究生入学考试试题高等代数 
 李扬 
 微信公众号: sxkyliyang}
\begin{enumerate}
  \item ( 10  分 )  计算  $n$  阶行列式  ( 其中  $a_{i}$  均不为零 )  的值 .
\end{enumerate}
$$
D_{n}=\left|\begin{array}{ccccc}
1+a_{1} & 1 & 1 & \cdots & 1 \\
1 & 1+a_{2} & 1 & \cdots & 1 \\
1 & 1 & 1+a_{3} & \cdots & 1 \\
\vdots & \vdots & \vdots & & \vdots \\
1 & 1 & 1 & \cdots & 1+a_{n}
\end{array}\right|
$$

\begin{enumerate}
  \setcounter{enumi}{2}
  \item (10  分 )
\end{enumerate}
(1) 叙述并证明艾森斯坦  (Eisenstein)  判别法 .

(2)  举例说明在有理数域上存在任意次数的不可约多项式  ( 给出理由 ).

( 3 )  设  $n>1$,  证明  $n$  个互不相同的素数的几何平均数一定为无理数 .

\begin{enumerate}
  \setcounter{enumi}{3}
  \item (20  分 ) $k$  取何值时 ,  线性方程组  $\left\{\begin{array}{l}k x+y+z=-2 ; \\ x+k y+z=k ; \\ x+y+k z=k^{2} .\end{array}\right.$,
\end{enumerate}
(1) 有唯一解 ;

(2)  无解 ;

(3) 有无穷多解 ?  并在此条件下求通解 .

\begin{enumerate}
  \setcounter{enumi}{4}
  \item ( 20  分 )  设矩阵  $A$  和  $B$  满足  $A B=2 A+3 B$.
\end{enumerate}
(1)  证明 : $A B=B A$.

( 2 )  若  $A=\left(\begin{array}{lll}0 & 0 & 0 \\ 0 & 1 & 0 \\ 0 & 0 & 2\end{array}\right)$,  求  $B$.

\begin{enumerate}
  \setcounter{enumi}{5}
  \item (20  分 )
\end{enumerate}
(1)  求一正交变换  $x=Q y$  将二次型  $f\left(x_{1}, x_{2}, x_{3}\right)=2 x_{1} x_{2}+2 x_{2} x_{3}+2 x_{1} x_{3}$  化为标准型 ,  并写出相应的标准   开䄯 .

(2)  在空间直角坐标系  $O x y z$  中 ,  二次曲面方程  $2 x y+2 y z+2 z x=1$  表示何种曲面 ?

\begin{enumerate}
  \setcounter{enumi}{6}
  \item ( 20  分 )  设  $A$  为  $n$  阶幂等阵  ( 即  $\left.A^{2}=A\right)$.
\end{enumerate}
(1)  证明  $A$  相似于对角阵 ;

(2) $r(A)+r(I-A)=n$,  其中  $I$  为  $n$  级单位阵 .

\begin{enumerate}
  \setcounter{enumi}{7}
  \item ( 20  分 )  给定  $\mathbb{R}^{4}$  的两个子空间 ,
\end{enumerate}
$$
\begin{gathered}
V_{1}=\left\{\left(x_{1}, x_{2}, x_{3}, x_{4}\right)^{\prime} \mid 2 x_{1}-x_{2}+4 x_{3}-3 x_{4}=0, x_{1}+x_{3}-x_{4}=0\right\} \\
V_{2}=\left\{\left(x_{1}, x_{2}, x_{3}, x_{4}\right)^{\prime} \mid 3 x_{1}+x_{2}+x_{3}=0,7 x_{1}+7 x_{3}-3 x_{4}=0\right\}
\end{gathered}
$$
 分别求  $V_{1}+V_{2}, V_{1} \cap V_{2}$  的一个基和维数 . 8. (10  分 )  设  $V$  是实数域上的  $n$  维线性空间 , $\mathscr{A}$  是  $V$  上的线性变换且  $\mathscr{A}^{2}=\mathscr{A}$,  证明 :

(1) $\operatorname{Ker} \mathscr{A}=\{\alpha-\mathscr{A}(\alpha) \mid \alpha \in V\}$;

(2) $V=\operatorname{Ker} \mathscr{A} \oplus \operatorname{Im} \mathscr{A}$.

\begin{enumerate}
  \setcounter{enumi}{9}
  \item (10  分 )  设  $\mathscr{A}$  是欧几里得空间  $V$  上的一个正交线性变换 ,  证明 :  若  $W$  是  $\mathscr{A}$  的不变子空间 ,  则正交补  $W^{\perp}$  也   是  $\mathscr{A}$  的不变子空间 .

  \item (10  分 )  设  $A$  为  $n$  级实对称正定阵 , $x$  为  $n$  维非零列向量 ,  证明 : $0<x^{\prime}\left(A+x x^{\prime}\right)^{-1} x<1$.

\end{enumerate}
\section{4. 大连理工大学 2013 年研究生入学考试试题高等代数 
 李扬 
 微信公众号: sxkyliyang}
\begin{enumerate}
  \item  计算行列式 
\end{enumerate}
$$
D_{n}=\left|\begin{array}{ccccccc}
1 & -1 & 0 & 0 & \cdots & 0 & 0 \\
a_{1} & 1-a_{1} & -1 & 0 & \cdots & 0 & 0 \\
0 & a_{2} & 1-a_{2} & -1 & \cdots & 0 & 0 \\
0 & 0 & a_{3} & 1-a_{3} & \cdots & 0 & 0 \\
\vdots & \vdots & \vdots & \vdots & & \vdots & \vdots \\
0 & 0 & 0 & 0 & \cdots & a_{n} & 1-a_{n}
\end{array}\right|
$$

\begin{enumerate}
  \setcounter{enumi}{2}
  \item  已知  $A, B$  都是  $n \times n$  可逆矩阵 , $D=\left(\begin{array}{cccc}A & 0 & 0 & 0 \\ E & B & 0 & 0 \\ 0 & 0 & B & E \\ 0 & 0 & 0 & A\end{array}\right)$,  其中  $E$  是  $n \times n$  单位矩阵 ,  求  $D$  的逆矩阵  $D^{-1}$.

  \item $a, b$  取何值时 , $x-1$  是多项式  $f(x)=\left(x^{2}+a x+3\right)\left(x^{2}-b\right)$  的重因式 .

  \item  设  $A=\left(\begin{array}{ccc}3 & 0 & 0 \\ -4 & 3 & 0 \\ 0 & -4 & 5\end{array}\right)$,

\end{enumerate}
(1)  若  $B=(E+A)^{-1}(E-A)$,  求  $E+B$.

(2)  若  $B A-4 E=A^{-1} B A$,  求  $B^{-1}$.

5 .  设  $A=\left(\begin{array}{lll}3 & 2 & 0 \\ 2 & 0 & 0 \\ 0 & 0 & 2\end{array}\right)$,  求一个正交矩阵  $Q$,  使得  $Q^{-1} A Q$  为对角形矩阵 .

\begin{enumerate}
  \setcounter{enumi}{6}
  \item (1)  写出一个  $3 \times 3$  矩阵  $A$  使得  $(x, y, z) A=(x+y,-y, 0)$,  其中  $x, y, z$  为实数 .
\end{enumerate}
( 2 )  若一个  $3 \times 3$  矩阵的特征值为  $-1,0,1$,  求  $A$  的秩及行列式  $|A+2 E|$.

\begin{enumerate}
  \setcounter{enumi}{7}
  \item  设  $q_{1}, q_{2}, q_{3}$  是三个四元实列向量 ,  并记  $Q=\left(q_{1}, q_{2}, q_{3}\right)$,  若  $q_{1}, q_{2}, q_{3}$  两两正交且长度相等 ,  则 
\end{enumerate}
(1)  求齐次线性方程组  $Q x=0$  的解空间的维数 ;

( 2 )  求  $Q x=q_{1}+2 q_{2}+4 q_{3}$  的最小二乘解 ;

(3) 设  $v$  是  $q_{1}, q_{2}, q_{3}$  生成的线性空间之外的一个四元实列向量 ,  通过对  $q_{1}, q_{2}, q_{3}, v$  进行施密特正交化 ,  写   出第四个正交列向量  $q_{4}$.

\begin{enumerate}
  \setcounter{enumi}{8}
  \item  设  $v_{1}$  是域  $F$  上的有限维线性空间  $V$  的子空间 , $V^{*}$  是  $V$  的对偶空间 ,  记  $V_{1}^{\perp}=\left\{f \in V^{*} \mid f(v)=0, \forall v \in V_{1}\right\}$.  证明 :
\end{enumerate}
(1) $V_{1}^{\perp}$  是  $V^{*}$  的子空间 ;

( 2 )  若  $V_{2}$  是  $V$  的另一个子空间 ,  则  $V_{1}^{\perp} \cap V_{2}^{\perp}=\left(V_{1}+V_{2}\right)^{\perp}$.

\begin{enumerate}
  \setcounter{enumi}{9}
  \item  设  $A$  和  $B$  都是  $n$  阶方阵 ,  且  $r(A)+r(B)<n$,  其中  $r(A)$  表示  $A$  的秩 ,  证明  $A$  和  $B$  至少有一个公共特征向   量 .

  \item  设  $A$  和  $B$  都是  $n$  阶正定阵 ,  证明  $A B$  是正定阵的充要条件是  $A B=B A$.

\end{enumerate}
\section{5. 大连理工大学 2015 年研究生入学考试试题高等代数 
 李扬 
 微信公众号: sxkyliyang}
\begin{enumerate}
  \item  填空题  ( 共  10  题 ,  每题  5  分 ,  共  50  分 )
\end{enumerate}
(1)  如果  $(x-1)^{2} \mid a x^{4}+b x^{2}+1$,  则  $a=$ $b=$

( 2 )  设  3  级矩阵  $A$  的列分块矩阵为  $A=\left(\alpha_{1}, \alpha_{2}, \alpha_{1}+\alpha_{2}\right)$,  若  $\alpha_{1}, \alpha_{2}$  线性无关 ,  且  $\beta=2 \alpha_{1}+\alpha_{2}$,  则线性方   程组  $A X=\beta$  的通解为 

(3)  若方阵  $A$  满足  $A^{2}+A-4 E=0$,  则  $(A+3 E)^{-1}=$

(4)  设  $A$  为三级矩阵 ,  且  $2 E-A, E-A,-E-A$  的秩都小于  $3, A$  的行列式  $|A|=$

(5)  已知矩阵  $A=\left(\begin{array}{ccc}2 & -5 & k \\ 1 & -4 & 1 \\ 0 & 0 & 1\end{array}\right)$  可对角化 ,  则  $k=$

(6)  在实线性空间  $\mathbb{R}^{3}$,  对于  $\mathbb{R}^{3}$  中的向量  $\alpha=\left(x_{1}, x_{2}, x_{3}\right), \beta=\left(y_{1}, y_{2}, y_{3}\right)$,  定义  $\mathbb{R}^{3}$  上的二元运算为  $(\alpha, \beta)=x_{1} y_{1}+x_{2} y_{2}+x_{3} y_{3}$,  则  $\mathbb{R}^{3}$  成为欧式空间 ,  在此欧式空间  $\mathbb{R}^{3}$  中 ,  向量  $\alpha=(1,2,2)$  的长度是  ,  向量  $\alpha=(1,2,2)$  与  $\beta=(2,-1,0)$  之间的夹角是 

(7)  设矩阵  $A$  的特征多项式为  $f(\lambda)=\lambda^{2}-5 \lambda+6$,  则  $A$  可逆 , $A^{-1}$  的特征多项式为 

(8)  设四级矩阵  $A$  的最小多项式为  $m(\lambda)=(\lambda-1)^{2}(\lambda-2)$,  写出  $A$  的所有可能的若尔当标准形 

(9)  设  $f$  是数域  $P$  上三维线性空间  $V$  上的一个线性函数 , $\varepsilon_{1}, \varepsilon_{2}, \varepsilon_{3}$  是  $V$  的一组基 ,  且 
$$
f\left(\varepsilon_{1}+\varepsilon_{3}\right)=f\left(\varepsilon_{1}-2 \varepsilon_{3}\right)=0, f\left(\varepsilon_{1}+\varepsilon_{2}\right)=1
$$
 则  $f\left(x_{1} \varepsilon_{1}+x_{2} \varepsilon_{2}+x_{3} \varepsilon_{3}\right)=$

(10 )  设  $f(\alpha, \beta)$  是数域  $P$  上三维线性空间  $V$  上的一个双线性函数 , $\varepsilon_{1}, \varepsilon_{2}, \varepsilon_{3}$  是  $V$  的一组基 ,  矩阵 
$$
A=\left(\begin{array}{lll}
1 & 0 & 1 \\
0 & 2 & 1 \\
2 & 1 & 0
\end{array}\right)
$$
 是  $f(\alpha, \beta)$  在  $\varepsilon_{1}, \varepsilon_{2}, \varepsilon_{3}$  下的度量矩阵 ,  设  $\alpha=2 \varepsilon_{1}+\varepsilon_{2}-\varepsilon_{3}, \beta=\varepsilon_{1}-\varepsilon_{2}$,  则  $f(\alpha, \beta)=$

\begin{enumerate}
  \setcounter{enumi}{2}
  \item (10  分 )  已知两个齐次线性方程组 :
\end{enumerate}
$$
\left\{\begin{array} { l } 
{ x _ { 1 } + 2 x _ { 2 } + 3 x _ { 3 } = 0 ; } \\
{ 2 x _ { 1 } + 3 x _ { 2 } + 5 x _ { 3 } = 0 ; } \\
{ x _ { 1 } + x _ { 2 } + a x _ { 3 } = 0 }
\end{array} \text { 与 } \left\{\begin{array}{l}
x_{1}+b x_{2}+c x_{3}=0 \\
2 x_{1}+b^{2} x_{2}+(c+1) x_{3}=0
\end{array}\right.\right.
$$
 同解 ,  求  $a, b, c$.

\begin{enumerate}
  \setcounter{enumi}{3}
  \item ( 10  分 )  设  $\eta_{0}$  是某非齐次线性方程组的一个特解 , $\eta_{1}, \cdots, \eta_{t}$  是其导出组的一个基础解系 ,  证明向量组  $\eta_{0}, \eta_{1}, \cdots, \eta_{t}$  线性无关 .

  \item (15  分 )  设  $A=\left(\begin{array}{lll}2 & 1 & 0 \\ 1 & 2 & 0 \\ 0 & 0 & t\end{array}\right), B=\left(\begin{array}{lll}1 & 0 & 0 \\ 0 & 1 & 1 \\ 0 & 1 & 0\end{array}\right)$.

\end{enumerate}
(1) $t$  为何值时 , $A$  与  $B$  等价 ; (2) $t$  为何值时 , $A$  与  $B$  合同 ;

(3) $t$  为何值时 , $A$  是正定矩阵 .

\begin{enumerate}
  \setcounter{enumi}{5}
  \item (15  分 )  设向量组  $\alpha_{1}, \cdots, \alpha_{s} ; \beta_{1}, \cdots, \beta_{t} ; \alpha_{1}, \cdots, \alpha_{s}, \beta_{1}, \cdots, \beta_{t}$  的秩分别为  $r_{1}, r_{2}, r_{3}$.  证明 :
\end{enumerate}
$$
\max \left(r_{1}, r_{2}\right) \leqslant r_{3} \leqslant r_{1}+r_{2}
$$

\begin{enumerate}
  \setcounter{enumi}{6}
  \item (10  分 )  设  $f\left(x_{1}, x_{2}, x_{3}\right)=2 x_{1}^{2}+2 x_{2}^{2}+2 x_{3}^{2}-6 x_{1} x_{2}-6 x_{1} x_{3}-6 x_{2} x_{3}$.
\end{enumerate}
(1) 用正交线性替换化下列二次型为标准形 ;

(2)  在空间直角坐标系  $O-X_{1} X_{2} X_{3}$  中 , $f\left(x_{1}, x_{2}, x_{3}\right)=1$  表示何种曲面 .

\begin{enumerate}
  \setcounter{enumi}{7}
  \item (10  分 )  设  3  维线性空间  $V$  上线性变换  $\mathscr{A}$  在基  $\varepsilon_{1}, \varepsilon_{2}, \varepsilon_{3}$  下的矩阵  $A=\left(\begin{array}{c}1 \\ 3\end{array}\right)$,  记  $L(V)$  为  $V$  上线   性变换全体 , $C(\mathscr{A})=\{\mathscr{B} \mid \mathscr{A} \mathscr{B}=\mathscr{B} \mathscr{A}, \mathscr{B} \in L(V)\}$.
\end{enumerate}
(1)  证明 : $C(\mathscr{A})$  是  $L(V)$  的子空间 ;

(2)  求  $C(\mathscr{A})$  的一组基和维数 .

\begin{enumerate}
  \setcounter{enumi}{8}
  \item (10  分 )  设  $\mathscr{A}, \mathscr{B}$  是  $n$  维线性空间  $V$  上线性变换 ,  且  $\mathscr{A}+\mathscr{B}=1_{v}\left(1_{v}\right.$  是  $V$  上的恒等变换  $)$,  且  $r(\mathscr{A})+r(\mathscr{B})=n$,  证明 :
\end{enumerate}
(1) $V=\mathscr{A}(V) \oplus \mathscr{B}(V)$;

(2) $\mathscr{A} \mathscr{B}=\mathscr{B} \mathscr{A}=0 ; \mathscr{A}^{2}=\mathscr{A}, \mathscr{B}^{2}=\mathscr{B}$.

\section{6. 大连理工大学 2016 年研究生入学考试试题高等代数 
 李扬 
 微信公众号: sxkyliyang}
\begin{enumerate}
  \item  填空题 .
\end{enumerate}
(1)  求多项式  $f(x)$  和  $g(x)$  的最大公因式  $d(x)$,  并求  $p(x)$  和  $q(x)$  使得 : $p(x) f(x)+q(x) g(x)=d(x)$.

(2)  若  $A^{2}=0$,  则  $|E-A|=$

(3)  求矩阵的特征多项式与特征值 .

(4)  最小二乘法 .

(5)  已知  $f(x)=1$,  给出矩阵  $A$,  求  $f(A)$  以及  $[f(A)]^{-1}$.

(6)  求对偶基 .

\begin{enumerate}
  \setcounter{enumi}{2}
  \item  证明 : $A^{\prime} A$  正定的充要条件是  $r(A)=n$.

  \item  证明 :  一个实二次型可以分解成两个实系数的一次齐次多项式的乘积的充分必要条件是它的秩等于  2  和符号   差等于  0 ,  或者秩等于  1 .

  \item  给出矩阵  $A$  和  $B, A$  和  $B$  相似 、 合同时求出矩阵中的末知参数 .

  \item  判断是否为线性空间 ,  求内积 ,  以及  $W^{\perp}$,  其中  $W=L\left(\left[\begin{array}{ll}0 & 1 \\ 0 & 0\end{array}\right]\right)$.

  \item  已知两个线性变换  $\mathscr{A}, \mathscr{B}$.

\end{enumerate}
(1)  求两个线性变换的特征值与特征向量 ;

(2)  求证线性变换在一组基下的矩阵是上三角矩阵 ;

\begin{enumerate}
  \setcounter{enumi}{7}
  \item  已知  $x_{1}, x_{2}, \cdots, x_{n}$  是欧式空间的一组基底 ,  证明存在一向量  $q$,  使得  $\left(q, x_{i}\right)>0(i=1,2, \cdots, n)$.
\end{enumerate}

\end{document}