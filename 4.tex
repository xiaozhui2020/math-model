\documentclass[10pt]{article}
\usepackage[utf8]{inputenc}
\usepackage[T1]{fontenc}
\usepackage{CJKutf8}
\usepackage{amsmath}
\usepackage{amsfonts}
\usepackage{amssymb}
\usepackage{mhchem}
\usepackage{stmaryrd}
\usepackage{bbold}
\usepackage{mathrsfs}
\usepackage{graphicx}
\usepackage[export]{adjustbox}
\graphicspath{ {./images/} }

\begin{document}
\section{7. 西北大学 2015 年研究生入学考试试题数学分析}
\begin{CJK}{UTF8}{mj}李扬\end{CJK}

\begin{CJK}{UTF8}{mj}微信公众号\end{CJK}: sxkyliyang

\begin{CJK}{UTF8}{mj}一\end{CJK}、\begin{CJK}{UTF8}{mj}填空题\end{CJK} (\begin{CJK}{UTF8}{mj}每小题\end{CJK} 6 \begin{CJK}{UTF8}{mj}分\end{CJK}, \begin{CJK}{UTF8}{mj}本题共\end{CJK} 60 \begin{CJK}{UTF8}{mj}分\end{CJK})

\begin{enumerate}
  \item \begin{CJK}{UTF8}{mj}设\end{CJK} $f^{\prime}\left(x^{3}+1\right)=1+3 x^{6}$, \begin{CJK}{UTF8}{mj}且\end{CJK} $f(0)=1$, \begin{CJK}{UTF8}{mj}则\end{CJK} $f(x)=$

  \item \begin{CJK}{UTF8}{mj}积分\end{CJK} $\int_{1}^{+\infty} \frac{\mathrm{d} x}{x^{2}(x+1)}=$

  \item \begin{CJK}{UTF8}{mj}设\end{CJK} $z=y^{x y}$, \begin{CJK}{UTF8}{mj}则\end{CJK} $\frac{\partial z}{\partial y}=$

  \item \begin{CJK}{UTF8}{mj}曲面\end{CJK} $\mathrm{e}^{z}=x^{2}+y z$ \begin{CJK}{UTF8}{mj}在点\end{CJK} $P(1,2,0)$ \begin{CJK}{UTF8}{mj}处的切平面方程为\end{CJK}

  \item \begin{CJK}{UTF8}{mj}幂级数\end{CJK} $\sum_{n=1}^{\infty} \frac{2^{n}+3^{n}}{n}(x-1)^{n}$ \begin{CJK}{UTF8}{mj}的收敛区间为\end{CJK}

  \item $\int_{0}^{1}\left(\ln \frac{1}{x}\right)^{2} \mathrm{~d} x=$

  \item \begin{CJK}{UTF8}{mj}设曲线\end{CJK} $C$ \begin{CJK}{UTF8}{mj}是\end{CJK} $(x-1)^{2}+y^{2}=1$ \begin{CJK}{UTF8}{mj}的上半圆周\end{CJK}, \begin{CJK}{UTF8}{mj}且以\end{CJK} $A(2,0)$ \begin{CJK}{UTF8}{mj}点为起点\end{CJK}, $B(0,0)$ \begin{CJK}{UTF8}{mj}点为终点\end{CJK}, \begin{CJK}{UTF8}{mj}则曲线积分\end{CJK} $\int_{C} 3 x^{2}(y+$ 1) $\mathrm{d} x+x^{3} \mathrm{~d} y=$

  \item \begin{CJK}{UTF8}{mj}设\end{CJK} $p \neq-1$, \begin{CJK}{UTF8}{mj}则\end{CJK} $\lim _{n \rightarrow \infty} \frac{1^{p}+2^{p}+\cdots+n^{p}}{n^{p+1}}=$

  \item \begin{CJK}{UTF8}{mj}设\end{CJK} $f(x)=\left\{\begin{array}{l}a x+b, 0 \leq x \leq 1 \\ \frac{\mathrm{e}^{x}-b}{x-1}, 1<x \leq 2\end{array}\right.$ \begin{CJK}{UTF8}{mj}在\end{CJK} $x=1$ \begin{CJK}{UTF8}{mj}连续\end{CJK}, \begin{CJK}{UTF8}{mj}则\end{CJK} $a=$ $b=$

  \item \begin{CJK}{UTF8}{mj}函数\end{CJK} $\sin ^{2} x$ \begin{CJK}{UTF8}{mj}的麦克劳林级数为\end{CJK}

\end{enumerate}
\begin{CJK}{UTF8}{mj}二\end{CJK}、\begin{CJK}{UTF8}{mj}计算与证明\end{CJK} (\begin{CJK}{UTF8}{mj}每题\end{CJK} 15\begin{CJK}{UTF8}{mj}分\end{CJK})

\begin{enumerate}
  \item \begin{CJK}{UTF8}{mj}求函数\end{CJK}
\end{enumerate}
$$
f(x)=\int_{0}^{x^{2}}(2-t) \mathrm{e}^{-t} \mathrm{~d} t
$$
\begin{CJK}{UTF8}{mj}的最大与最小值\end{CJK}.

\begin{enumerate}
  \setcounter{enumi}{2}
  \item \begin{CJK}{UTF8}{mj}求极限\end{CJK}
\end{enumerate}
$$
\lim _{x \rightarrow 0}\left(\frac{\sin x}{x}\right)^{\frac{1}{x^{2}}}
$$

\begin{enumerate}
  \setcounter{enumi}{3}
  \item \begin{CJK}{UTF8}{mj}求幂级数\end{CJK}
\end{enumerate}
$$
\sum_{n=1}^{\infty} \frac{n^{2}+1}{n} x^{n-1}
$$
\begin{CJK}{UTF8}{mj}的和函数\end{CJK} $S(x)$.

\begin{enumerate}
  \setcounter{enumi}{4}
  \item \begin{CJK}{UTF8}{mj}计算积分\end{CJK}
\end{enumerate}
$$
I=\iiint_{D}(x+y+z) \mathrm{d} x \mathrm{~d} y \mathrm{~d} z
$$
\begin{CJK}{UTF8}{mj}其中\end{CJK} $D$ \begin{CJK}{UTF8}{mj}是由平面\end{CJK} $x+y+z=1$ \begin{CJK}{UTF8}{mj}及三个坐标面围成\end{CJK}. 5. \begin{CJK}{UTF8}{mj}设\end{CJK} $f(x)$ \begin{CJK}{UTF8}{mj}在\end{CJK} $[0,+\infty)$ \begin{CJK}{UTF8}{mj}具有连续的二阶导数\end{CJK}, \begin{CJK}{UTF8}{mj}且当\end{CJK} $x \in(0,+\infty)$ \begin{CJK}{UTF8}{mj}时\end{CJK}, \begin{CJK}{UTF8}{mj}有\end{CJK} $|f(x)| \leq M_{0}<+\infty,\left|f^{\prime \prime}(x)\right| \leq M_{2}<+\infty$, \begin{CJK}{UTF8}{mj}证明\end{CJK}:
$$
\left|f^{\prime}(x)\right| \leq \frac{2 M_{0}}{t}+\frac{t}{2} M_{2}
$$
\begin{CJK}{UTF8}{mj}对\end{CJK} $\forall t>0, x>0$ \begin{CJK}{UTF8}{mj}成立\end{CJK};
$$
\left|f^{\prime}(x)\right| \leq 2 \sqrt{M_{0} M_{2}}, x \in[0,+\infty)
$$

\begin{enumerate}
  \setcounter{enumi}{6}
  \item \begin{CJK}{UTF8}{mj}设\end{CJK} $f_{0}(x)$ \begin{CJK}{UTF8}{mj}在\end{CJK} $[a, b]$ \begin{CJK}{UTF8}{mj}上连续\end{CJK}, $g(x, y)$ \begin{CJK}{UTF8}{mj}在闭区域\end{CJK} $D=\{a \leq x \leq b, a \leq y \leq b\}$ \begin{CJK}{UTF8}{mj}上连续\end{CJK}, $\forall x \in[a, b]$, \begin{CJK}{UTF8}{mj}令\end{CJK}
\end{enumerate}
$$
f_{n}(x)=\int_{a}^{x} g(x, y) f_{n-1}(y) \mathrm{d} y, n=1,2, \cdots .
$$
\begin{CJK}{UTF8}{mj}证明\end{CJK}: $f_{n}(x)$ \begin{CJK}{UTF8}{mj}在\end{CJK} $[a, b]$ \begin{CJK}{UTF8}{mj}上一致收敛于\end{CJK} 0 .

\section{8. 西北大学 2016 年研究生入学考试试题数学分析}
\begin{CJK}{UTF8}{mj}李扬\end{CJK}

\begin{CJK}{UTF8}{mj}微信公众号\end{CJK}: sxkyliyang

\begin{enumerate}
  \item (10 \begin{CJK}{UTF8}{mj}分\end{CJK}) \begin{CJK}{UTF8}{mj}求函数\end{CJK}
\end{enumerate}
$$
f(x)= \begin{cases}{\left[\frac{(1+x)^{\frac{1}{x}}}{\mathrm{e}}\right]^{\frac{1}{x}}} & x \neq 0 \\ 1, & x=0\end{cases}
$$
\begin{CJK}{UTF8}{mj}在\end{CJK} $x>-1$ \begin{CJK}{UTF8}{mj}上的间断点\end{CJK}, \begin{CJK}{UTF8}{mj}并说明类型\end{CJK}.

\begin{enumerate}
  \setcounter{enumi}{2}
  \item ( 15 \begin{CJK}{UTF8}{mj}分\end{CJK}) \begin{CJK}{UTF8}{mj}设\end{CJK} $f(x)$ \begin{CJK}{UTF8}{mj}在\end{CJK} $[0,1]$ \begin{CJK}{UTF8}{mj}连续\end{CJK}, \begin{CJK}{UTF8}{mj}证明\end{CJK}:
\end{enumerate}
$$
\lim _{n \rightarrow+\infty} \sqrt[n]{\sum_{i=1}^{n}\left|f\left(\frac{i}{n}\right)\right|^{n}}=\max _{0 \leq x \leq 1}|f(x)| .
$$

\begin{enumerate}
  \setcounter{enumi}{3}
  \item (15 \begin{CJK}{UTF8}{mj}分\end{CJK}) \begin{CJK}{UTF8}{mj}设\end{CJK} $f(x)$ \begin{CJK}{UTF8}{mj}在\end{CJK} $[0,1]$ \begin{CJK}{UTF8}{mj}上可导\end{CJK}, \begin{CJK}{UTF8}{mj}且\end{CJK} $f(1)-2 \int_{0}^{\frac{1}{2}} x f(x) \mathrm{d} x=0$. \begin{CJK}{UTF8}{mj}证明\end{CJK}: \begin{CJK}{UTF8}{mj}存在\end{CJK} $\xi \in(0,1)$, \begin{CJK}{UTF8}{mj}使\end{CJK}
\end{enumerate}
$$
f^{\prime}(\xi)=-\frac{f(\xi)}{\xi}
$$

\begin{enumerate}
  \setcounter{enumi}{4}
  \item (15 \begin{CJK}{UTF8}{mj}分\end{CJK}) (1) \begin{CJK}{UTF8}{mj}将\end{CJK} $\arctan x$ \begin{CJK}{UTF8}{mj}展开为\end{CJK} $x$ \begin{CJK}{UTF8}{mj}的幂级数并求收敛半径\end{CJK}.
\end{enumerate}
(2) \begin{CJK}{UTF8}{mj}证明\end{CJK}
$$
\pi=4-\frac{4}{3}+\frac{4}{5}-\cdots+(-1)^{n-1} \frac{4}{2 n-1}+\cdots
$$

\begin{enumerate}
  \setcounter{enumi}{5}
  \item (15 \begin{CJK}{UTF8}{mj}分\end{CJK}) \begin{CJK}{UTF8}{mj}计算\end{CJK}:
\end{enumerate}
$$
I=\oint_{C} \frac{x \mathrm{~d} y-y \mathrm{~d} x}{4 x^{2}+y^{2}}
$$
\begin{CJK}{UTF8}{mj}其中\end{CJK} $C$ \begin{CJK}{UTF8}{mj}是以\end{CJK} $(0,1)$ \begin{CJK}{UTF8}{mj}为圆心\end{CJK}, \begin{CJK}{UTF8}{mj}半径为\end{CJK} $r(r \neq 1)$ \begin{CJK}{UTF8}{mj}的圆周\end{CJK}.

\begin{enumerate}
  \setcounter{enumi}{6}
  \item ( 15 \begin{CJK}{UTF8}{mj}分\end{CJK}) \begin{CJK}{UTF8}{mj}求\end{CJK}
\end{enumerate}
$$
\lim _{t \rightarrow 0} \frac{1}{t^{6}} \iiint_{x^{2}+y^{2}+z^{2} \leq t^{2}} \sin \left(x^{2}+y^{2}+z^{2}\right)^{\frac{3}{2}} \mathrm{~d} x \mathrm{~d} y \mathrm{~d} z .
$$

\begin{enumerate}
  \setcounter{enumi}{7}
  \item ( 15 \begin{CJK}{UTF8}{mj}分\end{CJK}) \begin{CJK}{UTF8}{mj}设\end{CJK} $f(x)$ \begin{CJK}{UTF8}{mj}在\end{CJK} $[a,+\infty)$ \begin{CJK}{UTF8}{mj}上连续且\end{CJK}
\end{enumerate}
$$
\int_{0}^{+\infty} f(x) \mathrm{d} x
$$
\begin{CJK}{UTF8}{mj}收敛\end{CJK},

(1) \begin{CJK}{UTF8}{mj}问是否必有\end{CJK} $\lim _{x \rightarrow+\infty} f(x)=0$ ? \begin{CJK}{UTF8}{mj}为什么\end{CJK}?

(2) \begin{CJK}{UTF8}{mj}证明存在数列\end{CJK} $x_{n} \in[a,+\infty)$ \begin{CJK}{UTF8}{mj}满足\end{CJK} $\lim _{n \rightarrow+\infty} x_{n}=+\infty, \lim _{n \rightarrow \infty} f\left(x_{n}\right)=0$.

\begin{enumerate}
  \setcounter{enumi}{8}
  \item (15 \begin{CJK}{UTF8}{mj}分\end{CJK}) \begin{CJK}{UTF8}{mj}讨论函数\end{CJK}
\end{enumerate}
$$
f(x, y)= \begin{cases}\left(x^{2}+y^{2}\right) \sin \frac{1}{x^{2}+y^{2}}, & x^{2}+y^{2} \neq 0 \\ 0, & x^{2}+y^{2}=0 .\end{cases}
$$
\begin{CJK}{UTF8}{mj}在\end{CJK} $(0,0)$ \begin{CJK}{UTF8}{mj}点处的可微性及偏导数的连续性\end{CJK}.

\begin{enumerate}
  \setcounter{enumi}{9}
  \item ( 15 \begin{CJK}{UTF8}{mj}分\end{CJK}) \begin{CJK}{UTF8}{mj}证明\end{CJK}: $f(x)$ \begin{CJK}{UTF8}{mj}在区间\end{CJK} $I$ \begin{CJK}{UTF8}{mj}上一致连续的充要条件是\end{CJK}: \begin{CJK}{UTF8}{mj}对\end{CJK} $I$ \begin{CJK}{UTF8}{mj}上任意两个数列\end{CJK} $\left\{x_{n}^{\prime}\right\}$ \begin{CJK}{UTF8}{mj}与\end{CJK} $\left\{x_{n}^{\prime \prime}\right\}$, \begin{CJK}{UTF8}{mj}当\end{CJK} $x_{n}^{\prime}-x_{n}^{\prime \prime} \rightarrow 0$ \begin{CJK}{UTF8}{mj}有\end{CJK}
\end{enumerate}
$$
f\left(x_{n}^{\prime}\right)-f\left(x_{n}^{\prime \prime}\right) \rightarrow 0
$$

\begin{enumerate}
  \setcounter{enumi}{10}
  \item ( 20 \begin{CJK}{UTF8}{mj}分\end{CJK}) \begin{CJK}{UTF8}{mj}求积分\end{CJK}
\end{enumerate}
$$
I=\int_{0}^{1} \frac{\ln (1+x)}{1+x^{2}} \mathrm{~d} x
$$

\section{9. 西北大学 2017 年研究生入学考试试题数学分析}
\begin{CJK}{UTF8}{mj}李扬\end{CJK}

\begin{CJK}{UTF8}{mj}微信公众号\end{CJK}: sxkyliyang

\begin{enumerate}
  \item (10 \begin{CJK}{UTF8}{mj}分\end{CJK}) \begin{CJK}{UTF8}{mj}设\end{CJK}
\end{enumerate}
$$
f(x)= \begin{cases}\frac{\sin e x}{x}-(1+x)^{\frac{1}{x}}, & x>0 ; \\ \frac{1}{x^{a}} \int_{0}^{x}(1-\cos t) \mathrm{d} t, & x<0 .\end{cases}
$$
\begin{CJK}{UTF8}{mj}在\end{CJK} $x=0$ \begin{CJK}{UTF8}{mj}处连续\end{CJK}, \begin{CJK}{UTF8}{mj}求\end{CJK} $a, b$.

\begin{enumerate}
  \setcounter{enumi}{2}
  \item (15 \begin{CJK}{UTF8}{mj}分\end{CJK}) \begin{CJK}{UTF8}{mj}设\end{CJK} $f(x), g(x)$ \begin{CJK}{UTF8}{mj}在\end{CJK} $[a, b]$ \begin{CJK}{UTF8}{mj}上连续\end{CJK}, \begin{CJK}{UTF8}{mj}在\end{CJK} $(a, b)$ \begin{CJK}{UTF8}{mj}内可导\end{CJK}, \begin{CJK}{UTF8}{mj}且\end{CJK} $f(a)=f(b)=0$, \begin{CJK}{UTF8}{mj}证明\end{CJK}: \begin{CJK}{UTF8}{mj}存在\end{CJK} $\xi \in(a, b)$, \begin{CJK}{UTF8}{mj}使得\end{CJK}
\end{enumerate}
$$
f^{\prime}(\xi)+g^{\prime}(\xi) f(\xi)=0
$$

\begin{enumerate}
  \setcounter{enumi}{3}
  \item (15 \begin{CJK}{UTF8}{mj}分\end{CJK}) $f(x), g(x)$ \begin{CJK}{UTF8}{mj}在\end{CJK} $[a, b]$ \begin{CJK}{UTF8}{mj}上连续\end{CJK}, $f(x) \geq 0, g(x)>0$, \begin{CJK}{UTF8}{mj}证明\end{CJK}:
\end{enumerate}
$$
\lim _{n \rightarrow \infty}\left(\int_{a}^{b} f^{n}(x) g(x) \mathrm{d} x\right)^{\frac{1}{n}}=\max _{x \in[a, b]} f(x) .
$$

\begin{enumerate}
  \setcounter{enumi}{4}
  \item (15 \begin{CJK}{UTF8}{mj}分\end{CJK}) $f(x)$ \begin{CJK}{UTF8}{mj}在\end{CJK} $(-\infty,+\infty)$ \begin{CJK}{UTF8}{mj}上连续\end{CJK},
\end{enumerate}
$$
f_{n}(x)=\sum_{k=0}^{n-1} \frac{1}{n} f\left(x+\frac{k}{n}\right)
$$
\begin{CJK}{UTF8}{mj}证明\end{CJK}: $\left\{f_{n}(x)\right\}$ \begin{CJK}{UTF8}{mj}在任何有限区间上一致收敛\end{CJK}.

\begin{enumerate}
  \setcounter{enumi}{5}
  \item ( 15 \begin{CJK}{UTF8}{mj}分\end{CJK}) \begin{CJK}{UTF8}{mj}函数\end{CJK}
\end{enumerate}
$$
f(x, y)= \begin{cases}\left(x^{2}+y^{2}\right) \sin \frac{1}{x^{2}+y^{2}}, & x^{2}+y^{2} \neq 0 \\ 0, & x^{2}+y^{2}=0 .\end{cases}
$$
\begin{CJK}{UTF8}{mj}求\end{CJK} $f$ \begin{CJK}{UTF8}{mj}的偏导数并讨论连续性\end{CJK}, \begin{CJK}{UTF8}{mj}讨论\end{CJK} $f$ \begin{CJK}{UTF8}{mj}在\end{CJK} $(0,0)$ \begin{CJK}{UTF8}{mj}点的可微性\end{CJK}.

\begin{enumerate}
  \setcounter{enumi}{6}
  \item (15 \begin{CJK}{UTF8}{mj}分\end{CJK}) \begin{CJK}{UTF8}{mj}已知\end{CJK} $\sum_{n=1}^{+\infty} \frac{(-1)^{n-1}}{n^{2}}=\frac{\pi^{2}}{12}$, \begin{CJK}{UTF8}{mj}求\end{CJK}
\end{enumerate}
$$
I=\int_{0}^{+\infty} \frac{x}{1+\mathrm{e}^{x}} \mathrm{~d} x
$$

\begin{enumerate}
  \setcounter{enumi}{7}
  \item (15 \begin{CJK}{UTF8}{mj}分\end{CJK}) \begin{CJK}{UTF8}{mj}证明\end{CJK}
\end{enumerate}
$$
F(x)=\int_{0}^{+\infty} \frac{\sin x t}{1+t^{2}} \mathrm{~d} t
$$
\begin{CJK}{UTF8}{mj}在\end{CJK} $(0,+\infty)$ \begin{CJK}{UTF8}{mj}内连续且可微\end{CJK}.

\begin{enumerate}
  \setcounter{enumi}{8}
  \item ( 15 \begin{CJK}{UTF8}{mj}分\end{CJK}) \begin{CJK}{UTF8}{mj}求曲线\end{CJK} $y^{2}=x, y^{2}=4 x, x y=1, x y=2$ \begin{CJK}{UTF8}{mj}围成的图形\end{CJK} $D$ \begin{CJK}{UTF8}{mj}的面积\end{CJK}.

  \item ( 20 \begin{CJK}{UTF8}{mj}分\end{CJK}) \begin{CJK}{UTF8}{mj}设\end{CJK} $S$ \begin{CJK}{UTF8}{mj}为封闭光滑曲面\end{CJK}, \begin{CJK}{UTF8}{mj}向量\end{CJK} $\vec{l}$ \begin{CJK}{UTF8}{mj}为任意固定的已知向量\end{CJK}, \begin{CJK}{UTF8}{mj}求\end{CJK}

\end{enumerate}
$$
\iint_{S} \cos (\vec{l}, \vec{n}) \mathrm{d} S
$$
$\vec{n}$ \begin{CJK}{UTF8}{mj}为外法线单位方向向量\end{CJK}.

\section{0. 西北大学 2009 年研究生入学考试试题高等代数}
\begin{CJK}{UTF8}{mj}李扬\end{CJK}

\begin{CJK}{UTF8}{mj}微信公众号\end{CJK}: sxkyliyang

\begin{enumerate}
  \item ( 10 \begin{CJK}{UTF8}{mj}分\end{CJK}) \begin{CJK}{UTF8}{mj}证明\end{CJK} $x^{3 n}+x^{3 n+1}+x^{3 l+2}$ \begin{CJK}{UTF8}{mj}能被\end{CJK} $x^{2}+x+1$ \begin{CJK}{UTF8}{mj}整除\end{CJK}. $(n, l \in \mathbb{Z})$

  \item (10 \begin{CJK}{UTF8}{mj}分\end{CJK}) \begin{CJK}{UTF8}{mj}设\end{CJK} $f(x)$ \begin{CJK}{UTF8}{mj}是一个\end{CJK} $n$ \begin{CJK}{UTF8}{mj}次本原多项式\end{CJK}, \begin{CJK}{UTF8}{mj}试证对于任意整数\end{CJK} $a$, \begin{CJK}{UTF8}{mj}多项式\end{CJK}

\end{enumerate}
$$
a x^{n+1} f(x) \text { 及 } x f(x)+a
$$
\begin{CJK}{UTF8}{mj}都是本原多项式\end{CJK}.

\begin{enumerate}
  \setcounter{enumi}{3}
  \item ( 15 \begin{CJK}{UTF8}{mj}分\end{CJK}) \begin{CJK}{UTF8}{mj}设\end{CJK} $A$ \begin{CJK}{UTF8}{mj}为实对称矩阵\end{CJK}, \begin{CJK}{UTF8}{mj}证明\end{CJK}: \begin{CJK}{UTF8}{mj}对任意正奇数\end{CJK} $m$, \begin{CJK}{UTF8}{mj}必有实对称矩阵\end{CJK} $B$, \begin{CJK}{UTF8}{mj}使\end{CJK}
\end{enumerate}
$$
B^{m}=A
$$
\begin{CJK}{UTF8}{mj}成立\end{CJK}.

\begin{enumerate}
  \setcounter{enumi}{4}
  \item ( 15 \begin{CJK}{UTF8}{mj}分\end{CJK}) \begin{CJK}{UTF8}{mj}令方程组\end{CJK}
\end{enumerate}
$$
\left\{\begin{array}{l}
a_{11} x_{1}+a_{12} x_{2}+\cdots+a_{1 n} x_{n}=b_{1} \\
a_{21} x_{1}+a_{22} x_{2}+\cdots+a_{2 n} x_{n}=b_{2} \\
\vdots \\
a_{n 1} x_{1}+a_{n 2} x_{2}+\cdots+a_{n n} x_{n}=b_{n}
\end{array}\right.
$$
\begin{CJK}{UTF8}{mj}与\end{CJK}
$$
\left\{\begin{array}{l}
A_{11} x_{1}+A_{12} x_{2}+\cdots+A_{1 n} x_{n}=C_{1} \\
A_{21} x_{1}+A_{22} x_{2}+\cdots+A_{2 n} x_{n}=C_{2} \\
\vdots \\
A_{n 1} x_{1}+A_{n 2} x_{2}+\cdots+A_{n n} x_{n}=C_{n}
\end{array}\right.
$$
\begin{CJK}{UTF8}{mj}其中\end{CJK} $A_{i j}$ \begin{CJK}{UTF8}{mj}为\end{CJK} $a_{i j}$ \begin{CJK}{UTF8}{mj}在系数行列式\end{CJK} $D=\left|a_{i j}\right|$ \begin{CJK}{UTF8}{mj}中的代数余子式\end{CJK}, \begin{CJK}{UTF8}{mj}证明方程组\end{CJK} (1) \begin{CJK}{UTF8}{mj}有唯一解的充分必要条件是\end{CJK} (2) \begin{CJK}{UTF8}{mj}有\end{CJK} \begin{CJK}{UTF8}{mj}唯一解\end{CJK},

\begin{enumerate}
  \setcounter{enumi}{5}
  \item ( 15 \begin{CJK}{UTF8}{mj}分\end{CJK}) \begin{CJK}{UTF8}{mj}设\end{CJK} $V$ \begin{CJK}{UTF8}{mj}是复数域上的\end{CJK} $n$ \begin{CJK}{UTF8}{mj}维线性空间\end{CJK}, \begin{CJK}{UTF8}{mj}而线性变换\end{CJK} $\mathscr{A}$ \begin{CJK}{UTF8}{mj}在基\end{CJK} $\varepsilon_{1}, \varepsilon_{2}, \cdots, \varepsilon_{n}$ \begin{CJK}{UTF8}{mj}下的矩阵是一若当块矩阵\end{CJK}, \begin{CJK}{UTF8}{mj}证\end{CJK} \begin{CJK}{UTF8}{mj}明\end{CJK} $V$ \begin{CJK}{UTF8}{mj}中任意非零\end{CJK} $\mathscr{A}$ \begin{CJK}{UTF8}{mj}一子空间都包含\end{CJK} $\varepsilon_{n}$.

  \item ( 15 \begin{CJK}{UTF8}{mj}分\end{CJK}) \begin{CJK}{UTF8}{mj}计算\end{CJK} $n$ \begin{CJK}{UTF8}{mj}阶行列式\end{CJK}

\end{enumerate}
$$
\left|\begin{array}{cccc}
x+a_{1} & a_{2} & \cdots & a_{n} \\
a_{1} & x+a_{2} & \cdots & a_{n} \\
\vdots & \vdots & & \vdots \\
a_{1} & a_{2} & \cdots & x+a_{n}
\end{array}\right|,(x \neq 0) .
$$

\begin{enumerate}
  \setcounter{enumi}{7}
  \item (15 \begin{CJK}{UTF8}{mj}分\end{CJK}) \begin{CJK}{UTF8}{mj}设\end{CJK} $\alpha_{1}, \alpha_{2}, \cdots, \alpha_{n}$ \begin{CJK}{UTF8}{mj}是\end{CJK} $n$ \begin{CJK}{UTF8}{mj}维线性空间\end{CJK} $V$ \begin{CJK}{UTF8}{mj}的一组基\end{CJK}, $A$ \begin{CJK}{UTF8}{mj}是一\end{CJK} $n \times s$ \begin{CJK}{UTF8}{mj}矩阵\end{CJK},
\end{enumerate}
$$
\left(\beta_{1}, \beta_{2}, \cdots, \beta_{s}\right)=\left(\alpha_{1}, \alpha_{2}, \cdots, \alpha_{n}\right) A .
$$
\begin{CJK}{UTF8}{mj}证明\end{CJK}: $L\left(\beta_{1}, \beta_{2}, \cdots, \beta_{s}\right)$ \begin{CJK}{UTF8}{mj}的维数等于\end{CJK} $A$ \begin{CJK}{UTF8}{mj}的秩\end{CJK}.

\begin{enumerate}
  \setcounter{enumi}{8}
  \item ( 20 \begin{CJK}{UTF8}{mj}分\end{CJK}) \begin{CJK}{UTF8}{mj}已知二次型\end{CJK}
\end{enumerate}
$$
f\left(x_{1}, x_{2}, x_{3}\right)=2 x_{1}^{2}+3 x_{2}^{2}+3 x_{3}^{2}+2 a x_{2} x_{3}(a>0)
$$
\begin{CJK}{UTF8}{mj}通过正交变换可化为标准形\end{CJK} $y_{1}^{2}+2 y_{2}^{2}+5 y_{3}^{2}$. \begin{CJK}{UTF8}{mj}求参数\end{CJK} $a$ \begin{CJK}{UTF8}{mj}及所用的正交变换矩阵\end{CJK}. 9. (20 \begin{CJK}{UTF8}{mj}分\end{CJK}) \begin{CJK}{UTF8}{mj}给定\end{CJK} $P^{3}$ \begin{CJK}{UTF8}{mj}的两组基\end{CJK}
$$
\begin{gathered}
\varepsilon_{1}=(1,0,1), \varepsilon_{2}=(2,1,0), \varepsilon_{3}=(1,1,1) \\
\eta_{1}=(1,2,-1), \eta_{2}=(2,2,-1), \eta_{3}=(2,-1,-1)
\end{gathered}
$$
\begin{CJK}{UTF8}{mj}设\end{CJK} $\mathscr{A}$ \begin{CJK}{UTF8}{mj}是\end{CJK} $P^{3}$ \begin{CJK}{UTF8}{mj}的线性变换\end{CJK}, \begin{CJK}{UTF8}{mj}且\end{CJK} $\mathscr{A} \varepsilon_{i}=\eta_{i}(i=1,2,3)$.\\
(1) \begin{CJK}{UTF8}{mj}求\end{CJK} $\mathscr{A}$ \begin{CJK}{UTF8}{mj}在基\end{CJK} $\varepsilon_{1}, \varepsilon_{2}, \varepsilon_{3}$ \begin{CJK}{UTF8}{mj}下的矩阵\end{CJK};\\
(2) \begin{CJK}{UTF8}{mj}求\end{CJK} $\mathscr{A}$ \begin{CJK}{UTF8}{mj}在基\end{CJK} $\eta_{1}, \eta_{2}, \eta_{3}$ \begin{CJK}{UTF8}{mj}下的矩阵\end{CJK}.

\begin{enumerate}
  \setcounter{enumi}{10}
  \item ( 20 \begin{CJK}{UTF8}{mj}分\end{CJK}) \begin{CJK}{UTF8}{mj}设矩阵\end{CJK} $A$ \begin{CJK}{UTF8}{mj}的伴随矩阵\end{CJK}
\end{enumerate}
$$
A^{*}=\left[\begin{array}{cccc}
1 & 0 & 0 & 0 \\
0 & 1 & 0 & 0 \\
1 & 0 & 1 & 0 \\
0 & -3 & 0 & 8
\end{array}\right]
$$
\begin{CJK}{UTF8}{mj}且\end{CJK} $A B A^{-1}=B A^{-1}+3 E$, \begin{CJK}{UTF8}{mj}求矩阵\end{CJK} $B$.

\section{1. 西北大学 2010 年研究生入学考试试题高等代数 
 李扬 
 微信公众号: sxkyliyang}
\begin{enumerate}
  \item ( 15 \begin{CJK}{UTF8}{mj}分\end{CJK}) \begin{CJK}{UTF8}{mj}设\end{CJK}
\end{enumerate}
$$
x^{n}-1 \mid(x-1)\left[f_{1}\left(x^{n}\right)+x f_{2}\left(x^{n}\right)+x^{2} f_{3}\left(x^{n}\right)+\cdots+x^{n-2} f_{n-1}\left(x^{n}\right)\right],
$$
\begin{CJK}{UTF8}{mj}求证\end{CJK}: $x-1 \mid f_{i}(x)(i=1,2, \cdots, n-1)$.

\begin{enumerate}
  \setcounter{enumi}{2}
  \item (15 \begin{CJK}{UTF8}{mj}分\end{CJK}) \begin{CJK}{UTF8}{mj}设\end{CJK} $n \geq 2, f_{1}(x), f_{2}(x), \cdots, f_{n}(x)$ \begin{CJK}{UTF8}{mj}是关于\end{CJK} $x$ \begin{CJK}{UTF8}{mj}的次数小于或等于\end{CJK} $n-2$ \begin{CJK}{UTF8}{mj}的多项式\end{CJK}, $a_{1}, a_{2}, \cdots, a_{n}$ \begin{CJK}{UTF8}{mj}为任\end{CJK} \begin{CJK}{UTF8}{mj}意数\end{CJK}, \begin{CJK}{UTF8}{mj}证明\end{CJK}: \begin{CJK}{UTF8}{mj}行列式\end{CJK}
\end{enumerate}
$$
\left|\begin{array}{cccc}
f_{1}\left(a_{1}\right) & f_{2}\left(a_{1}\right) & \cdots & f_{n}\left(a_{1}\right) \\
f_{1}\left(a_{2}\right) & f_{2}\left(a_{2}\right) & \cdots & f_{n}\left(a_{2}\right) \\
\vdots & \vdots & & \vdots \\
f_{1}\left(a_{n}\right) & f_{2}\left(a_{n}\right) & \cdots & f_{n}\left(a_{n}\right)
\end{array}\right|=0 .
$$
\begin{CJK}{UTF8}{mj}并举例说明条件\end{CJK}“\begin{CJK}{UTF8}{mj}次数\end{CJK} $\leq n-2 "$ \begin{CJK}{UTF8}{mj}是不可缺少的\end{CJK}.

\begin{enumerate}
  \setcounter{enumi}{3}
  \item ( 15 \begin{CJK}{UTF8}{mj}分\end{CJK}) \begin{CJK}{UTF8}{mj}设线性方程组\end{CJK}
\end{enumerate}
$$
\left\{\begin{array}{l}
2 x_{1}+x_{2}-x_{3}=1 \\
x_{1}-x_{2}+x_{3}=2 \\
4 x_{1}+5 x_{2}-5 x_{3}=-1
\end{array}\right.
$$
\begin{CJK}{UTF8}{mj}与\end{CJK}
$$
\left\{\begin{array}{l}
a x_{1}+b x_{2}-x_{3}=0 \\
2 x_{1}-x_{2}+a x_{3}=3
\end{array}\right.
$$
\begin{CJK}{UTF8}{mj}同解\end{CJK}, \begin{CJK}{UTF8}{mj}求通解及\end{CJK} $a, b$.

\begin{enumerate}
  \setcounter{enumi}{4}
  \item ( 15 \begin{CJK}{UTF8}{mj}分\end{CJK}) \begin{CJK}{UTF8}{mj}设\end{CJK} $A$ \begin{CJK}{UTF8}{mj}为\end{CJK} $n$ \begin{CJK}{UTF8}{mj}阶反对称实矩阵\end{CJK}, \begin{CJK}{UTF8}{mj}证明\end{CJK}:
\end{enumerate}
(1) $E+A$ \begin{CJK}{UTF8}{mj}可逆\end{CJK};

(2) $Q=(E-A)(E+A)^{-1}$ \begin{CJK}{UTF8}{mj}是一个正交矩阵\end{CJK}.

\begin{enumerate}
  \setcounter{enumi}{5}
  \item ( 20 \begin{CJK}{UTF8}{mj}分\end{CJK}) \begin{CJK}{UTF8}{mj}试用正交线性替换将二次型\end{CJK}
\end{enumerate}
$$
f\left(x_{1}, x_{2}, x_{3}\right)=x_{1}^{2}-2 x_{2}^{2}-2 x_{3}^{2}-4 x_{1} x_{2}+4 x_{1} x_{3}+8 x_{2} x_{3}
$$
\begin{CJK}{UTF8}{mj}化成标准形\end{CJK}, \begin{CJK}{UTF8}{mj}并写出所用的正交线性替换和计算过程\end{CJK}.

\begin{enumerate}
  \setcounter{enumi}{6}
  \item (15 \begin{CJK}{UTF8}{mj}分\end{CJK}) \begin{CJK}{UTF8}{mj}设\end{CJK} $A$ \begin{CJK}{UTF8}{mj}为\end{CJK} $n \times n$ \begin{CJK}{UTF8}{mj}正定实对称矩阵\end{CJK}, $S$ \begin{CJK}{UTF8}{mj}为\end{CJK} $n \times n$ \begin{CJK}{UTF8}{mj}实反对称矩阵\end{CJK}, \begin{CJK}{UTF8}{mj}试证\end{CJK}: \begin{CJK}{UTF8}{mj}行列式\end{CJK}
\end{enumerate}
$$
|A+S|>0
$$

\begin{enumerate}
  \setcounter{enumi}{7}
  \item ( 20 \begin{CJK}{UTF8}{mj}分\end{CJK}) \begin{CJK}{UTF8}{mj}设\end{CJK} $V_{1}, V_{2}$ \begin{CJK}{UTF8}{mj}分别是齐次线性方程组\end{CJK}
\end{enumerate}
$$
x_{1}+x_{2}+\cdots+x_{n}=0
$$
\begin{CJK}{UTF8}{mj}与\end{CJK}
$$
x_{1}=x_{2}=\cdots=x_{n}
$$
\begin{CJK}{UTF8}{mj}的解空间\end{CJK}. \begin{CJK}{UTF8}{mj}证明\end{CJK} $n$ \begin{CJK}{UTF8}{mj}维实向量空间\end{CJK} $\mathbb{R}^{n}$ \begin{CJK}{UTF8}{mj}是\end{CJK} $V_{1}$ \begin{CJK}{UTF8}{mj}与\end{CJK} $V_{2}$ \begin{CJK}{UTF8}{mj}的直和\end{CJK}. 8. ( 20 \begin{CJK}{UTF8}{mj}分\end{CJK}) \begin{CJK}{UTF8}{mj}以\end{CJK} $\mathbb{R}_{n}[x]$ \begin{CJK}{UTF8}{mj}表示次数不超过\end{CJK} $n$ \begin{CJK}{UTF8}{mj}的实系数多项式构成的实向量空间\end{CJK}, \begin{CJK}{UTF8}{mj}其加法是多项式加法\end{CJK}, \begin{CJK}{UTF8}{mj}数乘运算是实\end{CJK} \begin{CJK}{UTF8}{mj}数乘多项式\end{CJK}, \begin{CJK}{UTF8}{mj}以\end{CJK} $D=\frac{\mathrm{d}}{\mathrm{d} x}$ \begin{CJK}{UTF8}{mj}表示求多项式导数的求导算子\end{CJK}, \begin{CJK}{UTF8}{mj}则\end{CJK} $D$ \begin{CJK}{UTF8}{mj}为\end{CJK} $\mathbb{R}_{n}[x]$ \begin{CJK}{UTF8}{mj}上的线性变换\end{CJK}.

(1) \begin{CJK}{UTF8}{mj}试证\end{CJK}: $1, x, \cdots, x^{n}$ \begin{CJK}{UTF8}{mj}是\end{CJK} $\mathbb{R}_{n}[x]$ \begin{CJK}{UTF8}{mj}上的一组基\end{CJK};

(2) \begin{CJK}{UTF8}{mj}求\end{CJK} $D$ \begin{CJK}{UTF8}{mj}在上述基下的矩阵\end{CJK};

(3) \begin{CJK}{UTF8}{mj}试证\end{CJK}: \begin{CJK}{UTF8}{mj}当\end{CJK} $n \geq 1$ \begin{CJK}{UTF8}{mj}时\end{CJK}, $D$ \begin{CJK}{UTF8}{mj}不能对角化\end{CJK}.

\begin{enumerate}
  \setcounter{enumi}{9}
  \item ( 20 \begin{CJK}{UTF8}{mj}分\end{CJK}) \begin{CJK}{UTF8}{mj}设\end{CJK} $V$ \begin{CJK}{UTF8}{mj}是\end{CJK} $n$ \begin{CJK}{UTF8}{mj}维欧式空间\end{CJK}, $\mathscr{A}$ \begin{CJK}{UTF8}{mj}是\end{CJK} $V$ \begin{CJK}{UTF8}{mj}上的线性变换\end{CJK}, \begin{CJK}{UTF8}{mj}并且满足条件\end{CJK}: \begin{CJK}{UTF8}{mj}对任意\end{CJK} $\alpha, \beta \in V$, \begin{CJK}{UTF8}{mj}有\end{CJK}
\end{enumerate}
$$
(\mathscr{A}(\alpha), \beta)=(\alpha, \mathscr{A}(\beta))
$$
\begin{CJK}{UTF8}{mj}其中\end{CJK} $(\xi, \eta)$ \begin{CJK}{UTF8}{mj}表示向量\end{CJK} $\xi, \eta$ \begin{CJK}{UTF8}{mj}的内积\end{CJK}.

(1) \begin{CJK}{UTF8}{mj}证明\end{CJK}: $\mathscr{A}$ \begin{CJK}{UTF8}{mj}的属于不同特征值的特征向量是相互正交的\end{CJK};

(2) \begin{CJK}{UTF8}{mj}证明\end{CJK}: \begin{CJK}{UTF8}{mj}若\end{CJK} $\mathscr{A}^{2}=\mathscr{A}$, \begin{CJK}{UTF8}{mj}则\end{CJK} $V_{0}=V_{1}^{\perp}$, \begin{CJK}{UTF8}{mj}其中\end{CJK} $V_{0}, V_{1}$ \begin{CJK}{UTF8}{mj}分别表示\end{CJK} $\mathscr{A}$ \begin{CJK}{UTF8}{mj}的关于特征值\end{CJK} 0 \begin{CJK}{UTF8}{mj}和\end{CJK} 1 \begin{CJK}{UTF8}{mj}的特征子空间\end{CJK}, $V_{1}^{\perp}$ \begin{CJK}{UTF8}{mj}表示\end{CJK} $V_{1}$ \begin{CJK}{UTF8}{mj}的正交补空间\end{CJK}.

\section{2. 西北大学 2011 年研究生入学考试试题高等代数 
 李扬 
 微信公众号: sxkyliyang}
\begin{enumerate}
  \item (15 \begin{CJK}{UTF8}{mj}分\end{CJK}) \begin{CJK}{UTF8}{mj}计算行列式\end{CJK}
\end{enumerate}
$$
D_{5}=\left|\begin{array}{ccccc}
1-a & a & 0 & 0 & 0 \\
-1 & 1-a & a & 0 & 0 \\
0 & -1 & 1-a & a & 0 \\
0 & 0 & -1 & 1-a & a \\
0 & 0 & 0 & -1 & 1-a
\end{array}\right|
$$

\begin{enumerate}
  \setcounter{enumi}{2}
  \item (15 \begin{CJK}{UTF8}{mj}分\end{CJK}) \begin{CJK}{UTF8}{mj}试问\end{CJK} $t$ \begin{CJK}{UTF8}{mj}取何值时\end{CJK}, \begin{CJK}{UTF8}{mj}实二次型\end{CJK}
\end{enumerate}
$$
f\left(x_{1}, x_{2}, x_{3}, x_{4}\right)=t\left(x_{1}^{2}+x_{2}^{2}+x_{3}^{2}\right)+2 x_{1} x_{2}+2 x_{1} x_{3}-2 x_{2} x_{3}+x_{4}^{2}
$$
\begin{CJK}{UTF8}{mj}为正定二次型\end{CJK}?

\begin{enumerate}
  \setcounter{enumi}{3}
  \item ( 15 \begin{CJK}{UTF8}{mj}分\end{CJK}) \begin{CJK}{UTF8}{mj}讨论当\end{CJK} $a, b$ \begin{CJK}{UTF8}{mj}为何值时\end{CJK}, \begin{CJK}{UTF8}{mj}方程组\end{CJK}
\end{enumerate}
$$
\left\{\begin{array}{l}
x_{1}+x_{2}+x_{3}+x_{4}=0 \\
x_{2}+2 x_{3}+2 x_{4}=1 \\
-x_{2}+(a-3) x_{3}-2 x_{4}=b \\
3 x_{1}+2 x_{2}+x_{3}+a x_{4}=-1
\end{array}\right.
$$
\begin{CJK}{UTF8}{mj}无解\end{CJK}; \begin{CJK}{UTF8}{mj}有唯一解\end{CJK}; \begin{CJK}{UTF8}{mj}有无穷多解\end{CJK}. \begin{CJK}{UTF8}{mj}当有无穷多解时\end{CJK}, \begin{CJK}{UTF8}{mj}求出一般解\end{CJK}.

\begin{enumerate}
  \setcounter{enumi}{4}
  \item ( 15 \begin{CJK}{UTF8}{mj}分\end{CJK}) \begin{CJK}{UTF8}{mj}设\end{CJK} $\varepsilon_{1}, \varepsilon_{2}, \varepsilon_{3}, \varepsilon_{4}$ \begin{CJK}{UTF8}{mj}是\end{CJK} 4 \begin{CJK}{UTF8}{mj}为线性空间\end{CJK} $V$ \begin{CJK}{UTF8}{mj}的一组基\end{CJK}, \begin{CJK}{UTF8}{mj}线性变换\end{CJK} $\mathscr{A}$ \begin{CJK}{UTF8}{mj}在这组基下的矩阵为\end{CJK}
\end{enumerate}
$$
\left(\begin{array}{cccc}
1 & 0 & 2 & 1 \\
-1 & 2 & 1 & 3 \\
1 & 2 & 5 & 5 \\
2 & -2 & 1 & -2
\end{array}\right)
$$
\begin{CJK}{UTF8}{mj}求\end{CJK} $\mathscr{A}$ \begin{CJK}{UTF8}{mj}的值域和核\end{CJK}.

\begin{enumerate}
  \setcounter{enumi}{5}
  \item ( 15 \begin{CJK}{UTF8}{mj}分\end{CJK}) \begin{CJK}{UTF8}{mj}设\end{CJK} $\xi_{1}, \cdots, \xi_{s}$ \begin{CJK}{UTF8}{mj}是某个齐次线性方程组的一个基础解系\end{CJK}, $\eta_{1}, \cdots, \eta_{k}$ \begin{CJK}{UTF8}{mj}是该齐次线性方程组的\end{CJK} $k$ \begin{CJK}{UTF8}{mj}个线性\end{CJK} \begin{CJK}{UTF8}{mj}无关的解\end{CJK}. \begin{CJK}{UTF8}{mj}证明\end{CJK}: \begin{CJK}{UTF8}{mj}若\end{CJK} $k<s$, \begin{CJK}{UTF8}{mj}则在\end{CJK} $\xi_{1}, \cdots, \xi_{s}$ \begin{CJK}{UTF8}{mj}中必可取出\end{CJK} $s-k$ \begin{CJK}{UTF8}{mj}个向量使得与\end{CJK} $\eta_{1}, \cdots, \eta_{k}$ \begin{CJK}{UTF8}{mj}共同构成该齐次线性\end{CJK} \begin{CJK}{UTF8}{mj}方程组的一个基础解系\end{CJK}.

  \item (15 \begin{CJK}{UTF8}{mj}分\end{CJK}) \begin{CJK}{UTF8}{mj}设方阵\end{CJK} $A$ \begin{CJK}{UTF8}{mj}满足\end{CJK}

\end{enumerate}
$$
A^{2}+2 A-3 E=0
$$
$E$ \begin{CJK}{UTF8}{mj}为单位矩阵\end{CJK}.

(1) \begin{CJK}{UTF8}{mj}证明\end{CJK}: $A+4 E$ \begin{CJK}{UTF8}{mj}可逆\end{CJK}, \begin{CJK}{UTF8}{mj}并求其逆矩阵\end{CJK};

(2) \begin{CJK}{UTF8}{mj}设\end{CJK} $n$ \begin{CJK}{UTF8}{mj}为自然数\end{CJK}, \begin{CJK}{UTF8}{mj}讨论\end{CJK} $A+n E$ \begin{CJK}{UTF8}{mj}的可逆性\end{CJK}.

\begin{enumerate}
  \setcounter{enumi}{7}
  \item (20 \begin{CJK}{UTF8}{mj}分\end{CJK}) \begin{CJK}{UTF8}{mj}设实二次型\end{CJK}
\end{enumerate}
$$
f\left(x_{1}, x_{2}, \cdots, x_{n}\right)=\sum_{i=1}^{s}\left(a_{i 1} x_{1}+a_{i 2} x_{2}+\cdots+a_{i n} x_{n}\right)^{2},
$$
\begin{CJK}{UTF8}{mj}证明\end{CJK}: $f\left(x_{1}, x_{2}, \cdots, x_{n}\right)$ \begin{CJK}{UTF8}{mj}的秩等于矩阵\end{CJK} $A=\left(a_{i j}\right)_{s \times n}$ \begin{CJK}{UTF8}{mj}的秩\end{CJK}. 8. ( 20 \begin{CJK}{UTF8}{mj}分\end{CJK}) \begin{CJK}{UTF8}{mj}求\end{CJK} 7 \begin{CJK}{UTF8}{mj}次多项式\end{CJK} $f(x)$ \begin{CJK}{UTF8}{mj}使得\end{CJK} $f(x)+1$ \begin{CJK}{UTF8}{mj}能被\end{CJK} $(x-1)^{4}$ \begin{CJK}{UTF8}{mj}整除\end{CJK}, $f(x)-1$ \begin{CJK}{UTF8}{mj}能被\end{CJK} $(x+1)^{4}$ \begin{CJK}{UTF8}{mj}整除\end{CJK}.

\begin{enumerate}
  \setcounter{enumi}{9}
  \item (20 \begin{CJK}{UTF8}{mj}分\end{CJK}) (1) \begin{CJK}{UTF8}{mj}求矩阵\end{CJK}
\end{enumerate}
$$
A=\left(\begin{array}{llll}
0 & 1 & 1 & 1 \\
0 & 0 & 1 & 1 \\
0 & 0 & 0 & 1 \\
0 & 0 & 0 & 0
\end{array}\right)
$$
\begin{CJK}{UTF8}{mj}的若尔当\end{CJK} (Jordan) \begin{CJK}{UTF8}{mj}的标准形\end{CJK}, \begin{CJK}{UTF8}{mj}并计算\end{CJK} $\mathrm{e}^{A}$.

(\begin{CJK}{UTF8}{mj}注\end{CJK}: \begin{CJK}{UTF8}{mj}通常定义\end{CJK} $\mathrm{e}^{A}=E+A+\frac{A^{2}}{2 !}+\frac{A^{3}}{3 !}+\cdots$ )

(2) \begin{CJK}{UTF8}{mj}设矩阵\end{CJK}
$$
B=\left(\begin{array}{ccc}
4 & 4.5 & -1 \\
-3 & -3.5 & 1 \\
-2 & -3 & 1.5
\end{array}\right)
$$
\begin{CJK}{UTF8}{mj}求\end{CJK} $B^{2011}$. (\begin{CJK}{UTF8}{mj}精确到小数点后\end{CJK} 4 \begin{CJK}{UTF8}{mj}位\end{CJK})

\section{3. 西北大学 2012 年研究生入学考试试题高等代数}
\begin{CJK}{UTF8}{mj}李扬\end{CJK}

\begin{CJK}{UTF8}{mj}微信公众号\end{CJK}: sxkyliyang

\begin{CJK}{UTF8}{mj}一\end{CJK}、\begin{CJK}{UTF8}{mj}填空题\end{CJK} (\begin{CJK}{UTF8}{mj}每小题\end{CJK} 5 \begin{CJK}{UTF8}{mj}分\end{CJK}, \begin{CJK}{UTF8}{mj}本题共\end{CJK} 30 \begin{CJK}{UTF8}{mj}分\end{CJK})

\begin{enumerate}
  \item \begin{CJK}{UTF8}{mj}多项式\end{CJK} $f(x)$ \begin{CJK}{UTF8}{mj}被\end{CJK} $a x-b(a \neq 0)$ \begin{CJK}{UTF8}{mj}除所得的余式为\end{CJK}

  \item \begin{CJK}{UTF8}{mj}设行列式\end{CJK} $D=\left|\begin{array}{cccc}3 & 0 & 4 & 0 \\ 2 & 2 & 2 & 2 \\ 0 & -7 & 0 & 0 \\ 5 & 3 & -2 & 2\end{array}\right|$. \begin{CJK}{UTF8}{mj}则第四行各元素的余子式之和的值为\end{CJK}

  \item \begin{CJK}{UTF8}{mj}设矩阵\end{CJK} $A=\left(\begin{array}{cccc}k & 1 & 1 & 1 \\ 1 & k & 1 & 1 \\ 1 & 1 & k & 1 \\ 1 & 1 & 1 & k\end{array}\right)$. \begin{CJK}{UTF8}{mj}且\end{CJK} $\mathrm{r}(A)=3$. \begin{CJK}{UTF8}{mj}则\end{CJK} $k=$

  \item \begin{CJK}{UTF8}{mj}设\end{CJK} $A$ \begin{CJK}{UTF8}{mj}为三阶矩阵\end{CJK}, $\mathrm{r}(A)=2$, \begin{CJK}{UTF8}{mj}而矩阵\end{CJK} $B=\left(\begin{array}{ccc}1 & 0 & 2 \\ 0 & 2 & 0 \\ -1 & 0 & 3\end{array}\right)$. \begin{CJK}{UTF8}{mj}则\end{CJK} $\mathrm{r}(A B)=$

  \item \begin{CJK}{UTF8}{mj}设\end{CJK} $n$ \begin{CJK}{UTF8}{mj}元实二次型\end{CJK} $f\left(x_{1}, x_{2}, \cdots, x_{n}\right)=\left(x_{1}+a_{1} x_{2}\right)^{2}+\left(x_{2}+a_{2} x_{3}\right)^{2}+\cdots+\left(x_{n-1}+a_{n-1} x_{n}\right)^{2}+\left(x_{n}+a_{n} x_{1}\right)^{2}$. \begin{CJK}{UTF8}{mj}当\end{CJK} $a_{1}, a_{2}, \cdots, a_{n}$ \begin{CJK}{UTF8}{mj}满足条件\end{CJK} \begin{CJK}{UTF8}{mj}时\end{CJK}, $f\left(x_{1}, x_{2}, \cdots, x_{n}\right)$ \begin{CJK}{UTF8}{mj}为正定二次型\end{CJK}.

  \item \begin{CJK}{UTF8}{mj}设\end{CJK} $A$ \begin{CJK}{UTF8}{mj}为\end{CJK} $n$ \begin{CJK}{UTF8}{mj}阶矩阵\end{CJK}, $|A| \neq 0, A^{*}$ \begin{CJK}{UTF8}{mj}为\end{CJK} $A$ \begin{CJK}{UTF8}{mj}的伴随矩阵\end{CJK}, $E$ \begin{CJK}{UTF8}{mj}为\end{CJK} $n$ \begin{CJK}{UTF8}{mj}阶单位矩阵\end{CJK}, \begin{CJK}{UTF8}{mj}若\end{CJK} $A$ \begin{CJK}{UTF8}{mj}有特征值\end{CJK} $\lambda$, \begin{CJK}{UTF8}{mj}则\end{CJK} $\left(A^{*}\right)^{2}+E$ \begin{CJK}{UTF8}{mj}必有特\end{CJK} \begin{CJK}{UTF8}{mj}征值\end{CJK}

\end{enumerate}
\begin{CJK}{UTF8}{mj}二\end{CJK}、\begin{CJK}{UTF8}{mj}解答题\end{CJK}.

\begin{enumerate}
  \item (10 \begin{CJK}{UTF8}{mj}分\end{CJK}) \begin{CJK}{UTF8}{mj}设有线性方程组\end{CJK}
\end{enumerate}
$$
\left\{\begin{array}{l}
x_{1}+\lambda x_{2}+\mu x_{3}+x_{4}=0 \\
2 x_{1}+x_{2}+x_{3}+5 x_{4}=0 \\
3 x_{1}+(2+\lambda) x_{2}+(4+\mu) x_{3}+4 x_{4}=1
\end{array}\right.
$$
\begin{CJK}{UTF8}{mj}已知\end{CJK} $(1,-1,1,-1)$ \begin{CJK}{UTF8}{mj}是该方程组的一个解\end{CJK}.

(1) \begin{CJK}{UTF8}{mj}求出方程组的全部解\end{CJK}, \begin{CJK}{UTF8}{mj}并用对应的齐次线性方程组的基础解系表出全部解\end{CJK};

(2) \begin{CJK}{UTF8}{mj}求出该方程组满足\end{CJK} $x_{2}=x_{3}$ \begin{CJK}{UTF8}{mj}的全部解\end{CJK}.

\begin{enumerate}
  \setcounter{enumi}{2}
  \item (10 \begin{CJK}{UTF8}{mj}分\end{CJK}) \begin{CJK}{UTF8}{mj}求一正交的正交替换\end{CJK}, \begin{CJK}{UTF8}{mj}化二次型\end{CJK}
\end{enumerate}
$$
f\left(x_{1}, x_{2}, x_{3}\right)=x_{1}^{2}+4 x_{2}^{2}+4 x_{3}^{2}-4 x_{1} x_{2}+4 x_{1} x_{3}-8 x_{2} x_{3}
$$
\begin{CJK}{UTF8}{mj}为标准形\end{CJK}.

\begin{enumerate}
  \setcounter{enumi}{3}
  \item ( 10 \begin{CJK}{UTF8}{mj}分\end{CJK}) \begin{CJK}{UTF8}{mj}已知\end{CJK} $P^{2 \times 2}$ \begin{CJK}{UTF8}{mj}的线性变换\end{CJK}
\end{enumerate}
$$
\mathscr{A}(X)=M X-X M
$$
\begin{CJK}{UTF8}{mj}其中\end{CJK} $M=\left(\begin{array}{ll}1 & 2 \\ 0 & 3\end{array}\right), X \in P^{2 \times 2}$. \begin{CJK}{UTF8}{mj}求线性变换\end{CJK} $\mathscr{A}$ \begin{CJK}{UTF8}{mj}的核与值域\end{CJK}.

\begin{enumerate}
  \setcounter{enumi}{4}
  \item (10 \begin{CJK}{UTF8}{mj}分\end{CJK}) \begin{CJK}{UTF8}{mj}设\end{CJK} $n \times n$ \begin{CJK}{UTF8}{mj}阶矩阵\end{CJK}
\end{enumerate}
$$
A=\left[\begin{array}{ccccc}
0 & 0 & \cdots & 0 & 1 \\
1 & 0 & \cdots & 0 & 0 \\
0 & 1 & \cdots & 0 & 0 \\
\vdots & \vdots & & \vdots & \vdots \\
0 & 0 & \cdots & 1 & 0
\end{array}\right]
$$
\begin{CJK}{UTF8}{mj}求\end{CJK} $A$ \begin{CJK}{UTF8}{mj}的若尔当\end{CJK} (Jordan) \begin{CJK}{UTF8}{mj}标准形\end{CJK}.

\begin{enumerate}
  \setcounter{enumi}{5}
  \item ( 20 \begin{CJK}{UTF8}{mj}分\end{CJK}) \begin{CJK}{UTF8}{mj}设\end{CJK} $V$ \begin{CJK}{UTF8}{mj}是数域\end{CJK} $P$ \begin{CJK}{UTF8}{mj}上的一个\end{CJK} $n$ \begin{CJK}{UTF8}{mj}维线性空间\end{CJK}, $\alpha_{1}, \alpha_{2}, \cdots, \alpha_{n}$ \begin{CJK}{UTF8}{mj}是\end{CJK} $V$ \begin{CJK}{UTF8}{mj}的一组基\end{CJK}, \begin{CJK}{UTF8}{mj}用\end{CJK} $W_{1}$ \begin{CJK}{UTF8}{mj}表示由\end{CJK} $\alpha_{1}+\alpha_{2}+\cdots+\alpha_{n}$ \begin{CJK}{UTF8}{mj}生成的子空间\end{CJK}, \begin{CJK}{UTF8}{mj}并令\end{CJK}
\end{enumerate}
$$
W_{2}=\left\{\sum_{i=1}^{n} k_{i} \alpha_{i} \mid \sum_{i=1}^{n} k_{i}=0, k_{i} \in P\right\}
$$
(1) \begin{CJK}{UTF8}{mj}证明\end{CJK}: $W_{2}$ \begin{CJK}{UTF8}{mj}是\end{CJK} $V$ \begin{CJK}{UTF8}{mj}的子空间\end{CJK};

(2) \begin{CJK}{UTF8}{mj}证明\end{CJK}: $V=W_{1} \oplus W_{2}$.

\begin{enumerate}
  \setcounter{enumi}{6}
  \item (20 \begin{CJK}{UTF8}{mj}分\end{CJK}) \begin{CJK}{UTF8}{mj}证明\end{CJK}: \begin{CJK}{UTF8}{mj}不存在正交矩阵\end{CJK} $A, B$, \begin{CJK}{UTF8}{mj}使得\end{CJK}
\end{enumerate}
$$
A^{2}=A B+B^{2}
$$

\begin{enumerate}
  \setcounter{enumi}{7}
  \item ( 20 \begin{CJK}{UTF8}{mj}分\end{CJK}) \begin{CJK}{UTF8}{mj}证明\end{CJK}:
\end{enumerate}
$$
\left(x^{2}+x+1\right) \mid\left(x^{2012}+x^{2011}+1\right) .
$$

\begin{enumerate}
  \setcounter{enumi}{8}
  \item ( 20 \begin{CJK}{UTF8}{mj}分\end{CJK}) \begin{CJK}{UTF8}{mj}设\end{CJK} $A$ \begin{CJK}{UTF8}{mj}是复数域上的方阵\end{CJK}, \begin{CJK}{UTF8}{mj}证明\end{CJK}:
\end{enumerate}
(1) $A$ \begin{CJK}{UTF8}{mj}的特征值均为零当且仅当存在正整数\end{CJK} $m$ \begin{CJK}{UTF8}{mj}使得\end{CJK}
$$
A^{m}=0
$$
(2) \begin{CJK}{UTF8}{mj}若存在正整数\end{CJK} $n$ \begin{CJK}{UTF8}{mj}使得\end{CJK} $A^{n}=0$, \begin{CJK}{UTF8}{mj}则必有\end{CJK}
$$
|A+E|=1 .
$$
(\begin{CJK}{UTF8}{mj}此处\end{CJK} $E$ \begin{CJK}{UTF8}{mj}表示与\end{CJK} $A$ \begin{CJK}{UTF8}{mj}同阶的单位矩阵\end{CJK}).

\section{4. 西北大学 2013 年研究生入学考试试题高等代数}
\begin{CJK}{UTF8}{mj}李扬\end{CJK}

\begin{CJK}{UTF8}{mj}微信公众号\end{CJK}: sxkyliyang

\begin{CJK}{UTF8}{mj}一\end{CJK}、\begin{CJK}{UTF8}{mj}填空题\end{CJK} (\begin{CJK}{UTF8}{mj}每小题\end{CJK} 5 \begin{CJK}{UTF8}{mj}分\end{CJK}, \begin{CJK}{UTF8}{mj}本题共\end{CJK} 30 \begin{CJK}{UTF8}{mj}分\end{CJK})

\begin{enumerate}
  \item \begin{CJK}{UTF8}{mj}设行列式\end{CJK} $D=\left|\begin{array}{cccc}2 & 3 & 4 & 5 \\ 1 & 2 & 3 & 4 \\ 1 & 4 & 9 & 16 \\ 1 & 8 & 27 & 64\end{array}\right|, A_{i j}$ \begin{CJK}{UTF8}{mj}表示第\end{CJK} $i$ \begin{CJK}{UTF8}{mj}行第\end{CJK} $j$ \begin{CJK}{UTF8}{mj}列元素的代数余子式\end{CJK}, \begin{CJK}{UTF8}{mj}则\end{CJK} $A_{11}+A_{12}+A_{13}+A_{14}=$

  \item \begin{CJK}{UTF8}{mj}设\end{CJK} $A=P^{-1} \Lambda P, D=\left(\begin{array}{cc}-1 & 3 \\ 1 & 2\end{array}\right), \Lambda=\left(\begin{array}{cc}-1 & 0 \\ 0 & 2\end{array}\right)$, \begin{CJK}{UTF8}{mj}则\end{CJK} $A^{10}=$

  \item \begin{CJK}{UTF8}{mj}设\end{CJK} $A=\left(\begin{array}{lll}1 & 0 & 0 \\ 2 & 2 & 0 \\ 3 & 3 & 3\end{array}\right)$ \begin{CJK}{UTF8}{mj}的伴随矩阵为\end{CJK} $A^{*}$, \begin{CJK}{UTF8}{mj}则\end{CJK} $\left(A^{*}\right)^{-1}=$

  \item \begin{CJK}{UTF8}{mj}设\end{CJK} $f(x)$ \begin{CJK}{UTF8}{mj}仅有实根\end{CJK}, $a$ \begin{CJK}{UTF8}{mj}是\end{CJK} $f^{\prime}(x)$ \begin{CJK}{UTF8}{mj}的重根\end{CJK}, \begin{CJK}{UTF8}{mj}则\end{CJK} $f(a)=$

  \item \begin{CJK}{UTF8}{mj}设\end{CJK} $V$ \begin{CJK}{UTF8}{mj}是\end{CJK} $n$ \begin{CJK}{UTF8}{mj}维线性空间\end{CJK}, $U$ \begin{CJK}{UTF8}{mj}与\end{CJK} $W$ \begin{CJK}{UTF8}{mj}是\end{CJK} $V$ \begin{CJK}{UTF8}{mj}的两个相异的\end{CJK} $n-1$ \begin{CJK}{UTF8}{mj}维子空间\end{CJK}, \begin{CJK}{UTF8}{mj}则\end{CJK} $\operatorname{dim}(U \cap W)=$

  \item \begin{CJK}{UTF8}{mj}设二次型\end{CJK} $f\left(x_{1}, x_{2}, x_{3}\right)=x_{1}^{2}+4 x_{2}^{2}+4 x_{3}^{2}+2 \lambda x_{1} x_{2}-2 x_{1} x_{3}+4 x_{2} x_{3}$, \begin{CJK}{UTF8}{mj}当\end{CJK} $\lambda$ \begin{CJK}{UTF8}{mj}满足条件\end{CJK} \begin{CJK}{UTF8}{mj}时\end{CJK}, $f\left(x_{1}, x_{2}, x_{3}\right)$ \begin{CJK}{UTF8}{mj}为正定二次型\end{CJK}.

\end{enumerate}
\section{一、解答题.}
\begin{enumerate}
  \item (15 \begin{CJK}{UTF8}{mj}分\end{CJK}) \begin{CJK}{UTF8}{mj}设\end{CJK} $f(x)$ \begin{CJK}{UTF8}{mj}和\end{CJK} $g(x)$ \begin{CJK}{UTF8}{mj}为数域\end{CJK} $P$ \begin{CJK}{UTF8}{mj}上的多项式\end{CJK}, $n$ \begin{CJK}{UTF8}{mj}为自然数\end{CJK}. \begin{CJK}{UTF8}{mj}证明\end{CJK}:
\end{enumerate}
$$
f(x) \text { 与 } g(x) \text { 互质 } \Leftrightarrow f\left(x^{n}\right) \text { 与 } g\left(x^{n}\right) \text { 互质. }
$$

\begin{enumerate}
  \setcounter{enumi}{2}
  \item (15 \begin{CJK}{UTF8}{mj}分\end{CJK}) \begin{CJK}{UTF8}{mj}设\end{CJK} $A$ \begin{CJK}{UTF8}{mj}为\end{CJK} $n$ \begin{CJK}{UTF8}{mj}阶方阵且\end{CJK} $A^{2}=A$, \begin{CJK}{UTF8}{mj}证明\end{CJK}:
\end{enumerate}
$$
\mathrm{r}(A)+\mathrm{r}(E-A)=n
$$

\begin{enumerate}
  \setcounter{enumi}{3}
  \item ( 10 \begin{CJK}{UTF8}{mj}分\end{CJK}) \begin{CJK}{UTF8}{mj}计算\end{CJK}
\end{enumerate}
$$
\left|\begin{array}{ccccc}
1 & 1 & 1 & 1 & 1 \\
a & a+1 & a+1 & a+1 & a+1 \\
b & a+b & 2 b & c+b & d+b \\
c & a^{2}+c & b^{2}+c & c^{2}+c & d^{2}+c \\
d & a^{4}+d & b^{4}+d & c^{4}+d & d^{4}+d
\end{array}\right|
$$

\begin{enumerate}
  \setcounter{enumi}{4}
  \item (15 \begin{CJK}{UTF8}{mj}分\end{CJK}) \begin{CJK}{UTF8}{mj}设\end{CJK} $p$ \begin{CJK}{UTF8}{mj}为素数\end{CJK}, \begin{CJK}{UTF8}{mj}将多项式\end{CJK}
\end{enumerate}
$$
x^{p+1}+x^{p}-x-1
$$
\begin{CJK}{UTF8}{mj}在有理数域上分解为不可约多项式的乘积\end{CJK}.

\begin{enumerate}
  \setcounter{enumi}{5}
  \item (15 \begin{CJK}{UTF8}{mj}分\end{CJK}) \begin{CJK}{UTF8}{mj}设\end{CJK} $\alpha_{1}=(1,0,2,3)^{\prime}, \alpha_{2}=(2,1,-1,0)^{\prime}, \alpha_{3}=(3,1,1,3)^{\prime}$, \begin{CJK}{UTF8}{mj}求一以\end{CJK} $L\left(\alpha_{1}, \alpha_{2}, \alpha_{3}\right)$ \begin{CJK}{UTF8}{mj}为解空间的齐次线性\end{CJK} \begin{CJK}{UTF8}{mj}方程组\end{CJK}. 6. ( 30 \begin{CJK}{UTF8}{mj}分\end{CJK}) $V$ \begin{CJK}{UTF8}{mj}是\end{CJK} $n$ \begin{CJK}{UTF8}{mj}维线性空间\end{CJK}, $\varepsilon_{1}, \varepsilon_{2}, \cdots, \varepsilon_{n}$ \begin{CJK}{UTF8}{mj}是\end{CJK} $V$ \begin{CJK}{UTF8}{mj}的一组基\end{CJK}, $V$ \begin{CJK}{UTF8}{mj}的线性变换\end{CJK} $\mathscr{A}$ \begin{CJK}{UTF8}{mj}定义为\end{CJK}
\end{enumerate}
$$
\left\{\begin{array}{l}
\mathscr{A} \varepsilon_{i}=\varepsilon_{i+1}, i=1,2, \cdots, n-1 \\
\mathscr{A} \varepsilon_{n}=0
\end{array}\right.
$$
(1) \begin{CJK}{UTF8}{mj}求\end{CJK} $\mathscr{A}$ \begin{CJK}{UTF8}{mj}在基\end{CJK} $\varepsilon_{1}, \varepsilon_{2}, \cdots, \varepsilon_{n}$ \begin{CJK}{UTF8}{mj}下的矩阵\end{CJK} $A$;

(2) \begin{CJK}{UTF8}{mj}求\end{CJK} $\mathscr{A}$ \begin{CJK}{UTF8}{mj}的值域和核的维数\end{CJK};

(3) \begin{CJK}{UTF8}{mj}证明\end{CJK}: $\mathscr{A}^{n}=0, \mathscr{A}^{n-1} \neq 0$.

\begin{CJK}{UTF8}{mj}若\end{CJK} $V$ \begin{CJK}{UTF8}{mj}的另一个线性变换\end{CJK} $\mathscr{B}$ \begin{CJK}{UTF8}{mj}满足\end{CJK} $\mathscr{B}^{n}=0$, \begin{CJK}{UTF8}{mj}而\end{CJK} $\mathscr{B}^{n-1} \neq 0$, \begin{CJK}{UTF8}{mj}试证存在\end{CJK} $V$ \begin{CJK}{UTF8}{mj}的一组基使\end{CJK} $\mathscr{B}$ \begin{CJK}{UTF8}{mj}在该组基下的矩阵也是\end{CJK} $A$.

\begin{enumerate}
  \setcounter{enumi}{7}
  \item ( 20 \begin{CJK}{UTF8}{mj}分\end{CJK}) \begin{CJK}{UTF8}{mj}设二次型\end{CJK}
\end{enumerate}
$$
f=k\left(x_{1}^{2}+x_{2}^{2}+x_{3}^{2}+x_{4}^{2}\right)+2 x_{1} x_{2}+2 x_{1} x_{3}-2 x_{1} x_{4}-2 x_{2} x_{3}+2 x_{2} x_{4}+2 x_{3} x_{4}
$$
(1) \begin{CJK}{UTF8}{mj}若\end{CJK} $f$ \begin{CJK}{UTF8}{mj}为正定二次型\end{CJK}, \begin{CJK}{UTF8}{mj}求\end{CJK} $k$ \begin{CJK}{UTF8}{mj}的取值范围\end{CJK};

(2) \begin{CJK}{UTF8}{mj}若\end{CJK} $f$ \begin{CJK}{UTF8}{mj}通过正交线性替换化为标准形\end{CJK} $g=3 y_{1}^{2}+3 y_{2}^{2}+3 y_{3}^{2}-y_{4}^{2}$, \begin{CJK}{UTF8}{mj}求\end{CJK} $k$ \begin{CJK}{UTF8}{mj}的值\end{CJK}.

\section{5. 西北大学 2014 年研究生入学考试试题高等代数}
\begin{CJK}{UTF8}{mj}李扬\end{CJK}

\begin{CJK}{UTF8}{mj}微信公众号\end{CJK}: sxkyliyang

\begin{CJK}{UTF8}{mj}一\end{CJK}、\begin{CJK}{UTF8}{mj}填空题\end{CJK} (\begin{CJK}{UTF8}{mj}每小题\end{CJK} 5 \begin{CJK}{UTF8}{mj}分\end{CJK}, \begin{CJK}{UTF8}{mj}本题共\end{CJK} 30 \begin{CJK}{UTF8}{mj}分\end{CJK})

\begin{enumerate}
  \item \begin{CJK}{UTF8}{mj}设\end{CJK} $f(x)=(x-2)^{2013}\left(x^{2}-x+2\right)^{2014}$, \begin{CJK}{UTF8}{mj}则\end{CJK} $f(x)$ \begin{CJK}{UTF8}{mj}展开式中各项系数之和为\end{CJK}

  \item \begin{CJK}{UTF8}{mj}设\end{CJK} $A$ \begin{CJK}{UTF8}{mj}是正交矩阵且\end{CJK} $|A|=-1$, \begin{CJK}{UTF8}{mj}则\end{CJK} $|E+A|=$

  \item \begin{CJK}{UTF8}{mj}设齐次线性方程组\end{CJK} $\left\{\begin{array}{l}\lambda x_{1}+x_{2}+\lambda^{2} x_{3}=0 ; \\ x_{1}+\lambda x_{2}+x_{3}=0 ; \\ x_{1}+x_{2}+\lambda x_{3}=0 .\end{array}\right.$ \begin{CJK}{UTF8}{mj}的系数矩阵记为\end{CJK} $A$, \begin{CJK}{UTF8}{mj}若存在三阶方阵\end{CJK} $B \neq 0$, \begin{CJK}{UTF8}{mj}使得\end{CJK} $A B=0$, \begin{CJK}{UTF8}{mj}则\end{CJK} $\lambda=$ $-|B|=$

  \item \begin{CJK}{UTF8}{mj}设\end{CJK} $\alpha_{1}=(2,1,-3)^{\prime}, \alpha_{2}=(1,0,-1)^{\prime}, \alpha_{3}=(3,2,-5)^{\prime}, \beta=(1,-3, a)^{\prime}, \gamma=(b,-2,1)^{\prime}$, \begin{CJK}{UTF8}{mj}向量组\end{CJK} $\alpha_{1}, \alpha_{2}$, $\alpha_{3}$ \begin{CJK}{UTF8}{mj}与向\end{CJK} \begin{CJK}{UTF8}{mj}量组\end{CJK} $\alpha_{1}, \alpha_{2}, \alpha_{3}, \beta, \gamma$ \begin{CJK}{UTF8}{mj}等价\end{CJK}, \begin{CJK}{UTF8}{mj}则\end{CJK} $a=$ ,$b=$

  \item \begin{CJK}{UTF8}{mj}设二次型\end{CJK} $f\left(x_{1}, x_{2}, x_{3}\right)=x_{1}^{2}+x_{2}^{2}+5 x_{3}^{2}+2 \lambda x_{1} x_{2}-2 x_{1} x_{3}+4 x_{2} x_{3}$, \begin{CJK}{UTF8}{mj}则\end{CJK} $\lambda$ \begin{CJK}{UTF8}{mj}满足条件\end{CJK} \begin{CJK}{UTF8}{mj}时\end{CJK}, $f\left(x_{1}, x_{2}, x_{3}\right)$ \begin{CJK}{UTF8}{mj}为正定二次型\end{CJK}.

  \item \begin{CJK}{UTF8}{mj}若四阶矩阵\end{CJK} $A$ \begin{CJK}{UTF8}{mj}与\end{CJK} $B$ \begin{CJK}{UTF8}{mj}相似\end{CJK}, \begin{CJK}{UTF8}{mj}矩阵\end{CJK} $A$ \begin{CJK}{UTF8}{mj}的特征值为\end{CJK} $\frac{1}{2}, \frac{1}{3}, \frac{1}{4}, \frac{1}{5}$, \begin{CJK}{UTF8}{mj}则行列式\end{CJK} $\left|B^{-1}-E\right|=$

\end{enumerate}
\begin{CJK}{UTF8}{mj}二\end{CJK}、\begin{CJK}{UTF8}{mj}解答题\end{CJK}.

\begin{enumerate}
  \item (15 \begin{CJK}{UTF8}{mj}分\end{CJK}) \begin{CJK}{UTF8}{mj}求满足\end{CJK}
\end{enumerate}
$$
f\left(x^{2}\right)=f(x) f(x+1)
$$
\begin{CJK}{UTF8}{mj}的非常数多项式\end{CJK} $f(x)$.

\begin{enumerate}
  \setcounter{enumi}{2}
  \item ( 20 \begin{CJK}{UTF8}{mj}分\end{CJK}) \begin{CJK}{UTF8}{mj}计算下列行列式\end{CJK}
\end{enumerate}
(1)
$$
\left|\begin{array}{cccc}
a_{1}^{n} & a_{1}^{n-1} b_{1} & \cdots & b_{1}^{n} \\
a_{2}^{n} & a_{2}^{n-1} b_{2} & \cdots & b_{2}^{n} \\
\vdots & \vdots & & \vdots \\
a_{n+1}^{n} & a_{n+1}^{n-1} b_{n+1} & \cdots & b_{n+1}^{n}
\end{array}\right|, a_{1} a_{2} \cdots a_{n+1} \neq 0
$$
(2)
$$
\left|\begin{array}{ccccc}
1 & a & a^{2} & \cdots & a^{n} \\
b_{11} & 1 & a & \cdots & a^{n-1} \\
b_{21} & b_{22} & 1 & \cdots & a^{n-2} \\
\vdots & \vdots & \vdots & & \vdots \\
b_{n 1} & b_{n 2} & b_{n 3} & \cdots & 1
\end{array}\right|
$$

\begin{enumerate}
  \setcounter{enumi}{3}
  \item (15 \begin{CJK}{UTF8}{mj}分\end{CJK}) \begin{CJK}{UTF8}{mj}设\end{CJK} $A$ \begin{CJK}{UTF8}{mj}为\end{CJK} $n \times n$ \begin{CJK}{UTF8}{mj}矩阵\end{CJK}, $A^{*}$ \begin{CJK}{UTF8}{mj}为\end{CJK} $A$ \begin{CJK}{UTF8}{mj}的伴随矩阵\end{CJK}, \begin{CJK}{UTF8}{mj}证明\end{CJK}: \begin{CJK}{UTF8}{mj}如果\end{CJK} $A^{2}=A$, \begin{CJK}{UTF8}{mj}那么\end{CJK}
\end{enumerate}
$$
\mathrm{r}\left(A^{*}-E\right)= \begin{cases}0, & \text { 当 } \mathrm{r}(A)=n ; \\ n-1, & \text { 当 } \mathrm{r}(A)=n-1 ; \\ n, & \text { 当 } \mathrm{r}(A)<n .\end{cases}
$$

\begin{enumerate}
  \setcounter{enumi}{4}
  \item ( 15 \begin{CJK}{UTF8}{mj}分\end{CJK}) \begin{CJK}{UTF8}{mj}设\end{CJK} $A$ \begin{CJK}{UTF8}{mj}为\end{CJK} $m$ \begin{CJK}{UTF8}{mj}行\end{CJK} 4 \begin{CJK}{UTF8}{mj}列矩阵\end{CJK}, $\mathrm{r}(A)=3, \alpha_{1}, \alpha_{2}, \alpha_{3}$ \begin{CJK}{UTF8}{mj}是方程组\end{CJK} $A x=0$ \begin{CJK}{UTF8}{mj}的解\end{CJK}, \begin{CJK}{UTF8}{mj}已知\end{CJK} $\alpha_{1}+\alpha_{2}=(2,2,4,6)^{\prime}, \alpha_{1}+$ $2 \alpha_{3}=(0,3,0,6)^{\prime}$, \begin{CJK}{UTF8}{mj}求\end{CJK} $A x=b$ \begin{CJK}{UTF8}{mj}的通解\end{CJK}. 5. ( 30 \begin{CJK}{UTF8}{mj}分\end{CJK}) \begin{CJK}{UTF8}{mj}设\end{CJK} $\mathscr{A}$ \begin{CJK}{UTF8}{mj}是线性空间\end{CJK} $V$ \begin{CJK}{UTF8}{mj}上的线性变换\end{CJK}, $\mathscr{A} V$ \begin{CJK}{UTF8}{mj}与\end{CJK} $\mathscr{A}^{-1}(0)$ \begin{CJK}{UTF8}{mj}分别表示\end{CJK} $\mathscr{A}$ \begin{CJK}{UTF8}{mj}的值域与核\end{CJK}, \begin{CJK}{UTF8}{mj}证明以下几条等价\end{CJK}:
\end{enumerate}
(1) $\mathscr{A}^{2}=\mathscr{A}$;

(2) $\mathscr{A} V=(\mathscr{E}-\mathscr{A})^{-1}(0)$, \begin{CJK}{UTF8}{mj}即\end{CJK} $\mathscr{A}$ \begin{CJK}{UTF8}{mj}的值域等于\end{CJK} $\mathscr{E}-\mathscr{A}$ \begin{CJK}{UTF8}{mj}的核\end{CJK};

(3) $V=\mathscr{A}^{-1}(0) \oplus(\mathscr{E}-\mathscr{A})^{-1}(0)$.

(4) $\mathscr{A}$ \begin{CJK}{UTF8}{mj}在\end{CJK} $V$ \begin{CJK}{UTF8}{mj}的某组基下的矩阵为\end{CJK}

\includegraphics[max width=\textwidth]{2022_04_18_3416d289b173eb9de8c1g-017}

(5) \begin{CJK}{UTF8}{mj}存在\end{CJK} $V$ \begin{CJK}{UTF8}{mj}的子空间\end{CJK} $U$ \begin{CJK}{UTF8}{mj}和\end{CJK} $W$, \begin{CJK}{UTF8}{mj}使得\end{CJK} $V=U+W$, \begin{CJK}{UTF8}{mj}且对任意的\end{CJK} $\alpha \in V$, \begin{CJK}{UTF8}{mj}当\end{CJK} $\alpha=\alpha_{1}+\alpha_{2}, \alpha_{1} \in V, \alpha_{2} \in W$ \begin{CJK}{UTF8}{mj}时有\end{CJK} $\mathscr{A} \alpha=\alpha_{1} .$

\begin{enumerate}
  \setcounter{enumi}{6}
  \item (10 \begin{CJK}{UTF8}{mj}分\end{CJK}) \begin{CJK}{UTF8}{mj}求复系数矩阵\end{CJK}
\end{enumerate}
$$
\left(\begin{array}{ccc}
3 & 0 & 8 \\
3 & -1 & 6 \\
-2 & 0 & -5
\end{array}\right)
$$
\begin{CJK}{UTF8}{mj}的若尔当标准型\end{CJK}.

\begin{enumerate}
  \setcounter{enumi}{7}
  \item ( 15 \begin{CJK}{UTF8}{mj}分\end{CJK}) \begin{CJK}{UTF8}{mj}设\end{CJK}
\end{enumerate}
$$
A=\left(\begin{array}{llll}
0 & 1 & 0 & 0 \\
1 & 0 & 0 & 0 \\
0 & 0 & 2 & 1 \\
0 & 0 & 1 & 2
\end{array}\right)
$$
\begin{CJK}{UTF8}{mj}求正交矩阵\end{CJK} $P$, \begin{CJK}{UTF8}{mj}使得\end{CJK} $(A P)^{\prime}(A P)$ \begin{CJK}{UTF8}{mj}为对角形矩阵\end{CJK}.

\section{6. 西北大学 2015 年研究生入学考试试题高等代数 
 李扬 
 微信公众号: sxkyliyang}
\begin{CJK}{UTF8}{mj}一\end{CJK}、\begin{CJK}{UTF8}{mj}填空题\end{CJK} (\begin{CJK}{UTF8}{mj}每小题\end{CJK} 5 \begin{CJK}{UTF8}{mj}分\end{CJK}, \begin{CJK}{UTF8}{mj}本题共\end{CJK} 30 \begin{CJK}{UTF8}{mj}分\end{CJK})

\begin{enumerate}
  \item \begin{CJK}{UTF8}{mj}多项式\end{CJK} $f\left(x^{3}+1\right)$ \begin{CJK}{UTF8}{mj}除以多项式\end{CJK} $x^{2}-1$ \begin{CJK}{UTF8}{mj}的余式\end{CJK} $r(x)=$

  \item \begin{CJK}{UTF8}{mj}设\end{CJK} $A, B$ \begin{CJK}{UTF8}{mj}是\end{CJK} $n$ \begin{CJK}{UTF8}{mj}阶方阵\end{CJK}, \begin{CJK}{UTF8}{mj}满足\end{CJK} $A^{2}=B^{2}=E$ \begin{CJK}{UTF8}{mj}且\end{CJK} $|A|+|B|=0$, \begin{CJK}{UTF8}{mj}则\end{CJK} $|A+B|=$

  \item \begin{CJK}{UTF8}{mj}设\end{CJK} $A=\left(\begin{array}{ccc}1 & 2 & -2 \\ 2 & -1 & \lambda \\ 3 & 1 & -1\end{array}\right)$, \begin{CJK}{UTF8}{mj}若三阶非零方阵\end{CJK} $B$ \begin{CJK}{UTF8}{mj}的各列均为方程组\end{CJK} $A \vec{x}=\overrightarrow{0}$ \begin{CJK}{UTF8}{mj}的解\end{CJK}, \begin{CJK}{UTF8}{mj}则\end{CJK} $\lambda=$

  \item \begin{CJK}{UTF8}{mj}设\end{CJK} $V$ \begin{CJK}{UTF8}{mj}为\end{CJK} $n$ \begin{CJK}{UTF8}{mj}维线性空间\end{CJK}, $U$ \begin{CJK}{UTF8}{mj}与\end{CJK} $W$ \begin{CJK}{UTF8}{mj}是\end{CJK} $V$ \begin{CJK}{UTF8}{mj}的两个相异的\end{CJK} $n-1$ \begin{CJK}{UTF8}{mj}维子空间\end{CJK}, \begin{CJK}{UTF8}{mj}则\end{CJK} $\operatorname{dim}(U \cap W)=$

\end{enumerate}
5 . \begin{CJK}{UTF8}{mj}设二次型\end{CJK} $f\left(x_{1}, x_{2}, x_{3}\right)=x_{1}^{2}+x_{2}^{2}+5 x_{3}^{2}+2 \lambda x_{1} x_{2}-2 x_{1} x_{3}-4 x_{2} x_{3}$, \begin{CJK}{UTF8}{mj}则\end{CJK} $\lambda$ \begin{CJK}{UTF8}{mj}满足条件\end{CJK} \begin{CJK}{UTF8}{mj}时\end{CJK}, $f\left(x_{1}, x_{2}, x_{3}\right)$ \begin{CJK}{UTF8}{mj}为正定二次型\end{CJK}.

\begin{enumerate}
  \setcounter{enumi}{6}
  \item \begin{CJK}{UTF8}{mj}若三阶方阵\end{CJK} $A$ \begin{CJK}{UTF8}{mj}的特征值为\end{CJK} $1,-1,2$, \begin{CJK}{UTF8}{mj}则\end{CJK} $\left|\left(A^{*}\right)^{2}+E\right|=$
\end{enumerate}
\begin{CJK}{UTF8}{mj}二\end{CJK}、\begin{CJK}{UTF8}{mj}解答题\end{CJK}.

\begin{enumerate}
  \item (10 \begin{CJK}{UTF8}{mj}分\end{CJK}) \begin{CJK}{UTF8}{mj}计算行列式\end{CJK}
\end{enumerate}
$$
\left|\begin{array}{cccc}
1+x & 1 & 1 & 1 \\
1 & 1-x & 1 & 1 \\
1 & 1 & 1+z & 1 \\
1 & 1 & 1 & 1-z
\end{array}\right|
$$

\begin{enumerate}
  \setcounter{enumi}{2}
  \item (20 \begin{CJK}{UTF8}{mj}分\end{CJK}) \begin{CJK}{UTF8}{mj}设\end{CJK} $n$ \begin{CJK}{UTF8}{mj}阶行列式\end{CJK}
\end{enumerate}
$$
D_{n}(x)=\left|\begin{array}{cccc}
a_{11}+x & a_{12}+x & \cdots & a_{1 n}+x \\
a_{21}+x & a_{22}+x & \cdots & a_{2 n}+x \\
\vdots & \vdots & & \vdots \\
a_{n 1}+x & a_{n 2}+x & \cdots & a_{n n}+x
\end{array}\right|
$$
(1) \begin{CJK}{UTF8}{mj}将\end{CJK} $D_{n}(x)$ \begin{CJK}{UTF8}{mj}表示为按\end{CJK} $x$ \begin{CJK}{UTF8}{mj}的幂排列的多项式\end{CJK};

(2) \begin{CJK}{UTF8}{mj}证明行列式\end{CJK} $D_{n}(0)$ \begin{CJK}{UTF8}{mj}所有元素的代数余子式的和等于行列式\end{CJK}
$$
D^{\prime}=\left|\begin{array}{cccc}
1 & 1 & \cdots & 1 \\
a_{21}-a_{11} & a_{22}-a_{12} & \cdots & a_{2 n}-a_{1 n} \\
a_{31}-a_{11} & a_{32}-a_{12} & \cdots & a_{3 n}-a_{1 n} \\
\vdots & \vdots & & \vdots \\
a_{n 1}-a_{11} & a_{n 2}-a_{12} & \cdots & a_{n n}-a_{1 n}
\end{array}\right|
$$

\begin{enumerate}
  \setcounter{enumi}{3}
  \item ( 15 \begin{CJK}{UTF8}{mj}分\end{CJK}) \begin{CJK}{UTF8}{mj}设\end{CJK}
\end{enumerate}
$$
f(x)=x^{3}-\left(a_{1}+a_{2}+a_{3}\right) x^{2}+\left(a_{1} a_{2}+a_{1} a_{3}+a_{2} a_{3}\right) x-\left(a_{1} a_{2} a_{3}+1\right),
$$
$a_{1}, a_{2}, a_{3}$ \begin{CJK}{UTF8}{mj}是互不相同的整数\end{CJK}, \begin{CJK}{UTF8}{mj}证明\end{CJK}: $f(x)$ \begin{CJK}{UTF8}{mj}在有理数域上不可约\end{CJK}.

\begin{enumerate}
  \setcounter{enumi}{4}
  \item (10 \begin{CJK}{UTF8}{mj}分\end{CJK}) \begin{CJK}{UTF8}{mj}设\end{CJK} $V_{1}$ \begin{CJK}{UTF8}{mj}与\end{CJK} $V_{2}$ \begin{CJK}{UTF8}{mj}分别是线性方程组\end{CJK} $x_{1}+x_{2}+\cdots+x_{n}=0$ \begin{CJK}{UTF8}{mj}与\end{CJK} $x_{1}=x_{2}=\cdots=x_{n}$ \begin{CJK}{UTF8}{mj}的解空间\end{CJK}, \begin{CJK}{UTF8}{mj}证明\end{CJK}:
\end{enumerate}
$$
P^{n}=V_{1} \oplus V_{2} .
$$

\begin{enumerate}
  \setcounter{enumi}{5}
  \item (10 \begin{CJK}{UTF8}{mj}分\end{CJK}) \begin{CJK}{UTF8}{mj}设\end{CJK} $A$ \begin{CJK}{UTF8}{mj}是正交矩阵且\end{CJK} $E+A$ \begin{CJK}{UTF8}{mj}可逆\end{CJK}, \begin{CJK}{UTF8}{mj}证明\end{CJK}: \begin{CJK}{UTF8}{mj}若\end{CJK} $B=(E-A)(E+A)^{-1}$, \begin{CJK}{UTF8}{mj}则\end{CJK}
\end{enumerate}
$$
B^{\prime}=-B
$$

\begin{enumerate}
  \setcounter{enumi}{6}
  \item (30 \begin{CJK}{UTF8}{mj}分\end{CJK}) \begin{CJK}{UTF8}{mj}设\end{CJK} $\mathscr{A}$ \begin{CJK}{UTF8}{mj}是线性空间\end{CJK} $V$ \begin{CJK}{UTF8}{mj}上的线性变换\end{CJK}, $\mathscr{A} V$ \begin{CJK}{UTF8}{mj}与\end{CJK} $\mathscr{A}^{-1}(0)$ \begin{CJK}{UTF8}{mj}分别表示\end{CJK} $\mathscr{A}$ \begin{CJK}{UTF8}{mj}的值域与核\end{CJK}, \begin{CJK}{UTF8}{mj}证明以下条等价\end{CJK}:
\end{enumerate}
(1) $\mathscr{A}^{2}=\mathscr{A}$;

(2) $V=\mathscr{A}^{-1}(0) \oplus(\mathscr{E}-\mathscr{A})^{-1}(0)$, \begin{CJK}{UTF8}{mj}这里\end{CJK} $\mathscr{E}$ \begin{CJK}{UTF8}{mj}表示恒等变换\end{CJK}.

\includegraphics[max width=\textwidth]{2022_04_18_3416d289b173eb9de8c1g-019}

\begin{enumerate}
  \setcounter{enumi}{7}
  \item ( 10 \begin{CJK}{UTF8}{mj}分\end{CJK}) \begin{CJK}{UTF8}{mj}设\end{CJK} $A$ \begin{CJK}{UTF8}{mj}为\end{CJK} $n \times n$ \begin{CJK}{UTF8}{mj}矩阵\end{CJK}, $A^{*}$ \begin{CJK}{UTF8}{mj}为\end{CJK} $A$ \begin{CJK}{UTF8}{mj}的伴随矩阵\end{CJK}, \begin{CJK}{UTF8}{mj}证明\end{CJK}: \begin{CJK}{UTF8}{mj}如果\end{CJK} $A^{2}=E$, \begin{CJK}{UTF8}{mj}那么\end{CJK}
\end{enumerate}
$$
\mathrm{r}\left(A^{*}+E\right)+\mathrm{r}\left(A^{*}-E\right)=n
$$

\begin{enumerate}
  \setcounter{enumi}{8}
  \item (15 \begin{CJK}{UTF8}{mj}分\end{CJK}) \begin{CJK}{UTF8}{mj}设求复系数矩阵\end{CJK}
\end{enumerate}
$$
\left(\begin{array}{rrr}
1 & -1 & 2 \\
3 & -3 & 6 \\
2 & -2 & 4
\end{array}\right)
$$
\begin{CJK}{UTF8}{mj}的初等因子\end{CJK}.

\begin{enumerate}
  \setcounter{enumi}{9}
  \item (15 \begin{CJK}{UTF8}{mj}分\end{CJK}) \begin{CJK}{UTF8}{mj}设\end{CJK} $A$ \begin{CJK}{UTF8}{mj}是二次型\end{CJK}
\end{enumerate}
$$
f=2 x_{1}^{2}+a x_{2}^{2}+a x_{3}^{2}+4 x_{2} x_{3}
$$
\begin{CJK}{UTF8}{mj}的矩阵\end{CJK}. \begin{CJK}{UTF8}{mj}已知\end{CJK} $a>1$ \begin{CJK}{UTF8}{mj}且\end{CJK} 1 \begin{CJK}{UTF8}{mj}是\end{CJK} $A$ \begin{CJK}{UTF8}{mj}的一个特征值\end{CJK}, \begin{CJK}{UTF8}{mj}试用正交变换化二次型\end{CJK} $f$ \begin{CJK}{UTF8}{mj}为标准形\end{CJK}.

\section{7. 西北大学 2016 年研究生入学考试试题高等代数}
\begin{CJK}{UTF8}{mj}李扬\end{CJK}

\begin{CJK}{UTF8}{mj}微信公众号\end{CJK}: sxkyliyang

\begin{CJK}{UTF8}{mj}一\end{CJK}、\begin{CJK}{UTF8}{mj}填空题\end{CJK} (\begin{CJK}{UTF8}{mj}每小题\end{CJK} 5 \begin{CJK}{UTF8}{mj}分\end{CJK}, \begin{CJK}{UTF8}{mj}本题共\end{CJK} 20 \begin{CJK}{UTF8}{mj}分\end{CJK})

\begin{enumerate}
  \item \begin{CJK}{UTF8}{mj}已知\end{CJK} $D=\left|\begin{array}{llll}a_{1} & a_{2} & a_{3} & a_{4} \\ a_{2} & a_{2} & a_{4} & a_{5} \\ a_{3} & a_{2} & a_{5} & a_{6} \\ a_{4} & a_{2} & a_{6} & a_{7}\end{array}\right|$, \begin{CJK}{UTF8}{mj}则第三列各元素的代数余子式之和的值为\end{CJK}

  \item \begin{CJK}{UTF8}{mj}设齐次线性方程组\end{CJK} $\left\{\begin{array}{l}x_{1}+2 x_{2}-2 x_{3}=0 ; \\ 2 x_{1}-x_{2}+\lambda x_{3}=0 ; \\ 3 x_{1}+x_{2}-x_{3}=0 .\end{array}\right.$ \begin{CJK}{UTF8}{mj}的系数矩阵为\end{CJK} $A$. \begin{CJK}{UTF8}{mj}若存在三阶非零矩阵\end{CJK} $B$ \begin{CJK}{UTF8}{mj}使得\end{CJK} $A B=0$, \begin{CJK}{UTF8}{mj}则\end{CJK} $\lambda=$

  \item \begin{CJK}{UTF8}{mj}设二次型\end{CJK} $f\left(x_{1}, x_{2}, x_{3}\right)=x_{1}^{2}+4 x_{2}^{2}+4 x_{3}^{2}+2 \lambda x_{1} x_{2}-2 x_{1} x_{3}+4 x_{2} x_{3}$, \begin{CJK}{UTF8}{mj}则\end{CJK} $\lambda$ \begin{CJK}{UTF8}{mj}满足条件\end{CJK} \begin{CJK}{UTF8}{mj}时\end{CJK}, $f\left(x_{1}, x_{2}, x_{3}\right)$ \begin{CJK}{UTF8}{mj}为正定二次型\end{CJK}.

  \item \begin{CJK}{UTF8}{mj}设\end{CJK} $A$ \begin{CJK}{UTF8}{mj}是\end{CJK} $n$ \begin{CJK}{UTF8}{mj}阶实对称矩阵\end{CJK}, \begin{CJK}{UTF8}{mj}满足\end{CJK} $A^{3}-3 A^{2}+3 A-2 E=0$. \begin{CJK}{UTF8}{mj}则\end{CJK} $A$ \begin{CJK}{UTF8}{mj}的特征值为\end{CJK}

\end{enumerate}
\begin{CJK}{UTF8}{mj}二\end{CJK}、\begin{CJK}{UTF8}{mj}解答题\end{CJK}.

\begin{enumerate}
  \item (10 \begin{CJK}{UTF8}{mj}分\end{CJK}) \begin{CJK}{UTF8}{mj}设有线性方程组\end{CJK}
\end{enumerate}
$$
\left\{\begin{array}{l}
x_{1}+\lambda x_{2}+\mu x_{3}+x_{4}=0 \\
2 x_{1}+x_{2}+x_{3}+2 x_{4}=0 \\
3 x_{1}+(2+\lambda) x_{2}+(4+\mu) x_{3}+4 x_{4}=1
\end{array}\right.
$$
\begin{CJK}{UTF8}{mj}已知\end{CJK} $(1,-1,1,-1)$ \begin{CJK}{UTF8}{mj}是该方程组的一个解\end{CJK}.

(1) \begin{CJK}{UTF8}{mj}求出方程组的全部解\end{CJK}, \begin{CJK}{UTF8}{mj}并用对应的齐次线性方程组的基础解系表出全部解\end{CJK};

(2) \begin{CJK}{UTF8}{mj}求出该方程组满足\end{CJK} $x_{2}=x_{3}$ \begin{CJK}{UTF8}{mj}的全部解\end{CJK}.

\begin{enumerate}
  \setcounter{enumi}{2}
  \item (10 \begin{CJK}{UTF8}{mj}分\end{CJK}) \begin{CJK}{UTF8}{mj}化矩阵\end{CJK}
\end{enumerate}
$$
A=\left[\begin{array}{cccc}
3 & -4 & 0 & 0 \\
4 & -5 & 0 & 0 \\
0 & 0 & 3 & -2 \\
0 & 0 & 2 & -1
\end{array}\right]
$$
\begin{CJK}{UTF8}{mj}为若尔当\end{CJK} (Jordan) \begin{CJK}{UTF8}{mj}标准形\end{CJK}.

\begin{enumerate}
  \setcounter{enumi}{3}
  \item (10 \begin{CJK}{UTF8}{mj}分\end{CJK}) \begin{CJK}{UTF8}{mj}设\end{CJK}
\end{enumerate}
$$
A=\left[\begin{array}{ccc}
1 & -2 & 2 \\
-2 & 4 & -4 \\
2 & -4 & 4
\end{array}\right]
$$
\begin{CJK}{UTF8}{mj}求正交矩阵\end{CJK} $T$ \begin{CJK}{UTF8}{mj}使得\end{CJK} $T^{\prime} A T=T^{-1} A T$ \begin{CJK}{UTF8}{mj}为对角阵\end{CJK}.

\begin{enumerate}
  \setcounter{enumi}{4}
  \item ( 10 \begin{CJK}{UTF8}{mj}分\end{CJK}) \begin{CJK}{UTF8}{mj}已知\end{CJK} $P^{2 \times 2}$ \begin{CJK}{UTF8}{mj}的线性变换\end{CJK}
\end{enumerate}
$$
\mathscr{A}(X)=M X-X M
$$
\begin{CJK}{UTF8}{mj}其中\end{CJK} $M=\left(\begin{array}{ll}1 & 2 \\ 0 & 3\end{array}\right), X \in P^{2 \times 2}$. \begin{CJK}{UTF8}{mj}求线性变换\end{CJK} $\mathscr{A}$ \begin{CJK}{UTF8}{mj}的核与值域\end{CJK}. 5. (10 \begin{CJK}{UTF8}{mj}分\end{CJK}) \begin{CJK}{UTF8}{mj}设\end{CJK}
$$
A=\left[\begin{array}{ccc}
1 & 0 & 0 \\
0 & 1 & 0 \\
3 & 1 & 2
\end{array}\right]
$$
$$
\begin{aligned}
& W=\left\{B \mid B \in P^{3 \times 3}, A B=B A\right\} \text {. }
\end{aligned}
$$
\begin{CJK}{UTF8}{mj}求\end{CJK} $W$ \begin{CJK}{UTF8}{mj}的维数与一组基\end{CJK}.

\begin{enumerate}
  \setcounter{enumi}{6}
  \item ( 20 \begin{CJK}{UTF8}{mj}分\end{CJK}) \begin{CJK}{UTF8}{mj}设\end{CJK} $A$ \begin{CJK}{UTF8}{mj}是复数域上的方阵\end{CJK}, \begin{CJK}{UTF8}{mj}证明\end{CJK}:
\end{enumerate}
(1) $A$ \begin{CJK}{UTF8}{mj}的特征值均为零当且仅当存在正整数\end{CJK} $m$, \begin{CJK}{UTF8}{mj}使得\end{CJK}
$$
A^{m}=0
$$
(2) \begin{CJK}{UTF8}{mj}若存在正整数\end{CJK} $n$, \begin{CJK}{UTF8}{mj}使得\end{CJK} $A^{n}=0$, \begin{CJK}{UTF8}{mj}则必有\end{CJK}
$$
|A+E|=1 \text {. }
$$
(\begin{CJK}{UTF8}{mj}此处\end{CJK} $E$ \begin{CJK}{UTF8}{mj}表示与\end{CJK} $A$ \begin{CJK}{UTF8}{mj}同阶的单位矩阵\end{CJK}).

\begin{enumerate}
  \setcounter{enumi}{7}
  \item ( 20 \begin{CJK}{UTF8}{mj}分\end{CJK}) \begin{CJK}{UTF8}{mj}证明\end{CJK}:
\end{enumerate}
$$
\left(x^{2}+x+1\right) \mid\left(x^{2015}+x^{2014}+1\right)
$$

\begin{enumerate}
  \setcounter{enumi}{8}
  \item (20 \begin{CJK}{UTF8}{mj}分\end{CJK}) \begin{CJK}{UTF8}{mj}设实二次型\end{CJK}
\end{enumerate}
$$
f\left(x_{1}, x_{2}, \cdots, x_{n}\right)=\sum_{i=1}^{s}\left(a_{i 1} x_{1}+a_{i 2} x_{2}+\cdots+a_{i n} x_{n}\right)^{2},
$$
\begin{CJK}{UTF8}{mj}证明\end{CJK}: $f\left(x_{1}, x_{2}, \cdots, x_{n}\right)$ \begin{CJK}{UTF8}{mj}的秩等于矩阵\end{CJK} $A=\left(a_{i j}\right)_{s \times n}$ \begin{CJK}{UTF8}{mj}的秩\end{CJK}.

\begin{enumerate}
  \setcounter{enumi}{9}
  \item ( 20 \begin{CJK}{UTF8}{mj}分\end{CJK}) \begin{CJK}{UTF8}{mj}求\end{CJK} 7 \begin{CJK}{UTF8}{mj}次多项式\end{CJK} $f(x)$ \begin{CJK}{UTF8}{mj}使得\end{CJK} $f(x)+1$ \begin{CJK}{UTF8}{mj}能被\end{CJK} $(x-1)^{4}$ \begin{CJK}{UTF8}{mj}整除\end{CJK}, $f(x)-1$ \begin{CJK}{UTF8}{mj}能被\end{CJK} $(x+1)^{4}$ \begin{CJK}{UTF8}{mj}整除\end{CJK}.
\end{enumerate}
\section{8. 西北大学 2017 年研究生入学考试试题高等代数 
 李扬 
 微信公众号: sxkyliyang}
\begin{CJK}{UTF8}{mj}一\end{CJK}、\begin{CJK}{UTF8}{mj}填空题\end{CJK} (\begin{CJK}{UTF8}{mj}每小题\end{CJK} 5 \begin{CJK}{UTF8}{mj}分\end{CJK}, \begin{CJK}{UTF8}{mj}本题共\end{CJK} 30 \begin{CJK}{UTF8}{mj}分\end{CJK})

\begin{enumerate}
  \item 1 \begin{CJK}{UTF8}{mj}是多项式\end{CJK} $x^{7}+2 x^{6}-6 x^{5}-8 x^{4}+17 x^{3}+6 x^{2}-20 x+8$ \begin{CJK}{UTF8}{mj}的\end{CJK} \begin{CJK}{UTF8}{mj}重根\end{CJK}.

  \item \begin{CJK}{UTF8}{mj}设\end{CJK} $k$ \begin{CJK}{UTF8}{mj}为给定的大于\end{CJK} 1 \begin{CJK}{UTF8}{mj}的正整数\end{CJK}, \begin{CJK}{UTF8}{mj}若向量组\end{CJK} $\alpha_{1}, \alpha_{2}, \alpha_{3}, \cdots, \alpha_{2 k}$ \begin{CJK}{UTF8}{mj}线性无关\end{CJK}, \begin{CJK}{UTF8}{mj}则向量组\end{CJK} $\alpha_{1}+\alpha_{2}, \alpha_{2}+$ $\alpha_{3}, \cdots, \alpha_{2 k-1}+\alpha_{2 k}, \alpha_{2 k}+\alpha_{1}$ \begin{CJK}{UTF8}{mj}线性\end{CJK}

  \item \begin{CJK}{UTF8}{mj}设\end{CJK} $A$ \begin{CJK}{UTF8}{mj}是\end{CJK} 3 \begin{CJK}{UTF8}{mj}阶矩阵\end{CJK}, $|A|=4, A^{*}$ \begin{CJK}{UTF8}{mj}是\end{CJK} $A$ \begin{CJK}{UTF8}{mj}的伴随矩阵\end{CJK}. \begin{CJK}{UTF8}{mj}若将\end{CJK} $A$ \begin{CJK}{UTF8}{mj}的第一行和第二行互换得到矩阵\end{CJK} $B$, \begin{CJK}{UTF8}{mj}则\end{CJK} $\left|B A^{*}\right|=$

  \item \begin{CJK}{UTF8}{mj}设\end{CJK} $\alpha_{1}=(1,1,0), \alpha_{2}=(1,0,1), \alpha_{3}=(0,1,1), \alpha_{4}=(1,2,3)$, \begin{CJK}{UTF8}{mj}令\end{CJK} $W_{1}$ \begin{CJK}{UTF8}{mj}和\end{CJK} $W_{2}$ \begin{CJK}{UTF8}{mj}分别为由\end{CJK} $\alpha_{1}, \alpha_{2}$ \begin{CJK}{UTF8}{mj}和\end{CJK} $\alpha_{3}, \alpha_{4}$ \begin{CJK}{UTF8}{mj}生成的\end{CJK} \begin{CJK}{UTF8}{mj}子空间\end{CJK}, \begin{CJK}{UTF8}{mj}则\end{CJK} $W_{1}+W_{2}$ \begin{CJK}{UTF8}{mj}的维数等于\end{CJK}

  \item \begin{CJK}{UTF8}{mj}设\end{CJK} $A$ \begin{CJK}{UTF8}{mj}为\end{CJK} 3 \begin{CJK}{UTF8}{mj}阶实对称矩阵\end{CJK}, $r(A)=2$, \begin{CJK}{UTF8}{mj}且\end{CJK} $A\left(\begin{array}{cc}1 & 1 \\ 0 & 0 \\ -1 & 1\end{array}\right)=\left(\begin{array}{cc}-1 & 1 \\ 0 & 0 \\ 1 & 1\end{array}\right)$. \begin{CJK}{UTF8}{mj}则\end{CJK} $A$ \begin{CJK}{UTF8}{mj}的特征值是\end{CJK}

  \item \begin{CJK}{UTF8}{mj}设二次型\end{CJK} $f\left(x_{1}, x_{2}, x_{3}\right)=x^{\prime} A x$ \begin{CJK}{UTF8}{mj}在正交变化\end{CJK} $x=Q y$ \begin{CJK}{UTF8}{mj}下的标准形为\end{CJK} $y_{1}^{2}+y_{2}^{2}$, \begin{CJK}{UTF8}{mj}且\end{CJK} $Q$ \begin{CJK}{UTF8}{mj}的第三列为\end{CJK} $\left(\frac{\sqrt{2}}{2}, 0, \frac{\sqrt{2}}{2}\right)$, \begin{CJK}{UTF8}{mj}则\end{CJK} $A=$

\end{enumerate}
\section{一、解答题.}
\begin{enumerate}
  \item (10 \begin{CJK}{UTF8}{mj}分\end{CJK}) \begin{CJK}{UTF8}{mj}设\end{CJK} $f_{i}(x), i=1,2,3,4$ \begin{CJK}{UTF8}{mj}为数域\end{CJK} $P$ \begin{CJK}{UTF8}{mj}上的多项式且满足\end{CJK}
\end{enumerate}
$$
x^{4}+x^{3}+x^{2}+x+1 \mid f_{1}\left(x^{5}\right)+x f_{2}\left(x^{5}\right)+x^{2} f_{3}\left(x^{5}\right)+x^{3} f_{4}\left(x^{5}\right),
$$
$A$ \begin{CJK}{UTF8}{mj}为\end{CJK} $n$ \begin{CJK}{UTF8}{mj}阶方阵且有一个特征值是\end{CJK} 1 . \begin{CJK}{UTF8}{mj}证明\end{CJK}: $f_{1}(A), f_{2}(A), f_{3}(A)$ \begin{CJK}{UTF8}{mj}和\end{CJK} $f_{4}(A)$ \begin{CJK}{UTF8}{mj}均不是可逆矩阵\end{CJK}.

\begin{enumerate}
  \setcounter{enumi}{2}
  \item (15 \begin{CJK}{UTF8}{mj}分\end{CJK}) \begin{CJK}{UTF8}{mj}设\end{CJK} $A$ \begin{CJK}{UTF8}{mj}是\end{CJK} $s \times n$ \begin{CJK}{UTF8}{mj}矩阵\end{CJK}, $B$ \begin{CJK}{UTF8}{mj}是\end{CJK} $m \times n$ \begin{CJK}{UTF8}{mj}矩阵\end{CJK}, \begin{CJK}{UTF8}{mj}且满足其次线性方程组\end{CJK} $A X=0$ \begin{CJK}{UTF8}{mj}的解都是\end{CJK} $B X=0$ \begin{CJK}{UTF8}{mj}的解\end{CJK}. \begin{CJK}{UTF8}{mj}试问\end{CJK}:
\end{enumerate}
(1) $A$ \begin{CJK}{UTF8}{mj}的秩和\end{CJK} $B$ \begin{CJK}{UTF8}{mj}的秩的大小关系是什么\end{CJK}? \begin{CJK}{UTF8}{mj}请说明你的理由\end{CJK}.

(2) \begin{CJK}{UTF8}{mj}若\end{CJK} $A$ \begin{CJK}{UTF8}{mj}的秩和\end{CJK} $B$ \begin{CJK}{UTF8}{mj}的秩相等\end{CJK}, $A X=0$ \begin{CJK}{UTF8}{mj}和\end{CJK} $B X=0$ \begin{CJK}{UTF8}{mj}是否是同解的\end{CJK}? \begin{CJK}{UTF8}{mj}为什么\end{CJK}?

\begin{enumerate}
  \setcounter{enumi}{3}
  \item (10 \begin{CJK}{UTF8}{mj}分\end{CJK}) \begin{CJK}{UTF8}{mj}计算下面行列式的值\end{CJK}:
\end{enumerate}
$$
\left|\begin{array}{cccccc}
1 & 3 & 3 & 3 & \cdots & 3 \\
3 & 2 & 3 & 3 & \cdots & 3 \\
3 & 3 & 3 & 3 & \cdots & 3 \\
3 & 3 & 3 & 4 & \cdots & 3 \\
\vdots & \vdots & \vdots & \vdots & & \vdots \\
3 & 3 & 3 & 3 & \cdots & n
\end{array}\right| .
$$

\begin{enumerate}
  \setcounter{enumi}{4}
  \item (15 \begin{CJK}{UTF8}{mj}分\end{CJK}) \begin{CJK}{UTF8}{mj}设\end{CJK} $V=\left\{a x^{3}+b x^{2}+c x+d \mid a, b, c, d \in \mathbb{R}\right\}$ \begin{CJK}{UTF8}{mj}是实数域\end{CJK} $R$ \begin{CJK}{UTF8}{mj}上的四维线性空间\end{CJK}. \begin{CJK}{UTF8}{mj}对任意的\end{CJK} $f(x) \in V$, \begin{CJK}{UTF8}{mj}定\end{CJK} X
\end{enumerate}
$$
\mathscr{A}(f(x))=3 f(x)+x f^{\prime}(x)
$$
(1) \begin{CJK}{UTF8}{mj}证明\end{CJK}: $\mathscr{A}$ \begin{CJK}{UTF8}{mj}是\end{CJK} $V$ \begin{CJK}{UTF8}{mj}上的线性变换\end{CJK};

(2) \begin{CJK}{UTF8}{mj}求\end{CJK} $\mathscr{A}$ \begin{CJK}{UTF8}{mj}的特征值和特征向量\end{CJK}.

\begin{enumerate}
  \setcounter{enumi}{5}
  \item ( 15 \begin{CJK}{UTF8}{mj}分\end{CJK}) \begin{CJK}{UTF8}{mj}已知矩阵\end{CJK}
\end{enumerate}
$$
A=\left[\begin{array}{cccc}
2 & -4 & 2 & 2 \\
-2 & 0 & 1 & 3 \\
-2 & -2 & 3 & 3 \\
-2 & -6 & 3 & 7
\end{array}\right]
$$
\begin{CJK}{UTF8}{mj}求可逆矩阵\end{CJK} $P$, \begin{CJK}{UTF8}{mj}使得\end{CJK} $P^{-1} A P$ \begin{CJK}{UTF8}{mj}为\end{CJK} Jordan \begin{CJK}{UTF8}{mj}标准形\end{CJK}.

\begin{enumerate}
  \setcounter{enumi}{6}
  \item (15 \begin{CJK}{UTF8}{mj}分\end{CJK}) \begin{CJK}{UTF8}{mj}设\end{CJK} $V_{1}$ \begin{CJK}{UTF8}{mj}是\end{CJK} $n$ \begin{CJK}{UTF8}{mj}维线性空间\end{CJK} $V_{1}$ \begin{CJK}{UTF8}{mj}的真子空间\end{CJK}. \begin{CJK}{UTF8}{mj}证明\end{CJK}: \begin{CJK}{UTF8}{mj}至少存在\end{CJK} $V$ \begin{CJK}{UTF8}{mj}的两个不同的子空间\end{CJK} $W_{1}$ \begin{CJK}{UTF8}{mj}和\end{CJK} $W_{2}$, \begin{CJK}{UTF8}{mj}使得\end{CJK}
\end{enumerate}
$$
V=V_{1} \oplus W_{1}=V_{1} \oplus W_{2}
$$

\begin{enumerate}
  \setcounter{enumi}{7}
  \item ( 15 \begin{CJK}{UTF8}{mj}分\end{CJK}) \begin{CJK}{UTF8}{mj}设\end{CJK}
\end{enumerate}
$$
A=\left(\begin{array}{ccc}
1 & 0 & 1 \\
0 & 1 & 1 \\
-1 & 0 & a
\end{array}\right)
$$
\begin{CJK}{UTF8}{mj}二次型\end{CJK} $f=X^{\prime} A^{\prime} A X$, \begin{CJK}{UTF8}{mj}已知\end{CJK} $\mathrm{r}\left(A^{\prime} A\right)=2$, \begin{CJK}{UTF8}{mj}试用正交线性替换化该二次型为标准形\end{CJK}.

\begin{enumerate}
  \setcounter{enumi}{8}
  \item (15 \begin{CJK}{UTF8}{mj}分\end{CJK}) \begin{CJK}{UTF8}{mj}证明\end{CJK}: $n$ \begin{CJK}{UTF8}{mj}阶矩阵\end{CJK} $A$ \begin{CJK}{UTF8}{mj}相似于对角矩阵的充分必要条件是对任意的\end{CJK} $\lambda$, \begin{CJK}{UTF8}{mj}若\end{CJK} $(\lambda E-A)^{2} X=0$, \begin{CJK}{UTF8}{mj}则\end{CJK}
\end{enumerate}
$$
(\lambda E-A) X=0,
$$
\begin{CJK}{UTF8}{mj}其中\end{CJK} $X$ \begin{CJK}{UTF8}{mj}为\end{CJK} $n$ \begin{CJK}{UTF8}{mj}维列向量\end{CJK}.

\begin{enumerate}
  \setcounter{enumi}{9}
  \item ( 10 \begin{CJK}{UTF8}{mj}分\end{CJK}) \begin{CJK}{UTF8}{mj}设\end{CJK} $A$ \begin{CJK}{UTF8}{mj}为\end{CJK} $n$ \begin{CJK}{UTF8}{mj}阶矩阵\end{CJK}, \begin{CJK}{UTF8}{mj}且满足\end{CJK} $\mathrm{r}(A)=\mathrm{r}\left(A^{2}\right)$. \begin{CJK}{UTF8}{mj}证明\end{CJK}:
\end{enumerate}
$$
\mathrm{r}\left(A^{2}\right)=\mathrm{r}\left(A^{3}\right)
$$

\section{1. 西南大学 2009 年研究生入学考试试题高等代数}
\begin{CJK}{UTF8}{mj}李扬\end{CJK}

\begin{CJK}{UTF8}{mj}微信公众号\end{CJK}: sxkyliyang

\begin{CJK}{UTF8}{mj}一\end{CJK}. \begin{CJK}{UTF8}{mj}填空\end{CJK}(\begin{CJK}{UTF8}{mj}每小题\end{CJK} 5 \begin{CJK}{UTF8}{mj}分\end{CJK}, \begin{CJK}{UTF8}{mj}共\end{CJK} 40 \begin{CJK}{UTF8}{mj}分\end{CJK})

(1) \begin{CJK}{UTF8}{mj}设\end{CJK} $A$ \begin{CJK}{UTF8}{mj}为\end{CJK} 3 \begin{CJK}{UTF8}{mj}阶方阵\end{CJK}, $|A|=\frac{1}{2}, A^{*}$ \begin{CJK}{UTF8}{mj}为\end{CJK} $A$ \begin{CJK}{UTF8}{mj}的伴随矩阵\end{CJK}, \begin{CJK}{UTF8}{mj}则\end{CJK} $\left|(2 A)^{-1}-5 A^{*}\right|=$

(2) \begin{CJK}{UTF8}{mj}设\end{CJK} $A=\left(\begin{array}{ccc}1 & -2 & 3 k \\ -1 & 2 k & -3 \\ k & -2 & 3\end{array}\right)$, \begin{CJK}{UTF8}{mj}若\end{CJK} $A$ \begin{CJK}{UTF8}{mj}的秩为\end{CJK} 2 , \begin{CJK}{UTF8}{mj}则\end{CJK} $k=$

(3) \begin{CJK}{UTF8}{mj}若复数域上多项式\end{CJK} $f(x)=x^{3}-3 x^{2}+x-t$ \begin{CJK}{UTF8}{mj}有重根\end{CJK}, \begin{CJK}{UTF8}{mj}则\end{CJK} $t=$

(4) $n$ \begin{CJK}{UTF8}{mj}元实二次型\end{CJK} $(n-1) \sum_{i=1}^{n} x_{i}^{2}-2 \sum_{1 \leqslant j<k \leqslant n} x_{j} x_{k}$ \begin{CJK}{UTF8}{mj}的符号差是\end{CJK}

(5) \begin{CJK}{UTF8}{mj}设\end{CJK} $A=\left(\begin{array}{cccc}1 & -a & 0 & 0 \\ 0 & 1 & -a & 0 \\ 0 & 0 & 1 & -a \\ 0 & 0 & 0 & 1\end{array}\right)$, \begin{CJK}{UTF8}{mj}则\end{CJK} $A^{-1}=$

(6) \begin{CJK}{UTF8}{mj}给定\end{CJK} $P^{3}$ \begin{CJK}{UTF8}{mj}中的线性变换\end{CJK} $\mathscr{A}$ \begin{CJK}{UTF8}{mj}如下\end{CJK}:
$$
\mathscr{A}:\left(x_{1}, x_{2}, x_{3}\right) \rightarrow\left(2 x_{1}-x_{2}, x_{2}+x_{3}, 2 x_{1}+x_{3}\right)
$$
\begin{CJK}{UTF8}{mj}则\end{CJK} $\operatorname{ker} \mathscr{A}=$

(7) \begin{CJK}{UTF8}{mj}令\end{CJK} $\mathscr{A}$ \begin{CJK}{UTF8}{mj}为\end{CJK} $\mathbb{R}^{4}$ \begin{CJK}{UTF8}{mj}的正交变换\end{CJK}, $\alpha=(2,0,-1,-2)$ \begin{CJK}{UTF8}{mj}为\end{CJK} $\mathscr{A}$ \begin{CJK}{UTF8}{mj}的一个特征向量\end{CJK}, \begin{CJK}{UTF8}{mj}则\end{CJK} $(\mathscr{A} \alpha, \mathscr{A} \alpha)=$

(8) \begin{CJK}{UTF8}{mj}设\end{CJK} $A$ \begin{CJK}{UTF8}{mj}是实数域上\end{CJK} $n$ \begin{CJK}{UTF8}{mj}阶反对称矩阵\end{CJK}, \begin{CJK}{UTF8}{mj}若\end{CJK} $n$ \begin{CJK}{UTF8}{mj}为奇数\end{CJK}, \begin{CJK}{UTF8}{mj}则\end{CJK} $A$ \begin{CJK}{UTF8}{mj}的特征根是\end{CJK}

\begin{CJK}{UTF8}{mj}二\end{CJK}. (30 \begin{CJK}{UTF8}{mj}分\end{CJK}) \begin{CJK}{UTF8}{mj}设\end{CJK} $A, B$ \begin{CJK}{UTF8}{mj}为\end{CJK} $n$ \begin{CJK}{UTF8}{mj}阶实对称矩阵\end{CJK}, $C$ \begin{CJK}{UTF8}{mj}为\end{CJK} $n$ \begin{CJK}{UTF8}{mj}阶实反对称矩阵\end{CJK}, \begin{CJK}{UTF8}{mj}且\end{CJK} $A^{2}+B^{2}=C^{2}$. \begin{CJK}{UTF8}{mj}证明\end{CJK}: $A=B=C=0$.

\begin{CJK}{UTF8}{mj}三\end{CJK}. (30 \begin{CJK}{UTF8}{mj}分\end{CJK}) \begin{CJK}{UTF8}{mj}设\end{CJK} $A, B, C$ \begin{CJK}{UTF8}{mj}为一个三角形的三个内角\end{CJK}. \begin{CJK}{UTF8}{mj}证明对任意实数\end{CJK} $x, y, z$ \begin{CJK}{UTF8}{mj}有\end{CJK}

(1) \begin{CJK}{UTF8}{mj}证明\end{CJK} $W$ \begin{CJK}{UTF8}{mj}是\end{CJK} $\mathbb{R}^{n \times n}$ \begin{CJK}{UTF8}{mj}的子空间\end{CJK};

(2) \begin{CJK}{UTF8}{mj}求\end{CJK} $\operatorname{dim} W$.

\begin{CJK}{UTF8}{mj}五\end{CJK}. ( 20 \begin{CJK}{UTF8}{mj}分\end{CJK}) \begin{CJK}{UTF8}{mj}设\end{CJK} $V$ \begin{CJK}{UTF8}{mj}是数域\end{CJK} $P$ \begin{CJK}{UTF8}{mj}上\end{CJK} $n$ \begin{CJK}{UTF8}{mj}维线性空间\end{CJK}, $T$ \begin{CJK}{UTF8}{mj}是\end{CJK} $V$ \begin{CJK}{UTF8}{mj}的可对角化线性变换\end{CJK}, $W$ \begin{CJK}{UTF8}{mj}是\end{CJK} $V$ \begin{CJK}{UTF8}{mj}的\end{CJK} $T$ \begin{CJK}{UTF8}{mj}不变子空间\end{CJK}.

(1) \begin{CJK}{UTF8}{mj}证明存在\end{CJK} $V$ \begin{CJK}{UTF8}{mj}的\end{CJK} $T$ \begin{CJK}{UTF8}{mj}不变子空间\end{CJK} $W^{\prime}$, \begin{CJK}{UTF8}{mj}使\end{CJK} $V=W \oplus W^{\prime}$;

(2) \begin{CJK}{UTF8}{mj}令\end{CJK} $\left.T\right|_{W}$ \begin{CJK}{UTF8}{mj}是\end{CJK} $T$ \begin{CJK}{UTF8}{mj}在\end{CJK} $W$ \begin{CJK}{UTF8}{mj}上的限制线性变换\end{CJK}, \begin{CJK}{UTF8}{mj}证明\end{CJK} $\left.T\right|_{W}$ \begin{CJK}{UTF8}{mj}也可对角化\end{CJK}.

\begin{CJK}{UTF8}{mj}六\end{CJK}. (20 \begin{CJK}{UTF8}{mj}分\end{CJK}) \begin{CJK}{UTF8}{mj}设\end{CJK} $A$ \begin{CJK}{UTF8}{mj}为复数域\end{CJK} $C$ \begin{CJK}{UTF8}{mj}上的\end{CJK} $n$ \begin{CJK}{UTF8}{mj}阶方阵\end{CJK}, $A$ \begin{CJK}{UTF8}{mj}的最小多项式无重根\end{CJK}, \begin{CJK}{UTF8}{mj}证明\end{CJK}: \begin{CJK}{UTF8}{mj}存在\end{CJK} $n$ \begin{CJK}{UTF8}{mj}阶方阵\end{CJK} $B$ \begin{CJK}{UTF8}{mj}使\end{CJK} $A=B^{2}$.
$$
\begin{aligned}
& x^{2}+y^{2}+z^{2} \geqslant 2 x y \cos A+2 x z \cos B+2 y z \cos C
\end{aligned}
$$
\includegraphics[max width=\textwidth]{2022_04_18_3416d289b173eb9de8c1g-024}
$$
\begin{aligned}
& W=\{f(A) \mid f(x) \in \mathbb{R}[x]\} 
\end{aligned}
$$

\section{2. 西南大学 2010 年研究生入学考试试题高等代数}
\begin{CJK}{UTF8}{mj}李扬\end{CJK}

\begin{CJK}{UTF8}{mj}微信公众号\end{CJK}: sxkyliyang

\begin{CJK}{UTF8}{mj}一\end{CJK}. \begin{CJK}{UTF8}{mj}填空\end{CJK}(\begin{CJK}{UTF8}{mj}每小题\end{CJK} 8 \begin{CJK}{UTF8}{mj}分\end{CJK}, \begin{CJK}{UTF8}{mj}共\end{CJK} 40 \begin{CJK}{UTF8}{mj}分\end{CJK})

(1) \begin{CJK}{UTF8}{mj}每一行和每一列只有一个元素为\end{CJK} 1 \begin{CJK}{UTF8}{mj}其余元素为零的\end{CJK} $n$ \begin{CJK}{UTF8}{mj}阶行列式\end{CJK} $(n \geqslant 2)$ \begin{CJK}{UTF8}{mj}共有\end{CJK} \begin{CJK}{UTF8}{mj}个\end{CJK}, \begin{CJK}{UTF8}{mj}所有\end{CJK} \begin{CJK}{UTF8}{mj}这些行列式的和等于\end{CJK}

(2) \begin{CJK}{UTF8}{mj}设\end{CJK} $A=\left(\begin{array}{ccc}1 & -2 & 3 k \\ -1 & 2 k & -3 \\ k & -2 & 3\end{array}\right)$, \begin{CJK}{UTF8}{mj}若齐次线性方程组\end{CJK} $A X=0$ \begin{CJK}{UTF8}{mj}有非零解\end{CJK}, \begin{CJK}{UTF8}{mj}则\end{CJK} $k=$

(3) \begin{CJK}{UTF8}{mj}设\end{CJK} $f(x), g(x)$ \begin{CJK}{UTF8}{mj}为复数域上两个最高项系数为\end{CJK} 1 \begin{CJK}{UTF8}{mj}的不同的\end{CJK} 3 \begin{CJK}{UTF8}{mj}次多项式\end{CJK}, \begin{CJK}{UTF8}{mj}若\end{CJK} $x^{4}+x^{2}+1 \mid f\left(x^{3}\right)+x^{4} g\left(x^{3}\right)$, \begin{CJK}{UTF8}{mj}则\end{CJK} $(f(x), g(x))=$

(4) $n$ \begin{CJK}{UTF8}{mj}元实二次型\end{CJK} $f\left(x_{1}, x_{2}, \cdots, x_{n}\right)=\sum_{1 \leqslant i<k \leqslant n}|i-k| x_{i} x_{k}$ \begin{CJK}{UTF8}{mj}的标准形是\end{CJK}

(5) \begin{CJK}{UTF8}{mj}在欧氏空间\end{CJK} $\mathbb{R}^{n}$ \begin{CJK}{UTF8}{mj}中定义变换\end{CJK} $\mathscr{A}$ :
$$
\mathscr{A} \alpha=\alpha-k(\alpha, \varepsilon) \varepsilon
$$
\begin{CJK}{UTF8}{mj}其中\end{CJK} $\varepsilon$ \begin{CJK}{UTF8}{mj}为单位向量\end{CJK}, $k$ \begin{CJK}{UTF8}{mj}为实数\end{CJK}. \begin{CJK}{UTF8}{mj}若\end{CJK} $\mathscr{A}$ \begin{CJK}{UTF8}{mj}为正交变换\end{CJK}, \begin{CJK}{UTF8}{mj}则\end{CJK} $k=$

\begin{CJK}{UTF8}{mj}二\end{CJK}. ( 30 \begin{CJK}{UTF8}{mj}分\end{CJK}) \begin{CJK}{UTF8}{mj}设\end{CJK} $A=\left(\begin{array}{ccc}1 & 0 & 2 \\ 0 & -1 & 1 \\ 0 & 1 & 0\end{array}\right), f(x)=2 x^{11}+2 x^{8}-8 x^{7}+3 x^{5}+x^{4}+17 x^{2}-4$, \begin{CJK}{UTF8}{mj}求\end{CJK} $(f(A))^{-1}$.

\begin{CJK}{UTF8}{mj}三\end{CJK}. (30 \begin{CJK}{UTF8}{mj}分\end{CJK}) \begin{CJK}{UTF8}{mj}设\end{CJK} $V$ \begin{CJK}{UTF8}{mj}是数域\end{CJK} $P$ \begin{CJK}{UTF8}{mj}上\end{CJK} $n$ \begin{CJK}{UTF8}{mj}维线性空间\end{CJK}, $T$ \begin{CJK}{UTF8}{mj}是\end{CJK} $V$ \begin{CJK}{UTF8}{mj}的线性变换\end{CJK}. $\lambda_{1}, \lambda_{2}, \cdots, \lambda_{k}$ \begin{CJK}{UTF8}{mj}是\end{CJK} $T$ \begin{CJK}{UTF8}{mj}的互不相同的特征\end{CJK} \begin{CJK}{UTF8}{mj}值\end{CJK}, $V_{\lambda_{i}}, i=1,2, \cdots, k$ \begin{CJK}{UTF8}{mj}是\end{CJK} $T$ \begin{CJK}{UTF8}{mj}的特征子空间\end{CJK}, \begin{CJK}{UTF8}{mj}且\end{CJK} $V=V_{\lambda_{1}} \oplus V_{\lambda_{2}} \oplus \cdots \oplus V_{\lambda_{k}}$. $W$ \begin{CJK}{UTF8}{mj}是\end{CJK} $V$ \begin{CJK}{UTF8}{mj}的\end{CJK} $T$ \begin{CJK}{UTF8}{mj}不变子空间\end{CJK}. \begin{CJK}{UTF8}{mj}证明\end{CJK}: $W$ \begin{CJK}{UTF8}{mj}中每个向量\end{CJK} $\eta$ \begin{CJK}{UTF8}{mj}可唯一表成\end{CJK} $\eta=\xi_{1}+\xi_{2}+\cdots+\xi_{k}$, \begin{CJK}{UTF8}{mj}其中\end{CJK} $\xi_{i} \in V_{\lambda_{i}} \cap W, i=1,2, \cdots, k$.

\begin{CJK}{UTF8}{mj}四\end{CJK}. ( 20 \begin{CJK}{UTF8}{mj}分\end{CJK}) \begin{CJK}{UTF8}{mj}设\end{CJK} $V$ \begin{CJK}{UTF8}{mj}是数域\end{CJK} $P$ \begin{CJK}{UTF8}{mj}上的\end{CJK} $n$ \begin{CJK}{UTF8}{mj}维线性空间\end{CJK}, $T$ \begin{CJK}{UTF8}{mj}是\end{CJK} $V$ \begin{CJK}{UTF8}{mj}的线性变换\end{CJK}. \begin{CJK}{UTF8}{mj}证明\end{CJK}: \begin{CJK}{UTF8}{mj}存在\end{CJK} $V$ \begin{CJK}{UTF8}{mj}的线性变换\end{CJK} $S$ \begin{CJK}{UTF8}{mj}使得\end{CJK} $T S T=T$.

\begin{CJK}{UTF8}{mj}五\end{CJK}. ( 20 \begin{CJK}{UTF8}{mj}分\end{CJK}) \begin{CJK}{UTF8}{mj}设\end{CJK} $A$ \begin{CJK}{UTF8}{mj}为\end{CJK} $n$ \begin{CJK}{UTF8}{mj}阶实对称阵\end{CJK}, $B$ \begin{CJK}{UTF8}{mj}为\end{CJK} $n$ \begin{CJK}{UTF8}{mj}阶实矩阵\end{CJK}, \begin{CJK}{UTF8}{mj}且\end{CJK} $B A+A B^{T}$ \begin{CJK}{UTF8}{mj}的特征值全大于零\end{CJK}, \begin{CJK}{UTF8}{mj}其中\end{CJK} $B^{T}$ \begin{CJK}{UTF8}{mj}为\end{CJK} $B$ \begin{CJK}{UTF8}{mj}的\end{CJK} \begin{CJK}{UTF8}{mj}转置\end{CJK}. \begin{CJK}{UTF8}{mj}证明\end{CJK}: $A$ \begin{CJK}{UTF8}{mj}可逆\end{CJK}.

\begin{CJK}{UTF8}{mj}六\end{CJK}. (10 \begin{CJK}{UTF8}{mj}分\end{CJK}) \begin{CJK}{UTF8}{mj}设\end{CJK} $X, B_{0}$ \begin{CJK}{UTF8}{mj}为\end{CJK} $n$ \begin{CJK}{UTF8}{mj}阶实矩阵\end{CJK}, \begin{CJK}{UTF8}{mj}按归纳法定义矩阵序列\end{CJK}
$$
B_{i}=B_{i-1} X-X B_{i-1}, \quad i=1,2,3, \cdots
$$
\begin{CJK}{UTF8}{mj}正明\end{CJK}: \begin{CJK}{UTF8}{mj}如果\end{CJK} $B_{n^{2}}=X$, \begin{CJK}{UTF8}{mj}那么\end{CJK} $X=0$.

\section{3. 西南大学 2011 年研究生入学考试试题高等代数}
\begin{CJK}{UTF8}{mj}李扬\end{CJK}

\begin{CJK}{UTF8}{mj}微信公众号\end{CJK}: sxkyliyang

\begin{CJK}{UTF8}{mj}一\end{CJK}. \begin{CJK}{UTF8}{mj}填空\end{CJK}(\begin{CJK}{UTF8}{mj}每小题\end{CJK} 6 \begin{CJK}{UTF8}{mj}分\end{CJK}, \begin{CJK}{UTF8}{mj}共\end{CJK} 60 \begin{CJK}{UTF8}{mj}分\end{CJK})

\begin{enumerate}
  \item \begin{CJK}{UTF8}{mj}设\end{CJK} $n \geqslant 3$, \begin{CJK}{UTF8}{mj}在由\end{CJK} $1,2, \cdots, n$ \begin{CJK}{UTF8}{mj}构成的\end{CJK} $n$ ! \begin{CJK}{UTF8}{mj}个\end{CJK} $n$ \begin{CJK}{UTF8}{mj}级排列中\end{CJK}, \begin{CJK}{UTF8}{mj}反序数等于\end{CJK} 2 \begin{CJK}{UTF8}{mj}的排列共有\end{CJK} \begin{CJK}{UTF8}{mj}个\end{CJK}.

  \item \begin{CJK}{UTF8}{mj}设\end{CJK} $A, B$ \begin{CJK}{UTF8}{mj}为\end{CJK} $n$ \begin{CJK}{UTF8}{mj}阶方阵\end{CJK}. \begin{CJK}{UTF8}{mj}若\end{CJK} $|A|=2,|B|=3,\left|A^{-1}+B\right|=6$, \begin{CJK}{UTF8}{mj}则\end{CJK} $\left|A+B^{-1}\right|=$

  \item \begin{CJK}{UTF8}{mj}设\end{CJK} $f(x)=x^{4}+x^{2}+a x+b, g(x)=x^{2}+x-2$. \begin{CJK}{UTF8}{mj}若\end{CJK} $(f(x), g(x))=g(x)$, \begin{CJK}{UTF8}{mj}则\end{CJK} $a=$ $b=$

  \item \begin{CJK}{UTF8}{mj}设\end{CJK} $A$ \begin{CJK}{UTF8}{mj}为三阶方阵\end{CJK}, $P=\left(\alpha_{1}, \alpha_{2}, \alpha_{3}\right)$ \begin{CJK}{UTF8}{mj}为三阶可逆阵\end{CJK}, \begin{CJK}{UTF8}{mj}并且\end{CJK} $P^{-1} A P=\left(\begin{array}{lll}2 & 0 & 0 \\ 0 & 2 & 0 \\ 0 & 0\end{array}\right)$ \begin{CJK}{UTF8}{mj}若\end{CJK} $Q=$ $\left(\alpha_{2}, \alpha_{3}, \alpha_{1}+\alpha_{2}\right)$, \begin{CJK}{UTF8}{mj}则\end{CJK} $Q^{-1} A Q=$

  \item \begin{CJK}{UTF8}{mj}设\end{CJK} $A=\left(\begin{array}{ccc}1 & 1 & 0 \\ 1 & k & 0 \\ 0 & 0 & k-2\end{array}\right)$ \begin{CJK}{UTF8}{mj}是三阶正定矩阵\end{CJK}, \begin{CJK}{UTF8}{mj}则\end{CJK} $k$ \begin{CJK}{UTF8}{mj}的取值范围是\end{CJK}

  \item \begin{CJK}{UTF8}{mj}设\end{CJK} $=\left(\begin{array}{ccc}1 & 1 & 0 \\ 0 & 1 & 0 \\ 0 & 0 & 1\end{array}\right)$ \begin{CJK}{UTF8}{mj}为复数域上三阶方阵\end{CJK}, \begin{CJK}{UTF8}{mj}则\end{CJK} $A$ \begin{CJK}{UTF8}{mj}的最小多项式为\end{CJK}

  \item \begin{CJK}{UTF8}{mj}二元实二次型\end{CJK} $f\left(x_{1}, x_{2}\right)=\left(x_{1}, x_{2}\right)\left(\begin{array}{ll}1 & 2 \\ 0 & 0\end{array}\right)\left(\begin{array}{l}x_{1} \\ x_{2}\end{array}\right)$ \begin{CJK}{UTF8}{mj}的秩\end{CJK} $=$

  \item \begin{CJK}{UTF8}{mj}设\end{CJK} $n$ \begin{CJK}{UTF8}{mj}元非齐次线性方程组\end{CJK} $A X=B$ \begin{CJK}{UTF8}{mj}无解\end{CJK}, \begin{CJK}{UTF8}{mj}其系数矩阵的秩为\end{CJK} 4 , \begin{CJK}{UTF8}{mj}则其增广矩阵的秩为\end{CJK}

  \item \begin{CJK}{UTF8}{mj}设矩阵\end{CJK} $A=\left(\alpha_{1}, \alpha_{2}, \alpha_{3}, \alpha_{4}\right)$, \begin{CJK}{UTF8}{mj}其中\end{CJK} $\alpha_{2}, \alpha_{3}, \alpha_{4}$ \begin{CJK}{UTF8}{mj}线性无关\end{CJK}, $\alpha_{1}=2 \alpha_{2}-\alpha_{3}$, \begin{CJK}{UTF8}{mj}向量\end{CJK} $\beta=\alpha_{1}+\alpha_{2}+\alpha_{3}+\alpha_{4}$, \begin{CJK}{UTF8}{mj}则非齐次线性方程组\end{CJK} $A X=\beta$ \begin{CJK}{UTF8}{mj}的通解为\end{CJK}

  \item \begin{CJK}{UTF8}{mj}设\end{CJK} $D$ \begin{CJK}{UTF8}{mj}为一个三阶行列式\end{CJK}, $D$ \begin{CJK}{UTF8}{mj}的元素为\end{CJK} 1 \begin{CJK}{UTF8}{mj}或\end{CJK} $-1$, \begin{CJK}{UTF8}{mj}则\end{CJK} $D$ \begin{CJK}{UTF8}{mj}的最大值为\end{CJK}

\end{enumerate}
\begin{CJK}{UTF8}{mj}二\end{CJK}. ( 20 \begin{CJK}{UTF8}{mj}分\end{CJK}) \begin{CJK}{UTF8}{mj}设\end{CJK} $A=\left(\begin{array}{lll}1 & 0 & 1 \\ 0 & 2 & 0 \\ 1 & 0 & 1\end{array}\right)$, \begin{CJK}{UTF8}{mj}且\end{CJK} $A B+E=A^{2}+B$, \begin{CJK}{UTF8}{mj}其中\end{CJK} $E$ \begin{CJK}{UTF8}{mj}为三阶单位矩阵\end{CJK}, \begin{CJK}{UTF8}{mj}求\end{CJK} $B$.

\begin{CJK}{UTF8}{mj}三\end{CJK} $(20$ \begin{CJK}{UTF8}{mj}分\end{CJK} $)$ \begin{CJK}{UTF8}{mj}设\end{CJK} $A$ \begin{CJK}{UTF8}{mj}为三阶实对称矩阵\end{CJK}, \begin{CJK}{UTF8}{mj}其特征值为\end{CJK} $\lambda_{1}=1, \lambda_{2}=-1, \lambda_{3}=0, \eta_{1}=\left(\begin{array}{c}1 \\ 2 \\ 2\end{array}\right)$ \begin{CJK}{UTF8}{mj}与\end{CJK} $\eta_{2}=\left(\begin{array}{c}2 \\ 1 \\ -2\end{array}\right)$ \begin{CJK}{UTF8}{mj}分别是\end{CJK} $A$ \begin{CJK}{UTF8}{mj}的属于特征值\end{CJK} $\lambda_{1}$ \begin{CJK}{UTF8}{mj}与\end{CJK} $\lambda_{2}$ \begin{CJK}{UTF8}{mj}的特征向量\end{CJK}. \begin{CJK}{UTF8}{mj}求矩阵\end{CJK} $A$.

\begin{CJK}{UTF8}{mj}四\end{CJK}. (20 \begin{CJK}{UTF8}{mj}分\end{CJK}) \begin{CJK}{UTF8}{mj}设\end{CJK} $P$ \begin{CJK}{UTF8}{mj}为数域\end{CJK}, $f(x), g(x) \in P[x], a, b, c, d \in P$, \begin{CJK}{UTF8}{mj}且\end{CJK} $a d-b c \neq 0$. \begin{CJK}{UTF8}{mj}证明\end{CJK}
$$
(a f(x)+b g(x), c f(x)+d g(x))=(f(x), g(x))
$$
\begin{CJK}{UTF8}{mj}五\end{CJK}. ( 20 \begin{CJK}{UTF8}{mj}分\end{CJK}) \begin{CJK}{UTF8}{mj}设\end{CJK} $A=\left(\begin{array}{ccc}0 & 2001 & 1 \\ 0 & 0 & 2011 \\ 0 & 0 & 0\end{array}\right)$, \begin{CJK}{UTF8}{mj}证明\end{CJK} $X^{2}=A$ \begin{CJK}{UTF8}{mj}无解\end{CJK}, \begin{CJK}{UTF8}{mj}这里\end{CJK} $X$ \begin{CJK}{UTF8}{mj}为三阶末知复矩阵\end{CJK}.

\begin{CJK}{UTF8}{mj}六\end{CJK}. (10 \begin{CJK}{UTF8}{mj}分\end{CJK}) \begin{CJK}{UTF8}{mj}设\end{CJK} $V$ \begin{CJK}{UTF8}{mj}是数域\end{CJK} $P$ \begin{CJK}{UTF8}{mj}上的\end{CJK} $n$ \begin{CJK}{UTF8}{mj}维线性空间\end{CJK}, $\tau$ \begin{CJK}{UTF8}{mj}是\end{CJK} $V$ \begin{CJK}{UTF8}{mj}的一个线性变换\end{CJK}, $\tau$ \begin{CJK}{UTF8}{mj}的特征多项式为\end{CJK} $f(\lambda)$. \begin{CJK}{UTF8}{mj}证明\end{CJK}: $f(\lambda)$ \begin{CJK}{UTF8}{mj}在\end{CJK} $P$ \begin{CJK}{UTF8}{mj}上不可约的充分必要条件是\end{CJK} $V$ \begin{CJK}{UTF8}{mj}无关于\end{CJK} $\tau$ \begin{CJK}{UTF8}{mj}的非平凡不变子空间\end{CJK}(\begin{CJK}{UTF8}{mj}通常称\end{CJK} $V$ \begin{CJK}{UTF8}{mj}的子空间\end{CJK} 0 \begin{CJK}{UTF8}{mj}和\end{CJK} $V$ \begin{CJK}{UTF8}{mj}为\end{CJK} $V$ \begin{CJK}{UTF8}{mj}的关于\end{CJK} $\tau$ \begin{CJK}{UTF8}{mj}的平凡不变子空间\end{CJK} $)$.

\section{4. 西南大学 2012 年研究生入学考试试题高等代数}
\begin{CJK}{UTF8}{mj}李扬\end{CJK}

\begin{CJK}{UTF8}{mj}微信公众号\end{CJK}: sxkyliyang

\begin{CJK}{UTF8}{mj}一\end{CJK}. \begin{CJK}{UTF8}{mj}填空\end{CJK}(\begin{CJK}{UTF8}{mj}每小题\end{CJK} 8 \begin{CJK}{UTF8}{mj}分\end{CJK}, \begin{CJK}{UTF8}{mj}共\end{CJK} 80 \begin{CJK}{UTF8}{mj}分\end{CJK})

\begin{enumerate}
  \item \begin{CJK}{UTF8}{mj}方程组\end{CJK} $\left\{\begin{array}{c}6 x^{4}-x^{3}-52 x^{2}+11 x+18=0 \\ 6 x^{3}-19 x^{2}+3 x+7=0\end{array}\right.$ \begin{CJK}{UTF8}{mj}在复数域的解是\end{CJK}

  \item \begin{CJK}{UTF8}{mj}行列式\end{CJK} $\left|\begin{array}{cccc}3 & 1 & 2 & 3 \\ 7 & -1 & 0 & 1 \\ -1 & 0 & 2 & 3 \\ 2 & 1 & -1 & -2\end{array}\right|$ \begin{CJK}{UTF8}{mj}的第一列元素的代数余子式的和是\end{CJK}

  \item $\left(\begin{array}{lll}\lambda & 1 & \\ & \lambda & 1 \\ & & \lambda\end{array}\right)^{n}=$

  \item $\left(\begin{array}{lll}1 & 2 & 3 \\ 2 & 2 & 1 \\ 3 & 4 & 3\end{array}\right)^{-1}=$

  \item \begin{CJK}{UTF8}{mj}设\end{CJK} $A$ \begin{CJK}{UTF8}{mj}为\end{CJK} 3 \begin{CJK}{UTF8}{mj}阶方阵\end{CJK}, $X$ \begin{CJK}{UTF8}{mj}为\end{CJK} 3 \begin{CJK}{UTF8}{mj}维列向量\end{CJK}, \begin{CJK}{UTF8}{mj}满足\end{CJK} $A^{3} X+A^{2} X+2 A X-3 X=0$, \begin{CJK}{UTF8}{mj}若向量组\end{CJK} $X, A X, A^{2} X$ \begin{CJK}{UTF8}{mj}线性\end{CJK} \begin{CJK}{UTF8}{mj}无关\end{CJK}, \begin{CJK}{UTF8}{mj}则\end{CJK} $|A|=$

  \item \begin{CJK}{UTF8}{mj}若\end{CJK} $P$ \begin{CJK}{UTF8}{mj}为数域\end{CJK}, $f$ \begin{CJK}{UTF8}{mj}为线性空间\end{CJK} $P^{3}$ \begin{CJK}{UTF8}{mj}的线性变换\end{CJK}, \begin{CJK}{UTF8}{mj}使\end{CJK}

\end{enumerate}
$$
f(x, y, z)=(x+y-z, x+y+z, x+y-2 z) .
$$
\begin{CJK}{UTF8}{mj}则\end{CJK} $f$ \begin{CJK}{UTF8}{mj}的象空间\end{CJK} $\operatorname{Im} f$ \begin{CJK}{UTF8}{mj}的维数是\end{CJK}

\begin{enumerate}
  \setcounter{enumi}{7}
  \item \begin{CJK}{UTF8}{mj}设\end{CJK} $A=\left(\begin{array}{ccc}4 & -1 & 2 \\ -9 & 4 & -6 \\ -9 & 3 & -5\end{array}\right)$, \begin{CJK}{UTF8}{mj}则\end{CJK} $A$ \begin{CJK}{UTF8}{mj}的\end{CJK} Jordan \begin{CJK}{UTF8}{mj}标准型为\end{CJK}

  \item \begin{CJK}{UTF8}{mj}设\end{CJK} $V$ \begin{CJK}{UTF8}{mj}为\end{CJK} $n$ \begin{CJK}{UTF8}{mj}维欧几里得空间\end{CJK}(\begin{CJK}{UTF8}{mj}欧式空间\end{CJK}), $\alpha$ \begin{CJK}{UTF8}{mj}为\end{CJK} $V$ \begin{CJK}{UTF8}{mj}中非零向量\end{CJK}, $\sigma_{\alpha}$ \begin{CJK}{UTF8}{mj}是关于\end{CJK} $\alpha$ \begin{CJK}{UTF8}{mj}的反射变换\end{CJK}, \begin{CJK}{UTF8}{mj}对\end{CJK} $\beta \in V$, \begin{CJK}{UTF8}{mj}有\end{CJK} $\sigma_{\alpha}(\beta)=\square$

  \item \begin{CJK}{UTF8}{mj}三元实二次型\end{CJK} $f\left(x_{1}, x_{2}, x_{3}\right)=2 x_{1} x_{2}+2 x_{1} x_{3}-6 x_{2} x_{3}$ \begin{CJK}{UTF8}{mj}的正惯性指数为\end{CJK}

  \item \begin{CJK}{UTF8}{mj}设\end{CJK} $A=\left(a_{1}, a_{2}, \cdots, a_{n}\right)$, \begin{CJK}{UTF8}{mj}其中\end{CJK} $a_{1}, a_{2}, \cdots, a_{n}$ \begin{CJK}{UTF8}{mj}为实数\end{CJK}, \begin{CJK}{UTF8}{mj}且不全为零\end{CJK}, $B=A^{T} A$, \begin{CJK}{UTF8}{mj}这里\end{CJK} $A^{T}$ \begin{CJK}{UTF8}{mj}是\end{CJK} $A$ \begin{CJK}{UTF8}{mj}的转置\end{CJK}. \begin{CJK}{UTF8}{mj}则\end{CJK} $B$ \begin{CJK}{UTF8}{mj}的全部特征值为\end{CJK}

\end{enumerate}
\begin{CJK}{UTF8}{mj}二\end{CJK}. (20 \begin{CJK}{UTF8}{mj}分\end{CJK}) \begin{CJK}{UTF8}{mj}设\end{CJK} $\alpha$ \begin{CJK}{UTF8}{mj}为一复数\end{CJK}, \begin{CJK}{UTF8}{mj}且是\end{CJK} $\mathbb{Q}[x]$ \begin{CJK}{UTF8}{mj}中某个非零多项式的根\end{CJK}, \begin{CJK}{UTF8}{mj}令\end{CJK}
$$
J=\{f(x) \in \mathbb{Q}[x] \mid f(\alpha)=0\}
$$
\begin{CJK}{UTF8}{mj}证明\end{CJK}:

(1) \begin{CJK}{UTF8}{mj}在\end{CJK} $J$ \begin{CJK}{UTF8}{mj}中存在唯一的最高项系数为\end{CJK} 1 \begin{CJK}{UTF8}{mj}的多项式\end{CJK} $p(x)$, \begin{CJK}{UTF8}{mj}使\end{CJK} $p(x)$ \begin{CJK}{UTF8}{mj}整除\end{CJK} $J$ \begin{CJK}{UTF8}{mj}中每一多项式\end{CJK} $f(x)$;

(2) $p(x)$ \begin{CJK}{UTF8}{mj}在\end{CJK} $\mathbb{Q}$ \begin{CJK}{UTF8}{mj}上不可约\end{CJK}.

\begin{CJK}{UTF8}{mj}三\end{CJK}. ( 20 \begin{CJK}{UTF8}{mj}分\end{CJK}) \begin{CJK}{UTF8}{mj}设\end{CJK} $V$ \begin{CJK}{UTF8}{mj}为\end{CJK} $n$ \begin{CJK}{UTF8}{mj}维欧几里得空间\end{CJK}, $\sigma$ \begin{CJK}{UTF8}{mj}为\end{CJK} $V$ \begin{CJK}{UTF8}{mj}的一个正交变换\end{CJK}, \begin{CJK}{UTF8}{mj}令\end{CJK}
$$
V_{1}=\{\alpha \in V \mid \sigma(\alpha)=\alpha\}, V_{2}=\{\alpha-\sigma(\alpha) \mid \alpha \in V\}
$$
(1) \begin{CJK}{UTF8}{mj}证明\end{CJK}: $V_{1}, V_{2}$ \begin{CJK}{UTF8}{mj}是\end{CJK} $V$ \begin{CJK}{UTF8}{mj}的子空间\end{CJK};

(2) \begin{CJK}{UTF8}{mj}证明\end{CJK}: $V=V_{1} \oplus V_{2}$. \begin{CJK}{UTF8}{mj}四\end{CJK}. (20 \begin{CJK}{UTF8}{mj}分\end{CJK}) \begin{CJK}{UTF8}{mj}设\end{CJK} $A, B$ \begin{CJK}{UTF8}{mj}为\end{CJK} $n$ \begin{CJK}{UTF8}{mj}阶实矩阵\end{CJK}, $A$ \begin{CJK}{UTF8}{mj}有\end{CJK} $n$ \begin{CJK}{UTF8}{mj}个互不相同的特征值\end{CJK}, \begin{CJK}{UTF8}{mj}且\end{CJK} $A B=B A$. \begin{CJK}{UTF8}{mj}证明存在非零实系数多项\end{CJK} \begin{CJK}{UTF8}{mj}式\end{CJK} $f(x)$, \begin{CJK}{UTF8}{mj}使\end{CJK} $B=f(A)$.

\begin{CJK}{UTF8}{mj}五\end{CJK}. (10 \begin{CJK}{UTF8}{mj}分\end{CJK}) \begin{CJK}{UTF8}{mj}设\end{CJK} $A, B, C$ \begin{CJK}{UTF8}{mj}为\end{CJK} $n$ \begin{CJK}{UTF8}{mj}阶方阵\end{CJK}, \begin{CJK}{UTF8}{mj}满足条件\end{CJK} $B C=0, r(A)<r(C)$. \begin{CJK}{UTF8}{mj}证明\end{CJK}: \begin{CJK}{UTF8}{mj}存在非零的\end{CJK} $n$ \begin{CJK}{UTF8}{mj}维列向量\end{CJK} $X$, \begin{CJK}{UTF8}{mj}使\end{CJK} $A X=B X$. \begin{CJK}{UTF8}{mj}其中\end{CJK} $r(A)$ \begin{CJK}{UTF8}{mj}表示矩阵\end{CJK} $A$ \begin{CJK}{UTF8}{mj}的秩\end{CJK}.

\section{5. 西南大学 2013 年研究生入学考试试题高等代数}
\begin{CJK}{UTF8}{mj}李扬\end{CJK}

\begin{CJK}{UTF8}{mj}微信公众号\end{CJK}: sxkyliyang

\begin{CJK}{UTF8}{mj}一\end{CJK}. ( 20 \begin{CJK}{UTF8}{mj}分\end{CJK}) \begin{CJK}{UTF8}{mj}设\end{CJK} $a_{1}, a_{2}, \cdots, a_{n}$ \begin{CJK}{UTF8}{mj}是数域\end{CJK} $P$ \begin{CJK}{UTF8}{mj}中\end{CJK} $n$ \begin{CJK}{UTF8}{mj}个互不相同的数\end{CJK}, $b_{1}, b_{2}, \cdots, b_{n}$ \begin{CJK}{UTF8}{mj}是数域\end{CJK} $P$ \begin{CJK}{UTF8}{mj}中任意\end{CJK} $n$ \begin{CJK}{UTF8}{mj}个数\end{CJK}. \begin{CJK}{UTF8}{mj}证明\end{CJK}:

(1) \begin{CJK}{UTF8}{mj}存在数域\end{CJK} $P$ \begin{CJK}{UTF8}{mj}上唯一的多项式\end{CJK} $f(x)=c_{n-1} x^{n-1}+\cdots+c_{1} x+c_{0}$ \begin{CJK}{UTF8}{mj}使得\end{CJK} $f\left(a_{i}\right)=b_{i}, i=1,2, \cdots, n$.

(2) \begin{CJK}{UTF8}{mj}取\end{CJK} $a_{1}, a_{2}, \cdots, a_{n}$ \begin{CJK}{UTF8}{mj}分别为\end{CJK} $1,2, \cdots, n, b_{1}=b_{2}=\cdots=b_{n}=1$, \begin{CJK}{UTF8}{mj}求出\end{CJK} (1) \begin{CJK}{UTF8}{mj}中的多项式\end{CJK} $f(x)$.

\begin{CJK}{UTF8}{mj}二\end{CJK}. ( 30 \begin{CJK}{UTF8}{mj}分\end{CJK}) \begin{CJK}{UTF8}{mj}已知矩阵\end{CJK} $A=\left(\begin{array}{ll}2 & 2 \\ 2 & a\end{array}\right), B=\left(\begin{array}{cc}4 & b \\ 3 & 1\end{array}\right)$, \begin{CJK}{UTF8}{mj}证明\end{CJK}:

(1) $A$ \begin{CJK}{UTF8}{mj}相似于\end{CJK} $B$ \begin{CJK}{UTF8}{mj}的充要条件是\end{CJK} $a=3, b=\frac{2}{3}$.

(2) $A$ \begin{CJK}{UTF8}{mj}合同于\end{CJK} $B$ \begin{CJK}{UTF8}{mj}的充要条件是\end{CJK} $a<2, b=3$.

\includegraphics[max width=\textwidth]{2022_04_18_3416d289b173eb9de8c1g-029}\\
$B=\left(\begin{array}{ccc}0 & 1 & 1 \\ 1 & 0 & 1 \\ 1 & 1 & 0\end{array}\right)$, \begin{CJK}{UTF8}{mj}求\end{CJK} $X$.

\begin{CJK}{UTF8}{mj}四\end{CJK}. ( 20 \begin{CJK}{UTF8}{mj}分\end{CJK}) \begin{CJK}{UTF8}{mj}设\end{CJK} $V$ \begin{CJK}{UTF8}{mj}是\end{CJK} $n$ \begin{CJK}{UTF8}{mj}维欧式空间\end{CJK}, $V_{1}, V_{2}$ \begin{CJK}{UTF8}{mj}是\end{CJK} $V$ \begin{CJK}{UTF8}{mj}的两个\end{CJK} $m(0<m<n)$ \begin{CJK}{UTF8}{mj}维子空间\end{CJK}, \begin{CJK}{UTF8}{mj}证明\end{CJK}: \begin{CJK}{UTF8}{mj}存在正交变换\end{CJK} $\tau$, \begin{CJK}{UTF8}{mj}使得\end{CJK} $\tau\left(V_{1}\right)=V_{2}$.

\begin{CJK}{UTF8}{mj}五\end{CJK}. ( 20 \begin{CJK}{UTF8}{mj}分\end{CJK}) \begin{CJK}{UTF8}{mj}设\end{CJK} $\alpha_{1}, \alpha_{2}, \cdots, \alpha_{r}$ \begin{CJK}{UTF8}{mj}与\end{CJK} $\beta_{1}, \beta_{2}, \cdots, \beta_{s}$ \begin{CJK}{UTF8}{mj}是数域\end{CJK} $P$ \begin{CJK}{UTF8}{mj}上线性空间\end{CJK} $V$ \begin{CJK}{UTF8}{mj}的两个向量组\end{CJK}. \begin{CJK}{UTF8}{mj}证明\end{CJK}: \begin{CJK}{UTF8}{mj}如果向量组\end{CJK} $\alpha_{1}, \alpha_{2}, \cdots, \alpha_{r}$ \begin{CJK}{UTF8}{mj}可由向量组\end{CJK} $\beta_{1}, \beta_{2}, \cdots, \beta_{s}$ \begin{CJK}{UTF8}{mj}线性表示\end{CJK}, \begin{CJK}{UTF8}{mj}并且\end{CJK} $r>s$, \begin{CJK}{UTF8}{mj}那么在向量组\end{CJK} $\alpha_{1}, \alpha_{2}, \cdots, \alpha_{r}$ \begin{CJK}{UTF8}{mj}中至少\end{CJK} \begin{CJK}{UTF8}{mj}有\end{CJK} $r-s$ \begin{CJK}{UTF8}{mj}个向量可以由其余向量线性表示\end{CJK}.

\begin{CJK}{UTF8}{mj}六\end{CJK}. ( 20 \begin{CJK}{UTF8}{mj}分\end{CJK}) \begin{CJK}{UTF8}{mj}设\end{CJK} $V=P^{n \times n}$ \begin{CJK}{UTF8}{mj}是数域\end{CJK} $P$ \begin{CJK}{UTF8}{mj}上的\end{CJK} $n \times n$ \begin{CJK}{UTF8}{mj}阶矩阵关于矩阵加法与矩阵的数乘构成的线性空间\end{CJK}.

(1) \begin{CJK}{UTF8}{mj}写出\end{CJK} $V$ \begin{CJK}{UTF8}{mj}的一个基及\end{CJK} $V$ \begin{CJK}{UTF8}{mj}的维数\end{CJK};

(2) \begin{CJK}{UTF8}{mj}令\end{CJK} $A=\left(\begin{array}{ccccc}0 & 0 & 0 & \cdots & a_{1} \\ 1 & 0 & 0 & \cdots & a_{2} \\ 0 & 1 & 0 & \cdots & a_{3} \\ \cdots & \cdots & \cdots & & \cdots \\ 0 & 0 & \cdots & 1 & a_{n}\end{array}\right) \in V$, \begin{CJK}{UTF8}{mj}求由\end{CJK} $E, A, A^{2}, \cdots, A^{m}$ \begin{CJK}{UTF8}{mj}生成的\end{CJK} $V$ \begin{CJK}{UTF8}{mj}的子空间\end{CJK} $W$ \begin{CJK}{UTF8}{mj}的维数及\end{CJK} \begin{CJK}{UTF8}{mj}一个基\end{CJK}, \begin{CJK}{UTF8}{mj}其中\end{CJK} $m$ \begin{CJK}{UTF8}{mj}是正整数\end{CJK}.

\begin{CJK}{UTF8}{mj}七\end{CJK}. ( 10 \begin{CJK}{UTF8}{mj}分\end{CJK}) \begin{CJK}{UTF8}{mj}设\end{CJK} $A, B$ \begin{CJK}{UTF8}{mj}为\end{CJK} $n$ \begin{CJK}{UTF8}{mj}阶实矩阵\end{CJK}, $A, B$ \begin{CJK}{UTF8}{mj}的\end{CJK} $n$ \begin{CJK}{UTF8}{mj}个特征值均大于\end{CJK} 0 . \begin{CJK}{UTF8}{mj}证明\end{CJK}: \begin{CJK}{UTF8}{mj}若\end{CJK} $A^{2}=B^{2}$, \begin{CJK}{UTF8}{mj}则\end{CJK} $A=B$.

\section{6. 西南大学 2014 年研究生入学考试试题高等代数}
\begin{CJK}{UTF8}{mj}李扬\end{CJK}

\begin{CJK}{UTF8}{mj}微信公众号\end{CJK}: sxkyliyang

\begin{CJK}{UTF8}{mj}一\end{CJK}. \begin{CJK}{UTF8}{mj}填空题\end{CJK}(\begin{CJK}{UTF8}{mj}每小题\end{CJK} 10 \begin{CJK}{UTF8}{mj}分\end{CJK}, \begin{CJK}{UTF8}{mj}共\end{CJK} 60 \begin{CJK}{UTF8}{mj}分\end{CJK})

\begin{enumerate}
  \item \begin{CJK}{UTF8}{mj}设\end{CJK} $f(x)=x^{3}-3 x^{2}-x-1, g(x)=3 x^{2}-2 x+1$, \begin{CJK}{UTF8}{mj}则\end{CJK} $g(x)$ \begin{CJK}{UTF8}{mj}除\end{CJK} $f(x)$ \begin{CJK}{UTF8}{mj}的余式为\end{CJK}

  \item \begin{CJK}{UTF8}{mj}四阶行列式\end{CJK} $\left|\begin{array}{llll}1 & 2 & 3 & 4 \\ 2 & 3 & 4 & 1 \\ 3 & 4 & 1 & 2 \\ 4 & 1 & 2 & 3\end{array}\right|=$

  \item $\left(\begin{array}{ccc}5 & 0 & 0 \\ 0 & 4 & 11 \\ 0 & 1 & 3\end{array}\right)^{-1}=$

  \item \begin{CJK}{UTF8}{mj}在线性空间\end{CJK} $P^{4}$ \begin{CJK}{UTF8}{mj}中\end{CJK}, \begin{CJK}{UTF8}{mj}向量\end{CJK} $\beta=(1,2,1,1)$ \begin{CJK}{UTF8}{mj}在基\end{CJK} $\alpha_{1}=(1,1,1,1), \alpha_{2}=(1,1,-1,-1), \alpha_{3}=(1,-1,1,-1), \alpha_{4}=$ $(1,-1,-1,1)$ \begin{CJK}{UTF8}{mj}下的坐标是\end{CJK}

  \item \begin{CJK}{UTF8}{mj}设\end{CJK} $P$ \begin{CJK}{UTF8}{mj}为数域\end{CJK}, $f$ \begin{CJK}{UTF8}{mj}为线性空间\end{CJK} $P^{3}$ \begin{CJK}{UTF8}{mj}的线性变换\end{CJK}, \begin{CJK}{UTF8}{mj}使\end{CJK}

\end{enumerate}
$$
f(x, y, z)=(x+y-z, x+y+z, x+y-2 z) .
$$
\begin{CJK}{UTF8}{mj}则\end{CJK} $f$ \begin{CJK}{UTF8}{mj}的核空间\end{CJK} ker $f$ \begin{CJK}{UTF8}{mj}的维数是\end{CJK}

\begin{enumerate}
  \setcounter{enumi}{6}
  \item \begin{CJK}{UTF8}{mj}用正交线性替换把实二次型\end{CJK} $f\left(x_{1}, x_{2}, x_{3}\right)=2 x_{1}^{2}-4 x_{1} x_{2}+x_{2}^{2}-4 x_{2} x_{3}$ \begin{CJK}{UTF8}{mj}化为标准形\end{CJK}, \begin{CJK}{UTF8}{mj}其标准形\end{CJK} \begin{CJK}{UTF8}{mj}为\end{CJK}
\end{enumerate}
\begin{CJK}{UTF8}{mj}二\end{CJK}. ( 20 \begin{CJK}{UTF8}{mj}分\end{CJK}) \begin{CJK}{UTF8}{mj}设\end{CJK} $A=\left(\begin{array}{ccc}2 & 1 & 0 \\ -1 & 0 & 0 \\ -2 & -1 & 2\end{array}\right) \in P^{3 \times 3}, E$ \begin{CJK}{UTF8}{mj}为\end{CJK} $n$ \begin{CJK}{UTF8}{mj}阶单位矩阵\end{CJK}.

(1) \begin{CJK}{UTF8}{mj}求\end{CJK} $A$ \begin{CJK}{UTF8}{mj}的最小多项式\end{CJK};

(2) \begin{CJK}{UTF8}{mj}证明\end{CJK} $E, A, A^{2}$ \begin{CJK}{UTF8}{mj}线性无关\end{CJK};

(3) \begin{CJK}{UTF8}{mj}把\end{CJK} $A^{3}$ \begin{CJK}{UTF8}{mj}表成\end{CJK} $E, A, A^{2}$ \begin{CJK}{UTF8}{mj}的线性组合\end{CJK}.

\begin{CJK}{UTF8}{mj}三\end{CJK}. (20 \begin{CJK}{UTF8}{mj}分\end{CJK}) \begin{CJK}{UTF8}{mj}设\end{CJK} $A, B, C$ \begin{CJK}{UTF8}{mj}为数域\end{CJK} $P$ \begin{CJK}{UTF8}{mj}上\end{CJK} $n$ \begin{CJK}{UTF8}{mj}阶方阵\end{CJK}, \begin{CJK}{UTF8}{mj}证明\end{CJK}: \begin{CJK}{UTF8}{mj}如果\end{CJK} $A B=0, A C=0, r(A)=n-1, B \neq 0$. \begin{CJK}{UTF8}{mj}那么\end{CJK} $C$ \begin{CJK}{UTF8}{mj}的列向量均可被\end{CJK} $B$ \begin{CJK}{UTF8}{mj}的列向量线性表示\end{CJK}, \begin{CJK}{UTF8}{mj}其中\end{CJK} $r(A)$ \begin{CJK}{UTF8}{mj}为\end{CJK} $A$ \begin{CJK}{UTF8}{mj}的秩\end{CJK}.

\begin{CJK}{UTF8}{mj}四\end{CJK}. (20 \begin{CJK}{UTF8}{mj}分\end{CJK}) \begin{CJK}{UTF8}{mj}设\end{CJK} $W$ \begin{CJK}{UTF8}{mj}是数域\end{CJK} $P$ \begin{CJK}{UTF8}{mj}上\end{CJK} $n$ \begin{CJK}{UTF8}{mj}维线性空间\end{CJK} $V$ \begin{CJK}{UTF8}{mj}的子空间\end{CJK}. \begin{CJK}{UTF8}{mj}证明\end{CJK}: \begin{CJK}{UTF8}{mj}存在\end{CJK} $V$ \begin{CJK}{UTF8}{mj}的线性变换\end{CJK} $\sigma, \tau$ \begin{CJK}{UTF8}{mj}使得\end{CJK} $\operatorname{Im}(\sigma)=W$, $T \operatorname{ker}(\tau)=W$.

\begin{CJK}{UTF8}{mj}五\end{CJK}. ( 20 \begin{CJK}{UTF8}{mj}分\end{CJK}) \begin{CJK}{UTF8}{mj}设\end{CJK} $A$ \begin{CJK}{UTF8}{mj}为\end{CJK} $n$ \begin{CJK}{UTF8}{mj}阶实方阵\end{CJK}, \begin{CJK}{UTF8}{mj}它的每个元素都是\end{CJK} $a$, \begin{CJK}{UTF8}{mj}且\end{CJK} $a$ \begin{CJK}{UTF8}{mj}为非零实数\end{CJK}. \begin{CJK}{UTF8}{mj}试证\end{CJK}:

(1) $n$ \begin{CJK}{UTF8}{mj}阶矩阵\end{CJK} $A+n a E$ \begin{CJK}{UTF8}{mj}可逆\end{CJK};

(2) \begin{CJK}{UTF8}{mj}存在一个\end{CJK} $n-1$ \begin{CJK}{UTF8}{mj}次实系数多项式\end{CJK} $f(x)$, \begin{CJK}{UTF8}{mj}使\end{CJK} $f(A)=(A+n a E)^{-1}$. \begin{CJK}{UTF8}{mj}其中\end{CJK} $E$ \begin{CJK}{UTF8}{mj}为\end{CJK} $n$ \begin{CJK}{UTF8}{mj}阶单位矩阵\end{CJK}.

\begin{CJK}{UTF8}{mj}六\end{CJK}. ( 10 \begin{CJK}{UTF8}{mj}分\end{CJK}) \begin{CJK}{UTF8}{mj}设复数域上\end{CJK} $n$ \begin{CJK}{UTF8}{mj}阶矩阵\end{CJK} $A$ \begin{CJK}{UTF8}{mj}的特征多项式为\end{CJK} $f(\lambda)=(\lambda-1)^{n}$. \begin{CJK}{UTF8}{mj}证明\end{CJK}: \begin{CJK}{UTF8}{mj}对任何正整数\end{CJK} $k, s$ \begin{CJK}{UTF8}{mj}有\end{CJK} $A^{k}$ \begin{CJK}{UTF8}{mj}与\end{CJK} $A^{s}$ \begin{CJK}{UTF8}{mj}相似\end{CJK}.

\section{7. 西南大学 2015 年研究生入学考试试题高等代数}
\begin{CJK}{UTF8}{mj}李扬\end{CJK}

\begin{CJK}{UTF8}{mj}微信公众号\end{CJK}: sxkyliyang

\begin{CJK}{UTF8}{mj}一\end{CJK}. \begin{CJK}{UTF8}{mj}填空\end{CJK}.

\begin{enumerate}
  \item \begin{CJK}{UTF8}{mj}排列\end{CJK} 217986354 \begin{CJK}{UTF8}{mj}的逆序数是\end{CJK}

  \item \begin{CJK}{UTF8}{mj}设\end{CJK} $n$ \begin{CJK}{UTF8}{mj}阶方阵\end{CJK} $A$ \begin{CJK}{UTF8}{mj}的特征值为\end{CJK} $2,4, \cdots, 2 n$, \begin{CJK}{UTF8}{mj}则行列式\end{CJK} $|3 E-A|=$ \begin{CJK}{UTF8}{mj}其中\end{CJK} $E$ \begin{CJK}{UTF8}{mj}为\end{CJK} $n$ \begin{CJK}{UTF8}{mj}级单位阵\end{CJK}.

  \item \begin{CJK}{UTF8}{mj}设\end{CJK} $A=\left(\begin{array}{cccc}1 & -a & 0 & 0 \\ 0 & 1 & -a & 0 \\ 0 & 0 & 1 & -a \\ 0 & 0 & 0 & 1\end{array}\right)$, \begin{CJK}{UTF8}{mj}则\end{CJK} $A^{-1}=$

  \item \begin{CJK}{UTF8}{mj}设\end{CJK} $A=\left(\begin{array}{ccc}1 & 2 & -2 \\ 4 & a & 3 \\ 3 & -1 & 1\end{array}\right), B$ \begin{CJK}{UTF8}{mj}为三阶非零方阵\end{CJK}, \begin{CJK}{UTF8}{mj}且\end{CJK} $A B=0$, \begin{CJK}{UTF8}{mj}则\end{CJK} $a=$

  \item \begin{CJK}{UTF8}{mj}设\end{CJK} $A=\left(\begin{array}{cc}A_{1} & 0 \\ 0 & A_{2}\end{array}\right)$ \begin{CJK}{UTF8}{mj}为准对角阵\end{CJK}, $A_{1}$ \begin{CJK}{UTF8}{mj}的特征多项式为\end{CJK} $(\lambda-1)(\lambda-2), A_{2}$ \begin{CJK}{UTF8}{mj}的特征多项式为\end{CJK} $(\lambda-2)(\lambda-3)$, \begin{CJK}{UTF8}{mj}则\end{CJK} $A$ \begin{CJK}{UTF8}{mj}的特征多项式为\end{CJK}

  \item \begin{CJK}{UTF8}{mj}设\end{CJK} $\mathscr{A}$ \begin{CJK}{UTF8}{mj}为\end{CJK} $P^{3}$ \begin{CJK}{UTF8}{mj}上的线性变换\end{CJK}, $\alpha=(1,1,1), \beta=(1,2,1)$, \begin{CJK}{UTF8}{mj}已知\end{CJK} $\mathscr{A}(\alpha)=(2,-1,1), \mathscr{A}(\beta)=(0,1,3)$, \begin{CJK}{UTF8}{mj}则\end{CJK} $\mathscr{A}(2 \alpha-\beta)=$

  \item \begin{CJK}{UTF8}{mj}设\end{CJK} $W=\left\{\left(\alpha_{1}, \alpha_{2}, \alpha_{3}\right) \mid \alpha_{1}+2 \alpha_{2}+3 \alpha_{3}=0\right\}$ \begin{CJK}{UTF8}{mj}是\end{CJK} $\mathbb{R}^{3}$ \begin{CJK}{UTF8}{mj}的子空间\end{CJK}, \begin{CJK}{UTF8}{mj}则\end{CJK} $\operatorname{dim} W=$

  \item \begin{CJK}{UTF8}{mj}给定\end{CJK} $P^{3}$ \begin{CJK}{UTF8}{mj}上的线性变换\end{CJK} $\mathscr{A}$ \begin{CJK}{UTF8}{mj}如下\end{CJK}:

\end{enumerate}
\begin{CJK}{UTF8}{mj}则\end{CJK} $\operatorname{ker} \mathscr{A}=$

\includegraphics[max width=\textwidth]{2022_04_18_3416d289b173eb9de8c1g-031}

\begin{enumerate}
  \setcounter{enumi}{9}
  \item \begin{CJK}{UTF8}{mj}实\end{CJK} $n$ \begin{CJK}{UTF8}{mj}元二次型\end{CJK} $f=\sum_{i=1}^{n} x_{i}^{2}+2 \sum_{i<j}^{n} x_{i} x_{j}$ \begin{CJK}{UTF8}{mj}的正惯性指数为\end{CJK}

  \item \begin{CJK}{UTF8}{mj}若复数域上的多项式\end{CJK} $f(x)=x^{3}-3 x^{2}+x-t$ \begin{CJK}{UTF8}{mj}有重根\end{CJK}, \begin{CJK}{UTF8}{mj}则\end{CJK} $t=$

\end{enumerate}
\begin{CJK}{UTF8}{mj}二\end{CJK}. \begin{CJK}{UTF8}{mj}已知实二次型\end{CJK} $f\left(x_{1}, x_{2}, x_{3}\right)=2 x_{1}^{2}+3 x_{2}^{2}+3 x_{3}^{2}+2 a x_{2} x_{3}(a>0)$ \begin{CJK}{UTF8}{mj}通过正交线性替换化成标准形\end{CJK} $f=y_{1}^{2}+2 y_{2}^{2}+5 y_{3}^{2}$, \begin{CJK}{UTF8}{mj}求参数\end{CJK} $a$ \begin{CJK}{UTF8}{mj}的值及所用的正交线性替换\end{CJK}.

\begin{CJK}{UTF8}{mj}设\end{CJK} $\alpha_{1}, \cdots, \alpha_{n}$ \begin{CJK}{UTF8}{mj}是\end{CJK} $n$ \begin{CJK}{UTF8}{mj}维欧氏空间的一个基\end{CJK}, \begin{CJK}{UTF8}{mj}证明\end{CJK}: \begin{CJK}{UTF8}{mj}对于任意\end{CJK} $n$ \begin{CJK}{UTF8}{mj}个实数\end{CJK} $b_{1}, \cdots, b_{n}$, \begin{CJK}{UTF8}{mj}恰有一个向量\end{CJK} $\alpha \in V$, \begin{CJK}{UTF8}{mj}使\end{CJK} \begin{CJK}{UTF8}{mj}得\end{CJK} $\left(\alpha, \alpha_{i}\right)=b_{i}, i=1, \cdots, n$.

\begin{CJK}{UTF8}{mj}四\end{CJK}. \begin{CJK}{UTF8}{mj}设\end{CJK} $A$ \begin{CJK}{UTF8}{mj}为\end{CJK} $n$ \begin{CJK}{UTF8}{mj}阶实对称矩阵\end{CJK}, \begin{CJK}{UTF8}{mj}其特征值为\end{CJK} $\lambda_{1}, \cdots, \lambda_{n}$. \begin{CJK}{UTF8}{mj}证明\end{CJK}:\begin{CJK}{UTF8}{mj}存在\end{CJK} $n$ \begin{CJK}{UTF8}{mj}级实矩阵\end{CJK} $P_{1}, \cdots, P_{n}$, \begin{CJK}{UTF8}{mj}使\end{CJK} $A=\lambda_{1} P_{1}+$ $\lambda_{2} P_{2}+\cdots+\lambda_{n} P_{n}$, \begin{CJK}{UTF8}{mj}且\end{CJK} $r\left(P_{i}\right)=1, i=1,2, \cdots, n$, \begin{CJK}{UTF8}{mj}其中\end{CJK} $r(A)$ \begin{CJK}{UTF8}{mj}表示矩阵\end{CJK} $A$ \begin{CJK}{UTF8}{mj}的秩\end{CJK}.

\begin{CJK}{UTF8}{mj}五\end{CJK}. \begin{CJK}{UTF8}{mj}设\end{CJK} $f(x) \in \mathbb{Z}[x], a, b, c, d$ \begin{CJK}{UTF8}{mj}为四个不同的整数\end{CJK}. \begin{CJK}{UTF8}{mj}证明\end{CJK}: \begin{CJK}{UTF8}{mj}若\end{CJK} $f(a)=f(b)=f(c)=f(d)=0$, \begin{CJK}{UTF8}{mj}则\end{CJK} $f(x)+1$ \begin{CJK}{UTF8}{mj}无整\end{CJK} \begin{CJK}{UTF8}{mj}数根\end{CJK}.

\begin{CJK}{UTF8}{mj}六\end{CJK}. \begin{CJK}{UTF8}{mj}设\end{CJK} $V$ \begin{CJK}{UTF8}{mj}是数域\end{CJK} $P$ \begin{CJK}{UTF8}{mj}上的\end{CJK} $n$ \begin{CJK}{UTF8}{mj}维线性空间\end{CJK}, $\mathscr{A}$ \begin{CJK}{UTF8}{mj}为\end{CJK} $V$ \begin{CJK}{UTF8}{mj}上的线性变换\end{CJK}. \begin{CJK}{UTF8}{mj}证明\end{CJK}: \begin{CJK}{UTF8}{mj}如果\end{CJK} $\mathscr{A}$ \begin{CJK}{UTF8}{mj}有\end{CJK} $n$ \begin{CJK}{UTF8}{mj}个互异的特征值\end{CJK}, \begin{CJK}{UTF8}{mj}那么\end{CJK} \begin{CJK}{UTF8}{mj}与\end{CJK} $\mathscr{A}$ \begin{CJK}{UTF8}{mj}可交换的\end{CJK} $V$ \begin{CJK}{UTF8}{mj}的线性变换都是\end{CJK} $\mathscr{E}, \mathscr{A}, \mathscr{A}^{2}, \cdots, \mathscr{A}^{n-1}$ \begin{CJK}{UTF8}{mj}的线性组合\end{CJK}.

\section{8. 西南大学 2016 年研究生入学考试试题高等代数}
\begin{CJK}{UTF8}{mj}李扬\end{CJK}

\begin{CJK}{UTF8}{mj}微信公众号\end{CJK}: sxkyliyang

\section{一. 填空题}
\begin{enumerate}
  \item \begin{CJK}{UTF8}{mj}若\end{CJK} $n$ \begin{CJK}{UTF8}{mj}级排列\end{CJK} $X_{1}, X_{2}, \cdots, X_{n-1}, X_{n}$ \begin{CJK}{UTF8}{mj}的逆序数为\end{CJK} $k$, \begin{CJK}{UTF8}{mj}则\end{CJK} $n$ \begin{CJK}{UTF8}{mj}级排列\end{CJK} $X_{n}, X_{n-1}, \cdots, X_{2}, X_{1}$ \begin{CJK}{UTF8}{mj}的逆序数\end{CJK} \begin{CJK}{UTF8}{mj}为\end{CJK}

  \item \begin{CJK}{UTF8}{mj}行列式\end{CJK} $\left|\begin{array}{cccc}1 & 1 & 1 & 1 \\ 1 & 2 & 3 & 4 \\ 1 & 4 & 9 & 6 \\ 1 & 8 & 27 & 65\end{array}\right|=$

  \item \begin{CJK}{UTF8}{mj}若\end{CJK} $f(x)=x^{3}-3 x^{2}+t x-1$ \begin{CJK}{UTF8}{mj}有一个二重根\end{CJK}, \begin{CJK}{UTF8}{mj}则\end{CJK} $t=$

  \item \begin{CJK}{UTF8}{mj}设\end{CJK} $A=\left(\begin{array}{llll}0 & 0 & 1 & 2 \\ 0 & 0 & 1 & 3 \\ 2 & 1 & 0 & 0 \\ 5 & 3 & 0 & 0\end{array}\right)$, \begin{CJK}{UTF8}{mj}则\end{CJK} $A^{-1}=$

  \item \begin{CJK}{UTF8}{mj}设\end{CJK} $V$ \begin{CJK}{UTF8}{mj}是数域\end{CJK} $P$ \begin{CJK}{UTF8}{mj}上的线性空间\end{CJK}, $V_{1}, V_{2}$ \begin{CJK}{UTF8}{mj}是\end{CJK} $V$ \begin{CJK}{UTF8}{mj}的子空间\end{CJK}, \begin{CJK}{UTF8}{mj}如果\end{CJK} $\operatorname{dim} V=9, \operatorname{dim} V_{1}=5, \operatorname{dim} V_{2}=6$, \begin{CJK}{UTF8}{mj}则\end{CJK} $\operatorname{dim}\left(V_{1} \cap V_{2}\right)$ \begin{CJK}{UTF8}{mj}的最小值是\end{CJK}

  \item \begin{CJK}{UTF8}{mj}若数域\end{CJK} $P$ \begin{CJK}{UTF8}{mj}上\end{CJK} $n$ \begin{CJK}{UTF8}{mj}阶方阵\end{CJK} $A$ \begin{CJK}{UTF8}{mj}的各行元素之和均为\end{CJK} 0 , \begin{CJK}{UTF8}{mj}且\end{CJK} $A$ \begin{CJK}{UTF8}{mj}的秩为\end{CJK} $n-1$, \begin{CJK}{UTF8}{mj}则齐次线性方程组\end{CJK} $A X=0$ \begin{CJK}{UTF8}{mj}的所有\end{CJK} \begin{CJK}{UTF8}{mj}解是\end{CJK}

  \item \begin{CJK}{UTF8}{mj}设\end{CJK} $A=\left(\begin{array}{lll}1 & 0 & 1 \\ 0 & 2 & 0 \\ 1 & 0 & 1\end{array}\right), B=(A-k E)^{2}$, \begin{CJK}{UTF8}{mj}若\end{CJK} $B$ \begin{CJK}{UTF8}{mj}正定\end{CJK}, \begin{CJK}{UTF8}{mj}则\end{CJK} $k$ \begin{CJK}{UTF8}{mj}的取值是\end{CJK}

  \item \begin{CJK}{UTF8}{mj}设\end{CJK} $A$ \begin{CJK}{UTF8}{mj}是数域\end{CJK} $P$ \begin{CJK}{UTF8}{mj}上的二阶方阵\end{CJK}, $a_{1}, a_{2}$ \begin{CJK}{UTF8}{mj}为\end{CJK} $P$ \begin{CJK}{UTF8}{mj}上线性无关的\end{CJK} 2 \begin{CJK}{UTF8}{mj}维列向量\end{CJK}, \begin{CJK}{UTF8}{mj}且\end{CJK} $A a_{1}=0, A a_{2}=2 a_{1}+a_{2}$, \begin{CJK}{UTF8}{mj}则\end{CJK} $A$ \begin{CJK}{UTF8}{mj}的特征值为\end{CJK}

  \item \begin{CJK}{UTF8}{mj}设\end{CJK} $V$ \begin{CJK}{UTF8}{mj}是复数域上\end{CJK} $n$ \begin{CJK}{UTF8}{mj}维线性空间\end{CJK}, $\mathscr{A}$ \begin{CJK}{UTF8}{mj}为\end{CJK} $V$ \begin{CJK}{UTF8}{mj}上的线性变换\end{CJK}, $\mathscr{E}$ \begin{CJK}{UTF8}{mj}是恒等变换\end{CJK}, $\lambda$ \begin{CJK}{UTF8}{mj}是\end{CJK} $\mathscr{A}$ \begin{CJK}{UTF8}{mj}的一个三重特征值\end{CJK}, \begin{CJK}{UTF8}{mj}则\end{CJK} $\operatorname{dim}(\mathscr{A}-\lambda \mathscr{E})^{3}=$

  \item \begin{CJK}{UTF8}{mj}若\end{CJK} $n$ \begin{CJK}{UTF8}{mj}元实二次型\end{CJK}

\end{enumerate}
$$
f=\left(x_{1}+a_{1} x_{2}\right)^{2}+\left(x_{2}+a_{2} x_{3}\right)^{2}+\cdots+\left(x_{n-1}+a_{n-1} x_{n}\right)^{2}+\left(x_{n}+a_{n} x_{1}\right)^{2}
$$
\begin{CJK}{UTF8}{mj}是正定的\end{CJK}, \begin{CJK}{UTF8}{mj}则\end{CJK} $a_{1}, a_{2}, \cdots, a_{n}$ \begin{CJK}{UTF8}{mj}满足的条件是\end{CJK}

\begin{CJK}{UTF8}{mj}二\end{CJK}. ( 20 \begin{CJK}{UTF8}{mj}分\end{CJK}) \begin{CJK}{UTF8}{mj}求实数域\end{CJK} $\mathbb{R}$ \begin{CJK}{UTF8}{mj}上齐次线性方程组\end{CJK} $\left\{\begin{array}{l}x_{1}+x_{2}+x_{3}+x_{4}=0 \\ x_{1}-x_{2}+x_{3}-x_{4}=0\end{array}\right.$ \begin{CJK}{UTF8}{mj}的解空间\end{CJK} $W$ \begin{CJK}{UTF8}{mj}在\end{CJK} $\mathbb{R}$ \begin{CJK}{UTF8}{mj}中的正交补的维\end{CJK} \begin{CJK}{UTF8}{mj}数和一个标准正交基\end{CJK}.

\begin{CJK}{UTF8}{mj}三\end{CJK}. ( 20 \begin{CJK}{UTF8}{mj}分\end{CJK}) \begin{CJK}{UTF8}{mj}设\end{CJK} $V$ \begin{CJK}{UTF8}{mj}是数域\end{CJK} $P$ \begin{CJK}{UTF8}{mj}上的\end{CJK} 3 \begin{CJK}{UTF8}{mj}维线性空间\end{CJK}, $\sigma$ \begin{CJK}{UTF8}{mj}为\end{CJK} $V$ \begin{CJK}{UTF8}{mj}的线性变换\end{CJK}, $\sigma$ \begin{CJK}{UTF8}{mj}在\end{CJK} $V$ \begin{CJK}{UTF8}{mj}的基\end{CJK} $\alpha_{1}, \alpha_{2}, \alpha_{3}$ \begin{CJK}{UTF8}{mj}下的矩阵为\end{CJK}
$$
A=\left(\begin{array}{ccc}
1 & 1 & -1 \\
2 & 1 & 0 \\
1 & 1 & 0
\end{array}\right) \text {. }
$$
(1) \begin{CJK}{UTF8}{mj}证明\end{CJK} $\sigma$ \begin{CJK}{UTF8}{mj}是可逆的\end{CJK};

(2) \begin{CJK}{UTF8}{mj}求\end{CJK} $\sigma^{-1}\left(\alpha_{1}+\alpha_{2}+\alpha_{3}\right)$.

\begin{CJK}{UTF8}{mj}四\end{CJK}. ( 20 \begin{CJK}{UTF8}{mj}分\end{CJK}) \begin{CJK}{UTF8}{mj}设\end{CJK} $A$ \begin{CJK}{UTF8}{mj}是\end{CJK} $n$ \begin{CJK}{UTF8}{mj}阶方阵\end{CJK}. \begin{CJK}{UTF8}{mj}证明存在\end{CJK} $n$ \begin{CJK}{UTF8}{mj}阶方阵\end{CJK} $B$, \begin{CJK}{UTF8}{mj}使得\end{CJK} $A B A=A$. \begin{CJK}{UTF8}{mj}五\end{CJK}. ( 15 \begin{CJK}{UTF8}{mj}分\end{CJK}) \begin{CJK}{UTF8}{mj}设\end{CJK} $A=\left(\begin{array}{cc}B & G \\ G^{T} & 0\end{array}\right)$, \begin{CJK}{UTF8}{mj}其中\end{CJK} $B$ \begin{CJK}{UTF8}{mj}是\end{CJK} $n$ \begin{CJK}{UTF8}{mj}阶正定阵\end{CJK}, $G$ \begin{CJK}{UTF8}{mj}为秩为\end{CJK} $M$ \begin{CJK}{UTF8}{mj}的\end{CJK} $n \times m$ \begin{CJK}{UTF8}{mj}实矩阵\end{CJK}.

\begin{CJK}{UTF8}{mj}证明\end{CJK}: $A$ \begin{CJK}{UTF8}{mj}有\end{CJK} $n$ \begin{CJK}{UTF8}{mj}个正的特征值\end{CJK}, $m$ \begin{CJK}{UTF8}{mj}个负的特征值\end{CJK}.

\begin{CJK}{UTF8}{mj}六\end{CJK}. (15 \begin{CJK}{UTF8}{mj}分\end{CJK}) \begin{CJK}{UTF8}{mj}设\end{CJK} $V$ \begin{CJK}{UTF8}{mj}是数域\end{CJK} $P$ \begin{CJK}{UTF8}{mj}上\end{CJK} $n$ \begin{CJK}{UTF8}{mj}维线性空间\end{CJK}, $\sigma$ \begin{CJK}{UTF8}{mj}是\end{CJK} $V$ \begin{CJK}{UTF8}{mj}上的线性变换\end{CJK}, $W$ \begin{CJK}{UTF8}{mj}是\end{CJK} $V$ \begin{CJK}{UTF8}{mj}的非平凡\end{CJK} $\sigma$ \begin{CJK}{UTF8}{mj}不变子空间\end{CJK}, $\left.\sigma\right|_{W}$ \begin{CJK}{UTF8}{mj}是\end{CJK} $\sigma$ \begin{CJK}{UTF8}{mj}在\end{CJK} $W$ \begin{CJK}{UTF8}{mj}上的限制线性变换\end{CJK}.

(1) \begin{CJK}{UTF8}{mj}证明\end{CJK}: \begin{CJK}{UTF8}{mj}若\end{CJK} $\sigma$ \begin{CJK}{UTF8}{mj}可对角化\end{CJK}, \begin{CJK}{UTF8}{mj}则\end{CJK} $\left.\sigma\right|_{W}$ \begin{CJK}{UTF8}{mj}也可对角化\end{CJK}.

(2)\begin{CJK}{UTF8}{mj}反之\end{CJK}, \begin{CJK}{UTF8}{mj}假设\end{CJK} $\left.\sigma\right|_{W}$ \begin{CJK}{UTF8}{mj}可对角化\end{CJK}, \begin{CJK}{UTF8}{mj}问\end{CJK} $\sigma$ \begin{CJK}{UTF8}{mj}是否可对角化\end{CJK}? \begin{CJK}{UTF8}{mj}若能\end{CJK}, \begin{CJK}{UTF8}{mj}请给出证明\end{CJK}; \begin{CJK}{UTF8}{mj}若不能\end{CJK}, \begin{CJK}{UTF8}{mj}请举出例子\end{CJK}.

\section{9. 西南大学 2017 年研究生入学考试试题高等代数}
\begin{CJK}{UTF8}{mj}李扬\end{CJK}

\begin{CJK}{UTF8}{mj}微信公众号\end{CJK}: sxkyliyang

\begin{CJK}{UTF8}{mj}一\end{CJK}. \begin{CJK}{UTF8}{mj}填空题\end{CJK}( 70 \begin{CJK}{UTF8}{mj}分\end{CJK})

\begin{enumerate}
  \item \begin{CJK}{UTF8}{mj}设\end{CJK} $f(x)=\left|\begin{array}{ccc}2 x & 1 & -1 \\ -x & -x & x \\ 1 & 2 & x\end{array}\right|$, \begin{CJK}{UTF8}{mj}则\end{CJK} $f(x)$ \begin{CJK}{UTF8}{mj}中\end{CJK} $x^{3}$ \begin{CJK}{UTF8}{mj}的系数是\end{CJK}

  \item \begin{CJK}{UTF8}{mj}若线性方程组\end{CJK} $\left(\begin{array}{ccc}1 & 2 & 1 \\ 2 & 3 & a+2 \\ 1 & a & -2\end{array}\right)\left(\begin{array}{l}x_{1} \\ x_{2} \\ x_{3}\end{array}\right)=\left(\begin{array}{l}1 \\ 3 \\ 0\end{array}\right)$ \begin{CJK}{UTF8}{mj}无解\end{CJK}, \begin{CJK}{UTF8}{mj}则\end{CJK} $a=$

  \item \begin{CJK}{UTF8}{mj}设\end{CJK} $m$ \begin{CJK}{UTF8}{mj}是大于\end{CJK} 1 \begin{CJK}{UTF8}{mj}的正整数\end{CJK}, $f(x)=x^{m-1}+x^{m-2}+\cdots+x^{2}+x+1, c$ \begin{CJK}{UTF8}{mj}是一个数\end{CJK}, \begin{CJK}{UTF8}{mj}若\end{CJK} $f(x) \mid\left(f\left(x^{m}\right)+c\right)$, \begin{CJK}{UTF8}{mj}则\end{CJK} $c=$

  \item \begin{CJK}{UTF8}{mj}设\end{CJK} $A=\left(\begin{array}{cccc}1 & -1 & 2 & -1 \\ 0 & 1 & -1 & 2 \\ 0 & 0 & 1 & -1 \\ 0 & 0 & 0 & 1\end{array}\right)$, \begin{CJK}{UTF8}{mj}则\end{CJK} $A^{-1}=$

  \item \begin{CJK}{UTF8}{mj}设\end{CJK} $A$ \begin{CJK}{UTF8}{mj}为\end{CJK} $n$ \begin{CJK}{UTF8}{mj}阶方阵\end{CJK}, $A^{*}$ \begin{CJK}{UTF8}{mj}为其伴随矩阵\end{CJK}, \begin{CJK}{UTF8}{mj}若\end{CJK} $|A|=2$, \begin{CJK}{UTF8}{mj}则\end{CJK} $\left|\left(A^{*}\right)^{*}\right|=$

  \item \begin{CJK}{UTF8}{mj}设齐次线性方程组\end{CJK} $\left\{\begin{array}{c}3 x_{1}+2 x_{2}-5 x_{3}+4 x_{4}=0 \\ 3 x_{1}-x_{2}+3 x_{3}-3 x_{4}=0 \\ 3 x_{1}+5 x_{2}-13 x_{3}+11 x_{4}=0\end{array}\right.$ \begin{CJK}{UTF8}{mj}的解空间为\end{CJK} $W$, \begin{CJK}{UTF8}{mj}则\end{CJK} $W$ \begin{CJK}{UTF8}{mj}的维数是\end{CJK}

  \item \begin{CJK}{UTF8}{mj}若\end{CJK} $A=\left(\begin{array}{ccc}1 & 1 & a \\ 4 & 1 & -6 \\ 0 & 0 & 3\end{array}\right)$ \begin{CJK}{UTF8}{mj}可对角化\end{CJK}, \begin{CJK}{UTF8}{mj}则\end{CJK} $a=$

  \item \begin{CJK}{UTF8}{mj}设\end{CJK} $A$ \begin{CJK}{UTF8}{mj}为\end{CJK} 3 \begin{CJK}{UTF8}{mj}阶方阵\end{CJK}, $X$ \begin{CJK}{UTF8}{mj}为\end{CJK} 3 \begin{CJK}{UTF8}{mj}维列向量\end{CJK}, \begin{CJK}{UTF8}{mj}满足\end{CJK} $A^{3} X=3 A X-2 A^{2} X$, \begin{CJK}{UTF8}{mj}若向量组\end{CJK} $X, A X, A^{2} X$ \begin{CJK}{UTF8}{mj}线性无关\end{CJK}, \begin{CJK}{UTF8}{mj}则\end{CJK} $|A+E|=$

  \item \begin{CJK}{UTF8}{mj}设\end{CJK} $V$ \begin{CJK}{UTF8}{mj}是复数域上的\end{CJK} $n$ \begin{CJK}{UTF8}{mj}维线性空间\end{CJK}, $\sigma$ \begin{CJK}{UTF8}{mj}为\end{CJK} $V$ \begin{CJK}{UTF8}{mj}上的线性变换\end{CJK}, \begin{CJK}{UTF8}{mj}则\end{CJK} $\operatorname{dim} \operatorname{ker} \sigma+\operatorname{dim} \operatorname{Im} \sigma=$

\end{enumerate}
10 . \begin{CJK}{UTF8}{mj}设\end{CJK} 3 \begin{CJK}{UTF8}{mj}元实二次型\end{CJK} $f=2 x_{1}^{2}+3 x_{2}^{2}+3 x_{3}^{2}+2 a x_{2} x_{3}(a>0)$ \begin{CJK}{UTF8}{mj}通过正交线性替换化成标准形为\end{CJK} $f=2 y_{1}^{2}+y_{2}^{2}+5 y_{3}^{2}$, \begin{CJK}{UTF8}{mj}则\end{CJK} $a=$

\begin{CJK}{UTF8}{mj}二\end{CJK}. (20 \begin{CJK}{UTF8}{mj}分\end{CJK}) \begin{CJK}{UTF8}{mj}若\end{CJK} $\alpha_{1}, \alpha_{2}, \cdots, \alpha_{s}$ \begin{CJK}{UTF8}{mj}为齐次线性方程组\end{CJK} $A X=0$ \begin{CJK}{UTF8}{mj}的一个基础解系\end{CJK}, \begin{CJK}{UTF8}{mj}但向量\end{CJK} $\beta$ \begin{CJK}{UTF8}{mj}满足\end{CJK} $A \beta \neq 0$. \begin{CJK}{UTF8}{mj}证明\end{CJK}: $\alpha_{1}+\beta, \alpha_{2}+\beta, \cdots, \alpha_{s}+\beta, \beta$ \begin{CJK}{UTF8}{mj}线性无关\end{CJK}.

15 \begin{CJK}{UTF8}{mj}分\end{CJK}) \begin{CJK}{UTF8}{mj}试求出所有适合下式的非零复多项式\end{CJK} $f(x)$,
$$
f(f(x))=(f(x))^{n}
$$
\begin{CJK}{UTF8}{mj}其中\end{CJK} $n$ \begin{CJK}{UTF8}{mj}是正整数\end{CJK}.

\begin{CJK}{UTF8}{mj}四\end{CJK}. (15 \begin{CJK}{UTF8}{mj}分\end{CJK}) \begin{CJK}{UTF8}{mj}设\end{CJK} $A=\left(a_{i j}\right)_{n \times n}$ \begin{CJK}{UTF8}{mj}为\end{CJK} $n$ \begin{CJK}{UTF8}{mj}阶实对称矩阵\end{CJK}, \begin{CJK}{UTF8}{mj}证明\end{CJK}:

(1) $A$ \begin{CJK}{UTF8}{mj}特征多项式恰有\end{CJK} $n$ \begin{CJK}{UTF8}{mj}个实根\end{CJK};

(2) \begin{CJK}{UTF8}{mj}令\end{CJK} $m, M$ \begin{CJK}{UTF8}{mj}分别为\end{CJK} $A$ \begin{CJK}{UTF8}{mj}的最小\end{CJK}, \begin{CJK}{UTF8}{mj}最大特征根\end{CJK}, \begin{CJK}{UTF8}{mj}则有\end{CJK} $m \leqslant \frac{a_{11}+a_{22}+\cdots+a_{n n}}{n} \leqslant M$.

\begin{CJK}{UTF8}{mj}五\end{CJK}. ( 15 \begin{CJK}{UTF8}{mj}分\end{CJK}) \begin{CJK}{UTF8}{mj}设\end{CJK} $V$ \begin{CJK}{UTF8}{mj}是数域\end{CJK} $P$ \begin{CJK}{UTF8}{mj}上的\end{CJK} $n$ \begin{CJK}{UTF8}{mj}维线性空间\end{CJK}, $f$ \begin{CJK}{UTF8}{mj}为\end{CJK} $V$ \begin{CJK}{UTF8}{mj}的一个线性变换\end{CJK}, $W_{1}, W_{2}$ \begin{CJK}{UTF8}{mj}是\end{CJK} $V$ \begin{CJK}{UTF8}{mj}的两个真子空间\end{CJK}, \begin{CJK}{UTF8}{mj}且\end{CJK} $V=W_{1} \oplus W_{2}(\oplus$ \begin{CJK}{UTF8}{mj}表示直和\end{CJK} $)$. \begin{CJK}{UTF8}{mj}证明\end{CJK}: $f$ \begin{CJK}{UTF8}{mj}可逆的充要条件是\end{CJK} $V=f\left(W_{1}\right) \oplus f\left(W_{2}\right)$.

\begin{CJK}{UTF8}{mj}六\end{CJK}. (15 \begin{CJK}{UTF8}{mj}分\end{CJK}) \begin{CJK}{UTF8}{mj}设\end{CJK} $A, B, A_{1}, B_{1}$ \begin{CJK}{UTF8}{mj}为数域\end{CJK} $P$ \begin{CJK}{UTF8}{mj}上\end{CJK} $n$ \begin{CJK}{UTF8}{mj}阶方阵\end{CJK}. $A$ \begin{CJK}{UTF8}{mj}与\end{CJK} $A_{1}$ \begin{CJK}{UTF8}{mj}相似\end{CJK}, $B$ \begin{CJK}{UTF8}{mj}与\end{CJK} $B_{1}$ \begin{CJK}{UTF8}{mj}相似\end{CJK}. \begin{CJK}{UTF8}{mj}证明\end{CJK}: \begin{CJK}{UTF8}{mj}若\end{CJK} $n$ \begin{CJK}{UTF8}{mj}为奇数\end{CJK}, \begin{CJK}{UTF8}{mj}且\end{CJK} $A B=0$, \begin{CJK}{UTF8}{mj}则\end{CJK} $A+A_{1}, B+B_{1}$ \begin{CJK}{UTF8}{mj}中至少有一个是不可逆的\end{CJK}.

\section{0. 西南大学 2009 年研究生入学考试试题数学分析}
\begin{CJK}{UTF8}{mj}李扬\end{CJK}

\begin{CJK}{UTF8}{mj}微信公众号\end{CJK}: sxkyliyang

\begin{CJK}{UTF8}{mj}一\end{CJK}. \begin{CJK}{UTF8}{mj}选择题\end{CJK} (\begin{CJK}{UTF8}{mj}本题共\end{CJK} 10 \begin{CJK}{UTF8}{mj}小题\end{CJK}, \begin{CJK}{UTF8}{mj}每小题\end{CJK} 5 \begin{CJK}{UTF8}{mj}分\end{CJK}, \begin{CJK}{UTF8}{mj}满分\end{CJK} 50 \begin{CJK}{UTF8}{mj}分\end{CJK})

\begin{enumerate}
  \item \begin{CJK}{UTF8}{mj}设\end{CJK} $\left\{a_{n}\right\},\left\{b_{n}\right\},\left\{c_{n}\right\}$ \begin{CJK}{UTF8}{mj}均为非负数列\end{CJK}, \begin{CJK}{UTF8}{mj}且\end{CJK} $\lim _{n \rightarrow \infty} a_{n}=0, \lim _{n \rightarrow \infty} b_{n}=1, \lim _{n \rightarrow \infty} c_{n}=\infty$, \begin{CJK}{UTF8}{mj}则有\end{CJK} ( ).\\
A. $a_{n}<b_{n}$ \begin{CJK}{UTF8}{mj}对任意\end{CJK} $n$ \begin{CJK}{UTF8}{mj}成立\end{CJK}.\\
B. $b_{n}<c_{n}$ \begin{CJK}{UTF8}{mj}对任意\end{CJK} $n$ \begin{CJK}{UTF8}{mj}成立\end{CJK}.\\
C. \begin{CJK}{UTF8}{mj}极限\end{CJK} $\lim _{n \rightarrow \infty} a_{n} c_{n}$ \begin{CJK}{UTF8}{mj}不存在\end{CJK}.\\
D. \begin{CJK}{UTF8}{mj}极限\end{CJK} $\lim _{n \rightarrow \infty} b_{n} c_{n}$ \begin{CJK}{UTF8}{mj}不存在\end{CJK}.

  \item \begin{CJK}{UTF8}{mj}设\end{CJK} $p_{n}=\frac{a_{n}+\left|a_{n}\right|}{2}, q_{n}=\frac{a_{n}-\left|a_{n}\right|}{2}, n=1,2, \cdots \cdots$, \begin{CJK}{UTF8}{mj}则下列命题正确的是\end{CJK}\\
A. \begin{CJK}{UTF8}{mj}若\end{CJK} $\sum_{n=1}^{\infty} a_{n}$ \begin{CJK}{UTF8}{mj}条件收敛\end{CJK}, \begin{CJK}{UTF8}{mj}则\end{CJK} $\sum_{n=1}^{\infty} p_{n}$ \begin{CJK}{UTF8}{mj}与\end{CJK} $\sum_{n=1}^{\infty} q_{n}$ \begin{CJK}{UTF8}{mj}都收敛\end{CJK}.\\
B. \begin{CJK}{UTF8}{mj}若\end{CJK} $\sum_{n=1}^{\infty} a_{n}$ \begin{CJK}{UTF8}{mj}条件收敛\end{CJK}, \begin{CJK}{UTF8}{mj}则\end{CJK} $\sum_{n=1}^{\infty} p_{n}$ \begin{CJK}{UTF8}{mj}与\end{CJK} $\sum_{n=1}^{\infty} q_{n}$ \begin{CJK}{UTF8}{mj}的敛散性都不定\end{CJK}.\\
C. \begin{CJK}{UTF8}{mj}若\end{CJK} $\sum_{n=1}^{\infty} a_{n}$ \begin{CJK}{UTF8}{mj}绝对收敛\end{CJK}, \begin{CJK}{UTF8}{mj}则\end{CJK} $\sum_{n=1}^{\infty} p_{n}$ \begin{CJK}{UTF8}{mj}与\end{CJK} $\sum_{n=1}^{\infty} q_{n}$ \begin{CJK}{UTF8}{mj}都收敛\end{CJK}.\\
D. \begin{CJK}{UTF8}{mj}若\end{CJK} $\sum_{n=1}^{\infty} a_{n}$ \begin{CJK}{UTF8}{mj}绝对收敛\end{CJK}, \begin{CJK}{UTF8}{mj}则\end{CJK} $\sum_{n=1}^{\infty} p_{n}$ \begin{CJK}{UTF8}{mj}与\end{CJK} $\sum_{n=1}^{\infty} q_{n}$ \begin{CJK}{UTF8}{mj}敛散性都不定\end{CJK}.

  \item \begin{CJK}{UTF8}{mj}设\end{CJK} $f(x)$ \begin{CJK}{UTF8}{mj}在\end{CJK} $[a, b]$ \begin{CJK}{UTF8}{mj}上连续\end{CJK}, \begin{CJK}{UTF8}{mj}且\end{CJK} $f(x)>0$ \begin{CJK}{UTF8}{mj}在\end{CJK} $[a, b]$ \begin{CJK}{UTF8}{mj}上恒成立\end{CJK}, \begin{CJK}{UTF8}{mj}则方程\end{CJK} $\int_{a}^{x} f(t) \mathrm{d} t+\int_{a}^{x} \frac{1}{f(t)} \mathrm{d} t=0$ \begin{CJK}{UTF8}{mj}在\end{CJK} $(a, b)$ \begin{CJK}{UTF8}{mj}内\end{CJK} \begin{CJK}{UTF8}{mj}根的个数为\end{CJK} ( ).\\
A. 0\\
B. 1\\
C. 2\\
D. 3

  \item \begin{CJK}{UTF8}{mj}设\end{CJK} $f(x)=\frac{\ln |x|}{|x-1|} \sin x$, \begin{CJK}{UTF8}{mj}则\end{CJK} $f(x)$ \begin{CJK}{UTF8}{mj}有\end{CJK} ( ) $)$.\\
A. \begin{CJK}{UTF8}{mj}一个可去间断点\end{CJK}, \begin{CJK}{UTF8}{mj}一个跳跃间断点\end{CJK}.\\
B. \begin{CJK}{UTF8}{mj}一个可去间断点\end{CJK},\begin{CJK}{UTF8}{mj}一个无穷间断点\end{CJK}.\\
C. \begin{CJK}{UTF8}{mj}两个可去间断点\end{CJK}.\\
D. \begin{CJK}{UTF8}{mj}两个无穷间断点\end{CJK}.

  \item \begin{CJK}{UTF8}{mj}设\end{CJK} $f(x)$ \begin{CJK}{UTF8}{mj}为\end{CJK} $(-\infty,+\infty)$ \begin{CJK}{UTF8}{mj}上的偶函数\end{CJK}, \begin{CJK}{UTF8}{mj}且有二阶导数\end{CJK}, \begin{CJK}{UTF8}{mj}若在\end{CJK} $(-\infty, 0)$ \begin{CJK}{UTF8}{mj}内\end{CJK} $f^{\prime}(x)>0$, \begin{CJK}{UTF8}{mj}且\end{CJK} $f^{\prime \prime}(x)<0$, \begin{CJK}{UTF8}{mj}则在\end{CJK} $(0,+\infty)$ \begin{CJK}{UTF8}{mj}内有\end{CJK}\\
A. $f^{\prime}(x)>0, f^{\prime \prime}(x)<0$.\\
B. $f^{\prime}(x)>0, f^{\prime \prime}(x)>0$.\\
C. $f^{\prime}(x)<0, f^{\prime \prime}(x)<0$.\\
D. $f^{\prime}(x)<0, f^{\prime \prime}(x)>0$.

  \item \begin{CJK}{UTF8}{mj}设\end{CJK} $f(x)$ \begin{CJK}{UTF8}{mj}在\end{CJK} $x=1$ \begin{CJK}{UTF8}{mj}点处可导\end{CJK}, \begin{CJK}{UTF8}{mj}且\end{CJK} $\lim _{x \rightarrow 0} \frac{f(1)-f(1-x)}{2 x}=1$, \begin{CJK}{UTF8}{mj}则曲线\end{CJK} $y=f(x)$ \begin{CJK}{UTF8}{mj}在点\end{CJK} $(1, f(1))$ \begin{CJK}{UTF8}{mj}处切线的斜率\end{CJK} \begin{CJK}{UTF8}{mj}为\end{CJK} ( ).\\
A. $-2$\\
B. $-1$\\
C. $\frac{1}{2}$\\
D. 2

  \item \begin{CJK}{UTF8}{mj}设\end{CJK} $f(x), \varphi(x)$ \begin{CJK}{UTF8}{mj}在点\end{CJK} $x=0$ \begin{CJK}{UTF8}{mj}的某个邻域内连续\end{CJK}, \begin{CJK}{UTF8}{mj}且当\end{CJK} $x \rightarrow 0$ \begin{CJK}{UTF8}{mj}时\end{CJK}, $f(x)$ \begin{CJK}{UTF8}{mj}是\end{CJK} $\varphi(x)$ \begin{CJK}{UTF8}{mj}的高阶无穷小量\end{CJK}, \begin{CJK}{UTF8}{mj}则当\end{CJK} $x \rightarrow 0$ \begin{CJK}{UTF8}{mj}时\end{CJK}, $\int_{0}^{x} f(t) \sin t \mathrm{~d} t$ \begin{CJK}{UTF8}{mj}是\end{CJK} $\int_{0}^{x} t \varphi(t) \mathrm{d} t$ \begin{CJK}{UTF8}{mj}的\end{CJK} ( ).\\
A. \begin{CJK}{UTF8}{mj}低阶无穷小量\end{CJK}.\\
B. \begin{CJK}{UTF8}{mj}高阶无穷小量\end{CJK}.\\
C. \begin{CJK}{UTF8}{mj}同阶但不等价的无穷小量\end{CJK}.\\
D. \begin{CJK}{UTF8}{mj}等价无穷小量\end{CJK}.

  \item \begin{CJK}{UTF8}{mj}下列广义积分发散的是\end{CJK} ( ).\\
A. $\int_{0}^{+\infty} e^{-x^{2}} \mathrm{~d} x$\\
B. $\int_{-1}^{1} \frac{1}{\sqrt{1-x^{2}}} \mathrm{~d} x$\\
C. $\int_{-1}^{1} \frac{1}{\sin x} \mathrm{~d} x$\\
D. $\int_{2}^{+\infty} \frac{1}{x \ln ^{2} x} \mathrm{~d} x$

  \item \begin{CJK}{UTF8}{mj}已知\end{CJK} $f(x, y)$ \begin{CJK}{UTF8}{mj}在点\end{CJK} $(0,0)$ \begin{CJK}{UTF8}{mj}的某邻域内连续\end{CJK}, \begin{CJK}{UTF8}{mj}且\end{CJK} $\lim _{(x, y) \rightarrow(0,0)} \frac{f(x, y)-x y}{\left(x^{2}+y^{2}\right)^{2}}=1$, \begin{CJK}{UTF8}{mj}则\end{CJK} ( ).\\
A. \begin{CJK}{UTF8}{mj}点\end{CJK} $(0,0)$ \begin{CJK}{UTF8}{mj}不是\end{CJK} $f(x, y)$ \begin{CJK}{UTF8}{mj}的极值点\end{CJK}.\\
B. \begin{CJK}{UTF8}{mj}点\end{CJK} $(0,0)$ \begin{CJK}{UTF8}{mj}是\end{CJK} $f(x, y)$ \begin{CJK}{UTF8}{mj}的极大值点\end{CJK}.\\
C. \begin{CJK}{UTF8}{mj}点\end{CJK} $(0,0)$ \begin{CJK}{UTF8}{mj}是\end{CJK} $f(x, y)$ \begin{CJK}{UTF8}{mj}的极小值点\end{CJK}.\\
D. \begin{CJK}{UTF8}{mj}根据所给条件无法判断点\end{CJK} $(0,0)$ \begin{CJK}{UTF8}{mj}是否为\end{CJK} $f(x, y)$ \begin{CJK}{UTF8}{mj}的极值点\end{CJK}.

  \item \begin{CJK}{UTF8}{mj}设曲线\end{CJK} $L$ \begin{CJK}{UTF8}{mj}为取顺时针方向的圆周\end{CJK} $x^{2}+y^{2}=a^{2}$, \begin{CJK}{UTF8}{mj}则\end{CJK} $\oint_{L} y \mathrm{~d} x-x \mathrm{~d} y$ \begin{CJK}{UTF8}{mj}的值为\end{CJK}\\
A. $2 \pi a^{2}$\\
B. $-2 \pi a^{2}$\\
C. $-\pi a^{2}$\\
D. $\pi a^{2}$

\end{enumerate}
\begin{CJK}{UTF8}{mj}二\end{CJK}. \begin{CJK}{UTF8}{mj}计算题\end{CJK} (\begin{CJK}{UTF8}{mj}本题共\end{CJK} 5 \begin{CJK}{UTF8}{mj}小题\end{CJK}, \begin{CJK}{UTF8}{mj}每小题\end{CJK} 7 \begin{CJK}{UTF8}{mj}分\end{CJK}, \begin{CJK}{UTF8}{mj}满分\end{CJK} 35 \begin{CJK}{UTF8}{mj}分\end{CJK})

\begin{enumerate}
  \item \begin{CJK}{UTF8}{mj}求极限\end{CJK}
\end{enumerate}
$$
\lim _{x \rightarrow 0}\left[\frac{1}{x}-\frac{\ln (1+x)}{x^{2}}\right]
$$

\begin{enumerate}
  \setcounter{enumi}{2}
  \item \begin{CJK}{UTF8}{mj}求极限\end{CJK}
\end{enumerate}
$$
\lim _{n \rightarrow \infty}\left(\frac{1}{n+1}+\frac{1}{n+2}+\cdots+\frac{1}{2 n}\right)
$$

\begin{enumerate}
  \setcounter{enumi}{3}
  \item \begin{CJK}{UTF8}{mj}求不定积分\end{CJK}
\end{enumerate}
\includegraphics[max width=\textwidth]{2022_04_18_3416d289b173eb9de8c1g-036}

\begin{CJK}{UTF8}{mj}求\end{CJK} $f(0)$.

\begin{enumerate}
  \setcounter{enumi}{5}
  \item \begin{CJK}{UTF8}{mj}已知\end{CJK} $f(x)$ \begin{CJK}{UTF8}{mj}连续\end{CJK},
\end{enumerate}
$$
\int_{0}^{x} t f(x-t) \mathrm{d} t=1-\cos x
$$

\section{1. 西南大学 2010 年研究生入学考试试题数学分析}
\begin{CJK}{UTF8}{mj}李扬\end{CJK}

\begin{CJK}{UTF8}{mj}微信公众号\end{CJK}: sxkyliyang

\begin{CJK}{UTF8}{mj}一\end{CJK}. \begin{CJK}{UTF8}{mj}选择题\end{CJK} (\begin{CJK}{UTF8}{mj}本题共\end{CJK} 10 \begin{CJK}{UTF8}{mj}小题\end{CJK}, \begin{CJK}{UTF8}{mj}每小题\end{CJK} 5 \begin{CJK}{UTF8}{mj}分\end{CJK}, \begin{CJK}{UTF8}{mj}满分\end{CJK} 50 \begin{CJK}{UTF8}{mj}分\end{CJK})

\begin{enumerate}
  \item \begin{CJK}{UTF8}{mj}设\end{CJK} $f$ \begin{CJK}{UTF8}{mj}在\end{CJK} $[a, b]$ \begin{CJK}{UTF8}{mj}上可导\end{CJK}, $x_{0} \in[a, b]$ \begin{CJK}{UTF8}{mj}是\end{CJK} $f$ \begin{CJK}{UTF8}{mj}的最大值点\end{CJK}, \begin{CJK}{UTF8}{mj}则\end{CJK} ( ).\\
A. $f^{\prime}\left(x_{0}\right)=0$.\\
B. $f^{\prime}\left(x_{0}\right) \neq 0 .$\\
C. \begin{CJK}{UTF8}{mj}当\end{CJK} $x_{0} \in(a, b)$ \begin{CJK}{UTF8}{mj}时\end{CJK}, $f^{\prime}\left(x_{0}\right)=0$.\\
D. \begin{CJK}{UTF8}{mj}以上都不对\end{CJK}.

  \item \begin{CJK}{UTF8}{mj}设\end{CJK} $\lim _{x \rightarrow x_{0}} f(x)$ \begin{CJK}{UTF8}{mj}及\end{CJK} $\lim _{x \rightarrow x_{0}} g(x)$ \begin{CJK}{UTF8}{mj}都不存在\end{CJK}, \begin{CJK}{UTF8}{mj}则\end{CJK} $(\quad)$.\\
A. $\lim _{x \rightarrow x_{0}}(f(x)+g(x))$ \begin{CJK}{UTF8}{mj}及\end{CJK} $\lim _{x \rightarrow x_{0}}(f(x)-g(x))$ \begin{CJK}{UTF8}{mj}一定都不存在\end{CJK}.\\
B. $\lim _{x \rightarrow x_{0}}(f(x)+g(x))$ \begin{CJK}{UTF8}{mj}及\end{CJK} $\lim _{x \rightarrow x_{0}}(f(x)-g(x))$ \begin{CJK}{UTF8}{mj}一定都存在\end{CJK}.\\
C. $\lim _{x \rightarrow x_{0}}(f(x)+g(x))$ \begin{CJK}{UTF8}{mj}及\end{CJK} $\lim _{x \rightarrow x_{0}}(f(x)-g(x))$ \begin{CJK}{UTF8}{mj}中有一个存在\end{CJK}, \begin{CJK}{UTF8}{mj}而另一个不存在\end{CJK}.\\
D. $\lim _{x \rightarrow x_{0}}(f(x)+g(x))$ \begin{CJK}{UTF8}{mj}及\end{CJK} $\lim _{x \rightarrow x_{0}}(f(x)-g(x))$ \begin{CJK}{UTF8}{mj}有可能都不存在\end{CJK}.

  \item $\lim _{n \rightarrow \infty}(\sqrt{n+2}-2 \sqrt{n+1}+\sqrt{n})=(\quad)$.\\
A. 0\\
B. 1\\
C. 2\\
D. 3

  \item $\lim _{x \rightarrow 0} \frac{x-\sin x}{x+\sin x}=(\quad)$.\\
A. 1\\
B. 0\\
D. \begin{CJK}{UTF8}{mj}不存在\end{CJK}

  \item “\begin{CJK}{UTF8}{mj}对任意给定的\end{CJK} $\varepsilon \in(0,1)$, \begin{CJK}{UTF8}{mj}总存在正整数\end{CJK} $N$, \begin{CJK}{UTF8}{mj}使得当\end{CJK} $n \geqslant N$ \begin{CJK}{UTF8}{mj}时\end{CJK}, \begin{CJK}{UTF8}{mj}恒有\end{CJK} $\left|x_{n}-a\right| \leqslant 2 \varepsilon$ " \begin{CJK}{UTF8}{mj}是数列\end{CJK} $\left\{x_{n}\right\}$ \begin{CJK}{UTF8}{mj}收玫于\end{CJK} $a$ \begin{CJK}{UTF8}{mj}的\end{CJK} ( ).\\
A. \begin{CJK}{UTF8}{mj}充分条件但非必要条件\end{CJK}.\\
B. \begin{CJK}{UTF8}{mj}必要条件但非充分条件\end{CJK}.\\
C. \begin{CJK}{UTF8}{mj}充分必要条件\end{CJK}\\
D. \begin{CJK}{UTF8}{mj}既非充分又非必要条件\end{CJK}.

  \item \begin{CJK}{UTF8}{mj}定义域为\end{CJK} $[a, b]$, \begin{CJK}{UTF8}{mj}值域为\end{CJK} $(-1,1)$ \begin{CJK}{UTF8}{mj}的连续函数\end{CJK} ( ).\\
A. \begin{CJK}{UTF8}{mj}定不存在\end{CJK}\\
B. \begin{CJK}{UTF8}{mj}可能存在\end{CJK}\\
C. \begin{CJK}{UTF8}{mj}存在且唯一\end{CJK}\\
D. \begin{CJK}{UTF8}{mj}存在\end{CJK}

  \item \begin{CJK}{UTF8}{mj}函数\end{CJK} $f(x)$ \begin{CJK}{UTF8}{mj}在\end{CJK} $[a, b]$ \begin{CJK}{UTF8}{mj}上可积的充分必要条件是\end{CJK} ( ) .\\
A. \begin{CJK}{UTF8}{mj}连续\end{CJK}\\
B. \begin{CJK}{UTF8}{mj}有界\end{CJK}\\
C. \begin{CJK}{UTF8}{mj}无间断点\end{CJK}\\
D. \begin{CJK}{UTF8}{mj}有原函数\end{CJK}

  \item \begin{CJK}{UTF8}{mj}函数\end{CJK} $f(x)$ \begin{CJK}{UTF8}{mj}是奇函数\end{CJK}, \begin{CJK}{UTF8}{mj}且在\end{CJK} $[-a, a]$ \begin{CJK}{UTF8}{mj}上可积\end{CJK}, \begin{CJK}{UTF8}{mj}则\end{CJK} ( ).\\
A. $\int_{-a}^{a} f(x) \mathrm{d} x=2 \int_{0}^{a} f(x) \mathrm{d} x$\\
B. $\int_{-a}^{a} f(x) \mathrm{d} x=0$\\
C. $\int_{-a}^{a} f(x) \mathrm{d} x=-2 \int_{0}^{a} f(x) \mathrm{d} x$\\
D. $\int_{-a}^{a} f(x) \mathrm{d} x=2 f(a)$ 9. \begin{CJK}{UTF8}{mj}级数\end{CJK} $\sum_{n=1}^{\infty} a_{n}$ \begin{CJK}{UTF8}{mj}收敛是\end{CJK} $\sum_{n=1}^{\infty} a_{n}$ \begin{CJK}{UTF8}{mj}部分和有界的\end{CJK} ( ).\\
A. \begin{CJK}{UTF8}{mj}必要条件\end{CJK}\\
B. \begin{CJK}{UTF8}{mj}充分条件\end{CJK}\\
C. \begin{CJK}{UTF8}{mj}充分必要条件\end{CJK}\\
D. \begin{CJK}{UTF8}{mj}无关条件\end{CJK}

  \item \begin{CJK}{UTF8}{mj}曲线\end{CJK} $y=\frac{x}{1-x^{2}}$ \begin{CJK}{UTF8}{mj}的渐近线有\end{CJK} ( ).\\
A. 1 \begin{CJK}{UTF8}{mj}条\end{CJK}\\
B. 2 \begin{CJK}{UTF8}{mj}条\end{CJK}\\
C. 3 \begin{CJK}{UTF8}{mj}条\end{CJK}\\
D. 4 \begin{CJK}{UTF8}{mj}条\end{CJK}

\end{enumerate}
\begin{CJK}{UTF8}{mj}二\end{CJK}. \begin{CJK}{UTF8}{mj}计算题\end{CJK} (\begin{CJK}{UTF8}{mj}每小题\end{CJK} 10 \begin{CJK}{UTF8}{mj}分\end{CJK}, \begin{CJK}{UTF8}{mj}共\end{CJK} 50 \begin{CJK}{UTF8}{mj}分\end{CJK})

\begin{enumerate}
  \item \begin{CJK}{UTF8}{mj}计算\end{CJK}
\end{enumerate}
$$
\int_{1}^{+\infty} \frac{\mathrm{d} x}{e^{1+x}+e^{3-x}}
$$

\begin{enumerate}
  \setcounter{enumi}{2}
  \item \begin{CJK}{UTF8}{mj}已知\end{CJK} $z=u^{v}, u=\ln \sqrt{x^{2}+y^{2}}, v=\arctan \frac{y}{x}$, \begin{CJK}{UTF8}{mj}求\end{CJK} $\mathrm{d} z$.

  \item \begin{CJK}{UTF8}{mj}设\end{CJK} $D$ \begin{CJK}{UTF8}{mj}为以\end{CJK} $O(0,0), A(1,2), B(2,1)$ \begin{CJK}{UTF8}{mj}为顶点的三角形区域\end{CJK}, \begin{CJK}{UTF8}{mj}求\end{CJK} $\iint_{D} x \mathrm{~d} x \mathrm{~d} y$.

  \item \begin{CJK}{UTF8}{mj}求数项级数\end{CJK}

\end{enumerate}
$$
\sum_{n=1}^{\infty} \frac{1}{2^{n}(2 n-1)}
$$
\begin{CJK}{UTF8}{mj}的和\end{CJK}.

\begin{enumerate}
  \setcounter{enumi}{5}
  \item \begin{CJK}{UTF8}{mj}求极限\end{CJK}
\end{enumerate}
$$
\lim _{n \rightarrow+\infty} \sin \left(\sqrt{n^{2}+1} \pi\right)
$$
\begin{CJK}{UTF8}{mj}三\end{CJK}. \begin{CJK}{UTF8}{mj}证明题\end{CJK} (\begin{CJK}{UTF8}{mj}共\end{CJK} 50 \begin{CJK}{UTF8}{mj}分\end{CJK})

\begin{enumerate}
  \item (10 \begin{CJK}{UTF8}{mj}分\end{CJK}) \begin{CJK}{UTF8}{mj}设\end{CJK} $f$ \begin{CJK}{UTF8}{mj}为\end{CJK} $(-\infty,+\infty)$ \begin{CJK}{UTF8}{mj}上的奇函数\end{CJK}, \begin{CJK}{UTF8}{mj}且对任意实数\end{CJK} $x$, \begin{CJK}{UTF8}{mj}有\end{CJK} $f(x+2)=f(x)+f(2)$. \begin{CJK}{UTF8}{mj}又设\end{CJK} $f(1)=a$, \begin{CJK}{UTF8}{mj}证\end{CJK} \begin{CJK}{UTF8}{mj}明\end{CJK}: \begin{CJK}{UTF8}{mj}对任意整数\end{CJK} $n$, \begin{CJK}{UTF8}{mj}有\end{CJK}
\end{enumerate}
$$
f(n)=n a .
$$

\begin{enumerate}
  \setcounter{enumi}{2}
  \item ( 20 \begin{CJK}{UTF8}{mj}分\end{CJK}) \begin{CJK}{UTF8}{mj}设\end{CJK}
\end{enumerate}
$$
x_{1}>0, x_{n+1}=\frac{x_{n}\left(x_{n}^{2}+3\right)}{3 x_{n}^{2}+1},(n=1,2, \cdots)
$$
\begin{CJK}{UTF8}{mj}证明\end{CJK}: \begin{CJK}{UTF8}{mj}数列\end{CJK} $\left\{x_{n}\right\}$ \begin{CJK}{UTF8}{mj}收敛\end{CJK}, \begin{CJK}{UTF8}{mj}并求\end{CJK} $\lim _{n \rightarrow \infty} x_{n}$.

\begin{enumerate}
  \setcounter{enumi}{3}
  \item ( 20 \begin{CJK}{UTF8}{mj}分\end{CJK}) \begin{CJK}{UTF8}{mj}设\end{CJK} $f(x)$ \begin{CJK}{UTF8}{mj}在\end{CJK} $[-2,2]$ \begin{CJK}{UTF8}{mj}上二阶可导\end{CJK}, \begin{CJK}{UTF8}{mj}且\end{CJK} $|f(x)| \leqslant 1,(-2 \leqslant x \leqslant 2)$, \begin{CJK}{UTF8}{mj}又设\end{CJK} $\frac{1}{2}\left[f^{\prime}(0)\right]^{2}+f^{3}(0)>\frac{3}{2}$. \begin{CJK}{UTF8}{mj}证明\end{CJK}: \begin{CJK}{UTF8}{mj}存在\end{CJK} $x_{0} \in(-2,2)$, \begin{CJK}{UTF8}{mj}使得\end{CJK}
\end{enumerate}
$$
f^{\prime \prime}\left(x_{0}\right)+3 f^{2}\left(x_{0}\right)=0
$$

\section{2. 西南大学 2011 年研究生入学考试试题数学分析}
\begin{CJK}{UTF8}{mj}李扬\end{CJK}

\begin{CJK}{UTF8}{mj}微信公众号\end{CJK}: sxkyliyang

\begin{CJK}{UTF8}{mj}一\end{CJK}. \begin{CJK}{UTF8}{mj}选择题\end{CJK} (\begin{CJK}{UTF8}{mj}本题共\end{CJK} 4 \begin{CJK}{UTF8}{mj}小题\end{CJK}, \begin{CJK}{UTF8}{mj}每小题\end{CJK} 5 \begin{CJK}{UTF8}{mj}分\end{CJK}, \begin{CJK}{UTF8}{mj}满分\end{CJK} 20 \begin{CJK}{UTF8}{mj}分\end{CJK})

\begin{enumerate}
  \item \begin{CJK}{UTF8}{mj}设函数\end{CJK} $f(x)$ \begin{CJK}{UTF8}{mj}在点\end{CJK} $x=a$ \begin{CJK}{UTF8}{mj}处可导\end{CJK}, \begin{CJK}{UTF8}{mj}则函数\end{CJK} $|f(x)|$ \begin{CJK}{UTF8}{mj}在点\end{CJK} $x=a$ \begin{CJK}{UTF8}{mj}处不可导的充分条件是\end{CJK} ( )\\
A. $f(a)=0$ \begin{CJK}{UTF8}{mj}且\end{CJK} $f^{\prime}(a)=0$.\\
B. $f(a)=0$ \begin{CJK}{UTF8}{mj}且\end{CJK} $f^{\prime}(a) \neq 0$.\\
C. $f(a)>0$ \begin{CJK}{UTF8}{mj}且\end{CJK} $f^{\prime}(a)>0$.\\
D. $f(a)<0$ \begin{CJK}{UTF8}{mj}且\end{CJK} $f^{\prime}(a)<0$.

  \item \begin{CJK}{UTF8}{mj}设周期函数\end{CJK} $f(x)$ \begin{CJK}{UTF8}{mj}在\end{CJK} $(-\infty,+\infty)$ \begin{CJK}{UTF8}{mj}内可导\end{CJK}, \begin{CJK}{UTF8}{mj}周期为\end{CJK} 4 , \begin{CJK}{UTF8}{mj}且\end{CJK} $\lim _{x \rightarrow 0} \frac{f(1)-f(1-x)}{2 x}=-1$, \begin{CJK}{UTF8}{mj}则曲线\end{CJK} $y=f(x)$ \begin{CJK}{UTF8}{mj}在点\end{CJK} $(5, f(5))$ \begin{CJK}{UTF8}{mj}处的斜率为\end{CJK} $(\quad)$.\\
A. $\frac{1}{2}$\\
B. 0\\
C. $-1$\\
D. $-2$

  \item \begin{CJK}{UTF8}{mj}设\end{CJK} $f(x), \varphi(x)$ \begin{CJK}{UTF8}{mj}在点\end{CJK} $x=0$ \begin{CJK}{UTF8}{mj}的某个邻域内连续\end{CJK}, \begin{CJK}{UTF8}{mj}且当\end{CJK} $x \rightarrow 0$ \begin{CJK}{UTF8}{mj}时\end{CJK}, $f(x)$ \begin{CJK}{UTF8}{mj}是\end{CJK} $\varphi(x)$ \begin{CJK}{UTF8}{mj}的高阶无穷小量\end{CJK}, \begin{CJK}{UTF8}{mj}则当\end{CJK} $x \rightarrow 0$ \begin{CJK}{UTF8}{mj}时\end{CJK}, $\int_{0}^{x} f(t) \sin t \mathrm{~d} t$ \begin{CJK}{UTF8}{mj}是\end{CJK} $\int_{0}^{x} t \varphi(t) \mathrm{d} t$ \begin{CJK}{UTF8}{mj}的\end{CJK} ( ).\\
A. \begin{CJK}{UTF8}{mj}低阶无穷小量\end{CJK}.\\
B. \begin{CJK}{UTF8}{mj}高阶无穷小量\end{CJK}.\\
C. \begin{CJK}{UTF8}{mj}同阶但不等价的无穷小量\end{CJK}.\\
D. \begin{CJK}{UTF8}{mj}等价无穷小量\end{CJK}.

  \item \begin{CJK}{UTF8}{mj}幂级数\end{CJK} $\sum_{n=0}^{\infty} a_{n} x^{n}$ \begin{CJK}{UTF8}{mj}在其收敛域内\end{CJK}\\
A. \begin{CJK}{UTF8}{mj}一定收敛\end{CJK}\\
B. \begin{CJK}{UTF8}{mj}每一点绝对收敛\end{CJK}\\
C. \begin{CJK}{UTF8}{mj}其和函数连续\end{CJK}\\
D. \begin{CJK}{UTF8}{mj}可微\end{CJK}

\end{enumerate}
\begin{CJK}{UTF8}{mj}二\end{CJK}. \begin{CJK}{UTF8}{mj}计算题\end{CJK} (\begin{CJK}{UTF8}{mj}本题共\end{CJK} 6 \begin{CJK}{UTF8}{mj}小题\end{CJK}, \begin{CJK}{UTF8}{mj}每小题\end{CJK} 10 \begin{CJK}{UTF8}{mj}分\end{CJK}, \begin{CJK}{UTF8}{mj}共\end{CJK} 60 \begin{CJK}{UTF8}{mj}分\end{CJK})

\begin{enumerate}
  \item \begin{CJK}{UTF8}{mj}设\end{CJK}
\end{enumerate}
$$
f(t)=\lim _{x \rightarrow \infty} t\left(\frac{x+t}{x-t}\right)^{x}
$$
\begin{CJK}{UTF8}{mj}求\end{CJK} $f^{\prime}(t)$.

\begin{enumerate}
  \setcounter{enumi}{2}
  \item \begin{CJK}{UTF8}{mj}求极限\end{CJK}
\end{enumerate}
$$
\lim _{n \rightarrow \infty} \frac{n^{5}}{e^{n}}
$$

\begin{enumerate}
  \setcounter{enumi}{3}
  \item \begin{CJK}{UTF8}{mj}求\end{CJK}
\end{enumerate}
$$
\int \frac{\ln (1+x)-\ln x}{x(x+1)} \mathrm{d} x
$$

\begin{enumerate}
  \setcounter{enumi}{4}
  \item \begin{CJK}{UTF8}{mj}设\end{CJK} $f(x, y)$ \begin{CJK}{UTF8}{mj}连续\end{CJK}, $D$ \begin{CJK}{UTF8}{mj}是由\end{CJK} $y=0, y=x^{2}, x=1$ \begin{CJK}{UTF8}{mj}所围成的平面区域\end{CJK}, \begin{CJK}{UTF8}{mj}且\end{CJK}
\end{enumerate}
$$
f(x, y)=x y+\iint_{D} f(x, y) \mathrm{d} x \mathrm{~d} y .
$$
\begin{CJK}{UTF8}{mj}求\end{CJK} $\iint_{D} f(x, y) \mathrm{d} x \mathrm{~d} y$ \begin{CJK}{UTF8}{mj}的值\end{CJK}.

\begin{enumerate}
  \setcounter{enumi}{5}
  \item \begin{CJK}{UTF8}{mj}将\end{CJK} $f(x)=2^{x}$ \begin{CJK}{UTF8}{mj}在\end{CJK} $x_{0}=0$ \begin{CJK}{UTF8}{mj}点展开成泰勒级数\end{CJK}.

  \item \begin{CJK}{UTF8}{mj}计算\end{CJK}

\end{enumerate}
$$
\int_{0}^{4} \frac{x+2}{\sqrt{2 x+1}} \mathrm{~d} x
$$

\section{三. 证明题 (共 70 分)}
\begin{enumerate}
  \item ( 10 \begin{CJK}{UTF8}{mj}分\end{CJK}) \begin{CJK}{UTF8}{mj}若\end{CJK} $f(x)$ \begin{CJK}{UTF8}{mj}为\end{CJK} $(-\infty,+\infty)$ \begin{CJK}{UTF8}{mj}上可导函数\end{CJK}, \begin{CJK}{UTF8}{mj}且方程\end{CJK} $f^{\prime}(x)=0$ \begin{CJK}{UTF8}{mj}至多有一个实根\end{CJK}. \begin{CJK}{UTF8}{mj}证明方程\end{CJK} $f(x)=0$ \begin{CJK}{UTF8}{mj}至多\end{CJK} \begin{CJK}{UTF8}{mj}有两个实根\end{CJK}.

  \item (10 \begin{CJK}{UTF8}{mj}分\end{CJK}) \begin{CJK}{UTF8}{mj}证明数列\end{CJK}

\end{enumerate}
$$
\left\{\frac{\ln (n+2)}{n+2}\right\}
$$
\begin{CJK}{UTF8}{mj}单调递减\end{CJK}.

\begin{enumerate}
  \setcounter{enumi}{3}
  \item ( 20 \begin{CJK}{UTF8}{mj}分\end{CJK} $)$ \begin{CJK}{UTF8}{mj}设\end{CJK} $f(x)$ \begin{CJK}{UTF8}{mj}在\end{CJK} $(-\infty,+\infty)$ \begin{CJK}{UTF8}{mj}上连续\end{CJK}, \begin{CJK}{UTF8}{mj}且\end{CJK} $\lim _{x \rightarrow 0} \frac{f(x)-\sin x}{x}=1$, \begin{CJK}{UTF8}{mj}令\end{CJK}
\end{enumerate}
$$
F(x)=\int_{0}^{1} f(x y) \mathrm{d} y .
$$
\begin{CJK}{UTF8}{mj}求\end{CJK} $F^{\prime}(x)$, \begin{CJK}{UTF8}{mj}并证明\end{CJK} $F^{\prime}(x)$ \begin{CJK}{UTF8}{mj}在\end{CJK} $(-\infty,+\infty)$ \begin{CJK}{UTF8}{mj}上连续\end{CJK}.

\begin{enumerate}
  \setcounter{enumi}{4}
  \item ( 10 \begin{CJK}{UTF8}{mj}分\end{CJK}) \begin{CJK}{UTF8}{mj}设\end{CJK} $f$ \begin{CJK}{UTF8}{mj}在\end{CJK} $[0,2]$ \begin{CJK}{UTF8}{mj}上有二阶连续的导数\end{CJK}, \begin{CJK}{UTF8}{mj}且\end{CJK} $f(1)=0$. \begin{CJK}{UTF8}{mj}证明\end{CJK}:
\end{enumerate}
$$
\left|\int_{0}^{2} f(x) \mathrm{d} x\right| \leqslant \frac{1}{3} M
$$
\begin{CJK}{UTF8}{mj}其中\end{CJK} $M=\max _{0 \leqslant x \leqslant 2}\left|f^{\prime \prime}(x)\right|$.

\begin{enumerate}
  \setcounter{enumi}{5}
  \item ( 20 \begin{CJK}{UTF8}{mj}分\end{CJK}) \begin{CJK}{UTF8}{mj}设\end{CJK} $f$ \begin{CJK}{UTF8}{mj}在\end{CJK} $[a, b]$ \begin{CJK}{UTF8}{mj}上连续\end{CJK}, \begin{CJK}{UTF8}{mj}在\end{CJK} $(a, b)$ \begin{CJK}{UTF8}{mj}可导\end{CJK}, \begin{CJK}{UTF8}{mj}且\end{CJK} $f(a)=f(b)=0$. \begin{CJK}{UTF8}{mj}证明存在\end{CJK} $\xi \in(a, b)$ \begin{CJK}{UTF8}{mj}使得\end{CJK}
\end{enumerate}
$$
f^{\prime}(\xi)-\frac{1}{2} f(\xi)=0
$$

\section{3. 西南大学 2012 年研究生入学考试试题数学分析}
\begin{CJK}{UTF8}{mj}李扬\end{CJK}

\begin{CJK}{UTF8}{mj}微信公众号\end{CJK}: sxkyliyang

\begin{CJK}{UTF8}{mj}一\end{CJK}. \begin{CJK}{UTF8}{mj}选择题\end{CJK} (\begin{CJK}{UTF8}{mj}本题共\end{CJK} 6 \begin{CJK}{UTF8}{mj}小题\end{CJK}, \begin{CJK}{UTF8}{mj}每小题\end{CJK} 5 \begin{CJK}{UTF8}{mj}分\end{CJK}, \begin{CJK}{UTF8}{mj}满分\end{CJK} 30 \begin{CJK}{UTF8}{mj}分\end{CJK})

\begin{enumerate}
  \item \begin{CJK}{UTF8}{mj}设有数列\end{CJK} $\left\{x_{n}\right\}$ \begin{CJK}{UTF8}{mj}及与\end{CJK} $n$ \begin{CJK}{UTF8}{mj}无关的常数\end{CJK} $a$, “\begin{CJK}{UTF8}{mj}对任意给定的\end{CJK} $\varepsilon \in(0,1)$, \begin{CJK}{UTF8}{mj}总存在正整数\end{CJK} $N$, \begin{CJK}{UTF8}{mj}使得当\end{CJK} $n \geqslant N$ \begin{CJK}{UTF8}{mj}时\end{CJK}, \begin{CJK}{UTF8}{mj}恒\end{CJK} \begin{CJK}{UTF8}{mj}有\end{CJK} $\left|x_{n}-a\right| \leqslant 2 \varepsilon$ " \begin{CJK}{UTF8}{mj}是数列\end{CJK} $\left\{x_{n}\right\}$ \begin{CJK}{UTF8}{mj}收敛于\end{CJK} $a$ \begin{CJK}{UTF8}{mj}的\end{CJK} ( ).\\
A. \begin{CJK}{UTF8}{mj}充分条件但非必要条件\end{CJK}.\\
B. \begin{CJK}{UTF8}{mj}必要条件但非充分条件\end{CJK}.\\
C. \begin{CJK}{UTF8}{mj}充分必要条件\end{CJK}.\\
D. \begin{CJK}{UTF8}{mj}既非充分又非必要条件\end{CJK}.

  \item \begin{CJK}{UTF8}{mj}设函数\end{CJK} $f(x)$ \begin{CJK}{UTF8}{mj}在区间\end{CJK} $(a, b)$ \begin{CJK}{UTF8}{mj}内可导\end{CJK}, \begin{CJK}{UTF8}{mj}则导函数\end{CJK} $f^{\prime}(x)$ \begin{CJK}{UTF8}{mj}在\end{CJK} $(a, b)$ \begin{CJK}{UTF8}{mj}内间断点的类型只可能是\end{CJK}\\
A. \begin{CJK}{UTF8}{mj}第一类间断点\end{CJK}.\\
B. \begin{CJK}{UTF8}{mj}第二类间断点\end{CJK}.\\
C. \begin{CJK}{UTF8}{mj}可去间断点\end{CJK}.\\
D. \begin{CJK}{UTF8}{mj}既有第一类也有第二类间断点\end{CJK}

  \item \begin{CJK}{UTF8}{mj}设\end{CJK} $\phi(x)$ \begin{CJK}{UTF8}{mj}在\end{CJK} $a$ \begin{CJK}{UTF8}{mj}点连续\end{CJK}, $f(x)=|x-a| \phi(x)$, \begin{CJK}{UTF8}{mj}则\end{CJK} $f^{\prime}(a)$ \begin{CJK}{UTF8}{mj}存在的条件是\end{CJK} ( ) .\\
A. $\phi(a)=0$\\
B. $\phi(a)=1$\\
C. $\phi(a)=-1$\\
D. $\phi(a)=a$

  \item \begin{CJK}{UTF8}{mj}点集\end{CJK} $S=\left\{(-1)^{n}+\frac{1}{n}, n \in N_{+}\right\}$\begin{CJK}{UTF8}{mj}的聚点是\end{CJK} ( ).\\
A. 0\\
B. 1\\
$\begin{array}{ll}\text { C. }-1 & \text { D. } 1 \text { 和 }-1\end{array}$

  \item $\lim _{n \rightarrow \infty} a_{n}=0$ \begin{CJK}{UTF8}{mj}是级数\end{CJK} $\sum_{n=1}^{\infty} a_{n}$ \begin{CJK}{UTF8}{mj}收敛的\end{CJK} ( )\\
A. \begin{CJK}{UTF8}{mj}充分条件但非必要条件\end{CJK}.\\
B. \begin{CJK}{UTF8}{mj}必要条件但非充分条件\end{CJK}.\\
C. \begin{CJK}{UTF8}{mj}充分必要条件\end{CJK}.\\
D. \begin{CJK}{UTF8}{mj}既非充分又非必要条件\end{CJK}.

  \item \begin{CJK}{UTF8}{mj}幂级数\end{CJK} $\sum_{n=1}^{\infty} \frac{(x-1)^{n}}{2^{n} n}$ \begin{CJK}{UTF8}{mj}的收敛域是\end{CJK} ( $)$.\\
A. $(-2,2)$\\
B. $(-1,3)$\\
C. $[-2,2)$\\
D. $[-1,3)$

\end{enumerate}
\begin{CJK}{UTF8}{mj}计算题\end{CJK} (\begin{CJK}{UTF8}{mj}本题共\end{CJK} 7 \begin{CJK}{UTF8}{mj}小题\end{CJK}, \begin{CJK}{UTF8}{mj}每小题\end{CJK} 10 \begin{CJK}{UTF8}{mj}分\end{CJK}, \begin{CJK}{UTF8}{mj}共\end{CJK} 70 \begin{CJK}{UTF8}{mj}分\end{CJK})

\begin{enumerate}
  \item \begin{CJK}{UTF8}{mj}求极限\end{CJK}
\end{enumerate}
$$
\lim _{x \rightarrow 0}\left(\frac{1}{x}-\frac{1}{e^{x}-1}+\frac{1}{x+1}\right)
$$

\begin{enumerate}
  \setcounter{enumi}{2}
  \item \begin{CJK}{UTF8}{mj}设\end{CJK}
\end{enumerate}
$$
\left\{\begin{array}{l}
x=e^{t} \cos t \\
y=e^{t} \sin t
\end{array}\right.
$$
\begin{CJK}{UTF8}{mj}求\end{CJK} $\frac{\mathrm{d}^{2} y}{\mathrm{~d} x^{2}}$.

\begin{enumerate}
  \setcounter{enumi}{3}
  \item \begin{CJK}{UTF8}{mj}设\end{CJK}
\end{enumerate}
$$
f(x)=\int_{1}^{x^{2}} \frac{\sin t}{t} \mathrm{~d} t
$$
\begin{CJK}{UTF8}{mj}求\end{CJK} $\int_{0}^{1} x f(x) \mathrm{d} x$. 4. \begin{CJK}{UTF8}{mj}设\end{CJK} $z=u^{2} v-u v^{2}, u=x \cos y, v=x \sin y$, \begin{CJK}{UTF8}{mj}求\end{CJK} $\frac{\partial z}{\partial x}$ \begin{CJK}{UTF8}{mj}和\end{CJK} $\frac{\partial z}{\partial y}$.

\begin{enumerate}
  \setcounter{enumi}{5}
  \item \begin{CJK}{UTF8}{mj}将函数\end{CJK} $f(x)=3^{x}$ \begin{CJK}{UTF8}{mj}在\end{CJK} $x_{0}=0$ \begin{CJK}{UTF8}{mj}点处展开成泰勒级数\end{CJK}.

  \item \begin{CJK}{UTF8}{mj}计算二重积分\end{CJK}

\end{enumerate}
$$
\iint_{D} x y \sqrt{x^{2}+y^{2}} \mathrm{~d} x \mathrm{~d} y
$$
\begin{CJK}{UTF8}{mj}其中\end{CJK} $D=\{(x, y) \mid 0 \leqslant x \leqslant 1,0 \leqslant y \leqslant x\}$.

\begin{enumerate}
  \setcounter{enumi}{7}
  \item \begin{CJK}{UTF8}{mj}计算\end{CJK}
\end{enumerate}
$$
\int_{L} x^{2} \mathrm{~d} s .
$$
\begin{CJK}{UTF8}{mj}其中\end{CJK} $L$ \begin{CJK}{UTF8}{mj}为球面\end{CJK} $x^{2}+y^{2}+z^{2}=a^{2}$ \begin{CJK}{UTF8}{mj}被平面\end{CJK} $x+y+z=0$ \begin{CJK}{UTF8}{mj}所截得的圆周\end{CJK}.

\begin{CJK}{UTF8}{mj}三\end{CJK}. \begin{CJK}{UTF8}{mj}证明题\end{CJK} (\begin{CJK}{UTF8}{mj}共\end{CJK} 50 \begin{CJK}{UTF8}{mj}分\end{CJK})

\begin{enumerate}
  \item ( 10 \begin{CJK}{UTF8}{mj}分\end{CJK}) \begin{CJK}{UTF8}{mj}设\end{CJK} $f(x)$ \begin{CJK}{UTF8}{mj}在开区间\end{CJK} $(0,1)$ \begin{CJK}{UTF8}{mj}内可导\end{CJK}, \begin{CJK}{UTF8}{mj}且当\end{CJK} $x \in(0,1)$ \begin{CJK}{UTF8}{mj}时有\end{CJK} $\left|f^{\prime}(x)\right| \leqslant 1$. \begin{CJK}{UTF8}{mj}证明\end{CJK}: \begin{CJK}{UTF8}{mj}数列\end{CJK} $\left\{f\left(\frac{1}{n}\right)\right\}$ \begin{CJK}{UTF8}{mj}收敛\end{CJK}.

  \item (10 \begin{CJK}{UTF8}{mj}分\end{CJK}) \begin{CJK}{UTF8}{mj}设\end{CJK} $f(x)$ \begin{CJK}{UTF8}{mj}在\end{CJK} $[a, b]$ \begin{CJK}{UTF8}{mj}上连续\end{CJK}, \begin{CJK}{UTF8}{mj}且存在数列\end{CJK} $\left\{x_{n}\right\} \subset[a, b]$, \begin{CJK}{UTF8}{mj}使得\end{CJK}

\end{enumerate}
$$
\lim _{n \rightarrow \infty} f\left(x_{n}\right)=A .
$$
\begin{CJK}{UTF8}{mj}证明\end{CJK}: \begin{CJK}{UTF8}{mj}存在\end{CJK} $x_{0} \in[a, b]$, \begin{CJK}{UTF8}{mj}使得\end{CJK} $f\left(x_{0}\right)=A$.

\begin{enumerate}
  \setcounter{enumi}{3}
  \item ( 15 \begin{CJK}{UTF8}{mj}分\end{CJK}) \begin{CJK}{UTF8}{mj}设二元函数\end{CJK} $f$ \begin{CJK}{UTF8}{mj}在\end{CJK} $\mathbb{R}^{2}$ \begin{CJK}{UTF8}{mj}上连续\end{CJK}. \begin{CJK}{UTF8}{mj}若\end{CJK} $\lim _{x^{2}+y^{2} \rightarrow+\infty} f(x, y)=0$. \begin{CJK}{UTF8}{mj}试研究\end{CJK} $f$ \begin{CJK}{UTF8}{mj}在\end{CJK} $\mathbb{R}^{2}$ \begin{CJK}{UTF8}{mj}上的最大值与最小值是\end{CJK} \begin{CJK}{UTF8}{mj}否至少有一个存在\end{CJK}? \begin{CJK}{UTF8}{mj}若至少存在一个\end{CJK}, \begin{CJK}{UTF8}{mj}请给出严格论证\end{CJK}, \begin{CJK}{UTF8}{mj}否则请举例说明\end{CJK}.

  \item ( 15 \begin{CJK}{UTF8}{mj}分\end{CJK}) \begin{CJK}{UTF8}{mj}设函数\end{CJK} $f(x)$ \begin{CJK}{UTF8}{mj}在\end{CJK} $[a, b]$ \begin{CJK}{UTF8}{mj}上连续\end{CJK}, \begin{CJK}{UTF8}{mj}在\end{CJK} $(a, b)$ \begin{CJK}{UTF8}{mj}内可导\end{CJK}. \begin{CJK}{UTF8}{mj}证明\end{CJK}. \begin{CJK}{UTF8}{mj}若\end{CJK} $f(x)$ \begin{CJK}{UTF8}{mj}在\end{CJK} $[a, b]$ \begin{CJK}{UTF8}{mj}上不是线性函数\end{CJK} (\begin{CJK}{UTF8}{mj}常数或\end{CJK} \begin{CJK}{UTF8}{mj}一次代数多项式\end{CJK}), \begin{CJK}{UTF8}{mj}则必存在\end{CJK} $\eta, \tau \in(a, b)$ \begin{CJK}{UTF8}{mj}使得\end{CJK}

\end{enumerate}
$$
f^{\prime}(\eta)<\frac{f(b)-f(a)}{b-a}<f^{\prime}(\tau) .
$$

\section{4. 西南大学 2013 年研究生入学考试试题数学分析}
\begin{CJK}{UTF8}{mj}李扬\end{CJK}

\begin{CJK}{UTF8}{mj}微信公众号\end{CJK}: sxkyliyang

\begin{CJK}{UTF8}{mj}一\end{CJK}. \begin{CJK}{UTF8}{mj}计算题\end{CJK}

\begin{enumerate}
  \item \begin{CJK}{UTF8}{mj}求极限\end{CJK}
\end{enumerate}
$$
\lim _{x \rightarrow 0} \frac{\ln \left(1+x^{2}\right)}{\sec x-\cos x} .
$$
2 . \begin{CJK}{UTF8}{mj}设函数\end{CJK} $y=y(x)$ \begin{CJK}{UTF8}{mj}由方程\end{CJK} $x y=e^{x+y}$ \begin{CJK}{UTF8}{mj}确定\end{CJK}. \begin{CJK}{UTF8}{mj}求\end{CJK} $\frac{\mathrm{d} y}{\mathrm{~d} x}$.

\begin{enumerate}
  \setcounter{enumi}{3}
  \item \begin{CJK}{UTF8}{mj}求\end{CJK} $\int \ln ^{2} x \mathrm{~d} x$.

  \item \begin{CJK}{UTF8}{mj}计算抛物线\end{CJK} $y^{2}=2 x$ \begin{CJK}{UTF8}{mj}与直线\end{CJK} $y=x-4$ \begin{CJK}{UTF8}{mj}所围成的图形的面积\end{CJK}.

\end{enumerate}
5 . \begin{CJK}{UTF8}{mj}设\end{CJK} $z=e^{x y^{2}}, x=t \cos t, y=t \sin t$, \begin{CJK}{UTF8}{mj}求\end{CJK} $\left.\frac{\mathrm{d} z}{\mathrm{~d} t}\right|_{t=\frac{\pi}{2}}$.

\begin{enumerate}
  \setcounter{enumi}{6}
  \item \begin{CJK}{UTF8}{mj}求幂级数\end{CJK}
\end{enumerate}
$$
\sum_{n=1}^{\infty}\left(1+\frac{1}{n}\right)^{n^{2}} x^{n}
$$
\begin{CJK}{UTF8}{mj}的收敛域\end{CJK}.

\begin{enumerate}
  \setcounter{enumi}{7}
  \item \begin{CJK}{UTF8}{mj}计算曲线积分\end{CJK} $\int_{L} y^{2} \mathrm{~d} s$. \begin{CJK}{UTF8}{mj}其中\end{CJK} $L$ \begin{CJK}{UTF8}{mj}为摆线\end{CJK} $x=a(t-\sin t), y=a(1-\cos t)(a>0)$ \begin{CJK}{UTF8}{mj}在\end{CJK} $t \in[0, \pi]$ \begin{CJK}{UTF8}{mj}间的一段\end{CJK}.

  \item \begin{CJK}{UTF8}{mj}计算二重积分\end{CJK}

\end{enumerate}
\includegraphics[max width=\textwidth]{2022_04_18_3416d289b173eb9de8c1g-043}

\begin{CJK}{UTF8}{mj}其中\end{CJK} $D=\left\{(x, y) \mid 0 \leqslant x \leqslant \frac{\pi}{2}, 0 \leqslant y \leqslant 1\right\}$.

\begin{CJK}{UTF8}{mj}三\end{CJK}. \begin{CJK}{UTF8}{mj}证明题\end{CJK}

\begin{enumerate}
  \item \begin{CJK}{UTF8}{mj}设\end{CJK} $c>0,0<x_{1}<\frac{1}{c}, x_{n+1}=x_{n}\left(2-c x_{n}\right), n=1,2, \cdots$. \begin{CJK}{UTF8}{mj}证明\end{CJK}: \begin{CJK}{UTF8}{mj}数列\end{CJK} $\left\{x_{n}\right\}$ \begin{CJK}{UTF8}{mj}收敛并求极限\end{CJK}.

  \item \begin{CJK}{UTF8}{mj}证明\end{CJK}: \begin{CJK}{UTF8}{mj}方程\end{CJK}

\end{enumerate}
$$
2 x^{5}+3 x^{3}+x-4=0
$$
\begin{CJK}{UTF8}{mj}有且只有一个肯根\end{CJK}.

\begin{enumerate}
  \setcounter{enumi}{3}
  \item \begin{CJK}{UTF8}{mj}证明\end{CJK}: \begin{CJK}{UTF8}{mj}若\end{CJK} $\sum_{n=1}^{\infty} a_{n}$ \begin{CJK}{UTF8}{mj}绝对收敛\end{CJK}, \begin{CJK}{UTF8}{mj}则\end{CJK} $\sum_{n=1}^{\infty} a_{n}\left(a_{1}+a_{2}+\cdots+a_{n}\right)$ \begin{CJK}{UTF8}{mj}也必绝对收敛\end{CJK}.

  \item (1) \begin{CJK}{UTF8}{mj}试举例说明\end{CJK}: \begin{CJK}{UTF8}{mj}即使二元函数在某一点存在对所有变量的偏导数\end{CJK}, \begin{CJK}{UTF8}{mj}也不能保证函数在该点连续\end{CJK}.

\end{enumerate}
(2) \begin{CJK}{UTF8}{mj}设\end{CJK} $f(x, y, z)$ \begin{CJK}{UTF8}{mj}在\end{CJK} $D=\left\{(x, y, z) \mid x^{2}+y^{2}+z^{2}<1\right\}$ \begin{CJK}{UTF8}{mj}内有定义\end{CJK}. \begin{CJK}{UTF8}{mj}若\end{CJK} $f(x, y, z)$ \begin{CJK}{UTF8}{mj}关于变量\end{CJK} $z$ \begin{CJK}{UTF8}{mj}是连续的\end{CJK}, \begin{CJK}{UTF8}{mj}并且\end{CJK} \begin{CJK}{UTF8}{mj}对\end{CJK} $\forall(x, y, z) \in D$, \begin{CJK}{UTF8}{mj}满足\end{CJK}
$$
\left|f_{x}(x, y, z)\right| \leqslant 1,\left|f_{y}(x, y, z)\right| \leqslant 1
$$
\begin{CJK}{UTF8}{mj}证明\end{CJK}: \begin{CJK}{UTF8}{mj}函数\end{CJK} $f(x, y, z)$ \begin{CJK}{UTF8}{mj}在区域\end{CJK} $D$ \begin{CJK}{UTF8}{mj}内连续\end{CJK}.

\section{5. 西南大学 2014 年研究生入学考试试题数学分析}
\begin{CJK}{UTF8}{mj}李扬\end{CJK}

\begin{CJK}{UTF8}{mj}微信公众号\end{CJK}: sxkyliyang

\section{一. 填空题}
\begin{enumerate}
  \item $\lim _{x \rightarrow 0} \frac{(1+x)^{\frac{1}{x}}-e}{x}=$

  \item \begin{CJK}{UTF8}{mj}设\end{CJK} $\left\{\begin{array}{l}x=e^{t} \cos t ; \\ y=e^{t} \sin t\end{array}\right.$ \begin{CJK}{UTF8}{mj}则\end{CJK} $\frac{\mathrm{d}^{2} y}{\mathrm{~d} x^{2}}=$

  \item \begin{CJK}{UTF8}{mj}设\end{CJK} $f(\ln x)=\frac{\ln (1+x)}{x}$, \begin{CJK}{UTF8}{mj}则\end{CJK} $\int f(x) \mathrm{d} x=$

  \item \begin{CJK}{UTF8}{mj}设\end{CJK} $f(x, y, z)=x^{2} y+x z^{3}$, \begin{CJK}{UTF8}{mj}则\end{CJK} $f$ \begin{CJK}{UTF8}{mj}在点\end{CJK} $(1,-1,2)$ \begin{CJK}{UTF8}{mj}的梯度为\end{CJK}

  \item \begin{CJK}{UTF8}{mj}设\end{CJK} $L$ \begin{CJK}{UTF8}{mj}为从点\end{CJK} $(1,1,1)$ \begin{CJK}{UTF8}{mj}到点\end{CJK} $(2,3,4)$ \begin{CJK}{UTF8}{mj}的直线段\end{CJK}, \begin{CJK}{UTF8}{mj}则\end{CJK} $\int_{L} x \mathrm{~d} x+y \mathrm{~d} y+z \mathrm{~d} z=$

  \item \begin{CJK}{UTF8}{mj}设\end{CJK} $D=\left\{(x, y) \mid x^{2}+y^{2} \leqslant 1\right\}$, \begin{CJK}{UTF8}{mj}则\end{CJK} $\iint_{D} \frac{\mathrm{d} \sigma}{\sqrt{1-x^{2}-y^{2}}}=$

  \item \begin{CJK}{UTF8}{mj}设\end{CJK} $S$ \begin{CJK}{UTF8}{mj}是球面\end{CJK} $x^{2}+y^{2}+z^{2}=1$ \begin{CJK}{UTF8}{mj}在\end{CJK} $x \geqslant 0, y \geqslant 0$ \begin{CJK}{UTF8}{mj}的部分并取球面外侧\end{CJK}, \begin{CJK}{UTF8}{mj}则\end{CJK} $\iint_{D} x y z \mathrm{~d} x \mathrm{~d} y=$

\end{enumerate}
\section{二.证明题}
\begin{enumerate}
  \item \begin{CJK}{UTF8}{mj}设\end{CJK} $S$ \begin{CJK}{UTF8}{mj}为有界数集\end{CJK}, \begin{CJK}{UTF8}{mj}证明\end{CJK}: \begin{CJK}{UTF8}{mj}若\end{CJK} $\sup S=a \notin S$, \begin{CJK}{UTF8}{mj}则存在严格递增数列\end{CJK} $\left\{x_{n}\right\} \subset S$, \begin{CJK}{UTF8}{mj}使得\end{CJK}

  \item \begin{CJK}{UTF8}{mj}证明\end{CJK}: \begin{CJK}{UTF8}{mj}函数\end{CJK} $f(x)=x^{2}$ \begin{CJK}{UTF8}{mj}在\end{CJK} $(-\infty,+\infty)$ \begin{CJK}{UTF8}{mj}上不一致连续\end{CJK}.

  \item \begin{CJK}{UTF8}{mj}设函数\end{CJK} $f(x)$ \begin{CJK}{UTF8}{mj}在\end{CJK} $x=0$ \begin{CJK}{UTF8}{mj}的某邻域内具有二阶连续导数\end{CJK}, \begin{CJK}{UTF8}{mj}且\end{CJK} $\lim _{x \rightarrow 0} \frac{f(x)}{x}=0$, \begin{CJK}{UTF8}{mj}证明\end{CJK}: \begin{CJK}{UTF8}{mj}级数\end{CJK}

\end{enumerate}
$$
\sum_{n=1}^{\infty} \sqrt{n} f\left(\frac{1}{n}\right)
$$
\begin{CJK}{UTF8}{mj}绝对收敛\end{CJK}.

\begin{enumerate}
  \setcounter{enumi}{4}
  \item \begin{CJK}{UTF8}{mj}设函数\end{CJK} $f(x)$ \begin{CJK}{UTF8}{mj}在\end{CJK} $(-1,1)$ \begin{CJK}{UTF8}{mj}内是二阶连续导数且\end{CJK} $f^{\prime \prime}(x) \neq 0$, \begin{CJK}{UTF8}{mj}证明\end{CJK}:
\end{enumerate}
(1) \begin{CJK}{UTF8}{mj}对于\end{CJK} $(-1,1)$ \begin{CJK}{UTF8}{mj}内任一\end{CJK} $x \neq 0$, \begin{CJK}{UTF8}{mj}存在唯一的\end{CJK} $\theta(x) \in[0,1)$ \begin{CJK}{UTF8}{mj}使得\end{CJK}
$$
f(x)=f(0)+x f^{\prime}(\theta(x) x)
$$
\begin{CJK}{UTF8}{mj}成立\end{CJK}.

(2) $\lim _{x \rightarrow 0} \theta(x)=\frac{1}{2}$.

\section{6. 西南大学 2015 年研究生入学考试试题数学分析}
\begin{CJK}{UTF8}{mj}李扬\end{CJK}

\begin{CJK}{UTF8}{mj}微信公众号\end{CJK}: sxkyliyang

\begin{CJK}{UTF8}{mj}一\end{CJK}. \begin{CJK}{UTF8}{mj}计算题\end{CJK} (\begin{CJK}{UTF8}{mj}本大题共\end{CJK} 7 \begin{CJK}{UTF8}{mj}小题\end{CJK}, \begin{CJK}{UTF8}{mj}每小题\end{CJK} 10 \begin{CJK}{UTF8}{mj}分\end{CJK}, \begin{CJK}{UTF8}{mj}共\end{CJK} 70 \begin{CJK}{UTF8}{mj}分\end{CJK})

\begin{enumerate}
  \item \begin{CJK}{UTF8}{mj}求极限\end{CJK} $\lim _{n \rightarrow \infty}\left(1-\frac{1}{2^{2}}\right)\left(1-\frac{1}{3^{2}}\right) \cdots\left(1-\frac{1}{n^{2}}\right)$.

  \item \begin{CJK}{UTF8}{mj}求极限\end{CJK} $\lim _{x \rightarrow 0}\left(\frac{\sin x}{x}\right)^{\frac{1}{x^{2}}}$.

  \item \begin{CJK}{UTF8}{mj}求\end{CJK} $\int_{-\frac{\pi}{2}}^{\frac{\pi}{2}}\left(x^{5}+2 x^{3}+3 x+1\right) \cos ^{2} x \mathrm{~d} x$.

  \item \begin{CJK}{UTF8}{mj}求曲面\end{CJK} $z=\arctan \frac{y}{x}$ \begin{CJK}{UTF8}{mj}在\end{CJK} $\left(1,1, \frac{\pi}{4}\right)$ \begin{CJK}{UTF8}{mj}的切平面方程及法线方程\end{CJK}.

  \item \begin{CJK}{UTF8}{mj}设\end{CJK} $F(x)=\int_{x}^{x^{2}} e^{-x y^{2}} \mathrm{~d} y$, \begin{CJK}{UTF8}{mj}求\end{CJK} $F^{\prime}(0)$.

  \item \begin{CJK}{UTF8}{mj}计算曲面积分\end{CJK} $\iint_{x}(x+y+z) \mathrm{d} S$, \begin{CJK}{UTF8}{mj}其中\end{CJK} $S$ \begin{CJK}{UTF8}{mj}为上半球面\end{CJK} $x^{2}+y^{2}+z^{2}=a^{2}, z \geqslant 0$.

  \item \begin{CJK}{UTF8}{mj}计算二重积分\end{CJK} $\iint_{D} x^{2} e^{-y^{2}} \mathrm{~d} \sigma$, \begin{CJK}{UTF8}{mj}其中\end{CJK} $D$ \begin{CJK}{UTF8}{mj}是由直线\end{CJK} $x=0, y=1, y=x$ \begin{CJK}{UTF8}{mj}围成的区域\end{CJK}.

\end{enumerate}
\begin{CJK}{UTF8}{mj}二\end{CJK}. \begin{CJK}{UTF8}{mj}证明题\end{CJK} (\begin{CJK}{UTF8}{mj}本大题共\end{CJK} 4 \begin{CJK}{UTF8}{mj}小题\end{CJK}, \begin{CJK}{UTF8}{mj}每小题\end{CJK} 20 \begin{CJK}{UTF8}{mj}分\end{CJK}, \begin{CJK}{UTF8}{mj}共\end{CJK} 80 \begin{CJK}{UTF8}{mj}分\end{CJK})

\begin{enumerate}
  \item \begin{CJK}{UTF8}{mj}设\end{CJK} $0 \leqslant c \leqslant 1, a_{1}=\frac{c}{2}, a_{n+1}=\frac{c}{2}+\frac{a_{n}^{2}}{2}, n=1,2, \cdots$. \begin{CJK}{UTF8}{mj}证明数列\end{CJK} $\left\{a_{n}\right\}$ \begin{CJK}{UTF8}{mj}收敛\end{CJK}, \begin{CJK}{UTF8}{mj}并求其极限\end{CJK}.

  \item \begin{CJK}{UTF8}{mj}设\end{CJK} $f(x)$ \begin{CJK}{UTF8}{mj}为闭区间\end{CJK} $[0, a]$ \begin{CJK}{UTF8}{mj}上连续\end{CJK} $(a>0)$, \begin{CJK}{UTF8}{mj}且\end{CJK} $f(0)=f(a)$. \begin{CJK}{UTF8}{mj}证明\end{CJK}: \begin{CJK}{UTF8}{mj}对任意自然数\end{CJK} $n$, \begin{CJK}{UTF8}{mj}至少存在一点\end{CJK} $\xi \in\left[\frac{a}{n}, a\right]$, \begin{CJK}{UTF8}{mj}使得\end{CJK} $f(\xi)=f\left(\xi-\frac{a}{n}\right)$.

  \item \begin{CJK}{UTF8}{mj}已知\end{CJK} $a_{n}>0$, \begin{CJK}{UTF8}{mj}级数\end{CJK} $\sum_{n=1}^{\infty} \frac{1}{a_{n}}$ \begin{CJK}{UTF8}{mj}发散\end{CJK}. \begin{CJK}{UTF8}{mj}证明\end{CJK}: \begin{CJK}{UTF8}{mj}级数\end{CJK} $\sum_{n=1}^{\infty} \frac{1}{a_{n}+1}$ \begin{CJK}{UTF8}{mj}发散\end{CJK}.

  \item \begin{CJK}{UTF8}{mj}设\end{CJK} $f(x, y)$ \begin{CJK}{UTF8}{mj}在开区域\end{CJK} $D \subset R^{2}$ \begin{CJK}{UTF8}{mj}上对\end{CJK} $x$ \begin{CJK}{UTF8}{mj}连续\end{CJK}, \begin{CJK}{UTF8}{mj}对\end{CJK} $y$ \begin{CJK}{UTF8}{mj}满足利普希茨条件\end{CJK}:

\end{enumerate}
$$
\left|f\left(x, y^{\prime}\right)-f\left(x, y^{\prime \prime}\right)\right| \leqslant L\left|y^{\prime}-y^{\prime \prime}\right|,
$$
\begin{CJK}{UTF8}{mj}其中\end{CJK} $\left(x, y^{\prime}\right),\left(x, y^{\prime \prime}\right) \in D, L$ \begin{CJK}{UTF8}{mj}为正常数\end{CJK}. \begin{CJK}{UTF8}{mj}证明\end{CJK}: $f$ \begin{CJK}{UTF8}{mj}在\end{CJK} $D$ \begin{CJK}{UTF8}{mj}上处处连续\end{CJK}.

\section{7. 西南大学 2016 年研究生入学考试试题数学分析}
\begin{CJK}{UTF8}{mj}李扬\end{CJK}

\begin{CJK}{UTF8}{mj}微信公众号\end{CJK}: sxkyliyang

\begin{CJK}{UTF8}{mj}一\end{CJK}. \begin{CJK}{UTF8}{mj}叙述题\end{CJK} $(10 \times 2=20$ \begin{CJK}{UTF8}{mj}分\end{CJK} $)$

\begin{enumerate}
  \item \begin{CJK}{UTF8}{mj}叙述\end{CJK} $x \rightarrow x_{0}$ \begin{CJK}{UTF8}{mj}的函数极限的归结原则\end{CJK}.

  \item \begin{CJK}{UTF8}{mj}函数\end{CJK} $f(x)$ \begin{CJK}{UTF8}{mj}在区间\end{CJK} $I$ \begin{CJK}{UTF8}{mj}上不一致收敛\end{CJK}.

\end{enumerate}
\begin{CJK}{UTF8}{mj}二\end{CJK}. \begin{CJK}{UTF8}{mj}计算\end{CJK} $(10 \times 7=70$ \begin{CJK}{UTF8}{mj}分\end{CJK} $)$

\begin{enumerate}
  \item \begin{CJK}{UTF8}{mj}求\end{CJK} $\lim _{n \rightarrow \infty}\left(\frac{1}{n^{2}+n+1}+\frac{2}{n^{2}+n+2}+\cdots+\frac{n}{n^{2}+n+n}\right)$;

  \item \begin{CJK}{UTF8}{mj}已知\end{CJK} $\lim _{x \rightarrow 0} \frac{\sqrt{1+f(x) \sin x}-1}{e^{3 x}-1}=2$, \begin{CJK}{UTF8}{mj}求\end{CJK} $\lim _{x \rightarrow 0} f(x)$ \begin{CJK}{UTF8}{mj}的值\end{CJK};

  \item \begin{CJK}{UTF8}{mj}计算不定积分\end{CJK} $\int x \arctan x \ln \left(1+x^{2}\right) \mathrm{d} x$;

  \item \begin{CJK}{UTF8}{mj}设\end{CJK} $z=f\left(x, \frac{x}{y}\right)$, \begin{CJK}{UTF8}{mj}求\end{CJK} $\frac{\partial^{2} z}{\partial x \partial y}$;

  \item \begin{CJK}{UTF8}{mj}求幂级数\end{CJK} $\sum_{n=1}^{\infty}(-1)^{n} \frac{x^{2 n+1}}{4 n^{2}-1}$ \begin{CJK}{UTF8}{mj}的收敛域与和函数\end{CJK};

  \item \begin{CJK}{UTF8}{mj}设\end{CJK} $f(x)$ \begin{CJK}{UTF8}{mj}在\end{CJK} $[0,1]$ \begin{CJK}{UTF8}{mj}上连续\end{CJK}, \begin{CJK}{UTF8}{mj}且\end{CJK} $\int_{0}^{1} f(x) d x=3$, \begin{CJK}{UTF8}{mj}求\end{CJK} $\int_{0}^{1} d x \int_{x}^{1} f(x) f(y) d y$.

  \item \begin{CJK}{UTF8}{mj}设\end{CJK} $S$ \begin{CJK}{UTF8}{mj}为椭球面\end{CJK} $\frac{x^{2}}{2}+\frac{y^{2}}{2}+z^{2}=1$ \begin{CJK}{UTF8}{mj}的上半部分\end{CJK}, $\pi$ \begin{CJK}{UTF8}{mj}为\end{CJK} $S$ \begin{CJK}{UTF8}{mj}在点\end{CJK} $P(x, y, z) \in S$ \begin{CJK}{UTF8}{mj}的切平面\end{CJK}, $\rho(x, y, z)$ \begin{CJK}{UTF8}{mj}为坐标原\end{CJK} \begin{CJK}{UTF8}{mj}点\end{CJK} $O$ \begin{CJK}{UTF8}{mj}到平面的距离\end{CJK}, \begin{CJK}{UTF8}{mj}求\end{CJK} $\iint_{S} \frac{z}{\rho(x, y, z)} \mathrm{d} S$ \begin{CJK}{UTF8}{mj}的值\end{CJK}.

\end{enumerate}
\begin{CJK}{UTF8}{mj}三\end{CJK}. \begin{CJK}{UTF8}{mj}证明\end{CJK} $(15 \times 4=60$ \begin{CJK}{UTF8}{mj}分\end{CJK} $)$

\begin{enumerate}
  \item \begin{CJK}{UTF8}{mj}叙述数列极限存在的柯西收敛准则\end{CJK}, \begin{CJK}{UTF8}{mj}并用其证明\end{CJK} $a_{n}=1+\frac{1}{2^{2}}+\cdots+\frac{1}{n^{2}}$ \begin{CJK}{UTF8}{mj}收敛\end{CJK}.

  \item \begin{CJK}{UTF8}{mj}设函数\end{CJK} $f(x)$ \begin{CJK}{UTF8}{mj}在\end{CJK} $[0,+\infty)$ \begin{CJK}{UTF8}{mj}上可导\end{CJK}, \begin{CJK}{UTF8}{mj}且\end{CJK} $f^{\prime}(x) \geqslant k>0, f(0)<0$, \begin{CJK}{UTF8}{mj}证明\end{CJK}: $f(x)$ \begin{CJK}{UTF8}{mj}在\end{CJK} $(0,+\infty)$ \begin{CJK}{UTF8}{mj}内有且只有一个零\end{CJK} \begin{CJK}{UTF8}{mj}点\end{CJK}.

  \item \begin{CJK}{UTF8}{mj}设\end{CJK} $f(x)$ \begin{CJK}{UTF8}{mj}在\end{CJK} $(-\infty,+\infty)$ \begin{CJK}{UTF8}{mj}内可微\end{CJK}, $f(x)>0$, \begin{CJK}{UTF8}{mj}且存在\end{CJK} $0<\lambda<1$ \begin{CJK}{UTF8}{mj}使得\end{CJK} $\left|f^{\prime}(x)\right| \leqslant \lambda f(x)$. \begin{CJK}{UTF8}{mj}任取\end{CJK} $a_{0} \in \mathbb{R}$, \begin{CJK}{UTF8}{mj}定义\end{CJK} $a_{n}=\ln f\left(a_{n-1}\right), n=1,2, \cdots$, \begin{CJK}{UTF8}{mj}证明\end{CJK}: \begin{CJK}{UTF8}{mj}级数\end{CJK} $\sum_{n=1}^{\infty}\left|a_{n}-a_{n-1}\right|$ \begin{CJK}{UTF8}{mj}收敛\end{CJK}.

  \item (1) \begin{CJK}{UTF8}{mj}试举例说明\end{CJK}: \begin{CJK}{UTF8}{mj}即使二元函数在某一点存在对所有变量的偏导数\end{CJK}, \begin{CJK}{UTF8}{mj}也不能保证该函数在该点可微\end{CJK}.

\end{enumerate}
(2) \begin{CJK}{UTF8}{mj}设函数\end{CJK} $z=f(x, y)$ \begin{CJK}{UTF8}{mj}的偏导在\end{CJK} $\left(x_{0}, y_{0}\right)$ \begin{CJK}{UTF8}{mj}某邻域内存在\end{CJK}, \begin{CJK}{UTF8}{mj}且\end{CJK} $f_{x}$ \begin{CJK}{UTF8}{mj}与\end{CJK} $f_{y}$ \begin{CJK}{UTF8}{mj}都在点\end{CJK} $\left(x_{0}, y_{0}\right)$ \begin{CJK}{UTF8}{mj}处连续\end{CJK}. \begin{CJK}{UTF8}{mj}证明\end{CJK}: \begin{CJK}{UTF8}{mj}函数\end{CJK} $f$ \begin{CJK}{UTF8}{mj}在\end{CJK} $\left(x_{0}, y_{0}\right)$ \begin{CJK}{UTF8}{mj}可微\end{CJK}.

\section{8. 西南大学 2017 年研究生入学考试试题数学分析}
\begin{CJK}{UTF8}{mj}李扬\end{CJK}

\begin{CJK}{UTF8}{mj}微信公众号\end{CJK}: sxkyliyang

\begin{CJK}{UTF8}{mj}一\end{CJK}. \begin{CJK}{UTF8}{mj}计算题\end{CJK}(\begin{CJK}{UTF8}{mj}每题\end{CJK} 10 \begin{CJK}{UTF8}{mj}分\end{CJK}, \begin{CJK}{UTF8}{mj}共\end{CJK} 9 \begin{CJK}{UTF8}{mj}题\end{CJK})

\begin{enumerate}
  \item \begin{CJK}{UTF8}{mj}求数列极限\end{CJK} $\lim _{n \rightarrow \infty} \frac{a^{2 n}}{1+a^{2 n}}$.

  \item \begin{CJK}{UTF8}{mj}求函数极限\end{CJK} $\lim _{x \rightarrow 1} \frac{\sqrt{x}-e^{\frac{x-1}{2}}}{\ln ^{2}(2 x-1)}$.

  \item \begin{CJK}{UTF8}{mj}计算\end{CJK} $\int_{0}^{\frac{\pi}{2}}(\sqrt{\tan x}+\sqrt{\cot x}) \mathrm{d} x$.

  \item $z=z(x, y)$ \begin{CJK}{UTF8}{mj}是由方程\end{CJK} $F\left(x+\frac{z}{y}, y+\frac{z}{x}\right)=0$ \begin{CJK}{UTF8}{mj}确定的隐函数\end{CJK}, \begin{CJK}{UTF8}{mj}求\end{CJK} $x \frac{\partial z}{\partial x}+y \frac{\partial z}{\partial y}$.

  \item \begin{CJK}{UTF8}{mj}求\end{CJK} $f(x)=a x-\ln x$ \begin{CJK}{UTF8}{mj}在\end{CJK} $(0,+\infty)$ \begin{CJK}{UTF8}{mj}内的极值\end{CJK}, \begin{CJK}{UTF8}{mj}并讨论方程\end{CJK} $a x=\ln x$ \begin{CJK}{UTF8}{mj}有两个正实根的条件\end{CJK}.

  \item \begin{CJK}{UTF8}{mj}计算不定积分\end{CJK} $\int\left[\ln (\ln x)+\frac{1}{\ln x}\right] \mathrm{d} x$.

  \item \begin{CJK}{UTF8}{mj}求曲线\end{CJK} $y=\frac{4}{x}$, \begin{CJK}{UTF8}{mj}直线\end{CJK} $y=x$ \begin{CJK}{UTF8}{mj}及\end{CJK} $y=4 x$ \begin{CJK}{UTF8}{mj}在第一象限中所围图形的面积\end{CJK}

  \item \begin{CJK}{UTF8}{mj}计算\end{CJK} $\int_{L} e^{x}[(1-\cos y) \mathrm{d} x-(y-\sin y) \mathrm{d} y]$, \begin{CJK}{UTF8}{mj}其中\end{CJK} $L$ \begin{CJK}{UTF8}{mj}是曲线\end{CJK} $y=\sin x$ \begin{CJK}{UTF8}{mj}从\end{CJK} $(0,0)$ \begin{CJK}{UTF8}{mj}到\end{CJK} $(\pi, 0)$ \begin{CJK}{UTF8}{mj}之间的部分\end{CJK}.

  \item \begin{CJK}{UTF8}{mj}计算\end{CJK} $\iiint_{V} y \sqrt{16-z^{2}} \mathrm{~d} x \mathrm{~d} y \mathrm{~d} z$, \begin{CJK}{UTF8}{mj}其中\end{CJK} $v$ \begin{CJK}{UTF8}{mj}是\end{CJK} $z=y^{2}, z=2 y^{2}, z=x, z=2 x$ \begin{CJK}{UTF8}{mj}和\end{CJK} $z=4$ \begin{CJK}{UTF8}{mj}围成的区域\end{CJK}.

\end{enumerate}
\begin{CJK}{UTF8}{mj}二\end{CJK}. \begin{CJK}{UTF8}{mj}证明题\end{CJK}(\begin{CJK}{UTF8}{mj}每题\end{CJK} 15 \begin{CJK}{UTF8}{mj}分\end{CJK}, \begin{CJK}{UTF8}{mj}共\end{CJK} 4 \begin{CJK}{UTF8}{mj}题\end{CJK})

\begin{enumerate}
  \item \begin{CJK}{UTF8}{mj}设\end{CJK} $x_{0}=0, x_{n+1}=\frac{2 x_{n}+3}{x_{n}+3}(n \geqslant 0)$, \begin{CJK}{UTF8}{mj}证明\end{CJK}: \begin{CJK}{UTF8}{mj}数列\end{CJK} $\left\{x_{n}\right\}$ \begin{CJK}{UTF8}{mj}收敛并求其极限\end{CJK}.

  \item \begin{CJK}{UTF8}{mj}设\end{CJK} $f(x)$ \begin{CJK}{UTF8}{mj}在\end{CJK} $[0,1]$ \begin{CJK}{UTF8}{mj}上连续可导\end{CJK}, \begin{CJK}{UTF8}{mj}证明\end{CJK}: $\lim _{n \rightarrow \infty} \int_{0}^{1} x^{n} f(x) \mathrm{d} x=f(1)$.

  \item \begin{CJK}{UTF8}{mj}证明函数列\end{CJK} $S_{n}(x)=(1-x) x^{n}$ \begin{CJK}{UTF8}{mj}在闭区间\end{CJK} $[0,1]$ \begin{CJK}{UTF8}{mj}上一致收敛\end{CJK}.

  \item \begin{CJK}{UTF8}{mj}设函数\end{CJK} $f$ \begin{CJK}{UTF8}{mj}在\end{CJK} $[a, b]$ \begin{CJK}{UTF8}{mj}上连续\end{CJK}, \begin{CJK}{UTF8}{mj}在\end{CJK} $(a, b)$ \begin{CJK}{UTF8}{mj}内可导\end{CJK}, $f(a)>f(b), c=\frac{f(b)-f(a)}{b-a}$. \begin{CJK}{UTF8}{mj}证明\end{CJK}: $f$ \begin{CJK}{UTF8}{mj}必须满足下列两个\end{CJK} \begin{CJK}{UTF8}{mj}性质中的一个\end{CJK}

\end{enumerate}
(1) \begin{CJK}{UTF8}{mj}对任意的\end{CJK} $x \in[a, b]$ \begin{CJK}{UTF8}{mj}都有\end{CJK} $f(x)-f(a)=c(x-a)$;

(2) \begin{CJK}{UTF8}{mj}存在\end{CJK} $\xi \in(a, b)$, \begin{CJK}{UTF8}{mj}使得\end{CJK} $f^{\prime}(\xi)>c$.

\section{1. 湘潭大学 2009 年研究生入学考试试题高等代数}
\begin{CJK}{UTF8}{mj}李扬\end{CJK}

\begin{CJK}{UTF8}{mj}微信公众号\end{CJK}: sxkyliyang

\begin{CJK}{UTF8}{mj}本试题共\end{CJK} 10 \begin{CJK}{UTF8}{mj}大题\end{CJK}, \begin{CJK}{UTF8}{mj}每个大题满分均为\end{CJK} 15 \begin{CJK}{UTF8}{mj}分\end{CJK}, \begin{CJK}{UTF8}{mj}总分共\end{CJK} 150 \begin{CJK}{UTF8}{mj}分\end{CJK}.

\begin{CJK}{UTF8}{mj}一\end{CJK}. \begin{CJK}{UTF8}{mj}计算\end{CJK} $n$ \begin{CJK}{UTF8}{mj}阶行列式\end{CJK} $D_{n}=\left|a_{i j}\right|$, \begin{CJK}{UTF8}{mj}其中\end{CJK} $\left|a_{i j}\right|=|i-j|$.

\begin{CJK}{UTF8}{mj}二\end{CJK}. \begin{CJK}{UTF8}{mj}设齐次线性方程组\end{CJK}
$$
\left\{\begin{array}{c}
a x_{1}+b x_{2}+\cdots+b x_{n}=0 \\
b x_{1}+a x_{2}+\cdots+b x_{n}=0 \\
\vdots \\
b x_{1}+b x_{2}+\cdots+a x_{n}=0
\end{array}\right.
$$
\begin{CJK}{UTF8}{mj}试讨论\end{CJK} $a, b$ \begin{CJK}{UTF8}{mj}为何值时\end{CJK}, \begin{CJK}{UTF8}{mj}方程组仅有零解\end{CJK}, \begin{CJK}{UTF8}{mj}有无穷多解\end{CJK}? \begin{CJK}{UTF8}{mj}在有无穷多解时\end{CJK}, \begin{CJK}{UTF8}{mj}求通解\end{CJK}.

\begin{CJK}{UTF8}{mj}三\end{CJK}. \begin{CJK}{UTF8}{mj}已知\end{CJK} $P^{2 \times 2}$ \begin{CJK}{UTF8}{mj}的子空间\end{CJK}
$$
W=\left\{\left(\begin{array}{ll}
x_{11} & x_{12} \\
x_{21} & x_{22}
\end{array}\right) \mid x_{11}+x_{22}=0, x_{i j} \in P, i, j=1,2\right\}
$$
\begin{CJK}{UTF8}{mj}和线性变换\end{CJK}
$$
\mathscr{A}(X)=B^{\prime} X-X^{\prime} B, \forall X \in P^{2 \times 2}, B=\left(\begin{array}{ll}
1 & 1 \\
0 & 1
\end{array}\right),
$$
$B^{\prime}$ \begin{CJK}{UTF8}{mj}和\end{CJK} $X^{\prime}$ \begin{CJK}{UTF8}{mj}分别表示\end{CJK} $B$ \begin{CJK}{UTF8}{mj}和\end{CJK} $X$ \begin{CJK}{UTF8}{mj}的转置\end{CJK}.

(1) \begin{CJK}{UTF8}{mj}求\end{CJK} $W$ \begin{CJK}{UTF8}{mj}的维数和一组基\end{CJK}.

(2) \begin{CJK}{UTF8}{mj}证明\end{CJK} $W$ \begin{CJK}{UTF8}{mj}是\end{CJK} $\mathscr{A}$ \begin{CJK}{UTF8}{mj}的不变子空间\end{CJK}.

\begin{CJK}{UTF8}{mj}四\end{CJK}. \begin{CJK}{UTF8}{mj}求矩阵\end{CJK}
$$
A=\left(\begin{array}{ccc}
1 & 2 & 0 \\
0 & 2 & 0 \\
-2 & -2 & -1
\end{array}\right)
$$
\begin{CJK}{UTF8}{mj}的行列式因子\end{CJK}, \begin{CJK}{UTF8}{mj}不变因子\end{CJK}, \begin{CJK}{UTF8}{mj}最小多项式\end{CJK}, \begin{CJK}{UTF8}{mj}初等因子\end{CJK}, \begin{CJK}{UTF8}{mj}若当标准形\end{CJK}.

\begin{CJK}{UTF8}{mj}五\end{CJK}. \begin{CJK}{UTF8}{mj}设\end{CJK} $A$ \begin{CJK}{UTF8}{mj}是数域\end{CJK} $P$ \begin{CJK}{UTF8}{mj}上的\end{CJK} $n(>0)$ \begin{CJK}{UTF8}{mj}阶矩阵\end{CJK}, $A^{2}=A, W_{1}$ \begin{CJK}{UTF8}{mj}为方程组\end{CJK} $A X=0$ \begin{CJK}{UTF8}{mj}的解空间\end{CJK}, $W_{2}$ \begin{CJK}{UTF8}{mj}为方程组\end{CJK} $(A-E) X=0$ \begin{CJK}{UTF8}{mj}的解空间\end{CJK}. \begin{CJK}{UTF8}{mj}证明\end{CJK}:
$$
P^{n}=W_{1} \oplus W_{2}
$$
\begin{CJK}{UTF8}{mj}六\end{CJK}. \begin{CJK}{UTF8}{mj}证明\end{CJK}: \begin{CJK}{UTF8}{mj}数域\end{CJK} $P$ \begin{CJK}{UTF8}{mj}上\end{CJK} $n(>0)$ \begin{CJK}{UTF8}{mj}次多项式\end{CJK} $f(x)$ \begin{CJK}{UTF8}{mj}的形式微商\end{CJK} $f^{\prime}(x) \mid f(x)$ \begin{CJK}{UTF8}{mj}的充要条件是\end{CJK} $f(x)$ \begin{CJK}{UTF8}{mj}有\end{CJK} $n$ \begin{CJK}{UTF8}{mj}重根\end{CJK}.

\begin{CJK}{UTF8}{mj}七\end{CJK}. \begin{CJK}{UTF8}{mj}设向量组\end{CJK} (I): $\alpha_{1}, \alpha_{2}, \alpha_{3}, \alpha_{4}$ \begin{CJK}{UTF8}{mj}的秩为\end{CJK} 3 ; \begin{CJK}{UTF8}{mj}向量组\end{CJK} (II) : $\alpha_{1}, \alpha_{2}, \alpha_{3}, \alpha_{5}$ \begin{CJK}{UTF8}{mj}的秩为\end{CJK} 4 . \begin{CJK}{UTF8}{mj}证明\end{CJK}: \begin{CJK}{UTF8}{mj}向量组\end{CJK}
$$
\alpha_{1}, \alpha_{2}, \alpha_{3}, \alpha_{5}-\alpha_{4}
$$
\begin{CJK}{UTF8}{mj}的秩为\end{CJK} 4 ,

\begin{CJK}{UTF8}{mj}八\end{CJK}. \begin{CJK}{UTF8}{mj}设实对称矩阵\end{CJK} $A$ \begin{CJK}{UTF8}{mj}的特征值的绝对值都是\end{CJK} 1 , \begin{CJK}{UTF8}{mj}证明\end{CJK}: $A$ \begin{CJK}{UTF8}{mj}为正交矩阵\end{CJK}.

\begin{CJK}{UTF8}{mj}九\end{CJK}. \begin{CJK}{UTF8}{mj}设\end{CJK} $\alpha$ \begin{CJK}{UTF8}{mj}是欧式空间\end{CJK} $V$ \begin{CJK}{UTF8}{mj}的一个非零向量\end{CJK}, $\alpha_{1}, \alpha_{2}, \cdots, \alpha_{n} \in V$ \begin{CJK}{UTF8}{mj}满足条件\end{CJK}
$$
\begin{gathered}
\left(\alpha_{i}, \alpha\right)>0(i=1,2, \cdots, n) \\
\left(\alpha_{i}, \alpha_{j}\right) \leqslant 0(i, j=1,2, \cdots, n ; i \neq j)
\end{gathered}
$$
\begin{CJK}{UTF8}{mj}证明\end{CJK}: $\alpha_{1}, \alpha_{2}, \cdots, \alpha_{n}$ \begin{CJK}{UTF8}{mj}线性无关\end{CJK} \begin{CJK}{UTF8}{mj}十\end{CJK}. \begin{CJK}{UTF8}{mj}设\end{CJK} $A$ \begin{CJK}{UTF8}{mj}为\end{CJK} $n(>0)$ \begin{CJK}{UTF8}{mj}阶方阵\end{CJK}.

(1) \begin{CJK}{UTF8}{mj}已知\end{CJK} $A$ \begin{CJK}{UTF8}{mj}的特征值为\end{CJK} $\lambda_{1}, \lambda_{2}, \cdots, \lambda_{n}$, \begin{CJK}{UTF8}{mj}求\end{CJK} $A^{5}$ \begin{CJK}{UTF8}{mj}和\end{CJK} $A^{*}$ \begin{CJK}{UTF8}{mj}的全部特征值\end{CJK}, \begin{CJK}{UTF8}{mj}其中\end{CJK} $A^{*}$ \begin{CJK}{UTF8}{mj}为\end{CJK} $A$ \begin{CJK}{UTF8}{mj}的伴随矩阵\end{CJK}.

(2) \begin{CJK}{UTF8}{mj}证明\end{CJK}: \begin{CJK}{UTF8}{mj}如果\end{CJK} $A$ \begin{CJK}{UTF8}{mj}是正定实对称矩阵\end{CJK}, \begin{CJK}{UTF8}{mj}则\end{CJK} $A^{5}+A^{*}$ \begin{CJK}{UTF8}{mj}也是正定实对称矩阵\end{CJK}.

\section{2. 湘潭大学 2010 年研究生入学考试试题高等代数}
\begin{CJK}{UTF8}{mj}李扬\end{CJK}

\begin{CJK}{UTF8}{mj}微信公众号\end{CJK}: sxkyliyang

\begin{CJK}{UTF8}{mj}本试题共\end{CJK} 10 \begin{CJK}{UTF8}{mj}大题\end{CJK}, \begin{CJK}{UTF8}{mj}每个大题满分均为\end{CJK} 15 \begin{CJK}{UTF8}{mj}分\end{CJK}, \begin{CJK}{UTF8}{mj}总分共\end{CJK} 150 \begin{CJK}{UTF8}{mj}分\end{CJK}.

\begin{CJK}{UTF8}{mj}一\end{CJK}. \begin{CJK}{UTF8}{mj}设\end{CJK} $f(x)$ \begin{CJK}{UTF8}{mj}是数域\end{CJK} $P$ \begin{CJK}{UTF8}{mj}上的一个多项式\end{CJK}, $a, b \in P, a \neq 0$.

(1) \begin{CJK}{UTF8}{mj}证明\end{CJK} $f(x)$ \begin{CJK}{UTF8}{mj}在\end{CJK} $P$ \begin{CJK}{UTF8}{mj}上可约的充分必要条件是\end{CJK} $f(a x+b)$ \begin{CJK}{UTF8}{mj}在\end{CJK} $P$ \begin{CJK}{UTF8}{mj}上可约\end{CJK}.

(2) \begin{CJK}{UTF8}{mj}举例说明当\end{CJK} $f(x)$ \begin{CJK}{UTF8}{mj}不可约时\end{CJK}, $f\left(x^{2}\right)$ \begin{CJK}{UTF8}{mj}可以可约\end{CJK}.

\begin{CJK}{UTF8}{mj}二\end{CJK}. \begin{CJK}{UTF8}{mj}证明如果\end{CJK} $A, B$ \begin{CJK}{UTF8}{mj}是半正定二次型的矩阵\end{CJK}, \begin{CJK}{UTF8}{mj}则\end{CJK} $A+B$ \begin{CJK}{UTF8}{mj}也是\end{CJK}.

\begin{CJK}{UTF8}{mj}三\end{CJK}. \begin{CJK}{UTF8}{mj}设\end{CJK} $A$ \begin{CJK}{UTF8}{mj}的伴随矩阵\end{CJK}
$$
A^{*}=\left(\begin{array}{cc}
-8 & 5 \\
-7 & 5
\end{array}\right)
$$
\begin{CJK}{UTF8}{mj}且\end{CJK} $A B A^{-1}=B A^{-1}+3 E$, \begin{CJK}{UTF8}{mj}求矩阵\end{CJK} $B$.

\begin{CJK}{UTF8}{mj}四\end{CJK}. \begin{CJK}{UTF8}{mj}叙述\end{CJK} Cramer \begin{CJK}{UTF8}{mj}法则并用矩阵方法加以证明解释\end{CJK}.

\begin{CJK}{UTF8}{mj}五\end{CJK}. \begin{CJK}{UTF8}{mj}证明任一非退化下三角矩阵可以表示成一个对角线上元素为\end{CJK} 1 \begin{CJK}{UTF8}{mj}的下三角矩阵左乘一个对角矩阵的乘积\end{CJK}, \begin{CJK}{UTF8}{mj}并且\end{CJK} \begin{CJK}{UTF8}{mj}这种表示方式唯一\end{CJK}.

\begin{CJK}{UTF8}{mj}六\end{CJK}. \begin{CJK}{UTF8}{mj}设\end{CJK} $\alpha_{1}, \alpha_{2}, \alpha_{3}, \alpha_{4}$ \begin{CJK}{UTF8}{mj}是线性空间\end{CJK} $V$ \begin{CJK}{UTF8}{mj}的一组基\end{CJK}.

(1) \begin{CJK}{UTF8}{mj}证明\end{CJK} $\beta_{1}=\alpha_{1}, \beta_{2}=\alpha_{1}+\alpha_{2}, \beta_{3}=\alpha_{1}+\alpha_{2}+\alpha_{3}, \beta_{4}=\alpha_{1}+\alpha_{2}+\alpha_{3}+\alpha_{4}$ \begin{CJK}{UTF8}{mj}也是\end{CJK} $V$ \begin{CJK}{UTF8}{mj}的一组基\end{CJK}.

(2) \begin{CJK}{UTF8}{mj}设\end{CJK} $\gamma$ \begin{CJK}{UTF8}{mj}在前一组基下的坐标是\end{CJK} $(4,3,2,1)$, \begin{CJK}{UTF8}{mj}求\end{CJK} $\gamma$ \begin{CJK}{UTF8}{mj}在后一组基下的坐标\end{CJK}.

\begin{CJK}{UTF8}{mj}七\end{CJK}. \begin{CJK}{UTF8}{mj}设\end{CJK} $\varepsilon_{1}=(1,0), \varepsilon_{2}=(0,1)$ \begin{CJK}{UTF8}{mj}是实线性空间\end{CJK} $\mathbb{R}^{2}$ \begin{CJK}{UTF8}{mj}中的单位向量\end{CJK}, $\alpha_{1}=(1,3), \alpha_{2}=(2,2), \mathscr{A}$ \begin{CJK}{UTF8}{mj}是\end{CJK} $\mathbb{R}^{2}$ \begin{CJK}{UTF8}{mj}的线性变换\end{CJK}, \begin{CJK}{UTF8}{mj}使\end{CJK} \begin{CJK}{UTF8}{mj}得\end{CJK}

(1) \begin{CJK}{UTF8}{mj}求\end{CJK} $\mathscr{A}$ \begin{CJK}{UTF8}{mj}在基\end{CJK} $\varepsilon_{1}, \varepsilon_{2}$ \begin{CJK}{UTF8}{mj}下的矩阵\end{CJK}.

\includegraphics[max width=\textwidth]{2022_04_18_3416d289b173eb9de8c1g-050}

(2) \begin{CJK}{UTF8}{mj}求\end{CJK} $\mathscr{A}$ \begin{CJK}{UTF8}{mj}的逆变换\end{CJK} $\mathscr{A}^{-1}$ \begin{CJK}{UTF8}{mj}在基\end{CJK} $\alpha_{1}, \alpha_{2}$ \begin{CJK}{UTF8}{mj}下的矩阵\end{CJK}.

(3) \begin{CJK}{UTF8}{mj}求\end{CJK} $\mathscr{A}$ \begin{CJK}{UTF8}{mj}的特征值与特征向量\end{CJK}.

(4) \begin{CJK}{UTF8}{mj}求\end{CJK} $\mathscr{A}$ \begin{CJK}{UTF8}{mj}的值域戈核\end{CJK}.

\begin{CJK}{UTF8}{mj}八\end{CJK}. \begin{CJK}{UTF8}{mj}证明数\end{CJK} $\lambda$ \begin{CJK}{UTF8}{mj}是方阵\end{CJK} $A$ \begin{CJK}{UTF8}{mj}的特征多项式的根的充分必要条件是\end{CJK} $\lambda$ \begin{CJK}{UTF8}{mj}也是\end{CJK} $A$ \begin{CJK}{UTF8}{mj}的最小多项式的根\end{CJK}.

\begin{CJK}{UTF8}{mj}九\end{CJK}. \begin{CJK}{UTF8}{mj}设复矩阵\end{CJK} $A$ \begin{CJK}{UTF8}{mj}的若当标准形为\end{CJK}
$$
J=\left(\begin{array}{lllll}
3 & & & & \\
1 & 3 & & & \\
& 0 & 3 & & \\
& & 1 & 3 & \\
& & & 0 & 3
\end{array}\right)
$$
\begin{CJK}{UTF8}{mj}证明\end{CJK} $A$ \begin{CJK}{UTF8}{mj}有三个线性无关的特征向量\end{CJK}, \begin{CJK}{UTF8}{mj}但是没有更多的特征向量线性无关\end{CJK}.

\begin{CJK}{UTF8}{mj}十\end{CJK}. \begin{CJK}{UTF8}{mj}设\end{CJK} $\alpha_{1}, \alpha_{2}, \cdots, \alpha_{n}$ \begin{CJK}{UTF8}{mj}为\end{CJK} $n$ \begin{CJK}{UTF8}{mj}维欧式空间\end{CJK} $V$ \begin{CJK}{UTF8}{mj}的一组基\end{CJK}, \begin{CJK}{UTF8}{mj}证明\end{CJK}: \begin{CJK}{UTF8}{mj}这组基是标准正交基的充分必要条件是\end{CJK}: \begin{CJK}{UTF8}{mj}对\end{CJK} $V$ \begin{CJK}{UTF8}{mj}中任\end{CJK} \begin{CJK}{UTF8}{mj}意向量\end{CJK} $\alpha$ \begin{CJK}{UTF8}{mj}都有\end{CJK}
$$
\alpha=\left(\alpha, \alpha_{1}\right) \alpha_{1}+\left(\alpha, \alpha_{2}\right) \alpha_{2}+\cdots+\left(\alpha, \alpha_{n}\right) \alpha_{n}
$$

\section{3. 湘潭大学 2011 年研究生入学考试试题高等代数}
\begin{CJK}{UTF8}{mj}李扬\end{CJK}

\begin{CJK}{UTF8}{mj}微信公众号\end{CJK}: sxkyliyang

\begin{CJK}{UTF8}{mj}本试题共\end{CJK} 10 \begin{CJK}{UTF8}{mj}大题\end{CJK}, \begin{CJK}{UTF8}{mj}每个大题满分均为\end{CJK} 15 \begin{CJK}{UTF8}{mj}分\end{CJK}, \begin{CJK}{UTF8}{mj}总分共\end{CJK} 150 \begin{CJK}{UTF8}{mj}分\end{CJK}.

\begin{CJK}{UTF8}{mj}一\end{CJK}. \begin{CJK}{UTF8}{mj}讨论多项式\end{CJK}
$$
f(x)=x^{4}+1
$$
\begin{CJK}{UTF8}{mj}在实数域\end{CJK}, \begin{CJK}{UTF8}{mj}复数域\end{CJK}, \begin{CJK}{UTF8}{mj}有理数域上的因式分解\end{CJK}.

\begin{CJK}{UTF8}{mj}二\end{CJK}. \begin{CJK}{UTF8}{mj}求\end{CJK} $n$ \begin{CJK}{UTF8}{mj}阶矩阵\end{CJK} $A$ \begin{CJK}{UTF8}{mj}的行列式和秩\end{CJK}, \begin{CJK}{UTF8}{mj}这里\end{CJK}
$$
A=\left(\begin{array}{cccc}
1 & 1 & \cdots & 1 \\
x_{1} & x_{2} & \cdots & x_{n} \\
x_{1}^{2} & x_{2}^{2} & \cdots & x_{n}^{2} \\
\vdots & \vdots & & \vdots \\
x_{1}^{n-2} & x_{2}^{n-2} & \cdots & x_{n}^{n-2} \\
x_{1}^{n} & x_{2}^{n} & \cdots & x_{n}^{n}
\end{array}\right),
$$
\begin{CJK}{UTF8}{mj}其中\end{CJK} $x_{i} \neq 0(i=1,2, \cdots, n)$ \begin{CJK}{UTF8}{mj}互异\end{CJK}.

\begin{CJK}{UTF8}{mj}三\end{CJK}. \begin{CJK}{UTF8}{mj}考虑齐次线性方程组\end{CJK} $A X=0$, \begin{CJK}{UTF8}{mj}其中\end{CJK} $X=\left(x_{1}, x_{2}, \cdots, x_{n}\right)^{\prime}, A=\left(a_{i j}\right)_{n \times n}, a_{i i}=a, a_{i j}=b(i \neq j), i, j=$ $1,2, \cdots, n, n \geqslant 2, a$ \begin{CJK}{UTF8}{mj}和\end{CJK} $b$ \begin{CJK}{UTF8}{mj}不同时为零\end{CJK}. \begin{CJK}{UTF8}{mj}试讨论当\end{CJK} $a, b$ \begin{CJK}{UTF8}{mj}为何值时\end{CJK}, \begin{CJK}{UTF8}{mj}方程组仅有零解\end{CJK}, \begin{CJK}{UTF8}{mj}有无穷多组非零解\end{CJK}? \begin{CJK}{UTF8}{mj}并求\end{CJK} \begin{CJK}{UTF8}{mj}出全部解\end{CJK}.

\begin{CJK}{UTF8}{mj}四\end{CJK}. \begin{CJK}{UTF8}{mj}设二次型\end{CJK}
$$
f\left(x_{1}, x_{2}, x_{3}\right)=a x_{1}^{2}+2 x_{2}^{2}-2 x_{3}^{2}+2 b x_{1} x_{3}(b<0),
$$
\begin{CJK}{UTF8}{mj}且二次型\end{CJK} $f$ \begin{CJK}{UTF8}{mj}的矩阵\end{CJK} $A$ \begin{CJK}{UTF8}{mj}的特征值之和为\end{CJK} 1 , \begin{CJK}{UTF8}{mj}特征值之积为\end{CJK} $-12$.

(1) \begin{CJK}{UTF8}{mj}求\end{CJK} $a, b$ \begin{CJK}{UTF8}{mj}的值\end{CJK}.

(2) \begin{CJK}{UTF8}{mj}求正交变换将二次型\end{CJK} $f$ \begin{CJK}{UTF8}{mj}化为标准型\end{CJK}.

\begin{CJK}{UTF8}{mj}五\end{CJK}. \begin{CJK}{UTF8}{mj}设\end{CJK} $P$ \begin{CJK}{UTF8}{mj}是数域\end{CJK}, \begin{CJK}{UTF8}{mj}集合\end{CJK}
$$
W=\left\{\left(\begin{array}{ll}
a & 0 \\
b & c
\end{array}\right) \mid a, b, c \in P\right\} .
$$
(1) \begin{CJK}{UTF8}{mj}证明\end{CJK}: $W$ \begin{CJK}{UTF8}{mj}对于矩阵加法和数量乘法构成一个线性空间\end{CJK}.

(2) \begin{CJK}{UTF8}{mj}令\end{CJK} $A=\left(\begin{array}{ll}1 & 0 \\ 1 & 1\end{array}\right)$, \begin{CJK}{UTF8}{mj}变换\end{CJK} $\mathscr{A}: W \rightarrow W$ \begin{CJK}{UTF8}{mj}定义为\end{CJK} $\mathscr{A}(X)=A X(\forall X \in W)$. \begin{CJK}{UTF8}{mj}证明\end{CJK}: $\mathscr{A}$ \begin{CJK}{UTF8}{mj}是\end{CJK} $W$ \begin{CJK}{UTF8}{mj}的线性变换\end{CJK}.

(3) \begin{CJK}{UTF8}{mj}写出\end{CJK} $W$ \begin{CJK}{UTF8}{mj}的一组基及维数\end{CJK}, \begin{CJK}{UTF8}{mj}求\end{CJK} $\mathscr{A}$ \begin{CJK}{UTF8}{mj}在这组基下的矩阵\end{CJK}.

(4) \begin{CJK}{UTF8}{mj}求出\end{CJK} $\mathscr{A}$ \begin{CJK}{UTF8}{mj}的特征值及相应的全部特征向量\end{CJK}.

\begin{CJK}{UTF8}{mj}六\end{CJK}. \begin{CJK}{UTF8}{mj}设复矩阵\end{CJK}
$$
A=\left(\begin{array}{ccc}
2 & 0 & 0 \\
a & 2 & 0 \\
b & 1 & -1
\end{array}\right)
$$
\begin{CJK}{UTF8}{mj}求\end{CJK} $A$ \begin{CJK}{UTF8}{mj}的行列式因子\end{CJK}, \begin{CJK}{UTF8}{mj}不变因子\end{CJK}, \begin{CJK}{UTF8}{mj}初等因子\end{CJK}, \begin{CJK}{UTF8}{mj}最小多项式\end{CJK}, Jordan \begin{CJK}{UTF8}{mj}标准形及\end{CJK} $A$ \begin{CJK}{UTF8}{mj}相似于对角矩阵的充分必要条件\end{CJK}.

\begin{CJK}{UTF8}{mj}七\end{CJK}. \begin{CJK}{UTF8}{mj}设\end{CJK} $A$ \begin{CJK}{UTF8}{mj}为\end{CJK} $n$ \begin{CJK}{UTF8}{mj}阶实对称矩阵\end{CJK}, \begin{CJK}{UTF8}{mj}则\end{CJK} $A$ \begin{CJK}{UTF8}{mj}的秩为\end{CJK} $n$ \begin{CJK}{UTF8}{mj}的充分必要条件是\end{CJK}: \begin{CJK}{UTF8}{mj}存在\end{CJK} $n$ \begin{CJK}{UTF8}{mj}阶实矩阵\end{CJK} $B$ \begin{CJK}{UTF8}{mj}使得\end{CJK} $A B+B^{\prime} A$ \begin{CJK}{UTF8}{mj}为正定矩阵\end{CJK}. \begin{CJK}{UTF8}{mj}八\end{CJK}. \begin{CJK}{UTF8}{mj}设\end{CJK} $\mathscr{A}$ \begin{CJK}{UTF8}{mj}是有限维线性空间\end{CJK} $V$ \begin{CJK}{UTF8}{mj}的可逆线性变换\end{CJK}, $W$ \begin{CJK}{UTF8}{mj}是\end{CJK} $V$ \begin{CJK}{UTF8}{mj}中\end{CJK} $\mathscr{A}$ \begin{CJK}{UTF8}{mj}的不变子空间\end{CJK}. \begin{CJK}{UTF8}{mj}证明\end{CJK}: $W$ \begin{CJK}{UTF8}{mj}也是\end{CJK} $V$ \begin{CJK}{UTF8}{mj}中\end{CJK} $\mathscr{A}{ }^{-1}$ \begin{CJK}{UTF8}{mj}的不变\end{CJK} \begin{CJK}{UTF8}{mj}子空间\end{CJK}.

\begin{CJK}{UTF8}{mj}九\end{CJK}. \begin{CJK}{UTF8}{mj}证明\end{CJK}: $n$ \begin{CJK}{UTF8}{mj}维欧式空间\end{CJK} $V$ \begin{CJK}{UTF8}{mj}的每一子空间\end{CJK} $V_{1}$ \begin{CJK}{UTF8}{mj}都有唯一的正交补\end{CJK}.

\begin{CJK}{UTF8}{mj}十\end{CJK}. \begin{CJK}{UTF8}{mj}设\end{CJK} $A, B$ \begin{CJK}{UTF8}{mj}均为\end{CJK} $n$ \begin{CJK}{UTF8}{mj}阶矩阵\end{CJK}, $A B=A+B$. \begin{CJK}{UTF8}{mj}证明\end{CJK}:

(1) $A, B$ \begin{CJK}{UTF8}{mj}的特征值均不为\end{CJK} 1 , \begin{CJK}{UTF8}{mj}且\end{CJK} $A B=B A$.

(2) \begin{CJK}{UTF8}{mj}若\end{CJK} $A, B$ \begin{CJK}{UTF8}{mj}均可对角化\end{CJK}, \begin{CJK}{UTF8}{mj}则存在可逆矩阵\end{CJK} $P$, \begin{CJK}{UTF8}{mj}使\end{CJK} $P^{-1} A P$ \begin{CJK}{UTF8}{mj}和\end{CJK} $P^{-1} B P$ \begin{CJK}{UTF8}{mj}同时为对角阵\end{CJK}.

\section{4. 湘潭大学 2012 年研究生入学考试试题高等代数}
\begin{CJK}{UTF8}{mj}李扬\end{CJK}

\begin{CJK}{UTF8}{mj}微信公众号\end{CJK}: sxkyliyang

\begin{CJK}{UTF8}{mj}本试题共\end{CJK} 10 \begin{CJK}{UTF8}{mj}大题\end{CJK}, \begin{CJK}{UTF8}{mj}每个大题满分均为\end{CJK} 15 \begin{CJK}{UTF8}{mj}分\end{CJK}, \begin{CJK}{UTF8}{mj}总分共\end{CJK} 150 \begin{CJK}{UTF8}{mj}分\end{CJK}.

\begin{CJK}{UTF8}{mj}一\end{CJK}. \begin{CJK}{UTF8}{mj}计算行列式\end{CJK}
$$
D_{n}=\left|\begin{array}{cccccc}
x+y & y & 0 & \cdots & 0 & 0 \\
x & x+y & y & \cdots & 0 & 0 \\
0 & x & x+y & \cdots & 0 & 0 \\
\vdots & \vdots & \vdots & & \vdots & \vdots \\
0 & 0 & 0 & \cdots & x+y & y \\
0 & 0 & 0 & \cdots & x & x+y
\end{array}\right|, n \geqslant 1
$$
\begin{CJK}{UTF8}{mj}二\end{CJK}. \begin{CJK}{UTF8}{mj}设向量组\end{CJK} $\beta, \alpha_{1}, \alpha_{2}, \cdots, \alpha_{n}$ \begin{CJK}{UTF8}{mj}线性相关\end{CJK}, \begin{CJK}{UTF8}{mj}及向量组\end{CJK} $\alpha_{1}, \alpha_{2}, \cdots, \alpha_{n}$ \begin{CJK}{UTF8}{mj}线性无关\end{CJK}. \begin{CJK}{UTF8}{mj}证明\end{CJK}: \begin{CJK}{UTF8}{mj}向量\end{CJK} $\beta$ \begin{CJK}{UTF8}{mj}可以由向量组\end{CJK} $\alpha_{1}, \alpha_{2}, \cdots, \alpha_{n}$ \begin{CJK}{UTF8}{mj}唯一线性表示\end{CJK}.

\begin{CJK}{UTF8}{mj}三\end{CJK}. \begin{CJK}{UTF8}{mj}设\end{CJK} $B$ \begin{CJK}{UTF8}{mj}为\end{CJK} $n$ \begin{CJK}{UTF8}{mj}级实矩阵\end{CJK}, $A$ \begin{CJK}{UTF8}{mj}为\end{CJK} $n$ \begin{CJK}{UTF8}{mj}级实对称矩阵且\end{CJK} $A$ \begin{CJK}{UTF8}{mj}的所有特征值两两均不相同\end{CJK}. \begin{CJK}{UTF8}{mj}证明\end{CJK}: $A$ \begin{CJK}{UTF8}{mj}与\end{CJK} $B$ \begin{CJK}{UTF8}{mj}可交换当且仅当\end{CJK} \begin{CJK}{UTF8}{mj}存在可逆矩阵\end{CJK} $S$ \begin{CJK}{UTF8}{mj}使得\end{CJK} $S^{-1} A S$ \begin{CJK}{UTF8}{mj}与\end{CJK} $S^{-1} B S$ \begin{CJK}{UTF8}{mj}同时为对角矩阵\end{CJK}.

\begin{CJK}{UTF8}{mj}四\end{CJK}. \begin{CJK}{UTF8}{mj}设\end{CJK} $A$ \begin{CJK}{UTF8}{mj}为\end{CJK} $n$ \begin{CJK}{UTF8}{mj}级方阵\end{CJK}, $X$ \begin{CJK}{UTF8}{mj}为\end{CJK} $n \times m$ \begin{CJK}{UTF8}{mj}的矩阵且\end{CJK} $X$ \begin{CJK}{UTF8}{mj}的秩为\end{CJK} $m(m \leqslant n)$. \begin{CJK}{UTF8}{mj}令\end{CJK} $\Gamma$ \begin{CJK}{UTF8}{mj}为矩阵\end{CJK} $X$ \begin{CJK}{UTF8}{mj}的列向量组生成的子空间\end{CJK}. \begin{CJK}{UTF8}{mj}证\end{CJK} \begin{CJK}{UTF8}{mj}明\end{CJK}: $\Gamma$ \begin{CJK}{UTF8}{mj}为\end{CJK} $A$ \begin{CJK}{UTF8}{mj}的不变子空间当且仅当存在\end{CJK} $m \times m$ \begin{CJK}{UTF8}{mj}的矩阵\end{CJK} $B$ \begin{CJK}{UTF8}{mj}使得\end{CJK}

\begin{CJK}{UTF8}{mj}五\end{CJK}. \begin{CJK}{UTF8}{mj}设齐次线性方程组\end{CJK}
$$
\left\{\begin{array}{l}
(a+b) x_{1}+b x_{2}+\cdots+b x_{n}=0 \\
2 b x_{1}+(a+2 b) x_{2}+\cdots+2 b x_{n}=0 \\
\vdots \\
n b x_{1}+n b x_{2}+\cdots+(a+n b) x_{n}=0
\end{array}\right.
$$
\begin{CJK}{UTF8}{mj}试讨论\end{CJK} $a, b$ \begin{CJK}{UTF8}{mj}为何值时\end{CJK}, \begin{CJK}{UTF8}{mj}方程组只有零解\end{CJK}, \begin{CJK}{UTF8}{mj}有非零解\end{CJK}? \begin{CJK}{UTF8}{mj}在有非零解时\end{CJK}, \begin{CJK}{UTF8}{mj}求出其通解\end{CJK}.

\begin{CJK}{UTF8}{mj}六\end{CJK}. \begin{CJK}{UTF8}{mj}设多项式\end{CJK}
$$
f(x)=6 x^{4}+3 b x^{3}+4 a x^{2}-10 x-1
$$
$$
g(x)=2 x^{4}+5 x^{3}+a x^{2}-b x+2,
$$
\begin{CJK}{UTF8}{mj}其中\end{CJK} $a, b$ \begin{CJK}{UTF8}{mj}为整数\end{CJK}. \begin{CJK}{UTF8}{mj}讨论\end{CJK} $a, b$ \begin{CJK}{UTF8}{mj}为何值时\end{CJK}, $f(x)$ \begin{CJK}{UTF8}{mj}与\end{CJK} $g(x)$ \begin{CJK}{UTF8}{mj}有公共有理根\end{CJK}, \begin{CJK}{UTF8}{mj}并求出相应的有理根\end{CJK}.

\begin{CJK}{UTF8}{mj}七\end{CJK}. \begin{CJK}{UTF8}{mj}设\end{CJK} $P$ \begin{CJK}{UTF8}{mj}为一数域\end{CJK}, \begin{CJK}{UTF8}{mj}集合\end{CJK}
$$
V=\left\{\left(\begin{array}{ll}
a & b \\
c & d
\end{array}\right) \mid a, b, c, d \in P, a+b=0\right\} .
$$
(1) \begin{CJK}{UTF8}{mj}证明\end{CJK}: $V$ \begin{CJK}{UTF8}{mj}对于矩阵的加法和数量乘法构成一个线性空间\end{CJK}.

(2) \begin{CJK}{UTF8}{mj}令\end{CJK} $A=\left(\begin{array}{ll}1 & 0 \\ 1 & 1\end{array}\right)$, \begin{CJK}{UTF8}{mj}变换\end{CJK} $\mathscr{A}: V \rightarrow V$ \begin{CJK}{UTF8}{mj}定义为\end{CJK}: $\mathscr{A}(X)=A X, \forall X \in V$. \begin{CJK}{UTF8}{mj}证明\end{CJK}: $\mathscr{A}$ \begin{CJK}{UTF8}{mj}是\end{CJK} $V$ \begin{CJK}{UTF8}{mj}的线性变换\end{CJK}.

(3) \begin{CJK}{UTF8}{mj}写出\end{CJK} $V$ \begin{CJK}{UTF8}{mj}的一组基\end{CJK}, \begin{CJK}{UTF8}{mj}并求\end{CJK} $\mathscr{A}$ \begin{CJK}{UTF8}{mj}在这一组基下的矩阵\end{CJK}.

(4) \begin{CJK}{UTF8}{mj}求出\end{CJK} $\mathscr{A}$ \begin{CJK}{UTF8}{mj}的特征值及相应的全部特征向量\end{CJK}. \begin{CJK}{UTF8}{mj}八\end{CJK}. \begin{CJK}{UTF8}{mj}设矩阵\end{CJK} $A$ \begin{CJK}{UTF8}{mj}的特征多项式及最小多项式分别为\end{CJK}
$$
\begin{aligned}
&f(\lambda)=(\lambda+1)^{2}(\lambda-1)(\lambda-2) \\
&m(\lambda)=(\lambda+1)(\lambda-1)(\lambda-2)
\end{aligned}
$$
\begin{CJK}{UTF8}{mj}分别求出\end{CJK} $A$ \begin{CJK}{UTF8}{mj}的行列式因子\end{CJK}, \begin{CJK}{UTF8}{mj}不变因子\end{CJK}, \begin{CJK}{UTF8}{mj}初等因子及\end{CJK} Jordan \begin{CJK}{UTF8}{mj}标准型\end{CJK}.

\begin{CJK}{UTF8}{mj}九\end{CJK}. \begin{CJK}{UTF8}{mj}证明\end{CJK}:

(1) \begin{CJK}{UTF8}{mj}如果\end{CJK} $A$ \begin{CJK}{UTF8}{mj}为\end{CJK} $n \times n$ \begin{CJK}{UTF8}{mj}的实正定矩阵\end{CJK}, \begin{CJK}{UTF8}{mj}则存在可逆的对称矩阵\end{CJK} $S$ \begin{CJK}{UTF8}{mj}使得\end{CJK} $A=S^{2}$.

(2) \begin{CJK}{UTF8}{mj}如果\end{CJK} $A$ \begin{CJK}{UTF8}{mj}为\end{CJK} $n \times n$ \begin{CJK}{UTF8}{mj}的实矩阵\end{CJK}, \begin{CJK}{UTF8}{mj}则\end{CJK} $A$ \begin{CJK}{UTF8}{mj}可表示为两个实正定矩阵乘积的充要条件是\end{CJK} $A$ \begin{CJK}{UTF8}{mj}可对角化且其所有特征值\end{CJK} \begin{CJK}{UTF8}{mj}都大于零\end{CJK}.

\begin{CJK}{UTF8}{mj}十\end{CJK}. \begin{CJK}{UTF8}{mj}如果矩阵的所有元素都为非负数\end{CJK}, \begin{CJK}{UTF8}{mj}那么就称该矩阵为非负矩阵\end{CJK}. \begin{CJK}{UTF8}{mj}设\end{CJK} $A=\left(a_{i j}\right)$ \begin{CJK}{UTF8}{mj}为\end{CJK} $n \times n$ \begin{CJK}{UTF8}{mj}的可逆非负矩阵且主对\end{CJK} \begin{CJK}{UTF8}{mj}角线上所有元素都大于零\end{CJK}. \begin{CJK}{UTF8}{mj}证明\end{CJK}: $A$ \begin{CJK}{UTF8}{mj}的逆矩阵为非负矩阵当且仅当\end{CJK} $A$ \begin{CJK}{UTF8}{mj}为对角阵\end{CJK}.

\section{5. 湘潭大学 2013 年研究生入学考试试题高等代数}
\begin{CJK}{UTF8}{mj}李扬\end{CJK}

\begin{CJK}{UTF8}{mj}微信公众号\end{CJK}: sxkyliyang

\begin{CJK}{UTF8}{mj}本试题共\end{CJK} 10 \begin{CJK}{UTF8}{mj}大题\end{CJK}, \begin{CJK}{UTF8}{mj}每个大题满分均为\end{CJK} 15 \begin{CJK}{UTF8}{mj}分\end{CJK}, \begin{CJK}{UTF8}{mj}总分共\end{CJK} 150 \begin{CJK}{UTF8}{mj}分\end{CJK}.

\begin{CJK}{UTF8}{mj}一\end{CJK}. \begin{CJK}{UTF8}{mj}计算\end{CJK} $n$ \begin{CJK}{UTF8}{mj}级行列式\end{CJK}
$$
D_{n}=\left|\begin{array}{ccccc}
x & a & \cdots & a & a \\
b & x & \cdots & a & a \\
\vdots & \vdots & & \vdots & \vdots \\
b & b & \cdots & x & a \\
b & b & \cdots & b & x
\end{array}\right|
$$
\begin{CJK}{UTF8}{mj}二\end{CJK}. \begin{CJK}{UTF8}{mj}设\end{CJK} $n$ \begin{CJK}{UTF8}{mj}为奇数\end{CJK}, $\alpha_{1}, \alpha_{2}, \cdots, \alpha_{n}$ \begin{CJK}{UTF8}{mj}为\end{CJK} $n$ \begin{CJK}{UTF8}{mj}维列向量组\end{CJK}. \begin{CJK}{UTF8}{mj}证明\end{CJK}: \begin{CJK}{UTF8}{mj}向量组\end{CJK} $\alpha_{1}, \alpha_{2}, \cdots, \alpha_{n}$ \begin{CJK}{UTF8}{mj}线性无关当且仅当向量组\end{CJK}
$$
\alpha_{1}+\alpha_{2}, \alpha_{2}+\alpha_{3}, \cdots, \alpha_{n-1}+\alpha_{n}, \alpha_{n}+\alpha_{1}
$$
\begin{CJK}{UTF8}{mj}线性无关\end{CJK}.

\begin{CJK}{UTF8}{mj}三\end{CJK}. \begin{CJK}{UTF8}{mj}设\end{CJK} $p$ \begin{CJK}{UTF8}{mj}是一个素数\end{CJK}, \begin{CJK}{UTF8}{mj}多项式\end{CJK}
$$
f(x)=x^{p-1}+x^{p-2}+\cdots+x+1 .
$$
\begin{CJK}{UTF8}{mj}证明\end{CJK}: $f(x)$ \begin{CJK}{UTF8}{mj}在有理数域上是不可约的\end{CJK}.

\begin{CJK}{UTF8}{mj}四\end{CJK}. \begin{CJK}{UTF8}{mj}设齐次线性方程组\end{CJK} $x_{1}+2 x_{2}+\cdots+n x_{n}=0$ \begin{CJK}{UTF8}{mj}与\end{CJK} $x_{1}=x_{2}=\cdots=x_{n}$ \begin{CJK}{UTF8}{mj}的解空间分别为\end{CJK} $V_{1}$ \begin{CJK}{UTF8}{mj}和\end{CJK} $V_{2}$. \begin{CJK}{UTF8}{mj}证明\end{CJK}:
$$
\mathbb{C}^{n}=V_{1} \oplus V_{2} .
$$
\begin{CJK}{UTF8}{mj}五\end{CJK}. \begin{CJK}{UTF8}{mj}设\end{CJK} $A$ \begin{CJK}{UTF8}{mj}为\end{CJK} $n$ \begin{CJK}{UTF8}{mj}级可逆方阵\end{CJK}, \begin{CJK}{UTF8}{mj}则\end{CJK} $A$ \begin{CJK}{UTF8}{mj}存在\end{CJK} $L U$ \begin{CJK}{UTF8}{mj}分解\end{CJK} (\begin{CJK}{UTF8}{mj}即\end{CJK} $A=L U$, \begin{CJK}{UTF8}{mj}其中\end{CJK} $L$ \begin{CJK}{UTF8}{mj}是主对角元素为\end{CJK} 1 \begin{CJK}{UTF8}{mj}的下三角矩阵\end{CJK}, $U$ \begin{CJK}{UTF8}{mj}是上三\end{CJK} \begin{CJK}{UTF8}{mj}角矩阵\end{CJK}) \begin{CJK}{UTF8}{mj}当且仅当\end{CJK} $A$ \begin{CJK}{UTF8}{mj}的所有顺序主子式不等于零\end{CJK}.

\begin{CJK}{UTF8}{mj}六\end{CJK}. \begin{CJK}{UTF8}{mj}设\end{CJK} $A$ \begin{CJK}{UTF8}{mj}为\end{CJK} $n$ \begin{CJK}{UTF8}{mj}级复方阵\end{CJK}. \begin{CJK}{UTF8}{mj}若\end{CJK} $A^{2}=A$, \begin{CJK}{UTF8}{mj}则称\end{CJK} $A$ \begin{CJK}{UTF8}{mj}为幂等矩阵\end{CJK}. \begin{CJK}{UTF8}{mj}证明\end{CJK}:

(1) \begin{CJK}{UTF8}{mj}幂等矩阵\end{CJK} $A$ \begin{CJK}{UTF8}{mj}的特征值只有\end{CJK} $\lambda_{0}=0$ \begin{CJK}{UTF8}{mj}或\end{CJK} $\lambda_{1}=1$.

(2) \begin{CJK}{UTF8}{mj}令\end{CJK} $V_{\lambda_{0}}$ \begin{CJK}{UTF8}{mj}和\end{CJK} $V_{\lambda_{1}}$ \begin{CJK}{UTF8}{mj}分别表示对应特征值\end{CJK} $\lambda_{0}=0$ \begin{CJK}{UTF8}{mj}和\end{CJK} $\lambda_{1}=1$ \begin{CJK}{UTF8}{mj}的特征子空间\end{CJK}, \begin{CJK}{UTF8}{mj}则向量空间\end{CJK}
$$
\mathbb{C}^{n}=V_{\lambda_{0}} \oplus V_{\lambda_{1}}
$$
\begin{CJK}{UTF8}{mj}七\end{CJK}. \begin{CJK}{UTF8}{mj}设\end{CJK} $P$ \begin{CJK}{UTF8}{mj}为一数域\end{CJK}, \begin{CJK}{UTF8}{mj}集合\end{CJK}
$$
V=\left\{\left(\begin{array}{lll}
a & 0 & b \\
0 & c & 0 \\
b & 0 & a
\end{array}\right) \mid a, b, c \in P\right\} .
$$
(1) \begin{CJK}{UTF8}{mj}证明\end{CJK}: $V$ \begin{CJK}{UTF8}{mj}对于矩阵的加法和数量乘法构成一个线性空间\end{CJK}.

(2) \begin{CJK}{UTF8}{mj}令\end{CJK} $A=\left(\begin{array}{lll}0 & 0 & 1 \\ 0 & 2 & 0 \\ 1 & 0 & 0\end{array}\right)$, \begin{CJK}{UTF8}{mj}变换\end{CJK} $\mathscr{A}: V \rightarrow V$ \begin{CJK}{UTF8}{mj}定义为\end{CJK}: $\mathscr{A}(X)=A X, \forall X \in V$. \begin{CJK}{UTF8}{mj}证明\end{CJK}: $\mathscr{A}$ \begin{CJK}{UTF8}{mj}是\end{CJK} $V$ \begin{CJK}{UTF8}{mj}的线性变换\end{CJK}.

(3) \begin{CJK}{UTF8}{mj}写出\end{CJK} $V$ \begin{CJK}{UTF8}{mj}的一组基\end{CJK}, \begin{CJK}{UTF8}{mj}并求\end{CJK} $\mathscr{A}$ \begin{CJK}{UTF8}{mj}在这一组基下的矩阵\end{CJK}.

(4) \begin{CJK}{UTF8}{mj}求出\end{CJK} $\mathscr{A}$ \begin{CJK}{UTF8}{mj}的特征值及相应的全部特征向量\end{CJK}. \begin{CJK}{UTF8}{mj}八\end{CJK}. \begin{CJK}{UTF8}{mj}令复矩阵\end{CJK}
$$
A=\left(\begin{array}{ccc}
1 & a & b \\
0 & 1 & c \\
0 & 0 & 2
\end{array}\right)
$$
(1) \begin{CJK}{UTF8}{mj}求出\end{CJK} $A$ \begin{CJK}{UTF8}{mj}的所有可能的若尔当标准型\end{CJK}.

(2) \begin{CJK}{UTF8}{mj}给出\end{CJK} $A$ \begin{CJK}{UTF8}{mj}可对角化的充要条件\end{CJK}.

\begin{CJK}{UTF8}{mj}九\end{CJK}. \begin{CJK}{UTF8}{mj}令\end{CJK}
$$
A=\left(\begin{array}{cccc}
1 & 1 & 1 & 1 \\
1 & 1 & -1 & -1 \\
1 & -1 & 1 & -1 \\
1 & -1 & -1 & 1
\end{array}\right)
$$
\begin{CJK}{UTF8}{mj}求正交矩阵\end{CJK} $Q$, \begin{CJK}{UTF8}{mj}使得\end{CJK} $Q^{\prime} A Q$ \begin{CJK}{UTF8}{mj}为对角矩阵\end{CJK}, \begin{CJK}{UTF8}{mj}其中\end{CJK} $Q^{\prime}$ \begin{CJK}{UTF8}{mj}表示\end{CJK} $Q$ \begin{CJK}{UTF8}{mj}的转置矩阵\end{CJK}; \begin{CJK}{UTF8}{mj}并求出对角阵\end{CJK}.

\begin{CJK}{UTF8}{mj}十\end{CJK}. \begin{CJK}{UTF8}{mj}设\end{CJK} $A=\left(a_{i j}\right)$ \begin{CJK}{UTF8}{mj}为\end{CJK} $n$ \begin{CJK}{UTF8}{mj}级实对称正定矩阵\end{CJK}, \begin{CJK}{UTF8}{mj}证明\end{CJK}:
$$
|A| \leqslant a_{11} a_{22} \cdots a_{n n}
$$
\begin{CJK}{UTF8}{mj}且等号成立当且仅当\end{CJK} $A$ \begin{CJK}{UTF8}{mj}为对角阵\end{CJK}.

\section{6. 湘潭大学 2014 年研究生入学考试试题高等代数}
\begin{CJK}{UTF8}{mj}李扬\end{CJK}

\begin{CJK}{UTF8}{mj}微信公众号\end{CJK}: sxkyliyang

\begin{CJK}{UTF8}{mj}本试题共\end{CJK} 10 \begin{CJK}{UTF8}{mj}大题\end{CJK}, \begin{CJK}{UTF8}{mj}每个大题满分均为\end{CJK} 15 \begin{CJK}{UTF8}{mj}分\end{CJK}, \begin{CJK}{UTF8}{mj}总分共\end{CJK} 150 \begin{CJK}{UTF8}{mj}分\end{CJK}.

\begin{CJK}{UTF8}{mj}一\end{CJK}. \begin{CJK}{UTF8}{mj}令\end{CJK} $n$ \begin{CJK}{UTF8}{mj}级行列式\end{CJK}
$$
|A|=\left|\begin{array}{ccccc}
0 & 1 & \cdots & 1 & 1 \\
1 & 2 & \cdots & 0 & 0 \\
\vdots & \vdots & & \vdots & \vdots \\
1 & 0 & \cdots & n-1 & 0 \\
1 & 0 & \cdots & 0 & n
\end{array}\right|, n \geqslant 2
$$
\begin{CJK}{UTF8}{mj}求\end{CJK} $\sum_{j=1}^{n} A_{1 j}$, \begin{CJK}{UTF8}{mj}其中\end{CJK} $A_{1 j}$ \begin{CJK}{UTF8}{mj}为行列式\end{CJK} $|A|$ \begin{CJK}{UTF8}{mj}中元素\end{CJK} $a_{1 j}$ \begin{CJK}{UTF8}{mj}的代数余子式\end{CJK}.

\begin{CJK}{UTF8}{mj}二\end{CJK}. \begin{CJK}{UTF8}{mj}令向量\end{CJK} $\alpha_{i}=\left(a_{i 1}, a_{i 2}, \cdots, a_{i n}\right), i=1,2, \cdots, s$ \begin{CJK}{UTF8}{mj}及\end{CJK} $\beta=\left(b_{1}, b_{2}, \cdots, b_{n}\right)$, \begin{CJK}{UTF8}{mj}证明\end{CJK}: \begin{CJK}{UTF8}{mj}线性方程组\end{CJK}
$$
\left\{\begin{array}{l}
a_{11} x_{1}+a_{12} x_{2}+\cdots+a_{1 n} x_{n}=0 \\
a_{21} x_{1}+a_{22} x_{2}+\cdots+a_{2 n} x_{n}=0 \\
\vdots \\
a_{s 1} x_{1}+a_{s 2} x_{2}+\cdots+a_{s n} x_{n}=0
\end{array}\right.
$$
\begin{CJK}{UTF8}{mj}的解全是方程\end{CJK} $b_{1} x_{1}+b_{2} x_{2}+\cdots+b_{n} x_{n}=0$ \begin{CJK}{UTF8}{mj}的解当且仅当\end{CJK} $\beta$ \begin{CJK}{UTF8}{mj}可由\end{CJK} $\alpha_{1}, \alpha_{2}, \cdots, \alpha_{s}$ \begin{CJK}{UTF8}{mj}线性表示\end{CJK}.

\begin{CJK}{UTF8}{mj}三\end{CJK}. \begin{CJK}{UTF8}{mj}证明\end{CJK}:

(1) \begin{CJK}{UTF8}{mj}令两组数\end{CJK} $\lambda_{1}, \lambda_{2}, \cdots, \lambda_{n}$ \begin{CJK}{UTF8}{mj}与\end{CJK} $b_{1}, b_{2}, \cdots, b_{n}$, \begin{CJK}{UTF8}{mj}如果\end{CJK} $\lambda_{1}, \lambda_{2}, \cdots, \lambda_{n}$ \begin{CJK}{UTF8}{mj}两两互不相同\end{CJK}, \begin{CJK}{UTF8}{mj}则存在多项式\end{CJK}
$$
p(x)=a_{0}+a_{1} x+\cdots+a_{n-1} x^{n-1}
$$
\begin{CJK}{UTF8}{mj}使得\end{CJK} $p\left(\lambda_{i}\right)=b_{i}, i=1,2, \cdots, n$.

(2) \begin{CJK}{UTF8}{mj}令两个\end{CJK} $n$ \begin{CJK}{UTF8}{mj}级矩阵\end{CJK} $A$ \begin{CJK}{UTF8}{mj}与\end{CJK} $B$, \begin{CJK}{UTF8}{mj}且\end{CJK} $A$ \begin{CJK}{UTF8}{mj}有\end{CJK} $n$ \begin{CJK}{UTF8}{mj}个互不相同的特征值\end{CJK}, \begin{CJK}{UTF8}{mj}如果\end{CJK} $A B=B A$, \begin{CJK}{UTF8}{mj}则存在次数最多为\end{CJK} $n-1$ \begin{CJK}{UTF8}{mj}次的\end{CJK} \begin{CJK}{UTF8}{mj}多项式\end{CJK} $p(x)$ \begin{CJK}{UTF8}{mj}使得\end{CJK} $p(A)=B$.

\begin{CJK}{UTF8}{mj}四\end{CJK}. \begin{CJK}{UTF8}{mj}令\end{CJK} $n$ \begin{CJK}{UTF8}{mj}级矩阵\end{CJK} $A=\left(a_{i j}\right)$ \begin{CJK}{UTF8}{mj}为正定矩阵\end{CJK}, \begin{CJK}{UTF8}{mj}证明\end{CJK}:

(1) $A$ \begin{CJK}{UTF8}{mj}的主对角元素均大于零\end{CJK}, \begin{CJK}{UTF8}{mj}且\end{CJK} $A$ \begin{CJK}{UTF8}{mj}的绝对值最大的元素必在主对角线上\end{CJK}.

(2) $n$ \begin{CJK}{UTF8}{mj}级矩阵\end{CJK} $B=\left(b_{i j}\right)$ \begin{CJK}{UTF8}{mj}为正定矩阵\end{CJK}, \begin{CJK}{UTF8}{mj}其中\end{CJK} $b_{i j}=a_{i j} / \sqrt{a_{i i} a_{j j}}$.

\begin{CJK}{UTF8}{mj}五\end{CJK}. \begin{CJK}{UTF8}{mj}令\end{CJK} $\mathbb{R}^{n \times n}$ \begin{CJK}{UTF8}{mj}表示实数域上全体\end{CJK} $n$ \begin{CJK}{UTF8}{mj}级矩阵所组成的线性空间\end{CJK}, \begin{CJK}{UTF8}{mj}令\end{CJK}
$$
V_{1}=\left\{A \in \mathbb{R}^{n \times n} \mid A=A^{\prime}\right\},
$$
\begin{CJK}{UTF8}{mj}其中\end{CJK} $A^{\prime}$ \begin{CJK}{UTF8}{mj}表示矩阵\end{CJK} $A$ \begin{CJK}{UTF8}{mj}的转置矩阵\end{CJK}.

(1) \begin{CJK}{UTF8}{mj}证明\end{CJK}: $V_{1}$ \begin{CJK}{UTF8}{mj}为\end{CJK} $\mathbb{R}^{n \times n}$ \begin{CJK}{UTF8}{mj}的子空间\end{CJK}.

(2) \begin{CJK}{UTF8}{mj}求\end{CJK} $\mathbb{R}^{n \times n}$ \begin{CJK}{UTF8}{mj}的子空间\end{CJK} $V_{2}$ \begin{CJK}{UTF8}{mj}使得\end{CJK} $\mathbb{R}^{n \times n}=V_{1} \oplus V_{2}$.

\begin{CJK}{UTF8}{mj}六\end{CJK}. \begin{CJK}{UTF8}{mj}令\end{CJK} $\mathscr{A}$ \begin{CJK}{UTF8}{mj}为有限维线性空间\end{CJK} $V$ \begin{CJK}{UTF8}{mj}的线性变换\end{CJK}, \begin{CJK}{UTF8}{mj}证明\end{CJK}: $\mathscr{A}^{2}$ \begin{CJK}{UTF8}{mj}的秩等于\end{CJK} $\mathscr{A}$ \begin{CJK}{UTF8}{mj}的秩当且仅当\end{CJK}
$$
V=\mathscr{A} V \oplus \mathscr{A}^{-1}(0),
$$
\begin{CJK}{UTF8}{mj}其中\end{CJK} $\mathscr{A} V$ \begin{CJK}{UTF8}{mj}表示\end{CJK} $\mathscr{A}$ \begin{CJK}{UTF8}{mj}的值域\end{CJK}, $\mathscr{A}^{-1}(0)$ \begin{CJK}{UTF8}{mj}表示\end{CJK} $\mathscr{A}$ \begin{CJK}{UTF8}{mj}的核\end{CJK}. \begin{CJK}{UTF8}{mj}七\end{CJK}. \begin{CJK}{UTF8}{mj}设\end{CJK} $P$ \begin{CJK}{UTF8}{mj}为一数域\end{CJK}, \begin{CJK}{UTF8}{mj}集合\end{CJK}
$$
V=\left\{\left(\begin{array}{cc}
a & b \\
-b & c
\end{array}\right) \mid a, b, c \in P\right\}
$$
(1) \begin{CJK}{UTF8}{mj}证明\end{CJK}: $V$ \begin{CJK}{UTF8}{mj}对于矩阵的加法和数量乘法构成一个线性空间\end{CJK}.

(2) \begin{CJK}{UTF8}{mj}令\end{CJK} $A=\left(\begin{array}{cc}1 & 1 \\ -1 & 1\end{array}\right)$, \begin{CJK}{UTF8}{mj}变换\end{CJK} $\mathscr{A}: V \rightarrow V$ \begin{CJK}{UTF8}{mj}定义为\end{CJK}: $\mathscr{A}(X)=A X+X A, \forall X \in V$. \begin{CJK}{UTF8}{mj}证明\end{CJK}: $\mathscr{A}$ \begin{CJK}{UTF8}{mj}是\end{CJK} $V$ \begin{CJK}{UTF8}{mj}的线性变换\end{CJK}.

(3) \begin{CJK}{UTF8}{mj}写出\end{CJK} $V$ \begin{CJK}{UTF8}{mj}的一组基\end{CJK}, \begin{CJK}{UTF8}{mj}并求\end{CJK} $\mathscr{A}$ \begin{CJK}{UTF8}{mj}在这一组基下的矩阵\end{CJK}.

\begin{CJK}{UTF8}{mj}八\end{CJK}. \begin{CJK}{UTF8}{mj}设\end{CJK}
$$
A=\left(\begin{array}{lll}
2 & 1 & 1 \\
1 & a & b \\
1 & b & 2
\end{array}\right)
$$
\begin{CJK}{UTF8}{mj}与\end{CJK}
$$
B=\left(\begin{array}{lll}
1 & 0 & 0 \\
0 & 1 & 0 \\
0 & 0 & 4
\end{array}\right)
$$
\begin{CJK}{UTF8}{mj}相似\end{CJK}.

(1) \begin{CJK}{UTF8}{mj}求\end{CJK} $a$ \begin{CJK}{UTF8}{mj}与\end{CJK} $b$ \begin{CJK}{UTF8}{mj}的值\end{CJK}.

(2) \begin{CJK}{UTF8}{mj}求正交矩阵\end{CJK} $Q$ \begin{CJK}{UTF8}{mj}使得\end{CJK} $Q^{\prime} A Q=B$.

\begin{CJK}{UTF8}{mj}九\end{CJK}. \begin{CJK}{UTF8}{mj}设正整数\end{CJK} $m>1$, \begin{CJK}{UTF8}{mj}多项式\end{CJK}
$$
f(x)=x^{5}+m x+1 .
$$
\begin{CJK}{UTF8}{mj}证明\end{CJK}: $f(x)$ \begin{CJK}{UTF8}{mj}在有理数域上是不可约的\end{CJK}.

\begin{CJK}{UTF8}{mj}十\end{CJK}. \begin{CJK}{UTF8}{mj}设矩阵\end{CJK} $A$ \begin{CJK}{UTF8}{mj}的特征多项式及最小多项式分别为\end{CJK}
$$
\begin{array}{r}
f(\lambda)=(\lambda+1)^{3}(\lambda-1)^{2}(\lambda-2), \\
m(\lambda)=(\lambda+1)^{2}(\lambda-1)(\lambda-2) .
\end{array}
$$
\begin{CJK}{UTF8}{mj}分别求出\end{CJK} $A$ \begin{CJK}{UTF8}{mj}的行列式因子\end{CJK}, \begin{CJK}{UTF8}{mj}不变因子\end{CJK}, \begin{CJK}{UTF8}{mj}初等因子及\end{CJK} Jordan \begin{CJK}{UTF8}{mj}标准型\end{CJK}.

\section{7. 湘潭大学 2015 年研究生入学考试试题高等代数}
\begin{CJK}{UTF8}{mj}李扬\end{CJK}

\begin{CJK}{UTF8}{mj}微信公众号\end{CJK}: sxkyliyang

\begin{CJK}{UTF8}{mj}本试题共\end{CJK} 10 \begin{CJK}{UTF8}{mj}大题\end{CJK}, \begin{CJK}{UTF8}{mj}每个大题满分均为\end{CJK} 15 \begin{CJK}{UTF8}{mj}分\end{CJK}, \begin{CJK}{UTF8}{mj}总分共\end{CJK} 150 \begin{CJK}{UTF8}{mj}分\end{CJK}.

\begin{CJK}{UTF8}{mj}一\end{CJK}. \begin{CJK}{UTF8}{mj}令\end{CJK} $A=\left(a_{i j}\right)$ \begin{CJK}{UTF8}{mj}为\end{CJK} $n(n \geqslant 2)$ \begin{CJK}{UTF8}{mj}级方阵\end{CJK}, $A^{*}$ \begin{CJK}{UTF8}{mj}为\end{CJK} $A$ \begin{CJK}{UTF8}{mj}的伴随矩阵\end{CJK}, \begin{CJK}{UTF8}{mj}证明\end{CJK}:

(1) \begin{CJK}{UTF8}{mj}如果\end{CJK} $|A|=0$, \begin{CJK}{UTF8}{mj}则\end{CJK} $\mathrm{r}\left(A^{*}\right)=1$ \begin{CJK}{UTF8}{mj}或\end{CJK} 0 .

(2) \begin{CJK}{UTF8}{mj}令\end{CJK} $M$ \begin{CJK}{UTF8}{mj}表示划掉\end{CJK} $A^{*}$ \begin{CJK}{UTF8}{mj}的第\end{CJK} $i$ \begin{CJK}{UTF8}{mj}行和第\end{CJK} $i$ \begin{CJK}{UTF8}{mj}列所得到的\end{CJK} $n-1$ \begin{CJK}{UTF8}{mj}阶子式\end{CJK}, \begin{CJK}{UTF8}{mj}则\end{CJK} $M=a_{i i}|A|^{n-2}$.

\begin{CJK}{UTF8}{mj}二\end{CJK}. \begin{CJK}{UTF8}{mj}令\end{CJK} $\varepsilon_{1}, \varepsilon_{2}, \cdots, \varepsilon_{n}$ \begin{CJK}{UTF8}{mj}为线性无关的\end{CJK} $n$ \begin{CJK}{UTF8}{mj}维列向量组\end{CJK}, \begin{CJK}{UTF8}{mj}讨论向量组\end{CJK}
$$
\varepsilon_{1}+\varepsilon_{2}, \varepsilon_{2}+\varepsilon_{3}, \cdots, \varepsilon_{n-1}+\varepsilon_{n}, \varepsilon_{n}+\varepsilon_{1}
$$
\begin{CJK}{UTF8}{mj}的线性相关性\end{CJK}.

\begin{CJK}{UTF8}{mj}三\end{CJK}. \begin{CJK}{UTF8}{mj}令\end{CJK} $A$ \begin{CJK}{UTF8}{mj}与\end{CJK} $B$ \begin{CJK}{UTF8}{mj}为\end{CJK} $n$ \begin{CJK}{UTF8}{mj}阶实对称矩阵且\end{CJK} $A B=B A$. \begin{CJK}{UTF8}{mj}证明\end{CJK}: \begin{CJK}{UTF8}{mj}存在\end{CJK} $n$ \begin{CJK}{UTF8}{mj}阶的正交矩阵\end{CJK} $Q$ \begin{CJK}{UTF8}{mj}使得\end{CJK} $Q^{\prime} A Q$ \begin{CJK}{UTF8}{mj}与\end{CJK} $Q^{\prime} B Q$ \begin{CJK}{UTF8}{mj}同时为对角\end{CJK} \begin{CJK}{UTF8}{mj}阵\end{CJK}.

\begin{CJK}{UTF8}{mj}四\end{CJK}. \begin{CJK}{UTF8}{mj}令\end{CJK} $A=\left(a_{i j}\right)$ \begin{CJK}{UTF8}{mj}为\end{CJK} $n(n \geqslant 2)$ \begin{CJK}{UTF8}{mj}级方阵\end{CJK}, $\lambda$ \begin{CJK}{UTF8}{mj}为\end{CJK} $A$ \begin{CJK}{UTF8}{mj}的任意特征值\end{CJK}, \begin{CJK}{UTF8}{mj}证明\end{CJK}:

(1) \begin{CJK}{UTF8}{mj}对于某正整数\end{CJK} $1 \leqslant k \leqslant n$ \begin{CJK}{UTF8}{mj}有\end{CJK}:
$$
\left|\lambda-a_{k k}\right| \leqslant \sum_{j=1, j \neq k}^{n}\left|a_{k j}\right|
$$
(2) \begin{CJK}{UTF8}{mj}如果\end{CJK} $A$ \begin{CJK}{UTF8}{mj}满足\end{CJK}:

\begin{CJK}{UTF8}{mj}则\end{CJK} $A$ \begin{CJK}{UTF8}{mj}是非奇异的\end{CJK}, \begin{CJK}{UTF8}{mj}即\end{CJK} $|A| \neq 0$.

\includegraphics[max width=\textwidth]{2022_04_18_3416d289b173eb9de8c1g-059}

\begin{CJK}{UTF8}{mj}五\end{CJK}. \begin{CJK}{UTF8}{mj}证明\end{CJK}: $n$ \begin{CJK}{UTF8}{mj}阶矩阵\end{CJK} $A$ \begin{CJK}{UTF8}{mj}为正定矩阵当且仅当存在\end{CJK} $n$ \begin{CJK}{UTF8}{mj}阶可逆矩阵\end{CJK} $Q$ \begin{CJK}{UTF8}{mj}使得\end{CJK}
$$
A=Q^{\prime} Q
$$
\begin{CJK}{UTF8}{mj}六\end{CJK}. \begin{CJK}{UTF8}{mj}令\end{CJK} $M_{n}(\mathbb{R})$ \begin{CJK}{UTF8}{mj}为实数域\end{CJK} $\mathbb{R}$ \begin{CJK}{UTF8}{mj}上的所有\end{CJK} $n$ \begin{CJK}{UTF8}{mj}阶矩阵构成的线性空间\end{CJK}, \begin{CJK}{UTF8}{mj}变换\end{CJK} $\mathscr{A}: M_{n}(\mathbb{R}) \rightarrow M_{n}(\mathbb{R})$ \begin{CJK}{UTF8}{mj}定义为\end{CJK}:
$$
\mathscr{A}(X)=A X+X A, \forall X \in M_{n}(\mathbb{R}),
$$
\begin{CJK}{UTF8}{mj}其中\end{CJK} $A \in M_{n}(\mathbb{R})$ \begin{CJK}{UTF8}{mj}为实对称矩阵\end{CJK}.

(1) \begin{CJK}{UTF8}{mj}给出\end{CJK} $M_{n}(\mathbb{R})$ \begin{CJK}{UTF8}{mj}的维数和一组基\end{CJK}.

(2) \begin{CJK}{UTF8}{mj}证明\end{CJK}: $\mathscr{A}$ \begin{CJK}{UTF8}{mj}为线性变换\end{CJK}.

(3) \begin{CJK}{UTF8}{mj}证明\end{CJK}: $\mathscr{A}$ \begin{CJK}{UTF8}{mj}可对角化\end{CJK}, \begin{CJK}{UTF8}{mj}即\end{CJK} $\mathscr{A}$ \begin{CJK}{UTF8}{mj}在\end{CJK} $M_{n}(\mathbb{R})$ \begin{CJK}{UTF8}{mj}的某一组基下的矩阵为对角阵\end{CJK}.

\begin{CJK}{UTF8}{mj}七\end{CJK}. \begin{CJK}{UTF8}{mj}令\end{CJK} $V$ \begin{CJK}{UTF8}{mj}为复数域上的\end{CJK} $n$ \begin{CJK}{UTF8}{mj}维线性空间\end{CJK}, \begin{CJK}{UTF8}{mj}线性变换\end{CJK} $\mathscr{A}$ \begin{CJK}{UTF8}{mj}在\end{CJK} $V$ \begin{CJK}{UTF8}{mj}中一组基\end{CJK} $\alpha_{1}, \alpha_{2}, \cdots, \alpha_{n}$ \begin{CJK}{UTF8}{mj}下的矩阵为一下三角的若尔\end{CJK} \begin{CJK}{UTF8}{mj}当块\end{CJK}, \begin{CJK}{UTF8}{mj}证明\end{CJK}:

(1) $V_{i}=L\left(\alpha_{i}, \alpha_{i+1}, \cdots, \alpha_{n}\right)$ \begin{CJK}{UTF8}{mj}为\end{CJK} $\mathscr{A}$ \begin{CJK}{UTF8}{mj}不变子空间\end{CJK}.

(2) $V$ \begin{CJK}{UTF8}{mj}中任意非零的\end{CJK} $\mathscr{A}$ \begin{CJK}{UTF8}{mj}不变子空间都包含有\end{CJK} $\alpha_{n}$.

\begin{CJK}{UTF8}{mj}八\end{CJK}. \begin{CJK}{UTF8}{mj}令\end{CJK} $n$ \begin{CJK}{UTF8}{mj}阶矩阵\end{CJK}
$$
A=\left(\begin{array}{cccc}
2 & a & \cdots & a \\
a & 2 & \cdots & a \\
\vdots & \vdots & & \vdots \\
a & a & \cdots & 2
\end{array}\right)
$$
(1) \begin{CJK}{UTF8}{mj}求\end{CJK} $A$ \begin{CJK}{UTF8}{mj}的所有特征值和特征向量\end{CJK}.

(2) \begin{CJK}{UTF8}{mj}求可逆矩阵\end{CJK} $P$ \begin{CJK}{UTF8}{mj}使得\end{CJK} $P^{-1} A P$ \begin{CJK}{UTF8}{mj}为对角阵\end{CJK}.

\begin{CJK}{UTF8}{mj}九\end{CJK}. \begin{CJK}{UTF8}{mj}令\end{CJK} $\alpha_{1}, \alpha_{2}, \cdots, \alpha_{n}$ \begin{CJK}{UTF8}{mj}为数域\end{CJK} $P$ \begin{CJK}{UTF8}{mj}上的\end{CJK} $n$ \begin{CJK}{UTF8}{mj}维线性空间\end{CJK} $V$ \begin{CJK}{UTF8}{mj}的一组基\end{CJK}, $V_{1}$ \begin{CJK}{UTF8}{mj}表示由\end{CJK} $\alpha_{1}+\alpha_{2}+\cdots+\alpha_{n}$ \begin{CJK}{UTF8}{mj}生成的子空间\end{CJK}, \begin{CJK}{UTF8}{mj}以及\end{CJK}
$$
V_{2}=\left\{\sum_{i=1}^{n} x_{i} \alpha_{i} \mid \sum_{i=1}^{n} x_{i}=0, x_{i} \in P\right\}
$$
\begin{CJK}{UTF8}{mj}证明\end{CJK}:

(1) $V_{2}$ \begin{CJK}{UTF8}{mj}为\end{CJK} $V$ \begin{CJK}{UTF8}{mj}的子空间\end{CJK}.

(2) $V=V_{1} \oplus V_{2}$.

\begin{CJK}{UTF8}{mj}十\end{CJK}. \begin{CJK}{UTF8}{mj}令\end{CJK} $A$ \begin{CJK}{UTF8}{mj}为\end{CJK} $n$ \begin{CJK}{UTF8}{mj}阶实对称矩阵\end{CJK}, \begin{CJK}{UTF8}{mj}其特征值为\end{CJK} $\lambda_{1} \geqslant \lambda_{2} \geqslant \cdots \geqslant \lambda_{n}$, \begin{CJK}{UTF8}{mj}其相应的标准正交特征向量为\end{CJK} $\alpha_{1}, \alpha_{2}, \cdots, \alpha_{n}$. \begin{CJK}{UTF8}{mj}记\end{CJK} $V_{i}^{j}=L\left(\alpha_{i}, \alpha_{i+1}, \cdots, \alpha_{j}\right)$ \begin{CJK}{UTF8}{mj}为实数域上的生成子空间\end{CJK}. \begin{CJK}{UTF8}{mj}证明\end{CJK}:
$$
\lambda_{i}=\max _{X \in V_{i}^{j}, X \neq 0} \frac{X^{\prime} A X}{X^{\prime} X}, \lambda_{j}=\min _{X \in V_{i}^{j}, X \neq 0} \frac{X^{\prime} A X}{X^{\prime} X}
$$

\section{8. 湘潭大学 2016 年研究生入学考试试题高等代数}
\begin{CJK}{UTF8}{mj}李扬\end{CJK}

\begin{CJK}{UTF8}{mj}微信公众号\end{CJK}: sxkyliyang

\begin{CJK}{UTF8}{mj}本试题共\end{CJK} 10 \begin{CJK}{UTF8}{mj}大题\end{CJK}, \begin{CJK}{UTF8}{mj}每个大题满分均为\end{CJK} 15 \begin{CJK}{UTF8}{mj}分\end{CJK}, \begin{CJK}{UTF8}{mj}总分共\end{CJK} 150 \begin{CJK}{UTF8}{mj}分\end{CJK}.

\begin{CJK}{UTF8}{mj}一\end{CJK}. \begin{CJK}{UTF8}{mj}令\end{CJK} $A$ \begin{CJK}{UTF8}{mj}为\end{CJK} $n$ \begin{CJK}{UTF8}{mj}阶复矩阵\end{CJK}, \begin{CJK}{UTF8}{mj}证明\end{CJK}:

(1) $A$ \begin{CJK}{UTF8}{mj}的最小多项式\end{CJK} $g(x)$ \begin{CJK}{UTF8}{mj}整除\end{CJK} $A$ \begin{CJK}{UTF8}{mj}的特征多项式\end{CJK} $f(x)$.

(2) \begin{CJK}{UTF8}{mj}如果\end{CJK} $A$ \begin{CJK}{UTF8}{mj}可逆\end{CJK}, \begin{CJK}{UTF8}{mj}则存在次数不超过\end{CJK} $n-1$ \begin{CJK}{UTF8}{mj}的多项式\end{CJK} $p(x)$ \begin{CJK}{UTF8}{mj}使得\end{CJK} $A^{-1}=p(A)$.

\begin{CJK}{UTF8}{mj}二\end{CJK}. \begin{CJK}{UTF8}{mj}令矩阵\end{CJK}
$$
C=\left(\begin{array}{ll}
A & 0 \\
0 & B
\end{array}\right) \text {, }
$$
\begin{CJK}{UTF8}{mj}其中\end{CJK} $A$ \begin{CJK}{UTF8}{mj}为\end{CJK} $n$ \begin{CJK}{UTF8}{mj}阶矩阵\end{CJK}, $B$ \begin{CJK}{UTF8}{mj}为\end{CJK} $m$ \begin{CJK}{UTF8}{mj}阶矩阵\end{CJK}. \begin{CJK}{UTF8}{mj}证明\end{CJK}: $C$ \begin{CJK}{UTF8}{mj}相似于对角阵当且仅当\end{CJK} $A$ \begin{CJK}{UTF8}{mj}与\end{CJK} $B$ \begin{CJK}{UTF8}{mj}都相似于对角阵\end{CJK}.

\begin{CJK}{UTF8}{mj}三\end{CJK}. \begin{CJK}{UTF8}{mj}令\end{CJK} $A=\left(a_{i j}\right)$ \begin{CJK}{UTF8}{mj}与\end{CJK} $B=Q_{1} A Q_{2}=\left(b_{i j}\right)$ \begin{CJK}{UTF8}{mj}均为\end{CJK} $n$ \begin{CJK}{UTF8}{mj}阶实矩阵\end{CJK}, \begin{CJK}{UTF8}{mj}其中\end{CJK} $Q_{1}$ \begin{CJK}{UTF8}{mj}与\end{CJK} $Q_{2}$ \begin{CJK}{UTF8}{mj}都是\end{CJK} $n$ \begin{CJK}{UTF8}{mj}阶正交矩阵\end{CJK}. \begin{CJK}{UTF8}{mj}证明\end{CJK}:

(1)
$$
\operatorname{tr}\left(A A^{\prime}\right)=\operatorname{tr}\left(A^{\prime} A\right)
$$
\begin{CJK}{UTF8}{mj}其中\end{CJK} $\operatorname{tr}(\cdot)$ \begin{CJK}{UTF8}{mj}表示为矩阵的迹\end{CJK}, $A^{\prime}$ \begin{CJK}{UTF8}{mj}表示为\end{CJK} $A$ \begin{CJK}{UTF8}{mj}的转置矩阵\end{CJK}.
$$
\sum_{i, j=1}^{n}\left|a_{i j}\right|^{2}=\sum_{i, j=1}^{n}\left|b_{i j}\right|^{2} .
$$
\begin{CJK}{UTF8}{mj}四\end{CJK}. \begin{CJK}{UTF8}{mj}令\end{CJK} $n$ \begin{CJK}{UTF8}{mj}阶复矩阵\end{CJK} $A$ \begin{CJK}{UTF8}{mj}的\end{CJK} Jordan \begin{CJK}{UTF8}{mj}标准型为\end{CJK} $J, J_{m}(\lambda)$ \begin{CJK}{UTF8}{mj}为\end{CJK} $J$ \begin{CJK}{UTF8}{mj}中含有特征值\end{CJK} $\lambda$, \begin{CJK}{UTF8}{mj}阶数为\end{CJK} $m$ \begin{CJK}{UTF8}{mj}的\end{CJK} Jordan \begin{CJK}{UTF8}{mj}块\end{CJK}. \begin{CJK}{UTF8}{mj}证明\end{CJK}:

(1) \begin{CJK}{UTF8}{mj}如果\end{CJK} $k \leqslant m$ \begin{CJK}{UTF8}{mj}时\end{CJK}, \begin{CJK}{UTF8}{mj}则\end{CJK}
$$
\mathrm{r}\left(J_{m}(\lambda)-\lambda I_{m}\right)^{k-1}=\mathrm{r}\left(J_{m}-\lambda I_{m}\right)^{k}+1
$$
\begin{CJK}{UTF8}{mj}其中\end{CJK} $I_{m}$ \begin{CJK}{UTF8}{mj}表示为\end{CJK} $m$ \begin{CJK}{UTF8}{mj}阶单位矩阵\end{CJK}.

(2) \begin{CJK}{UTF8}{mj}令\end{CJK} Jordan \begin{CJK}{UTF8}{mj}标准型\end{CJK} $J$ \begin{CJK}{UTF8}{mj}中\end{CJK} $J_{m}(\lambda)$ \begin{CJK}{UTF8}{mj}的块数为\end{CJK} $s$, \begin{CJK}{UTF8}{mj}则\end{CJK}
$$
s=\mathrm{r}\left(A-\lambda I_{n}\right)^{m-1}-2 \mathrm{r}\left(A-\lambda I_{n}\right)^{m}+\mathrm{r}\left(A-\lambda I_{n}\right)^{m+1} .
$$
\begin{CJK}{UTF8}{mj}五\end{CJK}. \begin{CJK}{UTF8}{mj}令\end{CJK} $A$ \begin{CJK}{UTF8}{mj}为\end{CJK} $n$ \begin{CJK}{UTF8}{mj}阶可逆的实矩阵\end{CJK}, \begin{CJK}{UTF8}{mj}证明\end{CJK}:

(1) \begin{CJK}{UTF8}{mj}存在\end{CJK} $n$ \begin{CJK}{UTF8}{mj}阶正交矩阵\end{CJK} $Q$ \begin{CJK}{UTF8}{mj}和上三角矩阵\end{CJK} $R$ \begin{CJK}{UTF8}{mj}使得\end{CJK} $A=Q R$.

(2) \begin{CJK}{UTF8}{mj}记\end{CJK} $A_{1}=A$, \begin{CJK}{UTF8}{mj}作变换\end{CJK}
$$
\left\{\begin{array}{l}
A_{k}=Q_{k} R_{k} ; \\
A_{k+1}=R_{k} Q_{k}
\end{array} \quad k=1,2, \cdots\right.
$$
\begin{CJK}{UTF8}{mj}其中\end{CJK} $Q_{k}$ \begin{CJK}{UTF8}{mj}为\end{CJK} $n$ \begin{CJK}{UTF8}{mj}阶正交矩阵和\end{CJK} $R_{k}$ \begin{CJK}{UTF8}{mj}为\end{CJK} $n$ \begin{CJK}{UTF8}{mj}阶上三角矩阵\end{CJK}, \begin{CJK}{UTF8}{mj}则\end{CJK} $A_{k}(k=1,2, \cdots)$ \begin{CJK}{UTF8}{mj}正交相似于\end{CJK} $A$.

\begin{CJK}{UTF8}{mj}六\end{CJK}. \begin{CJK}{UTF8}{mj}令\end{CJK} $P^{2 \times 2}$ \begin{CJK}{UTF8}{mj}表示数域\end{CJK} $P$ \begin{CJK}{UTF8}{mj}上全体\end{CJK} 2 \begin{CJK}{UTF8}{mj}阶矩阵所构成的线性空间\end{CJK}, \begin{CJK}{UTF8}{mj}定义\end{CJK} $P^{2 \times 2}$ \begin{CJK}{UTF8}{mj}上的线性变换为\end{CJK}:
$$
\mathscr{A}(X)=\left(\begin{array}{ll}
3 & 4 \\
5 & 2
\end{array}\right) X, \forall X \in P^{2 \times 2}
$$
(1) \begin{CJK}{UTF8}{mj}给出\end{CJK} $P^{2 \times 2}$ \begin{CJK}{UTF8}{mj}的一组基及线性变换\end{CJK} $\mathscr{A}$ \begin{CJK}{UTF8}{mj}在这一组基下的矩阵\end{CJK}.

(2) \begin{CJK}{UTF8}{mj}求\end{CJK} $P^{2 \times 2}$ \begin{CJK}{UTF8}{mj}的一组基\end{CJK}, \begin{CJK}{UTF8}{mj}使得线性变换\end{CJK} $\mathscr{A}$ \begin{CJK}{UTF8}{mj}在这一组基下的矩阵为对角阵\end{CJK}. \begin{CJK}{UTF8}{mj}七\end{CJK}. \begin{CJK}{UTF8}{mj}令\end{CJK} $P[x]_{n}$ \begin{CJK}{UTF8}{mj}为数域\end{CJK} $P$ \begin{CJK}{UTF8}{mj}上次数小于\end{CJK} $n$ \begin{CJK}{UTF8}{mj}的多项式所构成的线性空间\end{CJK}, \begin{CJK}{UTF8}{mj}定义\end{CJK} $P[x]_{n}$ \begin{CJK}{UTF8}{mj}上的变换为\end{CJK}
$$
\mathscr{A}(f(x))=(n-1) f(x)-x f^{\prime}(x), \forall f(x) \in P[x]_{n}
$$
\begin{CJK}{UTF8}{mj}其中\end{CJK} $f^{\prime}(x)$ \begin{CJK}{UTF8}{mj}为\end{CJK} $f(x)$ \begin{CJK}{UTF8}{mj}的微商\end{CJK}. \begin{CJK}{UTF8}{mj}证明\end{CJK}:

(1) $\mathscr{A}$ \begin{CJK}{UTF8}{mj}是\end{CJK} $P[x]_{n}$ \begin{CJK}{UTF8}{mj}上的线性变换\end{CJK}.

(2) \begin{CJK}{UTF8}{mj}求线性变换\end{CJK} $\mathscr{A}$ \begin{CJK}{UTF8}{mj}的值域\end{CJK} $\mathscr{A} P[x]_{n}$ \begin{CJK}{UTF8}{mj}与核\end{CJK} $\mathscr{A}^{-1}(0)$.

(3) \begin{CJK}{UTF8}{mj}证明\end{CJK}: $P[x]_{n}=\mathscr{A} P[x]_{n} \oplus \mathscr{A}^{-1}(0)$.

\begin{CJK}{UTF8}{mj}八\end{CJK}. \begin{CJK}{UTF8}{mj}令\end{CJK} $\mathscr{A}$ \begin{CJK}{UTF8}{mj}为\end{CJK} $n$ \begin{CJK}{UTF8}{mj}维欧式空间\end{CJK} $V$ \begin{CJK}{UTF8}{mj}上的线性变换\end{CJK}, \begin{CJK}{UTF8}{mj}如果对于任意\end{CJK} $\alpha, \beta \in V$ \begin{CJK}{UTF8}{mj}都有\end{CJK}:
$$
(\mathscr{A} \alpha, \beta)=(\alpha, \mathscr{A} \beta)
$$
\begin{CJK}{UTF8}{mj}则称\end{CJK} $\mathscr{A}$ \begin{CJK}{UTF8}{mj}为对称变换\end{CJK}. \begin{CJK}{UTF8}{mj}证明\end{CJK}:

(1) $\mathscr{A}$ \begin{CJK}{UTF8}{mj}为对称变换当且仅当\end{CJK} $\mathscr{A}$ \begin{CJK}{UTF8}{mj}在\end{CJK} $V$ \begin{CJK}{UTF8}{mj}的一组标准正交基下的矩阵是对称矩阵\end{CJK}.

(2) \begin{CJK}{UTF8}{mj}如果\end{CJK} $V_{1}$ \begin{CJK}{UTF8}{mj}是\end{CJK} $\mathscr{A}$-\begin{CJK}{UTF8}{mj}子空间\end{CJK}, \begin{CJK}{UTF8}{mj}则正交补\end{CJK} $V_{1}^{\perp}$ \begin{CJK}{UTF8}{mj}也是\end{CJK} $\mathscr{A}$-\begin{CJK}{UTF8}{mj}子空间\end{CJK}.

\begin{CJK}{UTF8}{mj}九\end{CJK}. \begin{CJK}{UTF8}{mj}令\end{CJK} $A$ \begin{CJK}{UTF8}{mj}为\end{CJK} $n$ \begin{CJK}{UTF8}{mj}阶实正定矩阵\end{CJK}, \begin{CJK}{UTF8}{mj}证明\end{CJK}:

(1) \begin{CJK}{UTF8}{mj}存在可逆的对称矩阵\end{CJK} $S$ \begin{CJK}{UTF8}{mj}使得\end{CJK} $A=S^{2}$.

(2) \begin{CJK}{UTF8}{mj}如果\end{CJK} $A, B$ \begin{CJK}{UTF8}{mj}均为\end{CJK} $n$ \begin{CJK}{UTF8}{mj}阶实正定矩阵\end{CJK}, \begin{CJK}{UTF8}{mj}且\end{CJK} $A B=B A$, \begin{CJK}{UTF8}{mj}则\end{CJK} $A B$ \begin{CJK}{UTF8}{mj}为正定矩阵\end{CJK}.

\begin{CJK}{UTF8}{mj}十\end{CJK}. \begin{CJK}{UTF8}{mj}令\end{CJK} $A$ \begin{CJK}{UTF8}{mj}为\end{CJK} $n$ \begin{CJK}{UTF8}{mj}阶实对称矩阵\end{CJK}, \begin{CJK}{UTF8}{mj}其特征值为\end{CJK} $\lambda_{1} \geqslant \lambda_{2} \geqslant \cdots \geqslant \lambda_{n}$, \begin{CJK}{UTF8}{mj}证明\end{CJK}:
$$
\lambda_{1}=\max _{X \in \mathbb{R}^{n}, X \neq 0} \frac{X^{\prime} A X}{X^{\prime} X}, \lambda_{n}=\min _{X \in \mathbb{R}^{n}, X \neq 0} \frac{X^{\prime} A X}{X^{\prime} X} .
$$

\section{9. 湘潭大学 2017 年研究生入学考试试题高等代数}
\begin{CJK}{UTF8}{mj}李扬\end{CJK}

\begin{CJK}{UTF8}{mj}微信公众号\end{CJK}: sxkyliyang

\begin{CJK}{UTF8}{mj}本试题共\end{CJK} 10 \begin{CJK}{UTF8}{mj}大题\end{CJK}, \begin{CJK}{UTF8}{mj}每个大题满分均为\end{CJK} 15 \begin{CJK}{UTF8}{mj}分\end{CJK}, \begin{CJK}{UTF8}{mj}总分共\end{CJK} 150 \begin{CJK}{UTF8}{mj}分\end{CJK}.

\begin{CJK}{UTF8}{mj}一\end{CJK}. \begin{CJK}{UTF8}{mj}令\end{CJK} $x_{1}, x_{2}, \cdots, x_{n}$ \begin{CJK}{UTF8}{mj}与\end{CJK} $y_{1}, y_{2}, \cdots, y_{n}$ \begin{CJK}{UTF8}{mj}为\end{CJK} $2 n$ \begin{CJK}{UTF8}{mj}个数\end{CJK}, \begin{CJK}{UTF8}{mj}其中\end{CJK} $x_{1}, x_{2}, \cdots, x_{n}$ \begin{CJK}{UTF8}{mj}互不相同\end{CJK}. \begin{CJK}{UTF8}{mj}证明\end{CJK}: \begin{CJK}{UTF8}{mj}存在次数不超过\end{CJK} $n-1$ \begin{CJK}{UTF8}{mj}次的多项式\end{CJK} $p(x)$ \begin{CJK}{UTF8}{mj}使得\end{CJK}:
$$
p\left(x_{i}\right)=y_{i}, i=1,2, \cdots, n
$$
\begin{CJK}{UTF8}{mj}二\end{CJK}. \begin{CJK}{UTF8}{mj}令矩阵\end{CJK} $G=E_{n}-u v^{\prime}$, \begin{CJK}{UTF8}{mj}其中\end{CJK} $E_{n}$ \begin{CJK}{UTF8}{mj}为单位矩阵\end{CJK}, $u, v$ \begin{CJK}{UTF8}{mj}为\end{CJK} $n$ \begin{CJK}{UTF8}{mj}维非零列向量\end{CJK}. \begin{CJK}{UTF8}{mj}记\end{CJK}
$$
N(G)=\{y \mid G y=0\},
$$
$L(u)$ \begin{CJK}{UTF8}{mj}表示由\end{CJK} $u$ \begin{CJK}{UTF8}{mj}生成的子空间\end{CJK}. \begin{CJK}{UTF8}{mj}证明\end{CJK}:

(1) $N(G)=L(u)$ \begin{CJK}{UTF8}{mj}当且仅当\end{CJK} $v^{\prime} u=1$.

(2) \begin{CJK}{UTF8}{mj}令\end{CJK} $x=\left(x_{1}, \cdots, x_{n}\right)^{\prime}$, \begin{CJK}{UTF8}{mj}其中\end{CJK} $x_{1} \neq 0$, \begin{CJK}{UTF8}{mj}则存在\end{CJK} $G=E_{n}-u v^{\prime}$ \begin{CJK}{UTF8}{mj}使得\end{CJK} $G x=x_{1} e_{1}$, \begin{CJK}{UTF8}{mj}其中\end{CJK} $e_{1}$ \begin{CJK}{UTF8}{mj}为\end{CJK} $E_{n}$ \begin{CJK}{UTF8}{mj}的第一列\end{CJK}.

\begin{CJK}{UTF8}{mj}三\end{CJK}. \begin{CJK}{UTF8}{mj}令\end{CJK} $\lambda$ \begin{CJK}{UTF8}{mj}为\end{CJK} $n$ \begin{CJK}{UTF8}{mj}阶矩阵\end{CJK} $A$ \begin{CJK}{UTF8}{mj}的特征值\end{CJK}, \begin{CJK}{UTF8}{mj}记\end{CJK}
$$
V_{\lambda}=\{x \mid A X=\lambda X\}
$$
\begin{CJK}{UTF8}{mj}证明\end{CJK}: \begin{CJK}{UTF8}{mj}如果\end{CJK} $V_{\lambda}$ \begin{CJK}{UTF8}{mj}的维数\end{CJK} $\geqslant k$, \begin{CJK}{UTF8}{mj}则\end{CJK} $\lambda$ \begin{CJK}{UTF8}{mj}为\end{CJK} $m$ \begin{CJK}{UTF8}{mj}阶顺序主子阵\end{CJK} $A_{m}$ \begin{CJK}{UTF8}{mj}的特征值\end{CJK}, \begin{CJK}{UTF8}{mj}其中\end{CJK} $m>n-k$.

\begin{CJK}{UTF8}{mj}四\end{CJK}. \begin{CJK}{UTF8}{mj}令\end{CJK} $A$ \begin{CJK}{UTF8}{mj}与\end{CJK} $B$ \begin{CJK}{UTF8}{mj}均为\end{CJK} $n$ \begin{CJK}{UTF8}{mj}级实正定矩阵\end{CJK}. \begin{CJK}{UTF8}{mj}证明\end{CJK}:

(1) \begin{CJK}{UTF8}{mj}存在可逆矩阵\end{CJK} $P$ \begin{CJK}{UTF8}{mj}使得\end{CJK}: $A=P^{\prime} P$.

(2) $B-A$ \begin{CJK}{UTF8}{mj}为正定矩阵当且仅当\end{CJK} $A B^{-1}$ \begin{CJK}{UTF8}{mj}的所有特征值均小于\end{CJK} 1 .

\begin{CJK}{UTF8}{mj}五\end{CJK}. \begin{CJK}{UTF8}{mj}令矩阵方程\end{CJK}

\includegraphics[max width=\textwidth]{2022_04_18_3416d289b173eb9de8c1g-063}

\begin{CJK}{UTF8}{mj}其中\end{CJK} $A$ \begin{CJK}{UTF8}{mj}为\end{CJK} $n$ \begin{CJK}{UTF8}{mj}阶矩阵\end{CJK}, $B$ \begin{CJK}{UTF8}{mj}为\end{CJK} $m$ \begin{CJK}{UTF8}{mj}阶矩阵\end{CJK}, $X$ \begin{CJK}{UTF8}{mj}为\end{CJK} $n \times m$ \begin{CJK}{UTF8}{mj}的末知矩阵\end{CJK}.

(1) \begin{CJK}{UTF8}{mj}叙述哈密顿\end{CJK}-\begin{CJK}{UTF8}{mj}凯莱定理\end{CJK}.

(2) \begin{CJK}{UTF8}{mj}证明\end{CJK}: \begin{CJK}{UTF8}{mj}如果\end{CJK} $A$ \begin{CJK}{UTF8}{mj}与\end{CJK} $B$ \begin{CJK}{UTF8}{mj}没有公共的特征值\end{CJK}, \begin{CJK}{UTF8}{mj}则该方程有唯一解\end{CJK}: $X=0_{n \times m}$.

\begin{CJK}{UTF8}{mj}六\end{CJK}. \begin{CJK}{UTF8}{mj}令\end{CJK} $n$ \begin{CJK}{UTF8}{mj}阶矩阵\end{CJK} $A$ \begin{CJK}{UTF8}{mj}的零特征值的个数为\end{CJK} $s$, \begin{CJK}{UTF8}{mj}其特征值子空间\end{CJK} $V=\{X \mid A X=0\}$ \begin{CJK}{UTF8}{mj}的维数为\end{CJK} $k$. \begin{CJK}{UTF8}{mj}证明\end{CJK}: $\mathrm{r}(A)=\mathrm{r}\left(A^{2}\right)$ \begin{CJK}{UTF8}{mj}当且仅当\end{CJK} $s=k$.

\begin{CJK}{UTF8}{mj}七\end{CJK}. \begin{CJK}{UTF8}{mj}令\end{CJK} $\lambda_{1} \geqslant \lambda_{2} \geqslant \cdots \geqslant \lambda_{n}$ \begin{CJK}{UTF8}{mj}为\end{CJK} $n$ \begin{CJK}{UTF8}{mj}阶实对称\end{CJK} $A$ \begin{CJK}{UTF8}{mj}的特征值\end{CJK}, \begin{CJK}{UTF8}{mj}其相应的单位正交特征向量为\end{CJK} $x_{1}, x_{2}, \cdots, x_{n}$, \begin{CJK}{UTF8}{mj}记\end{CJK} $V=L\left(x_{i}, x_{i+1}, \cdots, x_{j}\right)$, \begin{CJK}{UTF8}{mj}其中\end{CJK} $1 \leqslant i<j \leqslant n$. \begin{CJK}{UTF8}{mj}证明\end{CJK}:
$$
\lambda_{i}=\max _{0 \neq x \in V} \frac{x^{\prime} A x}{x^{\prime} x}, \lambda_{j}=\min _{0 \neq x \in V} \frac{x^{\prime} A x}{x^{\prime} x} .
$$
\begin{CJK}{UTF8}{mj}八\end{CJK}. \begin{CJK}{UTF8}{mj}令\end{CJK} $\lambda=\alpha+i \beta$ \begin{CJK}{UTF8}{mj}为\end{CJK} $n$ \begin{CJK}{UTF8}{mj}阶实矩阵\end{CJK} $A$ \begin{CJK}{UTF8}{mj}的特征值且\end{CJK} $\beta \neq 0$, \begin{CJK}{UTF8}{mj}及相应特征向量为\end{CJK} $v=x+i y$, \begin{CJK}{UTF8}{mj}其中\end{CJK} $i=\sqrt{-1}$. \begin{CJK}{UTF8}{mj}证明\end{CJK}:

(1) $L(x, y)$ \begin{CJK}{UTF8}{mj}是\end{CJK} $A$ \begin{CJK}{UTF8}{mj}的\end{CJK} 2 \begin{CJK}{UTF8}{mj}维不变子空间\end{CJK}.

(2) \begin{CJK}{UTF8}{mj}存在可逆矩阵\end{CJK} $X$ \begin{CJK}{UTF8}{mj}使得\end{CJK} $X^{-1} A X=\left(\begin{array}{cc}A_{11} & A_{12} \\ 0 & A_{22}\end{array}\right)$, \begin{CJK}{UTF8}{mj}其中\end{CJK} $A_{11}=\left(\begin{array}{cc}\alpha & \beta \\ -\beta & \alpha\end{array}\right)$. \begin{CJK}{UTF8}{mj}九\end{CJK}. \begin{CJK}{UTF8}{mj}令线性空间\end{CJK} $P^{2 \times 3}$ \begin{CJK}{UTF8}{mj}上变换\end{CJK}
$$
\mathscr{A}: X \rightarrow f(A) X
$$
\begin{CJK}{UTF8}{mj}其中\end{CJK} $f(x)=x^{2}+x+1, A=\left(\begin{array}{ll}1 & 1 \\ 1 & 1\end{array}\right)$. \begin{CJK}{UTF8}{mj}证明\end{CJK}:

(1) $\mathscr{A}$ \begin{CJK}{UTF8}{mj}是\end{CJK} $P^{2 \times 3}$ \begin{CJK}{UTF8}{mj}上的线性变换\end{CJK}.

(2) \begin{CJK}{UTF8}{mj}给出\end{CJK} $P^{2 \times 3}$ \begin{CJK}{UTF8}{mj}的一组基\end{CJK}, \begin{CJK}{UTF8}{mj}并求\end{CJK} $\mathscr{A}$ \begin{CJK}{UTF8}{mj}在一组基下的矩阵\end{CJK}.

(3) \begin{CJK}{UTF8}{mj}求出\end{CJK} $\mathscr{A}$ \begin{CJK}{UTF8}{mj}的特征值和相应特征向量\end{CJK}.

\begin{CJK}{UTF8}{mj}十\end{CJK}. \begin{CJK}{UTF8}{mj}令\end{CJK} $\alpha_{1}, \alpha_{2}, \cdots, \alpha_{n}$ \begin{CJK}{UTF8}{mj}为欧式空间\end{CJK} $\mathbb{R}^{n}$ \begin{CJK}{UTF8}{mj}的列向量\end{CJK}, \begin{CJK}{UTF8}{mj}定义\end{CJK} $n$ \begin{CJK}{UTF8}{mj}阶矩阵\end{CJK} $G=\left(g_{i j}\right)$, \begin{CJK}{UTF8}{mj}其中\end{CJK} $g_{i j}=\left(\alpha_{i}, \alpha_{j}\right)$ \begin{CJK}{UTF8}{mj}为\end{CJK} $\alpha_{i}$ \begin{CJK}{UTF8}{mj}与\end{CJK} $\alpha_{j}$ \begin{CJK}{UTF8}{mj}的内\end{CJK} \begin{CJK}{UTF8}{mj}积\end{CJK}. \begin{CJK}{UTF8}{mj}证明\end{CJK}:

(1) $G$ \begin{CJK}{UTF8}{mj}为正定矩阵当且仅当\end{CJK} $\alpha_{1}, \alpha_{2}, \cdots, \alpha_{n}$ \begin{CJK}{UTF8}{mj}线性无关\end{CJK}.

(2) \begin{CJK}{UTF8}{mj}令\end{CJK} $X=\left(\alpha_{1}, \alpha_{2}, \cdots, \alpha_{n}\right)$, \begin{CJK}{UTF8}{mj}则存在正定矩阵\end{CJK} $A$ \begin{CJK}{UTF8}{mj}使得\end{CJK}: $G=X^{\prime} A X$.

\section{0. 湘潭大学 2018 年研究生入学考试试题高等代数}
\begin{CJK}{UTF8}{mj}李扬\end{CJK}

\begin{CJK}{UTF8}{mj}微信公众号\end{CJK}: sxkyliyang

\begin{CJK}{UTF8}{mj}本试题共\end{CJK} 10 \begin{CJK}{UTF8}{mj}大题\end{CJK}, \begin{CJK}{UTF8}{mj}每个大题满分均为\end{CJK} 15 \begin{CJK}{UTF8}{mj}分\end{CJK}, \begin{CJK}{UTF8}{mj}总分共\end{CJK} 150 \begin{CJK}{UTF8}{mj}分\end{CJK}.

\begin{CJK}{UTF8}{mj}一\end{CJK}. \begin{CJK}{UTF8}{mj}已知向量\end{CJK} $\beta$ \begin{CJK}{UTF8}{mj}可以有向量组\end{CJK} $\alpha_{1}, \alpha_{2}, \cdots, \alpha_{s}$ \begin{CJK}{UTF8}{mj}线性表示\end{CJK}, \begin{CJK}{UTF8}{mj}但是不能由\end{CJK} $\alpha_{1}, \alpha_{2}, \cdots, \alpha_{s-1}$ \begin{CJK}{UTF8}{mj}线性表示\end{CJK}, \begin{CJK}{UTF8}{mj}证明向量组\end{CJK} $\alpha_{1}, \alpha_{2}, \cdots, \alpha_{s}$ \begin{CJK}{UTF8}{mj}与向量组\end{CJK} $\alpha_{1}, \alpha_{2}, \cdots, \alpha_{s-1}$ \begin{CJK}{UTF8}{mj}等价\end{CJK}.

\begin{CJK}{UTF8}{mj}二\end{CJK}. \begin{CJK}{UTF8}{mj}设四元齐次方程组\end{CJK} $(\mathrm{I})$ \begin{CJK}{UTF8}{mj}为\end{CJK}
$$
\left\{\begin{array}{l}
2 x_{1}+3 x_{2}-x_{3}=0 \\
x_{1}+2 x_{2}+x_{3}-x_{4}=0
\end{array}\right.
$$
\begin{CJK}{UTF8}{mj}并已知另一组四元齐次线性方程组\end{CJK} (II) \begin{CJK}{UTF8}{mj}的基础解析为\end{CJK}
$$
\eta_{1}=(2,-1, y+2,1)^{\prime}, \eta_{2}=(1,2,4, y+8)^{\prime}
$$
(1) \begin{CJK}{UTF8}{mj}求方程组\end{CJK} (I) \begin{CJK}{UTF8}{mj}的全部解\end{CJK}.

(2) \begin{CJK}{UTF8}{mj}求\end{CJK} $y$ \begin{CJK}{UTF8}{mj}为何值时\end{CJK}, \begin{CJK}{UTF8}{mj}两个方程组有公共的非零解\end{CJK}.

\begin{CJK}{UTF8}{mj}三\end{CJK}. \begin{CJK}{UTF8}{mj}设矩阵\end{CJK} $A=\left(a_{i j}\right)$ \begin{CJK}{UTF8}{mj}为数域\end{CJK} $P$ \begin{CJK}{UTF8}{mj}上的\end{CJK} $n$ \begin{CJK}{UTF8}{mj}级矩阵\end{CJK}, \begin{CJK}{UTF8}{mj}其中\end{CJK} $a_{i j}=a_{i}-b_{j}$, \begin{CJK}{UTF8}{mj}求\end{CJK} $A$ \begin{CJK}{UTF8}{mj}的行列式与特征多项式\end{CJK}.

\begin{CJK}{UTF8}{mj}四\end{CJK}. \begin{CJK}{UTF8}{mj}令\end{CJK} $A$ \begin{CJK}{UTF8}{mj}为\end{CJK} $n$ \begin{CJK}{UTF8}{mj}级复矩阵\end{CJK}, \begin{CJK}{UTF8}{mj}证明\end{CJK}

(1) $A$ \begin{CJK}{UTF8}{mj}的最小多项式\end{CJK} $g(x)$ \begin{CJK}{UTF8}{mj}整除\end{CJK} $A$ \begin{CJK}{UTF8}{mj}的特征多项式\end{CJK} $f(x)$.

(2) \begin{CJK}{UTF8}{mj}如果\end{CJK} $A$ \begin{CJK}{UTF8}{mj}可逆\end{CJK}, \begin{CJK}{UTF8}{mj}则存在次数不超过\end{CJK} $n-1$ \begin{CJK}{UTF8}{mj}的多项式\end{CJK} $p(x)$, \begin{CJK}{UTF8}{mj}使得\end{CJK} $A^{-1}=p(A)$.

\begin{CJK}{UTF8}{mj}五\end{CJK}. \begin{CJK}{UTF8}{mj}设\end{CJK} $A=\left(a_{i j}\right)$ \begin{CJK}{UTF8}{mj}为\end{CJK} $n(n \geqslant 2)$ \begin{CJK}{UTF8}{mj}级矩阵\end{CJK}, $A^{*}$ \begin{CJK}{UTF8}{mj}为\end{CJK} $A$ \begin{CJK}{UTF8}{mj}的伴随矩阵\end{CJK}, \begin{CJK}{UTF8}{mj}证明\end{CJK}

(1) \begin{CJK}{UTF8}{mj}如果\end{CJK} $|A|=0$, \begin{CJK}{UTF8}{mj}则\end{CJK} $r\left(A^{*}\right)=0$ \begin{CJK}{UTF8}{mj}或\end{CJK} 1 .

(2) \begin{CJK}{UTF8}{mj}令\end{CJK} $M$ \begin{CJK}{UTF8}{mj}表示划掉\end{CJK} $A^{*}$ \begin{CJK}{UTF8}{mj}的第\end{CJK} $i$ \begin{CJK}{UTF8}{mj}行和第\end{CJK} $i$ \begin{CJK}{UTF8}{mj}列所得到的\end{CJK} $n-1$ \begin{CJK}{UTF8}{mj}阶子式\end{CJK}, \begin{CJK}{UTF8}{mj}则\end{CJK} $M=a_{i i}|A|^{n-2}$. \begin{CJK}{UTF8}{mj}六\end{CJK}. \begin{CJK}{UTF8}{mj}令\end{CJK} $A=\left(a_{i j}\right)$ \begin{CJK}{UTF8}{mj}为\end{CJK} $n(n \geqslant 2)$ \begin{CJK}{UTF8}{mj}级矩阵\end{CJK}, $\lambda$ \begin{CJK}{UTF8}{mj}为\end{CJK} $A$ \begin{CJK}{UTF8}{mj}的特征值\end{CJK}, \begin{CJK}{UTF8}{mj}证明\end{CJK}

(1) \begin{CJK}{UTF8}{mj}对每个正整数\end{CJK} $k(1 \leqslant k \leqslant n)$, \begin{CJK}{UTF8}{mj}有\end{CJK} $\left|\lambda-a_{k k}\right| \leqslant \sum_{j \neq k}\left|a_{k j}\right|$.

(2) \begin{CJK}{UTF8}{mj}如果\end{CJK} $A$ \begin{CJK}{UTF8}{mj}满足\end{CJK} $\left|a_{i i}\right|>\sum_{j \neq i}\left|a_{i j}\right| \quad(i=1,2, \cdots, n)$, \begin{CJK}{UTF8}{mj}则\end{CJK} $A$ \begin{CJK}{UTF8}{mj}为非奇异矩阵\end{CJK}, \begin{CJK}{UTF8}{mj}即\end{CJK} $|A| \neq 0$.

\begin{CJK}{UTF8}{mj}七\end{CJK}. \begin{CJK}{UTF8}{mj}利用非退化线性替换将二次型\end{CJK}
$$
f\left(x, x_{2}, x_{3}\right)=a x_{1}^{2}+b x_{2}^{2}+a x_{3}^{2}+2 c x_{1} x_{3}
$$
\begin{CJK}{UTF8}{mj}化为标准型\end{CJK}, \begin{CJK}{UTF8}{mj}求出所做的非退化线性替换\end{CJK}, \begin{CJK}{UTF8}{mj}并指出\end{CJK} $a, b, c$ \begin{CJK}{UTF8}{mj}满足何种条件时\end{CJK}, $f$ \begin{CJK}{UTF8}{mj}为正定二次型\end{CJK}.

\begin{CJK}{UTF8}{mj}八\end{CJK}. \begin{CJK}{UTF8}{mj}令\end{CJK} $\mathscr{A}$ \begin{CJK}{UTF8}{mj}为数域\end{CJK} $P$ \begin{CJK}{UTF8}{mj}上\end{CJK} $n$ \begin{CJK}{UTF8}{mj}维线性空间\end{CJK} $V$ \begin{CJK}{UTF8}{mj}上的线性变换\end{CJK}, \begin{CJK}{UTF8}{mj}且\end{CJK} $\mathscr{A}^{2}=\mathscr{A}$, \begin{CJK}{UTF8}{mj}证明\end{CJK}

(1) $\mathscr{A}^{-1}(0)=\{\alpha-\mathscr{A} \alpha \mid \alpha \in V\}$.

(2) $V=\mathscr{A} V \oplus \mathscr{A}^{-1}(0)$.

\begin{CJK}{UTF8}{mj}九\end{CJK}. \begin{CJK}{UTF8}{mj}令\end{CJK} $\mathscr{A}$ \begin{CJK}{UTF8}{mj}为\end{CJK} $n$ \begin{CJK}{UTF8}{mj}维欧氏空间\end{CJK} $V$ \begin{CJK}{UTF8}{mj}上的线性变换\end{CJK}, \begin{CJK}{UTF8}{mj}如果对任意的\end{CJK} $\alpha, \beta \in V$, \begin{CJK}{UTF8}{mj}都有\end{CJK} $(\mathscr{A} \alpha, \beta)=(\alpha, \mathscr{A} \beta)$, \begin{CJK}{UTF8}{mj}则称\end{CJK} $\mathscr{A}$ \begin{CJK}{UTF8}{mj}为对称\end{CJK} \begin{CJK}{UTF8}{mj}变换\end{CJK}. \begin{CJK}{UTF8}{mj}证明\end{CJK}

(1) $\mathscr{A}$ \begin{CJK}{UTF8}{mj}为对称变换当且仅当\end{CJK} $\mathscr{A}$ \begin{CJK}{UTF8}{mj}在\end{CJK} $V$ \begin{CJK}{UTF8}{mj}的一组标准正交基下的矩阵为实对称矩阵\end{CJK}.

(2) \begin{CJK}{UTF8}{mj}对于对称变换\end{CJK} $\mathscr{A}$, \begin{CJK}{UTF8}{mj}如果\end{CJK} $V_{1}$ \begin{CJK}{UTF8}{mj}是\end{CJK} $\mathscr{A}$-\begin{CJK}{UTF8}{mj}子空间\end{CJK}, \begin{CJK}{UTF8}{mj}则\end{CJK} $V_{1}^{\perp}$ \begin{CJK}{UTF8}{mj}也是\end{CJK} $\mathscr{A}$-\begin{CJK}{UTF8}{mj}子空间\end{CJK}.

\begin{CJK}{UTF8}{mj}十\end{CJK}. \begin{CJK}{UTF8}{mj}设\end{CJK} $V$ \begin{CJK}{UTF8}{mj}是复数域上的\end{CJK} $n$ \begin{CJK}{UTF8}{mj}维线性空间\end{CJK}, $\mathscr{A}, \mathscr{B}$ \begin{CJK}{UTF8}{mj}是\end{CJK} $V$ \begin{CJK}{UTF8}{mj}上的线性变换\end{CJK}, \begin{CJK}{UTF8}{mj}且\end{CJK} $\mathscr{A} \mathscr{B}=\mathscr{B} \mathscr{A}$, \begin{CJK}{UTF8}{mj}证明\end{CJK}

(1) $\mathscr{A}, \mathscr{B}$ \begin{CJK}{UTF8}{mj}至少有一个公共的特征向量\end{CJK}.

(2) \begin{CJK}{UTF8}{mj}在\end{CJK} $V$ \begin{CJK}{UTF8}{mj}中存在一组基\end{CJK}, \begin{CJK}{UTF8}{mj}使得\end{CJK} $\mathscr{A}, \mathscr{B}$ \begin{CJK}{UTF8}{mj}在这组基下的矩阵同时为上三角形\end{CJK}.

\section{1. 湘潭大学 2009 年研究生入学考试试题数学分析}
\begin{CJK}{UTF8}{mj}李扬\end{CJK}

\begin{CJK}{UTF8}{mj}微信公众号\end{CJK}: sxkyliyang

\begin{CJK}{UTF8}{mj}一\end{CJK}. (12 \begin{CJK}{UTF8}{mj}分\end{CJK}) \begin{CJK}{UTF8}{mj}已知\end{CJK}
$$
f(x)=x^{3}+a x^{2}+b x
$$
\begin{CJK}{UTF8}{mj}在\end{CJK} $x=1$ \begin{CJK}{UTF8}{mj}处有极值\end{CJK} $-2$, \begin{CJK}{UTF8}{mj}试确定系数\end{CJK} $a, b$, \begin{CJK}{UTF8}{mj}并求函数\end{CJK} $f(x)$ \begin{CJK}{UTF8}{mj}的极值及曲线\end{CJK} $y=f(x)$ \begin{CJK}{UTF8}{mj}的拐点\end{CJK}.

\begin{CJK}{UTF8}{mj}二\end{CJK}. \begin{CJK}{UTF8}{mj}求下列极限\end{CJK} (\begin{CJK}{UTF8}{mj}每小题\end{CJK} 10 \begin{CJK}{UTF8}{mj}分\end{CJK}, \begin{CJK}{UTF8}{mj}共\end{CJK} 20 \begin{CJK}{UTF8}{mj}分\end{CJK})

(1)
$$
\lim _{x \rightarrow 0} \frac{1-x^{2}-\mathrm{e}^{-x^{2}}}{x \sin ^{3} 2 x}
$$
(2)
$$
\lim _{n \rightarrow \infty}\left(a^{n}+b^{n}+c^{n}\right)^{\frac{1}{n}},(a \geqslant 0, b \geqslant 0, c \geqslant 0) .
$$
\begin{CJK}{UTF8}{mj}三\end{CJK}. \begin{CJK}{UTF8}{mj}求下列积分\end{CJK} (\begin{CJK}{UTF8}{mj}每小题\end{CJK} 10 \begin{CJK}{UTF8}{mj}分\end{CJK}, \begin{CJK}{UTF8}{mj}共\end{CJK} 20 \begin{CJK}{UTF8}{mj}分\end{CJK})

(1)
$$
\int \frac{1}{(x-1)(x+1)^{2}} d x
$$
$(2)$

\includegraphics[max width=\textwidth]{2022_04_18_3416d289b173eb9de8c1g-066}

\begin{CJK}{UTF8}{mj}四\end{CJK}. (12 \begin{CJK}{UTF8}{mj}分\end{CJK}) $(1)$ \begin{CJK}{UTF8}{mj}设数列\end{CJK} $\left\{x_{n}\right\}$ \begin{CJK}{UTF8}{mj}满足压缩性条件\end{CJK}
$$
\left|x_{n+1}-x_{n}\right| \leqslant k\left|x_{n}-x_{n-1}\right|, 0<k<1, n=2,3, \cdots,
$$
\begin{CJK}{UTF8}{mj}证明\end{CJK}: \begin{CJK}{UTF8}{mj}数列\end{CJK} $\left\{x_{n}\right\}$ \begin{CJK}{UTF8}{mj}收敛\end{CJK}.

(2) \begin{CJK}{UTF8}{mj}设\end{CJK} $f(x)$ \begin{CJK}{UTF8}{mj}可微且\end{CJK} $\left|f^{\prime}(x)\right| \leqslant r<1, r$ \begin{CJK}{UTF8}{mj}是常数\end{CJK}. \begin{CJK}{UTF8}{mj}给定\end{CJK} $x_{0}$, \begin{CJK}{UTF8}{mj}令\end{CJK}
$$
x_{n}=f\left(x_{n-1}\right),(n=1,2, \cdots),
$$
\begin{CJK}{UTF8}{mj}证明\end{CJK}: \begin{CJK}{UTF8}{mj}数列\end{CJK} $\left\{x_{n}\right\}$ \begin{CJK}{UTF8}{mj}收敛\end{CJK}.

\begin{CJK}{UTF8}{mj}五\end{CJK}. (12 \begin{CJK}{UTF8}{mj}分\end{CJK}) \begin{CJK}{UTF8}{mj}证明\end{CJK} $f(x)=\sin x^{2}$ \begin{CJK}{UTF8}{mj}在\end{CJK} $[0, A]$ \begin{CJK}{UTF8}{mj}上一致连续\end{CJK} ( $A$ \begin{CJK}{UTF8}{mj}为任意有限正数\end{CJK}), \begin{CJK}{UTF8}{mj}但在\end{CJK} $(-\infty,+\infty)$ \begin{CJK}{UTF8}{mj}上不一致连续\end{CJK}.

\begin{CJK}{UTF8}{mj}六\end{CJK}. (10 \begin{CJK}{UTF8}{mj}分\end{CJK}) \begin{CJK}{UTF8}{mj}计算积分\end{CJK}
$$
I=\iint_{D}(\sqrt{x}+\sqrt{y}) \mathrm{d} x \mathrm{~d} y,
$$
\begin{CJK}{UTF8}{mj}其中\end{CJK} $D$ \begin{CJK}{UTF8}{mj}为坐标轴及抛物线\end{CJK} $\sqrt{x}+\sqrt{y}=1$ \begin{CJK}{UTF8}{mj}所围成的有界闭区域\end{CJK}.

\begin{CJK}{UTF8}{mj}七\end{CJK}. (12 \begin{CJK}{UTF8}{mj}分\end{CJK}) \begin{CJK}{UTF8}{mj}计算第二类曲线积分\end{CJK}
$$
I=\int_{L} x \mathrm{~d} x+y \mathrm{~d} y+(x+y-1) \mathrm{d} z
$$
$L$ \begin{CJK}{UTF8}{mj}是从点\end{CJK} $(1,1,1)$ \begin{CJK}{UTF8}{mj}到\end{CJK} $(2,3,4)$ \begin{CJK}{UTF8}{mj}的直线段\end{CJK}.

\begin{CJK}{UTF8}{mj}八\end{CJK}. (10 \begin{CJK}{UTF8}{mj}分\end{CJK}) \begin{CJK}{UTF8}{mj}设\end{CJK} $f(x)$ \begin{CJK}{UTF8}{mj}在\end{CJK} $x=a$ \begin{CJK}{UTF8}{mj}处有\end{CJK} $n$ \begin{CJK}{UTF8}{mj}阶连续导数\end{CJK}, \begin{CJK}{UTF8}{mj}且\end{CJK}
$$
f^{\prime}(a)=\cdots=f^{(n-1)}(a)=0, f^{(n)}(a) \neq 0,
$$
\begin{CJK}{UTF8}{mj}讨论\end{CJK} $f(x)$ \begin{CJK}{UTF8}{mj}在\end{CJK} $x=a$ \begin{CJK}{UTF8}{mj}处的极值情况\end{CJK}. \begin{CJK}{UTF8}{mj}九\end{CJK}. ( 10 \begin{CJK}{UTF8}{mj}分\end{CJK}) \begin{CJK}{UTF8}{mj}设\end{CJK} $u(x, y)$ \begin{CJK}{UTF8}{mj}的所有二阶偏导数都连续\end{CJK},
$$
\frac{\partial^{2} u}{\partial x^{2}}-\frac{\partial^{2} u}{\partial y^{2}}=0, u(x, 2 x)=x, u_{x}^{\prime}(x, 2 x)=x^{2}
$$
\begin{CJK}{UTF8}{mj}试求\end{CJK}: $u_{x x}^{\prime \prime}(x, 2 x), u_{x y}^{\prime \prime}(x, 2 x)$.

\begin{CJK}{UTF8}{mj}十\end{CJK}. (12 \begin{CJK}{UTF8}{mj}分\end{CJK}) \begin{CJK}{UTF8}{mj}设\end{CJK} $I_{n}=\int_{0}^{\frac{\pi}{4}} \tan ^{n} x \mathrm{~d} x, n$ \begin{CJK}{UTF8}{mj}为大于\end{CJK} 1 \begin{CJK}{UTF8}{mj}的整数\end{CJK}, \begin{CJK}{UTF8}{mj}计算\end{CJK} $I_{n}+I_{n-2}$, \begin{CJK}{UTF8}{mj}并证明\end{CJK}
$$
\frac{1}{2(n+1)}<I_{n}<\frac{1}{2(n-1)}
$$
$十$ \begin{CJK}{UTF8}{mj}十\end{CJK}. (10 \begin{CJK}{UTF8}{mj}分\end{CJK}) \begin{CJK}{UTF8}{mj}设\end{CJK} $f_{1}(x)$ \begin{CJK}{UTF8}{mj}在\end{CJK} $[a, b]$ \begin{CJK}{UTF8}{mj}上连续\end{CJK},
$$
f_{n+1}(x)=\int_{a}^{x} f_{n}(t) \mathrm{d} t, n=1,2, \cdots,
$$
\begin{CJK}{UTF8}{mj}证明\end{CJK}: \begin{CJK}{UTF8}{mj}函数序列\end{CJK} $\left\{f_{n}(x)\right\}$ \begin{CJK}{UTF8}{mj}在\end{CJK} $[a, b]$ \begin{CJK}{UTF8}{mj}上一致收敛于零\end{CJK}.

\begin{CJK}{UTF8}{mj}十二\end{CJK}. ( 10 \begin{CJK}{UTF8}{mj}分\end{CJK}) \begin{CJK}{UTF8}{mj}设\end{CJK} $(1) \int_{a}^{+\infty} f(x, y) \mathrm{d} x$ \begin{CJK}{UTF8}{mj}关于\end{CJK} $y \in[c, d]$ \begin{CJK}{UTF8}{mj}一致收敛\end{CJK}.

(2) $g(x, y)$ \begin{CJK}{UTF8}{mj}关于\end{CJK} $x$ \begin{CJK}{UTF8}{mj}单调\end{CJK}.

(3) $g(x, y)$ \begin{CJK}{UTF8}{mj}一致有界\end{CJK}.

\begin{CJK}{UTF8}{mj}证明\end{CJK}: \begin{CJK}{UTF8}{mj}含参变量的反常积分\end{CJK}
$$
\int_{a}^{+\infty} f(x, y) g(x, y) \mathrm{d} x
$$
\begin{CJK}{UTF8}{mj}关于\end{CJK} $y \in[c, d]$ \begin{CJK}{UTF8}{mj}一致收敛\end{CJK}.

\includegraphics[max width=\textwidth]{2022_04_18_3416d289b173eb9de8c1g-067}

\section{2. 湘潭大学 2010 年研究生入学考试试题数学分析}
\begin{CJK}{UTF8}{mj}李扬\end{CJK}

\begin{CJK}{UTF8}{mj}微信公众号\end{CJK}: sxkyliyang

\begin{CJK}{UTF8}{mj}一\end{CJK}. \begin{CJK}{UTF8}{mj}求下列极限\end{CJK} (\begin{CJK}{UTF8}{mj}每小题\end{CJK} 10 \begin{CJK}{UTF8}{mj}分\end{CJK}, \begin{CJK}{UTF8}{mj}共\end{CJK} 30 \begin{CJK}{UTF8}{mj}分\end{CJK})

(1)
$$
\lim _{x \rightarrow 0} \frac{\tan x-\sin x}{x^{3}}
$$
(2)
$$
\lim _{n \rightarrow \infty} n\left(\sqrt{n^{2}+1}-\sqrt{n^{2}-1}\right) .
$$
$(3)$
$$
\lim _{n \rightarrow \infty}\left(\frac{1}{n^{2}+n+1}+\frac{2}{n^{2}+n+2}+\cdots+\frac{n}{n^{2}+n+n}\right)
$$
\begin{CJK}{UTF8}{mj}二\end{CJK}. (10 \begin{CJK}{UTF8}{mj}分\end{CJK}) \begin{CJK}{UTF8}{mj}设\end{CJK}
$$
f(x)= \begin{cases}\frac{\ln (1+x)}{x}, & x>0 \\ 0, & x=0 \\ \frac{\sqrt{1+x}-\sqrt{1-x}}{x}, & -1 \leqslant x<0\end{cases}
$$
\begin{CJK}{UTF8}{mj}试研究函数\end{CJK} $f(x)$ \begin{CJK}{UTF8}{mj}在\end{CJK} $x=0$ \begin{CJK}{UTF8}{mj}点的连续性\end{CJK}.

\begin{CJK}{UTF8}{mj}三\end{CJK}. \begin{CJK}{UTF8}{mj}求下列积分\end{CJK} (\begin{CJK}{UTF8}{mj}每小题\end{CJK} 10 \begin{CJK}{UTF8}{mj}分\end{CJK}, \begin{CJK}{UTF8}{mj}共\end{CJK} 20 \begin{CJK}{UTF8}{mj}分\end{CJK})

(1)
$$
\int \frac{1+\ln x}{(x \ln x)^{2}} \mathrm{~d} x
$$
(2)

\begin{CJK}{UTF8}{mj}其中\end{CJK} $n$ \begin{CJK}{UTF8}{mj}为正整数\end{CJK}.

\includegraphics[max width=\textwidth]{2022_04_18_3416d289b173eb9de8c1g-068}

\begin{CJK}{UTF8}{mj}四\end{CJK}. \begin{CJK}{UTF8}{mj}叙述下列概念\end{CJK} (\begin{CJK}{UTF8}{mj}每小题\end{CJK} 5 \begin{CJK}{UTF8}{mj}分\end{CJK}, \begin{CJK}{UTF8}{mj}共\end{CJK} 20 \begin{CJK}{UTF8}{mj}分\end{CJK})

(1) \begin{CJK}{UTF8}{mj}反常积分\end{CJK} $\int_{a}^{+\infty} f(x) \mathrm{d} x$ \begin{CJK}{UTF8}{mj}的敛散性\end{CJK}.

(2) \begin{CJK}{UTF8}{mj}函数序列\end{CJK} $\left\{S_{n}(x)\right\}$ \begin{CJK}{UTF8}{mj}在\end{CJK} $D$ \begin{CJK}{UTF8}{mj}上一致收敛于\end{CJK} $S(x)$.

(3) \begin{CJK}{UTF8}{mj}函数\end{CJK} $f(x)$ \begin{CJK}{UTF8}{mj}在区间\end{CJK} $I$ \begin{CJK}{UTF8}{mj}上一致连续\end{CJK}.

$(4)$ Z\begin{CJK}{UTF8}{mj}元函数\end{CJK} $f(x, y)$ \begin{CJK}{UTF8}{mj}在\end{CJK} $\left(x_{0}, y_{0}\right)$ \begin{CJK}{UTF8}{mj}点对\end{CJK} $x$ \begin{CJK}{UTF8}{mj}的偏导数\end{CJK}.

\begin{CJK}{UTF8}{mj}五\end{CJK}. ( 10 \begin{CJK}{UTF8}{mj}分\end{CJK}) \begin{CJK}{UTF8}{mj}设\end{CJK} $f(x)$ \begin{CJK}{UTF8}{mj}在区间\end{CJK} $[a, b]$ \begin{CJK}{UTF8}{mj}上满足\end{CJK}:
$$
|f(x)-f(y)| \leqslant M|x-y|^{\alpha}, \forall x, y \in[a, b]
$$
\begin{CJK}{UTF8}{mj}其中\end{CJK} $M>0, \alpha>1$ \begin{CJK}{UTF8}{mj}为常数\end{CJK}, \begin{CJK}{UTF8}{mj}证明\end{CJK}: $f(x)$ \begin{CJK}{UTF8}{mj}在\end{CJK} $[a, b]$ \begin{CJK}{UTF8}{mj}上恒为常数\end{CJK}.

\begin{CJK}{UTF8}{mj}六\end{CJK}. (10 \begin{CJK}{UTF8}{mj}分\end{CJK}) \begin{CJK}{UTF8}{mj}计算三重积分\end{CJK}
$$
I=\iiint_{\Omega} x^{2} \cdot \sqrt{x^{2}+y^{2}} \mathrm{~d} x \mathrm{~d} y \mathrm{~d} z
$$
\begin{CJK}{UTF8}{mj}其中\end{CJK} $\Omega$ \begin{CJK}{UTF8}{mj}是曲面\end{CJK} $z=\sqrt{x^{2}+y^{2}}$ \begin{CJK}{UTF8}{mj}与\end{CJK} $z=x^{2}+y^{2}$ \begin{CJK}{UTF8}{mj}围成的有界区域\end{CJK}. \begin{CJK}{UTF8}{mj}七\end{CJK}. (10 \begin{CJK}{UTF8}{mj}分\end{CJK}) \begin{CJK}{UTF8}{mj}证明\end{CJK}:
$$
\sin x \cdot \sin y \cdot \sin (x+y) \leqslant \frac{3 \sqrt{3}}{8}, 0<x, y<\pi
$$
\begin{CJK}{UTF8}{mj}八\end{CJK}. (10 \begin{CJK}{UTF8}{mj}分\end{CJK}) \begin{CJK}{UTF8}{mj}计算第一类曲线积分\end{CJK}
$$
\int_{L} x^{2} \mathrm{~d} s
$$
\begin{CJK}{UTF8}{mj}其中\end{CJK} $L: x^{2}+y^{2}+z^{2}=a^{2}, x+y+z=0$.

\begin{CJK}{UTF8}{mj}九\end{CJK}. (10 \begin{CJK}{UTF8}{mj}分\end{CJK}) \begin{CJK}{UTF8}{mj}设\end{CJK} $\left\{a_{n}\right\},\left\{b_{n}\right\}$ \begin{CJK}{UTF8}{mj}满足条件\end{CJK} $\mathrm{e}^{a_{n}}=a_{n}+\mathrm{e}^{b_{n}}$ ( $n$ \begin{CJK}{UTF8}{mj}为正整数\end{CJK}) \begin{CJK}{UTF8}{mj}且\end{CJK} $\sum_{n=1}^{\infty} a_{n}^{2}$ \begin{CJK}{UTF8}{mj}收敛\end{CJK}. \begin{CJK}{UTF8}{mj}证明\end{CJK}: $\sum_{n=1}^{\infty} b_{n}$ \begin{CJK}{UTF8}{mj}收敛\end{CJK}.

\begin{CJK}{UTF8}{mj}十\end{CJK}. (10 \begin{CJK}{UTF8}{mj}分\end{CJK}) \begin{CJK}{UTF8}{mj}已知\end{CJK} $f(0)=\frac{\sqrt{2}}{2}$, \begin{CJK}{UTF8}{mj}求\end{CJK}
$$
f(x)=\int_{0}^{+\infty} \mathrm{e}^{-t^{2}} \cos 2 x t \mathrm{~d} t
$$
\begin{CJK}{UTF8}{mj}十一\end{CJK}. (10 \begin{CJK}{UTF8}{mj}分\end{CJK}) \begin{CJK}{UTF8}{mj}讨论\end{CJK}
$$
\sum_{n=1}^{\infty}\left(1+\frac{1}{2^{2}}+\cdots+\frac{1}{n^{2}}\right) \cdot \frac{\sin n x}{n}
$$
\begin{CJK}{UTF8}{mj}的敛散性\end{CJK}.

\section{3. 湘潭大学 2011 年研究生入学考试试题数学分析}
\begin{CJK}{UTF8}{mj}李扬\end{CJK}

\begin{CJK}{UTF8}{mj}微信公众号\end{CJK}: sxkyliyang

\begin{CJK}{UTF8}{mj}一\end{CJK}. (12 \begin{CJK}{UTF8}{mj}分\end{CJK}) \begin{CJK}{UTF8}{mj}写出\end{CJK} $\lim _{x \rightarrow x_{0}} f(x)=A$ \begin{CJK}{UTF8}{mj}的\end{CJK} $\varepsilon-\delta$ \begin{CJK}{UTF8}{mj}定义\end{CJK}, \begin{CJK}{UTF8}{mj}并用定义证明\end{CJK}:
$$
\lim _{x \rightarrow 3} x^{2}=9 .
$$
\begin{CJK}{UTF8}{mj}二\end{CJK}. \begin{CJK}{UTF8}{mj}求下列极限\end{CJK} (\begin{CJK}{UTF8}{mj}每小题\end{CJK} 11 \begin{CJK}{UTF8}{mj}分\end{CJK}, \begin{CJK}{UTF8}{mj}共\end{CJK} 22 \begin{CJK}{UTF8}{mj}分\end{CJK})

(1)
$$
\lim _{x \rightarrow 0}\left(\frac{1}{\ln (1+x)}-\frac{1}{x}\right) .
$$
$(2)$
$$
\lim _{n \rightarrow \infty}\left(1+2^{n}+3^{n}+4^{n}+5^{n}\right)^{\frac{1}{n}}
$$
\begin{CJK}{UTF8}{mj}三\end{CJK}. \begin{CJK}{UTF8}{mj}求下列积分\end{CJK} (\begin{CJK}{UTF8}{mj}每小题\end{CJK} 11 \begin{CJK}{UTF8}{mj}分\end{CJK}, \begin{CJK}{UTF8}{mj}共\end{CJK} 22 \begin{CJK}{UTF8}{mj}分\end{CJK})

(1)
$$
\int \frac{\ln ^{3} x}{x^{2}} \mathrm{~d} x .
$$
(2)
$$
\int_{0}^{1} \mathrm{e}^{\sqrt{x+1}} \mathrm{~d} x
$$
\begin{CJK}{UTF8}{mj}四\end{CJK}. (12 \begin{CJK}{UTF8}{mj}分\end{CJK}) \begin{CJK}{UTF8}{mj}设函数\end{CJK} $f(x)$ \begin{CJK}{UTF8}{mj}在\end{CJK} $(0,+\infty)$ \begin{CJK}{UTF8}{mj}上连续且满足\end{CJK} $f(x)=\sin x-\int_{0}^{\pi} f(x) \mathrm{d} x$, \begin{CJK}{UTF8}{mj}求\end{CJK}
$$
\int_{0}^{\pi} f(x) \mathrm{d} x
$$
\begin{CJK}{UTF8}{mj}五\end{CJK}. ( 12 \begin{CJK}{UTF8}{mj}分\end{CJK}) \begin{CJK}{UTF8}{mj}对于每个正整数\end{CJK} $n(n \geqslant 2)$, \begin{CJK}{UTF8}{mj}证明方程\end{CJK}

\begin{CJK}{UTF8}{mj}在\end{CJK} $(0,1)$ \begin{CJK}{UTF8}{mj}内有唯一的实根\end{CJK} $\bar{x}_{n}$, \begin{CJK}{UTF8}{mj}并求\end{CJK} $\lim _{n \rightarrow \infty} x_{n}$.

\begin{CJK}{UTF8}{mj}六\end{CJK}. (11 \begin{CJK}{UTF8}{mj}分\end{CJK}) \begin{CJK}{UTF8}{mj}判别级数\end{CJK}
$$
\sum_{n=1}^{\infty} \int_{0}^{\frac{1}{n}} \sqrt{\frac{x}{1+x}} \mathrm{~d} x
$$
\begin{CJK}{UTF8}{mj}的敛散性\end{CJK}.

\begin{CJK}{UTF8}{mj}七\end{CJK}. (12 \begin{CJK}{UTF8}{mj}分\end{CJK}) \begin{CJK}{UTF8}{mj}求由方程\end{CJK}
$$
x^{2}+2 x y+2 y^{2}=1
$$
\begin{CJK}{UTF8}{mj}所确定的隐函数\end{CJK} $y=y(x)$ \begin{CJK}{UTF8}{mj}的极值\end{CJK}.

\begin{CJK}{UTF8}{mj}八\end{CJK}. (12 \begin{CJK}{UTF8}{mj}分\end{CJK}) \begin{CJK}{UTF8}{mj}计算积分\end{CJK}
$$
I=\iint_{D} f(x y) \mathrm{d} x \mathrm{~d} y
$$
\begin{CJK}{UTF8}{mj}其中\end{CJK} $D$ \begin{CJK}{UTF8}{mj}由曲线\end{CJK} $x y=1, x y=2, y=x, y=4 x(x>0, y>0)$ \begin{CJK}{UTF8}{mj}所围成的区域\end{CJK}.

\begin{CJK}{UTF8}{mj}九\end{CJK}. (12 \begin{CJK}{UTF8}{mj}分\end{CJK}) \begin{CJK}{UTF8}{mj}讨论反常积分\end{CJK}
$$
\int_{1}^{+\infty} \frac{\sin x}{x^{p}} \mathrm{~d} x(p>0)
$$
\begin{CJK}{UTF8}{mj}的敛散性\end{CJK} (\begin{CJK}{UTF8}{mj}包括绝对收敛\end{CJK}, \begin{CJK}{UTF8}{mj}条件收敛和发散\end{CJK}). \begin{CJK}{UTF8}{mj}十\end{CJK}. (12 \begin{CJK}{UTF8}{mj}分\end{CJK}) \begin{CJK}{UTF8}{mj}设\end{CJK} $Q(x, y)$ \begin{CJK}{UTF8}{mj}在\end{CJK} $x y$ \begin{CJK}{UTF8}{mj}平面上具有连续偏导数\end{CJK}, \begin{CJK}{UTF8}{mj}曲线积分\end{CJK} $\int_{L} 2 x y \mathrm{~d} x+Q(x, y) \mathrm{d} y$ \begin{CJK}{UTF8}{mj}与路径无关\end{CJK}, \begin{CJK}{UTF8}{mj}并且对任意\end{CJK} $t$ \begin{CJK}{UTF8}{mj}恒有\end{CJK}
$$
\int_{(0,0)}^{(t, 1)} 2 x y \mathrm{~d} x+Q(x, y) \mathrm{d} y=\int_{(0,0)}^{(1, t)} 2 x y \mathrm{~d} x+Q(x, y) \mathrm{d} y
$$
\begin{CJK}{UTF8}{mj}求\end{CJK} $Q(x, y)$.

\begin{CJK}{UTF8}{mj}十一\end{CJK}. (11 \begin{CJK}{UTF8}{mj}分\end{CJK}) \begin{CJK}{UTF8}{mj}表述数列的\end{CJK} Cauchy \begin{CJK}{UTF8}{mj}收敛原理\end{CJK}, \begin{CJK}{UTF8}{mj}并利用此原理证明\end{CJK}: \begin{CJK}{UTF8}{mj}单调有界数列必定收敛\end{CJK}.

\section{4. 湘潭大学 2012 年研究生入学考试试题数学分析}
\begin{CJK}{UTF8}{mj}李扬\end{CJK}

\begin{CJK}{UTF8}{mj}微信公众号\end{CJK}: sxkyliyang

\begin{CJK}{UTF8}{mj}一\end{CJK}. \begin{CJK}{UTF8}{mj}求下列极限\end{CJK} (\begin{CJK}{UTF8}{mj}每小题\end{CJK} 10 \begin{CJK}{UTF8}{mj}分\end{CJK}, \begin{CJK}{UTF8}{mj}共\end{CJK} 40 \begin{CJK}{UTF8}{mj}分\end{CJK})

(1)

(2)
$$
\lim _{n \rightarrow \infty} \frac{1^{2}+3^{2}+5^{2}+\cdots+(2 n+1)^{2}}{n^{3}}
$$
(3) \begin{CJK}{UTF8}{mj}证明\end{CJK}: \begin{CJK}{UTF8}{mj}当\end{CJK} $t \neq 0$ \begin{CJK}{UTF8}{mj}时\end{CJK}
$$
\lim _{n \rightarrow \infty}\left(\cos \frac{t}{2}\right)\left(\cos \frac{t}{2^{2}}\right) \cdots\left(\cos \frac{t}{2^{n}}\right)=\frac{\sin t}{t}
$$
(4)
$$
\lim _{(x, y) \rightarrow(0,0)} \frac{\sin \left(x^{3}+y^{3}\right)}{x^{2}+y^{2}}
$$
\begin{CJK}{UTF8}{mj}二\end{CJK}. (1)( 7 \begin{CJK}{UTF8}{mj}分\end{CJK}) \begin{CJK}{UTF8}{mj}叙述数集\end{CJK} $S$ \begin{CJK}{UTF8}{mj}的上确界的定义\end{CJK}.

$(2)$ ( 8 \begin{CJK}{UTF8}{mj}分\end{CJK}) \begin{CJK}{UTF8}{mj}设\end{CJK} $S$ \begin{CJK}{UTF8}{mj}是一个有上界的数集\end{CJK}, \begin{CJK}{UTF8}{mj}用\end{CJK} $S_{a}$ \begin{CJK}{UTF8}{mj}表示\end{CJK} $S$ \begin{CJK}{UTF8}{mj}的一个平移\end{CJK}, \begin{CJK}{UTF8}{mj}即\end{CJK} $S_{a}=\{x+a \mid x \in S\}$, \begin{CJK}{UTF8}{mj}其中\end{CJK} $a$ \begin{CJK}{UTF8}{mj}是一个实数\end{CJK}, \begin{CJK}{UTF8}{mj}试证明\end{CJK}:
$$
\sup S_{a}=\sup S+a .
$$
\begin{CJK}{UTF8}{mj}其中\end{CJK} $\sup S$ \begin{CJK}{UTF8}{mj}表示数集\end{CJK} $S$ \begin{CJK}{UTF8}{mj}的上确界\end{CJK}.

\begin{CJK}{UTF8}{mj}三\end{CJK}. (1) ( 10 \begin{CJK}{UTF8}{mj}分\end{CJK}) \begin{CJK}{UTF8}{mj}设\end{CJK}
$$
f(\ln x)=\frac{\ln (1+x)}{x},
$$
\begin{CJK}{UTF8}{mj}求\end{CJK} $\int f(x) \mathrm{d} x$.

$(2)$ ( 10 \begin{CJK}{UTF8}{mj}分\end{CJK}) \begin{CJK}{UTF8}{mj}设函数\end{CJK} $f(x)$ \begin{CJK}{UTF8}{mj}在\end{CJK} $[0,1]$ \begin{CJK}{UTF8}{mj}上二阶可导\end{CJK}, \begin{CJK}{UTF8}{mj}且\end{CJK} $f^{\prime \prime}(x) \leqslant 0, \forall x \in[0,1]$, \begin{CJK}{UTF8}{mj}证明\end{CJK}:
$$
\int_{0}^{1} f\left(x^{2}\right) \mathrm{d} x \leqslant f\left(\frac{1}{3}\right) .
$$
\begin{CJK}{UTF8}{mj}四\end{CJK}. \begin{CJK}{UTF8}{mj}叙述下列概念\end{CJK} (\begin{CJK}{UTF8}{mj}每小题\end{CJK} 5 \begin{CJK}{UTF8}{mj}分\end{CJK}, \begin{CJK}{UTF8}{mj}共\end{CJK} 10 \begin{CJK}{UTF8}{mj}分\end{CJK})

(1) \begin{CJK}{UTF8}{mj}反常积分\end{CJK} $\int_{a}^{b} f(x) \mathrm{d} x$ ( $b$ \begin{CJK}{UTF8}{mj}为奇点\end{CJK} $)$ \begin{CJK}{UTF8}{mj}的敛散性\end{CJK}.

(2) \begin{CJK}{UTF8}{mj}函数序列\end{CJK} $\left\{S_{n}(x)\right\}$ \begin{CJK}{UTF8}{mj}在\end{CJK} $D$ \begin{CJK}{UTF8}{mj}上一致收敛于\end{CJK} $S(x)$.

\begin{CJK}{UTF8}{mj}五\end{CJK}. (10 \begin{CJK}{UTF8}{mj}分\end{CJK}) (1) \begin{CJK}{UTF8}{mj}叙述\end{CJK} $f(x)$ \begin{CJK}{UTF8}{mj}在区间\end{CJK} $I$ \begin{CJK}{UTF8}{mj}上一致连续的概念\end{CJK}.

(2) \begin{CJK}{UTF8}{mj}证明\end{CJK}: \begin{CJK}{UTF8}{mj}函数\end{CJK} $f(x)=\ln x$ \begin{CJK}{UTF8}{mj}在\end{CJK} $[1,+\infty)$ \begin{CJK}{UTF8}{mj}上一致连续\end{CJK}.

\begin{CJK}{UTF8}{mj}六\end{CJK}. (10 \begin{CJK}{UTF8}{mj}分\end{CJK}) \begin{CJK}{UTF8}{mj}计算积分\end{CJK}
$$
I=\iint_{D} \frac{x^{2}-y^{2}}{\sqrt{x+y+3}} \mathrm{~d} x \mathrm{~d} y
$$
\begin{CJK}{UTF8}{mj}其中\end{CJK} $D=\{(x, y)|| x|+| y \mid \leqslant 1\}$.

\begin{CJK}{UTF8}{mj}七\end{CJK}. (15 \begin{CJK}{UTF8}{mj}分\end{CJK}) \begin{CJK}{UTF8}{mj}求两曲面\end{CJK} $x+2 y=1$ \begin{CJK}{UTF8}{mj}和\end{CJK} $x^{2}+2 y^{2}+z^{2}=1$ \begin{CJK}{UTF8}{mj}的交线上距离原点最近的点的坐标\end{CJK}. \begin{CJK}{UTF8}{mj}八\end{CJK}. ( 10 \begin{CJK}{UTF8}{mj}分\end{CJK}) \begin{CJK}{UTF8}{mj}计算曲面积分\end{CJK}
$$
I=\iint_{\Sigma} \frac{a x \mathrm{~d} y \mathrm{~d} z+(a+z)^{2} \mathrm{~d} x \mathrm{~d} y}{\left(x^{2}+y^{2}+z^{2}\right)^{\frac{1}{2}}}(a>0),
$$
\begin{CJK}{UTF8}{mj}其中\end{CJK} $\Sigma$ \begin{CJK}{UTF8}{mj}是下半球面\end{CJK} $z=-\sqrt{a^{2}-x^{2}-y^{2}}$, \begin{CJK}{UTF8}{mj}方向取上侧\end{CJK}.

\begin{CJK}{UTF8}{mj}九\end{CJK}. (10 \begin{CJK}{UTF8}{mj}分\end{CJK}) \begin{CJK}{UTF8}{mj}证明\end{CJK}: \begin{CJK}{UTF8}{mj}正项级数\end{CJK}
$$
\sum_{n=2}^{\infty} \frac{1}{n \ln ^{p} n}
$$
\begin{CJK}{UTF8}{mj}在\end{CJK} $p>1$ \begin{CJK}{UTF8}{mj}时收敛\end{CJK}, \begin{CJK}{UTF8}{mj}在\end{CJK} $p \leqslant 1$ \begin{CJK}{UTF8}{mj}时发散\end{CJK}.

\begin{CJK}{UTF8}{mj}十\end{CJK}. ( 10 \begin{CJK}{UTF8}{mj}分\end{CJK}) \begin{CJK}{UTF8}{mj}证明\end{CJK}: \begin{CJK}{UTF8}{mj}函数\end{CJK}
$$
f(x)=\sum_{n=0}^{\infty} \frac{\cos n x}{n^{2}+1}
$$
\begin{CJK}{UTF8}{mj}在\end{CJK} $(0,2 \pi)$ \begin{CJK}{UTF8}{mj}上连续\end{CJK}, \begin{CJK}{UTF8}{mj}且有连续的导函数\end{CJK}.

\section{5. 湘潭大学 2013 年研究生入学考试试题数学分析}
\begin{CJK}{UTF8}{mj}李扬\end{CJK}

\begin{CJK}{UTF8}{mj}微信公众号\end{CJK}: sxkyliyang

\begin{CJK}{UTF8}{mj}一\end{CJK}. \begin{CJK}{UTF8}{mj}求下列极限\end{CJK} (\begin{CJK}{UTF8}{mj}每小题\end{CJK} 10 \begin{CJK}{UTF8}{mj}分\end{CJK}, \begin{CJK}{UTF8}{mj}共\end{CJK} 30 \begin{CJK}{UTF8}{mj}分\end{CJK})

(1)
$$
\lim _{n \rightarrow \infty}\left(\frac{1}{n+\sqrt{1}}+\frac{1}{n+\sqrt{2}}+\cdots+\frac{1}{n+\sqrt{n}}\right) .
$$
(2)
$$
\lim _{x \rightarrow \infty}\left(1+\frac{1}{x^{2}}\right)^{x}
$$
(3)
$$
\lim _{x \rightarrow 0} \frac{\sqrt{1+x}-\mathrm{e}^{\frac{x}{3}}}{\ln (1+2 x)} .
$$
\begin{CJK}{UTF8}{mj}二\end{CJK}. ( 12 \begin{CJK}{UTF8}{mj}分\end{CJK}) \begin{CJK}{UTF8}{mj}设\end{CJK} $0<x_{1}<1, x_{n+1}=1-\sqrt{1-x_{n}}, n=1,2,3, \cdots$. \begin{CJK}{UTF8}{mj}证明\end{CJK}: \begin{CJK}{UTF8}{mj}数列\end{CJK} $\left\{x_{n}\right\}$ \begin{CJK}{UTF8}{mj}收敛\end{CJK}, \begin{CJK}{UTF8}{mj}并求此极限\end{CJK}.

\begin{CJK}{UTF8}{mj}三\end{CJK}. (12 \begin{CJK}{UTF8}{mj}分\end{CJK}) \begin{CJK}{UTF8}{mj}证明\end{CJK} $\sin \frac{1}{x}$ \begin{CJK}{UTF8}{mj}在\end{CJK} $(0,1)$ \begin{CJK}{UTF8}{mj}上不一致连续\end{CJK}, \begin{CJK}{UTF8}{mj}但在\end{CJK} $(a, 1)(a>0)$ \begin{CJK}{UTF8}{mj}上一致连续\end{CJK}.

\begin{CJK}{UTF8}{mj}四\end{CJK}. (10 \begin{CJK}{UTF8}{mj}分\end{CJK}) \begin{CJK}{UTF8}{mj}设函数\end{CJK}
$$
f(x)= \begin{cases}x^{2}+b, & x>2 \\ a x+1, & x \leqslant 2\end{cases}
$$
\begin{CJK}{UTF8}{mj}确定\end{CJK} $a, b$, \begin{CJK}{UTF8}{mj}使得\end{CJK} $f(x)$ \begin{CJK}{UTF8}{mj}在\end{CJK} $x=2$ \begin{CJK}{UTF8}{mj}处可导\end{CJK}.

\begin{CJK}{UTF8}{mj}五\end{CJK}. \begin{CJK}{UTF8}{mj}求下列积分\end{CJK} (\begin{CJK}{UTF8}{mj}每小题\end{CJK} 10 \begin{CJK}{UTF8}{mj}分\end{CJK}, \begin{CJK}{UTF8}{mj}共\end{CJK} 20 \begin{CJK}{UTF8}{mj}分\end{CJK})

\includegraphics[max width=\textwidth]{2022_04_18_3416d289b173eb9de8c1g-074}

\begin{CJK}{UTF8}{mj}七\end{CJK}. (12 \begin{CJK}{UTF8}{mj}分\end{CJK}) \begin{CJK}{UTF8}{mj}计算积分\end{CJK}
$$
I=\iint_{D}(x+y) \mathrm{d} x \mathrm{~d} y
$$
$D$ \begin{CJK}{UTF8}{mj}是由曲线\end{CJK} $x^{2}+y^{2}=x+y$ \begin{CJK}{UTF8}{mj}所围成的闭区域\end{CJK}.

\begin{CJK}{UTF8}{mj}八\end{CJK}. (10 \begin{CJK}{UTF8}{mj}分\end{CJK}) \begin{CJK}{UTF8}{mj}计算曲线积分\end{CJK}
$$
I=\int_{L}(x y+y z+z x) \mathrm{d} s,
$$
\begin{CJK}{UTF8}{mj}其中\end{CJK} $L$ \begin{CJK}{UTF8}{mj}为球面\end{CJK} $x^{2}+y^{2}+z^{2}=4$ \begin{CJK}{UTF8}{mj}和平面\end{CJK} $x+y+z=0$ \begin{CJK}{UTF8}{mj}的交线\end{CJK}.

\begin{CJK}{UTF8}{mj}九\end{CJK}. (12 \begin{CJK}{UTF8}{mj}分\end{CJK}) \begin{CJK}{UTF8}{mj}计算曲面积分\end{CJK}
$$
I=\iint_{\Sigma} x^{3} \mathrm{~d} y \mathrm{~d} z+y^{3} \mathrm{~d} z \mathrm{~d} x+z^{3} \mathrm{~d} x \mathrm{~d} y,
$$
\begin{CJK}{UTF8}{mj}其中\end{CJK} $\Sigma$ \begin{CJK}{UTF8}{mj}为立方体\end{CJK} $0 \leqslant x, y, z \leqslant a$ \begin{CJK}{UTF8}{mj}的表面\end{CJK}, \begin{CJK}{UTF8}{mj}方向取内侧\end{CJK}. \begin{CJK}{UTF8}{mj}十\end{CJK}. (12 \begin{CJK}{UTF8}{mj}分\end{CJK}) \begin{CJK}{UTF8}{mj}求函数\end{CJK}
$$
f(x, y, z)=x-2 y+2 z
$$
\begin{CJK}{UTF8}{mj}在\end{CJK} $\Sigma: x^{2}+y^{2}+z^{2}=1$ \begin{CJK}{UTF8}{mj}上的最大值和最小值\end{CJK}.

\begin{CJK}{UTF8}{mj}十一\end{CJK}. ( 10 \begin{CJK}{UTF8}{mj}分\end{CJK}) \begin{CJK}{UTF8}{mj}讨论\end{CJK}
$$
\sum_{n=1}^{\infty}\left(1+\frac{1}{2}+\frac{1}{3}+\cdots+\frac{1}{n}\right) \cdot \frac{\sin n x}{n}
$$
\begin{CJK}{UTF8}{mj}的敛散性\end{CJK}.

\section{6. 湘潭大学 2014 年研究生入学考试试题数学分析}
\begin{CJK}{UTF8}{mj}李扬\end{CJK}

\begin{CJK}{UTF8}{mj}微信公众号\end{CJK}: sxkyliyang

\begin{CJK}{UTF8}{mj}一\end{CJK}. \begin{CJK}{UTF8}{mj}求下列极限\end{CJK} (\begin{CJK}{UTF8}{mj}每小题\end{CJK} 10 \begin{CJK}{UTF8}{mj}分\end{CJK}, \begin{CJK}{UTF8}{mj}共\end{CJK} 20 \begin{CJK}{UTF8}{mj}分\end{CJK})

(1)
$$
\lim _{n \rightarrow \infty}\left(1+\frac{1}{2}+\frac{1}{3}+\cdots+\frac{1}{n}\right)^{\frac{1}{n}} .
$$
(2)
$$
\lim _{x \rightarrow 0}(\cos x)^{\frac{1}{x^{2}}} .
$$
\begin{CJK}{UTF8}{mj}二\end{CJK}. ( 10 \begin{CJK}{UTF8}{mj}分\end{CJK}) \begin{CJK}{UTF8}{mj}设\end{CJK} $x_{1}=1, x_{n+1}=\sqrt{4+3 x_{n}}, n=1,2,3, \cdots$. \begin{CJK}{UTF8}{mj}证明\end{CJK}: \begin{CJK}{UTF8}{mj}数列\end{CJK} $\left\{x_{n}\right\}$ \begin{CJK}{UTF8}{mj}收敛\end{CJK}, \begin{CJK}{UTF8}{mj}并求此极限\end{CJK}

\begin{CJK}{UTF8}{mj}三\end{CJK}. (10 \begin{CJK}{UTF8}{mj}分\end{CJK}) \begin{CJK}{UTF8}{mj}证明\end{CJK} $\sin x^{2}$ \begin{CJK}{UTF8}{mj}在\end{CJK} $(-\infty,+\infty)$ \begin{CJK}{UTF8}{mj}上不一致连续\end{CJK}, \begin{CJK}{UTF8}{mj}但在\end{CJK} $[0, A](A>0)$ \begin{CJK}{UTF8}{mj}上一致连续\end{CJK}.

\begin{CJK}{UTF8}{mj}四\end{CJK}. (10 \begin{CJK}{UTF8}{mj}分\end{CJK}) \begin{CJK}{UTF8}{mj}讨论函数\end{CJK}
$$
f(x)= \begin{cases}x \mathrm{e}^{x}, & x>0 \\ a x^{2}, & x \leqslant 0\end{cases}
$$
\begin{CJK}{UTF8}{mj}在\end{CJK} $x=0$ \begin{CJK}{UTF8}{mj}处的可导性\end{CJK}.

\begin{CJK}{UTF8}{mj}五\end{CJK}. \begin{CJK}{UTF8}{mj}解答题\end{CJK} (\begin{CJK}{UTF8}{mj}每小题\end{CJK} 10 \begin{CJK}{UTF8}{mj}分\end{CJK}, \begin{CJK}{UTF8}{mj}共\end{CJK} 20 \begin{CJK}{UTF8}{mj}分\end{CJK})

(1) \begin{CJK}{UTF8}{mj}求\end{CJK}

\includegraphics[max width=\textwidth]{2022_04_18_3416d289b173eb9de8c1g-076}

(2) \begin{CJK}{UTF8}{mj}求曲线\end{CJK} $y^{2}=x$ \begin{CJK}{UTF8}{mj}和直线\end{CJK} $y=x$ \begin{CJK}{UTF8}{mj}所围成的图形的面积\end{CJK}.

\begin{CJK}{UTF8}{mj}六\end{CJK}. (10 \begin{CJK}{UTF8}{mj}分\end{CJK}) \begin{CJK}{UTF8}{mj}讨论\end{CJK}

\begin{CJK}{UTF8}{mj}的敛散性\end{CJK} ( $a$ \begin{CJK}{UTF8}{mj}是常数\end{CJK})

\includegraphics[max width=\textwidth]{2022_04_18_3416d289b173eb9de8c1g-076(1)}

\begin{CJK}{UTF8}{mj}七\end{CJK}. (10 \begin{CJK}{UTF8}{mj}分\end{CJK}) \begin{CJK}{UTF8}{mj}讨论级数\end{CJK}
$$
\sum_{n=1}^{\infty}\left(1-\cos \frac{\pi}{n}\right)
$$
\begin{CJK}{UTF8}{mj}的敛散性\end{CJK}.

\begin{CJK}{UTF8}{mj}八\end{CJK}. $(10$ \begin{CJK}{UTF8}{mj}分\end{CJK} $)$ \begin{CJK}{UTF8}{mj}求函数\end{CJK}
$$
f(x, y)=4 x+x y^{2}+y^{2}
$$
\begin{CJK}{UTF8}{mj}在区域\end{CJK} $D: x^{2}+y^{2} \leqslant 1$ \begin{CJK}{UTF8}{mj}上的最大值和最小值\end{CJK}.

\begin{CJK}{UTF8}{mj}九\end{CJK}. (13 \begin{CJK}{UTF8}{mj}分\end{CJK}) \begin{CJK}{UTF8}{mj}求积分\end{CJK}
$$
I=\iiint_{\Omega} \frac{z}{\left(x^{2}+y^{2}+z^{2}\right)^{\frac{3}{2}}} \mathrm{~d} x \mathrm{~d} y \mathrm{~d} z,
$$
\begin{CJK}{UTF8}{mj}其中\end{CJK} $\Omega$ \begin{CJK}{UTF8}{mj}是柱面\end{CJK} $x^{2}+y^{2}=1$ \begin{CJK}{UTF8}{mj}和雉面\end{CJK} $x^{2}+y^{2}=z^{2}(h>0)$ \begin{CJK}{UTF8}{mj}以及平面\end{CJK} $z=h(h>1)$ \begin{CJK}{UTF8}{mj}所围成的立体\end{CJK}.

\begin{CJK}{UTF8}{mj}十\end{CJK}. (15 \begin{CJK}{UTF8}{mj}分\end{CJK}) \begin{CJK}{UTF8}{mj}计算曲面积分\end{CJK}
$$
I=\iint_{\Sigma} \frac{\mathrm{d} S}{z}
$$
\begin{CJK}{UTF8}{mj}其中\end{CJK} $\Sigma$ \begin{CJK}{UTF8}{mj}是球面\end{CJK} $x^{2}+y^{2}+z^{2}=a^{2}$ \begin{CJK}{UTF8}{mj}在平面\end{CJK} $z=h(0<h<a)$ \begin{CJK}{UTF8}{mj}之上的部分\end{CJK}. \begin{CJK}{UTF8}{mj}十一\end{CJK}. (12 \begin{CJK}{UTF8}{mj}分\end{CJK}) \begin{CJK}{UTF8}{mj}求幂级数\end{CJK}
$$
\sum_{n=0}^{\infty} \frac{n^{2}+1}{2^{n} n !} x^{n}
$$
\begin{CJK}{UTF8}{mj}的和函数\end{CJK}.

\begin{CJK}{UTF8}{mj}十二\end{CJK}. (10 \begin{CJK}{UTF8}{mj}分\end{CJK}) \begin{CJK}{UTF8}{mj}证明\end{CJK}:
$$
\int_{0}^{+\infty} x \mathrm{e}^{-\alpha x} \mathrm{~d} x
$$
\begin{CJK}{UTF8}{mj}在\end{CJK} $0<\alpha_{0} \leqslant \alpha<+\infty$ \begin{CJK}{UTF8}{mj}上一致收敛\end{CJK}, \begin{CJK}{UTF8}{mj}但在\end{CJK} $0<\alpha<+\infty$ \begin{CJK}{UTF8}{mj}内不一致收敛\end{CJK}.

\section{7. 湘潭大学 2015 年研究生入学考试试题数学分析}
\begin{CJK}{UTF8}{mj}李扬\end{CJK}

\begin{CJK}{UTF8}{mj}微信公众号\end{CJK}: sxkyliyang

\begin{CJK}{UTF8}{mj}一\end{CJK}. \begin{CJK}{UTF8}{mj}求下列极限\end{CJK} (\begin{CJK}{UTF8}{mj}每小题\end{CJK} 10 \begin{CJK}{UTF8}{mj}分\end{CJK}, \begin{CJK}{UTF8}{mj}共\end{CJK} 30 \begin{CJK}{UTF8}{mj}分\end{CJK})

(1) \begin{CJK}{UTF8}{mj}设\end{CJK} $a>b>0$, \begin{CJK}{UTF8}{mj}求\end{CJK}
$$
\lim _{n \rightarrow \infty} \sqrt[n]{a^{n}+b^{n}}
$$
(2)
$$
\lim _{x \rightarrow 0}(\cos x)^{\frac{1}{x^{2}}}
$$
(3)
$$
\lim _{x \rightarrow 0} \frac{\tan x-\sin x}{x^{3}} .
$$
\begin{CJK}{UTF8}{mj}二\end{CJK}. ( 15 \begin{CJK}{UTF8}{mj}分\end{CJK}) \begin{CJK}{UTF8}{mj}设\end{CJK} $x_{1}=\sqrt{2}, x_{n+1}=\sqrt{2+x_{n}}, n=1,2,3, \cdots$. \begin{CJK}{UTF8}{mj}证明\end{CJK}: \begin{CJK}{UTF8}{mj}数列\end{CJK} $\left\{x_{n}\right\}$ \begin{CJK}{UTF8}{mj}收敛\end{CJK}, \begin{CJK}{UTF8}{mj}并求此极限\end{CJK}.

\begin{CJK}{UTF8}{mj}三\end{CJK}. (12 \begin{CJK}{UTF8}{mj}分\end{CJK}) \begin{CJK}{UTF8}{mj}设函数\end{CJK} $f(x)$ \begin{CJK}{UTF8}{mj}在有限开区间\end{CJK} $(a, b)$ \begin{CJK}{UTF8}{mj}上一致连续\end{CJK}. \begin{CJK}{UTF8}{mj}证明\end{CJK}: $f(x)$ \begin{CJK}{UTF8}{mj}在\end{CJK} $(a, b)$ \begin{CJK}{UTF8}{mj}上有界\end{CJK}.

\begin{CJK}{UTF8}{mj}四\end{CJK}. (13 \begin{CJK}{UTF8}{mj}分\end{CJK}) \begin{CJK}{UTF8}{mj}求函数\end{CJK}
$$
f(x)= \begin{cases}x^{2} \sin \frac{1}{x}, & x \neq 0 \\ 0, & x=0\end{cases}
$$
\begin{CJK}{UTF8}{mj}的导函数\end{CJK}.

\begin{CJK}{UTF8}{mj}五\end{CJK}. \begin{CJK}{UTF8}{mj}求下列积分\end{CJK} (\begin{CJK}{UTF8}{mj}每小题\end{CJK} 10 \begin{CJK}{UTF8}{mj}分\end{CJK}, \begin{CJK}{UTF8}{mj}共\end{CJK} 20 \begin{CJK}{UTF8}{mj}分\end{CJK})

(2)

\includegraphics[max width=\textwidth]{2022_04_18_3416d289b173eb9de8c1g-078}

\begin{CJK}{UTF8}{mj}六\end{CJK}. ( 10 \begin{CJK}{UTF8}{mj}分\end{CJK}) \begin{CJK}{UTF8}{mj}讨论级数\end{CJK}
$$
\sum_{n=1}^{\infty} \frac{n^{3}}{3^{n}}
$$
\begin{CJK}{UTF8}{mj}的敛散性\end{CJK}.

\begin{CJK}{UTF8}{mj}七\end{CJK}. (13 \begin{CJK}{UTF8}{mj}分\end{CJK}) \begin{CJK}{UTF8}{mj}求幂级数\end{CJK}
$$
\sum_{n=1}^{\infty} n(n+1) x^{n}
$$
\begin{CJK}{UTF8}{mj}的收敛半径及和函数\end{CJK}.

\begin{CJK}{UTF8}{mj}八\end{CJK}. (15 \begin{CJK}{UTF8}{mj}分\end{CJK}) \begin{CJK}{UTF8}{mj}计算\end{CJK}
$$
I=\iint_{D} x y \mathrm{~d} x \mathrm{~d} y,
$$
$D$ \begin{CJK}{UTF8}{mj}是由拋物线\end{CJK} $y^{2}=x$ \begin{CJK}{UTF8}{mj}和直线\end{CJK} $y=x-2$ \begin{CJK}{UTF8}{mj}所围成的闭区域\end{CJK}.

\begin{CJK}{UTF8}{mj}九\end{CJK}. (12 \begin{CJK}{UTF8}{mj}分\end{CJK}) \begin{CJK}{UTF8}{mj}证明\end{CJK}:
$$
\sin x \sin y \sin (x+y) \leqslant \frac{3 \sqrt{3}}{8}(0<x, y<\pi),
$$
\begin{CJK}{UTF8}{mj}并确定何时等号成立\end{CJK}. \begin{CJK}{UTF8}{mj}十\end{CJK}. (10 \begin{CJK}{UTF8}{mj}分\end{CJK}) \begin{CJK}{UTF8}{mj}设\end{CJK} $f(x, y)$ \begin{CJK}{UTF8}{mj}在\end{CJK} $D: x^{2}+y^{2} \leqslant 1$ \begin{CJK}{UTF8}{mj}上二次连续可微\end{CJK}, \begin{CJK}{UTF8}{mj}且满足\end{CJK} $\frac{\partial^{2} f}{\partial x^{2}}+\frac{\partial^{2} f}{\partial y^{2}}=\mathrm{e}^{-x^{2}-y^{2}}$. \begin{CJK}{UTF8}{mj}证明\end{CJK}:
$$
\iint_{D}\left(x \frac{\partial f}{\partial x}+y \frac{\partial f}{\partial y}\right) \mathrm{d} x \mathrm{~d} y=\frac{\pi}{2 \mathrm{e}}
$$

\section{8. 湘潭大学 2016 年研究生入学考试试题数学分析}
\begin{CJK}{UTF8}{mj}李扬\end{CJK}

\begin{CJK}{UTF8}{mj}微信公众号\end{CJK}: sxkyliyang

\begin{CJK}{UTF8}{mj}一\end{CJK}. \begin{CJK}{UTF8}{mj}求下列极限\end{CJK} (\begin{CJK}{UTF8}{mj}每小题\end{CJK} 10 \begin{CJK}{UTF8}{mj}分\end{CJK}, \begin{CJK}{UTF8}{mj}共\end{CJK} 30 \begin{CJK}{UTF8}{mj}分\end{CJK})

(1) \begin{CJK}{UTF8}{mj}设\end{CJK}
$$
x_{n}=\sum_{k=n^{2}}^{(n+1)^{2}} \frac{1}{\sqrt{k}}
$$
\begin{CJK}{UTF8}{mj}求\end{CJK} $\lim _{n \rightarrow \infty} x_{n}$.

(2)
$$
\lim _{x \rightarrow 0^{+}}(\sin x)^{x} .
$$
$(3)$
$$
\lim _{x \rightarrow 0} \frac{1-x^{2}-\mathrm{e}^{-x^{2}}}{x \sin ^{3} 2 x}
$$
\begin{CJK}{UTF8}{mj}二\end{CJK}. (15 \begin{CJK}{UTF8}{mj}分\end{CJK}) \begin{CJK}{UTF8}{mj}求函数\end{CJK}
$$
f(x)=x-\ln (1+x)
$$
\begin{CJK}{UTF8}{mj}的极值\end{CJK}, \begin{CJK}{UTF8}{mj}并确定其单调区间\end{CJK}.

\begin{CJK}{UTF8}{mj}三\end{CJK}. \begin{CJK}{UTF8}{mj}求下列积分\end{CJK} (\begin{CJK}{UTF8}{mj}每小题\end{CJK} 10 \begin{CJK}{UTF8}{mj}分\end{CJK}, \begin{CJK}{UTF8}{mj}共\end{CJK} 20 \begin{CJK}{UTF8}{mj}分\end{CJK})

(1)

(2)
$$
I=\int_{0}^{a} \sqrt{a^{2 x} \sin x \mathrm{~d} x .} \mathrm{d} x(a>0) .
$$
\begin{CJK}{UTF8}{mj}四\end{CJK}. (15 \begin{CJK}{UTF8}{mj}分\end{CJK}) \begin{CJK}{UTF8}{mj}设\end{CJK} $f(x)$ \begin{CJK}{UTF8}{mj}在\end{CJK} $[0,1]$ \begin{CJK}{UTF8}{mj}上连续\end{CJK}, \begin{CJK}{UTF8}{mj}在\end{CJK} $(0,1)$ \begin{CJK}{UTF8}{mj}内可导且\end{CJK} $0<f^{\prime}(x)<1, f(0)=0$, \begin{CJK}{UTF8}{mj}证明\end{CJK}:
$$
\left(\int_{0}^{1} f(x) \mathrm{d} x\right)^{2} \geqslant \int_{0}^{1} f^{3}(x) \mathrm{d} x
$$
\begin{CJK}{UTF8}{mj}五\end{CJK}. (15 \begin{CJK}{UTF8}{mj}分\end{CJK}) \begin{CJK}{UTF8}{mj}设\end{CJK}
$$
f_{n}(x)=x^{n}+x^{n-1}+\cdots+x^{2}+x, n=2,3, \cdots .
$$
\begin{CJK}{UTF8}{mj}证明\end{CJK}:

(1) \begin{CJK}{UTF8}{mj}方程\end{CJK} $f_{n}(x)=1$ \begin{CJK}{UTF8}{mj}在\end{CJK} $(0,1)$ \begin{CJK}{UTF8}{mj}内有唯一的实根\end{CJK} $x_{n}$.

(2) \begin{CJK}{UTF8}{mj}数列\end{CJK} $\left\{x_{n}\right\}$ \begin{CJK}{UTF8}{mj}有极限\end{CJK}, \begin{CJK}{UTF8}{mj}并求极限\end{CJK} $\lim _{n \rightarrow \infty} x_{n}$.

\begin{CJK}{UTF8}{mj}六\end{CJK}. ( 10 \begin{CJK}{UTF8}{mj}分\end{CJK}) \begin{CJK}{UTF8}{mj}设函数\end{CJK} $f$ \begin{CJK}{UTF8}{mj}连续可导\end{CJK}, $f(1)=1$, \begin{CJK}{UTF8}{mj}且当\end{CJK} $x \geqslant 1$ \begin{CJK}{UTF8}{mj}时\end{CJK}, \begin{CJK}{UTF8}{mj}有\end{CJK}
$$
f^{\prime}(x)=\frac{1}{x^{2}+f^{2}(x)} .
$$
\begin{CJK}{UTF8}{mj}证明\end{CJK}: $\lim _{x \rightarrow+\infty} f(x)$ \begin{CJK}{UTF8}{mj}存在\end{CJK}, \begin{CJK}{UTF8}{mj}且\end{CJK} $\lim _{x \rightarrow+\infty} f(x) \leqslant 1+\frac{\pi}{4}$.

\begin{CJK}{UTF8}{mj}七\end{CJK}. (12 \begin{CJK}{UTF8}{mj}分\end{CJK}) \begin{CJK}{UTF8}{mj}讨论正项级数\end{CJK}
$$
\sum_{n=1}^{\infty} \frac{p^{n} n !}{n^{n}}\left(p \in \mathbb{R}^{+}\right)
$$
\begin{CJK}{UTF8}{mj}的敛散性\end{CJK}. \begin{CJK}{UTF8}{mj}八\end{CJK}. (10 \begin{CJK}{UTF8}{mj}分\end{CJK}) \begin{CJK}{UTF8}{mj}计算\end{CJK}
$$
I=\int_{L} \frac{(x-y) \mathrm{d} x+(x+4 y) \mathrm{d} y}{x^{2}+4 y^{2}}
$$
\begin{CJK}{UTF8}{mj}其中\end{CJK} $L: x^{2}+y^{2}=1$, \begin{CJK}{UTF8}{mj}取逆时针方向\end{CJK}.

\begin{CJK}{UTF8}{mj}九\end{CJK}. (12 \begin{CJK}{UTF8}{mj}分\end{CJK}) \begin{CJK}{UTF8}{mj}证明不等式\end{CJK}:
$$
2 \pi(\sqrt{17}-4) \leqslant \iint_{x^{2}+y^{2} \leqslant 1} \frac{\mathrm{d} x \mathrm{~d} y}{\sqrt{16+\sin ^{2} x+\sin ^{2} y}} \leqslant \frac{\pi}{4}
$$
\begin{CJK}{UTF8}{mj}十\end{CJK}. (11 \begin{CJK}{UTF8}{mj}分\end{CJK}) \begin{CJK}{UTF8}{mj}讨论广义积分\end{CJK}
$$
\int_{1}^{+\infty} \frac{\cos 2 x}{x^{p}} \mathrm{~d} x(p>0)
$$
\begin{CJK}{UTF8}{mj}的敛散性\end{CJK} (\begin{CJK}{UTF8}{mj}包括绝对收敛\end{CJK}, \begin{CJK}{UTF8}{mj}条件收敛和发散\end{CJK}).

\section{9. 湘潭大学 2017 年研究生入学考试试题数学分析}
\begin{CJK}{UTF8}{mj}李扬\end{CJK}

\begin{CJK}{UTF8}{mj}微信公众号\end{CJK}: sxkyliyang

\begin{CJK}{UTF8}{mj}一\end{CJK}. \begin{CJK}{UTF8}{mj}求下列极限\end{CJK} (\begin{CJK}{UTF8}{mj}每小题\end{CJK} 10 \begin{CJK}{UTF8}{mj}分\end{CJK}, \begin{CJK}{UTF8}{mj}共\end{CJK} 30 \begin{CJK}{UTF8}{mj}分\end{CJK})

(1)
$$
\lim _{n \rightarrow \infty} \frac{2^{n} n !}{n^{n}}
$$
(2)

$\lim _{x \rightarrow+\infty}\left(\frac{2}{\pi} \arctan x\right)^{x} .$

(3) \begin{CJK}{UTF8}{mj}设\end{CJK} $x_{1}=a, x_{2}=b, x_{n+2}=\frac{x_{n}+x_{n+1}}{2}, n=1,2, \cdots$, \begin{CJK}{UTF8}{mj}求\end{CJK} $\lim _{n \rightarrow \infty} x_{n}$.

\begin{CJK}{UTF8}{mj}二\end{CJK}. (15 \begin{CJK}{UTF8}{mj}分\end{CJK}) \begin{CJK}{UTF8}{mj}设\end{CJK}
$$
x_{n}=\frac{\cos 1}{1 \cdot 2}+\frac{\cos 2}{2 \cdot 3}+\cdots+\frac{\cos n}{n \cdot(n+1)},
$$
\begin{CJK}{UTF8}{mj}证明\end{CJK}: \begin{CJK}{UTF8}{mj}数列\end{CJK} $\left\{x_{n}\right\}$ \begin{CJK}{UTF8}{mj}收敛\end{CJK}.

\begin{CJK}{UTF8}{mj}三\end{CJK}. ( 15 \begin{CJK}{UTF8}{mj}分\end{CJK}) \begin{CJK}{UTF8}{mj}证明\end{CJK}: \begin{CJK}{UTF8}{mj}函数\end{CJK} $f(x)=\ln x$ \begin{CJK}{UTF8}{mj}在\end{CJK} $[1,+\infty)$ \begin{CJK}{UTF8}{mj}上一致连续\end{CJK}.

\begin{CJK}{UTF8}{mj}四\end{CJK}. (13 \begin{CJK}{UTF8}{mj}分\end{CJK}) \begin{CJK}{UTF8}{mj}求曲线\end{CJK}
$$
f(x)=x^{n}\left(n \in \mathbb{N}^{+}\right)
$$
\begin{CJK}{UTF8}{mj}上过点\end{CJK} $(1,1)$ \begin{CJK}{UTF8}{mj}的切线与\end{CJK} $x$ \begin{CJK}{UTF8}{mj}轴交点的横坐标\end{CJK} $x_{n}$, \begin{CJK}{UTF8}{mj}并求极限\end{CJK} $\lim _{n \rightarrow \infty} f\left(x_{n}\right)$.

\begin{CJK}{UTF8}{mj}五\end{CJK}. \begin{CJK}{UTF8}{mj}求下列积分\end{CJK} (\begin{CJK}{UTF8}{mj}每小题\end{CJK} 10 \begin{CJK}{UTF8}{mj}分\end{CJK}, \begin{CJK}{UTF8}{mj}共\end{CJK} 20 \begin{CJK}{UTF8}{mj}分\end{CJK})

(1)

(2) \begin{CJK}{UTF8}{mj}设\end{CJK}

\includegraphics[max width=\textwidth]{2022_04_18_3416d289b173eb9de8c1g-082}

\begin{CJK}{UTF8}{mj}求\end{CJK} $\int_{1}^{3} f(x-2) \mathrm{d} x$

\begin{CJK}{UTF8}{mj}六\end{CJK}. ( 13 \begin{CJK}{UTF8}{mj}分\end{CJK}) \begin{CJK}{UTF8}{mj}讨论级数\end{CJK}
$$
\sum_{n=1}^{\infty} \frac{a^{n}}{1+a^{2 n}}(a>0)
$$
\begin{CJK}{UTF8}{mj}的敛散性\end{CJK}.

\begin{CJK}{UTF8}{mj}七\end{CJK}. (12 \begin{CJK}{UTF8}{mj}分\end{CJK}) \begin{CJK}{UTF8}{mj}求幂级数\end{CJK}
$$
\sum_{n=1}^{\infty} \frac{(-1)^{n}}{n \cdot \sqrt[n]{n}}\left(\frac{x}{2 x+1}\right)^{n}
$$
\begin{CJK}{UTF8}{mj}的收敛域\end{CJK}.

\begin{CJK}{UTF8}{mj}八\end{CJK}. (12 \begin{CJK}{UTF8}{mj}分\end{CJK}) \begin{CJK}{UTF8}{mj}计算\end{CJK}
$$
I=\iiint_{\Omega} z \mathrm{e}^{-\left(x^{2}+y^{2}+z^{2}\right)} \mathrm{d} x \mathrm{~d} y \mathrm{~d} z
$$
$\Omega$ \begin{CJK}{UTF8}{mj}为雉面\end{CJK} $z=\sqrt{x^{2}+y^{2}}$ \begin{CJK}{UTF8}{mj}和球面\end{CJK} $x^{2}+y^{2}+z^{2}=1$ \begin{CJK}{UTF8}{mj}所围成的闭区域\end{CJK}. \begin{CJK}{UTF8}{mj}九\end{CJK}. (10 \begin{CJK}{UTF8}{mj}分\end{CJK}) \begin{CJK}{UTF8}{mj}设\end{CJK} $f(x)=\mathrm{e}^{x^{2}} \int_{x}^{+\infty} \mathrm{e}^{-t^{2}} \mathrm{~d} t$, \begin{CJK}{UTF8}{mj}求证\end{CJK}:
$$
f(x) \leqslant \frac{\sqrt{\pi}}{2}(x \geqslant 0)
$$
\begin{CJK}{UTF8}{mj}十\end{CJK}. (10 \begin{CJK}{UTF8}{mj}分\end{CJK}) \begin{CJK}{UTF8}{mj}证明\end{CJK}:\begin{CJK}{UTF8}{mj}函数\end{CJK}
$$
F(\alpha)=\int_{1}^{+\infty} \frac{\cos x}{x^{\alpha}} \mathrm{d} x
$$
\begin{CJK}{UTF8}{mj}在\end{CJK} $(0,+\infty)$ \begin{CJK}{UTF8}{mj}上连续\end{CJK}.

\section{0. 湘潭大学 2018 年研究生入学考试试题数学分析}
\begin{CJK}{UTF8}{mj}李扬\end{CJK}

\begin{CJK}{UTF8}{mj}微信公众号\end{CJK}: sxkyliyang

\begin{CJK}{UTF8}{mj}一\end{CJK}. \begin{CJK}{UTF8}{mj}计算题\end{CJK}.

\begin{enumerate}
  \item \begin{CJK}{UTF8}{mj}求极限\end{CJK}
\end{enumerate}
$$
\lim _{n \rightarrow \infty}\left(1-\frac{1}{2 n}+\frac{1}{n^{2}}\right)^{n} .
$$

\begin{enumerate}
  \setcounter{enumi}{2}
  \item \begin{CJK}{UTF8}{mj}求极限\end{CJK}
\end{enumerate}
$$
\lim _{x \rightarrow 0} \frac{1-\cos 2 x}{x^{2}} .
$$

\begin{enumerate}
  \setcounter{enumi}{3}
  \item \begin{CJK}{UTF8}{mj}求不定积分\end{CJK}
\end{enumerate}
$$
\int \tan x \mathrm{~d} x
$$

\begin{enumerate}
  \setcounter{enumi}{4}
  \item \begin{CJK}{UTF8}{mj}已知\end{CJK} $f(x)=x^{2} \cos 2 x$, \begin{CJK}{UTF8}{mj}求\end{CJK} $f^{\prime \prime}(x)$.

  \item \begin{CJK}{UTF8}{mj}求定积分\end{CJK}

\end{enumerate}
$$
\int_{0}^{\frac{\pi}{2}} e^{x} \sin ^{2} x \mathrm{~d} x .
$$
\begin{CJK}{UTF8}{mj}二\end{CJK}. \begin{CJK}{UTF8}{mj}已知数列\end{CJK} $\left\{x_{n}\right\}$ \begin{CJK}{UTF8}{mj}满足\end{CJK} $x_{1}>0, x_{n+1}=1+\frac{x_{n}}{1+x_{n}}$, \begin{CJK}{UTF8}{mj}证明\end{CJK} $x_{n}$ \begin{CJK}{UTF8}{mj}收敛\end{CJK}, \begin{CJK}{UTF8}{mj}并求其极限\end{CJK}.

\begin{CJK}{UTF8}{mj}三\end{CJK}. \begin{CJK}{UTF8}{mj}证明不等式\end{CJK} $x-\frac{x^{2}}{2}<\ln (x+1)<x(x>0)$.

\begin{CJK}{UTF8}{mj}四\end{CJK}. \begin{CJK}{UTF8}{mj}已知函数\end{CJK}
$$
f(x)= \begin{cases}x^{2} \sin \frac{1}{x}, & x \neq 0 \\ 0, & x=0 .\end{cases}
$$
\begin{CJK}{UTF8}{mj}求函数\end{CJK} $f(x)$ \begin{CJK}{UTF8}{mj}的导函数\end{CJK} $f^{\prime}(x)$, \begin{CJK}{UTF8}{mj}并说明\end{CJK} $f^{\prime}(x)$ \begin{CJK}{UTF8}{mj}的连续性\end{CJK}, \begin{CJK}{UTF8}{mj}若有不连续点\end{CJK}, \begin{CJK}{UTF8}{mj}写出不连续点的类型并说明理由\end{CJK}.

\begin{CJK}{UTF8}{mj}五\end{CJK}. \begin{CJK}{UTF8}{mj}求由函数\end{CJK} $e^{x+y}-x y-e=0$ \begin{CJK}{UTF8}{mj}确定的隐函数\end{CJK} $y=y(x)$ \begin{CJK}{UTF8}{mj}在\end{CJK} $x=0$ \begin{CJK}{UTF8}{mj}处的切线方程\end{CJK}.

\begin{CJK}{UTF8}{mj}六\end{CJK}. \begin{CJK}{UTF8}{mj}求由\end{CJK} $y=x^{2}$ \begin{CJK}{UTF8}{mj}和\end{CJK} $x=y^{2}$ \begin{CJK}{UTF8}{mj}围成的区域的面积\end{CJK}.

\begin{CJK}{UTF8}{mj}七\end{CJK}. \begin{CJK}{UTF8}{mj}判断级数\end{CJK}
$$
\sum_{n=1}^{\infty} \int_{0}^{\frac{1}{n}} \ln (1+x) \mathrm{d} x
$$
\begin{CJK}{UTF8}{mj}的敛散性\end{CJK}.

\begin{CJK}{UTF8}{mj}八\end{CJK}. \begin{CJK}{UTF8}{mj}已知区域\end{CJK} $D=\{(x, y) \mid-1 \leqslant x \leqslant 1,0 \leqslant y \leqslant 2\}$, \begin{CJK}{UTF8}{mj}求重积分\end{CJK}
$$
\iint_{D} \sqrt{\left|y-x^{2}\right|} \mathrm{d} x \mathrm{~d} y .
$$
\begin{CJK}{UTF8}{mj}九\end{CJK}. \begin{CJK}{UTF8}{mj}求\end{CJK} $y=\frac{1}{x^{2}}$ \begin{CJK}{UTF8}{mj}在\end{CJK} $x=1$ \begin{CJK}{UTF8}{mj}处的幂级数展开式\end{CJK}.

\begin{CJK}{UTF8}{mj}十\end{CJK}. \begin{CJK}{UTF8}{mj}已知\end{CJK} $x, y$ \begin{CJK}{UTF8}{mj}是三角形的两个内角\end{CJK}, \begin{CJK}{UTF8}{mj}证明不等式\end{CJK}
$$
\sin x \sin y \sin (x+y) \leqslant \frac{3 \sqrt{3}}{8} .
$$
\begin{CJK}{UTF8}{mj}并说明等号成立的条件\end{CJK}.

\begin{CJK}{UTF8}{mj}十一\end{CJK}. \begin{CJK}{UTF8}{mj}计算曲线积分\end{CJK}
$$
\oint_{L} \frac{(x-y) \mathrm{d} x+(x+4 y) \mathrm{d} y}{x^{2}+4 y^{2}}
$$
\begin{CJK}{UTF8}{mj}其中\end{CJK} $L: x^{2}+y^{2}=1$, \begin{CJK}{UTF8}{mj}取逆时针方向\end{CJK}.

\section{1. 云南大学 2009 年研究生入学考试试题高等代数}
\begin{CJK}{UTF8}{mj}李扬\end{CJK}

\begin{CJK}{UTF8}{mj}微信公众号\end{CJK}: sxkyliyang

\begin{CJK}{UTF8}{mj}一\end{CJK}、 \begin{CJK}{UTF8}{mj}填空题\end{CJK} (\begin{CJK}{UTF8}{mj}共\end{CJK} 6 \begin{CJK}{UTF8}{mj}题\end{CJK}, \begin{CJK}{UTF8}{mj}每题\end{CJK} 5 \begin{CJK}{UTF8}{mj}分\end{CJK}, \begin{CJK}{UTF8}{mj}共\end{CJK} 30 \begin{CJK}{UTF8}{mj}分\end{CJK})

\begin{enumerate}
  \item \begin{CJK}{UTF8}{mj}设\end{CJK} $A$ \begin{CJK}{UTF8}{mj}是\end{CJK} $s$ \begin{CJK}{UTF8}{mj}阶方阵\end{CJK}, $|A|=m, B$ \begin{CJK}{UTF8}{mj}为\end{CJK} $t$ \begin{CJK}{UTF8}{mj}阶方阵\end{CJK}, $|B|=n, C=\left(\begin{array}{cc}0 & A \\ B & 0\end{array}\right)$, \begin{CJK}{UTF8}{mj}则\end{CJK} $|C|=$

  \item \begin{CJK}{UTF8}{mj}二次型\end{CJK} $f\left(x_{1}, x_{2}, x_{3}\right)=x_{1}^{2}+2 x_{2}^{2}+3 x_{3}^{2}-4 x_{1} x_{2}-4 x_{2} x_{3}$ \begin{CJK}{UTF8}{mj}的标准形是\end{CJK}

  \item \begin{CJK}{UTF8}{mj}复数域作为实数域上的线性空间是\end{CJK} \begin{CJK}{UTF8}{mj}维的\end{CJK}, \begin{CJK}{UTF8}{mj}写出一组基是\end{CJK}

\end{enumerate}
$4 .\left|\begin{array}{cccc}x_{1}-m & x_{2} & \cdots & x_{n} \\ x_{1} & x_{2}-m & \cdots & x_{n} \\ \vdots & \vdots & & \vdots \\ x_{1} & x_{2} & \cdots & x_{n}-m\end{array}\right|=$

\begin{enumerate}
  \setcounter{enumi}{5}
  \item 3 \begin{CJK}{UTF8}{mj}阶矩阵\end{CJK} $\left(\begin{array}{ccc}4 & 5 & 6 \\ 8 & 10 & 12 \\ 12 & 15 & 18\end{array}\right)$ \begin{CJK}{UTF8}{mj}在复数域上的三个特征值是\end{CJK} $\lambda_{1}=\underline{\longrightarrow}, \lambda_{2}=\square \lambda_{3}=$

  \item \begin{CJK}{UTF8}{mj}设\end{CJK} $\mathscr{A}$ \begin{CJK}{UTF8}{mj}是线性空间\end{CJK} $P^{3}$ \begin{CJK}{UTF8}{mj}的一个线性变换\end{CJK}, \begin{CJK}{UTF8}{mj}且\end{CJK} $\mathscr{A}\left(x_{1}, x_{2}, x_{3}\right)=\left(3 x_{1}-x_{3}, x_{1}+x_{3}, x_{2}\right)$, \begin{CJK}{UTF8}{mj}则\end{CJK} $\mathscr{A}$ \begin{CJK}{UTF8}{mj}在基\end{CJK} $\varepsilon_{1}=(1,0,0)$, $\varepsilon_{2}=(0,1,0), \varepsilon_{3}=(0,0,1)$ \begin{CJK}{UTF8}{mj}下的矩阵是\end{CJK}

\end{enumerate}
\begin{CJK}{UTF8}{mj}二\end{CJK}、 (15 \begin{CJK}{UTF8}{mj}分\end{CJK}) \begin{CJK}{UTF8}{mj}讨论\end{CJK} $a, b$ \begin{CJK}{UTF8}{mj}分别取何值时方程组\end{CJK}
$$
\left\{\begin{array}{l}
a x_{1}+b x_{2}+2 x_{3}=1 \\
a x_{1}+(2 b-1) x_{2}+3 x_{3}=1 \\
a x_{1}+b x_{2}+(b+3) x_{3}=2 b-1
\end{array}\right.
$$

\begin{enumerate}
  \item \begin{CJK}{UTF8}{mj}无解\end{CJK};

  \item \begin{CJK}{UTF8}{mj}有唯一解\end{CJK}, \begin{CJK}{UTF8}{mj}求其解\end{CJK};

  \item \begin{CJK}{UTF8}{mj}有无穷解\end{CJK}, \begin{CJK}{UTF8}{mj}求其通解\end{CJK}, \begin{CJK}{UTF8}{mj}说明解集合的几何意义\end{CJK}.

\end{enumerate}
\begin{CJK}{UTF8}{mj}三\end{CJK}、 $(20$ \begin{CJK}{UTF8}{mj}分\end{CJK} $)$ \begin{CJK}{UTF8}{mj}设\end{CJK}
$$
A=\left(\begin{array}{ccc}
2 & 2 & -2 \\
2 & 5 & -4 \\
-2 & -4 & 5
\end{array}\right)
$$
\begin{CJK}{UTF8}{mj}求一个正交矩阵\end{CJK} $T$ \begin{CJK}{UTF8}{mj}使得\end{CJK} $T^{\prime} A T$ \begin{CJK}{UTF8}{mj}是对角矩阵\end{CJK}, \begin{CJK}{UTF8}{mj}并写出此对角矩阵\end{CJK}.

\begin{CJK}{UTF8}{mj}四\end{CJK}、 ( 20 \begin{CJK}{UTF8}{mj}分\end{CJK}) \begin{CJK}{UTF8}{mj}在复数域上\end{CJK}, \begin{CJK}{UTF8}{mj}求矩阵\end{CJK}
$$
\left(\begin{array}{cccc}
3 & 1 & 0 & 0 \\
-4 & -1 & 0 & 0 \\
7 & 1 & 2 & 1 \\
-7 & -6 & -1 & 0
\end{array}\right)
$$
\begin{CJK}{UTF8}{mj}的若尔当标准形\end{CJK}.

\begin{CJK}{UTF8}{mj}五\end{CJK}、 (20 \begin{CJK}{UTF8}{mj}分\end{CJK}) \begin{CJK}{UTF8}{mj}设\end{CJK}
$$
A=\left(\begin{array}{lll}
1 & 0 & 0 \\
1 & 0 & 1 \\
0 & 1 & 0
\end{array}\right)
$$

\begin{enumerate}
  \item \begin{CJK}{UTF8}{mj}证明\end{CJK}: $A^{n}=A^{n-2}+A^{2}-E(n \geqslant 3)$;

  \item \begin{CJK}{UTF8}{mj}求\end{CJK} $A^{100}$.

\end{enumerate}
\begin{CJK}{UTF8}{mj}六\end{CJK}、 (15 \begin{CJK}{UTF8}{mj}分\end{CJK}) \begin{CJK}{UTF8}{mj}设\end{CJK} $A$ \begin{CJK}{UTF8}{mj}是数域\end{CJK} $P$ \begin{CJK}{UTF8}{mj}上的\end{CJK} $n$ \begin{CJK}{UTF8}{mj}阶可逆矩阵\end{CJK}, \begin{CJK}{UTF8}{mj}证明\end{CJK}: \begin{CJK}{UTF8}{mj}存在\end{CJK} $P$ \begin{CJK}{UTF8}{mj}上的多项式\end{CJK} $f(x)$ \begin{CJK}{UTF8}{mj}使得\end{CJK} $A^{-1}=f(A)$.

\begin{CJK}{UTF8}{mj}七\end{CJK}、 (15 \begin{CJK}{UTF8}{mj}分\end{CJK}) \begin{CJK}{UTF8}{mj}证明\end{CJK}: \begin{CJK}{UTF8}{mj}实二次型\end{CJK} $f\left(x_{1}, x_{2}, \cdots, x_{n}\right)=X^{\prime} A X$ \begin{CJK}{UTF8}{mj}在条件\end{CJK} $\sum_{i=1}^{n} x_{i}^{2}=1$ \begin{CJK}{UTF8}{mj}下的最大值等于\end{CJK} $A$ \begin{CJK}{UTF8}{mj}的最大的特征值\end{CJK}.

\begin{CJK}{UTF8}{mj}八\end{CJK}、 ( 20 \begin{CJK}{UTF8}{mj}分\end{CJK}) \begin{CJK}{UTF8}{mj}设\end{CJK} $\alpha, \beta$ \begin{CJK}{UTF8}{mj}是欧式空间\end{CJK} $V$ \begin{CJK}{UTF8}{mj}中的\end{CJK} 2 \begin{CJK}{UTF8}{mj}个向量\end{CJK}. \begin{CJK}{UTF8}{mj}证明\end{CJK}: $\alpha \perp \beta$ \begin{CJK}{UTF8}{mj}的充分必要条件是\end{CJK} $\forall t \in \mathbb{R},|\alpha+t \beta| \geqslant|\alpha|$.

\section{2. 云南大学 2010 年研究生入学考试试题高等代数}
\begin{CJK}{UTF8}{mj}李扬\end{CJK}

\begin{CJK}{UTF8}{mj}微信公众号\end{CJK}: sxkyliyang

\begin{CJK}{UTF8}{mj}一\end{CJK}、\begin{CJK}{UTF8}{mj}填空题\end{CJK} (\begin{CJK}{UTF8}{mj}共\end{CJK} 5 \begin{CJK}{UTF8}{mj}题\end{CJK}, \begin{CJK}{UTF8}{mj}每题\end{CJK} 6 \begin{CJK}{UTF8}{mj}分\end{CJK}, \begin{CJK}{UTF8}{mj}共\end{CJK} 30 \begin{CJK}{UTF8}{mj}分\end{CJK})

\begin{enumerate}
  \item \begin{CJK}{UTF8}{mj}设\end{CJK} 4 \begin{CJK}{UTF8}{mj}阶方阵\end{CJK} $A=\left(\alpha, \gamma_{2}, \gamma_{3}, \gamma_{4}\right), B=\left(\beta, \gamma_{2}, \gamma_{3}, \gamma_{4}\right)$, \begin{CJK}{UTF8}{mj}其中\end{CJK} $\alpha, \beta, \gamma_{2}, \gamma_{3}, \gamma_{4}$ \begin{CJK}{UTF8}{mj}均为\end{CJK} 4 \begin{CJK}{UTF8}{mj}维列向量\end{CJK}, \begin{CJK}{UTF8}{mj}且\end{CJK} $|A|=4$, $|B|=1$, \begin{CJK}{UTF8}{mj}则\end{CJK} $|A+B|=$

  \item \begin{CJK}{UTF8}{mj}设矩阵\end{CJK} $A, B$ \begin{CJK}{UTF8}{mj}满足\end{CJK} $A^{*} B A=2 B A-8 E$, \begin{CJK}{UTF8}{mj}其中\end{CJK} $A=\left(\begin{array}{ccc}1 & 0 & 0 \\ 0 & -2 & 0 \\ 0 & 0 & 1\end{array}\right), E$ \begin{CJK}{UTF8}{mj}为单位矩阵\end{CJK}, $A^{*}$ \begin{CJK}{UTF8}{mj}为\end{CJK} $A$ \begin{CJK}{UTF8}{mj}的伴随矩\end{CJK} \begin{CJK}{UTF8}{mj}阵\end{CJK}, \begin{CJK}{UTF8}{mj}则\end{CJK} $B=$

  \item \begin{CJK}{UTF8}{mj}设\end{CJK} $A$ \begin{CJK}{UTF8}{mj}是三阶实对称矩阵\end{CJK}, $A$ \begin{CJK}{UTF8}{mj}的特征值\end{CJK} $\lambda_{1}=\lambda_{2}=2, \lambda_{3}=-2$, \begin{CJK}{UTF8}{mj}则\end{CJK} $A^{2010}=$

  \item \begin{CJK}{UTF8}{mj}已知实二次型\end{CJK} $f\left(x_{1}, x_{2}, x_{3}\right)=x_{1}^{2}+4 x_{2}^{2}+4 x_{3}^{2}+2 \lambda x_{1} x_{2}-2 x_{1} x_{3}+4 x_{2} x_{3}$ \begin{CJK}{UTF8}{mj}为正定二次型\end{CJK}, \begin{CJK}{UTF8}{mj}则\end{CJK} $\lambda$ \begin{CJK}{UTF8}{mj}的取值范\end{CJK} \begin{CJK}{UTF8}{mj}围是\end{CJK}

  \item \begin{CJK}{UTF8}{mj}设\end{CJK} $V$ \begin{CJK}{UTF8}{mj}是\end{CJK} $\mathbb{R}$ \begin{CJK}{UTF8}{mj}上所有\end{CJK} $2 \times 2$ \begin{CJK}{UTF8}{mj}矩阵构成的线性空间\end{CJK}, \begin{CJK}{UTF8}{mj}则矩阵\end{CJK} $A=\left(\begin{array}{cc}2 & 3 \\ 4 & -7\end{array}\right)$ \begin{CJK}{UTF8}{mj}在\end{CJK} $V$ \begin{CJK}{UTF8}{mj}的一组基\end{CJK} $\left(\begin{array}{cc}1 & 1 \\ 1 & 1\end{array}\right)$, $\left(\begin{array}{cc}0 & -1 \\ 1 & 0\end{array}\right),\left(\begin{array}{cc}1 & -1 \\ 0 & 0\end{array}\right),\left(\begin{array}{cc}1 & 0 \\ 0 & 0\end{array}\right)$ \begin{CJK}{UTF8}{mj}下的坐标是\end{CJK}

\end{enumerate}
\begin{CJK}{UTF8}{mj}二\end{CJK}、 (10 \begin{CJK}{UTF8}{mj}分\end{CJK}) \begin{CJK}{UTF8}{mj}设\end{CJK} 4 \begin{CJK}{UTF8}{mj}阶行列式\end{CJK} $A=\left|\begin{array}{cccc}3 & -5 & 2 & d \\ a & b & c & d \\ a^{2} & b^{2} & c^{2} & d^{2} \\ a^{4} & b^{4} & c^{4} & d^{4}\end{array}\right|$, \begin{CJK}{UTF8}{mj}计算\end{CJK} $A_{11}+A_{12}+A_{13}+A_{14}$, \begin{CJK}{UTF8}{mj}其中\end{CJK} $A_{i j}$ \begin{CJK}{UTF8}{mj}是元素\end{CJK} $a_{i j}$ \begin{CJK}{UTF8}{mj}的代数余\end{CJK} \begin{CJK}{UTF8}{mj}子式\end{CJK}.

\begin{CJK}{UTF8}{mj}三\end{CJK}、 ( 15 \begin{CJK}{UTF8}{mj}分\end{CJK}) \begin{CJK}{UTF8}{mj}设多项式\end{CJK} $f(x), g(x)$ \begin{CJK}{UTF8}{mj}满足\end{CJK}: $f\left(x^{3}\right)+x g\left(x^{3}\right)=\left(x^{2}+x+1\right) h(x), h(x)$ \begin{CJK}{UTF8}{mj}为一多项式\end{CJK}. \begin{CJK}{UTF8}{mj}证明\end{CJK}: $x-1$ \begin{CJK}{UTF8}{mj}是\end{CJK} $f(x), g(x)$ \begin{CJK}{UTF8}{mj}的一个公因式\end{CJK}.

\begin{CJK}{UTF8}{mj}四\end{CJK}、(15 \begin{CJK}{UTF8}{mj}分\end{CJK}) \begin{CJK}{UTF8}{mj}已知线性方程组\end{CJK}
$$
\left\{\begin{array}{l}
x_{1}+x_{2}+2 x_{3}+3 x_{4}=1 \\
x_{1}+3 x_{2}+6 x_{3}+x_{4}=3 \\
3 x_{1}-x_{2}-a x_{3}+15 x_{4}=3 \\
x_{1}-5 x_{2}-10 x_{3}+12 x_{4}=b
\end{array}\right.
$$
\begin{CJK}{UTF8}{mj}问\end{CJK} $a, b$ \begin{CJK}{UTF8}{mj}取何值时\end{CJK}, \begin{CJK}{UTF8}{mj}方程组无解\end{CJK}? \begin{CJK}{UTF8}{mj}有惟一解\end{CJK}? \begin{CJK}{UTF8}{mj}有无穷多解\end{CJK}? \begin{CJK}{UTF8}{mj}在方程组有无穷多解时\end{CJK}, \begin{CJK}{UTF8}{mj}求其通解\end{CJK}.

\begin{CJK}{UTF8}{mj}五\end{CJK}、 ( 20 \begin{CJK}{UTF8}{mj}分\end{CJK}) \begin{CJK}{UTF8}{mj}用正交变换把下面二次型转化为标准形\end{CJK}, \begin{CJK}{UTF8}{mj}并判断二次型是否正定\end{CJK}.
$$
f\left(x_{1}, x_{2}, x_{3}, x_{4}\right)=2 x_{1} x_{2}+2 x_{1} x_{3}-2 x_{1} x_{4}-2 x_{2} x_{3}+2 x_{2} x_{4}+2 x_{3} x_{4}
$$
\begin{CJK}{UTF8}{mj}六\end{CJK}、 (10 \begin{CJK}{UTF8}{mj}分\end{CJK}) $\mathbb{R}^{3}$ \begin{CJK}{UTF8}{mj}中线性变换\end{CJK} $\mathscr{A}$ \begin{CJK}{UTF8}{mj}在基\end{CJK} $\varepsilon_{1}=\left(\begin{array}{l}1 \\ 0 \\ 0\end{array}\right), \varepsilon_{2}=\left(\begin{array}{l}0 \\ 1 \\ 0\end{array}\right), \varepsilon_{3}=\left(\begin{array}{cc}0 \\ 0 \\ 1\end{array}\right)$ \begin{CJK}{UTF8}{mj}下的矩阵\end{CJK} $A=\left(\begin{array}{ccc}-1 & 2 & 0 \\ 1 & 1 & -1 \\ 0 & -1\end{array}\right)$, \begin{CJK}{UTF8}{mj}求\end{CJK} $\mathscr{A}$ \begin{CJK}{UTF8}{mj}在基\end{CJK} $\eta_{1}=\left(\begin{array}{c}1 \\ 1 \\ 1\end{array}\right), \eta_{2}=\left(\begin{array}{l}1 \\ 1 \\ 0\end{array}\right), \eta_{3}=\left(\begin{array}{l}1 \\ 0 \\ 0\end{array}\right)$ \begin{CJK}{UTF8}{mj}下的矩阵\end{CJK}. \begin{CJK}{UTF8}{mj}七\end{CJK}、 ( 20 \begin{CJK}{UTF8}{mj}分\end{CJK}) \begin{CJK}{UTF8}{mj}设\end{CJK} $V=P^{2 \times 2}(V$ \begin{CJK}{UTF8}{mj}是数域\end{CJK} $P$ \begin{CJK}{UTF8}{mj}上所有\end{CJK} 2 \begin{CJK}{UTF8}{mj}阶方阵构成的线性空间\end{CJK} $), A=\left(\begin{array}{cc}1 & -1 \\ 0\end{array}\right) \in V$, \begin{CJK}{UTF8}{mj}且\end{CJK} $W=\{X \mid A X=X A, X \in V\} .$

\begin{enumerate}
  \item \begin{CJK}{UTF8}{mj}证明\end{CJK} $W$ \begin{CJK}{UTF8}{mj}是\end{CJK} $V$ \begin{CJK}{UTF8}{mj}的子空间\end{CJK};

  \item \begin{CJK}{UTF8}{mj}求\end{CJK} $W$ \begin{CJK}{UTF8}{mj}的维数\end{CJK};

  \item \begin{CJK}{UTF8}{mj}求\end{CJK} $V$ \begin{CJK}{UTF8}{mj}的一个线性变换\end{CJK} $\sigma$ \begin{CJK}{UTF8}{mj}使\end{CJK} $\sigma(V)=W$.

\end{enumerate}
\begin{CJK}{UTF8}{mj}八\end{CJK}、 (20 \begin{CJK}{UTF8}{mj}分\end{CJK}) \begin{CJK}{UTF8}{mj}设\end{CJK} $A$ \begin{CJK}{UTF8}{mj}是数域\end{CJK} $P$ \begin{CJK}{UTF8}{mj}上的\end{CJK} $n$ \begin{CJK}{UTF8}{mj}阶方阵\end{CJK}, $V_{1}=\left\{X \mid(A-E) X=0, X \in P^{n}\right\}, V_{2}=\left\{X \mid(A+E) X=0, X \in P^{n}\right\}$. \begin{CJK}{UTF8}{mj}证明\end{CJK}: $P^{n}=V_{1} \oplus V_{2}$ \begin{CJK}{UTF8}{mj}的充要条件是\end{CJK} $A^{2}=E$, \begin{CJK}{UTF8}{mj}其中\end{CJK} $E$ \begin{CJK}{UTF8}{mj}为\end{CJK} $n$ \begin{CJK}{UTF8}{mj}阶单位矩阵\end{CJK}.

\begin{CJK}{UTF8}{mj}九\end{CJK}、 ( 10 \begin{CJK}{UTF8}{mj}分\end{CJK}) \begin{CJK}{UTF8}{mj}设\end{CJK} $A$ \begin{CJK}{UTF8}{mj}为\end{CJK} $n$ \begin{CJK}{UTF8}{mj}阶正定矩阵\end{CJK}, $\alpha_{1}, \alpha_{2}, \cdots, \alpha_{n}, \beta$ \begin{CJK}{UTF8}{mj}为\end{CJK} $n$ \begin{CJK}{UTF8}{mj}维欧式空间中的列向量\end{CJK}. \begin{CJK}{UTF8}{mj}若已知\end{CJK} $\alpha_{i} \neq 0$ $(i=1,2, \cdots, n), \beta$ \begin{CJK}{UTF8}{mj}与\end{CJK} $\alpha_{1}, \alpha_{2}, \cdots, \alpha_{n}$ \begin{CJK}{UTF8}{mj}都正交\end{CJK}, \begin{CJK}{UTF8}{mj}且\end{CJK} $\alpha_{i}^{\prime} A \alpha_{j}=0(i \neq j, i, j=1,2, \cdots, n)$, \begin{CJK}{UTF8}{mj}其中\end{CJK} $\alpha_{i}^{\prime}$ \begin{CJK}{UTF8}{mj}是\end{CJK} $\alpha_{i}$ \begin{CJK}{UTF8}{mj}的转\end{CJK} \begin{CJK}{UTF8}{mj}置\end{CJK}. \begin{CJK}{UTF8}{mj}证明\end{CJK}: $\beta=0$.

\section{3. 云南大学 2011 年研究生入学考试试题高等代数}
\begin{CJK}{UTF8}{mj}李扬\end{CJK}

\begin{CJK}{UTF8}{mj}微信公众号\end{CJK}: sxkyliyang

\begin{CJK}{UTF8}{mj}一\end{CJK}、\begin{CJK}{UTF8}{mj}填空题\end{CJK} (\begin{CJK}{UTF8}{mj}共\end{CJK} 7 \begin{CJK}{UTF8}{mj}题\end{CJK}, \begin{CJK}{UTF8}{mj}每题\end{CJK} 5 \begin{CJK}{UTF8}{mj}分\end{CJK}, \begin{CJK}{UTF8}{mj}共\end{CJK} 35 \begin{CJK}{UTF8}{mj}分\end{CJK})

\begin{enumerate}
  \item \begin{CJK}{UTF8}{mj}设\end{CJK}
\end{enumerate}
$$
f(x)=\left|\begin{array}{cccc}
3 x & x & 1 & 2 \\
-1 & x & 1 & -1 \\
4 & -2 & x & 1 \\
1 & 3 & -1 & -x
\end{array}\right|
$$
\begin{CJK}{UTF8}{mj}则\end{CJK} $f(x)$ \begin{CJK}{UTF8}{mj}中\end{CJK} $x^{4}$ \begin{CJK}{UTF8}{mj}的系数是\end{CJK} ,$x^{3}$ \begin{CJK}{UTF8}{mj}的系数是\end{CJK}

\begin{enumerate}
  \setcounter{enumi}{2}
  \item \begin{CJK}{UTF8}{mj}如果\end{CJK} $n$ \begin{CJK}{UTF8}{mj}级排列\end{CJK} $x_{1} x_{2} \cdots x_{n-1} x_{n}$ \begin{CJK}{UTF8}{mj}的逆序数是\end{CJK} $k$, \begin{CJK}{UTF8}{mj}那么\end{CJK} $n$ \begin{CJK}{UTF8}{mj}级排列\end{CJK} $x_{n} x_{n-1} \cdots x_{2} x_{1}$ \begin{CJK}{UTF8}{mj}的逆序数是\end{CJK}

  \item \begin{CJK}{UTF8}{mj}设\end{CJK} $\mathbb{R}^{+}$\begin{CJK}{UTF8}{mj}是全体正实数的集合\end{CJK}, $\mathbb{R}$ \begin{CJK}{UTF8}{mj}是实数域\end{CJK}, \begin{CJK}{UTF8}{mj}任取\end{CJK} $a, b \in \mathbb{R}^{+}, m \in \mathbb{R}$, \begin{CJK}{UTF8}{mj}加法和数量乘法定义为\end{CJK}: $a \oplus b=a b$, $m \circ a=a^{m}$. \begin{CJK}{UTF8}{mj}则\end{CJK} $\mathbb{R}^{+}$\begin{CJK}{UTF8}{mj}构成实数域上的线性空间\end{CJK}, \begin{CJK}{UTF8}{mj}维数是\end{CJK} \begin{CJK}{UTF8}{mj}一组基是\end{CJK}

  \item \begin{CJK}{UTF8}{mj}如果\end{CJK} $(f(x), g(x))=1$, \begin{CJK}{UTF8}{mj}那么\end{CJK} $(f(x) g(x), f(x)+g(x))=$

  \item \begin{CJK}{UTF8}{mj}求微商\end{CJK} $\mathscr{D}(f(x))=f^{\prime}(x)$ \begin{CJK}{UTF8}{mj}是线性空间\end{CJK} $P[x]_{3}$ \begin{CJK}{UTF8}{mj}的一个线性变换\end{CJK} (\begin{CJK}{UTF8}{mj}其中\end{CJK} $f(x) \in P[x]_{3}$ ), \begin{CJK}{UTF8}{mj}则\end{CJK} $\mathscr{D}$ \begin{CJK}{UTF8}{mj}在基\end{CJK} $1, x, x^{2}$ \begin{CJK}{UTF8}{mj}下\end{CJK} \begin{CJK}{UTF8}{mj}的矩阵是\end{CJK}

  \item \begin{CJK}{UTF8}{mj}设\end{CJK} $\mathscr{A}$ \begin{CJK}{UTF8}{mj}是维线性空间\end{CJK} $V$ \begin{CJK}{UTF8}{mj}的线性变换\end{CJK}, \begin{CJK}{UTF8}{mj}那么\end{CJK} $\mathscr{A}$ \begin{CJK}{UTF8}{mj}的秩\end{CJK} $+\mathscr{A}$ \begin{CJK}{UTF8}{mj}的零度\end{CJK}= ,\begin{CJK}{UTF8}{mj}一般来说\end{CJK} $\mathscr{A} V+\mathscr{A}^{-1}(0) \neq V$, \begin{CJK}{UTF8}{mj}例如\end{CJK}:

  \item \begin{CJK}{UTF8}{mj}已知\end{CJK} 3 \begin{CJK}{UTF8}{mj}级方阵\end{CJK} $A$ \begin{CJK}{UTF8}{mj}的特征值为\end{CJK} $1,-1$ \begin{CJK}{UTF8}{mj}和\end{CJK} 2 , \begin{CJK}{UTF8}{mj}则\end{CJK} 3 \begin{CJK}{UTF8}{mj}级方阵\end{CJK} $B=A^{3}-2 A^{2}$ \begin{CJK}{UTF8}{mj}所有的特征值为\end{CJK} $|B|=$

\end{enumerate}
\begin{CJK}{UTF8}{mj}二\end{CJK}、 $(20$ \begin{CJK}{UTF8}{mj}分\end{CJK}) \begin{CJK}{UTF8}{mj}求线性方程组\end{CJK}
$$
\left\{\begin{array}{l}
2 x_{1}-2 x_{2}+x_{3}-x_{4}+x_{5}=1 \\
x_{1}+2 x_{2}-x_{3}+x_{4}-2 x_{5}=1 \\
4 x_{1}-10 x_{2}+5 x_{3}-5 x_{4}+7 x_{5}=1 \\
2 x_{1}-14 x_{2}+7 x_{3}-7 x_{4}+11 x_{5}=1
\end{array}\right.
$$
\begin{CJK}{UTF8}{mj}的全部解\end{CJK}, \begin{CJK}{UTF8}{mj}要求用其导出组的基础解系来表示\end{CJK}.

\begin{CJK}{UTF8}{mj}三\end{CJK}、 $(20$ \begin{CJK}{UTF8}{mj}分\end{CJK} $)$ \begin{CJK}{UTF8}{mj}设二次型\end{CJK}
$$
f\left(x_{1}, x_{2}, x_{3}\right)=2 x_{1}^{2}+2 x_{2}^{2}+2 x_{3}^{2}-2 x_{1} x_{2}-2 x_{1} x_{3}-2 x_{2} x_{3}
$$

\begin{enumerate}
  \item \begin{CJK}{UTF8}{mj}写出二次型的矩阵\end{CJK};

  \item \begin{CJK}{UTF8}{mj}用正交线性替换化二次型为标准形\end{CJK};

  \item \begin{CJK}{UTF8}{mj}求二次型的秩\end{CJK};

  \item \begin{CJK}{UTF8}{mj}判断二次型的正定性\end{CJK}.

\end{enumerate}
\begin{CJK}{UTF8}{mj}四\end{CJK}、(15 \begin{CJK}{UTF8}{mj}分\end{CJK}) \begin{CJK}{UTF8}{mj}计算行列式\end{CJK}
$$
\left|\begin{array}{ccccc}
x+1 & x & x & \cdots & x \\
x & x+2 & x & \cdots & x \\
x & x & x+2^{2} & \cdots & x \\
\vdots & \vdots & \vdots & & \vdots \\
x & x & x & \cdots & x+2^{n}
\end{array}\right|
$$
\begin{CJK}{UTF8}{mj}五\end{CJK}、 ( 15 \begin{CJK}{UTF8}{mj}分\end{CJK}) \begin{CJK}{UTF8}{mj}设\end{CJK} 3 \begin{CJK}{UTF8}{mj}级方阵\end{CJK}
$$
A=\left(\begin{array}{ccc}
\frac{1}{\sqrt{2}} & x & 0 \\
0 & 0 & 1 \\
y & z & 0
\end{array}\right)
$$

\begin{enumerate}
  \item \begin{CJK}{UTF8}{mj}当\end{CJK} $x, y, z$ \begin{CJK}{UTF8}{mj}满足何关系时\end{CJK}, $A$ \begin{CJK}{UTF8}{mj}为可逆矩阵\end{CJK}?

  \item \begin{CJK}{UTF8}{mj}当\end{CJK} $x, y, z$ \begin{CJK}{UTF8}{mj}取何值时\end{CJK}, $A$ \begin{CJK}{UTF8}{mj}为对称矩阵\end{CJK}?

  \item \begin{CJK}{UTF8}{mj}当\end{CJK} $x, y, z$ \begin{CJK}{UTF8}{mj}取何值时\end{CJK}, $A$ \begin{CJK}{UTF8}{mj}为正交矩阵\end{CJK}?

\end{enumerate}
\begin{CJK}{UTF8}{mj}六\end{CJK}、 ( 15 \begin{CJK}{UTF8}{mj}分\end{CJK}) \begin{CJK}{UTF8}{mj}设\end{CJK} $f(x)$ \begin{CJK}{UTF8}{mj}是一个整系数多项式\end{CJK}, \begin{CJK}{UTF8}{mj}证明\end{CJK}: \begin{CJK}{UTF8}{mj}如果\end{CJK} $f(0)$ \begin{CJK}{UTF8}{mj}与\end{CJK} $f(1)$ \begin{CJK}{UTF8}{mj}都是奇数\end{CJK}, \begin{CJK}{UTF8}{mj}那么\end{CJK} $(x)$ \begin{CJK}{UTF8}{mj}不能有整数根\end{CJK}.

\begin{CJK}{UTF8}{mj}七\end{CJK}、 (15 \begin{CJK}{UTF8}{mj}分\end{CJK}) \begin{CJK}{UTF8}{mj}设\end{CJK} $A, B$ \begin{CJK}{UTF8}{mj}都是\end{CJK} $n$ \begin{CJK}{UTF8}{mj}级实对称矩阵\end{CJK}, \begin{CJK}{UTF8}{mj}而且\end{CJK} $A$ \begin{CJK}{UTF8}{mj}是正定矩阵\end{CJK}. \begin{CJK}{UTF8}{mj}证明\end{CJK}: \begin{CJK}{UTF8}{mj}存在实可逆矩阵\end{CJK} $T$ \begin{CJK}{UTF8}{mj}使得\end{CJK} $T^{\prime}(A+B) T$ \begin{CJK}{UTF8}{mj}是对\end{CJK} \begin{CJK}{UTF8}{mj}角矩阵\end{CJK}.

\begin{CJK}{UTF8}{mj}八\end{CJK}、(15 \begin{CJK}{UTF8}{mj}分\end{CJK}) \begin{CJK}{UTF8}{mj}设\end{CJK} $\mathscr{A}$ \begin{CJK}{UTF8}{mj}是\end{CJK} $n$ \begin{CJK}{UTF8}{mj}维线性空间\end{CJK} $V$ \begin{CJK}{UTF8}{mj}的线性变换\end{CJK}, \begin{CJK}{UTF8}{mj}向量\end{CJK} $\alpha_{1}, \alpha_{2}, \cdots, \alpha_{k}$ \begin{CJK}{UTF8}{mj}是\end{CJK} $\mathscr{A}$ \begin{CJK}{UTF8}{mj}的分别属于不同特征值\end{CJK} $\lambda_{1}, \lambda_{2}, \cdots, \lambda_{k}$ \begin{CJK}{UTF8}{mj}的相应的特征向量\end{CJK}, \begin{CJK}{UTF8}{mj}其中\end{CJK} $1 \leqslant k \leqslant n$. \begin{CJK}{UTF8}{mj}证明\end{CJK}: \begin{CJK}{UTF8}{mj}如果\end{CJK} $W$ \begin{CJK}{UTF8}{mj}是\end{CJK} $\mathscr{A}$ \begin{CJK}{UTF8}{mj}的不变子空间\end{CJK}, \begin{CJK}{UTF8}{mj}而且\end{CJK} $\alpha_{1}+\alpha_{2}+\cdots+$ $\alpha_{k} \in W$, \begin{CJK}{UTF8}{mj}那么\end{CJK} $\operatorname{dim} W \geqslant k$.

\section{4. 云南大学 2012 年研究生入学考试试题高等代数}
\begin{CJK}{UTF8}{mj}李扬\end{CJK}

\begin{CJK}{UTF8}{mj}微信公众号\end{CJK}: sxkyliyang

\begin{CJK}{UTF8}{mj}一\end{CJK}、\begin{CJK}{UTF8}{mj}填空题\end{CJK} (\begin{CJK}{UTF8}{mj}共\end{CJK} 5 \begin{CJK}{UTF8}{mj}题\end{CJK}, \begin{CJK}{UTF8}{mj}每题\end{CJK} 6 \begin{CJK}{UTF8}{mj}分\end{CJK}, \begin{CJK}{UTF8}{mj}共\end{CJK} 30 \begin{CJK}{UTF8}{mj}分\end{CJK})

\begin{enumerate}
  \item \begin{CJK}{UTF8}{mj}设\end{CJK} $A=\left(\begin{array}{ll}1 & 0 \\ 2 & 1\end{array}\right), f(x)=3 x^{2}-5 x+4$. \begin{CJK}{UTF8}{mj}则\end{CJK} $f(A)=$

  \item \begin{CJK}{UTF8}{mj}设\end{CJK} $\alpha_{1}, \alpha_{2}, \alpha_{3}, \beta_{1}, \beta_{2}$ \begin{CJK}{UTF8}{mj}都是\end{CJK} 4 \begin{CJK}{UTF8}{mj}维列向量\end{CJK}. \begin{CJK}{UTF8}{mj}如果\end{CJK} 4 \begin{CJK}{UTF8}{mj}阶行列式\end{CJK} $\left|\alpha_{1} \quad \alpha_{2} \quad \alpha_{3} \quad \beta_{1}\right|=m,\left|\alpha_{1} \quad \beta_{2} \quad \alpha_{3} \quad \alpha_{2}\right|=$ $n$, \begin{CJK}{UTF8}{mj}那么\end{CJK} 4 \begin{CJK}{UTF8}{mj}阶行列式\end{CJK} $\left|\begin{array}{llll}\alpha_{1} & \alpha_{3} & \alpha_{2} & \beta_{1}+\beta_{2}\end{array}\right|=$

  \item \begin{CJK}{UTF8}{mj}在线性空间\end{CJK} $R[x]_{n}$ \begin{CJK}{UTF8}{mj}中\end{CJK}, \begin{CJK}{UTF8}{mj}令线性变换\end{CJK} $\mathscr{D}(f(x))=f^{\prime}(x)$, \begin{CJK}{UTF8}{mj}则\end{CJK} $\mathscr{D}$ \begin{CJK}{UTF8}{mj}的特征值是\end{CJK} , $\mathscr{D}$ \begin{CJK}{UTF8}{mj}的核是\end{CJK}

\end{enumerate}
$\mathscr{D}$ \begin{CJK}{UTF8}{mj}的值域是\end{CJK}

$1 .\left|\begin{array}{cccc}-\varepsilon^{4} & -\varepsilon^{4} & -\varepsilon^{4} & -\varepsilon^{4} \\ -\varepsilon^{3} & -\varepsilon^{3} & -\varepsilon^{3} & 4 \varepsilon^{3} \\ -\varepsilon^{2} & -\varepsilon^{2} & 4 \varepsilon^{2} & -\varepsilon^{2} \\ -\varepsilon & 4 \varepsilon & -\varepsilon & -\varepsilon\end{array}\right|=\underline{\text {. }}$.

\begin{enumerate}
  \setcounter{enumi}{4}
  \item \begin{CJK}{UTF8}{mj}多项式\end{CJK} $x^{6}+6 x^{3}-6 x-2$ \begin{CJK}{UTF8}{mj}在有理数域上是\end{CJK} $\quad$ (\begin{CJK}{UTF8}{mj}可约的或不可约的\end{CJK}).

  \item \begin{CJK}{UTF8}{mj}全体正实数的集合\end{CJK} $\mathbb{R}^{+}$, \begin{CJK}{UTF8}{mj}加法和数量乘法分别定义为\end{CJK}: $a \oplus b=a b, k \circ a=a^{k}$. \begin{CJK}{UTF8}{mj}则此线性空间维数是\end{CJK} ,\begin{CJK}{UTF8}{mj}一组基是\end{CJK}

\end{enumerate}
\begin{CJK}{UTF8}{mj}二\end{CJK}、 ( 20 \begin{CJK}{UTF8}{mj}分\end{CJK}) \begin{CJK}{UTF8}{mj}确定\end{CJK} $a$ \begin{CJK}{UTF8}{mj}的值使方程组\end{CJK}
$$
\left\{\begin{array}{l}
2 x_{1}-x_{2}+x_{3}+x_{4}=1 \\
x_{1}+2 x_{2}-x_{3}+4 x_{4}=2 \\
x_{1}+7 x_{2}-4 x_{3}+11 x_{4}=a
\end{array}\right.
$$
\begin{CJK}{UTF8}{mj}有解\end{CJK}, \begin{CJK}{UTF8}{mj}并求出全部解\end{CJK}, \begin{CJK}{UTF8}{mj}要求用其导出组的基础解系来表示\end{CJK}.

\begin{CJK}{UTF8}{mj}三\end{CJK}、 $(20$ \begin{CJK}{UTF8}{mj}分\end{CJK} $)$ \begin{CJK}{UTF8}{mj}设二次型\end{CJK}
$$
f\left(x_{1}, x_{2}, x_{3}\right)=x_{1}^{2}+2 x_{2}^{2}+3 x_{3}^{2}-4 x_{1} x_{2}-4 x_{2} x_{3}
$$

\begin{enumerate}
  \item \begin{CJK}{UTF8}{mj}写出此二次型的矩阵\end{CJK}; (3 \begin{CJK}{UTF8}{mj}分\end{CJK})

  \item \begin{CJK}{UTF8}{mj}求一正交线性替换将此化二次型为标准形\end{CJK}; ( 13 \begin{CJK}{UTF8}{mj}分\end{CJK})

  \item \begin{CJK}{UTF8}{mj}求二次型的秩\end{CJK}; (2 \begin{CJK}{UTF8}{mj}分\end{CJK})

  \item \begin{CJK}{UTF8}{mj}判断此二次型是正定\end{CJK}, \begin{CJK}{UTF8}{mj}负定还是不定二次型\end{CJK}? ( 2 \begin{CJK}{UTF8}{mj}分\end{CJK})

\end{enumerate}
\begin{CJK}{UTF8}{mj}四\end{CJK}、 ( 20 \begin{CJK}{UTF8}{mj}分\end{CJK}) \begin{CJK}{UTF8}{mj}设\end{CJK} $A=\left(\begin{array}{cccc}1 & 0 & 0 & 0 \\ -1 & -1 & -1 & 0 \\ 1 & 1 & 1 & 0 \\ 2 & 2 & 2 & 0\end{array}\right)$, \begin{CJK}{UTF8}{mj}求\end{CJK} $A$ \begin{CJK}{UTF8}{mj}的若尔当标准形\end{CJK}.

\begin{CJK}{UTF8}{mj}五\end{CJK}、 (20 \begin{CJK}{UTF8}{mj}分\end{CJK}) \begin{CJK}{UTF8}{mj}设\end{CJK} $A$ \begin{CJK}{UTF8}{mj}是\end{CJK} $n$ \begin{CJK}{UTF8}{mj}阶方阵\end{CJK}

\begin{enumerate}
  \item \begin{CJK}{UTF8}{mj}如果存在\end{CJK} $n$ \begin{CJK}{UTF8}{mj}阶方阵\end{CJK} $B$ \begin{CJK}{UTF8}{mj}使得\end{CJK} $A B=0$, \begin{CJK}{UTF8}{mj}证明\end{CJK}: $r(A)+r(B) \leqslant n ;(10$ \begin{CJK}{UTF8}{mj}分\end{CJK})

  \item \begin{CJK}{UTF8}{mj}如果对于自然数\end{CJK} $k$ \begin{CJK}{UTF8}{mj}满足\end{CJK} $r(A) \leqslant k \leqslant n$, \begin{CJK}{UTF8}{mj}证明\end{CJK}: \begin{CJK}{UTF8}{mj}存在\end{CJK} $n$ \begin{CJK}{UTF8}{mj}阶方阵\end{CJK} $C$ \begin{CJK}{UTF8}{mj}使得\end{CJK} $r(A)+r(C)=k$. ( 10 \begin{CJK}{UTF8}{mj}分\end{CJK})

\end{enumerate}
\begin{CJK}{UTF8}{mj}六\end{CJK}、( 20 \begin{CJK}{UTF8}{mj}分\end{CJK}) \begin{CJK}{UTF8}{mj}设向量\end{CJK} $\beta$ \begin{CJK}{UTF8}{mj}可以经向量组\end{CJK} $\alpha_{1}, \alpha_{2}, \cdots, \alpha_{m}$ \begin{CJK}{UTF8}{mj}线性表出\end{CJK}. \begin{CJK}{UTF8}{mj}证明\end{CJK}: $\beta$ \begin{CJK}{UTF8}{mj}可以经向量组\end{CJK} $\alpha_{1}, \alpha_{2}, \cdots, \alpha_{m}$ \begin{CJK}{UTF8}{mj}线性表\end{CJK} \begin{CJK}{UTF8}{mj}出的表示法唯一的充分必要条件是向量组\end{CJK} $\alpha_{1}, \alpha_{2}, \cdots, \alpha_{m}$ \begin{CJK}{UTF8}{mj}线性无关\end{CJK}. \begin{CJK}{UTF8}{mj}七\end{CJK}、 ( 10 \begin{CJK}{UTF8}{mj}分\end{CJK}) \begin{CJK}{UTF8}{mj}设不可约多项式\end{CJK} $p(x)$ \begin{CJK}{UTF8}{mj}是\end{CJK} $f(x)$ \begin{CJK}{UTF8}{mj}的\end{CJK} $k$ \begin{CJK}{UTF8}{mj}重因式\end{CJK} (\begin{CJK}{UTF8}{mj}其中自然数\end{CJK} $k \geqslant 2$ ), \begin{CJK}{UTF8}{mj}证明\end{CJK}: $p(x)$ \begin{CJK}{UTF8}{mj}是其微商\end{CJK} $f^{\prime}(x)$ \begin{CJK}{UTF8}{mj}的\end{CJK} $k-1$ \begin{CJK}{UTF8}{mj}重因式\end{CJK}.

\begin{CJK}{UTF8}{mj}八\end{CJK}、 (10 \begin{CJK}{UTF8}{mj}分\end{CJK}) \begin{CJK}{UTF8}{mj}证明\end{CJK}: \begin{CJK}{UTF8}{mj}不存在正交矩阵\end{CJK} $A, B$, \begin{CJK}{UTF8}{mj}使得\end{CJK} $A^{2}=A B+B^{2}$.

\section{5. 云南大学 2013 年研究生入学考试试题高等代数}
\begin{CJK}{UTF8}{mj}李扬\end{CJK}

\begin{CJK}{UTF8}{mj}微信公众号\end{CJK}: sxkyliyang

\begin{CJK}{UTF8}{mj}一\end{CJK}、\begin{CJK}{UTF8}{mj}填空题\end{CJK} (\begin{CJK}{UTF8}{mj}共\end{CJK} 6 \begin{CJK}{UTF8}{mj}题\end{CJK}, \begin{CJK}{UTF8}{mj}每题\end{CJK} 5 \begin{CJK}{UTF8}{mj}分\end{CJK}, \begin{CJK}{UTF8}{mj}共\end{CJK} 30 \begin{CJK}{UTF8}{mj}分\end{CJK})

\begin{enumerate}
  \item $\left|\begin{array}{ccc}k & 3 & 4 \\ -1 & k & 0 \\ 0 & k & 1\end{array}\right|=$

  \item \begin{CJK}{UTF8}{mj}设\end{CJK} $n$ \begin{CJK}{UTF8}{mj}阶矩阵\end{CJK} $A=\left(\begin{array}{cccc}1 & 2 & 3 & 4 \\ 0 & 2 & 3 & 4 \\ 0 & 0 & 3 & 4 \\ 0 & 0 & 0 & 4\end{array}\right), A^{*}$ \begin{CJK}{UTF8}{mj}为\end{CJK} $A$ \begin{CJK}{UTF8}{mj}的伴随矩阵\end{CJK}, \begin{CJK}{UTF8}{mj}则\end{CJK} $\left(A^{*}\right)^{-1}=$

  \item $\left|\begin{array}{cccc}x-1 & 1 & 0 & 0 \\ 0 & x & x-2 & -1 \\ 0 & x & x-2 & 2 \\ 2 & 0 & 0 & x-3\end{array}\right|=0$ \begin{CJK}{UTF8}{mj}的解为\end{CJK}

  \item \begin{CJK}{UTF8}{mj}设\end{CJK} $A$ \begin{CJK}{UTF8}{mj}是\end{CJK} 3 \begin{CJK}{UTF8}{mj}阶实对称矩阵\end{CJK}, $A$ \begin{CJK}{UTF8}{mj}的特征值\end{CJK} $\lambda_{1}=\lambda_{2}=\lambda_{3}=-2$, \begin{CJK}{UTF8}{mj}则\end{CJK} $A^{2013}=$

  \item \begin{CJK}{UTF8}{mj}二次型\end{CJK} $f\left(x_{1}, x_{2}, x_{3}\right)=\left(x_{1}+4 x_{2}+2 x_{3}\right)^{2}$ \begin{CJK}{UTF8}{mj}的矩阵\end{CJK} $=$

  \item \begin{CJK}{UTF8}{mj}向量组\end{CJK} $\alpha_{1}=(6,4,1,-1,2), \alpha_{2}=(1,0,2,3,-4), \alpha_{3}=(7,1,0,-1,3), \alpha_{4}=(1,4,-9,-16,22)$ \begin{CJK}{UTF8}{mj}的一个极\end{CJK} \begin{CJK}{UTF8}{mj}大线性无关组为\end{CJK}

\end{enumerate}
\begin{CJK}{UTF8}{mj}二\end{CJK}、 (10 \begin{CJK}{UTF8}{mj}分\end{CJK}) \begin{CJK}{UTF8}{mj}设多项式\end{CJK} $f(x)=x^{n}+a^{n}(n>1, a \neq 0)$ \begin{CJK}{UTF8}{mj}满足\end{CJK} $x+a \mid f(x)$. \begin{CJK}{UTF8}{mj}求\end{CJK} $n$ \begin{CJK}{UTF8}{mj}满足的条件\end{CJK}.

\begin{CJK}{UTF8}{mj}三\end{CJK}、 ( 20 \begin{CJK}{UTF8}{mj}分\end{CJK}) \begin{CJK}{UTF8}{mj}将二次型\end{CJK} $f\left(x_{1}, x_{2}, x_{3}\right)=2 x_{1} x_{2}+2 x_{1} x_{3}+2 x_{2} x_{3}$ \begin{CJK}{UTF8}{mj}通过正交变换化为标准形\end{CJK}, \begin{CJK}{UTF8}{mj}并求所用的正交变换矩\end{CJK} \begin{CJK}{UTF8}{mj}阵\end{CJK}.

\begin{CJK}{UTF8}{mj}四\end{CJK}、(15 \begin{CJK}{UTF8}{mj}分\end{CJK}) \begin{CJK}{UTF8}{mj}解线性方程组\end{CJK}
$$
\left\{\begin{array}{l}
3 a x_{1}+(2 a+1) x_{2}+(a+1) x_{3}=a \\
(2 a-1) x_{1}+(2 a-1) x_{2}+(a-2) x_{3}=a+1 \\
(4 a-1) x_{1}+3 a x_{2}+2 a x_{3}=1
\end{array}\right.
$$
\begin{CJK}{UTF8}{mj}五\end{CJK}、(15 \begin{CJK}{UTF8}{mj}分\end{CJK}) \begin{CJK}{UTF8}{mj}设\end{CJK} $\mathscr{A}$ \begin{CJK}{UTF8}{mj}是\end{CJK} $n$ \begin{CJK}{UTF8}{mj}维线性空间的一个线性变换\end{CJK}, \begin{CJK}{UTF8}{mj}如果\end{CJK} $\lambda_{1}, \lambda_{2}, \cdots, \lambda_{k}(1 \leqslant k \leqslant n)$ \begin{CJK}{UTF8}{mj}是\end{CJK} $A$ \begin{CJK}{UTF8}{mj}的不同的特征根\end{CJK}, $\xi_{1}, \xi_{2}, \cdots, \xi_{k}$ \begin{CJK}{UTF8}{mj}分别是\end{CJK} $A$ \begin{CJK}{UTF8}{mj}的属于\end{CJK} $\lambda_{1}, \lambda_{2}, \cdots, \lambda_{k}$ \begin{CJK}{UTF8}{mj}的特征向量\end{CJK}, \begin{CJK}{UTF8}{mj}问\end{CJK}: $\xi_{1}, \xi_{2}, \cdots, \xi_{k}$ \begin{CJK}{UTF8}{mj}是否线性相关\end{CJK}? \begin{CJK}{UTF8}{mj}请证明你的\end{CJK} \begin{CJK}{UTF8}{mj}结论\end{CJK}.

\begin{CJK}{UTF8}{mj}六\end{CJK}、(\begin{CJK}{UTF8}{mj}共\end{CJK} 2 \begin{CJK}{UTF8}{mj}题\end{CJK}, \begin{CJK}{UTF8}{mj}每题\end{CJK} 10 \begin{CJK}{UTF8}{mj}分\end{CJK}, \begin{CJK}{UTF8}{mj}共\end{CJK} 20 \begin{CJK}{UTF8}{mj}分\end{CJK}) \begin{CJK}{UTF8}{mj}设\end{CJK} $V$ \begin{CJK}{UTF8}{mj}是\end{CJK} $n$ \begin{CJK}{UTF8}{mj}维欧式空间\end{CJK}, $0 \neq \alpha \in V, V_{1}=\{x \in V \mid(x, \alpha)=0\}$.

\begin{enumerate}
  \item \begin{CJK}{UTF8}{mj}证明\end{CJK}: $V_{1}$ \begin{CJK}{UTF8}{mj}是\end{CJK} $V$ \begin{CJK}{UTF8}{mj}的子空间\end{CJK};

  \item \begin{CJK}{UTF8}{mj}求\end{CJK} $V_{1}$ \begin{CJK}{UTF8}{mj}的维数\end{CJK} $\operatorname{dim} V_{1}$.

\end{enumerate}
\begin{CJK}{UTF8}{mj}七\end{CJK}、 ( 20 \begin{CJK}{UTF8}{mj}分\end{CJK}) \begin{CJK}{UTF8}{mj}设\end{CJK} $A=\left(\begin{array}{cccc}1 & -3 & 0 & 3 \\ -2 & -6 & 0 & 13 \\ 0 & -3 & 1 & 3 \\ -1 & -4 & 0 & 8\end{array}\right)$, \begin{CJK}{UTF8}{mj}求\end{CJK} $A$ \begin{CJK}{UTF8}{mj}的行列式因子\end{CJK}, \begin{CJK}{UTF8}{mj}不变因子\end{CJK}, \begin{CJK}{UTF8}{mj}初等因子及\end{CJK} Jordan \begin{CJK}{UTF8}{mj}标准形\end{CJK}.

\begin{CJK}{UTF8}{mj}八\end{CJK}、 (10 \begin{CJK}{UTF8}{mj}分\end{CJK}) \begin{CJK}{UTF8}{mj}设\end{CJK} $A$ \begin{CJK}{UTF8}{mj}是\end{CJK} $n$ \begin{CJK}{UTF8}{mj}阶矩阵\end{CJK}, \begin{CJK}{UTF8}{mj}且\end{CJK} $A^{7}=0$. \begin{CJK}{UTF8}{mj}证明\end{CJK}: $E+2 A$ \begin{CJK}{UTF8}{mj}可逆并求其逆\end{CJK}. \begin{CJK}{UTF8}{mj}九\end{CJK}、 $(10$ \begin{CJK}{UTF8}{mj}分\end{CJK}) \begin{CJK}{UTF8}{mj}计算\end{CJK} $n$ \begin{CJK}{UTF8}{mj}阶行列式\end{CJK} $(n \geqslant 1)$
$$
D_{n}=\left|\begin{array}{cccc}
\sin \left(\alpha_{1}+\alpha_{1}\right) & \sin \left(\alpha_{1}+\alpha_{2}\right) & \cdots & \sin \left(\alpha_{1}+\alpha_{n}\right) \\
\sin \left(\alpha_{2}+\alpha_{1}\right) & \sin \left(\alpha_{2}+\alpha_{2}\right) & \cdots & \sin \left(\alpha_{2}+\alpha_{n}\right) \\
\vdots & \vdots & & \vdots \\
\sin \left(\alpha_{n}+\alpha_{1}\right) & \sin \left(\alpha_{n}+\alpha_{2}\right) & \cdots & \sin \left(\alpha_{n}+\alpha_{n}\right)
\end{array}\right|
$$

\section{6. 云南大学 2014 年研究生入学考试试题高等代数}
\begin{CJK}{UTF8}{mj}李扬\end{CJK}

\begin{CJK}{UTF8}{mj}微信公众号\end{CJK}: sxkyliyang

\begin{CJK}{UTF8}{mj}一\end{CJK}、\begin{CJK}{UTF8}{mj}填空题\end{CJK} (\begin{CJK}{UTF8}{mj}共\end{CJK} 5 \begin{CJK}{UTF8}{mj}题\end{CJK}, \begin{CJK}{UTF8}{mj}每题\end{CJK} 4 \begin{CJK}{UTF8}{mj}分\end{CJK}, \begin{CJK}{UTF8}{mj}共\end{CJK} 20 \begin{CJK}{UTF8}{mj}分\end{CJK})

\begin{enumerate}
  \item \begin{CJK}{UTF8}{mj}设\end{CJK} $A=\left(\begin{array}{cc}1 & -1 \\ 0 & 1\end{array}\right), f(x)=2 x^{2}-x+4$, \begin{CJK}{UTF8}{mj}则\end{CJK} $f(A)=$

  \item \begin{CJK}{UTF8}{mj}方程\end{CJK} $\left|\begin{array}{cccc}1 & 1 & 1 & 1 \\ 1 & a & a^{2} & a^{3} \\ 1 & b & b^{2} & b^{3} \\ x & x^{2} & x^{3} & x^{4}\end{array}\right|=0$ \begin{CJK}{UTF8}{mj}的所有根为\end{CJK}

  \item \begin{CJK}{UTF8}{mj}把复数域看作复数域上的线性空间的维数是\end{CJK} ,\begin{CJK}{UTF8}{mj}一组基是\end{CJK} ; \begin{CJK}{UTF8}{mj}把复数域看作实数域上的\end{CJK} \begin{CJK}{UTF8}{mj}线性空间的维数是\end{CJK} ,\begin{CJK}{UTF8}{mj}一组基是\end{CJK}

  \item \begin{CJK}{UTF8}{mj}设\end{CJK} $\alpha_{1}=(-1,1), \alpha_{2}=(-1,-1)$ \begin{CJK}{UTF8}{mj}为欧式空间\end{CJK} $\mathbb{R}^{2}$ \begin{CJK}{UTF8}{mj}的一组基\end{CJK}(\begin{CJK}{UTF8}{mj}内积为通常意义下的内积\end{CJK}), \begin{CJK}{UTF8}{mj}则这组基的度量\end{CJK} \begin{CJK}{UTF8}{mj}矩阵为\end{CJK}

  \item \begin{CJK}{UTF8}{mj}设\end{CJK} $P[x]_{3}=\{f(x) \in P[x] \mid \partial(f(x))<3$ \begin{CJK}{UTF8}{mj}或\end{CJK} $f(x)=0\}$ \begin{CJK}{UTF8}{mj}是对多项式的普通加法和数乘构成的线性空间\end{CJK}, $\mathscr{D}(f(x))=f^{\prime}(x)$ \begin{CJK}{UTF8}{mj}是求微商的线性变换\end{CJK}, \begin{CJK}{UTF8}{mj}则\end{CJK} $\mathscr{D}$ \begin{CJK}{UTF8}{mj}在基\end{CJK} $1, x, x^{2}$ \begin{CJK}{UTF8}{mj}下的矩阵是\end{CJK}

\end{enumerate}
\begin{CJK}{UTF8}{mj}二\end{CJK}、 $(20$ \begin{CJK}{UTF8}{mj}分\end{CJK}) \begin{CJK}{UTF8}{mj}当\end{CJK} $s, t$ \begin{CJK}{UTF8}{mj}取什么值时\end{CJK}, \begin{CJK}{UTF8}{mj}线性方程组\end{CJK}
$$
\left\{\begin{array}{l}
x_{1}+x_{2}+x_{3}+x_{4}+x_{5}=1 \\
3 x_{1}+2 x_{2}+x_{3}+x_{4}-3 x_{5}=s \\
x_{1}+2 x_{3}+2 x_{4}+6 x_{5}=3 \\
5 x_{1}+4 x_{2}+3 x_{3}+3 x_{4}-x_{5}=t
\end{array}\right.
$$
\begin{CJK}{UTF8}{mj}有解\end{CJK}? \begin{CJK}{UTF8}{mj}在有解的情形下\end{CJK}, \begin{CJK}{UTF8}{mj}用其导出组的基础解系表示其全部解\end{CJK}.

\begin{CJK}{UTF8}{mj}三\end{CJK}、 $(20$ \begin{CJK}{UTF8}{mj}分\end{CJK} $)$ \begin{CJK}{UTF8}{mj}设二次型\end{CJK}
$$
f\left(x_{1}, x_{2}, x_{3}\right)=2 x_{1}^{2}+2 x_{2}^{2}+2 x_{3}^{2}-6 x_{1} x_{2}-6 x_{1} x_{3}-6 x_{2} x_{3}
$$

\begin{enumerate}
  \item \begin{CJK}{UTF8}{mj}写出\end{CJK} $f\left(x_{1}, x_{2}, x_{3}\right)$ \begin{CJK}{UTF8}{mj}的矩阵\end{CJK};

  \item \begin{CJK}{UTF8}{mj}求正交线性替换化\end{CJK} $f\left(x_{1}, x_{2}, x_{3}\right)$ \begin{CJK}{UTF8}{mj}为标准形\end{CJK}, \begin{CJK}{UTF8}{mj}并指出其秩\end{CJK};

  \item \begin{CJK}{UTF8}{mj}判断\end{CJK} $f\left(x_{1}, x_{2}, x_{3}\right)$ \begin{CJK}{UTF8}{mj}的正定性\end{CJK}.

\end{enumerate}
\begin{CJK}{UTF8}{mj}四\end{CJK}、 ( 20 \begin{CJK}{UTF8}{mj}分\end{CJK}) \begin{CJK}{UTF8}{mj}在复数域上\end{CJK}, \begin{CJK}{UTF8}{mj}求矩阵\end{CJK}
$$
\left(\begin{array}{cccc}
1 & -3 & 0 & 3 \\
-2 & 6 & 0 & 13 \\
0 & -3 & 1 & 3 \\
-1 & -4 & 0 & 8
\end{array}\right)
$$
\begin{CJK}{UTF8}{mj}的若尔当标准形\end{CJK}.

\begin{CJK}{UTF8}{mj}五\end{CJK}、(20 \begin{CJK}{UTF8}{mj}分\end{CJK}) \begin{CJK}{UTF8}{mj}设\end{CJK} $\mathscr{A}$ \begin{CJK}{UTF8}{mj}是线性空间\end{CJK} $V$ \begin{CJK}{UTF8}{mj}的线性变换\end{CJK}, $\alpha$ \begin{CJK}{UTF8}{mj}是\end{CJK} $V$ \begin{CJK}{UTF8}{mj}中的非零向量\end{CJK}, $m$ \begin{CJK}{UTF8}{mj}是大于\end{CJK} 1 \begin{CJK}{UTF8}{mj}的正整数\end{CJK}, \begin{CJK}{UTF8}{mj}如果向量组\end{CJK} $\alpha, \mathscr{A}(\alpha), \cdots, \mathscr{A}^{m-1}(\alpha)$ \begin{CJK}{UTF8}{mj}线性无关\end{CJK}, \begin{CJK}{UTF8}{mj}而向量组\end{CJK} $\alpha, \mathscr{A}(\alpha), \cdots, \mathscr{A}^{m-1}(\alpha), \mathscr{A}^{m}(\alpha)$ \begin{CJK}{UTF8}{mj}线性相关\end{CJK}.

\begin{enumerate}
  \item \begin{CJK}{UTF8}{mj}证明\end{CJK}: \begin{CJK}{UTF8}{mj}由\end{CJK} $\alpha, \mathscr{A}(\alpha), \cdots, \mathscr{A}^{m-1}(\alpha)$ \begin{CJK}{UTF8}{mj}生成的子空间\end{CJK} $W$ \begin{CJK}{UTF8}{mj}是\end{CJK} $\mathscr{A}$ \begin{CJK}{UTF8}{mj}的不变子空间\end{CJK};

  \item \begin{CJK}{UTF8}{mj}求\end{CJK} $W$ \begin{CJK}{UTF8}{mj}的一组基\end{CJK}, \begin{CJK}{UTF8}{mj}并求\end{CJK} $\mathscr{A}$ \begin{CJK}{UTF8}{mj}在\end{CJK} $W$ \begin{CJK}{UTF8}{mj}的这组基下的矩阵\end{CJK}. \begin{CJK}{UTF8}{mj}六\end{CJK}、 ( 20 \begin{CJK}{UTF8}{mj}分\end{CJK}) \begin{CJK}{UTF8}{mj}设\end{CJK} $A$ \begin{CJK}{UTF8}{mj}是\end{CJK} $m \times n$ \begin{CJK}{UTF8}{mj}实矩阵\end{CJK}, $A^{\prime}$ \begin{CJK}{UTF8}{mj}是\end{CJK} $A$ \begin{CJK}{UTF8}{mj}的转置矩阵\end{CJK}, $H$ \begin{CJK}{UTF8}{mj}是\end{CJK} $m \times m$ \begin{CJK}{UTF8}{mj}实正定矩阵\end{CJK}, $X$ \begin{CJK}{UTF8}{mj}是\end{CJK} $n$ \begin{CJK}{UTF8}{mj}维列向量\end{CJK}, \begin{CJK}{UTF8}{mj}证明\end{CJK}:

  \item \begin{CJK}{UTF8}{mj}齐次线性方程组\end{CJK} $A X=0$ \begin{CJK}{UTF8}{mj}与\end{CJK} $A^{\prime} H A X=0$ \begin{CJK}{UTF8}{mj}同解\end{CJK};

  \item \begin{CJK}{UTF8}{mj}若\end{CJK} $n \times n$ \begin{CJK}{UTF8}{mj}矩阵\end{CJK} $B$ \begin{CJK}{UTF8}{mj}满足\end{CJK} $A^{\prime} H A B=A^{\prime} H A$, \begin{CJK}{UTF8}{mj}则必有\end{CJK} $A B=A$.

\end{enumerate}
\begin{CJK}{UTF8}{mj}七\end{CJK}、 $\left(20\right.$ \begin{CJK}{UTF8}{mj}分\end{CJK}) \begin{CJK}{UTF8}{mj}设\end{CJK} $n$ \begin{CJK}{UTF8}{mj}次多项式\end{CJK} $f(x)=a_{0} x^{n}+a_{1} x^{n-1}+\cdots+a_{n-1} x+a_{n}=0,\left(a_{0} \neq 0\right)$ \begin{CJK}{UTF8}{mj}是方阵\end{CJK} $A$ \begin{CJK}{UTF8}{mj}的最小多项式\end{CJK}, \begin{CJK}{UTF8}{mj}证\end{CJK} \begin{CJK}{UTF8}{mj}明\end{CJK}: $A$ \begin{CJK}{UTF8}{mj}可逆的充分必要条件是\end{CJK} $a_{n} \neq 0$.

\begin{CJK}{UTF8}{mj}八\end{CJK}、 ( 20 \begin{CJK}{UTF8}{mj}分\end{CJK}) \begin{CJK}{UTF8}{mj}设\end{CJK} $A$ \begin{CJK}{UTF8}{mj}为\end{CJK} $n$ \begin{CJK}{UTF8}{mj}阶实对称矩阵\end{CJK}, \begin{CJK}{UTF8}{mj}且\end{CJK} $A$ \begin{CJK}{UTF8}{mj}的主对角线上的元素之和等于正整数\end{CJK} $N$, \begin{CJK}{UTF8}{mj}求\end{CJK} $|E+2 A|$ \begin{CJK}{UTF8}{mj}的最大值\end{CJK}.

\section{7. 云南大学 2015 年研究生入学考试试题高等代数}
\begin{CJK}{UTF8}{mj}李扬\end{CJK}

\begin{CJK}{UTF8}{mj}微信公众号\end{CJK}: sxkyliyang

\begin{CJK}{UTF8}{mj}一\end{CJK}、\begin{CJK}{UTF8}{mj}填空题\end{CJK} (\begin{CJK}{UTF8}{mj}共\end{CJK} 6 \begin{CJK}{UTF8}{mj}题\end{CJK}, \begin{CJK}{UTF8}{mj}每题\end{CJK} 5 \begin{CJK}{UTF8}{mj}分\end{CJK}, \begin{CJK}{UTF8}{mj}共\end{CJK} 30 \begin{CJK}{UTF8}{mj}分\end{CJK})

\begin{enumerate}
  \item \begin{CJK}{UTF8}{mj}设\end{CJK} $A=\left(\begin{array}{ccc}738 & 427 & 327 \\ 3024 & 543 & 443 \\ -972 & 721 & 621\end{array}\right)$, \begin{CJK}{UTF8}{mj}则\end{CJK} $|A|=$

  \item \begin{CJK}{UTF8}{mj}已知\end{CJK} $2,4,6, \cdots, 2 n$ \begin{CJK}{UTF8}{mj}是\end{CJK} $n$ \begin{CJK}{UTF8}{mj}阶矩阵\end{CJK} $A$ \begin{CJK}{UTF8}{mj}的\end{CJK} $n$ \begin{CJK}{UTF8}{mj}个特征值\end{CJK}, \begin{CJK}{UTF8}{mj}则\end{CJK} $|A-3 E|=$

  \item \begin{CJK}{UTF8}{mj}二次型\end{CJK} $f(x, y, z)=4 x^{2}+4 y^{2}+4 z^{2}+2 x y+3 x z$ \begin{CJK}{UTF8}{mj}的矩阵是\end{CJK}

  \item \begin{CJK}{UTF8}{mj}设\end{CJK} $a_{1}, a_{2}, a_{3}$ \begin{CJK}{UTF8}{mj}是一组线性无关向量\end{CJK}, \begin{CJK}{UTF8}{mj}则\end{CJK} $a_{1}-a_{2}, a_{2}-a_{3}, a_{3}-a_{1}$ \begin{CJK}{UTF8}{mj}线性\end{CJK}

\end{enumerate}
5 . \begin{CJK}{UTF8}{mj}设矩阵\end{CJK} $A=\left(\begin{array}{cc}2 & 1 \\ -1 & 2\end{array}\right)$, \begin{CJK}{UTF8}{mj}且\end{CJK} $B A=B+2 E, E$ \begin{CJK}{UTF8}{mj}是单位矩阵\end{CJK}, \begin{CJK}{UTF8}{mj}则\end{CJK} $|B|=$

\begin{enumerate}
  \setcounter{enumi}{6}
  \item \begin{CJK}{UTF8}{mj}设复系数多项式\end{CJK} $f(x)$ \begin{CJK}{UTF8}{mj}没有重因式\end{CJK}, \begin{CJK}{UTF8}{mj}则\end{CJK} $\left(f(x)+f^{\prime}(x), f(x)\right)=$
\end{enumerate}
\begin{CJK}{UTF8}{mj}二\end{CJK}、 $\left(10\right.$ \begin{CJK}{UTF8}{mj}分\end{CJK}) \begin{CJK}{UTF8}{mj}多项式\end{CJK} $f(x)$ \begin{CJK}{UTF8}{mj}没有重因式\end{CJK}, \begin{CJK}{UTF8}{mj}如果\end{CJK} $f^{\prime}(x) \mid f(x)$, \begin{CJK}{UTF8}{mj}则\end{CJK} $f(x)$ \begin{CJK}{UTF8}{mj}有\end{CJK} $n$ \begin{CJK}{UTF8}{mj}重根\end{CJK}, \begin{CJK}{UTF8}{mj}其中\end{CJK} $n=\partial(f(x))$.

\begin{CJK}{UTF8}{mj}三\end{CJK}、 $(10$ \begin{CJK}{UTF8}{mj}分\end{CJK}) \begin{CJK}{UTF8}{mj}设\end{CJK} $A, B$ \begin{CJK}{UTF8}{mj}分别是\end{CJK} $s \times n$ \begin{CJK}{UTF8}{mj}与\end{CJK} $n \times m$ \begin{CJK}{UTF8}{mj}矩阵\end{CJK}, \begin{CJK}{UTF8}{mj}则\end{CJK} $r(A)+r(B)-n \leqslant r(A B)$.

\begin{CJK}{UTF8}{mj}四\end{CJK}、( 20 \begin{CJK}{UTF8}{mj}分\end{CJK}) \begin{CJK}{UTF8}{mj}设向量组\end{CJK} $\alpha_{1}, \alpha_{2}, \cdots, \alpha_{m}$ \begin{CJK}{UTF8}{mj}线性无关\end{CJK}, \begin{CJK}{UTF8}{mj}向量\end{CJK} $\beta_{1}$ \begin{CJK}{UTF8}{mj}可由它线性表出\end{CJK}, \begin{CJK}{UTF8}{mj}向量\end{CJK} $\beta_{2}$ \begin{CJK}{UTF8}{mj}不能由它线性表出\end{CJK}. \begin{CJK}{UTF8}{mj}证明\end{CJK}: $\alpha_{1}, \alpha_{2}, \cdots, \alpha_{m}, \beta_{1}+\beta_{2}$ \begin{CJK}{UTF8}{mj}线性无关\end{CJK}.

\begin{CJK}{UTF8}{mj}五\end{CJK}、 (20 \begin{CJK}{UTF8}{mj}分\end{CJK}) \begin{CJK}{UTF8}{mj}设齐次线性方程组\end{CJK}
$$
\left\{\begin{array}{l}
a x_{1}+b x_{2}+b x_{3}+\cdots+b x_{n}=0 \\
b x_{1}+a x_{2}+b x_{3}+\cdots+b x_{n}=0 \\
\cdots \cdots \\
b x_{1}+b x_{2}+b x_{3}+\cdots+a x_{n}=0
\end{array}\right.
$$
\begin{CJK}{UTF8}{mj}其中\end{CJK} $a \neq 0, b \neq 0$ ? \begin{CJK}{UTF8}{mj}试讨论\end{CJK} $a, b$ \begin{CJK}{UTF8}{mj}为何值时\end{CJK}, \begin{CJK}{UTF8}{mj}方程组仅有零解\end{CJK}, \begin{CJK}{UTF8}{mj}有无穷多解\end{CJK}? \begin{CJK}{UTF8}{mj}在解为无穷时\end{CJK}, \begin{CJK}{UTF8}{mj}求出全部解并用基\end{CJK} \begin{CJK}{UTF8}{mj}础解系表示出全部解\end{CJK}.

\begin{CJK}{UTF8}{mj}六\end{CJK}、 ( 20 \begin{CJK}{UTF8}{mj}分\end{CJK}) \begin{CJK}{UTF8}{mj}已知二次型\end{CJK} $f\left(x_{1}, x_{2}, x_{3}\right)=(1-a) x_{1}^{2}+(1-a) x_{2}^{2}+2 x_{3}^{2}+2(1+a) x_{1} x_{2}$ \begin{CJK}{UTF8}{mj}的秩为\end{CJK} 2 .

\begin{enumerate}
  \item \begin{CJK}{UTF8}{mj}求\end{CJK} $a$ \begin{CJK}{UTF8}{mj}的值\end{CJK};

  \item \begin{CJK}{UTF8}{mj}求正交变换\end{CJK} $X=Q Y$, \begin{CJK}{UTF8}{mj}把\end{CJK} $f\left(x_{1}, x_{2}, x_{3}\right)$ \begin{CJK}{UTF8}{mj}化为标准形\end{CJK};

  \item \begin{CJK}{UTF8}{mj}求方程\end{CJK} $f\left(x_{1}, x_{2}, x_{3}\right)=0$ \begin{CJK}{UTF8}{mj}的解\end{CJK}.

\end{enumerate}
\begin{CJK}{UTF8}{mj}七\end{CJK}、 (20 \begin{CJK}{UTF8}{mj}分\end{CJK}) \begin{CJK}{UTF8}{mj}求证\end{CJK}: \begin{CJK}{UTF8}{mj}不存在正交矩阵\end{CJK} $A, B$, \begin{CJK}{UTF8}{mj}使得\end{CJK} $A^{2}=A B+B^{2}$.

\begin{CJK}{UTF8}{mj}八\end{CJK}、 (20 \begin{CJK}{UTF8}{mj}分\end{CJK}) \begin{CJK}{UTF8}{mj}化\end{CJK} $A=\left(\begin{array}{cccc}3 & -4 & 0 & 0 \\ 4 & -5 & 0 & 0 \\ 0 & 0 & 3 & -2 \\ 0 & 0 & 2 & -1\end{array}\right)$ \begin{CJK}{UTF8}{mj}为\end{CJK} Jordan \begin{CJK}{UTF8}{mj}标准形\end{CJK}.

\begin{CJK}{UTF8}{mj}九\end{CJK}、 (20 \begin{CJK}{UTF8}{mj}分\end{CJK}) \begin{CJK}{UTF8}{mj}对称矩阵\end{CJK} $A=\left(\begin{array}{ccc}2 & -1 & -1 \\ -1 & 2 & -1 \\ -1 & -1 & 2\end{array}\right)$, \begin{CJK}{UTF8}{mj}求一正交矩阵\end{CJK} $T$, \begin{CJK}{UTF8}{mj}使得\end{CJK} $T^{-1} A T$ \begin{CJK}{UTF8}{mj}为对角形\end{CJK}.

\section{8. 云南大学 2016 年研究生入学考试试题高等代数}
\begin{CJK}{UTF8}{mj}李扬\end{CJK}

\begin{CJK}{UTF8}{mj}微信公众号\end{CJK}: sxkyliyang

\begin{CJK}{UTF8}{mj}一\end{CJK}、\begin{CJK}{UTF8}{mj}填空题\end{CJK}

\begin{enumerate}
  \item $f(x)$ \begin{CJK}{UTF8}{mj}除以\end{CJK} $a x-b(a \neq 0)$ \begin{CJK}{UTF8}{mj}的余式为\end{CJK}

  \item $\left|\begin{array}{ccc}2 & -3 & 1 \\ 1 & 1 & 1 \\ 3 & 2 & -2\end{array}\right|=$

  \item $A$ \begin{CJK}{UTF8}{mj}是\end{CJK} $4 \times 3$ \begin{CJK}{UTF8}{mj}矩阵\end{CJK}, $r(A)=2, B=\left(\begin{array}{ccc}1 & 0 & -1 \\ 0 & 2 & 0 \\ 2 & 0 & 3\end{array}\right), r(A B)=$

  \item \begin{CJK}{UTF8}{mj}欧式空间\end{CJK} $\mathbb{R}^{n}$ \begin{CJK}{UTF8}{mj}中\end{CJK}, \begin{CJK}{UTF8}{mj}满足\end{CJK} $\left\{\left(x_{1}, x_{2}, \cdots, x_{n}\right) \mid \sum_{i=1}^{n} x_{i}=0\right\}$ \begin{CJK}{UTF8}{mj}欧式空间的维数是\end{CJK}

  \item $f\left(x_{1}, x_{2}, x_{3}\right)=2 x_{1}^{2}+x_{2}^{2}+4 x_{3}^{2}+2 x_{1} x_{2}+t x_{2} x_{3}$ \begin{CJK}{UTF8}{mj}为正定二次型\end{CJK}, \begin{CJK}{UTF8}{mj}则\end{CJK} $t$ \begin{CJK}{UTF8}{mj}的取值范围是\end{CJK}

\end{enumerate}
\begin{CJK}{UTF8}{mj}二\end{CJK}、\begin{CJK}{UTF8}{mj}已知\end{CJK} $(x-1)^{2} \mid\left(a x^{4}+b x^{2}+1\right)$, \begin{CJK}{UTF8}{mj}求\end{CJK} $a, b$.

\begin{CJK}{UTF8}{mj}三\end{CJK}、\begin{CJK}{UTF8}{mj}已知行列式\end{CJK}
$$
D=\left|\begin{array}{cccc}
1 & 1 & 1 & 1 \\
2 & 1 & 1 & -3 \\
1 & 2 & 2 & 5 \\
4 & 3 & 2 & 1
\end{array}\right|
$$
\begin{CJK}{UTF8}{mj}求\end{CJK} $D$ \begin{CJK}{UTF8}{mj}的第\end{CJK} 4 \begin{CJK}{UTF8}{mj}列的代数余子式之和\end{CJK}.

\begin{CJK}{UTF8}{mj}四\end{CJK}、\begin{CJK}{UTF8}{mj}设矩阵\end{CJK} $A=\left(\begin{array}{ccc}1 & -2 & 0 \\ 1 & 2 & 0 \\ 0 & 0 & 2\end{array}\right)$, \begin{CJK}{UTF8}{mj}且\end{CJK} $B$ \begin{CJK}{UTF8}{mj}满足\end{CJK} $A^{2}+3 B=A B+9 E, E$ \begin{CJK}{UTF8}{mj}是\end{CJK} 3 \begin{CJK}{UTF8}{mj}阶单位矩阵\end{CJK}, \begin{CJK}{UTF8}{mj}求\end{CJK} $B$.

\begin{CJK}{UTF8}{mj}五\end{CJK}、\begin{CJK}{UTF8}{mj}当\end{CJK} $\lambda$ \begin{CJK}{UTF8}{mj}满足什么条件时\end{CJK}, \begin{CJK}{UTF8}{mj}方程组有解\end{CJK}, \begin{CJK}{UTF8}{mj}并求解\end{CJK}.
$$
\left\{\begin{array}{l}
(1+\lambda) x_{1}+x_{2}+x_{3}=1 \\
x_{1}+(1+\lambda) x_{2}+x_{3}=\lambda \\
x_{1}+x_{2}+(1+\lambda) x_{3}=\lambda^{2}
\end{array}\right.
$$
\begin{CJK}{UTF8}{mj}六\end{CJK}、\begin{CJK}{UTF8}{mj}用正交线性变换化二次型为标准形\end{CJK}.
$$
f\left(x_{1}, x_{2}, x_{3}\right)=x_{1}^{2}-2 x_{2}^{2}-2 x_{3}^{2}-4 x_{1} x_{2}+4 x_{1} x_{3}+8 x_{2} x_{3}
$$
\begin{CJK}{UTF8}{mj}七\end{CJK}、\begin{CJK}{UTF8}{mj}设\end{CJK} $A=\left(\begin{array}{cc}2 & -1 \\ -3 & 3\end{array}\right)$, \begin{CJK}{UTF8}{mj}有理系数多项式\end{CJK} $f(x)$ \begin{CJK}{UTF8}{mj}满足\end{CJK} $f(A)=0$ \begin{CJK}{UTF8}{mj}的充要条件是\end{CJK} $f(x)$ \begin{CJK}{UTF8}{mj}是\end{CJK} $x^{2}-5 x+3$ \begin{CJK}{UTF8}{mj}的倍式\end{CJK}.

\begin{CJK}{UTF8}{mj}八\end{CJK}、\begin{CJK}{UTF8}{mj}设线性空间\end{CJK} $V$ \begin{CJK}{UTF8}{mj}由全体\end{CJK} $n$ \begin{CJK}{UTF8}{mj}阶实矩阵所组成\end{CJK}, $V_{1}$ \begin{CJK}{UTF8}{mj}为全体实对称矩阵所组成的子空间\end{CJK}, $V_{2}$ \begin{CJK}{UTF8}{mj}为全体实反对称矩阵所\end{CJK} \begin{CJK}{UTF8}{mj}组成的子空间\end{CJK}, \begin{CJK}{UTF8}{mj}证明\end{CJK}: $V=V_{1} \oplus V_{2}$.

\begin{CJK}{UTF8}{mj}九\end{CJK}、 $A, B$ \begin{CJK}{UTF8}{mj}是正交变换\end{CJK}, $|A|+|B|=0$, \begin{CJK}{UTF8}{mj}则\end{CJK} $|A+B|=0$.

\section{9. 云南大学 2017 年研究生入学考试试题高等代数}
\begin{CJK}{UTF8}{mj}李扬\end{CJK}

\begin{CJK}{UTF8}{mj}微信公众号\end{CJK}: sxkyliyang

\begin{CJK}{UTF8}{mj}一\end{CJK}、\begin{CJK}{UTF8}{mj}填空题\end{CJK}

\begin{enumerate}
  \item \begin{CJK}{UTF8}{mj}行列式\end{CJK} $\left|\begin{array}{lll}0 & 4 & 0 \\ 3 & 0 & 0 \\ 0 & 0 & 6\end{array}\right|=$

  \item $A=\left(\begin{array}{llll}1 & 0 & 0 & 1 \\ 2 & 3 & 0 & 0 \\ 1 & 0 & 2 & 0 \\ 4 & 0 & 0 & 0\end{array}\right)$, \begin{CJK}{UTF8}{mj}求\end{CJK} $A^{-1}=$

  \item \begin{CJK}{UTF8}{mj}多项式\end{CJK} $F(x)=x^{4}+a x^{2}+3 x+b, g(x)=x^{2}+2 x+1$, \begin{CJK}{UTF8}{mj}且\end{CJK} $(f(x), g(x))=g(x)$, \begin{CJK}{UTF8}{mj}求\end{CJK} $a=$ ,$b=$

  \item $\alpha=(1,1,-1,1), \beta=(1,1,1,-1)$, \begin{CJK}{UTF8}{mj}求\end{CJK} $(\alpha, \beta)=$

\end{enumerate}
\begin{CJK}{UTF8}{mj}二\end{CJK}、 $A$ \begin{CJK}{UTF8}{mj}是\end{CJK} $n$ \begin{CJK}{UTF8}{mj}阶实矩阵\end{CJK}, $A^{8}=0$, \begin{CJK}{UTF8}{mj}证明\end{CJK}: $E+2 A$ \begin{CJK}{UTF8}{mj}可逆\end{CJK}.

\begin{CJK}{UTF8}{mj}三\end{CJK}、 \begin{CJK}{UTF8}{mj}计算行列式\end{CJK}
$$
\left|\begin{array}{ccccc}
1 & 1 & 1 & \cdots & 1 \\
\alpha_{1}-2 & \alpha_{2}-2 & \alpha_{3}-2 & \cdots & \alpha_{n}-2 \\
\alpha_{1}^{2}-3 \alpha_{1} & \alpha_{2}^{2}-3 \alpha_{2} & \alpha_{3}^{2}-3 \alpha_{3} & \cdots & \alpha_{n}^{2}-3 \alpha_{n} \\
\vdots & \vdots & \vdots & & \vdots \\
\alpha_{1}^{n-1}-n \alpha_{1}^{n-2} & \alpha_{2}^{n-1}-n \alpha_{2}^{n-2} & \alpha_{3}^{n-1}-n \alpha_{3}^{n-2} & \cdots & \alpha_{n}^{n-1}-n \alpha_{n}^{n-2}
\end{array}\right| .
$$
\begin{CJK}{UTF8}{mj}四\end{CJK}、\begin{CJK}{UTF8}{mj}方程组\end{CJK}
$$
\left\{\begin{array}{l}
x_{1}-x_{2}=a_{1} \\
x_{2}-x_{3}=a_{2} \\
\cdots \cdots \\
x_{n}-x_{1}=a_{n}
\end{array}\right.
$$
\begin{CJK}{UTF8}{mj}有解的充分必要条件是什么\end{CJK}? \begin{CJK}{UTF8}{mj}并解方程组\end{CJK} (\begin{CJK}{UTF8}{mj}用导出组基础解系来表示\end{CJK}).

\begin{CJK}{UTF8}{mj}五\end{CJK}、 $f(x), g(x)$ \begin{CJK}{UTF8}{mj}为整系数多项式\end{CJK}, \begin{CJK}{UTF8}{mj}证明\end{CJK} $(f(x), g(x))=1$ \begin{CJK}{UTF8}{mj}的充分必要条件是\end{CJK} $\left(f\left(x^{m}\right), g\left(x^{m}\right)\right)=1$.

\begin{CJK}{UTF8}{mj}六\end{CJK}、 $V$ \begin{CJK}{UTF8}{mj}是\end{CJK} $n$ \begin{CJK}{UTF8}{mj}维欧式空间\end{CJK}, $\alpha \in V, \alpha$ \begin{CJK}{UTF8}{mj}非零\end{CJK}, $\alpha^{\perp}=\{\beta \mid(\beta, \alpha)=0, \beta \in V\}$, \begin{CJK}{UTF8}{mj}证明\end{CJK}: $\alpha^{\perp}$ \begin{CJK}{UTF8}{mj}是\end{CJK} $V$ \begin{CJK}{UTF8}{mj}的子空间\end{CJK}, \begin{CJK}{UTF8}{mj}并求\end{CJK} $\alpha^{\perp}$ \begin{CJK}{UTF8}{mj}的维\end{CJK} \begin{CJK}{UTF8}{mj}数\end{CJK}.

\begin{CJK}{UTF8}{mj}七\end{CJK}、 $A=\left(\begin{array}{ccc}1 & -1 & -2 \\ -3 & 4 & 2 \\ 3 & -2 & 0\end{array}\right)$, \begin{CJK}{UTF8}{mj}求若尔当标准形\end{CJK}.

\begin{CJK}{UTF8}{mj}八\end{CJK}、 \begin{CJK}{UTF8}{mj}二次型\end{CJK} $f(X)=X^{\prime} A X, A=\left(\begin{array}{ccc}0 & 1 & 1 \\ 1 & 0 & \alpha \\ 1 & -3 & 0\end{array}\right)$, \begin{CJK}{UTF8}{mj}经正交变换\end{CJK} $X=P Y$ \begin{CJK}{UTF8}{mj}化为标准形\end{CJK} $f(y)=2 y_{1}^{2}+2 y_{2}^{2}+\beta y_{3}^{2}$, \begin{CJK}{UTF8}{mj}求\end{CJK} $\alpha, \beta$ \begin{CJK}{UTF8}{mj}的值及正交矩阵\end{CJK} $P$.

\begin{CJK}{UTF8}{mj}九\end{CJK}、 $A$ \begin{CJK}{UTF8}{mj}是\end{CJK} $n$ \begin{CJK}{UTF8}{mj}阶实对称矩阵\end{CJK}, \begin{CJK}{UTF8}{mj}且\end{CJK} $A$ \begin{CJK}{UTF8}{mj}正定\end{CJK}, \begin{CJK}{UTF8}{mj}证明\end{CJK}: $B^{\prime} A B$ \begin{CJK}{UTF8}{mj}正定的充分必要条件是\end{CJK} $r(B)=m$, \begin{CJK}{UTF8}{mj}其中\end{CJK} $B$ \begin{CJK}{UTF8}{mj}是\end{CJK} $n \times m$ \begin{CJK}{UTF8}{mj}阶实矩\end{CJK} \begin{CJK}{UTF8}{mj}阵\end{CJK}.

\section{0. 云南大学 2009 年研究生入学考试试题数学分析}
\begin{CJK}{UTF8}{mj}李扬\end{CJK}

\begin{CJK}{UTF8}{mj}微信公众号\end{CJK}: sxkyliyang

\begin{CJK}{UTF8}{mj}一\end{CJK}、\begin{CJK}{UTF8}{mj}填空题\end{CJK} (\begin{CJK}{UTF8}{mj}共\end{CJK} 5 \begin{CJK}{UTF8}{mj}题\end{CJK}, \begin{CJK}{UTF8}{mj}每题\end{CJK} 6 \begin{CJK}{UTF8}{mj}分\end{CJK}, \begin{CJK}{UTF8}{mj}共\end{CJK} 30 \begin{CJK}{UTF8}{mj}分\end{CJK})

\begin{enumerate}
  \item \begin{CJK}{UTF8}{mj}曲线\end{CJK} $y=\frac{2 x^{4}-5 x^{3}+3 x^{2}+1}{x^{2}(x-2)}$ \begin{CJK}{UTF8}{mj}共有\end{CJK} \begin{CJK}{UTF8}{mj}条渐近线\end{CJK}.

  \item \begin{CJK}{UTF8}{mj}设函数\end{CJK} $f^{\prime}(5 x-1)=e^{x}, f(-1)=5$, \begin{CJK}{UTF8}{mj}则\end{CJK} $f(x)=$

\end{enumerate}
3 . \begin{CJK}{UTF8}{mj}设\end{CJK} $x=\ln \left(1+t^{2}\right), y=t-\arctan t$, \begin{CJK}{UTF8}{mj}则\end{CJK} $\left.\mathrm{d}^{2} y\right|_{t=2}=$

\begin{enumerate}
  \setcounter{enumi}{4}
  \item \begin{CJK}{UTF8}{mj}已知级数\end{CJK} $\sum_{n=1}^{\infty} \frac{(-1)^{n}}{n^{p}}$ \begin{CJK}{UTF8}{mj}与广义积分\end{CJK} $\int_{0}^{+\infty} e^{(p-2) x} \mathrm{~d} x$ \begin{CJK}{UTF8}{mj}都收敛\end{CJK}, \begin{CJK}{UTF8}{mj}则\end{CJK} $p$ \begin{CJK}{UTF8}{mj}的取值范围是\end{CJK}

  \item \begin{CJK}{UTF8}{mj}设函数\end{CJK} $y=y(x)$ \begin{CJK}{UTF8}{mj}满足方程\end{CJK} $y^{\prime \prime}+4 y^{\prime}+4 y=0, y(0)=0, y^{\prime}(0)=1$, \begin{CJK}{UTF8}{mj}则积分\end{CJK} $\int_{0}^{+\infty} y \mathrm{~d} x=$ \begin{CJK}{UTF8}{mj}二\end{CJK}、 $\left(20\right.$ \begin{CJK}{UTF8}{mj}分\end{CJK}) \begin{CJK}{UTF8}{mj}设\end{CJK} $g(x)$ \begin{CJK}{UTF8}{mj}为连续函数\end{CJK}, $f(x)=\int_{0}^{\sin x} g\left(x^{2} t\right) \mathrm{d} x$, \begin{CJK}{UTF8}{mj}求\end{CJK} $f^{\prime}(x)$, \begin{CJK}{UTF8}{mj}并讨论\end{CJK} $f^{\prime}(x)$ \begin{CJK}{UTF8}{mj}的连续性\end{CJK}.

\end{enumerate}
\begin{CJK}{UTF8}{mj}三\end{CJK}、 $\left(15\right.$ \begin{CJK}{UTF8}{mj}分\end{CJK} ) \begin{CJK}{UTF8}{mj}设\end{CJK} $f(x)$ \begin{CJK}{UTF8}{mj}在\end{CJK} $[a, b]$ \begin{CJK}{UTF8}{mj}上可导\end{CJK}, \begin{CJK}{UTF8}{mj}且\end{CJK} $f(a)=f(b)=0, f^{\prime}(a) \cdot f^{\prime}(b)>0$, \begin{CJK}{UTF8}{mj}试证\end{CJK}: \begin{CJK}{UTF8}{mj}存在点\end{CJK} $\xi, \eta \in(a, b)$ \begin{CJK}{UTF8}{mj}满足\end{CJK} $\xi \neq \eta$, \begin{CJK}{UTF8}{mj}使得\end{CJK}
$$
f^{\prime}(\xi)=f^{\prime}(\eta)=0 .
$$
\begin{CJK}{UTF8}{mj}四\end{CJK}、 ( 20 \begin{CJK}{UTF8}{mj}分\end{CJK}) \begin{CJK}{UTF8}{mj}将函数\end{CJK} $f(x)=2+x(0 \leqslant x \leqslant 1)$ \begin{CJK}{UTF8}{mj}展开成周期为\end{CJK} 2 \begin{CJK}{UTF8}{mj}的余弦级数\end{CJK}, \begin{CJK}{UTF8}{mj}并求级数\end{CJK} $\sum_{n=1}^{\infty} \frac{1}{n^{2}}$ \begin{CJK}{UTF8}{mj}的和\end{CJK}.

\begin{CJK}{UTF8}{mj}五\end{CJK}、 ( 20 \begin{CJK}{UTF8}{mj}分\end{CJK}) \begin{CJK}{UTF8}{mj}设函数\end{CJK} $u=f\left(\sqrt{x^{2}+y^{2}}, z\right), f$ \begin{CJK}{UTF8}{mj}具有二阶连续偏导数\end{CJK}, \begin{CJK}{UTF8}{mj}且\end{CJK} $z=z(x, y)$ \begin{CJK}{UTF8}{mj}由方程\end{CJK} $x y+x+y-z=e^{z}$ \begin{CJK}{UTF8}{mj}确定\end{CJK}, \begin{CJK}{UTF8}{mj}求\end{CJK} $\frac{\partial^{2} u}{\partial x \partial y}$.

\begin{CJK}{UTF8}{mj}六\end{CJK}、 $(15$ \begin{CJK}{UTF8}{mj}分\end{CJK}) \begin{CJK}{UTF8}{mj}计算积分\end{CJK}
$$
\iint_{S} x\left(x^{2}+1\right) \mathrm{d} y \mathrm{~d} z+y\left(y^{2}+2\right) \mathrm{d} z \mathrm{~d} x+z\left(z^{2}+3\right) \mathrm{d} x \mathrm{~d} y
$$
\begin{CJK}{UTF8}{mj}其中\end{CJK} $S$ \begin{CJK}{UTF8}{mj}为上半球面\end{CJK} $z=\sqrt{1-x^{2}-y^{2}}$ \begin{CJK}{UTF8}{mj}的上侧\end{CJK}.

\begin{CJK}{UTF8}{mj}七\end{CJK}、 (15 \begin{CJK}{UTF8}{mj}分\end{CJK}) \begin{CJK}{UTF8}{mj}计算曲线积分\end{CJK}
$$
\int_{\widehat{A O B}}\left(e^{y}+12 x y\right) \mathrm{d} x+\left(x e^{y}-\cos y\right) \mathrm{d} y
$$
\begin{CJK}{UTF8}{mj}其中\end{CJK} $\widehat{A O B}$ \begin{CJK}{UTF8}{mj}是从点\end{CJK} $A(-1,1)$ \begin{CJK}{UTF8}{mj}沿曲线\end{CJK} $y=x^{2}$ \begin{CJK}{UTF8}{mj}到原点\end{CJK} $O(0,0)$, \begin{CJK}{UTF8}{mj}再沿\end{CJK} $O x$ \begin{CJK}{UTF8}{mj}轴到点\end{CJK} $B(3,0)$ \begin{CJK}{UTF8}{mj}的路径\end{CJK}.

\begin{CJK}{UTF8}{mj}八\end{CJK}、 (15 \begin{CJK}{UTF8}{mj}分\end{CJK}) \begin{CJK}{UTF8}{mj}设\end{CJK} $f(x)$ \begin{CJK}{UTF8}{mj}是\end{CJK} $(-\infty,+\infty)$ \begin{CJK}{UTF8}{mj}内的连续函数\end{CJK}, $f_{n}(x)=\sum_{k=1}^{n} \frac{1}{n} f\left(x+\frac{k}{n}\right), n=1,2, \cdots$, \begin{CJK}{UTF8}{mj}证明\end{CJK}: \begin{CJK}{UTF8}{mj}函数列\end{CJK} $\left\{f_{n}(x)\right\}$ \begin{CJK}{UTF8}{mj}在任意有限闭区间上一致收敛\end{CJK}.

\section{1. 云南大学 2010 年研究生入学考试试题数学分析}
\begin{CJK}{UTF8}{mj}李扬\end{CJK}

\begin{CJK}{UTF8}{mj}微信公众号\end{CJK}: sxkyliyang

\begin{CJK}{UTF8}{mj}一\end{CJK}、\begin{CJK}{UTF8}{mj}填空题\end{CJK} (\begin{CJK}{UTF8}{mj}共\end{CJK} 5 \begin{CJK}{UTF8}{mj}题\end{CJK}, \begin{CJK}{UTF8}{mj}每题\end{CJK} 6 \begin{CJK}{UTF8}{mj}分\end{CJK}, \begin{CJK}{UTF8}{mj}共\end{CJK} 30 \begin{CJK}{UTF8}{mj}分\end{CJK})

\begin{enumerate}
  \item \begin{CJK}{UTF8}{mj}设函数\end{CJK} $f(x)=\lim _{n \rightarrow \infty}\left[\left(1+\frac{x}{n}\right)\left(1+\frac{2 x}{n}\right) \cdots\left(1+\frac{n x}{n}\right)\right]^{\frac{1}{n}}$, \begin{CJK}{UTF8}{mj}则\end{CJK} $\lim _{x \rightarrow+\infty} \frac{f(x)}{x}=$
\end{enumerate}
2 . \begin{CJK}{UTF8}{mj}曲线\end{CJK} $y=\frac{2 x}{1+x^{2}}$ \begin{CJK}{UTF8}{mj}共有\end{CJK} \begin{CJK}{UTF8}{mj}个拐点\end{CJK}.

\begin{enumerate}
  \setcounter{enumi}{3}
  \item \begin{CJK}{UTF8}{mj}已知\end{CJK} $\frac{\mathrm{d}}{\mathrm{d} x} \int_{\sqrt{x}}^{2} f(t) \mathrm{d} t=\sqrt{x}(x>0)$, \begin{CJK}{UTF8}{mj}则\end{CJK} $f^{\prime}(x)=$

  \item \begin{CJK}{UTF8}{mj}设二元函数\end{CJK} $z=y^{3} \ln (2 e-x)+x^{3} \tan \sqrt{y-1}$, \begin{CJK}{UTF8}{mj}则\end{CJK} $\left.\frac{\partial z}{\partial x}\right|_{(x, 1)}=$

  \item \begin{CJK}{UTF8}{mj}设\end{CJK} $f(x, y)=\left\{\begin{array}{ll}x, & 0 \leqslant x \leqslant \frac{1}{2} ; \\ 3-2 x, & \frac{1}{2}<x<1 .\end{array}, S(x)=\sum_{n=1}^{\infty} b_{n} \sin n \pi x,-\infty<x<+\infty\right.$, \begin{CJK}{UTF8}{mj}其中\end{CJK} $b_{n}=2 \int_{0}^{1} f(x) \sin n \pi x \mathrm{~d} x,$, $n=1,2, \cdots$, \begin{CJK}{UTF8}{mj}则\end{CJK} $S\left(-\frac{7}{2}\right)^{2}=$

\end{enumerate}
\begin{CJK}{UTF8}{mj}二\end{CJK}、 $\left(20\right.$ \begin{CJK}{UTF8}{mj}分\end{CJK} ) \begin{CJK}{UTF8}{mj}设\end{CJK} $\lim _{n \rightarrow \infty} x_{n}=a$, \begin{CJK}{UTF8}{mj}按定义证明\end{CJK}: $\lim _{n \rightarrow \infty} \frac{x_{1}+x_{2}+\cdots+x_{n}}{n+\sqrt{n}}=a$.

\begin{CJK}{UTF8}{mj}三\end{CJK}、 $\left(15\right.$ \begin{CJK}{UTF8}{mj}分\end{CJK}) \begin{CJK}{UTF8}{mj}设\end{CJK} $f(x)$ \begin{CJK}{UTF8}{mj}在\end{CJK} $[0,1]$ \begin{CJK}{UTF8}{mj}上有二阶连续导数\end{CJK}, $\left|f^{\prime \prime}(x)\right| \leqslant 1, x \in[0,1], f(0)=f(1)$. \begin{CJK}{UTF8}{mj}证明\end{CJK}:
$$
\left|f^{\prime}(x)\right| \leqslant \frac{1}{2}, x \in(0,1)
$$
\begin{CJK}{UTF8}{mj}四\end{CJK}、 ( 20 \begin{CJK}{UTF8}{mj}分\end{CJK} $)$ \begin{CJK}{UTF8}{mj}令\end{CJK} $u=x+y, v=\frac{y}{x}, w=\frac{z}{x}$, \begin{CJK}{UTF8}{mj}取\end{CJK} $u, v$ \begin{CJK}{UTF8}{mj}为新的自变量\end{CJK}, \begin{CJK}{UTF8}{mj}为\end{CJK} $w=w(u, v)$ \begin{CJK}{UTF8}{mj}新的因变量\end{CJK}, \begin{CJK}{UTF8}{mj}变换方程\end{CJK}
$$
\frac{\partial^{2} z}{\partial x^{2}}-2 \frac{\partial^{2} z}{\partial x \partial y}+\frac{\partial^{2} z}{\partial y^{2}}=0
$$
\begin{CJK}{UTF8}{mj}并求出该方程的解\end{CJK}.

\begin{CJK}{UTF8}{mj}五\end{CJK}、 ( 20 \begin{CJK}{UTF8}{mj}分\end{CJK}) \begin{CJK}{UTF8}{mj}设有方程\end{CJK} $(x-1)^{n}+n x-n-1=0$, \begin{CJK}{UTF8}{mj}其中\end{CJK} $n$ \begin{CJK}{UTF8}{mj}为正整数\end{CJK}, \begin{CJK}{UTF8}{mj}证明此方程存在唯一正实根\end{CJK} $x_{n}$, \begin{CJK}{UTF8}{mj}并证明当\end{CJK} $a>0$ \begin{CJK}{UTF8}{mj}时\end{CJK}, \begin{CJK}{UTF8}{mj}级数\end{CJK} $\sum_{n=1}^{\infty}\left(\frac{x_{n}}{x_{n}+a}\right)^{n}$ \begin{CJK}{UTF8}{mj}收玫\end{CJK}.

\begin{CJK}{UTF8}{mj}六\end{CJK}、 ( 15 \begin{CJK}{UTF8}{mj}分\end{CJK}) \begin{CJK}{UTF8}{mj}计算曲面积分\end{CJK}
$$
\iint_{S} \frac{x \mathrm{~d} y \mathrm{~d} z+z \mathrm{~d} z \mathrm{~d} x+(z+1)^{2} \mathrm{~d} x \mathrm{~d} y}{\left(x^{2}+y^{2}+z^{2}\right)^{\frac{1}{2}}},
$$
\begin{CJK}{UTF8}{mj}其中\end{CJK} $S$ \begin{CJK}{UTF8}{mj}为下半球面\end{CJK} $z=-\sqrt{1-x^{2}-y^{2}}$ \begin{CJK}{UTF8}{mj}的内侧\end{CJK}.

\begin{CJK}{UTF8}{mj}七\end{CJK}、 (15 \begin{CJK}{UTF8}{mj}分\end{CJK}) \begin{CJK}{UTF8}{mj}计算曲线积分\end{CJK}
$$
\int_{L} \frac{x \mathrm{~d} y-y \mathrm{~d} x}{4 x^{2}+y^{2}}
$$
\begin{CJK}{UTF8}{mj}其中\end{CJK} $L$ \begin{CJK}{UTF8}{mj}为以\end{CJK} $(-1,0)$ \begin{CJK}{UTF8}{mj}为中心\end{CJK}, $R$ \begin{CJK}{UTF8}{mj}为半径的圆周\end{CJK} $(R \neq 1)$, \begin{CJK}{UTF8}{mj}方向为逆时针方向\end{CJK}.

\begin{CJK}{UTF8}{mj}八\end{CJK}、 $(15$ \begin{CJK}{UTF8}{mj}分\end{CJK} $)$ \begin{CJK}{UTF8}{mj}设\end{CJK} $f(x)$ \begin{CJK}{UTF8}{mj}在区间\end{CJK} $(a, b)$ \begin{CJK}{UTF8}{mj}上有连续的导数\end{CJK}, \begin{CJK}{UTF8}{mj}令\end{CJK}
$$
F_{n}(x)=\frac{n}{2}\left[f\left(x+\frac{1}{n}\right)-f\left(x-\frac{1}{n}\right)\right]
$$
\begin{CJK}{UTF8}{mj}设\end{CJK} $a<\beta<\gamma<b$. 1. \begin{CJK}{UTF8}{mj}证明\end{CJK} $\left\{F_{n}(x)\right\}$ \begin{CJK}{UTF8}{mj}在\end{CJK} $[\beta, \gamma]$ \begin{CJK}{UTF8}{mj}上是一致收敛的\end{CJK};

\begin{enumerate}
  \setcounter{enumi}{2}
  \item \begin{CJK}{UTF8}{mj}证明\end{CJK}: $\lim _{n \rightarrow \infty} \int_{\beta}^{\gamma} F_{n}(x) \mathrm{d} x=f(\gamma)-f(\beta)$.
\end{enumerate}
\section{2. 云南大学 2011 年研究生入学考试试题数学分析}
\begin{CJK}{UTF8}{mj}李扬\end{CJK}

\begin{CJK}{UTF8}{mj}微信公众号\end{CJK}: sxkyliyang

\begin{CJK}{UTF8}{mj}一\end{CJK}、\begin{CJK}{UTF8}{mj}填空题\end{CJK}

\begin{enumerate}
  \item $\lim _{x \rightarrow 0} \frac{\int_{0}^{x} x e^{-t^{2}} \mathrm{~d} t}{1-e^{-x^{2}}}=$

  \item \begin{CJK}{UTF8}{mj}在点\end{CJK} $x=1$ \begin{CJK}{UTF8}{mj}处取极大值\end{CJK} 6 , \begin{CJK}{UTF8}{mj}在点\end{CJK} $x=3$ \begin{CJK}{UTF8}{mj}处取极小值\end{CJK} 2 \begin{CJK}{UTF8}{mj}的次数最低的多项式为\end{CJK}

  \item $\int_{0}^{1} \mathrm{~d} y \int_{y}^{1} \sqrt{x^{2}-y^{2}} \mathrm{~d} x$ \begin{CJK}{UTF8}{mj}的值为\end{CJK}

  \item $\left(x+a x y-y^{2}\right) \mathrm{d} x+\left(x^{2}-2 x y-y^{2}\right) \mathrm{d} y$ \begin{CJK}{UTF8}{mj}是某个函数的微分\end{CJK}, \begin{CJK}{UTF8}{mj}则\end{CJK} $a=$

  \item \begin{CJK}{UTF8}{mj}设函数\end{CJK} $f(x)=x(\pi-x)(x \in[0, \pi])$ \begin{CJK}{UTF8}{mj}的傅里叶级数展开式为\end{CJK} $\frac{a_{0}}{2}+\sum_{n=1}^{\infty} a_{n} \cos n x$, \begin{CJK}{UTF8}{mj}则其中系数\end{CJK} $a_{3}$ \begin{CJK}{UTF8}{mj}的值为\end{CJK}

\end{enumerate}
\begin{CJK}{UTF8}{mj}二\end{CJK}、\begin{CJK}{UTF8}{mj}证明\end{CJK}: $x>1$ \begin{CJK}{UTF8}{mj}时\end{CJK}, \begin{CJK}{UTF8}{mj}有\end{CJK} $0<\frac{x}{2} \ln \frac{x+1}{x-1}-1<\frac{1}{3\left(x^{2}-1\right)}$.

\begin{CJK}{UTF8}{mj}三\end{CJK}、\begin{CJK}{UTF8}{mj}设函数\end{CJK} $f(x)$ \begin{CJK}{UTF8}{mj}在\end{CJK} $[a, b]$ \begin{CJK}{UTF8}{mj}上连续\end{CJK}, \begin{CJK}{UTF8}{mj}在\end{CJK} $(a, b)$ \begin{CJK}{UTF8}{mj}内可导\end{CJK}, \begin{CJK}{UTF8}{mj}且\end{CJK} $f(a)=f(b)=1$, \begin{CJK}{UTF8}{mj}证明\end{CJK}: \begin{CJK}{UTF8}{mj}存在\end{CJK} $\xi, \eta \in(a, b)$ \begin{CJK}{UTF8}{mj}使得\end{CJK}
$$
e^{\xi-\eta}\left[f^{2}(\xi)+2 f(\xi) f^{\prime}(\xi)\right]=1
$$
\begin{CJK}{UTF8}{mj}四\end{CJK}、\begin{CJK}{UTF8}{mj}将\end{CJK} $f(x)=\arctan \left(\frac{1-x}{1+x}\right)+\arctan \left(\frac{2 x}{1-x^{2}}\right)$ \begin{CJK}{UTF8}{mj}展开为\end{CJK} $x$ \begin{CJK}{UTF8}{mj}的幂函数并求\end{CJK} $\sum_{n=0}^{\infty} \frac{(-1)^{n}}{2 n+1}$ \begin{CJK}{UTF8}{mj}的和\end{CJK}.

\begin{CJK}{UTF8}{mj}五\end{CJK}、\begin{CJK}{UTF8}{mj}设函数\end{CJK} $u=f(v)$ \begin{CJK}{UTF8}{mj}在\end{CJK} $(0,+\infty)$ \begin{CJK}{UTF8}{mj}具有二阶连续导数\end{CJK}, \begin{CJK}{UTF8}{mj}且\end{CJK} $u=f\left(\ln \sqrt{x^{2}+y^{2}+z^{2}}\right)$ \begin{CJK}{UTF8}{mj}满足方程\end{CJK}
$$
\frac{\partial^{2} u}{\partial x^{2}}+\frac{\partial^{2} u}{\partial y^{2}}+\frac{\partial^{2} u}{\partial z^{2}}=\left(x^{2}+y^{2}+z^{2}\right)^{-\frac{3}{2}}
$$
\begin{CJK}{UTF8}{mj}且\end{CJK} $f(1)=-\frac{2}{e}, f^{\prime}(1)=\frac{1}{e}$, \begin{CJK}{UTF8}{mj}求\end{CJK} $f(v)$ \begin{CJK}{UTF8}{mj}的具体表达式\end{CJK}.

\begin{CJK}{UTF8}{mj}六\end{CJK}、\begin{CJK}{UTF8}{mj}计算第二类曲面积分\end{CJK}
$$
\iint_{s_{1}+s_{2}} x y \mathrm{~d} z \mathrm{~d} x+(z+1) \mathrm{d} x \mathrm{~d} y .
$$
\begin{CJK}{UTF8}{mj}其中\end{CJK} $s_{1}$ \begin{CJK}{UTF8}{mj}为柱面\end{CJK} $x^{2}+y^{2}=4, x \geqslant 0$ \begin{CJK}{UTF8}{mj}介于平面\end{CJK} $0<z \leqslant 1$ \begin{CJK}{UTF8}{mj}的部分\end{CJK}, \begin{CJK}{UTF8}{mj}法向量与\end{CJK} $x$ \begin{CJK}{UTF8}{mj}轴的正向成锐角\end{CJK}; $s_{2}$ \begin{CJK}{UTF8}{mj}为\end{CJK} $x o y$ \begin{CJK}{UTF8}{mj}平面\end{CJK} \begin{CJK}{UTF8}{mj}上\end{CJK} $x^{2}+y^{2} \leqslant 4, x \geqslant 0$ \begin{CJK}{UTF8}{mj}的部分\end{CJK}, \begin{CJK}{UTF8}{mj}法线方向朝下\end{CJK}.

\begin{CJK}{UTF8}{mj}七\end{CJK}、\begin{CJK}{UTF8}{mj}设\end{CJK} $0<a<1, x_{1}=\frac{c}{2}, x_{n+1}=\frac{c}{2}+\frac{x_{n}^{2}}{2}$. \begin{CJK}{UTF8}{mj}证明\end{CJK}:

\begin{enumerate}
  \item $\left\{x_{n}\right\}$ \begin{CJK}{UTF8}{mj}收敛\end{CJK};

  \item \begin{CJK}{UTF8}{mj}级数\end{CJK} $\sum_{n=0}^{\infty}\left(\frac{x_{n+1}}{x_{n}}-1\right)$ \begin{CJK}{UTF8}{mj}收敛\end{CJK}.

\end{enumerate}
\begin{CJK}{UTF8}{mj}八\end{CJK}、\begin{CJK}{UTF8}{mj}已知\end{CJK} $f(x)$ \begin{CJK}{UTF8}{mj}是\end{CJK} $[0,+\infty)$ \begin{CJK}{UTF8}{mj}上的正的连续函数\end{CJK}, \begin{CJK}{UTF8}{mj}且\end{CJK} $\int_{0}^{+\infty} \frac{1}{f^{2}(x)} \mathrm{d} x$ \begin{CJK}{UTF8}{mj}收敛\end{CJK}. \begin{CJK}{UTF8}{mj}证明\end{CJK}: $\lim _{A \rightarrow+\infty} \frac{1}{A^{2}} \int_{0}^{A} f^{2}(x) \mathrm{d} x=+\infty$.

\section{3. 云南大学 2012 年研究生入学考试试题数学分析}
\begin{CJK}{UTF8}{mj}李扬\end{CJK}

\begin{CJK}{UTF8}{mj}微信公众号\end{CJK}: sxkyliyang

\begin{CJK}{UTF8}{mj}一\end{CJK}、\begin{CJK}{UTF8}{mj}填空题\end{CJK} (\begin{CJK}{UTF8}{mj}共\end{CJK} 5 \begin{CJK}{UTF8}{mj}题\end{CJK}, \begin{CJK}{UTF8}{mj}每题\end{CJK} 6 \begin{CJK}{UTF8}{mj}分\end{CJK}, \begin{CJK}{UTF8}{mj}共\end{CJK} 30 \begin{CJK}{UTF8}{mj}分\end{CJK})

\begin{enumerate}
  \item \begin{CJK}{UTF8}{mj}已知\end{CJK} $\lim _{x \rightarrow 0} \frac{e^{a x}-1+b \ln \cos x}{\sin 2 x+\sqrt{1-c x^{2}}-1}=-1$, \begin{CJK}{UTF8}{mj}则\end{CJK} $a=$

  \item \begin{CJK}{UTF8}{mj}设\end{CJK} $F(y)=\int_{1}^{y}(x+y) \sin x^{2} \mathrm{~d} x$, \begin{CJK}{UTF8}{mj}则\end{CJK} $F^{\prime \prime}(y)=$

  \item \begin{CJK}{UTF8}{mj}设数列\end{CJK} $\left\{a_{n}\right\}$ \begin{CJK}{UTF8}{mj}单调减少\end{CJK}, $\lim _{n \rightarrow \infty} a_{n}=0, S_{n}=\sum_{k=1}^{n} a_{k}(n=1,2, \cdots)$ \begin{CJK}{UTF8}{mj}无界\end{CJK}, \begin{CJK}{UTF8}{mj}则幂级数\end{CJK} $\sum_{n=1}^{\infty} a_{n}(x-1)^{n}$ \begin{CJK}{UTF8}{mj}的收敛\end{CJK} \begin{CJK}{UTF8}{mj}区间为\end{CJK}

  \item \begin{CJK}{UTF8}{mj}广义积分\end{CJK} $\int_{0}^{+\infty} \frac{\mathrm{d} x}{\left(e^{x}+e^{-x}\right)^{2}}=$

  \item \begin{CJK}{UTF8}{mj}曲线\end{CJK} $\left(\frac{x^{2}}{9}+\frac{y^{2}}{16}\right)^{2}=x^{2}+y^{2}$ \begin{CJK}{UTF8}{mj}所围图形的面积等于\end{CJK}

\end{enumerate}
\begin{CJK}{UTF8}{mj}二\end{CJK}、 $(20$ \begin{CJK}{UTF8}{mj}分\end{CJK} $)$ \begin{CJK}{UTF8}{mj}求\end{CJK} $f(x)=\left\{\begin{array}{ll}e^{-\frac{1}{x^{2}}}, & x \neq 0 ; \\ 0 . & x=0 .\end{array}\right.$ \begin{CJK}{UTF8}{mj}的一阶\end{CJK}, \begin{CJK}{UTF8}{mj}二阶导数\end{CJK}, \begin{CJK}{UTF8}{mj}并讨论其连续性\end{CJK}.

\begin{CJK}{UTF8}{mj}三\end{CJK}、 $(15$ \begin{CJK}{UTF8}{mj}分\end{CJK} $)$ \begin{CJK}{UTF8}{mj}设\end{CJK} $f(x)$ \begin{CJK}{UTF8}{mj}在\end{CJK} $[1,+\infty)$ \begin{CJK}{UTF8}{mj}上处处有\end{CJK} $f^{\prime \prime}(x) \leqslant 0, f(1)=2, f^{\prime}(1)=-3$, \begin{CJK}{UTF8}{mj}证明\end{CJK}: \begin{CJK}{UTF8}{mj}在\end{CJK} $(1,+\infty)$ \begin{CJK}{UTF8}{mj}内方程\end{CJK} $f(x)=0$ \begin{CJK}{UTF8}{mj}有且只有一个实根\end{CJK}.

\begin{CJK}{UTF8}{mj}四\end{CJK}、(20 \begin{CJK}{UTF8}{mj}分\end{CJK}) \begin{CJK}{UTF8}{mj}设函数\end{CJK} $f(x)=x^{2}(-\pi \leqslant x \leqslant \pi)$ \begin{CJK}{UTF8}{mj}的傅里叶展开式为\end{CJK} $\sum_{n=0}^{\infty} a_{n} \cos n x$, \begin{CJK}{UTF8}{mj}试证\end{CJK}: \begin{CJK}{UTF8}{mj}级数\end{CJK} $\sum_{n=2}^{\infty}\left[n a_{n} \sin \left(n \pi+\frac{1}{\ln n}\right)\right]$ \begin{CJK}{UTF8}{mj}发散\end{CJK}.

\begin{CJK}{UTF8}{mj}五\end{CJK}、 $\left(20\right.$ \begin{CJK}{UTF8}{mj}分\end{CJK}) \begin{CJK}{UTF8}{mj}设\end{CJK} $z=r^{3} \cos \theta+\theta \ln r$, \begin{CJK}{UTF8}{mj}而\end{CJK} $r^{2}=x^{2}+y^{2}, \tan \theta=\frac{y}{x}$, \begin{CJK}{UTF8}{mj}求\end{CJK} $\frac{\partial^{2} z}{\partial x^{2}}+\frac{\partial^{2} z}{\partial y^{2}}$.

\begin{CJK}{UTF8}{mj}六\end{CJK}、 (15 \begin{CJK}{UTF8}{mj}分\end{CJK}) \begin{CJK}{UTF8}{mj}计算积分\end{CJK}
$$
\iiint_{\Omega}\left(\sin z-\sin \left(x^{2}+y^{2}\right)\right) \mathrm{d} x \mathrm{~d} y \mathrm{~d} z
$$
\begin{CJK}{UTF8}{mj}其中\end{CJK} $\Omega$ \begin{CJK}{UTF8}{mj}是由曲面\end{CJK} $z=x^{2}+y^{2}(0 \leqslant z \leqslant \pi)$ \begin{CJK}{UTF8}{mj}与平面\end{CJK} $z=\pi$ \begin{CJK}{UTF8}{mj}所围成的区域\end{CJK}.

\begin{CJK}{UTF8}{mj}七\end{CJK}、 $(15$ \begin{CJK}{UTF8}{mj}分\end{CJK} $)$ \begin{CJK}{UTF8}{mj}设曲线积分\end{CJK} $\int_{L}\left(y e^{x}+3 x^{2}\right) \mathrm{d} x+[\varphi(x)+2 y] \mathrm{d} y$ \begin{CJK}{UTF8}{mj}与路径无关\end{CJK}, \begin{CJK}{UTF8}{mj}其中\end{CJK} $\varphi(x)$ \begin{CJK}{UTF8}{mj}具有连续的导数\end{CJK}, \begin{CJK}{UTF8}{mj}且\end{CJK} $\varphi(0)=1$, \begin{CJK}{UTF8}{mj}计算\end{CJK} $\int_{(0,0)}^{(2,2)}\left(y e^{x}+3 x^{2}\right) \mathrm{d} x+[\varphi(x)+2 y] \mathrm{d} y$ \begin{CJK}{UTF8}{mj}的值\end{CJK}.

\begin{CJK}{UTF8}{mj}八\end{CJK}、(15 \begin{CJK}{UTF8}{mj}分\end{CJK}) \begin{CJK}{UTF8}{mj}设\end{CJK} $f(x)$ \begin{CJK}{UTF8}{mj}是闭区间\end{CJK} $[a, b]$ \begin{CJK}{UTF8}{mj}上的函数\end{CJK}, \begin{CJK}{UTF8}{mj}满足条件\end{CJK}: \begin{CJK}{UTF8}{mj}对每一点\end{CJK} $x \in[a, b]$, \begin{CJK}{UTF8}{mj}任取\end{CJK} $\varepsilon>0$, \begin{CJK}{UTF8}{mj}存在\end{CJK} $\delta>0, \delta$ \begin{CJK}{UTF8}{mj}与\end{CJK} $x$ \begin{CJK}{UTF8}{mj}和\end{CJK} $\varepsilon$ \begin{CJK}{UTF8}{mj}有关\end{CJK}, \begin{CJK}{UTF8}{mj}当\end{CJK} $y \in[a, b] \cap(x-\delta, x+\delta)$ \begin{CJK}{UTF8}{mj}时\end{CJK}, \begin{CJK}{UTF8}{mj}有\end{CJK} $f(y)>f(x)-\varepsilon$. \begin{CJK}{UTF8}{mj}试证\end{CJK}: $f(x)$ \begin{CJK}{UTF8}{mj}在\end{CJK} $[a, b]$ \begin{CJK}{UTF8}{mj}上有最小值\end{CJK}.

\section{4. 云南大学 2013 年研究生入学考试试题数学分析}
\begin{CJK}{UTF8}{mj}李扬\end{CJK}

\begin{CJK}{UTF8}{mj}微信公众号\end{CJK}: sxkyliyang

\begin{CJK}{UTF8}{mj}一\end{CJK}、\begin{CJK}{UTF8}{mj}填空题\end{CJK} (\begin{CJK}{UTF8}{mj}共\end{CJK} 5 \begin{CJK}{UTF8}{mj}题\end{CJK}, \begin{CJK}{UTF8}{mj}每题\end{CJK} 6 \begin{CJK}{UTF8}{mj}分\end{CJK}, \begin{CJK}{UTF8}{mj}共\end{CJK} 30 \begin{CJK}{UTF8}{mj}分\end{CJK})

\begin{enumerate}
  \item $\lim _{n \rightarrow \infty}\left[\left(\frac{n-1}{n}\right)^{n}+\frac{\sin n}{n}+\left(\frac{2 n}{n^{2}+1}+\frac{2 n}{n^{2}+2^{2}}+\cdots+\frac{2 n}{n^{2}+n^{2}}\right)\right]=$

  \item \begin{CJK}{UTF8}{mj}设\end{CJK} $f(x)$ \begin{CJK}{UTF8}{mj}是定义在整个实数轴上的连续可导函数\end{CJK}, $f(0)=0$, \begin{CJK}{UTF8}{mj}且当\end{CJK} $x \rightarrow 0$ \begin{CJK}{UTF8}{mj}时\end{CJK}, $F(x)=\int_{0}^{x}\left(x^{3}-t^{3}\right) f(t) \mathrm{d} t$ \begin{CJK}{UTF8}{mj}的导数与\end{CJK} $x^{4}$ \begin{CJK}{UTF8}{mj}为等价无穷小\end{CJK}, \begin{CJK}{UTF8}{mj}则\end{CJK} $f^{\prime}(0)=$

  \item \begin{CJK}{UTF8}{mj}曲线\end{CJK} $y=\sin x(0 \leqslant x \leqslant \pi)$ \begin{CJK}{UTF8}{mj}绕\end{CJK} $x$ \begin{CJK}{UTF8}{mj}轴旋转所得的旋转曲面的面积等于\end{CJK}

  \item $\int_{0}^{+\infty} e^{-x} \sin x \mathrm{~d} x+\int_{0}^{2} \frac{1}{\sqrt{|x-1|}} \mathrm{d} x=$

  \item \begin{CJK}{UTF8}{mj}级数\end{CJK} $\sum_{n=1}^{\infty} \frac{1}{n \cdot 2^{n}}$ \begin{CJK}{UTF8}{mj}的和等于\end{CJK}

\end{enumerate}
\begin{CJK}{UTF8}{mj}二\end{CJK}、 (15 \begin{CJK}{UTF8}{mj}分\end{CJK}) \begin{CJK}{UTF8}{mj}设\end{CJK} $f(x)=\lim _{n \rightarrow \infty} \frac{\ln \left(e^{n^{2}}+1+|x|^{n^{2}}\right)}{n^{2}}$, \begin{CJK}{UTF8}{mj}讨论\end{CJK} $f(x)$ \begin{CJK}{UTF8}{mj}在其定义域上的连续性\end{CJK}.

\begin{CJK}{UTF8}{mj}三\end{CJK}、 (15 \begin{CJK}{UTF8}{mj}分\end{CJK}) \begin{CJK}{UTF8}{mj}设\end{CJK} $x_{n} \neq 0(n=1,2,3, \cdots)$, \begin{CJK}{UTF8}{mj}且\end{CJK} $\lim _{n \rightarrow \infty} \frac{x_{n}}{n}=2$, \begin{CJK}{UTF8}{mj}讨论级数\end{CJK} $\sum_{n=1}^{\infty}(-1)^{n}\left(\frac{x_{n+1}+x_{n}}{x_{n} x_{n+1}}\right)$ \begin{CJK}{UTF8}{mj}的敛散性\end{CJK}, \begin{CJK}{UTF8}{mj}包括条件\end{CJK} \begin{CJK}{UTF8}{mj}收敛和绝对收敛\end{CJK}.

\begin{CJK}{UTF8}{mj}四\end{CJK}、(15 \begin{CJK}{UTF8}{mj}分\end{CJK}) \begin{CJK}{UTF8}{mj}设\end{CJK} $f(x)$ \begin{CJK}{UTF8}{mj}在\end{CJK} $[a,+\infty)$ \begin{CJK}{UTF8}{mj}上连续\end{CJK}, \begin{CJK}{UTF8}{mj}且\end{CJK} $\int_{a}^{+\infty} f(x) \mathrm{d} x$ \begin{CJK}{UTF8}{mj}收敛\end{CJK}, \begin{CJK}{UTF8}{mj}证明\end{CJK}: \begin{CJK}{UTF8}{mj}存在点列\end{CJK} $\left\{x_{n}\right\} \subset[a,+\infty)$, \begin{CJK}{UTF8}{mj}使得\end{CJK} $\left\{x_{n}\right\}$ \begin{CJK}{UTF8}{mj}满足\end{CJK} $\lim _{n \rightarrow \infty} x_{n}=+\infty$ \begin{CJK}{UTF8}{mj}且\end{CJK} $\lim _{n \rightarrow \infty} f\left(x_{n}\right)=0 .$

\begin{CJK}{UTF8}{mj}五\end{CJK}、 ( 20 \begin{CJK}{UTF8}{mj}分\end{CJK}) \begin{CJK}{UTF8}{mj}证明函数项级数\end{CJK} $\sum_{n=1}^{\infty} \frac{x+n(-1)^{n}}{n^{2}+x^{2}}$ \begin{CJK}{UTF8}{mj}在\end{CJK} $|x|<1$ \begin{CJK}{UTF8}{mj}处连续\end{CJK}.

\begin{CJK}{UTF8}{mj}六\end{CJK}、(15 \begin{CJK}{UTF8}{mj}分\end{CJK}) \begin{CJK}{UTF8}{mj}确定\end{CJK} $\beta$ \begin{CJK}{UTF8}{mj}的值\end{CJK}, \begin{CJK}{UTF8}{mj}使得\end{CJK}
$$
f(x, y)= \begin{cases}\left(x^{2}+y^{2}\right)^{\beta} \cos \frac{1}{x^{4}+y^{4}}, & (x, y) \neq(0,0) \\ 0 . & (x, y)=(0,0)\end{cases}
$$
\begin{CJK}{UTF8}{mj}在点\end{CJK} $(0,0)$ \begin{CJK}{UTF8}{mj}可微\end{CJK}.

\begin{CJK}{UTF8}{mj}七\end{CJK}、 (20 \begin{CJK}{UTF8}{mj}分\end{CJK}) \begin{CJK}{UTF8}{mj}求两曲面\end{CJK} $x+2 y=1$ \begin{CJK}{UTF8}{mj}和\end{CJK} $x^{2}+2 y^{2}+z^{2}=1$ \begin{CJK}{UTF8}{mj}的交线上距离原点最近的点\end{CJK}.

\begin{CJK}{UTF8}{mj}八\end{CJK}、 (20 \begin{CJK}{UTF8}{mj}分\end{CJK}) \begin{CJK}{UTF8}{mj}求曲面积分\end{CJK}
$$
I=\iint_{S}\left(y^{2}-x\right) \mathrm{d} y \mathrm{~d} z+\left(z^{2}-y\right) \mathrm{d} z \mathrm{~d} x+\left(x^{2}-z\right) \mathrm{d} x \mathrm{~d} y,
$$
\begin{CJK}{UTF8}{mj}其中\end{CJK} $S$ \begin{CJK}{UTF8}{mj}是曲面\end{CJK} $z=2-x^{2}-y^{2}(1 \leqslant z \leqslant 2)$ \begin{CJK}{UTF8}{mj}的上侧\end{CJK}.

\section{5. 云南大学 2014 年研究生入学考试试题数学分析}
\begin{CJK}{UTF8}{mj}李扬\end{CJK}

\begin{CJK}{UTF8}{mj}微信公众号\end{CJK}: sxkyliyang

\begin{CJK}{UTF8}{mj}一\end{CJK}、\begin{CJK}{UTF8}{mj}填空题\end{CJK} (\begin{CJK}{UTF8}{mj}共\end{CJK} 6 \begin{CJK}{UTF8}{mj}题\end{CJK}, \begin{CJK}{UTF8}{mj}每题\end{CJK} 5 \begin{CJK}{UTF8}{mj}分\end{CJK}, \begin{CJK}{UTF8}{mj}共\end{CJK} 30 \begin{CJK}{UTF8}{mj}分\end{CJK})

\begin{enumerate}
  \item $\lim _{x \rightarrow 0}\left[\frac{\tan 3 x}{2 x}+(1+x)^{\frac{2}{x}}+x^{3} \cos \frac{1}{x^{2}}\right]=$

  \item \begin{CJK}{UTF8}{mj}设\end{CJK} $u=x^{2} y+e^{x^{2} y}$, \begin{CJK}{UTF8}{mj}则\end{CJK} $u$ \begin{CJK}{UTF8}{mj}在点\end{CJK} $(1,1)$ \begin{CJK}{UTF8}{mj}处的全微分\end{CJK} $\left.\mathrm{d} u\right|_{(1,1)}=$

  \item \begin{CJK}{UTF8}{mj}若定义\end{CJK} $f(0,0)=$ , \begin{CJK}{UTF8}{mj}则函数\end{CJK} $f(x, y)=\frac{\left(x^{3}+y^{2}\right) e^{x+y}}{\sqrt{x^{3}+y^{2}+4}-2}$ \begin{CJK}{UTF8}{mj}在点\end{CJK} $(0,0)$ \begin{CJK}{UTF8}{mj}连续\end{CJK}.

  \item \begin{CJK}{UTF8}{mj}若\end{CJK} $f(x)$ \begin{CJK}{UTF8}{mj}是周期为\end{CJK} 6 \begin{CJK}{UTF8}{mj}的周期函数\end{CJK}, \begin{CJK}{UTF8}{mj}它在\end{CJK} $(-3,3]$ \begin{CJK}{UTF8}{mj}上定义为\end{CJK} $f(x)=\left\{\begin{array}{ll}8 x, & -3<x \leqslant 0 ; \\ x^{2} . & 0<x \leqslant 3 .\end{array}\right.$, \begin{CJK}{UTF8}{mj}则\end{CJK} $f(x)$ \begin{CJK}{UTF8}{mj}的\end{CJK} Fourier \begin{CJK}{UTF8}{mj}级数在\end{CJK} $x=3$ \begin{CJK}{UTF8}{mj}处收敛于\end{CJK}

  \item \begin{CJK}{UTF8}{mj}若积分\end{CJK} $\int_{0}^{6} \frac{\sin 2 x}{x^{p}} \mathrm{~d} x$ \begin{CJK}{UTF8}{mj}收敛\end{CJK}, \begin{CJK}{UTF8}{mj}则参数\end{CJK} $p$ \begin{CJK}{UTF8}{mj}的范围为\end{CJK}

  \item \begin{CJK}{UTF8}{mj}四个平面\end{CJK} $x+y+z=1, x=0, y=0, z=0$ \begin{CJK}{UTF8}{mj}所围成的四面体的体积等于\end{CJK}

\end{enumerate}
\begin{CJK}{UTF8}{mj}二\end{CJK}、(\begin{CJK}{UTF8}{mj}共\end{CJK} 2 \begin{CJK}{UTF8}{mj}题\end{CJK}, \begin{CJK}{UTF8}{mj}第\end{CJK} 1 \begin{CJK}{UTF8}{mj}题\end{CJK} 8 \begin{CJK}{UTF8}{mj}分\end{CJK}, \begin{CJK}{UTF8}{mj}第\end{CJK} 2 \begin{CJK}{UTF8}{mj}题\end{CJK} 7 \begin{CJK}{UTF8}{mj}分\end{CJK}, \begin{CJK}{UTF8}{mj}共\end{CJK} 15 \begin{CJK}{UTF8}{mj}分\end{CJK})

\begin{enumerate}
  \item \begin{CJK}{UTF8}{mj}证明\end{CJK} $f(x)=\cos x$ \begin{CJK}{UTF8}{mj}在\end{CJK} $(-\infty,+\infty)$ \begin{CJK}{UTF8}{mj}上一致连续\end{CJK};

  \item \begin{CJK}{UTF8}{mj}证明\end{CJK} $g(x)=\cos x^{2}$ \begin{CJK}{UTF8}{mj}在\end{CJK} $(-\infty,+\infty)$ \begin{CJK}{UTF8}{mj}上不一致连续\end{CJK}.

\end{enumerate}
\begin{CJK}{UTF8}{mj}三\end{CJK}、 (15 \begin{CJK}{UTF8}{mj}分\end{CJK}) \begin{CJK}{UTF8}{mj}设\end{CJK} $x_{0}=4, x_{n}=\frac{1}{2}\left(x_{n-1}+\frac{4}{x_{n-1}}\right)(n=1,2, \cdots)$, \begin{CJK}{UTF8}{mj}证明数列\end{CJK} $\left\{x_{n}\right\}$ \begin{CJK}{UTF8}{mj}的极限存在\end{CJK}, \begin{CJK}{UTF8}{mj}并求出该极限\end{CJK}.

\begin{CJK}{UTF8}{mj}四\end{CJK}、 $\left(15\right.$ \begin{CJK}{UTF8}{mj}分\end{CJK}) \begin{CJK}{UTF8}{mj}证明\end{CJK}: \begin{CJK}{UTF8}{mj}当\end{CJK} $x>0$ \begin{CJK}{UTF8}{mj}时\end{CJK}, $e^{x}>1+(1+x) \ln (1+x)$.

\begin{CJK}{UTF8}{mj}五\end{CJK}、 (15 \begin{CJK}{UTF8}{mj}分\end{CJK}) \begin{CJK}{UTF8}{mj}求不定积分\end{CJK} $\int e^{\max [1, x]} \mathrm{d} x$.

\begin{CJK}{UTF8}{mj}六\end{CJK}、( 15 \begin{CJK}{UTF8}{mj}分\end{CJK}) \begin{CJK}{UTF8}{mj}证明函数\end{CJK} $z=\left(1+e^{y}\right) \cos x-y e^{y}$ \begin{CJK}{UTF8}{mj}有无穷多个极大值点\end{CJK}, \begin{CJK}{UTF8}{mj}但无极小值点\end{CJK}.

\begin{CJK}{UTF8}{mj}七\end{CJK}、(15 \begin{CJK}{UTF8}{mj}分\end{CJK}) \begin{CJK}{UTF8}{mj}求幂级数\end{CJK} $\sum_{n=1}^{\infty} \frac{(-1)^{n+1}}{4 n^{2}+2 n} x^{2 n+1}$ \begin{CJK}{UTF8}{mj}的和函数\end{CJK}, \begin{CJK}{UTF8}{mj}并求级数\end{CJK} $\sum_{n=1}^{\infty} \frac{(-1)^{n+1}}{4 n^{2}+2 n}$ \begin{CJK}{UTF8}{mj}的和\end{CJK}.

\begin{CJK}{UTF8}{mj}八\end{CJK}、(15 \begin{CJK}{UTF8}{mj}分\end{CJK}) \begin{CJK}{UTF8}{mj}计算第一类曲面积分\end{CJK} $\iint_{S} z \mathrm{~d} S$, \begin{CJK}{UTF8}{mj}其中\end{CJK} $S$ \begin{CJK}{UTF8}{mj}为雉面\end{CJK} $z=\sqrt{x^{2}+y^{2}}$ \begin{CJK}{UTF8}{mj}被柱面\end{CJK} $x^{2}+y^{2}=4$ \begin{CJK}{UTF8}{mj}所割下的部分\end{CJK}.

\begin{CJK}{UTF8}{mj}九\end{CJK}、 $\left(15\right.$ \begin{CJK}{UTF8}{mj}分\end{CJK}) \begin{CJK}{UTF8}{mj}设\end{CJK} $f(x)$ \begin{CJK}{UTF8}{mj}在\end{CJK} $(-\infty,+\infty)$ \begin{CJK}{UTF8}{mj}上可微\end{CJK}, \begin{CJK}{UTF8}{mj}并且对任意的\end{CJK} $x \in(-\infty,+\infty)$, \begin{CJK}{UTF8}{mj}有\end{CJK} $f(x)>0,\left|f^{\prime}(x)\right| \leqslant \theta|f(x)|$, \begin{CJK}{UTF8}{mj}其中\end{CJK} $\theta \in(0,1)$. \begin{CJK}{UTF8}{mj}任取实数\end{CJK} $a_{0}$, \begin{CJK}{UTF8}{mj}定义\end{CJK} $a_{n}=\ln f\left(a_{n-1}\right)(n=1,2, \cdots)$, \begin{CJK}{UTF8}{mj}证明级数\end{CJK} $\sum_{n=1}^{\infty}\left|a_{n}-a_{n-1}\right|$ \begin{CJK}{UTF8}{mj}收敛\end{CJK}.

\section{6. 云南大学 2015 年研究生入学考试试题数学分析}
\begin{CJK}{UTF8}{mj}李扬\end{CJK}

\begin{CJK}{UTF8}{mj}微信公众号\end{CJK}: sxkyliyang

\begin{CJK}{UTF8}{mj}一\end{CJK}、\begin{CJK}{UTF8}{mj}填空题\end{CJK} (\begin{CJK}{UTF8}{mj}共\end{CJK} 6 \begin{CJK}{UTF8}{mj}题\end{CJK}, \begin{CJK}{UTF8}{mj}每题\end{CJK} 5 \begin{CJK}{UTF8}{mj}分\end{CJK}, \begin{CJK}{UTF8}{mj}共\end{CJK} 30 \begin{CJK}{UTF8}{mj}分\end{CJK})

\begin{enumerate}
  \item $\lim _{n \rightarrow \infty}\left(\frac{n}{n-1}\right)^{2-n}=$
\end{enumerate}
2 . \begin{CJK}{UTF8}{mj}设\end{CJK} $F(y)=\int_{0}^{y}(5 x+y) \cos x^{2} \mathrm{~d} x$, \begin{CJK}{UTF8}{mj}则\end{CJK} $F^{\prime}(y)=$

\begin{enumerate}
  \setcounter{enumi}{3}
  \item \begin{CJK}{UTF8}{mj}幂级数\end{CJK} $\sum_{n=1}^{\infty}\left(1+\frac{1}{n}\right)^{n(n+1)} x^{n}$ \begin{CJK}{UTF8}{mj}的收敛域为\end{CJK}

  \item \begin{CJK}{UTF8}{mj}求\end{CJK} $e^{-x}$ \begin{CJK}{UTF8}{mj}在\end{CJK} $x=0$ \begin{CJK}{UTF8}{mj}处的极大值为\end{CJK}

  \item \begin{CJK}{UTF8}{mj}广义积分\end{CJK} $\int_{0}^{+\infty} \frac{x e^{-x}}{\left(1+e^{-x}\right)^{2}} \mathrm{~d} x=$

\end{enumerate}
\begin{CJK}{UTF8}{mj}二\end{CJK}、 ( 15 \begin{CJK}{UTF8}{mj}分\end{CJK}) \begin{CJK}{UTF8}{mj}求不定积分\end{CJK} $\int \frac{\cos x \sin ^{3} x}{1+\cos ^{2} x} \mathrm{~d} x$.

\begin{CJK}{UTF8}{mj}三\end{CJK}、(15 \begin{CJK}{UTF8}{mj}分\end{CJK}) \begin{CJK}{UTF8}{mj}证明\end{CJK}: $\left\{x_{n}\right\}$ \begin{CJK}{UTF8}{mj}收敛\end{CJK}, \begin{CJK}{UTF8}{mj}其中\end{CJK} $x_{1}=1, x_{n+1}=\frac{1}{2}\left(x_{n}+\frac{1}{x_{n}}\right), n=1,2, \cdots$, \begin{CJK}{UTF8}{mj}并求\end{CJK} $\lim _{n \rightarrow \infty} x_{n}$.

\begin{CJK}{UTF8}{mj}四\end{CJK}、 (15 \begin{CJK}{UTF8}{mj}分\end{CJK}) \begin{CJK}{UTF8}{mj}证明\end{CJK}: $f(x)=\sum_{n=1}^{\infty} \frac{\sin n x}{n^{3}}$ \begin{CJK}{UTF8}{mj}在\end{CJK} $(-\infty,+\infty)$ \begin{CJK}{UTF8}{mj}上连续\end{CJK}.

\begin{CJK}{UTF8}{mj}五\end{CJK}、 ( 15 \begin{CJK}{UTF8}{mj}分\end{CJK}) \begin{CJK}{UTF8}{mj}求级数\end{CJK} $\sum_{n=1}^{\infty} \frac{n}{3^{n}} 2^{n}$ \begin{CJK}{UTF8}{mj}的和\end{CJK}.

\begin{CJK}{UTF8}{mj}六\end{CJK}、 $\left(20\right.$ \begin{CJK}{UTF8}{mj}分\end{CJK}) \begin{CJK}{UTF8}{mj}设\end{CJK} $f(x)=\left\{\begin{array}{ll}x, & 0 \leqslant x \leqslant \pi ; \\ 0 . & -\pi<x<0 .\end{array}\right.$, \begin{CJK}{UTF8}{mj}求\end{CJK} $f(x)$ \begin{CJK}{UTF8}{mj}的傅里叶级数展开式\end{CJK}.

\begin{CJK}{UTF8}{mj}七\end{CJK}、 $\left(20\right.$ \begin{CJK}{UTF8}{mj}分\end{CJK}) \begin{CJK}{UTF8}{mj}讨论\end{CJK} $f(x)=\left\{\begin{array}{ll}\frac{\sin x y}{\sqrt{x^{2}+y^{2}}}, & x^{2}+y^{2} \neq 0 ; \\ 0 . & x^{2}+y^{2}=0 .\end{array}\right.$ \begin{CJK}{UTF8}{mj}在\end{CJK} $(0,0)$ \begin{CJK}{UTF8}{mj}处的可微性\end{CJK}.

\begin{CJK}{UTF8}{mj}八\end{CJK}、 $(20$ \begin{CJK}{UTF8}{mj}分\end{CJK} $)$ \begin{CJK}{UTF8}{mj}计算\end{CJK}
$$
\iint_{\Sigma} x^{2} \mathrm{~d} y \mathrm{~d} z+y^{2} \mathrm{~d} z \mathrm{~d} x+z^{2} \mathrm{~d} x \mathrm{~d} y
$$
\begin{CJK}{UTF8}{mj}其中\end{CJK} $\Sigma: x^{2}+y^{2}+z^{2}=1(z \geqslant 0)$, \begin{CJK}{UTF8}{mj}取外侧\end{CJK}.

\section{7. 云南大学 2016 年研究生入学考试试题数学分析}
\begin{CJK}{UTF8}{mj}李扬\end{CJK}

\begin{CJK}{UTF8}{mj}微信公众号\end{CJK}: sxkyliyang

\begin{CJK}{UTF8}{mj}一\end{CJK}、\begin{CJK}{UTF8}{mj}填空题\end{CJK}

\begin{enumerate}
  \item $\lim _{n \rightarrow \infty}\left(\frac{3 n}{\sqrt{n^{4}+1}}+\frac{3 n}{\sqrt{n^{4}+2}}+\cdots+\frac{3 n}{\sqrt{n^{4}+n}}\right)=$

  \item \begin{CJK}{UTF8}{mj}已知\end{CJK} $\int e^{x} f(x) \mathrm{d} x=x e^{x}+C, C$ \begin{CJK}{UTF8}{mj}为常数\end{CJK}, \begin{CJK}{UTF8}{mj}则\end{CJK} $\int \frac{\mathrm{d} x}{f(x)}=$

  \item \begin{CJK}{UTF8}{mj}由\end{CJK} $y^{2}=1-x, 2 y=x+1$ \begin{CJK}{UTF8}{mj}所围成的图形的面积为\end{CJK}

  \item $u=x y-y^{2} z+e^{x}, u$ \begin{CJK}{UTF8}{mj}从点\end{CJK} $(1,0,2)$ \begin{CJK}{UTF8}{mj}到\end{CJK} $(2,1,-1)$ \begin{CJK}{UTF8}{mj}的方向导数是\end{CJK}

  \item $\int_{0}^{+\infty} e^{-4 x^{2}} \mathrm{~d} x=$

\end{enumerate}
\begin{CJK}{UTF8}{mj}二\end{CJK}、\begin{CJK}{UTF8}{mj}求极限\end{CJK} $\lim _{x \rightarrow 0} \frac{x e^{x}-\ln (1+x)}{x^{2}}$.

\begin{CJK}{UTF8}{mj}三\end{CJK}、\begin{CJK}{UTF8}{mj}证明\end{CJK}: $p>1, x \in[0,1]$, \begin{CJK}{UTF8}{mj}则\end{CJK} $\frac{1}{2^{p-1}} \leqslant x^{p}+(1-x)^{p} \leqslant 1$.

\begin{CJK}{UTF8}{mj}四\end{CJK}、\begin{CJK}{UTF8}{mj}证明\end{CJK}: \begin{CJK}{UTF8}{mj}设\end{CJK} $f(x, y)=\left\{\begin{array}{l}\left(x^{2}+y^{2}\right) \cos \frac{1}{\sqrt{x^{2}+y^{2}}}, \\ 0 .\end{array}\right.$

\begin{CJK}{UTF8}{mj}五\end{CJK}、\begin{CJK}{UTF8}{mj}判断级数\end{CJK} $\sum_{n=1}^{\infty}(-1)^{n+1} \frac{\ln n}{\sqrt{n}}$ \begin{CJK}{UTF8}{mj}的敛散性\end{CJK} (\begin{CJK}{UTF8}{mj}条件收敛还是绝对收敛\end{CJK}).

\begin{CJK}{UTF8}{mj}六\end{CJK}. \begin{CJK}{UTF8}{mj}证明\end{CJK} $f(x)=\sum_{n=1}^{\infty} \frac{1}{n^{2 x}}$ \begin{CJK}{UTF8}{mj}在\end{CJK} $\left(\frac{1}{2},+\infty\right)$ \begin{CJK}{UTF8}{mj}上连续\end{CJK}.

\begin{CJK}{UTF8}{mj}七\end{CJK}. \begin{CJK}{UTF8}{mj}计算三重积分\end{CJK}
$$
\iiint_{V} x^{2} \sqrt{x^{2}+y^{2}} \mathrm{~d} x \mathrm{~d} y \mathrm{~d} z
$$
$V$ \begin{CJK}{UTF8}{mj}是由\end{CJK} $z=\sqrt{x^{2}+y^{2}} 与 z=x^{2}+y^{2}$ \begin{CJK}{UTF8}{mj}所围成的区域\end{CJK}.

\begin{CJK}{UTF8}{mj}八\end{CJK}. \begin{CJK}{UTF8}{mj}计算积分\end{CJK}
$$
\int_{\widehat{A M O}}\left(e^{x} \sin y-4 y\right) \mathrm{d} x+\left(e^{x} \cos y-4\right) \mathrm{d} y,
$$
$\widehat{A M O}$ \begin{CJK}{UTF8}{mj}是从\end{CJK} $(2,0)$ \begin{CJK}{UTF8}{mj}经过上半圆\end{CJK} $x^{2}+y^{2}=2 x$ \begin{CJK}{UTF8}{mj}到点\end{CJK} $O(0,0)$ \begin{CJK}{UTF8}{mj}的路程\end{CJK}.

\begin{CJK}{UTF8}{mj}九\end{CJK}. $T>0, f(x)$ \begin{CJK}{UTF8}{mj}是\end{CJK} $[0,+\infty)$ \begin{CJK}{UTF8}{mj}上周期为\end{CJK} $T$ \begin{CJK}{UTF8}{mj}的连续函数\end{CJK}, \begin{CJK}{UTF8}{mj}证明\end{CJK} $\lim _{x \rightarrow+\infty} \frac{1}{x} \int_{0}^{x} f(t) \mathrm{d} t=\frac{1}{T} \int_{0}^{T} f(t) \mathrm{d} t$.

\section{8. 云南大学 2017 年研究生入学考试试题数学分析}
\begin{CJK}{UTF8}{mj}李扬\end{CJK}

\begin{CJK}{UTF8}{mj}微信公众号\end{CJK}: sxkyliyang

\begin{CJK}{UTF8}{mj}一\end{CJK}、\begin{CJK}{UTF8}{mj}填空题\end{CJK}

\begin{enumerate}
  \item \begin{CJK}{UTF8}{mj}求极限\end{CJK} $\lim _{x \rightarrow 0} \frac{\sqrt{1+2 \sin x}-x-1}{x \ln (1+x)}=$

  \item $\int_{-2}^{1} \max \left\{1, x^{2}\right\} \mathrm{d} x=$

  \item $e^{\frac{x}{z}}+e^{\frac{y}{z}}=4$, \begin{CJK}{UTF8}{mj}求点\end{CJK} $(\ln 2, \ln 2,1)$ \begin{CJK}{UTF8}{mj}处的法线方程\end{CJK}

  \item \begin{CJK}{UTF8}{mj}求\end{CJK} $x^{2}+y^{2}=\frac{z^{2}}{3}$ \begin{CJK}{UTF8}{mj}与\end{CJK} $x+y+z=2$ \begin{CJK}{UTF8}{mj}所界部分表面面积\end{CJK}

  \item $u=x y+e^{y}$, \begin{CJK}{UTF8}{mj}求\end{CJK} $\frac{\partial u}{\partial y}=$

\end{enumerate}
\begin{CJK}{UTF8}{mj}二\end{CJK}、 \begin{CJK}{UTF8}{mj}求极限\end{CJK} $\lim _{n \rightarrow \infty} \frac{1+\sqrt{2}+\sqrt[3]{3}+\cdots+\sqrt[n]{n}}{n}$.

\begin{CJK}{UTF8}{mj}三\end{CJK}、 $f_{n}(x)=\sin x+\sin ^{2} x+\cdots+\sin ^{n} x$

\begin{enumerate}
  \item \begin{CJK}{UTF8}{mj}证明\end{CJK}: $\forall$ \begin{CJK}{UTF8}{mj}正整数\end{CJK} $n, f_{n}(x)=1$ \begin{CJK}{UTF8}{mj}在\end{CJK} $\left(\frac{\pi}{6}, \frac{\pi}{2}\right]$ \begin{CJK}{UTF8}{mj}上有且只有一根\end{CJK};

  \item $x_{n} \in\left(\frac{\pi}{6}, \frac{\pi}{2}\right]$ \begin{CJK}{UTF8}{mj}是\end{CJK} $f_{n}(x)=1$ \begin{CJK}{UTF8}{mj}的根\end{CJK}, \begin{CJK}{UTF8}{mj}证明\end{CJK} $\lim _{n \rightarrow \infty} x_{n}=\frac{\pi}{6}$.

\end{enumerate}
\begin{CJK}{UTF8}{mj}四\end{CJK}、 $f(x)=n x(1-x)^{n}$, \begin{CJK}{UTF8}{mj}求\end{CJK} $f(x)$ \begin{CJK}{UTF8}{mj}在\end{CJK} $[0,1]$ \begin{CJK}{UTF8}{mj}上的最大值\end{CJK} $M(n)$, \begin{CJK}{UTF8}{mj}求\end{CJK} $\lim _{n \rightarrow \infty} M(n)$.

\begin{CJK}{UTF8}{mj}五\end{CJK}、\begin{CJK}{UTF8}{mj}求\end{CJK} $\int_{1}^{+\infty} \frac{\cos x}{x^{p}} \mathrm{~d} x$ \begin{CJK}{UTF8}{mj}的敛散性\end{CJK}. (\begin{CJK}{UTF8}{mj}包括绝对收敛和条件收敛\end{CJK})

\begin{CJK}{UTF8}{mj}六\end{CJK}、\begin{CJK}{UTF8}{mj}求\end{CJK} $f(x)=\left\{\begin{array}{ll}2, & 0 \leqslant x \leqslant 1 ; \\ 1 . & 1<x \leqslant 2 .\end{array}\right.$ \begin{CJK}{UTF8}{mj}的余弦级数及和函数\end{CJK}.

\begin{CJK}{UTF8}{mj}七\end{CJK}、\begin{CJK}{UTF8}{mj}求二重积分\end{CJK}
$$
\iint_{D}(\sqrt{x}+\sqrt{y}) \mathrm{d} x \mathrm{~d} y
$$
$D$ \begin{CJK}{UTF8}{mj}是由曲线\end{CJK} $\sqrt{x}+\sqrt{y}=1, x=0$ \begin{CJK}{UTF8}{mj}及\end{CJK} $y=0$ \begin{CJK}{UTF8}{mj}围成的\end{CJK}.

\begin{CJK}{UTF8}{mj}八\end{CJK}、 $f(x)$ \begin{CJK}{UTF8}{mj}在\end{CJK} $[1,+\infty)$ \begin{CJK}{UTF8}{mj}上连续\end{CJK}, \begin{CJK}{UTF8}{mj}且\end{CJK} $\lim _{x \rightarrow+\infty} f(x)=A$, \begin{CJK}{UTF8}{mj}证\end{CJK} $f(x)$ \begin{CJK}{UTF8}{mj}在\end{CJK} $[1,+\infty)$ \begin{CJK}{UTF8}{mj}上有界\end{CJK}.

\begin{CJK}{UTF8}{mj}九\end{CJK}、 $f(x)=\frac{1}{(2+x)\left(\frac{1}{2}-x\right)}$, \begin{CJK}{UTF8}{mj}求\end{CJK} $f^{(n)}(x)$, \begin{CJK}{UTF8}{mj}并证明\end{CJK} $\sum_{n=1}^{\infty} \frac{n !}{f^{(n)}(0)}$ \begin{CJK}{UTF8}{mj}收敛\end{CJK}.

\section{1. 浙江大学 2009 年研究生入学考试试题高等代数 
 李扬 
 微信公众号: sxkyliyang}
\begin{CJK}{UTF8}{mj}注\end{CJK}: \begin{CJK}{UTF8}{mj}本试卷满分为\end{CJK} 150 \begin{CJK}{UTF8}{mj}分\end{CJK}, \begin{CJK}{UTF8}{mj}共计\end{CJK} 10 \begin{CJK}{UTF8}{mj}道题\end{CJK}, \begin{CJK}{UTF8}{mj}每题满分\end{CJK} 15 \begin{CJK}{UTF8}{mj}分\end{CJK}, \begin{CJK}{UTF8}{mj}考试时间总计\end{CJK} 180 \begin{CJK}{UTF8}{mj}分钟\end{CJK}.

\begin{enumerate}
  \item \begin{CJK}{UTF8}{mj}设\end{CJK} $P$ \begin{CJK}{UTF8}{mj}是数域\end{CJK}, \begin{CJK}{UTF8}{mj}在\end{CJK} $n$ \begin{CJK}{UTF8}{mj}个变元的多项式环\end{CJK} $P\left[x_{1}, \cdots, x_{n}\right]$ \begin{CJK}{UTF8}{mj}中引入对第\end{CJK} $k$ \begin{CJK}{UTF8}{mj}个变元的偏导子\end{CJK}, \begin{CJK}{UTF8}{mj}由下列式子定义\end{CJK}:
\end{enumerate}
$$
\frac{\partial}{\partial x_{k}}\left(\sum_{i_{1} \cdots i_{n}} a_{i_{1} \cdots i_{n}} x_{1}^{i_{1}} \cdots x_{k}^{i_{k}} \cdots x_{n}^{i_{n}}\right)=\sum_{i_{1} \cdots i_{n}} i_{k} a_{i_{1} \cdots i_{n}} x_{1}^{i_{1}} \cdots x_{k}^{i_{k}-1} \cdots x_{n}^{i_{n}} .
$$
\begin{CJK}{UTF8}{mj}其中\end{CJK} $a_{i_{1} \cdots i_{n}}$ \begin{CJK}{UTF8}{mj}是数域\end{CJK} $P$ \begin{CJK}{UTF8}{mj}中的任意数\end{CJK}.

(1) \begin{CJK}{UTF8}{mj}证明\end{CJK}: \begin{CJK}{UTF8}{mj}在\end{CJK} $\frac{\partial}{\partial x_{k}}$ \begin{CJK}{UTF8}{mj}下取零值的多项式的集合是\end{CJK} $n-1$ \begin{CJK}{UTF8}{mj}个变元的多项式环\end{CJK} $P\left[x_{1}, \cdots, x_{k-1}, x_{k+1}, \cdots, x_{n}\right]$;

(2) \begin{CJK}{UTF8}{mj}设\end{CJK} $f\left(x_{1}, \cdots, x_{n}\right)$ \begin{CJK}{UTF8}{mj}是一个\end{CJK} $m$ \begin{CJK}{UTF8}{mj}次齐次多项式\end{CJK}, \begin{CJK}{UTF8}{mj}证明\end{CJK}:
$$
\sum_{k=1}^{n} x_{k} \frac{\partial f}{\partial x_{k}}=m f\left(x_{1}, \cdots, x_{n}\right) .
$$
\begin{CJK}{UTF8}{mj}该式称为欧拉恒等式\end{CJK}. \begin{CJK}{UTF8}{mj}反之\end{CJK}, \begin{CJK}{UTF8}{mj}证明\end{CJK}: \begin{CJK}{UTF8}{mj}对任意正整数\end{CJK} $m$, \begin{CJK}{UTF8}{mj}满足欧拉恒等式的多项式必为\end{CJK} $m$ \begin{CJK}{UTF8}{mj}次齐次多项式\end{CJK}.

\begin{enumerate}
  \setcounter{enumi}{2}
  \item \begin{CJK}{UTF8}{mj}设\end{CJK}
\end{enumerate}
$$
A=\left(\begin{array}{cccc}
2 a_{1} b_{1} & a_{1} b_{2}+a_{2} b_{1} & \cdots & a_{1} b_{n}+a_{n} b_{1} \\
a_{2} b_{1}+a_{1} b_{2} & 2 a_{2} b_{2} & \cdots & a_{2} b_{n}+a_{n} b_{2} \\
\vdots & \vdots & & \vdots \\
a_{n} b_{1}+a_{1} b_{n} & a_{n} b_{2}+a_{2} b_{n} & \cdots & 2 a_{n} b_{n}
\end{array}\right)
$$
\begin{CJK}{UTF8}{mj}计算\end{CJK} $A$ \begin{CJK}{UTF8}{mj}的行列式\end{CJK}.

\begin{enumerate}
  \setcounter{enumi}{3}
  \item \begin{CJK}{UTF8}{mj}设\end{CJK}
\end{enumerate}
$$
A=\left(\begin{array}{ccccc}
1 & -1 & 0 & -1 & -2 \\
-1 & 2 & 1 & 3 & 6 \\
0 & 1 & 1 & 2 & 4 \\
0 & -1 & -1 & 1 & 2
\end{array}\right)
$$
$\mathbb{R}^{5 \times 2}$ \begin{CJK}{UTF8}{mj}表示实数域\end{CJK} $\mathbb{R}$ \begin{CJK}{UTF8}{mj}上所有\end{CJK} $5 \times 2$ \begin{CJK}{UTF8}{mj}阶矩阵组成的线性空间\end{CJK}, $W=\left\{B \in \mathbb{R}^{5 \times 2} \mid A B=0\right\}$, \begin{CJK}{UTF8}{mj}证明\end{CJK}: $W$ \begin{CJK}{UTF8}{mj}是\end{CJK} $\mathbb{R}^{5 \times 2}$ \begin{CJK}{UTF8}{mj}的\end{CJK} \begin{CJK}{UTF8}{mj}子空间\end{CJK}, \begin{CJK}{UTF8}{mj}并求出它在\end{CJK} $\mathbb{R}$ \begin{CJK}{UTF8}{mj}上维数\end{CJK}.

\begin{enumerate}
  \setcounter{enumi}{4}
  \item \begin{CJK}{UTF8}{mj}设\end{CJK} $\alpha, \beta, \gamma, \delta$ \begin{CJK}{UTF8}{mj}是实数\end{CJK}, \begin{CJK}{UTF8}{mj}给出存在一个次数不超过\end{CJK} 2 \begin{CJK}{UTF8}{mj}的实系数多项式\end{CJK} $f(x)$ \begin{CJK}{UTF8}{mj}使得满足\end{CJK} $f(-1)=\alpha, f(1)=$ $\beta, f(3)=\gamma, f(0)=\delta$ \begin{CJK}{UTF8}{mj}的充要条件\end{CJK}.

  \item \begin{CJK}{UTF8}{mj}设\end{CJK} $x, y$ \begin{CJK}{UTF8}{mj}是两个非零实数\end{CJK}, $\Phi, \Psi$ \begin{CJK}{UTF8}{mj}是实数域上某个\end{CJK} $n$ \begin{CJK}{UTF8}{mj}维线性空间上的两个线性变换\end{CJK}, \begin{CJK}{UTF8}{mj}满足\end{CJK} $\Phi \circ \Psi=x \Phi+y \Psi$, \begin{CJK}{UTF8}{mj}证\end{CJK} \begin{CJK}{UTF8}{mj}明\end{CJK}:

\end{enumerate}
$$
\Phi \circ \Psi=\Psi \circ \Phi .
$$
\begin{CJK}{UTF8}{mj}这里\end{CJK} $\Phi \circ \Psi$ \begin{CJK}{UTF8}{mj}表示\end{CJK} $\Phi$ \begin{CJK}{UTF8}{mj}与\end{CJK} $\Psi$ \begin{CJK}{UTF8}{mj}从右到左的线性变换合成\end{CJK}.

\begin{enumerate}
  \setcounter{enumi}{6}
  \item \begin{CJK}{UTF8}{mj}设\end{CJK} $A$ \begin{CJK}{UTF8}{mj}是一个\end{CJK} $n$ \begin{CJK}{UTF8}{mj}阶实对称矩阵\end{CJK}, \begin{CJK}{UTF8}{mj}证明\end{CJK}: \begin{CJK}{UTF8}{mj}存在某个充分大的实数\end{CJK} $a$, \begin{CJK}{UTF8}{mj}使得\end{CJK} $\mathbb{R}^{n}$ \begin{CJK}{UTF8}{mj}关于运算\end{CJK} $(\alpha, \beta)=\alpha^{T}(A+a E) \beta$ \begin{CJK}{UTF8}{mj}构成\end{CJK} \begin{CJK}{UTF8}{mj}一个欧式空间\end{CJK}, \begin{CJK}{UTF8}{mj}其中\end{CJK} $\alpha^{T}$ \begin{CJK}{UTF8}{mj}表示列向量\end{CJK} $\alpha$ \begin{CJK}{UTF8}{mj}的转置\end{CJK}, $E$ \begin{CJK}{UTF8}{mj}是\end{CJK} $n$ \begin{CJK}{UTF8}{mj}阶单位矩阵\end{CJK}.

  \item \begin{CJK}{UTF8}{mj}设\end{CJK} $A$ \begin{CJK}{UTF8}{mj}是\end{CJK} $n$ \begin{CJK}{UTF8}{mj}阶复方阵\end{CJK}, \begin{CJK}{UTF8}{mj}零是\end{CJK} $A$ \begin{CJK}{UTF8}{mj}的\end{CJK} $k$ \begin{CJK}{UTF8}{mj}重特征值\end{CJK}, \begin{CJK}{UTF8}{mj}求证\end{CJK}: \begin{CJK}{UTF8}{mj}秩\end{CJK} $\left(A^{k}\right)=n-k$.

  \item \begin{CJK}{UTF8}{mj}用正交变换将矩阵\end{CJK}

\end{enumerate}
$$
A=\left(\begin{array}{ccc}
-1 & 3 & -3 \\
3 & -1 & -3 \\
-3 & -3 & 5
\end{array}\right)
$$
\begin{CJK}{UTF8}{mj}化成对角矩阵\end{CJK}, \begin{CJK}{UTF8}{mj}并求\end{CJK} $A^{3}+3 A^{2}+4 A+6 E$, \begin{CJK}{UTF8}{mj}其中\end{CJK} $E$ \begin{CJK}{UTF8}{mj}是\end{CJK} 3 \begin{CJK}{UTF8}{mj}阶单位矩阵\end{CJK}. 9. \begin{CJK}{UTF8}{mj}对\end{CJK} $n$ \begin{CJK}{UTF8}{mj}维欧式空间\end{CJK} $V$ \begin{CJK}{UTF8}{mj}上的线性变换\end{CJK} $\psi$, \begin{CJK}{UTF8}{mj}若存在固定的单位向量\end{CJK} $\eta \in V$, \begin{CJK}{UTF8}{mj}使对\end{CJK} $\alpha \in V$, \begin{CJK}{UTF8}{mj}有\end{CJK} $\psi(\alpha)=\alpha-2(\alpha, \eta) \eta$, \begin{CJK}{UTF8}{mj}则\end{CJK} \begin{CJK}{UTF8}{mj}称\end{CJK} $\psi$ \begin{CJK}{UTF8}{mj}是\end{CJK} $V$ \begin{CJK}{UTF8}{mj}上的镜面反射\end{CJK}, $\psi$ \begin{CJK}{UTF8}{mj}在\end{CJK} $V$ \begin{CJK}{UTF8}{mj}的标准正交基下的矩阵称为镜面反射矩阵\end{CJK}. \begin{CJK}{UTF8}{mj}证明\end{CJK}: $n$ \begin{CJK}{UTF8}{mj}阶实方阵\end{CJK} $A$ \begin{CJK}{UTF8}{mj}是镜面反射\end{CJK} \begin{CJK}{UTF8}{mj}矩阵当且仅当存在单位向量\end{CJK} $w=\left(w_{1}, \cdots, w_{n}\right)^{T} \in \mathbb{R}^{n}$, \begin{CJK}{UTF8}{mj}使得\end{CJK} $A=E_{n}-2 w w^{T}$.

10 . \begin{CJK}{UTF8}{mj}设\end{CJK} $A$ \begin{CJK}{UTF8}{mj}是\end{CJK} $n$ \begin{CJK}{UTF8}{mj}阶复方阵\end{CJK},

(1) \begin{CJK}{UTF8}{mj}证明\end{CJK}: $A$ \begin{CJK}{UTF8}{mj}的最小多项式等于\end{CJK} $A$ \begin{CJK}{UTF8}{mj}的特征矩阵\end{CJK} $\lambda E-A$ \begin{CJK}{UTF8}{mj}的最高次不变因子\end{CJK};

(2) \begin{CJK}{UTF8}{mj}求\end{CJK}
$$
A=\left(\begin{array}{ccc}
-1 & -2 & 6 \\
-1 & 0 & 3 \\
-1 & -1 & 4
\end{array}\right)
$$
\begin{CJK}{UTF8}{mj}的最小多项式\end{CJK}.

\section{2. 浙江大学 2010 年研究生入学考试试题高等代数 
 李扬 
 微信公众号: sxkyliyang}
\begin{CJK}{UTF8}{mj}注\end{CJK}: \begin{CJK}{UTF8}{mj}本试卷满分为\end{CJK} 150 \begin{CJK}{UTF8}{mj}分\end{CJK}, \begin{CJK}{UTF8}{mj}共计\end{CJK} 10 \begin{CJK}{UTF8}{mj}道题\end{CJK}, \begin{CJK}{UTF8}{mj}每题满分\end{CJK} 15 \begin{CJK}{UTF8}{mj}分\end{CJK}, \begin{CJK}{UTF8}{mj}考试时间总计\end{CJK} 180 \begin{CJK}{UTF8}{mj}分钟\end{CJK}.

\begin{enumerate}
  \item \begin{CJK}{UTF8}{mj}设多项式\end{CJK} $f_{1}(x), \cdots, f_{k}(x)$ \begin{CJK}{UTF8}{mj}的最大公因式等于\end{CJK} $1, A \in P^{n \times n}, X \in P^{n \times 1}$. \begin{CJK}{UTF8}{mj}求证\end{CJK}: \begin{CJK}{UTF8}{mj}如果对于\end{CJK} $1 \leqslant i \leqslant k$, \begin{CJK}{UTF8}{mj}总成立\end{CJK} $f_{i}(A) X=0$, \begin{CJK}{UTF8}{mj}则\end{CJK} $X=0$.

  \item \begin{CJK}{UTF8}{mj}设\end{CJK} $\beta_{1}=\left(\begin{array}{c}2 \\ 3 \\ 3 \\ -1\end{array}\right), \beta_{2}=\left(\begin{array}{l}1 \\ 1 \\ 2 \\ 0\end{array}\right), \beta_{3}=\left(\begin{array}{c}0 \\ 2 \\ -1 \\ -1\end{array}\right), \beta_{4}=\left(\begin{array}{c}0 \\ -1 \\ 2 \\ 2\end{array}\right)$. \begin{CJK}{UTF8}{mj}而\end{CJK} $\alpha_{1}=\left(\begin{array}{c}3 \\ 8 \\ 3 \\ -3\end{array}\right), \alpha_{2}=\left(\begin{array}{c}2 \\ 5 \\ 2 \\ -2\end{array}\right), \alpha_{3}=\left(\begin{array}{c}-1 \\ 4 \\ -4 \\ -2\end{array}\right)$ \begin{CJK}{UTF8}{mj}是由\end{CJK} $\beta_{1}, \beta_{2}, \beta_{3}, \beta_{4}$ \begin{CJK}{UTF8}{mj}张成的线性空间\end{CJK} $V=L\left(\beta_{1}, \beta_{2}, \beta_{3}, \beta_{4}\right)$ \begin{CJK}{UTF8}{mj}的一组基\end{CJK} (I).

\end{enumerate}
(1) \begin{CJK}{UTF8}{mj}求出该空间\end{CJK} $V$ \begin{CJK}{UTF8}{mj}由\end{CJK} $\beta_{1}, \beta_{2}, \beta_{3}, \beta_{4}$ \begin{CJK}{UTF8}{mj}中一部分向量组成的一组基\end{CJK} (II);

(2) \begin{CJK}{UTF8}{mj}并求出由基\end{CJK} (I) \begin{CJK}{UTF8}{mj}到基\end{CJK} (II) \begin{CJK}{UTF8}{mj}的过渡矩阵\end{CJK};

(3) \begin{CJK}{UTF8}{mj}求出\end{CJK} $\left\{\beta_{1}, \beta_{2}, \beta_{3}, \beta_{4}\right\}$ \begin{CJK}{UTF8}{mj}中除掉基\end{CJK} (II) \begin{CJK}{UTF8}{mj}的向量外\end{CJK}, \begin{CJK}{UTF8}{mj}剩余向量\end{CJK} $\beta_{i}$ \begin{CJK}{UTF8}{mj}在基\end{CJK} $\alpha_{1}, \alpha_{2}, \alpha_{3}$ \begin{CJK}{UTF8}{mj}下的坐标\end{CJK}.

\begin{enumerate}
  \setcounter{enumi}{3}
  \item \begin{CJK}{UTF8}{mj}设线性方程组\end{CJK}
\end{enumerate}
$$
\left\{\begin{array}{l}
x_{1}-x_{2}-x_{3}+x_{4}=1 \\
x_{1}+x_{2}-3 x_{3}+x_{4}=1
\end{array}\right.
$$
(1) \begin{CJK}{UTF8}{mj}求该方程组的通解\end{CJK};

(2) \begin{CJK}{UTF8}{mj}求全体解集合向量的极大线性无关组\end{CJK}.

\begin{enumerate}
  \setcounter{enumi}{4}
  \item \begin{CJK}{UTF8}{mj}设\end{CJK} $\Phi, \Psi$ \begin{CJK}{UTF8}{mj}是某数域上\end{CJK} $n$ \begin{CJK}{UTF8}{mj}维线性空间上的两个线性变换\end{CJK}, \begin{CJK}{UTF8}{mj}满足\end{CJK} $\Phi \circ \Psi=\Psi \circ \Phi$, \begin{CJK}{UTF8}{mj}而且存在正整数\end{CJK} $N$ \begin{CJK}{UTF8}{mj}使得\end{CJK} $\Phi{ }^{N}=0$ \begin{CJK}{UTF8}{mj}为零线性变换\end{CJK}, \begin{CJK}{UTF8}{mj}证明\end{CJK} $\Phi+\Psi$ \begin{CJK}{UTF8}{mj}为可逆线性变换的充分必要条件是\end{CJK} $\Psi$ \begin{CJK}{UTF8}{mj}为可逆线性变换\end{CJK}.

  \item \begin{CJK}{UTF8}{mj}设\end{CJK} $A$ \begin{CJK}{UTF8}{mj}是\end{CJK} $n$ \begin{CJK}{UTF8}{mj}阶实对称矩阵\end{CJK}, \begin{CJK}{UTF8}{mj}证明存在幂等矩阵\end{CJK} $B_{i}(i=1, \cdots, s)$ \begin{CJK}{UTF8}{mj}使\end{CJK} $A=\lambda_{1} B_{1}+\lambda_{2} B_{2}+\cdots+\lambda_{s} B_{s}$. (\begin{CJK}{UTF8}{mj}说明\end{CJK}: \begin{CJK}{UTF8}{mj}一个矩阵\end{CJK} $B$ \begin{CJK}{UTF8}{mj}被称为幂等矩阵\end{CJK}, \begin{CJK}{UTF8}{mj}如果\end{CJK} $B^{2}=B$.)

  \item \begin{CJK}{UTF8}{mj}用正交变换将矩阵\end{CJK}

\end{enumerate}
$$
A=\left(\begin{array}{lll}
1 & 2 & 2 \\
2 & 1 & 2 \\
2 & 2 & 1
\end{array}\right)
$$
\begin{CJK}{UTF8}{mj}化成对角矩阵\end{CJK}, \begin{CJK}{UTF8}{mj}并求\end{CJK} $A^{3}+3 A^{2}+4 A+6 E$, \begin{CJK}{UTF8}{mj}其中\end{CJK} $E$ \begin{CJK}{UTF8}{mj}是\end{CJK} 3 \begin{CJK}{UTF8}{mj}阶单位矩阵\end{CJK}.

\begin{enumerate}
  \setcounter{enumi}{7}
  \item \begin{CJK}{UTF8}{mj}设\end{CJK} $f(x)$ \begin{CJK}{UTF8}{mj}是复系数一元多项式\end{CJK}, \begin{CJK}{UTF8}{mj}对任意整数\end{CJK} $n$ \begin{CJK}{UTF8}{mj}有\end{CJK} $f(n)$ \begin{CJK}{UTF8}{mj}还是整数\end{CJK}, \begin{CJK}{UTF8}{mj}证明\end{CJK} $f(x)$ \begin{CJK}{UTF8}{mj}的系数都是有理数\end{CJK}, \begin{CJK}{UTF8}{mj}举例说明存在\end{CJK} \begin{CJK}{UTF8}{mj}不是整系数的多项式满足对任意整数\end{CJK} $n$ \begin{CJK}{UTF8}{mj}有\end{CJK} $f(n)$ \begin{CJK}{UTF8}{mj}还是整数\end{CJK}.

  \item \begin{CJK}{UTF8}{mj}设\end{CJK} $a, b$ \begin{CJK}{UTF8}{mj}是任意两个复数\end{CJK}, \begin{CJK}{UTF8}{mj}根据不同的\end{CJK} $a, b$, \begin{CJK}{UTF8}{mj}求\end{CJK} $n$ \begin{CJK}{UTF8}{mj}阶上三角矩阵\end{CJK}

\end{enumerate}
$$
A=\left(\begin{array}{ccccc}
a & b & \cdots & b & b \\
0 & a & \cdots & b & b \\
\vdots & \vdots & & \vdots & \vdots \\
0 & 0 & \cdots & a & b \\
0 & 0 & \cdots & 0 & a
\end{array}\right)
$$
\begin{CJK}{UTF8}{mj}的最小多项式和若当\end{CJK} (Jordan) \begin{CJK}{UTF8}{mj}标准型\end{CJK}.

\begin{enumerate}
  \setcounter{enumi}{9}
  \item \begin{CJK}{UTF8}{mj}设\end{CJK} $\alpha_{1}, \cdots, \alpha_{k}$ \begin{CJK}{UTF8}{mj}是欧氏空间\end{CJK} $V$ \begin{CJK}{UTF8}{mj}中的一组两两正交的单位向量\end{CJK}, $\alpha$ \begin{CJK}{UTF8}{mj}是\end{CJK} $V$ \begin{CJK}{UTF8}{mj}中任意一个向量\end{CJK}. \begin{CJK}{UTF8}{mj}证明\end{CJK} Bessel \begin{CJK}{UTF8}{mj}不等式\end{CJK}:
\end{enumerate}
$$
\sum_{i=1}^{k}\left(\alpha, \alpha_{i}\right)^{2} \leqslant|\alpha|^{2}
$$
\begin{CJK}{UTF8}{mj}并证明向量\end{CJK} $\beta=\alpha-\sum_{i=1}^{k}\left(\alpha, \alpha_{i}\right) \alpha_{i}$ \begin{CJK}{UTF8}{mj}与每个向量\end{CJK} $\alpha_{j}$ \begin{CJK}{UTF8}{mj}都正交\end{CJK}.

\begin{enumerate}
  \setcounter{enumi}{10}
  \item \begin{CJK}{UTF8}{mj}设复线性空间\end{CJK} $V$ \begin{CJK}{UTF8}{mj}有线性变换\end{CJK} $A$, \begin{CJK}{UTF8}{mj}且\end{CJK} $A$ \begin{CJK}{UTF8}{mj}的特征多项式\end{CJK} $f(\lambda)=\left(\lambda-\lambda_{1}\right)^{r_{1}}\left(\lambda-\lambda_{2}\right)^{r_{2}}$. \begin{CJK}{UTF8}{mj}证明\end{CJK}: \begin{CJK}{UTF8}{mj}根子空间\end{CJK} $\overline{V_{\lambda_{i}}}=$ $\operatorname{ker}\left(A-\lambda_{i} \mathrm{id}\right)^{r_{i}}(i=1,2)$ \begin{CJK}{UTF8}{mj}均为\end{CJK} $A-$ \begin{CJK}{UTF8}{mj}不变子空间\end{CJK}.
\end{enumerate}
\section{3. 浙江大学 2011 年研究生入学考试试题高等代数 
 李扬 
 微信公众号: sxkyliyang}
\begin{CJK}{UTF8}{mj}注\end{CJK}: \begin{CJK}{UTF8}{mj}本试卷满分为\end{CJK} 150 \begin{CJK}{UTF8}{mj}分\end{CJK}, \begin{CJK}{UTF8}{mj}共计\end{CJK} 10 \begin{CJK}{UTF8}{mj}道题\end{CJK}, \begin{CJK}{UTF8}{mj}每题满分\end{CJK} 15 \begin{CJK}{UTF8}{mj}分\end{CJK}, \begin{CJK}{UTF8}{mj}考试时间总计\end{CJK} 180 \begin{CJK}{UTF8}{mj}分钟\end{CJK}.

\begin{enumerate}
  \item \begin{CJK}{UTF8}{mj}如果\end{CJK} $\left(x^{2}+x+1\right) \mid\left(f_{1}\left(x^{3}\right)+x f_{2}\left(x^{3}\right)\right)$, \begin{CJK}{UTF8}{mj}且\end{CJK} $n$ \begin{CJK}{UTF8}{mj}阶方阵\end{CJK} $A$ \begin{CJK}{UTF8}{mj}有一个特征值等于\end{CJK} 1 , \begin{CJK}{UTF8}{mj}证明\end{CJK} $f_{1}(A), f_{2}(A)$ \begin{CJK}{UTF8}{mj}都不是可逆矩\end{CJK} \begin{CJK}{UTF8}{mj}阵\end{CJK}.

  \item \begin{CJK}{UTF8}{mj}解下列方程组\end{CJK}

\end{enumerate}
$$
\left\{\begin{array}{l}
x_{1}+x_{2}+x_{3}+x_{4}=6 \\
x_{1}^{2}+x_{2}^{2}+x_{3}^{2}+x_{4}^{2}=10 \\
x_{1}^{3}+x_{2}^{3}+x_{3}^{3}+x_{4}^{3}=18 \\
x_{1}^{4}+x_{2}^{4}+x_{3}^{4}+x_{4}^{4}=34
\end{array}\right.
$$

\begin{enumerate}
  \setcounter{enumi}{3}
  \item \begin{CJK}{UTF8}{mj}设\end{CJK} $n$ \begin{CJK}{UTF8}{mj}阶方阵\end{CJK} $A$ \begin{CJK}{UTF8}{mj}的伴随矩阵\end{CJK} $A^{*}$, \begin{CJK}{UTF8}{mj}当\end{CJK} $n>2$ \begin{CJK}{UTF8}{mj}时\end{CJK}, \begin{CJK}{UTF8}{mj}证明\end{CJK}:
\end{enumerate}
$$
\left(A^{*}\right)^{*}=|A|^{n-2} A .
$$

\begin{enumerate}
  \setcounter{enumi}{4}
  \item \begin{CJK}{UTF8}{mj}设\end{CJK} $n$ \begin{CJK}{UTF8}{mj}阶方阵\end{CJK} $A$ \begin{CJK}{UTF8}{mj}满足\end{CJK}
\end{enumerate}
$$
A^{T} A=E,|A|=-1
$$
\begin{CJK}{UTF8}{mj}证明\end{CJK} $A+E$ \begin{CJK}{UTF8}{mj}是不可逆矩阵\end{CJK}.

\begin{enumerate}
  \setcounter{enumi}{5}
  \item \begin{CJK}{UTF8}{mj}设\end{CJK} $e_{i}=(0, \cdots, 0,1,0, \cdots)^{T}, i=1,2, \cdots n$ \begin{CJK}{UTF8}{mj}是欧氏空间\end{CJK} $\mathbb{R}^{n}$ \begin{CJK}{UTF8}{mj}的常用基\end{CJK}, \begin{CJK}{UTF8}{mj}一个矩阵\end{CJK} $P$ \begin{CJK}{UTF8}{mj}被称为置换矩阵假如存在\end{CJK} $1,2, \cdots n$ \begin{CJK}{UTF8}{mj}一个全排列阶\end{CJK} $i_{1}, i_{2}, \cdots, i_{n}$ \begin{CJK}{UTF8}{mj}使得矩阵\end{CJK} $P=\left(e_{i_{1}}, e_{i_{2}}, \cdots, e_{i_{n}}\right)$, \begin{CJK}{UTF8}{mj}例如\end{CJK}
\end{enumerate}
$$
\left(\begin{array}{llll}
0 & 0 & 1 & 0 \\
1 & 0 & 0 & 0 \\
0 & 0 & 0 & 1 \\
0 & 1 & 0 & 0
\end{array}\right)
$$
\begin{CJK}{UTF8}{mj}就是一个四阶置换矩阵\end{CJK}. \begin{CJK}{UTF8}{mj}假如\end{CJK} $n$ \begin{CJK}{UTF8}{mj}阶方阵\end{CJK} $A$ \begin{CJK}{UTF8}{mj}的秩等于\end{CJK} $r$, \begin{CJK}{UTF8}{mj}证明存在置换矩阵\end{CJK} $P$ \begin{CJK}{UTF8}{mj}使得\end{CJK} $P A P=\left(\begin{array}{l}A_{1} \\ A_{2}\end{array}\right)$, \begin{CJK}{UTF8}{mj}其中\end{CJK} $A_{1}$ \begin{CJK}{UTF8}{mj}的\end{CJK} \begin{CJK}{UTF8}{mj}秩等于\end{CJK} $r$.

\begin{enumerate}
  \setcounter{enumi}{6}
  \item \begin{CJK}{UTF8}{mj}设\end{CJK} $V=\left\{a+b x+c x^{2} \mid a, b, c \in \mathbb{R}\right\}$ \begin{CJK}{UTF8}{mj}是实数域上三维线性空间\end{CJK}, \begin{CJK}{UTF8}{mj}定义\end{CJK}
\end{enumerate}
$$
T(f(x))=2 f(x)+x f^{\prime}(x) .
$$
\begin{CJK}{UTF8}{mj}证明\end{CJK} $T$ \begin{CJK}{UTF8}{mj}是\end{CJK} $V$ \begin{CJK}{UTF8}{mj}上的线性变换\end{CJK}, \begin{CJK}{UTF8}{mj}并求其特征值和特征向量\end{CJK}.

\begin{enumerate}
  \setcounter{enumi}{7}
  \item \begin{CJK}{UTF8}{mj}设\end{CJK} $B$ \begin{CJK}{UTF8}{mj}是实数域上\end{CJK} $n \times n$ \begin{CJK}{UTF8}{mj}矩阵\end{CJK}, $A=B^{T} B$, \begin{CJK}{UTF8}{mj}对任意一个大于零的常数\end{CJK} $a$, \begin{CJK}{UTF8}{mj}证明\end{CJK} $(\alpha, \beta)=\alpha^{T}(A+a E) \beta$ \begin{CJK}{UTF8}{mj}定义了\end{CJK} $\mathbb{R}^{n}$ \begin{CJK}{UTF8}{mj}一个内积使得\end{CJK} $\mathbb{R}^{n}$ \begin{CJK}{UTF8}{mj}成为欧氏空间\end{CJK}. \begin{CJK}{UTF8}{mj}其中\end{CJK} $\alpha^{T}$ \begin{CJK}{UTF8}{mj}表示列向量\end{CJK} $\alpha$ \begin{CJK}{UTF8}{mj}的转置\end{CJK}, $E$ \begin{CJK}{UTF8}{mj}是\end{CJK} $n$ \begin{CJK}{UTF8}{mj}阶单位矩阵\end{CJK}.

  \item \begin{CJK}{UTF8}{mj}试证明满足\end{CJK} $A^{m}=E_{n}$ \begin{CJK}{UTF8}{mj}的\end{CJK} $n$ \begin{CJK}{UTF8}{mj}阶方阵\end{CJK} $A$ \begin{CJK}{UTF8}{mj}都相似于一个对角矩阵\end{CJK}.

  \item \begin{CJK}{UTF8}{mj}假设\end{CJK} $A=\left(a_{i j}\right)_{n \times n}$ \begin{CJK}{UTF8}{mj}是半正定矩阵\end{CJK}, \begin{CJK}{UTF8}{mj}证明满足\end{CJK} $X^{T} A X=0$ \begin{CJK}{UTF8}{mj}的所有\end{CJK} $X$ \begin{CJK}{UTF8}{mj}组成\end{CJK} $\mathbb{R}^{n}$ \begin{CJK}{UTF8}{mj}的\end{CJK} $n-r(A)$ \begin{CJK}{UTF8}{mj}维子空间\end{CJK}.

  \item \begin{CJK}{UTF8}{mj}已知矩阵\end{CJK}

\end{enumerate}
$$
A=\left(\begin{array}{cccc}
2 & -4 & 2 & 2 \\
-2 & 0 & 1 & 3 \\
-2 & -2 & 3 & 3 \\
-2 & -6 & 3 & 7
\end{array}\right)
$$
\begin{CJK}{UTF8}{mj}求矩阵\end{CJK} $P$, \begin{CJK}{UTF8}{mj}使\end{CJK} $P^{-1} A P$ \begin{CJK}{UTF8}{mj}为若当\end{CJK} (Jordan) \begin{CJK}{UTF8}{mj}标准型\end{CJK}.

\section{4. 浙江大学 2012 年研究生入学考试试题高等代数 
 李扬 
 微信公众号: sxkyliyang}
\begin{CJK}{UTF8}{mj}注\end{CJK}: \begin{CJK}{UTF8}{mj}本试卷满分为\end{CJK} 150 \begin{CJK}{UTF8}{mj}分\end{CJK}, \begin{CJK}{UTF8}{mj}共计\end{CJK} 10 \begin{CJK}{UTF8}{mj}道题\end{CJK}, \begin{CJK}{UTF8}{mj}每题满分\end{CJK} 15 \begin{CJK}{UTF8}{mj}分\end{CJK}, \begin{CJK}{UTF8}{mj}考试时间总计\end{CJK} 180 \begin{CJK}{UTF8}{mj}分钟\end{CJK}.

\begin{enumerate}
  \item \begin{CJK}{UTF8}{mj}设\end{CJK} $E$ \begin{CJK}{UTF8}{mj}是\end{CJK} $n$ \begin{CJK}{UTF8}{mj}阶单位矩阵\end{CJK}, $M=\left(\begin{array}{cc}0 & E \\ -E & 0\end{array}\right)$, \begin{CJK}{UTF8}{mj}矩阵\end{CJK} $A$ \begin{CJK}{UTF8}{mj}满足\end{CJK} $A^{T} M A=M$, \begin{CJK}{UTF8}{mj}证明\end{CJK} $A$ \begin{CJK}{UTF8}{mj}的行列式等于\end{CJK} 1 .

  \item \begin{CJK}{UTF8}{mj}设\end{CJK} $A$ \begin{CJK}{UTF8}{mj}是\end{CJK} $n$ \begin{CJK}{UTF8}{mj}阶幂等矩阵满足\end{CJK}:

\end{enumerate}
(1) $A=A_{1}+A_{2}+\cdots+A_{s}$,

(2) $r(A)=r\left(A_{1}\right)+\cdots+r\left(A_{s}\right)$.

\begin{CJK}{UTF8}{mj}证明所有的\end{CJK} $A_{i}$ \begin{CJK}{UTF8}{mj}都相似于一个对角矩阵\end{CJK}, $A_{i}$ \begin{CJK}{UTF8}{mj}的特征值之和等于矩阵\end{CJK} $A_{i}$ \begin{CJK}{UTF8}{mj}的秩\end{CJK}.

\begin{enumerate}
  \setcounter{enumi}{3}
  \item \begin{CJK}{UTF8}{mj}设\end{CJK} $\phi$ \begin{CJK}{UTF8}{mj}是\end{CJK} $n$ \begin{CJK}{UTF8}{mj}维欧式空间的正交变换\end{CJK}, \begin{CJK}{UTF8}{mj}证明\end{CJK} $\phi$ \begin{CJK}{UTF8}{mj}最多可以表示为\end{CJK} $n+1$ \begin{CJK}{UTF8}{mj}个镜面反射的复合\end{CJK}.

  \item \begin{CJK}{UTF8}{mj}设\end{CJK} $A$ \begin{CJK}{UTF8}{mj}是\end{CJK} $n$ \begin{CJK}{UTF8}{mj}阶复矩阵\end{CJK}, \begin{CJK}{UTF8}{mj}证明存在常数项等于零的多项式\end{CJK} $g(\lambda), h(\lambda)$ \begin{CJK}{UTF8}{mj}使得\end{CJK} $g(A)$ \begin{CJK}{UTF8}{mj}是可以对角化的矩阵\end{CJK}, $h(A)$ \begin{CJK}{UTF8}{mj}是幂\end{CJK} \begin{CJK}{UTF8}{mj}零矩阵\end{CJK}, \begin{CJK}{UTF8}{mj}且\end{CJK} $A=g(A)+h(A)$.

  \item \begin{CJK}{UTF8}{mj}设\end{CJK}

\end{enumerate}
$$
A=\left(\begin{array}{ccc}
3 & 2 & -2 \\
k & -1 & -k \\
4 & 2 & -3
\end{array}\right)
$$
(1) \begin{CJK}{UTF8}{mj}当\end{CJK} $k$ \begin{CJK}{UTF8}{mj}为何值时\end{CJK}, \begin{CJK}{UTF8}{mj}存在\end{CJK} $P$ \begin{CJK}{UTF8}{mj}使得\end{CJK} $P^{-1} A P$ \begin{CJK}{UTF8}{mj}为对角矩阵\end{CJK}? \begin{CJK}{UTF8}{mj}并求出这样的矩阵\end{CJK} $P$ \begin{CJK}{UTF8}{mj}和对角矩阵\end{CJK};

(2) \begin{CJK}{UTF8}{mj}求\end{CJK} $k=2$ \begin{CJK}{UTF8}{mj}时矩阵\end{CJK} $A$ \begin{CJK}{UTF8}{mj}的\end{CJK} Jordan \begin{CJK}{UTF8}{mj}标准型\end{CJK}.

\begin{enumerate}
  \setcounter{enumi}{6}
  \item \begin{CJK}{UTF8}{mj}令二次型\end{CJK}
\end{enumerate}
$$
f\left(x_{1}, \cdots, x_{n}\right)=\sum_{i=1}^{m}\left(a_{i 1} x_{1}+\cdots+a_{i n} x_{n}\right)^{2}
$$
(1) \begin{CJK}{UTF8}{mj}求次二次型的方阵\end{CJK};

(2) \begin{CJK}{UTF8}{mj}当\end{CJK} $a_{i j}$ \begin{CJK}{UTF8}{mj}均为实数\end{CJK}, \begin{CJK}{UTF8}{mj}给出次二次型为正定的条件\end{CJK}.

\begin{enumerate}
  \setcounter{enumi}{7}
  \item \begin{CJK}{UTF8}{mj}令\end{CJK} $V$ \begin{CJK}{UTF8}{mj}和\end{CJK} $W$ \begin{CJK}{UTF8}{mj}是域\end{CJK} $K$ \begin{CJK}{UTF8}{mj}上的线性空间\end{CJK}, $\operatorname{Hom}_{K}(V, W)$ \begin{CJK}{UTF8}{mj}表示\end{CJK} $V$ \begin{CJK}{UTF8}{mj}到\end{CJK} $W$ \begin{CJK}{UTF8}{mj}所有线性映射组成的线性空间\end{CJK}. \begin{CJK}{UTF8}{mj}证明\end{CJK}: \begin{CJK}{UTF8}{mj}对\end{CJK} $\operatorname{Hom}_{K}(V, W)$, \begin{CJK}{UTF8}{mj}若\end{CJK} $\operatorname{Im} f \cap \operatorname{Im} g=\{0\}$, \begin{CJK}{UTF8}{mj}则\end{CJK} $f$ \begin{CJK}{UTF8}{mj}和\end{CJK} $g$ \begin{CJK}{UTF8}{mj}在\end{CJK} $\operatorname{Hom}_{K}(V, W)$ \begin{CJK}{UTF8}{mj}中是线性无关的\end{CJK}.

  \item \begin{CJK}{UTF8}{mj}令线性空间\end{CJK} $V=\operatorname{Im} f \oplus W$, \begin{CJK}{UTF8}{mj}其中\end{CJK} $W$ \begin{CJK}{UTF8}{mj}是\end{CJK} $V$ \begin{CJK}{UTF8}{mj}的线性变换\end{CJK} $f$ \begin{CJK}{UTF8}{mj}的不变子空间\end{CJK}.

\end{enumerate}
(1) \begin{CJK}{UTF8}{mj}证明\end{CJK}: $W \subseteq \operatorname{ker} f$;

(2) \begin{CJK}{UTF8}{mj}证明若\end{CJK} $V$ \begin{CJK}{UTF8}{mj}是有限维线性空间\end{CJK}, \begin{CJK}{UTF8}{mj}则\end{CJK} $W=\operatorname{ker} f$;

(3) \begin{CJK}{UTF8}{mj}举例说明\end{CJK}, \begin{CJK}{UTF8}{mj}当\end{CJK} $V$ \begin{CJK}{UTF8}{mj}是无限维的\end{CJK}, \begin{CJK}{UTF8}{mj}可能有\end{CJK} $W \subseteq \operatorname{ker} f$, \begin{CJK}{UTF8}{mj}且\end{CJK} $W \neq \operatorname{ker} f$.

\begin{enumerate}
  \setcounter{enumi}{9}
  \item \begin{CJK}{UTF8}{mj}令\end{CJK}
\end{enumerate}
$$
A=\left(\begin{array}{ccccc}
1 & 0 & -1 & 2 & 1 \\
-1 & 1 & 3 & -1 & 0 \\
-2 & 1 & 4 & -1 & 3 \\
3 & -1 & -5 & 1 & -6
\end{array}\right)
$$
(1) \begin{CJK}{UTF8}{mj}求\end{CJK} $5 \times 5$ \begin{CJK}{UTF8}{mj}阶秩为\end{CJK} 2 \begin{CJK}{UTF8}{mj}的矩阵\end{CJK} $M$, \begin{CJK}{UTF8}{mj}使得\end{CJK} $A M=0$ (\begin{CJK}{UTF8}{mj}零矩阵\end{CJK});

(2) \begin{CJK}{UTF8}{mj}假如\end{CJK} $B$ \begin{CJK}{UTF8}{mj}是满足\end{CJK} $A B=0$ \begin{CJK}{UTF8}{mj}的\end{CJK} $5 \times 5$ \begin{CJK}{UTF8}{mj}阶矩阵\end{CJK}, \begin{CJK}{UTF8}{mj}证明\end{CJK}: \begin{CJK}{UTF8}{mj}秩\end{CJK} $\operatorname{rank}(B) \leqslant 2$.

\begin{enumerate}
  \setcounter{enumi}{10}
  \item \begin{CJK}{UTF8}{mj}令\end{CJK} $T$ \begin{CJK}{UTF8}{mj}是有限维线性空间\end{CJK} $V$ \begin{CJK}{UTF8}{mj}上的线性变换\end{CJK}, \begin{CJK}{UTF8}{mj}设\end{CJK} $W$ \begin{CJK}{UTF8}{mj}是\end{CJK} $V$ \begin{CJK}{UTF8}{mj}的\end{CJK} $T$-\begin{CJK}{UTF8}{mj}不变子空间\end{CJK}. \begin{CJK}{UTF8}{mj}那么\end{CJK}, $\left.T\right|_{W}$ \begin{CJK}{UTF8}{mj}的最小多项式整除\end{CJK} $T$ \begin{CJK}{UTF8}{mj}的\end{CJK} \begin{CJK}{UTF8}{mj}最小多项式\end{CJK}.
\end{enumerate}
\section{5. 浙江大学 2014 年研究生入学考试试题高等代数 
 李扬 
 微信公众号: sxkyliyang}
\begin{CJK}{UTF8}{mj}注\end{CJK}: \begin{CJK}{UTF8}{mj}本试卷满分为\end{CJK} 150 \begin{CJK}{UTF8}{mj}分\end{CJK}, \begin{CJK}{UTF8}{mj}共计\end{CJK} 10 \begin{CJK}{UTF8}{mj}道题\end{CJK}, \begin{CJK}{UTF8}{mj}每题满分\end{CJK} 15 \begin{CJK}{UTF8}{mj}分\end{CJK}, \begin{CJK}{UTF8}{mj}考试时间总计\end{CJK} 180 \begin{CJK}{UTF8}{mj}分钟\end{CJK}.
$$
A=\left(\begin{array}{cc}
0 & E_{n} \\
E_{n} & 0
\end{array}\right), L=\left\{B \in M_{2 n}(\mathbb{R}) \mid A B=B A\right\}
$$
\begin{CJK}{UTF8}{mj}证明\end{CJK} $L$ \begin{CJK}{UTF8}{mj}为\end{CJK} $M_{2 n}(\mathbb{R})$ \begin{CJK}{UTF8}{mj}的子空间并计算其维数\end{CJK}.

$2 .$
$$
A=\left(\begin{array}{cc}
0 & E_{n} \\
E_{n} & 0
\end{array}\right) \text {. }
$$
\begin{CJK}{UTF8}{mj}请问\end{CJK} $A$ \begin{CJK}{UTF8}{mj}是否可对角化并给出理由\end{CJK}. \begin{CJK}{UTF8}{mj}若\end{CJK} $A$ \begin{CJK}{UTF8}{mj}可对角化为\end{CJK} $C$, \begin{CJK}{UTF8}{mj}给出可逆矩阵\end{CJK} $P$, \begin{CJK}{UTF8}{mj}使得\end{CJK} $P^{-1} A P=C$.

\begin{enumerate}
  \setcounter{enumi}{3}
  \item \begin{CJK}{UTF8}{mj}方阵\end{CJK} $A$ \begin{CJK}{UTF8}{mj}的特征多项式为\end{CJK}
\end{enumerate}
$$
f(\lambda)=(\lambda-2)^{3}(\lambda+3)^{2} .
$$
\begin{CJK}{UTF8}{mj}请给出\end{CJK} $A$ \begin{CJK}{UTF8}{mj}所有可能的\end{CJK} Jordan \begin{CJK}{UTF8}{mj}标准型\end{CJK}.

\begin{enumerate}
  \setcounter{enumi}{4}
  \item $\eta_{1}, \eta_{2}, \eta_{3}$ \begin{CJK}{UTF8}{mj}为\end{CJK} $A X=0$ \begin{CJK}{UTF8}{mj}的基础解系\end{CJK}, $A$ \begin{CJK}{UTF8}{mj}为\end{CJK} 3 \begin{CJK}{UTF8}{mj}行\end{CJK} 5 \begin{CJK}{UTF8}{mj}列实矩阵\end{CJK}. \begin{CJK}{UTF8}{mj}求证\end{CJK}: \begin{CJK}{UTF8}{mj}存在\end{CJK} $\mathbb{R}^{5}$ \begin{CJK}{UTF8}{mj}的一组基\end{CJK}, \begin{CJK}{UTF8}{mj}其包含\end{CJK} $\eta_{1}+\eta_{2}+\eta_{3}, \eta_{1}-$ $\eta_{2}+\eta_{3}, \eta_{1}+2 \eta_{2}+4 \eta_{3} .$

  \item $X, Y$ \begin{CJK}{UTF8}{mj}分别为\end{CJK} $m \times n$ \begin{CJK}{UTF8}{mj}和\end{CJK} $n \times m$ \begin{CJK}{UTF8}{mj}矩阵\end{CJK}, $Y X=E_{n}, A=E_{m}+X Y$, \begin{CJK}{UTF8}{mj}证明\end{CJK} $A$ \begin{CJK}{UTF8}{mj}相似于对角矩阵\end{CJK}.

  \item $A$ \begin{CJK}{UTF8}{mj}为\end{CJK} $n$ \begin{CJK}{UTF8}{mj}阶线性空间\end{CJK} $V$ \begin{CJK}{UTF8}{mj}的线性变换\end{CJK}, $\lambda_{1}, \lambda_{2}, \cdots, \lambda_{m}$ \begin{CJK}{UTF8}{mj}为\end{CJK} $A$ \begin{CJK}{UTF8}{mj}的不同特征值\end{CJK}, $V_{\lambda_{i}}$ \begin{CJK}{UTF8}{mj}为其特征子空间\end{CJK}. \begin{CJK}{UTF8}{mj}证明\end{CJK}: \begin{CJK}{UTF8}{mj}对任意\end{CJK} $V$ \begin{CJK}{UTF8}{mj}的子空间\end{CJK} $W$, \begin{CJK}{UTF8}{mj}有\end{CJK}

\end{enumerate}
$$
W=\left(W \cap V_{\lambda_{1}}\right) \oplus \cdots \oplus\left(W \cap V_{\lambda_{m}}\right)
$$

\begin{enumerate}
  \setcounter{enumi}{7}
  \item \begin{CJK}{UTF8}{mj}矩阵\end{CJK} $A, B$ \begin{CJK}{UTF8}{mj}均为\end{CJK} $m \times n$ \begin{CJK}{UTF8}{mj}矩阵\end{CJK}, $A X=0$ \begin{CJK}{UTF8}{mj}与\end{CJK} $B X=0$ \begin{CJK}{UTF8}{mj}同解\end{CJK}, \begin{CJK}{UTF8}{mj}求证\end{CJK} $A, B$ \begin{CJK}{UTF8}{mj}等价\end{CJK}. \begin{CJK}{UTF8}{mj}若\end{CJK} $A, B$ \begin{CJK}{UTF8}{mj}等价\end{CJK}, \begin{CJK}{UTF8}{mj}是否有\end{CJK} $A X=0$ \begin{CJK}{UTF8}{mj}与\end{CJK} $B X=0$ \begin{CJK}{UTF8}{mj}同解\end{CJK}? \begin{CJK}{UTF8}{mj}证明或举反例否定\end{CJK}.

  \item \begin{CJK}{UTF8}{mj}证明\end{CJK}: $A$ \begin{CJK}{UTF8}{mj}正定的充分必要条件是存在方阵\end{CJK} $B_{i}(i=1,2, \cdots, n), B_{i}$ \begin{CJK}{UTF8}{mj}中至少有一个非退化\end{CJK}, \begin{CJK}{UTF8}{mj}使得\end{CJK}

\end{enumerate}
$$
A=\sum_{i=1}^{n} B_{i} B_{i}^{T}
$$

\begin{enumerate}
  \setcounter{enumi}{9}
  \item \begin{CJK}{UTF8}{mj}定义\end{CJK} $\psi$ \begin{CJK}{UTF8}{mj}为\end{CJK} $[0,1]$ \begin{CJK}{UTF8}{mj}到\end{CJK} $n$ \begin{CJK}{UTF8}{mj}阶方阵全体组成的欧式空间的连续映射\end{CJK}, \begin{CJK}{UTF8}{mj}使得\end{CJK} $\psi(0)$ \begin{CJK}{UTF8}{mj}为第一类正交矩阵\end{CJK}, $\psi(1)$ \begin{CJK}{UTF8}{mj}为第二类正\end{CJK} \begin{CJK}{UTF8}{mj}交矩阵\end{CJK}. \begin{CJK}{UTF8}{mj}证明\end{CJK}: \begin{CJK}{UTF8}{mj}存在\end{CJK} $T_{0} \in(0,1)$, \begin{CJK}{UTF8}{mj}使得\end{CJK} $\psi\left(T_{0}\right)$ \begin{CJK}{UTF8}{mj}退化\end{CJK}.

  \item \begin{CJK}{UTF8}{mj}设\end{CJK} $g, h$ \begin{CJK}{UTF8}{mj}为复数域\end{CJK} $\mathbb{C}$ \begin{CJK}{UTF8}{mj}上\end{CJK} $n$ \begin{CJK}{UTF8}{mj}维线性空间\end{CJK} $V$ \begin{CJK}{UTF8}{mj}的线性变换\end{CJK}, $g h=h g$. \begin{CJK}{UTF8}{mj}求证\end{CJK} $g, h$ \begin{CJK}{UTF8}{mj}有公共的特征向量\end{CJK}. \begin{CJK}{UTF8}{mj}若不是在复数域\end{CJK} $\mathbb{C}$ \begin{CJK}{UTF8}{mj}上而是在实数域\end{CJK} $\mathbb{R}$ \begin{CJK}{UTF8}{mj}上\end{CJK}, \begin{CJK}{UTF8}{mj}则结论是否成立\end{CJK}? \begin{CJK}{UTF8}{mj}若成立\end{CJK}, \begin{CJK}{UTF8}{mj}给出理由\end{CJK}; \begin{CJK}{UTF8}{mj}不成立举出反例\end{CJK}.

\end{enumerate}
\section{6. 浙江大学 2015 年研究生入学考试试题高等代数 
 李扬 
 微信公众号: sxkyliyang}
\begin{CJK}{UTF8}{mj}注\end{CJK}: \begin{CJK}{UTF8}{mj}本试卷满分为\end{CJK} 150 \begin{CJK}{UTF8}{mj}分\end{CJK}, \begin{CJK}{UTF8}{mj}共计\end{CJK} 10 \begin{CJK}{UTF8}{mj}道题\end{CJK}, \begin{CJK}{UTF8}{mj}每题满分\end{CJK} 15 \begin{CJK}{UTF8}{mj}分\end{CJK}, \begin{CJK}{UTF8}{mj}考试时间总计\end{CJK} 180 \begin{CJK}{UTF8}{mj}分钟\end{CJK}.

\begin{enumerate}
  \item $A(t)$ \begin{CJK}{UTF8}{mj}矩阵各元素连续可微\end{CJK}, \begin{CJK}{UTF8}{mj}行列式恒为\end{CJK} $1, A(0)=E$, \begin{CJK}{UTF8}{mj}求证\end{CJK}: $A^{\prime}(0)$ \begin{CJK}{UTF8}{mj}的迹为\end{CJK} 0 . (\begin{CJK}{UTF8}{mj}求导是对各元素独立求的\end{CJK})

  \item \begin{CJK}{UTF8}{mj}线性空间上\end{CJK} $\left(a_{1}, a_{2}, \cdots, a_{s}\right)$ \begin{CJK}{UTF8}{mj}与\end{CJK} $\left(b_{1}, b_{2}, \cdots, b_{t}\right)$ \begin{CJK}{UTF8}{mj}是两个线性无关向量组\end{CJK}, $\left(a_{1}, a_{2}, \cdots, a_{s}\right)=\left(b_{1}, b_{2}, \cdots, b_{t}\right) A$, \begin{CJK}{UTF8}{mj}证\end{CJK}:

\end{enumerate}
$$
n-t-r(A) \leqslant s \leqslant \min \{r(A), t\}
$$

\begin{enumerate}
  \setcounter{enumi}{3}
  \item $f: v \rightarrow w$ \begin{CJK}{UTF8}{mj}为线性满映射\end{CJK}
\end{enumerate}
(1) $\forall \alpha \in w, f^{-1}(\alpha)=\beta+\operatorname{ker}(f)(\beta$ \begin{CJK}{UTF8}{mj}为\end{CJK} $v$ \begin{CJK}{UTF8}{mj}上任一个向量满足\end{CJK} $f(\beta)=\alpha)$;

(2) \begin{CJK}{UTF8}{mj}适当定义乘法和用下面定义的加法\end{CJK}:

\begin{CJK}{UTF8}{mj}证明\end{CJK}: $v / \operatorname{ker}(f)=\{\beta+\operatorname{ker}(f) \mid \beta \in v\}$ \begin{CJK}{UTF8}{mj}构成空间\end{CJK};

(3) \begin{CJK}{UTF8}{mj}适当定义同构映射\end{CJK}, \begin{CJK}{UTF8}{mj}证明\end{CJK}: $v / \operatorname{ker}(f)$ \begin{CJK}{UTF8}{mj}与\end{CJK} $\operatorname{Im}(f)$ \begin{CJK}{UTF8}{mj}同构\end{CJK}.

\begin{enumerate}
  \setcounter{enumi}{4}
  \item \begin{CJK}{UTF8}{mj}空间\end{CJK} $V$ \begin{CJK}{UTF8}{mj}上的线性变换\end{CJK} $f$, \begin{CJK}{UTF8}{mj}可以找到子空间\end{CJK} $U, W$, \begin{CJK}{UTF8}{mj}使得\end{CJK} $f$ \begin{CJK}{UTF8}{mj}在\end{CJK} $U$ \begin{CJK}{UTF8}{mj}上为可逆线性变换\end{CJK}, \begin{CJK}{UTF8}{mj}在\end{CJK} $W$ \begin{CJK}{UTF8}{mj}上为幂零线性变换\end{CJK}, \begin{CJK}{UTF8}{mj}且\end{CJK} $V=U \oplus W$.

  \item $\exists b \neq 0, A x=b$, \begin{CJK}{UTF8}{mj}证明\end{CJK}: $A^{*} x=b$ \begin{CJK}{UTF8}{mj}有解的充要条件为\end{CJK}

\end{enumerate}
$$
r(A)=n-1 .
$$

\begin{enumerate}
  \setcounter{enumi}{6}
  \item \begin{CJK}{UTF8}{mj}所有正交变换构成\end{CJK} $G$
\end{enumerate}
(1) $G$ \begin{CJK}{UTF8}{mj}关于线性变换的合成和逆变换封闭\end{CJK};

(2) $G$ \begin{CJK}{UTF8}{mj}为有限集还是无限集\end{CJK};

(3) $G$ \begin{CJK}{UTF8}{mj}是什么代数结构\end{CJK}.

\begin{enumerate}
  \setcounter{enumi}{7}
  \item $A$ \begin{CJK}{UTF8}{mj}为对称阵\end{CJK}, $A^{3}-6 A^{2}+11 A-6=0$ (I). \begin{CJK}{UTF8}{mj}求\end{CJK}
\end{enumerate}
$$
\max _{A} \max _{\|x\|=1} \lambda_{1} x_{1}^{2}+\lambda_{2} x_{2}^{2}+\lambda_{3} x_{3}^{2} .
$$
\begin{CJK}{UTF8}{mj}第一个极大值是对所有满足\end{CJK} I \begin{CJK}{UTF8}{mj}的矩阵\end{CJK} $A$ \begin{CJK}{UTF8}{mj}取的\end{CJK}.

\begin{enumerate}
  \setcounter{enumi}{8}
  \item $f(x)$ \begin{CJK}{UTF8}{mj}为一多项式\end{CJK}, $g(x)$ \begin{CJK}{UTF8}{mj}是\end{CJK} $A$ \begin{CJK}{UTF8}{mj}的最小多项式\end{CJK}, \begin{CJK}{UTF8}{mj}证明\end{CJK}: $f(A)$ \begin{CJK}{UTF8}{mj}可逆的充要条件是\end{CJK}
\end{enumerate}
$$
(f(x), g(x))=1 .
$$

\begin{enumerate}
  \setcounter{enumi}{9}
  \item $\lim _{n \rightarrow \infty} A^{n}=0 \Longleftrightarrow A$ \begin{CJK}{UTF8}{mj}的所有特征值\end{CJK} $|\lambda|<1$.

  \item \begin{CJK}{UTF8}{mj}双线性函数巴拉巴拉\end{CJK},

\end{enumerate}
(1) \begin{CJK}{UTF8}{mj}全迷向子空间关于以上定义的运算构成空间\end{CJK}?

(2) \begin{CJK}{UTF8}{mj}全迷向子空间含于其正交补\end{CJK}.

\section{7. 浙江大学 2016 年研究生入学考试试题高等代数 
 李扬 
 微信公众号: sxkyliyang}
\begin{CJK}{UTF8}{mj}注\end{CJK}: \begin{CJK}{UTF8}{mj}本试卷满分为\end{CJK} 150 \begin{CJK}{UTF8}{mj}分\end{CJK}, \begin{CJK}{UTF8}{mj}共计\end{CJK} 10 \begin{CJK}{UTF8}{mj}道题\end{CJK}, \begin{CJK}{UTF8}{mj}每题满分\end{CJK} 15 \begin{CJK}{UTF8}{mj}分\end{CJK}, \begin{CJK}{UTF8}{mj}考试时间总计\end{CJK} 180 \begin{CJK}{UTF8}{mj}分钟\end{CJK}.

\begin{enumerate}
  \item (15\begin{CJK}{UTF8}{mj}分\end{CJK}) \begin{CJK}{UTF8}{mj}已知矩阵\end{CJK} $A$ \begin{CJK}{UTF8}{mj}是\end{CJK} $n$ \begin{CJK}{UTF8}{mj}阶不可逆方阵\end{CJK}, $E$ \begin{CJK}{UTF8}{mj}是单位矩阵\end{CJK}, $A^{*}$ \begin{CJK}{UTF8}{mj}是\end{CJK} $A$ \begin{CJK}{UTF8}{mj}的伴随矩阵\end{CJK}. \begin{CJK}{UTF8}{mj}证明至多存在两个非零复数\end{CJK} $k$, \begin{CJK}{UTF8}{mj}使得\end{CJK} $k E+A^{*}$ \begin{CJK}{UTF8}{mj}为不可逆矩阵\end{CJK}.

  \item ( 15\begin{CJK}{UTF8}{mj}分\end{CJK}) \begin{CJK}{UTF8}{mj}设\end{CJK} $\mathbb{P}[x]$ \begin{CJK}{UTF8}{mj}是数域\end{CJK} $\mathbb{P}$ \begin{CJK}{UTF8}{mj}上的一元多项式全体\end{CJK}, $f_{1}(x), f_{2}(x), \cdots, f_{n}(x)$ \begin{CJK}{UTF8}{mj}和\end{CJK} $g_{1}(x), g_{2}(x), \cdots, g_{m}(x)$ \begin{CJK}{UTF8}{mj}是\end{CJK} $\mathbb{P}[x]$ \begin{CJK}{UTF8}{mj}中的\end{CJK} \begin{CJK}{UTF8}{mj}两组多项式\end{CJK}, \begin{CJK}{UTF8}{mj}且它们生成的子空间相同\end{CJK}, \begin{CJK}{UTF8}{mj}证明\end{CJK}:

\end{enumerate}
(1) $\mathbb{P}[x]$ \begin{CJK}{UTF8}{mj}不是该数域\end{CJK} $\mathbb{P}$ \begin{CJK}{UTF8}{mj}上的有限维线性空间\end{CJK}.

(2) $f_{1}(x), f_{2}(x), \cdots, f_{n}(x)$ \begin{CJK}{UTF8}{mj}的最大公因子等于\end{CJK} $g_{1}(x), g_{2}(x), \cdots, g_{s}(x)$ \begin{CJK}{UTF8}{mj}的最大公因子\end{CJK}.

\begin{enumerate}
  \setcounter{enumi}{3}
  \item (15\begin{CJK}{UTF8}{mj}分\end{CJK}) \begin{CJK}{UTF8}{mj}设\end{CJK} $\mathbb{R}[x]_{n+1}$ \begin{CJK}{UTF8}{mj}是次数小于等于\end{CJK} $n$ \begin{CJK}{UTF8}{mj}的实系数多项式全体\end{CJK}, $f(x)$ \begin{CJK}{UTF8}{mj}是\end{CJK} $n$ \begin{CJK}{UTF8}{mj}次多项式\end{CJK}. \begin{CJK}{UTF8}{mj}证明\end{CJK}: \begin{CJK}{UTF8}{mj}对\end{CJK} $\mathbb{R}[x]_{n+1}$ \begin{CJK}{UTF8}{mj}中的任意\end{CJK} \begin{CJK}{UTF8}{mj}多项式\end{CJK} $g(x)$, \begin{CJK}{UTF8}{mj}总存在常数\end{CJK} $c_{0}, c_{1}, \cdots, c_{n}$ \begin{CJK}{UTF8}{mj}使得\end{CJK}
\end{enumerate}
$$
g(x)=c_{0} f(x)+c_{1} f^{\prime}(x)+\cdots+c_{k} f^{(k)}(x)+\cdots+c_{n} f^{(n)}(x),
$$
\begin{CJK}{UTF8}{mj}其中\end{CJK} $f^{(k)}(x)$ \begin{CJK}{UTF8}{mj}是\end{CJK} $f(x)$ \begin{CJK}{UTF8}{mj}的\end{CJK} $k$ \begin{CJK}{UTF8}{mj}阶导数\end{CJK}.

\begin{enumerate}
  \setcounter{enumi}{4}
  \item ( 15 \begin{CJK}{UTF8}{mj}分\end{CJK}) \begin{CJK}{UTF8}{mj}设\end{CJK} $k$ \begin{CJK}{UTF8}{mj}是整数\end{CJK}, $\alpha$ \begin{CJK}{UTF8}{mj}是\end{CJK} $x^{4}+4 k x+1=0$ \begin{CJK}{UTF8}{mj}的一个根\end{CJK}, \begin{CJK}{UTF8}{mj}问\end{CJK}
\end{enumerate}
$$
\mathbb{Q}[\alpha]:=\left\{a_{0}+a_{1} \alpha+a_{2} \alpha^{2}+a_{3} \alpha^{3} \mid a_{i} \in \mathbb{Q}\right\}
$$
\begin{CJK}{UTF8}{mj}是否是数域\end{CJK}? \begin{CJK}{UTF8}{mj}如果是\end{CJK}, \begin{CJK}{UTF8}{mj}请给予证明\end{CJK}. \begin{CJK}{UTF8}{mj}假如不是\end{CJK}, \begin{CJK}{UTF8}{mj}请说明理由\end{CJK}. \begin{CJK}{UTF8}{mj}其中\end{CJK} $\mathbb{Q}$ \begin{CJK}{UTF8}{mj}是有理数域\end{CJK}.

\begin{enumerate}
  \setcounter{enumi}{5}
  \item ( 15 \begin{CJK}{UTF8}{mj}分\end{CJK}) \begin{CJK}{UTF8}{mj}设\end{CJK} $V_{1}, V_{2}$ \begin{CJK}{UTF8}{mj}是\end{CJK} $n$ \begin{CJK}{UTF8}{mj}维线性空间\end{CJK} $V$ \begin{CJK}{UTF8}{mj}的两个子空间\end{CJK}, \begin{CJK}{UTF8}{mj}且它们的维数之和等于\end{CJK} $n$. \begin{CJK}{UTF8}{mj}证明\end{CJK}: \begin{CJK}{UTF8}{mj}存在\end{CJK} $V$ \begin{CJK}{UTF8}{mj}上的线性变换\end{CJK} $\mathbb{T}$, \begin{CJK}{UTF8}{mj}使得\end{CJK} $\mathbb{T}$ \begin{CJK}{UTF8}{mj}的核和像分别等于\end{CJK} $V_{1}$ \begin{CJK}{UTF8}{mj}和\end{CJK} $V_{2}$.

  \item (15\begin{CJK}{UTF8}{mj}分\end{CJK}) \begin{CJK}{UTF8}{mj}已知矩阵\end{CJK} $A=\left(\begin{array}{lll}a & b & c \\ d & e & f \\ h & x & y\end{array}\right)$ \begin{CJK}{UTF8}{mj}的逆矩阵是\end{CJK} $A^{-1}=\left(\begin{array}{ccc}-1 & -2 & -1 \\ -2 & 1 & 0 \\ 0 & -3 & -1\end{array}\right), B=\left(\begin{array}{ccc}a-2 b & b-3 & -c \\ d-2 e & e-3 f & -f \\ h-2 x & x-3 y & -y\end{array}\right)$. \begin{CJK}{UTF8}{mj}求矩阵\end{CJK} $X$ \begin{CJK}{UTF8}{mj}满足\end{CJK}

\end{enumerate}
$$
X+\left(B\left(A^{T} B^{2}\right)^{-1} A^{T}\right)^{-1}=X\left(A^{2}\left(B^{T} A\right)^{-1} B^{T}\right)^{-1}(A+B) .
$$

\begin{enumerate}
  \setcounter{enumi}{7}
  \item (15\begin{CJK}{UTF8}{mj}分\end{CJK}) \begin{CJK}{UTF8}{mj}令\end{CJK} $\mathbb{T}$ \begin{CJK}{UTF8}{mj}是欧氏空间\end{CJK} $V$ \begin{CJK}{UTF8}{mj}上的线性变换\end{CJK}, \begin{CJK}{UTF8}{mj}而\end{CJK} $\mathbb{T}^{*}$ \begin{CJK}{UTF8}{mj}是\end{CJK} $\mathbb{T}$ \begin{CJK}{UTF8}{mj}的伴随线性变换\end{CJK}, \begin{CJK}{UTF8}{mj}即对任意的\end{CJK} $v, w \in V$ \begin{CJK}{UTF8}{mj}有\end{CJK} $\langle\mathbb{T}(v), w\rangle=$ $\left\langle v, \mathbb{T}^{*}(w)\right\rangle$.
\end{enumerate}
(1)\begin{CJK}{UTF8}{mj}当\end{CJK} $V$ \begin{CJK}{UTF8}{mj}为有限维欧氏空间\end{CJK}, $\mathbb{T}$ \begin{CJK}{UTF8}{mj}在一组单位正交基\end{CJK}(\begin{CJK}{UTF8}{mj}或称为标准正交基\end{CJK})\begin{CJK}{UTF8}{mj}下的矩阵为\end{CJK} $A$ \begin{CJK}{UTF8}{mj}时\end{CJK}, \begin{CJK}{UTF8}{mj}求\end{CJK} $\mathbb{T}^{*}$ \begin{CJK}{UTF8}{mj}在该基下的\end{CJK} \begin{CJK}{UTF8}{mj}矩阵\end{CJK}.

(2) \begin{CJK}{UTF8}{mj}证明\end{CJK}: $\left(\operatorname{Im}\left(\mathbb{T}^{*}\right)\right)^{\perp}=\operatorname{ker}(\mathbb{T})$.

\begin{enumerate}
  \setcounter{enumi}{8}
  \item (15\begin{CJK}{UTF8}{mj}分\end{CJK}) \begin{CJK}{UTF8}{mj}试证明\end{CJK}: \begin{CJK}{UTF8}{mj}正定矩阵\end{CJK} $A$ \begin{CJK}{UTF8}{mj}中绝对值最大的元素可以在主对角线上取到\end{CJK}.

  \item ( 15 \begin{CJK}{UTF8}{mj}分\end{CJK}) \begin{CJK}{UTF8}{mj}设\end{CJK} $\mathbb{T}$ \begin{CJK}{UTF8}{mj}是复数域上\end{CJK} $n$ \begin{CJK}{UTF8}{mj}维线性空间\end{CJK} $V$ \begin{CJK}{UTF8}{mj}的线性变换\end{CJK}, \begin{CJK}{UTF8}{mj}满足\end{CJK} $\mathbb{T}^{k}=\mathrm{id}_{V}$ ( $V$ \begin{CJK}{UTF8}{mj}上的恒等线性变换\end{CJK}), \begin{CJK}{UTF8}{mj}其中\end{CJK} $1 \leqslant k \leqslant n$, \begin{CJK}{UTF8}{mj}证明\end{CJK} $\mathbb{T}$ \begin{CJK}{UTF8}{mj}必然可以对角化\end{CJK}.

  \item (15\begin{CJK}{UTF8}{mj}分\end{CJK}) \begin{CJK}{UTF8}{mj}设有限维线性空间\end{CJK} $V$ \begin{CJK}{UTF8}{mj}有两个非平凡的子空间\end{CJK} $V_{1}, V_{2}$ \begin{CJK}{UTF8}{mj}使得\end{CJK} $V=V_{1} \oplus V_{2}, W$ \begin{CJK}{UTF8}{mj}为\end{CJK} $V$ \begin{CJK}{UTF8}{mj}的任意子空间\end{CJK}. \begin{CJK}{UTF8}{mj}证明\end{CJK}

\end{enumerate}
(1) $\left(W \cap V_{1}\right)+\left(W \cap V_{2}\right)$ \begin{CJK}{UTF8}{mj}是\end{CJK} $W$ \begin{CJK}{UTF8}{mj}的子空间\end{CJK}, $W$ \begin{CJK}{UTF8}{mj}是\end{CJK} $\left(W+V_{1}\right) \cap\left(W+V_{2}\right)$ \begin{CJK}{UTF8}{mj}的子空间\end{CJK}.

(2) \begin{CJK}{UTF8}{mj}商空间\end{CJK} $W /\left(W \cap V_{1}+W \cap V_{2}\right)$ \begin{CJK}{UTF8}{mj}的维数等于商空间\end{CJK} $\left(\left(W+V_{1}\right) \cap\left(W+V_{2}\right)\right) / W$ \begin{CJK}{UTF8}{mj}的维数\end{CJK}.

(3) \begin{CJK}{UTF8}{mj}利用上述结论证明\end{CJK} $W=\left(W \cap V_{1}\right) \oplus\left(W \cap V_{2}\right)$ \begin{CJK}{UTF8}{mj}充分必要条件是\end{CJK} $W=\left(W+V_{1}\right) \cap\left(W+V_{2}\right)$.

\section{8. 浙江大学 2009 年研究生入学考试试题数学分析 
 李扬 
 微信公众号: sxkyliyang}
\begin{enumerate}
  \item (\begin{CJK}{UTF8}{mj}每小题\end{CJK} 10 \begin{CJK}{UTF8}{mj}分\end{CJK}, \begin{CJK}{UTF8}{mj}共\end{CJK} 40 \begin{CJK}{UTF8}{mj}分\end{CJK}) \begin{CJK}{UTF8}{mj}计算\end{CJK}
\end{enumerate}
(1) $\int \frac{1}{a^{2} \cos ^{2} x+b^{2} \sin ^{2} x} \mathrm{~d} x(a b \neq 0)$.

(2) $\lim _{x \rightarrow 0} \frac{\int_{0}^{x} e^{\frac{t^{2}}{2}} \cos t \mathrm{~d} t-x}{\left(e^{x}-1\right)^{2}\left(1-\cos ^{2} x\right) \arctan x}$.

(3) $\int_{0}^{+\infty} \frac{\ln x}{1+x^{2}} \mathrm{~d} x$.

(4) $\iint_{D}(x+y) \operatorname{sgn}(x-y) \mathrm{d} x \mathrm{~d} y$. \begin{CJK}{UTF8}{mj}其中\end{CJK} $D=[0,1] \times[0,1]$.

\begin{enumerate}
  \setcounter{enumi}{2}
  \item ( 15 \begin{CJK}{UTF8}{mj}分\end{CJK}) \begin{CJK}{UTF8}{mj}如果\end{CJK} $f(x)$ \begin{CJK}{UTF8}{mj}在\end{CJK} $x_{0}$ \begin{CJK}{UTF8}{mj}的某邻域内可导\end{CJK}, \begin{CJK}{UTF8}{mj}且\end{CJK} $\lim _{x \rightarrow x_{0}} \frac{f^{\prime}(x)}{x-x_{0}}=\frac{1}{2}$. \begin{CJK}{UTF8}{mj}证明\end{CJK}: $f(x)$ \begin{CJK}{UTF8}{mj}在点\end{CJK} $x_{0}$ \begin{CJK}{UTF8}{mj}处取极小值\end{CJK}.

  \item ( 15 \begin{CJK}{UTF8}{mj}分\end{CJK}) \begin{CJK}{UTF8}{mj}设\end{CJK} $f(x, y, z)$ \begin{CJK}{UTF8}{mj}表从原点\end{CJK} $O(0,0,0)$ \begin{CJK}{UTF8}{mj}到椭圆面\end{CJK} $\Sigma: \frac{x^{2}}{a^{2}}+\frac{y^{2}}{b^{2}}+\frac{z^{2}}{c^{2}}=1(a, b, c>0)$ \begin{CJK}{UTF8}{mj}上点\end{CJK} $P(x, y, z)$ \begin{CJK}{UTF8}{mj}处的切\end{CJK} \begin{CJK}{UTF8}{mj}平面的距离\end{CJK}. \begin{CJK}{UTF8}{mj}求第一类曲面积分\end{CJK}:

\end{enumerate}
$$
\iint_{\Sigma} \frac{\mathrm{d} s}{f(x, y, z)}
$$

\begin{enumerate}
  \setcounter{enumi}{4}
  \item ( 20 \begin{CJK}{UTF8}{mj}分\end{CJK}) \begin{CJK}{UTF8}{mj}设\end{CJK} $f(x)$ \begin{CJK}{UTF8}{mj}在\end{CJK} $[a, b]$ \begin{CJK}{UTF8}{mj}上连续\end{CJK}, \begin{CJK}{UTF8}{mj}且\end{CJK} $\min _{x \in[a, b]} f(x)=1$. \begin{CJK}{UTF8}{mj}证明\end{CJK}:
\end{enumerate}
$$
\lim _{n \rightarrow \infty}\left(\int_{a}^{b} \frac{\mathrm{d} x}{(f(x))^{n}}\right)^{\frac{1}{n}}=1 .
$$

\begin{enumerate}
  \setcounter{enumi}{5}
  \item ( 20 \begin{CJK}{UTF8}{mj}分\end{CJK}) \begin{CJK}{UTF8}{mj}设对任意\end{CJK} $a>0, f(x)$ \begin{CJK}{UTF8}{mj}在\end{CJK} $[0, a]$ \begin{CJK}{UTF8}{mj}上黎曼可积\end{CJK}, \begin{CJK}{UTF8}{mj}且\end{CJK} $\lim _{x \rightarrow+\infty} f(x)=C$. \begin{CJK}{UTF8}{mj}证明\end{CJK}:
\end{enumerate}
$$
\lim _{t \rightarrow 0^{+}} t \int_{0}^{+\infty} e^{-t x} f(x) \mathrm{d} x=C .
$$

\begin{enumerate}
  \setcounter{enumi}{6}
  \item $\left(20\right.$ \begin{CJK}{UTF8}{mj}分\end{CJK}) \begin{CJK}{UTF8}{mj}证明\end{CJK} $f(x)=\frac{|\sin x|}{x}$ \begin{CJK}{UTF8}{mj}在\end{CJK} $(0,1)$ \begin{CJK}{UTF8}{mj}与\end{CJK} $(-1,0)$ \begin{CJK}{UTF8}{mj}上均一致连续\end{CJK}, \begin{CJK}{UTF8}{mj}但在\end{CJK} $(-1,0) \cup(0,1)$ \begin{CJK}{UTF8}{mj}中不一致连续\end{CJK}. (\begin{CJK}{UTF8}{mj}注\end{CJK}: \begin{CJK}{UTF8}{mj}称\end{CJK} $y=f(x)$ \begin{CJK}{UTF8}{mj}在集合\end{CJK} $D(D \subset \mathbb{R})$ \begin{CJK}{UTF8}{mj}上一致连续是指\end{CJK}: \begin{CJK}{UTF8}{mj}对\end{CJK} $\forall \varepsilon>0$, \begin{CJK}{UTF8}{mj}存在\end{CJK} $\delta>0$, \begin{CJK}{UTF8}{mj}使得对\end{CJK} $x^{\prime}, x^{\prime \prime} \in D$, \begin{CJK}{UTF8}{mj}当\end{CJK} $\left|x^{\prime}-x^{\prime \prime}\right|<\delta$ \begin{CJK}{UTF8}{mj}时\end{CJK}, \begin{CJK}{UTF8}{mj}有\end{CJK} $\left.\left|f\left(x^{\prime}\right)-f\left(x^{\prime \prime}\right)\right|<\varepsilon\right)$

  \item ( 20 \begin{CJK}{UTF8}{mj}分\end{CJK}) \begin{CJK}{UTF8}{mj}设\end{CJK} $f(x)$ \begin{CJK}{UTF8}{mj}在\end{CJK} $[a, b]$ \begin{CJK}{UTF8}{mj}上可导\end{CJK}, \begin{CJK}{UTF8}{mj}导函数\end{CJK} $f^{\prime}(x)$ \begin{CJK}{UTF8}{mj}在\end{CJK} $[a, b]$ \begin{CJK}{UTF8}{mj}上单调下降\end{CJK}, \begin{CJK}{UTF8}{mj}且\end{CJK} $f^{\prime}(b)>0$. \begin{CJK}{UTF8}{mj}证明\end{CJK}:

\end{enumerate}
$$
\left|\int_{a}^{b} \cos f(x) \mathrm{d} x\right| \leqslant \frac{2}{f^{\prime}(b)} .
$$

\section{9. 浙江大学 2010 年研究生入学考试试题数学分析}
\begin{CJK}{UTF8}{mj}李扬\end{CJK}

\begin{CJK}{UTF8}{mj}微信公众号\end{CJK}: sxkyliyang

\begin{enumerate}
  \item (\begin{CJK}{UTF8}{mj}每小题\end{CJK} 10 \begin{CJK}{UTF8}{mj}分\end{CJK}, \begin{CJK}{UTF8}{mj}共\end{CJK} 60 \begin{CJK}{UTF8}{mj}分\end{CJK}) \begin{CJK}{UTF8}{mj}计算下列极限和积分\end{CJK}
\end{enumerate}
(1) $\lim _{n \rightarrow \infty} \sum_{k=n^{2}}^{(n+1)^{2}} \frac{1}{\sqrt{k}}$.

(2) $\iint_{[0, \pi] \times[0,1]} y \sin (x y) \mathrm{d} x \mathrm{~d} y$.

(3) $\lim _{x \rightarrow 0} \frac{e^{x} \sin x-x(1+x)}{\sin ^{3} x}$.

(4) \begin{CJK}{UTF8}{mj}计算\end{CJK} $\iint_{\Sigma} z \mathrm{~d} x \mathrm{~d} y$. \begin{CJK}{UTF8}{mj}其中\end{CJK} $\Sigma$ \begin{CJK}{UTF8}{mj}是三角形\end{CJK} $\{(x, y, z): x, y, z \geqslant 0, x+y+z=1\}$, \begin{CJK}{UTF8}{mj}其法方向与\end{CJK} $(1,1,1)$ \begin{CJK}{UTF8}{mj}方向相同\end{CJK}.

(5) $\int_{0}^{2 \pi} \sqrt{1+\sin x} \mathrm{~d} x$.

(6) $\int_{0}^{1} \frac{\ln (1+x)}{1+x^{2}} \mathrm{~d} x$.

\begin{enumerate}
  \setcounter{enumi}{2}
  \item ( 15 \begin{CJK}{UTF8}{mj}分\end{CJK}) \begin{CJK}{UTF8}{mj}设\end{CJK} $a_{n}=\sin a_{n-1}, n \geqslant 2$, \begin{CJK}{UTF8}{mj}且\end{CJK} $a_{1}>0$, \begin{CJK}{UTF8}{mj}计算\end{CJK}:
\end{enumerate}
$$
\lim _{n \rightarrow \infty} \sqrt{\frac{n}{3}} a_{n} .
$$

\begin{enumerate}
  \setcounter{enumi}{3}
  \item (15 \begin{CJK}{UTF8}{mj}分\end{CJK}) \begin{CJK}{UTF8}{mj}设函数\end{CJK} $f(x)$ \begin{CJK}{UTF8}{mj}在\end{CJK} $(-\infty,+\infty)$ \begin{CJK}{UTF8}{mj}上连续\end{CJK}, $n$ \begin{CJK}{UTF8}{mj}为奇数\end{CJK}. \begin{CJK}{UTF8}{mj}证明\end{CJK}: \begin{CJK}{UTF8}{mj}若\end{CJK} $\lim _{x \rightarrow+\infty} \frac{f(x)}{x^{n}}=\lim _{x \rightarrow-\infty} \frac{f(x)}{x^{n}}=1$, \begin{CJK}{UTF8}{mj}则方程\end{CJK}
\end{enumerate}
$$
f(x)+x^{n}=0
$$
\begin{CJK}{UTF8}{mj}有实根\end{CJK}.

\begin{enumerate}
  \setcounter{enumi}{4}
  \item ( 20 \begin{CJK}{UTF8}{mj}分\end{CJK}) \begin{CJK}{UTF8}{mj}证明\end{CJK}
\end{enumerate}
$$
\int_{0}^{+\infty} \frac{\sin x y}{y} \mathrm{~d} y
$$
\begin{CJK}{UTF8}{mj}在\end{CJK} $[\delta,+\infty)$ \begin{CJK}{UTF8}{mj}上一致收敛\end{CJK} (\begin{CJK}{UTF8}{mj}其中\end{CJK} $\delta>0)$.

\begin{enumerate}
  \setcounter{enumi}{5}
  \item (20 \begin{CJK}{UTF8}{mj}分\end{CJK}) \begin{CJK}{UTF8}{mj}设\end{CJK} $f(x)$ \begin{CJK}{UTF8}{mj}连续\end{CJK}, \begin{CJK}{UTF8}{mj}证明\end{CJK} Poisson \begin{CJK}{UTF8}{mj}公式\end{CJK}:
\end{enumerate}
$$
\int_{x^{2}+y^{2}+z^{2}=1} f(a x+b y+c z) \mathrm{d} S=2 \pi \int_{-1}^{1} f\left(\sqrt{a^{2}+b^{2}+c^{2}} t\right) \mathrm{d} t .
$$

\begin{enumerate}
  \setcounter{enumi}{6}
  \item ( 20 \begin{CJK}{UTF8}{mj}分\end{CJK}) \begin{CJK}{UTF8}{mj}设\end{CJK} $\left\{a_{n}\right\}_{n \geqslant 1},\left\{b_{n}\right\}_{n \geqslant 1}$ \begin{CJK}{UTF8}{mj}为实数数列\end{CJK}, \begin{CJK}{UTF8}{mj}满足\end{CJK}
\end{enumerate}
(1) $\lim _{n \rightarrow \infty}\left|b_{n}\right|=\infty$;

(2) $\left\{\frac{1}{\left|b_{n}\right|} \sum_{i=1}^{n-1}\left|b_{i+1}-b_{i}\right|\right\}_{n \geqslant 1}$ \begin{CJK}{UTF8}{mj}有界\end{CJK}.

\begin{CJK}{UTF8}{mj}证明\end{CJK}: \begin{CJK}{UTF8}{mj}若\end{CJK} $\lim _{n \rightarrow \infty} \frac{a_{n+1}-a_{n}}{b_{n+1}-b_{n}}$ \begin{CJK}{UTF8}{mj}存在\end{CJK}, \begin{CJK}{UTF8}{mj}则\end{CJK} $\lim _{n \rightarrow \infty} \frac{a_{n}}{b_{n}}$ \begin{CJK}{UTF8}{mj}也存在\end{CJK}.

\section{0. 浙江大学 2011 年研究生入学考试试题数学分析 
 李扬 
 微信公众号: sxkyliyang}
\begin{enumerate}
  \item (\begin{CJK}{UTF8}{mj}每小题\end{CJK} 10 \begin{CJK}{UTF8}{mj}分\end{CJK}, \begin{CJK}{UTF8}{mj}共\end{CJK} 60 \begin{CJK}{UTF8}{mj}分\end{CJK}) \begin{CJK}{UTF8}{mj}计算题\end{CJK}
\end{enumerate}
(1) $\lim _{x \rightarrow 0} \frac{\tan x-\sin x}{\sin x^{3}}$.

(2) $\iint_{[0,2] \times[0,2]}[x+y] \mathrm{d} x \mathrm{~d} y$, \begin{CJK}{UTF8}{mj}其中\end{CJK} $[\alpha]$ \begin{CJK}{UTF8}{mj}表示\end{CJK} $\alpha$ \begin{CJK}{UTF8}{mj}的整数部分\end{CJK}.

(3) $F(x)=\int_{x}^{x^{2}} \frac{\sin x y}{y} \mathrm{~d} y, x>0$, \begin{CJK}{UTF8}{mj}求\end{CJK} $F^{\prime}(x)$.

(4) \begin{CJK}{UTF8}{mj}计算\end{CJK} $\iint_{\Sigma} y(x-z) \mathrm{d} y \mathrm{~d} z+x^{2} \mathrm{~d} z \mathrm{~d} x+\left(y^{2}+x z\right) \mathrm{d} x \mathrm{~d} y$. \begin{CJK}{UTF8}{mj}其中\end{CJK} $\Sigma$ \begin{CJK}{UTF8}{mj}是\end{CJK} $x=0, y=0, z=0, x=a, y=a, z=a$ \begin{CJK}{UTF8}{mj}这六个平面所围立方体表面\end{CJK}, \begin{CJK}{UTF8}{mj}正法方向为立方体表面外侧\end{CJK}.

(5) $\int_{0}^{1} \frac{\ln x}{1-x} \mathrm{~d} x$.

(6) $\int_{0}^{1} \frac{\arctan x}{x \sqrt{1-x^{2}}} \mathrm{~d} x$.

\begin{enumerate}
  \setcounter{enumi}{2}
  \item (15 \begin{CJK}{UTF8}{mj}分\end{CJK}) \begin{CJK}{UTF8}{mj}设函数\end{CJK}
\end{enumerate}
$$
f(x, y)= \begin{cases}\frac{x y}{\sqrt{x^{2}+y^{2}}}, & (x, y) \neq(0,0) \\ 0, & (x, y)=(0,0) .\end{cases}
$$
\begin{CJK}{UTF8}{mj}试证\end{CJK}: $f(x, y)$ \begin{CJK}{UTF8}{mj}在平面\end{CJK} $\mathbb{R}^{2}$ \begin{CJK}{UTF8}{mj}上连续\end{CJK}, \begin{CJK}{UTF8}{mj}偏导数\end{CJK} $f_{x}^{\prime}(x, y), f_{y}^{\prime}(x, y)$ \begin{CJK}{UTF8}{mj}有界\end{CJK}, $f(x, y)$ \begin{CJK}{UTF8}{mj}在原点\end{CJK} $(0,0)$ \begin{CJK}{UTF8}{mj}处不可微\end{CJK}.

\begin{enumerate}
  \setcounter{enumi}{3}
  \item (15 \begin{CJK}{UTF8}{mj}分\end{CJK}) \begin{CJK}{UTF8}{mj}设\end{CJK} $f(x)$ \begin{CJK}{UTF8}{mj}是\end{CJK} $[a, a+1]$ ( $a$ \begin{CJK}{UTF8}{mj}为常数\end{CJK}) \begin{CJK}{UTF8}{mj}上的连续正值函数\end{CJK}, \begin{CJK}{UTF8}{mj}记\end{CJK} $M=\max _{x \in[a, a+1]} f(x)$. \begin{CJK}{UTF8}{mj}证明\end{CJK}:
\end{enumerate}
$$
A_{n}=\sqrt[n]{\int_{a}^{a+1}[f(x)]^{n} \mathrm{~d} x}
$$
\begin{CJK}{UTF8}{mj}关于\end{CJK} $n$ \begin{CJK}{UTF8}{mj}单调递增\end{CJK}, \begin{CJK}{UTF8}{mj}且\end{CJK}
$$
\lim _{n \rightarrow \infty} \sqrt[n]{\int_{a}^{a+1}[f(x)]^{n} \mathrm{~d} x}=M
$$

\begin{enumerate}
  \setcounter{enumi}{4}
  \item (15 \begin{CJK}{UTF8}{mj}分\end{CJK}) \begin{CJK}{UTF8}{mj}设\end{CJK} $\operatorname{sh} x \cdot \operatorname{sh} y=1$, \begin{CJK}{UTF8}{mj}其中\end{CJK} $\operatorname{sh} x=\frac{e^{x}-e^{-x}}{2}$. \begin{CJK}{UTF8}{mj}计算\end{CJK}
\end{enumerate}
$$
\int_{0}^{+\infty} y(x) \mathrm{d} x
$$

\begin{enumerate}
  \setcounter{enumi}{5}
  \item ( 15 \begin{CJK}{UTF8}{mj}分\end{CJK}) \begin{CJK}{UTF8}{mj}讨论级数\end{CJK}
\end{enumerate}
$$
\sum_{n=1}^{\infty} \frac{(-1)^{n}}{\left(1+x^{2}\right)^{n}}
$$
\begin{CJK}{UTF8}{mj}在\end{CJK} $(-\infty,+\infty)$ \begin{CJK}{UTF8}{mj}上的收敛性和一致收敛性\end{CJK}.

\begin{enumerate}
  \setcounter{enumi}{6}
  \item (15 \begin{CJK}{UTF8}{mj}分\end{CJK}) \begin{CJK}{UTF8}{mj}设\end{CJK} $a_{1}, b_{1}$ \begin{CJK}{UTF8}{mj}为任意选定的实数\end{CJK}, $a_{n}, b_{n}$ \begin{CJK}{UTF8}{mj}定义为\end{CJK}:
\end{enumerate}
$$
\begin{cases}a_{n}=\int_{0}^{1} \max \left\{b_{n-1}, x\right\} \mathrm{d} x, & n=2,3, \cdots \\ b_{n}=\int_{0}^{1} \min \left\{a_{n-1}, x\right\} \mathrm{d} x, & n=2,3, \cdots\end{cases}
$$
\begin{CJK}{UTF8}{mj}试证\end{CJK}: $\lim _{n \rightarrow \infty} a_{n}=2-\sqrt{2}, \lim _{n \rightarrow \infty} b_{n}=\sqrt{2}-1 .$ 7. ( 15 \begin{CJK}{UTF8}{mj}分\end{CJK}) \begin{CJK}{UTF8}{mj}设\end{CJK} $a_{1} \in(0,1)$, \begin{CJK}{UTF8}{mj}且\end{CJK} $a_{n+1}=a_{n}\left(1-a_{n}\right), n \geqslant 1$, \begin{CJK}{UTF8}{mj}证明\end{CJK}:
$$
\lim _{n \rightarrow \infty} n a_{n}=1
$$

\section{1. 浙江大学 2012 年研究生入学考试试题数学分析}
\begin{CJK}{UTF8}{mj}李扬\end{CJK}

\begin{CJK}{UTF8}{mj}微信公众号\end{CJK}: sxkyliyang

\begin{enumerate}
  \item ( 15 \begin{CJK}{UTF8}{mj}分\end{CJK}) \begin{CJK}{UTF8}{mj}设\end{CJK} $n$ \begin{CJK}{UTF8}{mj}为正整数\end{CJK}, $f_{n}(x)=\int_{-1}^{1}\left(1-t^{2}\right)^{n} \cos x t \mathrm{~d} t, x \in \mathbb{R}$. \begin{CJK}{UTF8}{mj}证明\end{CJK}:
\end{enumerate}
$$
x^{2} f_{n}(x)=2 n(2 n-1) f_{n-1}(x)-4 n(n-1) f_{n-2}(x), n \geqslant 2
$$

\begin{enumerate}
  \setcounter{enumi}{2}
  \item (15 \begin{CJK}{UTF8}{mj}分\end{CJK}) \begin{CJK}{UTF8}{mj}设\end{CJK} $f$ \begin{CJK}{UTF8}{mj}在\end{CJK} $[0,1]$ \begin{CJK}{UTF8}{mj}上连续\end{CJK}, \begin{CJK}{UTF8}{mj}且对任意\end{CJK} $x, y \in[0,1]$ \begin{CJK}{UTF8}{mj}有\end{CJK} $f\left(\frac{x+y}{2}\right) \leqslant \frac{f(x)+f(y)}{2}$. \begin{CJK}{UTF8}{mj}证明\end{CJK}:
\end{enumerate}
$$
\int_{0}^{1} f(x) \mathrm{d} x \geqslant f\left(\frac{1}{2}\right)
$$

\begin{enumerate}
  \setcounter{enumi}{3}
  \item ( 20 \begin{CJK}{UTF8}{mj}分\end{CJK}) \begin{CJK}{UTF8}{mj}设实数\end{CJK} $\lambda,|\lambda|<1$, \begin{CJK}{UTF8}{mj}求\end{CJK}
\end{enumerate}
$$
f(\lambda)=\int_{0}^{\pi} \ln (1+\lambda \cos x) \mathrm{d} x
$$

\begin{enumerate}
  \setcounter{enumi}{4}
  \item ( 20 \begin{CJK}{UTF8}{mj}分\end{CJK}) \begin{CJK}{UTF8}{mj}设函数\end{CJK} $f: \mathbb{R}^{+} \rightarrow \mathbb{R}, a_{i}(i \geqslant 0)$ \begin{CJK}{UTF8}{mj}为实数\end{CJK}, \begin{CJK}{UTF8}{mj}且对充分大的\end{CJK} $x$, \begin{CJK}{UTF8}{mj}有\end{CJK}
\end{enumerate}
$$
f(x)=a_{0}+\frac{a_{1}}{x}+\cdots+\frac{a_{n}}{x^{n}}+\cdots
$$
\begin{CJK}{UTF8}{mj}证明\end{CJK}: $\sum_{n=1}^{\infty} f(n)$ \begin{CJK}{UTF8}{mj}收敛的充要条件是\end{CJK} $a_{0}=a_{1}=0$.

\begin{enumerate}
  \setcounter{enumi}{5}
  \item ( 20 \begin{CJK}{UTF8}{mj}分\end{CJK}) \begin{CJK}{UTF8}{mj}如果任意\end{CJK} $\varepsilon>0$, \begin{CJK}{UTF8}{mj}存在\end{CJK} $N$, \begin{CJK}{UTF8}{mj}当\end{CJK} $n, m>N$ \begin{CJK}{UTF8}{mj}时\end{CJK}, \begin{CJK}{UTF8}{mj}有\end{CJK} $\left|x_{m}-x_{n}\right|<\varepsilon$, \begin{CJK}{UTF8}{mj}则称数列\end{CJK} $\left\{x_{n}\right\}$ \begin{CJK}{UTF8}{mj}为\end{CJK} Cauchy \begin{CJK}{UTF8}{mj}数列\end{CJK}. \begin{CJK}{UTF8}{mj}证明\end{CJK}: \begin{CJK}{UTF8}{mj}函数\end{CJK} $f$ \begin{CJK}{UTF8}{mj}在有界区间\end{CJK} $A$ \begin{CJK}{UTF8}{mj}上一致连续的充要条件是对\end{CJK} $A$ \begin{CJK}{UTF8}{mj}中任意\end{CJK} Cauchy \begin{CJK}{UTF8}{mj}数列\end{CJK} $\left\{x_{n}\right\}$, \begin{CJK}{UTF8}{mj}数列\end{CJK} $\left\{f\left(x_{n}\right)\right\}$ \begin{CJK}{UTF8}{mj}为\end{CJK} Cauchy \begin{CJK}{UTF8}{mj}数\end{CJK} \begin{CJK}{UTF8}{mj}列\end{CJK}.

  \item (30 \begin{CJK}{UTF8}{mj}分\end{CJK}) \begin{CJK}{UTF8}{mj}设\end{CJK} $f(x)$ \begin{CJK}{UTF8}{mj}在\end{CJK} $x=0$ \begin{CJK}{UTF8}{mj}的邻域内有连续的一阶导数\end{CJK}, \begin{CJK}{UTF8}{mj}且\end{CJK} $f^{\prime}(0)=0, f^{\prime \prime}(0)=1$. \begin{CJK}{UTF8}{mj}求\end{CJK}

\end{enumerate}
$$
\lim _{x \rightarrow 0} \frac{f(x)-f(\ln (1+x))}{x^{3}} .
$$

\begin{enumerate}
  \setcounter{enumi}{7}
  \item (30 \begin{CJK}{UTF8}{mj}分\end{CJK}) \begin{CJK}{UTF8}{mj}设实数\end{CJK} $\lambda>-4$, \begin{CJK}{UTF8}{mj}数列\end{CJK} $\left\{x_{n}\right\}$ \begin{CJK}{UTF8}{mj}满足\end{CJK}
\end{enumerate}
$$
x_{1}=\frac{\lambda}{2}, x_{n+1}=x_{1}+\frac{x_{n}^{2}}{2}, n \geqslant 1 .
$$
\begin{CJK}{UTF8}{mj}试讨论数列\end{CJK} $\left\{x_{n}\right\}$ \begin{CJK}{UTF8}{mj}的收敛性\end{CJK}.

\section{2. 浙江大学 2013 年研究生入学考试试题数学分析 
 李扬 
 微信公众号: sxkyliyang}
\begin{enumerate}
  \item (\begin{CJK}{UTF8}{mj}每小题\end{CJK} 10 \begin{CJK}{UTF8}{mj}分\end{CJK}, \begin{CJK}{UTF8}{mj}共\end{CJK} 40 \begin{CJK}{UTF8}{mj}分\end{CJK})
\end{enumerate}
(1) $\lim _{x \rightarrow 0} \frac{\sin x-\arctan x}{\tan x-\arcsin x}$.

(2) $\int_{0}^{\pi} \frac{\cos 4 \theta}{1+\cos ^{2} \theta} \mathrm{d} \theta$.

(3) \begin{CJK}{UTF8}{mj}设\end{CJK} $D=\left\{(x, y) \mid x^{2}+y^{2} \leqslant \sqrt{3}, x \geqslant 0, y \geqslant 0\right\},\left[1+x^{2}+y^{2}\right]$ \begin{CJK}{UTF8}{mj}表示不超过\end{CJK} $1+x^{2}+y^{2}$ \begin{CJK}{UTF8}{mj}的最大整数\end{CJK}, \begin{CJK}{UTF8}{mj}计算二\end{CJK}. \begin{CJK}{UTF8}{mj}重积分\end{CJK} $\iint_{D} x y\left[1+x^{2}+y^{2}\right] \mathrm{d} x \mathrm{~d} y$.

(4) \begin{CJK}{UTF8}{mj}设\end{CJK} $S_{n}=\frac{1}{\sqrt{n}}\left(1+\frac{1}{\sqrt{2}}+\cdots+\frac{1}{\sqrt{n}}\right)$. \begin{CJK}{UTF8}{mj}求\end{CJK} $\lim _{n \rightarrow+\infty} S_{n}$.

\begin{enumerate}
  \setcounter{enumi}{2}
  \item ( 10 \begin{CJK}{UTF8}{mj}分\end{CJK}) \begin{CJK}{UTF8}{mj}论证是否存在定义在\end{CJK} $\mathbb{R}$ \begin{CJK}{UTF8}{mj}上的连续函数使得\end{CJK}
\end{enumerate}
$$
f(f(x))=e^{-x}
$$

\begin{enumerate}
  \setcounter{enumi}{3}
  \item ( 15 \begin{CJK}{UTF8}{mj}分\end{CJK}) \begin{CJK}{UTF8}{mj}讨论函数项级数\end{CJK}
\end{enumerate}
$$
\sum_{n=1}^{+\infty} \frac{\sqrt{n+1}-\sqrt{n}}{n^{x}}
$$
\begin{CJK}{UTF8}{mj}的收敛性与一致收敛性\end{CJK}.

\begin{enumerate}
  \setcounter{enumi}{4}
  \item (15 \begin{CJK}{UTF8}{mj}分\end{CJK}) \begin{CJK}{UTF8}{mj}设\end{CJK} $f(x), g(x), \varphi(x)$ \begin{CJK}{UTF8}{mj}均为\end{CJK} $[a, b]$ \begin{CJK}{UTF8}{mj}上的连续函数\end{CJK}, \begin{CJK}{UTF8}{mj}且\end{CJK} $g(x)$ \begin{CJK}{UTF8}{mj}为单调递增的\end{CJK}, $\varphi(x) \geqslant 0$, \begin{CJK}{UTF8}{mj}同时对于任意的\end{CJK} $x \in[a, b]$, \begin{CJK}{UTF8}{mj}有\end{CJK} $f(x) \leqslant g(x)+\int_{a}^{x} \varphi(t) f(t) \mathrm{d} t$. \begin{CJK}{UTF8}{mj}证明\end{CJK}: \begin{CJK}{UTF8}{mj}对于任意的\end{CJK} $x \in[a, b]$, \begin{CJK}{UTF8}{mj}都有\end{CJK}
\end{enumerate}
$$
f(x) \leqslant g(x) e^{\int_{a}^{x} \varphi(s) \mathrm{d} s}
$$

\begin{enumerate}
  \setcounter{enumi}{5}
  \item (5 \begin{CJK}{UTF8}{mj}分\end{CJK}) (1) $\lim _{n \rightarrow+\infty} \int_{0}^{\frac{\pi}{2}} \sin ^{n} x \mathrm{~d} x=0$;
\end{enumerate}
(10 \begin{CJK}{UTF8}{mj}分\end{CJK}) (2) $\lim _{n \rightarrow+\infty} \int_{0}^{\frac{\pi}{2}} \sin x^{n} \mathrm{~d} x=0 .$

\begin{enumerate}
  \setcounter{enumi}{6}
  \item ( 5 \begin{CJK}{UTF8}{mj}分\end{CJK}) (1) \begin{CJK}{UTF8}{mj}构造一个在闭区间\end{CJK} $[-1,1]$ \begin{CJK}{UTF8}{mj}上处处可微的函数\end{CJK}, \begin{CJK}{UTF8}{mj}使得它的导函数在\end{CJK} $[-1,1]$ \begin{CJK}{UTF8}{mj}上无界\end{CJK};
\end{enumerate}
( 15 \begin{CJK}{UTF8}{mj}分\end{CJK}) (2) \begin{CJK}{UTF8}{mj}设函数\end{CJK} $f(x)$ \begin{CJK}{UTF8}{mj}在\end{CJK} $(a, b)$ \begin{CJK}{UTF8}{mj}内可导\end{CJK}, \begin{CJK}{UTF8}{mj}证明存在\end{CJK} $(\alpha, \beta) \subset(a, b)$, \begin{CJK}{UTF8}{mj}使得\end{CJK} $f^{\prime}(x)$ \begin{CJK}{UTF8}{mj}在\end{CJK} $(\alpha, \beta)$ \begin{CJK}{UTF8}{mj}内有界\end{CJK}.

\begin{enumerate}
  \setcounter{enumi}{7}
  \item ( 15 \begin{CJK}{UTF8}{mj}分\end{CJK}) \begin{CJK}{UTF8}{mj}设二元函数\end{CJK} $f(x, y)$ \begin{CJK}{UTF8}{mj}的两个混合偏导数\end{CJK} $f_{x y}(x, y), f_{y x}(x, y)$ \begin{CJK}{UTF8}{mj}在\end{CJK} $(0,0)$ \begin{CJK}{UTF8}{mj}附近存在\end{CJK}, \begin{CJK}{UTF8}{mj}且\end{CJK} $f_{x y}(x, y)$ \begin{CJK}{UTF8}{mj}在\end{CJK} $(0,0)$ \begin{CJK}{UTF8}{mj}处连续\end{CJK}. \begin{CJK}{UTF8}{mj}证明\end{CJK}:
\end{enumerate}
$$
f_{x y}(0,0)=f_{y x}(0,0) .
$$

\begin{enumerate}
  \setcounter{enumi}{8}
  \item ( 20 \begin{CJK}{UTF8}{mj}分\end{CJK}) \begin{CJK}{UTF8}{mj}已知对于实数\end{CJK} $n \geqslant 2$, \begin{CJK}{UTF8}{mj}有公式\end{CJK} $\sum_{p \leqslant n} \frac{\ln p}{p}=\ln n+O(1)$, \begin{CJK}{UTF8}{mj}其中求和是对所有不超过\end{CJK} $n$ \begin{CJK}{UTF8}{mj}的素数\end{CJK} $p$ \begin{CJK}{UTF8}{mj}求和\end{CJK}. \begin{CJK}{UTF8}{mj}求\end{CJK} \begin{CJK}{UTF8}{mj}证\end{CJK}:
\end{enumerate}
$$
\sum_{p \leqslant n} \frac{1}{p}=C+\ln \ln n+O\left(\frac{1}{\ln n}\right) .
$$
\begin{CJK}{UTF8}{mj}其中求和也是对所有不超过\end{CJK} $n$ \begin{CJK}{UTF8}{mj}的素数\end{CJK} $p$ \begin{CJK}{UTF8}{mj}求和\end{CJK}, $C$ \begin{CJK}{UTF8}{mj}是某个与\end{CJK} $n$ \begin{CJK}{UTF8}{mj}无关的常数\end{CJK}.

\section{3. 浙江大学 2014 年研究生入学考试试题数学分析 
 李扬 
 微信公众号: sxkyliyang}
\begin{enumerate}
  \item (\begin{CJK}{UTF8}{mj}每小题\end{CJK} 10 \begin{CJK}{UTF8}{mj}分\end{CJK}, \begin{CJK}{UTF8}{mj}共\end{CJK} 40 \begin{CJK}{UTF8}{mj}分\end{CJK})
\end{enumerate}
(1) \begin{CJK}{UTF8}{mj}求\end{CJK}
$$
\lim _{x \rightarrow 1} \frac{e^{\frac{x-1}{2}}-\sqrt{x}}{\ln ^{2}(2 x-1)}
$$
(2) \begin{CJK}{UTF8}{mj}求\end{CJK}
$$
\int \frac{t^{2}}{(1-t)^{2013}} \mathrm{~d} t
$$
(3) \begin{CJK}{UTF8}{mj}求\end{CJK}
$$
\iint_{\mathbb{R}^{2}} e^{-\left(x^{2}+x y+y^{2}\right)} \mathrm{d} x \mathrm{~d} y
$$
(4) \begin{CJK}{UTF8}{mj}求\end{CJK}
$$
\iint_{S} x^{3} \mathrm{~d} y \mathrm{~d} z+y^{3} \mathrm{~d} z \mathrm{~d} x+z^{3} \mathrm{~d} x \mathrm{~d} y .
$$
\begin{CJK}{UTF8}{mj}其中\end{CJK} $S$ \begin{CJK}{UTF8}{mj}为\end{CJK} $x^{2}+y^{2}+z^{2}=1$ \begin{CJK}{UTF8}{mj}上半球面下侧\end{CJK}.

\begin{enumerate}
  \setcounter{enumi}{2}
  \item (1) \begin{CJK}{UTF8}{mj}用闭区间套定理证明有限覆盖定理\end{CJK}.
\end{enumerate}
(2) \begin{CJK}{UTF8}{mj}用有限覆盖定理证明\end{CJK}: \begin{CJK}{UTF8}{mj}对\end{CJK} $[a, b]$ \begin{CJK}{UTF8}{mj}上连续函数\end{CJK} $f(x), f(x)>0$, \begin{CJK}{UTF8}{mj}存在常数\end{CJK} $c, c>0$, \begin{CJK}{UTF8}{mj}使得\end{CJK} $f(x) \geqslant c, x \in[a, b]$.
$$
f(x, y)= \begin{cases}\frac{x y}{\left(x^{2}+y^{2}\right)^{\alpha}}, & (x, y) \neq(0,0) \\ 0, & (x, y)=(0,0)\end{cases}
$$
\begin{CJK}{UTF8}{mj}求满足条件的\end{CJK} $\alpha$, \begin{CJK}{UTF8}{mj}使得\end{CJK} $f$ \begin{CJK}{UTF8}{mj}在原点满足\end{CJK}\\
(1) \begin{CJK}{UTF8}{mj}连续\end{CJK};\\
(2) \begin{CJK}{UTF8}{mj}可微\end{CJK};\\
(3) \begin{CJK}{UTF8}{mj}方向导数存在\end{CJK}.

\begin{enumerate}
  \setcounter{enumi}{4}
  \item \begin{CJK}{UTF8}{mj}和函数\end{CJK}
\end{enumerate}
$$
\sum_{i=1}^{\infty} \frac{\ln \left(1+n^{2} x^{2}\right)}{n^{3}}, x \in[0,1] .
$$
\begin{CJK}{UTF8}{mj}证明其对\end{CJK} $x$ \begin{CJK}{UTF8}{mj}一致收敛并分析其连续性\end{CJK}、\begin{CJK}{UTF8}{mj}可积性和可微性\end{CJK}.

\begin{enumerate}
  \setcounter{enumi}{5}
  \item $f(x)$ \begin{CJK}{UTF8}{mj}可微\end{CJK}, \begin{CJK}{UTF8}{mj}则\end{CJK} $f^{\prime}(x)$ \begin{CJK}{UTF8}{mj}可积的充要条件是\end{CJK}: \begin{CJK}{UTF8}{mj}存在可积函数\end{CJK} $g(x)$, st
\end{enumerate}
$$
f(x)=f(a)+\int_{a}^{x} g(t) \mathrm{d} t .
$$

\begin{enumerate}
  \setcounter{enumi}{6}
  \item \begin{CJK}{UTF8}{mj}空间体积为\end{CJK} $V$ \begin{CJK}{UTF8}{mj}的\end{CJK} $\Omega, X_{0} \in \Omega, 0<\alpha<3$, \begin{CJK}{UTF8}{mj}证明\end{CJK}:
\end{enumerate}
$$
\int_{\Omega}\left|X-X_{0}\right|^{\alpha-3} \mathrm{~d} X \leqslant C V^{\frac{\alpha}{3}}
$$
\begin{CJK}{UTF8}{mj}其中\end{CJK} $C$ \begin{CJK}{UTF8}{mj}与\end{CJK} $\alpha$ \begin{CJK}{UTF8}{mj}有关\end{CJK}.

\begin{enumerate}
  \setcounter{enumi}{7}
  \item $f(x)$ \begin{CJK}{UTF8}{mj}在\end{CJK} $[0,1]$ \begin{CJK}{UTF8}{mj}单调\end{CJK}, \begin{CJK}{UTF8}{mj}证明\end{CJK}:
\end{enumerate}
$$
\lim _{y \rightarrow+\infty} \int_{0}^{1} f(x) \frac{\sin x y}{x} \mathrm{~d} x=\frac{\pi}{2} f(0+) .
$$

\begin{enumerate}
  \setcounter{enumi}{8}
  \item $f(x)$ \begin{CJK}{UTF8}{mj}在\end{CJK} $[a,+\infty)$ \begin{CJK}{UTF8}{mj}一致连续\end{CJK}, \begin{CJK}{UTF8}{mj}且对任意\end{CJK} $\xi>0$, \begin{CJK}{UTF8}{mj}序列\end{CJK} $\{f(n \xi)\}$ \begin{CJK}{UTF8}{mj}极限存在\end{CJK}, \begin{CJK}{UTF8}{mj}求证\end{CJK}: $\lim _{x \rightarrow \infty} f(x)$ \begin{CJK}{UTF8}{mj}存在\end{CJK}.
\end{enumerate}
\section{4. 浙江大学 2015 年研究生入学考试试题数学分析 
 李扬 
 微信公众号: sxkyliyang}
\begin{enumerate}
  \item \begin{CJK}{UTF8}{mj}求极限\end{CJK}
\end{enumerate}
$$
\lim _{n \rightarrow \infty} \frac{\left(n^{2}+1\right)\left(n^{2}+2\right) \cdots\left(n^{2}+n\right)}{\left(n^{2}-1\right)\left(n^{2}-2\right) \cdots\left(n^{2}-n\right)}
$$

\begin{enumerate}
  \setcounter{enumi}{2}
  \item \begin{CJK}{UTF8}{mj}求\end{CJK}
\end{enumerate}
$$
\lim _{x \rightarrow 0^{+}}\left[\frac{1}{x^{5}} \int_{0}^{x} e^{-t^{2}} \mathrm{~d} t+\frac{1}{3} \frac{1}{x^{2}}-\frac{1}{x^{4}}\right]
$$

\begin{enumerate}
  \setcounter{enumi}{3}
  \item \begin{CJK}{UTF8}{mj}设\end{CJK}
\end{enumerate}
$$
I(r)=\oint_{L} \frac{y}{x^{2}+y^{2}} \mathrm{~d} x-\frac{x}{x^{2}+y^{2}} \mathrm{~d} y,
$$
\begin{CJK}{UTF8}{mj}其中\end{CJK} $L$ \begin{CJK}{UTF8}{mj}为\end{CJK} $x^{2}+x y+y^{2}=r^{2}$, \begin{CJK}{UTF8}{mj}取正方向\end{CJK}. \begin{CJK}{UTF8}{mj}求\end{CJK} $\lim _{r \rightarrow \infty} I(r)$.

\begin{enumerate}
  \setcounter{enumi}{4}
  \item \begin{CJK}{UTF8}{mj}求\end{CJK}
\end{enumerate}
$$
\int_{e^{2 n \pi}}^{0} \sin \ln \frac{1}{x} \mathrm{~d} x
$$

\begin{enumerate}
  \setcounter{enumi}{5}
  \item \begin{CJK}{UTF8}{mj}考察\end{CJK} Riemann \begin{CJK}{UTF8}{mj}函数的连续性\end{CJK}, \begin{CJK}{UTF8}{mj}可微性和可积性\end{CJK}.

  \item $f$ \begin{CJK}{UTF8}{mj}为定义在某区域\end{CJK} $D \subset \mathbb{R}^{n}$ \begin{CJK}{UTF8}{mj}上的一个函数\end{CJK}, \begin{CJK}{UTF8}{mj}有一阶连续偏导数\end{CJK}, \begin{CJK}{UTF8}{mj}且偏导数有界\end{CJK}.

\end{enumerate}
(1) \begin{CJK}{UTF8}{mj}若\end{CJK} $D$ \begin{CJK}{UTF8}{mj}为凸区域\end{CJK}, \begin{CJK}{UTF8}{mj}证明\end{CJK}: $f$ \begin{CJK}{UTF8}{mj}一致连续\end{CJK}.

(2) \begin{CJK}{UTF8}{mj}考察\end{CJK} $D$ \begin{CJK}{UTF8}{mj}不是凸区域的情况\end{CJK}.

\begin{enumerate}
  \setcounter{enumi}{7}
  \item \begin{CJK}{UTF8}{mj}设\end{CJK} $\left\{f_{n}\right\}$ \begin{CJK}{UTF8}{mj}是\end{CJK} $\mathbb{R}$ \begin{CJK}{UTF8}{mj}上的函数列\end{CJK}, \begin{CJK}{UTF8}{mj}且对任意\end{CJK} $x \in \mathbb{R},\left\{f_{n}(x)\right\}$ \begin{CJK}{UTF8}{mj}有界\end{CJK}. \begin{CJK}{UTF8}{mj}证明\end{CJK}: \begin{CJK}{UTF8}{mj}存在一个开区间\end{CJK} $(a, b)$, \begin{CJK}{UTF8}{mj}使得\end{CJK} $\left\{f_{n}(x)\right\}$ \begin{CJK}{UTF8}{mj}在该区\end{CJK} \begin{CJK}{UTF8}{mj}间上一致有界\end{CJK}.

  \item (1) \begin{CJK}{UTF8}{mj}证明\end{CJK} $\Gamma(s)$ \begin{CJK}{UTF8}{mj}在\end{CJK} $(0, \infty)$ \begin{CJK}{UTF8}{mj}内无穷次可微\end{CJK}.

\end{enumerate}
(2) \begin{CJK}{UTF8}{mj}证明\end{CJK} $\Gamma(s), \ln \Gamma(s)$ \begin{CJK}{UTF8}{mj}都是严格凸函数\end{CJK}.

\begin{enumerate}
  \setcounter{enumi}{9}
  \item \begin{CJK}{UTF8}{mj}设\end{CJK} $f$ \begin{CJK}{UTF8}{mj}在\end{CJK} $\mathbb{R}$ \begin{CJK}{UTF8}{mj}上二阶可微\end{CJK}, $f, f^{\prime}, f^{\prime \prime}$ \begin{CJK}{UTF8}{mj}均\end{CJK} $\geqslant 0$, \begin{CJK}{UTF8}{mj}且存在\end{CJK} $c>0$ \begin{CJK}{UTF8}{mj}使得\end{CJK} $f^{\prime \prime}(x) \leqslant c f(x)$. \begin{CJK}{UTF8}{mj}证明\end{CJK}:
\end{enumerate}
(1) $\lim _{x \rightarrow-\infty} f^{\prime}(x)=0$.

(2) \begin{CJK}{UTF8}{mj}存在常数\end{CJK} $a$, \begin{CJK}{UTF8}{mj}使得\end{CJK} $f^{\prime}(x) \leqslant a f(x)$, \begin{CJK}{UTF8}{mj}并求出\end{CJK} $a$.

\begin{enumerate}
  \setcounter{enumi}{10}
  \item \begin{CJK}{UTF8}{mj}证明\end{CJK} Fejer \begin{CJK}{UTF8}{mj}定理\end{CJK}.

  \item \begin{CJK}{UTF8}{mj}设\end{CJK} $f$ \begin{CJK}{UTF8}{mj}在\end{CJK} $[A, B]$ \begin{CJK}{UTF8}{mj}上\end{CJK} Riemann \begin{CJK}{UTF8}{mj}可积\end{CJK}, $0<f<1$, \begin{CJK}{UTF8}{mj}对于任意的\end{CJK} $\varepsilon>0$, \begin{CJK}{UTF8}{mj}构造一个函数\end{CJK} $g$ \begin{CJK}{UTF8}{mj}使得\end{CJK}

\end{enumerate}
(1) $g$ \begin{CJK}{UTF8}{mj}是一个阶梯函数\end{CJK}, \begin{CJK}{UTF8}{mj}取值为\end{CJK} 0 \begin{CJK}{UTF8}{mj}或\end{CJK} 1 .

(2) \begin{CJK}{UTF8}{mj}对于\end{CJK} $\forall[a, b] \subset[A, B]$,
$$
\left|\int_{a}^{b}[f(x)-g(x)] \mathrm{d} x\right|<\varepsilon
$$

\section{5. 浙江大学 2016 年研究生入学考试试题数学分析 
 李扬 
 微信公众号: sxkyliyang}
\begin{enumerate}
  \item (\begin{CJK}{UTF8}{mj}每小题\end{CJK} 10 \begin{CJK}{UTF8}{mj}分\end{CJK}, \begin{CJK}{UTF8}{mj}共\end{CJK} 40 \begin{CJK}{UTF8}{mj}分\end{CJK})
\end{enumerate}
(1) $\lim _{n \rightarrow \infty} \frac{\sqrt[n]{(n+1)(n+2) \cdots(n+n)}}{n}$.

(2) $\int_{0}^{\frac{\pi}{2}} \frac{\sin (2 n+1) x}{\sin x} \mathrm{~d} x$ ( $n$ \begin{CJK}{UTF8}{mj}为自然数\end{CJK}).

(3) $\lim _{x \rightarrow 0} \frac{e^{x} \sin x-x(1+x)}{\cos x \ln (1-2 x)}$.

(4) $\iint_{D} x\left(1+y e^{x^{2}+y^{2}}\right) \mathrm{d} x \mathrm{~d} y$, \begin{CJK}{UTF8}{mj}区域\end{CJK} $D$ \begin{CJK}{UTF8}{mj}表示由\end{CJK} $y=x^{3}, x=-1, y=1$ \begin{CJK}{UTF8}{mj}所围成的区域\end{CJK}.

\begin{enumerate}
  \setcounter{enumi}{2}
  \item (\begin{CJK}{UTF8}{mj}每小题\end{CJK} 10 \begin{CJK}{UTF8}{mj}分\end{CJK}, \begin{CJK}{UTF8}{mj}共\end{CJK} 20 \begin{CJK}{UTF8}{mj}分\end{CJK})
\end{enumerate}
(1) $A, B$ \begin{CJK}{UTF8}{mj}表示数集\end{CJK}, $E=A \cup B$, \begin{CJK}{UTF8}{mj}求证\end{CJK}:
$$
\sup E=\max (\sup A, \sup B) .
$$
(2) \begin{CJK}{UTF8}{mj}已知数列\end{CJK} $x_{n}>0, \overline{\lim }_{n \rightarrow \infty} x_{n} \cdot \varlimsup_{n \rightarrow \infty} \frac{1}{x_{n}}=1$, \begin{CJK}{UTF8}{mj}求证\end{CJK}: $\left\{x_{n}\right\}$ \begin{CJK}{UTF8}{mj}收敛\end{CJK}.

\begin{enumerate}
  \setcounter{enumi}{3}
  \item ( 15 \begin{CJK}{UTF8}{mj}分\end{CJK}) \begin{CJK}{UTF8}{mj}用有限覆盖证明\end{CJK}: \begin{CJK}{UTF8}{mj}有界数列必有其收敛的子列\end{CJK}.

  \item ( 15 \begin{CJK}{UTF8}{mj}分\end{CJK}) $f(x)$ \begin{CJK}{UTF8}{mj}是定义在\end{CJK} $(a, b)$ \begin{CJK}{UTF8}{mj}上的函数\end{CJK}, \begin{CJK}{UTF8}{mj}对于\end{CJK} $(a, b)$ \begin{CJK}{UTF8}{mj}上的任意收敛序列\end{CJK} $\left\{x_{n}\right\}$, \begin{CJK}{UTF8}{mj}都有极限\end{CJK} $\lim _{n \rightarrow \infty} f\left(x_{n}\right)$ \begin{CJK}{UTF8}{mj}存在\end{CJK}, \begin{CJK}{UTF8}{mj}证\end{CJK} \begin{CJK}{UTF8}{mj}明\end{CJK} $f(x)$ \begin{CJK}{UTF8}{mj}在\end{CJK} $(a, b)$ \begin{CJK}{UTF8}{mj}上一致连续\end{CJK}.

  \item ( 15 \begin{CJK}{UTF8}{mj}分\end{CJK}) $f(x, y)$ \begin{CJK}{UTF8}{mj}是定义在\end{CJK} $[a, b] \times[c,+\infty)$ \begin{CJK}{UTF8}{mj}上的连续函数\end{CJK}, $I(x)=\int_{c}^{+\infty} f(x, y) \mathrm{d} y$ \begin{CJK}{UTF8}{mj}是\end{CJK} $[a, b]$ \begin{CJK}{UTF8}{mj}上的连续函数\end{CJK}, \begin{CJK}{UTF8}{mj}证明\end{CJK} $I(x)$ \begin{CJK}{UTF8}{mj}在\end{CJK} $[a, b]$ \begin{CJK}{UTF8}{mj}上一致收敛\end{CJK}.

  \item ( 15 \begin{CJK}{UTF8}{mj}分\end{CJK}) \begin{CJK}{UTF8}{mj}求函数的傅里叶展开式\end{CJK}, \begin{CJK}{UTF8}{mj}周期是\end{CJK} $2 \pi$ \begin{CJK}{UTF8}{mj}在\end{CJK} $(-\pi, \pi)$ \begin{CJK}{UTF8}{mj}上\end{CJK}, $f(x)=x^{3}$; \begin{CJK}{UTF8}{mj}并求级数\end{CJK} $\sum_{n=1}^{\infty} \frac{1}{n^{6}}$ \begin{CJK}{UTF8}{mj}的和\end{CJK}.

  \item (15 \begin{CJK}{UTF8}{mj}分\end{CJK}) $f(x)$ \begin{CJK}{UTF8}{mj}在\end{CJK} $[a, b]$ \begin{CJK}{UTF8}{mj}上一阶连续可导\end{CJK}, $A=\frac{1}{b-a} \int_{a}^{b} f(x) \mathrm{d} x$, \begin{CJK}{UTF8}{mj}证明\end{CJK}:

\end{enumerate}
$$
\int_{a}^{b}(f(x)-A)^{2} \mathrm{~d} x \leqslant(b-a)^{2} \int_{a}^{b}|f(x)|^{2} \mathrm{~d} x
$$

\begin{enumerate}
  \setcounter{enumi}{8}
  \item ( 15 \begin{CJK}{UTF8}{mj}分\end{CJK}) $\varphi(x)$ \begin{CJK}{UTF8}{mj}是连续函数\end{CJK}, \begin{CJK}{UTF8}{mj}函数列\end{CJK} $f_{n}(t)$ \begin{CJK}{UTF8}{mj}是\end{CJK} $[a, b]$ \begin{CJK}{UTF8}{mj}上的连续函数\end{CJK}, $K(x, t)$ \begin{CJK}{UTF8}{mj}是连续函数\end{CJK}, \begin{CJK}{UTF8}{mj}定义\end{CJK}
\end{enumerate}
$$
f_{0}(t)=\varphi(x), f_{n}(x)=\varphi(x)+\int_{a}^{b} K(x, t) f_{n-1}(t) \mathrm{d} t .
$$
\begin{CJK}{UTF8}{mj}证明函数列\end{CJK} $f_{n}(t)$ \begin{CJK}{UTF8}{mj}收敛于一个连续函数\end{CJK}.

\section{6. 浙江大学 2017 年研究生入学考试试题数学分析 
 李扬 
 微信公众号: sxkyliyang}
\begin{enumerate}
  \item (\begin{CJK}{UTF8}{mj}每小题\end{CJK} 10 \begin{CJK}{UTF8}{mj}分\end{CJK}, \begin{CJK}{UTF8}{mj}共\end{CJK} 40 \begin{CJK}{UTF8}{mj}分\end{CJK}) \begin{CJK}{UTF8}{mj}计算下面各题\end{CJK}
\end{enumerate}
(1) $\lim _{x \rightarrow 0} \frac{1-(\cos x)^{\sin x}}{x^{3}}$.

(2) $\int \sqrt{1+\sin x} \mathrm{~d} x$.

(3) $\iint_{x^{2}+4 y^{2} \leqslant 1}\left(x^{2}+y^{2}\right) \mathrm{d} x \mathrm{~d} y$.

(4) \begin{CJK}{UTF8}{mj}将\end{CJK} $f(x)=\frac{\pi}{2}-x$ \begin{CJK}{UTF8}{mj}在\end{CJK} $(0, \pi)$ \begin{CJK}{UTF8}{mj}上展开成余弦级数\end{CJK}.

\begin{enumerate}
  \setcounter{enumi}{2}
  \item (10 \begin{CJK}{UTF8}{mj}分\end{CJK}) \begin{CJK}{UTF8}{mj}利用\end{CJK} $\varepsilon-N$ \begin{CJK}{UTF8}{mj}语言证明\end{CJK}
\end{enumerate}
$$
\lim _{n \rightarrow \infty}\left((-1)^{n}+\frac{1}{n}\right)
$$
\begin{CJK}{UTF8}{mj}不存在\end{CJK}.

\begin{enumerate}
  \setcounter{enumi}{3}
  \item (10 \begin{CJK}{UTF8}{mj}分\end{CJK}) \begin{CJK}{UTF8}{mj}求\end{CJK} $f(x, y)=x^{2}+y^{2}-x y$ \begin{CJK}{UTF8}{mj}在区域\end{CJK} $|x|+|y| \leqslant 1$ \begin{CJK}{UTF8}{mj}上的最大值与最小值\end{CJK}.

  \item (15 \begin{CJK}{UTF8}{mj}分\end{CJK}) (1) \begin{CJK}{UTF8}{mj}叙述有限覆盖定理\end{CJK};

\end{enumerate}
(2) \begin{CJK}{UTF8}{mj}利用有限覆盖定理证明上确界存在性定理\end{CJK}.

\begin{enumerate}
  \setcounter{enumi}{5}
  \item ( 15 \begin{CJK}{UTF8}{mj}分\end{CJK}) \begin{CJK}{UTF8}{mj}设\end{CJK} $f(x)$ \begin{CJK}{UTF8}{mj}是\end{CJK} $(1,+\infty)$ \begin{CJK}{UTF8}{mj}上的单调函数\end{CJK}, $\int_{1}^{\infty} f(x) \mathrm{d} x$ \begin{CJK}{UTF8}{mj}收敛\end{CJK}, \begin{CJK}{UTF8}{mj}证明\end{CJK}:
\end{enumerate}
$$
\lim _{x \rightarrow+\infty} f(x)=0
$$
\begin{CJK}{UTF8}{mj}并且\end{CJK}
$$
f(x)=o\left(\frac{1}{x}\right)(x \rightarrow+\infty) .
$$

\begin{enumerate}
  \setcounter{enumi}{6}
  \item ( 15 \begin{CJK}{UTF8}{mj}分\end{CJK}) \begin{CJK}{UTF8}{mj}设\end{CJK} $f(x)$ \begin{CJK}{UTF8}{mj}对任意的自然数\end{CJK} $n$ \begin{CJK}{UTF8}{mj}和任意的\end{CJK} $x \in(-\infty,+\infty)$ \begin{CJK}{UTF8}{mj}都有\end{CJK}
\end{enumerate}
$$
\left.\left.\left|f(x)-\sum_{k=0}^{n} \frac{(-1)^{k}}{(2 k) !}\right| x\right|^{k}\left|\leqslant \frac{1}{(2 n+2) !}\right| x\right|^{n+1}
$$
\begin{CJK}{UTF8}{mj}求\end{CJK} $f(x)$ \begin{CJK}{UTF8}{mj}的解析表达式并证明\end{CJK} $f(x)$ \begin{CJK}{UTF8}{mj}在\end{CJK} $\mathbb{R}$ \begin{CJK}{UTF8}{mj}上一致连续\end{CJK}.

\begin{enumerate}
  \setcounter{enumi}{7}
  \item (15 \begin{CJK}{UTF8}{mj}分\end{CJK}) \begin{CJK}{UTF8}{mj}求含参量积分\end{CJK}
\end{enumerate}
$$
\int_{0}^{1} \frac{1}{x^{\alpha}} \sin \left(\frac{1}{x}\right) d x
$$
\begin{CJK}{UTF8}{mj}的一致收敛区间\end{CJK}.

\begin{enumerate}
  \setcounter{enumi}{8}
  \item ( 15 \begin{CJK}{UTF8}{mj}分\end{CJK}) \begin{CJK}{UTF8}{mj}设\end{CJK} $f(x)$ \begin{CJK}{UTF8}{mj}在\end{CJK} $\mathbb{R}$ \begin{CJK}{UTF8}{mj}上连续可微\end{CJK}, $f(0)=0$. \begin{CJK}{UTF8}{mj}并且\end{CJK} $\left|f^{\prime}(x)\right| \leqslant|f(x)|$ \begin{CJK}{UTF8}{mj}对任意的\end{CJK} $x \in \mathbb{R}$ \begin{CJK}{UTF8}{mj}成立\end{CJK}. \begin{CJK}{UTF8}{mj}证明\end{CJK}:
\end{enumerate}
$$
f(x) \equiv 0 .
$$

\begin{enumerate}
  \setcounter{enumi}{9}
  \item (15 \begin{CJK}{UTF8}{mj}分\end{CJK}) \begin{CJK}{UTF8}{mj}设有界数列\end{CJK} $\left\{x_{n}\right\}$, \begin{CJK}{UTF8}{mj}满足\end{CJK}
\end{enumerate}
$$
\underline{\lim }_{n \rightarrow \infty} x_{n}=A<B=\varlimsup_{n \rightarrow \infty} x_{n}, \lim _{n \rightarrow \infty}\left(x_{n+1}-x_{n}\right)=0 .
$$
\begin{CJK}{UTF8}{mj}证明\end{CJK} $\left\{x_{n}\right\}$ \begin{CJK}{UTF8}{mj}的聚点全体恰好构成\end{CJK} $[A, B]$.

\section{7. 浙江大学 2018 年研究生入学考试试题数学分析 
 李扬 
 微信公众号: sxkyliyang}
\begin{enumerate}
  \item (40 \begin{CJK}{UTF8}{mj}分\end{CJK}) \begin{CJK}{UTF8}{mj}计算下面各题\end{CJK}
\end{enumerate}
(1) \begin{CJK}{UTF8}{mj}求极限\end{CJK}
$$
\lim _{n \rightarrow \infty} \sum_{k=1}^{n-1}\left(1+\frac{k}{n}\right) \sin \frac{k \pi}{n^{2}}
$$
(2) \begin{CJK}{UTF8}{mj}求\end{CJK}
$$
\lim _{x \rightarrow 0} \frac{\ln \left(1+x+x^{2}\right)+\arcsin 3 x-5 x^{3}}{\sin 2 x+\tan ^{2} x-\left(e^{x}-1\right)^{5}}
$$
(3) \begin{CJK}{UTF8}{mj}求\end{CJK}
$$
\iint_{\Sigma} \frac{R x \mathrm{~d} y \mathrm{~d} z+(z+R)^{2} \mathrm{~d} x \mathrm{~d} y}{\sqrt{x^{2}+y^{2}+z^{2}}} .
$$
\begin{CJK}{UTF8}{mj}其中\end{CJK} $\Sigma$ \begin{CJK}{UTF8}{mj}为\end{CJK} $x^{2}+y^{2}+z^{2}=R^{2}$ \begin{CJK}{UTF8}{mj}的下半球面的上侧\end{CJK}, $R$ \begin{CJK}{UTF8}{mj}为一常数\end{CJK}.

\begin{enumerate}
  \setcounter{enumi}{2}
  \item (10 \begin{CJK}{UTF8}{mj}分\end{CJK}) (1) \begin{CJK}{UTF8}{mj}用极限定义叙述\end{CJK} $\lim _{x \rightarrow+\infty} f(x) \neq+\infty$.
\end{enumerate}
(2) $\lim _{x \rightarrow+\infty} \frac{x \sin x}{\sqrt{x}+1} \neq+\infty$.

\begin{enumerate}
  \setcounter{enumi}{3}
  \item (10 \begin{CJK}{UTF8}{mj}分\end{CJK}) \begin{CJK}{UTF8}{mj}证明有界闭集上的有限覆盖定理\end{CJK}.

  \item (15 \begin{CJK}{UTF8}{mj}分\end{CJK}) \begin{CJK}{UTF8}{mj}设函数列\end{CJK} $\left\{f_{n}(x)\right\}$ \begin{CJK}{UTF8}{mj}在\end{CJK} $(a, b)$ \begin{CJK}{UTF8}{mj}上一致连续\end{CJK}, \begin{CJK}{UTF8}{mj}并且\end{CJK} $f_{n}(x)$ \begin{CJK}{UTF8}{mj}一致收敛于\end{CJK} $f(x)$. \begin{CJK}{UTF8}{mj}证明\end{CJK} $f(x)$ \begin{CJK}{UTF8}{mj}在\end{CJK} $(a, b)$ \begin{CJK}{UTF8}{mj}上一致连续\end{CJK}.

  \item (15 \begin{CJK}{UTF8}{mj}分\end{CJK}) \begin{CJK}{UTF8}{mj}构造或者证明是否存在函数\end{CJK} $f(x)$ :

\end{enumerate}
(1) $f(x)$ \begin{CJK}{UTF8}{mj}在\end{CJK} $[0,1]$ \begin{CJK}{UTF8}{mj}上连续\end{CJK}, \begin{CJK}{UTF8}{mj}在\end{CJK} $(0,1)$ \begin{CJK}{UTF8}{mj}内可导\end{CJK}, $f(x)$ \begin{CJK}{UTF8}{mj}在\end{CJK} $(0,1)$ \begin{CJK}{UTF8}{mj}内有无数个零点\end{CJK}, \begin{CJK}{UTF8}{mj}且对任意的\end{CJK} $x \in(0,1)$ \begin{CJK}{UTF8}{mj}不存在\end{CJK} $x$ \begin{CJK}{UTF8}{mj}使\end{CJK} \begin{CJK}{UTF8}{mj}得\end{CJK} $f(x)=f^{\prime}(x)=0$.

(2) \begin{CJK}{UTF8}{mj}假如\end{CJK} $f(0)$ \begin{CJK}{UTF8}{mj}和\end{CJK} $f(1)$ \begin{CJK}{UTF8}{mj}都不等于\end{CJK} 0 , \begin{CJK}{UTF8}{mj}问上述的\end{CJK} $f(x)$ \begin{CJK}{UTF8}{mj}是否成立\end{CJK}?

\begin{enumerate}
  \setcounter{enumi}{6}
  \item (15 \begin{CJK}{UTF8}{mj}分\end{CJK}) \begin{CJK}{UTF8}{mj}设\end{CJK} $f(y)$ \begin{CJK}{UTF8}{mj}在\end{CJK} $[0,1]$ \begin{CJK}{UTF8}{mj}上连续\end{CJK}, \begin{CJK}{UTF8}{mj}且\end{CJK}
\end{enumerate}
$$
K(x, y)= \begin{cases}y(1-x), & y<x \\ x(1-y), & y \geqslant x\end{cases}
$$
\begin{CJK}{UTF8}{mj}令\end{CJK} $u(x)=\int_{0}^{1} K(x, y) f(y) \mathrm{d} y$, \begin{CJK}{UTF8}{mj}问\end{CJK} $u(x)$ \begin{CJK}{UTF8}{mj}在\end{CJK} $[0,1]$ \begin{CJK}{UTF8}{mj}上是否连续并且求\end{CJK} $\frac{\mathrm{d}^{2} u}{\mathrm{~d} x^{2}}$.

\begin{enumerate}
  \setcounter{enumi}{7}
  \item (15 \begin{CJK}{UTF8}{mj}分\end{CJK}) \begin{CJK}{UTF8}{mj}设级数\end{CJK} $\sum_{n=1}^{\infty} n a_{n}$ \begin{CJK}{UTF8}{mj}收敛\end{CJK}, \begin{CJK}{UTF8}{mj}定义\end{CJK} $x_{n}=a_{n+1}+2 a_{n+2}+\cdots+k a_{n+k}+\cdots,(n=1,2, \cdots)$.
\end{enumerate}
(1) \begin{CJK}{UTF8}{mj}问\end{CJK} $x_{n}$ \begin{CJK}{UTF8}{mj}是否有意义\end{CJK}?

(2) \begin{CJK}{UTF8}{mj}求证\end{CJK} $\lim _{n \rightarrow \infty} x_{n}=0$.

\begin{enumerate}
  \setcounter{enumi}{8}
  \item (15 \begin{CJK}{UTF8}{mj}分\end{CJK}) \begin{CJK}{UTF8}{mj}设函数集合\end{CJK}
\end{enumerate}
$$
S=\left\{f(x)\left|\sup _{x \in \mathbb{R}}\right| x^{m} \frac{\mathrm{d}^{k} f}{\mathrm{~d} x^{k}} \mid<+\infty, m, n \in \mathbb{N}\right\}
$$
\begin{CJK}{UTF8}{mj}若\end{CJK} $f(x) \in S$, \begin{CJK}{UTF8}{mj}求证\end{CJK}
$$
\hat{f}(x) \in S
$$
\begin{CJK}{UTF8}{mj}其中\end{CJK} $\hat{f}(x)=\int_{-\infty}^{+\infty} f(t) e^{-i x t} \mathrm{~d} t, f(x)=\int_{-\infty}^{+\infty} \hat{f}(t) e^{i x t} \mathrm{~d} t$.

\begin{enumerate}
  \setcounter{enumi}{9}
  \item (15 \begin{CJK}{UTF8}{mj}分\end{CJK}) \begin{CJK}{UTF8}{mj}设\end{CJK} $f(x)$ \begin{CJK}{UTF8}{mj}在\end{CJK} $(0,+\infty)$ \begin{CJK}{UTF8}{mj}上连续可微\end{CJK}, \begin{CJK}{UTF8}{mj}且存在\end{CJK} $L>0$, \begin{CJK}{UTF8}{mj}使得对\end{CJK} $\forall x, y \in(0,+\infty)$ \begin{CJK}{UTF8}{mj}都有\end{CJK}
\end{enumerate}
$$
\left|f^{\prime}(x)-f^{\prime}(y)\right|<L|x-y| .
$$
\begin{CJK}{UTF8}{mj}证明\end{CJK}: $\left(f^{\prime}(x)\right)^{2}<2 L f(x)$.

\section{1. 中国海洋大学 2009 年研究生入学考试试题高等代数 
 李扬 
 微信公众号: sxkyliyang}
\begin{CJK}{UTF8}{mj}一\end{CJK}. \begin{CJK}{UTF8}{mj}设\end{CJK} $f(x)$ \begin{CJK}{UTF8}{mj}和\end{CJK} $g(x)$ \begin{CJK}{UTF8}{mj}为非零多项式\end{CJK}, $f(x) g(x)+f(x)+g(x)=p(x)$ \begin{CJK}{UTF8}{mj}是一个不可约的多项式\end{CJK}. \begin{CJK}{UTF8}{mj}证明\end{CJK}:
$$
(f(x), g(x))=1 .
$$
\begin{CJK}{UTF8}{mj}二\end{CJK}. \begin{CJK}{UTF8}{mj}已知齐次线性方程组\end{CJK}
$$
\left\{\begin{array}{c}
\left(a_{1}+b\right) x_{1}+a_{2} x_{2}+\cdots+a_{n} x_{n}=0 \\
a_{1} x_{1}+\left(a_{2}+b\right) x_{2}+\cdots+a_{n} x_{n}=0 \\
\cdots \cdots \cdots \cdots \cdots \cdots \\
a_{1} x_{1}+a_{2} x_{2}+\cdots+\left(a_{n}+b\right) x_{n}=0
\end{array}\right.
$$
\begin{CJK}{UTF8}{mj}其中\end{CJK} $\sum_{i=1}^{n} a_{i} \neq 0$, \begin{CJK}{UTF8}{mj}试讨论\end{CJK} $a_{1}, a_{2}, \cdots, a_{n}, b$ \begin{CJK}{UTF8}{mj}满足什么条件时\end{CJK}

(1) \begin{CJK}{UTF8}{mj}方程组仅有零解\end{CJK}?

(2) \begin{CJK}{UTF8}{mj}方程组有非零解\end{CJK}? \begin{CJK}{UTF8}{mj}此时\end{CJK}, \begin{CJK}{UTF8}{mj}用基础解系表示所有解\end{CJK}.

\begin{CJK}{UTF8}{mj}三\end{CJK}. \begin{CJK}{UTF8}{mj}证明\end{CJK}: \begin{CJK}{UTF8}{mj}反对称实数矩阵的特征值是零或纯虚数\end{CJK}.

\begin{CJK}{UTF8}{mj}四\end{CJK}. \begin{CJK}{UTF8}{mj}设\end{CJK} $A, B, C, D$ \begin{CJK}{UTF8}{mj}均为\end{CJK} $n$ \begin{CJK}{UTF8}{mj}阶方阵\end{CJK}, \begin{CJK}{UTF8}{mj}且\end{CJK} $|A| \neq 0, A C=C A$. \begin{CJK}{UTF8}{mj}证明\end{CJK}:
$$
\left|\begin{array}{ll}
A & B \\
C & D
\end{array}\right|=|A D-C B| .
$$
\begin{CJK}{UTF8}{mj}五\end{CJK}. \begin{CJK}{UTF8}{mj}设\end{CJK} $A$ \begin{CJK}{UTF8}{mj}为\end{CJK} $m$ \begin{CJK}{UTF8}{mj}级实对称矩阵且正定\end{CJK}, $B$ \begin{CJK}{UTF8}{mj}为\end{CJK} $m \times n$ \begin{CJK}{UTF8}{mj}实矩阵\end{CJK}. \begin{CJK}{UTF8}{mj}试证\end{CJK}: $B^{\prime} A B$ \begin{CJK}{UTF8}{mj}为正定矩阵的充分必要条件是\end{CJK}
$$
\operatorname{rank} B=n
$$
\begin{CJK}{UTF8}{mj}六\end{CJK}. \begin{CJK}{UTF8}{mj}设\end{CJK} $W$ \begin{CJK}{UTF8}{mj}表示数域\end{CJK} $P$ \begin{CJK}{UTF8}{mj}上次数不超过\end{CJK} $n-1$ \begin{CJK}{UTF8}{mj}的一元多项式的集合\end{CJK}, $D$ \begin{CJK}{UTF8}{mj}为求导运算\end{CJK}.

(1)\begin{CJK}{UTF8}{mj}证明\end{CJK}: $W$ \begin{CJK}{UTF8}{mj}是一元多项式线性空间\end{CJK} $P[x]$ \begin{CJK}{UTF8}{mj}的子空间\end{CJK}.

(2) \begin{CJK}{UTF8}{mj}给出其一组基\end{CJK}, \begin{CJK}{UTF8}{mj}并证之\end{CJK}.

(3) \begin{CJK}{UTF8}{mj}证明\end{CJK}: $D$ \begin{CJK}{UTF8}{mj}是\end{CJK} $W$ \begin{CJK}{UTF8}{mj}上的线性变换\end{CJK}.

(4) \begin{CJK}{UTF8}{mj}分别给出\end{CJK} $W$ \begin{CJK}{UTF8}{mj}的可对角化与不可对角化的线性变换\end{CJK}, \begin{CJK}{UTF8}{mj}并阐述理由\end{CJK}.

\begin{CJK}{UTF8}{mj}七\end{CJK}. \begin{CJK}{UTF8}{mj}设\end{CJK} $A$ \begin{CJK}{UTF8}{mj}为\end{CJK} $n$ \begin{CJK}{UTF8}{mj}级实矩阵\end{CJK}, \begin{CJK}{UTF8}{mj}齐次线性方程组\end{CJK} $A x=0$ \begin{CJK}{UTF8}{mj}的解空间为\end{CJK} $W_{1},(A-E) x=0$ \begin{CJK}{UTF8}{mj}的解空间为\end{CJK} $W_{2}$. \begin{CJK}{UTF8}{mj}证明\end{CJK}: $A$ \begin{CJK}{UTF8}{mj}为幂等\end{CJK} \begin{CJK}{UTF8}{mj}阵\end{CJK} $\left(A^{2}=A\right)$ \begin{CJK}{UTF8}{mj}的充要条件是\end{CJK}
$$
\mathbb{R}^{n}=W_{1} \oplus W_{2}
$$
\begin{CJK}{UTF8}{mj}八\end{CJK}. \begin{CJK}{UTF8}{mj}设\end{CJK} $n$ \begin{CJK}{UTF8}{mj}级方阵\end{CJK} $A$ \begin{CJK}{UTF8}{mj}的所有元素都是\end{CJK} 1 , \begin{CJK}{UTF8}{mj}求\end{CJK} $A$ \begin{CJK}{UTF8}{mj}的最小多项式\end{CJK}.

\begin{CJK}{UTF8}{mj}九\end{CJK}. \begin{CJK}{UTF8}{mj}证明\end{CJK}: \begin{CJK}{UTF8}{mj}欧式空间中两组基的度量矩阵是合同的\end{CJK}.

\section{2. 中国海洋大学 2010 年研究生入学考试试题高等代数 
 李扬 
 微信公众号: sxkyliyang}
-. (15 \begin{CJK}{UTF8}{mj}分\end{CJK}) \begin{CJK}{UTF8}{mj}设\end{CJK} $f(x)$ \begin{CJK}{UTF8}{mj}是一个次数\end{CJK} $\geqslant 2$ \begin{CJK}{UTF8}{mj}的整系数多项式\end{CJK}, \begin{CJK}{UTF8}{mj}它不能分解为两个较低次数的整系数多项式的乘积\end{CJK}, \begin{CJK}{UTF8}{mj}求证\end{CJK}: $f(x)=0$ \begin{CJK}{UTF8}{mj}即使在复数域也不可能有重根\end{CJK}.

\begin{CJK}{UTF8}{mj}二\end{CJK}. ( 16 \begin{CJK}{UTF8}{mj}分\end{CJK}) \begin{CJK}{UTF8}{mj}设\end{CJK} $f_{i}(x)(i=1,2, \cdots, n)$ \begin{CJK}{UTF8}{mj}是次数不超过\end{CJK} $n-2$ \begin{CJK}{UTF8}{mj}的多项式\end{CJK}. \begin{CJK}{UTF8}{mj}证明\end{CJK}: \begin{CJK}{UTF8}{mj}对任意\end{CJK} $n$ \begin{CJK}{UTF8}{mj}个数\end{CJK} $a_{1}, a_{2}, \cdots, a_{n}$, \begin{CJK}{UTF8}{mj}线性\end{CJK} \begin{CJK}{UTF8}{mj}方程组\end{CJK}
$$
\left\{\begin{array}{c}
f_{1}\left(a_{1}\right) x_{1}+f_{1}\left(a_{2}\right) x_{2}+\cdots+f_{1}\left(a_{n}\right) x_{n}=0 \\
f_{2}\left(a_{1}\right) x_{1}+f_{2}\left(a_{2}\right) x_{2}+\cdots+f_{2}\left(a_{n}\right) x_{n}=0 \\
\cdots \cdots \cdots \cdots \cdots \\
f_{n}\left(a_{1}\right) x_{1}+f_{n}\left(a_{2}\right) x_{2}+\cdots+f_{n}\left(a_{n}\right) x_{n}=0
\end{array}\right.
$$
\begin{CJK}{UTF8}{mj}有非零解\end{CJK}.

\begin{CJK}{UTF8}{mj}三\end{CJK}. $(10$ \begin{CJK}{UTF8}{mj}分\end{CJK} $)$ \begin{CJK}{UTF8}{mj}若\end{CJK} $D=\left(d_{i j}\right)_{n \times n}$, \begin{CJK}{UTF8}{mj}定义\end{CJK}
$$
\operatorname{tr} D=\sum_{i=1}^{n} d_{i i}, A=\left(\begin{array}{ccc}
3 & -3 & 1 \\
1 & 0 & 0 \\
0 & 1 & 0
\end{array}\right)
$$
(1) \begin{CJK}{UTF8}{mj}求\end{CJK} $\operatorname{tr} A^{k}, k=1,2, \cdots$.

(2) \begin{CJK}{UTF8}{mj}证明\end{CJK}: $A$ \begin{CJK}{UTF8}{mj}不相似于任何对角矩阵\end{CJK}.

\begin{CJK}{UTF8}{mj}四\end{CJK}. (10 \begin{CJK}{UTF8}{mj}分\end{CJK}) \begin{CJK}{UTF8}{mj}正定矩阵的伴随矩阵也是正定矩阵\end{CJK}.

\begin{CJK}{UTF8}{mj}五\end{CJK}. ( 15 \begin{CJK}{UTF8}{mj}分\end{CJK}) \begin{CJK}{UTF8}{mj}试求满足\end{CJK} $\left(A^{*}\right)^{*}=A$ \begin{CJK}{UTF8}{mj}的一切\end{CJK} $n(n \geqslant 2)$ \begin{CJK}{UTF8}{mj}阶方阵\end{CJK} $A$. (\begin{CJK}{UTF8}{mj}其中\end{CJK} $A^{*}$ \begin{CJK}{UTF8}{mj}是\end{CJK} $A$ \begin{CJK}{UTF8}{mj}的伴随矩阵\end{CJK}).

\begin{CJK}{UTF8}{mj}六\end{CJK}. (16 \begin{CJK}{UTF8}{mj}分\end{CJK}) \begin{CJK}{UTF8}{mj}设\end{CJK} $A, C$ \begin{CJK}{UTF8}{mj}是\end{CJK} $n$ \begin{CJK}{UTF8}{mj}级正定矩阵\end{CJK}, \begin{CJK}{UTF8}{mj}已知\end{CJK} $B$ \begin{CJK}{UTF8}{mj}是矩阵方程\end{CJK}: $A X+X A^{\prime}=C$ \begin{CJK}{UTF8}{mj}的唯一解\end{CJK}, \begin{CJK}{UTF8}{mj}证明\end{CJK}:

(1) $B$ \begin{CJK}{UTF8}{mj}是实对称矩阵\end{CJK}.

(2) $B$ \begin{CJK}{UTF8}{mj}是正定的\end{CJK}.

\begin{CJK}{UTF8}{mj}七\end{CJK}. (16 \begin{CJK}{UTF8}{mj}分\end{CJK}) \begin{CJK}{UTF8}{mj}设\end{CJK} $A, B$ \begin{CJK}{UTF8}{mj}是两个\end{CJK} $n$ \begin{CJK}{UTF8}{mj}级实对称\end{CJK}, \begin{CJK}{UTF8}{mj}且\end{CJK} $B$ \begin{CJK}{UTF8}{mj}是正定矩阵\end{CJK}, \begin{CJK}{UTF8}{mj}证明\end{CJK}: \begin{CJK}{UTF8}{mj}存在\end{CJK} $n$ \begin{CJK}{UTF8}{mj}级实可逆矩阵\end{CJK} $T$, \begin{CJK}{UTF8}{mj}使\end{CJK} $T^{\prime} A T$ \begin{CJK}{UTF8}{mj}和\end{CJK} $T^{\prime} B T$ \begin{CJK}{UTF8}{mj}同\end{CJK} \begin{CJK}{UTF8}{mj}时为对角矩阵\end{CJK}.

\begin{CJK}{UTF8}{mj}八\end{CJK}. ( 16 \begin{CJK}{UTF8}{mj}分\end{CJK}) \begin{CJK}{UTF8}{mj}设\end{CJK} $\alpha_{1}, \alpha_{2}, \cdots, \alpha_{s}$ \begin{CJK}{UTF8}{mj}与\end{CJK} $\beta_{1}, \beta_{2}, \cdots, \beta_{t}$ \begin{CJK}{UTF8}{mj}是两组\end{CJK} $n$ \begin{CJK}{UTF8}{mj}维向量\end{CJK}, \begin{CJK}{UTF8}{mj}证明\end{CJK}: \begin{CJK}{UTF8}{mj}若这两个向量组都线性无关\end{CJK}, \begin{CJK}{UTF8}{mj}则空间\end{CJK} $L\left(\alpha_{1}, \alpha_{2}, \cdots, \alpha_{s}\right) \bigcap L\left(\beta_{1}, \beta_{2}, \cdots, \beta_{t}\right)$ \begin{CJK}{UTF8}{mj}的维数等于齐次线性方程组\end{CJK}
$$
\alpha_{1} x_{1}+\alpha_{2} x_{2}+\cdots+\alpha_{s} x_{s}+\beta_{1} y_{1}+\beta_{2} y_{2}+\cdots+\beta_{t} y_{t}=0
$$
\begin{CJK}{UTF8}{mj}的解空间的维数\end{CJK}.

\begin{CJK}{UTF8}{mj}九\end{CJK}. (16 \begin{CJK}{UTF8}{mj}分\end{CJK}) \begin{CJK}{UTF8}{mj}设\end{CJK} $A$ \begin{CJK}{UTF8}{mj}是\end{CJK} $n$ \begin{CJK}{UTF8}{mj}级实非奇异矩阵\end{CJK}. \begin{CJK}{UTF8}{mj}证明\end{CJK}: $A$ \begin{CJK}{UTF8}{mj}可以分解成\end{CJK}
$$
A=Q R
$$
\begin{CJK}{UTF8}{mj}其中\end{CJK} $Q$ \begin{CJK}{UTF8}{mj}为正交矩阵\end{CJK}, $R$ \begin{CJK}{UTF8}{mj}是一个对角线上全为正实数的上三角矩阵\end{CJK}, \begin{CJK}{UTF8}{mj}并且这种分解是唯一的\end{CJK}.

\begin{CJK}{UTF8}{mj}十\end{CJK}. (20 \begin{CJK}{UTF8}{mj}分\end{CJK}) \begin{CJK}{UTF8}{mj}在\end{CJK} $\mathbb{C}[x]_{n}$ \begin{CJK}{UTF8}{mj}中\end{CJK}, \begin{CJK}{UTF8}{mj}微分变换\end{CJK} $D: D(f(x))=f^{\prime}(x)$ \begin{CJK}{UTF8}{mj}是\end{CJK} $\mathbb{C}[x]_{n}$ \begin{CJK}{UTF8}{mj}上的线性变换\end{CJK}.

(1) \begin{CJK}{UTF8}{mj}求\end{CJK} $D$ \begin{CJK}{UTF8}{mj}的核及值域\end{CJK}.

(2) \begin{CJK}{UTF8}{mj}求\end{CJK} $D$ \begin{CJK}{UTF8}{mj}在基\end{CJK} $1,1+x, 1+x+x^{2}, \cdots, 1+x+\cdots+x^{n-1}$ \begin{CJK}{UTF8}{mj}下的矩阵\end{CJK} $A$.

(3) \begin{CJK}{UTF8}{mj}求\end{CJK} $A$ \begin{CJK}{UTF8}{mj}的若尔当标准形\end{CJK}.

\section{3. 中国海洋大学 2011 年研究生入学考试试题高等代数 
 李扬 
 微信公众号: sxkyliyang}
\begin{CJK}{UTF8}{mj}一\end{CJK}. \begin{CJK}{UTF8}{mj}填空题\end{CJK} ( 40 \begin{CJK}{UTF8}{mj}分\end{CJK})

\begin{enumerate}
  \item \begin{CJK}{UTF8}{mj}当\end{CJK} $k=$ \begin{CJK}{UTF8}{mj}时\end{CJK}, \begin{CJK}{UTF8}{mj}多项式\end{CJK} $f(x)=x^{3}+3 x^{2}+k x+1$ \begin{CJK}{UTF8}{mj}有重根\end{CJK}.

  \item $n$ \begin{CJK}{UTF8}{mj}级行列式\end{CJK} $\left|\begin{array}{cccccc}5 & 3 & 0 & \cdots & 0 & 0 \\ 2 & 5 & 3 & \cdots & 0 & 0 \\ 0 & 2 & 5 & \cdots & 0 & 0 \\ \vdots & \vdots & \vdots & & \vdots & \vdots \\ 0 & 0 & 0 & \cdots & 5 & 3 \\ 0 & 0 & 0 & \cdots & 2 & 5\end{array}\right|=$

  \item \begin{CJK}{UTF8}{mj}设\end{CJK} $A$ \begin{CJK}{UTF8}{mj}是\end{CJK} $n(n \geqslant 2)$ \begin{CJK}{UTF8}{mj}级方阵\end{CJK}, \begin{CJK}{UTF8}{mj}如果秩\end{CJK} $(A)=n-1$, \begin{CJK}{UTF8}{mj}且代数余子式\end{CJK} $A_{12} \neq 0$, \begin{CJK}{UTF8}{mj}则齐次线性方程组\end{CJK} $A x=0$ \begin{CJK}{UTF8}{mj}的通解是\end{CJK}

  \item \begin{CJK}{UTF8}{mj}设\end{CJK} $A$ \begin{CJK}{UTF8}{mj}是\end{CJK} $n(n \geqslant 2)$ \begin{CJK}{UTF8}{mj}级方阵\end{CJK}, \begin{CJK}{UTF8}{mj}满足\end{CJK} $A^{2}+3 A+E=0$, \begin{CJK}{UTF8}{mj}则\end{CJK} $(A-2 E)^{-1}=$

  \item \begin{CJK}{UTF8}{mj}设\end{CJK} $A$ \begin{CJK}{UTF8}{mj}是\end{CJK} 3 \begin{CJK}{UTF8}{mj}级实对称矩阵\end{CJK}, \begin{CJK}{UTF8}{mj}且\end{CJK} $|A|=2$, \begin{CJK}{UTF8}{mj}已知\end{CJK} $A$ \begin{CJK}{UTF8}{mj}的特征值是\end{CJK} $\frac{1}{2}+\frac{\sqrt{3}}{2} i$, \begin{CJK}{UTF8}{mj}则\end{CJK} $\left|A^{*}-2 A\right|=$

  \item \begin{CJK}{UTF8}{mj}已知\end{CJK} $A=\left(\begin{array}{lll}0 & 0 & 1 \\ x & 1 & 0 \\ 1 & 0 & 0\end{array}\right)$ \begin{CJK}{UTF8}{mj}有三个线性无关的特征向量\end{CJK}, \begin{CJK}{UTF8}{mj}则\end{CJK} $x=$

  \item \begin{CJK}{UTF8}{mj}设复数矩阵\end{CJK} $A=\left(\begin{array}{ccc}3 & 0 & 8 \\ 3 & -1 & 6 \\ -2 & 0 & -5\end{array}\right), A$ \begin{CJK}{UTF8}{mj}的\end{CJK} Jordan \begin{CJK}{UTF8}{mj}标准形为\end{CJK}

  \item \begin{CJK}{UTF8}{mj}设\end{CJK} $V_{1}, V_{2}$ \begin{CJK}{UTF8}{mj}都是线性空间\end{CJK} $V$ \begin{CJK}{UTF8}{mj}的子空间\end{CJK}, \begin{CJK}{UTF8}{mj}则\end{CJK} $V_{1} \bigcup V_{2}$ \begin{CJK}{UTF8}{mj}是子空间的充分必要条件是\end{CJK}

  \item \begin{CJK}{UTF8}{mj}设\end{CJK} $T$ \begin{CJK}{UTF8}{mj}是\end{CJK} $n$ \begin{CJK}{UTF8}{mj}维欧式空间\end{CJK} $V$ \begin{CJK}{UTF8}{mj}的正交变换\end{CJK}, \begin{CJK}{UTF8}{mj}那么\end{CJK} $T$ \begin{CJK}{UTF8}{mj}的核\end{CJK} $T^{-1}(0)$ \begin{CJK}{UTF8}{mj}的维数为\end{CJK}

  \item \begin{CJK}{UTF8}{mj}设\end{CJK} $\mathbb{R}^{2}$ \begin{CJK}{UTF8}{mj}中的内积\end{CJK} $(\alpha, \beta)=\alpha^{\prime} A \beta, A=\left(\begin{array}{ll}2 & 1 \\ 1 & 2\end{array}\right)$, \begin{CJK}{UTF8}{mj}则\end{CJK} $\left(\begin{array}{l}1 \\ 2\end{array}\right),\left(\begin{array}{l}0 \\ 1\end{array}\right)$ \begin{CJK}{UTF8}{mj}在此内积下的度量矩阵为\end{CJK}

\end{enumerate}
\begin{CJK}{UTF8}{mj}二\end{CJK}. (10 \begin{CJK}{UTF8}{mj}分\end{CJK}) \begin{CJK}{UTF8}{mj}对于非负整数\end{CJK} $n$, \begin{CJK}{UTF8}{mj}令多项式\end{CJK} $f_{n}(x)=x^{n+2}-(x+1)^{2 n+1}$, \begin{CJK}{UTF8}{mj}证明\end{CJK}:
$$
\left(x^{2}+x+1, f_{n}(x)\right)=1 .
$$
\begin{CJK}{UTF8}{mj}三\end{CJK}. ( 20 \begin{CJK}{UTF8}{mj}分\end{CJK}) \begin{CJK}{UTF8}{mj}设\end{CJK} $\alpha_{1}, \alpha_{2}, \cdots, \alpha_{m}$ \begin{CJK}{UTF8}{mj}是一向量组\end{CJK}, \begin{CJK}{UTF8}{mj}讨论向量组\end{CJK} $\alpha_{1}+\alpha_{2}, \alpha_{2}+\alpha_{3}, \cdots, \alpha_{m}+\alpha_{1}$ \begin{CJK}{UTF8}{mj}的线性相关性\end{CJK}.

\begin{CJK}{UTF8}{mj}四\end{CJK}. (10 \begin{CJK}{UTF8}{mj}分\end{CJK}) \begin{CJK}{UTF8}{mj}设\end{CJK} $A, B$ \begin{CJK}{UTF8}{mj}为\end{CJK} $n \times n$ \begin{CJK}{UTF8}{mj}级实矩阵\end{CJK}, \begin{CJK}{UTF8}{mj}如果\end{CJK} $A$ \begin{CJK}{UTF8}{mj}为可逆矩阵\end{CJK}, $B$ \begin{CJK}{UTF8}{mj}为\end{CJK} $n$ \begin{CJK}{UTF8}{mj}级实反对称矩阵\end{CJK}, \begin{CJK}{UTF8}{mj}证明\end{CJK}:
$$
\left|A^{\prime} A+B\right|>0 .
$$
\begin{CJK}{UTF8}{mj}五\end{CJK}. (20 \begin{CJK}{UTF8}{mj}分\end{CJK}) $n$ \begin{CJK}{UTF8}{mj}元线性方程组\end{CJK}
$$
\left(\begin{array}{ccccc}
a & 1 & 1 & \cdots & 1 \\
1 & a & 1 & \cdots & 1 \\
1 & 1 & a & \cdots & 1 \\
\vdots & \vdots & \vdots & & \vdots \\
1 & 1 & 1 & \cdots & a
\end{array}\right)\left(\begin{array}{c}
x_{1} \\
x_{2} \\
\vdots \\
x_{n}
\end{array}\right)=\left(\begin{array}{c}
b_{1} \\
b_{2} \\
\vdots \\
b_{n}
\end{array}\right)
$$
\begin{CJK}{UTF8}{mj}何时无解\end{CJK}, \begin{CJK}{UTF8}{mj}有唯一解\end{CJK}, \begin{CJK}{UTF8}{mj}有无穷多组解\end{CJK}? \begin{CJK}{UTF8}{mj}有解时\end{CJK}, \begin{CJK}{UTF8}{mj}求出解\end{CJK}. \begin{CJK}{UTF8}{mj}六\end{CJK}. (15 \begin{CJK}{UTF8}{mj}分\end{CJK}) \begin{CJK}{UTF8}{mj}设方阵\end{CJK}
$$
A=\left(\begin{array}{ccc}
3 & 1 & -1 \\
-6 & -2 & 3 \\
-2 & -1 & 1
\end{array}\right) \text {. }
$$
$A$ \begin{CJK}{UTF8}{mj}在实数域\end{CJK} $\mathbb{R}$ \begin{CJK}{UTF8}{mj}上是否相似于对角矩阵\end{CJK} (\begin{CJK}{UTF8}{mj}即有实方阵\end{CJK} $P$ \begin{CJK}{UTF8}{mj}使\end{CJK} $P^{-1} A P$ \begin{CJK}{UTF8}{mj}为对角形\end{CJK})? \begin{CJK}{UTF8}{mj}在复数域\end{CJK} $\mathbb{C}$ \begin{CJK}{UTF8}{mj}上呢\end{CJK}? \begin{CJK}{UTF8}{mj}给出证明\end{CJK}.

\begin{CJK}{UTF8}{mj}七\end{CJK}. (20 \begin{CJK}{UTF8}{mj}分\end{CJK}) \begin{CJK}{UTF8}{mj}设\end{CJK} $V$ \begin{CJK}{UTF8}{mj}是数域\end{CJK} $P$ \begin{CJK}{UTF8}{mj}上的线性空间\end{CJK}, $\mathscr{A}$ \begin{CJK}{UTF8}{mj}是\end{CJK} $V$ \begin{CJK}{UTF8}{mj}上的线性变换\end{CJK}, $f(x), g(x) \in P[x], h(x)=f(x) g(x)$, \begin{CJK}{UTF8}{mj}证明\end{CJK}:

(1) $\operatorname{ker} f(\mathscr{A})+\operatorname{ker} g(\mathscr{A}) \subseteq \operatorname{ker} h(\mathscr{A})$.

(2) \begin{CJK}{UTF8}{mj}若\end{CJK} $(f(x), g(x))=1$, \begin{CJK}{UTF8}{mj}则\end{CJK} $\operatorname{ker} h(\mathscr{A})=\operatorname{ker} f(\mathscr{A}) \oplus \operatorname{ker} g(\mathscr{A})$. (\begin{CJK}{UTF8}{mj}其中\end{CJK} $\operatorname{ker}(\mathscr{A})$ \begin{CJK}{UTF8}{mj}表示线性变换\end{CJK} $\mathscr{A}$ \begin{CJK}{UTF8}{mj}的核\end{CJK})

\begin{CJK}{UTF8}{mj}八\end{CJK}. ( 15 \begin{CJK}{UTF8}{mj}分\end{CJK}) \begin{CJK}{UTF8}{mj}在\end{CJK} $P[x]_{4}$ \begin{CJK}{UTF8}{mj}中定义内积\end{CJK}: $(f, g)=\int_{-1}^{1} f(x) g(x) \mathrm{d} x, f, g \in P[x]_{4}$, \begin{CJK}{UTF8}{mj}并定义一个线性变换\end{CJK} $\mathscr{A}: \mathscr{A} e_{i}=$ $\eta_{i}, i=1,2,3,4$ \begin{CJK}{UTF8}{mj}其中\end{CJK}
$$
\left\{\begin{array}{l}
e_{1}=\frac{1}{2}+\frac{1}{2} x+\frac{1}{2} x^{2}+\frac{1}{2} x^{3} \\
e_{2}=-\frac{1}{2}-\frac{1}{2} x+\frac{1}{2} x^{2}+\frac{1}{2} x^{3} \\
e_{3}=-\frac{1}{2}+\frac{1}{2} x-\frac{1}{2} x^{2}+\frac{1}{2} x^{3} \\
e_{4}=-\frac{1}{2}+\frac{1}{2} x+\frac{1}{2} x^{2}-\frac{1}{2} x^{3}
\end{array},\left\{\begin{array}{l}
\eta_{1}=2 x+x^{2}-x^{3} \\
\eta_{2}=-1-x^{2}+2 x^{3} \\
\eta_{3}=-2 x-x^{2}+x^{3} \\
\eta_{4}=1-4 x-x^{2}
\end{array}\right.\right.
$$
\begin{CJK}{UTF8}{mj}试求\end{CJK} $\mathscr{A}$ \begin{CJK}{UTF8}{mj}的核的标准正交基\end{CJK}.

\section{4. 中国海洋大学 2012 年研究生入学考试试题高等代数 
 李扬 
 微信公众号: sxkyliyang}
\begin{CJK}{UTF8}{mj}一\end{CJK}. \begin{CJK}{UTF8}{mj}填空题\end{CJK} (25 \begin{CJK}{UTF8}{mj}分\end{CJK})

\begin{enumerate}
  \item \begin{CJK}{UTF8}{mj}已知\end{CJK} $x+y+z=0, x y z \neq 0$, \begin{CJK}{UTF8}{mj}则\end{CJK} $\frac{x^{2}}{y z}+\frac{y^{2}}{x z}+\frac{z^{2}}{x y}=$

  \item $A$ \begin{CJK}{UTF8}{mj}为\end{CJK} $n$ \begin{CJK}{UTF8}{mj}级方阵\end{CJK}, $B$ \begin{CJK}{UTF8}{mj}为\end{CJK} $m$ \begin{CJK}{UTF8}{mj}级方阵\end{CJK}, \begin{CJK}{UTF8}{mj}记\end{CJK} $\left|\begin{array}{cc}A & 0 \\ 0 & B\end{array}\right|=p,\left|\begin{array}{cc}0 & A \\ B & 0\end{array}\right|=q$, \begin{CJK}{UTF8}{mj}则\end{CJK} $\frac{p}{q}$ \begin{CJK}{UTF8}{mj}值为\end{CJK}

  \item \begin{CJK}{UTF8}{mj}记\end{CJK} 4 \begin{CJK}{UTF8}{mj}阶矩阵\end{CJK} $A=\left(\alpha_{1}, \alpha_{2}, \alpha_{3}, \alpha_{4}\right), \alpha_{1}, \alpha_{2}, \alpha_{3}, \alpha_{4}$ \begin{CJK}{UTF8}{mj}为\end{CJK} $A$ \begin{CJK}{UTF8}{mj}的列向量\end{CJK}, \begin{CJK}{UTF8}{mj}其中\end{CJK} $\alpha_{1}, \alpha_{2}, \alpha_{3}$ \begin{CJK}{UTF8}{mj}线性无关\end{CJK}, $\alpha_{4}=\alpha_{1}-2 \alpha_{2}$. \begin{CJK}{UTF8}{mj}若\end{CJK} $\beta=\alpha_{1}-2 \alpha_{2}+5 \alpha_{3}+4 \alpha_{4}$, \begin{CJK}{UTF8}{mj}则线性方程组\end{CJK} $A X=\beta$ \begin{CJK}{UTF8}{mj}的通解为\end{CJK}

  \item \begin{CJK}{UTF8}{mj}设\end{CJK} $A=\left(\begin{array}{ccc}1 & 0 & 0 \\ 0 & 3 / 2 & 1 / 2 \\ 0 & 5 / 2 & 1\end{array}\right)$, \begin{CJK}{UTF8}{mj}则\end{CJK} $\left(A^{*}\right)^{-1}=$

  \item \begin{CJK}{UTF8}{mj}设复数矩阵\end{CJK} $A=\left(\begin{array}{ccc}3 & 1 & 0 \\ -4 & -1 & 0 \\ 4 & -8 & -2\end{array}\right), A$ \begin{CJK}{UTF8}{mj}的\end{CJK} Jordan \begin{CJK}{UTF8}{mj}标准形为\end{CJK}

\end{enumerate}
\begin{CJK}{UTF8}{mj}二\end{CJK}. \begin{CJK}{UTF8}{mj}判断题\end{CJK} (\begin{CJK}{UTF8}{mj}若结论正确\end{CJK}, \begin{CJK}{UTF8}{mj}请证明\end{CJK}; \begin{CJK}{UTF8}{mj}若错误\end{CJK}, \begin{CJK}{UTF8}{mj}举反例证明\end{CJK}. \begin{CJK}{UTF8}{mj}每题\end{CJK} 7 \begin{CJK}{UTF8}{mj}分\end{CJK}, \begin{CJK}{UTF8}{mj}共\end{CJK} 35 \begin{CJK}{UTF8}{mj}分\end{CJK})

\begin{enumerate}
  \item \begin{CJK}{UTF8}{mj}设\end{CJK} $A, B$ \begin{CJK}{UTF8}{mj}分别为\end{CJK} $n \times m, m \times n$ \begin{CJK}{UTF8}{mj}矩阵\end{CJK} $(n<m)$, \begin{CJK}{UTF8}{mj}且\end{CJK} $A B=E$, \begin{CJK}{UTF8}{mj}则\end{CJK} $B$ \begin{CJK}{UTF8}{mj}的列向量组线性无关\end{CJK}. ( )

  \item \begin{CJK}{UTF8}{mj}设\end{CJK} $A$ \begin{CJK}{UTF8}{mj}为\end{CJK} $m \times n$ \begin{CJK}{UTF8}{mj}矩阵\end{CJK}, $A x=0$ \begin{CJK}{UTF8}{mj}是非齐次线性方程组\end{CJK} $A x=b$ \begin{CJK}{UTF8}{mj}的导出组\end{CJK}, \begin{CJK}{UTF8}{mj}若\end{CJK} $A x=0$ \begin{CJK}{UTF8}{mj}只有零解\end{CJK}, \begin{CJK}{UTF8}{mj}则\end{CJK} $A x=b$ \begin{CJK}{UTF8}{mj}有\end{CJK} \begin{CJK}{UTF8}{mj}唯一解\end{CJK}. ( )

  \item \begin{CJK}{UTF8}{mj}设\end{CJK} $V_{1}, V_{2}$ \begin{CJK}{UTF8}{mj}是\end{CJK} $n$ \begin{CJK}{UTF8}{mj}维线性空间\end{CJK} $V$ \begin{CJK}{UTF8}{mj}的子空间\end{CJK}, \begin{CJK}{UTF8}{mj}满足\end{CJK} $V_{1} \bigcap V_{2}=\{0\}$, \begin{CJK}{UTF8}{mj}则\end{CJK} $\operatorname{dim} V_{1}+\operatorname{dim} V_{2}=n$. ( )

  \item \begin{CJK}{UTF8}{mj}若\end{CJK} $n$ \begin{CJK}{UTF8}{mj}级矩阵\end{CJK} $A$ \begin{CJK}{UTF8}{mj}与\end{CJK} $B$ \begin{CJK}{UTF8}{mj}有相同的特征值\end{CJK}, \begin{CJK}{UTF8}{mj}则\end{CJK} $A$ \begin{CJK}{UTF8}{mj}与\end{CJK} $B$ \begin{CJK}{UTF8}{mj}相似\end{CJK}. ( )

  \item \begin{CJK}{UTF8}{mj}如果\end{CJK} $n$ \begin{CJK}{UTF8}{mj}级可逆矩阵\end{CJK} $A$ \begin{CJK}{UTF8}{mj}与对角矩阵相似\end{CJK}, \begin{CJK}{UTF8}{mj}则\end{CJK} $A$ \begin{CJK}{UTF8}{mj}的伴随矩阵\end{CJK} $A^{*}$ \begin{CJK}{UTF8}{mj}也与对角矩阵相似\end{CJK}. ( )

\end{enumerate}
\begin{CJK}{UTF8}{mj}三\end{CJK}. (15 \begin{CJK}{UTF8}{mj}分\end{CJK}) \begin{CJK}{UTF8}{mj}试确定\end{CJK} $t$ \begin{CJK}{UTF8}{mj}的值\end{CJK}, \begin{CJK}{UTF8}{mj}使\end{CJK}
$$
f(x)=x^{3}-2 x^{2}+x+t
$$
\begin{CJK}{UTF8}{mj}有重根\end{CJK}, \begin{CJK}{UTF8}{mj}进而确定重根及重数\end{CJK}.

\begin{CJK}{UTF8}{mj}四\end{CJK}. ( 20 \begin{CJK}{UTF8}{mj}分\end{CJK}) $a$ \begin{CJK}{UTF8}{mj}取何值时\end{CJK}, \begin{CJK}{UTF8}{mj}线性方程组\end{CJK}
$$
\left\{\begin{array}{c}
(a+1) x_{1}+x_{2}+\cdots+x_{n}=0 \\
2 x_{1}+(a+2) x_{2}+\cdots+2 x_{n}=0 \\
\cdots \cdots \cdots \cdots \\
n x_{1}+n x_{2}+\cdots+(a+n) x_{n}=0
\end{array}\right.
$$
\begin{CJK}{UTF8}{mj}仅有零解\end{CJK}? \begin{CJK}{UTF8}{mj}有非零解\end{CJK}? \begin{CJK}{UTF8}{mj}有非零解时\end{CJK}, \begin{CJK}{UTF8}{mj}求通解\end{CJK}.

\begin{CJK}{UTF8}{mj}五\end{CJK}. ( 15 \begin{CJK}{UTF8}{mj}分\end{CJK}) \begin{CJK}{UTF8}{mj}设\end{CJK} $A, B, C$ \begin{CJK}{UTF8}{mj}都是\end{CJK} $n$ \begin{CJK}{UTF8}{mj}级矩阵\end{CJK}, $M=\left(\begin{array}{cc}A & A \\ C-B & C\end{array}\right)$.

(1) \begin{CJK}{UTF8}{mj}证明\end{CJK}: $M$ \begin{CJK}{UTF8}{mj}可逆的充分必要条件是\end{CJK} $A B$ \begin{CJK}{UTF8}{mj}可逆\end{CJK}.

(2) $M$ \begin{CJK}{UTF8}{mj}可逆时\end{CJK}, \begin{CJK}{UTF8}{mj}求其逆\end{CJK}.

\begin{CJK}{UTF8}{mj}六\end{CJK}. ( 10 \begin{CJK}{UTF8}{mj}分\end{CJK}) \begin{CJK}{UTF8}{mj}设\end{CJK} $A=\left(a_{i j}\right)$ \begin{CJK}{UTF8}{mj}是正定矩阵\end{CJK}, $b_{1}, b_{2}, \cdots, b_{n}$ \begin{CJK}{UTF8}{mj}是任意\end{CJK} $n$ \begin{CJK}{UTF8}{mj}个非零实数\end{CJK}. \begin{CJK}{UTF8}{mj}证明\end{CJK}: \begin{CJK}{UTF8}{mj}矩阵\end{CJK} $B=\left(a_{i j} b_{i} b_{j}\right)$ \begin{CJK}{UTF8}{mj}也是正定\end{CJK} \begin{CJK}{UTF8}{mj}矩阵\end{CJK}. \begin{CJK}{UTF8}{mj}七\end{CJK}. (15 \begin{CJK}{UTF8}{mj}分\end{CJK}) \begin{CJK}{UTF8}{mj}求\end{CJK} $2 n$ \begin{CJK}{UTF8}{mj}阶方阵\end{CJK} $A$ \begin{CJK}{UTF8}{mj}的最小多项式\end{CJK}, \begin{CJK}{UTF8}{mj}其中\end{CJK}
$$
A=\left(\begin{array}{cccccc}
a & & & & & b \\
& \ddots & & & \cdot & \\
& & a & b & & \\
& & b & a & & \\
& . & & & \ddots & \\
& & & & & a
\end{array}\right)
$$
\begin{CJK}{UTF8}{mj}八\end{CJK}. (15 \begin{CJK}{UTF8}{mj}分\end{CJK}) \begin{CJK}{UTF8}{mj}设\end{CJK} $n$ \begin{CJK}{UTF8}{mj}维欧式空间\end{CJK} $V$ \begin{CJK}{UTF8}{mj}的基\end{CJK} $\alpha_{1}, \alpha_{2}, \cdots, \alpha_{n}$ \begin{CJK}{UTF8}{mj}的度量矩阵为\end{CJK} $G, V$ \begin{CJK}{UTF8}{mj}的线性变换\end{CJK} $\mathscr{A}$ \begin{CJK}{UTF8}{mj}在该基下的矩阵为\end{CJK} $A$, \begin{CJK}{UTF8}{mj}证明\end{CJK}:

(1) \begin{CJK}{UTF8}{mj}若\end{CJK} $\mathscr{A}$ \begin{CJK}{UTF8}{mj}是正交变换\end{CJK}, \begin{CJK}{UTF8}{mj}则\end{CJK} $A^{\prime} G A=G$.

(2) \begin{CJK}{UTF8}{mj}若\end{CJK} $\mathscr{A}$ \begin{CJK}{UTF8}{mj}是对称变换\end{CJK}, \begin{CJK}{UTF8}{mj}则\end{CJK} $A^{\prime} G=G A$.

\section{5. 中国海洋大学 2013 年研究生入学考试试题高等代数}
\begin{CJK}{UTF8}{mj}李扬\end{CJK}

\begin{CJK}{UTF8}{mj}微信公众号\end{CJK}: sxkyliyang

\begin{CJK}{UTF8}{mj}一\end{CJK}. \begin{CJK}{UTF8}{mj}填空题\end{CJK}

\begin{enumerate}
  \item \begin{CJK}{UTF8}{mj}已知\end{CJK} $(x-1)^{2} \mid A x^{4}+B x^{2}+1$, \begin{CJK}{UTF8}{mj}则\end{CJK} $A=$ $B=$

  \item $n$ \begin{CJK}{UTF8}{mj}级行列式\end{CJK} $D=a \neq 0$, \begin{CJK}{UTF8}{mj}且\end{CJK} $D$ \begin{CJK}{UTF8}{mj}的每行元素之和均为\end{CJK} $b$, \begin{CJK}{UTF8}{mj}则\end{CJK} $D$ \begin{CJK}{UTF8}{mj}的第一行元素的代数余子式之和等于\end{CJK}

  \item \begin{CJK}{UTF8}{mj}记\end{CJK} 4 \begin{CJK}{UTF8}{mj}阶矩阵\end{CJK} $A=\left(\alpha_{1}, \alpha_{2}, \alpha_{3}, \alpha_{4}\right), \alpha_{1}, \alpha_{2}, \alpha_{3}, \alpha_{4}$ \begin{CJK}{UTF8}{mj}为\end{CJK} $A$ \begin{CJK}{UTF8}{mj}的列向量\end{CJK}, \begin{CJK}{UTF8}{mj}其中\end{CJK} $\alpha_{2}, \alpha_{3}, \alpha_{4}$ \begin{CJK}{UTF8}{mj}线性无关\end{CJK}, $\alpha_{1}=2 \alpha_{2}-\alpha_{3}$, \begin{CJK}{UTF8}{mj}若\end{CJK} $\beta=\alpha_{1}+\alpha_{2}+\alpha_{3}+\alpha_{4}$, \begin{CJK}{UTF8}{mj}则\end{CJK} $A X=\beta$ \begin{CJK}{UTF8}{mj}的通解为\end{CJK}

  \item \begin{CJK}{UTF8}{mj}设\end{CJK} $A$ \begin{CJK}{UTF8}{mj}是\end{CJK} 3 \begin{CJK}{UTF8}{mj}级实对称矩阵\end{CJK}, \begin{CJK}{UTF8}{mj}且\end{CJK} $|A|=-2$, \begin{CJK}{UTF8}{mj}已知\end{CJK} $1,-1$ \begin{CJK}{UTF8}{mj}是\end{CJK} $A$ \begin{CJK}{UTF8}{mj}的特征值\end{CJK}, \begin{CJK}{UTF8}{mj}则\end{CJK} $\left|A^{*}+2 A+2 E\right|=$

  \item \begin{CJK}{UTF8}{mj}设\end{CJK} $\alpha=(a, b, c), \beta=(x, y, z)$, \begin{CJK}{UTF8}{mj}已知\end{CJK} $\alpha^{\prime} \beta=\left(\begin{array}{ccc}-2 & 4 & -6 \\ 1 & -2 & 3 \\ -1 & 2 & -3\end{array}\right)$, \begin{CJK}{UTF8}{mj}则\end{CJK} $\alpha \beta^{\prime}=$

  \item \begin{CJK}{UTF8}{mj}与\end{CJK} $\left(\begin{array}{ll}1 & 1 \\ 0 & 1\end{array}\right)$ \begin{CJK}{UTF8}{mj}可交换的二阶矩阵为\end{CJK}

  \item \begin{CJK}{UTF8}{mj}复数域\end{CJK} $\mathbb{C}$ \begin{CJK}{UTF8}{mj}作为实数域\end{CJK} $\mathbb{R}$ \begin{CJK}{UTF8}{mj}上的线性空间\end{CJK}, $\mathbb{C}$ \begin{CJK}{UTF8}{mj}的维数是\end{CJK} \begin{CJK}{UTF8}{mj}它的一个基为\end{CJK}

\end{enumerate}
\begin{CJK}{UTF8}{mj}二\end{CJK}. \begin{CJK}{UTF8}{mj}判断题\end{CJK} (\begin{CJK}{UTF8}{mj}正确的给予证明\end{CJK}, \begin{CJK}{UTF8}{mj}错误的加以说明或举反例\end{CJK})

\begin{enumerate}
  \item $\alpha_{1}, \alpha_{2}, \cdots, \alpha_{m}(m>2)$ \begin{CJK}{UTF8}{mj}线性无关的充要条件是任意两个向量线性无关\end{CJK}. ( )

  \item \begin{CJK}{UTF8}{mj}设\end{CJK} $A$ \begin{CJK}{UTF8}{mj}为\end{CJK} $m \times n$ \begin{CJK}{UTF8}{mj}矩阵\end{CJK}, \begin{CJK}{UTF8}{mj}若\end{CJK} $\mathrm{r}(A)=n$, \begin{CJK}{UTF8}{mj}则\end{CJK} $A x=b$ \begin{CJK}{UTF8}{mj}有唯一解\end{CJK}. ( )

  \item \begin{CJK}{UTF8}{mj}对于任意全不为零的\end{CJK} $x_{1}, x_{2}, \cdots, x_{n}$, \begin{CJK}{UTF8}{mj}二次型\end{CJK} $f\left(x_{1}, x_{2}, \cdots, x_{n}\right)$ \begin{CJK}{UTF8}{mj}恒大于\end{CJK} 0 , \begin{CJK}{UTF8}{mj}则\end{CJK} $f\left(x_{1}, x_{2}, \cdots, x_{n}\right)$ \begin{CJK}{UTF8}{mj}正定\end{CJK}.

  \item \begin{CJK}{UTF8}{mj}若\end{CJK} $n$ \begin{CJK}{UTF8}{mj}级矩阵\end{CJK} $A$ \begin{CJK}{UTF8}{mj}与\end{CJK} $B$ \begin{CJK}{UTF8}{mj}有相同的最小多项式\end{CJK}, \begin{CJK}{UTF8}{mj}则\end{CJK} $A \sim B$. ( )

  \item \begin{CJK}{UTF8}{mj}线性变换\end{CJK} $\mathscr{A}$ \begin{CJK}{UTF8}{mj}的特征向量之和仍为\end{CJK} $\mathscr{A}$ \begin{CJK}{UTF8}{mj}的特征向量\end{CJK}. \begin{CJK}{UTF8}{mj}三\end{CJK}. \begin{CJK}{UTF8}{mj}设\end{CJK} $f(x)$ \begin{CJK}{UTF8}{mj}是一个整系数多项式\end{CJK}, \begin{CJK}{UTF8}{mj}试证\end{CJK}: \begin{CJK}{UTF8}{mj}若\end{CJK} $f(0)$ \begin{CJK}{UTF8}{mj}与\end{CJK} $f(1)$ \begin{CJK}{UTF8}{mj}都是奇数\end{CJK}, \begin{CJK}{UTF8}{mj}那么\end{CJK} $f(x)$ \begin{CJK}{UTF8}{mj}不能有整根\end{CJK}.

\end{enumerate}
\begin{CJK}{UTF8}{mj}四\end{CJK}. \begin{CJK}{UTF8}{mj}若\end{CJK} $A$ \begin{CJK}{UTF8}{mj}是\end{CJK} $n$ \begin{CJK}{UTF8}{mj}级正定矩阵\end{CJK}, $X=\left(x_{1}, x_{2}, \cdots, x_{n}\right)^{\prime}$, \begin{CJK}{UTF8}{mj}证明\end{CJK}:
$$
f(X)=\operatorname{det}\left(\begin{array}{cc}
0 & X^{\prime} \\
X & A
\end{array}\right)
$$
\begin{CJK}{UTF8}{mj}是一个负定二次型\end{CJK}.

\begin{CJK}{UTF8}{mj}五\end{CJK}. \begin{CJK}{UTF8}{mj}设\end{CJK} $M$ \begin{CJK}{UTF8}{mj}是数域\end{CJK} $P$ \begin{CJK}{UTF8}{mj}上形如\end{CJK}
$$
A=\left(\begin{array}{cccc}
a_{1} & a_{2} & \cdots & a_{n} \\
a_{n} & a_{1} & \cdots & a_{n-1} \\
\vdots & \vdots & & \vdots \\
a_{2} & a_{3} & \cdots & a_{1}
\end{array}\right)
$$
\begin{CJK}{UTF8}{mj}的循环矩阵的集合\end{CJK}.

(1)\begin{CJK}{UTF8}{mj}证明\end{CJK}: $M$ \begin{CJK}{UTF8}{mj}是线性空间\end{CJK} $P^{n \times n}$ \begin{CJK}{UTF8}{mj}的子空间\end{CJK}.

(2) \begin{CJK}{UTF8}{mj}证明\end{CJK}: \begin{CJK}{UTF8}{mj}任意的\end{CJK} $A, B$ \begin{CJK}{UTF8}{mj}属于\end{CJK} $M$, \begin{CJK}{UTF8}{mj}有\end{CJK} $A B=B A$.

(3) \begin{CJK}{UTF8}{mj}求\end{CJK} $M$ \begin{CJK}{UTF8}{mj}的维数和一组基\end{CJK}. \begin{CJK}{UTF8}{mj}六\end{CJK}. \begin{CJK}{UTF8}{mj}设\end{CJK} $\alpha, \beta$ \begin{CJK}{UTF8}{mj}是欧氏空间中的两个非零正交向量\end{CJK}, $\alpha=\left(a_{1}, a_{2}, \cdots, a_{n}\right), \beta=\left(b_{1}, b_{2}, \cdots, b_{n}\right)$.

(1) \begin{CJK}{UTF8}{mj}求\end{CJK} $A=\alpha^{\prime} \beta$ \begin{CJK}{UTF8}{mj}的特征值和特征向量\end{CJK}.

(2) $A$ \begin{CJK}{UTF8}{mj}能否对角化\end{CJK}? \begin{CJK}{UTF8}{mj}为什么\end{CJK}?

\begin{CJK}{UTF8}{mj}七\end{CJK}. \begin{CJK}{UTF8}{mj}设\end{CJK} $\mathscr{A}$ \begin{CJK}{UTF8}{mj}是数域\end{CJK} $P$ \begin{CJK}{UTF8}{mj}上的\end{CJK} $n$ \begin{CJK}{UTF8}{mj}维线性空间\end{CJK} $V$ \begin{CJK}{UTF8}{mj}上的线性变换\end{CJK}, \begin{CJK}{UTF8}{mj}且\end{CJK} $\mathscr{A}^{2}=\mathscr{A}$. \begin{CJK}{UTF8}{mj}证\end{CJK}:

(1) $\mathscr{A}^{-1}(0)=\{\zeta-\mathscr{A}(\zeta) \mid \zeta \in V\}$.

(2) $V=\mathscr{A}^{-1}(0) \oplus \mathscr{A}(V)$.

\section{6. 中国海洋大学 2014 年研究生入学考试试题高等代数 
 李扬 
 微信公众号: sxkyliyang}
\section{一. 填空题 ( 35 分)}
\begin{enumerate}
  \item \begin{CJK}{UTF8}{mj}用\end{CJK} $(x-1)^{2}$ \begin{CJK}{UTF8}{mj}去除\end{CJK} $x^{4}-2 x^{2}+3$, \begin{CJK}{UTF8}{mj}则余式为\end{CJK}

  \item \begin{CJK}{UTF8}{mj}设\end{CJK} 4 \begin{CJK}{UTF8}{mj}阶行列式\end{CJK} $D=4$, \begin{CJK}{UTF8}{mj}且\end{CJK} $D$ \begin{CJK}{UTF8}{mj}的每列元素之和均为\end{CJK} 2 , \begin{CJK}{UTF8}{mj}则\end{CJK} $A_{21}+A_{22}+A_{23}+A_{24}=$

  \item $n$ \begin{CJK}{UTF8}{mj}级方阵\end{CJK} $A$ \begin{CJK}{UTF8}{mj}的每行元素之和都是\end{CJK} 0 , \begin{CJK}{UTF8}{mj}且秩\end{CJK} $A=n-1$, \begin{CJK}{UTF8}{mj}则线性方程组\end{CJK} $A X=0$ \begin{CJK}{UTF8}{mj}的通解是\end{CJK}

  \item \begin{CJK}{UTF8}{mj}若线性变换\end{CJK} $\mathscr{A}$ \begin{CJK}{UTF8}{mj}在基\end{CJK} $\left\{\alpha_{1}, \alpha_{2}\right\}$ \begin{CJK}{UTF8}{mj}下的矩阵为\end{CJK} $\left(\begin{array}{cc}a & b \\ c & d\end{array}\right)$, \begin{CJK}{UTF8}{mj}那么线性变换\end{CJK} $\mathscr{A}$ \begin{CJK}{UTF8}{mj}在基\end{CJK} $\left\{\alpha_{2}, 3 \alpha_{1}\right\}$ \begin{CJK}{UTF8}{mj}下的矩阵为\end{CJK}

  \item \begin{CJK}{UTF8}{mj}已知\end{CJK} 6 \begin{CJK}{UTF8}{mj}阶复矩阵\end{CJK} $A$ \begin{CJK}{UTF8}{mj}的不变因子为\end{CJK} $1,1,1,1, \lambda+3,(\lambda+3)\left(\lambda^{2}+4\right)^{2}$. \begin{CJK}{UTF8}{mj}则\end{CJK} $A$ \begin{CJK}{UTF8}{mj}的全部初等因子为\end{CJK}

  \item \begin{CJK}{UTF8}{mj}平方等于单位矩阵的所有二阶矩阵为\end{CJK}

  \item \begin{CJK}{UTF8}{mj}设\end{CJK} $n$ \begin{CJK}{UTF8}{mj}级矩阵\end{CJK} $A$ \begin{CJK}{UTF8}{mj}的所有元素都是\end{CJK} 1 , \begin{CJK}{UTF8}{mj}则\end{CJK} $A$ \begin{CJK}{UTF8}{mj}的最小多项式为\end{CJK}

\end{enumerate}
\begin{CJK}{UTF8}{mj}二\end{CJK}. \begin{CJK}{UTF8}{mj}判断正误并说明理由或举反例\end{CJK} (\begin{CJK}{UTF8}{mj}每题\end{CJK} 7 \begin{CJK}{UTF8}{mj}分\end{CJK}, \begin{CJK}{UTF8}{mj}共\end{CJK} 35 \begin{CJK}{UTF8}{mj}分\end{CJK})

\begin{enumerate}
  \item \begin{CJK}{UTF8}{mj}如果\end{CJK} $\alpha$ \begin{CJK}{UTF8}{mj}是\end{CJK} $f^{\prime}(x)$ \begin{CJK}{UTF8}{mj}的\end{CJK} $m$ \begin{CJK}{UTF8}{mj}重根\end{CJK}, \begin{CJK}{UTF8}{mj}那么\end{CJK} $\alpha$ \begin{CJK}{UTF8}{mj}是\end{CJK} $f(x)$ \begin{CJK}{UTF8}{mj}的\end{CJK} $m+1$ \begin{CJK}{UTF8}{mj}重根\end{CJK}. ( )

  \item \begin{CJK}{UTF8}{mj}设\end{CJK} $A$ \begin{CJK}{UTF8}{mj}为\end{CJK} $m \times n$ \begin{CJK}{UTF8}{mj}矩阵\end{CJK}, \begin{CJK}{UTF8}{mj}若齐次方程组\end{CJK} $A x=0$ \begin{CJK}{UTF8}{mj}只有零解\end{CJK}, \begin{CJK}{UTF8}{mj}则\end{CJK} $A x=b$ \begin{CJK}{UTF8}{mj}有唯一解\end{CJK}. ( )

  \item \begin{CJK}{UTF8}{mj}二次型\end{CJK} $f\left(x_{1}, x_{2}, \cdots, x_{n}\right)$ \begin{CJK}{UTF8}{mj}的矩阵是唯一的\end{CJK}. ( )

  \item $V_{1}, V_{2}$ \begin{CJK}{UTF8}{mj}是数域\end{CJK} $P$ \begin{CJK}{UTF8}{mj}上线性空间\end{CJK} $V$ \begin{CJK}{UTF8}{mj}的两个子空间\end{CJK}, \begin{CJK}{UTF8}{mj}则\end{CJK} $V_{1} \bigcup V_{2}$ \begin{CJK}{UTF8}{mj}也是\end{CJK} $V$ \begin{CJK}{UTF8}{mj}的子空间\end{CJK}. ( )

  \item \begin{CJK}{UTF8}{mj}若\end{CJK} $\alpha_{1}, \alpha_{2}, \alpha_{3}, \alpha_{4}$ \begin{CJK}{UTF8}{mj}是数域\end{CJK} $\mathbb{F}$ \begin{CJK}{UTF8}{mj}上的\end{CJK} 4 \begin{CJK}{UTF8}{mj}维向量空间\end{CJK} $V$ \begin{CJK}{UTF8}{mj}的一组基\end{CJK}, \begin{CJK}{UTF8}{mj}那么\end{CJK} $\alpha_{1}, \alpha_{2}, \alpha_{2}+\alpha_{3}, \alpha_{3}+\alpha_{4}$ \begin{CJK}{UTF8}{mj}也是\end{CJK} $V$ \begin{CJK}{UTF8}{mj}的\end{CJK} \begin{CJK}{UTF8}{mj}一组基\end{CJK}. ( )

\end{enumerate}
\begin{CJK}{UTF8}{mj}三\end{CJK}. (15 \begin{CJK}{UTF8}{mj}分\end{CJK}) \begin{CJK}{UTF8}{mj}设数域\end{CJK} $P$ \begin{CJK}{UTF8}{mj}上多项式\end{CJK} $f(x), g(x), h(x), l(x)$ \begin{CJK}{UTF8}{mj}满足\end{CJK}
$$
\begin{aligned}
&\left(x^{2}+1\right) h(x)+(x+1) f(x)+(x+2) g(x)=0, \\
&\left(x^{2}+1\right) l(x)+(x-1) f(x)+(x-2) g(x)=0
\end{aligned}
$$
\begin{CJK}{UTF8}{mj}证明\end{CJK}: $\left(x^{2}+1\right)\left|f(x),\left(x^{2}+1\right)\right| g(x)$.

\begin{CJK}{UTF8}{mj}四\end{CJK}. ( 15 \begin{CJK}{UTF8}{mj}分\end{CJK}) \begin{CJK}{UTF8}{mj}设\end{CJK} $A=\left(a_{i j}\right)$ \begin{CJK}{UTF8}{mj}是\end{CJK} $n$ \begin{CJK}{UTF8}{mj}级正定矩阵\end{CJK}, \begin{CJK}{UTF8}{mj}证明\end{CJK}:
$$
|A| \leqslant \prod_{i=1}^{n} a_{i i}
$$
\begin{CJK}{UTF8}{mj}五\end{CJK}. ( 20 \begin{CJK}{UTF8}{mj}分\end{CJK}) $n$ \begin{CJK}{UTF8}{mj}阶矩阵\end{CJK}
$$
A=\left(\begin{array}{cccccc}
a & 1 & 0 & \cdots & 0 & 0 \\
0 & a & 1 & \cdots & 0 & 0 \\
0 & 0 & a & \cdots & 0 & 0 \\
\vdots & \vdots & \vdots & & \vdots & \vdots \\
0 & 0 & 0 & \cdots & a & 1 \\
0 & 0 & 0 & \cdots & 0 & a
\end{array}\right), a \in \mathbb{R} .
$$
(1) \begin{CJK}{UTF8}{mj}证明\end{CJK}: \begin{CJK}{UTF8}{mj}与\end{CJK} $A$ \begin{CJK}{UTF8}{mj}可交换的矩阵全体\end{CJK} $W$ \begin{CJK}{UTF8}{mj}构成\end{CJK} $\mathbb{R}^{n \times n}$ \begin{CJK}{UTF8}{mj}的子空间\end{CJK}.

(2) \begin{CJK}{UTF8}{mj}求\end{CJK} $W$ \begin{CJK}{UTF8}{mj}的一组基和维数\end{CJK}.

\begin{CJK}{UTF8}{mj}六\end{CJK}. (15 \begin{CJK}{UTF8}{mj}分\end{CJK}) \begin{CJK}{UTF8}{mj}设\end{CJK} $\mathscr{A}$ \begin{CJK}{UTF8}{mj}是\end{CJK} $n$ \begin{CJK}{UTF8}{mj}维欧式空间\end{CJK} $V$ \begin{CJK}{UTF8}{mj}的一个线性变换\end{CJK}, \begin{CJK}{UTF8}{mj}如果\end{CJK} $\mathscr{A}$ \begin{CJK}{UTF8}{mj}是正交变换又是对称变换\end{CJK}, \begin{CJK}{UTF8}{mj}证明\end{CJK}: $\mathscr{A}^{2}$ \begin{CJK}{UTF8}{mj}是单位变换\end{CJK}. \begin{CJK}{UTF8}{mj}七\end{CJK}. (15 \begin{CJK}{UTF8}{mj}分\end{CJK}) \begin{CJK}{UTF8}{mj}设\end{CJK} $\alpha_{1}, \alpha_{2}, \cdots, \alpha_{n}$ \begin{CJK}{UTF8}{mj}是数域\end{CJK} $P$ \begin{CJK}{UTF8}{mj}上向量空间\end{CJK} $V$ \begin{CJK}{UTF8}{mj}的一个基\end{CJK}, \begin{CJK}{UTF8}{mj}令\end{CJK} $\beta_{i}=\alpha_{1}+\alpha_{2}+\cdots+\alpha_{i}$, \begin{CJK}{UTF8}{mj}且\end{CJK} $W_{i}=L\left(\beta_{i}\right), i=$ $1,2, \cdots, n$, \begin{CJK}{UTF8}{mj}证明\end{CJK}:
$$
V=W_{1} \oplus W_{2} \oplus \cdots \oplus W_{n}
$$

\section{7. 中国海洋大学 2015 年研究生入学考试试题高等代数}
\begin{CJK}{UTF8}{mj}李扬\end{CJK}

\begin{CJK}{UTF8}{mj}微信公众号\end{CJK}: sxkyliyang

\begin{CJK}{UTF8}{mj}一\end{CJK}. \begin{CJK}{UTF8}{mj}填空题\end{CJK} (30 \begin{CJK}{UTF8}{mj}分\end{CJK})

\begin{enumerate}
  \item \begin{CJK}{UTF8}{mj}设矩阵\end{CJK} $A$ \begin{CJK}{UTF8}{mj}满足\end{CJK} $A^{2}+A-4 I=0, I$ \begin{CJK}{UTF8}{mj}为单位矩阵\end{CJK}, \begin{CJK}{UTF8}{mj}则\end{CJK} $(A-I)^{-1}=$
\end{enumerate}
2 . \begin{CJK}{UTF8}{mj}设矩阵\end{CJK} $A=\left(\begin{array}{lll}1 & a & a \\ a & 1 & a \\ a & a & 1\end{array}\right)$ \begin{CJK}{UTF8}{mj}的秩为\end{CJK} 2 , \begin{CJK}{UTF8}{mj}则\end{CJK} $a=$

\begin{enumerate}
  \setcounter{enumi}{3}
  \item \begin{CJK}{UTF8}{mj}设\end{CJK} 4 \begin{CJK}{UTF8}{mj}阶方阵\end{CJK} $A$ \begin{CJK}{UTF8}{mj}的秩为\end{CJK} 2 , \begin{CJK}{UTF8}{mj}则其伴随矩阵\end{CJK} $A^{*}$ \begin{CJK}{UTF8}{mj}的秩为\end{CJK}

  \item \begin{CJK}{UTF8}{mj}设三阶方阵\end{CJK} $A$ \begin{CJK}{UTF8}{mj}的特征值为\end{CJK} $1, \frac{1}{2}, \frac{1}{3}$, \begin{CJK}{UTF8}{mj}则\end{CJK} $\left|A^{*}+A^{-1}+I\right|=$

  \item \begin{CJK}{UTF8}{mj}二次型\end{CJK} $f\left(x_{1}, x_{2}, x_{3}\right)=x_{1}^{2}+x_{2}^{2}+x_{3}^{2}-x_{1} x_{2}-x_{1} x_{3}-x_{2} x_{3}$ \begin{CJK}{UTF8}{mj}的正惯性指数为\end{CJK}

  \item \begin{CJK}{UTF8}{mj}矩阵\end{CJK} $\left(\begin{array}{ccc}0 & 0 & 1 \\ 0 & 1 & 0 \\ 1 & 0 & 0\end{array}\right)$ \begin{CJK}{UTF8}{mj}的最小多项式是\end{CJK}

\end{enumerate}
\begin{CJK}{UTF8}{mj}二\end{CJK}. (10 \begin{CJK}{UTF8}{mj}分\end{CJK}) \begin{CJK}{UTF8}{mj}设\end{CJK} $A, B$ \begin{CJK}{UTF8}{mj}为实对称阵\end{CJK}, $A$ \begin{CJK}{UTF8}{mj}为正定的\end{CJK}, \begin{CJK}{UTF8}{mj}求证\end{CJK}: \begin{CJK}{UTF8}{mj}存在可逆阵\end{CJK} $P$, \begin{CJK}{UTF8}{mj}使得\end{CJK} $P^{\prime} A P=E, P^{\prime} B P$ \begin{CJK}{UTF8}{mj}为对角矩阵\end{CJK}. \begin{CJK}{UTF8}{mj}这里\end{CJK} $E$ \begin{CJK}{UTF8}{mj}为单位矩阵\end{CJK}.

\begin{CJK}{UTF8}{mj}三\end{CJK}. (30 \begin{CJK}{UTF8}{mj}分\end{CJK}) \begin{CJK}{UTF8}{mj}设\end{CJK}
$$
A=\left(\begin{array}{ccc}
\lambda & 1 & 1 \\
0 & \lambda-1 & 0 \\
1 & 1 & \lambda
\end{array}\right), b=\left(\begin{array}{l}
a \\
1 \\
1
\end{array}\right)
$$
\begin{CJK}{UTF8}{mj}已知线性方程组\end{CJK} $A x=b$ \begin{CJK}{UTF8}{mj}有无穷多解\end{CJK}. \begin{CJK}{UTF8}{mj}求\end{CJK}:

(1) $\lambda, a$;

(2) $A x=b$ \begin{CJK}{UTF8}{mj}的一般解\end{CJK}.

\begin{CJK}{UTF8}{mj}四\end{CJK}. (15 \begin{CJK}{UTF8}{mj}分\end{CJK}) $f(x)$ \begin{CJK}{UTF8}{mj}是次数大于\end{CJK} 0 \begin{CJK}{UTF8}{mj}的首一整系数多项式\end{CJK}, \begin{CJK}{UTF8}{mj}若\end{CJK} $f(0), f(1)$ \begin{CJK}{UTF8}{mj}都是奇数\end{CJK}. \begin{CJK}{UTF8}{mj}求证\end{CJK}: $f(x)$ \begin{CJK}{UTF8}{mj}没有有理根\end{CJK}.

\begin{CJK}{UTF8}{mj}五\end{CJK}. (30 \begin{CJK}{UTF8}{mj}分\end{CJK}) \begin{CJK}{UTF8}{mj}设\end{CJK}
$$
A=\left(\begin{array}{ccc}
1 & -2 & 2 \\
-2 & -2 & 4 \\
2 & 4 & -2
\end{array}\right)
$$
\begin{CJK}{UTF8}{mj}求正交矩阵\end{CJK} $T$, \begin{CJK}{UTF8}{mj}使\end{CJK} $T^{-1} A T$ \begin{CJK}{UTF8}{mj}为对角矩阵\end{CJK}.

\begin{CJK}{UTF8}{mj}六\end{CJK}. ( 15 \begin{CJK}{UTF8}{mj}分\end{CJK}) \begin{CJK}{UTF8}{mj}设\end{CJK} $A$ \begin{CJK}{UTF8}{mj}是\end{CJK} $n$ \begin{CJK}{UTF8}{mj}级矩阵且\end{CJK} $\mathrm{r}(A)=r$, \begin{CJK}{UTF8}{mj}求证\end{CJK}: $A^{2}=A$ \begin{CJK}{UTF8}{mj}的充要条件是存在秩等于\end{CJK} $r$ \begin{CJK}{UTF8}{mj}的\end{CJK} $n \times r$ \begin{CJK}{UTF8}{mj}矩阵\end{CJK} $S$ \begin{CJK}{UTF8}{mj}和秩等于\end{CJK} $r$ \begin{CJK}{UTF8}{mj}的\end{CJK} $r \times n$ \begin{CJK}{UTF8}{mj}矩阵\end{CJK} $T$, \begin{CJK}{UTF8}{mj}使\end{CJK} $A=S T, T S=I_{r}$. \begin{CJK}{UTF8}{mj}这里\end{CJK} $I_{r}$ \begin{CJK}{UTF8}{mj}为\end{CJK} $r$ \begin{CJK}{UTF8}{mj}阶单位矩阵\end{CJK}.

\begin{CJK}{UTF8}{mj}七\end{CJK}. (15 \begin{CJK}{UTF8}{mj}分\end{CJK}) \begin{CJK}{UTF8}{mj}设\end{CJK} $\mathscr{A}$ \begin{CJK}{UTF8}{mj}是数域\end{CJK} $\mathbb{F}$ \begin{CJK}{UTF8}{mj}上\end{CJK} $n$ \begin{CJK}{UTF8}{mj}维线性空间\end{CJK} $V$ \begin{CJK}{UTF8}{mj}的线性变换\end{CJK}, \begin{CJK}{UTF8}{mj}求证\end{CJK}: \begin{CJK}{UTF8}{mj}必存在正整数\end{CJK} $m$, \begin{CJK}{UTF8}{mj}使\end{CJK}
$$
\begin{gathered}
\operatorname{Im} \mathscr{A}^{m}=\operatorname{Im} \mathscr{A}^{m+1} \\
\operatorname{ker} \mathscr{A}^{m}=\operatorname{ker} \mathscr{A}^{m+1} \\
V=\operatorname{Im} \mathscr{A}^{m} \oplus \operatorname{ker} \mathscr{A}^{m}
\end{gathered}
$$

\section{8. 中国海洋大学 2016 年研究生入学考试试题高等代数 
 李扬 
 微信公众号: sxkyliyang}
\begin{CJK}{UTF8}{mj}一\end{CJK}. \begin{CJK}{UTF8}{mj}填空题\end{CJK} (\begin{CJK}{UTF8}{mj}每小题\end{CJK} 5 \begin{CJK}{UTF8}{mj}分\end{CJK}, \begin{CJK}{UTF8}{mj}共\end{CJK} 30 \begin{CJK}{UTF8}{mj}分\end{CJK})

\begin{enumerate}
  \item \begin{CJK}{UTF8}{mj}设\end{CJK} $f(x)=x^{3}-3 x+k$ \begin{CJK}{UTF8}{mj}有重根\end{CJK}, \begin{CJK}{UTF8}{mj}则\end{CJK} $k$ \begin{CJK}{UTF8}{mj}可能的取值是\end{CJK}

  \item \begin{CJK}{UTF8}{mj}矩阵\end{CJK} $A=\left(\begin{array}{ccc}0 & 0 & -1 \\ 0 & 1 & 4 \\ 2 & 1 & 5\end{array}\right), B=\left(\begin{array}{cc}A & A \\ 2 A & A\end{array}\right)$, \begin{CJK}{UTF8}{mj}则行列式\end{CJK} $|B|=$

  \item \begin{CJK}{UTF8}{mj}设\end{CJK} $A$ \begin{CJK}{UTF8}{mj}为\end{CJK} 3 \begin{CJK}{UTF8}{mj}阶方阵\end{CJK}, $A^{*}$ \begin{CJK}{UTF8}{mj}为\end{CJK} $A$ \begin{CJK}{UTF8}{mj}的伴随矩阵\end{CJK}. \begin{CJK}{UTF8}{mj}已知\end{CJK} $|A|=\frac{1}{2}$, \begin{CJK}{UTF8}{mj}则行列式\end{CJK} $\left|(2 A)^{-1}-2 A^{*}\right|=$

  \item $P[x]_{4}$ \begin{CJK}{UTF8}{mj}表示数域\end{CJK} $P$ \begin{CJK}{UTF8}{mj}上次数不超过\end{CJK} 4 \begin{CJK}{UTF8}{mj}的一元多项式构成的线性空间\end{CJK}. \begin{CJK}{UTF8}{mj}则多项式\end{CJK} $f_{1}(x)=1+4 x-2 x^{2}+$ $x^{3}, f_{2}(x)=-1+9 x-3 x^{2}+2 x^{3}, f_{3}(x)=-5+6 x+x^{3}, f_{4}(x)=5+7 x-5 x^{2}+2 x^{3}$ \begin{CJK}{UTF8}{mj}的秩为\end{CJK}

  \item \begin{CJK}{UTF8}{mj}已知\end{CJK} $\left(\begin{array}{cc}0 & 1 \\ -2 & 1\end{array}\right) X=\left(\begin{array}{cc}1 & 1 \\ -1 & 1\end{array}\right)$, \begin{CJK}{UTF8}{mj}则\end{CJK} $X=$

  \item \begin{CJK}{UTF8}{mj}矩阵\end{CJK} $\left(\begin{array}{ccc}1 & 1 & 0 \\ 0 & 1 & 1 \\ 0 & 0 & 1\end{array}\right)$ \begin{CJK}{UTF8}{mj}的最小多项式为\end{CJK}

\end{enumerate}
\begin{CJK}{UTF8}{mj}二\end{CJK}. ( 20 \begin{CJK}{UTF8}{mj}分\end{CJK}) $\lambda$ \begin{CJK}{UTF8}{mj}取何值时\end{CJK}, \begin{CJK}{UTF8}{mj}线性方程组\end{CJK}
$$
\left\{\begin{array}{l}
x_{1}+\left(\lambda^{2}+1\right) x_{2}+2 x_{3}=\lambda \\
\lambda x_{1}+\lambda x_{2}+(2 \lambda+1) x_{3}=0 \\
x_{1}+(2 \lambda+1) x_{2}+2 x_{3}=2
\end{array}\right.
$$
\begin{CJK}{UTF8}{mj}有解\end{CJK}? \begin{CJK}{UTF8}{mj}并在有解时求出其解\end{CJK}.

\begin{CJK}{UTF8}{mj}三\end{CJK}. (15 \begin{CJK}{UTF8}{mj}分\end{CJK}) \begin{CJK}{UTF8}{mj}证明\end{CJK}: \begin{CJK}{UTF8}{mj}向量组\end{CJK} $\alpha_{1}, \alpha_{2}, \cdots, \alpha_{m}$ \begin{CJK}{UTF8}{mj}线性无关的充要条件是向量组\end{CJK} $\alpha_{1}, \alpha_{1}+\alpha_{2}, \cdots, \alpha_{1}+\alpha_{2}+\cdots+\alpha_{m}$ \begin{CJK}{UTF8}{mj}线性无关\end{CJK}.

\begin{CJK}{UTF8}{mj}四\end{CJK}. (15 \begin{CJK}{UTF8}{mj}分\end{CJK}) \begin{CJK}{UTF8}{mj}已知二次型\end{CJK}
$$
f\left(x_{1}, x_{2}, x_{3}\right)=t x_{1}^{2}+t x_{2}^{2}+t x_{3}^{2}-4 x_{1} x_{2}-4 x_{1} x_{3}+4 x_{2} x_{3} .
$$
(1) \begin{CJK}{UTF8}{mj}若该二次型正定\end{CJK}, \begin{CJK}{UTF8}{mj}求\end{CJK} $t$ \begin{CJK}{UTF8}{mj}的取值范围\end{CJK}.

(2) \begin{CJK}{UTF8}{mj}若该二次型负定\end{CJK}, \begin{CJK}{UTF8}{mj}求\end{CJK} $t$ \begin{CJK}{UTF8}{mj}的取值范围\end{CJK}.

\begin{CJK}{UTF8}{mj}五\end{CJK}. ( 15 \begin{CJK}{UTF8}{mj}分\end{CJK}) \begin{CJK}{UTF8}{mj}设\end{CJK} $V_{1}, V_{2}$ \begin{CJK}{UTF8}{mj}是线性空间\end{CJK} $V$ \begin{CJK}{UTF8}{mj}的两个子空间\end{CJK}. \begin{CJK}{UTF8}{mj}证明\end{CJK}: $V_{1}+V_{2}=V_{1} \bigcup V_{2}$ \begin{CJK}{UTF8}{mj}的充分必要条件是\end{CJK}: $V_{1} \subset V_{2}$ \begin{CJK}{UTF8}{mj}或\end{CJK} $V_{2} \subset V_{1}$.

\begin{CJK}{UTF8}{mj}六\end{CJK}. (30 \begin{CJK}{UTF8}{mj}分\end{CJK}) \begin{CJK}{UTF8}{mj}已知矩阵\end{CJK}
$$
A=\left(\begin{array}{ccc}
2 & 0 & 0 \\
1 & 2 & -1 \\
1 & 0 & 1
\end{array}\right)
$$
(1) \begin{CJK}{UTF8}{mj}求\end{CJK} $A$ \begin{CJK}{UTF8}{mj}的特征根\end{CJK}.

(2) \begin{CJK}{UTF8}{mj}求可逆矩阵\end{CJK} $P$, \begin{CJK}{UTF8}{mj}使得\end{CJK} $P^{-1} A P$ \begin{CJK}{UTF8}{mj}为对角阵\end{CJK}.

(3) \begin{CJK}{UTF8}{mj}计算\end{CJK} $A^{n}, n$ \begin{CJK}{UTF8}{mj}为正整数\end{CJK}. \begin{CJK}{UTF8}{mj}七\end{CJK}. ( 15 \begin{CJK}{UTF8}{mj}分\end{CJK}) \begin{CJK}{UTF8}{mj}设\end{CJK} $\alpha_{1}, \alpha_{2}, \cdots, \alpha_{n}$ \begin{CJK}{UTF8}{mj}为\end{CJK} $n$ \begin{CJK}{UTF8}{mj}维欧式空间\end{CJK} $V$ \begin{CJK}{UTF8}{mj}的一组基\end{CJK}, $\left(\alpha_{i}, \alpha_{j}\right)$ \begin{CJK}{UTF8}{mj}表示\end{CJK} $\alpha_{i}, \alpha_{j}$ \begin{CJK}{UTF8}{mj}的内积\end{CJK}. \begin{CJK}{UTF8}{mj}证明\end{CJK}: \begin{CJK}{UTF8}{mj}这组基是标准\end{CJK} \begin{CJK}{UTF8}{mj}正交基的充分必要条件是\end{CJK}: \begin{CJK}{UTF8}{mj}对\end{CJK} $V$ \begin{CJK}{UTF8}{mj}中任意向量\end{CJK} $\alpha$ \begin{CJK}{UTF8}{mj}都有\end{CJK}
$$
\alpha=\left(\alpha, \alpha_{1}\right) \alpha_{1}+\left(\alpha, \alpha_{2}\right) \alpha_{2}+\cdots+\left(\alpha, \alpha_{n}\right) \alpha_{n}
$$
\begin{CJK}{UTF8}{mj}八\end{CJK}. ( 10 \begin{CJK}{UTF8}{mj}分\end{CJK}) \begin{CJK}{UTF8}{mj}设\end{CJK} $\mathscr{A}$ \begin{CJK}{UTF8}{mj}是\end{CJK} $n$ \begin{CJK}{UTF8}{mj}维线性空间\end{CJK} $V$ \begin{CJK}{UTF8}{mj}的一个线性变换\end{CJK}, $V_{1}$ \begin{CJK}{UTF8}{mj}是\end{CJK} $V$ \begin{CJK}{UTF8}{mj}的一个子空间\end{CJK}. \begin{CJK}{UTF8}{mj}设\end{CJK} $\mathscr{A}$ \begin{CJK}{UTF8}{mj}的秩为\end{CJK} $r, V_{1}$ \begin{CJK}{UTF8}{mj}的维数为\end{CJK} $s$, $\mathscr{A} V_{1}$ \begin{CJK}{UTF8}{mj}的维数为\end{CJK} $t$. \begin{CJK}{UTF8}{mj}证明\end{CJK}:
$$
t \geqslant r+s-n .
$$

\section{9. 中国海洋大学 2009 年研究生入学考试试题数学分析}
\begin{CJK}{UTF8}{mj}李扬\end{CJK}

\begin{CJK}{UTF8}{mj}微信公众号\end{CJK}: sxkyliyang

\begin{CJK}{UTF8}{mj}一\end{CJK}. ( 30 \begin{CJK}{UTF8}{mj}分\end{CJK}) \begin{CJK}{UTF8}{mj}求极限\end{CJK} (\begin{CJK}{UTF8}{mj}要求有计算过程\end{CJK})

(1)
$$
\lim _{x \rightarrow 0}\left(\frac{1}{x}-\frac{1}{e^{x}-1}\right)
$$
(2)
$$
\lim _{n \rightarrow \infty} \frac{\sqrt[n]{n !}}{n}
$$
$(3)$
$$
\lim _{n \rightarrow \infty} \int_{0}^{1} \frac{d x}{1+\left(1+\frac{x}{n}\right)^{n}}
$$
(4)
$$
\lim _{x \rightarrow 0}(\cos x)^{\frac{1}{\ln \left(1+x^{2}\right)}} .
$$
$(5)$
$$
\lim _{n \rightarrow \infty} \int_{0}^{\frac{\pi}{2}} \sin ^{n} x \mathrm{~d} x
$$
\begin{CJK}{UTF8}{mj}二\end{CJK}. ( 15 \begin{CJK}{UTF8}{mj}分\end{CJK}) \begin{CJK}{UTF8}{mj}叙述与举例\end{CJK} (\begin{CJK}{UTF8}{mj}要求有讨论过程\end{CJK})

(1) \begin{CJK}{UTF8}{mj}用肯定语气写出\end{CJK} $f(x)$ \begin{CJK}{UTF8}{mj}在\end{CJK} $[a, b]$ \begin{CJK}{UTF8}{mj}上不一致连续的充要条件\end{CJK}.

(2) \begin{CJK}{UTF8}{mj}举出函数\end{CJK} $f(x)$ \begin{CJK}{UTF8}{mj}满足条件\end{CJK}: \begin{CJK}{UTF8}{mj}在\end{CJK} $x=0$ \begin{CJK}{UTF8}{mj}时可导\end{CJK}, \begin{CJK}{UTF8}{mj}但在\end{CJK} 0 \begin{CJK}{UTF8}{mj}的任何邻域内不可导\end{CJK}.

(3) \begin{CJK}{UTF8}{mj}举出函数\end{CJK} $f(x, y)$ \begin{CJK}{UTF8}{mj}满足条件\end{CJK}: \begin{CJK}{UTF8}{mj}在\end{CJK} $(0,0)$ \begin{CJK}{UTF8}{mj}处连续\end{CJK}, \begin{CJK}{UTF8}{mj}两个偏导数存在\end{CJK} (\begin{CJK}{UTF8}{mj}并求出\end{CJK}), \begin{CJK}{UTF8}{mj}但在\end{CJK} $(0,0)$ \begin{CJK}{UTF8}{mj}不可微\end{CJK}.

\begin{CJK}{UTF8}{mj}三\end{CJK}. ( 24 \begin{CJK}{UTF8}{mj}分\end{CJK}) \begin{CJK}{UTF8}{mj}证明不等式\end{CJK}

(1) \begin{CJK}{UTF8}{mj}当\end{CJK} $0<x_{1}<x_{2}<\frac{\pi}{2}$ \begin{CJK}{UTF8}{mj}时\end{CJK}, \begin{CJK}{UTF8}{mj}有\end{CJK}
$$
x_{1} \sin x_{2}<x_{2} \sin x_{1} .
$$
(2) \begin{CJK}{UTF8}{mj}设\end{CJK} $f$ \begin{CJK}{UTF8}{mj}在\end{CJK} $[a, b]$ \begin{CJK}{UTF8}{mj}上有连续导函数\end{CJK}, $f(a)=f(b)=0, \int_{a}^{b} f^{2}(x) \mathrm{d} x=1$, \begin{CJK}{UTF8}{mj}则\end{CJK}
$$
\left(\int_{a}^{b} f^{\prime 2}(x) \mathrm{d} x\right)\left(\int_{a}^{b} x^{2} f^{2}(x) \mathrm{d} x\right) \geqslant \frac{1}{4} .
$$
\begin{CJK}{UTF8}{mj}四\end{CJK}. ( 10 \begin{CJK}{UTF8}{mj}分\end{CJK}) \begin{CJK}{UTF8}{mj}判断下面正项级数的敛散性\end{CJK}:
$$
\sum_{n=1}^{\infty}\left(e^{\frac{1}{n^{2}}}-\cos \frac{1}{n}\right)
$$
\begin{CJK}{UTF8}{mj}五\end{CJK}. ( 28 \begin{CJK}{UTF8}{mj}分\end{CJK}) \begin{CJK}{UTF8}{mj}求积分\end{CJK}

(1)
$$
\int_{-\infty}^{+\infty} e^{-x^{2}} \mathrm{~d} x
$$
(2)
$$
\int_{L}(y-z) \mathrm{d} x+(z-x) \mathrm{d} y+(x-y) \mathrm{d} z
$$
\begin{CJK}{UTF8}{mj}其中\end{CJK} $L$ \begin{CJK}{UTF8}{mj}为圆柱面\end{CJK} $x^{2}+y^{2}=a^{2}$ \begin{CJK}{UTF8}{mj}和平面\end{CJK} $\frac{x}{a}+\frac{z}{h}=1(a>0, h>0)$ \begin{CJK}{UTF8}{mj}的交线\end{CJK}, \begin{CJK}{UTF8}{mj}从\end{CJK} $x$ \begin{CJK}{UTF8}{mj}轴的正向看去\end{CJK}, \begin{CJK}{UTF8}{mj}是逆时针\end{CJK} \begin{CJK}{UTF8}{mj}方向\end{CJK}. \begin{CJK}{UTF8}{mj}六\end{CJK}. (15 \begin{CJK}{UTF8}{mj}分\end{CJK}) \begin{CJK}{UTF8}{mj}讨论下列积分在指定区间的一致收敛性\end{CJK}:
$$
\int_{0}^{1} x^{-\alpha} \sin \frac{1}{x} \mathrm{~d} x, \alpha \in(0,2) .
$$
\begin{CJK}{UTF8}{mj}七\end{CJK}. (13 \begin{CJK}{UTF8}{mj}分\end{CJK}) \begin{CJK}{UTF8}{mj}设\end{CJK} $f(x)$ \begin{CJK}{UTF8}{mj}在\end{CJK} $(-\infty,+\infty)$ \begin{CJK}{UTF8}{mj}上可导\end{CJK}, \begin{CJK}{UTF8}{mj}且\end{CJK} $\lim _{x \rightarrow \infty} f(x)=a$, \begin{CJK}{UTF8}{mj}求证\end{CJK}: $f^{\prime}(x)$ \begin{CJK}{UTF8}{mj}至少存在一个零点\end{CJK}.

\begin{CJK}{UTF8}{mj}八\end{CJK}. ( 15 \begin{CJK}{UTF8}{mj}分\end{CJK} $)$ \begin{CJK}{UTF8}{mj}考察函数项级数\end{CJK}
$$
\sum_{n=1}^{\infty} \frac{n x}{1+n^{4} x^{2}}
$$
(1) \begin{CJK}{UTF8}{mj}讨论该函数项级数在\end{CJK} $(-\infty,+\infty)$ \begin{CJK}{UTF8}{mj}是否一致收敛\end{CJK}.

(2) \begin{CJK}{UTF8}{mj}求出该函数项级数的一致收敛区间\end{CJK}.

(3) \begin{CJK}{UTF8}{mj}求出该函数项级数的和函数的连续区间\end{CJK}.

\section{0. 中国海洋大学 2010 年研究生入学考试试题数学分析}
\begin{CJK}{UTF8}{mj}李扬\end{CJK}

\begin{CJK}{UTF8}{mj}微信公众号\end{CJK}: sxkyliyang

\begin{CJK}{UTF8}{mj}一\end{CJK}. \begin{CJK}{UTF8}{mj}计算题\end{CJK} (\begin{CJK}{UTF8}{mj}要有计算过程\end{CJK})

(1)
$$
\lim _{x \rightarrow 0} \frac{\cos (\sin x)-\cos x}{x^{4}}
$$
(2)
$$
I=\int_{1}^{+\infty} \frac{1}{e^{1+x}+e^{3-x}} \mathrm{~d} x
$$
\begin{CJK}{UTF8}{mj}其中\end{CJK} $e$ \begin{CJK}{UTF8}{mj}是自然对数的底\end{CJK}.

(3) \begin{CJK}{UTF8}{mj}设\end{CJK} $z=z(x, y)$ \begin{CJK}{UTF8}{mj}是由方程\end{CJK}
$$
F\left(x y z, x^{2}+y^{2}+z^{2}\right)=0
$$
\begin{CJK}{UTF8}{mj}所确定的可微的隐函数\end{CJK}, \begin{CJK}{UTF8}{mj}其中\end{CJK} $F$ \begin{CJK}{UTF8}{mj}具有连续的偏导数\end{CJK}, \begin{CJK}{UTF8}{mj}求\end{CJK} $\operatorname{grad} z$ ( $\operatorname{grad} z$ \begin{CJK}{UTF8}{mj}为\end{CJK} $z$ \begin{CJK}{UTF8}{mj}的梯度向量\end{CJK}).

\begin{CJK}{UTF8}{mj}二\end{CJK}. \begin{CJK}{UTF8}{mj}判断题\end{CJK} (\begin{CJK}{UTF8}{mj}正确的给予证明\end{CJK}, \begin{CJK}{UTF8}{mj}错误的举出反例\end{CJK})

(1) \begin{CJK}{UTF8}{mj}若函数\end{CJK} $f(x)$ \begin{CJK}{UTF8}{mj}在\end{CJK} $(a, b)$ \begin{CJK}{UTF8}{mj}上可微\end{CJK}, \begin{CJK}{UTF8}{mj}且\end{CJK} $\lim _{x \rightarrow a^{+}} f(x)=\infty$, \begin{CJK}{UTF8}{mj}则有\end{CJK} $\lim _{x \rightarrow a^{+}} f^{\prime}(x)=\infty$. ( )

(2) \begin{CJK}{UTF8}{mj}若函数\end{CJK} $f(x)$ \begin{CJK}{UTF8}{mj}在\end{CJK} $[a, b]$ \begin{CJK}{UTF8}{mj}上可积\end{CJK}, \begin{CJK}{UTF8}{mj}则\end{CJK} $f(x)$ \begin{CJK}{UTF8}{mj}在\end{CJK} $[a, b]$ \begin{CJK}{UTF8}{mj}上一定存在原函数\end{CJK}. ( )

(3) \begin{CJK}{UTF8}{mj}对于任意的两个数列\end{CJK} $\left\{a_{n}\right\}$ \begin{CJK}{UTF8}{mj}和\end{CJK} $\left\{b_{n}\right\}$, \begin{CJK}{UTF8}{mj}有\end{CJK} $\sup \left\{a_{n}+b_{n}\right\} \leqslant \sup \left\{a_{n}\right\}+\sup \left\{b_{n}\right\}$. ( )

\begin{CJK}{UTF8}{mj}三\end{CJK}. \begin{CJK}{UTF8}{mj}设\end{CJK}
$$
I_{n}=\int_{1}^{1+\frac{1}{n}} \sqrt{1+x^{n}} \mathrm{~d} x
$$
(1) \begin{CJK}{UTF8}{mj}证明\end{CJK}: $\lim _{n \rightarrow \infty} I_{n}=0$.

(2) \begin{CJK}{UTF8}{mj}证明极限\end{CJK} $\lim _{n \rightarrow \infty} n I_{n}$ \begin{CJK}{UTF8}{mj}存在\end{CJK}, \begin{CJK}{UTF8}{mj}并求出此极限值\end{CJK}.

\begin{CJK}{UTF8}{mj}四\end{CJK}. \begin{CJK}{UTF8}{mj}证明\end{CJK}:

(1) \begin{CJK}{UTF8}{mj}当\end{CJK} $0 \leqslant x<1$ \begin{CJK}{UTF8}{mj}时\end{CJK}, \begin{CJK}{UTF8}{mj}有\end{CJK}
$$
\sum_{n=0}^{\infty} \int_{0}^{x} u^{n} \sin \pi u \mathrm{~d} u=\int_{0}^{x} \frac{\sin \pi u}{1-u} \mathrm{~d} u
$$
(2) \begin{CJK}{UTF8}{mj}级数\end{CJK}
$$
\sum_{n=0}^{\infty} \int_{0}^{x} u^{n} \sin \pi u \mathrm{~d} u
$$
\begin{CJK}{UTF8}{mj}在\end{CJK} $[0,1]$ \begin{CJK}{UTF8}{mj}上一致收敛\end{CJK}.

(3)
$$
\sum_{n=0}^{\infty} \int_{0}^{1} u^{n} \sin \pi u \mathrm{~d} u=\int_{0}^{1} \frac{\sin \pi u}{1-u} \mathrm{~d} u
$$
\begin{CJK}{UTF8}{mj}五\end{CJK}. (1) \begin{CJK}{UTF8}{mj}将函数\end{CJK} $f(x)=\ln \left(1+x+x^{2}\right)$ \begin{CJK}{UTF8}{mj}在点\end{CJK} $x=0$ \begin{CJK}{UTF8}{mj}处展开为幂级数\end{CJK}.

(2) \begin{CJK}{UTF8}{mj}计算\end{CJK} $f^{(n)}(0)$.

\begin{CJK}{UTF8}{mj}六\end{CJK}. \begin{CJK}{UTF8}{mj}在过点\end{CJK} $O(0,0)$ \begin{CJK}{UTF8}{mj}和点\end{CJK} $A(\pi, 0)$ \begin{CJK}{UTF8}{mj}的曲线族\end{CJK} $y=a \sin x(a>0)$ \begin{CJK}{UTF8}{mj}中\end{CJK}, \begin{CJK}{UTF8}{mj}求一条曲线\end{CJK} $L$, \begin{CJK}{UTF8}{mj}使得沿该曲线从点\end{CJK} $O$ \begin{CJK}{UTF8}{mj}到\end{CJK} $A$ \begin{CJK}{UTF8}{mj}的积\end{CJK} \begin{CJK}{UTF8}{mj}分\end{CJK}
$$
\int_{L}\left(1+y^{3}\right) \mathrm{d} x+(2 x+y) \mathrm{d} y
$$
\begin{CJK}{UTF8}{mj}的值最小\end{CJK}. \begin{CJK}{UTF8}{mj}七\end{CJK}. \begin{CJK}{UTF8}{mj}设\end{CJK} $f(x)$ \begin{CJK}{UTF8}{mj}在\end{CJK} $[a, b]$ \begin{CJK}{UTF8}{mj}上有定义\end{CJK}, $x_{0} \in[a, b]$. \begin{CJK}{UTF8}{mj}若对任意的\end{CJK} $\varepsilon>0$, \begin{CJK}{UTF8}{mj}存在\end{CJK} $\delta>0$, \begin{CJK}{UTF8}{mj}当\end{CJK} $\left|x-x_{0}\right|<\delta$ \begin{CJK}{UTF8}{mj}时\end{CJK}, \begin{CJK}{UTF8}{mj}有\end{CJK} $f(x)<f\left(x_{0}\right)+\varepsilon$, \begin{CJK}{UTF8}{mj}则称\end{CJK} $f(x)$ \begin{CJK}{UTF8}{mj}在点\end{CJK} $x_{0}$ \begin{CJK}{UTF8}{mj}处上半连续\end{CJK}. \begin{CJK}{UTF8}{mj}若\end{CJK} $f(x)$ \begin{CJK}{UTF8}{mj}在\end{CJK} $[a, b]$ \begin{CJK}{UTF8}{mj}上的每一点都上半连续\end{CJK}, \begin{CJK}{UTF8}{mj}则称\end{CJK} $f(x)$ \begin{CJK}{UTF8}{mj}为\end{CJK} $[a, b]$ \begin{CJK}{UTF8}{mj}上的一个上半连\end{CJK} \begin{CJK}{UTF8}{mj}续函数\end{CJK}. \begin{CJK}{UTF8}{mj}证明\end{CJK}: $[a, b]$ \begin{CJK}{UTF8}{mj}上的上半连续函数一定有界\end{CJK}.

\begin{CJK}{UTF8}{mj}八\end{CJK}. \begin{CJK}{UTF8}{mj}求椭球面\end{CJK}
$$
\frac{x^{2}}{a^{2}}+\frac{y^{2}}{b^{2}}+\frac{z^{2}}{c^{2}}=1
$$
\begin{CJK}{UTF8}{mj}与锥面\end{CJK}
$$
\frac{x^{2}}{a^{2}}+\frac{y^{2}}{b^{2}}-\frac{z^{2}}{c^{2}}=0(z>0)
$$
\begin{CJK}{UTF8}{mj}所围立体的体积\end{CJK}.

\begin{CJK}{UTF8}{mj}九\end{CJK}. \begin{CJK}{UTF8}{mj}设\end{CJK} $f(x, y)$ \begin{CJK}{UTF8}{mj}在\end{CJK} $x^{2}+y^{2}<1$ \begin{CJK}{UTF8}{mj}上二次连续可微\end{CJK}, \begin{CJK}{UTF8}{mj}且满足\end{CJK} $\frac{\partial^{2} f}{\partial x^{2}}+\frac{\partial^{2} f}{\partial y^{2}}=e^{-\left(x^{2}+y^{2}\right)}$, \begin{CJK}{UTF8}{mj}证明\end{CJK}:
$$
\iint_{x^{2}+y^{2} \leqslant 1}\left(x \frac{\partial f}{\partial x}+y \frac{\partial f}{\partial y}\right) \mathrm{d} x \mathrm{~d} y=\frac{\pi}{2 e} .
$$

\section{1. 中国海洋大学 2011 年研究生入学考试试题数学分析}
\begin{CJK}{UTF8}{mj}李扬\end{CJK}

\begin{CJK}{UTF8}{mj}微信公众号\end{CJK}: sxkyliyang

\begin{CJK}{UTF8}{mj}一\end{CJK}. (24 \begin{CJK}{UTF8}{mj}分\end{CJK}) \begin{CJK}{UTF8}{mj}计算题\end{CJK} (\begin{CJK}{UTF8}{mj}要求有计算过程\end{CJK})

(1)
$$
\lim _{x \rightarrow 0} \frac{x^{2}}{\sqrt{1+x \sin x}-\sqrt{\cos x}}
$$
(2) \begin{CJK}{UTF8}{mj}设\end{CJK}
$$
f(\ln t)=\frac{\ln (1+t)}{t}
$$
\begin{CJK}{UTF8}{mj}计算\end{CJK} $\int f(t) \mathrm{d} t$.

(3) \begin{CJK}{UTF8}{mj}求数项级数\end{CJK}
$$
\sum_{n=1}^{\infty} \frac{n}{(n+1) !}
$$
\begin{CJK}{UTF8}{mj}的和\end{CJK}.

\begin{CJK}{UTF8}{mj}二\end{CJK}. (18 \begin{CJK}{UTF8}{mj}分\end{CJK}) \begin{CJK}{UTF8}{mj}判断题\end{CJK} (\begin{CJK}{UTF8}{mj}正确的给予证明\end{CJK}, \begin{CJK}{UTF8}{mj}错误的举出反例\end{CJK})

(1) \begin{CJK}{UTF8}{mj}若函数\end{CJK} $f(x)$ \begin{CJK}{UTF8}{mj}在点\end{CJK} $x_{0}$ \begin{CJK}{UTF8}{mj}可微\end{CJK}, \begin{CJK}{UTF8}{mj}则\end{CJK} $f(x)$ \begin{CJK}{UTF8}{mj}在点\end{CJK} $x_{0}$ \begin{CJK}{UTF8}{mj}的某个邻域内连续\end{CJK}. ( )

(2) \begin{CJK}{UTF8}{mj}对于收玫的交错级数\end{CJK} $\sum_{n=1}^{\infty}(-1)^{n} u_{n}\left(u_{n}>0\right)$, \begin{CJK}{UTF8}{mj}且数列\end{CJK} $\left\{u_{n}\right\}$ \begin{CJK}{UTF8}{mj}单调减少\end{CJK}, \begin{CJK}{UTF8}{mj}则级数\end{CJK} $\sum_{n=1}^{\infty} \frac{1}{\left(1+u_{n}\right)^{n}}$ \begin{CJK}{UTF8}{mj}必收敛\end{CJK}. ( )

(3) \begin{CJK}{UTF8}{mj}若在\end{CJK} $[a, b]$ \begin{CJK}{UTF8}{mj}上有\end{CJK} $F^{\prime}(x)=f(x)$, \begin{CJK}{UTF8}{mj}则函数\end{CJK} $f(x)$ \begin{CJK}{UTF8}{mj}在\end{CJK} $[a, b]$ \begin{CJK}{UTF8}{mj}上可积\end{CJK}. ( )

\begin{CJK}{UTF8}{mj}三\end{CJK}. (12 \begin{CJK}{UTF8}{mj}分\end{CJK}) \begin{CJK}{UTF8}{mj}设\end{CJK} $f(x)$ \begin{CJK}{UTF8}{mj}在\end{CJK} $[a, b]$ \begin{CJK}{UTF8}{mj}上非负连续\end{CJK}, \begin{CJK}{UTF8}{mj}证明\end{CJK}:
$$
\lim _{n \rightarrow \infty}\left(\int_{a}^{b} f^{n}(x) \mathrm{d} x\right)^{\frac{1}{n}}=\max \{f(x) \mid x \in[a, b]\} .
$$
\begin{CJK}{UTF8}{mj}四\end{CJK}. \begin{CJK}{UTF8}{mj}讨论如下含参量广义积分的一致收敛区间\end{CJK}:
$$
\int_{1}^{+\infty} \frac{1}{t^{2-\alpha}} \sin t \mathrm{~d} t
$$
(1) \begin{CJK}{UTF8}{mj}对任意的\end{CJK} $\delta>0$, \begin{CJK}{UTF8}{mj}证明它在\end{CJK} $(-\infty, 2-\delta]$ \begin{CJK}{UTF8}{mj}上一致收敛\end{CJK}.

(2) \begin{CJK}{UTF8}{mj}对任意的\end{CJK} $\eta<2$, \begin{CJK}{UTF8}{mj}证明它在\end{CJK} $[\eta, 2)$ \begin{CJK}{UTF8}{mj}上非一致收敛\end{CJK}.

\begin{CJK}{UTF8}{mj}五\end{CJK}. (16 \begin{CJK}{UTF8}{mj}分\end{CJK}) \begin{CJK}{UTF8}{mj}设\end{CJK} $f(x)$ \begin{CJK}{UTF8}{mj}为定义在有限区间\end{CJK} $I$ \begin{CJK}{UTF8}{mj}上的函数\end{CJK}, \begin{CJK}{UTF8}{mj}对\end{CJK} $I$ \begin{CJK}{UTF8}{mj}内的任何柯西列\end{CJK} $\left\{x_{n}\right\},\left\{f\left(x_{n}\right)\right\}$ \begin{CJK}{UTF8}{mj}也是柯西列\end{CJK}. (\begin{CJK}{UTF8}{mj}柯西列\end{CJK} $\left\{x_{n}\right\}$ \begin{CJK}{UTF8}{mj}是指\end{CJK}: $\forall \varepsilon>0, \exists N \in N^{+}, \forall n, m>N$, \begin{CJK}{UTF8}{mj}有\end{CJK} $\left|x_{n}-x_{m}\right|<\varepsilon$ )

(1) \begin{CJK}{UTF8}{mj}证明\end{CJK}: $f(x)$ \begin{CJK}{UTF8}{mj}是\end{CJK} $I$ \begin{CJK}{UTF8}{mj}上的一致连续函数\end{CJK}.

(2) \begin{CJK}{UTF8}{mj}问题设的条件是否是\end{CJK} $f(x)$ \begin{CJK}{UTF8}{mj}在\end{CJK} $I$ \begin{CJK}{UTF8}{mj}上一致连续的充要条件\end{CJK}.

\begin{CJK}{UTF8}{mj}六\end{CJK}. ( 12 \begin{CJK}{UTF8}{mj}分\end{CJK}) \begin{CJK}{UTF8}{mj}设整数\end{CJK} $n>1$, \begin{CJK}{UTF8}{mj}证明不等式\end{CJK}:
$$
\frac{1}{2 n e}<\frac{1}{e}-\left(1-\frac{1}{n}\right)^{n}<\frac{1}{n e}
$$
\begin{CJK}{UTF8}{mj}七\end{CJK}. (18 \begin{CJK}{UTF8}{mj}分\end{CJK}) \begin{CJK}{UTF8}{mj}设\end{CJK}
$$
f(t)=\int_{0}^{2 \pi} e^{t \cos \theta} \cos (\sin \theta) \mathrm{d} \theta
$$
(1) \begin{CJK}{UTF8}{mj}求\end{CJK} $f^{\prime}(t), f^{\prime \prime}(t), \cdots, f^{(n)}(t)$ ( $n$ \begin{CJK}{UTF8}{mj}为正整数\end{CJK}).

(2) \begin{CJK}{UTF8}{mj}写出\end{CJK} $f(t)$ \begin{CJK}{UTF8}{mj}的麦克劳林公式\end{CJK}.

(3) \begin{CJK}{UTF8}{mj}证明\end{CJK}: $f(t)=2 \pi$.

\begin{CJK}{UTF8}{mj}八\end{CJK}. (15 \begin{CJK}{UTF8}{mj}分\end{CJK}) \begin{CJK}{UTF8}{mj}计算\end{CJK}
$$
I=\iint_{\Sigma} 4 x z \mathrm{~d} y \mathrm{~d} z-2 y z \mathrm{~d} z \mathrm{~d} x+\left(1-z^{2}\right) \mathrm{d} x \mathrm{~d} y .
$$
\begin{CJK}{UTF8}{mj}其中\end{CJK} $\Sigma$ \begin{CJK}{UTF8}{mj}是曲线\end{CJK} $z=e^{y}(0 \leqslant y \leqslant a)$ \begin{CJK}{UTF8}{mj}绕\end{CJK} $z$ \begin{CJK}{UTF8}{mj}轴旋转生成的旋转面\end{CJK}, \begin{CJK}{UTF8}{mj}取下侧\end{CJK}.

\begin{CJK}{UTF8}{mj}九\end{CJK}. ( 15 \begin{CJK}{UTF8}{mj}分\end{CJK}) \begin{CJK}{UTF8}{mj}柇球面\end{CJK}
$$
\frac{x^{2}}{3}+\frac{y^{2}}{2}+z^{2}=1
$$
\begin{CJK}{UTF8}{mj}被通过原点的平面\end{CJK} $2 x+y+z=0$ \begin{CJK}{UTF8}{mj}截成一个椭圆\end{CJK} $l$, \begin{CJK}{UTF8}{mj}求此椭圆的面积\end{CJK}.

\section{2. 中国海洋大学 2012 年研究生入学考试试题数学分析}
\begin{CJK}{UTF8}{mj}李扬\end{CJK}

\begin{CJK}{UTF8}{mj}微信公众号\end{CJK}: sxkyliyang

\begin{CJK}{UTF8}{mj}一\end{CJK}. (40 \begin{CJK}{UTF8}{mj}分\end{CJK}) \begin{CJK}{UTF8}{mj}计算\end{CJK} (\begin{CJK}{UTF8}{mj}要求有计算过程\end{CJK})

(1)
$$
\lim _{n \rightarrow \infty}\left(\frac{\sqrt[n]{a}+\sqrt[n]{b}+\sqrt[n]{c}}{3}\right)^{n}, a>b>c>0 .
$$
(2) \begin{CJK}{UTF8}{mj}设\end{CJK} $y=f(x)$ \begin{CJK}{UTF8}{mj}的三阶导数存在\end{CJK}, $f^{\prime}(x) \neq 0$, \begin{CJK}{UTF8}{mj}用\end{CJK} $f(x)$ \begin{CJK}{UTF8}{mj}的各阶导数表示其反函数\end{CJK} $x=\varphi(y)$ \begin{CJK}{UTF8}{mj}的三阶导数\end{CJK} $\varphi^{\prime \prime \prime}(y)$.

(3) \begin{CJK}{UTF8}{mj}求定积分\end{CJK}
$$
\int_{-1}^{1} f^{(n)}(x) p_{n-1}(x) \mathrm{d} x
$$
\begin{CJK}{UTF8}{mj}其中\end{CJK} $n \in Z^{+}, f(x)=\left(x^{2}-1\right)^{n}, p_{n-1}(x)=a_{0}+a_{1} x+\cdots+a_{n-1} x^{n-1}, a_{i} \in \mathbb{R}, i=1,2, \cdots, n-1$.

(4)
$$
\iint_{\Sigma} \frac{\vec{r} \mathrm{~d} \vec{S}}{|\vec{r}|^{3}}
$$
\begin{CJK}{UTF8}{mj}其中\end{CJK} $\Sigma: x^{2}+y^{2}+(z-R)^{2}=R^{2}$ \begin{CJK}{UTF8}{mj}的外侧\end{CJK}, $\vec{r}=(x, y, z)$.

(5) \begin{CJK}{UTF8}{mj}设\end{CJK}
$$
f(x)= \begin{cases}e^{-\frac{1}{x^{2}}}, & x \neq 0 \\ 0, & x=0\end{cases}
$$
\begin{CJK}{UTF8}{mj}求\end{CJK} $f^{(n)}(0)$.

\begin{CJK}{UTF8}{mj}二\end{CJK}. (40 \begin{CJK}{UTF8}{mj}分\end{CJK}) \begin{CJK}{UTF8}{mj}判断题\end{CJK} (\begin{CJK}{UTF8}{mj}正确的给予证明\end{CJK}, \begin{CJK}{UTF8}{mj}错误的加以说明\end{CJK})

(1) \begin{CJK}{UTF8}{mj}若交错级数\end{CJK} $\sum_{n=1}^{\infty}(-1)^{n} a_{n}, a_{n}>0$ \begin{CJK}{UTF8}{mj}满足\end{CJK} $\lim _{n \rightarrow \infty} a_{n}=0$, \begin{CJK}{UTF8}{mj}则级数收玫\end{CJK}. ( )

(2) \begin{CJK}{UTF8}{mj}若含参变量广义积分\end{CJK} $I(y)=\int_{0}^{+\infty} f(x, y) \mathrm{d} x$ \begin{CJK}{UTF8}{mj}在\end{CJK} $(a, b)$ \begin{CJK}{UTF8}{mj}的任意闭区间上一致收敛\end{CJK}, \begin{CJK}{UTF8}{mj}则有\end{CJK} $I(y)$ \begin{CJK}{UTF8}{mj}在\end{CJK} $(a, b)$ \begin{CJK}{UTF8}{mj}上连\end{CJK} \begin{CJK}{UTF8}{mj}续\end{CJK}. ( )

(3) \begin{CJK}{UTF8}{mj}如果\end{CJK} $f^{\prime}\left(x_{0}+0\right)$ \begin{CJK}{UTF8}{mj}存在\end{CJK}, \begin{CJK}{UTF8}{mj}则\end{CJK} $f_{+}^{\prime}\left(x_{0}\right)$ \begin{CJK}{UTF8}{mj}存在\end{CJK}, \begin{CJK}{UTF8}{mj}且二者相等\end{CJK}. ( )

(4) $f(x)$ \begin{CJK}{UTF8}{mj}在\end{CJK} $(1,+\infty)$ \begin{CJK}{UTF8}{mj}上二阶可导\end{CJK}, \begin{CJK}{UTF8}{mj}且\end{CJK} $\lim _{x \rightarrow+\infty} f(x)=\lim _{x \rightarrow+\infty} f^{\prime \prime}(x)=0$, \begin{CJK}{UTF8}{mj}则\end{CJK} $\lim _{x \rightarrow+\infty} f^{\prime}(x)=0$. ( )

\begin{CJK}{UTF8}{mj}三\end{CJK}. \begin{CJK}{UTF8}{mj}证明题\end{CJK}

(1) (15 \begin{CJK}{UTF8}{mj}分\end{CJK}) \begin{CJK}{UTF8}{mj}设\end{CJK} $f(x)$ \begin{CJK}{UTF8}{mj}在\end{CJK} $[0,2]$ \begin{CJK}{UTF8}{mj}上具有二阶导数\end{CJK}, \begin{CJK}{UTF8}{mj}且在\end{CJK} $[0,2]$ \begin{CJK}{UTF8}{mj}上有\end{CJK} $|f(x)| \leqslant 1,\left|f^{\prime \prime}(x)\right| \leqslant 1$, \begin{CJK}{UTF8}{mj}求证\end{CJK}: \begin{CJK}{UTF8}{mj}在\end{CJK} $[0,2]$ \begin{CJK}{UTF8}{mj}上\end{CJK}, $\left|f^{\prime}(x)\right| \leqslant 2$.

(2) ( 20 \begin{CJK}{UTF8}{mj}分\end{CJK}) \begin{CJK}{UTF8}{mj}若\end{CJK} $f(x)$ \begin{CJK}{UTF8}{mj}在\end{CJK} $[a,+\infty)$ \begin{CJK}{UTF8}{mj}上一致连续\end{CJK}, $\int_{a}^{+\infty} f(x) \mathrm{d} x$ \begin{CJK}{UTF8}{mj}收敛\end{CJK}, \begin{CJK}{UTF8}{mj}证明\end{CJK} $\lim _{x \rightarrow+\infty} f(x)=0$.

\begin{CJK}{UTF8}{mj}若\end{CJK} $f(x)$ \begin{CJK}{UTF8}{mj}在\end{CJK} $[a,+\infty)$ \begin{CJK}{UTF8}{mj}上连续\end{CJK}, $\int_{a}^{+\infty} f(x) \mathrm{d} x$ \begin{CJK}{UTF8}{mj}收敛\end{CJK}, \begin{CJK}{UTF8}{mj}是否有\end{CJK} $\lim _{x \rightarrow+\infty} f(x)=0$ ? \begin{CJK}{UTF8}{mj}说明理由\end{CJK}.

(3) (15 \begin{CJK}{UTF8}{mj}分\end{CJK}) \begin{CJK}{UTF8}{mj}对于\end{CJK} $n$ \begin{CJK}{UTF8}{mj}阶行列式\end{CJK} $A=\left|a_{i j}\right|$, \begin{CJK}{UTF8}{mj}用求条件极值的\end{CJK} Lagrange \begin{CJK}{UTF8}{mj}乘数法\end{CJK}, \begin{CJK}{UTF8}{mj}证明\end{CJK}: Hadamard \begin{CJK}{UTF8}{mj}不等式\end{CJK}:
$$
|A|^{2} \leqslant \prod_{i=1}^{n}\left(\sum_{j=1}^{n} a_{i j}^{2}\right) .
$$
(4) (20 \begin{CJK}{UTF8}{mj}分\end{CJK}) \begin{CJK}{UTF8}{mj}证明\end{CJK}: \begin{CJK}{UTF8}{mj}黎曼函数\end{CJK}
$$
R(x)= \begin{cases}\frac{1}{q}, & x \text { 为有理数, } x=\frac{p}{q},(p, q)=1, p>0 ; \\ 0, & x \text { 为无理数. }\end{cases}
$$
\begin{CJK}{UTF8}{mj}在任何有理点不连续\end{CJK}, \begin{CJK}{UTF8}{mj}在任何无理点连续\end{CJK}; \begin{CJK}{UTF8}{mj}并证明在任何有限闭区间\end{CJK} $[a, b]$ \begin{CJK}{UTF8}{mj}上黎曼函数均可积\end{CJK}. \begin{CJK}{UTF8}{mj}且积分值\end{CJK} \begin{CJK}{UTF8}{mj}为\end{CJK} 0 .

\section{3. 中国海洋大学 2013 年研究生入学考试试题数学分析}
\begin{CJK}{UTF8}{mj}李扬\end{CJK}

\begin{CJK}{UTF8}{mj}微信公众号\end{CJK}: sxkyliyang

\begin{CJK}{UTF8}{mj}一\end{CJK}. \begin{CJK}{UTF8}{mj}计算题\end{CJK} (\begin{CJK}{UTF8}{mj}要求有计算过程\end{CJK})

(1)
$$
\lim _{x \rightarrow 0} \frac{\cos (\sin x)-\cos x}{x^{4}}
$$
(2) \begin{CJK}{UTF8}{mj}设\end{CJK}
$$
f(x)=x+\ln x(x>0) .
$$
\begin{CJK}{UTF8}{mj}求\end{CJK} $f(x)$ \begin{CJK}{UTF8}{mj}的反函数\end{CJK} $x=\varphi(y)$ \begin{CJK}{UTF8}{mj}的一阶及二阶导数\end{CJK}.

(3)
$$
\int_{0}^{\pi} \frac{x \sin x}{1+\cos ^{2} x} \mathrm{~d} x .
$$
\begin{CJK}{UTF8}{mj}二\end{CJK}. \begin{CJK}{UTF8}{mj}判断题\end{CJK} (\begin{CJK}{UTF8}{mj}正确的给予证明\end{CJK}, \begin{CJK}{UTF8}{mj}错误的加以说明或举出反例\end{CJK})

(1) $\forall 0<x_{1}<x_{2}$, \begin{CJK}{UTF8}{mj}成立\end{CJK} $\frac{x_{2}-x_{1}}{x_{2}}<\ln \frac{x_{2}}{x_{1}}<\frac{x_{2}-x_{1}}{x_{1}}$.

(2) \begin{CJK}{UTF8}{mj}一元函数\end{CJK} $f(x)$ \begin{CJK}{UTF8}{mj}在\end{CJK} $x_{0}$ \begin{CJK}{UTF8}{mj}的导数\end{CJK} $f^{\prime}(x)$ \begin{CJK}{UTF8}{mj}的符号大于\end{CJK} 0 , \begin{CJK}{UTF8}{mj}则\end{CJK} $f(x)$ \begin{CJK}{UTF8}{mj}在\end{CJK} $x_{0}$ \begin{CJK}{UTF8}{mj}的一个充分小的邻域内单调\end{CJK}.

(3) \begin{CJK}{UTF8}{mj}二元函数\end{CJK} $f(x, y)$ \begin{CJK}{UTF8}{mj}在点\end{CJK} $\left(x_{0}, y_{0}\right)$ \begin{CJK}{UTF8}{mj}可微\end{CJK}, \begin{CJK}{UTF8}{mj}偏导数\end{CJK} $f_{x}^{\prime}(x, y), f_{y}^{\prime}(x, y)$ \begin{CJK}{UTF8}{mj}在点\end{CJK} $\left(x_{0}, y_{0}\right)$ \begin{CJK}{UTF8}{mj}连续\end{CJK}.

(4) \begin{CJK}{UTF8}{mj}设\end{CJK} $A, B$ \begin{CJK}{UTF8}{mj}是两个有上界的数集\end{CJK}, \begin{CJK}{UTF8}{mj}数集\end{CJK} $C=\{x+y \mid x \in A, y \in B\}$, \begin{CJK}{UTF8}{mj}则\end{CJK} $\sup C=\sup A+\sup B$.

\begin{CJK}{UTF8}{mj}三\end{CJK}. \begin{CJK}{UTF8}{mj}叙述魏尔斯特拉斯致密性定理内容\end{CJK}.

\begin{CJK}{UTF8}{mj}四\end{CJK}. \begin{CJK}{UTF8}{mj}若\end{CJK} $f(x)$ \begin{CJK}{UTF8}{mj}是\end{CJK} $[a, b]$ \begin{CJK}{UTF8}{mj}上的连续函数\end{CJK}, \begin{CJK}{UTF8}{mj}且对数列\end{CJK} $x_{n} \in[a, b]$, \begin{CJK}{UTF8}{mj}有\end{CJK}
$$
\left|f\left(x_{n}\right)\right| \leqslant \frac{1}{2}\left|f\left(x_{n-1}\right)\right| .
$$
$n=2,3, \cdots$, \begin{CJK}{UTF8}{mj}证明\end{CJK}: $f(x)$ \begin{CJK}{UTF8}{mj}在\end{CJK} $[a, b]$ \begin{CJK}{UTF8}{mj}存在零点\end{CJK}.

\begin{CJK}{UTF8}{mj}五\end{CJK}. \begin{CJK}{UTF8}{mj}假设数列\end{CJK} $\left\{a_{n}\right\}$ \begin{CJK}{UTF8}{mj}满足下列条件\end{CJK}: $0<a_{n}<1,\left(1-a_{n}\right) a_{n+1}>\frac{1}{4},(n=1,2, \cdots)$. \begin{CJK}{UTF8}{mj}证明\end{CJK}: $\left\{a_{n}\right\}$ \begin{CJK}{UTF8}{mj}单调递增且\end{CJK} $\lim _{n \rightarrow \infty} a_{n}=\frac{1}{2} .$

\begin{CJK}{UTF8}{mj}六\end{CJK}. \begin{CJK}{UTF8}{mj}证明\end{CJK}: \begin{CJK}{UTF8}{mj}若数列\end{CJK} $\left\{a_{n}\right\}$ (\begin{CJK}{UTF8}{mj}其中\end{CJK} $a_{n}>0, n=1,2,3, \cdots$ ) \begin{CJK}{UTF8}{mj}单调递减且\end{CJK} $\sum_{n=1}^{\infty}(-1)^{n} a_{n}$ \begin{CJK}{UTF8}{mj}发散\end{CJK}, \begin{CJK}{UTF8}{mj}则\end{CJK}
$$
\sum_{n=1}^{\infty}\left(\frac{1}{1+a_{n}}\right)^{n}
$$
\begin{CJK}{UTF8}{mj}收敛\end{CJK}.

\begin{CJK}{UTF8}{mj}七\end{CJK}. \begin{CJK}{UTF8}{mj}若\end{CJK} $P(n)$ \begin{CJK}{UTF8}{mj}与\end{CJK} $Q(n)$ \begin{CJK}{UTF8}{mj}分别是关于\end{CJK} $n$ \begin{CJK}{UTF8}{mj}的\end{CJK} $p$ \begin{CJK}{UTF8}{mj}次与\end{CJK} $q$ \begin{CJK}{UTF8}{mj}次多项式\end{CJK}, \begin{CJK}{UTF8}{mj}且\end{CJK} $Q(n) \neq 0$, \begin{CJK}{UTF8}{mj}则级数\end{CJK}
$$
\sum_{n=1}^{\infty} \frac{P(n)}{Q(n)}
$$
\begin{CJK}{UTF8}{mj}收敛的充要条件是\end{CJK} $q-p \geqslant 2$.

\begin{CJK}{UTF8}{mj}八\end{CJK}. \begin{CJK}{UTF8}{mj}无穷广义积分\end{CJK} $\int_{a}^{+\infty} f(x) \mathrm{d} x$ \begin{CJK}{UTF8}{mj}收敛且\end{CJK} $f(x)$ \begin{CJK}{UTF8}{mj}在\end{CJK} $[a,+\infty)$ \begin{CJK}{UTF8}{mj}严格单减\end{CJK}, \begin{CJK}{UTF8}{mj}则\end{CJK}

(1) $f(x) \geqslant 0$. (2) $\lim _{x \rightarrow+\infty} x f(x)=0$.

\begin{CJK}{UTF8}{mj}九\end{CJK}. \begin{CJK}{UTF8}{mj}设\end{CJK} $\sum_{n=1}^{\infty} a_{n}(x)$ \begin{CJK}{UTF8}{mj}是\end{CJK} $[a, b]$ \begin{CJK}{UTF8}{mj}上可微函数项级数\end{CJK}, \begin{CJK}{UTF8}{mj}且\end{CJK} $\sum_{n=1}^{\infty} a_{n}^{\prime}(x)$ \begin{CJK}{UTF8}{mj}的部分和函数列在\end{CJK} $[a, b]$ \begin{CJK}{UTF8}{mj}上一致有界\end{CJK}, \begin{CJK}{UTF8}{mj}证\end{CJK}: \begin{CJK}{UTF8}{mj}若\end{CJK} $\sum_{n=1}^{\infty} a_{n}(x)$ \begin{CJK}{UTF8}{mj}在\end{CJK} $[a, b]$ \begin{CJK}{UTF8}{mj}上收敛\end{CJK}, \begin{CJK}{UTF8}{mj}则必在\end{CJK} $[a, b]$ \begin{CJK}{UTF8}{mj}上一致收敛\end{CJK}.

\begin{CJK}{UTF8}{mj}十\end{CJK}. \begin{CJK}{UTF8}{mj}椭球面\end{CJK}
$$
\frac{x^{2}}{3}+y^{2}+\frac{z^{2}}{2}=1
$$
\begin{CJK}{UTF8}{mj}被通过原点的平面\end{CJK} $2 x+y+z=0$ \begin{CJK}{UTF8}{mj}截成一个椭圆\end{CJK} $C$, \begin{CJK}{UTF8}{mj}求此椭圆的面积\end{CJK}.

\begin{CJK}{UTF8}{mj}十一\end{CJK}. \begin{CJK}{UTF8}{mj}求\end{CJK}
$$
\begin{gathered}
I=\iint_{S} f(x, y, z) \mathrm{d} s . \\
\left.\mid x^{2}+y^{2}+z^{2}=1\right\}, f(x, y, z)= \begin{cases}x^{2}+y^{2}, & z \geqslant \sqrt{x^{2}+y^{2}} ; \\
0, & z<\sqrt{x^{2}+y^{2}} .\end{cases}
\end{gathered}
$$
$$
\begin{aligned}
& \text { 其中 } S=\left\{(x, y, z) \mid x^{2}+y^{2}+z^{2}=1\right\}, f(x, y, z)= \begin{cases}x^{2}+y^{2}, & z \geqslant \sqrt{x^{2}+y^{2}} \\ 0, & z<\sqrt{x^{2}+y^{2}} .\end{cases} 
\end{aligned}
$$

\section{4. 中国海洋大学 2014 年研究生入学考试试题数学分析}
\begin{CJK}{UTF8}{mj}李扬\end{CJK}

\begin{CJK}{UTF8}{mj}微信公众号\end{CJK}: sxkyliyang

\begin{CJK}{UTF8}{mj}一\end{CJK}. ( 24 \begin{CJK}{UTF8}{mj}分\end{CJK}) \begin{CJK}{UTF8}{mj}计算题\end{CJK} (\begin{CJK}{UTF8}{mj}要求有计算过程\end{CJK})
$$
\lim _{n \rightarrow \infty} \frac{n}{\sqrt[n]{n !}}
$$
$$
\int_{0}^{+\infty} \frac{\arctan b x-\arctan a x}{x} \mathrm{~d} x,(b>a>0) .
$$
$$
\lim _{x \rightarrow 0, y \rightarrow 0}\left(x^{2}+y^{2}\right)^{x^{2} y^{2}}
$$
\begin{CJK}{UTF8}{mj}二\end{CJK}. ( 40 \begin{CJK}{UTF8}{mj}分\end{CJK}) \begin{CJK}{UTF8}{mj}判断题\end{CJK} (\begin{CJK}{UTF8}{mj}正确的给孕证明\end{CJK}, \begin{CJK}{UTF8}{mj}错误的加以说明\end{CJK})

(1) \begin{CJK}{UTF8}{mj}若\end{CJK} $\lim _{n \rightarrow \infty} a_{n}=\infty$, \begin{CJK}{UTF8}{mj}则\end{CJK} $\lim _{n \rightarrow \infty} \frac{a_{1}+a_{2}+\cdots+a_{n}}{n}=\infty$. ( )

(2)\begin{CJK}{UTF8}{mj}函数\end{CJK} $f(x)$ \begin{CJK}{UTF8}{mj}在\end{CJK} $[a, b]$ \begin{CJK}{UTF8}{mj}上有定义\end{CJK}, \begin{CJK}{UTF8}{mj}并且对\end{CJK} $[a, b]$ \begin{CJK}{UTF8}{mj}内的任何\end{CJK} $x$, \begin{CJK}{UTF8}{mj}存在\end{CJK} $x$ \begin{CJK}{UTF8}{mj}的某邻域\end{CJK} $O_{x}$, \begin{CJK}{UTF8}{mj}使得\end{CJK} $f(x)$ \begin{CJK}{UTF8}{mj}在\end{CJK} $O_{x}$ \begin{CJK}{UTF8}{mj}内有界\end{CJK}, \begin{CJK}{UTF8}{mj}则\end{CJK} $f(x)$ \begin{CJK}{UTF8}{mj}在\end{CJK} $[a, b]$ \begin{CJK}{UTF8}{mj}上有界\end{CJK}.

(3) \begin{CJK}{UTF8}{mj}函数\end{CJK} $f(x)$ \begin{CJK}{UTF8}{mj}在\end{CJK} $(a, b)$ \begin{CJK}{UTF8}{mj}内处处有导数\end{CJK} $f^{\prime}(x)$, \begin{CJK}{UTF8}{mj}则\end{CJK} $f^{\prime}(x)$ \begin{CJK}{UTF8}{mj}可能有第一类不连续点\end{CJK}. ( )

(4) \begin{CJK}{UTF8}{mj}设每一函数\end{CJK} $S_{n}(x), n=1,2, \cdots$ \begin{CJK}{UTF8}{mj}均在点\end{CJK} $c$ \begin{CJK}{UTF8}{mj}右连续\end{CJK}, \begin{CJK}{UTF8}{mj}并且数列\end{CJK} $\left\{S_{n}(c)\right\}$ \begin{CJK}{UTF8}{mj}发散\end{CJK}, \begin{CJK}{UTF8}{mj}则在开区间\end{CJK} $(c, c+\delta)($ \begin{CJK}{UTF8}{mj}其中\end{CJK} $\delta>0)$ \begin{CJK}{UTF8}{mj}内\end{CJK}, \begin{CJK}{UTF8}{mj}函数列\end{CJK} $\left\{S_{n}(x)\right\}$ \begin{CJK}{UTF8}{mj}必不一致收敛\end{CJK}. ( )

\begin{CJK}{UTF8}{mj}三\end{CJK}. ( 15 \begin{CJK}{UTF8}{mj}分\end{CJK}) \begin{CJK}{UTF8}{mj}设\end{CJK} $f(x)$ \begin{CJK}{UTF8}{mj}在\end{CJK} $[a, b]$ \begin{CJK}{UTF8}{mj}上有定义\end{CJK}, $x_{0} \in[a, b]$, \begin{CJK}{UTF8}{mj}若对任意的\end{CJK} $\varepsilon>0$, \begin{CJK}{UTF8}{mj}存在\end{CJK} $\delta>0$, \begin{CJK}{UTF8}{mj}当\end{CJK} $\left|x-x_{0}\right|<\delta$ \begin{CJK}{UTF8}{mj}时\end{CJK}, \begin{CJK}{UTF8}{mj}有\end{CJK} $f(x)<f\left(x_{0}\right)+\varepsilon$, \begin{CJK}{UTF8}{mj}则称\end{CJK} $f(x)$ \begin{CJK}{UTF8}{mj}在点\end{CJK} $x_{0}$ \begin{CJK}{UTF8}{mj}处上半连续\end{CJK}. \begin{CJK}{UTF8}{mj}若\end{CJK} $f(x)$ \begin{CJK}{UTF8}{mj}在\end{CJK} $[a, b]$ \begin{CJK}{UTF8}{mj}上每一点都上半连续\end{CJK}, \begin{CJK}{UTF8}{mj}则称\end{CJK} $f(x)$ \begin{CJK}{UTF8}{mj}为\end{CJK} $[a, b]$ \begin{CJK}{UTF8}{mj}上的一个上半连续函数\end{CJK}. \begin{CJK}{UTF8}{mj}证明\end{CJK}: $[a, b]$ \begin{CJK}{UTF8}{mj}上的上半连续函数一定有界\end{CJK}.

\begin{CJK}{UTF8}{mj}四\end{CJK}. (16 \begin{CJK}{UTF8}{mj}分\end{CJK})

(1) \begin{CJK}{UTF8}{mj}讨论函数\end{CJK}
$$
f(x)=\frac{1}{\ln (1+x)}-\frac{1}{x}
$$
\begin{CJK}{UTF8}{mj}在\end{CJK} $(0,1)$ \begin{CJK}{UTF8}{mj}上的单调性\end{CJK}.

(2) \begin{CJK}{UTF8}{mj}补充\end{CJK} (1) \begin{CJK}{UTF8}{mj}中的函数在\end{CJK} $x=0$ \begin{CJK}{UTF8}{mj}点的函数值\end{CJK}, \begin{CJK}{UTF8}{mj}使之在这点连续\end{CJK}, \begin{CJK}{UTF8}{mj}求\end{CJK} $f(x)$ \begin{CJK}{UTF8}{mj}在\end{CJK} $[0,1]$ \begin{CJK}{UTF8}{mj}上的最大值和最小值\end{CJK}.

(3) \begin{CJK}{UTF8}{mj}求最小的\end{CJK} $\beta$ \begin{CJK}{UTF8}{mj}和最大的\end{CJK} $\alpha$, \begin{CJK}{UTF8}{mj}使得对任意的正整数\end{CJK} $n$, \begin{CJK}{UTF8}{mj}成立不等式\end{CJK}:
$$
\left(1+\frac{1}{n}\right)^{n+\alpha} \leqslant e \leqslant\left(1+\frac{1}{n}\right)^{n+\beta}
$$
\begin{CJK}{UTF8}{mj}五\end{CJK}. (10 \begin{CJK}{UTF8}{mj}分\end{CJK}) \begin{CJK}{UTF8}{mj}设\end{CJK} $\varphi(x), \psi(x)$ \begin{CJK}{UTF8}{mj}都有连续的二阶导数\end{CJK}. \begin{CJK}{UTF8}{mj}证明\end{CJK}: \begin{CJK}{UTF8}{mj}函数\end{CJK} $u=\varphi\left(\frac{y}{x}\right)+x \psi\left(\frac{y}{x}\right)$ \begin{CJK}{UTF8}{mj}满足方程\end{CJK}
$$
x^{2} \frac{\partial^{2} u}{\partial x^{2}}+2 x y \frac{\partial^{2} u}{\partial x \partial y}+y^{2} \frac{\partial^{2} u}{\partial y^{2}}=0 .
$$
\begin{CJK}{UTF8}{mj}六\end{CJK}. (15 \begin{CJK}{UTF8}{mj}分\end{CJK}) \begin{CJK}{UTF8}{mj}设正项级数\end{CJK} $\sum_{n=1}^{\infty} a_{n}$ \begin{CJK}{UTF8}{mj}收敛\end{CJK}, \begin{CJK}{UTF8}{mj}和为\end{CJK} $S$.

(1) \begin{CJK}{UTF8}{mj}求\end{CJK}
$$
\lim _{n \rightarrow \infty} \frac{a_{1}+2 a_{2}+\cdots+n a_{n}}{n^{2}} .
$$
(2) \begin{CJK}{UTF8}{mj}判断级数\end{CJK}
$$
\sum_{n=1}^{\infty} \frac{a_{1}+2 a_{2}+\cdots+n a_{n}}{n^{2}}
$$
\begin{CJK}{UTF8}{mj}的敛散性\end{CJK}.

\begin{CJK}{UTF8}{mj}七\end{CJK}. (15 \begin{CJK}{UTF8}{mj}分\end{CJK}) \begin{CJK}{UTF8}{mj}设\end{CJK} $a>0, c>0, a c-b^{2}>0$, \begin{CJK}{UTF8}{mj}则方程\end{CJK} $a x^{2}+2 b x y+c y^{2}=1$ \begin{CJK}{UTF8}{mj}表示椭圆\end{CJK}, \begin{CJK}{UTF8}{mj}利用拉格朗日乘子法证明该椭\end{CJK} \begin{CJK}{UTF8}{mj}圆的面积为\end{CJK} $\frac{\pi}{\sqrt{a c-b^{2}}}$.

\begin{CJK}{UTF8}{mj}八\end{CJK}. ( 15 \begin{CJK}{UTF8}{mj}分\end{CJK}) \begin{CJK}{UTF8}{mj}计算封闭曲面\end{CJK}
$$
\left(x^{2}+y^{2}+z^{2}\right)^{2}=a^{3} z(a>0)
$$
\begin{CJK}{UTF8}{mj}所围成立体的体积\end{CJK}.

\section{5. 中国海洋大学 2015 年研究生入学考试试题数学分析}
\begin{CJK}{UTF8}{mj}李扬\end{CJK}

\begin{CJK}{UTF8}{mj}微信公众号\end{CJK}: sxkyliyang

\begin{CJK}{UTF8}{mj}一\end{CJK}. (24 \begin{CJK}{UTF8}{mj}分\end{CJK}) \begin{CJK}{UTF8}{mj}计算题\end{CJK} (\begin{CJK}{UTF8}{mj}要求有计算过程\end{CJK})

(1)
$$
\lim _{n \rightarrow \infty} \sum_{k=1}^{n} \frac{n}{n^{2}+k^{2}}
$$
(2)
$$
\lim _{x \rightarrow 0, y \rightarrow 0} \frac{x^{3} y+x y^{4}+x^{2} y}{x+y} .
$$
$(3)$
$$
\int_{1}^{e} \frac{\mathrm{d} x}{x \sqrt{1-(\ln x)^{2}}}
$$
\begin{CJK}{UTF8}{mj}二\end{CJK}. ( 40 \begin{CJK}{UTF8}{mj}分\end{CJK}) \begin{CJK}{UTF8}{mj}判断题\end{CJK} (\begin{CJK}{UTF8}{mj}正确的给予证明\end{CJK}, \begin{CJK}{UTF8}{mj}错误的加以说明\end{CJK})

\begin{enumerate}
  \item \begin{CJK}{UTF8}{mj}设\end{CJK} $f(x)$ \begin{CJK}{UTF8}{mj}在\end{CJK} $(a, b)$ \begin{CJK}{UTF8}{mj}上连续\end{CJK}, \begin{CJK}{UTF8}{mj}且\end{CJK} $\lim _{x \rightarrow a^{+}} f(x)=\lim _{x \rightarrow b^{-}} f(x)=0$, \begin{CJK}{UTF8}{mj}则\end{CJK} $f(x)$ \begin{CJK}{UTF8}{mj}在\end{CJK} $(a, b)$ \begin{CJK}{UTF8}{mj}内存在最大值或最小值\end{CJK}. ( )

  \item \begin{CJK}{UTF8}{mj}设\end{CJK} $f(x)$ \begin{CJK}{UTF8}{mj}在\end{CJK} $(-\infty,+\infty)$ \begin{CJK}{UTF8}{mj}上连续\end{CJK}, \begin{CJK}{UTF8}{mj}则积分\end{CJK} $\varphi(y)=\int_{0}^{2 \pi} f(x+y) \mathrm{d} x$ \begin{CJK}{UTF8}{mj}与\end{CJK} $y$ \begin{CJK}{UTF8}{mj}无关当且仅当\end{CJK} $f(x)$ \begin{CJK}{UTF8}{mj}为周期是\end{CJK} $2 \pi$ \begin{CJK}{UTF8}{mj}的\end{CJK} \begin{CJK}{UTF8}{mj}周期函数\end{CJK}. ( )

  \item \begin{CJK}{UTF8}{mj}设函数\end{CJK} $f(x)$ \begin{CJK}{UTF8}{mj}在\end{CJK} $[0,1]$ \begin{CJK}{UTF8}{mj}上有三阶导数\end{CJK}, \begin{CJK}{UTF8}{mj}且\end{CJK} $f(0)=0, f(1)=\frac{1}{2}, f^{\prime}\left(\frac{1}{2}\right)=0$, \begin{CJK}{UTF8}{mj}则存在\end{CJK} $\xi \in(0,1)$, \begin{CJK}{UTF8}{mj}使得\end{CJK} $\left|f^{\prime \prime \prime}(\xi)\right| \geqslant 12$. ( )

  \item \begin{CJK}{UTF8}{mj}设\end{CJK} $f_{1}(x)$ \begin{CJK}{UTF8}{mj}在\end{CJK} $[a, b]$ \begin{CJK}{UTF8}{mj}上可积\end{CJK}, $f_{n+1}(x)=\int_{a}^{x} f_{n}(t) \mathrm{d} t, n=1,2, \cdots$, \begin{CJK}{UTF8}{mj}则函数列\end{CJK} $\left\{f_{n}(x)\right\}$ \begin{CJK}{UTF8}{mj}在\end{CJK} $[a, b]$ \begin{CJK}{UTF8}{mj}上不一致收敛\end{CJK} \begin{CJK}{UTF8}{mj}于零\end{CJK}. ( )

\end{enumerate}
\begin{CJK}{UTF8}{mj}三\end{CJK}. (15 \begin{CJK}{UTF8}{mj}分\end{CJK}) \begin{CJK}{UTF8}{mj}设函数\end{CJK} $f(x)$ \begin{CJK}{UTF8}{mj}在\end{CJK} $[a, b]$ \begin{CJK}{UTF8}{mj}上连续\end{CJK}, \begin{CJK}{UTF8}{mj}在\end{CJK} $(a, b)$ \begin{CJK}{UTF8}{mj}上二阶可导\end{CJK}, \begin{CJK}{UTF8}{mj}且\end{CJK} $f(a)=f(b)=0, f_{+}^{\prime}(a)>0, f_{-}^{\prime}(b)>0$, \begin{CJK}{UTF8}{mj}证明\end{CJK}:

(1) \begin{CJK}{UTF8}{mj}存在\end{CJK} $\xi \in(a, b)$, \begin{CJK}{UTF8}{mj}使得\end{CJK} $f^{\prime \prime}(\xi)=0$.

(2) \begin{CJK}{UTF8}{mj}存在\end{CJK} $\eta_{1}, \eta_{2} \in(a, b)$, \begin{CJK}{UTF8}{mj}使得\end{CJK} $f^{\prime \prime}\left(\eta_{1}\right)<0, f^{\prime \prime}\left(\eta_{2}\right)>0$.

\begin{CJK}{UTF8}{mj}四\end{CJK}. (15 \begin{CJK}{UTF8}{mj}分\end{CJK}) \begin{CJK}{UTF8}{mj}设变换\end{CJK} $\left\{\begin{array}{l}u=x+a \sqrt{y} \\ v=x+2 \sqrt{y}\end{array}\right.$ \begin{CJK}{UTF8}{mj}把方程\end{CJK}
$$
\frac{\partial^{2} z}{\partial x^{2}}-y \frac{\partial^{2} z}{\partial y^{2}}-\frac{1}{2} \frac{\partial z}{\partial y}=0 \text { (其中 } z \text { 具有连续的二阶偏导数) }
$$
\begin{CJK}{UTF8}{mj}化为\end{CJK} $\frac{\partial^{2} z}{\partial u \partial v}=0$, \begin{CJK}{UTF8}{mj}求常数\end{CJK} $a$ \begin{CJK}{UTF8}{mj}的值\end{CJK}.

\begin{CJK}{UTF8}{mj}五\end{CJK}. (15 \begin{CJK}{UTF8}{mj}分\end{CJK}) \begin{CJK}{UTF8}{mj}在椭球面\end{CJK} $2 x^{2}+2 y^{2}+z^{2}=1$ \begin{CJK}{UTF8}{mj}上求一点\end{CJK}, \begin{CJK}{UTF8}{mj}使函数\end{CJK}
$$
f(x, y, z)=x^{2}+y^{2}+z^{2}
$$
\begin{CJK}{UTF8}{mj}在该点处沿方向\end{CJK} $l=(1,-1,0)$ \begin{CJK}{UTF8}{mj}的方向导数最大\end{CJK}.

\begin{CJK}{UTF8}{mj}六\end{CJK}. (15 \begin{CJK}{UTF8}{mj}分\end{CJK}) \begin{CJK}{UTF8}{mj}证明函数\end{CJK}
$$
f(x)=\sum_{n=1}^{\infty} \frac{\sin n x}{n^{3}}
$$
\begin{CJK}{UTF8}{mj}在\end{CJK} $(-\infty,+\infty)$ \begin{CJK}{UTF8}{mj}内连续\end{CJK}, \begin{CJK}{UTF8}{mj}并且有连续导函数\end{CJK}. \begin{CJK}{UTF8}{mj}七\end{CJK}. (13 \begin{CJK}{UTF8}{mj}分\end{CJK}) \begin{CJK}{UTF8}{mj}计算曲线积分\end{CJK}
$$
\oint_{L} \frac{x y^{2} \mathrm{~d} y-x^{2} y \mathrm{~d} x}{x^{2}+y^{2}}
$$
\begin{CJK}{UTF8}{mj}其中\end{CJK} $L$ \begin{CJK}{UTF8}{mj}为圆周\end{CJK} $x^{2}+y^{2}=a^{2}(a>0)$ \begin{CJK}{UTF8}{mj}沿逆时针方向\end{CJK}.

\begin{CJK}{UTF8}{mj}八\end{CJK}. (13 \begin{CJK}{UTF8}{mj}分\end{CJK}) \begin{CJK}{UTF8}{mj}计算曲面积分\end{CJK}
$$
\iint_{S} y z \mathrm{~d} y \mathrm{~d} z+\left(x^{2}+z^{2}\right) y \mathrm{~d} z \mathrm{~d} x+x y \mathrm{~d} x \mathrm{~d} y
$$
\begin{CJK}{UTF8}{mj}其中\end{CJK} $S$ \begin{CJK}{UTF8}{mj}为曲面\end{CJK} $4-y=x^{2}+z^{2}$ \begin{CJK}{UTF8}{mj}上\end{CJK} $y \geqslant 0$ \begin{CJK}{UTF8}{mj}的部分\end{CJK}, \begin{CJK}{UTF8}{mj}取右侧\end{CJK}. 16. \begin{CJK}{UTF8}{mj}中国海洋大学\end{CJK} 2016 \begin{CJK}{UTF8}{mj}年研究生入学考试试题数学分析\end{CJK}

\begin{CJK}{UTF8}{mj}李扬\end{CJK}

\begin{CJK}{UTF8}{mj}微信公众号\end{CJK}: sxkyliyang

\begin{CJK}{UTF8}{mj}一\end{CJK}. (24 \begin{CJK}{UTF8}{mj}分\end{CJK}) \begin{CJK}{UTF8}{mj}计算题\end{CJK} (\begin{CJK}{UTF8}{mj}要求有计算过程\end{CJK})
$$
\lim _{x \rightarrow 0} \frac{1}{x^{4}}\left[\ln \left(1+\sin ^{2} x\right)-4 \sin ^{2} \frac{x}{2}\right]
$$
2 .
$$
\lim _{n \rightarrow \infty} \frac{1}{n} \sqrt[n]{n(n+1)(n+2) \cdots(n+n)}
$$
$$
\int_{0}^{+\infty}\left(\frac{\sin \alpha x}{x}\right)^{2} d x(\alpha>0)
$$
\begin{CJK}{UTF8}{mj}提示\end{CJK}: Dirichlet \begin{CJK}{UTF8}{mj}积分\end{CJK} $\int_{0}^{+\infty} \frac{\sin x}{x} \mathrm{~d} x=\frac{\pi}{2}$.

\begin{CJK}{UTF8}{mj}二\end{CJK}. (30 \begin{CJK}{UTF8}{mj}分\end{CJK}) \begin{CJK}{UTF8}{mj}判断题\end{CJK} (\begin{CJK}{UTF8}{mj}正确的给予证明\end{CJK}, \begin{CJK}{UTF8}{mj}错误的举出反例\end{CJK})

\begin{enumerate}
  \item \begin{CJK}{UTF8}{mj}如果定积分\end{CJK} $\int_{a}^{b} f(x) \mathrm{d} x=0$, \begin{CJK}{UTF8}{mj}则函数\end{CJK} $f(x)$ \begin{CJK}{UTF8}{mj}在\end{CJK} $[a, b]$ \begin{CJK}{UTF8}{mj}上或者恒等于\end{CJK} 0 , \begin{CJK}{UTF8}{mj}或者存在\end{CJK} $x_{1} x_{2} \in[a, b]$, \begin{CJK}{UTF8}{mj}使得\end{CJK} $f\left(x_{1}\right)>0, f\left(x_{2}\right)<0 .(\quad)$

  \item \begin{CJK}{UTF8}{mj}存在函数\end{CJK} $f(x)$, \begin{CJK}{UTF8}{mj}在所有\end{CJK} $x \neq 0$ \begin{CJK}{UTF8}{mj}处不连续\end{CJK}, \begin{CJK}{UTF8}{mj}而在\end{CJK} $x=0$ \begin{CJK}{UTF8}{mj}处可微\end{CJK}. ( )

  \item \begin{CJK}{UTF8}{mj}设\end{CJK} $\sum_{n=1}^{\infty} u_{n}$ \begin{CJK}{UTF8}{mj}是收敛的正项级数\end{CJK}, \begin{CJK}{UTF8}{mj}且数列\end{CJK} $\left\{u_{n}\right\}$ \begin{CJK}{UTF8}{mj}单调减少\end{CJK}, \begin{CJK}{UTF8}{mj}则一定有\end{CJK} $\lim _{n \rightarrow \infty} n u_{n}=0$. ( )

\end{enumerate}
\begin{CJK}{UTF8}{mj}三\end{CJK}. 1. (10 \begin{CJK}{UTF8}{mj}分\end{CJK}) \begin{CJK}{UTF8}{mj}设\end{CJK} $\left\{a_{n}\right\}$ \begin{CJK}{UTF8}{mj}为任一数列\end{CJK}, $\left\{b_{n}\right\}$ \begin{CJK}{UTF8}{mj}为趋于\end{CJK} $+\infty$ \begin{CJK}{UTF8}{mj}的单调增加数列\end{CJK}, \begin{CJK}{UTF8}{mj}且\end{CJK}
$$
\lim _{n \rightarrow \infty} \frac{a_{n}-a_{n-1}}{b_{n}-b_{n-1}}=l .
$$
$l$ \begin{CJK}{UTF8}{mj}为有限数\end{CJK}, \begin{CJK}{UTF8}{mj}证明\end{CJK}: $\lim _{n \rightarrow \infty} \frac{a_{n}}{b_{n}}=l$.

\begin{enumerate}
  \setcounter{enumi}{2}
  \item ( 20 \begin{CJK}{UTF8}{mj}分\end{CJK}) \begin{CJK}{UTF8}{mj}对数列\end{CJK} $x_{0}=\alpha, 0<\alpha<\frac{\pi}{2}, x_{n}=\sin x_{n-1}, n=1,2, \cdots$.
\end{enumerate}
(1) \begin{CJK}{UTF8}{mj}证明\end{CJK}: \begin{CJK}{UTF8}{mj}数列\end{CJK} $\left\{x_{n}\right\}$ \begin{CJK}{UTF8}{mj}存在极限并求出此极限值\end{CJK}.

(2) \begin{CJK}{UTF8}{mj}利用\end{CJK} 1 \begin{CJK}{UTF8}{mj}题的结论\end{CJK}, \begin{CJK}{UTF8}{mj}证明\end{CJK}: $\lim _{n \rightarrow \infty} n x_{n}^{2}=3$.

\begin{CJK}{UTF8}{mj}四\end{CJK}. ( 10 \begin{CJK}{UTF8}{mj}分\end{CJK}) \begin{CJK}{UTF8}{mj}若\end{CJK} $f(x)$ \begin{CJK}{UTF8}{mj}在\end{CJK} $[a, b]$ \begin{CJK}{UTF8}{mj}上可导\end{CJK}, \begin{CJK}{UTF8}{mj}且\end{CJK} $f_{+}^{\prime}(a)<1<f_{-}^{\prime}(b)$. \begin{CJK}{UTF8}{mj}证明\end{CJK}: \begin{CJK}{UTF8}{mj}至少存在一点\end{CJK} $\xi \in(a, b)$, \begin{CJK}{UTF8}{mj}使得\end{CJK} $f^{\prime}(\xi)=1$. (\begin{CJK}{UTF8}{mj}其中\end{CJK} $f_{+}^{\prime}(a)$ \begin{CJK}{UTF8}{mj}表示在点\end{CJK} $a$ \begin{CJK}{UTF8}{mj}的右导数\end{CJK}, $f_{-}^{\prime}(b)$ \begin{CJK}{UTF8}{mj}表示在点\end{CJK} $b$ \begin{CJK}{UTF8}{mj}的左导数\end{CJK} $)$

\begin{CJK}{UTF8}{mj}五\end{CJK}. \begin{CJK}{UTF8}{mj}选取适当的\end{CJK} $\alpha, \beta$ (\begin{CJK}{UTF8}{mj}即确定的\end{CJK} $\alpha, \beta$ \begin{CJK}{UTF8}{mj}的值\end{CJK}), \begin{CJK}{UTF8}{mj}使得变换\end{CJK} $\left\{\begin{array}{l}u=x+\alpha y \\ v=x+\beta y\end{array}\right.$ \begin{CJK}{UTF8}{mj}将方程\end{CJK}
$$
a \frac{\partial^{2} z}{\partial x^{2}}+2 b \frac{\partial^{2} z}{\partial x \partial y}+c \frac{\partial^{2} z}{\partial y^{2}}=0,\left(a, b, c \text { 都是常数, } b^{2}-a c=0, c \neq 0\right)
$$
\begin{CJK}{UTF8}{mj}作代换后的方程有简单的形式\end{CJK}. (\begin{CJK}{UTF8}{mj}本题分值和本题之后的题目信息暂无\end{CJK}, \begin{CJK}{UTF8}{mj}欢迎提供\end{CJK}.)

\section{1. 中国科学技术大学 2009 年研究生入学考试试题线性代数与解析 几何}
\begin{CJK}{UTF8}{mj}李扬\end{CJK}

\begin{CJK}{UTF8}{mj}微信公众号\end{CJK}: sxkyliyang

\begin{CJK}{UTF8}{mj}一\end{CJK}. \begin{CJK}{UTF8}{mj}填空\end{CJK}: $A=\left(\begin{array}{ccc}0 & \cdots & a_{1} \\ \vdots & \ddots & \vdots \\ a_{2 n} & \cdots & 0\end{array}\right), a_{2 k}=0, a_{2 k-1}=1(k=1,2, \cdots, n)$, \begin{CJK}{UTF8}{mj}求\end{CJK} $A$ \begin{CJK}{UTF8}{mj}的\end{CJK}Jordan \begin{CJK}{UTF8}{mj}标准型\end{CJK}.

\begin{CJK}{UTF8}{mj}二\end{CJK}. \begin{CJK}{UTF8}{mj}直线\end{CJK} $l_{1}, l_{2}$ :

\begin{enumerate}
  \item \begin{CJK}{UTF8}{mj}证明\end{CJK}: \begin{CJK}{UTF8}{mj}两直线异面\end{CJK};

  \item \begin{CJK}{UTF8}{mj}求两直线的公垂线\end{CJK}

\end{enumerate}
\begin{CJK}{UTF8}{mj}三\end{CJK}. \begin{CJK}{UTF8}{mj}向量\end{CJK} $\alpha_{1}, \alpha_{2}$ \begin{CJK}{UTF8}{mj}线性无关\end{CJK}, $\alpha_{3}, \alpha_{4}$ \begin{CJK}{UTF8}{mj}线性无关\end{CJK}, \begin{CJK}{UTF8}{mj}且\end{CJK} $\alpha_{1}$ \begin{CJK}{UTF8}{mj}正交于\end{CJK} $\alpha_{3}, \alpha_{4}, \alpha_{2}$ \begin{CJK}{UTF8}{mj}正交于\end{CJK} $\alpha_{3}, \alpha_{4}$, \begin{CJK}{UTF8}{mj}证明\end{CJK}: \begin{CJK}{UTF8}{mj}这四个向量线性无关\end{CJK}.

\begin{CJK}{UTF8}{mj}四\end{CJK}. $A$ \begin{CJK}{UTF8}{mj}和\end{CJK} $B$ \begin{CJK}{UTF8}{mj}是两个不同的方阵\end{CJK}, \begin{CJK}{UTF8}{mj}满足\end{CJK} $A^{3}=B^{3}, A B^{2}=B^{2} A$, \begin{CJK}{UTF8}{mj}问\end{CJK} $A^{2}+B^{2}$ \begin{CJK}{UTF8}{mj}是否可逆\end{CJK}, \begin{CJK}{UTF8}{mj}说明理由\end{CJK}.

\begin{CJK}{UTF8}{mj}五\end{CJK}. $\alpha_{1}, \cdots, \alpha_{n}$ \begin{CJK}{UTF8}{mj}是\end{CJK} $V$ \begin{CJK}{UTF8}{mj}的基\end{CJK}, \begin{CJK}{UTF8}{mj}对任意\end{CJK} $c_{1}, \cdots, c_{n}$, \begin{CJK}{UTF8}{mj}存在且唯一的存在向量\end{CJK} $\alpha$, \begin{CJK}{UTF8}{mj}使得\end{CJK} $\left(\alpha, \alpha_{i}\right)=c_{i}$.

\begin{CJK}{UTF8}{mj}六\end{CJK}. 3 \begin{CJK}{UTF8}{mj}阶方阵\end{CJK} $A$ \begin{CJK}{UTF8}{mj}的特征值分别为\end{CJK} $\lambda_{1}=\lambda_{2}=9, \lambda_{3}=-9$ \begin{CJK}{UTF8}{mj}和\end{CJK} $\lambda_{1}$ \begin{CJK}{UTF8}{mj}与\end{CJK} $\lambda_{2}$ \begin{CJK}{UTF8}{mj}分别对应的特征向量为\end{CJK} $\alpha_{1}=(1,0,-1), \alpha_{2}=$ $(?, ?, ?)$, \begin{CJK}{UTF8}{mj}求矩阵\end{CJK} $A$ \begin{CJK}{UTF8}{mj}以及使\end{CJK} $A$ \begin{CJK}{UTF8}{mj}对角化的矩阵\end{CJK} $P$.

\begin{CJK}{UTF8}{mj}七\end{CJK}. $A$ \begin{CJK}{UTF8}{mj}是复方阵\end{CJK}, \begin{CJK}{UTF8}{mj}线性变换\end{CJK} $T: X \rightarrow A X+X A$, \begin{CJK}{UTF8}{mj}证明\end{CJK}: \begin{CJK}{UTF8}{mj}如果\end{CJK} $A$ \begin{CJK}{UTF8}{mj}可对角化\end{CJK}, \begin{CJK}{UTF8}{mj}那么\end{CJK} $T$ \begin{CJK}{UTF8}{mj}也可以对角化\end{CJK}.

\begin{CJK}{UTF8}{mj}八\end{CJK}. $A$ \begin{CJK}{UTF8}{mj}是复方阵\end{CJK}, \begin{CJK}{UTF8}{mj}定义\end{CJK} $e^{A}=\sum_{k=0}^{+\infty} \frac{A^{k}}{k !}$, \begin{CJK}{UTF8}{mj}证明\end{CJK}: $\operatorname{det}\left(e^{A}\right)=e^{\operatorname{tr}(A)}$.

\section{2. 中国科学技术大学 2010 年研究生入学考试试题线性代数与解析 几何 
 李扬 
 微信公众号: sxkyliyang}
\begin{CJK}{UTF8}{mj}一\end{CJK}. \begin{CJK}{UTF8}{mj}填空题\end{CJK}(\begin{CJK}{UTF8}{mj}每空\end{CJK} 5 \begin{CJK}{UTF8}{mj}分\end{CJK}, \begin{CJK}{UTF8}{mj}共\end{CJK} 60 \begin{CJK}{UTF8}{mj}分\end{CJK})

\begin{enumerate}
  \item \begin{CJK}{UTF8}{mj}二次曲线\end{CJK} $x^{2}-4 x y+y^{2}+10 x-10 y+21=0$ \begin{CJK}{UTF8}{mj}的类型是\end{CJK} , \begin{CJK}{UTF8}{mj}通过转轴去掉其交叉项的转角角度是\end{CJK} - (\begin{CJK}{UTF8}{mj}只需要填写一个角度即可\end{CJK}).

  \item \begin{CJK}{UTF8}{mj}以曲线\end{CJK}

\end{enumerate}
$$
\left\{\begin{array}{l}
y=x^{2} \\
z=2
\end{array}\right.
$$
\begin{CJK}{UTF8}{mj}为准线\end{CJK}, \begin{CJK}{UTF8}{mj}原点为顶点的雉面方程为\end{CJK}

\begin{enumerate}
  \setcounter{enumi}{3}
  \item \begin{CJK}{UTF8}{mj}以\end{CJK} $x O y$ \begin{CJK}{UTF8}{mj}平面上的曲线\end{CJK} $f(x, y)=0$ \begin{CJK}{UTF8}{mj}绕\end{CJK} $x$ \begin{CJK}{UTF8}{mj}轴旋转所得的旋转面的方程是\end{CJK} . \begin{CJK}{UTF8}{mj}如果曲线方程是\end{CJK} $x^{2}-y^{2}-1=$ 0 , \begin{CJK}{UTF8}{mj}由此得到的曲面类型是\end{CJK}

  \item \begin{CJK}{UTF8}{mj}设\end{CJK} $\alpha_{1}, \alpha_{2}, \alpha_{3}, \alpha_{4}$ \begin{CJK}{UTF8}{mj}是线性空间\end{CJK} $V$ \begin{CJK}{UTF8}{mj}中\end{CJK} 4 \begin{CJK}{UTF8}{mj}个线性无关的向量\end{CJK}, \begin{CJK}{UTF8}{mj}则向量组\end{CJK} $\alpha_{1}+\alpha_{2}, \alpha_{2}+\alpha_{3}, \alpha_{3}+\alpha_{4}, \alpha_{4}+\alpha_{1}$ \begin{CJK}{UTF8}{mj}的秩等于\end{CJK}

  \item \begin{CJK}{UTF8}{mj}在\end{CJK} 3 \begin{CJK}{UTF8}{mj}维实向量空间\end{CJK} $\mathbb{R}^{3}$ \begin{CJK}{UTF8}{mj}中\end{CJK}, \begin{CJK}{UTF8}{mj}设\end{CJK} $\alpha_{1}=(-1,1,1)^{T}, \alpha_{2}=(1,-1,0)^{T}, \alpha_{3}=(1,0,-1)^{T}, \beta=(-4,3,4)^{T}$. \begin{CJK}{UTF8}{mj}则\end{CJK} $\beta$ \begin{CJK}{UTF8}{mj}在基\end{CJK} $\left\{\alpha_{1}, \alpha_{2}, \alpha_{3}\right\}$ \begin{CJK}{UTF8}{mj}下的坐标是\end{CJK}

\end{enumerate}
\includegraphics[max width=\textwidth]{2022_04_18_3416d289b173eb9de8c1g-159}

\begin{enumerate}
  \setcounter{enumi}{8}
  \item $\lambda-$ \begin{CJK}{UTF8}{mj}矩阵\end{CJK} $\left(\begin{array}{ccc}\lambda-1 & \lambda & \lambda^{2}-1 \\ 3 \lambda-1 & \lambda^{2}+2 \lambda & 3 \lambda^{2}-1 \\ \lambda+1 & \lambda^{2} & \lambda^{2}+1\end{array}\right)$ \begin{CJK}{UTF8}{mj}的\end{CJK} Smith \begin{CJK}{UTF8}{mj}标准型是\end{CJK}

  \item \begin{CJK}{UTF8}{mj}用\end{CJK} Gram-Schmidt \begin{CJK}{UTF8}{mj}正交化方法将\end{CJK} $\mathbb{R}^{3}$ (\begin{CJK}{UTF8}{mj}标准内积\end{CJK}) \begin{CJK}{UTF8}{mj}的基\end{CJK} $\left\{(1,1,1)^{T},(-1,0,-1)^{T},(-1,2,3)^{T}\right\}$ \begin{CJK}{UTF8}{mj}化成的标准正\end{CJK} \begin{CJK}{UTF8}{mj}交基是\end{CJK}

  \item \begin{CJK}{UTF8}{mj}定义所有\end{CJK} $n$ \begin{CJK}{UTF8}{mj}阶实方阵构成的实线性空间\end{CJK} $V$ \begin{CJK}{UTF8}{mj}上的对称双线性函数为\end{CJK} $f(X, Y)=\operatorname{tr}\left(X^{T} Y\right), X, Y \in V$, \begin{CJK}{UTF8}{mj}二次\end{CJK} \begin{CJK}{UTF8}{mj}型为\end{CJK} $Q(X)=f(X, X)$. \begin{CJK}{UTF8}{mj}则\end{CJK} $Q(X)$ \begin{CJK}{UTF8}{mj}的正负惯性指数分别为\end{CJK}

\end{enumerate}
\begin{CJK}{UTF8}{mj}二\end{CJK}. ( 10 \begin{CJK}{UTF8}{mj}分\end{CJK}) \begin{CJK}{UTF8}{mj}求如下线性方程组的通解\end{CJK}:
$$
\left\{\begin{array}{l}
x_{1}+x_{2}+x_{3}+x_{4}+x_{5}=1 \\
3 x_{1}+2 x_{2}+x_{3}+x_{4}-3 x_{5}=-2 \\
x_{2}+2 x_{3}+2 x_{4}+6 x_{5}=5 \\
5 x_{1}+4 x_{2}+3 x_{3}+3 x_{4}-x_{5}=0
\end{array}\right.
$$
\begin{CJK}{UTF8}{mj}三\end{CJK}. (15 \begin{CJK}{UTF8}{mj}分\end{CJK}) \begin{CJK}{UTF8}{mj}设空间上有直线\end{CJK} $l_{1}: \frac{x-1}{3}=\frac{y}{1}=\frac{z}{0}$ \begin{CJK}{UTF8}{mj}和\end{CJK} $l_{2}:(x, y, z)=(3+2 t, t, 3 t-3)$. \begin{CJK}{UTF8}{mj}设平面\end{CJK} $\pi$ \begin{CJK}{UTF8}{mj}与直线\end{CJK} $l_{1}, l_{2}$ \begin{CJK}{UTF8}{mj}平行\end{CJK}, \begin{CJK}{UTF8}{mj}且\end{CJK} $\pi$ \begin{CJK}{UTF8}{mj}与\end{CJK} $l_{1}$ \begin{CJK}{UTF8}{mj}的距离是\end{CJK} $\sqrt{91}$, \begin{CJK}{UTF8}{mj}求\end{CJK} $\pi$ \begin{CJK}{UTF8}{mj}的方程\end{CJK}.

\begin{CJK}{UTF8}{mj}四\end{CJK}. ( 10 \begin{CJK}{UTF8}{mj}分\end{CJK}) \begin{CJK}{UTF8}{mj}设\end{CJK} $\mathscr{A}: U \rightarrow V$ \begin{CJK}{UTF8}{mj}为数域\end{CJK} $F$ \begin{CJK}{UTF8}{mj}上的线性空间\end{CJK} $U$ \begin{CJK}{UTF8}{mj}到\end{CJK} $V$ \begin{CJK}{UTF8}{mj}上线性映射\end{CJK}. \begin{CJK}{UTF8}{mj}证明\end{CJK}:
$$
\operatorname{dim} \operatorname{Ker} \mathscr{A}+\operatorname{dim} \operatorname{Im} \mathscr{A}=\operatorname{dim} U
$$
\begin{CJK}{UTF8}{mj}五\end{CJK}. ( 15 \begin{CJK}{UTF8}{mj}分\end{CJK}) \begin{CJK}{UTF8}{mj}设\end{CJK} $A=\left(\begin{array}{ccc}2 & -1 & 1 \\ 2 & 2 & -1 \\ 1 & 2 & -1\end{array}\right)$, \begin{CJK}{UTF8}{mj}求方阵\end{CJK} $P$, \begin{CJK}{UTF8}{mj}使得\end{CJK} $P^{-1} A P$ \begin{CJK}{UTF8}{mj}为\end{CJK} $A$ \begin{CJK}{UTF8}{mj}的\end{CJK} Jordan \begin{CJK}{UTF8}{mj}标准形\end{CJK}.

\begin{CJK}{UTF8}{mj}六\end{CJK}. (10 \begin{CJK}{UTF8}{mj}分\end{CJK}) \begin{CJK}{UTF8}{mj}证明\end{CJK}: \begin{CJK}{UTF8}{mj}西矩阵的特征值模长为\end{CJK} 1 .

\begin{CJK}{UTF8}{mj}七\end{CJK}. (10 \begin{CJK}{UTF8}{mj}分\end{CJK}) \begin{CJK}{UTF8}{mj}设\end{CJK} $V$ \begin{CJK}{UTF8}{mj}是\end{CJK} $n$ \begin{CJK}{UTF8}{mj}维欧式空间\end{CJK}, $(,$, \begin{CJK}{UTF8}{mj}为其内积\end{CJK}, $V^{*}$ \begin{CJK}{UTF8}{mj}为其对偶空间\end{CJK}. \begin{CJK}{UTF8}{mj}证明\end{CJK}:

(1) \begin{CJK}{UTF8}{mj}对于每个给定的\end{CJK} $\alpha \in V$, \begin{CJK}{UTF8}{mj}映射\end{CJK} $f_{\alpha}: V \rightarrow \mathbb{R}, \beta \mapsto(\alpha, \beta)$ \begin{CJK}{UTF8}{mj}是\end{CJK} $V^{*}$ \begin{CJK}{UTF8}{mj}中的一个元素\end{CJK}.

(2) \begin{CJK}{UTF8}{mj}映射\end{CJK} $f: V \rightarrow V^{*}, \alpha \mapsto f_{\alpha}$ \begin{CJK}{UTF8}{mj}是\end{CJK} $n$ \begin{CJK}{UTF8}{mj}维线性空间\end{CJK} $V$ \begin{CJK}{UTF8}{mj}到\end{CJK} $V^{*}$ \begin{CJK}{UTF8}{mj}的同构映射\end{CJK}.

\begin{CJK}{UTF8}{mj}八\end{CJK}. ( 20 \begin{CJK}{UTF8}{mj}分\end{CJK}) \begin{CJK}{UTF8}{mj}设数域\end{CJK} $F$ \begin{CJK}{UTF8}{mj}上有限维空间\end{CJK} $V$ \begin{CJK}{UTF8}{mj}上线性变换\end{CJK} $\mathscr{A}$ \begin{CJK}{UTF8}{mj}和\end{CJK} $\mathscr{B}$ \begin{CJK}{UTF8}{mj}满足\end{CJK} $\mathscr{A} \mathscr{B}=a \mathscr{B} \mathscr{A}(a \in F, a \neq 1)$, \begin{CJK}{UTF8}{mj}且\end{CJK} $\mathscr{A}$ \begin{CJK}{UTF8}{mj}是可逆线性\end{CJK} \begin{CJK}{UTF8}{mj}变换\end{CJK}, \begin{CJK}{UTF8}{mj}证明\end{CJK}:

(1) $\mathscr{B}$ \begin{CJK}{UTF8}{mj}为幂零矩阵\end{CJK} (\begin{CJK}{UTF8}{mj}即存在正整数\end{CJK} $\left.n, \mathscr{B}^{n}=0\right)$.

(2) $\mathscr{A}$ \begin{CJK}{UTF8}{mj}和\end{CJK} $\mathscr{B}$ \begin{CJK}{UTF8}{mj}有一个公共特征向量\end{CJK}.

\section{3. 中国科学技术大学 2011 年研究生入学考试试题线性代数与解析 几何 
 李扬 
 微信公众号: sxkyliyang}
\begin{CJK}{UTF8}{mj}一\end{CJK}. \begin{CJK}{UTF8}{mj}填空题\end{CJK}(\begin{CJK}{UTF8}{mj}每小题\end{CJK} 5 \begin{CJK}{UTF8}{mj}分\end{CJK}, \begin{CJK}{UTF8}{mj}共\end{CJK} 50 \begin{CJK}{UTF8}{mj}分\end{CJK})

\begin{enumerate}
  \item \begin{CJK}{UTF8}{mj}两个平面\end{CJK} $z=x+2 y$ \begin{CJK}{UTF8}{mj}和\end{CJK} $z=-2 x-y$ \begin{CJK}{UTF8}{mj}的夹角等于\end{CJK}
\end{enumerate}
2 . \begin{CJK}{UTF8}{mj}点\end{CJK} $(0,2,1)$ \begin{CJK}{UTF8}{mj}到平面\end{CJK} $2 x-3 y+6 z=1$ \begin{CJK}{UTF8}{mj}的距离等于\end{CJK}

\begin{enumerate}
  \setcounter{enumi}{3}
  \item \begin{CJK}{UTF8}{mj}二次曲面\end{CJK} $x y+z^{2}=1$ \begin{CJK}{UTF8}{mj}的曲面类型是\end{CJK}
\end{enumerate}
$4 .\left(\begin{array}{lll}0 & 0 & 1 \\ 0 & 1 & 1 \\ 1 & 1 & 1\end{array}\right)^{-1}=$

\begin{enumerate}
  \setcounter{enumi}{5}
  \item \begin{CJK}{UTF8}{mj}设\end{CJK} $V$ \begin{CJK}{UTF8}{mj}是由\end{CJK} $A_{1}=\left(\begin{array}{cc}1 & 1 \\ 1 & -1\end{array}\right), A_{2}=\left(\begin{array}{cc}2 & 1 \\ 1 & -2\end{array}\right), A_{3}=\left(\begin{array}{cc}3 & 1 \\ 1 & -3\end{array}\right)$ \begin{CJK}{UTF8}{mj}生成的\end{CJK} $\mathbb{R}^{2 \times 2}$ \begin{CJK}{UTF8}{mj}的子空间\end{CJK}, \begin{CJK}{UTF8}{mj}则\end{CJK} $\operatorname{dim} V=$

  \item \begin{CJK}{UTF8}{mj}已知实线性空间\end{CJK} $V$ \begin{CJK}{UTF8}{mj}中的向量\end{CJK} $\alpha_{1}, \alpha_{2}, \alpha_{3}, \alpha_{4}$ \begin{CJK}{UTF8}{mj}线性无关\end{CJK}, \begin{CJK}{UTF8}{mj}则向量组\end{CJK} $\left\{\alpha_{1}+\alpha_{2}, \alpha_{2}+\alpha_{3}, \alpha_{3}+\alpha_{4}, \alpha_{4}+\alpha_{1}\right\}$ \begin{CJK}{UTF8}{mj}的秩等于\end{CJK}

  \item \begin{CJK}{UTF8}{mj}已知实方阵\end{CJK} $A=\left(\begin{array}{ccc}0 & 0 & a \\ 1 & 1 & 0 \\ 1 & 0 & 0\end{array}\right)$ \begin{CJK}{UTF8}{mj}与\end{CJK} $B=\left(\begin{array}{ccc}0 & 0 & a^{2} \\ 0 & 1 & 0 \\ 1 & 0 & 0\end{array}\right)$ \begin{CJK}{UTF8}{mj}相似\end{CJK}, \begin{CJK}{UTF8}{mj}则\end{CJK} $a=$

  \item $\left(\begin{array}{ccc}1 & 1 & 1 \\ 1 & \lambda & \lambda^{2} \\ 1 & \lambda^{2} & \lambda^{4}\end{array}\right)$ \begin{CJK}{UTF8}{mj}的初等因子组是\end{CJK}

  \item \begin{CJK}{UTF8}{mj}对\end{CJK} $\mathbb{R}^{4}$ \begin{CJK}{UTF8}{mj}中的向量\end{CJK} $\alpha_{1}=(1,0,1,0), \alpha_{2}=(0,-1,1,-1), \alpha_{3}=(1,1,1,1)$ \begin{CJK}{UTF8}{mj}作\end{CJK} Gram-Schmidt \begin{CJK}{UTF8}{mj}正交化和单位\end{CJK} \begin{CJK}{UTF8}{mj}化\end{CJK}, \begin{CJK}{UTF8}{mj}得到\end{CJK} $\beta_{1}, \beta_{2}, \beta_{3}$, \begin{CJK}{UTF8}{mj}则\end{CJK} $\beta_{3}=$

  \item \begin{CJK}{UTF8}{mj}已知实二次型\end{CJK} $Q(x, y, z)=a x^{2}+y^{2}+z^{2}+x y+y z+z x$ \begin{CJK}{UTF8}{mj}正定\end{CJK}, \begin{CJK}{UTF8}{mj}则实数\end{CJK} $a$ \begin{CJK}{UTF8}{mj}的取值范围是\end{CJK}

\end{enumerate}
\begin{CJK}{UTF8}{mj}二\end{CJK}. \begin{CJK}{UTF8}{mj}解答题\end{CJK}(\begin{CJK}{UTF8}{mj}共\end{CJK} 100 \begin{CJK}{UTF8}{mj}分\end{CJK}, \begin{CJK}{UTF8}{mj}请给出详细的计算和证明过程\end{CJK})

\begin{enumerate}
  \setcounter{enumi}{11}
  \item (15 \begin{CJK}{UTF8}{mj}分\end{CJK}) \begin{CJK}{UTF8}{mj}设点\end{CJK} $A(1,1,-1), B(-1,1,1), C(1,1,1)$, \begin{CJK}{UTF8}{mj}求\end{CJK} $\triangle A B C$ \begin{CJK}{UTF8}{mj}的外接圆的方程\end{CJK}.

  \item (15 \begin{CJK}{UTF8}{mj}分\end{CJK}) \begin{CJK}{UTF8}{mj}求线性方程组\end{CJK}

\end{enumerate}
$$
\left\{\begin{array}{l}
x_{1}+x_{2}+x_{3}+x_{4}+x_{5}=1 \\
3 x_{1}+2 x_{2}+x_{3}+x_{4}-3 x_{5}=-2 \\
x_{2}+2 x_{3}+2 x_{4}+6 x_{5}=5 \\
5 x_{1}+4 x_{2}+3 x_{3}+3 x_{4}-x_{5}=0
\end{array}\right.
$$
\begin{CJK}{UTF8}{mj}的通解\end{CJK}.

\begin{enumerate}
  \setcounter{enumi}{13}
  \item ( 15 \begin{CJK}{UTF8}{mj}分\end{CJK}) \begin{CJK}{UTF8}{mj}设\end{CJK} $n$ \begin{CJK}{UTF8}{mj}阶复方阵\end{CJK} $A$ \begin{CJK}{UTF8}{mj}的特征值全体为\end{CJK} $\lambda_{1}, \cdots, \lambda_{n}, f(x)$ \begin{CJK}{UTF8}{mj}是任意一个复系数多项式\end{CJK}. \begin{CJK}{UTF8}{mj}求证\end{CJK}: $f(A)$ \begin{CJK}{UTF8}{mj}的\end{CJK} \begin{CJK}{UTF8}{mj}特征值全体为\end{CJK} $f\left(\lambda_{1}\right), \cdots, f\left(\lambda_{n}\right)$.

  \item (15 \begin{CJK}{UTF8}{mj}分\end{CJK}) \begin{CJK}{UTF8}{mj}设\end{CJK} $\alpha_{1}, \cdots, \alpha_{n}$ \begin{CJK}{UTF8}{mj}是欧式空间\end{CJK} $V$ \begin{CJK}{UTF8}{mj}中任意\end{CJK} $n$ \begin{CJK}{UTF8}{mj}个向量\end{CJK}, $n \geqslant 1, G=\left(\begin{array}{cccc}\left(\alpha_{1}, \alpha_{1}\right) & \left(\alpha_{1}, \alpha_{2}\right) & \cdots & \left(\alpha_{1}, \alpha_{n}\right) \\ \left(\alpha_{2}, \alpha_{1}\right) & \left(\alpha_{2}, \alpha_{2}\right) & \cdots & \left(\alpha_{2}, \alpha_{n}\right) \\ \vdots & \vdots & \vdots \\ \left(\alpha_{n}, \alpha_{1}\right) & \left(\alpha_{n}, \alpha_{2}\right) & \cdots & \left(\alpha_{n}, \alpha_{n}\right)\end{array}\right)$, \begin{CJK}{UTF8}{mj}其中\end{CJK} $\left(\alpha_{i}, \alpha_{j}\right)$ \begin{CJK}{UTF8}{mj}是\end{CJK} $V$ \begin{CJK}{UTF8}{mj}的内积\end{CJK}.

\end{enumerate}
\begin{CJK}{UTF8}{mj}求证\end{CJK}: $G$ \begin{CJK}{UTF8}{mj}正定的充分必要条件是\end{CJK} $\alpha_{1}, \cdots, \alpha_{n}$ \begin{CJK}{UTF8}{mj}线性无关\end{CJK}. 15. (20 \begin{CJK}{UTF8}{mj}分\end{CJK}) \begin{CJK}{UTF8}{mj}设\end{CJK} $\mathscr{A}$ \begin{CJK}{UTF8}{mj}是无限维线性空间\end{CJK} $V$ \begin{CJK}{UTF8}{mj}的线性变换\end{CJK}, $\mathscr{B}$ \begin{CJK}{UTF8}{mj}是\end{CJK} $\mathscr{A}$ \begin{CJK}{UTF8}{mj}在\end{CJK} $\operatorname{Im} \mathscr{A}$ \begin{CJK}{UTF8}{mj}上的限制变换\end{CJK}. \begin{CJK}{UTF8}{mj}求证\end{CJK}: $V=\operatorname{Im} \mathscr{A} \oplus \mathrm{Ker} \mathscr{A}$ \begin{CJK}{UTF8}{mj}的充分必要条件是\end{CJK} $\mathscr{B}$ \begin{CJK}{UTF8}{mj}可逆\end{CJK}.

\begin{enumerate}
  \setcounter{enumi}{16}
  \item ( 20 \begin{CJK}{UTF8}{mj}分\end{CJK}) \begin{CJK}{UTF8}{mj}已知\end{CJK} $\mathbb{R}^{2}$ \begin{CJK}{UTF8}{mj}的线性变换\end{CJK} $\mathscr{A}$ \begin{CJK}{UTF8}{mj}把\end{CJK} $(1,0)$ \begin{CJK}{UTF8}{mj}映射到\end{CJK} $(0,1)$, \begin{CJK}{UTF8}{mj}把\end{CJK} $(0,1)$ \begin{CJK}{UTF8}{mj}映射到\end{CJK} $(2,1)$, \begin{CJK}{UTF8}{mj}并且把圆\end{CJK} $C: x^{2}+y^{2}=1$ \begin{CJK}{UTF8}{mj}映射成椭圆\end{CJK} $E$. \begin{CJK}{UTF8}{mj}求\end{CJK}:\\
(1) $E$ \begin{CJK}{UTF8}{mj}的方程\end{CJK};\\
(2) $E$ \begin{CJK}{UTF8}{mj}的长轴所在直线的方程\end{CJK};\\
(3) $E$ \begin{CJK}{UTF8}{mj}的面积\end{CJK}.
\end{enumerate}
\section{4. 中国科学技术大学 2012 年研究生入学考试试题线性代数与解析 几何 
 李扬 
 微信公众号: sxkyliyang}
\begin{CJK}{UTF8}{mj}一\end{CJK}. \begin{CJK}{UTF8}{mj}填空题\end{CJK}(\begin{CJK}{UTF8}{mj}每空\end{CJK} 5 \begin{CJK}{UTF8}{mj}分\end{CJK}, \begin{CJK}{UTF8}{mj}共\end{CJK} 50 \begin{CJK}{UTF8}{mj}分\end{CJK})

\begin{enumerate}
  \item \begin{CJK}{UTF8}{mj}在\end{CJK} $\mathbb{R}^{3}$ \begin{CJK}{UTF8}{mj}中\end{CJK}, \begin{CJK}{UTF8}{mj}直线\end{CJK} $x=y=z$ \begin{CJK}{UTF8}{mj}与平面\end{CJK} $z=x-y$ \begin{CJK}{UTF8}{mj}的夹角的余弦值等于\end{CJK}

  \item \begin{CJK}{UTF8}{mj}在\end{CJK} $\mathbb{R}^{3}$ \begin{CJK}{UTF8}{mj}中\end{CJK}, \begin{CJK}{UTF8}{mj}方程\end{CJK} $x y-y z+z x=1$ \begin{CJK}{UTF8}{mj}所表示的二次曲面类型为\end{CJK}

  \item \begin{CJK}{UTF8}{mj}在\end{CJK} $\mathbb{R}^{4}$ \begin{CJK}{UTF8}{mj}中\end{CJK}, \begin{CJK}{UTF8}{mj}设三点\end{CJK} $A, B, C$ \begin{CJK}{UTF8}{mj}的坐标分别为\end{CJK} $A(1,0,1,0), B(0,1,0,1), C(1,1,1,1)$, \begin{CJK}{UTF8}{mj}则\end{CJK} $\triangle A B C$ \begin{CJK}{UTF8}{mj}的面积等于\end{CJK}

  \item \begin{CJK}{UTF8}{mj}满足\end{CJK} $f(-1)=0, f(1)=4, f(2)=3, f(3)=16$ \begin{CJK}{UTF8}{mj}的次数最小的一元多项式\end{CJK} $f(x)=$

  \item \begin{CJK}{UTF8}{mj}使线性方程组\end{CJK}

\end{enumerate}
$$
\left\{\begin{array}{l}
a^{2} x_{1}+x_{2}+x_{3}=1 \\
x_{1}+a x_{2}+x_{3}=a \\
x_{1}+x_{2}+x_{3}=a^{2}
\end{array}\right.
$$
\begin{CJK}{UTF8}{mj}有解的实数\end{CJK} $a$ \begin{CJK}{UTF8}{mj}的取值范围是\end{CJK}

\begin{enumerate}
  \setcounter{enumi}{6}
  \item \begin{CJK}{UTF8}{mj}已知实方阵\end{CJK} $A$ \begin{CJK}{UTF8}{mj}的伴随矩阵\end{CJK} $A^{*}=\left(\begin{array}{llll}2 & 1 & 1 & 1 \\ 1 & 1 & 0 & 0 \\ 1 & 0 & 1 & 0 \\ 1 & 0 & 0 & 1\end{array}\right)$, \begin{CJK}{UTF8}{mj}则\end{CJK} $A=$

  \item \begin{CJK}{UTF8}{mj}已知复方阵\end{CJK} $A$ \begin{CJK}{UTF8}{mj}的特征方阵\end{CJK} $\lambda I-A$ \begin{CJK}{UTF8}{mj}的初等因子组为\end{CJK}

\end{enumerate}
$$
\left\{\lambda, \lambda+1, \lambda^{2}, \lambda^{2},(\lambda-1)^{2},(\lambda-1)^{3}\right\},
$$
\begin{CJK}{UTF8}{mj}则\end{CJK} $A$ \begin{CJK}{UTF8}{mj}的最小多项式\end{CJK} $d_{A}(\lambda)=$ , $\operatorname{rank}(A)=$ , $\operatorname{tr}(A)=$

\begin{enumerate}
  \setcounter{enumi}{8}
  \item \begin{CJK}{UTF8}{mj}设\end{CJK} $n \geqslant 2$, \begin{CJK}{UTF8}{mj}则实二次型\end{CJK} $Q\left(x_{1}, \cdots, x_{n}\right)=\sum_{i=1}^{n} x_{i}^{2}-\left(\sum_{i=1}^{n} x_{i}\right)^{2}$ \begin{CJK}{UTF8}{mj}的规范型为\end{CJK}
\end{enumerate}
\begin{CJK}{UTF8}{mj}二\end{CJK}. \begin{CJK}{UTF8}{mj}解答题\end{CJK}(\begin{CJK}{UTF8}{mj}共\end{CJK} 100 \begin{CJK}{UTF8}{mj}分\end{CJK}, \begin{CJK}{UTF8}{mj}需给出详细的计算和证明过程\end{CJK})

\begin{enumerate}
  \setcounter{enumi}{9}
  \item (15 \begin{CJK}{UTF8}{mj}分\end{CJK}) \begin{CJK}{UTF8}{mj}求\end{CJK} $\mathbb{R}^{3}$ \begin{CJK}{UTF8}{mj}中直线\end{CJK} $x-1=y-2=z-3$ \begin{CJK}{UTF8}{mj}与\end{CJK} $x=2 y=3 z$ \begin{CJK}{UTF8}{mj}的公垂线方程\end{CJK}.

  \item ( 15 \begin{CJK}{UTF8}{mj}分\end{CJK}) \begin{CJK}{UTF8}{mj}已知\end{CJK} $W_{1}, W_{2}$ \begin{CJK}{UTF8}{mj}是数域\end{CJK} $F$ \begin{CJK}{UTF8}{mj}上\end{CJK} $n$ \begin{CJK}{UTF8}{mj}维线性空间\end{CJK} $V$ \begin{CJK}{UTF8}{mj}的两个子空间\end{CJK}. \begin{CJK}{UTF8}{mj}求证\end{CJK}:

\end{enumerate}
$$
\operatorname{dim}\left(W_{1} \cap W_{2}\right)+\operatorname{dim}\left(W_{1}+W_{2}\right)=\operatorname{dim} W_{1}+\operatorname{dim} W_{2}
$$

\begin{enumerate}
  \setcounter{enumi}{11}
  \item ( 20 \begin{CJK}{UTF8}{mj}分\end{CJK}) \begin{CJK}{UTF8}{mj}设\end{CJK} $\mathscr{A}$ \begin{CJK}{UTF8}{mj}是数域\end{CJK} $F$ \begin{CJK}{UTF8}{mj}上\end{CJK} $n$ \begin{CJK}{UTF8}{mj}维线性空间\end{CJK} $V$ \begin{CJK}{UTF8}{mj}的线性变换\end{CJK}. \begin{CJK}{UTF8}{mj}已知\end{CJK} $\mathscr{A}$ \begin{CJK}{UTF8}{mj}的特征多项式\end{CJK} $\varphi_{\mathscr{A}}(\lambda)=f(\lambda) \cdot g(\lambda)$, \begin{CJK}{UTF8}{mj}其\end{CJK} \begin{CJK}{UTF8}{mj}中\end{CJK} $f(\lambda)$ \begin{CJK}{UTF8}{mj}与\end{CJK} $g(\lambda)$ \begin{CJK}{UTF8}{mj}是数域\end{CJK} $F$ \begin{CJK}{UTF8}{mj}上的两个互素的多项式\end{CJK}. \begin{CJK}{UTF8}{mj}求证\end{CJK}:
\end{enumerate}
(1) $\operatorname{Im} f(\mathscr{A})=\operatorname{Ker} g(\mathscr{A})$;

(2) $V=\operatorname{Im} f(\mathscr{A}) \oplus \operatorname{Im} g(\mathscr{A})$.

\begin{enumerate}
  \setcounter{enumi}{12}
  \item (15 \begin{CJK}{UTF8}{mj}分\end{CJK}) \begin{CJK}{UTF8}{mj}设\end{CJK} $Q(\mathbf{x})$ \begin{CJK}{UTF8}{mj}是\end{CJK} $n$ \begin{CJK}{UTF8}{mj}元实二次型\end{CJK}, $V=\left\{\mathbf{x} \in \mathbb{R}^{n} \mid Q(\mathbf{x})=0\right\}$. \begin{CJK}{UTF8}{mj}求证\end{CJK}: $V$ \begin{CJK}{UTF8}{mj}是\end{CJK} $\mathbb{R}^{n}$ \begin{CJK}{UTF8}{mj}的子空间\end{CJK} $\Leftrightarrow Q(\mathbf{x})$ \begin{CJK}{UTF8}{mj}是半正定或半负定的\end{CJK}.

  \item ( 20 \begin{CJK}{UTF8}{mj}分\end{CJK}) \begin{CJK}{UTF8}{mj}设\end{CJK} $\mathbb{R}^{2 \times 2}$ \begin{CJK}{UTF8}{mj}上的线性变换\end{CJK} $\mathscr{A}(X)=A X-X A$, \begin{CJK}{UTF8}{mj}其中\end{CJK} $A=\left(\begin{array}{ll}1 & 1 \\ 1 & 2\end{array}\right)$.

\end{enumerate}
(1) \begin{CJK}{UTF8}{mj}求证\end{CJK}: $f(X, Y)=\operatorname{tr}\left(X^{T} A Y\right)$ \begin{CJK}{UTF8}{mj}是\end{CJK} $\mathbb{R}^{2 \times 2}$ \begin{CJK}{UTF8}{mj}上的内积\end{CJK}; (2) \begin{CJK}{UTF8}{mj}求\end{CJK} $\operatorname{Im} \mathscr{A}$ \begin{CJK}{UTF8}{mj}在\end{CJK} $f$ \begin{CJK}{UTF8}{mj}下的一组标准正交基\end{CJK}.

\begin{enumerate}
  \setcounter{enumi}{14}
  \item (15 \begin{CJK}{UTF8}{mj}分\end{CJK}) \begin{CJK}{UTF8}{mj}设\end{CJK} $n \geqslant 2$, \begin{CJK}{UTF8}{mj}求如下\end{CJK} $n$ \begin{CJK}{UTF8}{mj}阶实方阵\end{CJK} $A=\left(a_{i j}\right)_{n \times n}$ \begin{CJK}{UTF8}{mj}的\end{CJK} Jordan \begin{CJK}{UTF8}{mj}标准形\end{CJK}.
\end{enumerate}
$$
A=\left(\begin{array}{cccc}
O & & 1 & 1 \\
& \ddots & \ddots & \\
1 & . \cdot & & \\
1 & & & O
\end{array}\right)
$$
\begin{CJK}{UTF8}{mj}即\end{CJK}
$$
a_{i j}= \begin{cases}1, & i+j \in\{n, n+1\} \\ 0, & i+j \notin\{n, n+1\}\end{cases}
$$

\section{5. 中国科学技术大学 2013 年研究生入学考试试题线性代数与解析 几何}
\begin{CJK}{UTF8}{mj}李扬\end{CJK}

\begin{CJK}{UTF8}{mj}微信公众号\end{CJK}: sxkyliyang

\begin{CJK}{UTF8}{mj}一\end{CJK}. \begin{CJK}{UTF8}{mj}填空题\end{CJK}(\begin{CJK}{UTF8}{mj}每空\end{CJK} 6 \begin{CJK}{UTF8}{mj}分\end{CJK}, \begin{CJK}{UTF8}{mj}共\end{CJK} 60 \begin{CJK}{UTF8}{mj}分\end{CJK}. \begin{CJK}{UTF8}{mj}答案需化简\end{CJK})

\begin{enumerate}
  \item \begin{CJK}{UTF8}{mj}两直线\end{CJK} $1-x=2 y=3 z$ \begin{CJK}{UTF8}{mj}与\end{CJK} $x=y+2=2 z+4$ \begin{CJK}{UTF8}{mj}的夹角为\end{CJK} , \begin{CJK}{UTF8}{mj}距离为\end{CJK}

  \item \begin{CJK}{UTF8}{mj}当实数\end{CJK} $a, b, c$ \begin{CJK}{UTF8}{mj}满足\end{CJK} \begin{CJK}{UTF8}{mj}时\end{CJK}, \begin{CJK}{UTF8}{mj}曲面\end{CJK} $z=a x^{2}+b x y+c y^{2}$ \begin{CJK}{UTF8}{mj}是椭圆抛物面\end{CJK}.

  \item \begin{CJK}{UTF8}{mj}实方阵\end{CJK} $\left(\begin{array}{llll}1 & 0 & 1 & 0 \\ 1 & 1 & 0 & 1 \\ 0 & 0 & 1 & 1 \\ 0 & 0 & 0 & 1\end{array}\right)$ \begin{CJK}{UTF8}{mj}的伴随方阵为\end{CJK}

  \item \begin{CJK}{UTF8}{mj}设\end{CJK} $V$ \begin{CJK}{UTF8}{mj}是次数\end{CJK} $\leqslant 3$ \begin{CJK}{UTF8}{mj}的实系数多项式\end{CJK} $f(x)$ \begin{CJK}{UTF8}{mj}全体在多项式的加法和数乘运算下构成的实数域上的线性\end{CJK} \begin{CJK}{UTF8}{mj}空间\end{CJK}. \begin{CJK}{UTF8}{mj}从\end{CJK} $V$ \begin{CJK}{UTF8}{mj}的基\end{CJK} $1, x-1,(x-1)^{2},(x-1)^{3}$ \begin{CJK}{UTF8}{mj}到\end{CJK} $1, x, x^{2}, x^{3}$ \begin{CJK}{UTF8}{mj}的过渡矩阵为\end{CJK} $V$ \begin{CJK}{UTF8}{mj}上的线性变换\end{CJK} $\mathscr{A}: f(x) \mapsto x f^{\prime}(x)$ \begin{CJK}{UTF8}{mj}在\end{CJK} $1, x-1,(x-1)^{2},(x-1)^{3}$ \begin{CJK}{UTF8}{mj}下的矩阵为\end{CJK} , $\mathscr{A}$ \begin{CJK}{UTF8}{mj}的最小多项式为\end{CJK}

  \item \begin{CJK}{UTF8}{mj}多项式矩阵\end{CJK} $\left(\begin{array}{llll}1 & 1 & 1 & \lambda \\ 1 & 1 & \lambda & 1 \\ 1 & \lambda & 1 & 1 \\ \lambda & 1 & 1 & 1\end{array}\right)$ \begin{CJK}{UTF8}{mj}的初等因子组为\end{CJK}

  \item \begin{CJK}{UTF8}{mj}实二次型\end{CJK} $Q\left(x_{1}, x_{2}, x_{3}, x_{4}\right)=x_{1} x_{2}+x_{1} x_{3}+x_{1} x_{4}+x_{2} x_{3}$ \begin{CJK}{UTF8}{mj}的规范型为\end{CJK}

\end{enumerate}
\begin{CJK}{UTF8}{mj}二\end{CJK}. \begin{CJK}{UTF8}{mj}解答题\end{CJK}(\begin{CJK}{UTF8}{mj}共\end{CJK} 90 \begin{CJK}{UTF8}{mj}分\end{CJK}. \begin{CJK}{UTF8}{mj}需写出详细的解答过程\end{CJK})

\begin{enumerate}
  \setcounter{enumi}{7}
  \item (15 \begin{CJK}{UTF8}{mj}分\end{CJK}) \begin{CJK}{UTF8}{mj}求\end{CJK} $x$ \begin{CJK}{UTF8}{mj}轴绕直线\end{CJK} $x=y=z-1$ \begin{CJK}{UTF8}{mj}旋转所得旋转曲面的一般方程\end{CJK}.

  \item ( 15 \begin{CJK}{UTF8}{mj}分\end{CJK}) \begin{CJK}{UTF8}{mj}设\end{CJK} $\mathscr{A}$ \begin{CJK}{UTF8}{mj}是数域\end{CJK} $F$ \begin{CJK}{UTF8}{mj}上的有限维线性空间\end{CJK} $V$ \begin{CJK}{UTF8}{mj}上的线性变换\end{CJK}, \begin{CJK}{UTF8}{mj}求证\end{CJK}:

\end{enumerate}
$$
\operatorname{dim}(\operatorname{Im} \mathscr{A} \cap \operatorname{ker} \mathscr{A})=\operatorname{rank} \mathscr{A}-\operatorname{rank} \mathscr{A}^{2}
$$

\begin{enumerate}
  \setcounter{enumi}{9}
  \item ( 20 \begin{CJK}{UTF8}{mj}分\end{CJK}) \begin{CJK}{UTF8}{mj}证明\end{CJK}: Cayley-Hamilton \begin{CJK}{UTF8}{mj}定理\end{CJK}: \begin{CJK}{UTF8}{mj}数域\end{CJK} $F$ \begin{CJK}{UTF8}{mj}上的任意方阵\end{CJK} $A$ \begin{CJK}{UTF8}{mj}的特征多项式都是\end{CJK} $A$ \begin{CJK}{UTF8}{mj}的零化多项式\end{CJK}.

  \item ( 20 \begin{CJK}{UTF8}{mj}分\end{CJK}) \begin{CJK}{UTF8}{mj}设\end{CJK} $A, B$ \begin{CJK}{UTF8}{mj}是\end{CJK} $n$ \begin{CJK}{UTF8}{mj}阶复方阵\end{CJK}, $\mathbb{C}^{n \times n}$ \begin{CJK}{UTF8}{mj}上的线性变换\end{CJK} $\mathscr{A}: X \mapsto A X-X B$. \begin{CJK}{UTF8}{mj}求证\end{CJK}: $\mathscr{A}$ \begin{CJK}{UTF8}{mj}可逆的充要条件是\end{CJK} $A$ \begin{CJK}{UTF8}{mj}与\end{CJK} $B$ \begin{CJK}{UTF8}{mj}无公共的特征值\end{CJK}.

  \item ( 20 \begin{CJK}{UTF8}{mj}分\end{CJK}) \begin{CJK}{UTF8}{mj}设\end{CJK} $\mathscr{A}$ \begin{CJK}{UTF8}{mj}是数域\end{CJK} $F$ \begin{CJK}{UTF8}{mj}上的线性空间\end{CJK} $V$ \begin{CJK}{UTF8}{mj}上的线性变换\end{CJK}, $f_{1}(x)$ \begin{CJK}{UTF8}{mj}和\end{CJK} $f_{2}(x)$ \begin{CJK}{UTF8}{mj}是\end{CJK} $F$ \begin{CJK}{UTF8}{mj}上的多项式\end{CJK}, $g(x)$ \begin{CJK}{UTF8}{mj}是\end{CJK} $f_{1}(x)$ \begin{CJK}{UTF8}{mj}和\end{CJK} $f_{2}(x)$ \begin{CJK}{UTF8}{mj}的最大公因式\end{CJK}, $h(x)$ \begin{CJK}{UTF8}{mj}是\end{CJK} $f_{1}(x)$ \begin{CJK}{UTF8}{mj}和\end{CJK} $f_{2}(x)$ \begin{CJK}{UTF8}{mj}的最小公倍式\end{CJK}. \begin{CJK}{UTF8}{mj}求证\end{CJK}: $\operatorname{Kerh}(\mathscr{A})=\operatorname{Ker} f_{1}(\mathscr{A}) \oplus \operatorname{Ker} f_{2}(\mathscr{A})$ \begin{CJK}{UTF8}{mj}的充要条件是\end{CJK} $g(\mathscr{A})$ \begin{CJK}{UTF8}{mj}可逆\end{CJK}.

\end{enumerate}
\section{6. 中国科学技术大学 2014 年研究生入学考试试题线性代数与解析 几何 
 李扬 
 微信公众号: sxkyliyang}
\begin{CJK}{UTF8}{mj}一\end{CJK}. \begin{CJK}{UTF8}{mj}填空题\end{CJK}(\begin{CJK}{UTF8}{mj}每空\end{CJK} 6 \begin{CJK}{UTF8}{mj}分\end{CJK}, \begin{CJK}{UTF8}{mj}共\end{CJK} 60 \begin{CJK}{UTF8}{mj}分\end{CJK}. \begin{CJK}{UTF8}{mj}需化简答案\end{CJK})

\begin{enumerate}
  \item \begin{CJK}{UTF8}{mj}原点到直线\end{CJK} $x+1=y+2=z+3$ \begin{CJK}{UTF8}{mj}的距离为\end{CJK}

  \item \begin{CJK}{UTF8}{mj}设点\end{CJK} $P(1,2,3)$ \begin{CJK}{UTF8}{mj}与原点关于平面\end{CJK} $\pi$ \begin{CJK}{UTF8}{mj}对称\end{CJK}, \begin{CJK}{UTF8}{mj}则\end{CJK} $\pi$ \begin{CJK}{UTF8}{mj}的方程为\end{CJK}

  \item \begin{CJK}{UTF8}{mj}椭圆\end{CJK} $x^{2}+x y+y^{2}=1$ \begin{CJK}{UTF8}{mj}的离心率为\end{CJK}

  \item \begin{CJK}{UTF8}{mj}设\end{CJK} $f(x)=x^{2}-2 a x+2$ \begin{CJK}{UTF8}{mj}整除\end{CJK} $g(x)=x^{4}+3 x^{2}+a x+2$, \begin{CJK}{UTF8}{mj}则常数\end{CJK} $a=$

  \item \begin{CJK}{UTF8}{mj}设\end{CJK} $A=\left(\begin{array}{cccc}0 & 1 & 0 & 1 \\ -1 & 0 & 1 & 0 \\ 0 & -1 & 0 & 1 \\ -1 & 0 & -1 & 0\end{array}\right)$, \begin{CJK}{UTF8}{mj}则\end{CJK} $A^{-1}=$

  \item \begin{CJK}{UTF8}{mj}设\end{CJK} $V$ \begin{CJK}{UTF8}{mj}是实数域上的线性空间\end{CJK}, $V$ \begin{CJK}{UTF8}{mj}中的向量组\end{CJK} $\left\{\alpha_{1}, \alpha_{2}, \cdots, \alpha_{2014}\right\}$ \begin{CJK}{UTF8}{mj}线性无关\end{CJK}, \begin{CJK}{UTF8}{mj}则向量组\end{CJK} $\left\{\alpha_{i}+\alpha_{j} \mid 1 \leqslant i<\right.$ $j \leqslant 2014\}$ \begin{CJK}{UTF8}{mj}的秩为\end{CJK}

  \item \begin{CJK}{UTF8}{mj}设方阵\end{CJK} $A$ \begin{CJK}{UTF8}{mj}的最小多项式\end{CJK} $d_{A}(x)=x^{3}(x+1)^{3}$, \begin{CJK}{UTF8}{mj}则\end{CJK} $B=A^{2}$ \begin{CJK}{UTF8}{mj}的最小多项式\end{CJK} $d_{B}(x)=$

  \item \begin{CJK}{UTF8}{mj}多项式矩阵\end{CJK} $\left(\begin{array}{ccc}x-x^{2} & 0 & 0 \\ 0 & x+x^{2} & 0 \\ 0 & 0 & 1-x^{2}\end{array}\right)$ \begin{CJK}{UTF8}{mj}的\end{CJK} Smith \begin{CJK}{UTF8}{mj}标准形是\end{CJK}

  \item \begin{CJK}{UTF8}{mj}设实二次型\end{CJK} $Q(x, y, z)=a x^{2}+y^{2}+z^{2}+a x y-x z$ \begin{CJK}{UTF8}{mj}正定\end{CJK}, \begin{CJK}{UTF8}{mj}则常数\end{CJK} $a$ \begin{CJK}{UTF8}{mj}的取值范围是\end{CJK}

\end{enumerate}
\begin{CJK}{UTF8}{mj}二\end{CJK}. \begin{CJK}{UTF8}{mj}解答题\end{CJK}(\begin{CJK}{UTF8}{mj}共\end{CJK} 90 \begin{CJK}{UTF8}{mj}分\end{CJK}. \begin{CJK}{UTF8}{mj}需写出详细的解答过程\end{CJK})

\begin{enumerate}
  \item (15 \begin{CJK}{UTF8}{mj}分\end{CJK}) \begin{CJK}{UTF8}{mj}设\end{CJK} $\mathbb{R}^{n \times n}$ \begin{CJK}{UTF8}{mj}上的线性变换\end{CJK} $\mathscr{A}(X)=A X A^{T}$, \begin{CJK}{UTF8}{mj}其中\end{CJK} $A$ \begin{CJK}{UTF8}{mj}是\end{CJK} $n$ \begin{CJK}{UTF8}{mj}阶实方阵\end{CJK}, $\operatorname{rank}(A)=r$, \begin{CJK}{UTF8}{mj}求\end{CJK} $\operatorname{Im} \mathscr{A}$ \begin{CJK}{UTF8}{mj}的维数\end{CJK} \begin{CJK}{UTF8}{mj}及其一组基\end{CJK}.

  \item ( 15 \begin{CJK}{UTF8}{mj}分\end{CJK}) \begin{CJK}{UTF8}{mj}设\end{CJK} $\mathscr{A}$ \begin{CJK}{UTF8}{mj}是线性空间\end{CJK} $V$ \begin{CJK}{UTF8}{mj}上的线性变换\end{CJK}. \begin{CJK}{UTF8}{mj}求证\end{CJK}: \begin{CJK}{UTF8}{mj}存在\end{CJK} $V$ \begin{CJK}{UTF8}{mj}的子空间\end{CJK} $W$ \begin{CJK}{UTF8}{mj}与\end{CJK} $\mathrm{Im} \mathscr{A}$ \begin{CJK}{UTF8}{mj}同构\end{CJK}, \begin{CJK}{UTF8}{mj}并且\end{CJK} $V=$ $W \oplus \operatorname{Ker} \mathscr{A} .$

  \item ( 20 \begin{CJK}{UTF8}{mj}分\end{CJK}) \begin{CJK}{UTF8}{mj}设\end{CJK} $A$ \begin{CJK}{UTF8}{mj}是数域\end{CJK} $F$ \begin{CJK}{UTF8}{mj}上的\end{CJK} $n$ \begin{CJK}{UTF8}{mj}阶方阵\end{CJK}, \begin{CJK}{UTF8}{mj}向量\end{CJK} $\alpha_{i}$ \begin{CJK}{UTF8}{mj}满足\end{CJK} $\left(\lambda_{i} I-A\right)^{n} \alpha_{i}=0, i=1,2$. \begin{CJK}{UTF8}{mj}求证\end{CJK}: \begin{CJK}{UTF8}{mj}若\end{CJK} $\lambda_{1} \neq \lambda_{2}$, \begin{CJK}{UTF8}{mj}则\end{CJK} $F[A]\left(\alpha_{1}+\alpha_{2}\right)=F[A] \alpha_{1} \oplus F[A] \alpha_{2} .$

\end{enumerate}
(\begin{CJK}{UTF8}{mj}注\end{CJK}: $F[A] \alpha=\{f(A) \alpha \mid f(x) \in F[x]\}$.)

\begin{enumerate}
  \setcounter{enumi}{4}
  \item ( 20 \begin{CJK}{UTF8}{mj}分\end{CJK}) \begin{CJK}{UTF8}{mj}已知欧式空间\end{CJK} $V$ \begin{CJK}{UTF8}{mj}上的非零线性变换\end{CJK} $\mathscr{A}$ \begin{CJK}{UTF8}{mj}保持向量的夹角不变\end{CJK}. \begin{CJK}{UTF8}{mj}求证\end{CJK}: \begin{CJK}{UTF8}{mj}存在正实数\end{CJK} $\lambda$ \begin{CJK}{UTF8}{mj}使得\end{CJK} $\lambda \mathscr{A}$ \begin{CJK}{UTF8}{mj}是\end{CJK} \begin{CJK}{UTF8}{mj}正交变换\end{CJK}.

  \item ( 20 \begin{CJK}{UTF8}{mj}分\end{CJK}) \begin{CJK}{UTF8}{mj}设复方阵\end{CJK} $A=\left(\begin{array}{lll}0 & 1 & \mathrm{i} \\ \mathrm{i} & 0 & 1 \\ 1 & \mathrm{i} & 0\end{array}\right)$, i \begin{CJK}{UTF8}{mj}是虚数单位\end{CJK}. \begin{CJK}{UTF8}{mj}求西方阵\end{CJK} $P$ \begin{CJK}{UTF8}{mj}使得\end{CJK} $P A$ \begin{CJK}{UTF8}{mj}是上三角方阵并且\end{CJK} $P A$ \begin{CJK}{UTF8}{mj}的\end{CJK} \begin{CJK}{UTF8}{mj}对角线元素都是正实数\end{CJK}.

\end{enumerate}
\section{7. 中国科学技术大学 2015 年研究生入学考试试题线性代数与解析 几何 
 李扬 
 微信公众号: sxkyliyang}
\begin{CJK}{UTF8}{mj}一\end{CJK}. \begin{CJK}{UTF8}{mj}填空题\end{CJK}(\begin{CJK}{UTF8}{mj}每空\end{CJK} 6 \begin{CJK}{UTF8}{mj}分\end{CJK}, \begin{CJK}{UTF8}{mj}共\end{CJK} 60 \begin{CJK}{UTF8}{mj}分\end{CJK}. \begin{CJK}{UTF8}{mj}需化简答案\end{CJK})

\begin{enumerate}
  \item \begin{CJK}{UTF8}{mj}点\end{CJK} $(1,0,1)$ \begin{CJK}{UTF8}{mj}关于直线\end{CJK} $2 x-1=y-1=2 z-1$ \begin{CJK}{UTF8}{mj}的对称点为\end{CJK}

  \item \begin{CJK}{UTF8}{mj}设直线\end{CJK} $l$ \begin{CJK}{UTF8}{mj}过点\end{CJK} $(3,7,-8)$, \begin{CJK}{UTF8}{mj}并且与直线\end{CJK} $l_{1}: x=\frac{y}{2}=-\frac{z}{3}$ \begin{CJK}{UTF8}{mj}以及直线\end{CJK} $l_{2}: \frac{x-1}{2}=y+2=z-3$ \begin{CJK}{UTF8}{mj}均有交点\end{CJK}. \begin{CJK}{UTF8}{mj}则\end{CJK} $l$ \begin{CJK}{UTF8}{mj}与\end{CJK} $l_{1}$ \begin{CJK}{UTF8}{mj}的交点为\end{CJK} ,$l$ \begin{CJK}{UTF8}{mj}与\end{CJK} $l_{2}$ \begin{CJK}{UTF8}{mj}的交点为\end{CJK}

  \item \begin{CJK}{UTF8}{mj}二次型\end{CJK} $x_{1} x_{2}-3 x_{2} x_{3}+x_{1} x_{3}$ \begin{CJK}{UTF8}{mj}的正惯性指数为\end{CJK}

\end{enumerate}
\includegraphics[max width=\textwidth]{2022_04_18_3416d289b173eb9de8c1g-167}

\begin{enumerate}
  \setcounter{enumi}{5}
  \item \begin{CJK}{UTF8}{mj}在线性空间\end{CJK} $M_{3}(\mathbb{R})$ \begin{CJK}{UTF8}{mj}中\end{CJK}(\begin{CJK}{UTF8}{mj}运算为矩阵的加法和数乘\end{CJK}), \begin{CJK}{UTF8}{mj}考虑线性子空间\end{CJK}
\end{enumerate}
$$
V=\left\{A \in M_{3}(\mathbb{R}) \mid A\left(\begin{array}{lll}
1 & 1 & 2 \\
0 & 1 & 1 \\
0 & 0 & 1
\end{array}\right)=\left(\begin{array}{lll}
1 & 1 & 2 \\
0 & 1 & 1 \\
0 & 0 & 1
\end{array}\right) A\right\}
$$
\begin{CJK}{UTF8}{mj}则维数\end{CJK} $\operatorname{dim} V=$

\begin{enumerate}
  \setcounter{enumi}{6}
  \item \begin{CJK}{UTF8}{mj}设\end{CJK} $A, B, C$ \begin{CJK}{UTF8}{mj}为\end{CJK} $n$ \begin{CJK}{UTF8}{mj}阶矩阵\end{CJK}, $I_{n}$ \begin{CJK}{UTF8}{mj}表示\end{CJK} $n$ \begin{CJK}{UTF8}{mj}阶单位矩阵\end{CJK}, \begin{CJK}{UTF8}{mj}则矩阵\end{CJK} $\left(\begin{array}{ccc}I_{n} & A & C \\ 0 & I_{n} & B \\ 0 & 0 & I_{n}\end{array}\right)$ \begin{CJK}{UTF8}{mj}的逆矩阵为\end{CJK}

  \item \begin{CJK}{UTF8}{mj}考虑矩阵\end{CJK} $A=\left(\begin{array}{cccc}1 & 1 & 1 & 1 \\ 1 & -1 & -1 & 1 \\ 1 & 1 & -1 & -1 \\ 1 & -1 & 1 & -1\end{array}\right)$, \begin{CJK}{UTF8}{mj}则\end{CJK} $A_{11}-2 A_{21}+3 A_{31}-4 A_{41}=\ldots$, \begin{CJK}{UTF8}{mj}其中\end{CJK} $A_{i j}$ \begin{CJK}{UTF8}{mj}表示相应的代数\end{CJK} \begin{CJK}{UTF8}{mj}余子式\end{CJK}, \begin{CJK}{UTF8}{mj}方阵\end{CJK} $A$ \begin{CJK}{UTF8}{mj}的秩为\end{CJK}

\end{enumerate}
\begin{CJK}{UTF8}{mj}二\end{CJK}. \begin{CJK}{UTF8}{mj}解答题\end{CJK}(\begin{CJK}{UTF8}{mj}共\end{CJK} 90 \begin{CJK}{UTF8}{mj}分\end{CJK}. \begin{CJK}{UTF8}{mj}需写出详细的解答过程\end{CJK})

\begin{enumerate}
  \item ( 15 \begin{CJK}{UTF8}{mj}分\end{CJK}) \begin{CJK}{UTF8}{mj}设\end{CJK} $A$ \begin{CJK}{UTF8}{mj}为行满秩的\end{CJK} $m \times n$ \begin{CJK}{UTF8}{mj}矩阵\end{CJK}, $m<n$. \begin{CJK}{UTF8}{mj}试证明\end{CJK}: \begin{CJK}{UTF8}{mj}存在\end{CJK} $n$ \begin{CJK}{UTF8}{mj}阶可逆方阵\end{CJK} $Q$ \begin{CJK}{UTF8}{mj}使得\end{CJK} $A=\left(I_{m}, 0\right) Q$, \begin{CJK}{UTF8}{mj}其中\end{CJK} $I_{m}$ \begin{CJK}{UTF8}{mj}为\end{CJK} $m$ \begin{CJK}{UTF8}{mj}阶单位矩阵\end{CJK}, 0 \begin{CJK}{UTF8}{mj}为\end{CJK} $m \times(n-m)$ \begin{CJK}{UTF8}{mj}零矩阵\end{CJK}.

  \item (15 \begin{CJK}{UTF8}{mj}分\end{CJK}) \begin{CJK}{UTF8}{mj}设\end{CJK} $P_{3}$ \begin{CJK}{UTF8}{mj}为由次数不超过\end{CJK} 3 \begin{CJK}{UTF8}{mj}的复系数多项式组成的线性空间\end{CJK}. \begin{CJK}{UTF8}{mj}考虑其上的线性变换\end{CJK}

\end{enumerate}
$$
\mathbb{A}=x \frac{\mathrm{d}}{\mathrm{d} x}: P_{3} \rightarrow P_{3} .
$$
\begin{CJK}{UTF8}{mj}试求\end{CJK} $\mathbb{A}$ \begin{CJK}{UTF8}{mj}的极小多项式\end{CJK}.

\begin{enumerate}
  \setcounter{enumi}{3}
  \item ( 20 \begin{CJK}{UTF8}{mj}分\end{CJK}) \begin{CJK}{UTF8}{mj}考虑行向量空间\end{CJK} $\mathbb{R}^{4}$ \begin{CJK}{UTF8}{mj}中向量组\end{CJK}
\end{enumerate}
$$
S=\left\{a_{1}=(1,-1,2,4), a_{2}=(0,3,1,2), a_{3}=(1,-1,2,0), a_{4}=(2,1,5,10), a_{5}=(3,0,7,14)\right\} .
$$
\begin{CJK}{UTF8}{mj}试求向量组\end{CJK} $S$ \begin{CJK}{UTF8}{mj}的所有极大线性无关组\end{CJK}.

\begin{enumerate}
  \setcounter{enumi}{4}
  \item ( 20 \begin{CJK}{UTF8}{mj}分\end{CJK}) \begin{CJK}{UTF8}{mj}考虑矩阵\end{CJK} $A=\left(\begin{array}{cccc}1 & 0 & 2 & 0 \\ 0 & 0 & 0 & 1 \\ 2 & 0 & 1 & 0 \\ 0 & 1 & 0 & 0\end{array}\right)$. \begin{CJK}{UTF8}{mj}试求正交方阵\end{CJK} $P$ \begin{CJK}{UTF8}{mj}使得\end{CJK} $P^{-1} A P$ \begin{CJK}{UTF8}{mj}为对角阵\end{CJK}.

  \item (20 \begin{CJK}{UTF8}{mj}分\end{CJK}) \begin{CJK}{UTF8}{mj}设\end{CJK} $A=\frac{1}{10}\left(\begin{array}{cc}9 & 7 \\ 1 & 3\end{array}\right), p_{0}, q_{0}$ \begin{CJK}{UTF8}{mj}为正实数\end{CJK}, \begin{CJK}{UTF8}{mj}满足\end{CJK} $p_{0}+q_{0}=1$. \begin{CJK}{UTF8}{mj}对于\end{CJK} $n \geqslant 1$, \begin{CJK}{UTF8}{mj}令\end{CJK} $\left(\begin{array}{c}p_{n} \\ q_{n}\end{array}\right)=A^{n}\left(\begin{array}{c}p_{0} \\ q_{0}\end{array}\right)$. \begin{CJK}{UTF8}{mj}试证明\end{CJK}: \begin{CJK}{UTF8}{mj}数列\end{CJK} $\left\{p_{n}\right\}_{n \geqslant 0}$ \begin{CJK}{UTF8}{mj}的极限存在\end{CJK}, \begin{CJK}{UTF8}{mj}并求出极限值\end{CJK}.

\end{enumerate}
\section{8. 中国科学技术大学 2016 年研究生入学考试试题线性代数与解析 几何}
\begin{CJK}{UTF8}{mj}李扬\end{CJK}

\begin{CJK}{UTF8}{mj}微信公众号\end{CJK}: sxkyliyang

\begin{CJK}{UTF8}{mj}一\end{CJK}. \begin{CJK}{UTF8}{mj}填空题\end{CJK}(\begin{CJK}{UTF8}{mj}每空\end{CJK} 6 \begin{CJK}{UTF8}{mj}分\end{CJK}, \begin{CJK}{UTF8}{mj}共\end{CJK} 60 \begin{CJK}{UTF8}{mj}分\end{CJK}, \begin{CJK}{UTF8}{mj}需化简答案\end{CJK})

\begin{enumerate}
  \item \begin{CJK}{UTF8}{mj}经过直线\end{CJK} $x=y=2-z$ \begin{CJK}{UTF8}{mj}且与平面\end{CJK} $x+2 y+3 z=5$ \begin{CJK}{UTF8}{mj}垂直的平面方程是\end{CJK}

  \item \begin{CJK}{UTF8}{mj}给定空间直角坐标系中点\end{CJK} $A(1,0,1), B(0,1,-1), C(2,1,2)$ \begin{CJK}{UTF8}{mj}和\end{CJK} $D(5,4,1)$. \begin{CJK}{UTF8}{mj}则四面体\end{CJK} $\mathrm{ABCD}$ \begin{CJK}{UTF8}{mj}的体积为\end{CJK} \begin{CJK}{UTF8}{mj}点\end{CJK} $D$ \begin{CJK}{UTF8}{mj}到平面\end{CJK} $\mathrm{ABC}$ \begin{CJK}{UTF8}{mj}的距离为\end{CJK}

  \item \begin{CJK}{UTF8}{mj}二次型\end{CJK} $\left(x_{1}-x_{2}\right)^{2}+\left(x_{2}-x_{3}\right)^{2}+\left(x_{3}-x_{4}\right)^{2}+\left(x_{4}-x_{5}\right)^{2}$ \begin{CJK}{UTF8}{mj}的正惯性指数为\end{CJK}

\end{enumerate}
\includegraphics[max width=\textwidth]{2022_04_18_3416d289b173eb9de8c1g-168}\\
, \begin{CJK}{UTF8}{mj}其中\end{CJK} $I_{n}$ \begin{CJK}{UTF8}{mj}表示\end{CJK} $n$ \begin{CJK}{UTF8}{mj}阶单位矩阵\end{CJK}.

\begin{enumerate}
  \setcounter{enumi}{5}
  \item \begin{CJK}{UTF8}{mj}考虑列分块方阵\end{CJK} $A=\left(\beta_{1}, \beta_{2}, \beta_{3}\right)$ \begin{CJK}{UTF8}{mj}以及\end{CJK} $B=\left(2 \beta_{3}, \beta_{2},-\beta_{1}\right)$. \begin{CJK}{UTF8}{mj}若\end{CJK} $\operatorname{det} A=2$, \begin{CJK}{UTF8}{mj}则\end{CJK} $\operatorname{det} B=$

  \item \begin{CJK}{UTF8}{mj}由向量\end{CJK} $(0,1,4,14),(1,2,3,4),(1,1,0,-5),(3,2,1,-4)$ \begin{CJK}{UTF8}{mj}以及\end{CJK} $(2,1,1,1)$ \begin{CJK}{UTF8}{mj}生成\end{CJK} $\mathbb{R}^{4}$ \begin{CJK}{UTF8}{mj}的子空间维数为\end{CJK}

  \item $\lambda$ - \begin{CJK}{UTF8}{mj}矩阵\end{CJK} $\left(\begin{array}{ccc}0 & \lambda^{2}-\lambda & 0 \\ \lambda^{2}+\lambda & 0 & 0 \\ 0 & 0 & \lambda^{3}-\lambda\end{array}\right)$ \begin{CJK}{UTF8}{mj}的\end{CJK} Smith \begin{CJK}{UTF8}{mj}标准型为\end{CJK}

  \item \begin{CJK}{UTF8}{mj}若\end{CJK} 1 \begin{CJK}{UTF8}{mj}为方阵\end{CJK} $\left(\begin{array}{cccc}0 & 0 & 0 & -1 \\ 1 & 0 & 0 & 2 \\ 0 & 1 & 0 & -3 \\ 0 & 0 & 1 & a\end{array}\right)$ \begin{CJK}{UTF8}{mj}的特征值\end{CJK}, \begin{CJK}{UTF8}{mj}则\end{CJK} $a=$

\end{enumerate}
\begin{CJK}{UTF8}{mj}二\end{CJK}. \begin{CJK}{UTF8}{mj}答题\end{CJK}(\begin{CJK}{UTF8}{mj}共\end{CJK} 90 \begin{CJK}{UTF8}{mj}分\end{CJK}, \begin{CJK}{UTF8}{mj}需写出详细的解答过程\end{CJK})

\begin{enumerate}
  \item (15 \begin{CJK}{UTF8}{mj}分\end{CJK}) \begin{CJK}{UTF8}{mj}设\end{CJK} $A$ \begin{CJK}{UTF8}{mj}为\end{CJK} $m \times n$ \begin{CJK}{UTF8}{mj}矩阵\end{CJK}, $B$ \begin{CJK}{UTF8}{mj}为\end{CJK} $n \times m$ \begin{CJK}{UTF8}{mj}矩阵\end{CJK}, $a$ \begin{CJK}{UTF8}{mj}为非零复数\end{CJK}. \begin{CJK}{UTF8}{mj}试证明\end{CJK}: $a$ \begin{CJK}{UTF8}{mj}为\end{CJK} $A B$ \begin{CJK}{UTF8}{mj}的特征值当且仅当\end{CJK} $a$ \begin{CJK}{UTF8}{mj}为\end{CJK} $B A$ \begin{CJK}{UTF8}{mj}的特征值\end{CJK}. \begin{CJK}{UTF8}{mj}若\end{CJK} $a=0$, \begin{CJK}{UTF8}{mj}该结论还成立吗\end{CJK}? \begin{CJK}{UTF8}{mj}请论证或举例说明\end{CJK}.

  \item (15 \begin{CJK}{UTF8}{mj}分\end{CJK}) \begin{CJK}{UTF8}{mj}设\end{CJK} $n$ \begin{CJK}{UTF8}{mj}阶正交方阵\end{CJK} $A$ \begin{CJK}{UTF8}{mj}和\end{CJK} $B$ \begin{CJK}{UTF8}{mj}满足\end{CJK} $\operatorname{det} A=-\operatorname{det} B$. \begin{CJK}{UTF8}{mj}试证明\end{CJK}: $\operatorname{det}(A+B)=0$.

  \item ( 20 \begin{CJK}{UTF8}{mj}分\end{CJK}) \begin{CJK}{UTF8}{mj}考虑\end{CJK} $2 \times 2$ \begin{CJK}{UTF8}{mj}实方阵全体\end{CJK} $M_{2}(R)$, \begin{CJK}{UTF8}{mj}对于任给的两个\end{CJK} 2 \begin{CJK}{UTF8}{mj}阶方阵\end{CJK} $A, B$, \begin{CJK}{UTF8}{mj}我们定义\end{CJK} $\langle A, B\rangle=\operatorname{tr}\left(A B^{t}\right)$. \begin{CJK}{UTF8}{mj}这\end{CJK} \begin{CJK}{UTF8}{mj}里\end{CJK} $\operatorname{tr}$ \begin{CJK}{UTF8}{mj}表示迹\end{CJK}, $t$ \begin{CJK}{UTF8}{mj}表示矩阵转置\end{CJK}.

\end{enumerate}
(1) \begin{CJK}{UTF8}{mj}试证明\end{CJK}: $\langle-,-\rangle$ \begin{CJK}{UTF8}{mj}是\end{CJK} $M_{2}(R)$ \begin{CJK}{UTF8}{mj}上的一个内积\end{CJK}.

(2) \begin{CJK}{UTF8}{mj}在该内积下\end{CJK}, \begin{CJK}{UTF8}{mj}试计算向量组\end{CJK} $\left\{\left(\begin{array}{ll}1 & 0 \\ 0 & 1\end{array}\right),\left(\begin{array}{cc}1 & 1 \\ 0 & 0\end{array}\right),\left(\begin{array}{cc}0 & 0 \\ 1 & 1\end{array}\right),\left(\begin{array}{cc}1 & 0 \\ 0 & -1\end{array}\right)\right\}$ \begin{CJK}{UTF8}{mj}的\end{CJK} Gram-Schmidt \begin{CJK}{UTF8}{mj}标准正交化\end{CJK}.

\begin{enumerate}
  \setcounter{enumi}{4}
  \item ( 20 \begin{CJK}{UTF8}{mj}分\end{CJK}) \begin{CJK}{UTF8}{mj}考虑\end{CJK} $[0,1]$ \begin{CJK}{UTF8}{mj}区间上的连续实函数组成的实线性空间\end{CJK} $V$, \begin{CJK}{UTF8}{mj}设\end{CJK} $n \geqslant 1, W$ \begin{CJK}{UTF8}{mj}为\end{CJK} $\left\{1, x, x^{2}, \cdots, x^{n}\right\}$ \begin{CJK}{UTF8}{mj}生成的\end{CJK} \begin{CJK}{UTF8}{mj}子空间\end{CJK}, \begin{CJK}{UTF8}{mj}考虑线性变换\end{CJK} $\mathbb{A}=(x+1) \frac{\mathrm{d}}{\mathrm{d} x}: W \rightarrow W$. \begin{CJK}{UTF8}{mj}试证明\end{CJK}: $\operatorname{dim} W=n$, \begin{CJK}{UTF8}{mj}且\end{CJK} $\mathbb{A}$ \begin{CJK}{UTF8}{mj}可对角化\end{CJK}, \begin{CJK}{UTF8}{mj}并求出\end{CJK} $\mathbb{A}$ \begin{CJK}{UTF8}{mj}的所有\end{CJK} \begin{CJK}{UTF8}{mj}特征向量\end{CJK}.

  \item ( 20 \begin{CJK}{UTF8}{mj}分\end{CJK}) \begin{CJK}{UTF8}{mj}设\end{CJK} $e$ \begin{CJK}{UTF8}{mj}为自然常数\end{CJK}. \begin{CJK}{UTF8}{mj}对于\end{CJK} $n$ \begin{CJK}{UTF8}{mj}阶复方阵\end{CJK} $A$, \begin{CJK}{UTF8}{mj}我们定义\end{CJK} $e^{A}=I_{n}+A+\frac{A^{2}}{2 !}+\frac{A^{3}}{3 !}+\cdots$. \begin{CJK}{UTF8}{mj}试证明\end{CJK}:

\end{enumerate}
(1) \begin{CJK}{UTF8}{mj}方阵\end{CJK} $e^{A}$ \begin{CJK}{UTF8}{mj}定义合理\end{CJK};

(2) $\operatorname{det}\left(e^{A}\right)=e^{\operatorname{tr}(A)}$.

\section{9. 中国科学技术大学 2009 年研究生入学考试试题数学分析}
\begin{CJK}{UTF8}{mj}李扬\end{CJK}

\begin{CJK}{UTF8}{mj}微信公众号\end{CJK}: sxkyliyang

\begin{enumerate}
  \item \begin{CJK}{UTF8}{mj}判断\end{CJK}.
\end{enumerate}
(1) $\sum_{n=0}^{\infty} \frac{(1+2 \mathrm{i})^{n}}{3^{n}-2^{n}}$ \begin{CJK}{UTF8}{mj}绝对收敛\end{CJK}.

(2) $F$ \begin{CJK}{UTF8}{mj}一致收敛的充要条件是\end{CJK} $f$ \begin{CJK}{UTF8}{mj}把\end{CJK} Cauchy \begin{CJK}{UTF8}{mj}列映成\end{CJK} Cauchy \begin{CJK}{UTF8}{mj}列\end{CJK}.

\begin{enumerate}
  \setcounter{enumi}{2}
  \item \begin{CJK}{UTF8}{mj}填空\end{CJK}.
\end{enumerate}
(1) $f=1-x$ \begin{CJK}{UTF8}{mj}在\end{CJK} $x-1$ \begin{CJK}{UTF8}{mj}处展开后的级数\end{CJK}, \begin{CJK}{UTF8}{mj}问其收敛点集是什么\end{CJK}.

(2) $\sin \left(x^{2}\right)=x$ \begin{CJK}{UTF8}{mj}有\end{CJK}

(3) $1-\frac{1}{2}-\frac{1}{4}+\frac{1}{3}-\frac{1}{6}-\frac{1}{8}+\cdots+\frac{1}{2 n-1}-\frac{1}{4 n-2}-\frac{1}{4 n}+\cdots$ \begin{CJK}{UTF8}{mj}的和是\end{CJK}

\begin{enumerate}
  \setcounter{enumi}{3}
  \item $f:[0,1] \rightarrow \mathbb{R}$, \begin{CJK}{UTF8}{mj}单调递增且\end{CJK} $f([0,1])$ \begin{CJK}{UTF8}{mj}是闭集\end{CJK}, \begin{CJK}{UTF8}{mj}证明\end{CJK}: $f$ \begin{CJK}{UTF8}{mj}在\end{CJK} $[0,1]$ \begin{CJK}{UTF8}{mj}上连续\end{CJK}.

  \item $f$ \begin{CJK}{UTF8}{mj}在\end{CJK} $[0,1]$ \begin{CJK}{UTF8}{mj}上连续\end{CJK}, \begin{CJK}{UTF8}{mj}且\end{CJK} $\int_{0}^{1} f(x) x^{n} \mathrm{~d} x=0, n=0,1,2, \cdots$, \begin{CJK}{UTF8}{mj}证明\end{CJK}: $f \equiv 0$.

  \item \begin{CJK}{UTF8}{mj}是否存在原函数\end{CJK} $f$, \begin{CJK}{UTF8}{mj}使得\end{CJK} $\mathrm{d} f$ \begin{CJK}{UTF8}{mj}满足如下等式\end{CJK}:

\end{enumerate}
$$
\mathrm{d} f=\frac{x \mathrm{~d} y-y \mathrm{~d} x}{\sqrt{x^{2}+y^{2}}}
$$

\begin{enumerate}
  \setcounter{enumi}{6}
  \item $f: \mathbb{N} \rightarrow \mathbb{N}$, \begin{CJK}{UTF8}{mj}且\end{CJK} $f^{-1}(\mathbb{N})$ \begin{CJK}{UTF8}{mj}是有限集\end{CJK}, $\lim _{n \rightarrow \infty} x_{n}$ \begin{CJK}{UTF8}{mj}存在\end{CJK}, \begin{CJK}{UTF8}{mj}证明\end{CJK}: $\lim _{n \rightarrow \infty} x_{f(n)}$ \begin{CJK}{UTF8}{mj}存在\end{CJK}.

  \item $S=\left\{(x, y, z) \in \mathbb{R}^{3} \mid x y^{2} z^{3}=1\right\}$

\end{enumerate}
(1) \begin{CJK}{UTF8}{mj}证明\end{CJK}: $S$ \begin{CJK}{UTF8}{mj}在\end{CJK} $\mathbb{R}^{3}$ \begin{CJK}{UTF8}{mj}确定一张隐式的曲面\end{CJK}, \begin{CJK}{UTF8}{mj}并求出一个在点\end{CJK} $(1,1,1)$ \begin{CJK}{UTF8}{mj}附近的参数方程\end{CJK}.

(2) $S$ \begin{CJK}{UTF8}{mj}是否连通\end{CJK}, \begin{CJK}{UTF8}{mj}是否紧致\end{CJK}?

(3) \begin{CJK}{UTF8}{mj}点\end{CJK} $q \in S,|q|$ \begin{CJK}{UTF8}{mj}是\end{CJK} $q$ \begin{CJK}{UTF8}{mj}到原点的距离\end{CJK}, \begin{CJK}{UTF8}{mj}点\end{CJK} $p$ \begin{CJK}{UTF8}{mj}满足\end{CJK} $|p|=\inf _{q \in S}|q|$, \begin{CJK}{UTF8}{mj}求\end{CJK} $p$ \begin{CJK}{UTF8}{mj}组成的集合\end{CJK}.

\begin{enumerate}
  \setcounter{enumi}{8}
  \item \begin{CJK}{UTF8}{mj}证明恒等式\end{CJK}:
\end{enumerate}
$$
\pi \sum_{n=-\infty}^{+\infty} e^{2 \pi|n|}=\sum_{n=-\infty}^{+\infty} \frac{1}{n^{2}+1}
$$
$9 .$
$$
\Gamma(s)=\int_{0}^{+\infty} x^{s-1} e^{-x} \mathrm{~d} x, S=\left\{(x, y, z) \mid x^{2}+y^{2}+z^{2}=1\right\}
$$
\begin{CJK}{UTF8}{mj}用\end{CJK} $\Gamma(s)$ \begin{CJK}{UTF8}{mj}表示第一型积分\end{CJK} $\int_{s}\left(x^{2}+y^{2}\right)^{a} \mathrm{~d} \sigma$, \begin{CJK}{UTF8}{mj}其中\end{CJK} $a>-1$. 10. \begin{CJK}{UTF8}{mj}中国科学技术大学\end{CJK} 2010 \begin{CJK}{UTF8}{mj}年研究生入学考试试题数学分析\end{CJK}

\begin{CJK}{UTF8}{mj}李扬\end{CJK}

\begin{CJK}{UTF8}{mj}微信公众号\end{CJK}: sxkyliyang

\begin{enumerate}
  \item ( 15 \begin{CJK}{UTF8}{mj}分\end{CJK}) \begin{CJK}{UTF8}{mj}设函数\end{CJK} $f(x):[0,+\infty) \rightarrow[0,+\infty)$ \begin{CJK}{UTF8}{mj}是一致连续的\end{CJK}, $\alpha \in(0,1]$. \begin{CJK}{UTF8}{mj}求证\end{CJK}: \begin{CJK}{UTF8}{mj}函数\end{CJK} $g(x)=f^{\alpha}(x)$ \begin{CJK}{UTF8}{mj}也在\end{CJK} $[0,+\infty)$ \begin{CJK}{UTF8}{mj}上一致连续\end{CJK}.

  \item ( 15 \begin{CJK}{UTF8}{mj}分\end{CJK}) \begin{CJK}{UTF8}{mj}设\end{CJK} $f(x, y)$ \begin{CJK}{UTF8}{mj}在\end{CJK} $\mathbb{R}^{2} \backslash\{(0,0)\}$ \begin{CJK}{UTF8}{mj}上可微\end{CJK}, \begin{CJK}{UTF8}{mj}在\end{CJK} $(0,0)$ \begin{CJK}{UTF8}{mj}处连续\end{CJK}, \begin{CJK}{UTF8}{mj}且\end{CJK}

\end{enumerate}
$$
\lim _{(x, y) \rightarrow(0,0)} \frac{\partial f}{\partial x}(x, y)=0, \lim _{(x, y) \rightarrow(0,0)} \frac{\partial f}{\partial y}(x, y)=0 .
$$
\begin{CJK}{UTF8}{mj}求证\end{CJK}: $f(x, y)$ \begin{CJK}{UTF8}{mj}在\end{CJK} $(0,0)$ \begin{CJK}{UTF8}{mj}处可微\end{CJK}.

\begin{enumerate}
  \setcounter{enumi}{3}
  \item ( 15 \begin{CJK}{UTF8}{mj}分\end{CJK}) \begin{CJK}{UTF8}{mj}设\end{CJK} $x_{0} \in\left(1, \frac{3}{2}\right), x_{1}=x_{0}^{2}, x_{n+1}=\sqrt{x_{n}}+\frac{x_{n-1}}{2}, n=1,2, \cdots$. \begin{CJK}{UTF8}{mj}求证\end{CJK}: \begin{CJK}{UTF8}{mj}数列\end{CJK} $\left\{x_{n}\right\}$ \begin{CJK}{UTF8}{mj}收敛\end{CJK}, \begin{CJK}{UTF8}{mj}并求其极限\end{CJK}.

  \item (15 \begin{CJK}{UTF8}{mj}分\end{CJK}) \begin{CJK}{UTF8}{mj}设\end{CJK} $f(x)$ \begin{CJK}{UTF8}{mj}在\end{CJK} $(-\infty,+\infty)$ \begin{CJK}{UTF8}{mj}上有连续的导函数\end{CJK}, $f(0)=0$, \begin{CJK}{UTF8}{mj}且曲线积分\end{CJK}

\end{enumerate}
$$
\int_{C}\left(e^{x}+f(x)\right) y \mathrm{~d} x+f(x) \mathrm{d} y
$$
\begin{CJK}{UTF8}{mj}与路径无关\end{CJK}. \begin{CJK}{UTF8}{mj}求\end{CJK}
$$
\int_{(0,0)}^{(1,1)}\left(e^{x}+f(x)\right) y \mathrm{~d} x+f(x) \mathrm{d} y
$$

\begin{enumerate}
  \setcounter{enumi}{5}
  \item ( 15 \begin{CJK}{UTF8}{mj}分\end{CJK}) \begin{CJK}{UTF8}{mj}设\end{CJK} $\alpha>1$. \begin{CJK}{UTF8}{mj}求证\end{CJK}:\begin{CJK}{UTF8}{mj}以下含参变量\end{CJK} $x$ \begin{CJK}{UTF8}{mj}的无穷积分\end{CJK}
\end{enumerate}
$$
f(x)=\int_{1}^{+\infty} \frac{\arctan (t x)}{t^{\alpha}} \mathrm{d} t
$$
\begin{CJK}{UTF8}{mj}定义了\end{CJK} $(0,+\infty)$ \begin{CJK}{UTF8}{mj}上的一个可微函数\end{CJK}, \begin{CJK}{UTF8}{mj}且满足\end{CJK}
$$
x f^{\prime}(x)-(\alpha-1) f(x)+\arctan x=0 .
$$

\begin{enumerate}
  \setcounter{enumi}{6}
  \item (15 \begin{CJK}{UTF8}{mj}分\end{CJK}) \begin{CJK}{UTF8}{mj}设\end{CJK} $a, b, c$ \begin{CJK}{UTF8}{mj}都是正数\end{CJK}. \begin{CJK}{UTF8}{mj}计算曲面积分\end{CJK}
\end{enumerate}
$$
\iint_{S} x^{3} \mathrm{~d} y \mathrm{~d} z+y^{3} \mathrm{~d} z \mathrm{~d} x+z^{3} \mathrm{~d} x \mathrm{~d} y
$$
\begin{CJK}{UTF8}{mj}其中\end{CJK} $S$ \begin{CJK}{UTF8}{mj}是上半椭球面\end{CJK} $\frac{x^{2}}{a^{2}}+\frac{y^{2}}{b^{2}}+\frac{z^{2}}{c^{2}}=1, z \geqslant 0$, \begin{CJK}{UTF8}{mj}方向朝上\end{CJK}.

\begin{enumerate}
  \setcounter{enumi}{7}
  \item (15 \begin{CJK}{UTF8}{mj}分\end{CJK}) \begin{CJK}{UTF8}{mj}设\end{CJK} $f(x)$ \begin{CJK}{UTF8}{mj}是定义在实轴上以\end{CJK} $2 \pi$ \begin{CJK}{UTF8}{mj}为周期的奇函数\end{CJK}, \begin{CJK}{UTF8}{mj}又\end{CJK} $f(x)$ \begin{CJK}{UTF8}{mj}有连续的导函数且满足\end{CJK} $f^{\prime}(x)=f\left(\frac{\pi}{2}-x\right)$. \begin{CJK}{UTF8}{mj}试求\end{CJK} $f(x)$.

  \item (15 \begin{CJK}{UTF8}{mj}分\end{CJK}) \begin{CJK}{UTF8}{mj}设\end{CJK} $\sum_{n=1}^{\infty} a_{n}$ \begin{CJK}{UTF8}{mj}是一个收敛的正项级数\end{CJK}. \begin{CJK}{UTF8}{mj}求证\end{CJK}: $\sum_{n=1}^{\infty} a_{n}^{1-\frac{1}{n}}$ \begin{CJK}{UTF8}{mj}也收敛\end{CJK}.

  \item (15 \begin{CJK}{UTF8}{mj}分\end{CJK}) \begin{CJK}{UTF8}{mj}设函数\end{CJK} $f(x)$ \begin{CJK}{UTF8}{mj}在\end{CJK} $[0,+\infty)$ \begin{CJK}{UTF8}{mj}上二阶可导\end{CJK}, $f(0) \geqslant 0, f^{\prime}(0) \geqslant 0$, \begin{CJK}{UTF8}{mj}且满足\end{CJK} $f(x) \leqslant f^{\prime \prime}(x)$. \begin{CJK}{UTF8}{mj}求证\end{CJK}:

\end{enumerate}
$$
f(x) \geqslant f(0)+f^{\prime}(0) x .
$$

\begin{enumerate}
  \setcounter{enumi}{10}
  \item ( 15 \begin{CJK}{UTF8}{mj}分\end{CJK}) \begin{CJK}{UTF8}{mj}设\end{CJK} $\left\{a_{n}\right\}$ \begin{CJK}{UTF8}{mj}和\end{CJK} $\left\{b_{n}\right\}$ \begin{CJK}{UTF8}{mj}都是正数列\end{CJK}, \begin{CJK}{UTF8}{mj}满足\end{CJK} $\lim _{n \rightarrow \infty} \frac{b_{n}}{n}=0$ \begin{CJK}{UTF8}{mj}及\end{CJK}
\end{enumerate}
$$
\lim _{n \rightarrow \infty} b_{n}\left(\frac{a_{n}}{a_{n+1}}-1\right)=\lambda>0 .
$$
\begin{CJK}{UTF8}{mj}求证\end{CJK}:

(1) $\lim _{n \rightarrow \infty} a_{n}=0$;

(2) \begin{CJK}{UTF8}{mj}级数\end{CJK} $\sum_{n=1}^{\infty} a_{n}$ \begin{CJK}{UTF8}{mj}收敛\end{CJK}. 11. \begin{CJK}{UTF8}{mj}中国科学技术大学\end{CJK} 2011 \begin{CJK}{UTF8}{mj}年研究生入学考试试题数学分析\end{CJK}

\begin{CJK}{UTF8}{mj}李扬\end{CJK}

\begin{CJK}{UTF8}{mj}微信公众号\end{CJK}: sxkyliyang

\begin{enumerate}
  \item (15 \begin{CJK}{UTF8}{mj}分\end{CJK}) \begin{CJK}{UTF8}{mj}计算\end{CJK}.\\
(1) $\lim _{x \rightarrow+\infty} x\left\{\left(1+\frac{1}{x}\right)^{x}-e\right\}$;\\
(2) $\int_{0}^{\frac{\pi}{2}} \sin ^{7} x \mathrm{~d} x$.

  \item (15 \begin{CJK}{UTF8}{mj}分\end{CJK}) \begin{CJK}{UTF8}{mj}回答下列问题\end{CJK}, \begin{CJK}{UTF8}{mj}举例说明或者证明你的结论\end{CJK}.

\end{enumerate}
(1) \begin{CJK}{UTF8}{mj}是否存在\end{CJK} $\mathbb{R}$ \begin{CJK}{UTF8}{mj}上处处不连续的函数\end{CJK}, \begin{CJK}{UTF8}{mj}它的绝对值却是处处连续的函数\end{CJK}?

(2) \begin{CJK}{UTF8}{mj}设\end{CJK} $f, g$ \begin{CJK}{UTF8}{mj}是\end{CJK} $\mathbb{R}$ \begin{CJK}{UTF8}{mj}上的连续函数\end{CJK}, \begin{CJK}{UTF8}{mj}如果\end{CJK} $f(x)=g(x)$ \begin{CJK}{UTF8}{mj}对所有有理数\end{CJK} $x$ \begin{CJK}{UTF8}{mj}成立\end{CJK}, \begin{CJK}{UTF8}{mj}是否可以断言\end{CJK} $f(x)=g(x)$ \begin{CJK}{UTF8}{mj}在\end{CJK} $\mathbb{R}$ \begin{CJK}{UTF8}{mj}上\end{CJK} \begin{CJK}{UTF8}{mj}成立\end{CJK}?

(3) \begin{CJK}{UTF8}{mj}无穷区间上的连续函数是否能用多项式一致逼近\end{CJK}?

\begin{enumerate}
  \setcounter{enumi}{3}
  \item ( 15 \begin{CJK}{UTF8}{mj}分\end{CJK}) \begin{CJK}{UTF8}{mj}设\end{CJK} $a, b, c, d$ \begin{CJK}{UTF8}{mj}是\end{CJK} 4 \begin{CJK}{UTF8}{mj}个不等于\end{CJK} 1 \begin{CJK}{UTF8}{mj}的正数\end{CJK}, \begin{CJK}{UTF8}{mj}满足\end{CJK} $a b c d=1$, \begin{CJK}{UTF8}{mj}问\end{CJK}:
\end{enumerate}
$$
a^{2010}+b^{2010}+c^{2010}+d^{2010} \text { 和 } a^{2011}+b^{2011}+c^{2011}+d^{2011}
$$
\begin{CJK}{UTF8}{mj}哪个数大\end{CJK}? \begin{CJK}{UTF8}{mj}为什么\end{CJK}?

\begin{enumerate}
  \setcounter{enumi}{4}
  \item (15 \begin{CJK}{UTF8}{mj}分\end{CJK}) \begin{CJK}{UTF8}{mj}设\end{CJK} $f:[a, b] \rightarrow[a, b]$.
\end{enumerate}
(1) \begin{CJK}{UTF8}{mj}如果\end{CJK} $f$ \begin{CJK}{UTF8}{mj}是\end{CJK} $[a, b]$ \begin{CJK}{UTF8}{mj}上的连续函数\end{CJK}, \begin{CJK}{UTF8}{mj}证明\end{CJK}: \begin{CJK}{UTF8}{mj}存在\end{CJK} $\xi \in[a, b]$ \begin{CJK}{UTF8}{mj}使得\end{CJK} $f(\xi)=\xi$;

(2) \begin{CJK}{UTF8}{mj}如果\end{CJK} $f$ \begin{CJK}{UTF8}{mj}是\end{CJK} $[a, b]$ \begin{CJK}{UTF8}{mj}上的递增函数\end{CJK}, \begin{CJK}{UTF8}{mj}证明\end{CJK}: \begin{CJK}{UTF8}{mj}存在\end{CJK} $\xi \in[a, b]$ \begin{CJK}{UTF8}{mj}使得\end{CJK} $f(\xi)=\xi$.

\begin{enumerate}
  \setcounter{enumi}{5}
  \item ( 20 \begin{CJK}{UTF8}{mj}分\end{CJK})
\end{enumerate}
(1) \begin{CJK}{UTF8}{mj}把周期为\end{CJK} $2 \pi$ \begin{CJK}{UTF8}{mj}的函数\end{CJK}
$$
f(x)=x^{2}-\pi^{2}, x \in[-\pi, \pi]
$$
\begin{CJK}{UTF8}{mj}展开为\end{CJK} Fourier \begin{CJK}{UTF8}{mj}级数\end{CJK}.

(2) \begin{CJK}{UTF8}{mj}利用上面的级数\end{CJK}, \begin{CJK}{UTF8}{mj}计算下列级数的和\end{CJK}
$$
\sum_{n=1}^{\infty} \frac{1}{n^{2}}, \sum_{n=1}^{\infty}(-1)^{n-1} \frac{1}{n^{2}}, \sum_{n=1}^{\infty} \frac{1}{n^{4}}
$$
(3) \begin{CJK}{UTF8}{mj}求级数\end{CJK} $\sum_{n=1}^{\infty}(-1)^{n} \frac{\cos n x}{n^{2}}$ \begin{CJK}{UTF8}{mj}的和\end{CJK}.

\begin{enumerate}
  \setcounter{enumi}{6}
  \item (15 \begin{CJK}{UTF8}{mj}分\end{CJK}) \begin{CJK}{UTF8}{mj}证明\end{CJK}: \begin{CJK}{UTF8}{mj}含参变量积分\end{CJK}
\end{enumerate}
$$
F(u)=\int_{0}^{+\infty} \frac{\sin \left(u x^{2}\right)}{x} \mathrm{~d} x
$$
\begin{CJK}{UTF8}{mj}在\end{CJK} $(0,+\infty)$ \begin{CJK}{UTF8}{mj}上不一致连续\end{CJK}, \begin{CJK}{UTF8}{mj}但在\end{CJK} $(0,+\infty)$ \begin{CJK}{UTF8}{mj}上连续\end{CJK}.

\begin{enumerate}
  \setcounter{enumi}{7}
  \item ( 15 \begin{CJK}{UTF8}{mj}分\end{CJK}) \begin{CJK}{UTF8}{mj}设\end{CJK} $\left\{x_{n}\right\}$ \begin{CJK}{UTF8}{mj}是一个非负的数列\end{CJK}, \begin{CJK}{UTF8}{mj}满足\end{CJK}
\end{enumerate}
$$
x_{n+1} \leqslant x_{n}+\frac{1}{n^{2}}, n=1,2, \cdots,
$$
\begin{CJK}{UTF8}{mj}证明\end{CJK}: $\left\{x_{n}\right\}$ \begin{CJK}{UTF8}{mj}收敛\end{CJK}. 8. (15 \begin{CJK}{UTF8}{mj}分\end{CJK}) \begin{CJK}{UTF8}{mj}若\end{CJK} $\sum_{n=1}^{\infty} a_{n}=A$, \begin{CJK}{UTF8}{mj}证明\end{CJK}:
$$
\sum_{n=1}^{\infty} \frac{a_{1}+2 a_{2}+\cdots+n a_{n}}{n(n+1)}=A
$$

\begin{enumerate}
  \setcounter{enumi}{9}
  \item (15 \begin{CJK}{UTF8}{mj}分\end{CJK}) \begin{CJK}{UTF8}{mj}设\end{CJK} $f$ \begin{CJK}{UTF8}{mj}是从区间\end{CJK} $[0,1]$ \begin{CJK}{UTF8}{mj}映到\end{CJK} $[0,1]$ \begin{CJK}{UTF8}{mj}的函数\end{CJK}, \begin{CJK}{UTF8}{mj}其图像\end{CJK} $\{(x, f(x)): x \in[0,1]\}$ \begin{CJK}{UTF8}{mj}是单位正方形\end{CJK} $[0,1] \times[0,1]$ \begin{CJK}{UTF8}{mj}的\end{CJK} \begin{CJK}{UTF8}{mj}闭子集\end{CJK}. \begin{CJK}{UTF8}{mj}证明\end{CJK}: $f$ \begin{CJK}{UTF8}{mj}是连续函数\end{CJK}.

  \item ( 10 \begin{CJK}{UTF8}{mj}分\end{CJK}) \begin{CJK}{UTF8}{mj}设\end{CJK} $D$ \begin{CJK}{UTF8}{mj}是封闭光滑曲线\end{CJK} $L$ \begin{CJK}{UTF8}{mj}围成的区域\end{CJK}, $f(x, y)$ \begin{CJK}{UTF8}{mj}在\end{CJK} $\bar{D}$ \begin{CJK}{UTF8}{mj}上有二阶连续偏导数\end{CJK}, \begin{CJK}{UTF8}{mj}且\end{CJK}

\end{enumerate}
$$
a \frac{\partial^{2} f}{\partial x^{2}}+b \frac{\partial^{2} f}{\partial y^{2}}=0
$$
\begin{CJK}{UTF8}{mj}其中\end{CJK} $a, b>0$. \begin{CJK}{UTF8}{mj}若\end{CJK} $f$ \begin{CJK}{UTF8}{mj}在\end{CJK} $L$ \begin{CJK}{UTF8}{mj}上等于常数\end{CJK} $C$, \begin{CJK}{UTF8}{mj}证明\end{CJK}: $f$ \begin{CJK}{UTF8}{mj}在\end{CJK} $D$ \begin{CJK}{UTF8}{mj}上恒等于\end{CJK} $C$.

\section{2. 中国科学技术大学 2012 年研究生入学考试试题数学分析 
 李扬 
 微信公众号: sxkyliyang}
\begin{enumerate}
  \item (15 \begin{CJK}{UTF8}{mj}分\end{CJK}) \begin{CJK}{UTF8}{mj}在下面三个问题中\end{CJK}, \begin{CJK}{UTF8}{mj}如果答案是肯定的\end{CJK}, \begin{CJK}{UTF8}{mj}请举出相应的例子\end{CJK}; \begin{CJK}{UTF8}{mj}如果答案是否定的\end{CJK}, \begin{CJK}{UTF8}{mj}请给出证明\end{CJK}.
\end{enumerate}
(1) \begin{CJK}{UTF8}{mj}是否存在两个发散的正数列\end{CJK}, \begin{CJK}{UTF8}{mj}它们的和是一个收敛数列\end{CJK}?

(2) \begin{CJK}{UTF8}{mj}是否存在\end{CJK} $[a, b]$ \begin{CJK}{UTF8}{mj}上不恒等于\end{CJK} 0 \begin{CJK}{UTF8}{mj}的连续函数\end{CJK}, \begin{CJK}{UTF8}{mj}它在\end{CJK} $[a, b]$ \begin{CJK}{UTF8}{mj}中的有理点处都取\end{CJK} 0 \begin{CJK}{UTF8}{mj}值\end{CJK}?

(3) \begin{CJK}{UTF8}{mj}是否存在这样的数列\end{CJK} $\left\{a_{n}\right\}$, \begin{CJK}{UTF8}{mj}它满足\end{CJK}
$$
\lim _{n \rightarrow \infty} \frac{a_{n}}{n}=0 \text { 但是 } \lim _{n \rightarrow \infty} \frac{\max \left\{a_{1}, \cdots, a_{n}\right\}}{n} \neq 0 \text { ? }
$$

\begin{enumerate}
  \setcounter{enumi}{2}
  \item ( 15 \begin{CJK}{UTF8}{mj}分\end{CJK}) \begin{CJK}{UTF8}{mj}设数列\end{CJK} $\left\{a_{n}\right\}$ \begin{CJK}{UTF8}{mj}满足\end{CJK}
\end{enumerate}
$$
\lim _{n \rightarrow \infty} a_{2 n-1}=a, \lim _{n \rightarrow \infty} a_{2 n}=b .
$$
\begin{CJK}{UTF8}{mj}证明\end{CJK}:
$$
\lim _{n \rightarrow \infty} \frac{a_{1}+\cdots+a_{n}}{n}=\frac{a+b}{2} .
$$

\begin{enumerate}
  \setcounter{enumi}{3}
  \item ( 15 \begin{CJK}{UTF8}{mj}分\end{CJK}) \begin{CJK}{UTF8}{mj}函数\end{CJK}
\end{enumerate}
$$
f(x)=\int_{x}^{x^{2}} \frac{1}{t} \ln \left(\frac{t-1}{32}\right) \mathrm{d} t, x \in(1,+\infty)
$$
\begin{CJK}{UTF8}{mj}在何处取最小值\end{CJK}?

\begin{enumerate}
  \setcounter{enumi}{4}
  \item (15 \begin{CJK}{UTF8}{mj}分\end{CJK}) \begin{CJK}{UTF8}{mj}设函数\end{CJK} $f$ \begin{CJK}{UTF8}{mj}在\end{CJK} $(0,+\infty)$ \begin{CJK}{UTF8}{mj}上可微\end{CJK}, \begin{CJK}{UTF8}{mj}且\end{CJK} $f^{\prime}(x)=O(x), x \rightarrow+\infty$. \begin{CJK}{UTF8}{mj}证明\end{CJK}:
\end{enumerate}
$$
f(x)=O\left(x^{2}\right), x \rightarrow+\infty .
$$

\begin{enumerate}
  \setcounter{enumi}{5}
  \item (15 \begin{CJK}{UTF8}{mj}分\end{CJK})
\end{enumerate}
(1) \begin{CJK}{UTF8}{mj}把周期为\end{CJK} $2 \pi$ \begin{CJK}{UTF8}{mj}的函数\end{CJK}
$$
f(x)=\left(\frac{\pi-x}{2}\right)^{2},(0 \leqslant x \leqslant 2 \pi)
$$
\begin{CJK}{UTF8}{mj}展开为\end{CJK} Fourier \begin{CJK}{UTF8}{mj}级数\end{CJK}.

(2) \begin{CJK}{UTF8}{mj}利用上面的级数\end{CJK}, \begin{CJK}{UTF8}{mj}计算下列级数的和\end{CJK}:
$$
\sum_{n=1}^{\infty} \frac{1}{n^{2}}, \sum_{n=1}^{\infty}(-1)^{n-1} \frac{1}{n^{2}}, \sum_{n=1}^{\infty} \frac{1}{n^{4}}
$$
(3) \begin{CJK}{UTF8}{mj}求级数\end{CJK} $\sum_{n=1}^{\infty} \frac{\cos n x}{n^{2}}$ \begin{CJK}{UTF8}{mj}的和\end{CJK}.

\begin{enumerate}
  \setcounter{enumi}{6}
  \item (15 \begin{CJK}{UTF8}{mj}分\end{CJK})
\end{enumerate}
(1) \begin{CJK}{UTF8}{mj}计算幂级数\end{CJK} $\sum_{n=0}^{\infty}\left(n^{2}+1\right) 3^{n} x^{n}$ \begin{CJK}{UTF8}{mj}的和\end{CJK}.

(2) \begin{CJK}{UTF8}{mj}证明\end{CJK}: \begin{CJK}{UTF8}{mj}级数\end{CJK}
$$
\sum_{n=1}^{\infty} x^{\alpha} e^{-n x^{2}}
$$
\begin{CJK}{UTF8}{mj}当\end{CJK} $\alpha>2$ \begin{CJK}{UTF8}{mj}时在\end{CJK} $(0,+\infty)$ \begin{CJK}{UTF8}{mj}中一致收敛\end{CJK}. 7. ( 15 \begin{CJK}{UTF8}{mj}分\end{CJK}) \begin{CJK}{UTF8}{mj}设\end{CJK} $z=f(x, y)$ \begin{CJK}{UTF8}{mj}在\end{CJK} $\mathbb{R}^{2}$ \begin{CJK}{UTF8}{mj}上有连续的二阶偏导数\end{CJK}, \begin{CJK}{UTF8}{mj}且满足方程\end{CJK}
$$
6 \frac{\partial^{2} z}{\partial x^{2}}+\frac{\partial^{2} z}{\partial x \partial y}-\frac{\partial^{2} z}{\partial y^{2}}=0
$$
\begin{CJK}{UTF8}{mj}试确定\end{CJK} $a$ \begin{CJK}{UTF8}{mj}的值\end{CJK}, \begin{CJK}{UTF8}{mj}使得在变换\end{CJK}
$$
\xi=x-2 y, \eta=x+a y(a \neq-2)
$$
\begin{CJK}{UTF8}{mj}下方程\end{CJK} (1) \begin{CJK}{UTF8}{mj}被简化为\end{CJK}
$$
\frac{\partial^{2} z}{\partial \xi \partial \eta}=0
$$
\begin{CJK}{UTF8}{mj}并由此求偏微分方程\end{CJK} (1) \begin{CJK}{UTF8}{mj}的解\end{CJK}.

\begin{enumerate}
  \setcounter{enumi}{8}
  \item ( 15 \begin{CJK}{UTF8}{mj}分\end{CJK}) \begin{CJK}{UTF8}{mj}设\end{CJK} $D$ \begin{CJK}{UTF8}{mj}是\end{CJK} $\mathbb{R}^{3}$ \begin{CJK}{UTF8}{mj}中的有界闭区域\end{CJK}, $f$ \begin{CJK}{UTF8}{mj}在\end{CJK} $D$ \begin{CJK}{UTF8}{mj}上连续且有偏导数\end{CJK}. \begin{CJK}{UTF8}{mj}如果在\end{CJK} $D$ \begin{CJK}{UTF8}{mj}上有\end{CJK}
\end{enumerate}
$$
\frac{\partial f}{\partial x}+\frac{\partial f}{\partial y}+\frac{\partial f}{\partial z}=f,\left.f\right|_{\partial D}=0(\partial D \text { 记 } D \text { 的边界 }) .
$$
\begin{CJK}{UTF8}{mj}证明\end{CJK}: $f$ \begin{CJK}{UTF8}{mj}在\end{CJK} $D$ \begin{CJK}{UTF8}{mj}上恒等于\end{CJK} 0 .

\begin{enumerate}
  \setcounter{enumi}{9}
  \item ( 15 \begin{CJK}{UTF8}{mj}分\end{CJK})
\end{enumerate}
(1) \begin{CJK}{UTF8}{mj}计算\end{CJK}
$$
\iiint_{V} \sqrt{a^{2}-x^{2}-y^{2}-z^{2}} \mathrm{~d} x \mathrm{~d} y \mathrm{~d} z
$$
\begin{CJK}{UTF8}{mj}其中\end{CJK} $V$ \begin{CJK}{UTF8}{mj}是\end{CJK} $\mathbb{R}^{3}$ \begin{CJK}{UTF8}{mj}中以原点为中心\end{CJK}, $a$ \begin{CJK}{UTF8}{mj}为半径的球\end{CJK}.

(2) \begin{CJK}{UTF8}{mj}设\end{CJK} $S$ \begin{CJK}{UTF8}{mj}是\end{CJK} $\mathbb{R}^{3}$ \begin{CJK}{UTF8}{mj}中不通过原点的光滑封闭曲面\end{CJK}, $S$ \begin{CJK}{UTF8}{mj}上点\end{CJK} $P$ \begin{CJK}{UTF8}{mj}处的外单位法向量\end{CJK} $\vec{n}=(\cos \alpha, \cos \beta, \cos \gamma)$. \begin{CJK}{UTF8}{mj}试就原\end{CJK} \begin{CJK}{UTF8}{mj}点在\end{CJK} $S$ \begin{CJK}{UTF8}{mj}所包围区域的外部或内部两种情形计算曲面积分\end{CJK}
$$
\iint_{S} \frac{x \cos \alpha+y \cos \beta+z \cos \gamma}{\left(a x^{2}+b y^{2}+c z^{2}\right)^{\frac{3}{2}}} \mathrm{~d} \sigma
$$
\begin{CJK}{UTF8}{mj}其中\end{CJK} $a, b, c$ \begin{CJK}{UTF8}{mj}都是正数\end{CJK}.

\begin{enumerate}
  \setcounter{enumi}{10}
  \item (15 \begin{CJK}{UTF8}{mj}分\end{CJK}) \begin{CJK}{UTF8}{mj}设\end{CJK} $f:(0,+\infty) \rightarrow(0,+\infty)$ \begin{CJK}{UTF8}{mj}是一个单调增加的函数\end{CJK}. \begin{CJK}{UTF8}{mj}如果\end{CJK}
\end{enumerate}
$$
\lim _{x \rightarrow+\infty} \frac{f(2 t)}{f(t)}=1,
$$
\begin{CJK}{UTF8}{mj}证明\end{CJK}: \begin{CJK}{UTF8}{mj}对任意\end{CJK} $m>0$ \begin{CJK}{UTF8}{mj}都有\end{CJK}
$$
\lim _{t \rightarrow+\infty} \frac{f(m t)}{f(t)}=1 .
$$

\begin{enumerate}
  \setcounter{enumi}{13}
  \item \begin{CJK}{UTF8}{mj}中国科学技术大学\end{CJK} 2013 \begin{CJK}{UTF8}{mj}年研究生入学考试试题数学分析\end{CJK}
\end{enumerate}
\begin{CJK}{UTF8}{mj}李扬\end{CJK}

\begin{CJK}{UTF8}{mj}微信公众号\end{CJK}: sxkyliyang

\begin{enumerate}
  \item ( 15 \begin{CJK}{UTF8}{mj}分\end{CJK}) \begin{CJK}{UTF8}{mj}回答下列问题\end{CJK}, \begin{CJK}{UTF8}{mj}并说明肯定或否定的理由\end{CJK}.
\end{enumerate}
(1) \begin{CJK}{UTF8}{mj}如果\end{CJK} $\sum_{n=1}^{\infty} u_{n}(x)$ \begin{CJK}{UTF8}{mj}在\end{CJK} $(a, b)$ \begin{CJK}{UTF8}{mj}的任一闭子空间\end{CJK} $[\alpha, \beta] \subset(a, b)$ \begin{CJK}{UTF8}{mj}中一致收敛\end{CJK}, \begin{CJK}{UTF8}{mj}能否断定它在\end{CJK} $(a, b)$ \begin{CJK}{UTF8}{mj}中处处收敛\end{CJK}? \begin{CJK}{UTF8}{mj}能否断定它在\end{CJK} $(a, b)$ \begin{CJK}{UTF8}{mj}中一致收敛\end{CJK}?

(2) \begin{CJK}{UTF8}{mj}点集\end{CJK} $E=\left\{(x, y) \in \mathbb{R}^{2}: x y>0\right\}$ \begin{CJK}{UTF8}{mj}是不是\end{CJK} $\mathbb{R}^{2}$ \begin{CJK}{UTF8}{mj}中区域\end{CJK}? \begin{CJK}{UTF8}{mj}是不是\end{CJK} $\mathbb{R}^{2}$ \begin{CJK}{UTF8}{mj}中开集\end{CJK}?

\begin{enumerate}
  \setcounter{enumi}{2}
  \item (15 \begin{CJK}{UTF8}{mj}分\end{CJK}) \begin{CJK}{UTF8}{mj}设\end{CJK} $\lim _{n \rightarrow \infty} a_{n}=a$.
\end{enumerate}
(1) \begin{CJK}{UTF8}{mj}试用\end{CJK} $\varepsilon-N$ \begin{CJK}{UTF8}{mj}语言证明\end{CJK}: $\lim _{n \rightarrow \infty} \frac{a_{1}+\cdots+a_{n}}{n}=a$.

(2) \begin{CJK}{UTF8}{mj}证明\end{CJK}: $\lim _{n \rightarrow \infty} \frac{a_{1}+\frac{a_{2}}{2}+\cdots+\frac{a_{n}}{n}}{\ln n}=a$.

\begin{enumerate}
  \setcounter{enumi}{3}
  \item (15 \begin{CJK}{UTF8}{mj}分\end{CJK}) \begin{CJK}{UTF8}{mj}计算积分\end{CJK}
\end{enumerate}
(1) $\int_{0}^{+\infty} \frac{x-\sin x}{x^{3}} \mathrm{~d} x$.

(2) $\iint_{x^{2}+y^{2} \leqslant 1}\left(3 x y^{2}-x^{2}\right) \mathrm{d} x \mathrm{~d} y$.

\begin{enumerate}
  \setcounter{enumi}{4}
  \item (15 \begin{CJK}{UTF8}{mj}分\end{CJK}) \begin{CJK}{UTF8}{mj}证明\end{CJK}: \begin{CJK}{UTF8}{mj}不等式\end{CJK}
\end{enumerate}
$$
\frac{1}{3} \tan x+\frac{2}{3} \sin x>x
$$
\begin{CJK}{UTF8}{mj}对所有的\end{CJK} $x \in\left(0, \frac{\pi}{2}\right)$ \begin{CJK}{UTF8}{mj}成立\end{CJK}.

\begin{enumerate}
  \setcounter{enumi}{5}
  \item (15 \begin{CJK}{UTF8}{mj}分\end{CJK}) \begin{CJK}{UTF8}{mj}设\end{CJK}
\end{enumerate}
$$
f(x, y)= \begin{cases}\frac{x y^{2}}{x^{2}+y^{2}}, & (x, y) \neq(0,0) \\ 0, & (x, y)=(0,0)\end{cases}
$$
\begin{CJK}{UTF8}{mj}证明\end{CJK}:

(1) $f$ \begin{CJK}{UTF8}{mj}在\end{CJK} $(0,0)$ \begin{CJK}{UTF8}{mj}处连续\end{CJK};

(2) $f$ \begin{CJK}{UTF8}{mj}在\end{CJK} $(0,0)$ \begin{CJK}{UTF8}{mj}处沿任意方向的方向导数都存在\end{CJK};

(3) $f$ \begin{CJK}{UTF8}{mj}在\end{CJK} $(0,0)$ \begin{CJK}{UTF8}{mj}处不可微\end{CJK}.

\begin{enumerate}
  \setcounter{enumi}{6}
  \item (15 \begin{CJK}{UTF8}{mj}分\end{CJK}) \begin{CJK}{UTF8}{mj}证明\end{CJK}: \begin{CJK}{UTF8}{mj}函数\end{CJK} $f(x)=\frac{\sin x}{x}$ \begin{CJK}{UTF8}{mj}在\end{CJK} $(0,+\infty)$ \begin{CJK}{UTF8}{mj}中一致连续\end{CJK}.

  \item ( 15 \begin{CJK}{UTF8}{mj}分\end{CJK}) \begin{CJK}{UTF8}{mj}把周期为\end{CJK} $2 \pi$ \begin{CJK}{UTF8}{mj}的函数\end{CJK}

\end{enumerate}
$$
f(x)=x^{2}, x \in[-\pi, \pi]
$$
\begin{CJK}{UTF8}{mj}展开为\end{CJK} Fourier \begin{CJK}{UTF8}{mj}级数\end{CJK}, \begin{CJK}{UTF8}{mj}并计算下列级数的和\end{CJK}
$$
\sum_{n=1}^{\infty} \frac{1}{n^{2}}, \sum_{n=1}^{\infty} \frac{1}{n^{4}}, \sum_{n=1}^{\infty}(-1)^{n} \frac{\cos n x}{n^{2}}
$$

\begin{enumerate}
  \setcounter{enumi}{8}
  \item (15 \begin{CJK}{UTF8}{mj}分\end{CJK}) \begin{CJK}{UTF8}{mj}设\end{CJK} $f$ \begin{CJK}{UTF8}{mj}是\end{CJK} $[a, b]$ \begin{CJK}{UTF8}{mj}上的正值可积函数\end{CJK}. \begin{CJK}{UTF8}{mj}证明\end{CJK}: \begin{CJK}{UTF8}{mj}存在\end{CJK} $c \in(a, b)$ \begin{CJK}{UTF8}{mj}使得\end{CJK}
\end{enumerate}
$$
\int_{a}^{c} f(x) \mathrm{d} x=\int_{c}^{b} f(x) \mathrm{d} x=\frac{1}{2} \int_{a}^{b} f(x) \mathrm{d} x .
$$

\begin{enumerate}
  \setcounter{enumi}{9}
  \item ( 15 \begin{CJK}{UTF8}{mj}分\end{CJK}) \begin{CJK}{UTF8}{mj}设\end{CJK} $f \in C^{1}\left(\mathbb{R}^{3}\right), a, b, c$ \begin{CJK}{UTF8}{mj}是非零实数\end{CJK}. \begin{CJK}{UTF8}{mj}证明\end{CJK}: \begin{CJK}{UTF8}{mj}在\end{CJK} $\mathbb{R}^{3}$ \begin{CJK}{UTF8}{mj}上成立\end{CJK}
\end{enumerate}
$$
\frac{1}{a} \frac{\partial f}{\partial x}=\frac{1}{b} \frac{\partial f}{\partial y}=\frac{1}{c} \frac{\partial f}{\partial z}
$$
\begin{CJK}{UTF8}{mj}的充分必要条件是存在\end{CJK} $g \in C^{1}(\mathbb{R})$ \begin{CJK}{UTF8}{mj}使得\end{CJK}
$$
f(x, y, z)=g(a x+b y+c z)
$$

\begin{enumerate}
  \setcounter{enumi}{10}
  \item ( 15 \begin{CJK}{UTF8}{mj}分\end{CJK}) \begin{CJK}{UTF8}{mj}设\end{CJK} $a>0, a c-b^{2}>0, \alpha>\frac{1}{2}$. \begin{CJK}{UTF8}{mj}证明\end{CJK}:
\end{enumerate}
$$
\int_{-\infty}^{+\infty} \frac{\mathrm{d} x}{\left(a x^{2}+2 b x+c\right)^{\alpha}}=\frac{\left(a c-b^{2}\right)^{\frac{1}{2}-\alpha}}{a^{1-\alpha}} \frac{\Gamma\left(\alpha-\frac{1}{2}\right)}{\Gamma(\alpha)} \sqrt{\pi}
$$
\begin{CJK}{UTF8}{mj}其中\end{CJK} $\Gamma$ \begin{CJK}{UTF8}{mj}是\end{CJK} Gamma \begin{CJK}{UTF8}{mj}函数\end{CJK}. 14. \begin{CJK}{UTF8}{mj}中国科学技术大学\end{CJK} 2014 \begin{CJK}{UTF8}{mj}年研究生入学考试试题数学分析\end{CJK}

\begin{CJK}{UTF8}{mj}李扬\end{CJK}

\begin{CJK}{UTF8}{mj}微信公众号\end{CJK}: sxkyliyang

\begin{enumerate}
  \item ( 15 \begin{CJK}{UTF8}{mj}分\end{CJK}) \begin{CJK}{UTF8}{mj}回答下列问题\end{CJK}, \begin{CJK}{UTF8}{mj}举例说明或证明你的结论\end{CJK}.
\end{enumerate}
(1) \begin{CJK}{UTF8}{mj}如果\end{CJK} $\int_{0}^{+\infty} f(x) \mathrm{d} x$ \begin{CJK}{UTF8}{mj}收敛\end{CJK}, \begin{CJK}{UTF8}{mj}能否断言\end{CJK} $\lim _{x \rightarrow+\infty} f(x)=0$ ?

(2) \begin{CJK}{UTF8}{mj}如果\end{CJK} $\sum_{n=1}^{\infty} a_{n}^{2}<+\infty$, \begin{CJK}{UTF8}{mj}能否断言\end{CJK} $\sum_{n=1}^{\infty} a_{n}$ \begin{CJK}{UTF8}{mj}和\end{CJK} $\sum_{n=1}^{\infty}(-1)^{n-1} a_{n}$ \begin{CJK}{UTF8}{mj}至少有一个收敛\end{CJK}?

\begin{enumerate}
  \setcounter{enumi}{2}
  \item (15 \begin{CJK}{UTF8}{mj}分\end{CJK}) \begin{CJK}{UTF8}{mj}计算下列极限的值\end{CJK}.
\end{enumerate}
(1) $\lim _{x \rightarrow+\infty}\left(x-x^{2} \ln \left(1+\frac{1}{x}\right)\right)$.

(2) $\lim _{x \rightarrow 0} \frac{\cos x-e^{-\frac{x^{2}}{2}}}{\sin ^{4} x}$.

\begin{enumerate}
  \setcounter{enumi}{3}
  \item (15 \begin{CJK}{UTF8}{mj}分\end{CJK}) \begin{CJK}{UTF8}{mj}计算下列积分\end{CJK}.
\end{enumerate}
(1) $\int_{0}^{1}\left(\int_{0}^{1} \max (x, y) \mathrm{d} y\right) \mathrm{d} x$.

(2) $\int_{0}^{1} \frac{\ln x}{x^{\alpha}} \mathrm{d} x, \alpha<1$.

\begin{enumerate}
  \setcounter{enumi}{4}
  \item (15 \begin{CJK}{UTF8}{mj}分\end{CJK}) \begin{CJK}{UTF8}{mj}设\end{CJK} $\left\{a_{n}\right\}_{n=1}^{\infty}$ \begin{CJK}{UTF8}{mj}是一个非负数列\end{CJK}, \begin{CJK}{UTF8}{mj}满足\end{CJK}
\end{enumerate}
$$
a_{n+1} \leqslant a_{n}+\frac{1}{n^{2}}, n=1,2, \cdots .
$$
\begin{CJK}{UTF8}{mj}证明\end{CJK}: $\left\{a_{n}\right\}$ \begin{CJK}{UTF8}{mj}收敛\end{CJK}.

\begin{enumerate}
  \setcounter{enumi}{5}
  \item (15 \begin{CJK}{UTF8}{mj}分\end{CJK}) \begin{CJK}{UTF8}{mj}设\end{CJK} $f$ \begin{CJK}{UTF8}{mj}在\end{CJK} $(0,+\infty)$ \begin{CJK}{UTF8}{mj}上有三阶导数\end{CJK}, \begin{CJK}{UTF8}{mj}如果\end{CJK} $\lim _{x \rightarrow+\infty} f(x)$ \begin{CJK}{UTF8}{mj}和\end{CJK} $\lim _{x \rightarrow+\infty} f^{\prime \prime \prime}(x)$ \begin{CJK}{UTF8}{mj}都存在且有限\end{CJK}, \begin{CJK}{UTF8}{mj}证明\end{CJK}:
\end{enumerate}
$$
\lim _{x \rightarrow+\infty} f^{\prime}(x)=\lim _{x \rightarrow+\infty} f^{\prime \prime}(x)=\lim _{x \rightarrow+\infty} f^{\prime \prime \prime}(x)=0 .
$$

\begin{enumerate}
  \setcounter{enumi}{6}
  \item (15 \begin{CJK}{UTF8}{mj}分\end{CJK}) \begin{CJK}{UTF8}{mj}设\end{CJK} $f, g$ \begin{CJK}{UTF8}{mj}都是\end{CJK} $[a, b]$ \begin{CJK}{UTF8}{mj}中的连续函数\end{CJK}, \begin{CJK}{UTF8}{mj}证明\end{CJK}: \begin{CJK}{UTF8}{mj}存在\end{CJK} $\xi \in[a, b]$, \begin{CJK}{UTF8}{mj}使得\end{CJK}
\end{enumerate}
$$
g(\xi) \int_{a}^{\xi} f(x) \mathrm{d} x=f(\xi) \int_{\xi}^{b} g(x) \mathrm{d} x .
$$

\begin{enumerate}
  \setcounter{enumi}{7}
  \item (15 \begin{CJK}{UTF8}{mj}分\end{CJK}) \begin{CJK}{UTF8}{mj}设\end{CJK} $\lim _{n \rightarrow \infty} a_{n}=a \in \mathbb{R}$, \begin{CJK}{UTF8}{mj}证明\end{CJK}:
\end{enumerate}
(1) \begin{CJK}{UTF8}{mj}幂级数\end{CJK} $\sum_{n=0}^{\infty} a_{n} x^{n}$ \begin{CJK}{UTF8}{mj}的收敛半径\end{CJK} $R \geqslant 1$.

(2) \begin{CJK}{UTF8}{mj}设\end{CJK} $f(x)=\sum_{n=0}^{\infty} a_{n} x^{n}$, \begin{CJK}{UTF8}{mj}那么\end{CJK} $\lim _{x \rightarrow 1^{-}}(1-x) f(x)=a$.

(3) $\lim _{x \rightarrow 1^{-}}(1-x) \int_{0}^{x} \frac{f(t)}{1-t} \mathrm{~d} t=a$.

\begin{enumerate}
  \setcounter{enumi}{8}
  \item (15 \begin{CJK}{UTF8}{mj}分\end{CJK}) \begin{CJK}{UTF8}{mj}设\end{CJK} $z=z(x, y)$ \begin{CJK}{UTF8}{mj}在\end{CJK} $\mathbb{R}^{2}$ \begin{CJK}{UTF8}{mj}上有连续的二阶导数\end{CJK}, \begin{CJK}{UTF8}{mj}且满足方程\end{CJK}
\end{enumerate}
$$
14 \frac{\partial^{2} z}{\partial x^{2}}+5 \frac{\partial^{2} z}{\partial x \partial y}-\frac{\partial^{2} z}{\partial y^{2}}=0
$$
\begin{CJK}{UTF8}{mj}试确定\end{CJK} $\lambda$ \begin{CJK}{UTF8}{mj}的值\end{CJK}, \begin{CJK}{UTF8}{mj}使得在变换\end{CJK}
$$
\xi=x+\lambda y, \eta=x-2 y,(\lambda \neq 2)
$$
\begin{CJK}{UTF8}{mj}下\end{CJK}, \begin{CJK}{UTF8}{mj}方程被化简为\end{CJK}
$$
\frac{\partial^{2} z}{\partial \xi \partial \eta}=0
$$
\begin{CJK}{UTF8}{mj}并由此求出偏微分方程\end{CJK} (1) \begin{CJK}{UTF8}{mj}的解\end{CJK}.

\begin{enumerate}
  \setcounter{enumi}{9}
  \item (15 \begin{CJK}{UTF8}{mj}分\end{CJK}) \begin{CJK}{UTF8}{mj}设\end{CJK}
\end{enumerate}
$$
\vec{F}=\left(a-\frac{1}{y}+\frac{y}{z}, \frac{x}{z}+\frac{b x}{y^{2}},-\frac{c x y}{z^{2}}\right)
$$
\begin{CJK}{UTF8}{mj}其中\end{CJK} $a, b, c$ \begin{CJK}{UTF8}{mj}是三个常数\end{CJK}.\\
(1)\begin{CJK}{UTF8}{mj}问\end{CJK} $a, b, c$ \begin{CJK}{UTF8}{mj}取何值时\end{CJK}, $\vec{F}$ \begin{CJK}{UTF8}{mj}为有势场\end{CJK}.\\
(2) \begin{CJK}{UTF8}{mj}当\end{CJK} $\vec{F}$ \begin{CJK}{UTF8}{mj}为有势场时\end{CJK}, \begin{CJK}{UTF8}{mj}求出它的势函数\end{CJK}.

\begin{enumerate}
  \setcounter{enumi}{10}
  \item ( 15 \begin{CJK}{UTF8}{mj}分\end{CJK}) \begin{CJK}{UTF8}{mj}设\end{CJK} $u(x, y)$ \begin{CJK}{UTF8}{mj}在\end{CJK} $\mathbb{R}^{2}$ \begin{CJK}{UTF8}{mj}上有连续的二阶偏导数\end{CJK}, \begin{CJK}{UTF8}{mj}且恒取正值\end{CJK}. \begin{CJK}{UTF8}{mj}证明\end{CJK}: $u$ \begin{CJK}{UTF8}{mj}满足方程\end{CJK}
\end{enumerate}
$$
u \frac{\partial^{2} u}{\partial x \partial y}=\frac{\partial u}{\partial x} \frac{\partial u}{\partial y}
$$
\begin{CJK}{UTF8}{mj}的充分必要条件为\end{CJK} $u(x, y)=f(x) g(y)$. 15. \begin{CJK}{UTF8}{mj}中国科学技术大学\end{CJK} 2015 \begin{CJK}{UTF8}{mj}年研究生入学考试试题数学分析\end{CJK}

\begin{CJK}{UTF8}{mj}李扬\end{CJK}

\begin{CJK}{UTF8}{mj}微信公众号\end{CJK}: sxkyliyang

\begin{enumerate}
  \item (15 \begin{CJK}{UTF8}{mj}分\end{CJK}) \begin{CJK}{UTF8}{mj}求极限\end{CJK}
\end{enumerate}
$$
\lim _{x \rightarrow+\infty}\left(\sin \frac{1}{x}\right) \int_{0}^{x}|\sin t| \mathrm{d} t .
$$

\begin{enumerate}
  \setcounter{enumi}{2}
  \item ( 15 \begin{CJK}{UTF8}{mj}分\end{CJK}) \begin{CJK}{UTF8}{mj}求二元函数\end{CJK} $F(x, y)=\frac{x}{\sqrt{1+x^{2}}}+\frac{y}{\sqrt{1+y^{2}}}$ \begin{CJK}{UTF8}{mj}在闭区域\end{CJK} $x \geqslant 0, y \geqslant 0, x+y \leqslant 1$ \begin{CJK}{UTF8}{mj}上的最大值\end{CJK}.

  \item ( 15 \begin{CJK}{UTF8}{mj}分\end{CJK}) \begin{CJK}{UTF8}{mj}设\end{CJK} $a, b$ \begin{CJK}{UTF8}{mj}是正数\end{CJK}. \begin{CJK}{UTF8}{mj}计算二重积分\end{CJK}

\end{enumerate}
$$
\iint_{D}\left(x^{2}+y^{2}\right) \mathrm{d} x \mathrm{~d} y
$$
\begin{CJK}{UTF8}{mj}其中\end{CJK} $D$ \begin{CJK}{UTF8}{mj}是椭圆盘\end{CJK} $\frac{x^{2}}{a^{2}}+\frac{y^{2}}{b^{2}} \leqslant 1$.

\begin{enumerate}
  \setcounter{enumi}{4}
  \item (15 \begin{CJK}{UTF8}{mj}分\end{CJK}) \begin{CJK}{UTF8}{mj}设\end{CJK} $R>0$. \begin{CJK}{UTF8}{mj}计算曲面积分\end{CJK}
\end{enumerate}
$$
\iint_{S}\left(x y^{2}+\frac{1}{3} x^{3}\right) \mathrm{d} y \mathrm{~d} z+y z^{2} \mathrm{~d} z \mathrm{~d} x+R^{3} \mathrm{~d} x \mathrm{~d} y
$$
\begin{CJK}{UTF8}{mj}其中\end{CJK} $S$ \begin{CJK}{UTF8}{mj}是上半球面\end{CJK} $x^{2}+y^{2}+z^{2}=R^{2}(z \geqslant 0)$ \begin{CJK}{UTF8}{mj}方向朝上\end{CJK}.

\begin{enumerate}
  \setcounter{enumi}{5}
  \item ( 15 \begin{CJK}{UTF8}{mj}分\end{CJK} $)$ \begin{CJK}{UTF8}{mj}计算广义积分\end{CJK}
\end{enumerate}
$$
\int_{0}^{+\infty} \frac{1}{1+x^{n}} \mathrm{~d} x(n>1)
$$

\begin{enumerate}
  \setcounter{enumi}{6}
  \item (15 \begin{CJK}{UTF8}{mj}分\end{CJK}) \begin{CJK}{UTF8}{mj}设\end{CJK} $n>0$. \begin{CJK}{UTF8}{mj}求证不等式\end{CJK}
\end{enumerate}
$$
\frac{1}{2 n+2}<\int_{0}^{\frac{\pi}{4}} \tan ^{n} x \mathrm{~d} x<\frac{1}{2 n} .
$$

\begin{enumerate}
  \setcounter{enumi}{7}
  \item ( 15 \begin{CJK}{UTF8}{mj}分\end{CJK}) \begin{CJK}{UTF8}{mj}设\end{CJK} $\alpha \in(0,1),\left\{a_{n}\right\}$ \begin{CJK}{UTF8}{mj}是正严格递增数列\end{CJK}, \begin{CJK}{UTF8}{mj}且\end{CJK} $\left\{a_{n+1}-a_{n}\right\}$ \begin{CJK}{UTF8}{mj}有界\end{CJK}. \begin{CJK}{UTF8}{mj}求极限\end{CJK} $\lim _{n \rightarrow \infty}\left(a_{n+1}^{\alpha}-a_{n}^{\alpha}\right)$.

  \item (15 \begin{CJK}{UTF8}{mj}分\end{CJK}) \begin{CJK}{UTF8}{mj}讨论级数\end{CJK} $\sum_{n=1}^{\infty}(\sqrt{n+1}-\sqrt{n})^{\alpha} \cos n$ \begin{CJK}{UTF8}{mj}的条件收敛性和绝对收敛性\end{CJK}.

  \item ( 15 \begin{CJK}{UTF8}{mj}分\end{CJK}) \begin{CJK}{UTF8}{mj}设\end{CJK} $f(x)$ \begin{CJK}{UTF8}{mj}是区间\end{CJK} $[0,1]$ \begin{CJK}{UTF8}{mj}上的连续函数并满足\end{CJK} $0 \leqslant f(x) \leqslant x$. \begin{CJK}{UTF8}{mj}求证\end{CJK}:

\end{enumerate}
$$
\int_{0}^{1} x^{2} f(x) \mathrm{d} x \geqslant\left(\int_{0}^{1} f(x) \mathrm{d} x\right)^{2}
$$
\begin{CJK}{UTF8}{mj}并求使上式成为等式的所有连续函数\end{CJK} $f(x)$.

\begin{enumerate}
  \setcounter{enumi}{10}
  \item ( 15 \begin{CJK}{UTF8}{mj}分\end{CJK}) \begin{CJK}{UTF8}{mj}设\end{CJK} $f(x)$ \begin{CJK}{UTF8}{mj}在\end{CJK} $[a,+\infty)$ \begin{CJK}{UTF8}{mj}上有连续的导函数\end{CJK}, \begin{CJK}{UTF8}{mj}且\end{CJK}
\end{enumerate}
$$
\lim _{x \rightarrow+\infty} \sup \left|f(x)+f^{\prime}(x)\right| \leqslant M<+\infty .
$$
\begin{CJK}{UTF8}{mj}求证\end{CJK}: $\lim _{x \rightarrow+\infty} \sup |f(x)| \leqslant M$. 16. \begin{CJK}{UTF8}{mj}中国科学技术大学\end{CJK} 2016 \begin{CJK}{UTF8}{mj}年研究生入学考试试题数学分析\end{CJK}

\begin{CJK}{UTF8}{mj}李扬\end{CJK}

\begin{CJK}{UTF8}{mj}微信公众号\end{CJK}: sxkyliyang

\begin{enumerate}
  \item (15 \begin{CJK}{UTF8}{mj}分\end{CJK}) \begin{CJK}{UTF8}{mj}求极限\end{CJK}
\end{enumerate}
$$
\lim _{x \rightarrow+\infty} \frac{1}{x^{4}+|\sin x|} \int_{0}^{x^{2}} \frac{t^{3}}{1+t^{2}} \mathrm{~d} t .
$$

\begin{enumerate}
  \setcounter{enumi}{2}
  \item (15 \begin{CJK}{UTF8}{mj}分\end{CJK}) \begin{CJK}{UTF8}{mj}数列\end{CJK} $\left\{a_{n}\right\}$ \begin{CJK}{UTF8}{mj}满足\end{CJK}
\end{enumerate}
$$
a_{1}=3, a_{n+1}=\frac{1}{1+a_{n}}, n \geqslant 1
$$
\begin{CJK}{UTF8}{mj}证明\end{CJK}: \begin{CJK}{UTF8}{mj}数列\end{CJK} $\left\{a_{n}\right\}$ \begin{CJK}{UTF8}{mj}收敛\end{CJK}, \begin{CJK}{UTF8}{mj}并求其极限\end{CJK}.

\begin{enumerate}
  \setcounter{enumi}{3}
  \item (15 \begin{CJK}{UTF8}{mj}分\end{CJK}) \begin{CJK}{UTF8}{mj}证明\end{CJK}: $\int_{0}^{+\infty} \frac{\sin x}{x} e^{-x u} \mathrm{~d} x$ \begin{CJK}{UTF8}{mj}在\end{CJK} $[0,+\infty)$ \begin{CJK}{UTF8}{mj}一致收敛\end{CJK}.

  \item ( 15 \begin{CJK}{UTF8}{mj}分\end{CJK}) \begin{CJK}{UTF8}{mj}计算积分\end{CJK}

\end{enumerate}
$$
\iint_{S} z\left(x^{2}+y^{2}\right)^{3} \sqrt{1-\left(x^{2}+y^{2}\right)} \mathrm{d} S
$$
\begin{CJK}{UTF8}{mj}其中\end{CJK} $S$ \begin{CJK}{UTF8}{mj}是上半球面\end{CJK} $x^{2}+y^{2}+z^{2}=1, z>0$.

\begin{enumerate}
  \setcounter{enumi}{5}
  \item (15 \begin{CJK}{UTF8}{mj}分\end{CJK}) \begin{CJK}{UTF8}{mj}设函数\end{CJK} $f(x)$ \begin{CJK}{UTF8}{mj}以\end{CJK} $2 \pi$ \begin{CJK}{UTF8}{mj}为周期\end{CJK},
\end{enumerate}
$$
f(x)=1-x, x \in[-\pi, \pi) .
$$
\begin{CJK}{UTF8}{mj}求\end{CJK} $f(x)$ \begin{CJK}{UTF8}{mj}的\end{CJK} Fourier \begin{CJK}{UTF8}{mj}级数\end{CJK}, \begin{CJK}{UTF8}{mj}说明其\end{CJK} Fourier \begin{CJK}{UTF8}{mj}级数是否一致收敛\end{CJK}.

\begin{enumerate}
  \setcounter{enumi}{6}
  \item ( 15 \begin{CJK}{UTF8}{mj}分\end{CJK}) \begin{CJK}{UTF8}{mj}证明\end{CJK}: \begin{CJK}{UTF8}{mj}若函数\end{CJK} $f(x)$ \begin{CJK}{UTF8}{mj}的导数\end{CJK} $f^{\prime}(x)$ \begin{CJK}{UTF8}{mj}在区间\end{CJK} $(0,1)$ \begin{CJK}{UTF8}{mj}内有界\end{CJK}, \begin{CJK}{UTF8}{mj}则函数\end{CJK} $f(x)$ \begin{CJK}{UTF8}{mj}在区间\end{CJK} $(0,1)$ \begin{CJK}{UTF8}{mj}内有界\end{CJK}.

  \item (15 \begin{CJK}{UTF8}{mj}分\end{CJK}) \begin{CJK}{UTF8}{mj}设\end{CJK} $f(x)$ \begin{CJK}{UTF8}{mj}在区间\end{CJK} $[0,3]$ \begin{CJK}{UTF8}{mj}上连续\end{CJK}, \begin{CJK}{UTF8}{mj}在\end{CJK} $(0,3)$ \begin{CJK}{UTF8}{mj}内可导\end{CJK}, \begin{CJK}{UTF8}{mj}且满足\end{CJK}

\end{enumerate}
$$
f(0)+f(1)+f(2)=3, f(3)=1 .
$$
\begin{CJK}{UTF8}{mj}证明\end{CJK}: \begin{CJK}{UTF8}{mj}在区间\end{CJK} $(0,3)$ \begin{CJK}{UTF8}{mj}内存在一点\end{CJK} $\xi$, \begin{CJK}{UTF8}{mj}使得\end{CJK} $f^{\prime}(\xi)=0$.

\begin{enumerate}
  \setcounter{enumi}{8}
  \item (15 \begin{CJK}{UTF8}{mj}分\end{CJK}) \begin{CJK}{UTF8}{mj}设\end{CJK} $S$ \begin{CJK}{UTF8}{mj}是由椭圆\end{CJK} $\frac{x^{2}}{a^{2}}+\frac{y^{2}}{b^{2}}=1$ \begin{CJK}{UTF8}{mj}的切线与\end{CJK} 2 \begin{CJK}{UTF8}{mj}个坐标轴围成的区域的面积\end{CJK}, \begin{CJK}{UTF8}{mj}求\end{CJK} $S$ \begin{CJK}{UTF8}{mj}的最小值\end{CJK}.

  \item (15 \begin{CJK}{UTF8}{mj}分\end{CJK}) \begin{CJK}{UTF8}{mj}设函数\end{CJK} $f(x)$ \begin{CJK}{UTF8}{mj}在区间\end{CJK} $(0,1)$ \begin{CJK}{UTF8}{mj}上为凸函数\end{CJK}, \begin{CJK}{UTF8}{mj}即任给\end{CJK} $(0,1)$ \begin{CJK}{UTF8}{mj}中的两点\end{CJK} $x_{1}, x_{2}$, \begin{CJK}{UTF8}{mj}以及任意\end{CJK} $t \in(0,1)$ \begin{CJK}{UTF8}{mj}有\end{CJK}

\end{enumerate}
$$
f\left((1-t) x_{1}+t x_{2}\right) \leqslant(1-t) f\left(x_{1}\right)+t f\left(x_{2}\right)
$$
\begin{CJK}{UTF8}{mj}证明\end{CJK}: \begin{CJK}{UTF8}{mj}函数\end{CJK} $f(x)$ \begin{CJK}{UTF8}{mj}在区间\end{CJK} $(0,1)$ \begin{CJK}{UTF8}{mj}上连续\end{CJK}.

\begin{enumerate}
  \setcounter{enumi}{10}
  \item (15 \begin{CJK}{UTF8}{mj}分\end{CJK}) \begin{CJK}{UTF8}{mj}设\end{CJK}
\end{enumerate}
$$
F(x, y) \in C^{\infty}\left(\mathbb{R}^{2}\right), F(0,0)=0, F_{x}^{\prime}(0,0)=0, F_{y}^{\prime}(0,0) F_{x x}^{\prime \prime}(0,0)>0 .
$$
\begin{CJK}{UTF8}{mj}证明\end{CJK}: \begin{CJK}{UTF8}{mj}由\end{CJK} $F(x, y)=0$ \begin{CJK}{UTF8}{mj}确定的隐函数\end{CJK} $y=f(x)$ \begin{CJK}{UTF8}{mj}在\end{CJK} $x=0$ \begin{CJK}{UTF8}{mj}附近满足\end{CJK}
$$
f(x) \leqslant f(0)-\frac{1}{4} \frac{F_{x x}^{\prime \prime}(0,0)}{F_{y}^{\prime}(0,0)} x^{2} .
$$

\begin{enumerate}
  \setcounter{enumi}{17}
  \item \begin{CJK}{UTF8}{mj}中国科学技术大学\end{CJK} 2017 \begin{CJK}{UTF8}{mj}年研究生入学考试试题数学分析\end{CJK}
\end{enumerate}
\begin{CJK}{UTF8}{mj}李扬\end{CJK}

\begin{CJK}{UTF8}{mj}微信公众号\end{CJK}: sxkyliyang

\begin{enumerate}
  \item (15 \begin{CJK}{UTF8}{mj}分\end{CJK}) \begin{CJK}{UTF8}{mj}求极限\end{CJK}
\end{enumerate}
$$
\lim _{x \rightarrow 0} \frac{\int_{0}^{x^{2}} \sin t \mathrm{~d} t}{\tan x^{4}} .
$$

\begin{enumerate}
  \setcounter{enumi}{2}
  \item (15 \begin{CJK}{UTF8}{mj}分\end{CJK}) \begin{CJK}{UTF8}{mj}求第二型曲面积分\end{CJK}
\end{enumerate}
$$
\iint_{S} x^{3} \mathrm{~d} y \mathrm{~d} z+y^{3} \mathrm{~d} z \mathrm{~d} x+\left(z^{3}+1\right) \mathrm{d} x \mathrm{~d} y,
$$
\begin{CJK}{UTF8}{mj}其中\end{CJK}, $S$ \begin{CJK}{UTF8}{mj}是上半球面\end{CJK} $x^{2}+y^{2}+z^{2}=1$, \begin{CJK}{UTF8}{mj}方向沿球面外法向量向外\end{CJK}.

\begin{enumerate}
  \setcounter{enumi}{3}
  \item ( 15 \begin{CJK}{UTF8}{mj}分\end{CJK}) \begin{CJK}{UTF8}{mj}证明\end{CJK}:
\end{enumerate}
$$
\frac{2}{\pi} \int_{0}^{+\infty} \frac{\sin ^{2} u}{u^{2}} \cos (2 u x) \mathrm{d} u= \begin{cases}1-x, & x \in[0,1] \\ 0, & x>1\end{cases}
$$

\begin{enumerate}
  \setcounter{enumi}{4}
  \item ( 15 \begin{CJK}{UTF8}{mj}分\end{CJK}) \begin{CJK}{UTF8}{mj}设\end{CJK} $\alpha>0,\left\{a_{n}\right\}$ \begin{CJK}{UTF8}{mj}是递增趋于正无穷的正数列\end{CJK}. \begin{CJK}{UTF8}{mj}求证\end{CJK}:
\end{enumerate}
(1) $\frac{a_{k+1}-a_{k}}{a_{k+1}^{\alpha+1}} \leqslant \int_{a_{k}}^{a_{k+1}} \frac{1}{x^{\alpha+1}} \mathrm{~d} x$.

(2) $\sum_{k=1}^{\infty} \frac{a_{k+1}-a_{k}}{a_{k+1} a_{k}^{\alpha}}$ \begin{CJK}{UTF8}{mj}收敛\end{CJK}.

\begin{enumerate}
  \setcounter{enumi}{5}
  \item ( 15 \begin{CJK}{UTF8}{mj}分\end{CJK} $)$ \begin{CJK}{UTF8}{mj}设\end{CJK} $f(x, y) \in C^{1}\left(\mathbb{R}^{2}\right)$.
\end{enumerate}
(1) \begin{CJK}{UTF8}{mj}证明\end{CJK}: \begin{CJK}{UTF8}{mj}对于任意的\end{CJK} $a, b, c, d \in \mathbb{R}$, \begin{CJK}{UTF8}{mj}都存在\end{CJK} $(\xi, \eta) \in \mathbb{R}^{2}$, \begin{CJK}{UTF8}{mj}使得\end{CJK}
$$
f(a, b)-f(c, d)=(a-c) \frac{\partial f}{\partial x}(\xi, \eta)+(b-d) \frac{\partial f}{\partial y}(\xi, \eta) .
$$
(2) \begin{CJK}{UTF8}{mj}若\end{CJK} $\frac{\partial f}{\partial x}=\frac{\partial f}{\partial y}$, \begin{CJK}{UTF8}{mj}且\end{CJK} $f(x, 0)>0$ \begin{CJK}{UTF8}{mj}对任意的\end{CJK} $x \in \mathbb{R}$ \begin{CJK}{UTF8}{mj}都成立\end{CJK}, \begin{CJK}{UTF8}{mj}证明\end{CJK}: \begin{CJK}{UTF8}{mj}对于任意的\end{CJK} $(x, y) \in \mathbb{R}^{2}$, \begin{CJK}{UTF8}{mj}都有\end{CJK}
$$
f(x, y)>0
$$

\begin{enumerate}
  \setcounter{enumi}{6}
  \item ( 15 \begin{CJK}{UTF8}{mj}分\end{CJK}) \begin{CJK}{UTF8}{mj}证明\end{CJK}:
\end{enumerate}
$$
\int_{0}^{+\infty} \sin \left(x^{2}\right) \mathrm{d} x
$$
\begin{CJK}{UTF8}{mj}条件收敛\end{CJK}.

\begin{enumerate}
  \setcounter{enumi}{7}
  \item (15 \begin{CJK}{UTF8}{mj}分\end{CJK}) \begin{CJK}{UTF8}{mj}设\end{CJK} $D$ \begin{CJK}{UTF8}{mj}是光滑封闭曲线\end{CJK} $L$ \begin{CJK}{UTF8}{mj}所围的区域\end{CJK}, \begin{CJK}{UTF8}{mj}函数\end{CJK} $f(x, y)$ \begin{CJK}{UTF8}{mj}在\end{CJK} $\bar{D}$ \begin{CJK}{UTF8}{mj}上有二阶连续偏导数\end{CJK}, \begin{CJK}{UTF8}{mj}且满足\end{CJK} $\frac{\partial^{2} f}{\partial x^{2}}+\frac{\partial^{2} f}{\partial y^{2}}=0$.
\end{enumerate}
(1) \begin{CJK}{UTF8}{mj}求证\end{CJK}:
$$
\oint_{L}-f(x, y) \frac{\partial f}{\partial y} \mathrm{~d} x+f(x, y) \frac{\partial f}{\partial x} \mathrm{~d} y \geqslant 0
$$
(2) \begin{CJK}{UTF8}{mj}若\end{CJK} $f$ \begin{CJK}{UTF8}{mj}在\end{CJK} $L$ \begin{CJK}{UTF8}{mj}上恒为常数\end{CJK} $c$, \begin{CJK}{UTF8}{mj}求证\end{CJK}: $f$ \begin{CJK}{UTF8}{mj}在\end{CJK} $D$ \begin{CJK}{UTF8}{mj}上也恒为常数\end{CJK} $c$.

\begin{enumerate}
  \setcounter{enumi}{8}
  \item ( 15 \begin{CJK}{UTF8}{mj}分\end{CJK}) \begin{CJK}{UTF8}{mj}设\end{CJK} $f:[0,+\infty) \rightarrow[0,+\infty)$ \begin{CJK}{UTF8}{mj}是一致连续的\end{CJK}, $\alpha \in(0,1]$. \begin{CJK}{UTF8}{mj}求证\end{CJK}:\begin{CJK}{UTF8}{mj}函数\end{CJK} $g(x)=f^{\alpha}(x)$ \begin{CJK}{UTF8}{mj}也在\end{CJK} $[0,+\infty)$ \begin{CJK}{UTF8}{mj}上一致连\end{CJK} \begin{CJK}{UTF8}{mj}续\end{CJK}.

  \item (15 \begin{CJK}{UTF8}{mj}分\end{CJK}) \begin{CJK}{UTF8}{mj}设\end{CJK} $F(u, v) \in C^{1}(\mathbb{R})^{2}$, \begin{CJK}{UTF8}{mj}且\end{CJK} $F\left(x-\frac{z}{y}, y-\frac{z}{x}\right)=0$. \begin{CJK}{UTF8}{mj}证明\end{CJK}:

\end{enumerate}
$$
\left(x F_{u}+y F_{v}\right)\left(x y+z-x \frac{\partial z}{\partial x}-y \frac{\partial z}{\partial y}\right)=0 .
$$

\begin{enumerate}
  \setcounter{enumi}{10}
  \item (15 \begin{CJK}{UTF8}{mj}分\end{CJK}) \begin{CJK}{UTF8}{mj}设区间\end{CJK} $I=[0,1], f_{n}(x)$ \begin{CJK}{UTF8}{mj}与\end{CJK} $f(x)$ \begin{CJK}{UTF8}{mj}均是\end{CJK} $I$ \begin{CJK}{UTF8}{mj}上的连续函数\end{CJK}, $n, 1,2, \cdots$. \begin{CJK}{UTF8}{mj}且\end{CJK}
\end{enumerate}
$$
\begin{gathered}
f_{n}(x) \geqslant f_{n+1}(x), \forall x \in[0,1] . \\
\lim _{n \rightarrow \infty} f_{n}(x)=f(x)
\end{gathered}
$$
\begin{CJK}{UTF8}{mj}求证\end{CJK}:

(1) \begin{CJK}{UTF8}{mj}对于任意的\end{CJK} $\varepsilon>0$,
$$
I=\bigcup_{n=1}^{\infty}\left\{x \mid x \in I, f_{n}(x)-f(x)<\varepsilon\right\}
$$
(2) $f_{n}(x)$ \begin{CJK}{UTF8}{mj}一致收敛于\end{CJK} $f(x)$. 18. \begin{CJK}{UTF8}{mj}中国科学技术大学\end{CJK} 2018 \begin{CJK}{UTF8}{mj}年研究生入学考试试题数学分析\end{CJK}

\begin{CJK}{UTF8}{mj}李扬\end{CJK}

\begin{CJK}{UTF8}{mj}微信公众号\end{CJK}: sxkyliyang

\begin{enumerate}
  \item (1) \begin{CJK}{UTF8}{mj}求极限\end{CJK}
\end{enumerate}
$$
\lim _{x \rightarrow-\infty}|x|^{\arctan x+\frac{\pi}{2}} ;
$$
(2) \begin{CJK}{UTF8}{mj}已知\end{CJK} $a_{k}$ \begin{CJK}{UTF8}{mj}为正数数列\end{CJK}, \begin{CJK}{UTF8}{mj}且\end{CJK} $\left|\sum_{k=1}^{\infty} \frac{\sin \left(a_{k} x\right)}{k^{2}}\right| \leqslant|\tan x| . x \in(-1,1)$. \begin{CJK}{UTF8}{mj}证明\end{CJK}:
$$
a_{k}=o\left(k^{2}\right), k \rightarrow+\infty
$$

\begin{enumerate}
  \setcounter{enumi}{2}
  \item \begin{CJK}{UTF8}{mj}设\end{CJK} $\Phi(x)$ \begin{CJK}{UTF8}{mj}为周期为\end{CJK} 1 \begin{CJK}{UTF8}{mj}的黎曼函数\end{CJK}.
\end{enumerate}
(1) \begin{CJK}{UTF8}{mj}求\end{CJK} $\Phi(x)$ \begin{CJK}{UTF8}{mj}的连续点和间断点的类型\end{CJK};

(2) \begin{CJK}{UTF8}{mj}计算积分\end{CJK}
$$
\int_{0}^{1} \Phi(x) \mathrm{d} x
$$

\begin{enumerate}
  \setcounter{enumi}{3}
  \item \begin{CJK}{UTF8}{mj}已知\end{CJK} $\Omega$ \begin{CJK}{UTF8}{mj}为\end{CJK} $\mathbb{R}^{3}$ \begin{CJK}{UTF8}{mj}中的有界域\end{CJK}, $\vec{n}$ \begin{CJK}{UTF8}{mj}为单位向量\end{CJK}. \begin{CJK}{UTF8}{mj}求证\end{CJK}: \begin{CJK}{UTF8}{mj}存在以\end{CJK} $\vec{n}$ \begin{CJK}{UTF8}{mj}为法向量的平面平分\end{CJK} $\Omega$ \begin{CJK}{UTF8}{mj}的体积\end{CJK}.

  \item \begin{CJK}{UTF8}{mj}已知\end{CJK} $f(x)$ \begin{CJK}{UTF8}{mj}为周期等于\end{CJK} $2 \pi$ \begin{CJK}{UTF8}{mj}的奇函数\end{CJK}, \begin{CJK}{UTF8}{mj}当\end{CJK} $x \in(0, \pi)$ \begin{CJK}{UTF8}{mj}时\end{CJK}, $f(x)=-1$. \begin{CJK}{UTF8}{mj}试利用\end{CJK} $f$ \begin{CJK}{UTF8}{mj}的\end{CJK} Fourier \begin{CJK}{UTF8}{mj}级数计算\end{CJK} $\sum_{n=1}^{\infty} \frac{1}{(2 n-1)^{2}}$.

  \item \begin{CJK}{UTF8}{mj}设\end{CJK} $\varphi(x)$ \begin{CJK}{UTF8}{mj}为有势场\end{CJK} $F(x, y, z)=\left(x^{2}-y, y^{2}-x,-z^{2}\right)$ \begin{CJK}{UTF8}{mj}下的势函数\end{CJK}, \begin{CJK}{UTF8}{mj}求三重积分\end{CJK}

\end{enumerate}
$$
\iiint_{\Omega} \varphi(x, y, z) \mathrm{d} x \mathrm{~d} y \mathrm{~d} z
$$
\begin{CJK}{UTF8}{mj}其中\end{CJK} $\Omega$ \begin{CJK}{UTF8}{mj}为\end{CJK} $x^{2}+y^{2}+z^{2} \leqslant 1$.

\begin{enumerate}
  \setcounter{enumi}{6}
  \item \begin{CJK}{UTF8}{mj}已知\end{CJK} $f \in C^{2}[0,1], f(0)=f(1)=0$, \begin{CJK}{UTF8}{mj}且\end{CJK} $f(x)$ \begin{CJK}{UTF8}{mj}在\end{CJK} $x_{0}$ \begin{CJK}{UTF8}{mj}处取得最小值\end{CJK} $-1$,
\end{enumerate}
(1) \begin{CJK}{UTF8}{mj}求\end{CJK} $f(x)$ \begin{CJK}{UTF8}{mj}在\end{CJK} $x=x_{0}$ \begin{CJK}{UTF8}{mj}处的\end{CJK} Lagrange \begin{CJK}{UTF8}{mj}余项的\end{CJK} Taylor \begin{CJK}{UTF8}{mj}展开式\end{CJK};

(2) \begin{CJK}{UTF8}{mj}证明\end{CJK}: \begin{CJK}{UTF8}{mj}存在\end{CJK} $\xi \in(0,1)$ \begin{CJK}{UTF8}{mj}使得\end{CJK} $f^{\prime \prime}(\xi)=8$.

\begin{enumerate}
  \setcounter{enumi}{7}
  \item \begin{CJK}{UTF8}{mj}已知\end{CJK} $D_{t}=\left\{(x, y) \in \mathbb{R}^{2} \mid(x-t)^{2}+(y-t)^{2} \leqslant 1, y \geqslant t\right\}, f(t)=\iint_{D_{t}} \sqrt{x^{2}+y^{2}} \mathrm{~d} x \mathrm{~d} y$, \begin{CJK}{UTF8}{mj}计算\end{CJK} $f^{\prime}(0)$.

  \item \begin{CJK}{UTF8}{mj}已知\end{CJK} $u(x) \in C[0,1], u(x) \in C^{2}(0,1), u^{\prime \prime}(x) \geqslant 0$, \begin{CJK}{UTF8}{mj}令\end{CJK} $v(x)=u(x)+\varepsilon x^{2}, \varepsilon>0$.

\end{enumerate}
(1) \begin{CJK}{UTF8}{mj}证明\end{CJK}: $v(x)$ \begin{CJK}{UTF8}{mj}为\end{CJK} $(0,1)$ \begin{CJK}{UTF8}{mj}上的严格凸函数\end{CJK};

(2) \begin{CJK}{UTF8}{mj}证明\end{CJK}: $u(x)$ \begin{CJK}{UTF8}{mj}的最大值于端点处取得\end{CJK}.

\begin{enumerate}
  \setcounter{enumi}{9}
  \item \begin{CJK}{UTF8}{mj}已知\end{CJK} $B_{r}=\left\{(x, y) \in \mathbb{R}^{2} \mid x^{2}+y^{2} \leqslant r^{2}\right\}, B=B_{1}, u(x, y) \in C(\bar{B}) \cap C^{2}(B), \triangle u=\frac{\partial^{2} u}{\partial x^{2}}+\frac{\partial^{2} u}{\partial y^{2}}$
\end{enumerate}
(1) \begin{CJK}{UTF8}{mj}当\end{CJK} $\triangle u \geqslant 0, \forall(x, y) \in B$, \begin{CJK}{UTF8}{mj}证明\end{CJK}: $u(x, y)$ \begin{CJK}{UTF8}{mj}在\end{CJK} $\bar{B}$ \begin{CJK}{UTF8}{mj}上的最大值于边界\end{CJK} $\partial B$ \begin{CJK}{UTF8}{mj}上达到\end{CJK};

(2) \begin{CJK}{UTF8}{mj}当\end{CJK} $\triangle u=0, \forall(x, y) \in B$, \begin{CJK}{UTF8}{mj}证明\end{CJK}: $\frac{\mathrm{d}}{\mathrm{d} r}\left(\frac{1}{2 \pi r} \int_{\partial B_{r}} u(x, y) \mathrm{d} S\right)=0, \forall r \in(0,1)$.

(3) \begin{CJK}{UTF8}{mj}证明\end{CJK}: $u(0)=\frac{1}{2 \pi r} \int_{\partial B_{r}} u(x, y) \mathrm{d} S$.

\begin{enumerate}
  \setcounter{enumi}{10}
  \item \begin{CJK}{UTF8}{mj}已知\end{CJK} $u(x, t)$ \begin{CJK}{UTF8}{mj}具有二阶连续偏导\end{CJK}, \begin{CJK}{UTF8}{mj}且满足\end{CJK} $u_{t t}(x, t)=u_{x x}(x, t)$, \begin{CJK}{UTF8}{mj}记\end{CJK} $F(t)=\int_{t}^{2-t}\left(u_{t}^{2}(x, t)+u_{x}^{2}(x, t)\right) \mathrm{d} x$, \begin{CJK}{UTF8}{mj}证\end{CJK} \begin{CJK}{UTF8}{mj}明\end{CJK}: $\frac{\mathrm{d} F(t)}{\mathrm{d} t} \leqslant 0$.
\end{enumerate}
\section{1. 中国科学院大学 2009 年研究生入学考试试题高等代数 李扬 微信公众号: sxkyliyang}
\begin{CJK}{UTF8}{mj}一\end{CJK}、\begin{CJK}{UTF8}{mj}设\end{CJK} $A$ \begin{CJK}{UTF8}{mj}是\end{CJK} $n$ \begin{CJK}{UTF8}{mj}阶可逆矩阵\end{CJK}, $\alpha, \beta$ \begin{CJK}{UTF8}{mj}是\end{CJK} $n$ \begin{CJK}{UTF8}{mj}维列向量\end{CJK}, \begin{CJK}{UTF8}{mj}且\end{CJK} $1+\beta^{\prime} A^{-1} \alpha \neq 0$. \begin{CJK}{UTF8}{mj}证明\end{CJK}: $A+\alpha \beta^{\prime}$ \begin{CJK}{UTF8}{mj}可逆\end{CJK}, \begin{CJK}{UTF8}{mj}并求其逆\end{CJK}.

\begin{CJK}{UTF8}{mj}二\end{CJK}、\begin{CJK}{UTF8}{mj}证明\end{CJK}: \begin{CJK}{UTF8}{mj}如果两个实矩阵\end{CJK} $A, B$ \begin{CJK}{UTF8}{mj}在复数域上相似\end{CJK}, \begin{CJK}{UTF8}{mj}则它们在实数域上也相似\end{CJK}.

\section{2. 中国科学院大学 2010 年研究生入学考试试题高等代数 李扬 微信公众号: sxkyliyang}
\begin{CJK}{UTF8}{mj}一\end{CJK}、 (20 \begin{CJK}{UTF8}{mj}分\end{CJK}) \begin{CJK}{UTF8}{mj}设\end{CJK} $A, B$ \begin{CJK}{UTF8}{mj}分别是\end{CJK} $n \times m$ \begin{CJK}{UTF8}{mj}和\end{CJK} $m \times n$ \begin{CJK}{UTF8}{mj}矩阵\end{CJK}, $I_{k}$ \begin{CJK}{UTF8}{mj}是\end{CJK} $k$ \begin{CJK}{UTF8}{mj}阶单位矩阵\end{CJK}.

(1) \begin{CJK}{UTF8}{mj}求证\end{CJK}: $\left|I_{n}-A B\right|=\left|I_{m}-B A\right|$.

(2) \begin{CJK}{UTF8}{mj}计算行列式\end{CJK}
$$
D_{n}=\left|\begin{array}{cccc}
1+a_{1}+x_{1} & a_{1}+x_{2} & \cdots & a_{1}+x_{n} \\
a_{2}+x_{1} & 1+a_{2}+x_{2} & \cdots & a_{2}+x_{n} \\
\vdots & \vdots & \ddots & \vdots \\
a_{n}+x_{1} & a_{n}+x_{2} & \cdots & 1+a_{n}+x_{n}
\end{array}\right| .
$$
\begin{CJK}{UTF8}{mj}二\end{CJK}、 (20 \begin{CJK}{UTF8}{mj}分\end{CJK}) \begin{CJK}{UTF8}{mj}已知\end{CJK} 3 \begin{CJK}{UTF8}{mj}阶正交阵\end{CJK} $A$ \begin{CJK}{UTF8}{mj}的行列式为\end{CJK} 1 , \begin{CJK}{UTF8}{mj}求证\end{CJK}: $A$ \begin{CJK}{UTF8}{mj}的特征多项式一定为\end{CJK}
$$
f(\lambda)=\lambda^{3}-a \lambda^{2}+a \lambda-1,
$$
\begin{CJK}{UTF8}{mj}其中\end{CJK}, $a \in \mathbb{R}$ \begin{CJK}{UTF8}{mj}且\end{CJK} $-1 \leq a \leq 3$. (\begin{CJK}{UTF8}{mj}可与实数比较大小的数只有实数\end{CJK}, \begin{CJK}{UTF8}{mj}故条件\end{CJK} $a \in \mathbb{R}$ \begin{CJK}{UTF8}{mj}可略去\end{CJK})

\begin{CJK}{UTF8}{mj}三\end{CJK}、(20 \begin{CJK}{UTF8}{mj}分\end{CJK}) \begin{CJK}{UTF8}{mj}设\end{CJK} $A, B$ \begin{CJK}{UTF8}{mj}是\end{CJK} $n$ \begin{CJK}{UTF8}{mj}阶方阵\end{CJK}, $A$ \begin{CJK}{UTF8}{mj}可逆\end{CJK}, $B$ \begin{CJK}{UTF8}{mj}幂零\end{CJK}, $A B=B A$.

(1) \begin{CJK}{UTF8}{mj}求证\end{CJK}: $A+B$ \begin{CJK}{UTF8}{mj}可逆\end{CJK}.

(2) \begin{CJK}{UTF8}{mj}试举例说明要使\end{CJK} $(1)$ \begin{CJK}{UTF8}{mj}结论成立\end{CJK}, \begin{CJK}{UTF8}{mj}条件\end{CJK} $A B=B A$ \begin{CJK}{UTF8}{mj}是不可缺少的\end{CJK}.

\begin{CJK}{UTF8}{mj}四\end{CJK}、( 20 \begin{CJK}{UTF8}{mj}分\end{CJK}) \begin{CJK}{UTF8}{mj}求证\end{CJK}: \begin{CJK}{UTF8}{mj}任意\end{CJK} $n$ \begin{CJK}{UTF8}{mj}阶实方阵\end{CJK} $A$ \begin{CJK}{UTF8}{mj}的特征向量\end{CJK}, \begin{CJK}{UTF8}{mj}也是其伴随矩阵\end{CJK} $A^{*}$ \begin{CJK}{UTF8}{mj}的特征向量\end{CJK}.

\begin{CJK}{UTF8}{mj}五\end{CJK}、(30 \begin{CJK}{UTF8}{mj}分\end{CJK}) (1) $n$ \begin{CJK}{UTF8}{mj}阶方阵能表示成\end{CJK} $A=H+K$, \begin{CJK}{UTF8}{mj}其中\end{CJK} $H=\bar{H}^{T}, K=-\bar{K}^{T}$, \begin{CJK}{UTF8}{mj}矩阵\end{CJK} $\bar{B}^{T}$ \begin{CJK}{UTF8}{mj}表示矩阵\end{CJK} $B$ \begin{CJK}{UTF8}{mj}的共\end{CJK} \begin{CJK}{UTF8}{mj}轭转置\end{CJK}. \begin{CJK}{UTF8}{mj}设\end{CJK} $a, h, k$ \begin{CJK}{UTF8}{mj}分别是\end{CJK} $A, H, K$ \begin{CJK}{UTF8}{mj}中元素最大模\end{CJK}, \begin{CJK}{UTF8}{mj}若\end{CJK} $z=x+i y(x, y \in \mathbb{R})$ \begin{CJK}{UTF8}{mj}是\end{CJK} $A$ \begin{CJK}{UTF8}{mj}的任意特征值\end{CJK}, \begin{CJK}{UTF8}{mj}求证\end{CJK}:

$|z| \leq n a,|x| \leq n h,|y| \leq n k .$

(2) \begin{CJK}{UTF8}{mj}求证\end{CJK}: Hermite \begin{CJK}{UTF8}{mj}矩阵的特征值都是实数\end{CJK}.

(3) \begin{CJK}{UTF8}{mj}求证\end{CJK}: \begin{CJK}{UTF8}{mj}反对称矩阵的非零特征值都是纯虚数\end{CJK}.

\begin{CJK}{UTF8}{mj}六\end{CJK}、(15 \begin{CJK}{UTF8}{mj}分\end{CJK}) \begin{CJK}{UTF8}{mj}设\end{CJK} $\mathcal{A}$ \begin{CJK}{UTF8}{mj}是\end{CJK} $n$ \begin{CJK}{UTF8}{mj}维实线性空间\end{CJK} $V$ \begin{CJK}{UTF8}{mj}的线性变换\end{CJK}, $n \geq 1$. \begin{CJK}{UTF8}{mj}求证\end{CJK}: $\mathcal{A}$ \begin{CJK}{UTF8}{mj}至少存在一个一维或者二维的不变\end{CJK} \begin{CJK}{UTF8}{mj}子空间\end{CJK}。

\begin{CJK}{UTF8}{mj}七\end{CJK}、 ( 20 \begin{CJK}{UTF8}{mj}分\end{CJK}) \begin{CJK}{UTF8}{mj}设循环矩阵\end{CJK} $C$ \begin{CJK}{UTF8}{mj}为\end{CJK}
$$
\left(\begin{array}{cccc}
c_{0} & c_{1} & \cdots & c_{n-1} \\
c_{n-1} & c_{0} & \cdots & c_{n-2} \\
\vdots & \vdots & \ddots & \vdots \\
c_{1} & c_{2} & \cdots & c_{0}
\end{array}\right)
$$
(1) \begin{CJK}{UTF8}{mj}求\end{CJK} $C$ \begin{CJK}{UTF8}{mj}的全部特征值以及相应的特征向量\end{CJK}.

(2) \begin{CJK}{UTF8}{mj}求\end{CJK} $|C|$.

\begin{CJK}{UTF8}{mj}八\end{CJK}、 $\left(15\right.$ \begin{CJK}{UTF8}{mj}分\end{CJK}) \begin{CJK}{UTF8}{mj}设\end{CJK} $M_{n}(\mathbb{C})$ \begin{CJK}{UTF8}{mj}是复数域上所有\end{CJK} $n$ \begin{CJK}{UTF8}{mj}阶方阵组成的线性空间\end{CJK}, $T: M_{n}(\mathbb{C}) \rightarrow \mathbb{C}$ \begin{CJK}{UTF8}{mj}是线性映射\end{CJK}, \begin{CJK}{UTF8}{mj}满足\end{CJK} $T(A B)=T(B A)$, \begin{CJK}{UTF8}{mj}求证\end{CJK}: $\forall A \in M_{n}(\mathbb{C})$, \begin{CJK}{UTF8}{mj}总存在\end{CJK} $\lambda \in \mathbb{C}$ \begin{CJK}{UTF8}{mj}使得\end{CJK} $T(A)=\lambda \operatorname{tr}(A)$.

\section{3. 中国科学院大学 2011 年研究生入学考试试题高等代数 李扬 
 微信公众号: sxkyliyang}
\begin{CJK}{UTF8}{mj}一\end{CJK}、(20 \begin{CJK}{UTF8}{mj}分\end{CJK}) \begin{CJK}{UTF8}{mj}设\end{CJK} $\frac{p}{q}$ \begin{CJK}{UTF8}{mj}是既约分数\end{CJK}, $f(x)=a_{n} x_{n}+a_{n-1} x_{n-1}+\cdots+a_{1} x+a_{0}$ \begin{CJK}{UTF8}{mj}是整系数多项式\end{CJK}, \begin{CJK}{UTF8}{mj}而且\end{CJK} $f\left(\frac{p}{q}\right)=0$. \begin{CJK}{UTF8}{mj}证明\end{CJK}

(1) $p \mid a_{0}$, \begin{CJK}{UTF8}{mj}而\end{CJK} $q \mid a_{n}$.

(2) \begin{CJK}{UTF8}{mj}对任意正整数\end{CJK} $m$, \begin{CJK}{UTF8}{mj}有\end{CJK} $(p-m q) \mid f(m)$.

\begin{CJK}{UTF8}{mj}二\end{CJK}、 ( 20 \begin{CJK}{UTF8}{mj}分\end{CJK}) \begin{CJK}{UTF8}{mj}设\end{CJK} $n$ \begin{CJK}{UTF8}{mj}阶方阵\end{CJK} $A_{n}=(|i-j|)_{1 \leq i, j \leq n}$, \begin{CJK}{UTF8}{mj}其行列式记为\end{CJK} $D_{n}$. \begin{CJK}{UTF8}{mj}试证明\end{CJK}
$$
D_{n}+4 D_{n-1}+4 D_{n-2}=0,
$$
\begin{CJK}{UTF8}{mj}并由此求出行列式\end{CJK} $D_{n}$.

\begin{CJK}{UTF8}{mj}三\end{CJK}、 (16 \begin{CJK}{UTF8}{mj}分\end{CJK}) \begin{CJK}{UTF8}{mj}已知二阶矩阵\end{CJK} $A=\left(\begin{array}{ll}a & b \\ c & d\end{array}\right)$ \begin{CJK}{UTF8}{mj}的特征多项式为\end{CJK} $(\lambda-1)^{2}$, \begin{CJK}{UTF8}{mj}试求\end{CJK} $A^{2011}-2011 A$.

\begin{CJK}{UTF8}{mj}四\end{CJK}、(20 \begin{CJK}{UTF8}{mj}分\end{CJK}) \begin{CJK}{UTF8}{mj}设\end{CJK} $\alpha, \beta, \gamma$ \begin{CJK}{UTF8}{mj}是\end{CJK} 3 \begin{CJK}{UTF8}{mj}维线性空间\end{CJK} $V$ \begin{CJK}{UTF8}{mj}的一组基\end{CJK}, \begin{CJK}{UTF8}{mj}线性变换\end{CJK} $\mathcal{A}$ \begin{CJK}{UTF8}{mj}满足\end{CJK}
$$
\left\{\begin{array}{l}
\mathcal{A}(\alpha+2 \beta+\gamma)=\alpha \\
\mathcal{A}(3 \beta+4 \gamma)=\beta \\
\mathcal{A}(4 \beta+5 \gamma)=\gamma
\end{array}\right.
$$
\begin{CJK}{UTF8}{mj}试求\end{CJK} $\mathcal{A}$ \begin{CJK}{UTF8}{mj}在基\end{CJK} $\alpha, 2 \beta+\gamma, \gamma$ \begin{CJK}{UTF8}{mj}下的矩阵\end{CJK}.

\begin{CJK}{UTF8}{mj}五\end{CJK}、 (24 \begin{CJK}{UTF8}{mj}分\end{CJK}) \begin{CJK}{UTF8}{mj}已知矩阵\end{CJK} $A=\left(\begin{array}{ccc}2 & 2 & -2 \\ 2 & 5 & -4 \\ -2 & -4 & 5\end{array}\right)$.

(1) \begin{CJK}{UTF8}{mj}求\end{CJK} $A$ \begin{CJK}{UTF8}{mj}的特征多项式\end{CJK}, \begin{CJK}{UTF8}{mj}并确定其是否有重根\end{CJK}.

(2) \begin{CJK}{UTF8}{mj}求一个正交矩阵\end{CJK} $P$ \begin{CJK}{UTF8}{mj}使得\end{CJK} $P A P^{-1}$ \begin{CJK}{UTF8}{mj}为对角阵\end{CJK}.

(3) \begin{CJK}{UTF8}{mj}令\end{CJK} $V$ \begin{CJK}{UTF8}{mj}是所有与\end{CJK} $A$ \begin{CJK}{UTF8}{mj}可交换的实矩阵全体\end{CJK}, \begin{CJK}{UTF8}{mj}证明\end{CJK} $V$ \begin{CJK}{UTF8}{mj}是一个实数域上的线性空间\end{CJK}, \begin{CJK}{UTF8}{mj}并确定\end{CJK} $V$ \begin{CJK}{UTF8}{mj}的维数\end{CJK}. \begin{CJK}{UTF8}{mj}六\end{CJK}、 (20 \begin{CJK}{UTF8}{mj}分\end{CJK}) \begin{CJK}{UTF8}{mj}设\end{CJK} $A, B$ \begin{CJK}{UTF8}{mj}是两个\end{CJK} $n$ \begin{CJK}{UTF8}{mj}阶复方阵\end{CJK}, $n>1$.

(1) \begin{CJK}{UTF8}{mj}如果\end{CJK} $A B=B A$, \begin{CJK}{UTF8}{mj}证明\end{CJK} $A, B$ \begin{CJK}{UTF8}{mj}有公共的特征向量\end{CJK}.

(2) \begin{CJK}{UTF8}{mj}如果\end{CJK} $A B-B A=\mu B$, \begin{CJK}{UTF8}{mj}其中\end{CJK} $\mu$ \begin{CJK}{UTF8}{mj}是一个非零复数\end{CJK}, \begin{CJK}{UTF8}{mj}那么\end{CJK} $A, B$ \begin{CJK}{UTF8}{mj}是否会有公共的特征向量\end{CJK}? \begin{CJK}{UTF8}{mj}回答\end{CJK} “\begin{CJK}{UTF8}{mj}是\end{CJK}”\begin{CJK}{UTF8}{mj}请\end{CJK} \begin{CJK}{UTF8}{mj}给出证明\end{CJK}; \begin{CJK}{UTF8}{mj}回答\end{CJK} “\begin{CJK}{UTF8}{mj}否\end{CJK}”\begin{CJK}{UTF8}{mj}请给出反例\end{CJK}.

\begin{CJK}{UTF8}{mj}七\end{CJK}、(15 \begin{CJK}{UTF8}{mj}分\end{CJK}) \begin{CJK}{UTF8}{mj}设\end{CJK} $A$ \begin{CJK}{UTF8}{mj}是\end{CJK} $n$ \begin{CJK}{UTF8}{mj}阶实方阵\end{CJK}, \begin{CJK}{UTF8}{mj}其特征多项式有如下分解\end{CJK}
$$
p(\lambda)=\operatorname{det}(\lambda E-A)=\left(\lambda-\lambda_{1}\right)^{r_{1}}\left(\lambda-\lambda_{2}\right)^{r_{2}} \cdots\left(\lambda-\lambda_{s}\right)^{r_{s}},
$$
\begin{CJK}{UTF8}{mj}其中\end{CJK} $E$ \begin{CJK}{UTF8}{mj}为\end{CJK} $n$ \begin{CJK}{UTF8}{mj}阶单位方阵\end{CJK}, \begin{CJK}{UTF8}{mj}诸\end{CJK} $\lambda_{i}$ \begin{CJK}{UTF8}{mj}两两不相等\end{CJK}. \begin{CJK}{UTF8}{mj}试证明\end{CJK} $A$ \begin{CJK}{UTF8}{mj}的\end{CJK} Jordan \begin{CJK}{UTF8}{mj}标准形中以\end{CJK} $\lambda_{i}$ \begin{CJK}{UTF8}{mj}为特征值的\end{CJK} Jordan \begin{CJK}{UTF8}{mj}块\end{CJK} \begin{CJK}{UTF8}{mj}的个数等于特征子空间\end{CJK} $V_{\lambda_{i}}$ \begin{CJK}{UTF8}{mj}的维数\end{CJK}.

\begin{CJK}{UTF8}{mj}八\end{CJK}、 ( 15 \begin{CJK}{UTF8}{mj}分\end{CJK}) \begin{CJK}{UTF8}{mj}设\end{CJK} $A$ \begin{CJK}{UTF8}{mj}是\end{CJK} $n$ \begin{CJK}{UTF8}{mj}阶实方阵\end{CJK}, \begin{CJK}{UTF8}{mj}证明\end{CJK} $A$ \begin{CJK}{UTF8}{mj}为实对称阵当且仅当\end{CJK} $A A^{T}=A^{2}$, \begin{CJK}{UTF8}{mj}其中\end{CJK} $A^{T}$ \begin{CJK}{UTF8}{mj}表示矩阵\end{CJK} $A$ \begin{CJK}{UTF8}{mj}的转置\end{CJK}.

\section{4. 中国科学院大学 2012 年研究生入学考试试题高等代数 
 李扬 
 微信公众号: sxkyliyang}
\begin{CJK}{UTF8}{mj}一\end{CJK}、 (15 \begin{CJK}{UTF8}{mj}分\end{CJK}) \begin{CJK}{UTF8}{mj}证明\end{CJK}: \begin{CJK}{UTF8}{mj}多项式\end{CJK} $f(x)=1+\frac{x}{1 !}+\frac{x^{2}}{2 !}+\cdots+\frac{x^{n}}{n !}$ \begin{CJK}{UTF8}{mj}没有重根\end{CJK}.

\begin{CJK}{UTF8}{mj}二\end{CJK}、 (20 \begin{CJK}{UTF8}{mj}分\end{CJK}) \begin{CJK}{UTF8}{mj}设多项式\end{CJK} $g(x)=p^{k}(x) g_{1}(x)(k \geq 1)$, \begin{CJK}{UTF8}{mj}多项式\end{CJK} $p(x)$ \begin{CJK}{UTF8}{mj}与\end{CJK} $g_{1}(x)$ \begin{CJK}{UTF8}{mj}互素\end{CJK}. \begin{CJK}{UTF8}{mj}证明\end{CJK}: \begin{CJK}{UTF8}{mj}对任意的多项式\end{CJK} $f(x)$ \begin{CJK}{UTF8}{mj}有\end{CJK}
$$
\frac{f(x)}{g(x)}=\frac{r(x)}{p^{k} x}+\frac{f_{1}(x)}{p^{k-1} g_{1}(x)},
$$
\begin{CJK}{UTF8}{mj}其中\end{CJK} $r(x), f_{1}(x)$ \begin{CJK}{UTF8}{mj}都是多项式\end{CJK}, $r(x)=0$ \begin{CJK}{UTF8}{mj}或\end{CJK} $\operatorname{deg}(r(x))<\operatorname{deg}(p(x))$.

\begin{CJK}{UTF8}{mj}三\end{CJK}、 (20 \begin{CJK}{UTF8}{mj}分\end{CJK}) \begin{CJK}{UTF8}{mj}已知\end{CJK} $n$ \begin{CJK}{UTF8}{mj}阶方阵\end{CJK}
$$
A=\left(\begin{array}{cccc}
a_{1}^{2} & a_{1} a_{2}+1 & \cdots & a_{1} a_{n}+1 \\
a_{2} a_{1}+1 & a_{2}^{2} & \cdots & a_{2} a_{n}+1 \\
\cdots & \cdots & \cdots & \cdots \\
a_{n} a_{1}+1 & a_{n} a_{2}+1 & \cdots & a_{n}^{2}
\end{array}\right),
$$
\begin{CJK}{UTF8}{mj}其中\end{CJK} $\sum_{i=1}^{n} a_{i}=1, \sum_{i=1}^{n} a_{i}^{2}=n$.

(1) \begin{CJK}{UTF8}{mj}求\end{CJK} $A$ \begin{CJK}{UTF8}{mj}的全部特征值\end{CJK}.

(2) \begin{CJK}{UTF8}{mj}求\end{CJK} $A$ \begin{CJK}{UTF8}{mj}的行列式\end{CJK} $\operatorname{det}(A)$ \begin{CJK}{UTF8}{mj}和迹\end{CJK} $\operatorname{tr}(A)$.

\begin{CJK}{UTF8}{mj}四\end{CJK}、( 15 \begin{CJK}{UTF8}{mj}分\end{CJK}) \begin{CJK}{UTF8}{mj}设数域\end{CJK} $\mathbb{K}$ \begin{CJK}{UTF8}{mj}上的\end{CJK} $n$ \begin{CJK}{UTF8}{mj}阶方阵\end{CJK} $A$ \begin{CJK}{UTF8}{mj}满足\end{CJK} $A^{2}=A$, \begin{CJK}{UTF8}{mj}而\end{CJK} $V_{1}$, $V_{2}$ \begin{CJK}{UTF8}{mj}分别是齐次线性方程组\end{CJK} $A x=0$ \begin{CJK}{UTF8}{mj}和\end{CJK} $\left(A-I_{n}\right) x=0$ \begin{CJK}{UTF8}{mj}在\end{CJK} $\mathbb{K}^{n}$ \begin{CJK}{UTF8}{mj}中的解空间\end{CJK}, \begin{CJK}{UTF8}{mj}证明\end{CJK}: $\mathbb{K}^{n}=V_{1} \oplus V_{2}$, \begin{CJK}{UTF8}{mj}其中\end{CJK} $I_{n}$ \begin{CJK}{UTF8}{mj}代表\end{CJK} $n$ \begin{CJK}{UTF8}{mj}阶单位矩阵\end{CJK}, $\oplus$ \begin{CJK}{UTF8}{mj}表示直和\end{CJK}.

\begin{CJK}{UTF8}{mj}五\end{CJK}、 ( 20 \begin{CJK}{UTF8}{mj}分\end{CJK}) \begin{CJK}{UTF8}{mj}设\end{CJK} $n$ \begin{CJK}{UTF8}{mj}阶矩阵\end{CJK} $A$ \begin{CJK}{UTF8}{mj}可逆\end{CJK}, $\alpha, \beta$ \begin{CJK}{UTF8}{mj}均为\end{CJK} $n$ \begin{CJK}{UTF8}{mj}维列向量\end{CJK}, \begin{CJK}{UTF8}{mj}且\end{CJK} $1+\beta^{T} A^{-1} \alpha \neq 0$, \begin{CJK}{UTF8}{mj}其中\end{CJK} $\beta^{T}$ \begin{CJK}{UTF8}{mj}表示\end{CJK} $\beta$ \begin{CJK}{UTF8}{mj}的转置\end{CJK}.

(1) \begin{CJK}{UTF8}{mj}证明矩阵\end{CJK} $P=\left(\begin{array}{cc}A & \alpha \\ -\beta^{T} & 1\end{array}\right)$ \begin{CJK}{UTF8}{mj}可逆\end{CJK}, \begin{CJK}{UTF8}{mj}并求其逆矩阵\end{CJK}.

(2) \begin{CJK}{UTF8}{mj}证明矩阵\end{CJK} $Q=A+\alpha \beta^{T}$ \begin{CJK}{UTF8}{mj}可逆\end{CJK}, \begin{CJK}{UTF8}{mj}并求其逆矩阵\end{CJK}.

\begin{CJK}{UTF8}{mj}六\end{CJK}、 ( 20 \begin{CJK}{UTF8}{mj}分\end{CJK}) \begin{CJK}{UTF8}{mj}证明\end{CJK}: \begin{CJK}{UTF8}{mj}任何复数方阵\end{CJK} $A$ \begin{CJK}{UTF8}{mj}都与它的转置矩阵\end{CJK} $A^{T}$ \begin{CJK}{UTF8}{mj}相似\end{CJK}.

\begin{CJK}{UTF8}{mj}七\end{CJK}、(22 \begin{CJK}{UTF8}{mj}分\end{CJK}) \begin{CJK}{UTF8}{mj}在二阶实数矩阵构成的线性空间\end{CJK} $\mathbb{R}^{2 \times 2}$ \begin{CJK}{UTF8}{mj}中定义\end{CJK}:
$$
(A, B)=\operatorname{tr}\left(A^{T} B\right), \forall A, B \in \mathbb{R}^{2 \times 2},
$$
\begin{CJK}{UTF8}{mj}其中\end{CJK}, $A^{T}$ \begin{CJK}{UTF8}{mj}表示矩阵\end{CJK} $A$ \begin{CJK}{UTF8}{mj}的转置\end{CJK}, $\operatorname{tr}(X)$ \begin{CJK}{UTF8}{mj}表示矩阵\end{CJK} $X$ \begin{CJK}{UTF8}{mj}的迹\end{CJK}.

(1) \begin{CJK}{UTF8}{mj}证明\end{CJK} $(A, B)$ \begin{CJK}{UTF8}{mj}是线性空间\end{CJK} $\mathbb{R}^{2 \times 2}$ \begin{CJK}{UTF8}{mj}的内积\end{CJK}.

(2) \begin{CJK}{UTF8}{mj}设\end{CJK} $W$ \begin{CJK}{UTF8}{mj}是由\end{CJK} $A_{1}=\left(\begin{array}{ll}1 & 1 \\ 0 & 0\end{array}\right), A_{2}=\left(\begin{array}{ll}0 & 1 \\ 1 & 1\end{array}\right)$ \begin{CJK}{UTF8}{mj}生成的子空间\end{CJK}. \begin{CJK}{UTF8}{mj}求\end{CJK} $W^{\perp}$ \begin{CJK}{UTF8}{mj}的一组标准正交基\end{CJK}.

\begin{CJK}{UTF8}{mj}八\end{CJK}、(18 \begin{CJK}{UTF8}{mj}分\end{CJK}) \begin{CJK}{UTF8}{mj}设\end{CJK} $T_{1}, T_{2}, \cdots, T_{n}$ \begin{CJK}{UTF8}{mj}是数域\end{CJK} $\mathbb{F}$ \begin{CJK}{UTF8}{mj}上线性空间\end{CJK} $V$ \begin{CJK}{UTF8}{mj}的非零线性变换\end{CJK}, \begin{CJK}{UTF8}{mj}试证明存在向量\end{CJK} $\alpha \in V$, \begin{CJK}{UTF8}{mj}使得\end{CJK} $T_{i}(\alpha) \neq 0, i=1,2, \cdots, n .$

\section{5. 中国科学院大学 2013 年研究生入学考试试题高等代数 
 李扬 
 微信公众号: sxkyliyang}
\begin{CJK}{UTF8}{mj}一\end{CJK}、(15 \begin{CJK}{UTF8}{mj}分\end{CJK}) \begin{CJK}{UTF8}{mj}求下面\end{CJK} $n+1$ \begin{CJK}{UTF8}{mj}阶行列式的值\end{CJK}:
$$
D=\left|\begin{array}{cccccc}
s_{0} & s_{1} & s_{2} & \cdots & s_{n-1} & 1 \\
s_{1} & s_{2} & s_{3} & \cdots & s_{n} & x \\
s_{2} & s_{3} & s_{4} & \cdots & s_{n+1} & x^{2} \\
\cdots & \cdots & \cdots & \cdots & \cdots & \cdots \\
s_{n} & s_{n+1} & s_{n+2} & \cdots & s_{2 n-1} & x^{n}
\end{array}\right|
$$
\begin{CJK}{UTF8}{mj}其中\end{CJK}, $s_{k}=x_{1}^{k}+x_{2}^{k}+\cdots+x_{n}^{k}$.

\begin{CJK}{UTF8}{mj}二\end{CJK}、 ( 15 \begin{CJK}{UTF8}{mj}分\end{CJK}) \begin{CJK}{UTF8}{mj}假设矩阵\end{CJK} $A$ \begin{CJK}{UTF8}{mj}与\end{CJK} $B$ \begin{CJK}{UTF8}{mj}没有公共的特征根\end{CJK}, $f(x)$ \begin{CJK}{UTF8}{mj}是矩阵\end{CJK} $A$ \begin{CJK}{UTF8}{mj}的特征多项式\end{CJK}, \begin{CJK}{UTF8}{mj}证明以下结论\end{CJK}:

(1) \begin{CJK}{UTF8}{mj}矩阵\end{CJK} $f(B)$ \begin{CJK}{UTF8}{mj}可逆\end{CJK}.

(2) \begin{CJK}{UTF8}{mj}矩阵方程\end{CJK} $A X=X B$ \begin{CJK}{UTF8}{mj}只有零解\end{CJK}.

\begin{CJK}{UTF8}{mj}三\end{CJK}、 (15 \begin{CJK}{UTF8}{mj}分\end{CJK}) \begin{CJK}{UTF8}{mj}设\end{CJK} $A=\left(a_{i, j}\right)_{1 \leq i, j \leq n}$ \begin{CJK}{UTF8}{mj}是斜对称方阵\end{CJK}, \begin{CJK}{UTF8}{mj}即\end{CJK} $a_{i, j}=a_{n-j+1, n-i+1}(i, j=1,2, \cdots, n)$, \begin{CJK}{UTF8}{mj}证明\end{CJK}:\begin{CJK}{UTF8}{mj}若\end{CJK} $A$ \begin{CJK}{UTF8}{mj}可逆\end{CJK}, \begin{CJK}{UTF8}{mj}则其逆矩阵也是斜对称方阵\end{CJK}.

\begin{CJK}{UTF8}{mj}四\end{CJK}、 (20 \begin{CJK}{UTF8}{mj}分\end{CJK}) \begin{CJK}{UTF8}{mj}设二次曲面\end{CJK} $x^{2}+a y^{2}+z^{2}+2 b x y+2 x z+2 y z=4$ \begin{CJK}{UTF8}{mj}可以经由正交变换\end{CJK}
$$
\left(\begin{array}{l}
x \\
y \\
z
\end{array}\right)=P\left(\begin{array}{l}
\xi \\
\eta \\
\zeta
\end{array}\right)
$$
\begin{CJK}{UTF8}{mj}化成椭圆柱面方程\end{CJK} $\eta^{2}+4 \zeta^{2}=4$, \begin{CJK}{UTF8}{mj}试求\end{CJK} $a, b$ \begin{CJK}{UTF8}{mj}和正交矩阵\end{CJK} $P$.

\begin{CJK}{UTF8}{mj}五\end{CJK}、 ( 15 \begin{CJK}{UTF8}{mj}分\end{CJK}) \begin{CJK}{UTF8}{mj}假设\end{CJK} 3 \begin{CJK}{UTF8}{mj}阶实方阵\end{CJK} $A$ \begin{CJK}{UTF8}{mj}满足\end{CJK}: $A^{2}=E, A \neq \pm E$, \begin{CJK}{UTF8}{mj}其中\end{CJK} $E$ \begin{CJK}{UTF8}{mj}为单位方阵\end{CJK}. \begin{CJK}{UTF8}{mj}证明\end{CJK} $(\operatorname{tr}(A))^{2}=1$, \begin{CJK}{UTF8}{mj}其中\end{CJK} $\operatorname{tr}(A)$ \begin{CJK}{UTF8}{mj}表示矩阵\end{CJK} $A$ \begin{CJK}{UTF8}{mj}的迹\end{CJK}.

\begin{CJK}{UTF8}{mj}六\end{CJK}、 (15 \begin{CJK}{UTF8}{mj}分\end{CJK}) \begin{CJK}{UTF8}{mj}设\end{CJK} $A$ \begin{CJK}{UTF8}{mj}为\end{CJK} $n$ \begin{CJK}{UTF8}{mj}阶半正定实矩阵\end{CJK}. \begin{CJK}{UTF8}{mj}证明\end{CJK}: $|A+2013 E| \geq 2013^{n}$, \begin{CJK}{UTF8}{mj}等号成立当且仅当\end{CJK} $A=0$, \begin{CJK}{UTF8}{mj}其中\end{CJK} $E$ \begin{CJK}{UTF8}{mj}是\end{CJK} \begin{CJK}{UTF8}{mj}单位矩阵\end{CJK}.

\begin{CJK}{UTF8}{mj}七\end{CJK}、(15 \begin{CJK}{UTF8}{mj}分\end{CJK}) \begin{CJK}{UTF8}{mj}证明\end{CJK}:\begin{CJK}{UTF8}{mj}任何一个实方阵均可表示成两个对称矩阵的乘积\end{CJK}, \begin{CJK}{UTF8}{mj}其中至少有一个矩阵可逆\end{CJK}.

\begin{CJK}{UTF8}{mj}八\end{CJK}、 ( 15 \begin{CJK}{UTF8}{mj}分\end{CJK}) \begin{CJK}{UTF8}{mj}设\end{CJK} $A$ \begin{CJK}{UTF8}{mj}是一个\end{CJK} $3 \times 3$ \begin{CJK}{UTF8}{mj}正交矩阵\end{CJK}, \begin{CJK}{UTF8}{mj}证明\end{CJK}: $A$ \begin{CJK}{UTF8}{mj}可以写成\end{CJK} $C R$, \begin{CJK}{UTF8}{mj}其中\end{CJK} $C$ \begin{CJK}{UTF8}{mj}对应于\end{CJK} $\mathbb{R}^{3}$ \begin{CJK}{UTF8}{mj}中的旋转变换\end{CJK}, $R$ \begin{CJK}{UTF8}{mj}对\end{CJK} \begin{CJK}{UTF8}{mj}应于\end{CJK} $\mathbb{R}^{3}$ \begin{CJK}{UTF8}{mj}中的恒等变换或对应于\end{CJK} $\mathbb{R}^{3}$ \begin{CJK}{UTF8}{mj}中的镜面反射变换\end{CJK}, \begin{CJK}{UTF8}{mj}其中\end{CJK} $\mathbb{R}$ \begin{CJK}{UTF8}{mj}表示实数域\end{CJK}.

\begin{CJK}{UTF8}{mj}九\end{CJK}、 ( 10 \begin{CJK}{UTF8}{mj}分\end{CJK}) \begin{CJK}{UTF8}{mj}设\end{CJK} $V$ \begin{CJK}{UTF8}{mj}是数域\end{CJK} $\mathbb{F}$ \begin{CJK}{UTF8}{mj}上的有限维向量空间\end{CJK}, $\phi$ \begin{CJK}{UTF8}{mj}是\end{CJK} $V$ \begin{CJK}{UTF8}{mj}上的线性变换\end{CJK}. \begin{CJK}{UTF8}{mj}证明\end{CJK}: $V$ \begin{CJK}{UTF8}{mj}能够分解成两个子空\end{CJK} \begin{CJK}{UTF8}{mj}间的直和\end{CJK} $V=U \oplus W$. \begin{CJK}{UTF8}{mj}其中\end{CJK} $U, W$ \begin{CJK}{UTF8}{mj}满足\end{CJK}:\begin{CJK}{UTF8}{mj}对任意的\end{CJK} $u \in U$, \begin{CJK}{UTF8}{mj}存在正整数\end{CJK} $k$ \begin{CJK}{UTF8}{mj}使得\end{CJK} $\phi^{k}(u)=0$; \begin{CJK}{UTF8}{mj}对任意的\end{CJK} $w \in W$, \begin{CJK}{UTF8}{mj}存在\end{CJK} $v_{m} \in V$, \begin{CJK}{UTF8}{mj}使得\end{CJK} $w=\phi^{m}\left(v_{m}\right)$ \begin{CJK}{UTF8}{mj}对所有的正整数\end{CJK} $m$.

\begin{CJK}{UTF8}{mj}十\end{CJK}、(15 \begin{CJK}{UTF8}{mj}分\end{CJK}) \begin{CJK}{UTF8}{mj}设\end{CJK} $V$ \begin{CJK}{UTF8}{mj}是实数域\end{CJK} $\mathbb{R}$ \begin{CJK}{UTF8}{mj}上的\end{CJK} $n$ \begin{CJK}{UTF8}{mj}维线性空间\end{CJK}, $\phi$ \begin{CJK}{UTF8}{mj}是\end{CJK} $V$ \begin{CJK}{UTF8}{mj}上的线性变换\end{CJK}, \begin{CJK}{UTF8}{mj}满足\end{CJK} $\phi^{2}=-\varepsilon(\varepsilon$ \begin{CJK}{UTF8}{mj}是\end{CJK} $V$ \begin{CJK}{UTF8}{mj}上的恒等变\end{CJK} \begin{CJK}{UTF8}{mj}换\end{CJK}).

(1) \begin{CJK}{UTF8}{mj}证明\end{CJK} $n$ \begin{CJK}{UTF8}{mj}是偶数\end{CJK}.

$(2)$ \begin{CJK}{UTF8}{mj}若\end{CJK} $\psi$ \begin{CJK}{UTF8}{mj}是\end{CJK} $V$ \begin{CJK}{UTF8}{mj}上的线性变换\end{CJK}, \begin{CJK}{UTF8}{mj}满足\end{CJK} $\psi \phi=\phi \psi$, \begin{CJK}{UTF8}{mj}证明\end{CJK} $\operatorname{det}(\psi) \geq 0$.

\section{6. 中国科学院大学 2014 年研究生入学考试试题高等代数 李扬 微信公众号: sxkyliyang}
\begin{CJK}{UTF8}{mj}一\end{CJK}、\begin{CJK}{UTF8}{mj}多项式\end{CJK} $f(x)=f_{0}\left(x^{n}\right)+x f_{1}\left(x^{n}\right)+\cdots+x^{n-1} f_{n-1}\left(x^{n}\right)$, \begin{CJK}{UTF8}{mj}且\end{CJK} $x^{n}+x^{n-1}+x^{n-2}+\cdots+x+1$ \begin{CJK}{UTF8}{mj}整除\end{CJK} $f(x)$, \begin{CJK}{UTF8}{mj}求\end{CJK} \begin{CJK}{UTF8}{mj}证\end{CJK}: $f_{i}(1)=0$.

\begin{CJK}{UTF8}{mj}二\end{CJK}、\begin{CJK}{UTF8}{mj}设\end{CJK} $f(x), g(x)$ \begin{CJK}{UTF8}{mj}分别是\end{CJK} $m, n$ \begin{CJK}{UTF8}{mj}次多项式\end{CJK}(\begin{CJK}{UTF8}{mj}原题中\end{CJK} $m=2, n=3$ ), \begin{CJK}{UTF8}{mj}证明\end{CJK}:

(1) \begin{CJK}{UTF8}{mj}存在次数低于\end{CJK} $n$ \begin{CJK}{UTF8}{mj}的多项式\end{CJK} $u(x)$ \begin{CJK}{UTF8}{mj}与次数低于\end{CJK} $m$ \begin{CJK}{UTF8}{mj}的多项式\end{CJK} $v(x)$, \begin{CJK}{UTF8}{mj}使得\end{CJK} $u(x) f(x)+v(x) g(x)=r e s(f(x), g(x))$.

(2) $(f(x), g(x))=1$ \begin{CJK}{UTF8}{mj}当且仅当\end{CJK} $\operatorname{res}(f(x), g(x)) \neq 0$.

\begin{CJK}{UTF8}{mj}注\end{CJK}:\begin{CJK}{UTF8}{mj}对任意的多项式\end{CJK} $f(x)=a_{n} x^{n}+a_{n-1} x^{n-1}+\cdots+a_{1} x+a_{0}, g(x)=b_{m} x^{m}+b_{m-1} x^{m-1}+\cdots+b_{1} x+b_{0}$, \begin{CJK}{UTF8}{mj}我们定义\end{CJK} $f(x), g(x)$ \begin{CJK}{UTF8}{mj}的结式\end{CJK} $r e s(f(x), g(x))$ \begin{CJK}{UTF8}{mj}为由两个多项式系数形成的\end{CJK} Sylvester \begin{CJK}{UTF8}{mj}矩阵的行列式\end{CJK}:
$$
\left(\begin{array}{cccccccc}
a_{n} & a_{n-1} & \cdots & \cdots & a_{1} & a_{0} & & \\
& \ddots & \ddots & \ddots & \ddots & \ddots & \ddots & \\
& & a_{n} & a_{n-1} & \cdots & \cdots & a_{1} & a_{0} \\
b_{m} & b_{m-1} & \cdots & b_{1} & b_{0} & & & \\
& \ddots & \ddots & \ddots & \ddots & \ddots & & \\
& & b_{m} & b_{m-1} & \cdots & b_{1} & b_{0} &
\end{array}\right) .
$$
\begin{CJK}{UTF8}{mj}三\end{CJK}、\begin{CJK}{UTF8}{mj}已知\end{CJK} $c^{2}-4 a b \neq 0$, \begin{CJK}{UTF8}{mj}计算行列式\end{CJK}
$$
\left|\begin{array}{ccccc}
c & a & & & \\
b & c & a & & \\
& \ddots & \ddots & \ddots & \\
& & b & c & a \\
& & & b & c
\end{array}\right| .
$$
\begin{CJK}{UTF8}{mj}四\end{CJK}、 $A$ \begin{CJK}{UTF8}{mj}为\end{CJK} $m \times n$ \begin{CJK}{UTF8}{mj}矩阵\end{CJK}, $\operatorname{Rank}(A)=k$, \begin{CJK}{UTF8}{mj}证明\end{CJK}:

(1) \begin{CJK}{UTF8}{mj}若\end{CJK} $A=A_{1}+A_{2}+\cdots+A_{l}$, \begin{CJK}{UTF8}{mj}且\end{CJK} $\operatorname{Rank}\left(A_{i}\right)=1, i=1,2, \cdots, l$, \begin{CJK}{UTF8}{mj}则\end{CJK} $l \geq k$.

(2) \begin{CJK}{UTF8}{mj}存在秩为\end{CJK} 1 \begin{CJK}{UTF8}{mj}的矩阵\end{CJK} $A_{1}, A_{2}, \cdots, A_{k}$ \begin{CJK}{UTF8}{mj}使得\end{CJK} $A=A_{1}+A_{2}+\cdots+A_{k}$.

\begin{CJK}{UTF8}{mj}五\end{CJK}、\begin{CJK}{UTF8}{mj}证明\end{CJK}: \begin{CJK}{UTF8}{mj}任何一个复矩阵可以表示为两个对称矩阵的乘积\end{CJK}, \begin{CJK}{UTF8}{mj}且其中一个为可逆矩阵\end{CJK}.

\begin{CJK}{UTF8}{mj}六\end{CJK}、 $x \neq 0$, \begin{CJK}{UTF8}{mj}矩阵\end{CJK} $A=\left(\begin{array}{cc}1 & \frac{x}{n} \\ -\frac{x}{n} & 1\end{array}\right)$, \begin{CJK}{UTF8}{mj}计算极限\end{CJK} $\lim _{x \rightarrow 0} \lim _{n \rightarrow \infty}\left(A^{n}-E\right)$.

\begin{CJK}{UTF8}{mj}七\end{CJK}、\begin{CJK}{UTF8}{mj}矩阵\end{CJK} $A=\left(a_{i j}\right)_{n \times n}, B=\left(b_{i j}\right)_{n \times n}$, \begin{CJK}{UTF8}{mj}定义\end{CJK} $C=\left(a_{i j} b_{i j}\right)_{n \times n}$. \begin{CJK}{UTF8}{mj}证明\end{CJK}:\begin{CJK}{UTF8}{mj}若\end{CJK} $A, B$ \begin{CJK}{UTF8}{mj}半正定\end{CJK}, \begin{CJK}{UTF8}{mj}则\end{CJK} $C$ \begin{CJK}{UTF8}{mj}半正定\end{CJK}.

\begin{CJK}{UTF8}{mj}八\end{CJK}、\begin{CJK}{UTF8}{mj}设\end{CJK} $u$ \begin{CJK}{UTF8}{mj}是\end{CJK} Euclid \begin{CJK}{UTF8}{mj}空间\end{CJK} $\mathbb{R}^{5}$ \begin{CJK}{UTF8}{mj}中的单位向量\end{CJK}, \begin{CJK}{UTF8}{mj}定义\end{CJK} $T_{u}(x)=x-2(x, u) x$. \begin{CJK}{UTF8}{mj}现设\end{CJK} $\alpha, \beta$ \begin{CJK}{UTF8}{mj}是\end{CJK} $\mathbb{R}^{5}$ \begin{CJK}{UTF8}{mj}中线性无关的两个\end{CJK} \begin{CJK}{UTF8}{mj}单位向量\end{CJK}, \begin{CJK}{UTF8}{mj}问当\end{CJK} $\alpha, \beta$ \begin{CJK}{UTF8}{mj}满足什么条件时\end{CJK}, \begin{CJK}{UTF8}{mj}存在正整数\end{CJK} $k$ \begin{CJK}{UTF8}{mj}使得\end{CJK} $\left(T_{\alpha} T_{\beta}\right)^{k}$ \begin{CJK}{UTF8}{mj}为单位映射\end{CJK}.

\section{7. 中国科学院大学 2015 年研究生入学考试试题高等代数 
 李扬 
 微信公众号: sxkyliyang}
\begin{CJK}{UTF8}{mj}一\end{CJK}、\begin{CJK}{UTF8}{mj}设\end{CJK} $x_{1}, x_{2}, \cdots, x_{n}$ \begin{CJK}{UTF8}{mj}互异\end{CJK}, $f(x)=\left(x-x_{1}\right)\left(x-x_{2}\right) \cdots\left(x-x_{n}\right), s_{k}=x_{1}^{k}+x_{2}^{k}+\cdots+x_{n}^{k}$, \begin{CJK}{UTF8}{mj}证明\end{CJK}

(1) $\left[f^{\prime}(x)\right]^{2}-f(x) f^{\prime \prime}(x)$ \begin{CJK}{UTF8}{mj}无实根\end{CJK}.

(2) $x^{k+1} f^{\prime}(x)=\left(s_{0} x^{k}+s_{1} x^{k-1}+\cdots+s_{k-1} x+s_{k}\right) f(x)+g(x), g(x)$ \begin{CJK}{UTF8}{mj}的次数小于\end{CJK} $n$.

\begin{CJK}{UTF8}{mj}二\end{CJK}、 $A_{n}=\left(a^{|i-j|}\right)_{n \times n}$, \begin{CJK}{UTF8}{mj}求\end{CJK} $\left|A_{n}\right|$ \begin{CJK}{UTF8}{mj}和\end{CJK} $\left|A_{n}^{*}\right|$.

\begin{CJK}{UTF8}{mj}三\end{CJK}、 $A=\left(\begin{array}{ccc}a & 0 & 1 \\ -2 & 0 & 1 \\ 2 & b & -1\end{array}\right), B=\left(\begin{array}{ccc}2 & -1 & 0 \\ 0 & 0 & c \\ 2 & b & 1\end{array}\right)$,

(1) $a, b, c$ \begin{CJK}{UTF8}{mj}为何值时\end{CJK}, $A B=B A$.

(2) $A B=B A$ \begin{CJK}{UTF8}{mj}时\end{CJK}, \begin{CJK}{UTF8}{mj}求\end{CJK} $A, B$ \begin{CJK}{UTF8}{mj}公共的单位特征向量\end{CJK}.

\begin{CJK}{UTF8}{mj}四\end{CJK}、\begin{CJK}{UTF8}{mj}若\end{CJK} $A^{2}=-E$, \begin{CJK}{UTF8}{mj}则\end{CJK} $r(A+i E)$ \begin{CJK}{UTF8}{mj}与\end{CJK} $r(A-i E)$ \begin{CJK}{UTF8}{mj}满足什么关系\end{CJK}, \begin{CJK}{UTF8}{mj}并证明\end{CJK}.

\begin{CJK}{UTF8}{mj}五\end{CJK}、\begin{CJK}{UTF8}{mj}已知\end{CJK} $A=\left(\begin{array}{ccc}-1 & -2 & 6 \\ -1 & 0 & 3 \\ -1 & -1 & 4\end{array}\right)$, \begin{CJK}{UTF8}{mj}求\end{CJK} $A^{2015}$.

\begin{CJK}{UTF8}{mj}六\end{CJK}、\begin{CJK}{UTF8}{mj}若\end{CJK} $A, B$ \begin{CJK}{UTF8}{mj}无公共特征根\end{CJK}, $f(\lambda)$ \begin{CJK}{UTF8}{mj}为\end{CJK} $A$ \begin{CJK}{UTF8}{mj}的特征多项式\end{CJK}. \begin{CJK}{UTF8}{mj}证明\end{CJK}:

(1) $f(B)$ \begin{CJK}{UTF8}{mj}可逆\end{CJK}.

(2) $A X=X B$ \begin{CJK}{UTF8}{mj}只有零解\end{CJK}.

\begin{CJK}{UTF8}{mj}七\end{CJK}、\begin{CJK}{UTF8}{mj}线性空间\end{CJK} $V$ \begin{CJK}{UTF8}{mj}上的两个线性变换\end{CJK} $f, g$ \begin{CJK}{UTF8}{mj}满足\end{CJK} $\operatorname{Ker} f \subseteq \operatorname{Ker} g$, \begin{CJK}{UTF8}{mj}证明\end{CJK}

(1) \begin{CJK}{UTF8}{mj}存在线性变换\end{CJK} $h$, \begin{CJK}{UTF8}{mj}使得\end{CJK} $g=h f$.

(2) \begin{CJK}{UTF8}{mj}若\end{CJK} $\operatorname{Ker} f=\operatorname{Ker} g$, \begin{CJK}{UTF8}{mj}则存在线性变换\end{CJK} $h$, \begin{CJK}{UTF8}{mj}使得\end{CJK} $g=h f$.

\begin{CJK}{UTF8}{mj}八\end{CJK}、 $V$ \begin{CJK}{UTF8}{mj}是\end{CJK} $n$ \begin{CJK}{UTF8}{mj}维复向量空间\end{CJK}, $(-,-): V \times V \rightarrow \mathbb{C}$ \begin{CJK}{UTF8}{mj}是反对称的非退化双线性型\end{CJK}, $\varphi: V \rightarrow V$ \begin{CJK}{UTF8}{mj}是一个线性变换\end{CJK}, \begin{CJK}{UTF8}{mj}满足\end{CJK} $(\varphi(u), \varphi(v))=(u, v), \forall u, v \in V$. \begin{CJK}{UTF8}{mj}证明\end{CJK}

(1) $V$ \begin{CJK}{UTF8}{mj}的维数是偶数\end{CJK}.

(2) $\varphi$ \begin{CJK}{UTF8}{mj}是可逆的线性变换\end{CJK}.

(3) \begin{CJK}{UTF8}{mj}若\end{CJK} $\lambda$ \begin{CJK}{UTF8}{mj}是\end{CJK} $\varphi$ \begin{CJK}{UTF8}{mj}的特征值\end{CJK}, \begin{CJK}{UTF8}{mj}则\end{CJK} $\frac{1}{\lambda}$ \begin{CJK}{UTF8}{mj}也是\end{CJK} $\varphi$ \begin{CJK}{UTF8}{mj}的特征值\end{CJK}.

\section{8. 中国科学院大学 2016 年研究生入学考试试题高等代数 
 李扬 
 微信公众号: sxkyliyang}
\begin{CJK}{UTF8}{mj}一\end{CJK}、(20 \begin{CJK}{UTF8}{mj}分\end{CJK}) \begin{CJK}{UTF8}{mj}设\end{CJK} $a_{i}+b_{j} \neq 0$, \begin{CJK}{UTF8}{mj}求以下矩阵的行列式值\end{CJK}:
$$
A=\left(\begin{array}{cccc}
\left(a_{1}+b_{1}\right)^{-1} & \left(a_{1}+b_{2}\right)^{-1} & \cdots & \left(a_{1}+b_{n}\right)^{-1} \\
\left(a_{2}+b_{1}\right)^{-1} & \left(a_{2}+b_{2}\right)^{-1} & \cdots & \left(a_{2}+b_{n}\right)^{-1} \\
\cdots & \cdots & \cdots & \cdots \\
\left(a_{n}+b_{1}\right)^{-1} & \left(a_{n}+b_{2}\right)^{-1} & \cdots & \left(a_{n}+b_{n}\right)^{-1}
\end{array}\right) .
$$
\begin{CJK}{UTF8}{mj}二\end{CJK}、 (20 \begin{CJK}{UTF8}{mj}分\end{CJK}) \begin{CJK}{UTF8}{mj}已知二次型\end{CJK} $f\left(x_{1}, x_{2}, x_{3}\right)=5 x_{1}^{2}+5 x_{2}^{2}+\beta x_{3}^{2}-2 x_{1} x_{2}+6 x_{1} x_{3}-6 x_{2} x_{3}$ \begin{CJK}{UTF8}{mj}的秩为\end{CJK} 2 .

(1) \begin{CJK}{UTF8}{mj}求\end{CJK} $\beta$ \begin{CJK}{UTF8}{mj}的值\end{CJK}.

(2) \begin{CJK}{UTF8}{mj}求一实正交变换\end{CJK}, \begin{CJK}{UTF8}{mj}将上述二次型化为标准形\end{CJK}, \begin{CJK}{UTF8}{mj}并求出标准形\end{CJK}.

\begin{CJK}{UTF8}{mj}三\end{CJK}、 ( 16 \begin{CJK}{UTF8}{mj}分\end{CJK}) \begin{CJK}{UTF8}{mj}矩阵\end{CJK} $A$ \begin{CJK}{UTF8}{mj}的\end{CJK} $n-1$ \begin{CJK}{UTF8}{mj}阶子式不全为零\end{CJK}, \begin{CJK}{UTF8}{mj}给出齐次方程组\end{CJK} $A x=0$ \begin{CJK}{UTF8}{mj}的一组解\end{CJK}, \begin{CJK}{UTF8}{mj}并求出方程的所有解\end{CJK}, \begin{CJK}{UTF8}{mj}其\end{CJK} \begin{CJK}{UTF8}{mj}中\end{CJK}
$$
A=\left(\begin{array}{cccc}
a_{11} & a_{12} & \cdots & a_{1 n} \\
a_{21} & a_{22} & \cdots & a_{2 n} \\
\cdots & \cdots & \cdots & \cdots \\
a_{(n-1), 1} & a_{(n-1), 2} & \cdots & a_{(n-1), n}
\end{array}\right)
$$
\begin{CJK}{UTF8}{mj}四\end{CJK}、(18 \begin{CJK}{UTF8}{mj}分\end{CJK}) \begin{CJK}{UTF8}{mj}设\end{CJK} $V$ \begin{CJK}{UTF8}{mj}是\end{CJK} $n$ \begin{CJK}{UTF8}{mj}维线性空间\end{CJK}, $V_{1}, V_{2}$ \begin{CJK}{UTF8}{mj}是\end{CJK} $V$ \begin{CJK}{UTF8}{mj}的子空间\end{CJK}, \begin{CJK}{UTF8}{mj}且\end{CJK}
$$
\operatorname{dim}\left(V_{1}+V_{2}\right)=\operatorname{dim}\left(V_{1} \cap V_{2}\right)+1
$$
\begin{CJK}{UTF8}{mj}求证\end{CJK}: $\quad V_{1}+V_{2}=V_{1}, V_{1} \cap V_{2}=V_{2}$ \begin{CJK}{UTF8}{mj}或者\end{CJK} $V_{1}+V_{2}=V_{2}, V_{1} \cap V_{2}=V_{1}$.

\begin{CJK}{UTF8}{mj}五\end{CJK}、 (16 \begin{CJK}{UTF8}{mj}分\end{CJK}) \begin{CJK}{UTF8}{mj}证明与\end{CJK} $n$ \begin{CJK}{UTF8}{mj}阶若儿当块\end{CJK} $J=\left(\begin{array}{cccc}\lambda & 1 & & \\ & \lambda & \ddots & \\ & \ddots & 1 \\ & & \lambda\end{array}\right)$ \begin{CJK}{UTF8}{mj}可交换的矩阵必为\end{CJK} $J$ \begin{CJK}{UTF8}{mj}的多项式\end{CJK}.

\begin{CJK}{UTF8}{mj}六\end{CJK}、(15 \begin{CJK}{UTF8}{mj}分\end{CJK}) \begin{CJK}{UTF8}{mj}若\end{CJK} $n$ \begin{CJK}{UTF8}{mj}阶方阵\end{CJK} $A$ \begin{CJK}{UTF8}{mj}的每行每列恰有一个元素为\end{CJK} 1 \begin{CJK}{UTF8}{mj}或\end{CJK} $-1$, \begin{CJK}{UTF8}{mj}其余元素均为零\end{CJK}. \begin{CJK}{UTF8}{mj}证明\end{CJK}: \begin{CJK}{UTF8}{mj}存在正整数\end{CJK} $m$ \begin{CJK}{UTF8}{mj}使\end{CJK} \begin{CJK}{UTF8}{mj}得\end{CJK} $A^{m}=E$, \begin{CJK}{UTF8}{mj}其中\end{CJK} $E$ \begin{CJK}{UTF8}{mj}为单位矩阵\end{CJK}.

\begin{CJK}{UTF8}{mj}七\end{CJK}、 ( 15 \begin{CJK}{UTF8}{mj}分\end{CJK}) \begin{CJK}{UTF8}{mj}设\end{CJK} $A$ \begin{CJK}{UTF8}{mj}是一个\end{CJK} $n$ \begin{CJK}{UTF8}{mj}阶复矩阵\end{CJK}, \begin{CJK}{UTF8}{mj}定义\end{CJK} $M_{n}(\mathbb{C})$ \begin{CJK}{UTF8}{mj}上的线性变换\end{CJK} $\mathcal{T}(X):=A X-X A$. \begin{CJK}{UTF8}{mj}若\end{CJK} $A$ \begin{CJK}{UTF8}{mj}的特征值为\end{CJK} $\lambda_{1}, \lambda_{2}, \cdots, \lambda_{n}$ (\begin{CJK}{UTF8}{mj}不考虑重根\end{CJK}). \begin{CJK}{UTF8}{mj}证明\end{CJK}: $\mathcal{T}$ \begin{CJK}{UTF8}{mj}的特征值必可写成\end{CJK} $\lambda_{i}-\lambda_{j}(1 \leq i, j \leq n)$ \begin{CJK}{UTF8}{mj}的形式\end{CJK}.

\begin{CJK}{UTF8}{mj}八\end{CJK}、 ( 15 \begin{CJK}{UTF8}{mj}分\end{CJK}) \begin{CJK}{UTF8}{mj}设\end{CJK} $A, B$ \begin{CJK}{UTF8}{mj}是两个\end{CJK} $n$ \begin{CJK}{UTF8}{mj}阶复矩阵\end{CJK}, \begin{CJK}{UTF8}{mj}如果\end{CJK} $A B-B A=2 B$, \begin{CJK}{UTF8}{mj}证明\end{CJK}

(1) \begin{CJK}{UTF8}{mj}存在\end{CJK} $n$ \begin{CJK}{UTF8}{mj}维列向量\end{CJK} $u$ \begin{CJK}{UTF8}{mj}和常数\end{CJK} $\mu$, \begin{CJK}{UTF8}{mj}使得\end{CJK} $A u=\mu u, B u=0$.

(2) \begin{CJK}{UTF8}{mj}矩阵\end{CJK} $A, B$ \begin{CJK}{UTF8}{mj}可同时上三角化\end{CJK}.

\begin{CJK}{UTF8}{mj}九\end{CJK}、 ( 15 \begin{CJK}{UTF8}{mj}分\end{CJK}) \begin{CJK}{UTF8}{mj}设多项式\end{CJK} $g(x)=p^{k}(x) g_{1}(x)(k \geq 1)$, \begin{CJK}{UTF8}{mj}多项式\end{CJK} $p(x)$ \begin{CJK}{UTF8}{mj}与\end{CJK} $g_{1}(x)$ \begin{CJK}{UTF8}{mj}互素\end{CJK}. \begin{CJK}{UTF8}{mj}证明\end{CJK}: \begin{CJK}{UTF8}{mj}对于任意多项式\end{CJK} $f(x)$ \begin{CJK}{UTF8}{mj}有\end{CJK}
$$
\frac{f(x)}{g(x)}=\frac{r(x)}{p^{k}(x)}+\frac{f_{1}(x)}{p^{k-1} g_{1}(x)}
$$
\begin{CJK}{UTF8}{mj}其中\end{CJK} $r(x), f_{1}(x)$ \begin{CJK}{UTF8}{mj}都是多项式\end{CJK}, $r(x)=0$ \begin{CJK}{UTF8}{mj}或\end{CJK} $r(x)$ \begin{CJK}{UTF8}{mj}的次数小于\end{CJK} $p(x)$.

\section{9. 中国科学院大学 2017 年研究生入学考试试题高等代数 
 李扬 
 微信公众号: sxkyliyang}
\begin{CJK}{UTF8}{mj}一\end{CJK}、 (15 \begin{CJK}{UTF8}{mj}分\end{CJK}) \begin{CJK}{UTF8}{mj}证明\end{CJK}: \begin{CJK}{UTF8}{mj}若实系数多项式\end{CJK} $f(x)$ \begin{CJK}{UTF8}{mj}对所有的实数\end{CJK} $x$ \begin{CJK}{UTF8}{mj}均有\end{CJK} $f(x) \geq 0$, \begin{CJK}{UTF8}{mj}则\end{CJK} $f(x)$ \begin{CJK}{UTF8}{mj}可以写成两个实系数多项\end{CJK} \begin{CJK}{UTF8}{mj}式的平方和\end{CJK} $[g(x)]^{2}+[h(x)]^{2}$.

\begin{CJK}{UTF8}{mj}二\end{CJK}、 ( 15 \begin{CJK}{UTF8}{mj}分\end{CJK}) \begin{CJK}{UTF8}{mj}设\end{CJK} $f_{i}, i=1,2, \cdots, m(m<n)$ \begin{CJK}{UTF8}{mj}是\end{CJK} $n$ \begin{CJK}{UTF8}{mj}维线性空间\end{CJK} $V$ \begin{CJK}{UTF8}{mj}上的\end{CJK} $m$ \begin{CJK}{UTF8}{mj}个线性函数\end{CJK}, \begin{CJK}{UTF8}{mj}即\end{CJK} $f_{i}(a \alpha+b \beta)=$ $a f_{i}(\alpha)+b f_{i}(\beta)$. \begin{CJK}{UTF8}{mj}证明\end{CJK}: \begin{CJK}{UTF8}{mj}存在一个非零向量\end{CJK} $\alpha \in V$, \begin{CJK}{UTF8}{mj}使得\end{CJK} $f_{i}(\alpha)=0$.

\begin{CJK}{UTF8}{mj}三\end{CJK}、 (20 \begin{CJK}{UTF8}{mj}分\end{CJK}) \begin{CJK}{UTF8}{mj}求\end{CJK}
$$
\left|\begin{array}{ccccc}
1-a_{1} & a_{2} & & & \\
-1 & 1-a_{2} & a_{3} & & \\
& \ddots & \ddots & \ddots & \\
& & \ddots & \ddots & a_{n} \\
& & & -1 & 1-a_{n}
\end{array}\right|
$$
\begin{CJK}{UTF8}{mj}四\end{CJK}、 ( 20 \begin{CJK}{UTF8}{mj}分\end{CJK}) \begin{CJK}{UTF8}{mj}设\end{CJK} $f(x)=x^{\prime} A x$ \begin{CJK}{UTF8}{mj}是实二次型\end{CJK}, \begin{CJK}{UTF8}{mj}若存在\end{CJK} $n$ \begin{CJK}{UTF8}{mj}维实向量\end{CJK} $x_{1} \neq x_{2}$ \begin{CJK}{UTF8}{mj}使得\end{CJK} $f\left(x_{1}\right)+f\left(x_{2}\right)=0$, \begin{CJK}{UTF8}{mj}证明\end{CJK}: \begin{CJK}{UTF8}{mj}存在\end{CJK} $n$ \begin{CJK}{UTF8}{mj}维实向量\end{CJK} $x_{0} \neq 0$ \begin{CJK}{UTF8}{mj}使得\end{CJK} $f\left(x_{0}\right)=0$.

\begin{CJK}{UTF8}{mj}五\end{CJK}、 (15 \begin{CJK}{UTF8}{mj}分\end{CJK}) \begin{CJK}{UTF8}{mj}已知\end{CJK} $A$ \begin{CJK}{UTF8}{mj}为\end{CJK} $n$ \begin{CJK}{UTF8}{mj}阶幂等矩阵\end{CJK}, \begin{CJK}{UTF8}{mj}即\end{CJK} $A^{2}=A$.

(1) \begin{CJK}{UTF8}{mj}证明\end{CJK} $A$ \begin{CJK}{UTF8}{mj}的\end{CJK} Jordan \begin{CJK}{UTF8}{mj}标准形是\end{CJK} $\left(\begin{array}{cc}E_{r} & 0 \\ 0 & 0\end{array}\right)$, \begin{CJK}{UTF8}{mj}其中\end{CJK} $r=r(A)$.

(2) $\mathcal{R}\left(E_{n}-A\right)=\mathcal{N}(A)$, \begin{CJK}{UTF8}{mj}其中\end{CJK} $\mathcal{R}(B)$ \begin{CJK}{UTF8}{mj}是\end{CJK} $B$ \begin{CJK}{UTF8}{mj}的列向量张成的线性空间\end{CJK}, $\mathcal{N}(B)$ \begin{CJK}{UTF8}{mj}为\end{CJK} $B$ \begin{CJK}{UTF8}{mj}的解空间\end{CJK}, \begin{CJK}{UTF8}{mj}即\end{CJK} $\mathcal{N}(B)=\{x \mid B x=0\} .$

\begin{CJK}{UTF8}{mj}六\end{CJK}、 ( 15 \begin{CJK}{UTF8}{mj}分\end{CJK}) \begin{CJK}{UTF8}{mj}已知\end{CJK} $A$ \begin{CJK}{UTF8}{mj}为\end{CJK} $n$ \begin{CJK}{UTF8}{mj}阶可逆的反对称矩阵\end{CJK}, $B=\left(\begin{array}{cc}A & v \\ v^{\prime} & 0\end{array}\right)$, \begin{CJK}{UTF8}{mj}其中\end{CJK} $v$ \begin{CJK}{UTF8}{mj}为\end{CJK} $n$ \begin{CJK}{UTF8}{mj}维列向量\end{CJK}, \begin{CJK}{UTF8}{mj}求\end{CJK} $r(B)$.

\begin{CJK}{UTF8}{mj}七\end{CJK}、 (15 \begin{CJK}{UTF8}{mj}分\end{CJK}) \begin{CJK}{UTF8}{mj}设\end{CJK}
$$
\left(\begin{array}{c}
x_{3 n} \\
x_{3 n+1} \\
x_{3 n+2}
\end{array}\right)=\left(\begin{array}{ccc}
3 & -2 & 1 \\
4 & -1 & 0 \\
4 & -3 & 2
\end{array}\right)\left(\begin{array}{l}
x_{3 n-3} \\
x_{3 n-2} \\
x_{3 n-1}
\end{array}\right)
$$
\begin{CJK}{UTF8}{mj}给定初值\end{CJK} $a_{0}=5, a_{1}=7, a_{2}=8$, \begin{CJK}{UTF8}{mj}求\end{CJK} $x_{n}$ \begin{CJK}{UTF8}{mj}的通项\end{CJK}.

\begin{CJK}{UTF8}{mj}八\end{CJK}、 ( 18 \begin{CJK}{UTF8}{mj}分\end{CJK}) \begin{CJK}{UTF8}{mj}若\end{CJK} $n$ \begin{CJK}{UTF8}{mj}维线性空间\end{CJK} $V$ \begin{CJK}{UTF8}{mj}有两个子空间\end{CJK} $U_{1}$ \begin{CJK}{UTF8}{mj}和\end{CJK} $U_{2}$, \begin{CJK}{UTF8}{mj}维数\end{CJK} $\operatorname{dim} U_{1} \leq m, \operatorname{dim} U_{2} \leq m, m<n$. \begin{CJK}{UTF8}{mj}证明\end{CJK}: $V$ \begin{CJK}{UTF8}{mj}中\end{CJK} \begin{CJK}{UTF8}{mj}存在子空间\end{CJK} $W$, \begin{CJK}{UTF8}{mj}且\end{CJK} $\operatorname{dim} W=n-m$, \begin{CJK}{UTF8}{mj}满足\end{CJK} $W \cap U_{1}=W \cap U_{2}=\{0\}$.

\begin{CJK}{UTF8}{mj}九\end{CJK}、 $(17$ \begin{CJK}{UTF8}{mj}分\end{CJK}) \begin{CJK}{UTF8}{mj}设\end{CJK} $A$ \begin{CJK}{UTF8}{mj}是\end{CJK} $n$ \begin{CJK}{UTF8}{mj}阶实对称矩阵\end{CJK}, \begin{CJK}{UTF8}{mj}且\end{CJK}
$$
A=\left(\begin{array}{ccccc}
a_{1} & b_{1} & & & \\
b_{1} & a_{2} & b_{2} & & \\
& b_{2} & \ddots & \ddots & \\
& & \ddots & \ddots & b_{n-1} \\
& & & b_{n-1} & a_{n}
\end{array}\right), b_{j} \neq 0
$$
(1) \begin{CJK}{UTF8}{mj}证明\end{CJK} $r(A) \geq n-1$.

(2) \begin{CJK}{UTF8}{mj}证明\end{CJK} $A$ \begin{CJK}{UTF8}{mj}的特征值各不相同\end{CJK}.

\section{0. 中国科学院大学 2018 年研究生入学考试试题高等代数 李扬 微信公众号: sxkyliyang}
\begin{CJK}{UTF8}{mj}一\end{CJK}、( 20 \begin{CJK}{UTF8}{mj}分\end{CJK}) \begin{CJK}{UTF8}{mj}设\end{CJK} $p(x), q(x), r(x)$ \begin{CJK}{UTF8}{mj}都是数域\end{CJK} $\mathbb{K}$ \begin{CJK}{UTF8}{mj}上的正次数多项式\end{CJK}, \begin{CJK}{UTF8}{mj}而且\end{CJK} $p(x)$ \begin{CJK}{UTF8}{mj}与\end{CJK} $q(x)$ \begin{CJK}{UTF8}{mj}互素\end{CJK}, $\operatorname{deg}(r(x))<$ $\operatorname{deg}(p(x))+\operatorname{deg}(q(x))$. \begin{CJK}{UTF8}{mj}证明\end{CJK}: \begin{CJK}{UTF8}{mj}存在数域\end{CJK} $\mathbb{K}$ \begin{CJK}{UTF8}{mj}上的多项式\end{CJK} $u(x), v(x)$ \begin{CJK}{UTF8}{mj}满足\end{CJK} $\operatorname{deg}(u(x))<\operatorname{deg}(p(x)), \operatorname{deg}(v(x))<$ $\operatorname{deg}(q(x))$, \begin{CJK}{UTF8}{mj}使得\end{CJK} $\frac{r(x)}{p(x) q(x)}=\frac{u(x)}{p(x)}+\frac{v(x)}{q(x)}$.

\begin{CJK}{UTF8}{mj}二\end{CJK}、(20 \begin{CJK}{UTF8}{mj}分\end{CJK}) \begin{CJK}{UTF8}{mj}设\end{CJK} $n$ \begin{CJK}{UTF8}{mj}阶方阵\end{CJK} $M_{n}=(|i-j|)_{1 \leq i, j \leq n}$, \begin{CJK}{UTF8}{mj}令\end{CJK} $D_{n}=\operatorname{det}\left(M_{n}\right)\left(M_{n}\right.$ \begin{CJK}{UTF8}{mj}的行列式\end{CJK})

(1) \begin{CJK}{UTF8}{mj}计算\end{CJK} $D_{4}$.

(2) \begin{CJK}{UTF8}{mj}证明\end{CJK} $D_{n}$ \begin{CJK}{UTF8}{mj}满足递推关系\end{CJK} $D_{n}=-4 D_{n-1}-4 D_{n-2}$.

(3) \begin{CJK}{UTF8}{mj}求\end{CJK} $n$ \begin{CJK}{UTF8}{mj}阶方阵\end{CJK} $A_{n}=\left(\left|\frac{1}{i}-\frac{1}{j}\right|\right)_{1 \leq i, j \leq n}$ \begin{CJK}{UTF8}{mj}的行列式\end{CJK} $\operatorname{det}\left(A_{n}\right)$.

\begin{CJK}{UTF8}{mj}三\end{CJK}、(20 \begin{CJK}{UTF8}{mj}分\end{CJK}) \begin{CJK}{UTF8}{mj}设\end{CJK} $A, B$ \begin{CJK}{UTF8}{mj}均是\end{CJK} $n$ \begin{CJK}{UTF8}{mj}阶方阵\end{CJK}, \begin{CJK}{UTF8}{mj}满足\end{CJK} $A B=0$, \begin{CJK}{UTF8}{mj}证明\end{CJK}

(1) $\operatorname{rank}(A)+\operatorname{rank}(B) \leq n$;

(2) \begin{CJK}{UTF8}{mj}对于方阵\end{CJK} $A$ \begin{CJK}{UTF8}{mj}和正整数\end{CJK} $k(\operatorname{rank}(A) \leq k \leq n)$, \begin{CJK}{UTF8}{mj}必存在方阵\end{CJK} $B$, \begin{CJK}{UTF8}{mj}使得\end{CJK} $\operatorname{rank}(A)+\operatorname{rank}(B)=k$.

\begin{CJK}{UTF8}{mj}四\end{CJK}、( 20 \begin{CJK}{UTF8}{mj}分\end{CJK}) \begin{CJK}{UTF8}{mj}通过正交变换将下面的实二次型化成标准形\end{CJK}:
$$
q\left(x_{1}, x_{2}, x_{3}\right)=5 x_{1}^{2}+5 x_{2}^{2}+5 x_{3}^{2}-2 x_{1} x_{2}-2 x_{2} x_{3}-2 x_{1} x_{3}
$$
\begin{CJK}{UTF8}{mj}五\end{CJK}、(20 \begin{CJK}{UTF8}{mj}分\end{CJK}) \begin{CJK}{UTF8}{mj}设\end{CJK} $A, B$ \begin{CJK}{UTF8}{mj}是两个\end{CJK} $n$ \begin{CJK}{UTF8}{mj}阶实矩阵\end{CJK}, \begin{CJK}{UTF8}{mj}并且\end{CJK} $A$ \begin{CJK}{UTF8}{mj}是对称正定矩阵\end{CJK}, $B$ \begin{CJK}{UTF8}{mj}是反对称矩阵\end{CJK}, \begin{CJK}{UTF8}{mj}证明\end{CJK}: $A+B$ \begin{CJK}{UTF8}{mj}是可逆\end{CJK} \begin{CJK}{UTF8}{mj}矩阵\end{CJK}.

\begin{CJK}{UTF8}{mj}六\end{CJK}、( 20 \begin{CJK}{UTF8}{mj}分\end{CJK}) \begin{CJK}{UTF8}{mj}设\end{CJK} $A$ \begin{CJK}{UTF8}{mj}是\end{CJK} $n$ \begin{CJK}{UTF8}{mj}阶复矩阵\end{CJK}, \begin{CJK}{UTF8}{mj}且\end{CJK} $A=\left(\begin{array}{c}A_{1} \\ A_{2}\end{array}\right)$, \begin{CJK}{UTF8}{mj}令\end{CJK} $V_{1}=\left\{x \in \mathbb{C}^{n} \mid A_{1} x=0\right\}, V_{2}=\left\{x \in \mathbb{C}^{n} \mid A_{2} x=0\right\}$, \begin{CJK}{UTF8}{mj}证\end{CJK} \begin{CJK}{UTF8}{mj}明\end{CJK}: \begin{CJK}{UTF8}{mj}矩阵\end{CJK} $A$ \begin{CJK}{UTF8}{mj}可逆的充分必要条件是向量空间\end{CJK} $\mathbb{C}^{n}$ \begin{CJK}{UTF8}{mj}能够表示成子空间\end{CJK} $V_{1}$ \begin{CJK}{UTF8}{mj}与\end{CJK} $V_{2}$ \begin{CJK}{UTF8}{mj}的直和\end{CJK}, \begin{CJK}{UTF8}{mj}即\end{CJK} $\mathbb{C}^{n}=V_{1} \oplus V_{2}$.

\begin{CJK}{UTF8}{mj}七\end{CJK}、 ( 15 \begin{CJK}{UTF8}{mj}分\end{CJK}) \begin{CJK}{UTF8}{mj}证明\end{CJK}: 8 \begin{CJK}{UTF8}{mj}个满足\end{CJK} $A^{3}=0$ \begin{CJK}{UTF8}{mj}的\end{CJK} 5 \begin{CJK}{UTF8}{mj}阶复数矩阵中必有两个相似\end{CJK}.

\begin{CJK}{UTF8}{mj}八\end{CJK}、(15 \begin{CJK}{UTF8}{mj}分\end{CJK}) $\mathbb{R}$ \begin{CJK}{UTF8}{mj}上所有\end{CJK} $n(n \geq 2)$ \begin{CJK}{UTF8}{mj}阶方阵构成的线性空间\end{CJK} $V=\mathbb{R}^{n \times n}$ \begin{CJK}{UTF8}{mj}上的线性变换\end{CJK} $f: V \rightarrow V$ \begin{CJK}{UTF8}{mj}定义为\end{CJK}
$$
f(A)=A+A^{\prime}, \forall A \in V
$$
\begin{CJK}{UTF8}{mj}其中\end{CJK} $A^{\prime}$ \begin{CJK}{UTF8}{mj}为\end{CJK} $A$ \begin{CJK}{UTF8}{mj}的转置\end{CJK}. \begin{CJK}{UTF8}{mj}求\end{CJK} $f$ \begin{CJK}{UTF8}{mj}的特征值\end{CJK}, \begin{CJK}{UTF8}{mj}特征子空间\end{CJK}, \begin{CJK}{UTF8}{mj}最小多项式\end{CJK}. 11. \begin{CJK}{UTF8}{mj}中国科学院大学\end{CJK} 2010 \begin{CJK}{UTF8}{mj}年研究生入学考试试题数学分析\end{CJK}

\begin{CJK}{UTF8}{mj}李扬\end{CJK}

\begin{CJK}{UTF8}{mj}微信公众号\end{CJK}: sxkyliyang

\begin{CJK}{UTF8}{mj}一\end{CJK}、\begin{CJK}{UTF8}{mj}计算\end{CJK}:

(1) $\lim _{x \rightarrow 0} \frac{\int_{0}^{\sin ^{2} x} \ln (1+t) \mathrm{d} t}{\sqrt{1+x^{4}}-1}$.

(2) $\iint_{|x|+|y| \leq 1}|x y| \mathrm{d} x \mathrm{~d} y$.

\begin{CJK}{UTF8}{mj}二\end{CJK}、 (1) \begin{CJK}{UTF8}{mj}令\end{CJK}
$$
f(x)= \begin{cases}x^{2} \sin \frac{1}{x}, & x \neq 0 \\ 0, & x=0\end{cases}
$$
\begin{CJK}{UTF8}{mj}求\end{CJK} $f^{\prime}(0)$, \begin{CJK}{UTF8}{mj}并证明\end{CJK} $f^{\prime}(x)$ \begin{CJK}{UTF8}{mj}在\end{CJK} $x=0$ \begin{CJK}{UTF8}{mj}处不连续\end{CJK}.

(2) \begin{CJK}{UTF8}{mj}若\end{CJK} $\lambda=\sum_{k=1}^{n} \frac{1}{k}$, \begin{CJK}{UTF8}{mj}证明\end{CJK} $e^{\lambda}>n+1$.

\begin{CJK}{UTF8}{mj}三\end{CJK}、\begin{CJK}{UTF8}{mj}若\end{CJK} $f(x)$ \begin{CJK}{UTF8}{mj}在\end{CJK} $[0,1]$ \begin{CJK}{UTF8}{mj}上连续\end{CJK}, \begin{CJK}{UTF8}{mj}在\end{CJK} $(0,1)$ \begin{CJK}{UTF8}{mj}上二次可微\end{CJK}, \begin{CJK}{UTF8}{mj}并且\end{CJK}
$$
f(0)=f\left(\frac{1}{4}\right)=0, \int_{\frac{1}{4}}^{1} f(y) \mathrm{d} y=\frac{3}{4} f(1) .
$$
\begin{CJK}{UTF8}{mj}证明\end{CJK}: \begin{CJK}{UTF8}{mj}存在\end{CJK} $\xi \in(0,1)$ \begin{CJK}{UTF8}{mj}使得\end{CJK} $f^{\prime \prime}(\xi)=0$.

\begin{CJK}{UTF8}{mj}四\end{CJK}、\begin{CJK}{UTF8}{mj}求级数\end{CJK} $\sum_{n=1}^{\infty} \frac{n}{(n+1) !}$ \begin{CJK}{UTF8}{mj}的和\end{CJK}.

\begin{CJK}{UTF8}{mj}芏\end{CJK}、\begin{CJK}{UTF8}{mj}证明\end{CJK}:
$$
\frac{2 n}{3} \sqrt{n}<\sum_{k=1}^{n} \sqrt{k}<\left(\frac{2 n}{3}+\frac{1}{2}\right) \sqrt{n} .
$$
\begin{CJK}{UTF8}{mj}六\end{CJK}、\begin{CJK}{UTF8}{mj}计算\end{CJK}:
$$
\iiint_{V}\left(x^{3}+y^{3}+z^{3}\right) \mathrm{d} x \mathrm{~d} y \mathrm{~d} z
$$
\begin{CJK}{UTF8}{mj}其中\end{CJK} $V$ \begin{CJK}{UTF8}{mj}表示曲面\end{CJK} $x^{2}+y^{2}+z^{2}-2 a(x+y+z)+2 a^{2}=0(a>0)$ \begin{CJK}{UTF8}{mj}所围成的区域\end{CJK}.

\begin{CJK}{UTF8}{mj}七\end{CJK}、\begin{CJK}{UTF8}{mj}应用\end{CJK} Green \begin{CJK}{UTF8}{mj}公式计算积分\end{CJK}
$$
\mathrm{I}=\oint_{L} \frac{e^{x}(x \sin y-y \cos y) \mathrm{d} x+e^{x}(x \cos y+y \sin y) \mathrm{d} y}{x^{2}+y^{2}},
$$
\begin{CJK}{UTF8}{mj}其中\end{CJK} $L$ \begin{CJK}{UTF8}{mj}是包围原点的简单光滑闭曲线\end{CJK}, \begin{CJK}{UTF8}{mj}逆时针方向\end{CJK}.

\begin{CJK}{UTF8}{mj}八\end{CJK}、\begin{CJK}{UTF8}{mj}设\end{CJK} $f(x)$ \begin{CJK}{UTF8}{mj}定义在\end{CJK} $(-\infty,+\infty)$ \begin{CJK}{UTF8}{mj}上\end{CJK}, \begin{CJK}{UTF8}{mj}且在\end{CJK} $x=0$ \begin{CJK}{UTF8}{mj}连续\end{CJK}, \begin{CJK}{UTF8}{mj}并且对所有的\end{CJK} $x, y \in(-\infty,+\infty)$ \begin{CJK}{UTF8}{mj}有\end{CJK}
$$
f(x+y)=f(x)+f(y)
$$
\begin{CJK}{UTF8}{mj}证明\end{CJK}: $f(x)$ \begin{CJK}{UTF8}{mj}在\end{CJK} $(-\infty,+\infty)$ \begin{CJK}{UTF8}{mj}上连续且\end{CJK} $f(x)=f(1) x$.

\begin{CJK}{UTF8}{mj}九\end{CJK}、\begin{CJK}{UTF8}{mj}证明\end{CJK}
$$
\int_{0}^{1} \frac{\mathrm{d} x}{x^{x}}=\sum_{n=1}^{\infty} \frac{1}{n^{n}}
$$
\begin{CJK}{UTF8}{mj}十\end{CJK}、\begin{CJK}{UTF8}{mj}设函数\end{CJK} $f(x)$ \begin{CJK}{UTF8}{mj}在\end{CJK} $[0,1]$ \begin{CJK}{UTF8}{mj}上连续且\end{CJK} $f(x)>0$, \begin{CJK}{UTF8}{mj}讨论函数\end{CJK} $g(y)=\int_{0}^{1} \frac{y f(x)}{x^{2}+y^{2}} \mathrm{~d} x$ \begin{CJK}{UTF8}{mj}在\end{CJK} $(-\infty,+\infty)$ \begin{CJK}{UTF8}{mj}上的连续性\end{CJK}. 12. \begin{CJK}{UTF8}{mj}中国科学院大学\end{CJK} 2011 \begin{CJK}{UTF8}{mj}年研究生入学考试试题数学分析\end{CJK} \begin{CJK}{UTF8}{mj}李扬\end{CJK} \begin{CJK}{UTF8}{mj}微信公众号\end{CJK}: sxkyliyang

\begin{CJK}{UTF8}{mj}一\end{CJK}、(30 \begin{CJK}{UTF8}{mj}分\end{CJK})

(1) \begin{CJK}{UTF8}{mj}计算\end{CJK} $\lim _{x \rightarrow \infty}\left(\frac{1}{x}+2^{\frac{1}{x}}\right)^{x}$.

(2) \begin{CJK}{UTF8}{mj}计算\end{CJK} $\lim _{n \rightarrow \infty} \int_{0}^{1} \ln \left(1+x^{n}\right) \mathrm{d} x$.

(3) \begin{CJK}{UTF8}{mj}证明极限\end{CJK} $\lim _{x \rightarrow 0}\left(\frac{2-e^{\frac{1}{x}}}{1+e^{\frac{2}{x}}}+\frac{x}{|x|}\right)$ \begin{CJK}{UTF8}{mj}存在\end{CJK}, \begin{CJK}{UTF8}{mj}并求其值\end{CJK}.

\begin{CJK}{UTF8}{mj}二\end{CJK}、 (20 \begin{CJK}{UTF8}{mj}分\end{CJK}) \begin{CJK}{UTF8}{mj}求数列\end{CJK} $1, \sqrt{2}, \sqrt[3]{3}, \cdots, \sqrt[n]{n}$ \begin{CJK}{UTF8}{mj}中最大的一项\end{CJK}.

\begin{CJK}{UTF8}{mj}三\end{CJK}、 (15 \begin{CJK}{UTF8}{mj}分\end{CJK}) \begin{CJK}{UTF8}{mj}设函数\end{CJK} $f(x)$ \begin{CJK}{UTF8}{mj}满足\end{CJK} $f^{\prime \prime}(x)<0$ (\begin{CJK}{UTF8}{mj}当\end{CJK} $\left.x>0\right), f(0)=0$. \begin{CJK}{UTF8}{mj}证明对于所有\end{CJK} $x_{1}>0, x_{2}>0$ \begin{CJK}{UTF8}{mj}有\end{CJK} $f\left(x_{1}+x_{2}\right)<f\left(x_{1}\right)+f\left(x_{2}\right) .$

\begin{CJK}{UTF8}{mj}四\end{CJK}、(20 \begin{CJK}{UTF8}{mj}分\end{CJK}) \begin{CJK}{UTF8}{mj}设函数\end{CJK} $f(x)$ \begin{CJK}{UTF8}{mj}在\end{CJK} $[0,+\infty)$ \begin{CJK}{UTF8}{mj}内有界可微\end{CJK}, \begin{CJK}{UTF8}{mj}试问下列命题中哪个必定成立\end{CJK}(\begin{CJK}{UTF8}{mj}要说明理由\end{CJK}), \begin{CJK}{UTF8}{mj}哪个不成\end{CJK} \begin{CJK}{UTF8}{mj}立\end{CJK}(\begin{CJK}{UTF8}{mj}可由反例说明\end{CJK}):

(1) $\lim _{x \rightarrow \infty} f(x)=0$ \begin{CJK}{UTF8}{mj}蕴含\end{CJK} $\lim _{x \rightarrow \infty} f^{\prime}(x)=0$.

(2) $\lim _{x \rightarrow \infty} f^{\prime}(x)$ \begin{CJK}{UTF8}{mj}存在蕴含\end{CJK} $\lim _{x \rightarrow \infty} f^{\prime}(x)=0$.

\begin{CJK}{UTF8}{mj}五\end{CJK}、( 20 \begin{CJK}{UTF8}{mj}分\end{CJK}) \begin{CJK}{UTF8}{mj}过抛物线\end{CJK} $y=x^{2}$ \begin{CJK}{UTF8}{mj}上的一点\end{CJK} $\left(a, a^{2}\right)$ \begin{CJK}{UTF8}{mj}作切线\end{CJK}, \begin{CJK}{UTF8}{mj}确定\end{CJK} $a$ \begin{CJK}{UTF8}{mj}使得该切线与另一抛物线\end{CJK} $y=-x^{2}+4 x-1$ \begin{CJK}{UTF8}{mj}所围成的图形面积最小\end{CJK}, \begin{CJK}{UTF8}{mj}并求出最小面积值\end{CJK}.

\begin{CJK}{UTF8}{mj}六\end{CJK}、 (15 \begin{CJK}{UTF8}{mj}分\end{CJK}) \begin{CJK}{UTF8}{mj}计算曲线积分\end{CJK}
$$
\oint_{C}\left((x+1)^{2}+(y-2)^{2}\right) \mathrm{d} s,
$$
\begin{CJK}{UTF8}{mj}其中\end{CJK} $C$ \begin{CJK}{UTF8}{mj}表示曲面\end{CJK} $x^{2}+y^{2}+z^{2}=1$ \begin{CJK}{UTF8}{mj}与\end{CJK} $x+y+z=1$ \begin{CJK}{UTF8}{mj}的交线\end{CJK}.

\begin{CJK}{UTF8}{mj}七\end{CJK}、(15 \begin{CJK}{UTF8}{mj}分\end{CJK}) \begin{CJK}{UTF8}{mj}设函数列\end{CJK} $\left\{f_{n}(x)\right\}_{n \geq 0}$ \begin{CJK}{UTF8}{mj}在区间\end{CJK} I \begin{CJK}{UTF8}{mj}上一致收敛\end{CJK}, \begin{CJK}{UTF8}{mj}而且对每个\end{CJK} $n \geq 0, f_{n}(x)$ \begin{CJK}{UTF8}{mj}在\end{CJK} I \begin{CJK}{UTF8}{mj}上有界\end{CJK}. \begin{CJK}{UTF8}{mj}证明函数列\end{CJK} $\left\{f_{n}(x)\right\}_{n \geq 0}$ \begin{CJK}{UTF8}{mj}在区间\end{CJK} I \begin{CJK}{UTF8}{mj}上一致有界\end{CJK}, \begin{CJK}{UTF8}{mj}即存在常数\end{CJK} $M>0$, \begin{CJK}{UTF8}{mj}使得对所有的\end{CJK} $n \geq 0$ \begin{CJK}{UTF8}{mj}及\end{CJK} $x \in \mathrm{I}$ \begin{CJK}{UTF8}{mj}有\end{CJK} $\left|f_{n}(x)\right| \leq M$.

\begin{CJK}{UTF8}{mj}八\end{CJK}、 $\left(15\right.$ \begin{CJK}{UTF8}{mj}分\end{CJK}) \begin{CJK}{UTF8}{mj}设\end{CJK} $\left\{a_{k}\right\}_{k \geq 0},\left\{b_{k}\right\}_{k \geq 0},\left\{\xi_{k}\right\}_{k \geq 0}$ \begin{CJK}{UTF8}{mj}为非负数列\end{CJK}, \begin{CJK}{UTF8}{mj}而且对任意的\end{CJK} $k \geq 0$, \begin{CJK}{UTF8}{mj}有\end{CJK} $a_{k+1}^{2} \leq\left(a_{k}+b_{k}\right)^{2}-\xi_{k}^{2}$.

(1) \begin{CJK}{UTF8}{mj}证明\end{CJK} $\sum_{i=1}^{k} \xi_{i}^{2} \leq\left(a_{1}+\sum_{i=0}^{k} b_{i}\right)^{2}$.

(2) \begin{CJK}{UTF8}{mj}若数列\end{CJK} $\left\{b_{k}\right\}$ \begin{CJK}{UTF8}{mj}还满足\end{CJK} $\sum_{k=0}^{\infty} b_{k}^{2}<+\infty$, \begin{CJK}{UTF8}{mj}则\end{CJK} $\lim _{k \rightarrow \infty} \frac{1}{k} \sum_{i=1}^{k} \xi_{i}^{2}=0$. 13. \begin{CJK}{UTF8}{mj}中国科学院大学\end{CJK} 2012 \begin{CJK}{UTF8}{mj}年研究生入学考试试题数学分析\end{CJK}

\begin{CJK}{UTF8}{mj}李扬\end{CJK}

\begin{CJK}{UTF8}{mj}微信公众号\end{CJK}: sxkyliyang

\begin{CJK}{UTF8}{mj}一\end{CJK}、 (30 \begin{CJK}{UTF8}{mj}分\end{CJK}) \begin{CJK}{UTF8}{mj}计算极限\end{CJK}:

(1) $\lim _{n \rightarrow \infty} n^{3}\left(2 \sin \frac{1}{n}-\sin \frac{2}{n}\right)$.

(2) $\lim _{n \rightarrow \infty}\left(\sqrt{\cos \frac{1}{x^{2}}}\right)^{x^{4}}$.

\begin{CJK}{UTF8}{mj}二\end{CJK}、 (30 \begin{CJK}{UTF8}{mj}分\end{CJK}) \begin{CJK}{UTF8}{mj}计算积分\end{CJK}:

(1) $\mathrm{I}=\int_{0}^{\frac{\pi}{2}} \frac{\mathrm{d} x}{1+\tan ^{3} x}$.

(2) $\mathrm{J}=\iint_{S} x\left(1+y f\left(x^{2}+y^{2}\right)\right) \mathrm{d} x \mathrm{~d} y$, \begin{CJK}{UTF8}{mj}其中\end{CJK} $S$ \begin{CJK}{UTF8}{mj}为由曲线\end{CJK} $y=x^{3}, y=1, x=-1$ \begin{CJK}{UTF8}{mj}所围成的区域\end{CJK}, $f(x)$ \begin{CJK}{UTF8}{mj}为\end{CJK} \begin{CJK}{UTF8}{mj}实值连续函数\end{CJK}.

\begin{CJK}{UTF8}{mj}三\end{CJK}、 ( 15 \begin{CJK}{UTF8}{mj}分\end{CJK}) \begin{CJK}{UTF8}{mj}求下列幂级数的收玫域\end{CJK}:
$$
\sum_{n=1}^{\infty} \frac{x^{n}}{1+\frac{1}{2}+\cdots+\frac{1}{n}} .
$$
\begin{CJK}{UTF8}{mj}四\end{CJK}、( 15 \begin{CJK}{UTF8}{mj}分\end{CJK}) \begin{CJK}{UTF8}{mj}证明\end{CJK}: \begin{CJK}{UTF8}{mj}函数列\end{CJK} $s_{n}(x)=\frac{x}{1+n^{2} x^{2}}(n \geq 1)$ \begin{CJK}{UTF8}{mj}在区间\end{CJK} $(-\infty,+\infty)$ \begin{CJK}{UTF8}{mj}上的一致收敛\end{CJK}; \begin{CJK}{UTF8}{mj}函数列\end{CJK} $t_{n}(x)=$ $\frac{n x}{1+n^{2} x^{2}}(n \geq 1)$ \begin{CJK}{UTF8}{mj}在区间\end{CJK} $(0,1)$ \begin{CJK}{UTF8}{mj}上不一致收敛\end{CJK}.

\begin{CJK}{UTF8}{mj}五\end{CJK}、 $(15$ \begin{CJK}{UTF8}{mj}分\end{CJK}) \begin{CJK}{UTF8}{mj}设在区间\end{CJK} $[a, b]$ \begin{CJK}{UTF8}{mj}上\end{CJK}, $f(x)$ \begin{CJK}{UTF8}{mj}连续\end{CJK}, $g(x)$ \begin{CJK}{UTF8}{mj}可积\end{CJK}, \begin{CJK}{UTF8}{mj}并且\end{CJK} $f(x)>0, g(x)>0$. \begin{CJK}{UTF8}{mj}证明\end{CJK}
$$
\lim _{n \rightarrow \infty}\left(\int_{a}^{b} f^{n}(x) g(x) \mathrm{d} x\right)^{\frac{1}{n}}=\max _{a \leq x \leq b} f(x) .
$$
\begin{CJK}{UTF8}{mj}六\end{CJK}、 $\left(15\right.$ \begin{CJK}{UTF8}{mj}分\end{CJK} ) \begin{CJK}{UTF8}{mj}设在区间\end{CJK} $[0, a]$ \begin{CJK}{UTF8}{mj}上\end{CJK}, $f(x)$ \begin{CJK}{UTF8}{mj}二次可导\end{CJK}, \begin{CJK}{UTF8}{mj}并且\end{CJK} $|f(x)| \leq 1,\left|f^{\prime \prime}(x)\right| \leq 1$, \begin{CJK}{UTF8}{mj}证明\end{CJK}: \begin{CJK}{UTF8}{mj}当\end{CJK} $x \in[0, a]$ \begin{CJK}{UTF8}{mj}时\end{CJK}, \begin{CJK}{UTF8}{mj}有\end{CJK} $\left|f^{\prime}(x)\right| \leq \frac{a}{2}+\frac{2}{a} .$

\begin{CJK}{UTF8}{mj}七\end{CJK}、(15 \begin{CJK}{UTF8}{mj}分\end{CJK}) \begin{CJK}{UTF8}{mj}设\end{CJK} $n$ \begin{CJK}{UTF8}{mj}是一个正整数\end{CJK}. \begin{CJK}{UTF8}{mj}证明\end{CJK}: \begin{CJK}{UTF8}{mj}方程\end{CJK} $x^{n}+n x-1=0$ \begin{CJK}{UTF8}{mj}有唯一的正实根\end{CJK} $x_{n}$, \begin{CJK}{UTF8}{mj}并且当\end{CJK} $\alpha>1$ \begin{CJK}{UTF8}{mj}时\end{CJK}, \begin{CJK}{UTF8}{mj}级数\end{CJK} $\sum_{n=1}^{\infty} x_{n}^{\alpha}$ \begin{CJK}{UTF8}{mj}收敛\end{CJK}.

\begin{CJK}{UTF8}{mj}八\end{CJK}、 ( 15 \begin{CJK}{UTF8}{mj}分\end{CJK}) \begin{CJK}{UTF8}{mj}设\end{CJK} $\rho(x, y, z)$ \begin{CJK}{UTF8}{mj}是原点\end{CJK} $O$ \begin{CJK}{UTF8}{mj}到椭球面\end{CJK} $\frac{x^{2}}{2}+\frac{y^{2}}{2}+z^{2}=1$ \begin{CJK}{UTF8}{mj}的上半部分\end{CJK} (\begin{CJK}{UTF8}{mj}即满足\end{CJK} $z \geq 0$ \begin{CJK}{UTF8}{mj}的部分\end{CJK}) $\Sigma$ \begin{CJK}{UTF8}{mj}的任一\end{CJK} \begin{CJK}{UTF8}{mj}点\end{CJK} $(x, y, z)$ \begin{CJK}{UTF8}{mj}处的切面的距离\end{CJK}, \begin{CJK}{UTF8}{mj}求积分\end{CJK}
$$
\iint_{\Sigma} \frac{z}{\rho(x, y, z)} \mathrm{d} S .
$$

\begin{enumerate}
  \setcounter{enumi}{14}
  \item \begin{CJK}{UTF8}{mj}中国科学院大学\end{CJK} 2013 \begin{CJK}{UTF8}{mj}年研究生入学考试试题数学分析\end{CJK}
\end{enumerate}
\begin{CJK}{UTF8}{mj}李扬\end{CJK}

\begin{CJK}{UTF8}{mj}微信公众号\end{CJK}: sxkyliyang

\begin{CJK}{UTF8}{mj}一\end{CJK}、 (25 \begin{CJK}{UTF8}{mj}分\end{CJK}) \begin{CJK}{UTF8}{mj}计算\end{CJK}:

(1) ( 15 \begin{CJK}{UTF8}{mj}分\end{CJK}) $\lim _{n \rightarrow \infty} \sin ^{2}\left(\pi \sqrt{n^{2}+n}\right)$.

(2) ( 10 \begin{CJK}{UTF8}{mj}分\end{CJK}) $\lim _{n \rightarrow \infty} a_{n}$, \begin{CJK}{UTF8}{mj}其中设\end{CJK} $a_{1}=1, a_{n+1}=1+\frac{1}{a_{n}},(n \geq 1)$.

\begin{CJK}{UTF8}{mj}二\end{CJK}、 $\left(15\right.$ \begin{CJK}{UTF8}{mj}分\end{CJK}) \begin{CJK}{UTF8}{mj}设\end{CJK} $f(x)$ \begin{CJK}{UTF8}{mj}连续\end{CJK}, $g(x)=\int_{0}^{x} f(x-t) \sin t \mathrm{~d} t$, \begin{CJK}{UTF8}{mj}试证\end{CJK}
$$
g^{\prime \prime}(x)+g(x)=f(x), g(0)=g^{\prime}(0)=0 .
$$
\begin{CJK}{UTF8}{mj}三\end{CJK}、 ( 15 \begin{CJK}{UTF8}{mj}分\end{CJK}) \begin{CJK}{UTF8}{mj}求曲线\end{CJK} $y=e^{x}$ \begin{CJK}{UTF8}{mj}曲率的最大值\end{CJK}.

\begin{CJK}{UTF8}{mj}四\end{CJK}、(30 \begin{CJK}{UTF8}{mj}分\end{CJK}) \begin{CJK}{UTF8}{mj}计算积分\end{CJK}\\
(1) $\mathrm{I}=\int_{\frac{1}{4}}^{\frac{1}{2}} \mathrm{~d} y \int_{\frac{1}{2}}^{\sqrt{y}} e^{\frac{y}{x}} \mathrm{~d} x+\int_{\frac{1}{2}}^{1} \mathrm{~d} y \int_{y}^{\sqrt{y}} e^{\frac{y}{x}} \mathrm{~d} x$.\\
(2) $\mathrm{J}=\iint_{\Omega}\left|x^{2}+y^{2}-1\right| \mathrm{d} x \mathrm{~d} y$, \begin{CJK}{UTF8}{mj}其中\end{CJK} $\Omega=\{(x, y) \mid 0 \leq x \leq 1,0 \leq y \leq 1\}$.

\begin{CJK}{UTF8}{mj}五\end{CJK}、 ( 15 \begin{CJK}{UTF8}{mj}分\end{CJK}) \begin{CJK}{UTF8}{mj}讨论级数\end{CJK} $\sum_{n=1}^{\infty} \frac{n^{2}}{\left(x+\frac{1}{n}\right)^{n}}$ \begin{CJK}{UTF8}{mj}的收敛性和一致收玫性\end{CJK}(\begin{CJK}{UTF8}{mj}包括内闭一致收玫性\end{CJK}).

\begin{CJK}{UTF8}{mj}六\end{CJK}、 (20 \begin{CJK}{UTF8}{mj}分\end{CJK})

(1) \begin{CJK}{UTF8}{mj}证明\end{CJK}: \begin{CJK}{UTF8}{mj}当\end{CJK} $0<x<\frac{\pi}{2}$ \begin{CJK}{UTF8}{mj}时\end{CJK}, $\frac{2}{\pi}<\frac{\sin x}{x}<1$.

(2) \begin{CJK}{UTF8}{mj}设函数\end{CJK} $f(x)$ \begin{CJK}{UTF8}{mj}在闭区间\end{CJK} $[a, b]$ \begin{CJK}{UTF8}{mj}上二次可微\end{CJK}, \begin{CJK}{UTF8}{mj}且\end{CJK} $f^{\prime \prime}(x)<0$, \begin{CJK}{UTF8}{mj}则\end{CJK}
$$
\frac{f(a)+f(b)}{2} \leq \frac{1}{b-a} \int_{a}^{b} f(t) \mathrm{d} t .
$$
\begin{CJK}{UTF8}{mj}七\end{CJK}、 (15 \begin{CJK}{UTF8}{mj}分\end{CJK}) \begin{CJK}{UTF8}{mj}求函数\end{CJK} $f(x, y)=x^{2}+y^{2}+\frac{3}{2} x+1$ \begin{CJK}{UTF8}{mj}在集合\end{CJK} $G=\left\{(x, y) \in \mathbb{R}^{2} \mid 4 x^{2}+y^{2}-1=0\right\}$ \begin{CJK}{UTF8}{mj}上的最值\end{CJK}.

\begin{CJK}{UTF8}{mj}八\end{CJK}、 $\left(15\right.$ \begin{CJK}{UTF8}{mj}分\end{CJK}) \begin{CJK}{UTF8}{mj}设无穷实数列\end{CJK} $\left\{a_{n}\right\},\left\{b_{n}\right\}$ \begin{CJK}{UTF8}{mj}满足\end{CJK}
$$
a_{n+1}=b_{n}-\frac{n a_{n}}{2 n+1},(n=1,2, \cdots) .
$$
\begin{CJK}{UTF8}{mj}试证\end{CJK}:

(1) \begin{CJK}{UTF8}{mj}若\end{CJK} $\left\{b_{n}\right\}$ \begin{CJK}{UTF8}{mj}有界\end{CJK}, \begin{CJK}{UTF8}{mj}则\end{CJK} $\left\{a_{n}\right\}$ \begin{CJK}{UTF8}{mj}也有界\end{CJK}.

(2) \begin{CJK}{UTF8}{mj}若\end{CJK} $\left\{b_{n}\right\}$ \begin{CJK}{UTF8}{mj}收玫\end{CJK}, \begin{CJK}{UTF8}{mj}则\end{CJK} $\left\{a_{n}\right\}$ \begin{CJK}{UTF8}{mj}也收玫\end{CJK}. 15. \begin{CJK}{UTF8}{mj}中国科学院大学\end{CJK} 2014 \begin{CJK}{UTF8}{mj}年研究生入学考试试题数学分析\end{CJK}

\begin{CJK}{UTF8}{mj}李扬\end{CJK}

\begin{CJK}{UTF8}{mj}微信公众号\end{CJK}: sxkyliyang

\begin{CJK}{UTF8}{mj}一\end{CJK}、\begin{CJK}{UTF8}{mj}计算极限\end{CJK} $\lim _{n \rightarrow \infty}\left(\frac{\sin \frac{\pi}{n}}{n+1}+\frac{\sin \frac{2 \pi}{n}}{n+\frac{1}{2}}+\cdots+\frac{\sin \pi}{n+\frac{1}{n}}\right)$. (\begin{CJK}{UTF8}{mj}不知道原题是不是这样的\end{CJK}, \begin{CJK}{UTF8}{mj}这里给出第一项和最后一\end{CJK} \begin{CJK}{UTF8}{mj}项\end{CJK})

\begin{CJK}{UTF8}{mj}二\end{CJK}、\begin{CJK}{UTF8}{mj}已知函数\end{CJK} $f(x)=(1+x)^{\frac{1}{x}}$, \begin{CJK}{UTF8}{mj}计算\end{CJK} $f^{(i)}(0), i=1,2,3$.

\begin{CJK}{UTF8}{mj}三\end{CJK}、\begin{CJK}{UTF8}{mj}已知函数\end{CJK} $f(x)$ \begin{CJK}{UTF8}{mj}的反函数是\end{CJK} $\varphi(y)$, \begin{CJK}{UTF8}{mj}写出用\end{CJK} $f^{\prime}, f^{\prime \prime}, f^{\prime \prime \prime}$ \begin{CJK}{UTF8}{mj}表示\end{CJK} $\varphi^{\prime}, \varphi^{\prime \prime}, \varphi^{\prime \prime \prime}$ \begin{CJK}{UTF8}{mj}的表达式\end{CJK}.

\begin{CJK}{UTF8}{mj}四\end{CJK}、\begin{CJK}{UTF8}{mj}函数\end{CJK} $f(x)$ \begin{CJK}{UTF8}{mj}在\end{CJK} $[0,1]$ \begin{CJK}{UTF8}{mj}上连续\end{CJK}, $(0,1)$ \begin{CJK}{UTF8}{mj}上可导\end{CJK}, $f(0)=0, f(1)=1 / 2$, \begin{CJK}{UTF8}{mj}证明\end{CJK}: \begin{CJK}{UTF8}{mj}存在\end{CJK} $\xi, \eta \in(0,1)$ \begin{CJK}{UTF8}{mj}使得\end{CJK} $f(\xi)+$ $f^{\prime}(\eta)=\xi+\eta$.

\begin{CJK}{UTF8}{mj}五\end{CJK}、\begin{CJK}{UTF8}{mj}已知\end{CJK} $a_{0}>0, a_{n}=\sqrt{a_{n-1}+6}$, \begin{CJK}{UTF8}{mj}求极限\end{CJK} $\lim _{n \rightarrow \infty} a_{n}$.

\begin{CJK}{UTF8}{mj}只\end{CJK}、\begin{CJK}{UTF8}{mj}证明\end{CJK}:
$$
1 \leq \iint_{D}\left(\sin x^{2}+\cos y^{2}\right) \mathrm{d} x \mathrm{~d} y \leq \sqrt{2}
$$
\begin{CJK}{UTF8}{mj}其中\end{CJK} $D=\{(x, y) \mid 0 \leq x \leq 1,0 \leq y \leq 1\}$.

\begin{CJK}{UTF8}{mj}七\end{CJK}、\begin{CJK}{UTF8}{mj}求由\end{CJK} $z=x+y$ \begin{CJK}{UTF8}{mj}和\end{CJK} $z=x^{2}+y^{2}$ \begin{CJK}{UTF8}{mj}围成的几何体的体积\end{CJK}.

\begin{CJK}{UTF8}{mj}八\end{CJK}、\begin{CJK}{UTF8}{mj}讨论函数项级数\end{CJK} $\sum_{n=0}^{\infty} \frac{\left(x^{2}+x+1\right)^{n}}{n(n+1)}$ \begin{CJK}{UTF8}{mj}的收玫性与一致收敛性\end{CJK}.

\begin{CJK}{UTF8}{mj}九\end{CJK}、 (1) \begin{CJK}{UTF8}{mj}函数\end{CJK} $u(x)$ \begin{CJK}{UTF8}{mj}在\end{CJK} $[0,1]$ \begin{CJK}{UTF8}{mj}连续\end{CJK}, \begin{CJK}{UTF8}{mj}且\end{CJK} $u(x)$ \begin{CJK}{UTF8}{mj}绝对可积\end{CJK}, \begin{CJK}{UTF8}{mj}求证\end{CJK}:
$$
\sup _{x \in[0,1]}|u(x)| \leq \int_{0}^{1}|u(x)| \mathrm{d} x+\int_{0}^{1}\left|u^{\prime}(x)\right| \mathrm{d} x .
$$
(2) \begin{CJK}{UTF8}{mj}二元函数\end{CJK} $u(x, y)$ \begin{CJK}{UTF8}{mj}在\end{CJK} $D=\{(x, y) \mid 0 \leq x \leq 1,0 \leq y \leq 1\}$ \begin{CJK}{UTF8}{mj}上连续\end{CJK}, \begin{CJK}{UTF8}{mj}且偏导数\end{CJK} $\frac{\partial u}{\partial x}, \frac{\partial u}{\partial y}, \frac{\partial^{2} u}{\partial x \partial y}$ \begin{CJK}{UTF8}{mj}绝对可积\end{CJK}, \begin{CJK}{UTF8}{mj}求证\end{CJK}:
$$
\sup _{(x, y) \in D}|u| \leq \iint_{D}|u| \mathrm{d} x \mathrm{~d} y+\iint_{D}\left(\left|\frac{\partial u}{\partial x}\right|+\left|\frac{\partial u}{\partial y}\right|\right) \mathrm{d} x \mathrm{~d} y+\iint_{D}\left|\frac{\partial^{2} u}{\partial x \partial y}\right| \mathrm{d} x \mathrm{~d} y
$$

\begin{enumerate}
  \setcounter{enumi}{16}
  \item \begin{CJK}{UTF8}{mj}中国科学院大学\end{CJK} 2015 \begin{CJK}{UTF8}{mj}年研究生入学考试试题数学分析\end{CJK}
\end{enumerate}
\begin{CJK}{UTF8}{mj}李扬\end{CJK}

\begin{CJK}{UTF8}{mj}微信公众号\end{CJK}: sxkyliyang

\begin{CJK}{UTF8}{mj}一\end{CJK}、 (10 \begin{CJK}{UTF8}{mj}分\end{CJK}) \begin{CJK}{UTF8}{mj}已知数列\end{CJK} $x_{n}$ \begin{CJK}{UTF8}{mj}且\end{CJK} $\lim _{n \rightarrow \infty} x_{n}=+\infty$, \begin{CJK}{UTF8}{mj}证明\end{CJK}: $\lim _{n \rightarrow \infty} \frac{x_{1}+x_{2}+\cdots+x_{n}}{n}=+\infty$.

\begin{CJK}{UTF8}{mj}二\end{CJK}、 $\left(10\right.$ \begin{CJK}{UTF8}{mj}分\end{CJK}) \begin{CJK}{UTF8}{mj}求极限\end{CJK} $\lim _{(x, y) \rightarrow(0,0)} x^{2} y^{2} \ln \left(x^{2}+y^{2}\right)$.

\begin{CJK}{UTF8}{mj}三\end{CJK}、 $(10$ \begin{CJK}{UTF8}{mj}分\end{CJK}) \begin{CJK}{UTF8}{mj}已知\end{CJK}
$$
f(x)= \begin{cases}a x+b, & x<0 \\ x^{2}+1, & x \geq 0\end{cases}
$$
\begin{CJK}{UTF8}{mj}若\end{CJK} $f(x)$ \begin{CJK}{UTF8}{mj}在\end{CJK} $x=0$ \begin{CJK}{UTF8}{mj}处导数连续\end{CJK}, \begin{CJK}{UTF8}{mj}求\end{CJK} $a, b$ \begin{CJK}{UTF8}{mj}的值\end{CJK}.

\begin{CJK}{UTF8}{mj}四\end{CJK}、 (15 \begin{CJK}{UTF8}{mj}分\end{CJK}) \begin{CJK}{UTF8}{mj}已知\end{CJK} $f(x)$ \begin{CJK}{UTF8}{mj}在\end{CJK} $[a, b]$ \begin{CJK}{UTF8}{mj}上连续\end{CJK}, $(a, b)$ \begin{CJK}{UTF8}{mj}内可导\end{CJK}, \begin{CJK}{UTF8}{mj}且\end{CJK} $f(a)=f(b)=0$, \begin{CJK}{UTF8}{mj}存在\end{CJK} $c \in(a, b)$ \begin{CJK}{UTF8}{mj}使得\end{CJK} $f(c)>0$, \begin{CJK}{UTF8}{mj}证\end{CJK} \begin{CJK}{UTF8}{mj}明\end{CJK}: $\exists \xi \in(a, b)$ \begin{CJK}{UTF8}{mj}使得\end{CJK} $f^{\prime \prime}(\xi)<0$.

\begin{CJK}{UTF8}{mj}五\end{CJK}、 (15 \begin{CJK}{UTF8}{mj}分\end{CJK}) \begin{CJK}{UTF8}{mj}证明\end{CJK}: $f(x)=\sum_{n=1}^{\infty} \cdots$ \begin{CJK}{UTF8}{mj}连续\end{CJK}, \begin{CJK}{UTF8}{mj}且有连续的导函数\end{CJK}.

\begin{CJK}{UTF8}{mj}六\end{CJK}、 (15 \begin{CJK}{UTF8}{mj}分\end{CJK}) \begin{CJK}{UTF8}{mj}已知一个半径为\end{CJK} $r$ \begin{CJK}{UTF8}{mj}的球\end{CJK}, \begin{CJK}{UTF8}{mj}求与之相切的正圆雉的最小体积\end{CJK}.

\begin{CJK}{UTF8}{mj}七\end{CJK}、 (15 \begin{CJK}{UTF8}{mj}分\end{CJK}) \begin{CJK}{UTF8}{mj}设\end{CJK} $p(x)$ \begin{CJK}{UTF8}{mj}是不超过三次的多项式\end{CJK}, \begin{CJK}{UTF8}{mj}证明\end{CJK}: \begin{CJK}{UTF8}{mj}对任意的\end{CJK} $a, b$ \begin{CJK}{UTF8}{mj}有\end{CJK} $\int_{a}^{b} f(x) \mathrm{d} x=\cdots$.

\begin{CJK}{UTF8}{mj}八\end{CJK}、(15 \begin{CJK}{UTF8}{mj}分\end{CJK}) \begin{CJK}{UTF8}{mj}求\end{CJK} $\iint_{D} x y \mathrm{~d} x \mathrm{~d} y$, \begin{CJK}{UTF8}{mj}其中\end{CJK} $D$ \begin{CJK}{UTF8}{mj}由\end{CJK} $y=x^{2}, y=\frac{1}{2} x$ \begin{CJK}{UTF8}{mj}围成\end{CJK} $x>0$ \begin{CJK}{UTF8}{mj}的部分\end{CJK}.

\begin{CJK}{UTF8}{mj}九\end{CJK}、 ( 15 \begin{CJK}{UTF8}{mj}分\end{CJK}) \begin{CJK}{UTF8}{mj}已知空间中两个点\end{CJK} $A(1,0,-1), B(0,1,1)$, \begin{CJK}{UTF8}{mj}求线段\end{CJK} $A B$ \begin{CJK}{UTF8}{mj}旋转一周与\end{CJK} $z=1, z=-1$ \begin{CJK}{UTF8}{mj}所围体积\end{CJK}.

\begin{CJK}{UTF8}{mj}十\end{CJK}、 (15 \begin{CJK}{UTF8}{mj}分\end{CJK}) \begin{CJK}{UTF8}{mj}已知\end{CJK} $f(x)$ \begin{CJK}{UTF8}{mj}在\end{CJK} $[0,1]$ \begin{CJK}{UTF8}{mj}上单调递减\end{CJK}, \begin{CJK}{UTF8}{mj}证明\end{CJK}: $\forall \lambda \in(0,1)$, \begin{CJK}{UTF8}{mj}有\end{CJK} $\int_{0}^{\lambda} f(x) \mathrm{d} x \geq \lambda \int_{0}^{1} f(x) \mathrm{d} x$.

\begin{CJK}{UTF8}{mj}十一\end{CJK}、(15 \begin{CJK}{UTF8}{mj}分\end{CJK}) \begin{CJK}{UTF8}{mj}证明\end{CJK}: $\frac{\pi}{4}\left(1-\frac{1}{e}\right)<\left(\int_{0}^{1} e^{-x^{2}} \mathrm{~d} x\right)^{2}<\frac{16}{25}$ 17. \begin{CJK}{UTF8}{mj}中国科学院大学\end{CJK} 2016 \begin{CJK}{UTF8}{mj}年研究生入学考试试题数学分析\end{CJK}

\begin{CJK}{UTF8}{mj}李扬\end{CJK}

\begin{CJK}{UTF8}{mj}微信公众号\end{CJK}: sxkyliyang

\begin{CJK}{UTF8}{mj}一\end{CJK}、(20 \begin{CJK}{UTF8}{mj}分\end{CJK}) \begin{CJK}{UTF8}{mj}计算极限\end{CJK}
$$
\lim _{x \rightarrow 0}\left(\frac{e^{x}+e^{2 x}+\cdots+e^{n x}}{n}\right)^{\frac{1}{x}} .
$$
\begin{CJK}{UTF8}{mj}二\end{CJK}、 (20 \begin{CJK}{UTF8}{mj}分\end{CJK}) \begin{CJK}{UTF8}{mj}求定积分\end{CJK}
$$
\mathrm{I}=\int_{0}^{1} \log (1+\sqrt{x}) \mathrm{d} x .
$$
\begin{CJK}{UTF8}{mj}三\end{CJK}、 (15 \begin{CJK}{UTF8}{mj}分\end{CJK}) \begin{CJK}{UTF8}{mj}求二重极限\end{CJK}
$$
\lim _{\substack{x \rightarrow \infty \\ y \rightarrow \infty}} \frac{x+y}{x^{2}-x y+y^{2}} .
$$
\begin{CJK}{UTF8}{mj}四\end{CJK}、(12 \begin{CJK}{UTF8}{mj}分\end{CJK}) $f(x)$ \begin{CJK}{UTF8}{mj}是\end{CJK} $[a, b]$ \begin{CJK}{UTF8}{mj}上的连续正函数\end{CJK}, \begin{CJK}{UTF8}{mj}求证存在\end{CJK} $\xi \in(a, b)$, \begin{CJK}{UTF8}{mj}使得\end{CJK}
$$
\int_{a}^{\xi} f(x) \mathrm{d} x=\int_{\xi}^{b} f(x) \mathrm{d} x=\frac{1}{2} \int_{a}^{b} f(x) \mathrm{d} x
$$
\begin{CJK}{UTF8}{mj}五\end{CJK}、( 15 \begin{CJK}{UTF8}{mj}分\end{CJK}) \begin{CJK}{UTF8}{mj}求以下曲面所围立体的体积\end{CJK}
$$
\mathrm{S}_{1}: \frac{x^{2}}{a^{2}}+\frac{y^{2}}{b^{2}}+\frac{z^{2}}{c^{2}}=1, \mathrm{~S}_{2}: \frac{x^{2}}{a^{2}}+\frac{y^{2}}{b^{2}}=\frac{z^{2}}{c^{2}}(z \geq 0) .
$$
\begin{CJK}{UTF8}{mj}六\end{CJK}、( 12 \begin{CJK}{UTF8}{mj}分\end{CJK}) $f(x)$ \begin{CJK}{UTF8}{mj}是\end{CJK} $[a, b]$ \begin{CJK}{UTF8}{mj}上的连续函数\end{CJK}, \begin{CJK}{UTF8}{mj}且\end{CJK} $f(x)$ \begin{CJK}{UTF8}{mj}单调递增\end{CJK}, \begin{CJK}{UTF8}{mj}求证\end{CJK}:
$$
\int_{a}^{b} t f(t) \mathrm{d} t \geq \frac{a+b}{2} \int_{a}^{b} f(t) \mathrm{d} t .
$$
\begin{CJK}{UTF8}{mj}七\end{CJK}、 (12 \begin{CJK}{UTF8}{mj}分\end{CJK}) \begin{CJK}{UTF8}{mj}若数列\end{CJK} $\left\{a_{n}\right\},\left\{b_{n}\right\}$ \begin{CJK}{UTF8}{mj}满足以下条件\end{CJK}:

(a) $a_{1} \geq a_{2} \geq \cdots$ \begin{CJK}{UTF8}{mj}且\end{CJK} $\lim _{n \rightarrow \infty} a_{n}=0$;

(b) \begin{CJK}{UTF8}{mj}存在正数\end{CJK} $M$, \begin{CJK}{UTF8}{mj}对任意的正整数\end{CJK} $n$, \begin{CJK}{UTF8}{mj}均有\end{CJK} $\left|\sum_{k=1}^{n} b_{k}\right| \leq M$.

\begin{CJK}{UTF8}{mj}证明级数\end{CJK} $\sum_{n=1}^{\infty} a_{n} b_{n}$ \begin{CJK}{UTF8}{mj}收敛\end{CJK}.

\begin{CJK}{UTF8}{mj}八\end{CJK}、(15 \begin{CJK}{UTF8}{mj}分\end{CJK}) \begin{CJK}{UTF8}{mj}设\end{CJK} $0 \leq a<b / 2, f(x)$ \begin{CJK}{UTF8}{mj}在\end{CJK} $[a, b]$ \begin{CJK}{UTF8}{mj}上连续\end{CJK}, \begin{CJK}{UTF8}{mj}在\end{CJK} $(a, b)$ \begin{CJK}{UTF8}{mj}可导且\end{CJK} $f(a)=a, f(b)=b$.

(a) \begin{CJK}{UTF8}{mj}求证存在\end{CJK} $\xi \in(a, b)$ \begin{CJK}{UTF8}{mj}使得\end{CJK} $f(\xi)=b-\xi$.

(b) \begin{CJK}{UTF8}{mj}若\end{CJK} $a=0$, \begin{CJK}{UTF8}{mj}求证存在\end{CJK} $\alpha, \beta \in(a, b), \alpha \neq \beta$, \begin{CJK}{UTF8}{mj}使得\end{CJK} $f^{\prime}(\alpha) f^{\prime}(\beta)=1$.

\begin{CJK}{UTF8}{mj}九\end{CJK}、 (15 \begin{CJK}{UTF8}{mj}分\end{CJK}) \begin{CJK}{UTF8}{mj}求椭圆\end{CJK} $x^{2}+4 y^{2}=4$ \begin{CJK}{UTF8}{mj}上到直线\end{CJK} $2 x+3 y=6$ \begin{CJK}{UTF8}{mj}距离最短的点\end{CJK}, \begin{CJK}{UTF8}{mj}并求其最短距离\end{CJK}.

\begin{CJK}{UTF8}{mj}十\end{CJK}、 (15 \begin{CJK}{UTF8}{mj}分\end{CJK}) \begin{CJK}{UTF8}{mj}半径为\end{CJK} $R$ \begin{CJK}{UTF8}{mj}的球面\end{CJK} $S$ \begin{CJK}{UTF8}{mj}的球心在单位球面\end{CJK} $x^{2}+y^{2}+z^{2}=1$ \begin{CJK}{UTF8}{mj}上\end{CJK}, \begin{CJK}{UTF8}{mj}求球面\end{CJK} $S$ \begin{CJK}{UTF8}{mj}在单位球内面积的最大\end{CJK} \begin{CJK}{UTF8}{mj}值\end{CJK}, \begin{CJK}{UTF8}{mj}并求出此时的\end{CJK} $R$. 18. \begin{CJK}{UTF8}{mj}中国科学院大学\end{CJK} 2017 \begin{CJK}{UTF8}{mj}年研究生入学考试试题数学分析\end{CJK}

\begin{CJK}{UTF8}{mj}李扬\end{CJK}

\begin{CJK}{UTF8}{mj}微信公众号\end{CJK}: sxkyliyang

\begin{CJK}{UTF8}{mj}一\end{CJK}、(10 \begin{CJK}{UTF8}{mj}分\end{CJK}) \begin{CJK}{UTF8}{mj}计算极限\end{CJK}
$$
\lim _{x \rightarrow+\infty} x^{\frac{3}{2}}(\sqrt{2+x}-2 \sqrt{1+x}+\sqrt{x}) .
$$
\begin{CJK}{UTF8}{mj}二\end{CJK}、 $\left(10\right.$ \begin{CJK}{UTF8}{mj}分\end{CJK}) \begin{CJK}{UTF8}{mj}已知\end{CJK} $a_{n+1}\left(a_{n}+1\right)=1, a_{0}=0$, \begin{CJK}{UTF8}{mj}证明数列的极限存在\end{CJK}, \begin{CJK}{UTF8}{mj}并求出极限值\end{CJK}.

\begin{CJK}{UTF8}{mj}三\end{CJK}、 ( 15 \begin{CJK}{UTF8}{mj}分\end{CJK}) $f(x)$ \begin{CJK}{UTF8}{mj}三次连续可微\end{CJK}, \begin{CJK}{UTF8}{mj}令\end{CJK} $u(x, y, z)=f(x y z)$, \begin{CJK}{UTF8}{mj}求\end{CJK} $\phi(t)=\frac{\partial^{3} u}{\partial x \partial y \partial z}$ \begin{CJK}{UTF8}{mj}的具体表达式\end{CJK}, \begin{CJK}{UTF8}{mj}其中\end{CJK} $t=x y z$.

\begin{CJK}{UTF8}{mj}四\end{CJK}、(15 \begin{CJK}{UTF8}{mj}分\end{CJK}) \begin{CJK}{UTF8}{mj}求\end{CJK}
$$
\int \frac{\mathrm{d} x}{1+x^{4}}
$$
\begin{CJK}{UTF8}{mj}五\end{CJK}、 (15 \begin{CJK}{UTF8}{mj}分\end{CJK}) \begin{CJK}{UTF8}{mj}已知\end{CJK} $f(x)$ \begin{CJK}{UTF8}{mj}在\end{CJK} $[0,1]$ \begin{CJK}{UTF8}{mj}上二阶连续可微\end{CJK}, \begin{CJK}{UTF8}{mj}并且\end{CJK} $|f(x)| \leq a,\left|f^{\prime \prime}(x)\right| \leq b$, \begin{CJK}{UTF8}{mj}证明\end{CJK} $f^{\prime}(x) \leq 2 a+\frac{b}{2}$.

\begin{CJK}{UTF8}{mj}六\end{CJK}、 (15 \begin{CJK}{UTF8}{mj}分\end{CJK}) \begin{CJK}{UTF8}{mj}已知\end{CJK} $f(x)$ \begin{CJK}{UTF8}{mj}有界且可微\end{CJK}, \begin{CJK}{UTF8}{mj}假设\end{CJK} $\lim _{x \rightarrow \infty} f^{\prime}(x)$ \begin{CJK}{UTF8}{mj}存在\end{CJK}, \begin{CJK}{UTF8}{mj}求证\end{CJK} $\lim _{x \rightarrow \infty} f^{\prime}(x)=0$.

\begin{CJK}{UTF8}{mj}七\end{CJK}、 (15 \begin{CJK}{UTF8}{mj}分\end{CJK}) \begin{CJK}{UTF8}{mj}求二重积分\end{CJK} $\iint_{D}\left|x^{2}+y^{2}-1\right| \mathrm{d} x \mathrm{~d} y$, \begin{CJK}{UTF8}{mj}其中\end{CJK} $D=\{(x, y) \mid 0 \leq x \leq 1,0 \leq y \leq 1\}$.

\begin{CJK}{UTF8}{mj}八\end{CJK}、 (15 \begin{CJK}{UTF8}{mj}分\end{CJK}) \begin{CJK}{UTF8}{mj}已知\end{CJK} $a_{n}=\sum_{k=1}^{n} \ln (k+1)$, \begin{CJK}{UTF8}{mj}证明\end{CJK} $\sum_{n=1}^{\infty} \frac{1}{a_{n}}$ \begin{CJK}{UTF8}{mj}发散\end{CJK}.

\begin{CJK}{UTF8}{mj}九\end{CJK}、 (15 \begin{CJK}{UTF8}{mj}分\end{CJK}) \begin{CJK}{UTF8}{mj}已知\end{CJK} $n$ \begin{CJK}{UTF8}{mj}为整数\end{CJK}, $a$ \begin{CJK}{UTF8}{mj}为常数\end{CJK}, $\mathrm{I}_{n}(a)=\int_{0}^{\infty} \frac{\mathrm{d} x}{1+n x^{a}}$.

(1) \begin{CJK}{UTF8}{mj}试讨论\end{CJK} $a$ \begin{CJK}{UTF8}{mj}对玫散性的影响\end{CJK}.

(2) \begin{CJK}{UTF8}{mj}当\end{CJK} $a$ \begin{CJK}{UTF8}{mj}在使积分收玫的情况下\end{CJK}, \begin{CJK}{UTF8}{mj}求\end{CJK} $\lim _{n \rightarrow \infty} \mathrm{I}_{n}(a)$.

\begin{CJK}{UTF8}{mj}十\end{CJK}、 (15 \begin{CJK}{UTF8}{mj}分\end{CJK}) \begin{CJK}{UTF8}{mj}在\end{CJK} $[a, b]$ \begin{CJK}{UTF8}{mj}上\end{CJK} $(0<a<b)$, \begin{CJK}{UTF8}{mj}证明下面的不等式成立\end{CJK}
$$
\int_{a}^{b}\left(x^{2}+1\right) e^{-x^{2}} \mathrm{~d} x \geq e^{-a^{2}}-e^{-b^{2}} .
$$
\begin{CJK}{UTF8}{mj}十一\end{CJK}、(10 \begin{CJK}{UTF8}{mj}分\end{CJK}) \begin{CJK}{UTF8}{mj}求\end{CJK} $f(x)=e^{x}+e^{-x}+2 \cos x$ \begin{CJK}{UTF8}{mj}的极值\end{CJK}. 19. \begin{CJK}{UTF8}{mj}中国科学院大学\end{CJK} 2018 \begin{CJK}{UTF8}{mj}年研究生入学考试试题数学分析\end{CJK}

\begin{CJK}{UTF8}{mj}李扬\end{CJK}

\begin{CJK}{UTF8}{mj}微信公众号\end{CJK}: sxkyliyang

\begin{CJK}{UTF8}{mj}一\end{CJK}、 (15 \begin{CJK}{UTF8}{mj}分\end{CJK}) \begin{CJK}{UTF8}{mj}计算极限\end{CJK} $\lim _{x \rightarrow \infty}\left(\sin \frac{1}{x}+\cos \frac{1}{x}\right)^{x}$.

\begin{CJK}{UTF8}{mj}二\end{CJK}、 (15 \begin{CJK}{UTF8}{mj}分\end{CJK}) \begin{CJK}{UTF8}{mj}计算极限\end{CJK} $\lim _{x \rightarrow 0}\left(\frac{4+e^{\frac{1}{x}}}{2+e^{\frac{4}{x}}}+\frac{\sin x}{|x|}\right)$.

\begin{CJK}{UTF8}{mj}三\end{CJK}、 ( 15 \begin{CJK}{UTF8}{mj}分\end{CJK}) \begin{CJK}{UTF8}{mj}判断\end{CJK}(\begin{CJK}{UTF8}{mj}并证明\end{CJK}) \begin{CJK}{UTF8}{mj}函数\end{CJK} $f(x, y)=\sqrt{|x y|}$ \begin{CJK}{UTF8}{mj}在点\end{CJK} $(0,0)$ \begin{CJK}{UTF8}{mj}处的可微性\end{CJK}.

\begin{CJK}{UTF8}{mj}四\end{CJK}、(15 \begin{CJK}{UTF8}{mj}分\end{CJK}) \begin{CJK}{UTF8}{mj}求三个实常数\end{CJK} $a, b, c$, \begin{CJK}{UTF8}{mj}使得下式成立\end{CJK}
$$
\lim _{x \rightarrow 0} \frac{1}{\tan x-a x} \int_{b}^{x} \frac{s^{2}}{\sqrt{1-s^{2}}} \mathrm{~d} s=c .
$$
\begin{CJK}{UTF8}{mj}五\end{CJK}、 (15 \begin{CJK}{UTF8}{mj}分\end{CJK}) \begin{CJK}{UTF8}{mj}计算不定积分\end{CJK}
$$
\int \frac{\mathrm{d} x}{\sin ^{6} x+\cos ^{6} x} .
$$
\begin{CJK}{UTF8}{mj}六\end{CJK}、 (15 \begin{CJK}{UTF8}{mj}分\end{CJK}) \begin{CJK}{UTF8}{mj}设函数\end{CJK} $f(x)$ \begin{CJK}{UTF8}{mj}在\end{CJK} $[-1,1]$ \begin{CJK}{UTF8}{mj}二次连续可微\end{CJK}, $f(0)=0$, \begin{CJK}{UTF8}{mj}证明\end{CJK}:
$$
\left|\int_{-1}^{1} f(x) \mathrm{d} x\right| \leq \frac{M}{3},
$$
\begin{CJK}{UTF8}{mj}其中\end{CJK} $M=\max _{x \in[-1,1]}\left|f^{\prime \prime}(x)\right|$.

\begin{CJK}{UTF8}{mj}七\end{CJK}、 (15 \begin{CJK}{UTF8}{mj}分\end{CJK}) \begin{CJK}{UTF8}{mj}求曲线\end{CJK} $y=\frac{1}{2} x^{2}$ \begin{CJK}{UTF8}{mj}上的点\end{CJK}, \begin{CJK}{UTF8}{mj}使得曲线在该点处的法线被曲线所截得的线段长度最短\end{CJK}.

\begin{CJK}{UTF8}{mj}八\end{CJK}、(15 \begin{CJK}{UTF8}{mj}分\end{CJK}) \begin{CJK}{UTF8}{mj}设\end{CJK} $x>0$, \begin{CJK}{UTF8}{mj}证明\end{CJK}
$$
\sqrt{x+1}-\sqrt{x}=\frac{1}{2 \sqrt{x+\theta}},
$$
\begin{CJK}{UTF8}{mj}其中\end{CJK} $\theta=\theta(x)>0$, \begin{CJK}{UTF8}{mj}并且\end{CJK} $\lim _{x \rightarrow 0} \theta(x)=\frac{1}{4}$.

\begin{CJK}{UTF8}{mj}九\end{CJK}、 (15 \begin{CJK}{UTF8}{mj}分\end{CJK}) \begin{CJK}{UTF8}{mj}设\end{CJK}
$$
u_{n}(x)=\frac{(-1)^{n}}{\left(n^{2}-n+1\right)^{x}} \quad(n \geq 0),
$$
\begin{CJK}{UTF8}{mj}求函数\end{CJK} $f(x)=\sum_{n=0}^{\infty} u_{n}(x)$ \begin{CJK}{UTF8}{mj}的绝对收玫\end{CJK}、\begin{CJK}{UTF8}{mj}条件收玫以及发散的区域\end{CJK}.

\begin{CJK}{UTF8}{mj}十\end{CJK}、 (15 \begin{CJK}{UTF8}{mj}分\end{CJK}) \begin{CJK}{UTF8}{mj}证明\end{CJK}
$$
\frac{1}{5}<\int_{0}^{1} \frac{x e^{x}}{\sqrt{x^{2}-x+25}} \mathrm{~d} x<\frac{2 \sqrt{11}}{\sqrt{33}} .
$$

\section{1. 中山大学 2009 年研究生入学考试试题高等代数}
\begin{CJK}{UTF8}{mj}李扬\end{CJK}

\begin{CJK}{UTF8}{mj}微信公众号\end{CJK}: sxkyliyang

\begin{enumerate}
  \item (10 \begin{CJK}{UTF8}{mj}分\end{CJK}) \begin{CJK}{UTF8}{mj}计算行列式\end{CJK}
\end{enumerate}
$$
D=\left|\begin{array}{cccccc}
1 & a_{1} & a_{1}^{2} & \cdots & a_{1}^{n-2} & a_{1}^{n} \\
1 & a_{2} & a_{2}^{2} & \cdots & a_{2}^{n-2} & a_{2}^{n} \\
\vdots & \vdots & \vdots & & \vdots & \vdots \\
1 & a_{n} & a_{n}^{2} & \cdots & a_{n}^{n-2} & a_{n}^{n}
\end{array}\right| .
$$

\begin{enumerate}
  \setcounter{enumi}{2}
  \item (20 \begin{CJK}{UTF8}{mj}分\end{CJK}) \begin{CJK}{UTF8}{mj}证明\end{CJK}:
\end{enumerate}
(1) \begin{CJK}{UTF8}{mj}对任意矩阵\end{CJK} $A$, \begin{CJK}{UTF8}{mj}矩阵方程\end{CJK} $A X A=A$ \begin{CJK}{UTF8}{mj}都有解\end{CJK};

(2) \begin{CJK}{UTF8}{mj}如果矩阵方程\end{CJK} $A Y=C$ \begin{CJK}{UTF8}{mj}和\end{CJK} $Z B=C$ \begin{CJK}{UTF8}{mj}有解\end{CJK}, \begin{CJK}{UTF8}{mj}则方程\end{CJK} $A X B=C$ \begin{CJK}{UTF8}{mj}有解\end{CJK}.

\begin{enumerate}
  \setcounter{enumi}{3}
  \item ( 20 \begin{CJK}{UTF8}{mj}分\end{CJK}) \begin{CJK}{UTF8}{mj}设\end{CJK} $f: \mathbb{R}^{2} \rightarrow \mathbb{R}$ \begin{CJK}{UTF8}{mj}是线性映射\end{CJK}.
\end{enumerate}
(1) \begin{CJK}{UTF8}{mj}证明\end{CJK}: \begin{CJK}{UTF8}{mj}存在\end{CJK} $a, b \in \mathbb{R}$ \begin{CJK}{UTF8}{mj}使得对任意\end{CJK} $(x, y) \in \mathbb{R}^{2}$ \begin{CJK}{UTF8}{mj}有\end{CJK} $f(x, y)=a x+b y$;

(2) \begin{CJK}{UTF8}{mj}已知\end{CJK} $f(1,1)=3, f(1,0)=4$, \begin{CJK}{UTF8}{mj}求\end{CJK} $f(2,1)$.

\begin{enumerate}
  \setcounter{enumi}{4}
  \item (15 \begin{CJK}{UTF8}{mj}分\end{CJK}) \begin{CJK}{UTF8}{mj}设\end{CJK} $V$ \begin{CJK}{UTF8}{mj}是\end{CJK} $\mathbb{F}$ \begin{CJK}{UTF8}{mj}上\end{CJK} $n$ \begin{CJK}{UTF8}{mj}维线性空间\end{CJK}, $f$ \begin{CJK}{UTF8}{mj}是\end{CJK} $V$ \begin{CJK}{UTF8}{mj}上的一个非零线性函数\end{CJK}. \begin{CJK}{UTF8}{mj}证明\end{CJK}: \begin{CJK}{UTF8}{mj}存在\end{CJK} $V$ \begin{CJK}{UTF8}{mj}的一个基\end{CJK} $\left(e_{1}, e_{2}, \cdots, e_{n}\right)$, \begin{CJK}{UTF8}{mj}使得对任意向量\end{CJK} $x=\sum_{i=1}^{n} x_{i} e_{i} \in V$, \begin{CJK}{UTF8}{mj}有\end{CJK} $f(x)=x_{1}$.

  \item ( 15 \begin{CJK}{UTF8}{mj}分\end{CJK}) \begin{CJK}{UTF8}{mj}求一个次数最低的多项式\end{CJK} $f(x) \in \mathbb{F}[x]$, \begin{CJK}{UTF8}{mj}使得它被\end{CJK} $(x-1)^{2}$ \begin{CJK}{UTF8}{mj}除所得余式为\end{CJK} $2 x$, \begin{CJK}{UTF8}{mj}而被\end{CJK} $(x-2)^{3}$ \begin{CJK}{UTF8}{mj}除余式为\end{CJK} $3 x$.

  \item (15 \begin{CJK}{UTF8}{mj}分\end{CJK}) \begin{CJK}{UTF8}{mj}设\end{CJK} $f(x) \in \mathbb{F}[x]$ \begin{CJK}{UTF8}{mj}是一个次数大于零的多项式\end{CJK}. \begin{CJK}{UTF8}{mj}证明\end{CJK}: $f(x)$ \begin{CJK}{UTF8}{mj}不可约的充分必要条件是\end{CJK}: \begin{CJK}{UTF8}{mj}由\end{CJK} " $f(x)$ \begin{CJK}{UTF8}{mj}整除两\end{CJK} \begin{CJK}{UTF8}{mj}个多项式的乘积\end{CJK}" \begin{CJK}{UTF8}{mj}可推出\end{CJK} " $f(x)$ \begin{CJK}{UTF8}{mj}必整除其中的一个\end{CJK}".

  \item ( 15 \begin{CJK}{UTF8}{mj}分\end{CJK}) \begin{CJK}{UTF8}{mj}求复矩阵\end{CJK}

\end{enumerate}
$$
A=\left(\begin{array}{ccc}
1 & 2 & 0 \\
0 & 2 & 0 \\
-2 & -2 & 1
\end{array}\right)
$$
\begin{CJK}{UTF8}{mj}的若当标准形及最小多项式\end{CJK}.

\begin{enumerate}
  \setcounter{enumi}{8}
  \item (10 \begin{CJK}{UTF8}{mj}分\end{CJK}) \begin{CJK}{UTF8}{mj}证明\end{CJK}: $n$ \begin{CJK}{UTF8}{mj}元实二次型\end{CJK} $q(X)=X^{T} A X$ \begin{CJK}{UTF8}{mj}是半正定的当且仅当矩阵\end{CJK} $A$ \begin{CJK}{UTF8}{mj}的任意主子式非负\end{CJK}.

  \item (10 \begin{CJK}{UTF8}{mj}分\end{CJK}) \begin{CJK}{UTF8}{mj}设\end{CJK} $\sigma$ \begin{CJK}{UTF8}{mj}是\end{CJK} $n$ \begin{CJK}{UTF8}{mj}维欧式空间\end{CJK} $V$ \begin{CJK}{UTF8}{mj}上的一个对称变换\end{CJK}. \begin{CJK}{UTF8}{mj}证明\end{CJK}: $\sigma$ \begin{CJK}{UTF8}{mj}的核\end{CJK} $\operatorname{ker} \sigma$ \begin{CJK}{UTF8}{mj}的正交补等于\end{CJK} $\sigma$ \begin{CJK}{UTF8}{mj}的象\end{CJK} $\operatorname{Im} \sigma$.

  \item ( 20 \begin{CJK}{UTF8}{mj}分\end{CJK}) \begin{CJK}{UTF8}{mj}设实矩阵\end{CJK}

\end{enumerate}
$$
A=\left(\begin{array}{lll}
4 & 2 & 2 \\
2 & 4 & 2 \\
2 & 2 & 4
\end{array}\right)
$$
(1) \begin{CJK}{UTF8}{mj}求正交矩阵\end{CJK} $P$, \begin{CJK}{UTF8}{mj}使\end{CJK} $P^{-1} A P$ \begin{CJK}{UTF8}{mj}为对角矩阵\end{CJK};

(2) \begin{CJK}{UTF8}{mj}求正交矩阵\end{CJK} $Q$ \begin{CJK}{UTF8}{mj}及上三角矩阵\end{CJK} $R$ \begin{CJK}{UTF8}{mj}使\end{CJK} $A=Q R$.

\begin{CJK}{UTF8}{mj}注\end{CJK}: \begin{CJK}{UTF8}{mj}试卷中\end{CJK} $\mathbb{F}$ \begin{CJK}{UTF8}{mj}为一数域\end{CJK}, $\mathbb{R}$ \begin{CJK}{UTF8}{mj}表示实数域\end{CJK}, $I$ \begin{CJK}{UTF8}{mj}表示单位矩阵\end{CJK}.

\section{2. 中山大学 2010 年研究生入学考试试题高等代数 
 李扬 
 微信公众号: sxkyliyang}
\begin{CJK}{UTF8}{mj}一\end{CJK}. \begin{CJK}{UTF8}{mj}填空题\end{CJK} (\begin{CJK}{UTF8}{mj}每小题\end{CJK} 10 \begin{CJK}{UTF8}{mj}分\end{CJK}. \begin{CJK}{UTF8}{mj}只写答案\end{CJK}, \begin{CJK}{UTF8}{mj}不写计算过程\end{CJK}. ) \begin{CJK}{UTF8}{mj}请把答案按顺序写在答题纸上\end{CJK}

\begin{enumerate}
  \item \begin{CJK}{UTF8}{mj}设\end{CJK} $U=\left\{A \in M_{2}(\mathbb{F}): a_{11}+a_{12}=0\right\}, V=\left\{A \in M_{2}(\mathbb{F}): a_{11}+a_{21}=0\right\}$, \begin{CJK}{UTF8}{mj}则\end{CJK} $U+V$ \begin{CJK}{UTF8}{mj}的维数等于\end{CJK}
\end{enumerate}
$\left(M_{2}(\mathbb{F})\right.$ \begin{CJK}{UTF8}{mj}表示数域\end{CJK} $\mathbb{F}$ \begin{CJK}{UTF8}{mj}上所有\end{CJK} 2 \begin{CJK}{UTF8}{mj}阶方阵构成的\end{CJK} $\mathbb{F}$ \begin{CJK}{UTF8}{mj}上线性空间\end{CJK}.)

\begin{enumerate}
  \setcounter{enumi}{2}
  \item \begin{CJK}{UTF8}{mj}设\end{CJK} $e_{1}=(1,0,2), e_{2}=(1,2,1), e_{3}=(0,2,1) \in \mathbb{R}^{3},\left(f_{1}, f_{2}, f_{3}\right)$ \begin{CJK}{UTF8}{mj}与\end{CJK} $\left(e_{1}, e_{2}, e_{3}\right)$ \begin{CJK}{UTF8}{mj}互为对偶基\end{CJK}, \begin{CJK}{UTF8}{mj}则对于\end{CJK} $x=$ $\left(x_{1}, x_{2}, x_{3}\right) \in \mathbb{R}^{3}$, \begin{CJK}{UTF8}{mj}有\end{CJK} $f_{1}(x)=$ ,$f_{2}(x)=$ ,$f_{3}(x)=$

  \item \begin{CJK}{UTF8}{mj}设\end{CJK} $A=\left(a_{i j}\right)_{n \times n}$ \begin{CJK}{UTF8}{mj}的所有对角元都等于\end{CJK} 2 , \begin{CJK}{UTF8}{mj}当\end{CJK} $|i-j|=1$ \begin{CJK}{UTF8}{mj}时\end{CJK}, $a_{i j}=-1$, \begin{CJK}{UTF8}{mj}其他元都是\end{CJK} 0 , \begin{CJK}{UTF8}{mj}则\end{CJK} $A$ \begin{CJK}{UTF8}{mj}的行列式\end{CJK} $\operatorname{det} A$ \begin{CJK}{UTF8}{mj}等\end{CJK} \begin{CJK}{UTF8}{mj}于\end{CJK}

  \item \begin{CJK}{UTF8}{mj}设\end{CJK} $f(x)$ \begin{CJK}{UTF8}{mj}是数域\end{CJK} $\mathbb{F}$ \begin{CJK}{UTF8}{mj}上的\end{CJK} $n$ \begin{CJK}{UTF8}{mj}次多项式\end{CJK}, \begin{CJK}{UTF8}{mj}令\end{CJK} $(f)=\{g(x): g \in \mathbb{F}[x], f \mid g\}$, \begin{CJK}{UTF8}{mj}则商空间\end{CJK} $\mathbb{F}[x] /(f)$ \begin{CJK}{UTF8}{mj}的维数等于\end{CJK}

  \item \begin{CJK}{UTF8}{mj}已知线性变换\end{CJK} $\sigma: \mathbb{R}^{3} \rightarrow \mathbb{R}^{3}, \sigma(x, y, z)=(x+2 y+2 z, 2 x+y+2 z, 2 x+2 y+z)$, \begin{CJK}{UTF8}{mj}则\end{CJK} $\sigma$ \begin{CJK}{UTF8}{mj}的特征值为\end{CJK} \begin{CJK}{UTF8}{mj}对应的特征向量为\end{CJK}

  \item \begin{CJK}{UTF8}{mj}设\end{CJK} $A=\left(\begin{array}{ccc}3 & 0 & 8 \\ 3 & -1 & 6 \\ -2 & 0 & -5\end{array}\right)$, \begin{CJK}{UTF8}{mj}则\end{CJK} $A$ \begin{CJK}{UTF8}{mj}的若当标准形为\end{CJK}

  \item \begin{CJK}{UTF8}{mj}实二次型\end{CJK} $q\left(x_{1}, x_{2}, x_{3}\right)=2 x_{1} x_{2}-6 x_{2} x_{3}+2 x_{1} x_{3}$ \begin{CJK}{UTF8}{mj}的符号差等于\end{CJK}

  \item \begin{CJK}{UTF8}{mj}设\end{CJK} $f(x)=x^{4}+2 x^{3}-x^{2}-4 x-2, g(x)=x^{4}+x^{3}-x^{2}-2 x-2$, \begin{CJK}{UTF8}{mj}则它们的首一最大公因式\end{CJK} $(f, g)=$

  \item \begin{CJK}{UTF8}{mj}设\end{CJK} $x=(1,2,2,3), y=(3,1,5,1) \in \mathbb{R}^{4}$, \begin{CJK}{UTF8}{mj}则\end{CJK} $x$ \begin{CJK}{UTF8}{mj}与\end{CJK} $y$ \begin{CJK}{UTF8}{mj}的夹角\end{CJK} $\angle(x, y)=$

  \item \begin{CJK}{UTF8}{mj}设\end{CJK} $W=\{(x, y, z): x+y-2 z=0\} \subseteq \mathbb{R}^{3}$, \begin{CJK}{UTF8}{mj}则\end{CJK} $W$ \begin{CJK}{UTF8}{mj}的正交补\end{CJK} $W^{\perp}=$

\end{enumerate}
\begin{CJK}{UTF8}{mj}二\end{CJK}. \begin{CJK}{UTF8}{mj}证明题\end{CJK} (\begin{CJK}{UTF8}{mj}每小题\end{CJK} 10 \begin{CJK}{UTF8}{mj}分\end{CJK}. \begin{CJK}{UTF8}{mj}写出详细步骤\end{CJK}. )

\begin{enumerate}
  \item \begin{CJK}{UTF8}{mj}设\end{CJK} $A$ \begin{CJK}{UTF8}{mj}为数域\end{CJK} $\mathbb{F}$ \begin{CJK}{UTF8}{mj}上\end{CJK} $m \times n$ \begin{CJK}{UTF8}{mj}矩阵\end{CJK}, \begin{CJK}{UTF8}{mj}定义\end{CJK}
\end{enumerate}
$$
L_{A}: \mathbb{F}^{n} \rightarrow \mathbb{F}^{m}, x \mapsto A x .
$$
\begin{CJK}{UTF8}{mj}证明\end{CJK}: $L_{A}$ \begin{CJK}{UTF8}{mj}是单射当且仅当\end{CJK} $A$ \begin{CJK}{UTF8}{mj}的列向量组线性无关\end{CJK}; $L_{A}$ \begin{CJK}{UTF8}{mj}是满射当且仅当\end{CJK} $A$ \begin{CJK}{UTF8}{mj}的行向量组线性无关\end{CJK}.

\begin{enumerate}
  \setcounter{enumi}{2}
  \item \begin{CJK}{UTF8}{mj}设\end{CJK} $f(x), g(x)$ \begin{CJK}{UTF8}{mj}是数域\end{CJK} $\mathbb{F}$ \begin{CJK}{UTF8}{mj}上的多项式\end{CJK}, $m(x)=[f, g]$ \begin{CJK}{UTF8}{mj}是它们的首一最小公倍式\end{CJK}, $\sigma$ \begin{CJK}{UTF8}{mj}是\end{CJK} $\mathbb{F}$ \begin{CJK}{UTF8}{mj}上线性空间\end{CJK} $V$ \begin{CJK}{UTF8}{mj}的一个线\end{CJK} \begin{CJK}{UTF8}{mj}性变换\end{CJK}. \begin{CJK}{UTF8}{mj}证明\end{CJK}:
\end{enumerate}
$$
\operatorname{ker} f(\sigma)+\operatorname{ker} g(\sigma)=\operatorname{ker} m(\sigma)
$$

\begin{enumerate}
  \setcounter{enumi}{3}
  \item \begin{CJK}{UTF8}{mj}设\end{CJK} $\sigma$ \begin{CJK}{UTF8}{mj}是复线性空间\end{CJK} $V$ \begin{CJK}{UTF8}{mj}的一个线性变换\end{CJK}. \begin{CJK}{UTF8}{mj}证明\end{CJK}: $\sigma$ \begin{CJK}{UTF8}{mj}相似于对角矩阵当且仅当对任意\end{CJK} $\sigma$ \begin{CJK}{UTF8}{mj}子空间\end{CJK} $U$ \begin{CJK}{UTF8}{mj}都有\end{CJK} $\sigma$ \begin{CJK}{UTF8}{mj}子空间\end{CJK} $U^{\prime}$ \begin{CJK}{UTF8}{mj}使得\end{CJK} $V=U \oplus U^{\prime}$.

  \item \begin{CJK}{UTF8}{mj}设\end{CJK} $A, B$ \begin{CJK}{UTF8}{mj}为\end{CJK} $n$ \begin{CJK}{UTF8}{mj}阶实对称矩阵\end{CJK}, \begin{CJK}{UTF8}{mj}且\end{CJK} $B$ \begin{CJK}{UTF8}{mj}是正定矩阵\end{CJK}. \begin{CJK}{UTF8}{mj}证明\end{CJK}: \begin{CJK}{UTF8}{mj}存在实可逆矩阵\end{CJK} $C$ \begin{CJK}{UTF8}{mj}使得\end{CJK} $C^{T} A C$ \begin{CJK}{UTF8}{mj}和\end{CJK} $C^{T} B C$ \begin{CJK}{UTF8}{mj}都是实对角\end{CJK} \begin{CJK}{UTF8}{mj}矩阵\end{CJK}. ( $C^{T}$ \begin{CJK}{UTF8}{mj}表示\end{CJK} $C$ \begin{CJK}{UTF8}{mj}的转置\end{CJK})

  \item \begin{CJK}{UTF8}{mj}设\end{CJK} $\sigma$ \begin{CJK}{UTF8}{mj}是\end{CJK} $n$ \begin{CJK}{UTF8}{mj}维欧式空间\end{CJK} $V$ \begin{CJK}{UTF8}{mj}的一个正规变换\end{CJK}, \begin{CJK}{UTF8}{mj}且满足条件\end{CJK}: $\sigma^{2}+\mathrm{id}_{V}=0$. \begin{CJK}{UTF8}{mj}证明\end{CJK}: \begin{CJK}{UTF8}{mj}对任意\end{CJK} $x \in V$, \begin{CJK}{UTF8}{mj}有\end{CJK} $|x|=|\sigma(x)|=$ $\left|\sigma^{*}(x)\right|$. ( $\sigma^{*}$ \begin{CJK}{UTF8}{mj}表示\end{CJK} $\sigma$ \begin{CJK}{UTF8}{mj}的伴随变换\end{CJK}, $|x|$ \begin{CJK}{UTF8}{mj}表示\end{CJK} $x$ \begin{CJK}{UTF8}{mj}的长度\end{CJK}. )

\end{enumerate}
\section{3. 中山大学 2011 年研究生入学考试试题高等代数 
 李扬 
 微信公众号: sxkyliyang}
\begin{CJK}{UTF8}{mj}注\end{CJK}: $\mathbb{R}$ \begin{CJK}{UTF8}{mj}与\end{CJK} $\mathbb{C}$ \begin{CJK}{UTF8}{mj}分别表示实数域和复数域\end{CJK}.

\begin{CJK}{UTF8}{mj}一\end{CJK}. \begin{CJK}{UTF8}{mj}只写答案及主要计算过程\end{CJK} (\begin{CJK}{UTF8}{mj}每小题\end{CJK} 15 \begin{CJK}{UTF8}{mj}分\end{CJK}, \begin{CJK}{UTF8}{mj}共\end{CJK} 90 \begin{CJK}{UTF8}{mj}分\end{CJK})

\begin{enumerate}
  \item \begin{CJK}{UTF8}{mj}设\end{CJK}
\end{enumerate}
$$
A=\left(\begin{array}{ccc}
1 & 1 & -1 \\
0 & 2 & 2 \\
1 & -1 & 0
\end{array}\right), B=\left(\begin{array}{ccc}
1 & -1 & 1 \\
1 & 1 & 0 \\
2 & 1 & 1
\end{array}\right)
$$
\begin{CJK}{UTF8}{mj}求矩阵\end{CJK} $X$ \begin{CJK}{UTF8}{mj}使得\end{CJK} $A X=B$.

\begin{enumerate}
  \setcounter{enumi}{2}
  \item \begin{CJK}{UTF8}{mj}设\end{CJK} $A=\left(a_{i j}\right)_{n \times n}$ \begin{CJK}{UTF8}{mj}的对角元为\end{CJK} 2 , \begin{CJK}{UTF8}{mj}当\end{CJK} $|i-j|=1$ \begin{CJK}{UTF8}{mj}时\end{CJK}, $a_{i j}=-1$, \begin{CJK}{UTF8}{mj}其他元为\end{CJK} 0 . \begin{CJK}{UTF8}{mj}求\end{CJK} $A^{-1}$.

  \item \begin{CJK}{UTF8}{mj}设\end{CJK}

\end{enumerate}
$$
\begin{gathered}
f(x)=x^{3}+x^{2}+x+1 \\
g(x)=x^{3}+2 x^{2}+3 x+4 .
\end{gathered}
$$
$V$ \begin{CJK}{UTF8}{mj}是数域\end{CJK} $\mathbb{F}$ \begin{CJK}{UTF8}{mj}上的次数小于\end{CJK} 4 \begin{CJK}{UTF8}{mj}的多项式组成的线性空间\end{CJK}, \begin{CJK}{UTF8}{mj}令\end{CJK} $U$ \begin{CJK}{UTF8}{mj}为由\end{CJK} $\{f, g\}$ \begin{CJK}{UTF8}{mj}生成的子空间\end{CJK}. \begin{CJK}{UTF8}{mj}求商空间\end{CJK} $V / U$ \begin{CJK}{UTF8}{mj}的一\end{CJK} \begin{CJK}{UTF8}{mj}组基\end{CJK}.

\begin{enumerate}
  \setcounter{enumi}{4}
  \item \begin{CJK}{UTF8}{mj}已知线性变换\end{CJK}
\end{enumerate}
$$
\sigma: \mathbb{R}^{2} \rightarrow \mathbb{R}^{2}, \sigma(x, y)=(2 x+y, x+2 y)
$$
\begin{CJK}{UTF8}{mj}求\end{CJK} $\sigma$ \begin{CJK}{UTF8}{mj}的特征值和特征向量\end{CJK}.

\begin{enumerate}
  \setcounter{enumi}{5}
  \item \begin{CJK}{UTF8}{mj}设\end{CJK}
\end{enumerate}
$$
\begin{gathered}
f(x)=4 x^{4}-2 x^{3}-16 x^{2}+5 x+9 ; \\
g(x)=2 x^{3}-x^{2}-5 x+4 .
\end{gathered}
$$
\begin{CJK}{UTF8}{mj}求\end{CJK} $f$ \begin{CJK}{UTF8}{mj}与\end{CJK} $g$ \begin{CJK}{UTF8}{mj}的首一最大公因式\end{CJK} $(f, g)$.

\begin{enumerate}
  \setcounter{enumi}{6}
  \item \begin{CJK}{UTF8}{mj}设\end{CJK} $x=(1,-1,0), y=(1,0,-1), z=(4,2,0) \in \mathbb{R}^{3}$. \begin{CJK}{UTF8}{mj}求\end{CJK} $z$ \begin{CJK}{UTF8}{mj}到由\end{CJK} $\{x, y\}$ \begin{CJK}{UTF8}{mj}生成的子空间\end{CJK} $U$ \begin{CJK}{UTF8}{mj}的距离\end{CJK}.
\end{enumerate}
\begin{CJK}{UTF8}{mj}二\end{CJK}. \begin{CJK}{UTF8}{mj}要求写出详细步骤\end{CJK}.

\begin{enumerate}
  \item (10 \begin{CJK}{UTF8}{mj}分\end{CJK}) \begin{CJK}{UTF8}{mj}设\end{CJK} $f(x)=(x-3)^{2}, g(x)=x-1$ \begin{CJK}{UTF8}{mj}是\end{CJK} $\mathbb{R}$ \begin{CJK}{UTF8}{mj}上的两个多项式\end{CJK}. \begin{CJK}{UTF8}{mj}定义\end{CJK} $\mathbb{R}$ \begin{CJK}{UTF8}{mj}上线性空间\end{CJK} $\mathbb{R}^{3}$ \begin{CJK}{UTF8}{mj}的线性变换\end{CJK} $\sigma$ \begin{CJK}{UTF8}{mj}如下\end{CJK}: $\sigma: \mathbb{R}^{3} \rightarrow \mathbb{R}^{3}, \sigma(x, y, z)=(2 x+y, x+2 y, 3 z)$. \begin{CJK}{UTF8}{mj}证明\end{CJK}:
\end{enumerate}
$$
\mathbb{R}^{3}=\operatorname{ker} f(\sigma) \oplus \operatorname{ker} g(\sigma) .
$$

\begin{enumerate}
  \setcounter{enumi}{2}
  \item (10 \begin{CJK}{UTF8}{mj}分\end{CJK}) \begin{CJK}{UTF8}{mj}设\end{CJK} $A, B \in M_{n}(\mathbb{R})$. \begin{CJK}{UTF8}{mj}证明\end{CJK}: $A$ \begin{CJK}{UTF8}{mj}与\end{CJK} $B$ \begin{CJK}{UTF8}{mj}在\end{CJK} $\mathbb{R}$ \begin{CJK}{UTF8}{mj}上相似当且仅当\end{CJK} $A$ \begin{CJK}{UTF8}{mj}与\end{CJK} $B$ \begin{CJK}{UTF8}{mj}在\end{CJK} $\mathbb{C}$ \begin{CJK}{UTF8}{mj}上相似\end{CJK}.

  \item ( 10 \begin{CJK}{UTF8}{mj}分\end{CJK}) \begin{CJK}{UTF8}{mj}设\end{CJK} $n$ \begin{CJK}{UTF8}{mj}元实二次型\end{CJK} $q(X)=X^{T} A X$ \begin{CJK}{UTF8}{mj}满足条件\end{CJK}: $q(X)=0$ \begin{CJK}{UTF8}{mj}当且仅当\end{CJK} $X=0$. \begin{CJK}{UTF8}{mj}证明\end{CJK}: $q(X)$ \begin{CJK}{UTF8}{mj}是正定的或者是\end{CJK} \begin{CJK}{UTF8}{mj}负定的\end{CJK}.

  \item ( 12 \begin{CJK}{UTF8}{mj}分\end{CJK}) \begin{CJK}{UTF8}{mj}设\end{CJK} $V$ \begin{CJK}{UTF8}{mj}为数域\end{CJK} $\mathbb{F}$ \begin{CJK}{UTF8}{mj}上线性空间\end{CJK}, $S$ \begin{CJK}{UTF8}{mj}和\end{CJK} $T$ \begin{CJK}{UTF8}{mj}是\end{CJK} $V$ \begin{CJK}{UTF8}{mj}的子空间\end{CJK}, $f$ \begin{CJK}{UTF8}{mj}是\end{CJK} $V$ \begin{CJK}{UTF8}{mj}上的线性变换\end{CJK}. $V^{*}$ \begin{CJK}{UTF8}{mj}表示\end{CJK} $V$ \begin{CJK}{UTF8}{mj}的对偶空间\end{CJK}, $S^{0}$ \begin{CJK}{UTF8}{mj}表示\end{CJK} $S$ \begin{CJK}{UTF8}{mj}的零化子\end{CJK}, \begin{CJK}{UTF8}{mj}即\end{CJK} $S^{0}=\left\{f \in V^{*}: f(S)=0\right\}$, \begin{CJK}{UTF8}{mj}而\end{CJK} $f^{*}$ \begin{CJK}{UTF8}{mj}表示\end{CJK} $f$ \begin{CJK}{UTF8}{mj}的转置\end{CJK}, \begin{CJK}{UTF8}{mj}即\end{CJK} $f^{*}: V^{*} \rightarrow V^{*}, g \mapsto g f, \forall g \in V^{*}$. \begin{CJK}{UTF8}{mj}证明\end{CJK}:

\end{enumerate}
(1) $(S \cap T)^{0}=S^{0}+T^{0}$;

(2) $\operatorname{Im} f^{*}=(\operatorname{ker} f)^{0}$. 5. (18 \begin{CJK}{UTF8}{mj}分\end{CJK}) \begin{CJK}{UTF8}{mj}设\end{CJK}
$$
A=\left(\begin{array}{ccc}
1 & 1 & 1 \\
-1 & 0 & 1 \\
0 & -1 & 1
\end{array}\right)
$$
(1) \begin{CJK}{UTF8}{mj}求正交矩阵\end{CJK} $Q$ \begin{CJK}{UTF8}{mj}及主对角元大于零的上三角矩阵\end{CJK} $T$ \begin{CJK}{UTF8}{mj}使得\end{CJK} $A=Q T$;\\
(2) \begin{CJK}{UTF8}{mj}求正定矩阵\end{CJK} $P$ \begin{CJK}{UTF8}{mj}及正交矩阵\end{CJK} $O$ \begin{CJK}{UTF8}{mj}使得\end{CJK} $A=P O$;\\
(3) \begin{CJK}{UTF8}{mj}求正交矩阵\end{CJK} $U$ \begin{CJK}{UTF8}{mj}及正交矩阵\end{CJK} $V$ \begin{CJK}{UTF8}{mj}使得\end{CJK} $U A V$ \begin{CJK}{UTF8}{mj}为对角矩阵\end{CJK}.

\section{4. 中山大学 2012 年研究生入学考试试题高等代数 
 李扬 
 微信公众号: sxkyliyang}
\begin{CJK}{UTF8}{mj}注\end{CJK}: \begin{CJK}{UTF8}{mj}这里\end{CJK} $\mathbb{R}$ \begin{CJK}{UTF8}{mj}表示实数域\end{CJK}, $I$ \begin{CJK}{UTF8}{mj}表示单位矩阵\end{CJK}, $\mathrm{id} V$ \begin{CJK}{UTF8}{mj}表示\end{CJK} $V$ \begin{CJK}{UTF8}{mj}的恒等变换\end{CJK}, \begin{CJK}{UTF8}{mj}矩阵\end{CJK} $C$ \begin{CJK}{UTF8}{mj}的转置记为\end{CJK} $C^{T}$.

\begin{CJK}{UTF8}{mj}一\end{CJK}. \begin{CJK}{UTF8}{mj}计算题\end{CJK} (\begin{CJK}{UTF8}{mj}每小题\end{CJK} 15 \begin{CJK}{UTF8}{mj}分\end{CJK}, \begin{CJK}{UTF8}{mj}共\end{CJK} 90 \begin{CJK}{UTF8}{mj}分\end{CJK})

\begin{enumerate}
  \item \begin{CJK}{UTF8}{mj}设\end{CJK} $a=\left(a_{1}, a_{2}, \cdots, a_{n}\right), b=\left(b_{1}, b_{2}, \cdots, b_{n}\right) \in \mathbb{R}^{n}$, \begin{CJK}{UTF8}{mj}计算矩阵\end{CJK} $A=I-a^{T} b$ \begin{CJK}{UTF8}{mj}的行列式\end{CJK}.

  \item \begin{CJK}{UTF8}{mj}设\end{CJK} $n$ \begin{CJK}{UTF8}{mj}阶实矩阵\end{CJK} $A$ \begin{CJK}{UTF8}{mj}的主对角元都为\end{CJK} 0 , \begin{CJK}{UTF8}{mj}其余元都为\end{CJK} 1 . \begin{CJK}{UTF8}{mj}求\end{CJK} $A$ \begin{CJK}{UTF8}{mj}的特征值与特征向量\end{CJK}.

  \item \begin{CJK}{UTF8}{mj}设\end{CJK}

\end{enumerate}
$$
A=\left(\begin{array}{cccc}
2 & -1 & 0 & 1 \\
0 & 3 & -1 & 0 \\
0 & 1 & 1 & 0 \\
0 & -1 & 0 & 3
\end{array}\right)
$$
\begin{CJK}{UTF8}{mj}求\end{CJK} $A$ \begin{CJK}{UTF8}{mj}的若当标准形\end{CJK}.

\begin{enumerate}
  \setcounter{enumi}{4}
  \item \begin{CJK}{UTF8}{mj}设\end{CJK}
\end{enumerate}
$$
\begin{gathered}
f(x)=x^{6}-7 x^{4}+8 x^{3}-7 x+7 \\
g(x)=3 x^{5}-7 x^{3}+3 x^{2}-7
\end{gathered}
$$
\begin{CJK}{UTF8}{mj}求\end{CJK} $f$ \begin{CJK}{UTF8}{mj}与\end{CJK} $g$ \begin{CJK}{UTF8}{mj}的首一最大公因式\end{CJK} $(f, g)$.

\begin{enumerate}
  \setcounter{enumi}{5}
  \item \begin{CJK}{UTF8}{mj}设\end{CJK}
\end{enumerate}
$$
A=\left(\begin{array}{ccc}
3 & 1 & 1 \\
2 & 4 & 2 \\
-1 & -1 & 1
\end{array}\right)
$$
\begin{CJK}{UTF8}{mj}计算\end{CJK} $A^{10}$.

\begin{enumerate}
  \setcounter{enumi}{6}
  \item \begin{CJK}{UTF8}{mj}设\end{CJK}
\end{enumerate}
$$
A=\left(\begin{array}{lll}
3 & 0 & 0 \\
0 & 2 & 1 \\
0 & 1 & 2
\end{array}\right)
$$
\begin{CJK}{UTF8}{mj}求一正交矩阵\end{CJK} $Q$ \begin{CJK}{UTF8}{mj}使得\end{CJK} $Q^{-1} A Q$ \begin{CJK}{UTF8}{mj}为对角矩阵\end{CJK}, \begin{CJK}{UTF8}{mj}并求正定矩阵\end{CJK} $B$ \begin{CJK}{UTF8}{mj}使得\end{CJK} $A=B^{2}$.

\begin{CJK}{UTF8}{mj}二\end{CJK}. \begin{CJK}{UTF8}{mj}证明题\end{CJK} (\begin{CJK}{UTF8}{mj}每小题\end{CJK} 10 \begin{CJK}{UTF8}{mj}分\end{CJK}, \begin{CJK}{UTF8}{mj}共\end{CJK} 60 \begin{CJK}{UTF8}{mj}分\end{CJK})

\begin{enumerate}
  \item \begin{CJK}{UTF8}{mj}令\end{CJK} $S=\left\{\left(t, t^{2}, t^{3}\right): t \in \mathbb{R}\right\}$. \begin{CJK}{UTF8}{mj}证明\end{CJK}: $\operatorname{span}(S)=\mathbb{R}^{3}$.

  \item \begin{CJK}{UTF8}{mj}设\end{CJK} $A, B \in M_{n}(\mathbb{R}), A^{2}=A, B^{2}=B$, \begin{CJK}{UTF8}{mj}且\end{CJK} $I-(A+B)$ \begin{CJK}{UTF8}{mj}可逆\end{CJK}. \begin{CJK}{UTF8}{mj}证明\end{CJK}: $A$ \begin{CJK}{UTF8}{mj}与\end{CJK} $B$ \begin{CJK}{UTF8}{mj}的秩相等\end{CJK}.

  \item \begin{CJK}{UTF8}{mj}令\end{CJK}

\end{enumerate}
$$
S=\left\{A B-B A: A, B \in M_{n}(\mathbb{F})\right\} .
$$
\begin{CJK}{UTF8}{mj}证明\end{CJK}: $S$ \begin{CJK}{UTF8}{mj}张成的子空间\end{CJK} $W=\operatorname{span}(S)$ \begin{CJK}{UTF8}{mj}的维数等于\end{CJK} $n^{2}-1$, \begin{CJK}{UTF8}{mj}并且给出它的一个基\end{CJK}.

\begin{enumerate}
  \setcounter{enumi}{4}
  \item \begin{CJK}{UTF8}{mj}设\end{CJK} $V$ \begin{CJK}{UTF8}{mj}为数域\end{CJK} $\mathbb{F}$ \begin{CJK}{UTF8}{mj}上线性空间\end{CJK}, $W$ \begin{CJK}{UTF8}{mj}是\end{CJK} $V$ \begin{CJK}{UTF8}{mj}的子空间\end{CJK}. $V^{*}$ \begin{CJK}{UTF8}{mj}表示\end{CJK} $V$ \begin{CJK}{UTF8}{mj}的对偶空间\end{CJK}, $W^{0}$ \begin{CJK}{UTF8}{mj}表示\end{CJK} $W$ \begin{CJK}{UTF8}{mj}的零化子\end{CJK}, \begin{CJK}{UTF8}{mj}即\end{CJK} $W^{0}=\{f \in$ $\left.V^{*}: f(W)=0\right\}$. \begin{CJK}{UTF8}{mj}证明\end{CJK}: $W^{*} \cong V^{*} / W^{0}$.

  \item \begin{CJK}{UTF8}{mj}设\end{CJK} $q(X)=X^{T} A X$ \begin{CJK}{UTF8}{mj}为\end{CJK} $n$ \begin{CJK}{UTF8}{mj}元实二次型\end{CJK}. \begin{CJK}{UTF8}{mj}如果\end{CJK} $A$ \begin{CJK}{UTF8}{mj}的所有特征值都属于区间\end{CJK} $[a, b]$. \begin{CJK}{UTF8}{mj}证明\end{CJK}: $A-t I$ \begin{CJK}{UTF8}{mj}对应的二次型当\end{CJK} $t>b$ \begin{CJK}{UTF8}{mj}时是负定的\end{CJK}; \begin{CJK}{UTF8}{mj}当\end{CJK} $t<a$ \begin{CJK}{UTF8}{mj}时是正定的\end{CJK}.

  \item \begin{CJK}{UTF8}{mj}设\end{CJK} $\sigma$ \begin{CJK}{UTF8}{mj}是\end{CJK} $n$ \begin{CJK}{UTF8}{mj}维欧式空间\end{CJK} $V$ \begin{CJK}{UTF8}{mj}的一个正规变换\end{CJK}, \begin{CJK}{UTF8}{mj}且满足条件\end{CJK}: $\sigma^{2}=\mathrm{id}_{V}$. \begin{CJK}{UTF8}{mj}证明\end{CJK}: $\sigma$ \begin{CJK}{UTF8}{mj}既是对称变换\end{CJK}, \begin{CJK}{UTF8}{mj}也是正交变换\end{CJK}.

\end{enumerate}
\section{5. 中山大学 2013 年研究生入学考试试题高等代数 
 李扬 
 微信公众号: sxkyliyang}
\begin{enumerate}
  \item \begin{CJK}{UTF8}{mj}设\end{CJK} $E$ \begin{CJK}{UTF8}{mj}为数域\end{CJK}, $\mathbb{F} \subset E$, \begin{CJK}{UTF8}{mj}且\end{CJK} $E$ \begin{CJK}{UTF8}{mj}作为\end{CJK} $\mathbb{F}$ \begin{CJK}{UTF8}{mj}上的线性空间\end{CJK}, \begin{CJK}{UTF8}{mj}维数为\end{CJK} $m$. \begin{CJK}{UTF8}{mj}设\end{CJK} $V$ \begin{CJK}{UTF8}{mj}为\end{CJK} $E$ \begin{CJK}{UTF8}{mj}上的\end{CJK} $n$ \begin{CJK}{UTF8}{mj}维线性空间\end{CJK}. \begin{CJK}{UTF8}{mj}证明\end{CJK}: $V$ \begin{CJK}{UTF8}{mj}作为\end{CJK} $\mathbb{F}$ \begin{CJK}{UTF8}{mj}上的线性空间维数为\end{CJK} $m n$.

  \item \begin{CJK}{UTF8}{mj}设\end{CJK} $f$ \begin{CJK}{UTF8}{mj}是\end{CJK} $\mathbb{F}$ \begin{CJK}{UTF8}{mj}上线性空间\end{CJK} $M_{n}(\mathbb{F})$ \begin{CJK}{UTF8}{mj}到\end{CJK} $\mathbb{F}$ \begin{CJK}{UTF8}{mj}的线性映射\end{CJK}, $f(I)=n$ \begin{CJK}{UTF8}{mj}且对任意的矩阵\end{CJK} $A, B \in M_{n}(\mathbb{F})$ \begin{CJK}{UTF8}{mj}有\end{CJK} $f(A B)=f(B A)$. \begin{CJK}{UTF8}{mj}证明\end{CJK}: $f=\operatorname{tr}$. (\begin{CJK}{UTF8}{mj}注\end{CJK}: $\operatorname{tr}$ \begin{CJK}{UTF8}{mj}为迹函数\end{CJK}, $\operatorname{tr}(A)=\sum_{i=1}^{n} a_{i i}$.)

  \item \begin{CJK}{UTF8}{mj}设\end{CJK} $A, B \in M_{n}(\mathbb{F}), \operatorname{rank}(A)<n$, \begin{CJK}{UTF8}{mj}且\end{CJK} $A=B_{1} B_{2} \cdots B_{k}$, \begin{CJK}{UTF8}{mj}其中\end{CJK} $B_{i}{ }^{2}=B_{i}, i=1,2, \cdots, k$. \begin{CJK}{UTF8}{mj}证明\end{CJK}:

\end{enumerate}
$$
\operatorname{rank}(I-A) \leq k(n-\operatorname{rank}(A))
$$

\begin{enumerate}
  \setcounter{enumi}{4}
  \item \begin{CJK}{UTF8}{mj}设\end{CJK} $A \in \mathbb{F}^{m \times n}$. \begin{CJK}{UTF8}{mj}若对任意\end{CJK} $n$ \begin{CJK}{UTF8}{mj}维向量\end{CJK} $b \in \mathbb{F}^{n}$, \begin{CJK}{UTF8}{mj}线性方程组\end{CJK} $A X=b$ \begin{CJK}{UTF8}{mj}有解\end{CJK}. \begin{CJK}{UTF8}{mj}证明\end{CJK}:
\end{enumerate}
$$
\operatorname{rank}(A)=m
$$

\begin{enumerate}
  \setcounter{enumi}{5}
  \item \begin{CJK}{UTF8}{mj}设\end{CJK} $f(x)=x^{3}, g(x)=(1-x)^{2}$.
\end{enumerate}
(1) \begin{CJK}{UTF8}{mj}求\end{CJK} $u(x), v(x)$ \begin{CJK}{UTF8}{mj}使\end{CJK}
$$
(f(x), g(x))=u(x) f(x)+v(x) g(x) .
$$
(2) \begin{CJK}{UTF8}{mj}设\end{CJK} $r_{1}(x)=x+2, r_{2}(x)=1$. \begin{CJK}{UTF8}{mj}求一多项式\end{CJK} $h(x)$ \begin{CJK}{UTF8}{mj}使下列同余方程式成立\end{CJK}:
$$
h(x) \equiv r_{1}(x)(\bmod f(x)), h(x) \equiv r_{2}(x)(\bmod g(x))
$$

\begin{enumerate}
  \setcounter{enumi}{6}
  \item \begin{CJK}{UTF8}{mj}设\end{CJK} $\sigma$ \begin{CJK}{UTF8}{mj}是\end{CJK} $\mathbb{F}$ \begin{CJK}{UTF8}{mj}上线性空间\end{CJK} $V$ \begin{CJK}{UTF8}{mj}上的线性变换\end{CJK}. $W$ \begin{CJK}{UTF8}{mj}是\end{CJK} $\sigma$ \begin{CJK}{UTF8}{mj}的不变子空间\end{CJK}. $\lambda_{1}, \cdots, \lambda_{m}$ \begin{CJK}{UTF8}{mj}是\end{CJK} $\sigma$ \begin{CJK}{UTF8}{mj}的两两不同的特征根\end{CJK}, $\alpha_{1}, \cdots, \alpha_{m}$ \begin{CJK}{UTF8}{mj}分别是属于\end{CJK} $\lambda_{1}, \cdots, \lambda_{m}$ \begin{CJK}{UTF8}{mj}的根向量\end{CJK}. \begin{CJK}{UTF8}{mj}若\end{CJK} $\alpha=\alpha_{1}+\cdots+\alpha_{m} \in W$, \begin{CJK}{UTF8}{mj}证明\end{CJK}:
\end{enumerate}
$$
\alpha_{i} \in W, i=1, \cdots, m .
$$

\begin{enumerate}
  \setcounter{enumi}{7}
  \item \begin{CJK}{UTF8}{mj}设复矩阵\end{CJK}
\end{enumerate}
$$
A=\left(\begin{array}{cccc}
0 & -2 & 3 & 2 \\
1 & 1 & -1 & -1 \\
0 & 0 & 2 & 0 \\
1 & -1 & 0 & 1
\end{array}\right)
$$
\begin{CJK}{UTF8}{mj}求\end{CJK} $A$ \begin{CJK}{UTF8}{mj}的\end{CJK} Jordan \begin{CJK}{UTF8}{mj}标准型和最小多项式\end{CJK}.

\begin{enumerate}
  \setcounter{enumi}{8}
  \item \begin{CJK}{UTF8}{mj}设\end{CJK} $W$ \begin{CJK}{UTF8}{mj}为下列实线性方程组的解空间\end{CJK}. \begin{CJK}{UTF8}{mj}分别求\end{CJK} $W$ \begin{CJK}{UTF8}{mj}与\end{CJK} $W^{\perp}\left(W\right.$ \begin{CJK}{UTF8}{mj}的正交补\end{CJK}) \begin{CJK}{UTF8}{mj}的一个标准正交基\end{CJK}: $2 x_{1}+x_{2}-x_{3}+x_{4}=$ $0, x_{1}+x_{2}-x_{3}=0$.
\end{enumerate}
9 . \begin{CJK}{UTF8}{mj}设实矩阵\end{CJK}
$$
A=\left(\begin{array}{ccc}
3 & -2 & -4 \\
-2 & 6 & -2 \\
-4 & -2 & 3
\end{array}\right)
$$
\begin{CJK}{UTF8}{mj}求正交矩阵\end{CJK} $P$ \begin{CJK}{UTF8}{mj}使\end{CJK} $P^{-1} A P$ \begin{CJK}{UTF8}{mj}为对角矩阵\end{CJK}.

\begin{enumerate}
  \setcounter{enumi}{10}
  \item \begin{CJK}{UTF8}{mj}设\end{CJK} $A, B$ \begin{CJK}{UTF8}{mj}都是\end{CJK} $n$ \begin{CJK}{UTF8}{mj}阶实矩阵\end{CJK}, \begin{CJK}{UTF8}{mj}其中\end{CJK} $A$ \begin{CJK}{UTF8}{mj}正定\end{CJK}, $B$ \begin{CJK}{UTF8}{mj}半正定\end{CJK}. \begin{CJK}{UTF8}{mj}证明\end{CJK}:
\end{enumerate}
$$
\operatorname{det}(A+B) \geq \operatorname{det} A
$$

\section{6. 中山大学 2014 年研究生入学考试试题高等代数 
 李扬 
 微信公众号: sxkyliyang}
\begin{enumerate}
  \item ( 20 \begin{CJK}{UTF8}{mj}分\end{CJK}) \begin{CJK}{UTF8}{mj}设\end{CJK} $n$ \begin{CJK}{UTF8}{mj}阶实方阵\end{CJK} $A$ \begin{CJK}{UTF8}{mj}的主对角元为\end{CJK} 0 , \begin{CJK}{UTF8}{mj}其他元为\end{CJK} 1 .
\end{enumerate}
(1) \begin{CJK}{UTF8}{mj}求\end{CJK} $A$ \begin{CJK}{UTF8}{mj}的行列式\end{CJK} $\operatorname{det} A$ \begin{CJK}{UTF8}{mj}及\end{CJK} $A$ \begin{CJK}{UTF8}{mj}的逆\end{CJK} $A^{-1}$;

(2) \begin{CJK}{UTF8}{mj}求\end{CJK} $A$ \begin{CJK}{UTF8}{mj}的特征值\end{CJK}, \begin{CJK}{UTF8}{mj}特征向量及最小多项式\end{CJK}.

\begin{enumerate}
  \setcounter{enumi}{2}
  \item ( 20 \begin{CJK}{UTF8}{mj}分\end{CJK}) \begin{CJK}{UTF8}{mj}设数域\end{CJK} $\mathbb{F}$ \begin{CJK}{UTF8}{mj}上多项式\end{CJK} $f(x)=x^{4}+3 x^{3}-x^{2}-4 x-3, g(x)=3 x^{3}+10 x^{2}+2 x-3$.
\end{enumerate}
(1) \begin{CJK}{UTF8}{mj}求\end{CJK} $f$ \begin{CJK}{UTF8}{mj}与\end{CJK} $g$ \begin{CJK}{UTF8}{mj}的首一最大公因式\end{CJK} $d=(f, g)$;

(2) \begin{CJK}{UTF8}{mj}令\end{CJK} $U=\{u f+v g: u, v \in \mathbb{F}[x]\}$, \begin{CJK}{UTF8}{mj}求商空间\end{CJK} $\mathbb{F}[x] / U$ \begin{CJK}{UTF8}{mj}的维数\end{CJK}.

\begin{enumerate}
  \setcounter{enumi}{3}
  \item ( 20 \begin{CJK}{UTF8}{mj}分\end{CJK}) \begin{CJK}{UTF8}{mj}设\end{CJK} $A, A_{1}, \cdots, A_{k}$ \begin{CJK}{UTF8}{mj}都是数域\end{CJK} $\mathbb{F}$ \begin{CJK}{UTF8}{mj}上的\end{CJK} $n$ \begin{CJK}{UTF8}{mj}阶方阵\end{CJK}, \begin{CJK}{UTF8}{mj}且\end{CJK} $A_{1}+\cdots+A_{k}=I_{n}$. \begin{CJK}{UTF8}{mj}证明\end{CJK}:
\end{enumerate}
(1) \begin{CJK}{UTF8}{mj}若\end{CJK} $A^{2}=A$, \begin{CJK}{UTF8}{mj}则\end{CJK} $A$ \begin{CJK}{UTF8}{mj}的迹等于\end{CJK} $A$ \begin{CJK}{UTF8}{mj}的秩\end{CJK}, \begin{CJK}{UTF8}{mj}即\end{CJK} $\operatorname{tr}(A)=\operatorname{rank}(A)$;

(2) $A^{2}=A$ \begin{CJK}{UTF8}{mj}当且仅当\end{CJK} $\operatorname{rank}(A)+\operatorname{rank}\left(I_{n}-A\right)=n$;

(3) $A_{i}^{2}=A_{i}, i=1, \cdots, k$ \begin{CJK}{UTF8}{mj}当且仅当\end{CJK} $\operatorname{rank}\left(A_{1}\right)+\cdots+\operatorname{rank}\left(A_{k}\right)=n$.

\begin{enumerate}
  \setcounter{enumi}{4}
  \item (10 \begin{CJK}{UTF8}{mj}分\end{CJK}) \begin{CJK}{UTF8}{mj}设\end{CJK} $A, B, C$ \begin{CJK}{UTF8}{mj}是数域\end{CJK} $\mathbb{F}$ \begin{CJK}{UTF8}{mj}上的\end{CJK} 2 \begin{CJK}{UTF8}{mj}阶方阵\end{CJK}, \begin{CJK}{UTF8}{mj}记\end{CJK} $[A, B]=A B-B A$. \begin{CJK}{UTF8}{mj}证明\end{CJK}: $\left[[A, B]^{2}, C\right]=0$.

  \item ( 10 \begin{CJK}{UTF8}{mj}分\end{CJK}) \begin{CJK}{UTF8}{mj}设\end{CJK} $V$ \begin{CJK}{UTF8}{mj}是数域\end{CJK} $\mathbb{F}$ \begin{CJK}{UTF8}{mj}上的\end{CJK} $n$ \begin{CJK}{UTF8}{mj}维线性空间\end{CJK}, $\sigma$ \begin{CJK}{UTF8}{mj}和\end{CJK} $\tau$ \begin{CJK}{UTF8}{mj}都是\end{CJK} $V$ \begin{CJK}{UTF8}{mj}上的线性变换\end{CJK}, $V$ \begin{CJK}{UTF8}{mj}是\end{CJK} $\sigma$ \begin{CJK}{UTF8}{mj}循环子空间\end{CJK}, \begin{CJK}{UTF8}{mj}且\end{CJK} $\sigma \tau=\tau \sigma$. \begin{CJK}{UTF8}{mj}证明\end{CJK}: \begin{CJK}{UTF8}{mj}存在某个多项式\end{CJK} $f(x)$ \begin{CJK}{UTF8}{mj}使得\end{CJK} $\tau=f(\sigma)$.

  \item (10 \begin{CJK}{UTF8}{mj}分\end{CJK}) \begin{CJK}{UTF8}{mj}求复矩阵\end{CJK}

\end{enumerate}
$$
A=\left(\begin{array}{ccc}
13 & 16 & 14 \\
-6 & -7 & -6 \\
-6 & -8 & -7
\end{array}\right)
$$
\begin{CJK}{UTF8}{mj}的若当标准形\end{CJK}.

\begin{enumerate}
  \setcounter{enumi}{7}
  \item ( 10 \begin{CJK}{UTF8}{mj}分\end{CJK}) \begin{CJK}{UTF8}{mj}设\end{CJK} $q(X)=X^{T} A X$ \begin{CJK}{UTF8}{mj}为\end{CJK} $n$ \begin{CJK}{UTF8}{mj}元实二次型\end{CJK}, \begin{CJK}{UTF8}{mj}令\end{CJK} $V=\left\{X \in \mathbb{R}^{n}: q(X)=0\right\}$. \begin{CJK}{UTF8}{mj}证明\end{CJK}: \begin{CJK}{UTF8}{mj}二次型\end{CJK} $q(X)$ \begin{CJK}{UTF8}{mj}是半正定或\end{CJK} \begin{CJK}{UTF8}{mj}者半负定的充要条件为\end{CJK} $V$ \begin{CJK}{UTF8}{mj}是\end{CJK} $\mathbb{R}^{n}$ \begin{CJK}{UTF8}{mj}的子空间\end{CJK}, \begin{CJK}{UTF8}{mj}这里\end{CJK} $\mathbb{R}$ \begin{CJK}{UTF8}{mj}是实数域\end{CJK}.

  \item ( 10 \begin{CJK}{UTF8}{mj}分\end{CJK}) \begin{CJK}{UTF8}{mj}设\end{CJK} $V$ \begin{CJK}{UTF8}{mj}为数域\end{CJK} $\mathbb{F}$ \begin{CJK}{UTF8}{mj}上一个\end{CJK} $n$ \begin{CJK}{UTF8}{mj}维线性空间\end{CJK}, $f$ \begin{CJK}{UTF8}{mj}是\end{CJK} $V$ \begin{CJK}{UTF8}{mj}上的一个双线性函数\end{CJK}, \begin{CJK}{UTF8}{mj}令\end{CJK}

\end{enumerate}
$$
\begin{aligned}
&V_{1}=\{x \in V: f(x, y)=0, \forall y \in V\} \\
&V_{2}=\{y \in V: f(x, y)=0, \forall x \in V\}
\end{aligned}
$$
\begin{CJK}{UTF8}{mj}证明\end{CJK}: $V_{1}$ \begin{CJK}{UTF8}{mj}和\end{CJK} $V_{2}$ \begin{CJK}{UTF8}{mj}都是\end{CJK} $V$ \begin{CJK}{UTF8}{mj}的子空间\end{CJK}, \begin{CJK}{UTF8}{mj}且维数相等\end{CJK}, \begin{CJK}{UTF8}{mj}即\end{CJK} $\operatorname{dim} V_{1}=\operatorname{dim} V_{2}$.

\begin{enumerate}
  \setcounter{enumi}{9}
  \item ( 20 \begin{CJK}{UTF8}{mj}分\end{CJK}) \begin{CJK}{UTF8}{mj}给定\end{CJK} 4 \begin{CJK}{UTF8}{mj}维标准欧式空间\end{CJK} $\mathbb{R}^{4}$ \begin{CJK}{UTF8}{mj}的一个基\end{CJK} $\left(e_{1}, e_{2}, e_{3}, e_{4}\right)$, \begin{CJK}{UTF8}{mj}以此基作为列向量组的矩阵记为\end{CJK} $A$, \begin{CJK}{UTF8}{mj}其中\end{CJK} $e_{1}=(1,1,0,0), e_{2}=(1,0,1,0), e_{3}=(-1,0,0,1), e_{4}=(1,-1,-1,1) .$
\end{enumerate}
(1) \begin{CJK}{UTF8}{mj}用正交化方法求\end{CJK} $\mathbb{R}^{4}$ \begin{CJK}{UTF8}{mj}的一个标准正交基\end{CJK} $\left(e_{1}^{\prime}, e_{2}^{\prime}, e_{3}^{\prime}, e_{4}^{\prime}\right)$;

(2) \begin{CJK}{UTF8}{mj}求正交矩阵\end{CJK} $Q$ \begin{CJK}{UTF8}{mj}及主对角元大于零的上三角矩阵\end{CJK} $T$ \begin{CJK}{UTF8}{mj}使得\end{CJK} $A=Q T$.

\begin{enumerate}
  \setcounter{enumi}{10}
  \item ( 20 \begin{CJK}{UTF8}{mj}分\end{CJK}) \begin{CJK}{UTF8}{mj}设实矩阵\end{CJK}
\end{enumerate}
$$
A=\left(\begin{array}{ccc}
1 & 1 & 1 \\
-1 & 0 & 1 \\
0 & -1 & 1
\end{array}\right)
$$
$A^{T}$ \begin{CJK}{UTF8}{mj}表示\end{CJK} $A$ \begin{CJK}{UTF8}{mj}的转置\end{CJK}.

(1) \begin{CJK}{UTF8}{mj}求正定矩阵\end{CJK} $B$ \begin{CJK}{UTF8}{mj}使得\end{CJK} $A A^{T}=B^{2}$;

(2) \begin{CJK}{UTF8}{mj}求正定矩阵\end{CJK} $C$ \begin{CJK}{UTF8}{mj}及正交矩阵\end{CJK} $D$ \begin{CJK}{UTF8}{mj}使得\end{CJK} $A=C D$;

(3) \begin{CJK}{UTF8}{mj}求正交矩阵\end{CJK} $P$ \begin{CJK}{UTF8}{mj}及正交矩阵\end{CJK} $Q$ \begin{CJK}{UTF8}{mj}使得\end{CJK} $P A Q$ \begin{CJK}{UTF8}{mj}为对角矩阵\end{CJK}.

\section{7. 中山大学 2015 年研究生入学考试试题高等代数}
\begin{CJK}{UTF8}{mj}李扬\end{CJK}

\begin{CJK}{UTF8}{mj}微信公众号\end{CJK}: sxkyliyang

\begin{CJK}{UTF8}{mj}符号说明\end{CJK}: \begin{CJK}{UTF8}{mj}试卷中\end{CJK} $\mathbb{R}$ \begin{CJK}{UTF8}{mj}表示实数域\end{CJK}, $\mathbb{C}$ \begin{CJK}{UTF8}{mj}表示复数域\end{CJK}

\begin{enumerate}
  \item ( 20 \begin{CJK}{UTF8}{mj}分\end{CJK} $)$ \begin{CJK}{UTF8}{mj}求下列\end{CJK} $n$ \begin{CJK}{UTF8}{mj}阶实矩阵的行列式\end{CJK}:
\end{enumerate}
(1) $A=\left(a_{i j}\right)$, \begin{CJK}{UTF8}{mj}其中\end{CJK} $a_{i j}= \begin{cases}1, & i \neq j, \\ 2, & i=j ; \\ 0, & \text { 其他. }\end{cases}$

(2) $B=\left(b_{i j}\right)$, \begin{CJK}{UTF8}{mj}其中\end{CJK} $b_{i j}=f_{j}\left(a_{i}\right), f_{j}(x)$ \begin{CJK}{UTF8}{mj}为首一的\end{CJK} $j-1$ \begin{CJK}{UTF8}{mj}次实系数多项式\end{CJK}, $a_{1}, \cdots, a_{n}$ \begin{CJK}{UTF8}{mj}为两两不同的实数\end{CJK}.

\begin{enumerate}
  \setcounter{enumi}{2}
  \item ( 20 \begin{CJK}{UTF8}{mj}分\end{CJK}) \begin{CJK}{UTF8}{mj}已知实多项式\end{CJK}
\end{enumerate}
$$
\begin{gathered}
f(x)=x^{4}+2 x^{3}-x^{2}-4 x-2 \\
g(x)=x^{4}+x^{3}-x^{2}-2 x-2
\end{gathered}
$$
(1) \begin{CJK}{UTF8}{mj}求\end{CJK} $f(x)$ \begin{CJK}{UTF8}{mj}的全部有理根及相应的重数\end{CJK};

(2) \begin{CJK}{UTF8}{mj}求\end{CJK} $f(x)$ \begin{CJK}{UTF8}{mj}与\end{CJK} $g(x)$ \begin{CJK}{UTF8}{mj}的首一的最大公因式\end{CJK} $(f, g)$.

\begin{enumerate}
  \setcounter{enumi}{3}
  \item ( 20 \begin{CJK}{UTF8}{mj}分\end{CJK}) \begin{CJK}{UTF8}{mj}设\end{CJK} 3 \begin{CJK}{UTF8}{mj}阶复矩阵\end{CJK}
\end{enumerate}
$$
A=\left(\begin{array}{ccc}
2 & 3 & 2 \\
1 & 8 & 2 \\
-2 & -14 & -3
\end{array}\right)
$$
\begin{CJK}{UTF8}{mj}定义\end{CJK} $\mathbb{C}^{3}$ \begin{CJK}{UTF8}{mj}上的线性变换\end{CJK} $\sigma$ \begin{CJK}{UTF8}{mj}为\end{CJK}: $\sigma(\alpha)=A \alpha$, \begin{CJK}{UTF8}{mj}对任意的\end{CJK} $\alpha \in \mathbb{C}^{3}$. \begin{CJK}{UTF8}{mj}求\end{CJK} $\sigma$ \begin{CJK}{UTF8}{mj}的最小多项式以及\end{CJK} Jordan \begin{CJK}{UTF8}{mj}标准形\end{CJK}.

\begin{enumerate}
  \setcounter{enumi}{4}
  \item ( 20 \begin{CJK}{UTF8}{mj}分\end{CJK}) \begin{CJK}{UTF8}{mj}记\end{CJK} $\mathbb{R}[x]_{5}$ \begin{CJK}{UTF8}{mj}为次数小于\end{CJK} 5 \begin{CJK}{UTF8}{mj}的实多项式全体构成的向量空间\end{CJK}, \begin{CJK}{UTF8}{mj}在\end{CJK} $\mathbb{R}[x]_{5}$ \begin{CJK}{UTF8}{mj}上定义双线性函数如下\end{CJK}
\end{enumerate}
$$
(f(x), g(x))=\int_{-1}^{1} f(x) g(x) \mathrm{d} x .
$$
(1) \begin{CJK}{UTF8}{mj}证明\end{CJK}: \begin{CJK}{UTF8}{mj}上式定义了\end{CJK} $\mathbb{R}[x]_{5}$ \begin{CJK}{UTF8}{mj}上一个正定的对称双线性函数\end{CJK};

(2) \begin{CJK}{UTF8}{mj}用\end{CJK} Gram - Schmidt \begin{CJK}{UTF8}{mj}方法由\end{CJK} $1, x, x^{2}, x^{3}$ \begin{CJK}{UTF8}{mj}求\end{CJK} $\mathbb{R}[x]_{5}$ \begin{CJK}{UTF8}{mj}的一个正交向量组\end{CJK};

(3) \begin{CJK}{UTF8}{mj}求一个形如\end{CJK} $f(x)=a+b x^{2}-x^{4}$ \begin{CJK}{UTF8}{mj}的多项式\end{CJK}, \begin{CJK}{UTF8}{mj}使它与所有次数低于\end{CJK} 4 \begin{CJK}{UTF8}{mj}的实多项式正交\end{CJK}.

\begin{enumerate}
  \setcounter{enumi}{5}
  \item ( 20 \begin{CJK}{UTF8}{mj}分\end{CJK}) \begin{CJK}{UTF8}{mj}设\end{CJK} $A, B \in M_{n}(\mathbb{C})$ \begin{CJK}{UTF8}{mj}为幂等矩阵\end{CJK}, \begin{CJK}{UTF8}{mj}即\end{CJK} $A^{2}=A, B^{2}=B$.
\end{enumerate}
(1) \begin{CJK}{UTF8}{mj}证明\end{CJK}: $A-B$ \begin{CJK}{UTF8}{mj}为幂等矩阵当且仅当\end{CJK} $A B=B A=B$;

(2) \begin{CJK}{UTF8}{mj}证明\end{CJK}: \begin{CJK}{UTF8}{mj}若\end{CJK} $A B=B A$, \begin{CJK}{UTF8}{mj}则\end{CJK} $A B$ \begin{CJK}{UTF8}{mj}为幂等矩阵\end{CJK}. \begin{CJK}{UTF8}{mj}反之\end{CJK}, \begin{CJK}{UTF8}{mj}若\end{CJK} $A B$ \begin{CJK}{UTF8}{mj}为幂等矩阵\end{CJK}, \begin{CJK}{UTF8}{mj}是否必有\end{CJK} $A B=B A$ ? \begin{CJK}{UTF8}{mj}试证明或给出\end{CJK} \begin{CJK}{UTF8}{mj}反例\end{CJK}.

\begin{enumerate}
  \setcounter{enumi}{6}
  \item (10 \begin{CJK}{UTF8}{mj}分\end{CJK}) \begin{CJK}{UTF8}{mj}设\end{CJK} $A_{1}, \cdots, A_{m}$ \begin{CJK}{UTF8}{mj}为\end{CJK} $m$ \begin{CJK}{UTF8}{mj}个两两可换的互不相同的\end{CJK} $n$ \begin{CJK}{UTF8}{mj}阶实对称矩阵\end{CJK}, \begin{CJK}{UTF8}{mj}且\end{CJK} $\operatorname{tr}\left(A_{i} A_{j}\right)=\delta_{i j}, 1 \leq i, j \leq n$, \begin{CJK}{UTF8}{mj}这\end{CJK} \begin{CJK}{UTF8}{mj}里\end{CJK} $\operatorname{tr}(A)$ \begin{CJK}{UTF8}{mj}表示矩阵\end{CJK} $A$ \begin{CJK}{UTF8}{mj}的迹\end{CJK}, \begin{CJK}{UTF8}{mj}即它的对角元之和\end{CJK}, \begin{CJK}{UTF8}{mj}试证明\end{CJK} $m \leq n$.

  \item ( 20 \begin{CJK}{UTF8}{mj}分\end{CJK}) \begin{CJK}{UTF8}{mj}设\end{CJK} $n$ \begin{CJK}{UTF8}{mj}阶实矩阵\end{CJK} $A=\left(a_{i j}\right)$ \begin{CJK}{UTF8}{mj}半正定\end{CJK}.

\end{enumerate}
(1) \begin{CJK}{UTF8}{mj}证明\end{CJK}: \begin{CJK}{UTF8}{mj}存在\end{CJK} $\alpha_{i} \in \mathbb{R}^{n}, i=1, \cdots, n$, \begin{CJK}{UTF8}{mj}使得\end{CJK} $a_{i j}$ \begin{CJK}{UTF8}{mj}等于\end{CJK} $\alpha_{i}$ \begin{CJK}{UTF8}{mj}与\end{CJK} $\alpha_{j}$ \begin{CJK}{UTF8}{mj}的内积\end{CJK};

(2) \begin{CJK}{UTF8}{mj}证明\end{CJK}: $2 n$ \begin{CJK}{UTF8}{mj}阶矩阵\end{CJK} $\left(\begin{array}{cc}A & A \\ A & A\end{array}\right)$ \begin{CJK}{UTF8}{mj}半正定\end{CJK};

(3) \begin{CJK}{UTF8}{mj}若实矩阵\end{CJK} $B=\left(b_{i j}\right)$ \begin{CJK}{UTF8}{mj}也半正定\end{CJK}, \begin{CJK}{UTF8}{mj}令\end{CJK} $d_{i j}=a_{i j} b_{i j}$. \begin{CJK}{UTF8}{mj}证明\end{CJK}: \begin{CJK}{UTF8}{mj}矩阵\end{CJK} $D=\left(d_{i j}\right)$ \begin{CJK}{UTF8}{mj}半正定\end{CJK}.

\begin{enumerate}
  \setcounter{enumi}{8}
  \item ( 20 \begin{CJK}{UTF8}{mj}分\end{CJK}) \begin{CJK}{UTF8}{mj}设\end{CJK} $A \in M_{n}(\mathbb{C})$. \begin{CJK}{UTF8}{mj}定义\end{CJK} $M_{n}(\mathbb{C})$ \begin{CJK}{UTF8}{mj}上的线性变换\end{CJK} $\sigma, \tau$ \begin{CJK}{UTF8}{mj}为\end{CJK} $\sigma(X)=A X, \tau(X)=A X-X A$, \begin{CJK}{UTF8}{mj}对任意的\end{CJK} $X \in M_{n}(\mathbb{C}) .$
\end{enumerate}
(1) \begin{CJK}{UTF8}{mj}设\end{CJK} $A$ \begin{CJK}{UTF8}{mj}的秩为\end{CJK} $r$, \begin{CJK}{UTF8}{mj}求\end{CJK} $\operatorname{dim} \operatorname{ker} \sigma$;

(2) \begin{CJK}{UTF8}{mj}若\end{CJK} $B \in M_{n}(\mathbb{C})$, \begin{CJK}{UTF8}{mj}满足\end{CJK} $\tau(B)=B$. \begin{CJK}{UTF8}{mj}证明\end{CJK}: $B$ \begin{CJK}{UTF8}{mj}的特征值都是零\end{CJK}, \begin{CJK}{UTF8}{mj}且矩阵\end{CJK} $A$ \begin{CJK}{UTF8}{mj}与\end{CJK} $B$ \begin{CJK}{UTF8}{mj}至少有一个公共的特征向量\end{CJK}.

\section{8. 中山大学 2016 年研究生入学考试试题高等代数 
 李扬 
 微信公众号: sxkyliyang}
\begin{CJK}{UTF8}{mj}符号说明\end{CJK}: \begin{CJK}{UTF8}{mj}试卷中\end{CJK} $\mathbb{Q}, \mathbb{R}, \mathbb{C}$ \begin{CJK}{UTF8}{mj}分别表示有理数域\end{CJK}, \begin{CJK}{UTF8}{mj}实数域和复数域\end{CJK}; $\mathbb{F}$ \begin{CJK}{UTF8}{mj}表示一般数域\end{CJK}.

\begin{enumerate}
  \item ( 20 \begin{CJK}{UTF8}{mj}分\end{CJK}) \begin{CJK}{UTF8}{mj}设\end{CJK}
\end{enumerate}
$$
\begin{aligned}
&f(x)=x^{4}+x^{3}+x^{2}+x+1 \\
&g(x)=x^{4}-x^{3}+x^{2}-x+1
\end{aligned}
$$
(1) \begin{CJK}{UTF8}{mj}求\end{CJK} $f(x)$ \begin{CJK}{UTF8}{mj}与\end{CJK} $g(x)$ \begin{CJK}{UTF8}{mj}的首一最大公因式\end{CJK} $(f, g)$, \begin{CJK}{UTF8}{mj}并求\end{CJK} $u(x), v(x)$ \begin{CJK}{UTF8}{mj}使\end{CJK} $u f+v g=(f, g)$;

(2) \begin{CJK}{UTF8}{mj}把\end{CJK} $g(x)$ \begin{CJK}{UTF8}{mj}分别在数域\end{CJK} $\mathbb{C}, \mathbb{R}$ \begin{CJK}{UTF8}{mj}及\end{CJK} $\mathbb{Q}$ \begin{CJK}{UTF8}{mj}上分解为不可约因式的乘积\end{CJK}.

\begin{enumerate}
  \setcounter{enumi}{2}
  \item ( 20 \begin{CJK}{UTF8}{mj}分\end{CJK}) \begin{CJK}{UTF8}{mj}设\end{CJK}
\end{enumerate}
$$
A=\frac{1}{2}\left(\begin{array}{cccc}
1 & 1 & 1 & 1 \\
1 & 1 & -1 & -1 \\
1 & -1 & 1 & -1 \\
1 & -1 & -1 & 1
\end{array}\right)
$$
\begin{CJK}{UTF8}{mj}求正交矩阵\end{CJK} $P$ \begin{CJK}{UTF8}{mj}使\end{CJK} $P^{-1} A P$ \begin{CJK}{UTF8}{mj}为对角矩阵\end{CJK}.

\begin{enumerate}
  \setcounter{enumi}{3}
  \item ( 20 \begin{CJK}{UTF8}{mj}分\end{CJK}) \begin{CJK}{UTF8}{mj}设\end{CJK}
\end{enumerate}
$$
A=\left(\begin{array}{cccc}
0 & a & b & c \\
-a & 0 & h & -g \\
-b & -h & 0 & f \\
-c & g & -f & 0
\end{array}\right) \in M_{4}(\mathbb{R})
$$
(1) \begin{CJK}{UTF8}{mj}计算\end{CJK} $A$ \begin{CJK}{UTF8}{mj}的行列式\end{CJK} $\operatorname{det} A$;

(2) \begin{CJK}{UTF8}{mj}设\end{CJK} $\lambda \in \mathbb{R}$, \begin{CJK}{UTF8}{mj}证明\end{CJK}: \begin{CJK}{UTF8}{mj}线性方程组\end{CJK} $(\lambda I+A) X=0$ \begin{CJK}{UTF8}{mj}有解的充分必要条件是\end{CJK} $\lambda=a f+b g+c h=0$.

\begin{enumerate}
  \setcounter{enumi}{4}
  \item (10 \begin{CJK}{UTF8}{mj}分\end{CJK}) \begin{CJK}{UTF8}{mj}设\end{CJK} $A \in M_{n}(\mathbb{F}), \alpha, \beta \in \mathbb{F}^{n \times 1}$. \begin{CJK}{UTF8}{mj}证明\end{CJK}: $\operatorname{det}\left(A+\alpha \beta^{T}\right)=\operatorname{det} A+\beta^{T} \operatorname{adj}(A) \alpha$, \begin{CJK}{UTF8}{mj}这里\end{CJK} $\operatorname{adj}(A)$ \begin{CJK}{UTF8}{mj}表示矩阵\end{CJK} $A$ \begin{CJK}{UTF8}{mj}的\end{CJK} \begin{CJK}{UTF8}{mj}伴随矩阵\end{CJK}.

  \item ( 10 \begin{CJK}{UTF8}{mj}分\end{CJK}) \begin{CJK}{UTF8}{mj}设\end{CJK}

\end{enumerate}
$$
A=\left(\begin{array}{ccc}
3 & 1 & 0 \\
-4 & -1 & 0 \\
7 & 1 & 2
\end{array}\right)
$$
\begin{CJK}{UTF8}{mj}求\end{CJK} $A$ \begin{CJK}{UTF8}{mj}的\end{CJK} Jordan \begin{CJK}{UTF8}{mj}标准形和最小多项式\end{CJK}.

\begin{enumerate}
  \setcounter{enumi}{6}
  \item ( 10 \begin{CJK}{UTF8}{mj}分\end{CJK}) \begin{CJK}{UTF8}{mj}设\end{CJK} $A \in M_{n}(\mathbb{F})$ \begin{CJK}{UTF8}{mj}在\end{CJK} $\mathbb{F}$ \begin{CJK}{UTF8}{mj}上有\end{CJK} $n$ \begin{CJK}{UTF8}{mj}个不同的特征值\end{CJK}, \begin{CJK}{UTF8}{mj}令\end{CJK} $W=\left\{B \in M_{n}(\mathbb{F}) \mid A B=B A\right\}$. \begin{CJK}{UTF8}{mj}求\end{CJK} $\operatorname{dim} W$.

  \item $\left(10\right.$ \begin{CJK}{UTF8}{mj}分\end{CJK}) \begin{CJK}{UTF8}{mj}设线性映射\end{CJK} $\varphi: M_{n}(\mathbb{F}) \rightarrow M_{k}(\mathbb{F})$ \begin{CJK}{UTF8}{mj}满足\end{CJK}: $\forall A, B \in M_{n}(\mathbb{F}), \varphi(A B)=\varphi(A) \varphi(B)$ \begin{CJK}{UTF8}{mj}及\end{CJK} $\varphi\left(I_{n}\right)=I_{k}$. \begin{CJK}{UTF8}{mj}证明\end{CJK}: \begin{CJK}{UTF8}{mj}若\end{CJK} $\lambda$ \begin{CJK}{UTF8}{mj}是\end{CJK} $\varphi(A)$ \begin{CJK}{UTF8}{mj}的特征值\end{CJK}, \begin{CJK}{UTF8}{mj}则\end{CJK} $\lambda$ \begin{CJK}{UTF8}{mj}也是\end{CJK} $A$ \begin{CJK}{UTF8}{mj}的特征值\end{CJK}.

  \item (10 \begin{CJK}{UTF8}{mj}分\end{CJK}) \begin{CJK}{UTF8}{mj}设\end{CJK} $\sigma$ \begin{CJK}{UTF8}{mj}为\end{CJK} $n$ \begin{CJK}{UTF8}{mj}维欧氏空间\end{CJK} $V$ \begin{CJK}{UTF8}{mj}上的投影变换\end{CJK}, \begin{CJK}{UTF8}{mj}即\end{CJK} $\sigma^{2}=\sigma$. \begin{CJK}{UTF8}{mj}证明\end{CJK}: \begin{CJK}{UTF8}{mj}若\end{CJK} $\forall \alpha \in V,|\sigma(\alpha)| \leq|\alpha|$, \begin{CJK}{UTF8}{mj}则\end{CJK} $\operatorname{ker} \sigma \perp \operatorname{Im} \sigma$.

  \item ( 20 \begin{CJK}{UTF8}{mj}分\end{CJK}) \begin{CJK}{UTF8}{mj}记\end{CJK} $V=M_{n}(\mathbb{R}), U=\left\{A \in V \mid A^{T}=A\right\}, W=\left\{B \in V \mid B^{T}=-B\right\}$. \begin{CJK}{UTF8}{mj}在\end{CJK} $V$ \begin{CJK}{UTF8}{mj}上定义二元函数\end{CJK} $f: V \times V \rightarrow \mathbb{R}, f(A, B)=\operatorname{tr}\left(A B^{T}\right), \forall A, B \in V .$

\end{enumerate}
(1) \begin{CJK}{UTF8}{mj}证明\end{CJK}: $(V, f)$ \begin{CJK}{UTF8}{mj}是欧氏空间\end{CJK};

(2) \begin{CJK}{UTF8}{mj}证明\end{CJK}: $U \perp W, V=U \oplus W$;

(3) \begin{CJK}{UTF8}{mj}设\end{CJK} $A \in V$, \begin{CJK}{UTF8}{mj}试求\end{CJK} $B \in U$ \begin{CJK}{UTF8}{mj}使\end{CJK} $A$ \begin{CJK}{UTF8}{mj}与\end{CJK} $B$ \begin{CJK}{UTF8}{mj}的距离最短\end{CJK}, \begin{CJK}{UTF8}{mj}即\end{CJK} $\forall D \in U$, \begin{CJK}{UTF8}{mj}有\end{CJK} $d(A, B) \leq d(A, D)$. 10. ( 20 \begin{CJK}{UTF8}{mj}分\end{CJK}) \begin{CJK}{UTF8}{mj}我们称一个\end{CJK} $n$ \begin{CJK}{UTF8}{mj}阶复方阵\end{CJK} $A$ \begin{CJK}{UTF8}{mj}为半正定的\end{CJK}, \begin{CJK}{UTF8}{mj}如果\end{CJK} $\forall X \in \mathbb{C}^{n}, X^{*} A X \geq 0$; \begin{CJK}{UTF8}{mj}称一个线性映射\end{CJK} $\varphi: M_{n}(\mathbb{C}) \rightarrow$ $M_{k}(\mathbb{C})$ \begin{CJK}{UTF8}{mj}为非负的\end{CJK}, \begin{CJK}{UTF8}{mj}若\end{CJK} $A$ \begin{CJK}{UTF8}{mj}半正定可推出\end{CJK} $\varphi(A)$ \begin{CJK}{UTF8}{mj}半正定\end{CJK}. \begin{CJK}{UTF8}{mj}证明\end{CJK}:

(1) \begin{CJK}{UTF8}{mj}若\end{CJK} $A$ \begin{CJK}{UTF8}{mj}半正定\end{CJK}, \begin{CJK}{UTF8}{mj}则\end{CJK} $A^{*}=A$, \begin{CJK}{UTF8}{mj}且\end{CJK} $A$ \begin{CJK}{UTF8}{mj}的特征值为非负实数\end{CJK};

(2) \begin{CJK}{UTF8}{mj}若\end{CJK} $\varphi: M_{n}(\mathbb{C}) \rightarrow M_{k}(\mathbb{C})$ \begin{CJK}{UTF8}{mj}为非负的\end{CJK}, \begin{CJK}{UTF8}{mj}则\end{CJK} $\forall A \in M_{n}(\mathbb{C}), \varphi\left(A^{*}\right)=\varphi(A)^{*}$.

\begin{CJK}{UTF8}{mj}注\end{CJK}: \begin{CJK}{UTF8}{mj}若\end{CJK} $A$ \begin{CJK}{UTF8}{mj}为复矩阵\end{CJK}, $A^{*}$ \begin{CJK}{UTF8}{mj}表示\end{CJK} $A$ \begin{CJK}{UTF8}{mj}的共轭转置\end{CJK}. \begin{CJK}{UTF8}{mj}即\end{CJK} $A^{*}$ \begin{CJK}{UTF8}{mj}的\end{CJK} $(i, j)$ \begin{CJK}{UTF8}{mj}等于\end{CJK} $A$ \begin{CJK}{UTF8}{mj}的\end{CJK} $(j, i)$ \begin{CJK}{UTF8}{mj}元的共轭\end{CJK}.

\section{9. 中山大学 2017 年研究生入学考试试题高等代数 
 李扬 
 微信公众号: sxkyliyang}
\begin{CJK}{UTF8}{mj}符号说明\end{CJK}: \begin{CJK}{UTF8}{mj}试卷中\end{CJK} $\mathbb{Q}, \mathbb{R}, \mathbb{C}$ \begin{CJK}{UTF8}{mj}分别表示有理数域\end{CJK}, \begin{CJK}{UTF8}{mj}实数域和复数域\end{CJK}; $\mathbb{F}$ \begin{CJK}{UTF8}{mj}表示一般数域\end{CJK}.

\begin{enumerate}
  \item ( 20 \begin{CJK}{UTF8}{mj}分\end{CJK}) \begin{CJK}{UTF8}{mj}设\end{CJK}
\end{enumerate}
$$
\begin{gathered}
f(x)=x^{4}+3 x-2 \\
g(x)=3 x^{3}-x^{2}-7 x+4
\end{gathered}
$$
(1) \begin{CJK}{UTF8}{mj}求\end{CJK} $f(x)$ \begin{CJK}{UTF8}{mj}与\end{CJK} $g(x)$ \begin{CJK}{UTF8}{mj}的首一最大公因式\end{CJK} $(f, g)$, \begin{CJK}{UTF8}{mj}并求\end{CJK} $u(x), v(x)$ \begin{CJK}{UTF8}{mj}使\end{CJK} $u f+v g=(f, g)$;

(2) \begin{CJK}{UTF8}{mj}把\end{CJK} $g(x)$ \begin{CJK}{UTF8}{mj}分别在数域\end{CJK} $\mathbb{C}, \mathbb{R}$ \begin{CJK}{UTF8}{mj}及\end{CJK} $\mathbb{Q}$ \begin{CJK}{UTF8}{mj}上分解为不可约因式的乘积\end{CJK}.

\begin{enumerate}
  \setcounter{enumi}{2}
  \item ( 20 \begin{CJK}{UTF8}{mj}分\end{CJK}) \begin{CJK}{UTF8}{mj}求下列矩阵的行列式\end{CJK}:
\end{enumerate}
(1) $A=\left(\begin{array}{cccc}1823 & 823 & 23 & 3 \\ 1549 & 549 & 49 & 9 \\ 1667 & 667 & 67 & 7 \\ 1986 & 986 & 86 & 6\end{array}\right), \quad(2) \quad B=\left(\begin{array}{cccc}a & 1 & \cdots & 1 \\ 1 & a & \cdots & 1 \\ \vdots & \vdots & & \vdots \\ 1 & 1 & \cdots & a\end{array}\right)_{n \times n}$.

\begin{enumerate}
  \setcounter{enumi}{3}
  \item ( 20 \begin{CJK}{UTF8}{mj}分\end{CJK}) \begin{CJK}{UTF8}{mj}设有数域\end{CJK} $\mathbb{F}$ \begin{CJK}{UTF8}{mj}上的齐次线性方程组\end{CJK}
\end{enumerate}
$$
\left\{\begin{array}{c}
a x_{1}+x_{2}+\cdots+x_{n}=0 \\
x_{1}+a x_{2}+\cdots+x_{n}=0 \\
\vdots \\
x_{1}+x_{2}+\cdots+a x_{n}=0
\end{array}\right.
$$
\begin{CJK}{UTF8}{mj}其中\end{CJK} $n \geq 2$, \begin{CJK}{UTF8}{mj}试讨论\end{CJK} $a$ \begin{CJK}{UTF8}{mj}为何值时\end{CJK}, \begin{CJK}{UTF8}{mj}方程组仅有零解\end{CJK}, \begin{CJK}{UTF8}{mj}有无穷多解\end{CJK}? \begin{CJK}{UTF8}{mj}对于有无穷多解的情况给出方程组的基础解\end{CJK} \begin{CJK}{UTF8}{mj}系\end{CJK}.

\begin{enumerate}
  \setcounter{enumi}{4}
  \item (20 \begin{CJK}{UTF8}{mj}分\end{CJK}) \begin{CJK}{UTF8}{mj}设有实矩阵\end{CJK}
\end{enumerate}
$$
A=\left(\begin{array}{lll}
2 & 1 & 1 \\
1 & 2 & 1 \\
1 & 1 & 2
\end{array}\right)
$$
(1) \begin{CJK}{UTF8}{mj}求正交矩阵\end{CJK} $P$ \begin{CJK}{UTF8}{mj}使\end{CJK} $P^{T} A P$ \begin{CJK}{UTF8}{mj}为对角矩阵\end{CJK}, \begin{CJK}{UTF8}{mj}这里\end{CJK} $P^{T}$ \begin{CJK}{UTF8}{mj}指\end{CJK} $P$ \begin{CJK}{UTF8}{mj}的转置\end{CJK}.

(2) \begin{CJK}{UTF8}{mj}试求正定矩阵\end{CJK} $B$ \begin{CJK}{UTF8}{mj}使\end{CJK} $B^{2}=A$.

\begin{enumerate}
  \setcounter{enumi}{5}
  \item ( 20 \begin{CJK}{UTF8}{mj}分\end{CJK}) \begin{CJK}{UTF8}{mj}设复矩阵\end{CJK}
\end{enumerate}
$$
A=\left(\begin{array}{ccc}
3 & 0 & 8 \\
3 & -1 & 6 \\
-2 & 0 & -5
\end{array}\right)
$$
(1) \begin{CJK}{UTF8}{mj}求\end{CJK} $A$ \begin{CJK}{UTF8}{mj}的\end{CJK} Jordan \begin{CJK}{UTF8}{mj}标准形\end{CJK}.

(2) \begin{CJK}{UTF8}{mj}求\end{CJK} $A$ \begin{CJK}{UTF8}{mj}的最小多项式以及\end{CJK} $A^{100}$.

\begin{enumerate}
  \setcounter{enumi}{6}
  \item ( 10 \begin{CJK}{UTF8}{mj}分\end{CJK}) \begin{CJK}{UTF8}{mj}设\end{CJK} $A$ \begin{CJK}{UTF8}{mj}为数域\end{CJK} $\mathbb{F}$ \begin{CJK}{UTF8}{mj}上的一个\end{CJK} $n$ \begin{CJK}{UTF8}{mj}阶方阵\end{CJK}. \begin{CJK}{UTF8}{mj}证明\end{CJK} $A$ \begin{CJK}{UTF8}{mj}的秩为\end{CJK} 1 \begin{CJK}{UTF8}{mj}当且仅当存在非零的\end{CJK} $n$ \begin{CJK}{UTF8}{mj}维列向量\end{CJK} $\alpha, \beta$, \begin{CJK}{UTF8}{mj}使得\end{CJK} $A=\alpha \beta^{T}$.

  \item (10 \begin{CJK}{UTF8}{mj}分\end{CJK}) \begin{CJK}{UTF8}{mj}设\end{CJK} $\sigma, \tau$ \begin{CJK}{UTF8}{mj}为\end{CJK} $n$ \begin{CJK}{UTF8}{mj}维线性空间\end{CJK} $V$ \begin{CJK}{UTF8}{mj}上的线性变换\end{CJK}, $\mathrm{id}_{V}$ \begin{CJK}{UTF8}{mj}是\end{CJK} $V$ \begin{CJK}{UTF8}{mj}上的恒等变换\end{CJK}. \begin{CJK}{UTF8}{mj}证明\end{CJK}: \begin{CJK}{UTF8}{mj}若\end{CJK} $\sigma \tau=\mathrm{id}_{V}$, \begin{CJK}{UTF8}{mj}则\end{CJK} $\tau \sigma=\mathrm{id}_{V}$. 8. ( 20 \begin{CJK}{UTF8}{mj}分\end{CJK}) \begin{CJK}{UTF8}{mj}设\end{CJK} $V$ \begin{CJK}{UTF8}{mj}为一个\end{CJK} $n$ \begin{CJK}{UTF8}{mj}维欧几里得空间\end{CJK}, $\sigma$ \begin{CJK}{UTF8}{mj}为\end{CJK} $V$ \begin{CJK}{UTF8}{mj}上的一个线性变换\end{CJK}. \begin{CJK}{UTF8}{mj}若有单位向量\end{CJK} $\eta$ \begin{CJK}{UTF8}{mj}使得\end{CJK} $\sigma(\alpha)=\alpha-2(\eta, \alpha) \eta$, \begin{CJK}{UTF8}{mj}则称\end{CJK} $\sigma$ \begin{CJK}{UTF8}{mj}为镜面反射\end{CJK}. \begin{CJK}{UTF8}{mj}这里\end{CJK} $(\eta, \alpha)$ \begin{CJK}{UTF8}{mj}表示\end{CJK} $\eta$ \begin{CJK}{UTF8}{mj}和\end{CJK} $\alpha$ \begin{CJK}{UTF8}{mj}的内积\end{CJK}.

\end{enumerate}
(1) \begin{CJK}{UTF8}{mj}若\end{CJK} $\sigma$ \begin{CJK}{UTF8}{mj}是镜面反射\end{CJK}, \begin{CJK}{UTF8}{mj}证明\end{CJK}: $V$ \begin{CJK}{UTF8}{mj}有正交分解\end{CJK} $V=\operatorname{ker}\left(\mathrm{id}_{V}+\sigma\right) \oplus \operatorname{ker}\left(\mathrm{id}_{V}-\sigma\right)$. \begin{CJK}{UTF8}{mj}这里\end{CJK} $\mathrm{id}_{V}$ \begin{CJK}{UTF8}{mj}表示\end{CJK} $V$ \begin{CJK}{UTF8}{mj}上的恒等变换\end{CJK}. \begin{CJK}{UTF8}{mj}对于线性变换\end{CJK} $\sigma, \operatorname{ker} \sigma$ \begin{CJK}{UTF8}{mj}表示\end{CJK} $\sigma$ \begin{CJK}{UTF8}{mj}的核空间\end{CJK}.

(2) \begin{CJK}{UTF8}{mj}若\end{CJK} $\alpha, \beta$ \begin{CJK}{UTF8}{mj}为\end{CJK} $V$ \begin{CJK}{UTF8}{mj}上两个线性无关的单位向量\end{CJK}, \begin{CJK}{UTF8}{mj}求一个镜面反射\end{CJK} $\tau$ \begin{CJK}{UTF8}{mj}使得\end{CJK} $\tau(\alpha)=\beta$.

\begin{enumerate}
  \setcounter{enumi}{9}
  \item ( 10 \begin{CJK}{UTF8}{mj}分\end{CJK}) \begin{CJK}{UTF8}{mj}设\end{CJK} $A$ \begin{CJK}{UTF8}{mj}是一个\end{CJK} $n$ \begin{CJK}{UTF8}{mj}阶实矩阵\end{CJK}, \begin{CJK}{UTF8}{mj}其有\end{CJK} $n$ \begin{CJK}{UTF8}{mj}个绝对值小于\end{CJK} 1 \begin{CJK}{UTF8}{mj}的实特征值\end{CJK}, \begin{CJK}{UTF8}{mj}证明\end{CJK}:
\end{enumerate}
$$
\ln (\operatorname{det}(I-A))=-\sum_{k=1}^{\infty} \frac{1}{k} \operatorname{tr}\left(A^{k}\right) .
$$
\begin{CJK}{UTF8}{mj}其中\end{CJK} $\ln$ \begin{CJK}{UTF8}{mj}为自然对数\end{CJK}, $I$ \begin{CJK}{UTF8}{mj}表示\end{CJK} $n$ \begin{CJK}{UTF8}{mj}阶单位矩阵\end{CJK}, $\operatorname{det} A$ \begin{CJK}{UTF8}{mj}表示\end{CJK} $A$ \begin{CJK}{UTF8}{mj}的行列式\end{CJK}, $\operatorname{tr}(A)$ \begin{CJK}{UTF8}{mj}表示\end{CJK} $A$ \begin{CJK}{UTF8}{mj}的迹\end{CJK}, \begin{CJK}{UTF8}{mj}即其对角线上元素之\end{CJK} \begin{CJK}{UTF8}{mj}和\end{CJK}.

\section{0. 中山大学 2009 年研究生入学考试试题数学分析 
 李扬 
 微信公众号: sxkyliyang}
\begin{enumerate}
  \item (\begin{CJK}{UTF8}{mj}每小题\end{CJK} 6 \begin{CJK}{UTF8}{mj}分\end{CJK}, \begin{CJK}{UTF8}{mj}共\end{CJK} 48 \begin{CJK}{UTF8}{mj}分\end{CJK})
\end{enumerate}
(1) \begin{CJK}{UTF8}{mj}求\end{CJK}
$$
\lim _{x \rightarrow \infty}\left(x-x^{2} \ln \left(1+\frac{1}{x}\right)\right)
$$
(2) $\left\{\begin{array}{l}x=\cos \left(t^{2}\right) \\ y=\int_{0}^{2} \frac{\sin u}{u} \mathrm{~d} u\end{array}\right.$, \begin{CJK}{UTF8}{mj}求\end{CJK} $\frac{\mathrm{d} y}{\mathrm{~d} x}$.

(3) \begin{CJK}{UTF8}{mj}求\end{CJK}
$$
\int \frac{1-\ln x}{\ln ^{2} x} \mathrm{~d} x
$$
(4) \begin{CJK}{UTF8}{mj}求\end{CJK}
$$
\int_{-1}^{1}|x-a| e^{x} \mathrm{~d} x,|a|<1
$$
(5) \begin{CJK}{UTF8}{mj}设\end{CJK} $z=u v+\sin t, u=e^{t}, v=\cos t$, \begin{CJK}{UTF8}{mj}求\end{CJK} $\frac{\mathrm{d} z}{\mathrm{~d} t}$.

(6) \begin{CJK}{UTF8}{mj}设\end{CJK} $u=\varphi(x+\psi(y))$, \begin{CJK}{UTF8}{mj}其中\end{CJK} $\varphi, \psi$ \begin{CJK}{UTF8}{mj}二阶可微\end{CJK}, $x, y$ \begin{CJK}{UTF8}{mj}为自变量\end{CJK}, \begin{CJK}{UTF8}{mj}求\end{CJK} $\mathrm{d}^{2} u$.

(7) \begin{CJK}{UTF8}{mj}求级数\end{CJK}
$$
\sum_{n=1}^{\infty} \cos ^{n} x
$$
\begin{CJK}{UTF8}{mj}在收敛域上的和函数\end{CJK}.

(8) \begin{CJK}{UTF8}{mj}判别级数\end{CJK}
$$
\sum_{n=1}^{\infty} \frac{1}{n^{1+\frac{1}{n}}}
$$
\begin{CJK}{UTF8}{mj}的敛散性\end{CJK}.

\begin{enumerate}
  \setcounter{enumi}{2}
  \item (12 \begin{CJK}{UTF8}{mj}分\end{CJK}) \begin{CJK}{UTF8}{mj}将区间\end{CJK} $[1,2]$ \begin{CJK}{UTF8}{mj}作\end{CJK} $n$ \begin{CJK}{UTF8}{mj}等分\end{CJK}, \begin{CJK}{UTF8}{mj}分点为\end{CJK} $1=x_{0}<x_{1}<\cdots<x_{n}=2$, \begin{CJK}{UTF8}{mj}求\end{CJK} $\lim _{n \rightarrow \infty} \sqrt[n]{x_{1} x_{2} \cdots x_{n}}$.

  \item (16 \begin{CJK}{UTF8}{mj}分\end{CJK}) \begin{CJK}{UTF8}{mj}计算\end{CJK}

\end{enumerate}
$$
I=\int_{\ell} \frac{(x+y) \mathrm{d} x+(y-x) \mathrm{d} y}{x^{2}+y^{2}} .
$$
\begin{CJK}{UTF8}{mj}其中\end{CJK} $\ell$ \begin{CJK}{UTF8}{mj}是从点\end{CJK} $A(-1,0)$ \begin{CJK}{UTF8}{mj}到点\end{CJK} $B(1,0)$ \begin{CJK}{UTF8}{mj}的一条不通过原点的光滑曲线\end{CJK}: $y=f(x), x \in[-1,1]$, \begin{CJK}{UTF8}{mj}且当\end{CJK} $x \in(-1,1)$ \begin{CJK}{UTF8}{mj}时\end{CJK}, $f(x)>0$.

\begin{enumerate}
  \setcounter{enumi}{4}
  \item (16 \begin{CJK}{UTF8}{mj}分\end{CJK}) \begin{CJK}{UTF8}{mj}计算\end{CJK}
\end{enumerate}
$$
\iint_{\Sigma} x^{2} \mathrm{~d} y \mathrm{~d} z+y^{2} \mathrm{~d} z \mathrm{~d} x+z^{2} \mathrm{~d} x \mathrm{~d} y
$$
\begin{CJK}{UTF8}{mj}其中\end{CJK} $\Sigma$ \begin{CJK}{UTF8}{mj}为曲面\end{CJK} $x^{2}+y^{2}=z^{2}$ \begin{CJK}{UTF8}{mj}介于平面\end{CJK} $z=0$ \begin{CJK}{UTF8}{mj}和\end{CJK} $z=h(h>0)$ \begin{CJK}{UTF8}{mj}之间的部分取下侧\end{CJK}.

\begin{enumerate}
  \setcounter{enumi}{5}
  \item ( 16 \begin{CJK}{UTF8}{mj}分\end{CJK}) \begin{CJK}{UTF8}{mj}设\end{CJK} $f(x)$ \begin{CJK}{UTF8}{mj}在\end{CJK} $[1, \infty)$ \begin{CJK}{UTF8}{mj}连续\end{CJK}, $f^{\prime \prime}(x) \leq 0, f(1)=2, f^{\prime}(1)=-3$. \begin{CJK}{UTF8}{mj}证明\end{CJK}: $f(x)=0$ \begin{CJK}{UTF8}{mj}在\end{CJK} $(1, \infty)$ \begin{CJK}{UTF8}{mj}有且仅有一个实\end{CJK} \begin{CJK}{UTF8}{mj}根\end{CJK}.

  \item ( 16 \begin{CJK}{UTF8}{mj}分\end{CJK}) \begin{CJK}{UTF8}{mj}设函数\end{CJK} $f(x)$ \begin{CJK}{UTF8}{mj}在\end{CJK} $(-\infty, \infty)$ \begin{CJK}{UTF8}{mj}连续\end{CJK}, \begin{CJK}{UTF8}{mj}试证\end{CJK}: \begin{CJK}{UTF8}{mj}对一切\end{CJK} $x$ \begin{CJK}{UTF8}{mj}满足\end{CJK} $f(2 x)=f(x) e^{x}$ \begin{CJK}{UTF8}{mj}的充要条件是\end{CJK} $f(x)=f(0) e^{x}$.

  \item (16 \begin{CJK}{UTF8}{mj}分\end{CJK}) \begin{CJK}{UTF8}{mj}求椭球面\end{CJK}

\end{enumerate}
$$
\frac{x^{2}}{a^{2}}+\frac{y^{2}}{b^{2}}+\frac{z^{2}}{c^{2}}=1
$$
\begin{CJK}{UTF8}{mj}在第一卦限部分的切平面与三坐标平面围成的四面体的最小体积\end{CJK}. 8. ( 10 \begin{CJK}{UTF8}{mj}分\end{CJK}) \begin{CJK}{UTF8}{mj}讨论\end{CJK}
$$
\sum_{n=1}^{\infty} \frac{\cos \left(\frac{\pi}{2} \ln n\right)}{n}
$$
\begin{CJK}{UTF8}{mj}的敛散性\end{CJK}.

\section{1. 中山大学 2010 年研究生入学考试试题数学分析 
 李扬 
 微信公众号: sxkyliyang}
\begin{enumerate}
  \item (\begin{CJK}{UTF8}{mj}每小题\end{CJK} 6 \begin{CJK}{UTF8}{mj}分\end{CJK}, \begin{CJK}{UTF8}{mj}共\end{CJK} 48 \begin{CJK}{UTF8}{mj}分\end{CJK})
\end{enumerate}
(1) \begin{CJK}{UTF8}{mj}求极限\end{CJK}
$$
\lim _{n \rightarrow \infty} \frac{1}{n} \sqrt[n]{n(n+1) \cdots(2 n+1)}
$$
(2) \begin{CJK}{UTF8}{mj}计算不定积分\end{CJK}
$$
\int \max (|x|, 1) \mathrm{d} x
$$
(3) \begin{CJK}{UTF8}{mj}已知\end{CJK} $f(x)=\int_{0}^{x} \frac{\sin t}{\pi-t} \mathrm{~d} t$, \begin{CJK}{UTF8}{mj}求定积分\end{CJK}
$$
\int_{0}^{\pi} f(x) \mathrm{d} x
$$
(4) \begin{CJK}{UTF8}{mj}求二元函数极限\end{CJK}
$$
\lim _{x \rightarrow 0, y \rightarrow 0}\left(x^{2}+y^{2}\right)^{x^{2} y^{2}}
$$
(5) \begin{CJK}{UTF8}{mj}求二次积分\end{CJK}
$$
\int_{0}^{1} \mathrm{~d} y \int_{y}^{1} e^{x^{2}} \mathrm{~d} x
$$
(6) \begin{CJK}{UTF8}{mj}计算\end{CJK}
$$
I=\oint_{L} \frac{x \mathrm{~d} y-y \mathrm{~d} x}{x^{2}+y^{2}} .
$$
\begin{CJK}{UTF8}{mj}其中\end{CJK} $L$ \begin{CJK}{UTF8}{mj}为一条无重点\end{CJK}, \begin{CJK}{UTF8}{mj}分段光滑且不经过原点的连续封闭曲线\end{CJK}, $L$ \begin{CJK}{UTF8}{mj}的方向为逆时针方向\end{CJK}.

(7) \begin{CJK}{UTF8}{mj}讨论函数项级数\end{CJK}
$$
\sum_{n=1}^{\infty} \frac{(1-\cos x) \sin n x}{\sqrt{n+x}}
$$
\begin{CJK}{UTF8}{mj}在\end{CJK} $[0,2 \pi]$ \begin{CJK}{UTF8}{mj}上的一致收敛性\end{CJK}.

(8) \begin{CJK}{UTF8}{mj}计算\end{CJK}
$$
\iint_{s}\left(x^{2}+y^{2}\right) \mathrm{d} S
$$
\begin{CJK}{UTF8}{mj}其中\end{CJK} $S$ \begin{CJK}{UTF8}{mj}为曲面\end{CJK} $z=\sqrt{x^{2}+y^{2}}$ \begin{CJK}{UTF8}{mj}与平面\end{CJK} $z=1$ \begin{CJK}{UTF8}{mj}所围几何体的表面\end{CJK}.

\begin{enumerate}
  \setcounter{enumi}{2}
  \item (12 \begin{CJK}{UTF8}{mj}分\end{CJK}) \begin{CJK}{UTF8}{mj}单位圆盘中切去圆心角为\end{CJK} $\theta$ \begin{CJK}{UTF8}{mj}的扇形\end{CJK}, \begin{CJK}{UTF8}{mj}余下部分粘合成一雉面\end{CJK}. \begin{CJK}{UTF8}{mj}问\end{CJK} $\theta$ \begin{CJK}{UTF8}{mj}为多少时\end{CJK}, \begin{CJK}{UTF8}{mj}该雉面加上底面所围成\end{CJK} \begin{CJK}{UTF8}{mj}的雉体体积最大\end{CJK}?

  \item (16 \begin{CJK}{UTF8}{mj}分\end{CJK}) \begin{CJK}{UTF8}{mj}设\end{CJK} $f(x)$ \begin{CJK}{UTF8}{mj}在\end{CJK} $x=0$ \begin{CJK}{UTF8}{mj}某邻域内有二阶连续导数\end{CJK}, \begin{CJK}{UTF8}{mj}且\end{CJK} $\lim _{x \rightarrow 0} \frac{f(x)}{x}=0$. \begin{CJK}{UTF8}{mj}证明\end{CJK}:

\end{enumerate}
$$
\sum_{n=1}^{\infty} f\left(\frac{1}{n}\right)
$$
\begin{CJK}{UTF8}{mj}绝对收敛\end{CJK}.

\begin{enumerate}
  \setcounter{enumi}{4}
  \item (16 \begin{CJK}{UTF8}{mj}分\end{CJK}) \begin{CJK}{UTF8}{mj}设\end{CJK}
\end{enumerate}
$$
f(x, y)= \begin{cases}\left(x^{2}+y^{2}\right)^{p} \sin \frac{1}{x^{2}+y^{2}}, & x^{2}+y^{2} \neq 0 \\ 0, & x^{2}+y^{2}=0 .\end{cases}
$$
\begin{CJK}{UTF8}{mj}其中\end{CJK} $p$ \begin{CJK}{UTF8}{mj}为正数\end{CJK}. \begin{CJK}{UTF8}{mj}试分别确定\end{CJK} $p$ \begin{CJK}{UTF8}{mj}的值\end{CJK}, \begin{CJK}{UTF8}{mj}使得如下结论分别成立\end{CJK}:

(1) $f(x, y)$ \begin{CJK}{UTF8}{mj}在点\end{CJK} $(0,0)$ \begin{CJK}{UTF8}{mj}处连续\end{CJK};

(2) $f_{x}(0,0)$ \begin{CJK}{UTF8}{mj}与\end{CJK} $f_{y}(0,0)$ \begin{CJK}{UTF8}{mj}都存在\end{CJK};

(3) $f_{x}(x, y)$ \begin{CJK}{UTF8}{mj}与\end{CJK} $f_{y}(x, y)$ \begin{CJK}{UTF8}{mj}在\end{CJK} $(0,0)$ \begin{CJK}{UTF8}{mj}点连续\end{CJK}. 5. (16 \begin{CJK}{UTF8}{mj}分\end{CJK}) \begin{CJK}{UTF8}{mj}计算由曲面\end{CJK}
$$
\left(\frac{x}{a}+\frac{y}{b}\right)^{2}+\left(\frac{z}{c}\right)^{2}=1,(x \geq 0, y \geq 0, z \geq 0, a>0, b>0, c>0)
$$
\begin{CJK}{UTF8}{mj}所围几何体之体积\end{CJK}, \begin{CJK}{UTF8}{mj}其中\end{CJK} $a, b, c$ \begin{CJK}{UTF8}{mj}为正常数\end{CJK}.

\begin{enumerate}
  \setcounter{enumi}{6}
  \item (16 \begin{CJK}{UTF8}{mj}分\end{CJK}) \begin{CJK}{UTF8}{mj}求幂级数\end{CJK}
\end{enumerate}
$$
\sum_{n=1}^{\infty} \frac{n^{2}+1}{n ! 2^{n}} x^{n}
$$
\begin{CJK}{UTF8}{mj}的收敛范围\end{CJK}, \begin{CJK}{UTF8}{mj}并求其和函数\end{CJK}.

\begin{enumerate}
  \setcounter{enumi}{7}
  \item (16 \begin{CJK}{UTF8}{mj}分\end{CJK}) \begin{CJK}{UTF8}{mj}设\end{CJK} $u=f(r)$, \begin{CJK}{UTF8}{mj}其中\end{CJK} $r=\sqrt{x^{2}+y^{2}+z^{2}}$. \begin{CJK}{UTF8}{mj}变换方程\end{CJK}: $\frac{\partial^{2} u}{\partial x^{2}}+\frac{\partial^{2} u}{\partial y^{2}}+\frac{\partial^{2} u}{\partial z^{2}}=0$, \begin{CJK}{UTF8}{mj}使其成为关于\end{CJK} $f(r)$ \begin{CJK}{UTF8}{mj}的方程\end{CJK}.

  \item ( 10 \begin{CJK}{UTF8}{mj}分\end{CJK}) \begin{CJK}{UTF8}{mj}判别级数\end{CJK}

\end{enumerate}
$$
\sqrt{2}+\sqrt{2-\sqrt{2}}+\sqrt{2-\sqrt{2+\sqrt{2}}}+\sqrt{2-\sqrt{2+\sqrt{2+\sqrt{2}}}}+\cdots
$$
\begin{CJK}{UTF8}{mj}的收敛性\end{CJK}.

\section{2. 中山大学 2011 年研究生入学考试试题数学分析 
 李扬 
 微信公众号: sxkyliyang}
\begin{enumerate}
  \item (\begin{CJK}{UTF8}{mj}每小题\end{CJK} 6 \begin{CJK}{UTF8}{mj}分\end{CJK}, \begin{CJK}{UTF8}{mj}共\end{CJK} 48 \begin{CJK}{UTF8}{mj}分\end{CJK})
\end{enumerate}
(1) \begin{CJK}{UTF8}{mj}求极限\end{CJK}
$$
\lim _{x \rightarrow 0} \frac{\sqrt{1-x^{2}}-1}{x \tan x}
$$
(2) \begin{CJK}{UTF8}{mj}计算积分\end{CJK}
$$
\int_{0}^{\frac{\pi}{2}} \frac{\sin x \cos x}{1+\sin ^{4} x} \mathrm{~d} x
$$
(3) \begin{CJK}{UTF8}{mj}已知\end{CJK} $\sum_{n=1}^{\infty}(-1)^{n} a_{n}=A, \sum_{n=1}^{\infty} a_{2 n-1}=B$, \begin{CJK}{UTF8}{mj}求级数\end{CJK} $\sum_{n=1}^{\infty} a_{n}$ \begin{CJK}{UTF8}{mj}的和\end{CJK}.

(4) \begin{CJK}{UTF8}{mj}计算\end{CJK}
$$
\iint_{\Omega}\left(2 x+\frac{4}{3} y+z\right) \mathrm{d} S
$$
\begin{CJK}{UTF8}{mj}其中\end{CJK} $\Omega$ \begin{CJK}{UTF8}{mj}为平面\end{CJK} $\frac{x}{2}+\frac{y}{3}+\frac{z}{4}=1$ \begin{CJK}{UTF8}{mj}在第一卦限部分\end{CJK}.

(5) \begin{CJK}{UTF8}{mj}计算\end{CJK}
$$
\int_{L} \sqrt{x^{2}+y^{2}} \mathrm{~d} x+y\left(x y+\ln \left(x+\sqrt{x^{2}+y^{2}}\right)\right) \mathrm{d} y .
$$
\begin{CJK}{UTF8}{mj}其中\end{CJK} $L$ \begin{CJK}{UTF8}{mj}为曲线\end{CJK} $y=\sin x(0 \leq x \leq \pi)$ \begin{CJK}{UTF8}{mj}按\end{CJK} $x$ \begin{CJK}{UTF8}{mj}增大方向\end{CJK}.

(6) \begin{CJK}{UTF8}{mj}判别级数\end{CJK}
$$
\sum_{n=1}^{\infty} \frac{(-1)^{n}}{\sqrt{n}-\ln n}
$$
\begin{CJK}{UTF8}{mj}是绝对收玫\end{CJK}, \begin{CJK}{UTF8}{mj}条件收敛\end{CJK}, \begin{CJK}{UTF8}{mj}还是发散\end{CJK}?

(7) \begin{CJK}{UTF8}{mj}设\end{CJK} $\left\{\begin{array}{l}x=t^{3}-3 t \\ y=t^{2}+2 t\end{array}\right.$, \begin{CJK}{UTF8}{mj}求二阶导数\end{CJK} $\frac{\mathrm{d}^{2} y}{\mathrm{~d} x^{2}}$.

(8) \begin{CJK}{UTF8}{mj}求数列极限\end{CJK}
$$
\lim _{n \rightarrow \infty} \frac{1}{2} \cdot \frac{3}{4} \cdots \frac{2 n-1}{2 n} .
$$

\begin{enumerate}
  \setcounter{enumi}{2}
  \item (12 \begin{CJK}{UTF8}{mj}分\end{CJK}) \begin{CJK}{UTF8}{mj}设\end{CJK}
\end{enumerate}
$$
f(x, y)=\sqrt{|x y|}
$$
\begin{CJK}{UTF8}{mj}求偏导函数\end{CJK} $\frac{\partial f}{\partial x}, \frac{\partial f}{\partial y}$, \begin{CJK}{UTF8}{mj}指出它们的定义域及连续性\end{CJK}, \begin{CJK}{UTF8}{mj}并讨论\end{CJK} $f(x, y)$ \begin{CJK}{UTF8}{mj}在点\end{CJK} $(0,0)$ \begin{CJK}{UTF8}{mj}处的可微性\end{CJK}.

\begin{enumerate}
  \setcounter{enumi}{3}
  \item ( 16 \begin{CJK}{UTF8}{mj}分\end{CJK}) \begin{CJK}{UTF8}{mj}设\end{CJK} $f(x)$ \begin{CJK}{UTF8}{mj}满足\end{CJK}:
\end{enumerate}
(1) $-\infty<a \leq f(x) \leq b<+\infty$;

(2) $|f(x)-f(y)| \leq L|x-y|, 0<L<1, x, y \in[a, b]$;

\begin{CJK}{UTF8}{mj}任取\end{CJK} $x_{1} \in[a, b]$, \begin{CJK}{UTF8}{mj}作序列\end{CJK} $x_{n+1}=\frac{1}{2}\left(x_{n}+f\left(x_{n}\right)\right), n=1,2, \cdots$. \begin{CJK}{UTF8}{mj}求证\end{CJK}: $\left\{x_{n}\right\}$ \begin{CJK}{UTF8}{mj}收敛\end{CJK}, \begin{CJK}{UTF8}{mj}且其极限\end{CJK} $\xi \in[a, b]$ \begin{CJK}{UTF8}{mj}满足\end{CJK}: $f(\xi)=\xi$.

\begin{enumerate}
  \setcounter{enumi}{4}
  \item (16 \begin{CJK}{UTF8}{mj}分\end{CJK}) \begin{CJK}{UTF8}{mj}设正项数列\end{CJK} $\left\{x_{n}\right\}$ \begin{CJK}{UTF8}{mj}单调增\end{CJK}, \begin{CJK}{UTF8}{mj}且\end{CJK} $\lim _{n \rightarrow \infty} x_{n}=+\infty$, \begin{CJK}{UTF8}{mj}证明\end{CJK}:
\end{enumerate}
$$
\sum_{n=1}^{\infty}\left(1-\frac{x_{n}}{x_{n+1}}\right)
$$
\begin{CJK}{UTF8}{mj}发散\end{CJK}.

\begin{enumerate}
  \setcounter{enumi}{5}
  \item ( 16 \begin{CJK}{UTF8}{mj}分\end{CJK}) \begin{CJK}{UTF8}{mj}已知\end{CJK} $P$ \begin{CJK}{UTF8}{mj}是\end{CJK} $\angle A O B$ \begin{CJK}{UTF8}{mj}内一固定点\end{CJK}, $\angle A O P=\alpha, \angle B O P=\beta$, \begin{CJK}{UTF8}{mj}线段长度\end{CJK} $\overline{O P}=L$, \begin{CJK}{UTF8}{mj}过\end{CJK} $P$ \begin{CJK}{UTF8}{mj}的直线交射线\end{CJK} $O A$ \begin{CJK}{UTF8}{mj}和\end{CJK} $O B$ \begin{CJK}{UTF8}{mj}于点\end{CJK} $X$ \begin{CJK}{UTF8}{mj}和\end{CJK} $Y$. \begin{CJK}{UTF8}{mj}求线段长度乘积\end{CJK} $\overline{P X} \cdot \overline{P Y}$ \begin{CJK}{UTF8}{mj}的最小值\end{CJK}, \begin{CJK}{UTF8}{mj}说明取最值时\end{CJK} $X, Y$ \begin{CJK}{UTF8}{mj}的位置\end{CJK}. 6. (16 \begin{CJK}{UTF8}{mj}分\end{CJK}) \begin{CJK}{UTF8}{mj}计算曲面积分\end{CJK}
\end{enumerate}
$$
I=\iint_{\Omega} 4 z \mathrm{~d} y \mathrm{~d} z-2 z y \mathrm{~d} z \mathrm{~d} x+\left(1-z^{2}\right) \mathrm{d} x \mathrm{~d} y
$$
\begin{CJK}{UTF8}{mj}其中\end{CJK} $\Omega$ \begin{CJK}{UTF8}{mj}是由曲线\end{CJK} $\left\{\begin{array}{c}z=e^{y} \\ x=0\end{array} \quad,(0 \leq y \leq a)\right.$ \begin{CJK}{UTF8}{mj}绕\end{CJK} $z$ \begin{CJK}{UTF8}{mj}轴旋转一周所成曲面的下侧\end{CJK}.

\begin{enumerate}
  \setcounter{enumi}{7}
  \item (16 \begin{CJK}{UTF8}{mj}分\end{CJK}) \begin{CJK}{UTF8}{mj}设\end{CJK} $f_{1}(x)=f(x)=\frac{x}{\sqrt{1+x^{2}}}, f_{n+1}(x)=f\left(f_{n}(x)\right), n=1,2, \cdots$. \begin{CJK}{UTF8}{mj}证明\end{CJK}: \begin{CJK}{UTF8}{mj}函数项级数\end{CJK}
\end{enumerate}
$$
f_{1}(x)+\sum_{n=1}^{\infty}\left(f_{n+1}(x)-f_{n}(x)\right)
$$
\begin{CJK}{UTF8}{mj}在\end{CJK} $(-\infty,+\infty)$ \begin{CJK}{UTF8}{mj}上一致收敛于零\end{CJK}.

\begin{enumerate}
  \setcounter{enumi}{8}
  \item (10 \begin{CJK}{UTF8}{mj}分\end{CJK}) \begin{CJK}{UTF8}{mj}设\end{CJK} $0<x<1$, \begin{CJK}{UTF8}{mj}求\end{CJK}
\end{enumerate}
$$
\sum_{n=0}^{\infty} \frac{x^{2^{n}}}{1-x^{2^{n+1}}}
$$
\begin{CJK}{UTF8}{mj}的和函数\end{CJK}.

\section{3. 中山大学 2012 年研究生入学考试试题数学分析 
 李扬 
 微信公众号: sxkyliyang}
\begin{enumerate}
  \item (\begin{CJK}{UTF8}{mj}每小题\end{CJK} 6 \begin{CJK}{UTF8}{mj}分\end{CJK}, \begin{CJK}{UTF8}{mj}共\end{CJK} 48 \begin{CJK}{UTF8}{mj}分\end{CJK})
\end{enumerate}
(1) \begin{CJK}{UTF8}{mj}求极限\end{CJK}
$$
\lim _{x \rightarrow 0}\left(1+x^{2} e^{x}\right)^{\frac{1}{1-\cos x}} .
$$
(2) \begin{CJK}{UTF8}{mj}给定\end{CJK} $a_{0}, a_{1}$, \begin{CJK}{UTF8}{mj}并设\end{CJK} $a_{n}=\frac{1}{2}\left(a_{n-1}+a_{n-2}\right), n \geq 2$, \begin{CJK}{UTF8}{mj}求\end{CJK}: $\lim _{n \rightarrow \infty} a_{n}$.

(3) \begin{CJK}{UTF8}{mj}求\end{CJK}
$$
I_{n}=\int_{0}^{n \pi} x|\sin x| \mathrm{d} x
$$
(4) \begin{CJK}{UTF8}{mj}设\end{CJK} $g(x), f(x, y)$ \begin{CJK}{UTF8}{mj}均二阶可微\end{CJK}, $u(x, y)=y g(\cos x)+f\left(e^{x}, x y\right)$, \begin{CJK}{UTF8}{mj}求\end{CJK} $\frac{\partial u}{\partial x}, \frac{\partial^{2} u}{\partial x \partial y}$.

(5) \begin{CJK}{UTF8}{mj}已知二椭圆抛物面为\end{CJK} $\Sigma_{1}: z=x^{2}+2 y^{2}+1, \Sigma_{2}: z=2\left(x^{2}+3 y^{2}\right)$, \begin{CJK}{UTF8}{mj}计算\end{CJK} $\Sigma_{1}$ \begin{CJK}{UTF8}{mj}被\end{CJK} $\Sigma_{2}$ \begin{CJK}{UTF8}{mj}截下部分的曲面面积\end{CJK}.

(6) \begin{CJK}{UTF8}{mj}求曲线积分\end{CJK}
$$
\oint_{C} \frac{(x+4 y) \mathrm{d} y+(x-y) \mathrm{d} x}{x^{2}+4 y^{2}} .
$$
\begin{CJK}{UTF8}{mj}其中\end{CJK} $C$ \begin{CJK}{UTF8}{mj}为以原点为圆心单位圆\end{CJK}, \begin{CJK}{UTF8}{mj}并取正向\end{CJK}.

(7) \begin{CJK}{UTF8}{mj}判断级数\end{CJK}
$$
\sum_{n=1}^{\infty} n !\left(\frac{-e}{n}\right)^{n}
$$
\begin{CJK}{UTF8}{mj}的敛散性\end{CJK}.

(8) \begin{CJK}{UTF8}{mj}设\end{CJK} $a_{n}>0, \lim _{n \rightarrow \infty} a_{n}=a>0$, \begin{CJK}{UTF8}{mj}讨论级数\end{CJK}
$$
\sum_{n=1}^{\infty}\left(\frac{a}{a_{n}}\right)^{n}
$$
\begin{CJK}{UTF8}{mj}的敛散性\end{CJK}.

\begin{enumerate}
  \setcounter{enumi}{2}
  \item (16 \begin{CJK}{UTF8}{mj}分\end{CJK}) \begin{CJK}{UTF8}{mj}给出函数\end{CJK} $f(x)=x\left[x^{-1}\right]$ \begin{CJK}{UTF8}{mj}在\end{CJK} $(0,+\infty)$ \begin{CJK}{UTF8}{mj}上的不连续点\end{CJK}, \begin{CJK}{UTF8}{mj}其中\end{CJK} $\left[x^{-1}\right]$ \begin{CJK}{UTF8}{mj}表示\end{CJK} $x^{-1}$ \begin{CJK}{UTF8}{mj}的整数部分\end{CJK}.

  \item (16 \begin{CJK}{UTF8}{mj}分\end{CJK}) \begin{CJK}{UTF8}{mj}设\end{CJK}

\end{enumerate}
$$
f(x, y)=\left\{\begin{array}{ll}
\frac{x^{\alpha} y}{x^{2}+y^{2}}, & (x, y) \neq(0,0) ; \\
0, & (x, y)=(0,0) .
\end{array} \quad, \alpha>0 .\right.
$$
\begin{CJK}{UTF8}{mj}问\end{CJK} $\alpha$ \begin{CJK}{UTF8}{mj}取何值时能使\end{CJK} $f(x, y)$ \begin{CJK}{UTF8}{mj}在点\end{CJK} $(0,0)$ \begin{CJK}{UTF8}{mj}可微\end{CJK}?

\begin{enumerate}
  \setcounter{enumi}{4}
  \item ( 16 \begin{CJK}{UTF8}{mj}分\end{CJK}) \begin{CJK}{UTF8}{mj}计算曲线积分\end{CJK}
\end{enumerate}
$$
\oint_{L}(y+1) \mathrm{d} x+(z+2) \mathrm{d} y+(x+3) \mathrm{d} z .
$$
\begin{CJK}{UTF8}{mj}其中\end{CJK} $L$ \begin{CJK}{UTF8}{mj}是球面\end{CJK} $x^{2}+y^{2}+z^{2}=R^{2}$ \begin{CJK}{UTF8}{mj}被平面\end{CJK} $x+y+z=0$ \begin{CJK}{UTF8}{mj}所截得的圆周\end{CJK}, \begin{CJK}{UTF8}{mj}从\end{CJK} $x$ \begin{CJK}{UTF8}{mj}轴正向看去\end{CJK}, $L$ \begin{CJK}{UTF8}{mj}是逆时针方向\end{CJK}.

\begin{enumerate}
  \setcounter{enumi}{5}
  \item ( 16 \begin{CJK}{UTF8}{mj}分\end{CJK}) \begin{CJK}{UTF8}{mj}讨论函数项级数\end{CJK}
\end{enumerate}
$$
\sum_{n=1}^{\infty} \frac{x^{n}}{n \ln n}
$$
\begin{CJK}{UTF8}{mj}在\end{CJK} $[0,1)$ \begin{CJK}{UTF8}{mj}上的一致收敛性\end{CJK}.

\begin{enumerate}
  \setcounter{enumi}{6}
  \item (16 \begin{CJK}{UTF8}{mj}分\end{CJK}) \begin{CJK}{UTF8}{mj}设\end{CJK} $f(x)$ \begin{CJK}{UTF8}{mj}在\end{CJK} $[a, b]$ \begin{CJK}{UTF8}{mj}上有二阶连续导数\end{CJK}, $f\left(\frac{a+b}{2}\right)=0$, \begin{CJK}{UTF8}{mj}求证\end{CJK}:
\end{enumerate}
$$
\left|\int_{a}^{b} f(x) \mathrm{d} x\right| \leq \frac{(b-a)^{3}}{24} \max _{a \leq x \leq b}\left|f^{\prime \prime}(x)\right| .
$$

\begin{enumerate}
  \setcounter{enumi}{7}
  \item (10 \begin{CJK}{UTF8}{mj}分\end{CJK}) \begin{CJK}{UTF8}{mj}证明不等式\end{CJK}
\end{enumerate}
$$
y x^{y}(1-x)<e^{-1}
$$
\begin{CJK}{UTF8}{mj}其中\end{CJK} $(x, y) \in D=\{(x, y) \mid 0<x<1, y>0\}$.

\begin{enumerate}
  \setcounter{enumi}{8}
  \item (12 \begin{CJK}{UTF8}{mj}分\end{CJK}) \begin{CJK}{UTF8}{mj}设\end{CJK} $x>a$ \begin{CJK}{UTF8}{mj}时\end{CJK} $g(x)>0, f(x)$ \begin{CJK}{UTF8}{mj}和\end{CJK} $g(x)$ \begin{CJK}{UTF8}{mj}在任意有限区间\end{CJK} $[a, b]$ \begin{CJK}{UTF8}{mj}上可积\end{CJK}, $\int_{a}^{+\infty} g(x) \mathrm{d} x$ \begin{CJK}{UTF8}{mj}发散\end{CJK}, \begin{CJK}{UTF8}{mj}且\end{CJK} $\lim _{x \rightarrow+\infty} \frac{f(x)}{g(x)}=$ 0 , \begin{CJK}{UTF8}{mj}证明\end{CJK}:
\end{enumerate}
$$
\lim _{x \rightarrow+\infty} \frac{\int_{a}^{x} f(x) \mathrm{d} x}{\int_{a}^{x} g(x) \mathrm{d} x}=0 .
$$

\section{4. 中山大学 2013 年研究生入学考试试题数学分析 
 李扬 
 微信公众号: sxkyliyang}
\begin{enumerate}
  \item ( 24 \begin{CJK}{UTF8}{mj}分\end{CJK}) \begin{CJK}{UTF8}{mj}计算下列极限\end{CJK}:
\end{enumerate}
(1) \begin{CJK}{UTF8}{mj}设\end{CJK}
$$
x_{n}=\sqrt[n]{\left[1+\left(\frac{1}{n}\right)^{2}\right]\left[1+\left(\frac{2}{n}\right)^{2}\right] \cdots\left[1+\left(\frac{n}{n}\right)^{2}\right]}
$$
\begin{CJK}{UTF8}{mj}求\end{CJK} $\lim _{n \rightarrow \infty} x_{n}$.

(2) $\lim _{n \rightarrow \infty} n^{2}\left(x^{\frac{1}{n}}-x^{\frac{1}{n+1}}\right)$, \begin{CJK}{UTF8}{mj}其中\end{CJK} $x>0$.

(3) $\lim _{m \rightarrow \infty} \frac{\left(\sum_{i=d}^{m} i^{d}\right)-\frac{m^{d+1}}{d+1}}{m^{d}}$, \begin{CJK}{UTF8}{mj}其中\end{CJK} $d>0$.

\begin{enumerate}
  \setcounter{enumi}{2}
  \item ( 20 \begin{CJK}{UTF8}{mj}分\end{CJK}) (1) \begin{CJK}{UTF8}{mj}叙述数列\end{CJK} $\left\{a_{n}\right\}$ \begin{CJK}{UTF8}{mj}收敛的柯西收敛准则并证明之\end{CJK}.
\end{enumerate}
(2) \begin{CJK}{UTF8}{mj}用柯西收敛准则证明\end{CJK}: \begin{CJK}{UTF8}{mj}数列\end{CJK} $a_{n}=\frac{1}{2 \ln 2}+\frac{1}{3 \ln 3}+\cdots+\frac{1}{n \ln n}$ \begin{CJK}{UTF8}{mj}趋于无穷大\end{CJK}.

\begin{enumerate}
  \setcounter{enumi}{3}
  \item ( 20 \begin{CJK}{UTF8}{mj}分\end{CJK}) \begin{CJK}{UTF8}{mj}证明\end{CJK}:
\end{enumerate}
(1) $f(x)=\sin \sqrt{x}$ \begin{CJK}{UTF8}{mj}在\end{CJK} $[0, \infty)$ \begin{CJK}{UTF8}{mj}上一致连续\end{CJK}.

(2) $g(x)=\sin x^{2}$ \begin{CJK}{UTF8}{mj}在\end{CJK} $[0, \infty)$ \begin{CJK}{UTF8}{mj}上不一致连续\end{CJK}.

\begin{enumerate}
  \setcounter{enumi}{4}
  \item (16 \begin{CJK}{UTF8}{mj}分\end{CJK}) \begin{CJK}{UTF8}{mj}设\end{CJK} $x_{1}=-1, x_{n+1}=-1+\frac{x_{n}^{2}}{2}(n=1,2, \cdots)$, \begin{CJK}{UTF8}{mj}证明\end{CJK} $\lim _{n \rightarrow \infty} x_{n}$ \begin{CJK}{UTF8}{mj}存在\end{CJK}.
\end{enumerate}
5 . ( 10 \begin{CJK}{UTF8}{mj}分\end{CJK}) \begin{CJK}{UTF8}{mj}设\end{CJK} $a_{n}>0, n=1,2, \cdots$, \begin{CJK}{UTF8}{mj}证明\end{CJK}
$$
\varlimsup_{n \rightarrow \infty} n\left(\frac{1+a_{n+1}}{a_{n}}-1\right) \geq 1
$$

\begin{enumerate}
  \setcounter{enumi}{6}
  \item ( 10 \begin{CJK}{UTF8}{mj}分\end{CJK}) \begin{CJK}{UTF8}{mj}设\end{CJK} $0<x<1$, \begin{CJK}{UTF8}{mj}求\end{CJK}
\end{enumerate}
$$
S(x)=\sum_{k=1}^{\infty} x^{k}(1-x)^{2 k}
$$
\begin{CJK}{UTF8}{mj}的极值\end{CJK}

\begin{enumerate}
  \setcounter{enumi}{7}
  \item ( 10 \begin{CJK}{UTF8}{mj}分\end{CJK}) \begin{CJK}{UTF8}{mj}计算\end{CJK}
\end{enumerate}
$$
\int_{C} \frac{(x+y) \mathrm{d} x-(x-y) \mathrm{d} y}{x^{2}+y^{2}} .
$$
\begin{CJK}{UTF8}{mj}其中\end{CJK} $C$ \begin{CJK}{UTF8}{mj}是一条从\end{CJK} $(-1,0)$ \begin{CJK}{UTF8}{mj}到\end{CJK} $(1,0)$ \begin{CJK}{UTF8}{mj}不经过原点的光滑曲线\end{CJK}: $y=f(x),-1 \leq x \leq 1$.

\begin{enumerate}
  \setcounter{enumi}{8}
  \item (12 \begin{CJK}{UTF8}{mj}分\end{CJK}) \begin{CJK}{UTF8}{mj}计算\end{CJK}
\end{enumerate}
$$
\iint_{S} y z \mathrm{~d} x \mathrm{~d} y+z x \mathrm{~d} y \mathrm{~d} z+x y \mathrm{~d} z \mathrm{~d} x .
$$
\begin{CJK}{UTF8}{mj}其中\end{CJK} $S$ \begin{CJK}{UTF8}{mj}是由\end{CJK} $x^{2}+y^{2}=1$,\begin{CJK}{UTF8}{mj}三个坐标平面及\end{CJK} $z=2-x^{2}-y^{2}$ \begin{CJK}{UTF8}{mj}所围立体图形在第一卦限的外侧\end{CJK}.

\begin{enumerate}
  \setcounter{enumi}{9}
  \item ( 12 \begin{CJK}{UTF8}{mj}分\end{CJK}) \begin{CJK}{UTF8}{mj}讨论级数\end{CJK}
\end{enumerate}
$$
\sum_{k=1}^{\infty} \frac{\sin k x}{k}
$$
\begin{CJK}{UTF8}{mj}在\end{CJK} $[0,2 \pi]$ \begin{CJK}{UTF8}{mj}上的一致收敛性\end{CJK}.

\begin{enumerate}
  \setcounter{enumi}{10}
  \item (16 \begin{CJK}{UTF8}{mj}分\end{CJK}) (1) \begin{CJK}{UTF8}{mj}分别将函数\end{CJK} $f(x)=\frac{\pi-x}{2}$ \begin{CJK}{UTF8}{mj}和\end{CJK} $g(x)=\left\{\begin{array}{l}(\pi-1) x, 0 \leq x \leq 1 \\ \pi-x, \quad 1<x \leq \pi\end{array}\right.$ \begin{CJK}{UTF8}{mj}在\end{CJK} $[0, \pi]$ \begin{CJK}{UTF8}{mj}按正弦\end{CJK} Fourier \begin{CJK}{UTF8}{mj}级数展开\end{CJK}.
\end{enumerate}
(2) \begin{CJK}{UTF8}{mj}证明\end{CJK}: $\sum_{n=1}^{\infty} \frac{\sin n}{n}=\sum_{n=1}^{\infty}\left(\frac{\sin n}{n}\right)^{2}$.

\section{5. 中山大学 2014 年研究生入学考试试题数学分析 
 李扬 
 微信公众号: sxkyliyang}
\begin{enumerate}
  \item ( 30 \begin{CJK}{UTF8}{mj}分\end{CJK})\begin{CJK}{UTF8}{mj}计算\end{CJK}
\end{enumerate}
(1) $\int \frac{1}{x} \sqrt{\frac{x+2}{x-2}} \mathrm{~d} x$.

(2) $\int_{\Gamma} x y \mathrm{~d} s$, \begin{CJK}{UTF8}{mj}其中\end{CJK} $\Gamma$ \begin{CJK}{UTF8}{mj}为\end{CJK} $x^{2}+y^{2}+z^{2}=a^{2}$ \begin{CJK}{UTF8}{mj}与\end{CJK} $x+y+z=0$ \begin{CJK}{UTF8}{mj}的交线\end{CJK}.

(3) $\lim _{n \rightarrow \infty}\left(\int_{0}^{\pi} x^{2013} \sin ^{n} x \mathrm{~d} x\right)^{\frac{1}{n}}$.

\begin{enumerate}
  \setcounter{enumi}{2}
  \item ( 10 \begin{CJK}{UTF8}{mj}分\end{CJK}) \begin{CJK}{UTF8}{mj}设\end{CJK} $f(x)$ \begin{CJK}{UTF8}{mj}在\end{CJK} $(-\infty,+\infty)$ \begin{CJK}{UTF8}{mj}上连续\end{CJK}, \begin{CJK}{UTF8}{mj}并且\end{CJK} $\int_{-\infty}^{+\infty} f(x) \mathrm{d} x$ \begin{CJK}{UTF8}{mj}收敛\end{CJK}, \begin{CJK}{UTF8}{mj}对任意\end{CJK} $a, b, c \in \mathbb{R}$ \begin{CJK}{UTF8}{mj}都有\end{CJK} $\int_{a}^{a+c} f(x) \mathrm{d} x=$ $\int_{b}^{b+c} f(x) \mathrm{d} x$ \begin{CJK}{UTF8}{mj}证明\end{CJK}: $f(x) \equiv 0 .$

  \item ( 15 \begin{CJK}{UTF8}{mj}分\end{CJK}) \begin{CJK}{UTF8}{mj}表格填空\end{CJK}:

\end{enumerate}
\begin{tabular}{|c|l|l|l|}
\hline
$\sum_{n=3}^{\infty} a^{n} n^{b}(\ln n)^{c}$ & \begin{CJK}{UTF8}{mj}绝对收敛\end{CJK} & \begin{CJK}{UTF8}{mj}条件收敛\end{CJK} & \begin{CJK}{UTF8}{mj}发散\end{CJK} \\
\hline
\begin{CJK}{UTF8}{mj}参数\end{CJK} $a, b, c$ \begin{CJK}{UTF8}{mj}的取值范围\end{CJK} &  &  &  \\
\hline
\end{tabular}

\begin{enumerate}
  \setcounter{enumi}{4}
  \item (10 \begin{CJK}{UTF8}{mj}分\end{CJK}) \begin{CJK}{UTF8}{mj}求方程组\end{CJK}
\end{enumerate}
$$
\left\{\begin{array}{l}
u+v=x+y \\
\frac{\sin u}{\sin v}=\frac{x}{y}
\end{array}\right.
$$
\begin{CJK}{UTF8}{mj}所确定的隐函数\end{CJK} $\left\{\begin{array}{l}u=u(x, y) \\ v=v(x, y)\end{array}\right.$ \begin{CJK}{UTF8}{mj}的微分\end{CJK} $\mathrm{d} u, \mathrm{~d} v$.

\begin{enumerate}
  \setcounter{enumi}{5}
  \item (10 \begin{CJK}{UTF8}{mj}分\end{CJK}) \begin{CJK}{UTF8}{mj}讨论广义积分\end{CJK}
\end{enumerate}
$$
\int_{0}^{\infty} \sin x^{2} \mathrm{~d} x ; \int_{0}^{\infty} \int_{0}^{\infty} \sin \left(x^{2}+y^{2}\right) \mathrm{d} x \mathrm{~d} y
$$
\begin{CJK}{UTF8}{mj}的敛散性\end{CJK}. \begin{CJK}{UTF8}{mj}其中\end{CJK} $\int_{0}^{\infty} \int_{0}^{\infty} \sin \left(x^{2}+y^{2}\right) \mathrm{d} x \mathrm{~d} y=\lim _{r \rightarrow \infty} \underset{x^{2}+y^{2} \leq r^{2} ; x, y \geq 0}{ } \iint_{\cos } \sin \left(x^{2}+y^{2}\right) \mathrm{d} x \mathrm{~d} y$.

\begin{enumerate}
  \setcounter{enumi}{6}
  \item (15 \begin{CJK}{UTF8}{mj}分\end{CJK}) \begin{CJK}{UTF8}{mj}讨论函数\end{CJK}
\end{enumerate}
$$
f(x, y)= \begin{cases}x^{\frac{3}{2}} \sin \frac{y}{x}, & x \neq 0 \\ 0, & x=0\end{cases}
$$
\begin{CJK}{UTF8}{mj}的可微性\end{CJK}.

\begin{enumerate}
  \setcounter{enumi}{7}
  \item (15 \begin{CJK}{UTF8}{mj}分\end{CJK}) \begin{CJK}{UTF8}{mj}讨论积分\end{CJK}
\end{enumerate}
$$
f(x)=\int_{0}^{\infty} e^{-x\left(t+\frac{1}{t}\right)} \mathrm{d} t
$$
\begin{CJK}{UTF8}{mj}的收敛域及\end{CJK} $f(x)$ \begin{CJK}{UTF8}{mj}的连续性\end{CJK}.

\begin{enumerate}
  \setcounter{enumi}{8}
  \item (10 \begin{CJK}{UTF8}{mj}分\end{CJK}) \begin{CJK}{UTF8}{mj}半径为\end{CJK} $r$ \begin{CJK}{UTF8}{mj}的球的中心在单位球\end{CJK} $x^{2}+y^{2}+z^{2}=1$ \begin{CJK}{UTF8}{mj}的表面上\end{CJK}, \begin{CJK}{UTF8}{mj}问\end{CJK} $r$ \begin{CJK}{UTF8}{mj}取何值时该球位于单位球内部分的表\end{CJK} \begin{CJK}{UTF8}{mj}面积最大\end{CJK}?

  \item (15 \begin{CJK}{UTF8}{mj}分\end{CJK}) \begin{CJK}{UTF8}{mj}设\end{CJK} $a>0$, \begin{CJK}{UTF8}{mj}求\end{CJK} $y^{2}=\frac{x^{3}}{2 a-x}$ \begin{CJK}{UTF8}{mj}与\end{CJK} $x=2 a$ \begin{CJK}{UTF8}{mj}所围成的面积\end{CJK}. 10. (15 \begin{CJK}{UTF8}{mj}分\end{CJK}) \begin{CJK}{UTF8}{mj}讨论\end{CJK}

\end{enumerate}
$$
f(x)=x \sin x
$$
\begin{CJK}{UTF8}{mj}在\end{CJK} $[1, \infty)$ \begin{CJK}{UTF8}{mj}上是否一致连续\end{CJK}, \begin{CJK}{UTF8}{mj}并说明理由\end{CJK}.

\begin{enumerate}
  \setcounter{enumi}{11}
  \item (15 \begin{CJK}{UTF8}{mj}分\end{CJK}) \begin{CJK}{UTF8}{mj}设\end{CJK} $f(x)=\int_{x}^{x^{2}}\left(1+\frac{1}{2 t}\right)^{t}\left(e^{\frac{1}{\sqrt{t}}}-1\right) \mathrm{d} t,(x>0)$. \begin{CJK}{UTF8}{mj}求\end{CJK}
\end{enumerate}
$$
\lim _{n \rightarrow \infty} f(n) \sin \frac{1}{n}
$$

\section{6. 中山大学 2015 年研究生入学考试试题数学分析 
 李扬 
 微信公众号: sxkyliyang}
\begin{enumerate}
  \item (\begin{CJK}{UTF8}{mj}每小题\end{CJK} 15 \begin{CJK}{UTF8}{mj}分\end{CJK}, \begin{CJK}{UTF8}{mj}共\end{CJK} 60 \begin{CJK}{UTF8}{mj}分\end{CJK}) \begin{CJK}{UTF8}{mj}计算下列各题\end{CJK}:
\end{enumerate}
(1) \begin{CJK}{UTF8}{mj}求极限\end{CJK}
$$
\lim _{x \rightarrow 0}\left(\frac{\sin x}{x}\right)^{\frac{1}{1-\cos x}}
$$
(2) \begin{CJK}{UTF8}{mj}已知\end{CJK} $y=e^{x} \cdot x^{\sin x}$, \begin{CJK}{UTF8}{mj}求\end{CJK} $\frac{\mathrm{d} y}{\mathrm{~d} x}$.

(3) \begin{CJK}{UTF8}{mj}设\end{CJK} $f(x, y, z)=x y^{2} z^{3}$, \begin{CJK}{UTF8}{mj}且\end{CJK} $z=z(x, y)$ \begin{CJK}{UTF8}{mj}是由方程\end{CJK} $x^{2}+y^{2}+z^{2}-3 x y z=0$ \begin{CJK}{UTF8}{mj}所确定的隐函数\end{CJK}, \begin{CJK}{UTF8}{mj}求\end{CJK} $f_{x}(1,1,1)$.

(4) \begin{CJK}{UTF8}{mj}求\end{CJK} $u=x-2 y+2 z$ \begin{CJK}{UTF8}{mj}在条件\end{CJK} $x^{2}+y^{2}+z^{2}=1$ \begin{CJK}{UTF8}{mj}下的条件极值\end{CJK} (\begin{CJK}{UTF8}{mj}含极大值和极小值\end{CJK}).

\begin{enumerate}
  \setcounter{enumi}{2}
  \item ( 15 \begin{CJK}{UTF8}{mj}分\end{CJK}) \begin{CJK}{UTF8}{mj}将\end{CJK} $\frac{1}{x^{2}+4 x+3}$ \begin{CJK}{UTF8}{mj}展开为\end{CJK} $x-1$ \begin{CJK}{UTF8}{mj}的幂级数\end{CJK}.

  \item ( 15 \begin{CJK}{UTF8}{mj}分\end{CJK}) \begin{CJK}{UTF8}{mj}求抛物面\end{CJK}

\end{enumerate}
$$
z=x^{2}+y^{2}
$$
\begin{CJK}{UTF8}{mj}与抛物柱面\end{CJK}
$$
y=x^{2}
$$
\begin{CJK}{UTF8}{mj}的交线上的点\end{CJK} $P(1,1,2)$ \begin{CJK}{UTF8}{mj}处的切线方程和平面方程\end{CJK}.

\begin{enumerate}
  \setcounter{enumi}{4}
  \item (15 \begin{CJK}{UTF8}{mj}分\end{CJK}) \begin{CJK}{UTF8}{mj}计算三重积分\end{CJK}
\end{enumerate}
$$
\iiint_{\Omega}(x+y+z) \mathrm{d} x \mathrm{~d} y \mathrm{~d} z .
$$
\begin{CJK}{UTF8}{mj}其中\end{CJK} $\Omega$ \begin{CJK}{UTF8}{mj}是由平面\end{CJK} $x+y+z=1$ \begin{CJK}{UTF8}{mj}与三个坐标面围成的区域\end{CJK}.

\begin{enumerate}
  \setcounter{enumi}{5}
  \item ( 20 \begin{CJK}{UTF8}{mj}分\end{CJK}) \begin{CJK}{UTF8}{mj}如下图所示\end{CJK}, \begin{CJK}{UTF8}{mj}设当\end{CJK} $0 \leq x \leq a$ \begin{CJK}{UTF8}{mj}时\end{CJK}, \begin{CJK}{UTF8}{mj}直线\end{CJK} $y=a x(0<a<1)$ \begin{CJK}{UTF8}{mj}与抛物线\end{CJK} $y=x^{2}$ \begin{CJK}{UTF8}{mj}所围成的图形面积为\end{CJK} $S_{1}$. \begin{CJK}{UTF8}{mj}当\end{CJK} $a \leq x \leq 1$ \begin{CJK}{UTF8}{mj}时\end{CJK}, \begin{CJK}{UTF8}{mj}直线\end{CJK} $y=a x(0<a<1)$, \begin{CJK}{UTF8}{mj}抛物线\end{CJK} $y=x^{2}$ \begin{CJK}{UTF8}{mj}与直线\end{CJK} $x=1$ \begin{CJK}{UTF8}{mj}所围成的图形的面积为\end{CJK} $S_{2}$.
\end{enumerate}
\includegraphics[max width=\textwidth]{2022_04_18_3416d289b173eb9de8c1g-226}

(1) \begin{CJK}{UTF8}{mj}确定\end{CJK} $a$ \begin{CJK}{UTF8}{mj}的值\end{CJK}, \begin{CJK}{UTF8}{mj}使\end{CJK} $S=S_{1}+S_{2}$ \begin{CJK}{UTF8}{mj}达到最小\end{CJK}, \begin{CJK}{UTF8}{mj}并求出最小值\end{CJK}.

(2) \begin{CJK}{UTF8}{mj}求当\end{CJK} $S=S_{1}+S_{2}$ \begin{CJK}{UTF8}{mj}达到最小值时\end{CJK}, \begin{CJK}{UTF8}{mj}该图形绕\end{CJK} $x$ \begin{CJK}{UTF8}{mj}轴旋转一周所得到的旋转体的体积\end{CJK}.

\begin{enumerate}
  \setcounter{enumi}{6}
  \item ( 25 \begin{CJK}{UTF8}{mj}分\end{CJK}) \begin{CJK}{UTF8}{mj}设\end{CJK} $f(x)$ \begin{CJK}{UTF8}{mj}在\end{CJK} $(-\infty,+\infty)$ \begin{CJK}{UTF8}{mj}内有一阶连续导数\end{CJK}, $L$ \begin{CJK}{UTF8}{mj}是上半平面\end{CJK} $y>0$ \begin{CJK}{UTF8}{mj}内的有向分段光滑曲线\end{CJK}, \begin{CJK}{UTF8}{mj}其起点为\end{CJK} $(1,4)$, \begin{CJK}{UTF8}{mj}终点为\end{CJK} $(2,2)$, \begin{CJK}{UTF8}{mj}记\end{CJK}
\end{enumerate}
$$
I=\int_{L} \frac{1}{y}\left[1+y^{2} f(x y)\right] \mathrm{d} x+\frac{x}{y^{2}}\left[y^{2} f(x y)-1\right] \mathrm{d} y
$$
(1) \begin{CJK}{UTF8}{mj}证明曲线积分\end{CJK} $I$ \begin{CJK}{UTF8}{mj}与路径无关\end{CJK};

(2) \begin{CJK}{UTF8}{mj}求曲线积分\end{CJK} $I$ \begin{CJK}{UTF8}{mj}的值\end{CJK}.

\section{7. 中山大学 2016 年研究生入学考试试题数学分析 
 李扬 
 微信公众号: sxkyliyang}
\begin{enumerate}
  \item (\begin{CJK}{UTF8}{mj}每小题\end{CJK} 8 \begin{CJK}{UTF8}{mj}分\end{CJK}, \begin{CJK}{UTF8}{mj}共\end{CJK} 56 \begin{CJK}{UTF8}{mj}分\end{CJK}) \begin{CJK}{UTF8}{mj}解答下面各题\end{CJK}
\end{enumerate}
(1) \begin{CJK}{UTF8}{mj}求极限\end{CJK}
$$
\lim _{x \rightarrow 0}\left(\frac{1}{x}-\frac{1}{\sin x}\right) \frac{1}{\sin x}
$$
(2) \begin{CJK}{UTF8}{mj}求极限\end{CJK}
$$
\lim _{n \rightarrow \infty}(n !)^{\frac{1}{n^{2}}}
$$
(3) \begin{CJK}{UTF8}{mj}设\end{CJK} $y=x^{2} \cos 3 x$, \begin{CJK}{UTF8}{mj}求\end{CJK} $y^{(50)}(x)$.

(4) \begin{CJK}{UTF8}{mj}求曲线\end{CJK}
$$
y=\ln \left(1-x^{2}\right), 0 \leq x \leq \frac{1}{2}
$$
\begin{CJK}{UTF8}{mj}的弧长\end{CJK}.

(5) \begin{CJK}{UTF8}{mj}计算\end{CJK}
$$
\int_{0}^{2} \mathrm{~d} x \int_{x}^{2} e^{-y^{2}} \mathrm{~d} y
$$
(6) \begin{CJK}{UTF8}{mj}求幂级数\end{CJK}
$$
\sum_{n=0}^{\infty} \frac{n^{2}-1}{2^{n}} x^{n}
$$
\begin{CJK}{UTF8}{mj}的收敛区间与和函数\end{CJK}.

(7) \begin{CJK}{UTF8}{mj}设\end{CJK} $f(r)$ \begin{CJK}{UTF8}{mj}是定义在\end{CJK} $[0,1]$ \begin{CJK}{UTF8}{mj}上的单调递减连续函数\end{CJK}, \begin{CJK}{UTF8}{mj}定义\end{CJK}
$$
F(t)=\frac{3}{4 \pi t^{3}} \iiint_{x^{2}+y^{2}+z^{2} \leq t^{2}} f\left(\sqrt{x^{2}+y^{2}+z^{2}}\right) \mathrm{d} x \mathrm{~d} y \mathrm{~d} z, t \in(0,1] .
$$
\begin{CJK}{UTF8}{mj}求\end{CJK} $F(t)$ \begin{CJK}{UTF8}{mj}在\end{CJK} $(0,1]$ \begin{CJK}{UTF8}{mj}中的最小值\end{CJK}.

\begin{enumerate}
  \setcounter{enumi}{2}
  \item (10 \begin{CJK}{UTF8}{mj}分\end{CJK}) \begin{CJK}{UTF8}{mj}证明\end{CJK}: (1) \begin{CJK}{UTF8}{mj}级数\end{CJK}
\end{enumerate}
$$
\sum_{n=1}^{\infty} \frac{(-1)^{n} x^{2}}{\left(1+x^{2}\right)^{n}}
$$
\begin{CJK}{UTF8}{mj}在\end{CJK} $[-1,1]$ \begin{CJK}{UTF8}{mj}上一致收敛\end{CJK}.

(2) \begin{CJK}{UTF8}{mj}级数\end{CJK}
$$
\sum_{n=1}^{\infty} \frac{x^{2}}{\left(1+x^{2}\right)^{n}}
$$
\begin{CJK}{UTF8}{mj}在\end{CJK} $[-1,1]$ \begin{CJK}{UTF8}{mj}上不一致收敛\end{CJK}.

\begin{enumerate}
  \setcounter{enumi}{3}
  \item (10 \begin{CJK}{UTF8}{mj}分\end{CJK}) \begin{CJK}{UTF8}{mj}设函数\end{CJK} $f(x)$ \begin{CJK}{UTF8}{mj}在\end{CJK} $[0,1]$ \begin{CJK}{UTF8}{mj}上二阶可导且\end{CJK} $f^{\prime \prime}(x) \leq 0$. \begin{CJK}{UTF8}{mj}证明\end{CJK}:
\end{enumerate}
$$
\int_{0}^{1} f(x) \mathrm{d} x \leq f\left(\frac{1}{2}\right)
$$

\begin{enumerate}
  \setcounter{enumi}{4}
  \item (10 \begin{CJK}{UTF8}{mj}分\end{CJK}) \begin{CJK}{UTF8}{mj}函数\end{CJK}
\end{enumerate}
$$
f(x)=x^{\frac{1}{8}} \sin x
$$
\begin{CJK}{UTF8}{mj}在\end{CJK} $[0,+\infty)$ \begin{CJK}{UTF8}{mj}上是否一致连续\end{CJK}? \begin{CJK}{UTF8}{mj}试说明理由\end{CJK}.

\begin{enumerate}
  \setcounter{enumi}{5}
  \item (10 \begin{CJK}{UTF8}{mj}分\end{CJK}) \begin{CJK}{UTF8}{mj}设函数\end{CJK} $f(x)$ \begin{CJK}{UTF8}{mj}在\end{CJK} $[0,1]$ \begin{CJK}{UTF8}{mj}上可导且导函数连续\end{CJK}, \begin{CJK}{UTF8}{mj}证明\end{CJK}:
\end{enumerate}
$$
\lim _{n \rightarrow \infty} n \int_{0}^{1} x^{n} f(x) \mathrm{d} x=f(1) .
$$

\begin{enumerate}
  \setcounter{enumi}{6}
  \item (10 \begin{CJK}{UTF8}{mj}分\end{CJK}) \begin{CJK}{UTF8}{mj}设\end{CJK} $D$ \begin{CJK}{UTF8}{mj}是两条直线\end{CJK} $y=x, y=4 x$ \begin{CJK}{UTF8}{mj}和两条双曲线\end{CJK} $x y=1, x y=4$ \begin{CJK}{UTF8}{mj}所围成的区域\end{CJK}, $F(u)$ \begin{CJK}{UTF8}{mj}是具有连续导数\end{CJK} \begin{CJK}{UTF8}{mj}的一元函数\end{CJK}, \begin{CJK}{UTF8}{mj}记\end{CJK} $f(u)=F^{\prime}(u)$, \begin{CJK}{UTF8}{mj}证明\end{CJK}:
\end{enumerate}
$$
\oint_{\partial D} \frac{F(x y)}{y} \mathrm{~d} y=\int_{1}^{4} f(u) \mathrm{d} u \cdot \ln 2 .
$$
\begin{CJK}{UTF8}{mj}其中\end{CJK} $\partial D$ \begin{CJK}{UTF8}{mj}表示\end{CJK} $D$ \begin{CJK}{UTF8}{mj}的边界\end{CJK}, $\partial D$ \begin{CJK}{UTF8}{mj}的方向为逆时针方向\end{CJK}.

\begin{enumerate}
  \setcounter{enumi}{7}
  \item ( 10 \begin{CJK}{UTF8}{mj}分\end{CJK}) \begin{CJK}{UTF8}{mj}函数\end{CJK}
\end{enumerate}
$$
f(x, y)= \begin{cases}\left(1-\cos \frac{x^{2}}{y}\right) \sqrt{x^{2}+y^{2}}, & y \neq 0 \\ 0, & y=0 .\end{cases}
$$
$f(x, y)$ \begin{CJK}{UTF8}{mj}在\end{CJK} $(0,0)$ \begin{CJK}{UTF8}{mj}点可微吗\end{CJK}? \begin{CJK}{UTF8}{mj}证明你的结论\end{CJK}.

\begin{enumerate}
  \setcounter{enumi}{8}
  \item ( 10 \begin{CJK}{UTF8}{mj}分\end{CJK}) \begin{CJK}{UTF8}{mj}设\end{CJK} $x_{0}=1, x_{n+1}=\frac{3+2 x_{n}}{3+x_{n}}, n \geq 0$. \begin{CJK}{UTF8}{mj}证明\end{CJK}: \begin{CJK}{UTF8}{mj}序列\end{CJK} $\left\{x_{n}\right\}$ \begin{CJK}{UTF8}{mj}收敛并求其极限\end{CJK}.

  \item (10 \begin{CJK}{UTF8}{mj}分\end{CJK}) \begin{CJK}{UTF8}{mj}求第一类曲面积分\end{CJK}

\end{enumerate}
$$
\iint_{\Sigma} z \mathrm{~d} S .
$$
\begin{CJK}{UTF8}{mj}其中\end{CJK} $\Sigma$ \begin{CJK}{UTF8}{mj}是球面\end{CJK} $x^{2}+y^{2}+z^{2}=4$ \begin{CJK}{UTF8}{mj}被平面\end{CJK} $z=1$ \begin{CJK}{UTF8}{mj}截出的顶部\end{CJK}.

\begin{enumerate}
  \setcounter{enumi}{10}
  \item ( 8 \begin{CJK}{UTF8}{mj}分\end{CJK}) \begin{CJK}{UTF8}{mj}一点\end{CJK} $A$ \begin{CJK}{UTF8}{mj}位于半径为\end{CJK} $a$ \begin{CJK}{UTF8}{mj}的圆内\end{CJK}, \begin{CJK}{UTF8}{mj}它到圆心的距离为\end{CJK} $b$, \begin{CJK}{UTF8}{mj}试计算从\end{CJK} $A$ \begin{CJK}{UTF8}{mj}向圆的所有切线作垂线\end{CJK}, \begin{CJK}{UTF8}{mj}其垂足的轨\end{CJK} \begin{CJK}{UTF8}{mj}迹所包围的面积\end{CJK}.

  \item (6 \begin{CJK}{UTF8}{mj}分\end{CJK}) \begin{CJK}{UTF8}{mj}如果存在数列\end{CJK} $\left\{x_{n}\right\}$ \begin{CJK}{UTF8}{mj}的子列\end{CJK} $\left\{x_{n_{k}}\right\}$ \begin{CJK}{UTF8}{mj}使得\end{CJK} $\lim _{k \rightarrow \infty} x_{n_{k}}=a$, \begin{CJK}{UTF8}{mj}则称\end{CJK} $a$ \begin{CJK}{UTF8}{mj}为数列\end{CJK} $\left\{x_{n}\right\}$ \begin{CJK}{UTF8}{mj}的极限点\end{CJK}. \begin{CJK}{UTF8}{mj}设数列\end{CJK} $\left\{x_{n}\right\}$ \begin{CJK}{UTF8}{mj}有\end{CJK} \begin{CJK}{UTF8}{mj}界且\end{CJK} $\lim _{n \rightarrow \infty}\left(x_{n+1}-x_{n}\right)=0$, \begin{CJK}{UTF8}{mj}证明\end{CJK}: \begin{CJK}{UTF8}{mj}当\end{CJK} $\left\{x_{n}\right\}$ \begin{CJK}{UTF8}{mj}不收敛时其极限点集为有界闭区间\end{CJK}.

\end{enumerate}
\section{8. 中山大学 2017 年研究生入学考试试题数学分析 
 李扬 
 微信公众号: sxkyliyang}
\begin{enumerate}
  \item (\begin{CJK}{UTF8}{mj}每小题\end{CJK} 9 \begin{CJK}{UTF8}{mj}分\end{CJK}, \begin{CJK}{UTF8}{mj}共\end{CJK} 54 \begin{CJK}{UTF8}{mj}分\end{CJK}) \begin{CJK}{UTF8}{mj}计算题\end{CJK}: \begin{CJK}{UTF8}{mj}请写出必要的计算过程\end{CJK}.
\end{enumerate}
(1) $\lim _{x \rightarrow+\infty}\left(\cos \frac{1}{x}\right)^{x^{2}}$.

(2) $\lim _{n \rightarrow+\infty} \sum_{k=1}^{n} \frac{1}{n} \sin \frac{k \pi}{n}$.

(3) $\iint_{x^{2}+y^{2} \leq 1} e^{-x^{2}-y^{2}} \mathrm{~d} x \mathrm{~d} y$.

(4) $\int_{0}^{2} \mathrm{~d} y \int_{\frac{y}{2}}^{1} x^{3} \cos \left(x^{5}\right) \mathrm{d} x$.

(5) $\oint_{C} y \mathrm{~d} x+z \mathrm{~d} y+x \mathrm{~d} z$, \begin{CJK}{UTF8}{mj}其中\end{CJK} $C$ \begin{CJK}{UTF8}{mj}为球面\end{CJK} $x^{2}+y^{2}+z^{2}=a^{2}$ \begin{CJK}{UTF8}{mj}和平面\end{CJK} $x+y+z=0$ \begin{CJK}{UTF8}{mj}的交线\end{CJK}, \begin{CJK}{UTF8}{mj}从\end{CJK} $o x$ \begin{CJK}{UTF8}{mj}轴正向看沿逆\end{CJK} \begin{CJK}{UTF8}{mj}时针方向\end{CJK}.

(6) \begin{CJK}{UTF8}{mj}求级数\end{CJK}
$$
\sum_{n=1}^{\infty} \frac{(2 n-1)^{2}}{n !} x^{2 n-1}
$$
\begin{CJK}{UTF8}{mj}的和函数\end{CJK}.

\begin{enumerate}
  \setcounter{enumi}{2}
  \item (10 \begin{CJK}{UTF8}{mj}分\end{CJK}) \begin{CJK}{UTF8}{mj}判断下列函数是否在\end{CJK} $(0,+\infty)$ \begin{CJK}{UTF8}{mj}上一致连续\end{CJK}, \begin{CJK}{UTF8}{mj}并说明理由\end{CJK}.
\end{enumerate}
(1) $f(x)=\sqrt{x} \ln x$;

(2) $f(x)=x \ln x$.

\begin{enumerate}
  \setcounter{enumi}{3}
  \item ( 10 \begin{CJK}{UTF8}{mj}分\end{CJK}) \begin{CJK}{UTF8}{mj}如果\end{CJK} $u_{n}>0, n=1,2, \cdots$, \begin{CJK}{UTF8}{mj}为单调递增数列\end{CJK}. \begin{CJK}{UTF8}{mj}证明\end{CJK}: \begin{CJK}{UTF8}{mj}级数\end{CJK}
\end{enumerate}
$$
\sum_{n=1}^{\infty}\left(1-\frac{u_{n}}{u_{n+1}}\right)
$$
\begin{CJK}{UTF8}{mj}当\end{CJK} $u_{n}$ \begin{CJK}{UTF8}{mj}有界时收敛\end{CJK}, \begin{CJK}{UTF8}{mj}而当\end{CJK} $u_{n}$ \begin{CJK}{UTF8}{mj}无界时发散\end{CJK}.

\begin{enumerate}
  \setcounter{enumi}{4}
  \item (10 \begin{CJK}{UTF8}{mj}分\end{CJK}) \begin{CJK}{UTF8}{mj}求证\end{CJK}: \begin{CJK}{UTF8}{mj}方程\end{CJK}
\end{enumerate}
$$
e^{x}=a x^{2}+b x+c
$$
\begin{CJK}{UTF8}{mj}的根不超过三个\end{CJK}.

\begin{enumerate}
  \setcounter{enumi}{5}
  \item (10 \begin{CJK}{UTF8}{mj}分\end{CJK}) $f(x)$ \begin{CJK}{UTF8}{mj}在\end{CJK} $[a, b]$ \begin{CJK}{UTF8}{mj}上连续\end{CJK}, \begin{CJK}{UTF8}{mj}在\end{CJK} $(a, b)$ \begin{CJK}{UTF8}{mj}上右导数存在\end{CJK}, \begin{CJK}{UTF8}{mj}且\end{CJK} $f(a)=f(b)$. \begin{CJK}{UTF8}{mj}求证\end{CJK}: \begin{CJK}{UTF8}{mj}存在\end{CJK} $\xi \in(a, b)$, \begin{CJK}{UTF8}{mj}使得\end{CJK} $f_{+}^{\prime}(\xi) \leq 0$.

  \item (10 \begin{CJK}{UTF8}{mj}分\end{CJK}) \begin{CJK}{UTF8}{mj}判别广义积分\end{CJK}

\end{enumerate}
$$
\int_{0}^{+\infty} \frac{\ln (1+x)}{x^{p}} \mathrm{~d} x
$$
\begin{CJK}{UTF8}{mj}的收敛性\end{CJK}, \begin{CJK}{UTF8}{mj}并说明理由\end{CJK}.

\begin{enumerate}
  \setcounter{enumi}{7}
  \item ( 10 \begin{CJK}{UTF8}{mj}分\end{CJK}) \begin{CJK}{UTF8}{mj}讨论函数项级数\end{CJK}
\end{enumerate}
$$
\sum_{n=1}^{\infty} \frac{x^{2}}{\left(1+x^{2}\right)^{n}}
$$
\begin{CJK}{UTF8}{mj}在\end{CJK} $(-\infty,+\infty)$ \begin{CJK}{UTF8}{mj}上的一致收敛性\end{CJK}.

\begin{enumerate}
  \setcounter{enumi}{8}
  \item (10 \begin{CJK}{UTF8}{mj}分\end{CJK}) \begin{CJK}{UTF8}{mj}把函数\end{CJK} $f(x)=(x-\pi)^{2}$ \begin{CJK}{UTF8}{mj}在\end{CJK} $(0, \pi)$ \begin{CJK}{UTF8}{mj}上展开成余弦级数\end{CJK}, \begin{CJK}{UTF8}{mj}并求级数\end{CJK} $\sum_{n=1}^{\infty} \frac{1}{n^{2}}$ \begin{CJK}{UTF8}{mj}的和\end{CJK}. 9. ( 10 \begin{CJK}{UTF8}{mj}分\end{CJK}) \begin{CJK}{UTF8}{mj}计算\end{CJK}
\end{enumerate}
$$
\iint_{S}\left(z^{2}+x\right) \mathrm{d} y \mathrm{~d} z-z \mathrm{~d} x \mathrm{~d} y .
$$
\begin{CJK}{UTF8}{mj}其中\end{CJK} $S$ \begin{CJK}{UTF8}{mj}为曲面\end{CJK} $z=\frac{x^{2}+y^{2}}{2}, 0 \leq z \leq 2$ \begin{CJK}{UTF8}{mj}下侧\end{CJK}.

\begin{enumerate}
  \setcounter{enumi}{10}
  \item (10 \begin{CJK}{UTF8}{mj}分\end{CJK}) \begin{CJK}{UTF8}{mj}设\end{CJK} $f(x)>0, x \in[0,1]$, \begin{CJK}{UTF8}{mj}证明\end{CJK}:
\end{enumerate}
$$
\iint_{[0,1] \times[0,1]} \frac{f(x)}{f(y)} \mathrm{d} x \mathrm{~d} y \geq 1
$$

\begin{enumerate}
  \setcounter{enumi}{11}
  \item (6 \begin{CJK}{UTF8}{mj}分\end{CJK}) \begin{CJK}{UTF8}{mj}设\end{CJK} $\left\{p_{n}(x)\right\}$ \begin{CJK}{UTF8}{mj}为多项式序列\end{CJK}. \begin{CJK}{UTF8}{mj}若级数\end{CJK}
\end{enumerate}
$$
p_{1}(x)+\sum_{n=1}^{\infty}\left(p_{n+1}(x)-p_{n}(x)\right)
$$
\begin{CJK}{UTF8}{mj}在\end{CJK} $(-\infty,+\infty)$ \begin{CJK}{UTF8}{mj}上一致收敛于\end{CJK} $f(x)$. \begin{CJK}{UTF8}{mj}证明\end{CJK}: $f(x)$ \begin{CJK}{UTF8}{mj}必为一多项式\end{CJK}.

\section{第 1 章 数学分析}
\section{$1.11996$ 年}
1.(10 \begin{CJK}{UTF8}{mj}分\end{CJK}) \begin{CJK}{UTF8}{mj}证明\end{CJK}: \begin{CJK}{UTF8}{mj}若\end{CJK}
$$
x_{n} \leqslant z_{n} \leqslant y_{n}, \quad \lim _{n \rightarrow \infty} z_{n}=r, \quad \lim _{n \rightarrow \infty}\left(x_{n}-y_{n}\right)=0,
$$
\begin{CJK}{UTF8}{mj}则\end{CJK}
$$
\lim _{n \rightarrow \infty} x_{n}=\lim _{n \rightarrow \infty} y_{n}=r .
$$
2.(12 \begin{CJK}{UTF8}{mj}分\end{CJK}) \begin{CJK}{UTF8}{mj}证明\end{CJK}: \begin{CJK}{UTF8}{mj}若\end{CJK} $f(x)$ \begin{CJK}{UTF8}{mj}在\end{CJK} $[a,+\infty)$ \begin{CJK}{UTF8}{mj}上连续\end{CJK}, \begin{CJK}{UTF8}{mj}在\end{CJK} $(a,+\infty)$ \begin{CJK}{UTF8}{mj}内可导\end{CJK}, \begin{CJK}{UTF8}{mj}且\end{CJK}
$$
f(a)<0, \quad f^{\prime}(x) \geqslant K>0(x>a, K \text { 为常数) }
$$
\begin{CJK}{UTF8}{mj}则\end{CJK} $f(x)$ \begin{CJK}{UTF8}{mj}在\end{CJK} $(a,+\infty)$ \begin{CJK}{UTF8}{mj}内有县仅韦一个零点\end{CJK}.

$3 .(12$ \begin{CJK}{UTF8}{mj}分\end{CJK}) \begin{CJK}{UTF8}{mj}设\end{CJK}
$$
f(t)=\left(\int_{0}^{t} \mathrm{e}^{-x^{2}} \mathrm{~d} x\right)^{2}, \quad g(t)=\int_{0}^{1} \frac{\mathrm{e}^{-t^{2}\left(1+x^{2}\right)}}{1+x^{2}} \mathrm{~d} x
$$
\begin{CJK}{UTF8}{mj}试证\end{CJK}:
$$
f(t)+g(t) \equiv \frac{\pi}{4} .
$$
4.(10 \begin{CJK}{UTF8}{mj}分\end{CJK}) \begin{CJK}{UTF8}{mj}用\end{CJK} Lagrange \begin{CJK}{UTF8}{mj}乘数法证明\end{CJK}: \begin{CJK}{UTF8}{mj}以\end{CJK} $a, b, c, d$ \begin{CJK}{UTF8}{mj}为边长的凸四边形\end{CJK}, \begin{CJK}{UTF8}{mj}当它的面积最大时\end{CJK}, \begin{CJK}{UTF8}{mj}四顶点\end{CJK} \begin{CJK}{UTF8}{mj}共圆\end{CJK}.

5.(12 \begin{CJK}{UTF8}{mj}分\end{CJK}) \begin{CJK}{UTF8}{mj}设\end{CJK} $f(x, y, z)=\sqrt{x^{2}+y^{2}+z^{2}}, \Omega \subset \mathbb{R}^{3}$ \begin{CJK}{UTF8}{mj}由\end{CJK} $z \geqslant \sqrt{x^{2}+y^{2}}$ \begin{CJK}{UTF8}{mj}和\end{CJK} $4 \leqslant x^{2}+y^{2}+z^{2} \leqslant 16$ \begin{CJK}{UTF8}{mj}所确定\end{CJK}. \begin{CJK}{UTF8}{mj}试计算函数\end{CJK} $f$ \begin{CJK}{UTF8}{mj}关于区域\end{CJK} $\Omega$ \begin{CJK}{UTF8}{mj}的积分平均值\end{CJK}:
$$
M=\frac{1}{V_{\Omega}} \iiint_{\Omega} f(x, y, z) \mathrm{d} x \mathrm{~d} y \mathrm{~d} z,
$$
\begin{CJK}{UTF8}{mj}其中\end{CJK} $V_{\Omega}$ \begin{CJK}{UTF8}{mj}是\end{CJK} $\Omega$ \begin{CJK}{UTF8}{mj}的体积\end{CJK}.

\begin{enumerate}
  \setcounter{enumi}{6}
  \item (20 \begin{CJK}{UTF8}{mj}分\end{CJK}) \begin{CJK}{UTF8}{mj}设\end{CJK} $f(x)$ \begin{CJK}{UTF8}{mj}在\end{CJK} $[1,+\infty)$ \begin{CJK}{UTF8}{mj}上单调递增\end{CJK}, \begin{CJK}{UTF8}{mj}且有极限\end{CJK} $\lim _{x \rightarrow+\infty} f(x)=A$. \begin{CJK}{UTF8}{mj}证明\end{CJK}:
\end{enumerate}
(1) $\sum_{n=1}^{\infty}[f(n+1)-f(n)]$ \begin{CJK}{UTF8}{mj}收敛\end{CJK}, \begin{CJK}{UTF8}{mj}并求其和\end{CJK};

(2) \begin{CJK}{UTF8}{mj}又若\end{CJK} $f(x)$ \begin{CJK}{UTF8}{mj}在\end{CJK} $(1,+\infty)$ \begin{CJK}{UTF8}{mj}内二阶可导\end{CJK}, \begin{CJK}{UTF8}{mj}且\end{CJK} $f^{\prime \prime}(x)<0$, \begin{CJK}{UTF8}{mj}则级数\end{CJK} $\sum_{n=2}^{\infty} f^{\prime}(n)$ \begin{CJK}{UTF8}{mj}也收敛\end{CJK}.

7.(12 \begin{CJK}{UTF8}{mj}分\end{CJK}) \begin{CJK}{UTF8}{mj}求函数项级数\end{CJK}
$$
f(x)=\sum_{n=1}^{\infty} n\left(x+\frac{1}{n}\right)^{n}
$$
\begin{CJK}{UTF8}{mj}的收敛域\end{CJK}, \begin{CJK}{UTF8}{mj}并讨论该级数的一致收敛性\end{CJK}.

8.(12 \begin{CJK}{UTF8}{mj}分\end{CJK}) \begin{CJK}{UTF8}{mj}证明\end{CJK}: \begin{CJK}{UTF8}{mj}若\end{CJK} $f(x)$ \begin{CJK}{UTF8}{mj}在区间\end{CJK} $I$ \begin{CJK}{UTF8}{mj}上连续\end{CJK}, $E$ \begin{CJK}{UTF8}{mj}为\end{CJK} $I$ \begin{CJK}{UTF8}{mj}的任一有界闭子集\end{CJK}, \begin{CJK}{UTF8}{mj}则\end{CJK} $f(E)$ \begin{CJK}{UTF8}{mj}必为闭集\end{CJK}.

\section{$1.21997$ 年}
1.(12 \begin{CJK}{UTF8}{mj}分\end{CJK}) \begin{CJK}{UTF8}{mj}设\end{CJK} $f(x)$ \begin{CJK}{UTF8}{mj}是区间\end{CJK} $I$ \begin{CJK}{UTF8}{mj}上的连续函数\end{CJK}.

\begin{CJK}{UTF8}{mj}证明\end{CJK}: \begin{CJK}{UTF8}{mj}若\end{CJK} $f(x)$ \begin{CJK}{UTF8}{mj}为一一映射\end{CJK}, \begin{CJK}{UTF8}{mj}则\end{CJK} $f(x)$ \begin{CJK}{UTF8}{mj}在区间\end{CJK} $I$ \begin{CJK}{UTF8}{mj}上严格单调\end{CJK}.

2.(12 \begin{CJK}{UTF8}{mj}分\end{CJK}) \begin{CJK}{UTF8}{mj}设\end{CJK} $D(x)= \begin{cases}1, & x \text { 为有理数; } \\ 0, & x \text { 为无理数. }\end{cases}$

\begin{CJK}{UTF8}{mj}证明\end{CJK}: \begin{CJK}{UTF8}{mj}若\end{CJK} $f(x), D(x) f(x)$ \begin{CJK}{UTF8}{mj}在点\end{CJK} $x=0$ \begin{CJK}{UTF8}{mj}处都可导\end{CJK}, \begin{CJK}{UTF8}{mj}且\end{CJK} $f(0)=0$, \begin{CJK}{UTF8}{mj}则\end{CJK} $f^{\prime}(0)=0$.

3.(16 \begin{CJK}{UTF8}{mj}分\end{CJK}) \begin{CJK}{UTF8}{mj}考察函数\end{CJK} $f(x)=x \ln x$ \begin{CJK}{UTF8}{mj}的凸性\end{CJK}, \begin{CJK}{UTF8}{mj}并由此证明不等式\end{CJK}:
$$
a^{a} b^{b} \geqslant(a b)^{\frac{a+b}{2}}(a>0, b>0) .
$$
4.(16 \begin{CJK}{UTF8}{mj}分\end{CJK}) \begin{CJK}{UTF8}{mj}设级数\end{CJK} $\sum_{n=1}^{\infty} a_{n} \sqrt{n}$ \begin{CJK}{UTF8}{mj}收敛\end{CJK}, \begin{CJK}{UTF8}{mj}试就\end{CJK} $\sum_{n=1}^{\infty} a_{n}$ \begin{CJK}{UTF8}{mj}为正项级数和一般项级数两种情况分别证明\end{CJK} $\sum_{n=1}^{\infty} a_{n} \sqrt{n+\sqrt{n}}$ \begin{CJK}{UTF8}{mj}也收敛\end{CJK}.

5.(20 \begin{CJK}{UTF8}{mj}分\end{CJK}) \begin{CJK}{UTF8}{mj}设方程\end{CJK} $F(x, y)=0$ \begin{CJK}{UTF8}{mj}满足隐函数定理条件\end{CJK}, \begin{CJK}{UTF8}{mj}并由此确定了隐函数\end{CJK} $y=f(x)$. \begin{CJK}{UTF8}{mj}又设\end{CJK} $F(x, y)$ \begin{CJK}{UTF8}{mj}具有连续的二阶偏导数\end{CJK}.

(1) \begin{CJK}{UTF8}{mj}求\end{CJK} $f^{\prime \prime}(x)$;

(2) \begin{CJK}{UTF8}{mj}若\end{CJK} $F\left(x_{0}, y_{0}\right)=0, y_{0}=f\left(x_{0}\right)$ \begin{CJK}{UTF8}{mj}为\end{CJK} $f(x)$ \begin{CJK}{UTF8}{mj}的一个极值\end{CJK}, \begin{CJK}{UTF8}{mj}试证明\end{CJK}:

\begin{CJK}{UTF8}{mj}当\end{CJK} $F_{y}\left(x_{0}, y_{0}\right)$ \begin{CJK}{UTF8}{mj}与\end{CJK} $F_{x x}\left(x_{0}, y_{0}\right)$ \begin{CJK}{UTF8}{mj}同号时\end{CJK}, $f\left(x_{0}\right)$ \begin{CJK}{UTF8}{mj}为极大值\end{CJK};

\begin{CJK}{UTF8}{mj}当\end{CJK} $F_{y}\left(x_{0}, y_{0}\right)$ \begin{CJK}{UTF8}{mj}与\end{CJK} $F_{x x}\left(x_{0}, y_{0}\right)$ \begin{CJK}{UTF8}{mj}异号时\end{CJK}, $f\left(x_{0}\right)$ \begin{CJK}{UTF8}{mj}为极小值\end{CJK}.

(3) \begin{CJK}{UTF8}{mj}对方程\end{CJK} $x^{2}+x y+y^{2}=27$, \begin{CJK}{UTF8}{mj}在隐函数形式下\end{CJK} (\begin{CJK}{UTF8}{mj}不解出\end{CJK} $y$ ) \begin{CJK}{UTF8}{mj}求\end{CJK} $y=f(x)$ \begin{CJK}{UTF8}{mj}的极值\end{CJK}, \begin{CJK}{UTF8}{mj}并用\end{CJK} (2) \begin{CJK}{UTF8}{mj}的结论判\end{CJK} \begin{CJK}{UTF8}{mj}别极大或极小\end{CJK}.

6.(12 \begin{CJK}{UTF8}{mj}分\end{CJK}) \begin{CJK}{UTF8}{mj}改变累次积分\end{CJK}
$$
I=\int_{2}^{4} \mathrm{~d} x \int_{\frac{4}{x}}^{\frac{4 x-20}{x-8}}(y-4) \mathrm{d} y
$$
\begin{CJK}{UTF8}{mj}的积分次序\end{CJK}, \begin{CJK}{UTF8}{mj}并求其值\end{CJK}.

7.(12 \begin{CJK}{UTF8}{mj}分\end{CJK}) \begin{CJK}{UTF8}{mj}计算曲面积分\end{CJK}
$$
I=\iint_{S}\left(x^{2} \cos \alpha+y^{2} \cos \beta+z^{2} \cos \gamma\right) \mathrm{d} s
$$
\begin{CJK}{UTF8}{mj}其中\end{CJK} $S$ \begin{CJK}{UTF8}{mj}为雉面\end{CJK} $z=\sqrt{x^{2}+y^{2}}$ \begin{CJK}{UTF8}{mj}上介于\end{CJK} $0 \leqslant z \leqslant h$ \begin{CJK}{UTF8}{mj}的一块\end{CJK}, $\{\cos \alpha, \cos \beta, \cos \gamma\}$ \begin{CJK}{UTF8}{mj}为\end{CJK} $S$ \begin{CJK}{UTF8}{mj}的下侧法向的方向\end{CJK} \begin{CJK}{UTF8}{mj}余弦\end{CJK}.

\section{$1.31998$ 年}
\begin{enumerate}
  \item \begin{CJK}{UTF8}{mj}简答题\end{CJK} (20 \begin{CJK}{UTF8}{mj}分\end{CJK})
\end{enumerate}
(1). \begin{CJK}{UTF8}{mj}用定义验证\end{CJK}:
$$
\lim _{n \rightarrow \infty} \frac{3 n^{2}+2}{2 n^{2}+n+1}=\frac{3}{2}
$$
(2). \begin{CJK}{UTF8}{mj}设\end{CJK} $f(x)=\left\{\begin{array}{ll}\cos x, & x<0, \\ \ln \left(1+x^{2}\right), & x \geqslant 0,\end{array}\right.$ \begin{CJK}{UTF8}{mj}求\end{CJK} $f^{\prime}(x)$.

(3). \begin{CJK}{UTF8}{mj}计算\end{CJK}
$$
\int \frac{x^{3}}{\sqrt{1+x^{2}}} d x
$$
2.(12 \begin{CJK}{UTF8}{mj}分\end{CJK}) \begin{CJK}{UTF8}{mj}设\end{CJK} $f(x)$ \begin{CJK}{UTF8}{mj}有连续的二阶导数\end{CJK}, $f(\pi)=2$, \begin{CJK}{UTF8}{mj}且\end{CJK}
$$
\int_{0}^{\pi}\left[f(x)+f^{\prime \prime}(x)\right] \sin x \mathrm{~d} x=5
$$
\begin{CJK}{UTF8}{mj}求\end{CJK} $f(0)$.

3.(20 \begin{CJK}{UTF8}{mj}分\end{CJK}) (1). \begin{CJK}{UTF8}{mj}已知\end{CJK} $\sum_{n=1}^{\infty} a_{n}$ \begin{CJK}{UTF8}{mj}为发散的一般项级数\end{CJK}, \begin{CJK}{UTF8}{mj}试证明\end{CJK}: $\sum_{n=1}^{\infty}\left(1+\frac{1}{n}\right) a_{n}$ \begin{CJK}{UTF8}{mj}也是发散级数\end{CJK}.

(2). \begin{CJK}{UTF8}{mj}证明\end{CJK}: \begin{CJK}{UTF8}{mj}级数\end{CJK} $\sum_{n=1}^{\infty} 2^{n} \sin \frac{1}{3^{n} x}$ \begin{CJK}{UTF8}{mj}在\end{CJK} $(0,+\infty)$ \begin{CJK}{UTF8}{mj}上处处收敛\end{CJK},\begin{CJK}{UTF8}{mj}但不一致收敛\end{CJK}.

4.(12 \begin{CJK}{UTF8}{mj}分\end{CJK}) \begin{CJK}{UTF8}{mj}设\end{CJK}
$$
F(t)=\iiint_{V} f\left(x^{2}+y^{2}+z^{2}\right) \mathrm{d} x \mathrm{~d} y \mathrm{~d} z
$$
\begin{CJK}{UTF8}{mj}其中\end{CJK} $V=\left\{(x, y, z): x^{2}+y^{2}+z^{2} \leq t^{2}\right\}$, \begin{CJK}{UTF8}{mj}且\end{CJK} $f$ \begin{CJK}{UTF8}{mj}为连续函数\end{CJK}, $f(1)=1$. \begin{CJK}{UTF8}{mj}证明\end{CJK}: $F^{\prime}(1)=4 \pi$.

5.(12 \begin{CJK}{UTF8}{mj}分\end{CJK}) \begin{CJK}{UTF8}{mj}设\end{CJK} $D$ \begin{CJK}{UTF8}{mj}为由两条抛物线\end{CJK} $y=x^{2}-1$ \begin{CJK}{UTF8}{mj}与\end{CJK} $y=-x^{2}+1$ \begin{CJK}{UTF8}{mj}所围成的闭区域\end{CJK}, \begin{CJK}{UTF8}{mj}试求\end{CJK} $D$ \begin{CJK}{UTF8}{mj}内一椭圆\end{CJK} $\frac{x^{2}}{a^{2}}+\frac{y^{2}}{b^{2}}=1$, \begin{CJK}{UTF8}{mj}使其面积是大\end{CJK}.

6.(12 \begin{CJK}{UTF8}{mj}分\end{CJK}) \begin{CJK}{UTF8}{mj}设\end{CJK} $u(x, y)$ \begin{CJK}{UTF8}{mj}具有二阶连续偏导数\end{CJK}, $F(s, t)$ \begin{CJK}{UTF8}{mj}具有一阶连续偏导数\end{CJK}, \begin{CJK}{UTF8}{mj}且满足\end{CJK}
$$
F\left(u_{x}^{\prime}, u_{y}^{\prime}\right)=0, \quad\left(F_{s}^{\prime}\right)^{2}+\left(F_{t}^{\prime}\right)^{2} \neq 0 .
$$
\begin{CJK}{UTF8}{mj}证明\end{CJK}: $u_{x x}^{\prime \prime} u_{y y}^{\prime \prime}-\left(u_{x y}^{\prime \prime}\right)^{2}=0$.

7.(12 \begin{CJK}{UTF8}{mj}分\end{CJK}) \begin{CJK}{UTF8}{mj}设\end{CJK} $f(x)$ \begin{CJK}{UTF8}{mj}为\end{CJK} $(-\infty,+\infty)$ \begin{CJK}{UTF8}{mj}的周期函数\end{CJK}, \begin{CJK}{UTF8}{mj}其周期可小于任意小的正数\end{CJK}. \begin{CJK}{UTF8}{mj}证明\end{CJK}: \begin{CJK}{UTF8}{mj}若\end{CJK} $f(x)$ \begin{CJK}{UTF8}{mj}在\end{CJK} $(-\infty,+\infty)$ \begin{CJK}{UTF8}{mj}上连续\end{CJK}, \begin{CJK}{UTF8}{mj}则\end{CJK} $f(x)$ \begin{CJK}{UTF8}{mj}必为常数\end{CJK}.

\section{$1.41999$ 年}
1.(10 \begin{CJK}{UTF8}{mj}分\end{CJK}) \begin{CJK}{UTF8}{mj}设\end{CJK} $a>0,0<x_{1}<a, x_{n+1}=x_{n}\left(2-\frac{x_{n}}{a}\right), n \in \mathbb{N}_{+}$, \begin{CJK}{UTF8}{mj}证明\end{CJK}: $\left\{a_{n}\right\}$ \begin{CJK}{UTF8}{mj}收敛\end{CJK}, \begin{CJK}{UTF8}{mj}并求其极限\end{CJK}.

\begin{enumerate}
  \setcounter{enumi}{2}
  \item (10 \begin{CJK}{UTF8}{mj}分\end{CJK}) \begin{CJK}{UTF8}{mj}设\end{CJK} $f(x)$ \begin{CJK}{UTF8}{mj}是区间\end{CJK} $I$ \begin{CJK}{UTF8}{mj}上的连续函数\end{CJK}.
\end{enumerate}
\begin{CJK}{UTF8}{mj}证明\end{CJK}: \begin{CJK}{UTF8}{mj}若\end{CJK} $f(x)$ \begin{CJK}{UTF8}{mj}为一一映射\end{CJK}, \begin{CJK}{UTF8}{mj}则\end{CJK} $f(x)$ \begin{CJK}{UTF8}{mj}在区间\end{CJK} $I$ \begin{CJK}{UTF8}{mj}上严格单调\end{CJK}.

3.(10 \begin{CJK}{UTF8}{mj}分\end{CJK}) \begin{CJK}{UTF8}{mj}用条件极值的方法证明不等式\end{CJK}:
$$
\frac{x_{1}^{2}+x_{2}^{2}+\cdots+x_{n}^{2}}{n} \geqslant\left(\frac{x_{1}+x_{2}+\cdots+x_{n}}{n}\right)^{2}\left(x_{k}>0, k=1,2, \cdots, n\right) \text {. }
$$
4.(10 \begin{CJK}{UTF8}{mj}分\end{CJK}) \begin{CJK}{UTF8}{mj}设\end{CJK} $f(x)$ \begin{CJK}{UTF8}{mj}在\end{CJK} $(a,+\infty)$ \begin{CJK}{UTF8}{mj}上可导\end{CJK}, \begin{CJK}{UTF8}{mj}且\end{CJK} $\lim _{x \rightarrow+\infty} f^{\prime}(x)=+\infty$, \begin{CJK}{UTF8}{mj}证明\end{CJK}: $f(x)$ \begin{CJK}{UTF8}{mj}在\end{CJK} $(a,+\infty)$ \begin{CJK}{UTF8}{mj}上非一致连续\end{CJK}.

5.(15 \begin{CJK}{UTF8}{mj}分\end{CJK}) \begin{CJK}{UTF8}{mj}设\end{CJK} $f(x)$ \begin{CJK}{UTF8}{mj}在\end{CJK} $[a, b]$ \begin{CJK}{UTF8}{mj}上一阶可导\end{CJK}, \begin{CJK}{UTF8}{mj}且\end{CJK} $f(x) \geqslant 0, f^{\prime \prime}(x)<0$, \begin{CJK}{UTF8}{mj}证明\end{CJK}:
$$
f(x) \leqslant \frac{2}{b-a} \int_{a}^{b} f(t) \mathrm{d} t, x \in[a, b] .
$$
6.(15 \begin{CJK}{UTF8}{mj}分\end{CJK}) \begin{CJK}{UTF8}{mj}设\end{CJK} $f(x, y)$ \begin{CJK}{UTF8}{mj}在\end{CJK} $D=[a, b] \times[c, d]$ \begin{CJK}{UTF8}{mj}上有二阶连续偏导数\end{CJK}.

(1). \begin{CJK}{UTF8}{mj}通过计算验证\end{CJK}:
$$
\iint_{D} f_{x y}^{\prime \prime}(x, y) \mathrm{d} x \mathrm{~d} y=\iint_{D} f_{y x}^{\prime \prime}(x, y) \mathrm{d} x \mathrm{~d} y
$$
(2). \begin{CJK}{UTF8}{mj}利用\end{CJK} (1) \begin{CJK}{UTF8}{mj}证明\end{CJK}: $f_{x y}^{\prime \prime}(x, y)=f_{y x}^{\prime \prime}(x, y),(x, y) \in D$.

\begin{enumerate}
  \setcounter{enumi}{7}
  \item (15 \begin{CJK}{UTF8}{mj}分\end{CJK}) \begin{CJK}{UTF8}{mj}设对每个\end{CJK} $n, f_{n}(x)$ \begin{CJK}{UTF8}{mj}在\end{CJK} $[a, b]$ \begin{CJK}{UTF8}{mj}上有界\end{CJK}, \begin{CJK}{UTF8}{mj}且当\end{CJK} $n \rightarrow \infty$ \begin{CJK}{UTF8}{mj}时\end{CJK}, $f_{n}(x) \rightrightarrows f(x), x \in[a, b]$. \begin{CJK}{UTF8}{mj}证明\end{CJK}:
\end{enumerate}
(1). $f(x)$ \begin{CJK}{UTF8}{mj}在\end{CJK} $[a, b]$ \begin{CJK}{UTF8}{mj}上有界\end{CJK};

(2). $\lim _{n \rightarrow \infty} \sup _{a \leqslant x \leqslant b} f_{n}(x)=\sup _{a \leqslant x \leqslant b} f(x),\left(=\sup _{a \leqslant x \leqslant b} \lim _{n \rightarrow \infty} f_{n}(x)\right)$.

8.(15 \begin{CJK}{UTF8}{mj}分\end{CJK}) \begin{CJK}{UTF8}{mj}设\end{CJK} $S \subset \mathbb{R}^{2}, p_{1}$ \begin{CJK}{UTF8}{mj}为\end{CJK} $S$ \begin{CJK}{UTF8}{mj}的内点\end{CJK}, $p_{2}$ \begin{CJK}{UTF8}{mj}为\end{CJK} $S$ \begin{CJK}{UTF8}{mj}的任一外点\end{CJK} (\begin{CJK}{UTF8}{mj}假设内点和外点都存在\end{CJK}). \begin{CJK}{UTF8}{mj}证明\end{CJK}: \begin{CJK}{UTF8}{mj}连接\end{CJK} $p_{1} 与 p_{2}$ \begin{CJK}{UTF8}{mj}的直线段至少与\end{CJK} $S$ \begin{CJK}{UTF8}{mj}的边界\end{CJK} $\partial S$ \begin{CJK}{UTF8}{mj}有一个交点\end{CJK}.

\section{$1.52000$ 年}
1.(24 \begin{CJK}{UTF8}{mj}分\end{CJK}) \begin{CJK}{UTF8}{mj}计算题\end{CJK}:

(1). \begin{CJK}{UTF8}{mj}求\end{CJK}
$$
I=\lim _{x \rightarrow 0}\left(\frac{1}{\ln (1+x)}-\frac{1}{x}\right)
$$
(2). \begin{CJK}{UTF8}{mj}敘\end{CJK}
$$
\int \frac{\cos x \cdot \sin ^{3} x}{1+\cos ^{2} x} d x
$$
(3). \begin{CJK}{UTF8}{mj}设\end{CJK} $z=z(x, y)$ \begin{CJK}{UTF8}{mj}是由方程\end{CJK} $F\left(x y z, x^{2}+y^{2}+z^{2}\right)=0$ \begin{CJK}{UTF8}{mj}所确定的可微隐函数\end{CJK}, \begin{CJK}{UTF8}{mj}试求\end{CJK} $\operatorname{grad} z$.

2.(14 \begin{CJK}{UTF8}{mj}分\end{CJK}) \begin{CJK}{UTF8}{mj}证明\end{CJK}:

(1) $\left\{\left(1+\frac{1}{n}\right)^{1+n}\right\}$ \begin{CJK}{UTF8}{mj}为递减数列\end{CJK};

(2)
$$
\frac{1}{1+n}<\ln \left(1+\frac{1}{n}\right)<\frac{1}{n}, n=1,2, \cdots .
$$
3.(12 \begin{CJK}{UTF8}{mj}分\end{CJK}) \begin{CJK}{UTF8}{mj}设\end{CJK} $f(x)$ \begin{CJK}{UTF8}{mj}在\end{CJK} $[a, b]$ \begin{CJK}{UTF8}{mj}中任意两点之间都具有介值性\end{CJK},\begin{CJK}{UTF8}{mj}而且\end{CJK} $f$ \begin{CJK}{UTF8}{mj}在\end{CJK} $(a, b)$ \begin{CJK}{UTF8}{mj}内可导\end{CJK},
$$
\left|f^{\prime}(x)\right| \leqslant K, \quad(K \text { 为正常数 }), x \in(a, b) .
$$
\begin{CJK}{UTF8}{mj}证明\end{CJK}: $f$ \begin{CJK}{UTF8}{mj}在点\end{CJK} $a$ \begin{CJK}{UTF8}{mj}右连续\end{CJK}, \begin{CJK}{UTF8}{mj}在点\end{CJK} $b$ \begin{CJK}{UTF8}{mj}左连续\end{CJK}.

4.(14 \begin{CJK}{UTF8}{mj}分\end{CJK}) \begin{CJK}{UTF8}{mj}设\end{CJK} $I_{n}=\int_{0}^{1}\left(1-x^{2}\right)^{n} \mathrm{~d} x$, \begin{CJK}{UTF8}{mj}证明\end{CJK}:
$$
I_{n}=\frac{2 n}{2 n+1} I_{n-1}, n=2,3, \cdots
$$
(2)
$$
I_{n} \geqslant \frac{2}{3 \sqrt{n}}, n=1,2, \cdots
$$
5.(12 \begin{CJK}{UTF8}{mj}分\end{CJK}) \begin{CJK}{UTF8}{mj}设\end{CJK} $S$ \begin{CJK}{UTF8}{mj}为一旋转曲面\end{CJK}, \begin{CJK}{UTF8}{mj}由光滑曲线段\end{CJK}
$$
\begin{cases}y=f(x), x \in[a, b], & (f(x) \geqslant 0) \\ z=0 & \end{cases}
$$
\begin{CJK}{UTF8}{mj}绕\end{CJK} $x$ \begin{CJK}{UTF8}{mj}轴旋转而成\end{CJK}, \begin{CJK}{UTF8}{mj}试用二重积分计算曲面面积的方法\end{CJK}, \begin{CJK}{UTF8}{mj}导出\end{CJK} $S$ \begin{CJK}{UTF8}{mj}的面积公式为\end{CJK}
$$
A=2 \pi \int_{a}^{b} f(x) \sqrt{1+\left[f^{\prime}(x)\right]^{2}} \mathrm{~d} x .
$$
(\begin{CJK}{UTF8}{mj}提示\end{CJK}: \begin{CJK}{UTF8}{mj}具空间解析几何知道\end{CJK} $S$ \begin{CJK}{UTF8}{mj}的方程为\end{CJK} $y^{2}+z^{2}=f^{2}(x)$.)

\begin{enumerate}
  \setcounter{enumi}{6}
  \item (24 \begin{CJK}{UTF8}{mj}分\end{CJK}) \begin{CJK}{UTF8}{mj}级数问题\end{CJK}:
\end{enumerate}
(1) \begin{CJK}{UTF8}{mj}设\end{CJK} $f(x)=\left\{\begin{array}{l}\frac{\sin x}{x}, x \neq 0 \\ 1, x=0\end{array}\right.$, \begin{CJK}{UTF8}{mj}求\end{CJK} $f^{(k)}(0), k=1,2, \cdots$.

(2) \begin{CJK}{UTF8}{mj}设\end{CJK} $\sum_{n=1}^{\infty} a_{n}$ \begin{CJK}{UTF8}{mj}收敛\end{CJK}, $\lim _{x \rightarrow \infty} n a_{n}=0$, \begin{CJK}{UTF8}{mj}证明\end{CJK}:
$$
\sum_{n=1}^{\infty} n\left(a_{n}-a_{n+1}\right)=\sum_{n=1}^{\infty} a_{n}
$$
(3) \begin{CJK}{UTF8}{mj}设\end{CJK} $\left\{f_{n}(x)\right\}$ \begin{CJK}{UTF8}{mj}为\end{CJK} $[a, b]$ \begin{CJK}{UTF8}{mj}上的连续函数序列\end{CJK}, \begin{CJK}{UTF8}{mj}且\end{CJK}
$$
f_{n}(x) \rightrightarrows f(x), \quad x \in[a, b]
$$
\begin{CJK}{UTF8}{mj}证明\end{CJK}: \begin{CJK}{UTF8}{mj}若\end{CJK} $f(x)$ \begin{CJK}{UTF8}{mj}在\end{CJK} $[a, b]$ \begin{CJK}{UTF8}{mj}上无零点\end{CJK}, \begin{CJK}{UTF8}{mj}则当\end{CJK} $n$ \begin{CJK}{UTF8}{mj}充分大时\end{CJK}, $f_{n}(x)$ \begin{CJK}{UTF8}{mj}在也无零点\end{CJK}, \begin{CJK}{UTF8}{mj}且\end{CJK}
$$
\frac{1}{f_{n}(x)} \rightrightarrows \frac{1}{f(x)}, \quad x \in[a, b]
$$

\section{$1.62001$ 年}
1.(30 \begin{CJK}{UTF8}{mj}分\end{CJK}) \begin{CJK}{UTF8}{mj}简单计算题\end{CJK}:

(1). \begin{CJK}{UTF8}{mj}验证\end{CJK}: \begin{CJK}{UTF8}{mj}当\end{CJK} $x \rightarrow+\infty$ \begin{CJK}{UTF8}{mj}时\end{CJK}, $2 x \int_{0}^{x} e^{t^{2}} \mathrm{~d} t$ \begin{CJK}{UTF8}{mj}与\end{CJK} $e^{x^{2}}$ \begin{CJK}{UTF8}{mj}为等价无穷大量\end{CJK}.

(2). \begin{CJK}{UTF8}{mj}求不定积分\end{CJK}
$$
\int \frac{\ln (1+x)}{x^{2}} d x
$$
(3). \begin{CJK}{UTF8}{mj}求曲线积分\end{CJK}
$$
I=\int_{\widehat{O A}}\left(y^{2}-\cos y\right) \mathrm{d} x+x \sin y \mathrm{~d} y
$$
\begin{CJK}{UTF8}{mj}其中有向曲线\end{CJK} $\widehat{O A}$ \begin{CJK}{UTF8}{mj}为沿着正弦曲线\end{CJK} $y=\sin x$ \begin{CJK}{UTF8}{mj}从点\end{CJK} $O(0,0)$ \begin{CJK}{UTF8}{mj}到点\end{CJK} $A(\pi, 0)$.

(4). \begin{CJK}{UTF8}{mj}设\end{CJK} $f$ \begin{CJK}{UTF8}{mj}为可微函数\end{CJK}, $u=f\left(x^{2}+y^{2}+z^{2}\right)$, \begin{CJK}{UTF8}{mj}并有方程\end{CJK} $3 x+2 y^{2}+z^{3}=6 x y z$, \begin{CJK}{UTF8}{mj}试对以下两种情况分\end{CJK} \begin{CJK}{UTF8}{mj}别计算\end{CJK} $\frac{\partial u}{\partial x}$ \begin{CJK}{UTF8}{mj}在点\end{CJK} $P_{0}(1,1,1)$ \begin{CJK}{UTF8}{mj}处的值\end{CJK}:

(i) \begin{CJK}{UTF8}{mj}由方程确定的隐函数\end{CJK} $z=z(x, y)$;

(ii) \begin{CJK}{UTF8}{mj}由方程确定的隐函数\end{CJK} $y=y(x, z)$.

2.(12 \begin{CJK}{UTF8}{mj}分\end{CJK}) \begin{CJK}{UTF8}{mj}求由椭球面\end{CJK} $\frac{x^{2}}{a^{2}}+\frac{y^{2}}{b^{2}}+\frac{z^{2}}{c^{2}}=1$ \begin{CJK}{UTF8}{mj}与雉面\end{CJK} $\frac{x^{2}}{a^{2}}+\frac{y^{2}}{b^{2}}-\frac{z^{2}}{c^{2}}=0(z \geqslant 0)$ \begin{CJK}{UTF8}{mj}所围成立体的体积\end{CJK}.

3.(12 \begin{CJK}{UTF8}{mj}分\end{CJK}) \begin{CJK}{UTF8}{mj}证明\end{CJK}: \begin{CJK}{UTF8}{mj}若函数\end{CJK} $f(x)$ \begin{CJK}{UTF8}{mj}在有限区间\end{CJK} $(a, b)$ \begin{CJK}{UTF8}{mj}内可导\end{CJK}, \begin{CJK}{UTF8}{mj}但无界\end{CJK}, \begin{CJK}{UTF8}{mj}则其导函数\end{CJK} $f^{\prime}(x)$ \begin{CJK}{UTF8}{mj}在\end{CJK} $(a, b)$ \begin{CJK}{UTF8}{mj}内\end{CJK} \begin{CJK}{UTF8}{mj}亦无界\end{CJK}.

4.(12 \begin{CJK}{UTF8}{mj}分\end{CJK}) \begin{CJK}{UTF8}{mj}证明\end{CJK}: \begin{CJK}{UTF8}{mj}若\end{CJK} $\sum_{n=1}^{\infty} a_{n}$ \begin{CJK}{UTF8}{mj}绝对收敛\end{CJK}, \begin{CJK}{UTF8}{mj}则\end{CJK} $\sum_{n=1}^{\infty} a_{n}\left(a_{1}+\cdots+a_{n}\right)$ \begin{CJK}{UTF8}{mj}亦必绝对收敛\end{CJK}.

5.(17 \begin{CJK}{UTF8}{mj}分\end{CJK}) \begin{CJK}{UTF8}{mj}设函数\end{CJK} $f(x)$ \begin{CJK}{UTF8}{mj}在\end{CJK} $[0,1]$ \begin{CJK}{UTF8}{mj}上连续\end{CJK}. \begin{CJK}{UTF8}{mj}证明\end{CJK}:

(1) $\left\{x^{n}\right\}$ \begin{CJK}{UTF8}{mj}在\end{CJK} $[0,1]$ \begin{CJK}{UTF8}{mj}上不一致收敛\end{CJK};

(2) $\left\{f(x) x^{n}\right\}$ \begin{CJK}{UTF8}{mj}在\end{CJK} $[0,1]$ \begin{CJK}{UTF8}{mj}上一致收敛\end{CJK} $\Leftrightarrow f(1)=0$.

6.(17 \begin{CJK}{UTF8}{mj}分\end{CJK})\begin{CJK}{UTF8}{mj}设函数\end{CJK} $f(x)$ \begin{CJK}{UTF8}{mj}在\end{CJK} $[a, b]$ \begin{CJK}{UTF8}{mj}上无界\end{CJK}, \begin{CJK}{UTF8}{mj}证明\end{CJK}:

(1) \begin{CJK}{UTF8}{mj}存在\end{CJK} $\left\{x_{n}\right\} \subset[a, b]$, \begin{CJK}{UTF8}{mj}使\end{CJK}
$$
\lim _{n \rightarrow \infty} f\left(x_{n}\right)=\infty,
$$
(2) \begin{CJK}{UTF8}{mj}存在\end{CJK} $c \in[a, b]$, \begin{CJK}{UTF8}{mj}使得\end{CJK}: $\forall \delta>0, f(x)$ \begin{CJK}{UTF8}{mj}在\end{CJK} $(c-\delta, c+\delta) \cap[a, b]$ \begin{CJK}{UTF8}{mj}上无界\end{CJK}. $1.72002$ \begin{CJK}{UTF8}{mj}年\end{CJK}

1.(12 \begin{CJK}{UTF8}{mj}分\end{CJK}) \begin{CJK}{UTF8}{mj}计算\end{CJK}:

(1).
$$
\lim _{n \rightarrow \infty} \frac{2 n+\sin \left(n^{2}\right)}{2 n^{2}+n-100}
$$
(2).
$$
\lim _{x \rightarrow 0}\left(\frac{\sin x}{e^{x^{2}}-1}-\frac{1}{x}\right)
$$
(3). \begin{CJK}{UTF8}{mj}设\end{CJK} $F$ \begin{CJK}{UTF8}{mj}为\end{CJK} $\mathbb{R}^{3}$ \begin{CJK}{UTF8}{mj}上的可微函数\end{CJK}, \begin{CJK}{UTF8}{mj}由方程\end{CJK} $F\left(x y, y z^{2}, x^{3} z\right)=0$ \begin{CJK}{UTF8}{mj}确定了\end{CJK} $z$ \begin{CJK}{UTF8}{mj}为\end{CJK} $x$ \begin{CJK}{UTF8}{mj}与\end{CJK} $y$ \begin{CJK}{UTF8}{mj}的函数\end{CJK}, \begin{CJK}{UTF8}{mj}求\end{CJK} $z_{x}, z_{y}$ \begin{CJK}{UTF8}{mj}在点\end{CJK} $(1,1)$ \begin{CJK}{UTF8}{mj}的值\end{CJK}.

\begin{enumerate}
  \setcounter{enumi}{2}
  \item (15 \begin{CJK}{UTF8}{mj}分\end{CJK}) \begin{CJK}{UTF8}{mj}设函数\end{CJK} $f(x), g(x)$ \begin{CJK}{UTF8}{mj}均在\end{CJK} $(a, b)$ \begin{CJK}{UTF8}{mj}内有连续导数\end{CJK}, \begin{CJK}{UTF8}{mj}且\end{CJK}
\end{enumerate}
$$
F(x)=f^{\prime}(x) g(x)-g^{\prime}(x) f(x)>0, \quad \forall x \in(a, b),
$$
\begin{CJK}{UTF8}{mj}求证\end{CJK}: (1). $f(x), g(x)$ \begin{CJK}{UTF8}{mj}不可能有相同的零点\end{CJK};

(2). $f(x)$ \begin{CJK}{UTF8}{mj}的相邻零点之间必有\end{CJK} $g(x)$ \begin{CJK}{UTF8}{mj}的零点\end{CJK};

(3). \begin{CJK}{UTF8}{mj}在\end{CJK} $f(x)$ \begin{CJK}{UTF8}{mj}的每个极值点\end{CJK} $x_{0} \in(a, b)$, \begin{CJK}{UTF8}{mj}存在\end{CJK} $x_{0}$ \begin{CJK}{UTF8}{mj}的某邻域\end{CJK}, \begin{CJK}{UTF8}{mj}使得\end{CJK} $g(x)$ \begin{CJK}{UTF8}{mj}在该邻域中是严格单调的\end{CJK}. 3.(15 \begin{CJK}{UTF8}{mj}分\end{CJK}) \begin{CJK}{UTF8}{mj}设初始值\end{CJK} $a_{1} \in R$ \begin{CJK}{UTF8}{mj}给定\end{CJK}, \begin{CJK}{UTF8}{mj}用递推公式\end{CJK}
$$
a_{n+1}=\frac{2 a_{n}^{3}}{1+a_{n}^{4}}, \quad n=1,2, \cdots
$$
\begin{CJK}{UTF8}{mj}得到数列\end{CJK} $\left\{a_{n}\right\}$.

(1). \begin{CJK}{UTF8}{mj}求证数列\end{CJK} $\left\{a_{n}\right\}$ \begin{CJK}{UTF8}{mj}收敛\end{CJK};

(2). \begin{CJK}{UTF8}{mj}求\end{CJK} $\left\{a_{n}\right\}$ \begin{CJK}{UTF8}{mj}所有可能的极限值\end{CJK};

(3). \begin{CJK}{UTF8}{mj}试将实数轴\end{CJK} $R$ \begin{CJK}{UTF8}{mj}分成若干个小区间\end{CJK}, \begin{CJK}{UTF8}{mj}使得当且仅当在同一区间取初始值\end{CJK}, $\left\{a_{n}\right\}$ \begin{CJK}{UTF8}{mj}都收敛于相同\end{CJK} \begin{CJK}{UTF8}{mj}的极限值\end{CJK}.

\begin{enumerate}
  \setcounter{enumi}{4}
  \item (12 \begin{CJK}{UTF8}{mj}分\end{CJK}) \begin{CJK}{UTF8}{mj}设\end{CJK} $a>c>0$, \begin{CJK}{UTF8}{mj}求椭球体\end{CJK} $\frac{x^{2}+y^{2}}{a^{2}}+\frac{z^{2}}{c^{2}}=1$ \begin{CJK}{UTF8}{mj}的表面积\end{CJK}.
\end{enumerate}
5.(18 \begin{CJK}{UTF8}{mj}分\end{CJK}) \begin{CJK}{UTF8}{mj}设数列\end{CJK} $\left\{a_{n}\right\}$ \begin{CJK}{UTF8}{mj}有界但不收敛\end{CJK}, \begin{CJK}{UTF8}{mj}求证\end{CJK}:

(1). \begin{CJK}{UTF8}{mj}对于任何\end{CJK} $x>0, \sum_{n=1}^{\infty} a_{n} e^{-n x}$ \begin{CJK}{UTF8}{mj}收敛\end{CJK};

(2). \begin{CJK}{UTF8}{mj}对于任何\end{CJK} $\delta>0, \sum_{n=1}^{\infty} a_{n} e^{-n x}$ \begin{CJK}{UTF8}{mj}在\end{CJK} $[\delta,+\infty)$ \begin{CJK}{UTF8}{mj}上一致收敛\end{CJK};

(3). $\sum_{n=1}^{\infty} a_{n} e^{-n x}$ \begin{CJK}{UTF8}{mj}在\end{CJK} $(0,+\infty)$ \begin{CJK}{UTF8}{mj}上不一致收敛\end{CJK}.

6.(12 \begin{CJK}{UTF8}{mj}分\end{CJK}) \begin{CJK}{UTF8}{mj}设函数\end{CJK} $f(x)$ \begin{CJK}{UTF8}{mj}在\end{CJK} $[0,1]$ \begin{CJK}{UTF8}{mj}上连续\end{CJK}, \begin{CJK}{UTF8}{mj}求证\end{CJK}:
$$
\lim _{x \rightarrow 0^{+}} \int_{0}^{1} \frac{x}{t^{2}+x^{2}} f(t) \mathrm{d} t=\frac{\pi}{2} f(0) .
$$
7.(16 \begin{CJK}{UTF8}{mj}分\end{CJK}) \begin{CJK}{UTF8}{mj}设函数\end{CJK} $f$ \begin{CJK}{UTF8}{mj}在\end{CJK} $[0, a]$ \begin{CJK}{UTF8}{mj}上严格递增\end{CJK}, \begin{CJK}{UTF8}{mj}且有连续导数\end{CJK}, $f(0)=0$. \begin{CJK}{UTF8}{mj}设\end{CJK} $g$ \begin{CJK}{UTF8}{mj}是\end{CJK} $f$ \begin{CJK}{UTF8}{mj}的反函数\end{CJK}, \begin{CJK}{UTF8}{mj}求证\end{CJK}:

(1). \begin{CJK}{UTF8}{mj}对于任何\end{CJK} $x \in[0, a]$, \begin{CJK}{UTF8}{mj}都有\end{CJK}
$$
\int_{0}^{f(x)}(x-g(u)) \mathrm{d} u=\int_{0}^{x} f(t) \mathrm{d} t
$$
(2). \begin{CJK}{UTF8}{mj}当\end{CJK} $0 \leqslant x \leqslant a, 0 \leqslant y \leqslant f(a)$ \begin{CJK}{UTF8}{mj}时\end{CJK}, \begin{CJK}{UTF8}{mj}下列不等式成立\end{CJK}:
$$
x y \leqslant \int_{0}^{x} f(t) \mathrm{d} t+\int_{0}^{y} g(u) \mathrm{d} u,
$$
\begin{CJK}{UTF8}{mj}其中当且仅当\end{CJK} $y=f(x)$ \begin{CJK}{UTF8}{mj}时\end{CJK}, \begin{CJK}{UTF8}{mj}等式成立\end{CJK}. $1.82003$ \begin{CJK}{UTF8}{mj}年\end{CJK}

\begin{CJK}{UTF8}{mj}一\end{CJK}、\begin{CJK}{UTF8}{mj}简答题\end{CJK} (\begin{CJK}{UTF8}{mj}只需与出正确答案\end{CJK}), (\begin{CJK}{UTF8}{mj}每小题\end{CJK} 5 \begin{CJK}{UTF8}{mj}分\end{CJK}, \begin{CJK}{UTF8}{mj}共\end{CJK} 30 \begin{CJK}{UTF8}{mj}分\end{CJK})

(1)
$$
\lim _{x \rightarrow 1} \frac{\sin ^{2}(1-x)}{(x-1)^{2}(x+2)}=
$$
(2) $y=\arccos \left(\frac{1}{1+x^{2}}\right)$, \begin{CJK}{UTF8}{mj}则\end{CJK} $y^{\prime}=$

(3)
$$
\int \ln ^{2} x \mathrm{~d} x=
$$
(4) $z=y^{x} \sin \left(\frac{x}{y}\right)$, \begin{CJK}{UTF8}{mj}则\end{CJK} $\mathrm{d} z=$

(5) $D=\left\{(x, y) \mid x^{2}+y^{2} \leqslant 1\right\}$, \begin{CJK}{UTF8}{mj}则\end{CJK} $\iint_{D} e^{x^{2}+y^{2}} \mathrm{~d} x \mathrm{~d} y=$

(6) $L=\left\{(x, y) \mid x^{2}+y^{2}=1\right\}$, \begin{CJK}{UTF8}{mj}取顺时针方向\end{CJK}, \begin{CJK}{UTF8}{mj}则\end{CJK} $\oint_{L} x \mathrm{~d} y-y \mathrm{~d} x=$

\begin{CJK}{UTF8}{mj}二\end{CJK}、\begin{CJK}{UTF8}{mj}判断下列命题是否正确\end{CJK}, \begin{CJK}{UTF8}{mj}若正确给出证明\end{CJK}, \begin{CJK}{UTF8}{mj}若错误举出反例\end{CJK}, \begin{CJK}{UTF8}{mj}每小题\end{CJK} 5 \begin{CJK}{UTF8}{mj}分\end{CJK}, \begin{CJK}{UTF8}{mj}共\end{CJK} 20 \begin{CJK}{UTF8}{mj}分\end{CJK}

(1) \begin{CJK}{UTF8}{mj}若\end{CJK} $\lim _{n \rightarrow \infty} x_{n}=0$, \begin{CJK}{UTF8}{mj}则\end{CJK} $\lim _{n \rightarrow \infty} \sqrt[n]{x_{n}}=0$;

(2) \begin{CJK}{UTF8}{mj}若\end{CJK} $f(x)$ \begin{CJK}{UTF8}{mj}在\end{CJK} $(0,+\infty)$ \begin{CJK}{UTF8}{mj}上可导\end{CJK}, \begin{CJK}{UTF8}{mj}且\end{CJK} $f^{\prime}(x)$ \begin{CJK}{UTF8}{mj}有界\end{CJK}, \begin{CJK}{UTF8}{mj}则\end{CJK} $f(x)$ \begin{CJK}{UTF8}{mj}在\end{CJK} $(0,+\infty)$ \begin{CJK}{UTF8}{mj}上一致连续\end{CJK};

(3) \begin{CJK}{UTF8}{mj}若\end{CJK} $f(x)$ \begin{CJK}{UTF8}{mj}在\end{CJK} $[a, b]$ \begin{CJK}{UTF8}{mj}上可积\end{CJK}, $F(x)=\int_{-1}^{x} f(t) \mathrm{d} t$ \begin{CJK}{UTF8}{mj}在\end{CJK} $x_{0} \in(a, b)$ \begin{CJK}{UTF8}{mj}可导\end{CJK}, \begin{CJK}{UTF8}{mj}则\end{CJK} $F^{\prime}\left(x_{0}\right)=f\left(x_{0}\right)$;

(4) \begin{CJK}{UTF8}{mj}若\end{CJK} $\sum_{n=1}^{\infty}\left(a_{2 n-1}+a_{2 n}\right)$ \begin{CJK}{UTF8}{mj}收敛\end{CJK}, \begin{CJK}{UTF8}{mj}且\end{CJK} $\lim _{n \rightarrow \infty} a_{n}=0$, \begin{CJK}{UTF8}{mj}则\end{CJK} $\sum_{n=1}^{\infty} a_{n}$ \begin{CJK}{UTF8}{mj}收敛\end{CJK}.

\begin{CJK}{UTF8}{mj}三\end{CJK}.(17 \begin{CJK}{UTF8}{mj}分\end{CJK}) \begin{CJK}{UTF8}{mj}令\end{CJK}
$$
f(x)=\lim _{t \rightarrow x}\left(\frac{\sin t}{\sin x}\right)^{\frac{x}{\sin t-\sin x}},
$$
\begin{CJK}{UTF8}{mj}求\end{CJK} $f(x)$ \begin{CJK}{UTF8}{mj}的间断点\end{CJK}, \begin{CJK}{UTF8}{mj}并指出间断点的类型\end{CJK}.

\begin{CJK}{UTF8}{mj}四\end{CJK}.(17 \begin{CJK}{UTF8}{mj}分\end{CJK}) \begin{CJK}{UTF8}{mj}设\end{CJK} $f^{\prime}(x)$ \begin{CJK}{UTF8}{mj}在\end{CJK} $[0, a]$ \begin{CJK}{UTF8}{mj}上连续\end{CJK}, $f(0)=0$. \begin{CJK}{UTF8}{mj}证明\end{CJK}:
$$
\left|\int_{0}^{a} f(x) \mathrm{d} x\right| \leqslant \frac{M a^{2}}{2}, \quad \text { (其中 } M=\max _{0 \leqslant x \leqslant a}\left|f^{\prime}(x)\right| \text { ). }
$$
\begin{CJK}{UTF8}{mj}杢\end{CJK}.(17 \begin{CJK}{UTF8}{mj}分\end{CJK}) \begin{CJK}{UTF8}{mj}若函数\end{CJK} $f(x, y)$ \begin{CJK}{UTF8}{mj}在\end{CJK}. $\mathbb{R}^{2}$ \begin{CJK}{UTF8}{mj}上对\end{CJK} $x$ \begin{CJK}{UTF8}{mj}连续\end{CJK}, \begin{CJK}{UTF8}{mj}且存在\end{CJK} $L>0, \forall x, y^{\prime}, y^{\prime \prime} \in \mathbb{R}$,
$$
\left|f\left(x, y^{\prime}\right)-f\left(x, y^{\prime \prime}\right)\right| \leqslant L\left|y^{\prime}-y^{\prime \prime}\right| \text {. }
$$
\begin{CJK}{UTF8}{mj}证明\end{CJK}: $f(x, y)$ \begin{CJK}{UTF8}{mj}在\end{CJK} $\mathbb{R}^{2}$ \begin{CJK}{UTF8}{mj}上连续\end{CJK}.

\begin{CJK}{UTF8}{mj}六\end{CJK}. $(17$ \begin{CJK}{UTF8}{mj}分\end{CJK}) \begin{CJK}{UTF8}{mj}求下列积分\end{CJK}
$$
I=\iint_{S} f(x, y, z) \mathrm{d} S
$$
\begin{CJK}{UTF8}{mj}其中\end{CJK} $S=\left\{(x, y, z) \mid x^{2}+y^{2}+z^{2}=a^{2},(a>0)\right\}$,
$$
f(x, y, z)=\left\{\begin{array}{ll}
x^{2}+y^{2}, & z \geqslant \sqrt{x^{2}+y^{2}} \\
0, & z<\sqrt{x^{2}+y^{2}}
\end{array} .\right.
$$
\begin{CJK}{UTF8}{mj}七\end{CJK}.(17 \begin{CJK}{UTF8}{mj}分\end{CJK}) \begin{CJK}{UTF8}{mj}设\end{CJK} $0<r<1, x \in R$, \begin{CJK}{UTF8}{mj}求证\end{CJK}: (1)
$$
\frac{1-r^{2}}{1-2 r \cos x+r^{2}}=1+2 \sum_{n=1}^{\infty} r^{n} \cos n x
$$
(2)
$$
\int_{0}^{\pi} \ln \left(1-2 r \cos x+r^{2}\right) d x=0
$$
\begin{CJK}{UTF8}{mj}八\end{CJK}.(15 \begin{CJK}{UTF8}{mj}分\end{CJK}) $a>0, b>0, a_{1}=a, a_{2}=b, a_{n+2}=2+\frac{1}{a_{n+1}^{2}}+\frac{1}{a_{n}^{2}}, n=1,2, \cdots .$ \begin{CJK}{UTF8}{mj}求证\end{CJK}: $\left\{a_{n}\right\}$ \begin{CJK}{UTF8}{mj}收敛\end{CJK}. $1.92004$ \begin{CJK}{UTF8}{mj}年\end{CJK}

\begin{CJK}{UTF8}{mj}一\end{CJK}、\begin{CJK}{UTF8}{mj}求解下列各题\end{CJK} (\begin{CJK}{UTF8}{mj}每小题\end{CJK} 5 \begin{CJK}{UTF8}{mj}分\end{CJK}, \begin{CJK}{UTF8}{mj}共\end{CJK} 30 \begin{CJK}{UTF8}{mj}分\end{CJK})

(1).
$$
\lim _{x \rightarrow 0}\left(\cos x-\frac{x^{2}}{2}\right)^{\frac{1}{x^{2}}}
$$
(2). \begin{CJK}{UTF8}{mj}若\end{CJK} $y=e^{-\ln ^{2} x}+x \sin (\arctan x)$, \begin{CJK}{UTF8}{mj}求\end{CJK} $y^{\prime}$;

(3). \begin{CJK}{UTF8}{mj}求\end{CJK}
$$
\int \frac{x e^{-x}}{(1-x)^{2}} d x
$$
(4). \begin{CJK}{UTF8}{mj}求幕级数\end{CJK} $\sum_{n=1}^{\infty} n x^{n}$ \begin{CJK}{UTF8}{mj}的和函数\end{CJK} $f(x)$;

(5). $L$ \begin{CJK}{UTF8}{mj}为过点\end{CJK} $O(0,0)$ \begin{CJK}{UTF8}{mj}和点\end{CJK} $A\left(\frac{\pi}{2}, a\right)$ \begin{CJK}{UTF8}{mj}的曲线\end{CJK} $y=a \sin x(a>0)$, \begin{CJK}{UTF8}{mj}求\end{CJK}
$$
\int_{L}\left(x+y^{3}\right) d x+(2+y) d y .
$$
(6). \begin{CJK}{UTF8}{mj}求曲面积分\end{CJK}
$$
\iint_{S}(2 x+z) d y d z+z d x d y
$$
\begin{CJK}{UTF8}{mj}其中\end{CJK} $S=\left\{(x, y, z) \mid z=x^{2}+y^{2},(0 \leqslant z \leqslant 1)\right\}$, \begin{CJK}{UTF8}{mj}取上侧\end{CJK}.

\begin{CJK}{UTF8}{mj}二\end{CJK}、\begin{CJK}{UTF8}{mj}判断下列命题是否正确\end{CJK}, \begin{CJK}{UTF8}{mj}若正确给出证明\end{CJK}, \begin{CJK}{UTF8}{mj}若错误举出反例\end{CJK} (\begin{CJK}{UTF8}{mj}每小题\end{CJK} 5 \begin{CJK}{UTF8}{mj}分\end{CJK}, \begin{CJK}{UTF8}{mj}共\end{CJK} 30 \begin{CJK}{UTF8}{mj}分\end{CJK})

(1) \begin{CJK}{UTF8}{mj}若\end{CJK} $\left\{x_{n}, n=1,2, \cdots\right\}$ \begin{CJK}{UTF8}{mj}是互不相等的非无穷大数列\end{CJK}, \begin{CJK}{UTF8}{mj}则\end{CJK} $\left\{x_{n}\right\}$ \begin{CJK}{UTF8}{mj}至少存在一个聚点\end{CJK} $x_{0} \in(-\infty,+\infty)$.

(2) \begin{CJK}{UTF8}{mj}若\end{CJK} $f(x)$ \begin{CJK}{UTF8}{mj}在\end{CJK} $(a, b)$ \begin{CJK}{UTF8}{mj}上连续有界\end{CJK}, \begin{CJK}{UTF8}{mj}则\end{CJK} $f(x)$ \begin{CJK}{UTF8}{mj}在\end{CJK} $(a, b)$ \begin{CJK}{UTF8}{mj}上一致连续\end{CJK}.

(3) \begin{CJK}{UTF8}{mj}右\end{CJK} $f(x), g(x)$ \begin{CJK}{UTF8}{mj}在\end{CJK} $[0,1]$ \begin{CJK}{UTF8}{mj}上可积\end{CJK}, \begin{CJK}{UTF8}{mj}则\end{CJK}
$$
\lim _{n \rightarrow \infty} \frac{1}{n} \sum_{i=1}^{n} f\left(\frac{i}{n}\right) g\left(\frac{i-1}{n}\right)=\int_{0}^{1} f(x) g(x) \mathrm{d} x .
$$
(4) \begin{CJK}{UTF8}{mj}若\end{CJK} $\sum_{n=1}^{\infty} a_{n}$ \begin{CJK}{UTF8}{mj}收敛\end{CJK}, \begin{CJK}{UTF8}{mj}则\end{CJK} $\sum_{n=1}^{\infty} a_{n}^{2}$ \begin{CJK}{UTF8}{mj}收敛\end{CJK}.

(5) \begin{CJK}{UTF8}{mj}若在\end{CJK} $R^{2}$ \begin{CJK}{UTF8}{mj}上定义的函数\end{CJK} $f(x, y)$ \begin{CJK}{UTF8}{mj}存在偏导数\end{CJK} $f_{x}(x, y), f_{y}(x, y)$, \begin{CJK}{UTF8}{mj}且\end{CJK} $f_{x}(x, y), f_{y}(x, y)$ \begin{CJK}{UTF8}{mj}在\end{CJK} $(0,0)$ \begin{CJK}{UTF8}{mj}点连续\end{CJK}, \begin{CJK}{UTF8}{mj}则\end{CJK} $f(x, y)$ \begin{CJK}{UTF8}{mj}在\end{CJK} $(0,0)$ \begin{CJK}{UTF8}{mj}点可微\end{CJK}.

(6) \begin{CJK}{UTF8}{mj}函数\end{CJK} $f(x, y)$ \begin{CJK}{UTF8}{mj}在\end{CJK} $\mathbb{R}^{2}$ \begin{CJK}{UTF8}{mj}上连续\end{CJK}, $D_{r}\left(x_{0}, y_{0}\right)=\left\{(x, y) \mid\left(x-x_{0}\right)^{2}+\left(y-y_{0}\right)^{2} \leqslant r^{2}\right\}$, \begin{CJK}{UTF8}{mj}若\end{CJK} $\forall\left(x_{0}, y_{0}\right), \forall r>0$ , \begin{CJK}{UTF8}{mj}均有\end{CJK} $\iint_{D_{r}} f(x, y) \mathrm{d} \sigma=0$, \begin{CJK}{UTF8}{mj}则\end{CJK}
$$
f(x, y) \equiv 0, \quad(x, y) \in \mathbb{R}^{2}
$$

\section{三、证明下列各题 (每小题 15 分, 共 90 分)}
(1). \begin{CJK}{UTF8}{mj}函数\end{CJK} $f(x)$ \begin{CJK}{UTF8}{mj}在\end{CJK} $(-\infty,+\infty)$ \begin{CJK}{UTF8}{mj}上连续\end{CJK}, \begin{CJK}{UTF8}{mj}且\end{CJK} $\lim _{x \rightarrow \infty} f(x)=A$, \begin{CJK}{UTF8}{mj}求证\end{CJK}: $f(x)$ \begin{CJK}{UTF8}{mj}在\end{CJK} $(-\infty,+\infty)$ \begin{CJK}{UTF8}{mj}上有最大值或最\end{CJK} \begin{CJK}{UTF8}{mj}小值\end{CJK}.

(2). \begin{CJK}{UTF8}{mj}求证不等式\end{CJK}:
$$
2^{x} \geqslant 1+x^{2}, \quad x \in[0,1]
$$
(3). \begin{CJK}{UTF8}{mj}设\end{CJK} $f_{n}(x), n=1,2, \cdots$ \begin{CJK}{UTF8}{mj}在\end{CJK} $[a, b]$ \begin{CJK}{UTF8}{mj}上连续\end{CJK}, \begin{CJK}{UTF8}{mj}且\end{CJK} $f_{n}(x)$ \begin{CJK}{UTF8}{mj}在\end{CJK} $[a, b]$ \begin{CJK}{UTF8}{mj}上一致收敛于\end{CJK} $f(x)$. \begin{CJK}{UTF8}{mj}求证\end{CJK}: \begin{CJK}{UTF8}{mj}若\end{CJK}
$$
f(x)>0, \quad \forall x \in[a, b],
$$
\begin{CJK}{UTF8}{mj}则\end{CJK} $\exists N \in \mathbb{N}_{+}, \delta>0$, \begin{CJK}{UTF8}{mj}使\end{CJK}
$$
f_{n}(x)>\delta, \quad \forall x \in[a, b], n>N
$$
(4). \begin{CJK}{UTF8}{mj}设正项级数\end{CJK} $\sum_{n=1}^{\infty} a_{n}$ \begin{CJK}{UTF8}{mj}收敛\end{CJK}, \begin{CJK}{UTF8}{mj}且\end{CJK}
$$
0 \leqslant a_{k} \leqslant 100 a_{n}, n=k+1, k+2, \cdots .
$$
\begin{CJK}{UTF8}{mj}求证\end{CJK}: $\lim _{n \rightarrow \infty} n a_{n}=0$.

(5). \begin{CJK}{UTF8}{mj}若函数\end{CJK} $f(x)$ \begin{CJK}{UTF8}{mj}在\end{CJK} $[1,+\infty)$ \begin{CJK}{UTF8}{mj}上一致连续\end{CJK}, \begin{CJK}{UTF8}{mj}求证\end{CJK}: $\frac{f(x)}{x}$ \begin{CJK}{UTF8}{mj}在\end{CJK} $[1,+\infty)$ \begin{CJK}{UTF8}{mj}上有界\end{CJK}.

(6). \begin{CJK}{UTF8}{mj}设\end{CJK} $P(x, y, z), Q(x, y, z), R(x, y, z)$ \begin{CJK}{UTF8}{mj}在\end{CJK} $R^{3}$ \begin{CJK}{UTF8}{mj}有连续偏导数\end{CJK}, \begin{CJK}{UTF8}{mj}而且对以任意点\end{CJK} $\left(x_{0}, y_{0}, z_{0}\right)$ \begin{CJK}{UTF8}{mj}为中心\end{CJK}, \begin{CJK}{UTF8}{mj}以任意正数\end{CJK} $r$ \begin{CJK}{UTF8}{mj}为半径的上半球面\end{CJK} $S_{r}=\left\{(x, y, z) \mid\left(x-x_{0}\right)^{2}+\left(y-y_{0}\right)^{2}+\left(z-z_{0}\right)^{2}=r^{2}, z \geqslant z_{0}\right\}$, \begin{CJK}{UTF8}{mj}恒有\end{CJK}
$$
\iint_{S_{r}} P(x, y, z) \mathrm{d} y \mathrm{~d} z+Q(x, y, z) \mathrm{d} z \mathrm{~d} x+R(x, y, z) \mathrm{d} x \mathrm{~d} y=0 .
$$
\begin{CJK}{UTF8}{mj}求证\end{CJK}: $\forall(x, y, z), R(x, y, z)=0, P_{x}(x, y, z)+Q_{y}(x, y, z)=0$.

\section{$1.102005$ 年}
\section{一、判断下列命题是否正确, 若正确给出证明, 若错误举出反例 (每小题 6 分, 共 24 分)}
(1) $\lim _{n \rightarrow \infty} a_{n}=A$ \begin{CJK}{UTF8}{mj}的一个充要条件是\end{CJK}: \begin{CJK}{UTF8}{mj}存在正整数\end{CJK} $N$, \begin{CJK}{UTF8}{mj}对于任意正数\end{CJK} $\varepsilon$, \begin{CJK}{UTF8}{mj}当\end{CJK} $n>N$ \begin{CJK}{UTF8}{mj}时均有\end{CJK} $\left|a_{n}-A\right|<$ $\varepsilon .$

(2) \begin{CJK}{UTF8}{mj}设\end{CJK} $f(x)$ \begin{CJK}{UTF8}{mj}在\end{CJK} $[a,+\infty)$ \begin{CJK}{UTF8}{mj}上连续\end{CJK}, $f(x)$ \begin{CJK}{UTF8}{mj}在\end{CJK} $[a,+\infty)$ \begin{CJK}{UTF8}{mj}上一致连续\end{CJK}, \begin{CJK}{UTF8}{mj}那么\end{CJK} $f^{2}(x)$ \begin{CJK}{UTF8}{mj}在\end{CJK} $[a,+\infty)$ \begin{CJK}{UTF8}{mj}上一致连续\end{CJK}.

(3) \begin{CJK}{UTF8}{mj}设\end{CJK} $a_{n}>0, \lim _{n \rightarrow \infty} n a_{n}=0$, \begin{CJK}{UTF8}{mj}那么正项级数\end{CJK} $\sum_{n=1}^{\infty} a_{n}$ \begin{CJK}{UTF8}{mj}收敛\end{CJK}.

(4) $f(x, y)$ \begin{CJK}{UTF8}{mj}在点\end{CJK} $\left(x_{0}, y_{0}\right)$ \begin{CJK}{UTF8}{mj}沿任意方向导数都存在\end{CJK}, \begin{CJK}{UTF8}{mj}则函数\end{CJK} $f(x, y)$ \begin{CJK}{UTF8}{mj}在点\end{CJK} $\left(x_{0}, y_{0}\right)$ \begin{CJK}{UTF8}{mj}连续\end{CJK}.

\section{二、求解下列各题 (每小题 8 分, 共 64 分)}
(1). \begin{CJK}{UTF8}{mj}求极限\end{CJK} $\lim _{x \rightarrow 0}\left(\frac{1}{x^{2}}-\frac{1}{\sin ^{2} x}\right)$.

(2). \begin{CJK}{UTF8}{mj}求极限\end{CJK} $\lim _{n \rightarrow \infty} \sqrt[n]{\sin ^{2} n+2 \cos ^{2} n}$.

(3). \begin{CJK}{UTF8}{mj}求曲线\end{CJK} $x^{y}=x^{2} y$ \begin{CJK}{UTF8}{mj}在\end{CJK} $(1,1)$ \begin{CJK}{UTF8}{mj}点的切线方程\end{CJK}.

(4). \begin{CJK}{UTF8}{mj}设\end{CJK} $f(x)$ \begin{CJK}{UTF8}{mj}在\end{CJK} $R$ \begin{CJK}{UTF8}{mj}上连续\end{CJK}, $g(t)=\int_{t^{2}}^{e^{t}} f(x) \mathrm{d} x$, \begin{CJK}{UTF8}{mj}求\end{CJK} $g^{\prime}(t)$.

(5).\begin{CJK}{UTF8}{mj}求\end{CJK}
$$
\iint_{x^{2}+y^{2} \leqslant 1}|3 x+4 y| d x d y
$$
(6). \begin{CJK}{UTF8}{mj}设\end{CJK} $f(1,1)=1, f_{x}^{\prime}(1,1)=a, f_{y}^{\prime}(1,1)=b, g(x)=f(x, f(x, f(x, y)))$, \begin{CJK}{UTF8}{mj}求\end{CJK} $g^{\prime}(1)$.

(7). \begin{CJK}{UTF8}{mj}设\end{CJK} $S$ \begin{CJK}{UTF8}{mj}是有向曲面\end{CJK} $\frac{x^{2}}{a^{2}}+\frac{y^{2}}{b^{2}}+\frac{z^{2}}{c^{2}}=1$ \begin{CJK}{UTF8}{mj}外侧\end{CJK}, \begin{CJK}{UTF8}{mj}求第二型曲面积分\end{CJK} $\iint_{S} z \mathrm{~d} x \mathrm{~d} y$.

(8). \begin{CJK}{UTF8}{mj}求椭球面\end{CJK}
$$
\frac{x^{2}}{a^{2}}+\frac{y^{2}}{b^{2}}+\frac{z^{2}}{c^{2}}=1, \quad(x, y, z>0)
$$
\begin{CJK}{UTF8}{mj}的切平面与三个坐标平面所围成的几何体的最小体积\end{CJK}.

\section{三、证明下列各题 (第 1 题至第 4 题每题 12 分, 第 5 题 14 分, 共 62 分)}
(1). \begin{CJK}{UTF8}{mj}设\end{CJK} $f(x)$ \begin{CJK}{UTF8}{mj}在有限区间\end{CJK} $(a, b)$ \begin{CJK}{UTF8}{mj}上一致连续\end{CJK}, \begin{CJK}{UTF8}{mj}求证\end{CJK}: $f(x)$ \begin{CJK}{UTF8}{mj}在\end{CJK} $(a, b)$ \begin{CJK}{UTF8}{mj}上有界\end{CJK}.

(2). \begin{CJK}{UTF8}{mj}已知\end{CJK}
$$
x_{2 n-1}=\frac{1}{n}, x_{2 n}=\int_{n}^{n+1} \frac{1}{x} \mathrm{~d} x, \quad n=1,2, \cdots
$$
\begin{CJK}{UTF8}{mj}求证\end{CJK}:\begin{CJK}{UTF8}{mj}级数\end{CJK} $\sum_{n=1}^{\infty}(-1)^{n} x_{n}$ \begin{CJK}{UTF8}{mj}条件收敛\end{CJK}.

(3). \begin{CJK}{UTF8}{mj}设\end{CJK} $f(x)$ \begin{CJK}{UTF8}{mj}在区间\end{CJK} $[a, b]$ \begin{CJK}{UTF8}{mj}上连续\end{CJK}, $f(x)>0$. \begin{CJK}{UTF8}{mj}求证\end{CJK}:\begin{CJK}{UTF8}{mj}函数列\end{CJK} $\{\sqrt[n]{f(x)}\}$ \begin{CJK}{UTF8}{mj}在\end{CJK} $[a, b]$ \begin{CJK}{UTF8}{mj}上一致收敛于\end{CJK} 1 .

(4). \begin{CJK}{UTF8}{mj}设\end{CJK} $f(x, y)$ \begin{CJK}{UTF8}{mj}在\end{CJK} $[a, b] \times[c, d]$ \begin{CJK}{UTF8}{mj}上连续\end{CJK}, \begin{CJK}{UTF8}{mj}求证\end{CJK}: $g(y)=\max _{x \in[a, b]} f(x, y)$ \begin{CJK}{UTF8}{mj}在\end{CJK} $[c, d]$ \begin{CJK}{UTF8}{mj}上连续\end{CJK}.

(5). \begin{CJK}{UTF8}{mj}设函数\end{CJK} $f(x)$ \begin{CJK}{UTF8}{mj}在区间\end{CJK} $[a,+\infty)$ \begin{CJK}{UTF8}{mj}上有界连续\end{CJK},\begin{CJK}{UTF8}{mj}且对于任意的实数\end{CJK} $c$, \begin{CJK}{UTF8}{mj}方程\end{CJK} $f(x)=c$ \begin{CJK}{UTF8}{mj}至多只有有\end{CJK} \begin{CJK}{UTF8}{mj}限个解\end{CJK}, \begin{CJK}{UTF8}{mj}求证\end{CJK}: $\lim _{x \rightarrow+\infty} f(x)$ \begin{CJK}{UTF8}{mj}存在\end{CJK}.

\section{$1.112006$ 年}
\begin{CJK}{UTF8}{mj}一\end{CJK}、\begin{CJK}{UTF8}{mj}判断下列命题是否正确\end{CJK}, \begin{CJK}{UTF8}{mj}若正确给出证明\end{CJK}, \begin{CJK}{UTF8}{mj}若错误举出反例\end{CJK} (\begin{CJK}{UTF8}{mj}每小题\end{CJK} 6 \begin{CJK}{UTF8}{mj}分\end{CJK}, \begin{CJK}{UTF8}{mj}共\end{CJK} 30 \begin{CJK}{UTF8}{mj}分\end{CJK})

(1) \begin{CJK}{UTF8}{mj}设数列\end{CJK} $\left\{a_{n}\right\}$ \begin{CJK}{UTF8}{mj}满足条件\end{CJK}: $\forall \varepsilon>0, \exists N$, \begin{CJK}{UTF8}{mj}使\end{CJK} $\forall n>N,\left|a_{n}-a_{N}\right|<\varepsilon$, \begin{CJK}{UTF8}{mj}则\end{CJK} $\left\{a_{n}\right\}$ \begin{CJK}{UTF8}{mj}收敛\end{CJK}.

(2) \begin{CJK}{UTF8}{mj}设\end{CJK} $f(x)$ \begin{CJK}{UTF8}{mj}在\end{CJK} $(a, b)$ \begin{CJK}{UTF8}{mj}上可导\end{CJK}, \begin{CJK}{UTF8}{mj}若设\end{CJK} $f^{\prime}(x)$ \begin{CJK}{UTF8}{mj}在\end{CJK} $(a, b)$ \begin{CJK}{UTF8}{mj}上有界\end{CJK},\begin{CJK}{UTF8}{mj}则\end{CJK} $f(x)$ \begin{CJK}{UTF8}{mj}在\end{CJK} $(a, b)$ \begin{CJK}{UTF8}{mj}上有界\end{CJK}.

(3) \begin{CJK}{UTF8}{mj}如果对任何\end{CJK} $n \in \mathbb{N}_{+}, a_{n}>0$, \begin{CJK}{UTF8}{mj}且\end{CJK} $\lim _{n \rightarrow \infty} a_{n}=0$, \begin{CJK}{UTF8}{mj}则级数\end{CJK} $\sum_{n=1}^{\infty}(-1)^{n} a_{n}$ \begin{CJK}{UTF8}{mj}收敛\end{CJK}.

(4) \begin{CJK}{UTF8}{mj}设\end{CJK} $f(x)$ \begin{CJK}{UTF8}{mj}在\end{CJK} $[a, b]$ \begin{CJK}{UTF8}{mj}上可积\end{CJK}, \begin{CJK}{UTF8}{mj}且\end{CJK} $\int_{a}^{b} f(x) \mathrm{d} x>0$, \begin{CJK}{UTF8}{mj}则存在\end{CJK} $[c, d] \subset[a, b]$, \begin{CJK}{UTF8}{mj}使得\end{CJK} $\forall x \in[c, d]$, $f(x)>0$.

(5) \begin{CJK}{UTF8}{mj}设\end{CJK} $f(x, y)$ \begin{CJK}{UTF8}{mj}在\end{CJK} $\left(x_{0}, y_{0}\right)$ \begin{CJK}{UTF8}{mj}的某邻域内连续\end{CJK}, \begin{CJK}{UTF8}{mj}且在\end{CJK} $\left(x_{0}, y_{0}\right)$ \begin{CJK}{UTF8}{mj}处有偏导数\end{CJK} $f_{x}^{\prime}\left(x_{0}, y_{0}\right), f_{y}^{\prime}\left(x_{0}, y_{0}\right)$, \begin{CJK}{UTF8}{mj}则\end{CJK} $f(x, y)$ \begin{CJK}{UTF8}{mj}在\end{CJK} $\left(x_{0}, y_{0}\right)$ \begin{CJK}{UTF8}{mj}处可微\end{CJK}.

\begin{CJK}{UTF8}{mj}二\end{CJK}、\begin{CJK}{UTF8}{mj}求解下列各题\end{CJK} (\begin{CJK}{UTF8}{mj}每小题\end{CJK} 6 \begin{CJK}{UTF8}{mj}分\end{CJK}, \begin{CJK}{UTF8}{mj}共\end{CJK} 30 \begin{CJK}{UTF8}{mj}分\end{CJK})

(1). \begin{CJK}{UTF8}{mj}设\end{CJK} $0<a<b$, \begin{CJK}{UTF8}{mj}求\end{CJK}
$$
\lim _{n \rightarrow \infty} \sqrt[n]{a^{n}+b^{n}} .
$$
(2). \begin{CJK}{UTF8}{mj}求\end{CJK} $f(x)=\int_{0}^{x} \frac{1-\cos t}{t} \mathrm{~d} t$ \begin{CJK}{UTF8}{mj}的麦克劳林级数展开式\end{CJK}.

(3). \begin{CJK}{UTF8}{mj}戊\end{CJK}
$$
\int_{0}^{1} x^{2} \ln ^{2} x \mathrm{dx}
$$
(4). \begin{CJK}{UTF8}{mj}设\end{CJK} $z=f(u)$, \begin{CJK}{UTF8}{mj}方程\end{CJK}
$$
u=\phi(u)+\int_{y}^{x} P(t) \mathrm{d} t
$$
\begin{CJK}{UTF8}{mj}定义了隐函数\end{CJK} $u=u(x, y)$, \begin{CJK}{UTF8}{mj}其中\end{CJK} $f(u), \phi(u)$ \begin{CJK}{UTF8}{mj}可微\end{CJK}, $P(t), \phi^{\prime}(u)$ \begin{CJK}{UTF8}{mj}连续\end{CJK}, \begin{CJK}{UTF8}{mj}且\end{CJK} $\phi^{\prime}(u) \neq 1$, \begin{CJK}{UTF8}{mj}求\end{CJK}
$$
P(y) \frac{\partial z}{\partial x}+P(x) \frac{\partial z}{\partial y} \text {. }
$$
(5). \begin{CJK}{UTF8}{mj}求\end{CJK} $\iint_{\Sigma}\left(y^{2}+z^{2}\right) \mathrm{d} S$, \begin{CJK}{UTF8}{mj}其中\end{CJK} $\Sigma=\left\{(x, y, z) \mid x^{2}+y^{2}+z^{2}=1\right\}$.

\begin{CJK}{UTF8}{mj}三\end{CJK}、\begin{CJK}{UTF8}{mj}证明下列各题\end{CJK} (\begin{CJK}{UTF8}{mj}每小题\end{CJK} 18 \begin{CJK}{UTF8}{mj}分\end{CJK}, \begin{CJK}{UTF8}{mj}共\end{CJK} 90 \begin{CJK}{UTF8}{mj}分\end{CJK})

(1). \begin{CJK}{UTF8}{mj}设\end{CJK} $\delta>0, f(x)$ \begin{CJK}{UTF8}{mj}在\end{CJK} $(-\delta, \delta)$ \begin{CJK}{UTF8}{mj}上具有连续的二阶导函数\end{CJK} $f^{\prime \prime}(x), f(0)=0$. \begin{CJK}{UTF8}{mj}若\end{CJK} $g(x)=$ $\left\{\begin{array}{l}f^{\prime}(0), x=0 \\ \frac{f(x)}{x}, x \neq 0\end{array}\right.$ \begin{CJK}{UTF8}{mj}求证\end{CJK}: $g(x)$ \begin{CJK}{UTF8}{mj}在\end{CJK} $(-\delta, \delta)$ \begin{CJK}{UTF8}{mj}上有连续的导函数\end{CJK}.

(2). \begin{CJK}{UTF8}{mj}设\end{CJK} $f_{n}(x)$ \begin{CJK}{UTF8}{mj}是\end{CJK} $[0,1]$ \begin{CJK}{UTF8}{mj}上连续函数\end{CJK}, \begin{CJK}{UTF8}{mj}且在\end{CJK} $[0,1]$ \begin{CJK}{UTF8}{mj}上一致收敛于\end{CJK} $f(x)$, \begin{CJK}{UTF8}{mj}求证\end{CJK}:
$$
\lim _{n \rightarrow \infty} \int_{0}^{1-\frac{1}{n}} f_{n}(x) \mathrm{d} x=\int_{0}^{1} f(x) \mathrm{d} x .
$$
(3). \begin{CJK}{UTF8}{mj}设\end{CJK} $f(x)$ \begin{CJK}{UTF8}{mj}是\end{CJK} $[0,+\infty)$ \begin{CJK}{UTF8}{mj}上一致连续\end{CJK}, \begin{CJK}{UTF8}{mj}且\end{CJK} $\forall \delta>0$, $\lim _{n \rightarrow \infty} f(n \delta)=0$. \begin{CJK}{UTF8}{mj}求证\end{CJK}: $\lim _{x \rightarrow+\infty} f(x)=0$.

(4). \begin{CJK}{UTF8}{mj}设\end{CJK} $f(x)$ \begin{CJK}{UTF8}{mj}是\end{CJK} $[0,+\infty)$ \begin{CJK}{UTF8}{mj}上连续有界\end{CJK}, \begin{CJK}{UTF8}{mj}求证\end{CJK}:
$$
\lim _{n \rightarrow \infty} \sqrt[n]{\int_{0}^{n}|f(x)|^{n} \mathrm{~d} x}=\sup _{x \in[0,+\infty)}\{|f(x)|\} .
$$
(5). \begin{CJK}{UTF8}{mj}设\end{CJK} $f(x, y, z)$ \begin{CJK}{UTF8}{mj}是定义在开区域\end{CJK} $D$ \begin{CJK}{UTF8}{mj}上的有连续偏导数的三元函数\end{CJK}, \begin{CJK}{UTF8}{mj}且\end{CJK}
$$
f_{x}^{2}(x, y, z)+f_{y}^{2}(x, y, z)+f_{z}^{2}(x, y, z) \neq 0, \quad(\forall(x, y, z) \in D)
$$
$S$ \begin{CJK}{UTF8}{mj}是由\end{CJK} $f(x, y, z)=0$ \begin{CJK}{UTF8}{mj}定义的封闭的光滑曲面\end{CJK}. \begin{CJK}{UTF8}{mj}若\end{CJK} $P, Q \in S$, \begin{CJK}{UTF8}{mj}且\end{CJK} $P 与 Q$ \begin{CJK}{UTF8}{mj}与间的距离是\end{CJK} $S$ \begin{CJK}{UTF8}{mj}中任意两点之\end{CJK} \begin{CJK}{UTF8}{mj}间距离的是大值\end{CJK}.

\begin{CJK}{UTF8}{mj}求证\end{CJK}: \begin{CJK}{UTF8}{mj}过\end{CJK} $P$ \begin{CJK}{UTF8}{mj}的\end{CJK} $S$ \begin{CJK}{UTF8}{mj}的切平面与过\end{CJK} $Q$ \begin{CJK}{UTF8}{mj}的\end{CJK} $S$ \begin{CJK}{UTF8}{mj}的切平面互相平行\end{CJK}, \begin{CJK}{UTF8}{mj}且垂直于过\end{CJK} $P$ \begin{CJK}{UTF8}{mj}与\end{CJK} $Q$ \begin{CJK}{UTF8}{mj}的连线\end{CJK}.

\section{$1.122007$ 年}
\begin{CJK}{UTF8}{mj}一\end{CJK}、\begin{CJK}{UTF8}{mj}判断下列命题是否正确\end{CJK}, \begin{CJK}{UTF8}{mj}若正确给出证明\end{CJK}, \begin{CJK}{UTF8}{mj}若错误举出反例\end{CJK} (\begin{CJK}{UTF8}{mj}每小题\end{CJK} 6 \begin{CJK}{UTF8}{mj}分\end{CJK}, \begin{CJK}{UTF8}{mj}共\end{CJK} 30 \begin{CJK}{UTF8}{mj}分\end{CJK})

(1) \begin{CJK}{UTF8}{mj}设\end{CJK} $f(x)$ \begin{CJK}{UTF8}{mj}在\end{CJK} $x_{0}$ \begin{CJK}{UTF8}{mj}的邻域\end{CJK} $U\left(x_{0}\right)$ \begin{CJK}{UTF8}{mj}内有定义且有界\end{CJK}. \begin{CJK}{UTF8}{mj}若\end{CJK} $\lim _{x \rightarrow x_{0}} f(x)$ \begin{CJK}{UTF8}{mj}不存在\end{CJK}, \begin{CJK}{UTF8}{mj}则在数列\end{CJK} $\left\{x_{n}\right\} \subset U\left(x_{0}\right)$, $\left\{y_{n}\right\} \subset U\left(x_{0}\right)$, \begin{CJK}{UTF8}{mj}使得\end{CJK} $\lim _{n \rightarrow \infty} x_{n}=\lim _{n \rightarrow \infty} y_{n}=x_{0}$, \begin{CJK}{UTF8}{mj}而\end{CJK} $\lim _{n \rightarrow \infty} f\left(x_{n}\right)$ \begin{CJK}{UTF8}{mj}和\end{CJK} $\lim _{n \rightarrow \infty} f\left(y_{n}\right)$ \begin{CJK}{UTF8}{mj}都存在但不相等\end{CJK}.

(2) $f(x)$ \begin{CJK}{UTF8}{mj}在\end{CJK} $(a, b)$ \begin{CJK}{UTF8}{mj}上可导\end{CJK}, \begin{CJK}{UTF8}{mj}且\end{CJK} $f^{\prime}(x)$ \begin{CJK}{UTF8}{mj}在\end{CJK} $(a, b)$ \begin{CJK}{UTF8}{mj}上有界\end{CJK}, \begin{CJK}{UTF8}{mj}则\end{CJK} $f(x)$ \begin{CJK}{UTF8}{mj}在\end{CJK} $(a, b)$ \begin{CJK}{UTF8}{mj}上有界\end{CJK}.

(3) \begin{CJK}{UTF8}{mj}设数项级数\end{CJK} $\sum_{n=1}^{\infty} a_{n}$ \begin{CJK}{UTF8}{mj}收敛\end{CJK}, \begin{CJK}{UTF8}{mj}则级数\end{CJK} $\sum_{n=1}^{\infty} a_{n}^{2}$ \begin{CJK}{UTF8}{mj}收敛\end{CJK}.

(4) \begin{CJK}{UTF8}{mj}设\end{CJK} $f(x)$ \begin{CJK}{UTF8}{mj}在\end{CJK} $[a, b]$ \begin{CJK}{UTF8}{mj}上有连续的导函数\end{CJK}, $[a, b] \subset(-\pi, \pi), f(a)=f(b)=0$. \begin{CJK}{UTF8}{mj}若\end{CJK}
$$
\left.\begin{array}{l}
A_{n}=\frac{1}{\pi} \int_{a}^{b} f(x) \cos n x \mathrm{~d} x \\
B_{n}=\frac{1}{\pi} \int_{a}^{b} f(x) \sin n x \mathrm{~d} x
\end{array}\right\} n=1,2, \cdots,
$$
\begin{CJK}{UTF8}{mj}则对任意\end{CJK} $x \in[a, b], f(x)=\frac{A_{0}}{2}+\sum_{n=1}^{\infty}\left(A_{n} \cos n x+B_{n} \sin x\right)$.

(5) \begin{CJK}{UTF8}{mj}若\end{CJK} $f(x, y)$ \begin{CJK}{UTF8}{mj}在\end{CJK} $\left(x_{0}, y_{0}\right)$ \begin{CJK}{UTF8}{mj}上连续\end{CJK}, \begin{CJK}{UTF8}{mj}且\end{CJK} $f_{x}\left(x_{0}, y_{0}\right)=f_{y}\left(x_{0}, y_{0}\right)=0$, \begin{CJK}{UTF8}{mj}则\end{CJK} $f(x, y)$ \begin{CJK}{UTF8}{mj}在\end{CJK} $\left(x_{0}, y_{0}\right)$ \begin{CJK}{UTF8}{mj}上可微\end{CJK}.

\begin{CJK}{UTF8}{mj}二\end{CJK}、\begin{CJK}{UTF8}{mj}求解下列各题\end{CJK} (\begin{CJK}{UTF8}{mj}每小题\end{CJK} 8 \begin{CJK}{UTF8}{mj}分\end{CJK}, \begin{CJK}{UTF8}{mj}共\end{CJK} $\mathbf{4 0}$ \begin{CJK}{UTF8}{mj}分\end{CJK})

(1). \begin{CJK}{UTF8}{mj}计算\end{CJK}
$$
\lim _{x \rightarrow 0} \frac{\sqrt{1+\tan x}-\sqrt{1+\sin x}}{x^{2} \sin 2 x}
$$
(2). \begin{CJK}{UTF8}{mj}设\end{CJK} $a, b$ \begin{CJK}{UTF8}{mj}为非零常数\end{CJK}, \begin{CJK}{UTF8}{mj}求\end{CJK}
$$
\int_{0}^{\frac{\pi}{2}} \frac{\mathrm{d} x}{a^{2} \sin ^{2} x+b^{2} \cos ^{2} x}
$$
(3). \begin{CJK}{UTF8}{mj}求幂级数\end{CJK} $\sum_{n=0}^{\infty}(-1)^{n} \frac{x^{2 n+1}}{2 n+1}$ \begin{CJK}{UTF8}{mj}的收敛域与和函数\end{CJK}.

(4). \begin{CJK}{UTF8}{mj}设\end{CJK} $f(x)$ \begin{CJK}{UTF8}{mj}在\end{CJK} $(-\infty,+\infty)$ \begin{CJK}{UTF8}{mj}上有连续的二阶导函数\end{CJK}, $z=x f\left(\frac{x}{y}\right)+2 y f\left(\frac{y}{x}\right)$. \begin{CJK}{UTF8}{mj}求\end{CJK} $\frac{\partial z}{\partial x}, \frac{\partial z}{\partial y}, \frac{\partial z}{\partial x \partial y}$.

(5). \begin{CJK}{UTF8}{mj}求\end{CJK} $\iint_{S} \frac{\mathrm{d} s}{z}$, \begin{CJK}{UTF8}{mj}其中\end{CJK} $S$ \begin{CJK}{UTF8}{mj}是球面\end{CJK} $x^{2}+y^{2}+z^{2}=a^{2}$ \begin{CJK}{UTF8}{mj}被平面\end{CJK} $z=h(0<h<a)$ \begin{CJK}{UTF8}{mj}截得的球冠部分\end{CJK}.

\section{三、证明下列各题 (每小题 16 分, 共 80 分)}
(1). \begin{CJK}{UTF8}{mj}设\end{CJK} $\left\{a_{n}\right\}$ \begin{CJK}{UTF8}{mj}是一列有界的正实数列\end{CJK}, $a=\sup \left\{a_{1}, a_{2}, \cdots\right\}$. \begin{CJK}{UTF8}{mj}求证\end{CJK}:
$$
\lim _{n \rightarrow \infty}\left(\sum_{k=1}^{n} a_{k}^{n}\right)^{\frac{1}{n}}=a .
$$
(2). \begin{CJK}{UTF8}{mj}设\end{CJK} $f(x)$ \begin{CJK}{UTF8}{mj}是定义在\end{CJK} $[a, b]$ \begin{CJK}{UTF8}{mj}上的函数\end{CJK}, \begin{CJK}{UTF8}{mj}满足\end{CJK}: \begin{CJK}{UTF8}{mj}对任意\end{CJK} $x_{0} \in[a, b]$, \begin{CJK}{UTF8}{mj}存在\end{CJK} $\delta_{x_{0}}>0, \varepsilon_{x_{0}}>0$, \begin{CJK}{UTF8}{mj}使得在\end{CJK} $\left(x_{0}-\delta_{x_{0}}, x_{0}+\delta_{x_{0}}\right) \cap[a, b]$ \begin{CJK}{UTF8}{mj}有\end{CJK} $f(x)>\varepsilon_{x_{0}}$. \begin{CJK}{UTF8}{mj}求证\end{CJK}: \begin{CJK}{UTF8}{mj}存在\end{CJK} $\varepsilon>0$, \begin{CJK}{UTF8}{mj}使得在\end{CJK} $[a, b]$ \begin{CJK}{UTF8}{mj}上有\end{CJK} $f(x)>\varepsilon$.

(3). \begin{CJK}{UTF8}{mj}设\end{CJK} $f(x)$ \begin{CJK}{UTF8}{mj}是定义在\end{CJK} $(-\infty,+\infty)$ \begin{CJK}{UTF8}{mj}上的连续函数\end{CJK}, \begin{CJK}{UTF8}{mj}且\end{CJK} $\int_{0}^{+\infty} f(x) \mathrm{d} x$ \begin{CJK}{UTF8}{mj}收敛\end{CJK}. \begin{CJK}{UTF8}{mj}若含参变量反常积分\end{CJK} $I(y)=\int_{0}^{+\infty} f(x+y) \mathrm{d} x$ \begin{CJK}{UTF8}{mj}在\end{CJK} $(-\infty,+\infty)$ \begin{CJK}{UTF8}{mj}上一致收敛\end{CJK}. \begin{CJK}{UTF8}{mj}求证\end{CJK}: $f(x)=0, \forall x \in(-\infty,+\infty)$.

(4). \begin{CJK}{UTF8}{mj}设\end{CJK} $\left\{f_{n}(x)\right\}$ \begin{CJK}{UTF8}{mj}是定义在\end{CJK} $[-1,1]$ \begin{CJK}{UTF8}{mj}上的连续函数列\end{CJK}, \begin{CJK}{UTF8}{mj}且\end{CJK}

(i). $\lim _{n \rightarrow \infty} \int_{-1}^{1} f_{n}(x) \mathrm{d} x=1$;

(ii). \begin{CJK}{UTF8}{mj}对任意\end{CJK} $\delta>0, f_{n}(x)$ \begin{CJK}{UTF8}{mj}在\end{CJK} $[-1,-\delta] \cup[\delta, 1]$ \begin{CJK}{UTF8}{mj}上一致收敛与零\end{CJK}.

\begin{CJK}{UTF8}{mj}求证\end{CJK}: \begin{CJK}{UTF8}{mj}对任意的\end{CJK} $[-1,1]$ \begin{CJK}{UTF8}{mj}上的连续函数\end{CJK} $g(x), \lim _{n \rightarrow \infty} \int_{-1}^{1} f_{n}(x) g(x) \mathrm{d} x=g(0)$. (5). \begin{CJK}{UTF8}{mj}设\end{CJK} $f(x, y)$ \begin{CJK}{UTF8}{mj}在\end{CJK} $D=\left\{(x, y) \mid x^{2}+y^{2} \leqslant 1\right\}$ \begin{CJK}{UTF8}{mj}上有连续的偏导数\end{CJK}, \begin{CJK}{UTF8}{mj}且在\end{CJK} $T=\left\{(x, y) \mid x^{2}+y^{2}=1\right\}$ \begin{CJK}{UTF8}{mj}上\end{CJK} \begin{CJK}{UTF8}{mj}恒为零\end{CJK}. \begin{CJK}{UTF8}{mj}求证\end{CJK}:
$$
\left|\iint_{D} f(x, y) \mathrm{d} x \mathrm{~d} y\right| \leqslant \frac{\pi}{3} \max _{(x, y) \in D}\left[\left(\frac{\partial f}{\partial x}\right)^{2}+\left(\frac{\partial f}{\partial y}\right)^{2}\right]^{\frac{1}{2}}
$$

\section{$1.132008$ 年}
\begin{CJK}{UTF8}{mj}一\end{CJK}、\begin{CJK}{UTF8}{mj}判断下列命题是否正确\end{CJK}, \begin{CJK}{UTF8}{mj}若正确给出证明\end{CJK}, \begin{CJK}{UTF8}{mj}若错误举出反例\end{CJK}(\begin{CJK}{UTF8}{mj}每小题\end{CJK} 6 \begin{CJK}{UTF8}{mj}分\end{CJK}, \begin{CJK}{UTF8}{mj}共\end{CJK} 36 \begin{CJK}{UTF8}{mj}分\end{CJK})

(1) \begin{CJK}{UTF8}{mj}数列\end{CJK} $\left\{a_{n}\right\}$ \begin{CJK}{UTF8}{mj}收敛的充要条件是\end{CJK} $\forall \varepsilon>0, \exists N \in \mathbb{N}_{+}$, \begin{CJK}{UTF8}{mj}使得当\end{CJK} $n>N$ \begin{CJK}{UTF8}{mj}时\end{CJK}, \begin{CJK}{UTF8}{mj}恒有\end{CJK} $\left|a_{2 n}-a_{n}\right|<\varepsilon$

(2) \begin{CJK}{UTF8}{mj}若\end{CJK} $f(x, y)$ \begin{CJK}{UTF8}{mj}在\end{CJK} $\left(x_{0}, y_{0}\right)$ \begin{CJK}{UTF8}{mj}处可微\end{CJK}, \begin{CJK}{UTF8}{mj}则在\end{CJK} $\left(x_{0}, y_{0}\right)$ \begin{CJK}{UTF8}{mj}的某个邻域内\end{CJK} $\frac{\partial f}{\partial x}, \frac{\partial f}{\partial y}$ \begin{CJK}{UTF8}{mj}存在\end{CJK}.

(3) \begin{CJK}{UTF8}{mj}设\end{CJK} $f(x)$ \begin{CJK}{UTF8}{mj}在\end{CJK} $[a, b]$ \begin{CJK}{UTF8}{mj}上连续\end{CJK}, \begin{CJK}{UTF8}{mj}且\end{CJK} $\int_{a}^{b} f(x) \mathrm{d} x=0$, \begin{CJK}{UTF8}{mj}则\end{CJK} $f(x)$ \begin{CJK}{UTF8}{mj}在\end{CJK} $[a, b]$ \begin{CJK}{UTF8}{mj}上有零点\end{CJK}.

(4) \begin{CJK}{UTF8}{mj}设级数\end{CJK} $\sum_{n=1}^{\infty} a_{n}$ \begin{CJK}{UTF8}{mj}收敛\end{CJK}, \begin{CJK}{UTF8}{mj}则\end{CJK} $\sum_{n=1}^{\infty} \frac{a_{n}}{n}$ \begin{CJK}{UTF8}{mj}收敛\end{CJK}.

(5) \begin{CJK}{UTF8}{mj}设\end{CJK} $f(x, y)$ \begin{CJK}{UTF8}{mj}在\end{CJK} $\left(x_{0}, y_{0}\right)$ \begin{CJK}{UTF8}{mj}的某个邻域内有定义且\end{CJK}
$$
\lim _{x \rightarrow x_{0}} \lim _{y \rightarrow y_{0}} f(x, y)=\lim _{y \rightarrow y_{0}} \lim _{x \rightarrow x_{0}} f(x, y)=f\left(x_{0}, y_{0}\right)
$$
\begin{CJK}{UTF8}{mj}则\end{CJK} $f(x, y)$ \begin{CJK}{UTF8}{mj}在\end{CJK} $\left(x_{0}, y_{0}\right)$ \begin{CJK}{UTF8}{mj}处连续\end{CJK}.

(6) \begin{CJK}{UTF8}{mj}对任意给定的\end{CJK} $x_{0} \in \mathbb{R}$, \begin{CJK}{UTF8}{mj}任意给定的严格增加正整数列\end{CJK} $n_{k}, k=1,2, \cdots$, \begin{CJK}{UTF8}{mj}存在定义在\end{CJK} $\mathbb{R}$ \begin{CJK}{UTF8}{mj}上的连\end{CJK} \begin{CJK}{UTF8}{mj}续函数\end{CJK} $f(x)$ \begin{CJK}{UTF8}{mj}使得\end{CJK} $f^{\left(n_{k}\right)}\left(x_{0}\right)=0, k=1,2, \cdots$.

\begin{CJK}{UTF8}{mj}二\end{CJK}、\begin{CJK}{UTF8}{mj}求解下列各题\end{CJK} (\begin{CJK}{UTF8}{mj}每小题\end{CJK} 10 \begin{CJK}{UTF8}{mj}分\end{CJK}, \begin{CJK}{UTF8}{mj}共\end{CJK} 30 \begin{CJK}{UTF8}{mj}分\end{CJK})

(1) \begin{CJK}{UTF8}{mj}求\end{CJK}
$$
\lim _{x \rightarrow 0} \frac{[1+(x+1) \sin x]^{\frac{1}{4}}-1}{e^{x}-1}
$$
(2) \begin{CJK}{UTF8}{mj}设\end{CJK} $z=z(x, y)$ \begin{CJK}{UTF8}{mj}是由方程组\end{CJK} $\left\{\begin{array}{l}x=e^{u} \cos v \\ y=e^{u} \sin v \text { 所确定的隐函数. 求 } \frac{\partial z}{\partial x}, \frac{\partial z}{\partial y}, \frac{\partial z}{\partial x \partial y} \text {. } \\ z=u v\end{array}\right.$

(3) i\begin{CJK}{UTF8}{mj}十算\end{CJK}
$$
\iint_{S_{i}} \frac{x-1}{r^{3}} \mathrm{~d} y \mathrm{~d} z+\frac{y-2}{r^{3}} \mathrm{~d} z \mathrm{~d} x+\frac{z-3}{r^{3}} \mathrm{~d} x \mathrm{~d} y, \quad(i=1,2)
$$
\begin{CJK}{UTF8}{mj}其中\end{CJK}
$$
\begin{gathered}
r=\sqrt{(x-1)^{2}+(y-2)^{2}+(z-3)^{2}}, \\
S_{1}=\left\{(x, y) \mid(x-1)^{2}+(y-2)^{2}+(z-3)^{2}=1\right\}, \\
S_{2}=\left\{(x, y) \mid \frac{(x-1)^{2}}{1}+\frac{(y-2)^{2}}{2}+\frac{(z-3)^{2}}{3}=1\right\},
\end{gathered}
$$
\begin{CJK}{UTF8}{mj}积分沿曲面的外侧\end{CJK}.

\begin{CJK}{UTF8}{mj}三\end{CJK}、\begin{CJK}{UTF8}{mj}证明下列各题\end{CJK} (\begin{CJK}{UTF8}{mj}每小题\end{CJK} 14 \begin{CJK}{UTF8}{mj}分\end{CJK}, \begin{CJK}{UTF8}{mj}共\end{CJK} 84 \begin{CJK}{UTF8}{mj}分\end{CJK})

(1) \begin{CJK}{UTF8}{mj}设级数\end{CJK} $\sum_{n=1}^{\infty} a_{n}$ \begin{CJK}{UTF8}{mj}收敛于\end{CJK} $A$ (\begin{CJK}{UTF8}{mj}有限数\end{CJK}). \begin{CJK}{UTF8}{mj}证明\end{CJK}:
$$
\lim _{n \rightarrow \infty} \frac{a_{n}+2 a_{n-1}+\cdots+(n-1) a_{2}+n a_{1}}{n}=A .
$$
(2) \begin{CJK}{UTF8}{mj}设\end{CJK} $f(x)$ \begin{CJK}{UTF8}{mj}在\end{CJK} $[a, b]$ \begin{CJK}{UTF8}{mj}上的不连续点都是第一类间断点\end{CJK}. \begin{CJK}{UTF8}{mj}证明\end{CJK}: $f(x)$ \begin{CJK}{UTF8}{mj}在\end{CJK} $[a, b]$ \begin{CJK}{UTF8}{mj}上有界\end{CJK}.

(3) \begin{CJK}{UTF8}{mj}设函数列\end{CJK} $\left\{f_{n}(x)\right\}$ \begin{CJK}{UTF8}{mj}在\end{CJK} $[a, b]$ \begin{CJK}{UTF8}{mj}上一致收敛于\end{CJK} $f(x),\left\{g_{n}(x)\right\}$ \begin{CJK}{UTF8}{mj}在\end{CJK} $[a, b]$ \begin{CJK}{UTF8}{mj}上一致收敛于\end{CJK} $g(x)$. \begin{CJK}{UTF8}{mj}证明\end{CJK}: $\max \left\{f_{n}(x), g_{n}(x)\right\}$ \begin{CJK}{UTF8}{mj}在\end{CJK} $[a, b]$ \begin{CJK}{UTF8}{mj}上一致收敛于\end{CJK} $\max \{f(x), g(x)\}$.

(4) \begin{CJK}{UTF8}{mj}设数列\end{CJK} $\left\{a_{n}\right\}$ \begin{CJK}{UTF8}{mj}为\end{CJK} $(a, b)$ \begin{CJK}{UTF8}{mj}中互不相同的点列\end{CJK}, $a_{n}$ \begin{CJK}{UTF8}{mj}为函数\end{CJK} $f_{n}(x)$ \begin{CJK}{UTF8}{mj}在\end{CJK} $(a, b)$ \begin{CJK}{UTF8}{mj}上的唯一间断点\end{CJK}, $\left\{f_{n}(x)\right.$ : $n=1,2, \cdots\}$ \begin{CJK}{UTF8}{mj}在\end{CJK} $(a, b)$ \begin{CJK}{UTF8}{mj}上一致有界\end{CJK}. \begin{CJK}{UTF8}{mj}证明\end{CJK}: \begin{CJK}{UTF8}{mj}函数\end{CJK} $h(x)=\sum_{n=1}^{\infty} \frac{f_{n}(x)}{2^{n}}$ \begin{CJK}{UTF8}{mj}在\end{CJK} $(a, b)$ \begin{CJK}{UTF8}{mj}内的间断点集为\end{CJK} $\left\{a_{n}: n=\right.$ $1,2, \cdots\}$.

(5) \begin{CJK}{UTF8}{mj}设\end{CJK} $f(x)=\sum_{n=1}^{\infty} n e^{-n} \cos n x, x \in[0,2 \pi]$, \begin{CJK}{UTF8}{mj}证明\end{CJK}:

(i) $f(x)$ \begin{CJK}{UTF8}{mj}在\end{CJK} $[0,2 \pi]$ \begin{CJK}{UTF8}{mj}上连续\end{CJK};

(ii) $f^{\prime}(x)$ \begin{CJK}{UTF8}{mj}在\end{CJK} $[0,2 \pi]$ \begin{CJK}{UTF8}{mj}上存在且连续\end{CJK};

(iii) $\max _{x \in[0,2 \pi]}|f(x)|=\frac{e}{(e-1)^{2}}$.

(6)(i) \begin{CJK}{UTF8}{mj}设\end{CJK} $f(x)$ \begin{CJK}{UTF8}{mj}在\end{CJK} $(-\infty,+\infty)$ \begin{CJK}{UTF8}{mj}上可导\end{CJK}, \begin{CJK}{UTF8}{mj}存在数列\end{CJK} $\left\{x_{n}\right\},\left\{y_{n}\right\}$, \begin{CJK}{UTF8}{mj}且\end{CJK} $\lim _{n \rightarrow \infty} x_{n}=+\infty, \lim _{n \rightarrow \infty} y_{n}=-\infty$. \begin{CJK}{UTF8}{mj}证明\end{CJK}: \begin{CJK}{UTF8}{mj}若\end{CJK} $\lim _{n \rightarrow \infty} f\left(x_{n}\right)=\lim _{n \rightarrow \infty} f\left(y_{n}\right)=\beta \in(-\infty,+\infty)$, \begin{CJK}{UTF8}{mj}则存在\end{CJK} $\xi \in(-\infty,+\infty)$, \begin{CJK}{UTF8}{mj}使得\end{CJK} $f^{\prime}(\xi)=0$.

(ii) \begin{CJK}{UTF8}{mj}设\end{CJK} $f(x), g(x)$ \begin{CJK}{UTF8}{mj}在\end{CJK} $(-\infty,+\infty)$ \begin{CJK}{UTF8}{mj}上可导\end{CJK}, \begin{CJK}{UTF8}{mj}且\end{CJK} $g^{\prime}(x) \neq 0, x \in(-\infty,+\infty)$, \begin{CJK}{UTF8}{mj}存在数列\end{CJK} $\left\{x_{n}\right\},\left\{y_{n}\right\},\left\{x_{n}^{\prime}\right\}$, $\left\{y_{n}^{\prime}\right\}$, \begin{CJK}{UTF8}{mj}满足\end{CJK}:

\begin{CJK}{UTF8}{mj}证明\end{CJK}: \begin{CJK}{UTF8}{mj}存在\end{CJK} $\xi \in(-\infty,+\infty)$, \begin{CJK}{UTF8}{mj}使\end{CJK} $\frac{f^{\prime}(\xi)}{g^{\prime}(\xi)}=\frac{B-A}{b-a}$.
$$
\begin{aligned}
& \lim _{n \rightarrow \infty} x_{n}=+\infty, \quad \lim _{n \rightarrow \infty} y_{n}=-\infty, \quad \lim _{n \rightarrow \infty} x_{n}^{\prime}=+\infty, \quad \lim _{n \rightarrow \infty} y_{n}^{\prime}=-\infty \text {; } \\
& \lim _{n \rightarrow \infty} f\left(x_{n}\right)=B \in(-\infty,+\infty), \quad \lim _{n \rightarrow \infty} f\left(y_{n}\right)=A \in(-\infty,+\infty) \\
& \lim _{n \rightarrow \infty} g\left(x_{n}^{\prime}\right)=b \in(-\infty,+\infty), \quad \lim _{n \rightarrow \infty} g\left(y_{n}^{\prime}\right)=a \in(-\infty,+\infty) . 
\end{aligned}
$$

\section{$1.142009$ 年}
\begin{CJK}{UTF8}{mj}一\end{CJK}、\begin{CJK}{UTF8}{mj}判断下列命题是否正确\end{CJK}, \begin{CJK}{UTF8}{mj}若正确给出证明\end{CJK}, \begin{CJK}{UTF8}{mj}若错误举出反例\end{CJK}(\begin{CJK}{UTF8}{mj}每小题\end{CJK} 9 \begin{CJK}{UTF8}{mj}分\end{CJK}, \begin{CJK}{UTF8}{mj}共\end{CJK} 54 \begin{CJK}{UTF8}{mj}分\end{CJK})

(1) \begin{CJK}{UTF8}{mj}设\end{CJK} $\lim _{x \rightarrow a} g(x)=A, \lim _{y \rightarrow A} f(y)=B$, \begin{CJK}{UTF8}{mj}此处\end{CJK} $a, A, B$ \begin{CJK}{UTF8}{mj}均为实数\end{CJK}. \begin{CJK}{UTF8}{mj}则\end{CJK}
$$
\lim _{x \rightarrow a} f(g(x))=B
$$
(2) \begin{CJK}{UTF8}{mj}设\end{CJK} $f(x)$ \begin{CJK}{UTF8}{mj}为闭区间\end{CJK} $[a, b]$ \begin{CJK}{UTF8}{mj}上不恒为\end{CJK} 0 \begin{CJK}{UTF8}{mj}的连续函数\end{CJK}, $D(x)$ \begin{CJK}{UTF8}{mj}为\end{CJK} Dirichlet \begin{CJK}{UTF8}{mj}函数\end{CJK}, \begin{CJK}{UTF8}{mj}则\end{CJK} $f(x) D(x)$ \begin{CJK}{UTF8}{mj}在\end{CJK} $[a, b]$ \begin{CJK}{UTF8}{mj}上不可积\end{CJK}.

(3) \begin{CJK}{UTF8}{mj}存在实数\end{CJK} $a_{0}, a_{n}, b_{n}(n=1,2, \cdots)$ \begin{CJK}{UTF8}{mj}使得\end{CJK}
$$
\frac{a_{0}}{2}+\sum_{n=1}^{\infty}\left(a_{n} \cos n x+b_{n} \sin n x\right)=\left\{\begin{array}{l}
1, x \in[1,2] \\
0, x \in[4,5]
\end{array} .\right.
$$
(4) \begin{CJK}{UTF8}{mj}已知\end{CJK} $f(x)$ \begin{CJK}{UTF8}{mj}在\end{CJK} $x=2$ \begin{CJK}{UTF8}{mj}处连续\end{CJK}, \begin{CJK}{UTF8}{mj}且\end{CJK} $\lim _{x \rightarrow 2} \frac{f(x)}{x-2}=1$, \begin{CJK}{UTF8}{mj}则\end{CJK} $f(x)$ \begin{CJK}{UTF8}{mj}在\end{CJK} $x=2$ \begin{CJK}{UTF8}{mj}处可导\end{CJK}.

(5) \begin{CJK}{UTF8}{mj}如果\end{CJK} $f(x)$ \begin{CJK}{UTF8}{mj}在\end{CJK} $x_{0}$ \begin{CJK}{UTF8}{mj}处可导\end{CJK}, \begin{CJK}{UTF8}{mj}则\end{CJK} $f(x)$ \begin{CJK}{UTF8}{mj}在\end{CJK} $x_{0}$ \begin{CJK}{UTF8}{mj}的一个邻域上连续\end{CJK}.

(6) \begin{CJK}{UTF8}{mj}若多项式函数列\end{CJK} $\left\{f_{n}(x)\right\}$ \begin{CJK}{UTF8}{mj}在\end{CJK} $(-\infty,+\infty)$ \begin{CJK}{UTF8}{mj}上一致收敛于函数\end{CJK} $f(x)$, \begin{CJK}{UTF8}{mj}则\end{CJK} $f(x)$ \begin{CJK}{UTF8}{mj}必是多项式函数\end{CJK}.

\section{二、求解下列各题 (每小题 12 分, 共 36 分)}
(1) \begin{CJK}{UTF8}{mj}设\end{CJK} $a>0, a \neq 1$. \begin{CJK}{UTF8}{mj}求极限\end{CJK}
$$
\lim _{x \rightarrow+\infty}\left(\frac{a^{x}-1}{(a-1) x}\right)^{\frac{1}{x}}
$$
(2) \begin{CJK}{UTF8}{mj}设圆盘\end{CJK} $(x-a)^{2}+(y-b)^{2} \leqslant R^{2}$ \begin{CJK}{UTF8}{mj}上各点的密度等于该点到圆心的距离\end{CJK}, \begin{CJK}{UTF8}{mj}求此圆盘的质量\end{CJK}.

(3) \begin{CJK}{UTF8}{mj}设\end{CJK} $S$ \begin{CJK}{UTF8}{mj}为\end{CJK} $\mathbb{R}^{3}$ \begin{CJK}{UTF8}{mj}中封闭光滑曲面\end{CJK}, l \begin{CJK}{UTF8}{mj}为任意固定方向\end{CJK}, $\mathbf{n}$ \begin{CJK}{UTF8}{mj}为曲面\end{CJK} $S$ \begin{CJK}{UTF8}{mj}的外法线方向\end{CJK}. \begin{CJK}{UTF8}{mj}求\end{CJK}
$$
\iint_{S} \cos (\mathbf{n}, \mathbf{l}) \mathrm{d} S \text {. }
$$

\section{三、证明下列各题 (每小题 10 分, 共 60 分)}
(1) \begin{CJK}{UTF8}{mj}设\end{CJK} $P_{0}$ \begin{CJK}{UTF8}{mj}是曲面\end{CJK} $S: \frac{x^{2}}{a^{2}}+\frac{y^{2}}{b^{2}}+\frac{z^{2}}{c^{2}}=1$ \begin{CJK}{UTF8}{mj}外的\end{CJK} $一$ \begin{CJK}{UTF8}{mj}点\end{CJK}, $P_{1} \in S$. \begin{CJK}{UTF8}{mj}若\end{CJK} $\left|\overline{P_{0} P_{1}}\right|=\max _{P \in S}\left|\overline{P_{0} P}\right|$, \begin{CJK}{UTF8}{mj}求证直线\end{CJK} $P_{0} P$ \begin{CJK}{UTF8}{mj}是\end{CJK} $S$ \begin{CJK}{UTF8}{mj}在\end{CJK} $P_{1}$ \begin{CJK}{UTF8}{mj}处的法线\end{CJK}.

(2) \begin{CJK}{UTF8}{mj}设函数\end{CJK}
$$
f(x, y)= \begin{cases}\frac{y^{3} \sin \frac{y}{x}}{x^{2}+y^{2}}, & x \neq 0 \\ 0, & x=0\end{cases}
$$
\begin{CJK}{UTF8}{mj}求证\end{CJK}: \begin{CJK}{UTF8}{mj}在原点处\end{CJK} $f(x, y)$ \begin{CJK}{UTF8}{mj}连续\end{CJK}、\begin{CJK}{UTF8}{mj}沿任何方向的方向导数存在\end{CJK}、\begin{CJK}{UTF8}{mj}但不可微\end{CJK}.

(3) \begin{CJK}{UTF8}{mj}设\end{CJK} $a<b, c<d$ \begin{CJK}{UTF8}{mj}均为实数\end{CJK}. \begin{CJK}{UTF8}{mj}已知\end{CJK} $f(x)$ \begin{CJK}{UTF8}{mj}在\end{CJK} $(a, b)$ \begin{CJK}{UTF8}{mj}上单调\end{CJK}, \begin{CJK}{UTF8}{mj}其值域为\end{CJK} $c<d$, \begin{CJK}{UTF8}{mj}求证\end{CJK} $f(x)$ \begin{CJK}{UTF8}{mj}在\end{CJK} $(a, b)$ \begin{CJK}{UTF8}{mj}上致连续\end{CJK}.

(4) \begin{CJK}{UTF8}{mj}设数列满足条件\end{CJK}:
$$
\forall n \in \mathbb{N}, a_{n}>0 \text { 且 } \lim _{n \rightarrow \infty} \frac{a_{n}}{a_{n+2}+a_{n+4}}=0 \text {. }
$$
\begin{CJK}{UTF8}{mj}求证\end{CJK} $\left\{a_{n}\right\}$ \begin{CJK}{UTF8}{mj}为无界数列\end{CJK}.

(5) \begin{CJK}{UTF8}{mj}设\end{CJK} $f(x)$ \begin{CJK}{UTF8}{mj}在\end{CJK} $[0,+\infty)$ \begin{CJK}{UTF8}{mj}上连续且有界\end{CJK}. \begin{CJK}{UTF8}{mj}证明对于任意正数\end{CJK} $T$, \begin{CJK}{UTF8}{mj}存在\end{CJK} $x_{n} \rightarrow+\infty$, \begin{CJK}{UTF8}{mj}使得\end{CJK}
$$
\lim _{n \rightarrow \infty}\left(f\left(x_{n}+T\right)-f\left(x_{n}\right)\right)=0
$$
(6) \begin{CJK}{UTF8}{mj}设函数\end{CJK} $f(x)$ \begin{CJK}{UTF8}{mj}在闭区间\end{CJK} $[a, b](a<b)$ \begin{CJK}{UTF8}{mj}上可积\end{CJK}, $\int_{a}^{b} f(x) \mathrm{d} x=0$. \begin{CJK}{UTF8}{mj}证明\end{CJK}: \begin{CJK}{UTF8}{mj}若对任意的\end{CJK} $x \in[a, b]$ \begin{CJK}{UTF8}{mj}有\end{CJK} $f(x) \neq 0$, \begin{CJK}{UTF8}{mj}则存在\end{CJK} $[c, d] \subseteq[a, b]$, \begin{CJK}{UTF8}{mj}使得对任意的\end{CJK} $x \in[c, d]$ \begin{CJK}{UTF8}{mj}有\end{CJK} $f(x)>0$.

\section{$1.152010$ 年}
\section{一、求解下夗各题 (每小题 12 分, 共 60 分)}
(1) \begin{CJK}{UTF8}{mj}设曲线\end{CJK} $\Gamma: x=x(t)=t^{2}, y=y(t)=e^{t}+2, z=z(t)=t+\cos t, t \in \mathbb{R}$. \begin{CJK}{UTF8}{mj}试求\end{CJK} $\Gamma$ \begin{CJK}{UTF8}{mj}在点\end{CJK} $(x(0), y(0), z(0))$ \begin{CJK}{UTF8}{mj}处的切线方程与法平面方程\end{CJK}.

(2) \begin{CJK}{UTF8}{mj}求由方程\end{CJK} $x^{2}+2 x y+2 y^{2}=1$ \begin{CJK}{UTF8}{mj}所确定的隐函数\end{CJK} $y=f(x)$ \begin{CJK}{UTF8}{mj}的极值\end{CJK}.

(3) \begin{CJK}{UTF8}{mj}计算\end{CJK}
$$
\iint_{S}(x-a)^{3} \mathrm{~d} y \mathrm{~d} z+(y-b)^{3} \mathrm{~d} z \mathrm{~d} x+(z-c)^{3} \mathrm{~d} x \mathrm{~d} y
$$
\begin{CJK}{UTF8}{mj}其中\end{CJK} $S$ \begin{CJK}{UTF8}{mj}是球面\end{CJK} $(x-a)^{2}+(y-b)^{2}+(z-c)^{2}=1$ \begin{CJK}{UTF8}{mj}的外侧\end{CJK}.

(4)\begin{CJK}{UTF8}{mj}求函数\end{CJK} $f(x)=\frac{e^{x}}{2-2 x}$ \begin{CJK}{UTF8}{mj}在\end{CJK} $x=0$ \begin{CJK}{UTF8}{mj}处的泰勒展开式\end{CJK}, \begin{CJK}{UTF8}{mj}并求\end{CJK} $f^{(n)}(0)$.

(5) \begin{CJK}{UTF8}{mj}设\end{CJK} $g(x, y)=\int_{0}^{+\infty} \frac{\arctan (x t) \cdot \arctan (y t)}{t^{2}} \mathrm{~d} t,(x, y) \in(0,+\infty) \times(0,+\infty)$, \begin{CJK}{UTF8}{mj}试求\end{CJK} $g_{x y}(x, y)$.

\section{二、证明下列各题 (每小题 12 分, 共 60 分)}
(1) \begin{CJK}{UTF8}{mj}已知\end{CJK} $f(x, y)$ \begin{CJK}{UTF8}{mj}在点\end{CJK} $\left(x_{0}, y_{0}\right)$ \begin{CJK}{UTF8}{mj}可微且\end{CJK} $f\left(x_{0}, y_{0}\right)=0, g(x, y)$ \begin{CJK}{UTF8}{mj}在点\end{CJK} $\left(x_{0}, y_{0}\right)$ \begin{CJK}{UTF8}{mj}连续\end{CJK}. \begin{CJK}{UTF8}{mj}试证\end{CJK} $f(x, y) g(x, y)$ \begin{CJK}{UTF8}{mj}在点\end{CJK} $\left(x_{0}, y_{0}\right)$ \begin{CJK}{UTF8}{mj}可微\end{CJK}, \begin{CJK}{UTF8}{mj}且\end{CJK} $\mathrm{d}(f g)\left(x_{0}, y_{0}\right)=g\left(x_{0}, y_{0}\right) \mathrm{d} f\left(x_{0}, y_{0}\right)$.

(2) \begin{CJK}{UTF8}{mj}设\end{CJK} $f(x)$ \begin{CJK}{UTF8}{mj}为定义在\end{CJK} $[a,+\infty)$ \begin{CJK}{UTF8}{mj}上的正值连续函数\end{CJK}. \begin{CJK}{UTF8}{mj}证明\end{CJK}: \begin{CJK}{UTF8}{mj}若\end{CJK} $\lim _{x \rightarrow+\infty} \frac{f(x+1)}{f(x)}=q<1$, \begin{CJK}{UTF8}{mj}则反常积分\end{CJK} $\int_{a}^{+\infty} f(x) \mathrm{d} x$ \begin{CJK}{UTF8}{mj}收敛\end{CJK}.

(3) \begin{CJK}{UTF8}{mj}证明\end{CJK}: (i) $\forall n \in \mathbb{N}_{+}$, \begin{CJK}{UTF8}{mj}关于\end{CJK} $x$ \begin{CJK}{UTF8}{mj}的方程\end{CJK} $\sum_{k=1}^{n} e^{k x}=n+1$ \begin{CJK}{UTF8}{mj}在\end{CJK} $[0,1]$ \begin{CJK}{UTF8}{mj}上必定存在唯一实根\end{CJK} (\begin{CJK}{UTF8}{mj}记为\end{CJK} $a_{n}$ );

(ii) \begin{CJK}{UTF8}{mj}数列\end{CJK} $\left\{a_{n}\right\}$ \begin{CJK}{UTF8}{mj}必定收敛\end{CJK}, \begin{CJK}{UTF8}{mj}并求其极限\end{CJK}.

(4) \begin{CJK}{UTF8}{mj}设\end{CJK} $f(x, y)$ \begin{CJK}{UTF8}{mj}在\end{CJK} $[a, b] \times[c, d](a<b, c<d)$ \begin{CJK}{UTF8}{mj}上连续\end{CJK}, \begin{CJK}{UTF8}{mj}令\end{CJK}
$$
M(x)=\max _{y \in[c, d]} f(x, y), x \in[a, b]
$$
\begin{CJK}{UTF8}{mj}证明\end{CJK}: $M(x)$ \begin{CJK}{UTF8}{mj}在\end{CJK} $[a, b]$ \begin{CJK}{UTF8}{mj}上连续\end{CJK}.

(5) \begin{CJK}{UTF8}{mj}设\end{CJK} $f(x)$ \begin{CJK}{UTF8}{mj}在\end{CJK} $[a, b]$ \begin{CJK}{UTF8}{mj}上可导\end{CJK} $(a<b)$. \begin{CJK}{UTF8}{mj}证明\end{CJK}: $f(x)$ \begin{CJK}{UTF8}{mj}在\end{CJK} $[a, b]$ \begin{CJK}{UTF8}{mj}上一致可导的充要条件是\end{CJK} $f^{\prime}(x)$ \begin{CJK}{UTF8}{mj}在\end{CJK} $[a, b]$ \begin{CJK}{UTF8}{mj}上连续\end{CJK}.

\begin{CJK}{UTF8}{mj}说明\end{CJK} $f(x)$ \begin{CJK}{UTF8}{mj}在\end{CJK} $[a, b]$ \begin{CJK}{UTF8}{mj}上致可导是指\end{CJK}: $\forall \varepsilon>0, \exists \delta>0$, \begin{CJK}{UTF8}{mj}使得\end{CJK} $\forall x, y \in[a, b]$, \begin{CJK}{UTF8}{mj}只要\end{CJK} $0<|x-y|<\delta$, \begin{CJK}{UTF8}{mj}就有\end{CJK} $\left|\frac{f(x)-f(y)}{x-y}-f^{\prime}(x)\right|<\varepsilon$ \begin{CJK}{UTF8}{mj}成立\end{CJK}.

\begin{CJK}{UTF8}{mj}三\end{CJK}、 $(\mathbf{3 0}$ \begin{CJK}{UTF8}{mj}分\end{CJK} $)$ \begin{CJK}{UTF8}{mj}设可积函数列\end{CJK} $\left\{f_{n}(x)\right\}$ \begin{CJK}{UTF8}{mj}在\end{CJK} $[a, b]$ \begin{CJK}{UTF8}{mj}上一致收敛于\end{CJK} $f(x)$. \begin{CJK}{UTF8}{mj}证明\end{CJK}:

(1) $f(x)$ \begin{CJK}{UTF8}{mj}在\end{CJK} $[a, b]$ \begin{CJK}{UTF8}{mj}上可积\end{CJK}, \begin{CJK}{UTF8}{mj}且\end{CJK} $\lim _{n \rightarrow \infty} \int_{a}^{b} f_{n}(x) \mathrm{d} x=\int_{a}^{b} f(x) \mathrm{d} x$;

(2) $\left\{f_{n}(x)\right\}$ \begin{CJK}{UTF8}{mj}在\end{CJK} $[a, b]$ \begin{CJK}{UTF8}{mj}上一致可积\end{CJK} $\left(\right.$ \begin{CJK}{UTF8}{mj}致可积是指\end{CJK}: $\forall \varepsilon>0, \exists \delta>0$, \begin{CJK}{UTF8}{mj}使得对任意分割\end{CJK} $T: a=x_{0}<$ $x_{1}<\cdots<x_{k}=b$, \begin{CJK}{UTF8}{mj}只要\end{CJK} $\max _{1 \leqslant i \leqslant k} \Delta x_{i}<\delta$, \begin{CJK}{UTF8}{mj}就有\end{CJK}
$$
\left|\int_{a}^{b} f_{n}(x) \mathrm{d} x-\sum_{i=1}^{k} f_{n}\left(\xi_{i}\right) \Delta x_{i}\right|<\varepsilon
$$
\begin{CJK}{UTF8}{mj}对任意\end{CJK} $\xi_{i} \in\left[x_{i-1}, x_{i}\right], 1 \leqslant i \leqslant k$, \begin{CJK}{UTF8}{mj}及任意\end{CJK} $n \in \mathbb{N}_{+}$\begin{CJK}{UTF8}{mj}都成立\end{CJK});

(3) \begin{CJK}{UTF8}{mj}举例说明\end{CJK} (2) \begin{CJK}{UTF8}{mj}的逆不真\end{CJK}.

\section{$1.162011$ 年}
\section{一、求解下列各题 (每小题 12 分,共 48 分)}
(1) \begin{CJK}{UTF8}{mj}设\end{CJK} $f(x, y)$ \begin{CJK}{UTF8}{mj}在\end{CJK} $D=[0,1] \times[0,1]$ \begin{CJK}{UTF8}{mj}上连续\end{CJK}, \begin{CJK}{UTF8}{mj}求\end{CJK}
$$
\lim _{n \rightarrow \infty}\left(\iint_{D}|f(x, y)|^{n} \mathrm{~d} x \mathrm{~d} y\right)^{\frac{1}{n}} .
$$
(2) \begin{CJK}{UTF8}{mj}求幂级数\end{CJK} $\sum_{n=1}^{\infty} n x^{2 n+1}$ \begin{CJK}{UTF8}{mj}的收敛域及和函数\end{CJK}.

(3) \begin{CJK}{UTF8}{mj}求椭球面\end{CJK} $\frac{x^{2}}{3}+(y-1)^{2}+\frac{z^{2}}{2}$ \begin{CJK}{UTF8}{mj}被平面\end{CJK} $2 x+y+z=1$ \begin{CJK}{UTF8}{mj}截成的椭圆的面积\end{CJK}.

(4) \begin{CJK}{UTF8}{mj}设\end{CJK} $f(x)$ \begin{CJK}{UTF8}{mj}在\end{CJK} $\mathbb{R}$ \begin{CJK}{UTF8}{mj}连续可微\end{CJK}, $L$ \begin{CJK}{UTF8}{mj}为逐段光滑闭曲线\end{CJK}, \begin{CJK}{UTF8}{mj}求\end{CJK}
$$
\oint_{L} f(\sin x+\sin y)(\cos x d x+\cos y d y)
$$

\section{二、证明下列各题 (每小题 12 分, 共 72 分)}
(1)\begin{CJK}{UTF8}{mj}用\end{CJK} $\varepsilon-\delta$ \begin{CJK}{UTF8}{mj}语㐭证明\end{CJK}
$$
\lim _{x \rightarrow 2^{-}}\left(\sqrt{\frac{1}{2-x}+2}-\sqrt{\frac{1}{2-x}-2}\right)=0 .
$$
(2) \begin{CJK}{UTF8}{mj}证明\end{CJK}: $F(\alpha)=\int_{0}^{+\infty} \frac{x^{2}}{1+2 x^{\alpha}} \mathrm{d} x$ \begin{CJK}{UTF8}{mj}在\end{CJK} $(3,+\infty)$ \begin{CJK}{UTF8}{mj}上连续\end{CJK}.

(3) \begin{CJK}{UTF8}{mj}设\end{CJK} $f(x)$ \begin{CJK}{UTF8}{mj}在\end{CJK} $[a, b](a<b$ \begin{CJK}{UTF8}{mj}为实数\end{CJK} $)$ \begin{CJK}{UTF8}{mj}上连续\end{CJK}, \begin{CJK}{UTF8}{mj}且\end{CJK} $f(a) f(b)<0$, \begin{CJK}{UTF8}{mj}试用有限覆盖定理证明存在\end{CJK} $\xi \in(a, b)$, \begin{CJK}{UTF8}{mj}使得\end{CJK} $f(\xi)=0$.

(4) \begin{CJK}{UTF8}{mj}设\end{CJK} $f(x) \in C[a,+\infty)$ (\begin{CJK}{UTF8}{mj}即\end{CJK} $f(x)$ \begin{CJK}{UTF8}{mj}在\end{CJK} $[a,+\infty)$ \begin{CJK}{UTF8}{mj}上连续\end{CJK}) \begin{CJK}{UTF8}{mj}且\end{CJK} $\int_{a}^{+\infty}|f(x)| \mathrm{d} x$ \begin{CJK}{UTF8}{mj}收敛\end{CJK}, \begin{CJK}{UTF8}{mj}则存在数列\end{CJK} $\left\{x_{n}\right\} \subset$ $[a,+\infty)$ \begin{CJK}{UTF8}{mj}满足条件\end{CJK}: $\lim _{n \rightarrow \infty} x_{n}=+\infty$ \begin{CJK}{UTF8}{mj}及\end{CJK} $\lim _{n \rightarrow \infty} x_{n} f\left(x_{n}\right)=0$.

(5) \begin{CJK}{UTF8}{mj}设\end{CJK} $f(x, y, z)$ \begin{CJK}{UTF8}{mj}在\end{CJK} $[a,+\infty) \times[b,+\infty) \times[c,+\infty)$ \begin{CJK}{UTF8}{mj}上连续且无下界\end{CJK}, \begin{CJK}{UTF8}{mj}对于任意的\end{CJK} $s \in \mathbb{R}, f(x, y, z)=s$ \begin{CJK}{UTF8}{mj}的解集为有界集\end{CJK}, \begin{CJK}{UTF8}{mj}证明\end{CJK}: $\lim _{(x, y, z) \rightarrow(+\infty,+\infty,+\infty)} f(x, y, z)=-\infty$.

(6) \begin{CJK}{UTF8}{mj}设\end{CJK} $\left\{I_{n}\right\}$ \begin{CJK}{UTF8}{mj}为\end{CJK} $\mathbb{R}^{2}$ \begin{CJK}{UTF8}{mj}中一列闭集并且满足\end{CJK}: \begin{CJK}{UTF8}{mj}对任意的\end{CJK} $n \in \mathbb{N}, I_{n} \subseteq I_{n+1},{ }_{n=1}^{\infty} I_{n}=(0,1) \times(0,1)$. \begin{CJK}{UTF8}{mj}证明\end{CJK} \begin{CJK}{UTF8}{mj}存在\end{CJK} $P_{0} \in(0,1) \times(0,1), r>0$ \begin{CJK}{UTF8}{mj}及\end{CJK} $k \in \mathbb{N}$, \begin{CJK}{UTF8}{mj}使得闭球\end{CJK} $B\left(P_{0} ; r\right) \subseteq I_{k}$.

\section{三、 证明下列结论 (每小题 10 分, 共 20 分)}
(1) \begin{CJK}{UTF8}{mj}设\end{CJK} $f(x)$ \begin{CJK}{UTF8}{mj}为\end{CJK} $\mathbb{R}$ \begin{CJK}{UTF8}{mj}上的实值函数\end{CJK}, \begin{CJK}{UTF8}{mj}证明\end{CJK}: $f(x)$ \begin{CJK}{UTF8}{mj}的极值所组成的集合是至多可数的\end{CJK}.

(2) \begin{CJK}{UTF8}{mj}设\end{CJK} $f(x)$ \begin{CJK}{UTF8}{mj}为\end{CJK} $\mathbb{R}$ \begin{CJK}{UTF8}{mj}上的连续函数且\end{CJK} $\mathbb{R}$ \begin{CJK}{UTF8}{mj}上每一点均为\end{CJK} $f(x)$ \begin{CJK}{UTF8}{mj}的极值点\end{CJK}, \begin{CJK}{UTF8}{mj}求证\end{CJK}: $f(x)$ \begin{CJK}{UTF8}{mj}为常值函数\end{CJK}.

\begin{CJK}{UTF8}{mj}四\end{CJK}、 $\left(\mathbf{1 0}\right.$ \begin{CJK}{UTF8}{mj}分\end{CJK} ) \begin{CJK}{UTF8}{mj}设\end{CJK} $f: \mathbb{R} \rightarrow \mathbb{R}$ \begin{CJK}{UTF8}{mj}是连续函数\end{CJK}, \begin{CJK}{UTF8}{mj}记\end{CJK} $f^{n}=f \circ f \circ \cdots \circ f$ \begin{CJK}{UTF8}{mj}为\end{CJK} $f(x)$ \begin{CJK}{UTF8}{mj}的\end{CJK} $n$ \begin{CJK}{UTF8}{mj}次复合\end{CJK}. \begin{CJK}{UTF8}{mj}又设\end{CJK} $f(x)$ \begin{CJK}{UTF8}{mj}没有不动点\end{CJK}, \begin{CJK}{UTF8}{mj}即对任意\end{CJK} $x_{0} \in \mathbb{R}$ \begin{CJK}{UTF8}{mj}均有\end{CJK} $f\left(x_{0}\right) \neq x_{0}$. \begin{CJK}{UTF8}{mj}证明\end{CJK}: \begin{CJK}{UTF8}{mj}对任意\end{CJK} $x_{0} \in \mathbb{R}$, \begin{CJK}{UTF8}{mj}数列\end{CJK}
$$
f^{n}\left(x_{0}\right), \quad n=1,2, \ldots
$$
\begin{CJK}{UTF8}{mj}是无界的\end{CJK}.

\section{$1.172012$ 年}
\begin{CJK}{UTF8}{mj}一\end{CJK}、\begin{CJK}{UTF8}{mj}判断下列命题是否正确\end{CJK}, \begin{CJK}{UTF8}{mj}若正确给出证明\end{CJK}, \begin{CJK}{UTF8}{mj}若错误举出反例\end{CJK} (\begin{CJK}{UTF8}{mj}每小题\end{CJK} 8 \begin{CJK}{UTF8}{mj}分\end{CJK}, \begin{CJK}{UTF8}{mj}共\end{CJK} 32 \begin{CJK}{UTF8}{mj}分\end{CJK})

(1). \begin{CJK}{UTF8}{mj}如果函数\end{CJK} $f(x)$ \begin{CJK}{UTF8}{mj}在区间\end{CJK} $[a, b]$ \begin{CJK}{UTF8}{mj}上可积\end{CJK}, \begin{CJK}{UTF8}{mj}则函数\end{CJK} $f(x)$ \begin{CJK}{UTF8}{mj}在\end{CJK} $[a, b]$ \begin{CJK}{UTF8}{mj}上存在原函数\end{CJK}.

(2). \begin{CJK}{UTF8}{mj}如果函数\end{CJK} $f(x, y, z)$ \begin{CJK}{UTF8}{mj}是有向光滑曲面\end{CJK} $S$ \begin{CJK}{UTF8}{mj}上的非负连续函数\end{CJK}, \begin{CJK}{UTF8}{mj}则\end{CJK} $\iint_{S} f(x, y, z) \mathrm{d} y \mathrm{~d} z \geqslant 0$.

(3). \begin{CJK}{UTF8}{mj}设函数\end{CJK} $f(x)$ \begin{CJK}{UTF8}{mj}在点\end{CJK} $x_{0}$ \begin{CJK}{UTF8}{mj}的空心邻域\end{CJK} $U_{+}^{o}\left(x_{0}\right)$ \begin{CJK}{UTF8}{mj}内有定义\end{CJK}. \begin{CJK}{UTF8}{mj}如果对于任何满足条件\end{CJK}
$$
\left\{x_{n}\right\} \subset U_{+}^{o}\left(x_{0}\right), \lim _{n \rightarrow+\infty} x_{n}=x_{0}, x_{n+1}<x_{n}\left(n \in \mathbb{N}_{+}\right)
$$
\begin{CJK}{UTF8}{mj}的数列\end{CJK} $\left\{x_{n}\right\}$ \begin{CJK}{UTF8}{mj}都有\end{CJK} $\lim _{n \rightarrow+\infty} f\left(x_{n}\right)$ \begin{CJK}{UTF8}{mj}存在\end{CJK}, \begin{CJK}{UTF8}{mj}则\end{CJK} $\lim _{x \rightarrow x_{0}^{+}} f(x)$ \begin{CJK}{UTF8}{mj}存在\end{CJK}.

(4). \begin{CJK}{UTF8}{mj}如果级数\end{CJK} $\sum_{n=1}^{\infty} a_{n}^{2}$ \begin{CJK}{UTF8}{mj}收敛\end{CJK}, \begin{CJK}{UTF8}{mj}则级数\end{CJK} $\sum_{n=1}^{\infty} \frac{a_{n}}{n}$ \begin{CJK}{UTF8}{mj}收敛\end{CJK}.

\begin{CJK}{UTF8}{mj}二\end{CJK}、\begin{CJK}{UTF8}{mj}求解下列各题\end{CJK} (\begin{CJK}{UTF8}{mj}每小题\end{CJK} 8 \begin{CJK}{UTF8}{mj}分\end{CJK}, \begin{CJK}{UTF8}{mj}共\end{CJK} 40 \begin{CJK}{UTF8}{mj}分\end{CJK})

(1). \begin{CJK}{UTF8}{mj}求极限\end{CJK}
$$
\lim _{x \rightarrow 0} \frac{e^{x} \sin x-x(1+x)}{x^{3}}
$$
(2). \begin{CJK}{UTF8}{mj}设\end{CJK} $p>0,0<\alpha<1$. \begin{CJK}{UTF8}{mj}讨论级数\end{CJK} $\sum_{n=1}^{\infty}\left(\left(1+\frac{p}{n}\right)^{\alpha}-1-\frac{1}{n}\right)$ \begin{CJK}{UTF8}{mj}的敛散性\end{CJK}.

(3). \begin{CJK}{UTF8}{mj}计算积分\end{CJK} $\iint_{S} y z \mathrm{~d} x \mathrm{~d} y+z x \mathrm{~d} y \mathrm{~d} z+x y \mathrm{~d} z \mathrm{~d} x$, \begin{CJK}{UTF8}{mj}其中\end{CJK} $S$ \begin{CJK}{UTF8}{mj}是由\end{CJK} $z=1, x^{2}+y^{2}=1$ \begin{CJK}{UTF8}{mj}以及三个坐标面\end{CJK} \begin{CJK}{UTF8}{mj}所围成的第一卦限部分的外侧\end{CJK}.

(4). \begin{CJK}{UTF8}{mj}求函数\end{CJK} $f(x, y)=x+x y^{2}+y^{2}$ \begin{CJK}{UTF8}{mj}在\end{CJK} $x^{2}+y^{2} \leqslant 1$ \begin{CJK}{UTF8}{mj}上的最大值和最小值\end{CJK}.

(5). \begin{CJK}{UTF8}{mj}讨论无穷积分\end{CJK} $\int_{1}^{+\infty} \frac{\left(e^{\frac{1}{x^{2}}}-1\right)^{a}}{\ln ^{b}\left(1+\frac{1}{x}\right)}$ \begin{CJK}{UTF8}{mj}的敛散性\end{CJK}, \begin{CJK}{UTF8}{mj}其中\end{CJK} $a, b$ \begin{CJK}{UTF8}{mj}为常数\end{CJK}.

\begin{CJK}{UTF8}{mj}三\end{CJK}、\begin{CJK}{UTF8}{mj}证明下列各题\end{CJK} (\begin{CJK}{UTF8}{mj}每小题\end{CJK} 13 \begin{CJK}{UTF8}{mj}分\end{CJK}, \begin{CJK}{UTF8}{mj}共\end{CJK} 78 \begin{CJK}{UTF8}{mj}分\end{CJK})

(1). \begin{CJK}{UTF8}{mj}设函数\end{CJK} $f(x, y, z)$ \begin{CJK}{UTF8}{mj}和\end{CJK} $g(x, y, z)$ \begin{CJK}{UTF8}{mj}都在可求长的连续曲线\end{CJK} $L \subset \mathbb{R}^{3}$ \begin{CJK}{UTF8}{mj}上连续\end{CJK}, \begin{CJK}{UTF8}{mj}且\end{CJK} $g(x, y, z)$ \begin{CJK}{UTF8}{mj}是非负\end{CJK} \begin{CJK}{UTF8}{mj}的\end{CJK}. \begin{CJK}{UTF8}{mj}证明\end{CJK}: \begin{CJK}{UTF8}{mj}存在\end{CJK} $(\xi, \eta, \zeta) \in L$, \begin{CJK}{UTF8}{mj}使得\end{CJK}
$$
\int_{L} f(x, y, z) g(x, y, z) \mathrm{d} s=f(\xi, \eta, \zeta) \int_{L} g(x, y, z) \mathrm{d} s
$$
(2). \begin{CJK}{UTF8}{mj}设\end{CJK} $f(x)$ \begin{CJK}{UTF8}{mj}为\end{CJK} $[0,+\infty)$ \begin{CJK}{UTF8}{mj}上的连续可微的凹函数\end{CJK}, \begin{CJK}{UTF8}{mj}且\end{CJK} $\lim _{x \rightarrow+\infty} f(x)$ \begin{CJK}{UTF8}{mj}存在\end{CJK}. \begin{CJK}{UTF8}{mj}证明\end{CJK}: $\lim _{x \rightarrow+\infty} f^{\prime}(x)=0$.

(3). \begin{CJK}{UTF8}{mj}设\end{CJK} $f(x)$ \begin{CJK}{UTF8}{mj}在\end{CJK} $[0,1]$ \begin{CJK}{UTF8}{mj}上连续可微\end{CJK}, \begin{CJK}{UTF8}{mj}且\end{CJK}
$$
f(0)=0,0<f^{\prime}(x)<1
$$
\begin{CJK}{UTF8}{mj}证明\end{CJK}:
$$
\int_{0}^{t}(f(x))^{3} \mathrm{~d} x \leqslant\left(\int_{0}^{t} f(x) \mathrm{d} x\right)^{2}, \quad t \in[0,1]
$$
(4). \begin{CJK}{UTF8}{mj}设函数\end{CJK}
$$
f(x)=\sum_{n=1}^{\infty} \frac{\left|x-\frac{1}{n}\right|}{2^{n}}, \quad x \in(0,+\infty) .
$$
\begin{CJK}{UTF8}{mj}证明\end{CJK}: (i) \begin{CJK}{UTF8}{mj}函数\end{CJK} $f(x)$ \begin{CJK}{UTF8}{mj}在点\end{CJK} $x \neq \frac{1}{n}(n=1,2, \cdots)$ \begin{CJK}{UTF8}{mj}处可导\end{CJK}.

(ii) \begin{CJK}{UTF8}{mj}函数\end{CJK} $f(x)$ \begin{CJK}{UTF8}{mj}在点\end{CJK} $x=\frac{1}{n}(n=1,2, \cdots)$ \begin{CJK}{UTF8}{mj}处不可导\end{CJK}. (5). \begin{CJK}{UTF8}{mj}设\end{CJK} $D$ \begin{CJK}{UTF8}{mj}为有界闭区域\end{CJK}, $f(x, y)$ \begin{CJK}{UTF8}{mj}在\end{CJK} $D$ \begin{CJK}{UTF8}{mj}上连续\end{CJK}、\begin{CJK}{UTF8}{mj}一阶偏导数存在\end{CJK}, \begin{CJK}{UTF8}{mj}且\end{CJK}
$$
\frac{\partial f}{\partial x}+\frac{\partial f}{\partial y}=f,\left.\quad f\right|_{\partial D}=0
$$
\begin{CJK}{UTF8}{mj}证明\end{CJK}: $f(x, y)=0,(x, y) \in D$.

(6). \begin{CJK}{UTF8}{mj}设函数\end{CJK} $f(x)$ \begin{CJK}{UTF8}{mj}在\end{CJK} $[0,1]$ \begin{CJK}{UTF8}{mj}上连续可微\end{CJK}, \begin{CJK}{UTF8}{mj}且满足\end{CJK}
$$
f^{\prime \prime}(x) \leqslant 0, \quad 0 \leqslant f(x) \leqslant 1, \quad f(0)=f(1)=0
$$
\begin{CJK}{UTF8}{mj}证明\end{CJK}: \begin{CJK}{UTF8}{mj}曲线\end{CJK} $C: y=f(x), x \in[0,1]$ \begin{CJK}{UTF8}{mj}的弧长\end{CJK} $s \leqslant 3$.

\section{$1.182013$ 年}
\begin{CJK}{UTF8}{mj}一\end{CJK}、\begin{CJK}{UTF8}{mj}判断下列命题是否正确\end{CJK}, \begin{CJK}{UTF8}{mj}若正确给出证明\end{CJK}, \begin{CJK}{UTF8}{mj}若错误举出反例\end{CJK}(\begin{CJK}{UTF8}{mj}每小题\end{CJK} 6 \begin{CJK}{UTF8}{mj}分\end{CJK}, \begin{CJK}{UTF8}{mj}共\end{CJK} 36 \begin{CJK}{UTF8}{mj}分\end{CJK})

(1). \begin{CJK}{UTF8}{mj}如果对任何\end{CJK} $n \in \mathbb{N}_{+}, a_{n}>0$, \begin{CJK}{UTF8}{mj}且\end{CJK} $\lim _{n \rightarrow \infty} a_{n}=0$, \begin{CJK}{UTF8}{mj}则级数\end{CJK} $\sum_{n=1}^{\infty}(-1)^{n} a_{n}$ \begin{CJK}{UTF8}{mj}收敛\end{CJK}.

(2). \begin{CJK}{UTF8}{mj}如果函数\end{CJK} $f(u)$ \begin{CJK}{UTF8}{mj}在点\end{CJK} $u_{0}$ \begin{CJK}{UTF8}{mj}连续\end{CJK}, \begin{CJK}{UTF8}{mj}且\end{CJK} $\lim _{x \rightarrow x_{0}} g(x)=u_{0}$, \begin{CJK}{UTF8}{mj}则\end{CJK} $\lim _{x \rightarrow x_{0}} f(g(x))=f\left(u_{0}\right)$.

(3). \begin{CJK}{UTF8}{mj}如果对任何\end{CJK} $n \in \mathbb{N}_{+}, u_{n}(x)$ \begin{CJK}{UTF8}{mj}在\end{CJK} $[a, b]$ \begin{CJK}{UTF8}{mj}上连续\end{CJK}, \begin{CJK}{UTF8}{mj}级数\end{CJK} $\sum_{n=1}^{\infty} u_{n}(x)$ \begin{CJK}{UTF8}{mj}在\end{CJK} $[a, b]$ \begin{CJK}{UTF8}{mj}上收敛\end{CJK}, \begin{CJK}{UTF8}{mj}且和函数在\end{CJK} $[a, b]$ \begin{CJK}{UTF8}{mj}上连续\end{CJK}, \begin{CJK}{UTF8}{mj}则级数\end{CJK} $\sum_{n=1}^{\infty} u_{n}(x)$ \begin{CJK}{UTF8}{mj}在\end{CJK} $[a, b]$ \begin{CJK}{UTF8}{mj}上一致收敛\end{CJK}.

(4). \begin{CJK}{UTF8}{mj}如果函数\end{CJK} $f(x)$ \begin{CJK}{UTF8}{mj}在有限区间\end{CJK} $(a, b)$ \begin{CJK}{UTF8}{mj}上连续且有界\end{CJK}, \begin{CJK}{UTF8}{mj}则\end{CJK} $f(x)$ \begin{CJK}{UTF8}{mj}在\end{CJK} $(a, b)$ \begin{CJK}{UTF8}{mj}上一致连续\end{CJK}.

(5). \begin{CJK}{UTF8}{mj}如果函数\end{CJK} $f(x, y)$ \begin{CJK}{UTF8}{mj}在点\end{CJK} $P_{0}\left(x_{0}, y_{0}\right)$ \begin{CJK}{UTF8}{mj}处的两个二阶偏导数\end{CJK} $f_{x y}\left(P_{0}\right)$ \begin{CJK}{UTF8}{mj}与\end{CJK} $f_{y x}\left(P_{0}\right)$ \begin{CJK}{UTF8}{mj}都存在\end{CJK}, \begin{CJK}{UTF8}{mj}则\end{CJK} $f_{x y}\left(P_{0}\right)=f_{y x}\left(P_{0}\right) .$

(6). \begin{CJK}{UTF8}{mj}如果\end{CJK} $f(x)$ \begin{CJK}{UTF8}{mj}在\end{CJK} $[a,+\infty)$ \begin{CJK}{UTF8}{mj}上可导\end{CJK}, \begin{CJK}{UTF8}{mj}且\end{CJK} $\lim _{x \rightarrow+\infty} f(x)=A \in \mathbb{R}$, \begin{CJK}{UTF8}{mj}则\end{CJK} $\lim _{x \rightarrow+\infty} f^{\prime}(x)=0$.

\begin{CJK}{UTF8}{mj}二\end{CJK}、\begin{CJK}{UTF8}{mj}求解下列各题\end{CJK} (\begin{CJK}{UTF8}{mj}每小题\end{CJK} 9 \begin{CJK}{UTF8}{mj}分\end{CJK}, \begin{CJK}{UTF8}{mj}共\end{CJK} 36 \begin{CJK}{UTF8}{mj}分\end{CJK})

(1). \begin{CJK}{UTF8}{mj}求和\end{CJK}
$$
\sum_{n=2}^{\infty} \frac{(-1)^{n}}{n^{2}+n-2}
$$
(2). \begin{CJK}{UTF8}{mj}计算积分\end{CJK} $\int_{L} x y \mathrm{~d} s$, \begin{CJK}{UTF8}{mj}其中\end{CJK} $L$ \begin{CJK}{UTF8}{mj}为曲线\end{CJK} $x^{2}+y^{2}+z^{2}=1$ \begin{CJK}{UTF8}{mj}与平面\end{CJK} $x+y+z=0$ \begin{CJK}{UTF8}{mj}的交线在第一卦限\end{CJK} \begin{CJK}{UTF8}{mj}中的咅分\end{CJK}.

(3). \begin{CJK}{UTF8}{mj}计算积分\end{CJK}
$$
\int_{0}^{1} \frac{1}{\left[\frac{1}{x}\right]} d x
$$
(4). \begin{CJK}{UTF8}{mj}求极限\end{CJK}
$$
\lim _{x \rightarrow 0^{+}} \frac{\int_{1}^{+\infty} \frac{e^{-x y}-1}{y^{3}} \mathrm{~d} y}{\ln (1+x)}
$$

\section{三、证明下列各题 (每小题 13 分, 共 78 分)}
(1). \begin{CJK}{UTF8}{mj}证明\end{CJK}: \begin{CJK}{UTF8}{mj}对于任何自然数\end{CJK} $n$, \begin{CJK}{UTF8}{mj}有\end{CJK}
$$
0<e-\left(1+\frac{1}{n}\right)^{n}<\frac{3}{n} \text {. }
$$
(2). \begin{CJK}{UTF8}{mj}设\end{CJK} $g(x)$ \begin{CJK}{UTF8}{mj}在\end{CJK} $[a,+\infty)$ \begin{CJK}{UTF8}{mj}上不变号\end{CJK}, $\int_{a}^{+\infty} g(x) \mathrm{d} x$ \begin{CJK}{UTF8}{mj}收敛\end{CJK}, \begin{CJK}{UTF8}{mj}且\end{CJK} $f(x)$ \begin{CJK}{UTF8}{mj}在\end{CJK} $[a,+\infty)$ \begin{CJK}{UTF8}{mj}上连续有界\end{CJK}. \begin{CJK}{UTF8}{mj}证明\end{CJK}: \begin{CJK}{UTF8}{mj}存\end{CJK} \begin{CJK}{UTF8}{mj}在\end{CJK} $\xi \in[a,+\infty)$, \begin{CJK}{UTF8}{mj}使得\end{CJK}
$$
\int_{a}^{+\infty} f(x) g(x) \mathrm{d} x=f(\xi) \int_{a}^{+\infty} g(x) \mathrm{d} x
$$
(3). \begin{CJK}{UTF8}{mj}设\end{CJK} $f(x)$ \begin{CJK}{UTF8}{mj}在\end{CJK} $[a,+\infty)$ \begin{CJK}{UTF8}{mj}上可微\end{CJK}, $0 \leqslant q<1$, \begin{CJK}{UTF8}{mj}且\end{CJK} $\lim _{x \rightarrow+\infty} x^{p} f^{\prime}(x)=A \in \mathbb{R}$. \begin{CJK}{UTF8}{mj}证明\end{CJK}: $f(x)$ \begin{CJK}{UTF8}{mj}在\end{CJK} $[a,+\infty)$ \begin{CJK}{UTF8}{mj}上\end{CJK} \begin{CJK}{UTF8}{mj}一致连续\end{CJK}.

(4). \begin{CJK}{UTF8}{mj}设函数\end{CJK} $f(x)$ \begin{CJK}{UTF8}{mj}在\end{CJK} $(-\infty,+\infty)$ \begin{CJK}{UTF8}{mj}上连续\end{CJK}, $a$ \begin{CJK}{UTF8}{mj}和\end{CJK} $b$ \begin{CJK}{UTF8}{mj}是常数\end{CJK}. \begin{CJK}{UTF8}{mj}证明\end{CJK}:
$$
\iint_{x^{2}+y^{2} \leqslant 1} f(a x+b y) \mathrm{d} x \mathrm{~d} y=\iint_{u^{2}+v^{2} \leqslant 1} f\left(u \sqrt{a^{2}+b^{2}}\right) \mathrm{d} u \mathrm{~d} v
$$
(5). \begin{CJK}{UTF8}{mj}设函数\end{CJK} $f(x, y)$ \begin{CJK}{UTF8}{mj}和\end{CJK} $g(x, y)$ \begin{CJK}{UTF8}{mj}在区域\end{CJK} $[0,1] \times[0,1]$ \begin{CJK}{UTF8}{mj}上连续\end{CJK}, \begin{CJK}{UTF8}{mj}且存在\end{CJK} $[0,1] \times[0,1]$ \begin{CJK}{UTF8}{mj}中的点列\end{CJK} $\left\{P_{n}\left(x_{n}, y_{n}\right)\right\}$ \begin{CJK}{UTF8}{mj}使得\end{CJK}
$$
f\left(x_{n}, y_{n}\right)=g\left(x_{n+1}, y_{n+1}\right), \quad n \in \mathbb{N}_{+}
$$
\begin{CJK}{UTF8}{mj}证明\end{CJK}: \begin{CJK}{UTF8}{mj}存在\end{CJK} $\left(x_{0}, y_{0}\right) \in[0,1] \times[0,1]$, \begin{CJK}{UTF8}{mj}使得\end{CJK} $f\left(x_{0}, y_{0}\right)=g\left(x_{0}, y_{0}\right)$.

(6). \begin{CJK}{UTF8}{mj}设\end{CJK} $f(x)$ \begin{CJK}{UTF8}{mj}在\end{CJK} $(-\infty,+\infty)$ \begin{CJK}{UTF8}{mj}上连续可微\end{CJK}, $\lim _{x \rightarrow+\infty} f^{\prime}(x)=A \in \mathbb{R}$, \begin{CJK}{UTF8}{mj}且对于任何\end{CJK} $x \in(-\infty,+\infty)$ \begin{CJK}{UTF8}{mj}有\end{CJK}
$$
f(x+1)-f(x)=f^{\prime}(x) .
$$
\begin{CJK}{UTF8}{mj}证明\end{CJK}: $f(x)$ \begin{CJK}{UTF8}{mj}是线性函数\end{CJK}, \begin{CJK}{UTF8}{mj}即存在常数\end{CJK} $a$ \begin{CJK}{UTF8}{mj}和\end{CJK} $b$ \begin{CJK}{UTF8}{mj}使得\end{CJK} $f(x)=a x+b, x \in(-\infty,+\infty)$.

\section{$1.192014$ 年}
\begin{CJK}{UTF8}{mj}一\end{CJK}、\begin{CJK}{UTF8}{mj}判断下列命题是否正确\end{CJK}, \begin{CJK}{UTF8}{mj}若正确给出证明\end{CJK}, \begin{CJK}{UTF8}{mj}若错误举出反例\end{CJK}(\begin{CJK}{UTF8}{mj}每小题\end{CJK} 6 \begin{CJK}{UTF8}{mj}分\end{CJK}, \begin{CJK}{UTF8}{mj}共\end{CJK} 36 \begin{CJK}{UTF8}{mj}分\end{CJK})

(1). \begin{CJK}{UTF8}{mj}如果\end{CJK} $\forall p \in \mathbb{N}_{+}$\begin{CJK}{UTF8}{mj}有\end{CJK} $\lim _{n \rightarrow \infty}\left(a_{n+p}-a_{n}\right)=0$, \begin{CJK}{UTF8}{mj}则数列\end{CJK} $\left\{a_{n}\right\}$ \begin{CJK}{UTF8}{mj}收敛\end{CJK}.

(2). \begin{CJK}{UTF8}{mj}如果偏导数\end{CJK} $f_{x}\left(x_{0}, y_{0}\right)$ \begin{CJK}{UTF8}{mj}和\end{CJK} $f_{y}\left(x_{0}, y_{0}\right)$ \begin{CJK}{UTF8}{mj}都存在\end{CJK}, \begin{CJK}{UTF8}{mj}则\end{CJK} $f(x, y)$ \begin{CJK}{UTF8}{mj}在\end{CJK} $P_{0}\left(x_{0}, y_{0}\right)$ \begin{CJK}{UTF8}{mj}连续\end{CJK}.

(3). \begin{CJK}{UTF8}{mj}如果级数\end{CJK} $\sum_{n=1}^{+\infty} u_{n}(x)$ \begin{CJK}{UTF8}{mj}在区间\end{CJK} $[a, b]$ \begin{CJK}{UTF8}{mj}上一致收敛\end{CJK}, \begin{CJK}{UTF8}{mj}则对于任何\end{CJK} $x \in[a, b]$, \begin{CJK}{UTF8}{mj}级数\end{CJK} $\sum_{n=1}^{+\infty} u_{n}(x)$ \begin{CJK}{UTF8}{mj}是绝对\end{CJK} \begin{CJK}{UTF8}{mj}收敛的\end{CJK}.

(4). \begin{CJK}{UTF8}{mj}如果函数\end{CJK} $f(x)$ \begin{CJK}{UTF8}{mj}在\end{CJK} $[a, b]$ \begin{CJK}{UTF8}{mj}上存在原函数\end{CJK}, \begin{CJK}{UTF8}{mj}则\end{CJK} $f(x)$ \begin{CJK}{UTF8}{mj}在\end{CJK} $[a, b]$ \begin{CJK}{UTF8}{mj}上可积\end{CJK}.

(5). \begin{CJK}{UTF8}{mj}如果函数\end{CJK} $f(x)$ \begin{CJK}{UTF8}{mj}在有限开区间\end{CJK} $(x, y)$ \begin{CJK}{UTF8}{mj}上连续\end{CJK}, $f^{2}(x)$ \begin{CJK}{UTF8}{mj}在\end{CJK} $(x, y)$ \begin{CJK}{UTF8}{mj}上一致连续\end{CJK}, \begin{CJK}{UTF8}{mj}则\end{CJK} $f(x)$ \begin{CJK}{UTF8}{mj}在\end{CJK} $(x, y)$ \begin{CJK}{UTF8}{mj}上\end{CJK} \begin{CJK}{UTF8}{mj}致连终\end{CJK}.

(6). \begin{CJK}{UTF8}{mj}如果\end{CJK} $\int_{a}^{+\infty} f(x) \mathrm{d} x$ \begin{CJK}{UTF8}{mj}收敛\end{CJK}, \begin{CJK}{UTF8}{mj}且\end{CJK} $\varphi(x)$ \begin{CJK}{UTF8}{mj}在\end{CJK} $[a,+\infty)$ \begin{CJK}{UTF8}{mj}上有界\end{CJK}, \begin{CJK}{UTF8}{mj}则\end{CJK} $\int_{a}^{+\infty} f(x) \varphi(x) \mathrm{d} x$ \begin{CJK}{UTF8}{mj}收敛\end{CJK}.

\begin{CJK}{UTF8}{mj}二\end{CJK}、\begin{CJK}{UTF8}{mj}求解下列各题\end{CJK} (\begin{CJK}{UTF8}{mj}每小题\end{CJK} 9 \begin{CJK}{UTF8}{mj}分\end{CJK},\begin{CJK}{UTF8}{mj}共\end{CJK} 36 \begin{CJK}{UTF8}{mj}分\end{CJK})

(1). \begin{CJK}{UTF8}{mj}求极限\end{CJK}
$$
\lim _{x \rightarrow 0}\left[\frac{e^{x}+e^{2 x}+\cdots+e^{n x}}{n}\right]^{\frac{1}{x}}
$$
(2). \begin{CJK}{UTF8}{mj}设函数\end{CJK} $f(x)$ \begin{CJK}{UTF8}{mj}在\end{CJK} $(-\infty,+\infty)$ \begin{CJK}{UTF8}{mj}上连续可导\end{CJK}, \begin{CJK}{UTF8}{mj}计算积分\end{CJK}
$$
\iint_{\Sigma} x^{3} \mathrm{~d} y \mathrm{~d} z+\left(\frac{1}{z} f\left(\frac{y}{z}\right)+y^{3}\right) \mathrm{d} z \mathrm{~d} x+\left(\frac{1}{y} f\left(\frac{y}{z}\right)+z^{3}\right) \mathrm{d} x \mathrm{~d} y,
$$
\begin{CJK}{UTF8}{mj}其中\end{CJK} $\Sigma$ \begin{CJK}{UTF8}{mj}为雉面\end{CJK} $y^{2}+z^{2}=x^{2}(x>0)$ \begin{CJK}{UTF8}{mj}与球面\end{CJK} $x^{2}+y^{2}+z^{2}=1$ \begin{CJK}{UTF8}{mj}及\end{CJK} $x^{2}+y^{2}+z^{2}=4$ \begin{CJK}{UTF8}{mj}所围成立体的表面\end{CJK}.

(3). \begin{CJK}{UTF8}{mj}将函数\end{CJK} $f(x)=\arctan \frac{4+x^{2}}{4-x^{2}}$ \begin{CJK}{UTF8}{mj}展成\end{CJK} $x$ \begin{CJK}{UTF8}{mj}的幂级数\end{CJK}, \begin{CJK}{UTF8}{mj}并求级数\end{CJK} $\sum_{n=0}^{\infty} \frac{(-1)^{n}}{(2 n+1) 2^{2 n+1}}$ \begin{CJK}{UTF8}{mj}的和\end{CJK}.

(4). \begin{CJK}{UTF8}{mj}设函数\end{CJK} $f(x, y)$ \begin{CJK}{UTF8}{mj}在\end{CJK} $D=\left\{(x, y) \mid x^{2}+y^{2} \leqslant 1\right\}$ \begin{CJK}{UTF8}{mj}上具有二阶偏导数\end{CJK}, \begin{CJK}{UTF8}{mj}且满足\end{CJK}
$$
\frac{\partial^{2} f}{\partial x^{2}}+\frac{\partial^{2} f}{\partial y^{2}}=e^{-\left(x^{2}+y^{2}\right)},(x, y) \in D
$$
\begin{CJK}{UTF8}{mj}求积分\end{CJK} $\iint_{D}\left(x \frac{\partial f}{\partial x}+y \frac{\partial f}{\partial y}\right) \mathrm{d} x \mathrm{~d} y$.

\begin{CJK}{UTF8}{mj}三\end{CJK}、\begin{CJK}{UTF8}{mj}证明下列各题\end{CJK} (\begin{CJK}{UTF8}{mj}每小题\end{CJK} 13 \begin{CJK}{UTF8}{mj}分\end{CJK}, \begin{CJK}{UTF8}{mj}共\end{CJK} 78 \begin{CJK}{UTF8}{mj}分\end{CJK})

(1). \begin{CJK}{UTF8}{mj}设函数\end{CJK} $f(x)$ \begin{CJK}{UTF8}{mj}在\end{CJK} $[0,1]$ \begin{CJK}{UTF8}{mj}上二阶可导\end{CJK}, \begin{CJK}{UTF8}{mj}且\end{CJK} $f^{\prime \prime}(x)>0, x \in[0,1]$. \begin{CJK}{UTF8}{mj}证明\end{CJK}:
$$
\int_{0}^{1} f\left(x^{n}\right) \mathrm{d} x \geqslant f\left(\frac{1}{n+1}\right), \forall n \in \mathbb{N}_{+}
$$
(2). \begin{CJK}{UTF8}{mj}设有数列\end{CJK} $\left\{a_{n}\right\}$ :
$$
a_{1}=1, a_{2}=4, a_{n+1}=\frac{a_{n-1}+a_{n}}{2}, n=2,3,4 \cdots
$$
\begin{CJK}{UTF8}{mj}证明\end{CJK}: \begin{CJK}{UTF8}{mj}数列\end{CJK} $\left\{a_{n}\right\}$ \begin{CJK}{UTF8}{mj}收敛\end{CJK}.

(3). \begin{CJK}{UTF8}{mj}设函数\end{CJK} $f(x)$ \begin{CJK}{UTF8}{mj}定义在\end{CJK} $[1,+\infty)$ \begin{CJK}{UTF8}{mj}上\end{CJK}, $f(1)=1$, \begin{CJK}{UTF8}{mj}且当\end{CJK} $x \geqslant 1$ \begin{CJK}{UTF8}{mj}时\end{CJK}, \begin{CJK}{UTF8}{mj}有\end{CJK}
$$
f^{\prime}(x)=\frac{1}{x^{2}+f^{2}(x)}
$$
\begin{CJK}{UTF8}{mj}证明\end{CJK}: $\lim _{x \rightarrow+\infty} f(x)$ \begin{CJK}{UTF8}{mj}有限且其极限小于\end{CJK} $1+\frac{\pi}{4}$.

(4). \begin{CJK}{UTF8}{mj}设函数\end{CJK} $f(x)$ \begin{CJK}{UTF8}{mj}在\end{CJK} $[0,+\infty)$ \begin{CJK}{UTF8}{mj}上一致连续\end{CJK}, \begin{CJK}{UTF8}{mj}且\end{CJK} $\forall x \in[0,1]$, \begin{CJK}{UTF8}{mj}数列\end{CJK} $\{f(x+n)\}$ \begin{CJK}{UTF8}{mj}收敛于\end{CJK} 0 . \begin{CJK}{UTF8}{mj}证明\end{CJK}:
$$
\lim _{x \rightarrow+\infty} f(x)=0 .
$$
(5). \begin{CJK}{UTF8}{mj}设有级数\end{CJK} $\sum_{n=1}^{\infty} a_{n}$. \begin{CJK}{UTF8}{mj}如果对任何收敛于\end{CJK} 0 \begin{CJK}{UTF8}{mj}的数列\end{CJK} $\left\{b_{n}\right\}$, \begin{CJK}{UTF8}{mj}级数\end{CJK} $\sum_{n=1}^{\infty} a_{n} b_{n}$ \begin{CJK}{UTF8}{mj}收敛\end{CJK}. \begin{CJK}{UTF8}{mj}证明\end{CJK}: \begin{CJK}{UTF8}{mj}级数\end{CJK} $\sum_{n=1}^{\infty} a_{n}$ \begin{CJK}{UTF8}{mj}绝对收敛\end{CJK}.

(6). \begin{CJK}{UTF8}{mj}设\end{CJK} $\left\{f_{n}(x)\right\}$ \begin{CJK}{UTF8}{mj}为定义在\end{CJK} $[a, b]$ \begin{CJK}{UTF8}{mj}上的函数列\end{CJK}, \begin{CJK}{UTF8}{mj}且存在常数\end{CJK} $L \geqslant 0$ \begin{CJK}{UTF8}{mj}满足\end{CJK}:
$$
\left|f_{n}(x)-f_{n}(y)\right| \leqslant L|x-y|, x, y \in[a, b], n \in \mathbb{N}_{+} .
$$
\begin{CJK}{UTF8}{mj}证明\end{CJK}: (i). \begin{CJK}{UTF8}{mj}若\end{CJK} $\left\{f_{n}(x)\right\}$ \begin{CJK}{UTF8}{mj}在\end{CJK} $[a, b]$ \begin{CJK}{UTF8}{mj}上收敛于\end{CJK} $f(x)$, \begin{CJK}{UTF8}{mj}则\end{CJK} $\left\{f_{n}(x)\right\}$ \begin{CJK}{UTF8}{mj}在\end{CJK} $[a, b]$ \begin{CJK}{UTF8}{mj}上一致收敛于\end{CJK} $f(x)$,

(ii). \begin{CJK}{UTF8}{mj}若对于任何\end{CJK} $x \in[a, b]$, \begin{CJK}{UTF8}{mj}数列\end{CJK} $\left\{f_{n}(x)\right\}$ \begin{CJK}{UTF8}{mj}都有界\end{CJK}, \begin{CJK}{UTF8}{mj}则\end{CJK} $\left\{f_{n}(x)\right\}$ \begin{CJK}{UTF8}{mj}存在\end{CJK} $[a, b]$ \begin{CJK}{UTF8}{mj}上一致收敛的子列\end{CJK}.

\section{$1.202015$ 年}
\begin{CJK}{UTF8}{mj}一\end{CJK}、\begin{CJK}{UTF8}{mj}判断下列命题是否正确\end{CJK}, \begin{CJK}{UTF8}{mj}若正确给出证明\end{CJK}, \begin{CJK}{UTF8}{mj}若错误举出反例\end{CJK}(\begin{CJK}{UTF8}{mj}每小题\end{CJK} 6 \begin{CJK}{UTF8}{mj}分\end{CJK}, \begin{CJK}{UTF8}{mj}共\end{CJK} 36 \begin{CJK}{UTF8}{mj}分\end{CJK})

(1). \begin{CJK}{UTF8}{mj}如果\end{CJK} $\forall \varepsilon>0, \exists N \in \mathbb{N}_{+}$, \begin{CJK}{UTF8}{mj}当\end{CJK} $n>N$ \begin{CJK}{UTF8}{mj}时\end{CJK}, \begin{CJK}{UTF8}{mj}有\end{CJK} $\left|a_{n}-a_{N}\right|<\varepsilon$, \begin{CJK}{UTF8}{mj}则数列\end{CJK} $\left\{a_{n}\right\}$ \begin{CJK}{UTF8}{mj}收敛\end{CJK}.

(2). \begin{CJK}{UTF8}{mj}如果函数列\end{CJK} $\left\{f_{n}(x)\right\}$ \begin{CJK}{UTF8}{mj}在\end{CJK} $[a, b]$ \begin{CJK}{UTF8}{mj}上一致收敛于连续函数\end{CJK} $f(x)$, \begin{CJK}{UTF8}{mj}则\end{CJK} $\forall n \in \mathbb{N}_{+}$, \begin{CJK}{UTF8}{mj}均有\end{CJK} $\left\{f_{n}(x)\right\}$ \begin{CJK}{UTF8}{mj}在\end{CJK} $[a, b]$ \begin{CJK}{UTF8}{mj}上连续\end{CJK}.

(3). \begin{CJK}{UTF8}{mj}如果函数\end{CJK} $f(x)$ \begin{CJK}{UTF8}{mj}在\end{CJK} $x_{0}$ \begin{CJK}{UTF8}{mj}点连续\end{CJK}, \begin{CJK}{UTF8}{mj}且\end{CJK}
$$
\lim _{n \rightarrow \infty} \frac{f\left(x_{0}+\frac{1}{n}\right)-f\left(x_{0}\right)}{\frac{1}{n}}
$$
\begin{CJK}{UTF8}{mj}存在\end{CJK}, \begin{CJK}{UTF8}{mj}则\end{CJK} $f(x)$ \begin{CJK}{UTF8}{mj}在\end{CJK} $x_{0}$ \begin{CJK}{UTF8}{mj}点的右导数存在\end{CJK}.

(4). \begin{CJK}{UTF8}{mj}如果函数\end{CJK} $f(x), g(x)$ \begin{CJK}{UTF8}{mj}在\end{CJK} $[a, b]$ \begin{CJK}{UTF8}{mj}上连续\end{CJK}, \begin{CJK}{UTF8}{mj}则\end{CJK} $\exists \xi \in[a, b]$, \begin{CJK}{UTF8}{mj}使得\end{CJK}
$$
\int_{a}^{b} f(x) g(x) \mathrm{d} x=f(\xi) \int_{a}^{b} g(x) \mathrm{d} x
$$
(5). \begin{CJK}{UTF8}{mj}如果函数\end{CJK} $f(x, y)$ \begin{CJK}{UTF8}{mj}的偏导数在点\end{CJK} $P_{0}\left(x_{0}, y_{0}\right)$ \begin{CJK}{UTF8}{mj}的某邻域内存在且有界\end{CJK},\begin{CJK}{UTF8}{mj}则\end{CJK} $f(x, y)$ \begin{CJK}{UTF8}{mj}在点\end{CJK} $P_{0}\left(x_{0}, y_{0}\right)$ \begin{CJK}{UTF8}{mj}连续\end{CJK}.

(6). \begin{CJK}{UTF8}{mj}如果函数\end{CJK} $f(x)$ \begin{CJK}{UTF8}{mj}在\end{CJK} $[a,+\infty)$ \begin{CJK}{UTF8}{mj}上的非负连续\end{CJK}, \begin{CJK}{UTF8}{mj}且\end{CJK} $\int_{a}^{+\infty} f(x) \mathrm{d} x$ \begin{CJK}{UTF8}{mj}收敛\end{CJK}, \begin{CJK}{UTF8}{mj}则\end{CJK} $\lim _{x \rightarrow+\infty} f(x)=0$.

\section{二、求解下列各题 (每小题 9 分, 共 36 分)}
(1).
$$
\lim _{n \rightarrow \infty} \frac{2^{n+1} n !}{n^{n}}
$$
(2). \begin{CJK}{UTF8}{mj}计算积分\end{CJK}
$$
\iint_{S}\left(x^{2}+y-z^{3}\right) d s
$$
\begin{CJK}{UTF8}{mj}其中\end{CJK} $S$ \begin{CJK}{UTF8}{mj}为\end{CJK} $[-1,1] \times[-1,1] \times[-1,1]$ \begin{CJK}{UTF8}{mj}的表面\end{CJK}.

(3). \begin{CJK}{UTF8}{mj}计算积分\end{CJK}
$$
\int_{-1}^{1}\left|x-x^{2}\right| d x
$$
(4). \begin{CJK}{UTF8}{mj}戊\end{CJK}
$$
\sum_{n=1}^{+\infty} \frac{(-1)^{n-1}}{n(n+2)} x^{n-1}
$$
\begin{CJK}{UTF8}{mj}的和函数\end{CJK}.

\section{三、证明下列各题 (每小题 13 分, 共 78 分)}
(1). \begin{CJK}{UTF8}{mj}设函数\end{CJK} $f(x)$ \begin{CJK}{UTF8}{mj}在\end{CJK} $[a, b]$ \begin{CJK}{UTF8}{mj}内每一点的左右极限都存在\end{CJK}, \begin{CJK}{UTF8}{mj}证明\end{CJK}: $f(x)$ \begin{CJK}{UTF8}{mj}在\end{CJK} $[a, b]$ \begin{CJK}{UTF8}{mj}上有界\end{CJK}.

(2). \begin{CJK}{UTF8}{mj}设函数\end{CJK} $g(x)$ \begin{CJK}{UTF8}{mj}在\end{CJK} $[a,+\infty)$ \begin{CJK}{UTF8}{mj}上一致连续\end{CJK}, $f(x)$ \begin{CJK}{UTF8}{mj}在\end{CJK} $[a, \infty)$ \begin{CJK}{UTF8}{mj}上连续\end{CJK}, \begin{CJK}{UTF8}{mj}且\end{CJK}
$$
\lim _{x \rightarrow+\infty}[f(x)-g(x)]=0
$$
\begin{CJK}{UTF8}{mj}证明\end{CJK}: $f(x)$ \begin{CJK}{UTF8}{mj}在\end{CJK} $[a,+\infty)$ \begin{CJK}{UTF8}{mj}上一致连续\end{CJK}.

(3). \begin{CJK}{UTF8}{mj}设正项级数\end{CJK} $\sum_{n=1}^{+\infty} a_{n}$ \begin{CJK}{UTF8}{mj}收敛\end{CJK}, \begin{CJK}{UTF8}{mj}且\end{CJK}
$$
a_{k} \leqslant a_{n} \quad \forall n<k \leqslant 2 n
$$
\begin{CJK}{UTF8}{mj}证明\end{CJK}: $\lim _{n \rightarrow \infty} n a_{n}=0$. (4). \begin{CJK}{UTF8}{mj}设函数\end{CJK} $f(x)$ \begin{CJK}{UTF8}{mj}在\end{CJK} $(-\infty,+\infty)$ \begin{CJK}{UTF8}{mj}上三阶可导\end{CJK}, \begin{CJK}{UTF8}{mj}且\end{CJK} $f(x), f^{\prime \prime \prime}(x)$ \begin{CJK}{UTF8}{mj}在\end{CJK} $(-\infty,+\infty)$ \begin{CJK}{UTF8}{mj}上都有界\end{CJK}, \begin{CJK}{UTF8}{mj}证明\end{CJK}: $f^{\prime}(x), f^{\prime \prime}(x)$ \begin{CJK}{UTF8}{mj}在\end{CJK} $(-\infty,+\infty)$ \begin{CJK}{UTF8}{mj}上有界\end{CJK}.

(5). \begin{CJK}{UTF8}{mj}设\end{CJK} $\forall b>0$, \begin{CJK}{UTF8}{mj}函数\end{CJK} $f(x)$ \begin{CJK}{UTF8}{mj}在\end{CJK} $[0, b]$ \begin{CJK}{UTF8}{mj}上可积\end{CJK}, \begin{CJK}{UTF8}{mj}且\end{CJK} $\lim _{x \rightarrow+\infty} f(x)=\alpha$, \begin{CJK}{UTF8}{mj}令\end{CJK}
$$
F(t)=t \int_{0}^{+\infty} e^{-t x} f(x) \mathrm{d} x
$$
\begin{CJK}{UTF8}{mj}证明\end{CJK}: $\lim _{t \rightarrow 0^{+}} F(t)=\alpha$.

(6). \begin{CJK}{UTF8}{mj}设函数\end{CJK} $f(x)$ \begin{CJK}{UTF8}{mj}在\end{CJK} $[0,1]$ \begin{CJK}{UTF8}{mj}上连续\end{CJK}, \begin{CJK}{UTF8}{mj}在\end{CJK} $(0,1)$ \begin{CJK}{UTF8}{mj}内二阶可导\end{CJK}, \begin{CJK}{UTF8}{mj}且\end{CJK} $\int_{0}^{1} f(x) \mathrm{d} x=0$,
$$
f(0) f(1)>0, f^{\prime \prime}(x)>0
$$
\begin{CJK}{UTF8}{mj}证明\end{CJK}: (i). $f(x)$ \begin{CJK}{UTF8}{mj}在\end{CJK} $[0,1]$ \begin{CJK}{UTF8}{mj}内有且仅有两个零点\end{CJK},

(ii). \begin{CJK}{UTF8}{mj}存在\end{CJK} $\xi \in(0,1)$, \begin{CJK}{UTF8}{mj}使得\end{CJK} $f^{\prime}(\xi)=\int_{0}^{\xi} f(t) \mathrm{d} t$.

\section{$1.212016$ 年}
\begin{CJK}{UTF8}{mj}一\end{CJK}、\begin{CJK}{UTF8}{mj}判断下列命题是否正确\end{CJK}, \begin{CJK}{UTF8}{mj}若正确给岕证明\end{CJK}, \begin{CJK}{UTF8}{mj}若错误举㚎反例\end{CJK} (\begin{CJK}{UTF8}{mj}每小题\end{CJK} 6 \begin{CJK}{UTF8}{mj}分\end{CJK}, \begin{CJK}{UTF8}{mj}共\end{CJK} 36 \begin{CJK}{UTF8}{mj}分\end{CJK})

\begin{enumerate}
  \item \begin{CJK}{UTF8}{mj}对数列\end{CJK} $\left\{a_{n}\right\}$ \begin{CJK}{UTF8}{mj}的任意两个子列\end{CJK} $\left\{a_{n_{k}}\right\}$ \begin{CJK}{UTF8}{mj}与\end{CJK} $\left\{a_{m_{k}}\right\}$ \begin{CJK}{UTF8}{mj}均有\end{CJK} $\lim _{k \rightarrow \infty}\left(a_{n_{k}}-a_{m_{k}}\right)=0$, \begin{CJK}{UTF8}{mj}则\end{CJK} $\left\{a_{n}\right\}$ \begin{CJK}{UTF8}{mj}收敛\end{CJK}.

  \item \begin{CJK}{UTF8}{mj}如果\end{CJK} $f(x, y)$ \begin{CJK}{UTF8}{mj}在\end{CJK} $\left(x_{0}, y_{0}\right)$ \begin{CJK}{UTF8}{mj}沿任意方向的方向导数都存在\end{CJK}, \begin{CJK}{UTF8}{mj}偏导数\end{CJK} $f_{x}\left(x_{0}, y_{0}\right)$ \begin{CJK}{UTF8}{mj}与\end{CJK} $f_{y}\left(x_{0}, y_{0}\right)$ \begin{CJK}{UTF8}{mj}均存\end{CJK} \begin{CJK}{UTF8}{mj}庄\end{CJK}.

  \item \begin{CJK}{UTF8}{mj}设函数\end{CJK} $f(x)$ \begin{CJK}{UTF8}{mj}在\end{CJK} $[a,+\infty)$ \begin{CJK}{UTF8}{mj}上连续\end{CJK}, \begin{CJK}{UTF8}{mj}当\end{CJK} $x \rightarrow+\infty$ \begin{CJK}{UTF8}{mj}时\end{CJK}, $f(x)$ \begin{CJK}{UTF8}{mj}以\end{CJK} $y=c x+d$ \begin{CJK}{UTF8}{mj}为渐近线\end{CJK}, \begin{CJK}{UTF8}{mj}则\end{CJK} $f(x)$ \begin{CJK}{UTF8}{mj}在\end{CJK} $[a,+\infty)$ \begin{CJK}{UTF8}{mj}上一致连续\end{CJK}.

  \item \begin{CJK}{UTF8}{mj}如果级数\end{CJK} $\sum_{n=1}^{\infty} a_{n}$ \begin{CJK}{UTF8}{mj}收敛\end{CJK}, \begin{CJK}{UTF8}{mj}数列\end{CJK} $\left\{b_{n}\right\}$ \begin{CJK}{UTF8}{mj}满足\end{CJK} $\lim _{n \rightarrow \infty} b_{n}=1$, \begin{CJK}{UTF8}{mj}则\end{CJK} $\sum_{n=1}^{\infty} a_{n} b_{n}$ \begin{CJK}{UTF8}{mj}收敛\end{CJK}.

  \item \begin{CJK}{UTF8}{mj}如果函数\end{CJK} $f(x)$ \begin{CJK}{UTF8}{mj}在区间\end{CJK} $I$ \begin{CJK}{UTF8}{mj}上有原函数\end{CJK}, \begin{CJK}{UTF8}{mj}则\end{CJK} $f(x)$ \begin{CJK}{UTF8}{mj}在\end{CJK} $I$ \begin{CJK}{UTF8}{mj}上无第二类间断点\end{CJK}.

  \item \begin{CJK}{UTF8}{mj}如果函数\end{CJK} $f(x)$ \begin{CJK}{UTF8}{mj}在\end{CJK} $[a,+\infty)$ \begin{CJK}{UTF8}{mj}的任意子区间上可积\end{CJK}, \begin{CJK}{UTF8}{mj}且\end{CJK} $\forall \varepsilon>0, B>0$, \begin{CJK}{UTF8}{mj}都存在\end{CJK} $A(\geqslant a)$, \begin{CJK}{UTF8}{mj}使得\end{CJK} $\left|\int_{A}^{A+B} f(x) \mathrm{d} x\right|<\varepsilon$, \begin{CJK}{UTF8}{mj}则\end{CJK} $\int_{a}^{+\infty} f(x) \mathrm{d} x$ \begin{CJK}{UTF8}{mj}收敛\end{CJK}.

\end{enumerate}
\begin{CJK}{UTF8}{mj}二\end{CJK}、\begin{CJK}{UTF8}{mj}求解下列各题\end{CJK} (\begin{CJK}{UTF8}{mj}每小题\end{CJK} 8 \begin{CJK}{UTF8}{mj}分\end{CJK}, \begin{CJK}{UTF8}{mj}共\end{CJK} 40 \begin{CJK}{UTF8}{mj}分\end{CJK})

\begin{enumerate}
  \setcounter{enumi}{7}
  \item $\lim _{x \rightarrow+\infty}\left(\sin \frac{1}{x}+\cos \frac{1}{x}\right)^{x}$.

  \item $\int_{0}^{2 \pi} \sqrt{1+\cos x} d x$.

\end{enumerate}
$9 .$
$$
\iint_{\Sigma} 4 x z \mathrm{~d} y \mathrm{~d} z-2 y z \mathrm{~d} z \mathrm{~d} x+\left(1-z^{2}\right) \mathrm{d} x \mathrm{~d} y,
$$
\begin{CJK}{UTF8}{mj}其中\end{CJK} $\Sigma$ \begin{CJK}{UTF8}{mj}为\end{CJK} $z=e^{y}, y \in[0, a]$ \begin{CJK}{UTF8}{mj}绕\end{CJK} $z$ \begin{CJK}{UTF8}{mj}轴旋转一周形成的曲面\end{CJK}, \begin{CJK}{UTF8}{mj}方向取下侧\end{CJK}.

\begin{enumerate}
  \setcounter{enumi}{10}
  \item \begin{CJK}{UTF8}{mj}求级数\end{CJK} $\sum_{n=0}^{+\infty}(-1)^{n} \frac{1}{3 n+1}$ \begin{CJK}{UTF8}{mj}的和\end{CJK}.

  \item \begin{CJK}{UTF8}{mj}设\end{CJK} $f(r)$ \begin{CJK}{UTF8}{mj}为\end{CJK} $(0,+\infty)$ \begin{CJK}{UTF8}{mj}上的二阶连续可导函数\end{CJK}, $f(1)=f^{\prime}(1)=1, u(x, y)=f\left(\sqrt{x^{2}+y^{2}}\right)$ \begin{CJK}{UTF8}{mj}满足\end{CJK} Laplace \begin{CJK}{UTF8}{mj}条件\end{CJK}

\end{enumerate}
$$
\frac{\partial^{2} u}{\partial x^{2}}+\frac{\partial^{2} u}{\partial y^{2}}=0 .
$$
\begin{CJK}{UTF8}{mj}试确定\end{CJK} $f(r)$ \begin{CJK}{UTF8}{mj}所满足的微分方程\end{CJK}, \begin{CJK}{UTF8}{mj}并求出\end{CJK} $f(r)$ \begin{CJK}{UTF8}{mj}的解析式\end{CJK}.

\begin{CJK}{UTF8}{mj}三\end{CJK}、\begin{CJK}{UTF8}{mj}证明下列各题\end{CJK} (\begin{CJK}{UTF8}{mj}第\end{CJK} 12-15 \begin{CJK}{UTF8}{mj}题\end{CJK},\begin{CJK}{UTF8}{mj}每小题\end{CJK} 13 \begin{CJK}{UTF8}{mj}分\end{CJK}, \begin{CJK}{UTF8}{mj}第\end{CJK} $16-17$ \begin{CJK}{UTF8}{mj}题\end{CJK},\begin{CJK}{UTF8}{mj}每小题\end{CJK} 11 \begin{CJK}{UTF8}{mj}分\end{CJK}, \begin{CJK}{UTF8}{mj}共\end{CJK} 74 \begin{CJK}{UTF8}{mj}分\end{CJK})

\begin{enumerate}
  \setcounter{enumi}{12}
  \item \begin{CJK}{UTF8}{mj}设函数\end{CJK} $f(x)$ \begin{CJK}{UTF8}{mj}在\end{CJK} $[a, b]$ \begin{CJK}{UTF8}{mj}上连续\end{CJK}, $\forall x \in[a, b], \exists y \in[a, b]$, \begin{CJK}{UTF8}{mj}使得\end{CJK} $|f(y)| \leqslant \frac{1}{2}|f(x)|$. \begin{CJK}{UTF8}{mj}证明\end{CJK}: \begin{CJK}{UTF8}{mj}存在\end{CJK} $\xi \in[a, b]$, \begin{CJK}{UTF8}{mj}使得\end{CJK} $f(\xi)=0$.

  \item \begin{CJK}{UTF8}{mj}设函数\end{CJK} $f(x)$ \begin{CJK}{UTF8}{mj}在\end{CJK} $[a,+\infty)$ \begin{CJK}{UTF8}{mj}上有定义\end{CJK}, \begin{CJK}{UTF8}{mj}当\end{CJK} $\lim _{x \rightarrow+\infty} f(x)$ \begin{CJK}{UTF8}{mj}存在\end{CJK}, \begin{CJK}{UTF8}{mj}又\end{CJK} $f^{\prime}(x)$ \begin{CJK}{UTF8}{mj}在\end{CJK} $(a,+\infty)$ \begin{CJK}{UTF8}{mj}上存在且一致连\end{CJK} \begin{CJK}{UTF8}{mj}续\end{CJK}, \begin{CJK}{UTF8}{mj}试证明\end{CJK}: $\lim _{x \rightarrow+\infty} f^{\prime}(x)=0$.

  \item \begin{CJK}{UTF8}{mj}设函数\end{CJK} $f(x)$ \begin{CJK}{UTF8}{mj}在\end{CJK} $[a, b]$ \begin{CJK}{UTF8}{mj}上二阶可导\end{CJK}, \begin{CJK}{UTF8}{mj}且\end{CJK} $f^{\prime \prime}(x) \geqslant 0, f(a) \leqslant 0, f(b) \leqslant 0$. \begin{CJK}{UTF8}{mj}证明\end{CJK}:

\end{enumerate}
$$
f(x) \geqslant \frac{2}{b-a} \int_{a}^{b} f(t) \mathrm{d} t, \quad \forall x \in[a, b] .
$$

\begin{enumerate}
  \setcounter{enumi}{15}
  \item \begin{CJK}{UTF8}{mj}设函数\end{CJK} $f(x)$ \begin{CJK}{UTF8}{mj}在\end{CJK} $\Omega(t)=\left\{(x, y, z) \mid x^{2}+y^{2}+z^{2} \leqslant t^{2}\right\}$ \begin{CJK}{UTF8}{mj}上连续\end{CJK}, $S=\left\{(x, y, z) \mid x^{2}+y^{2}+z^{2}=t^{2}\right\}$, \begin{CJK}{UTF8}{mj}证\end{CJK} \begin{CJK}{UTF8}{mj}明\end{CJK}:
\end{enumerate}
$$
\frac{\mathrm{d}}{\mathrm{d} t} \iiint_{\Omega(t)} f(x, y, z) \mathrm{d} V=\iint_{S} f(x, y, z) \mathrm{d} S .
$$

\begin{enumerate}
  \setcounter{enumi}{16}
  \item \begin{CJK}{UTF8}{mj}设函数\end{CJK} $f(x)$ \begin{CJK}{UTF8}{mj}在\end{CJK} $[0,1]$ \begin{CJK}{UTF8}{mj}上连续\end{CJK}, \begin{CJK}{UTF8}{mj}试证明\end{CJK}:
\end{enumerate}
$$
\lim _{n \rightarrow \infty} \int_{0}^{1} \frac{n}{1+n^{2} x^{2}} f(x) d x=\frac{\pi}{2} f(0)
$$

\begin{enumerate}
  \setcounter{enumi}{17}
  \item \begin{CJK}{UTF8}{mj}设\end{CJK} $\left\{a_{n}\right\}$ \begin{CJK}{UTF8}{mj}是单调递减的正数列\end{CJK}, \begin{CJK}{UTF8}{mj}证明\end{CJK}: \begin{CJK}{UTF8}{mj}函数项级数\end{CJK} $\sum_{n=0}^{+\infty} a_{n} \sin n x$ \begin{CJK}{UTF8}{mj}在\end{CJK} $[0, \pi]$ \begin{CJK}{UTF8}{mj}上一致收敛的充要\end{CJK} \begin{CJK}{UTF8}{mj}条件为\end{CJK} $\lim _{n \rightarrow \infty} n a_{n}=0$.
\end{enumerate}
\section{$1.222017$ 年}
\begin{CJK}{UTF8}{mj}一\end{CJK}、\begin{CJK}{UTF8}{mj}判断下列命题是否正确\end{CJK}, \begin{CJK}{UTF8}{mj}若正确给出证明\end{CJK},\begin{CJK}{UTF8}{mj}若错误举㚎反例\end{CJK} (\begin{CJK}{UTF8}{mj}每小题\end{CJK} 6 \begin{CJK}{UTF8}{mj}分\end{CJK}, \begin{CJK}{UTF8}{mj}共\end{CJK} 36 \begin{CJK}{UTF8}{mj}分\end{CJK})

\begin{enumerate}
  \item \begin{CJK}{UTF8}{mj}如果函数\end{CJK} $f(x)$ \begin{CJK}{UTF8}{mj}在\end{CJK} $[a, b]$ \begin{CJK}{UTF8}{mj}上可积\end{CJK}, \begin{CJK}{UTF8}{mj}则\end{CJK} $f(x)$ \begin{CJK}{UTF8}{mj}在\end{CJK} $[a, b]$ \begin{CJK}{UTF8}{mj}上至多有有限个不连续点\end{CJK}.

  \item \begin{CJK}{UTF8}{mj}如果数列\end{CJK} $\left\{a_{n}\right\}$ \begin{CJK}{UTF8}{mj}单调\end{CJK}, \begin{CJK}{UTF8}{mj}且\end{CJK} $\lim _{n \rightarrow \infty}\left(a_{n+1}-a_{n}\right)=0$, \begin{CJK}{UTF8}{mj}则\end{CJK} $\left\{a_{n}\right\}$ \begin{CJK}{UTF8}{mj}收敛\end{CJK}.

  \item \begin{CJK}{UTF8}{mj}如果函数\end{CJK} $f(x, y)$ \begin{CJK}{UTF8}{mj}在\end{CJK} $\left(x_{0}, y_{0}\right)$ \begin{CJK}{UTF8}{mj}处连续\end{CJK}, \begin{CJK}{UTF8}{mj}偏导数\end{CJK} $f_{x}\left(x_{0}, y_{0}\right)$ \begin{CJK}{UTF8}{mj}与\end{CJK} $f_{y}\left(x_{0}, y_{0}\right)$ \begin{CJK}{UTF8}{mj}均存在\end{CJK}, \begin{CJK}{UTF8}{mj}则\end{CJK} $f(x, y)$ \begin{CJK}{UTF8}{mj}在\end{CJK} $\left(x_{0}, y_{0}\right)$ \begin{CJK}{UTF8}{mj}处可微\end{CJK}.

  \item \begin{CJK}{UTF8}{mj}设函数\end{CJK} $f(x)$ \begin{CJK}{UTF8}{mj}在\end{CJK} $(0,1)$ \begin{CJK}{UTF8}{mj}上一致连续\end{CJK}, \begin{CJK}{UTF8}{mj}则\end{CJK} $f(x)$ \begin{CJK}{UTF8}{mj}在\end{CJK} $(0,1)$ \begin{CJK}{UTF8}{mj}上有界\end{CJK}.

  \item \begin{CJK}{UTF8}{mj}如果\end{CJK} $u_{n}(x)(n=1,2, \cdots)$ \begin{CJK}{UTF8}{mj}在\end{CJK} $[a, b]$ \begin{CJK}{UTF8}{mj}上连续\end{CJK}, \begin{CJK}{UTF8}{mj}且函数项级数\end{CJK} $\sum_{n=1}^{\infty} u_{n}(x)$ \begin{CJK}{UTF8}{mj}在\end{CJK} $(a, b)$ \begin{CJK}{UTF8}{mj}上致收敛\end{CJK}, \begin{CJK}{UTF8}{mj}则\end{CJK} $\sum_{n=1}^{\infty} u_{n}(x)$ \begin{CJK}{UTF8}{mj}在\end{CJK} $[a, b]$ \begin{CJK}{UTF8}{mj}上一致收敛\end{CJK}.

  \item \begin{CJK}{UTF8}{mj}如果含参变量积分\end{CJK} $\int_{a}^{+\infty} f(x, y) \mathrm{d} x$ \begin{CJK}{UTF8}{mj}在\end{CJK} $[a, b]$ \begin{CJK}{UTF8}{mj}致收敛\end{CJK}, \begin{CJK}{UTF8}{mj}则在\end{CJK} $[a, b]$ \begin{CJK}{UTF8}{mj}上处处绝对收敛\end{CJK}.

\end{enumerate}
\begin{CJK}{UTF8}{mj}二\end{CJK}、\begin{CJK}{UTF8}{mj}求解下列各题\end{CJK}(\begin{CJK}{UTF8}{mj}每小题\end{CJK} 8 \begin{CJK}{UTF8}{mj}分\end{CJK}, \begin{CJK}{UTF8}{mj}共\end{CJK} 40 \begin{CJK}{UTF8}{mj}分\end{CJK})

\begin{enumerate}
  \setcounter{enumi}{7}
  \item \begin{CJK}{UTF8}{mj}求下面积分的值\end{CJK}
\end{enumerate}
$$
\iint_{D} \sqrt{y^{2}-x y} \mathrm{~d} \sigma
$$
\begin{CJK}{UTF8}{mj}其中\end{CJK} $D$ \begin{CJK}{UTF8}{mj}是由\end{CJK} $y=x, y=1, x=0$ \begin{CJK}{UTF8}{mj}围成的封闭图形\end{CJK}.

\begin{enumerate}
  \setcounter{enumi}{8}
  \item \begin{CJK}{UTF8}{mj}求极限\end{CJK}
\end{enumerate}
$$
\lim _{x \rightarrow 0}\left[\frac{e^{x}+e^{2 x}+\cdots+e^{n x}}{n}\right]^{\frac{1}{x}}
$$

\begin{enumerate}
  \setcounter{enumi}{9}
  \item \begin{CJK}{UTF8}{mj}设函数\end{CJK} $y=y(x), z=z(x)$ \begin{CJK}{UTF8}{mj}由方程\end{CJK} $z=x f(x+y)$ \begin{CJK}{UTF8}{mj}和\end{CJK} $F(x, y, z)=0$ \begin{CJK}{UTF8}{mj}所确定\end{CJK}, $f$ \begin{CJK}{UTF8}{mj}一阶连续可导\end{CJK}, $F$ \begin{CJK}{UTF8}{mj}具有一阶连续偏导数\end{CJK}, \begin{CJK}{UTF8}{mj}求\end{CJK} $\frac{\mathrm{d} z}{\mathrm{~d} x}$.
\end{enumerate}
$10 .$
$$
\iint_{\Sigma}\left(x^{2}-y z\right) \mathrm{d} y \mathrm{~d} z+\left(y^{2}-z x\right) \mathrm{d} z \mathrm{~d} x+\left(z^{2}-x y\right) \mathrm{d} x \mathrm{~d} y
$$
\begin{CJK}{UTF8}{mj}其中\end{CJK} $\Sigma$ \begin{CJK}{UTF8}{mj}为曲面\end{CJK} $z=1-\sqrt{x^{2}+y^{2}}$ \begin{CJK}{UTF8}{mj}被平面\end{CJK} $z=0, z=1$ \begin{CJK}{UTF8}{mj}所截的部分\end{CJK}, \begin{CJK}{UTF8}{mj}方向取下侧\end{CJK}.

\begin{enumerate}
  \setcounter{enumi}{11}
  \item \begin{CJK}{UTF8}{mj}求级数\end{CJK} $\sum_{n=1}^{+\infty} \frac{1}{3^{n}+(-2)^{n}} \cdot \frac{x^{n}}{n}$ \begin{CJK}{UTF8}{mj}的收敛域\end{CJK}.
\end{enumerate}
\begin{CJK}{UTF8}{mj}三\end{CJK}、\begin{CJK}{UTF8}{mj}证明下列各题\end{CJK} (\begin{CJK}{UTF8}{mj}第\end{CJK} 12-15 \begin{CJK}{UTF8}{mj}题\end{CJK},\begin{CJK}{UTF8}{mj}每小题\end{CJK} 13 \begin{CJK}{UTF8}{mj}分\end{CJK},\begin{CJK}{UTF8}{mj}第\end{CJK} 16-17 \begin{CJK}{UTF8}{mj}题\end{CJK},\begin{CJK}{UTF8}{mj}每小题\end{CJK} 11 \begin{CJK}{UTF8}{mj}分\end{CJK},\begin{CJK}{UTF8}{mj}共\end{CJK} 74 \begin{CJK}{UTF8}{mj}分\end{CJK})

\begin{enumerate}
  \setcounter{enumi}{12}
  \item \begin{CJK}{UTF8}{mj}设函数\end{CJK} $f(x)$ \begin{CJK}{UTF8}{mj}在\end{CJK} $[a,+\infty)$ \begin{CJK}{UTF8}{mj}上一致连续\end{CJK}, $\int_{a}^{+\infty} f(x) \mathrm{d} x$ \begin{CJK}{UTF8}{mj}收敛\end{CJK}, \begin{CJK}{UTF8}{mj}证明\end{CJK}: $\lim _{x \rightarrow+\infty} f(x)=0$.

  \item \begin{CJK}{UTF8}{mj}设函数\end{CJK} $f(x)$ \begin{CJK}{UTF8}{mj}在\end{CJK} $[0,+\infty)$ \begin{CJK}{UTF8}{mj}上一致连续\end{CJK}, \begin{CJK}{UTF8}{mj}对任意的\end{CJK} $x>0$, \begin{CJK}{UTF8}{mj}均有\end{CJK} $\lim _{n \rightarrow \infty} f(x+n)=0$, \begin{CJK}{UTF8}{mj}证明\end{CJK}: $\{f(x+n)\}$ \begin{CJK}{UTF8}{mj}在\end{CJK} $[0,1]$ \begin{CJK}{UTF8}{mj}上一致收敛于\end{CJK} 0 .

  \item \begin{CJK}{UTF8}{mj}设函数\end{CJK} $f(x)$ \begin{CJK}{UTF8}{mj}在\end{CJK} $[a, b]$ \begin{CJK}{UTF8}{mj}上二阶连续可导\end{CJK}, \begin{CJK}{UTF8}{mj}且\end{CJK} $f\left(\frac{a+b}{2}\right)=0, M=\max _{x \in[a, b]}\left\{\left|f^{\prime \prime}(x)\right|\right\}$. \begin{CJK}{UTF8}{mj}证明\end{CJK}:

\end{enumerate}
$$
\left|\int_{a}^{b} f(x) \mathrm{d} x\right| \leqslant \frac{M}{24}(b-a)^{3} \text {. }
$$

\begin{enumerate}
  \setcounter{enumi}{15}
  \item (1). \begin{CJK}{UTF8}{mj}证明\end{CJK}: \begin{CJK}{UTF8}{mj}级数\end{CJK} $\sum_{n=1}^{\infty} n e^{-n x}$ \begin{CJK}{UTF8}{mj}在\end{CJK} $(0,+\infty)$ \begin{CJK}{UTF8}{mj}上处处收敛\end{CJK}, \begin{CJK}{UTF8}{mj}但非一致收敛\end{CJK};
\end{enumerate}
(2). \begin{CJK}{UTF8}{mj}证明\end{CJK}: \begin{CJK}{UTF8}{mj}其和函数为\end{CJK} $S(x)=\frac{e^{x}}{\left(e^{x}-1\right)^{2}}, x>0$. 16. \begin{CJK}{UTF8}{mj}平面区域\end{CJK} $D$ \begin{CJK}{UTF8}{mj}由光滑封闭曲线\end{CJK} $L$ \begin{CJK}{UTF8}{mj}围成\end{CJK}, $\vec{n}$ \begin{CJK}{UTF8}{mj}为\end{CJK} $L$ \begin{CJK}{UTF8}{mj}上任意一点处的外法向向量\end{CJK}, $u(x, y) \in C^{2}(D)$. \begin{CJK}{UTF8}{mj}证明\end{CJK}:
$$
\iint_{D}\left(u_{x}^{2}+u_{y}^{2}\right) \mathrm{d} \sigma=\oint_{L} u \frac{\partial u}{\partial \vec{n}} \mathrm{~d} s-\iint_{D} u\left(u_{x x}+u_{y y}\right) \mathrm{d} \sigma
$$
\includegraphics[max width=\textwidth]{2022_04_18_3416d289b173eb9de8c1g-266}

\begin{enumerate}
  \setcounter{enumi}{17}
  \item \begin{CJK}{UTF8}{mj}设函数\end{CJK} $f(x)$ \begin{CJK}{UTF8}{mj}在\end{CJK} $(0,+\infty)$ \begin{CJK}{UTF8}{mj}上可导\end{CJK}, $\lim _{x \rightarrow+\infty}[f(x+1)-f(x)]=0$. \begin{CJK}{UTF8}{mj}证明\end{CJK}:
\end{enumerate}
(1). $\lim _{x \rightarrow+\infty} \frac{f(x)}{x}=0$;

(2). \begin{CJK}{UTF8}{mj}在\end{CJK} $(0,+\infty)$ \begin{CJK}{UTF8}{mj}上存在单调递增趋于\end{CJK} $+\infty$ \begin{CJK}{UTF8}{mj}的数列\end{CJK} $\left\{x_{n}\right\}$, \begin{CJK}{UTF8}{mj}使得\end{CJK} $\lim _{n \rightarrow+\infty} f^{\prime}\left(x_{n}\right)=0$.

\section{$1.232018$ 年}
\section{一、判断下列命题是否正确,若正确给出证明,若错误举出反例 (每小题 5 分,共 30 分)}
\begin{enumerate}
  \item \begin{CJK}{UTF8}{mj}若对任意的\end{CJK} $N$, \begin{CJK}{UTF8}{mj}总存在\end{CJK} $\varepsilon>0$, \begin{CJK}{UTF8}{mj}当\end{CJK} $n>N$ \begin{CJK}{UTF8}{mj}时\end{CJK}, \begin{CJK}{UTF8}{mj}有\end{CJK} $\left|a_{n}-a\right|<\varepsilon$ \begin{CJK}{UTF8}{mj}成立\end{CJK}, \begin{CJK}{UTF8}{mj}则\end{CJK} $\lim _{n \rightarrow \infty} a_{n}=a$.

  \item \begin{CJK}{UTF8}{mj}若函数\end{CJK} $f(x)$ \begin{CJK}{UTF8}{mj}在区间\end{CJK} $I$ \begin{CJK}{UTF8}{mj}上单调且存在原函数\end{CJK}, \begin{CJK}{UTF8}{mj}则\end{CJK} $f(x)$ \begin{CJK}{UTF8}{mj}在\end{CJK} $I$ \begin{CJK}{UTF8}{mj}上连续\end{CJK}.

  \item \begin{CJK}{UTF8}{mj}若函数\end{CJK} $f(x, y)$ \begin{CJK}{UTF8}{mj}在\end{CJK} $\left(x_{0}, y_{0}\right)$ \begin{CJK}{UTF8}{mj}处沿任意方向的方向导数均存在\end{CJK}, \begin{CJK}{UTF8}{mj}则\end{CJK} $f(x, y)$ \begin{CJK}{UTF8}{mj}在该点处连续\end{CJK}.

  \item \begin{CJK}{UTF8}{mj}若函数\end{CJK} $f(x)$ \begin{CJK}{UTF8}{mj}和\end{CJK} $g(x)$ \begin{CJK}{UTF8}{mj}在\end{CJK} $(0,1)$ \begin{CJK}{UTF8}{mj}内均一致连续\end{CJK}, \begin{CJK}{UTF8}{mj}则\end{CJK} $f(x) g(x)$ \begin{CJK}{UTF8}{mj}在\end{CJK} $(0,1)$ \begin{CJK}{UTF8}{mj}内也致连续\end{CJK}.

  \item \begin{CJK}{UTF8}{mj}若\end{CJK} $a_{n} \leqslant b_{n} \leqslant c_{n}(n=1,2, \ldots)$ \begin{CJK}{UTF8}{mj}且级数\end{CJK} $\sum_{n=1}^{+\infty} a_{n}$ \begin{CJK}{UTF8}{mj}与\end{CJK} $\sum_{n=1}^{+\infty} c_{n}$ \begin{CJK}{UTF8}{mj}都收敛\end{CJK}, \begin{CJK}{UTF8}{mj}则级数\end{CJK} $\sum_{n=1}^{+\infty} b_{n}$ \begin{CJK}{UTF8}{mj}也收敛\end{CJK}.

  \item \begin{CJK}{UTF8}{mj}若广义积分\end{CJK} $\int_{a}^{+\infty} f(x) \mathrm{d} x$ \begin{CJK}{UTF8}{mj}与\end{CJK} $\int_{a}^{+\infty} g(x) \mathrm{d} x$ \begin{CJK}{UTF8}{mj}皆绝对收敛\end{CJK}, \begin{CJK}{UTF8}{mj}则\end{CJK} $\int_{a}^{+\infty} f(x) g(x) \mathrm{d} x$ \begin{CJK}{UTF8}{mj}收敛\end{CJK}.

\end{enumerate}
\section{二、求解下列各题 (每小题 9 分, 共 45 分)}
\begin{enumerate}
  \setcounter{enumi}{7}
  \item \begin{CJK}{UTF8}{mj}计算极限\end{CJK}
\end{enumerate}
$$
\lim _{n \rightarrow \infty} \frac{1}{n}\left(\sqrt{1+\cos \frac{\pi}{n}}+\sqrt{1+\cos \frac{2 \pi}{n}}+\cdots+\sqrt{1+\cos \frac{n \pi}{n}}\right) \text {. }
$$

\begin{enumerate}
  \setcounter{enumi}{8}
  \item \begin{CJK}{UTF8}{mj}设函数\end{CJK} $z=f\left(x y, \frac{x}{y}\right)+g\left(\frac{y}{x}\right)$, \begin{CJK}{UTF8}{mj}其中\end{CJK} $f$ \begin{CJK}{UTF8}{mj}具有二阶连续偏导数\end{CJK}, $g$ \begin{CJK}{UTF8}{mj}一阶连续可导\end{CJK}, \begin{CJK}{UTF8}{mj}求\end{CJK} $\frac{\partial^{2} z}{\partial x \partial y}$.

  \item \begin{CJK}{UTF8}{mj}设函数\end{CJK}

\end{enumerate}
$$
f(x)=\lim _{a \rightarrow x}\left(\frac{\sin a}{\sin x}\right)^{\frac{x}{\sin a-\sin x}}
$$
\begin{CJK}{UTF8}{mj}求\end{CJK} $f(x)$ \begin{CJK}{UTF8}{mj}的所有间断点并判断其类型\end{CJK}.

\begin{enumerate}
  \setcounter{enumi}{10}
  \item \begin{CJK}{UTF8}{mj}求级数\end{CJK} $\sum_{n=1}^{+\infty} \sum_{k=1}^{n} \frac{k}{2^{n-1}}$ \begin{CJK}{UTF8}{mj}的和\end{CJK}.

  \item \begin{CJK}{UTF8}{mj}计算曲面积分\end{CJK}

\end{enumerate}
$$
\iint_{\Sigma} \frac{a x \mathrm{~d} y \mathrm{~d} z+(z+a)^{2} \mathrm{~d} x \mathrm{~d} y}{\sqrt{x^{2}+y^{2}+z^{2}}}
$$
\begin{CJK}{UTF8}{mj}其中\end{CJK} $\sum$ \begin{CJK}{UTF8}{mj}为下半球面\end{CJK} $z=-\sqrt{a^{2}-x^{2}-y^{2}}$ \begin{CJK}{UTF8}{mj}的上侧\end{CJK}, $a>0$ \begin{CJK}{UTF8}{mj}为常数\end{CJK}.

\section{三、证明下列各题(第 12-16 题, 每小题 13 分, 第 17 题 10 分, 共 75 分)}
\begin{enumerate}
  \setcounter{enumi}{12}
  \item \begin{CJK}{UTF8}{mj}已知函数列\end{CJK} $\left\{f_{n}(x)\right\}$ \begin{CJK}{UTF8}{mj}在\end{CJK} $[a, b]$ \begin{CJK}{UTF8}{mj}上收敛于连续函数\end{CJK} $f(x)$. \begin{CJK}{UTF8}{mj}证明\end{CJK}: $\left\{f_{n}(x)\right\}$ \begin{CJK}{UTF8}{mj}在\end{CJK} $[a, b]$ \begin{CJK}{UTF8}{mj}上一致收敛\end{CJK} \begin{CJK}{UTF8}{mj}于\end{CJK} $f(x)$ \begin{CJK}{UTF8}{mj}的充要条件为\end{CJK}: \begin{CJK}{UTF8}{mj}对任意数列\end{CJK} $\left\{x_{n}\right\} \subset[a, b]$, \begin{CJK}{UTF8}{mj}且\end{CJK} $\lim _{n \rightarrow+\infty} x_{n}=x_{0}$, \begin{CJK}{UTF8}{mj}有\end{CJK}
\end{enumerate}
$$
\lim _{n \rightarrow+\infty} f_{n}\left(x_{n}\right)=f\left(x_{0}\right)
$$

\begin{enumerate}
  \setcounter{enumi}{13}
  \item \begin{CJK}{UTF8}{mj}设函数\end{CJK} $f(x)$ \begin{CJK}{UTF8}{mj}在\end{CJK} $[0,1]$ \begin{CJK}{UTF8}{mj}上可导\end{CJK}, \begin{CJK}{UTF8}{mj}且\end{CJK} $f(0)=0,0 \leqslant f^{\prime}(x) \leqslant 1$, \begin{CJK}{UTF8}{mj}证明不等式\end{CJK}
\end{enumerate}
$$
\left(\int_{0}^{1} f(x) \mathrm{d} x\right)^{2} \geq \int_{0}^{1} f^{3}(x) \mathrm{d} x
$$

\begin{enumerate}
  \setcounter{enumi}{14}
  \item \begin{CJK}{UTF8}{mj}已知函数\end{CJK} $f(x)$ \begin{CJK}{UTF8}{mj}在\end{CJK} $[0,+\infty)$ \begin{CJK}{UTF8}{mj}上连续\end{CJK}, \begin{CJK}{UTF8}{mj}且\end{CJK} $\lim _{x \rightarrow+\infty} f(x)=A \in \mathbb{R}$, \begin{CJK}{UTF8}{mj}证\end{CJK}:\\
(1). $f(x)$ \begin{CJK}{UTF8}{mj}在\end{CJK} $[0,+\infty)$ \begin{CJK}{UTF8}{mj}上有界且一致连续\end{CJK};\\
(2). $\lim _{x \rightarrow+\infty} \frac{1}{x} \int_{0}^{x} f(t) \mathrm{d} t=A$.

  \item \begin{CJK}{UTF8}{mj}设函数\end{CJK}

\end{enumerate}
$$
F(x)=\int_{0}^{+\infty} \frac{\left(1-e^{-x t}\right) \cos t}{t} \mathrm{~d} t
$$
\begin{CJK}{UTF8}{mj}试证\end{CJK}: $F(x)$ \begin{CJK}{UTF8}{mj}在\end{CJK} $[0,+\infty)$ \begin{CJK}{UTF8}{mj}上连续\end{CJK}, \begin{CJK}{UTF8}{mj}在\end{CJK} $(0,+\infty)$ \begin{CJK}{UTF8}{mj}内可导\end{CJK}. 16. \begin{CJK}{UTF8}{mj}设级数\end{CJK} $\sum_{n=1}^{+\infty} u_{n}(x)$ \begin{CJK}{UTF8}{mj}在\end{CJK} $[a, b]$ \begin{CJK}{UTF8}{mj}上处处收敛\end{CJK},\begin{CJK}{UTF8}{mj}每一个\end{CJK} $u_{n}(x)$ \begin{CJK}{UTF8}{mj}均在\end{CJK} $[a, b]$ \begin{CJK}{UTF8}{mj}上连续可导\end{CJK}, \begin{CJK}{UTF8}{mj}且存在常数\end{CJK} $M>0$, \begin{CJK}{UTF8}{mj}使得\end{CJK} $\left|\sum_{k=1}^{n} u_{k}^{\prime}(x)\right| \leqslant M$ \begin{CJK}{UTF8}{mj}对任意\end{CJK} $x \in[a, b]$ \begin{CJK}{UTF8}{mj}及\end{CJK} $n \in \mathbb{N}_{+}$\begin{CJK}{UTF8}{mj}成立\end{CJK}, \begin{CJK}{UTF8}{mj}证明\end{CJK}: $\sum_{n=1}^{+\infty} u_{n}(x)$ \begin{CJK}{UTF8}{mj}在\end{CJK} $[a, b]$ \begin{CJK}{UTF8}{mj}上一致收\end{CJK} \begin{CJK}{UTF8}{mj}剑\end{CJK}.

\begin{enumerate}
  \setcounter{enumi}{17}
  \item \begin{CJK}{UTF8}{mj}设函数\end{CJK} $f(x)$ \begin{CJK}{UTF8}{mj}在\end{CJK} $[0,1]$ \begin{CJK}{UTF8}{mj}上可导\end{CJK}, \begin{CJK}{UTF8}{mj}且\end{CJK} $f(0)=0, f(1)=1$, \begin{CJK}{UTF8}{mj}又\end{CJK} $p_{1}, p_{2}, \ldots, p_{n}$ \begin{CJK}{UTF8}{mj}为\end{CJK} $n$ \begin{CJK}{UTF8}{mj}个正数\end{CJK}, \begin{CJK}{UTF8}{mj}证明\end{CJK}: \begin{CJK}{UTF8}{mj}在\end{CJK} $(0,1)$ \begin{CJK}{UTF8}{mj}内存在一组互异的数\end{CJK} $x_{1}, x_{2}, \ldots, x_{n}$, \begin{CJK}{UTF8}{mj}使得\end{CJK}
\end{enumerate}
$$
\sum_{i=1}^{n} \frac{p_{i}}{f^{\prime}\left(x_{i}\right)}=\sum_{i=1}^{n} p_{i}
$$

\section{$1.242019$ 年}
\begin{CJK}{UTF8}{mj}一\end{CJK}、\begin{CJK}{UTF8}{mj}判断下列命题是否正确\end{CJK}, \begin{CJK}{UTF8}{mj}若正确给出证明\end{CJK}, \begin{CJK}{UTF8}{mj}若错误举出反例\end{CJK} (\begin{CJK}{UTF8}{mj}每小题\end{CJK} 5 \begin{CJK}{UTF8}{mj}分\end{CJK},\begin{CJK}{UTF8}{mj}共\end{CJK} 30 \begin{CJK}{UTF8}{mj}分\end{CJK})

\begin{enumerate}
  \item \begin{CJK}{UTF8}{mj}如果函数\end{CJK} $f(x)$ \begin{CJK}{UTF8}{mj}在有限区间\end{CJK} $(a, b)$ \begin{CJK}{UTF8}{mj}上连续且有界\end{CJK}, \begin{CJK}{UTF8}{mj}则\end{CJK} $f(x)$ \begin{CJK}{UTF8}{mj}在\end{CJK} $(a, b)$ \begin{CJK}{UTF8}{mj}上一致连续\end{CJK}.

  \item \begin{CJK}{UTF8}{mj}设\end{CJK} $f(x)$ \begin{CJK}{UTF8}{mj}在\end{CJK} $[a,+\infty)$ \begin{CJK}{UTF8}{mj}上可导\end{CJK}, $f(a)=f(+\infty)$, \begin{CJK}{UTF8}{mj}则存在\end{CJK} $\xi \in(a,+\infty)$, \begin{CJK}{UTF8}{mj}使得\end{CJK} $f^{\prime}(\xi)=0$.

  \item \begin{CJK}{UTF8}{mj}设\end{CJK} $f(x)$ \begin{CJK}{UTF8}{mj}为闭区间\end{CJK} $[a, b]$ \begin{CJK}{UTF8}{mj}上不恒为\end{CJK} 0 \begin{CJK}{UTF8}{mj}的连续函数\end{CJK}, $D(x)$ \begin{CJK}{UTF8}{mj}为\end{CJK} Dirichlet \begin{CJK}{UTF8}{mj}函数\end{CJK}, \begin{CJK}{UTF8}{mj}则\end{CJK} $f(x) D(x)$ \begin{CJK}{UTF8}{mj}在\end{CJK} $[a, b]$ \begin{CJK}{UTF8}{mj}上不可积\end{CJK}.

  \item \begin{CJK}{UTF8}{mj}若\end{CJK} $\int_{a}^{+\infty} f(x) \mathrm{d} x$ \begin{CJK}{UTF8}{mj}与\end{CJK} $\int_{a}^{+\infty} f^{\prime}(x) \mathrm{d} x$ \begin{CJK}{UTF8}{mj}均收敛\end{CJK}, \begin{CJK}{UTF8}{mj}则\end{CJK} $\lim _{x \rightarrow+\infty} f(x)=0$.

  \item \begin{CJK}{UTF8}{mj}设函数\end{CJK} $f(x)$ \begin{CJK}{UTF8}{mj}在点\end{CJK} $x_{0}$ \begin{CJK}{UTF8}{mj}的某空心邻域内有定义\end{CJK}. \begin{CJK}{UTF8}{mj}如果对于任何严格单调收敛于\end{CJK} $x_{0}$ \begin{CJK}{UTF8}{mj}的数列\end{CJK} $\left\{x_{n}\right\}$ \begin{CJK}{UTF8}{mj}都有\end{CJK} $\lim _{n \rightarrow+\infty} f\left(x_{n}\right)$ \begin{CJK}{UTF8}{mj}存在且相等\end{CJK}, \begin{CJK}{UTF8}{mj}则\end{CJK} $\lim _{x \rightarrow x_{0}} f(x)$ \begin{CJK}{UTF8}{mj}存在\end{CJK}.

  \item \begin{CJK}{UTF8}{mj}若函数列\end{CJK} $\left\{f_{n}(x)\right\},\left\{g_{n}(x)\right\}$ \begin{CJK}{UTF8}{mj}在\end{CJK} $I$ \begin{CJK}{UTF8}{mj}上一致收敛\end{CJK}, \begin{CJK}{UTF8}{mj}则\end{CJK} $\left\{f_{n}(x) g_{n}(x)\right\}$ \begin{CJK}{UTF8}{mj}在\end{CJK} $I$ \begin{CJK}{UTF8}{mj}上必一致收敛\end{CJK}.

\end{enumerate}
\begin{CJK}{UTF8}{mj}二\end{CJK}、\begin{CJK}{UTF8}{mj}求解下列各题\end{CJK} (\begin{CJK}{UTF8}{mj}每小题\end{CJK} 9 \begin{CJK}{UTF8}{mj}分\end{CJK}, \begin{CJK}{UTF8}{mj}共\end{CJK} 45 \begin{CJK}{UTF8}{mj}分\end{CJK})

\begin{enumerate}
  \setcounter{enumi}{7}
  \item \begin{CJK}{UTF8}{mj}设\end{CJK} $f(x)=\left\{\begin{array}{ll}\frac{g(x)}{x}, & x \neq 0 \\ 0, & x=0\end{array}\right.$ \begin{CJK}{UTF8}{mj}且\end{CJK} $g(0)=g^{\prime}(0)=0, g^{\prime \prime}(0)=2$, \begin{CJK}{UTF8}{mj}求\end{CJK} $f^{\prime}(0)$.

  \item \begin{CJK}{UTF8}{mj}求积分\end{CJK} $\int_{0}^{+\infty} e^{-x^{2}} \mathrm{~d} x$ \begin{CJK}{UTF8}{mj}的值\end{CJK}.

  \item \begin{CJK}{UTF8}{mj}将函数\end{CJK} $f(x)=\arctan \frac{1-2 x}{1+2 x}$ \begin{CJK}{UTF8}{mj}展成\end{CJK} $x$ \begin{CJK}{UTF8}{mj}的幂级数\end{CJK}, \begin{CJK}{UTF8}{mj}并求级数\end{CJK} $\sum_{n=0}^{\infty} \frac{(-1)^{n}}{(2 n+1)}$ \begin{CJK}{UTF8}{mj}的和\end{CJK}.

  \item \begin{CJK}{UTF8}{mj}设\end{CJK} $u=x+y+z, S=\left\{(x, y, z) \mid x^{2}+y^{2}+z^{2}=1\right\}, \overrightarrow{\mathbf{n}}$ \begin{CJK}{UTF8}{mj}为球面\end{CJK} $S$ \begin{CJK}{UTF8}{mj}的外侧法向量\end{CJK}. \begin{CJK}{UTF8}{mj}求\end{CJK} $\frac{\partial u}{\partial \overrightarrow{\mathbf{n}}}$ \begin{CJK}{UTF8}{mj}的表达式\end{CJK} \begin{CJK}{UTF8}{mj}及\end{CJK} $\frac{\partial u}{\partial \overrightarrow{\mathbf{n}}}$ \begin{CJK}{UTF8}{mj}在\end{CJK} $S$ \begin{CJK}{UTF8}{mj}上的最值\end{CJK}.

  \item \begin{CJK}{UTF8}{mj}设\end{CJK} $P(x, y, z)=Q(x, y, z)=R(x, y, z)=f\left(\left(x^{2}+y^{2}\right) z\right), f$ \begin{CJK}{UTF8}{mj}有连续导数\end{CJK}, \begin{CJK}{UTF8}{mj}求极限\end{CJK}

\end{enumerate}
$$
\lim _{t \rightarrow 0^{+}} \frac{\iint_{\Omega} P(x, y, z) \mathrm{d} y \mathrm{~d} z+Q(x, y, z) \mathrm{d} z \mathrm{~d} x+R(x, y, z) \mathrm{d} x \mathrm{~d} y}{t^{4}},
$$
\begin{CJK}{UTF8}{mj}其中\end{CJK} $\Omega$ \begin{CJK}{UTF8}{mj}为圆柱\end{CJK} $\left\{(x, y, z) \mid x^{2}+y^{2} \leqslant t^{2}, z \in[0,1]\right\}$ \begin{CJK}{UTF8}{mj}的外表面\end{CJK}, \begin{CJK}{UTF8}{mj}方向取外侧\end{CJK}.

\begin{CJK}{UTF8}{mj}三\end{CJK}、\begin{CJK}{UTF8}{mj}证明下列各题\end{CJK}(\begin{CJK}{UTF8}{mj}第\end{CJK} 12-16 \begin{CJK}{UTF8}{mj}题\end{CJK}, \begin{CJK}{UTF8}{mj}每小题\end{CJK} 13 \begin{CJK}{UTF8}{mj}分\end{CJK},\begin{CJK}{UTF8}{mj}第\end{CJK} 17 \begin{CJK}{UTF8}{mj}题\end{CJK} 10 \begin{CJK}{UTF8}{mj}分\end{CJK}, \begin{CJK}{UTF8}{mj}共\end{CJK} 75 \begin{CJK}{UTF8}{mj}分\end{CJK})

\begin{enumerate}
  \setcounter{enumi}{12}
  \item \begin{CJK}{UTF8}{mj}若函数\end{CJK} $f(x)$ \begin{CJK}{UTF8}{mj}在有限区间\end{CJK} $I$ \begin{CJK}{UTF8}{mj}上有定义\end{CJK}. \begin{CJK}{UTF8}{mj}证明\end{CJK}: $f(x)$ \begin{CJK}{UTF8}{mj}在\end{CJK} $I$ \begin{CJK}{UTF8}{mj}上一致连续的充要条件是\end{CJK} $f$ \begin{CJK}{UTF8}{mj}将\end{CJK} Cauchy \begin{CJK}{UTF8}{mj}列映为\end{CJK} Cauchy \begin{CJK}{UTF8}{mj}列\end{CJK}.

  \item \begin{CJK}{UTF8}{mj}设\end{CJK} $f(x, y)=\left\{\begin{array}{ll}y \cdot \arctan \frac{1}{\sqrt{x^{2}+y^{2}}} & ,(x, y) \neq(0,0) \\ 0 & ,(x, y)=(0,0)\end{array}\right.$. \begin{CJK}{UTF8}{mj}证明\end{CJK}: $f(x, y)$ \begin{CJK}{UTF8}{mj}在\end{CJK} $(0,0)$ \begin{CJK}{UTF8}{mj}点处连续\end{CJK}, \begin{CJK}{UTF8}{mj}并讨论\end{CJK} \begin{CJK}{UTF8}{mj}其偏导数的存在性及可微性\end{CJK}.

  \item \begin{CJK}{UTF8}{mj}讨论广义积分\end{CJK} $\int_{0}^{+\infty} e^{-x} x^{\alpha-1}|\ln x|^{n} \mathrm{~d} x\left(\alpha>0, n \in \mathbb{Z}_{+}\right)$\begin{CJK}{UTF8}{mj}的敛散性\end{CJK}.

  \item \begin{CJK}{UTF8}{mj}若函数\end{CJK} $f(x)$ \begin{CJK}{UTF8}{mj}在\end{CJK} $[a, b]$ \begin{CJK}{UTF8}{mj}上可微\end{CJK}, $f(a)=0$, \begin{CJK}{UTF8}{mj}存在实数\end{CJK} $A>0$, \begin{CJK}{UTF8}{mj}使得\end{CJK} $\forall x \in[a, b],\left|f^{\prime}(x)\right| \leqslant A|f(x)|$. \begin{CJK}{UTF8}{mj}证明\end{CJK}: $f(x) \equiv 0(x \in[a, b])$.

  \item \begin{CJK}{UTF8}{mj}已知数列\end{CJK} $a_{n} \geqslant 0, S_{n}=\sum_{k=1}^{n} a_{k}, n=1,2, \ldots$

\end{enumerate}
(1) \begin{CJK}{UTF8}{mj}当级数\end{CJK} $\sum_{n=1}^{\infty} a_{n}$ \begin{CJK}{UTF8}{mj}收敛时\end{CJK}. \begin{CJK}{UTF8}{mj}证明\end{CJK}: \begin{CJK}{UTF8}{mj}函数项级数\end{CJK} $\sum_{n=1}^{\infty} a_{n} e^{-S_{n} x}$ \begin{CJK}{UTF8}{mj}的收敛域为\end{CJK} $(-\infty,+\infty), \forall a \in \mathbb{R}$, $\sum_{n=1}^{\infty} a_{n} e^{-S_{n} x}$ \begin{CJK}{UTF8}{mj}在\end{CJK} $[a,+\infty)$ \begin{CJK}{UTF8}{mj}上一致收敛域\end{CJK}, \begin{CJK}{UTF8}{mj}在\end{CJK} $(-\infty, a]$ \begin{CJK}{UTF8}{mj}上非一致收敛\end{CJK}.

(2) \begin{CJK}{UTF8}{mj}当级数\end{CJK} $\sum_{n=1}^{\infty} a_{n}$ \begin{CJK}{UTF8}{mj}发散时\end{CJK}, \begin{CJK}{UTF8}{mj}讨论\end{CJK} $\sum_{n=1}^{\infty} a_{n} e^{-S_{n} x}$ \begin{CJK}{UTF8}{mj}的收敛域\end{CJK}, \begin{CJK}{UTF8}{mj}一致收敛域及非\end{CJK} \begin{CJK}{UTF8}{mj}致收敛域\end{CJK}. 17. \begin{CJK}{UTF8}{mj}设\end{CJK} $f(x)$ \begin{CJK}{UTF8}{mj}在\end{CJK} $[0,1]$ \begin{CJK}{UTF8}{mj}上二阶可导\end{CJK}, $f(0)=f(1)=0, f(x) \neq 0(x \in(0,1))$, \begin{CJK}{UTF8}{mj}且\end{CJK} $\int_{0}^{1}\left|\frac{f^{\prime \prime}(x)}{f(x)}\right| \mathrm{d} x$ \begin{CJK}{UTF8}{mj}存\end{CJK} \begin{CJK}{UTF8}{mj}在\end{CJK}. \begin{CJK}{UTF8}{mj}试证\end{CJK}: $\int_{0}^{1}\left|\frac{f^{\prime \prime}(x)}{f(x)}\right| d x \geqslant 4$.

\section{$1.252020$ 年}
\begin{CJK}{UTF8}{mj}一\end{CJK}、\begin{CJK}{UTF8}{mj}判断下列命题是否正确\end{CJK}, \begin{CJK}{UTF8}{mj}若正确给出证明\end{CJK}, \begin{CJK}{UTF8}{mj}若错误举出反例\end{CJK} (\begin{CJK}{UTF8}{mj}每小题\end{CJK} 6 \begin{CJK}{UTF8}{mj}分\end{CJK}, \begin{CJK}{UTF8}{mj}共\end{CJK} 36 \begin{CJK}{UTF8}{mj}分\end{CJK})

\begin{enumerate}
  \item $\lim _{n \rightarrow \infty} a_{n}=A \in \mathbb{R}$ \begin{CJK}{UTF8}{mj}的充要条件是\end{CJK}: \begin{CJK}{UTF8}{mj}对任何正整数\end{CJK} $k, \exists N>0$, \begin{CJK}{UTF8}{mj}当\end{CJK} $n>N$ \begin{CJK}{UTF8}{mj}时有\end{CJK}
\end{enumerate}
$$
\left|a_{n}-A\right|<\frac{k}{k^{2}+1}
$$

\begin{enumerate}
  \setcounter{enumi}{2}
  \item \begin{CJK}{UTF8}{mj}若\end{CJK} $f(x)$ \begin{CJK}{UTF8}{mj}在\end{CJK} $x=0$ \begin{CJK}{UTF8}{mj}的某邻域商有定义\end{CJK}, \begin{CJK}{UTF8}{mj}且\end{CJK} $\lim _{x \rightarrow 0} \frac{f(x)-f(-x)}{2 x}$ \begin{CJK}{UTF8}{mj}存在\end{CJK}, \begin{CJK}{UTF8}{mj}则\end{CJK} $f^{\prime}(0)$ \begin{CJK}{UTF8}{mj}存在\end{CJK}.

  \item \begin{CJK}{UTF8}{mj}若\end{CJK} $f(x)$ \begin{CJK}{UTF8}{mj}在\end{CJK} $[a, b]$ \begin{CJK}{UTF8}{mj}可积\end{CJK}, \begin{CJK}{UTF8}{mj}则\end{CJK} $f(x)$ \begin{CJK}{UTF8}{mj}在\end{CJK} $[a, b]$ \begin{CJK}{UTF8}{mj}存在原函数\end{CJK}.

  \item \begin{CJK}{UTF8}{mj}若\end{CJK} $f(x)$ \begin{CJK}{UTF8}{mj}在\end{CJK} $[0,1]$ \begin{CJK}{UTF8}{mj}连续且\end{CJK} $\int_{0}^{1} f^{2}(x) \mathrm{d} x=0$, \begin{CJK}{UTF8}{mj}则\end{CJK} $f(x)$ \begin{CJK}{UTF8}{mj}在\end{CJK} $[0,1]$ \begin{CJK}{UTF8}{mj}上恒等于\end{CJK} 0 .

  \item \begin{CJK}{UTF8}{mj}若级数\end{CJK} $\sum_{n=1}^{+\infty} a_{n}$ \begin{CJK}{UTF8}{mj}和\end{CJK} $\sum_{n=1}^{+\infty} b_{n}$ \begin{CJK}{UTF8}{mj}均收敛\end{CJK}, \begin{CJK}{UTF8}{mj}则\end{CJK} $\sum_{n=1}^{+\infty} a_{n} b_{n}$ \begin{CJK}{UTF8}{mj}也收敛\end{CJK}.

  \item \begin{CJK}{UTF8}{mj}设\end{CJK} $x_{0}$ \begin{CJK}{UTF8}{mj}是\end{CJK} $f(x)$ \begin{CJK}{UTF8}{mj}的一个极小值点\end{CJK}, \begin{CJK}{UTF8}{mj}则一定存在\end{CJK} $\delta>0$ \begin{CJK}{UTF8}{mj}使\end{CJK} $f(x)$ \begin{CJK}{UTF8}{mj}在\end{CJK} $\left(x_{0}-\delta, x_{0}\right)$ \begin{CJK}{UTF8}{mj}上单调递减\end{CJK}, \begin{CJK}{UTF8}{mj}在\end{CJK} $\left(x_{0}, x_{0}+\delta\right)$ \begin{CJK}{UTF8}{mj}上单调递增\end{CJK}.

\end{enumerate}
\begin{CJK}{UTF8}{mj}二\end{CJK}、\begin{CJK}{UTF8}{mj}求解下列各题\end{CJK} (\begin{CJK}{UTF8}{mj}每小题\end{CJK} 8 \begin{CJK}{UTF8}{mj}分\end{CJK}, \begin{CJK}{UTF8}{mj}共\end{CJK} 40 \begin{CJK}{UTF8}{mj}分\end{CJK})
$$
\int_{0}^{2 \pi} \sqrt{1+\sin x} d x
$$
$$
\lim _{x \rightarrow 0} \frac{\sin (\sin x)-\tan (\tan x)}{x^{3}}
$$

\begin{enumerate}
  \setcounter{enumi}{9}
  \item \begin{CJK}{UTF8}{mj}求和\end{CJK}
\end{enumerate}
$$
\sum_{n=0}^{\infty}(-1)^{n} \frac{1}{3^{n}(2 n+1)}
$$

\begin{enumerate}
  \setcounter{enumi}{10}
  \item \begin{CJK}{UTF8}{mj}求\end{CJK}
\end{enumerate}
$$
\iint_{\Sigma}\left(z^{2}+x\right) \mathrm{d} y \mathrm{~d} z+\sqrt{z} \mathrm{~d} x \mathrm{~d} y
$$
\begin{CJK}{UTF8}{mj}其中\end{CJK} $\Sigma$ \begin{CJK}{UTF8}{mj}为抛物面\end{CJK} $z=\left(x^{2}+y^{2}\right) / 2$ \begin{CJK}{UTF8}{mj}在平面\end{CJK} $z=0$ \begin{CJK}{UTF8}{mj}与\end{CJK} $z=2$ \begin{CJK}{UTF8}{mj}之间的部分\end{CJK}, \begin{CJK}{UTF8}{mj}方向取下侧\end{CJK}.

\begin{enumerate}
  \setcounter{enumi}{11}
  \item \begin{CJK}{UTF8}{mj}已知\end{CJK} $\lim _{n \rightarrow+\infty} a_{n}=A$, \begin{CJK}{UTF8}{mj}求\end{CJK}
\end{enumerate}
$$
\lim _{n \rightarrow+\infty} \frac{a_{n+1}}{n+1}+\cdots+\frac{a_{2 n}}{2 n} .
$$
\begin{CJK}{UTF8}{mj}三\end{CJK}、\begin{CJK}{UTF8}{mj}证明下列各题\end{CJK}(\begin{CJK}{UTF8}{mj}第\end{CJK} 12 \begin{CJK}{UTF8}{mj}题\end{CJK} 14 \begin{CJK}{UTF8}{mj}分\end{CJK}, $13-16$ \begin{CJK}{UTF8}{mj}题\end{CJK} 15 \begin{CJK}{UTF8}{mj}分\end{CJK}。\begin{CJK}{UTF8}{mj}共\end{CJK} 74 \begin{CJK}{UTF8}{mj}分\end{CJK})

\begin{enumerate}
  \setcounter{enumi}{12}
  \item \begin{CJK}{UTF8}{mj}设\end{CJK} $a_{n}>0(n=1,2, \ldots), S_{n}=a_{1}+\cdots+a_{n}$, \begin{CJK}{UTF8}{mj}证明\end{CJK}: $\sum_{n=1}^{+\infty} a_{n}$ \begin{CJK}{UTF8}{mj}与\end{CJK} $\sum_{n=1}^{+\infty} \frac{a_{n}}{S_{n}}$ \begin{CJK}{UTF8}{mj}相同的敛散性\end{CJK}.

  \item \begin{CJK}{UTF8}{mj}已知\end{CJK} $\left\{a_{n}\right\}$ \begin{CJK}{UTF8}{mj}有界且每一项非负\end{CJK}, \begin{CJK}{UTF8}{mj}证明\end{CJK}:

\end{enumerate}
$$
\lim _{n \rightarrow \infty} \sqrt[n]{a_{1}^{n}+\cdots+a_{n}^{n}}=\sup _{n \geq 1} a_{n}
$$

\begin{enumerate}
  \setcounter{enumi}{14}
  \item \begin{CJK}{UTF8}{mj}㝊\end{CJK}
\end{enumerate}
$$
Q(x)=\left\{\begin{array}{cl}
q, & x=\frac{p}{q} \in(0,1) \\
0, & (0,1) \text { 上的其它点 }
\end{array} \quad \text { 里 } p \text { 是互素的正整数, }\right. \text {, }
$$
\begin{CJK}{UTF8}{mj}证明\end{CJK}: \begin{CJK}{UTF8}{mj}对任意\end{CJK} $x_{0} \in(0,1)$ \begin{CJK}{UTF8}{mj}以及任意\end{CJK} $\delta>0$, \begin{CJK}{UTF8}{mj}满足\end{CJK} $U\left(x_{0} ; \delta\right) \subset(0,1) . Q(x)$ \begin{CJK}{UTF8}{mj}在\end{CJK} $U\left(x_{0} ; \delta\right)$ \begin{CJK}{UTF8}{mj}上无界\end{CJK}.

\begin{enumerate}
  \setcounter{enumi}{15}
  \item $u_{n}(x)$ \begin{CJK}{UTF8}{mj}在\end{CJK} $[a, b]$ \begin{CJK}{UTF8}{mj}连续\end{CJK}, \begin{CJK}{UTF8}{mj}且\end{CJK} $u_{n}(x) \geq 0, n=1,2 \ldots$, \begin{CJK}{UTF8}{mj}设\end{CJK} $\sum_{n=1}^{+\infty} u_{n}(x)$ \begin{CJK}{UTF8}{mj}在\end{CJK} $[a, b]$ \begin{CJK}{UTF8}{mj}上收敛\end{CJK}. \begin{CJK}{UTF8}{mj}记\end{CJK} $f(x)=$ $\sum_{n=1}^{+\infty} u_{n}(x)$. \begin{CJK}{UTF8}{mj}证明\end{CJK}: $f(x)$ \begin{CJK}{UTF8}{mj}在\end{CJK} $[a, b]$ \begin{CJK}{UTF8}{mj}上有最小值\end{CJK}.

  \item \begin{CJK}{UTF8}{mj}设\end{CJK} $f(x)$ \begin{CJK}{UTF8}{mj}是定义在\end{CJK} $[0, \infty)$ \begin{CJK}{UTF8}{mj}的非负函数且可导\end{CJK}, \begin{CJK}{UTF8}{mj}满足\end{CJK} $\int_{0}^{+\infty} f(x) \mathrm{d} x$ \begin{CJK}{UTF8}{mj}收敛\end{CJK}. \begin{CJK}{UTF8}{mj}证明\end{CJK}: $\exists x_{n} \rightarrow+\infty$, \begin{CJK}{UTF8}{mj}使\end{CJK} \begin{CJK}{UTF8}{mj}待\end{CJK}

\end{enumerate}
$$
\lim _{n \rightarrow \infty}\left[f^{2}\left(x_{n}\right)+f^{\prime}\left(x_{n}\right)^{2}\right]=0
$$

\section{$1.262021$ 年}
\begin{CJK}{UTF8}{mj}一\end{CJK}、\begin{CJK}{UTF8}{mj}判断下列命题是否正确\end{CJK}, \begin{CJK}{UTF8}{mj}若正确给出证明\end{CJK}, \begin{CJK}{UTF8}{mj}若错误举出反例\end{CJK}(\begin{CJK}{UTF8}{mj}每小题\end{CJK} 6 \begin{CJK}{UTF8}{mj}分\end{CJK}, \begin{CJK}{UTF8}{mj}共\end{CJK} 30 \begin{CJK}{UTF8}{mj}分\end{CJK})

\begin{enumerate}
  \item \begin{CJK}{UTF8}{mj}数列\end{CJK} $\left\{a_{n}\right\}$ \begin{CJK}{UTF8}{mj}收敛的充要条件是\end{CJK} $\forall \varepsilon>0, \exists N>0$, \begin{CJK}{UTF8}{mj}当\end{CJK} $n>N$ \begin{CJK}{UTF8}{mj}时\end{CJK}, \begin{CJK}{UTF8}{mj}有\end{CJK} $\left|a_{n}-a_{2 n}\right|<\varepsilon$.

  \item \begin{CJK}{UTF8}{mj}若函数\end{CJK} $f(x)$ \begin{CJK}{UTF8}{mj}在闭区间\end{CJK} $[0,2]$ \begin{CJK}{UTF8}{mj}上连续\end{CJK}, \begin{CJK}{UTF8}{mj}且有\end{CJK} $f(0)=f(2)$, \begin{CJK}{UTF8}{mj}则方程\end{CJK} $f(x)-f(x+1)=0$ \begin{CJK}{UTF8}{mj}有解\end{CJK}.

  \item \begin{CJK}{UTF8}{mj}右函数\end{CJK} $f(x)$ \begin{CJK}{UTF8}{mj}在\end{CJK} $[a, b]$ \begin{CJK}{UTF8}{mj}上存在原函数\end{CJK}, \begin{CJK}{UTF8}{mj}则\end{CJK} $f(x)$ \begin{CJK}{UTF8}{mj}在\end{CJK} $[a, b]$ \begin{CJK}{UTF8}{mj}上黎曼可积\end{CJK}.

  \item \begin{CJK}{UTF8}{mj}若无穷积分\end{CJK} $\int_{1}^{+\infty} f(x) \mathrm{d} x$ \begin{CJK}{UTF8}{mj}收敛\end{CJK}, \begin{CJK}{UTF8}{mj}且\end{CJK} $f(x)$ \begin{CJK}{UTF8}{mj}在\end{CJK} $[1,+\infty)$ \begin{CJK}{UTF8}{mj}上连续\end{CJK}, \begin{CJK}{UTF8}{mj}则\end{CJK} $\lim _{x \rightarrow+\infty} f(x)=0$.

  \item \begin{CJK}{UTF8}{mj}若函数\end{CJK} $f(x)$ \begin{CJK}{UTF8}{mj}在\end{CJK} $(-1,1)$ \begin{CJK}{UTF8}{mj}上有定义\end{CJK}, \begin{CJK}{UTF8}{mj}在\end{CJK} $(-1,0) \cup(0,1)$ \begin{CJK}{UTF8}{mj}上可导\end{CJK}, \begin{CJK}{UTF8}{mj}且\end{CJK} $\lim _{x \rightarrow 0} f^{\prime}(x)$ \begin{CJK}{UTF8}{mj}存在\end{CJK}, \begin{CJK}{UTF8}{mj}则\end{CJK} $f^{\prime}(0)$ \begin{CJK}{UTF8}{mj}也存\end{CJK} \begin{CJK}{UTF8}{mj}庄\end{CJK}.

\end{enumerate}
\begin{CJK}{UTF8}{mj}二\end{CJK}、\begin{CJK}{UTF8}{mj}求解下列各题\end{CJK}(\begin{CJK}{UTF8}{mj}每小题\end{CJK} 9 \begin{CJK}{UTF8}{mj}分\end{CJK}, \begin{CJK}{UTF8}{mj}共\end{CJK} 45 \begin{CJK}{UTF8}{mj}分\end{CJK})

\begin{enumerate}
  \setcounter{enumi}{6}
  \item \begin{CJK}{UTF8}{mj}求极限\end{CJK}
\end{enumerate}
$$
\lim _{n \rightarrow \infty} \frac{\sqrt[n]{n(n+1)(n+2) \cdots(2 n-1)}}{n}
$$

\begin{enumerate}
  \setcounter{enumi}{7}
  \item \begin{CJK}{UTF8}{mj}假设二元函数\end{CJK} $u=f(x, y)$ \begin{CJK}{UTF8}{mj}满足\end{CJK} $u_{x x}+u_{y y}=0$. \begin{CJK}{UTF8}{mj}令\end{CJK} $v=f\left(\frac{x}{x^{2}+y^{2}}, \frac{y}{x^{2}+y^{2}}\right)$, \begin{CJK}{UTF8}{mj}且\end{CJK} $x^{2}+y^{2} \neq 0$. \begin{CJK}{UTF8}{mj}计算证\end{CJK} \begin{CJK}{UTF8}{mj}日明\end{CJK}
\end{enumerate}
$$
v_{x x}+v_{y y}=0
$$

\begin{enumerate}
  \setcounter{enumi}{8}
  \item \begin{CJK}{UTF8}{mj}求\end{CJK} $\int_{L} \frac{\mathrm{d} y-\mathrm{d} x}{x-y+1}$, \begin{CJK}{UTF8}{mj}其中\end{CJK} $L$ \begin{CJK}{UTF8}{mj}表示\end{CJK} $x^{2}+y^{2}=2 x$ \begin{CJK}{UTF8}{mj}沿\end{CJK} $x$ \begin{CJK}{UTF8}{mj}增长的方向\end{CJK}.

  \item \begin{CJK}{UTF8}{mj}将\end{CJK} $f(x)=(x-1)^{2}$ \begin{CJK}{UTF8}{mj}在\end{CJK} $(0,1)$ \begin{CJK}{UTF8}{mj}上展成余弦级数\end{CJK}, \begin{CJK}{UTF8}{mj}并证明\end{CJK} $\sum_{n=1}^{\infty} \frac{1}{n^{2}}=\frac{\pi}{6}$.

\end{enumerate}
10 . \begin{CJK}{UTF8}{mj}设\end{CJK} $f(x, y)$ \begin{CJK}{UTF8}{mj}在\end{CJK} $D=\left\{(x, y) \mid x^{2}+y^{2} \leqslant 1\right\}$ \begin{CJK}{UTF8}{mj}上非负连续\end{CJK}, \begin{CJK}{UTF8}{mj}求\end{CJK}
$$
\lim _{n \rightarrow \infty}\left(\iint_{D} f^{n}(x, y) \mathrm{d} x \mathrm{~d} y\right)^{1 / n} .
$$
\begin{CJK}{UTF8}{mj}三\end{CJK}、\begin{CJK}{UTF8}{mj}证明下夗各题\end{CJK} (\begin{CJK}{UTF8}{mj}每小题\end{CJK} 15 \begin{CJK}{UTF8}{mj}分\end{CJK}, \begin{CJK}{UTF8}{mj}共\end{CJK} 75 \begin{CJK}{UTF8}{mj}分\end{CJK})

\begin{enumerate}
  \setcounter{enumi}{11}
  \item \begin{CJK}{UTF8}{mj}若数项级数\end{CJK} $\sum_{n=1}^{+\infty} a_{n}$ \begin{CJK}{UTF8}{mj}收敛\end{CJK}, \begin{CJK}{UTF8}{mj}且\end{CJK} $\left\{a_{n}\right\}$ \begin{CJK}{UTF8}{mj}单调\end{CJK}, \begin{CJK}{UTF8}{mj}则\end{CJK} $\lim _{n \rightarrow \infty} n a_{n}=0$.

  \item \begin{CJK}{UTF8}{mj}证明\end{CJK}:

\end{enumerate}
$$
\int_{0}^{+\infty} \frac{e^{-t} \sin (t x)}{t} \mathrm{~d} t=\arctan x
$$

\begin{enumerate}
  \setcounter{enumi}{13}
  \item \begin{CJK}{UTF8}{mj}设函数\end{CJK} $f(u)$ \begin{CJK}{UTF8}{mj}在闭区间\end{CJK} $I$ \begin{CJK}{UTF8}{mj}上连续\end{CJK}, $\left\{g_{n}(x)\right\}$ \begin{CJK}{UTF8}{mj}在\end{CJK} $[a, b]$ \begin{CJK}{UTF8}{mj}上一致收敛\end{CJK}, \begin{CJK}{UTF8}{mj}且\end{CJK} $\forall n>N, x \in[a, b]$ \begin{CJK}{UTF8}{mj}有\end{CJK} $g_{n}(x) \in I$, \begin{CJK}{UTF8}{mj}证明\end{CJK}: $\left\{f\left(g_{n}(x)\right)\right\}$ \begin{CJK}{UTF8}{mj}一致收敛\end{CJK}.

  \item \begin{CJK}{UTF8}{mj}如果函数\end{CJK} $f(x)$ \begin{CJK}{UTF8}{mj}在\end{CJK} $[a, b]$ \begin{CJK}{UTF8}{mj}上连续\end{CJK}, \begin{CJK}{UTF8}{mj}且存在常数\end{CJK} $\tau \in(0,1)$ \begin{CJK}{UTF8}{mj}满足\end{CJK} $\forall x \in[a, b], \exists y \in[a, b]$, \begin{CJK}{UTF8}{mj}使得\end{CJK} $|f(x)| \leqslant \tau|f(y)|$, \begin{CJK}{UTF8}{mj}证明\end{CJK}: \begin{CJK}{UTF8}{mj}存在\end{CJK} $\xi \in[a, b]$ \begin{CJK}{UTF8}{mj}使得\end{CJK} $f(\xi)=0$.

\end{enumerate}
15 . \begin{CJK}{UTF8}{mj}设\end{CJK} $f(x)$ \begin{CJK}{UTF8}{mj}在\end{CJK} $[a,+\infty)$ \begin{CJK}{UTF8}{mj}上连续可微\end{CJK}, \begin{CJK}{UTF8}{mj}且\end{CJK}
$$
f(x+1)-f(x)=f^{\prime}(x), \quad \forall x \in[a,+\infty), \quad \lim _{x \rightarrow+\infty} f^{\prime}(x)=A
$$
\begin{CJK}{UTF8}{mj}证明\end{CJK}: $f^{\prime}(x) \equiv A, \forall x \in[a,+\infty)$ $1.272022$ \begin{CJK}{UTF8}{mj}年\end{CJK}

\section{一、求解下列各题 (每小题 $\mathbf{1 0}$ 分, 共 $\mathbf{5 0}$ 分)}
\begin{enumerate}
  \item \begin{CJK}{UTF8}{mj}已知函数\end{CJK} $f(x)$ \begin{CJK}{UTF8}{mj}在\end{CJK} $x=0$ \begin{CJK}{UTF8}{mj}处可导\end{CJK}, \begin{CJK}{UTF8}{mj}且\end{CJK} $f(a) \neq 0$. \begin{CJK}{UTF8}{mj}求极限\end{CJK}
\end{enumerate}
$$
\lim _{n \rightarrow \infty}\left(\frac{f(a+1 / n)}{f(a)}\right)^{n}
$$

\begin{enumerate}
  \setcounter{enumi}{2}
  \item \begin{CJK}{UTF8}{mj}判断极限\end{CJK}
\end{enumerate}
$$
\lim _{(x, y) \rightarrow(0,0)} \frac{x y}{\sqrt{x+y-1}-1}
$$
\begin{CJK}{UTF8}{mj}的存在性\end{CJK}, \begin{CJK}{UTF8}{mj}若存在\end{CJK}, \begin{CJK}{UTF8}{mj}给出证明过程\end{CJK}, \begin{CJK}{UTF8}{mj}若不存在\end{CJK}, \begin{CJK}{UTF8}{mj}说明理由\end{CJK}.

\begin{enumerate}
  \setcounter{enumi}{3}
  \item \begin{CJK}{UTF8}{mj}求不定积分\end{CJK}
\end{enumerate}
$$
\int \frac{\arctan x}{x^{2}\left(1+x^{2}\right)} d x
$$

\begin{enumerate}
  \setcounter{enumi}{4}
  \item \begin{CJK}{UTF8}{mj}设函数\end{CJK} $z=z(x, y)$ \begin{CJK}{UTF8}{mj}由方程\end{CJK} $e^{2 y z}+x+y^{2}+z=\frac{7}{4}$ \begin{CJK}{UTF8}{mj}所确定\end{CJK}, \begin{CJK}{UTF8}{mj}求\end{CJK} $\left.\mathrm{d} z\right|_{\left(\frac{1}{2}, \frac{1}{2}\right)}, \frac{\partial^{2} z}{\partial x \partial y} \mid\left(\frac{1}{2}, \frac{1}{2}\right)$.

  \item \begin{CJK}{UTF8}{mj}计算曲面积分\end{CJK}:

\end{enumerate}
$$
\iint_{\Sigma}(2 x+z) d y d z+z d x d y
$$
\begin{CJK}{UTF8}{mj}其中\end{CJK} $\Sigma$ \begin{CJK}{UTF8}{mj}为曲面\end{CJK} $z=x^{2}+y^{2}(0 \leq z \leq 1)$, \begin{CJK}{UTF8}{mj}方向取上侧\end{CJK}.

\begin{CJK}{UTF8}{mj}二\end{CJK}、\begin{CJK}{UTF8}{mj}证明下列各题\end{CJK}(\begin{CJK}{UTF8}{mj}第\end{CJK} 6-11 \begin{CJK}{UTF8}{mj}题\end{CJK}, \begin{CJK}{UTF8}{mj}每题\end{CJK} 15 \begin{CJK}{UTF8}{mj}分\end{CJK}, \begin{CJK}{UTF8}{mj}第\end{CJK} 12 \begin{CJK}{UTF8}{mj}题\end{CJK} 10 \begin{CJK}{UTF8}{mj}分\end{CJK}, \begin{CJK}{UTF8}{mj}共\end{CJK} 100 \begin{CJK}{UTF8}{mj}分\end{CJK}。)

\begin{enumerate}
  \setcounter{enumi}{6}
  \item (1). \begin{CJK}{UTF8}{mj}证明\end{CJK}:\begin{CJK}{UTF8}{mj}当\end{CJK} $x>0$ \begin{CJK}{UTF8}{mj}时\end{CJK},
\end{enumerate}
$$
\frac{x}{1+x}<\ln (1+x)<x
$$
(2). \begin{CJK}{UTF8}{mj}谋\end{CJK}
$$
a_{n}=1+\frac{1}{2}+\frac{1}{3}+\cdots+\frac{1}{n}-\ln n, \quad n=1,2, \ldots
$$
\begin{CJK}{UTF8}{mj}证明\end{CJK}: \begin{CJK}{UTF8}{mj}数列\end{CJK} $\left\{a_{n}\right\}$ \begin{CJK}{UTF8}{mj}收敛\end{CJK}.

\begin{enumerate}
  \setcounter{enumi}{7}
  \item \begin{CJK}{UTF8}{mj}讨论函数\end{CJK}
\end{enumerate}
$$
f(x)= \begin{cases}x(1-x), & x \text { 为有理数 } \\ x(1+x), & x \text { 为无理数, }\end{cases}
$$
\begin{CJK}{UTF8}{mj}的连续性与可微性\end{CJK}.

\begin{enumerate}
  \setcounter{enumi}{8}
  \item \begin{CJK}{UTF8}{mj}设\end{CJK} $n$ \begin{CJK}{UTF8}{mj}是正整数\end{CJK},\begin{CJK}{UTF8}{mj}证明方程\end{CJK} $x^{n}+n x-1=0$ \begin{CJK}{UTF8}{mj}有唯一的正实数根\end{CJK} $x_{n}$. \begin{CJK}{UTF8}{mj}进而\end{CJK}, \begin{CJK}{UTF8}{mj}判断级数\end{CJK} $\sum_{n=1}^{+\infty} x_{n}^{\alpha}(\alpha>1)$ \begin{CJK}{UTF8}{mj}的敛散惶\end{CJK}.

  \item \begin{CJK}{UTF8}{mj}已知函数\end{CJK} $f(x)$ \begin{CJK}{UTF8}{mj}在区间\end{CJK} $(-\delta, \delta)$ \begin{CJK}{UTF8}{mj}上有界\end{CJK}, \begin{CJK}{UTF8}{mj}且\end{CJK} $\forall x \in(-\delta, \delta)$, \begin{CJK}{UTF8}{mj}有\end{CJK} $f(x)=3 f(x / 2)$. \begin{CJK}{UTF8}{mj}证明\end{CJK}: $f(x)$ \begin{CJK}{UTF8}{mj}在\end{CJK} $x=0$ \begin{CJK}{UTF8}{mj}处连续\end{CJK}.

  \item \begin{CJK}{UTF8}{mj}设\end{CJK} $f(x, y)$ \begin{CJK}{UTF8}{mj}在\end{CJK} $[0,+\infty)$ \begin{CJK}{UTF8}{mj}和\end{CJK} $[c, d]$ \begin{CJK}{UTF8}{mj}上连续\end{CJK}, \begin{CJK}{UTF8}{mj}且无穷积分\end{CJK} $\int_{0}^{+\infty} f(x, y) \mathrm{d} x$ \begin{CJK}{UTF8}{mj}在\end{CJK} $[c, d)$ \begin{CJK}{UTF8}{mj}上致收敛\end{CJK}. \begin{CJK}{UTF8}{mj}证\end{CJK} \begin{CJK}{UTF8}{mj}明\end{CJK}: \begin{CJK}{UTF8}{mj}无穷积分\end{CJK} $\int_{0}^{+\infty} f(x, d) \mathrm{d} x$ \begin{CJK}{UTF8}{mj}收敛\end{CJK}.

  \item \begin{CJK}{UTF8}{mj}设函数\end{CJK} $f(x)$ \begin{CJK}{UTF8}{mj}在\end{CJK} $(-\infty,+\infty)$ \begin{CJK}{UTF8}{mj}上连续\end{CJK}, $f_{n}(x)=\sum_{n=1}^{n} \frac{1}{n} f\left(x+\frac{k}{n}\right), n=1,2, \ldots$ \begin{CJK}{UTF8}{mj}证明\end{CJK}:

\end{enumerate}
(1). \begin{CJK}{UTF8}{mj}函数列\end{CJK} $\left\{f_{n}(x)\right\}$ \begin{CJK}{UTF8}{mj}在\end{CJK} $(-\infty,+\infty)$ \begin{CJK}{UTF8}{mj}上处处收敛\end{CJK}, \begin{CJK}{UTF8}{mj}并给出其极限函数\end{CJK}.

(2). \begin{CJK}{UTF8}{mj}函数列\end{CJK} $\left\{f_{n}(x)\right\}$ \begin{CJK}{UTF8}{mj}在任何有限区间\end{CJK} $[a, b]$ \begin{CJK}{UTF8}{mj}上\end{CJK}“\begin{CJK}{UTF8}{mj}致收敛\end{CJK}. 12. \begin{CJK}{UTF8}{mj}已知函数\end{CJK} $f(x)$ \begin{CJK}{UTF8}{mj}在区间\end{CJK} $[a, b]$ \begin{CJK}{UTF8}{mj}上非负连续\end{CJK}, \begin{CJK}{UTF8}{mj}且严格递增\end{CJK}, \begin{CJK}{UTF8}{mj}并对任意正整数\end{CJK} $n$, \begin{CJK}{UTF8}{mj}存在\end{CJK} $x_{n} \in[a, b]$, \begin{CJK}{UTF8}{mj}使得\end{CJK}
$$
f^{n}\left(x_{n}\right)=\frac{1}{b-a} \int_{a}^{b} f^{n}(x) \mathrm{d} x
$$
\begin{CJK}{UTF8}{mj}证明数列\end{CJK} $\left\{x_{n}\right\}$ \begin{CJK}{UTF8}{mj}收敛\end{CJK}.

\section{第 2 章 高等代数}
\section{$2.11996$ 年}
\begin{enumerate}
  \item (10 \begin{CJK}{UTF8}{mj}分\end{CJK}) \begin{CJK}{UTF8}{mj}计算下列行列式的值\end{CJK}:
\end{enumerate}
$$
D_{n}=\left|\begin{array}{cccccc}
1+x & y & 0 & \cdots & 0 & 0 \\
z & 1+x & y & \cdots & 0 & 0 \\
0 & z & 1+x & \cdots & 0 & 0 \\
\vdots & \vdots & \vdots & & \vdots & \vdots \\
0 & 0 & 0 & \cdots & 1+x & y \\
0 & 0 & 0 & \cdots & z & 1+x
\end{array}\right| \text { 其中 } x
$$

\begin{enumerate}
  \setcounter{enumi}{2}
  \item (10 \begin{CJK}{UTF8}{mj}分\end{CJK}) \begin{CJK}{UTF8}{mj}已知\end{CJK} $A$ \begin{CJK}{UTF8}{mj}为\end{CJK} $n$ \begin{CJK}{UTF8}{mj}阶复矩阵\end{CJK}, \begin{CJK}{UTF8}{mj}且\end{CJK} $A^{3}=2 E, B=A^{2}-2 A+2 E$. \begin{CJK}{UTF8}{mj}试证\end{CJK}: $B$ \begin{CJK}{UTF8}{mj}为满秩方阵\end{CJK}, \begin{CJK}{UTF8}{mj}并求\end{CJK} $B^{-1}$.

  \item $(15$ \begin{CJK}{UTF8}{mj}分\end{CJK}) \begin{CJK}{UTF8}{mj}设有线性方程组\end{CJK}

\end{enumerate}
$$
\left\{\begin{array}{l}
a_{11} x_{1}+a_{12} x_{2}+\ldots+a_{1 n} x_{n}=b_{1} \\
a_{21} x_{1}+a_{22} x_{2}+\ldots+a_{2 n} x_{n}=b_{2} \\
\ldots \ldots \ldots \ldots \ldots \ldots . \ldots \ldots \\
a_{n 1} x_{1}+a_{n 2} x_{2}+\ldots+a_{n n} x_{n}=b_{n}
\end{array}\right.
$$
\begin{CJK}{UTF8}{mj}与\end{CJK}

\includegraphics[max width=\textwidth]{2022_04_18_3416d289b173eb9de8c1g-276}

\begin{CJK}{UTF8}{mj}其中\end{CJK} $A_{i j}$ \begin{CJK}{UTF8}{mj}是\end{CJK} $a_{i j}$ \begin{CJK}{UTF8}{mj}在\end{CJK} (1) \begin{CJK}{UTF8}{mj}的系数矩阵\end{CJK} $A=\left(a_{i j}\right)$ \begin{CJK}{UTF8}{mj}中的代数余子式\end{CJK}. \begin{CJK}{UTF8}{mj}证明\end{CJK}: \begin{CJK}{UTF8}{mj}方程组\end{CJK} (1) \begin{CJK}{UTF8}{mj}有唯一解\end{CJK} $\Leftrightarrow$ \begin{CJK}{UTF8}{mj}方程\end{CJK} \begin{CJK}{UTF8}{mj}组\end{CJK} (2) \begin{CJK}{UTF8}{mj}有唯一解\end{CJK}.

\begin{enumerate}
  \setcounter{enumi}{4}
  \item (15 \begin{CJK}{UTF8}{mj}分\end{CJK}) \begin{CJK}{UTF8}{mj}设\end{CJK} $f(x), g(x)$ \begin{CJK}{UTF8}{mj}是数域\end{CJK} $P$ \begin{CJK}{UTF8}{mj}上两个一元多项式\end{CJK}, $k$ \begin{CJK}{UTF8}{mj}为给定的正整数\end{CJK}. \begin{CJK}{UTF8}{mj}求证\end{CJK}:\begin{CJK}{UTF8}{mj}如果\end{CJK} $f^{k}(x) \mid g^{k}(x)$, \begin{CJK}{UTF8}{mj}则\end{CJK} $f(x) \mid g(x)$.

  \item (15 \begin{CJK}{UTF8}{mj}分\end{CJK}) \begin{CJK}{UTF8}{mj}设\end{CJK} $A=\left(\begin{array}{lll}1 & x & 1 \\ x & 1 & y \\ 1 & y & 1\end{array}\right)$ \begin{CJK}{UTF8}{mj}与\end{CJK} $B=\left(\begin{array}{lll}0 & 0 & 0 \\ 0 & 1 & 0 \\ 0 & 0 & 2\end{array}\right)$ \begin{CJK}{UTF8}{mj}相似\end{CJK},

\end{enumerate}
(1) \begin{CJK}{UTF8}{mj}求\end{CJK} $x, y$ \begin{CJK}{UTF8}{mj}的值\end{CJK}.

(2) \begin{CJK}{UTF8}{mj}求一个正交矩阵\end{CJK} $T$, \begin{CJK}{UTF8}{mj}使\end{CJK} $T^{-1} A T=T^{\prime} A T=B$.

\begin{CJK}{UTF8}{mj}注\end{CJK}: \begin{CJK}{UTF8}{mj}学科教学论考生不必做第\end{CJK} 6 \begin{CJK}{UTF8}{mj}题\end{CJK}, \begin{CJK}{UTF8}{mj}其他学科考生不必做第\end{CJK} 7 \begin{CJK}{UTF8}{mj}题\end{CJK}。

\begin{enumerate}
  \setcounter{enumi}{6}
  \item (20 \begin{CJK}{UTF8}{mj}分\end{CJK}) \begin{CJK}{UTF8}{mj}设\end{CJK} $\mathscr{A}$ \begin{CJK}{UTF8}{mj}为线性空间\end{CJK} $V$ \begin{CJK}{UTF8}{mj}上线性变换\end{CJK}, \begin{CJK}{UTF8}{mj}且\end{CJK} $\mathscr{A}^{2}=\mathscr{A}$. \begin{CJK}{UTF8}{mj}求证\end{CJK}:
\end{enumerate}
(1) $\mathscr{A}$ \begin{CJK}{UTF8}{mj}的特征值只能是\end{CJK} 1 \begin{CJK}{UTF8}{mj}或\end{CJK} 0 ;

(2) \begin{CJK}{UTF8}{mj}若用\end{CJK} $V_{0}, V_{1}$ \begin{CJK}{UTF8}{mj}分别表示对应于特征值\end{CJK} 0 \begin{CJK}{UTF8}{mj}和\end{CJK} 1 \begin{CJK}{UTF8}{mj}的特征子空间\end{CJK}, \begin{CJK}{UTF8}{mj}则\end{CJK} $V_{0}=\operatorname{Ker}(\mathscr{A}), V_{1}=\operatorname{Im}(\mathscr{A})$; (3) $V=V_{0} \oplus V_{1}=\operatorname{Ker}(\mathscr{A}) \oplus \operatorname{Im}(\mathscr{A})$;

(4) $\mathscr{A}$ \begin{CJK}{UTF8}{mj}只有\end{CJK} 0 \begin{CJK}{UTF8}{mj}特征值\end{CJK} $\Leftrightarrow \mathscr{A}=\mathscr{O}$.

\begin{enumerate}
  \setcounter{enumi}{7}
  \item (20 \begin{CJK}{UTF8}{mj}分\end{CJK}) \begin{CJK}{UTF8}{mj}设\end{CJK} $A$ \begin{CJK}{UTF8}{mj}是一方阵\end{CJK}, $g(\lambda)$ \begin{CJK}{UTF8}{mj}是\end{CJK} $A$ \begin{CJK}{UTF8}{mj}的最小多项式\end{CJK}, $f(\lambda)$ \begin{CJK}{UTF8}{mj}为任一次数大于\end{CJK} 0 \begin{CJK}{UTF8}{mj}的多项式\end{CJK}. \begin{CJK}{UTF8}{mj}证明\end{CJK}: \begin{CJK}{UTF8}{mj}方\end{CJK} \begin{CJK}{UTF8}{mj}阵\end{CJK} $f(A)$ \begin{CJK}{UTF8}{mj}非奇异\end{CJK} $\Leftrightarrow(f(\lambda), g(\lambda))=1$.
\end{enumerate}
\section{$2.21997$ 年}
\begin{enumerate}
  \item (10 \begin{CJK}{UTF8}{mj}分\end{CJK}) \begin{CJK}{UTF8}{mj}计算下列行列式的值\end{CJK}:
\end{enumerate}
$$
D_{n}=\left|\begin{array}{cccc}
1 & 1 & \cdots & 1 \\
x_{1}\left(x_{1}-1\right) & x_{2}\left(x_{2}-1\right) & \cdots & x_{n}\left(x_{n}-1\right) \\
x_{1}^{2}\left(x_{1}-1\right) & x_{2}^{2}\left(x_{2}-1\right) & \cdots & x_{n}^{2}\left(x_{n}-1\right) \\
\vdots & \vdots & \vdots \\
x_{1}^{n-1}\left(x_{1}-1\right) & x_{2}^{n-1}\left(x_{2}-1\right) & \cdots & x_{n}^{n-1}\left(x_{n}-1\right)
\end{array}\right| .
$$

\begin{enumerate}
  \setcounter{enumi}{2}
  \item (10 \begin{CJK}{UTF8}{mj}分\end{CJK}) \begin{CJK}{UTF8}{mj}设\end{CJK} $A=\left(\begin{array}{cccc}3 & -2 & 0 & 0 \\ -2 & 0 & 0 & 0 \\ 0 & 0 & 5 & -2 \\ 0 & 0 & -2 & 2\end{array}\right)$, \begin{CJK}{UTF8}{mj}求正交矩阵\end{CJK} $T$, \begin{CJK}{UTF8}{mj}使\end{CJK} $T^{\prime} A T=T^{-1} A T$ \begin{CJK}{UTF8}{mj}为对角形矩阵\end{CJK}, \begin{CJK}{UTF8}{mj}并写\end{CJK} \begin{CJK}{UTF8}{mj}出这个对角形矩阵\end{CJK}.

  \item (15 \begin{CJK}{UTF8}{mj}分\end{CJK}) \begin{CJK}{UTF8}{mj}设\end{CJK} $A=\left(\begin{array}{ccc}2 & 0 & 0 \\ a & 2 & 0 \\ b & c & -1\end{array}\right)$ \begin{CJK}{UTF8}{mj}是复矩阵\end{CJK}, \begin{CJK}{UTF8}{mj}求\end{CJK} $A$ \begin{CJK}{UTF8}{mj}的一切可能的\end{CJK} Jordan \begin{CJK}{UTF8}{mj}标准型\end{CJK}, \begin{CJK}{UTF8}{mj}并给出一个\end{CJK} $A$ \begin{CJK}{UTF8}{mj}可\end{CJK} \begin{CJK}{UTF8}{mj}对角化的充要条件\end{CJK}.

  \item (15 \begin{CJK}{UTF8}{mj}分\end{CJK}) \begin{CJK}{UTF8}{mj}已知\end{CJK} 3 \begin{CJK}{UTF8}{mj}阶实矩阵\end{CJK} $A=\left(a_{i j}\right)$ \begin{CJK}{UTF8}{mj}满足条件\end{CJK} $a_{i j}=A_{i j}(i, j=1,2,3)$, \begin{CJK}{UTF8}{mj}其中\end{CJK} $A_{i j}$ \begin{CJK}{UTF8}{mj}是\end{CJK} $a_{i j}$ \begin{CJK}{UTF8}{mj}的代数余\end{CJK} \begin{CJK}{UTF8}{mj}子式\end{CJK}, \begin{CJK}{UTF8}{mj}且\end{CJK} $a_{33}=-1$, \begin{CJK}{UTF8}{mj}求\end{CJK}

\end{enumerate}
(1). $|A|$

(2). \begin{CJK}{UTF8}{mj}方程组\end{CJK} $A\left(\begin{array}{l}x_{1} \\ x_{2} \\ x_{3}\end{array}\right)=\left(\begin{array}{l}0 \\ 0 \\ 1\end{array}\right)$ \begin{CJK}{UTF8}{mj}的解\end{CJK}.

\begin{enumerate}
  \setcounter{enumi}{5}
  \item (15 \begin{CJK}{UTF8}{mj}分\end{CJK}) \begin{CJK}{UTF8}{mj}证明\end{CJK}: \begin{CJK}{UTF8}{mj}一个非零复数\end{CJK} $\alpha$ \begin{CJK}{UTF8}{mj}是某一有理系数非零多项式的根当且仅当存在一个有理系数\end{CJK} \begin{CJK}{UTF8}{mj}多项式\end{CJK} $f(x)$ \begin{CJK}{UTF8}{mj}使得\end{CJK} $\frac{1}{\alpha}=f(\alpha)$.
\end{enumerate}
\begin{CJK}{UTF8}{mj}注\end{CJK}:\begin{CJK}{UTF8}{mj}学科教学论考生不必做第\end{CJK} $6 、 7$ \begin{CJK}{UTF8}{mj}题\end{CJK},\begin{CJK}{UTF8}{mj}其他学科考生不必做第\end{CJK} $8 、 9$ \begin{CJK}{UTF8}{mj}题\end{CJK}。

\begin{enumerate}
  \setcounter{enumi}{6}
  \item (15 \begin{CJK}{UTF8}{mj}分\end{CJK}) \begin{CJK}{UTF8}{mj}设\end{CJK} $A$ \begin{CJK}{UTF8}{mj}是\end{CJK} $n$ \begin{CJK}{UTF8}{mj}阶反対称阵\end{CJK}. \begin{CJK}{UTF8}{mj}证明\end{CJK}:
\end{enumerate}
(1) \begin{CJK}{UTF8}{mj}当\end{CJK} $n$ \begin{CJK}{UTF8}{mj}为奇数时\end{CJK} $|A|=0$; \begin{CJK}{UTF8}{mj}当\end{CJK} $n$ \begin{CJK}{UTF8}{mj}为偶数时\end{CJK} $|A|$ \begin{CJK}{UTF8}{mj}是一实数的完全平方\end{CJK}.

(2) $A$ \begin{CJK}{UTF8}{mj}的秩为偶数\end{CJK}.

\begin{enumerate}
  \setcounter{enumi}{7}
  \item (15 \begin{CJK}{UTF8}{mj}分\end{CJK}) \begin{CJK}{UTF8}{mj}设\end{CJK} $V$ \begin{CJK}{UTF8}{mj}是有限维欧式空间\end{CJK}, \begin{CJK}{UTF8}{mj}内积记为\end{CJK} $(\alpha, \beta)$. \begin{CJK}{UTF8}{mj}又设\end{CJK} $\mathscr{A}$ \begin{CJK}{UTF8}{mj}是\end{CJK} $V$ \begin{CJK}{UTF8}{mj}的一个正交变换\end{CJK}. \begin{CJK}{UTF8}{mj}记\end{CJK}
\end{enumerate}
$$
V_{1}=\{\alpha \mid \mathscr{A} \alpha=\alpha, \alpha \in V\}, V_{2}=\{\alpha-\mathscr{A} \alpha \mid \alpha \in V\}
$$
\begin{CJK}{UTF8}{mj}求证\end{CJK}: (1) $V_{1}, V_{2}$ \begin{CJK}{UTF8}{mj}是\end{CJK} $V$ \begin{CJK}{UTF8}{mj}的子空间\end{CJK};

(2) $V=V_{1} V_{2}$.

\begin{enumerate}
  \setcounter{enumi}{8}
  \item (15 \begin{CJK}{UTF8}{mj}分\end{CJK}) \begin{CJK}{UTF8}{mj}设\end{CJK} $n$ \begin{CJK}{UTF8}{mj}阶实数方阵的特征值全是实数且\end{CJK} $A$ \begin{CJK}{UTF8}{mj}的\end{CJK} 1 \begin{CJK}{UTF8}{mj}阶主子式之和为\end{CJK} 0,2 \begin{CJK}{UTF8}{mj}阶主子式之和也\end{CJK} \begin{CJK}{UTF8}{mj}为\end{CJK} 0 . \begin{CJK}{UTF8}{mj}求证\end{CJK}: $A^{n}=0$.

  \item (15 \begin{CJK}{UTF8}{mj}分\end{CJK}) \begin{CJK}{UTF8}{mj}设\end{CJK} $A, B$ \begin{CJK}{UTF8}{mj}均为正定矩阵\end{CJK}, \begin{CJK}{UTF8}{mj}证明\end{CJK}:

\end{enumerate}
(1). \begin{CJK}{UTF8}{mj}方程\end{CJK} $|\lambda A-B|=0$ \begin{CJK}{UTF8}{mj}的根均大于\end{CJK} 0 ;

(2). \begin{CJK}{UTF8}{mj}方程\end{CJK} $|\lambda A-B|=0$ \begin{CJK}{UTF8}{mj}的根均等于\end{CJK} 1 \begin{CJK}{UTF8}{mj}当且仅当\end{CJK} $A=B$.

\section{$2.31998$ 年}
\begin{enumerate}
  \item (10 \begin{CJK}{UTF8}{mj}分\end{CJK}) \begin{CJK}{UTF8}{mj}计算下列行列式\end{CJK}:
\end{enumerate}
$$
D=\left|\begin{array}{ccccc}
2^{n}-2 & 2^{n-1}-2 & \ldots & 2^{3}-2 & 2^{2}-2 \\
3^{n}-3 & 3^{n-1}-3 & \ldots & 3^{3}-3 & 3^{2}-3 \\
\vdots & \vdots & 3 & \vdots & \vdots \\
n^{n}-n & n^{n-1}-n & \ldots & n^{3}-n & n^{2}-n
\end{array}\right| .
$$

\begin{enumerate}
  \setcounter{enumi}{2}
  \item (10 \begin{CJK}{UTF8}{mj}分\end{CJK}) \begin{CJK}{UTF8}{mj}证明\end{CJK}: \begin{CJK}{UTF8}{mj}方程组\end{CJK}
\end{enumerate}
$$
\left\{\begin{array}{l}
a_{11} x_{1}+a_{12} x_{2}+\ldots+a_{1 n} x_{n}=0 \\
a_{21} x_{1}+a_{22} x_{2}+\ldots+a_{2 n} x_{n}=0 \\
\ldots \ldots \ldots \ldots \ldots \ldots . \ldots \ldots \\
a_{s 1} x_{1}+a_{s 2} x_{2}+\ldots+a_{s n} x_{n}=0
\end{array}\right.
$$
\begin{CJK}{UTF8}{mj}的解全是方程组\end{CJK}
$$
b_{1} x_{1}+b_{2} x_{2}+\ldots+b_{n} x_{n}=0
$$
\begin{CJK}{UTF8}{mj}的解的充要条件是\end{CJK}: $\beta=\left(b_{1}, b_{2} \ldots b_{n}\right)^{T}$ \begin{CJK}{UTF8}{mj}可由向量组\end{CJK} $\alpha_{1}, \alpha_{2} \ldots \alpha_{s}$ \begin{CJK}{UTF8}{mj}线性表示\end{CJK}, \begin{CJK}{UTF8}{mj}其中\end{CJK} $\alpha_{i}=\left(a_{i 1}, a_{i 2} \ldots a_{i n}\right)^{T}(i=$ $1,2 \ldots s)$.

\begin{enumerate}
  \setcounter{enumi}{3}
  \item (15 \begin{CJK}{UTF8}{mj}分\end{CJK}) \begin{CJK}{UTF8}{mj}设\end{CJK} $f(x)=x^{3}+a x^{2}+b x+c$ \begin{CJK}{UTF8}{mj}是整系数多项式\end{CJK}, \begin{CJK}{UTF8}{mj}证明\end{CJK}: \begin{CJK}{UTF8}{mj}若\end{CJK} $a c+b c$ \begin{CJK}{UTF8}{mj}为奇数\end{CJK}, \begin{CJK}{UTF8}{mj}则\end{CJK} $f(x)$ \begin{CJK}{UTF8}{mj}在有理\end{CJK} \begin{CJK}{UTF8}{mj}数域上不可约\end{CJK}.

  \item (15 \begin{CJK}{UTF8}{mj}分\end{CJK}) \begin{CJK}{UTF8}{mj}设\end{CJK} $A$ \begin{CJK}{UTF8}{mj}是\end{CJK} $n$ \begin{CJK}{UTF8}{mj}阶非奇异实对称矩阵\end{CJK}, $B$ \begin{CJK}{UTF8}{mj}是反对称矩阵\end{CJK}, \begin{CJK}{UTF8}{mj}且\end{CJK} $A B=B A$. \begin{CJK}{UTF8}{mj}证明\end{CJK}: $A+B$ \begin{CJK}{UTF8}{mj}必是非\end{CJK} \begin{CJK}{UTF8}{mj}奇异的\end{CJK}.

  \item (15 \begin{CJK}{UTF8}{mj}分\end{CJK}) \begin{CJK}{UTF8}{mj}设\end{CJK} $A$ \begin{CJK}{UTF8}{mj}为\end{CJK} $n$ \begin{CJK}{UTF8}{mj}阶方阵\end{CJK}, $f(\lambda)=|\lambda E-A|$ \begin{CJK}{UTF8}{mj}是\end{CJK} $A$ \begin{CJK}{UTF8}{mj}的特征多项式\end{CJK}, \begin{CJK}{UTF8}{mj}令\end{CJK} $g(\lambda)=\frac{f(\lambda)}{\left(f(\lambda), f^{\prime}(\lambda)\right)}$. \begin{CJK}{UTF8}{mj}证\end{CJK} \begin{CJK}{UTF8}{mj}明\end{CJK}: $A$ \begin{CJK}{UTF8}{mj}与一对角矩阵相似的充要条件是\end{CJK} $g(A)=0$.

\end{enumerate}
\begin{CJK}{UTF8}{mj}注\end{CJK}:\begin{CJK}{UTF8}{mj}学科教学论考生不必做第\end{CJK} 6、7 \begin{CJK}{UTF8}{mj}题\end{CJK},\begin{CJK}{UTF8}{mj}其他学科考生不必做第\end{CJK} $8 、 9$ \begin{CJK}{UTF8}{mj}题\end{CJK}。

\begin{enumerate}
  \setcounter{enumi}{6}
  \item (15 \begin{CJK}{UTF8}{mj}分\end{CJK}) \begin{CJK}{UTF8}{mj}设\end{CJK} $\mathscr{A}$ \begin{CJK}{UTF8}{mj}是\end{CJK} $n$ \begin{CJK}{UTF8}{mj}欧式空间\end{CJK} $V$ \begin{CJK}{UTF8}{mj}上的线性变换\end{CJK}, $\mathscr{A}^{*}$ \begin{CJK}{UTF8}{mj}是同一空间\end{CJK} $V$ \begin{CJK}{UTF8}{mj}的变换\end{CJK}, \begin{CJK}{UTF8}{mj}且\end{CJK} $\forall \alpha, \beta \in V$, \begin{CJK}{UTF8}{mj}有\end{CJK} $(\mathscr{A}(\alpha), \beta)=\left(\alpha, \mathscr{A}^{*}(\beta)\right)$. \begin{CJK}{UTF8}{mj}证明\end{CJK}:
\end{enumerate}
(1) $\mathscr{A}^{*}$ \begin{CJK}{UTF8}{mj}是\end{CJK} $V$ \begin{CJK}{UTF8}{mj}上的线性变换\end{CJK};

(2) $\mathscr{A}$ \begin{CJK}{UTF8}{mj}的核等于\end{CJK} $\mathscr{A}^{*}$ \begin{CJK}{UTF8}{mj}的值域的正交补\end{CJK}.

\begin{enumerate}
  \setcounter{enumi}{7}
  \item (15 \begin{CJK}{UTF8}{mj}分\end{CJK}) \begin{CJK}{UTF8}{mj}证明\end{CJK}: \begin{CJK}{UTF8}{mj}任意方阵可以表示成两个对称方阵之积\end{CJK}, \begin{CJK}{UTF8}{mj}其中一个是可逆的\end{CJK}.

  \item (15 \begin{CJK}{UTF8}{mj}分\end{CJK}) \begin{CJK}{UTF8}{mj}设\end{CJK} $f(x)$ \begin{CJK}{UTF8}{mj}是数域\end{CJK} $P$ \begin{CJK}{UTF8}{mj}上的多项式\end{CJK}, \begin{CJK}{UTF8}{mj}且有\end{CJK} $f(x)=f_{1}(x) f_{2}(x),\left(f_{1}(x), f_{2}(x)\right)=1$. \begin{CJK}{UTF8}{mj}又设\end{CJK} $V$ \begin{CJK}{UTF8}{mj}是\end{CJK} $P$ \begin{CJK}{UTF8}{mj}上的\end{CJK} $n$ \begin{CJK}{UTF8}{mj}维线性空间\end{CJK}, $\mathscr{A}$ \begin{CJK}{UTF8}{mj}是\end{CJK} $V$ \begin{CJK}{UTF8}{mj}的一个线性变换\end{CJK}. $\varpi$ \begin{CJK}{UTF8}{mj}为\end{CJK} $f(\mathscr{A})$ \begin{CJK}{UTF8}{mj}的核\end{CJK}, $\varpi_{1}$ \begin{CJK}{UTF8}{mj}为\end{CJK} $f_{1}(\mathscr{A})$ \begin{CJK}{UTF8}{mj}的核\end{CJK}, $\varpi_{2}$ \begin{CJK}{UTF8}{mj}为\end{CJK} $f_{2}(\mathscr{A})$ \begin{CJK}{UTF8}{mj}的核\end{CJK}. \begin{CJK}{UTF8}{mj}证明\end{CJK}: $\varpi=\varpi_{1} \varpi_{2}$.

  \item (15 \begin{CJK}{UTF8}{mj}分\end{CJK}) \begin{CJK}{UTF8}{mj}设\end{CJK} $a+b i(i=\sqrt{-1})$ \begin{CJK}{UTF8}{mj}是\end{CJK} $n$ \begin{CJK}{UTF8}{mj}阶实方阵\end{CJK} $A$ \begin{CJK}{UTF8}{mj}的任意特征值\end{CJK}, \begin{CJK}{UTF8}{mj}其中\end{CJK} $a, b$ \begin{CJK}{UTF8}{mj}是实数\end{CJK}. \begin{CJK}{UTF8}{mj}证明\end{CJK}: \begin{CJK}{UTF8}{mj}若\end{CJK} $A+A^{T}$ \begin{CJK}{UTF8}{mj}的\end{CJK} $n$ \begin{CJK}{UTF8}{mj}个特征值是\end{CJK} $\mu_{1}, \mu_{2}, \cdots, \mu_{n}$, \begin{CJK}{UTF8}{mj}则必有\end{CJK}

\end{enumerate}
$$
\frac{1}{2} \min _{1 \leqslant i \leqslant n} \mu_{i} \leqslant a \leqslant \frac{1}{2} \max _{1 \leqslant i \leqslant n} \mu_{i}
$$

\section{$2.41999$ 年}
\begin{enumerate}
  \item (15 \begin{CJK}{UTF8}{mj}分\end{CJK}) \begin{CJK}{UTF8}{mj}计算下列行列式\end{CJK}:
\end{enumerate}
$$
D_{n}=\left|\begin{array}{ccccccc}
n-1 & n-2 & \cdots & 3 & 2 & 1 & 0 \\
n-2 & n-3 & \cdots & 2 & 1 & 0 & 1 \\
n-3 & n-4 & \ldots & 1 & 0 & 1 & 2 \\
\vdots & \vdots & & \vdots & \vdots & \vdots & \vdots \\
1 & 0 & \cdots & n-5 & n-4 & n-3 & n-2 \\
0 & 1 & \cdots & n-4 & n-3 & n-2 & n-1
\end{array}\right|
$$

\begin{enumerate}
  \setcounter{enumi}{2}
  \item (15 \begin{CJK}{UTF8}{mj}分\end{CJK}) \begin{CJK}{UTF8}{mj}设\end{CJK} $p$ \begin{CJK}{UTF8}{mj}是一个素数\end{CJK},\begin{CJK}{UTF8}{mj}多项式\end{CJK} $f(x)=x^{p-1}+x^{p-2}+\cdots+x+1$. \begin{CJK}{UTF8}{mj}证明\end{CJK}: $f(x)$ \begin{CJK}{UTF8}{mj}在有理数域上\end{CJK} \begin{CJK}{UTF8}{mj}不可约\end{CJK}.

  \item (15 \begin{CJK}{UTF8}{mj}分\end{CJK}) \begin{CJK}{UTF8}{mj}设\end{CJK} $A=\left(\begin{array}{lll}1 & x & 1 \\ x & 1 & y \\ 1 & y & 1\end{array}\right)$ \begin{CJK}{UTF8}{mj}与\end{CJK} $B=\left(\begin{array}{lll}0 & 0 & 0 \\ 0 & 1 & 0 \\ 0 & 0 & 2\end{array}\right)$ \begin{CJK}{UTF8}{mj}相似\end{CJK},

\end{enumerate}
(1) \begin{CJK}{UTF8}{mj}求\end{CJK} $x, y$ \begin{CJK}{UTF8}{mj}的值\end{CJK}.

(2) \begin{CJK}{UTF8}{mj}求一个正交矩阵\end{CJK} $T$, \begin{CJK}{UTF8}{mj}使\end{CJK} $T^{-1} A T=T^{\prime} A T=B$.

\begin{enumerate}
  \setcounter{enumi}{4}
  \item (15 \begin{CJK}{UTF8}{mj}分\end{CJK}) \begin{CJK}{UTF8}{mj}设\end{CJK} $A$ \begin{CJK}{UTF8}{mj}是\end{CJK} $n$ \begin{CJK}{UTF8}{mj}阶实矩阵\end{CJK}, $A^{T}$ \begin{CJK}{UTF8}{mj}是\end{CJK} $A$ \begin{CJK}{UTF8}{mj}的转置矩阵\end{CJK}, \begin{CJK}{UTF8}{mj}求证\end{CJK}:
\end{enumerate}
(1) $\operatorname{rank}\left(A^{T} A\right)=\operatorname{rank}\left(A A^{T}\right)=\operatorname{rank}(A)$;

(2) $A^{T} A$ \begin{CJK}{UTF8}{mj}正定\end{CJK} $\Leftrightarrow|A| \neq 0$.

\begin{enumerate}
  \setcounter{enumi}{5}
  \item (20 \begin{CJK}{UTF8}{mj}分\end{CJK}) \begin{CJK}{UTF8}{mj}设\end{CJK} $A$ \begin{CJK}{UTF8}{mj}是\end{CJK} $n$ \begin{CJK}{UTF8}{mj}阶矩阵\end{CJK}, \begin{CJK}{UTF8}{mj}证明\end{CJK}:
\end{enumerate}
(1) $A$ \begin{CJK}{UTF8}{mj}的特征多项式\end{CJK} $f(x)$ \begin{CJK}{UTF8}{mj}与\end{CJK} $A$ \begin{CJK}{UTF8}{mj}的最小多项式\end{CJK} $m(x)$ \begin{CJK}{UTF8}{mj}的根相同\end{CJK};

(2) \begin{CJK}{UTF8}{mj}若\end{CJK} $A$ \begin{CJK}{UTF8}{mj}的特征根互异\end{CJK}, \begin{CJK}{UTF8}{mj}则\end{CJK} $m(x)=f(x)$.

\begin{CJK}{UTF8}{mj}注\end{CJK}:\begin{CJK}{UTF8}{mj}学科教学论考生不必做第\end{CJK} 7 \begin{CJK}{UTF8}{mj}题\end{CJK}, \begin{CJK}{UTF8}{mj}其他学科考生不必做第\end{CJK} 6 \begin{CJK}{UTF8}{mj}题\end{CJK}。

\begin{enumerate}
  \setcounter{enumi}{6}
  \item (20 \begin{CJK}{UTF8}{mj}分\end{CJK}) \begin{CJK}{UTF8}{mj}设\end{CJK} $V$ \begin{CJK}{UTF8}{mj}是数域\end{CJK} $\mathbb{F}$ \begin{CJK}{UTF8}{mj}上的一线性空间\end{CJK}, $\mathscr{A}$ \begin{CJK}{UTF8}{mj}是\end{CJK} $V$ \begin{CJK}{UTF8}{mj}上一线性变换\end{CJK}, $\mathbb{F}[x]$ \begin{CJK}{UTF8}{mj}是数域\end{CJK} $\mathbb{F}$ \begin{CJK}{UTF8}{mj}上一元多项式\end{CJK} \begin{CJK}{UTF8}{mj}的集合\end{CJK}.
\end{enumerate}
\begin{CJK}{UTF8}{mj}证明\end{CJK}: \begin{CJK}{UTF8}{mj}设\end{CJK} $d(x)$ \begin{CJK}{UTF8}{mj}是\end{CJK} $f(x), g(x)$ \begin{CJK}{UTF8}{mj}的最大公因式\end{CJK}, $f(x), g(x) \in F[x]$, \begin{CJK}{UTF8}{mj}则\end{CJK}
$$
\operatorname{Ker} d(\mathscr{A})=\operatorname{Ker} f(\mathscr{A}) \cap \operatorname{Ker}(\mathscr{A})
$$

\begin{enumerate}
  \setcounter{enumi}{7}
  \item (20 \begin{CJK}{UTF8}{mj}分\end{CJK}) \begin{CJK}{UTF8}{mj}设\end{CJK} $n$ \begin{CJK}{UTF8}{mj}维欧式空间\end{CJK} $V$ \begin{CJK}{UTF8}{mj}的线性变换\end{CJK} $\mathscr{A}$ \begin{CJK}{UTF8}{mj}满足\end{CJK} $\mathscr{A}^{3}+\mathscr{A}=0$.
\end{enumerate}
\begin{CJK}{UTF8}{mj}证明\end{CJK}: $\mathscr{A}$ \begin{CJK}{UTF8}{mj}的迹\end{CJK} (\begin{CJK}{UTF8}{mj}即\end{CJK} $\mathscr{A}$ \begin{CJK}{UTF8}{mj}在\end{CJK} $V$ \begin{CJK}{UTF8}{mj}的某一基下对应的矩阵的迹数\end{CJK}) \begin{CJK}{UTF8}{mj}等于零\end{CJK}.

\section{$2.52000$ 年}
\begin{enumerate}
  \item (15 \begin{CJK}{UTF8}{mj}分\end{CJK}) \begin{CJK}{UTF8}{mj}已知下列非齐次线性方程组\end{CJK} $(I),(I I)$\\
(I) $\left\{\begin{array}{l}x_{1}+x_{2}-2 x_{4}=-6, \\ 4 x_{1}-x_{2}-x_{3}-x_{4}=1, \\ 3 x_{1}-x_{2}-x_{3}=3\end{array}\right.$\\
$(I I)\left\{\begin{array}{l}x_{1}+m x_{2}-x_{3}-x_{4}=-5 \\ n x_{2}-x_{3}-2 x_{4}=-11 \\ x_{3}-2 x_{4}=-t+1\end{array}\right.$
\end{enumerate}
(1) \begin{CJK}{UTF8}{mj}解方程组\end{CJK} (I), \begin{CJK}{UTF8}{mj}用其导出组的基础解系表示通解\end{CJK}.

(2) \begin{CJK}{UTF8}{mj}当方程组\end{CJK} $(I I)$ \begin{CJK}{UTF8}{mj}中的参数\end{CJK} $m, n, t$ \begin{CJK}{UTF8}{mj}为何值时\end{CJK}, \begin{CJK}{UTF8}{mj}方程组\end{CJK} $(I)$ \begin{CJK}{UTF8}{mj}与\end{CJK} $(I I)$ \begin{CJK}{UTF8}{mj}同解\end{CJK}.

\begin{enumerate}
  \setcounter{enumi}{2}
  \item (15 \begin{CJK}{UTF8}{mj}分\end{CJK}) \begin{CJK}{UTF8}{mj}设\end{CJK} $n$ \begin{CJK}{UTF8}{mj}阶\end{CJK} $A, B$ \begin{CJK}{UTF8}{mj}方程满足条件\end{CJK} $A+B=A B$.
\end{enumerate}
(1) \begin{CJK}{UTF8}{mj}证明\end{CJK}: $A-E$ \begin{CJK}{UTF8}{mj}为可逆矩阵\end{CJK};

(2) \begin{CJK}{UTF8}{mj}证明\end{CJK}: $A B=B A$;

(3) \begin{CJK}{UTF8}{mj}已知\end{CJK} $B=\left(\begin{array}{ccc}1 & -3 & 0 \\ 2 & 1 & 0 \\ 0 & 0 & 2\end{array}\right)$, \begin{CJK}{UTF8}{mj}求\end{CJK} $A$.

\begin{enumerate}
  \setcounter{enumi}{3}
  \item (15 \begin{CJK}{UTF8}{mj}分\end{CJK}) \begin{CJK}{UTF8}{mj}设\end{CJK} $\alpha=\left(a_{1}, a_{2}, \cdots, a_{n}\right), \beta=\left(b_{1}, b_{2}, \cdots, b_{n}\right)$ \begin{CJK}{UTF8}{mj}都是非零向量\end{CJK}, \begin{CJK}{UTF8}{mj}且满足\end{CJK} $\alpha \beta^{T}=0$. \begin{CJK}{UTF8}{mj}令\end{CJK} $A=\alpha^{T} \beta$.
\end{enumerate}
(1) \begin{CJK}{UTF8}{mj}求\end{CJK} $A^{2}$;

(2) \begin{CJK}{UTF8}{mj}求矩阵\end{CJK} $A$ \begin{CJK}{UTF8}{mj}的特征值与特征向量\end{CJK};

(3) \begin{CJK}{UTF8}{mj}说明\end{CJK} $A$ \begin{CJK}{UTF8}{mj}是否与对角矩阵相似\end{CJK}.

\begin{enumerate}
  \setcounter{enumi}{4}
  \item (15 \begin{CJK}{UTF8}{mj}分\end{CJK}) \begin{CJK}{UTF8}{mj}求一可逆线性替换\end{CJK}, \begin{CJK}{UTF8}{mj}把\end{CJK}
\end{enumerate}
$$
2 x_{1}^{2}-2 x_{1} x_{2}+5 x_{2}^{2}-4 x_{1} x_{3}+4 x_{3}^{2}
$$
\begin{CJK}{UTF8}{mj}与\end{CJK}
$$
\frac{3}{2} x_{1}^{2}-2 x_{1} x_{3}+3 x_{2}^{2}-4 x_{2} x_{3}+2 x_{3}^{2}
$$
\begin{CJK}{UTF8}{mj}同时化成标准型\end{CJK}.

\begin{enumerate}
  \setcounter{enumi}{5}
  \item (10 \begin{CJK}{UTF8}{mj}分\end{CJK}) \begin{CJK}{UTF8}{mj}试证\end{CJK}: \begin{CJK}{UTF8}{mj}设\end{CJK} $f(x)$ \begin{CJK}{UTF8}{mj}为整系数多项式\end{CJK}, \begin{CJK}{UTF8}{mj}且\end{CJK} $f(1)=f(2)=f(3)=p(p$ \begin{CJK}{UTF8}{mj}为素数\end{CJK}), \begin{CJK}{UTF8}{mj}则不存在整数\end{CJK} $m$, \begin{CJK}{UTF8}{mj}使\end{CJK} $f(m)=2 p$.

  \item (15 \begin{CJK}{UTF8}{mj}分\end{CJK}) \begin{CJK}{UTF8}{mj}设\end{CJK} $A$ \begin{CJK}{UTF8}{mj}是\end{CJK} $n$ \begin{CJK}{UTF8}{mj}阶半正定矩阵\end{CJK}, $B$ \begin{CJK}{UTF8}{mj}为\end{CJK} $n$ \begin{CJK}{UTF8}{mj}阶正定矩阵\end{CJK}, \begin{CJK}{UTF8}{mj}证明\end{CJK}: $|A+B| \geqslant|B|$, \begin{CJK}{UTF8}{mj}等号成立\end{CJK} $\Leftrightarrow A=0$.

  \item (15 \begin{CJK}{UTF8}{mj}分\end{CJK}) \begin{CJK}{UTF8}{mj}设\end{CJK} $\mathscr{A}$ \begin{CJK}{UTF8}{mj}是线性空间\end{CJK} $V$ \begin{CJK}{UTF8}{mj}上的线性变换\end{CJK}. \begin{CJK}{UTF8}{mj}证明\end{CJK}:

\end{enumerate}
$$
\operatorname{rank}\left(\mathscr{A}^{2}\right)=\operatorname{rank}(\mathscr{A}) \Leftrightarrow \mathscr{A} V \cap \mathscr{A}^{-1}(0)=\{0\} .
$$

\section{$2.62001$ 年}
\begin{enumerate}
  \item (15 \begin{CJK}{UTF8}{mj}分\end{CJK}) \begin{CJK}{UTF8}{mj}计算\end{CJK} $n$ \begin{CJK}{UTF8}{mj}阶行列式\end{CJK}
\end{enumerate}
$$
D_{n}=\left|\begin{array}{cccccc}
0 & a_{2} & a_{3} & \cdots & a_{n-1} & a_{n} \\
b_{1} & 0 & a_{3} & \cdots & a_{n-1} & a_{n} \\
b_{1} & b_{2} & 0 & \cdots & a_{n-1} & a_{n} \\
\vdots & \vdots & \vdots & & \vdots & \vdots \\
b_{1} & b_{2} & b_{3} & \cdots & 0 & a_{n} \\
b_{1} & b_{2} & b_{3} & \cdots & a_{n-1} & 0
\end{array}\right|
$$

\begin{enumerate}
  \setcounter{enumi}{2}
  \item (15 \begin{CJK}{UTF8}{mj}分\end{CJK}) \begin{CJK}{UTF8}{mj}设\end{CJK} $\beta_{1}=\alpha_{1}+\alpha_{2}, \beta_{2}=\alpha_{2}+\alpha_{3}, \cdots, \beta_{n-1}=\alpha_{n-1}+\alpha_{n}, \beta_{n}=\alpha_{n}+\alpha_{1}$. \begin{CJK}{UTF8}{mj}试证\end{CJK}:
\end{enumerate}
(1) \begin{CJK}{UTF8}{mj}当\end{CJK} $n$ \begin{CJK}{UTF8}{mj}为偶数时\end{CJK}, $\beta_{1}, \beta_{2}, \cdots, \beta_{n}$ \begin{CJK}{UTF8}{mj}线性相关\end{CJK};

(2) \begin{CJK}{UTF8}{mj}当\end{CJK} $n$ \begin{CJK}{UTF8}{mj}为奇数时\end{CJK}, $\beta_{1}, \beta_{2}, \cdots, \beta_{n}$ \begin{CJK}{UTF8}{mj}线性无关\end{CJK} $\Leftrightarrow \alpha_{1}, \alpha_{2}, \cdots, \alpha_{n}$ \begin{CJK}{UTF8}{mj}线性无关\end{CJK}.

\begin{enumerate}
  \setcounter{enumi}{3}
  \item (15 \begin{CJK}{UTF8}{mj}分\end{CJK}) \begin{CJK}{UTF8}{mj}设\end{CJK} $a_{1}, a_{2}, \cdots, a_{n}$ \begin{CJK}{UTF8}{mj}是互异的整数\end{CJK}, \begin{CJK}{UTF8}{mj}求证\end{CJK}:
\end{enumerate}
$$
f(x)=\left(x-a_{1}\right)\left(x-a_{2}\right) \cdots\left(x-a_{n}\right)-1
$$
\begin{CJK}{UTF8}{mj}在有理数域上不可约\end{CJK}.

\begin{enumerate}
  \setcounter{enumi}{4}
  \item (20 \begin{CJK}{UTF8}{mj}分\end{CJK}) \begin{CJK}{UTF8}{mj}设\end{CJK} $A_{1}, A_{2}, A_{3}$ \begin{CJK}{UTF8}{mj}是三个非零得三阶方阵\end{CJK}, \begin{CJK}{UTF8}{mj}且\end{CJK} $A_{i}^{2}=A_{i}(i=1,2,3), A_{i} A_{j}=0(i \neq j)$. \begin{CJK}{UTF8}{mj}证明\end{CJK}:
\end{enumerate}
(1) $A_{i}$ \begin{CJK}{UTF8}{mj}的特征值只有\end{CJK} 1 \begin{CJK}{UTF8}{mj}和\end{CJK} 0 ;

(2) $A_{i}$ \begin{CJK}{UTF8}{mj}属于特征值\end{CJK} 1 \begin{CJK}{UTF8}{mj}的特征向量是\end{CJK} $A_{j}$ \begin{CJK}{UTF8}{mj}属于特征值\end{CJK} 0 \begin{CJK}{UTF8}{mj}的的特征向量\end{CJK};

(3) \begin{CJK}{UTF8}{mj}若\end{CJK} $X_{1}, X_{2}, X_{3}$ \begin{CJK}{UTF8}{mj}是\end{CJK} $A_{1}, A_{2}, A_{3}$ \begin{CJK}{UTF8}{mj}属于特征值\end{CJK} 1 \begin{CJK}{UTF8}{mj}的特征向量\end{CJK}, \begin{CJK}{UTF8}{mj}则\end{CJK} $X_{1}, X_{2}, X_{3}$ \begin{CJK}{UTF8}{mj}线性无关\end{CJK}.

\begin{enumerate}
  \setcounter{enumi}{5}
  \item (15 \begin{CJK}{UTF8}{mj}分\end{CJK}) \begin{CJK}{UTF8}{mj}若存在正整数\end{CJK} $m$, \begin{CJK}{UTF8}{mj}使\end{CJK} $n$ \begin{CJK}{UTF8}{mj}阶方阵\end{CJK} $A$ \begin{CJK}{UTF8}{mj}满足\end{CJK} $A^{m}=E$, \begin{CJK}{UTF8}{mj}则称\end{CJK} $A$ \begin{CJK}{UTF8}{mj}为周期矩阵\end{CJK}. \begin{CJK}{UTF8}{mj}证明\end{CJK}:
\end{enumerate}
(1) \begin{CJK}{UTF8}{mj}复数域上的周期矩阵相似于对角阵\end{CJK}.

(2) $A$ \begin{CJK}{UTF8}{mj}的所有特征根都是\end{CJK} $m$ \begin{CJK}{UTF8}{mj}次单位根\end{CJK}.

\begin{enumerate}
  \setcounter{enumi}{6}
  \item (20 \begin{CJK}{UTF8}{mj}分\end{CJK}) \begin{CJK}{UTF8}{mj}已知\end{CJK} $A$ \begin{CJK}{UTF8}{mj}为\end{CJK} $m \times n$ \begin{CJK}{UTF8}{mj}矩阵\end{CJK}, \begin{CJK}{UTF8}{mj}且\end{CJK} $m<n$. \begin{CJK}{UTF8}{mj}证明\end{CJK}: $A A^{T}$ \begin{CJK}{UTF8}{mj}正定\end{CJK} $\Leftrightarrow \operatorname{rank}(A)=m$.
\end{enumerate}
\section{$2.72002$ 年}
\begin{enumerate}
  \item (15 \begin{CJK}{UTF8}{mj}分\end{CJK}) i\begin{CJK}{UTF8}{mj}十算行列式\end{CJK}
\end{enumerate}
$$
D_{n}=\left|\begin{array}{cccccc}
x & 4 & 4 & 4 & \cdots & 4 \\
1 & x & 2 & 2 & \cdots & 2 \\
1 & 2 & x & 2 & \cdots & 2 \\
1 & 2 & 2 & x & \cdots & 2 \\
\vdots & \vdots & \vdots & \vdots & & \vdots \\
1 & 2 & 2 & 2 & \cdots & x
\end{array}\right| .
$$

\begin{enumerate}
  \setcounter{enumi}{2}
  \item (15 \begin{CJK}{UTF8}{mj}分\end{CJK}) \begin{CJK}{UTF8}{mj}设\end{CJK} $2 n$ \begin{CJK}{UTF8}{mj}阶实对称矩阵\end{CJK}
\end{enumerate}
$$
A=\left(\begin{array}{ccccc}
0 & 0 & \cdots & 0 & 1 \\
0 & 0 & \cdots & 1 & 0 \\
\vdots & \vdots & . & \vdots & \vdots \\
0 & 1 & \cdots & 0 & 0 \\
1 & 0 & \cdots & 0 & 0
\end{array}\right)
$$
\begin{CJK}{UTF8}{mj}试求正交矩阵\end{CJK} $T$, \begin{CJK}{UTF8}{mj}使\end{CJK} $T^{T} A T=T^{-1} A T=B$ \begin{CJK}{UTF8}{mj}为对角形矩阵\end{CJK}, \begin{CJK}{UTF8}{mj}并求矩阵\end{CJK} $B$.

\begin{enumerate}
  \setcounter{enumi}{3}
  \item (20 \begin{CJK}{UTF8}{mj}分\end{CJK}) \begin{CJK}{UTF8}{mj}设\end{CJK} $\mathscr{A}$ \begin{CJK}{UTF8}{mj}为数域\end{CJK} $K$ \begin{CJK}{UTF8}{mj}上\end{CJK} $n$ \begin{CJK}{UTF8}{mj}维线性空间\end{CJK} $V$ \begin{CJK}{UTF8}{mj}的一个线性变换\end{CJK}, \begin{CJK}{UTF8}{mj}满足\end{CJK} $\mathscr{A}^{2}=\mathscr{A}, A$ \begin{CJK}{UTF8}{mj}为\end{CJK} $\mathscr{A}$ \begin{CJK}{UTF8}{mj}在\end{CJK} $V$ \begin{CJK}{UTF8}{mj}的\end{CJK} \begin{CJK}{UTF8}{mj}某组貹下的矩阵\end{CJK}, $\operatorname{rank}(A)=r$.
\end{enumerate}
(1) \begin{CJK}{UTF8}{mj}证明\end{CJK}: $\mathscr{A}+\mathscr{E}$ \begin{CJK}{UTF8}{mj}为\end{CJK} $V$ \begin{CJK}{UTF8}{mj}的可逆线性变换\end{CJK};

(2) \begin{CJK}{UTF8}{mj}证明\end{CJK}: $\operatorname{rank}(A)=\operatorname{Tr}(A)$;

(3) \begin{CJK}{UTF8}{mj}试求\end{CJK} $|2 E-A|$.

\begin{enumerate}
  \setcounter{enumi}{4}
  \item (20 \begin{CJK}{UTF8}{mj}分\end{CJK}) \begin{CJK}{UTF8}{mj}设\end{CJK} $B$ \begin{CJK}{UTF8}{mj}是\end{CJK} $n$ \begin{CJK}{UTF8}{mj}阶正定矩阵\end{CJK}, $C$ \begin{CJK}{UTF8}{mj}为秩为\end{CJK} $m$ \begin{CJK}{UTF8}{mj}的\end{CJK} $n \times m$ \begin{CJK}{UTF8}{mj}阶实矩阵\end{CJK}, $n>m$. \begin{CJK}{UTF8}{mj}令\end{CJK}
\end{enumerate}
$$
A=\left(\begin{array}{cc}
B & C \\
C^{T} & 0
\end{array}\right)
$$
\begin{CJK}{UTF8}{mj}证明\end{CJK}: $A$ \begin{CJK}{UTF8}{mj}有\end{CJK} $n$ \begin{CJK}{UTF8}{mj}个正的特征值\end{CJK}, \begin{CJK}{UTF8}{mj}有\end{CJK} $m$ \begin{CJK}{UTF8}{mj}个负的特征值\end{CJK}.

\begin{enumerate}
  \setcounter{enumi}{5}
  \item (15 \begin{CJK}{UTF8}{mj}分\end{CJK}) \begin{CJK}{UTF8}{mj}设\end{CJK} $f(x)$ \begin{CJK}{UTF8}{mj}为实系数多项式\end{CJK}. \begin{CJK}{UTF8}{mj}证明\end{CJK}: \begin{CJK}{UTF8}{mj}如果对任意的\end{CJK} $c$ \begin{CJK}{UTF8}{mj}都有\end{CJK} $f(c) \geqslant 0$, \begin{CJK}{UTF8}{mj}则存在实系数多项\end{CJK} \begin{CJK}{UTF8}{mj}式\end{CJK} $g(x), h(x)$, \begin{CJK}{UTF8}{mj}使\end{CJK}
\end{enumerate}
$$
f(x)=g^{2}(x)+h^{2}(x)
$$

\begin{enumerate}
  \setcounter{enumi}{6}
  \item (15 \begin{CJK}{UTF8}{mj}分\end{CJK}) \begin{CJK}{UTF8}{mj}设\end{CJK} $A, B$ \begin{CJK}{UTF8}{mj}都是\end{CJK} $n$ \begin{CJK}{UTF8}{mj}阶方阵\end{CJK}, $\operatorname{rank}(A)=n-1$. \begin{CJK}{UTF8}{mj}证明\end{CJK}: \begin{CJK}{UTF8}{mj}如果\end{CJK} $A B=B A=0$, \begin{CJK}{UTF8}{mj}则存在多项式\end{CJK} $g(x)$, \begin{CJK}{UTF8}{mj}使\end{CJK} $B=g(A)$.
\end{enumerate}
\section{$2.82003$ 年}
\section{1. 填空、是非、选择题:(每小题 4 分, 共 60 分)}
(1) \begin{CJK}{UTF8}{mj}如果排列\end{CJK} $j_{1} j_{2} j_{3} j_{4} j_{5}$ \begin{CJK}{UTF8}{mj}的逆序数为\end{CJK} 4 , \begin{CJK}{UTF8}{mj}则排列\end{CJK} $j_{5} j_{4} j_{3} j_{2} j_{1}$ \begin{CJK}{UTF8}{mj}的逆序数等于\end{CJK}

(2) \begin{CJK}{UTF8}{mj}设\end{CJK} $A$ \begin{CJK}{UTF8}{mj}是\end{CJK} 3 \begin{CJK}{UTF8}{mj}阶方阵\end{CJK}. \begin{CJK}{UTF8}{mj}如果\end{CJK} $|-2 A|=56$, \begin{CJK}{UTF8}{mj}则\end{CJK} $|A|=$

(3) \begin{CJK}{UTF8}{mj}多贡式\end{CJK}
$$
f(x)=\left|\begin{array}{cccc}
2 & -3 & 3 & x \\
4 & 9 & 9 & x^{2} \\
8 & -27 & 27 & x^{3} \\
16 & 81 & 81 & x^{4}
\end{array}\right|
$$
\begin{CJK}{UTF8}{mj}的全部根为\end{CJK}

(4) \begin{CJK}{UTF8}{mj}设\end{CJK} $A=B+C$, \begin{CJK}{UTF8}{mj}其中\end{CJK}
$$
\left(\begin{array}{ccc}
1 & 11 & -3 \\
7 & 7 & 3 \\
11 & -3 & 8
\end{array}\right)
$$
$B$ \begin{CJK}{UTF8}{mj}为对称矩阵\end{CJK}, $C$ \begin{CJK}{UTF8}{mj}为反对称矩阵\end{CJK}, \begin{CJK}{UTF8}{mj}那么\end{CJK} $B=$

(5) \begin{CJK}{UTF8}{mj}设\end{CJK} $r$ \begin{CJK}{UTF8}{mj}与\end{CJK} $s$ \begin{CJK}{UTF8}{mj}分别是某线性方程组系数矩阵和增广矩阵的秩\end{CJK}, \begin{CJK}{UTF8}{mj}若该方程组无解\end{CJK}, \begin{CJK}{UTF8}{mj}则\end{CJK}\\
(A) $s=r-1$;\\
(B) $s=r$;\\
(C) $s=r+1$;\\
(D) $r$ \begin{CJK}{UTF8}{mj}与\end{CJK} $s$ \begin{CJK}{UTF8}{mj}的关系不确定\end{CJK}.

(6) \begin{CJK}{UTF8}{mj}以下命题中错误的是\end{CJK}\\
(A) \begin{CJK}{UTF8}{mj}包今零向量的向量组必线性相关\end{CJK};\\
(B) \begin{CJK}{UTF8}{mj}如果向量组\end{CJK} $\alpha_{1}, \cdots, \alpha_{r}$ \begin{CJK}{UTF8}{mj}线性无关\end{CJK}, \begin{CJK}{UTF8}{mj}且\end{CJK} $\alpha_{1}, \cdots, \alpha_{r}, \beta$ \begin{CJK}{UTF8}{mj}线性相关\end{CJK}, \begin{CJK}{UTF8}{mj}则\end{CJK} $\beta$ \begin{CJK}{UTF8}{mj}可表为\end{CJK} $\alpha_{1}, \cdots, \alpha_{r}$ \begin{CJK}{UTF8}{mj}的线\end{CJK} \begin{CJK}{UTF8}{mj}性组合\end{CJK};

(C) \begin{CJK}{UTF8}{mj}如果向量组\end{CJK} $\alpha_{1}, \cdots, \alpha_{r}, \alpha_{r+1}$ \begin{CJK}{UTF8}{mj}线性无关\end{CJK}, \begin{CJK}{UTF8}{mj}则\end{CJK} $\alpha_{1}, \cdots, \alpha_{r}$ \begin{CJK}{UTF8}{mj}也线性无关\end{CJK};

(D) \begin{CJK}{UTF8}{mj}如果向量组\end{CJK} $\alpha_{1}, \cdots, \alpha_{r}$ \begin{CJK}{UTF8}{mj}线性相关\end{CJK}, \begin{CJK}{UTF8}{mj}则每个\end{CJK} $\alpha_{i}$ \begin{CJK}{UTF8}{mj}均可由其余向量线性表出\end{CJK}.

(7) \begin{CJK}{UTF8}{mj}设向量组\end{CJK} $\alpha_{1}, \alpha_{2}, \alpha_{3}, \alpha_{4}, \alpha_{5}, \alpha_{6}, \alpha_{7}$ \begin{CJK}{UTF8}{mj}的秩为\end{CJK} 5 , \begin{CJK}{UTF8}{mj}其部分组\end{CJK} $\alpha_{2}, \alpha_{3}, \alpha_{4}, \alpha_{5}, \alpha_{6}$ \begin{CJK}{UTF8}{mj}的秩为\end{CJK} $t$, \begin{CJK}{UTF8}{mj}则\end{CJK} $t$ \begin{CJK}{UTF8}{mj}不可能\end{CJK} \begin{CJK}{UTF8}{mj}取到下面那个数\end{CJK}\\
(A) 2 ;\\
(B) 3 ;\\
(C) 4 ;\\
(D) 5 .

(8) \begin{CJK}{UTF8}{mj}以下向量组中线性无关的是\end{CJK}\\
(A) $\alpha_{1}=(1,2,3,4), \alpha_{2}=(4,3,2,1), \alpha_{3}=(1,1,1,1)$;\\
(B) $\alpha_{1}=(2,-2,0,0), \alpha_{2}=(0,1,-1,0), \alpha_{3}=(0,0,3,-3)$;\\
(C) $\alpha_{1}=(1,2,4), \alpha_{2}=(1,3,9), \alpha_{3}=(1,4,16), \alpha_{4}=(1,5,25)$;\\
(D) $\alpha_{1}=(1,-1,1,-1), \alpha_{2}=(0,0,0,0), \alpha_{3}=(-1,2,-3,4)$.

(9) \begin{CJK}{UTF8}{mj}如果\end{CJK} $U$ \begin{CJK}{UTF8}{mj}和\end{CJK} $W$ \begin{CJK}{UTF8}{mj}是线性空间\end{CJK} $V$ \begin{CJK}{UTF8}{mj}的维数相等的子空间\end{CJK}, \begin{CJK}{UTF8}{mj}且子空间\end{CJK} $U+W$ \begin{CJK}{UTF8}{mj}和\end{CJK} $U \cap W$ \begin{CJK}{UTF8}{mj}的维数分别\end{CJK} \begin{CJK}{UTF8}{mj}是\end{CJK} 8 \begin{CJK}{UTF8}{mj}和\end{CJK} 2 , \begin{CJK}{UTF8}{mj}则\end{CJK} $U$ \begin{CJK}{UTF8}{mj}的维数等于\end{CJK} (10) \begin{CJK}{UTF8}{mj}已知\end{CJK}
$$
A^{-1}=\left(\begin{array}{ccc}
2 & -1 & 0 \\
-1 & 2 & -1 \\
0 & -2 & 2
\end{array}\right)
$$
\begin{CJK}{UTF8}{mj}则\end{CJK} $\left(A^{T}\right)^{-1}=$

(11) \begin{CJK}{UTF8}{mj}已知\end{CJK} 3 \begin{CJK}{UTF8}{mj}阶矩阵\end{CJK} $A$ \begin{CJK}{UTF8}{mj}的特征值为\end{CJK} $4,5,6$, \begin{CJK}{UTF8}{mj}则\end{CJK} $|A|=$

(12) \begin{CJK}{UTF8}{mj}实数域上\end{CJK} 3 \begin{CJK}{UTF8}{mj}阶复反对称矩阵关于矩阵通常的加法和乘法构成线性空间\end{CJK}, \begin{CJK}{UTF8}{mj}则此线性空间的\end{CJK} \begin{CJK}{UTF8}{mj}维数\end{CJK}

(13) \begin{CJK}{UTF8}{mj}设\end{CJK} $A, B, C, D$ \begin{CJK}{UTF8}{mj}是同阶方阵\end{CJK}, \begin{CJK}{UTF8}{mj}且\end{CJK} $A$ \begin{CJK}{UTF8}{mj}与\end{CJK} $B$ \begin{CJK}{UTF8}{mj}相似\end{CJK}, $C$ \begin{CJK}{UTF8}{mj}与\end{CJK} $D$ \begin{CJK}{UTF8}{mj}相似\end{CJK}, \begin{CJK}{UTF8}{mj}则\end{CJK} $A+C$ \begin{CJK}{UTF8}{mj}与\end{CJK} $B+D$ \begin{CJK}{UTF8}{mj}相似\end{CJK}.

(14) \begin{CJK}{UTF8}{mj}如果欧式空间上的线性变换\end{CJK} $\mathscr{A}$ \begin{CJK}{UTF8}{mj}将每个正交基映成正交基\end{CJK},\begin{CJK}{UTF8}{mj}则\end{CJK} $\mathscr{A}$ \begin{CJK}{UTF8}{mj}是正交变换\end{CJK}.

(15) \begin{CJK}{UTF8}{mj}设\end{CJK} $A$ \begin{CJK}{UTF8}{mj}是\end{CJK} $n$ \begin{CJK}{UTF8}{mj}阶实方阵\end{CJK}, \begin{CJK}{UTF8}{mj}则\end{CJK} $A$ \begin{CJK}{UTF8}{mj}与\end{CJK} $A^{T} A$ \begin{CJK}{UTF8}{mj}有相同的秩\end{CJK}.

\begin{enumerate}
  \setcounter{enumi}{2}
  \item (12 \begin{CJK}{UTF8}{mj}分\end{CJK}) \begin{CJK}{UTF8}{mj}计算\end{CJK} $n$ \begin{CJK}{UTF8}{mj}阶行列式\end{CJK}:
\end{enumerate}
$$
D_{n}=\left|\begin{array}{ccccc}
1 & 2 & 3 & \cdots & n \\
x & 1 & 2 & \cdots & n-1 \\
x & x & 2 & \cdots & n-2 \\
\vdots & \vdots & \vdots & \ddots & \vdots \\
x & x & x & \cdots & 1
\end{array}\right|
$$

\begin{enumerate}
  \setcounter{enumi}{3}
  \item (12 \begin{CJK}{UTF8}{mj}分\end{CJK}) \begin{CJK}{UTF8}{mj}已知矩阵\end{CJK} $A$ \begin{CJK}{UTF8}{mj}的特征多项式\end{CJK} $f(\lambda)=\lambda^{3}-2 \lambda^{2}-\lambda+2$, \begin{CJK}{UTF8}{mj}试求\end{CJK}: \begin{CJK}{UTF8}{mj}矩阵\end{CJK} $A^{3}$ \begin{CJK}{UTF8}{mj}的特征多项式\end{CJK} $g(\lambda)$.

  \item (12 \begin{CJK}{UTF8}{mj}分\end{CJK}) \begin{CJK}{UTF8}{mj}设矩阵\end{CJK}

\end{enumerate}
$$
A=\left(\begin{array}{ccc}
3 & 0 & 8 \\
3 & -1 & 6 \\
-2 & 0 & -5
\end{array}\right)
$$
\begin{CJK}{UTF8}{mj}求\end{CJK} (1) $A$ \begin{CJK}{UTF8}{mj}的不变因子\end{CJK}; (2) $A$ \begin{CJK}{UTF8}{mj}的初等因子\end{CJK}; (3) $A$ \begin{CJK}{UTF8}{mj}的\end{CJK} Jordan \begin{CJK}{UTF8}{mj}标准型\end{CJK}.

\begin{enumerate}
  \setcounter{enumi}{5}
  \item (14 \begin{CJK}{UTF8}{mj}分\end{CJK}) \begin{CJK}{UTF8}{mj}已知矩阵\end{CJK}
\end{enumerate}
$$
A=\left(\begin{array}{cccc}
-1 & -3 & 3 & -3 \\
-3 & -1 & -3 & 3 \\
3 & -3 & -1 & -3 \\
-3 & 3 & -3 & -1
\end{array}\right)
$$
\begin{CJK}{UTF8}{mj}试求正交矩阵\end{CJK} $T$, \begin{CJK}{UTF8}{mj}使\end{CJK} $T^{T} A T$ \begin{CJK}{UTF8}{mj}为对角形矩阵\end{CJK}.

\begin{enumerate}
  \setcounter{enumi}{6}
  \item (15 \begin{CJK}{UTF8}{mj}分\end{CJK}) \begin{CJK}{UTF8}{mj}设\end{CJK} $f(x), g(x)$ \begin{CJK}{UTF8}{mj}是数域\end{CJK} $K$ \begin{CJK}{UTF8}{mj}上互素的多项式\end{CJK}, $H$ \begin{CJK}{UTF8}{mj}为\end{CJK} $K$ \begin{CJK}{UTF8}{mj}上的\end{CJK} $n$ \begin{CJK}{UTF8}{mj}阶矩阵\end{CJK}, $A=f(H), B=g(H)$. \begin{CJK}{UTF8}{mj}证明\end{CJK}: \begin{CJK}{UTF8}{mj}方程\end{CJK} $A B X=0$ \begin{CJK}{UTF8}{mj}的每一个解均可唯一的表为\end{CJK} $X=Y+Z$ \begin{CJK}{UTF8}{mj}的形式\end{CJK},\begin{CJK}{UTF8}{mj}其中\end{CJK} $Y, Z$ \begin{CJK}{UTF8}{mj}分别为方程\end{CJK} $B Y=0$ \begin{CJK}{UTF8}{mj}与\end{CJK} $A Z=0$ \begin{CJK}{UTF8}{mj}的解\end{CJK}.

  \item (15 \begin{CJK}{UTF8}{mj}分\end{CJK}) \begin{CJK}{UTF8}{mj}设\end{CJK} $A$ \begin{CJK}{UTF8}{mj}是实对称矩阵\end{CJK}. \begin{CJK}{UTF8}{mj}证明\end{CJK}:

\end{enumerate}
(1)\begin{CJK}{UTF8}{mj}当实数\end{CJK} $\lambda$ \begin{CJK}{UTF8}{mj}充分大之后\end{CJK}, $\lambda E+A$ \begin{CJK}{UTF8}{mj}是正定的\end{CJK};

(2) $A$ \begin{CJK}{UTF8}{mj}是半正定的\end{CJK} $\Leftrightarrow$ \begin{CJK}{UTF8}{mj}对任意的正实数\end{CJK} $\lambda, \lambda E+A$ \begin{CJK}{UTF8}{mj}都是正定的\end{CJK}.

\begin{enumerate}
  \setcounter{enumi}{8}
  \item (10 \begin{CJK}{UTF8}{mj}分\end{CJK}) \begin{CJK}{UTF8}{mj}证明\end{CJK}: \begin{CJK}{UTF8}{mj}特征值全为实数的正交矩阵是对称矩阵\end{CJK}.
\end{enumerate}
\section{$2.92004$ 年}
\section{1. 填空、是非、选择题:(每小题 4 分, 共 60 分)}
(1) \begin{CJK}{UTF8}{mj}设\end{CJK} $K$ \begin{CJK}{UTF8}{mj}是实数\end{CJK}, $T$ \begin{CJK}{UTF8}{mj}是正交矩阵\end{CJK}, \begin{CJK}{UTF8}{mj}若\end{CJK} $k T$ \begin{CJK}{UTF8}{mj}也是正交矩阵\end{CJK}, \begin{CJK}{UTF8}{mj}则\end{CJK} $k=$

(2) \begin{CJK}{UTF8}{mj}实对称矩阵\end{CJK} $A$ \begin{CJK}{UTF8}{mj}正定的充分必要条件是\end{CJK}\\
(A) $A^{-1}$ \begin{CJK}{UTF8}{mj}正定\end{CJK};\\
(B) $A$ \begin{CJK}{UTF8}{mj}的特征值都非负\end{CJK};\\
(C) $A$ \begin{CJK}{UTF8}{mj}的秩为\end{CJK} $n$;\\
(D) $A$ \begin{CJK}{UTF8}{mj}的所有\end{CJK} $k$ \begin{CJK}{UTF8}{mj}阶子式都大于\end{CJK} 0 .

(3) \begin{CJK}{UTF8}{mj}设\end{CJK} $A$ \begin{CJK}{UTF8}{mj}为\end{CJK} $n$ \begin{CJK}{UTF8}{mj}阶可逆矩阵\end{CJK}, $\lambda$ \begin{CJK}{UTF8}{mj}为\end{CJK} $A$ \begin{CJK}{UTF8}{mj}的特征值\end{CJK}, \begin{CJK}{UTF8}{mj}则\end{CJK} \begin{CJK}{UTF8}{mj}必为\end{CJK} $A$ \begin{CJK}{UTF8}{mj}的伴随矩阵\end{CJK} $A^{*}$ \begin{CJK}{UTF8}{mj}的特征值\end{CJK}.\\
(A) $\lambda^{-1}|A|^{n}$;\\
(B) $\lambda^{-1}|A|$;\\
(C) $\lambda|A|^{n}$\\
(D) $\lambda|A|$.

(4) \begin{CJK}{UTF8}{mj}设\end{CJK} $V=\left\{A \in M_{2}(\mathbb{R}) \mid \operatorname{Tr}(A)=0\right\}$ \begin{CJK}{UTF8}{mj}是关于矩阵的加法和数乘构成的实线性空间\end{CJK}, \begin{CJK}{UTF8}{mj}则\end{CJK} $\operatorname{dim} V=$

(5) \begin{CJK}{UTF8}{mj}设\end{CJK} 8 \begin{CJK}{UTF8}{mj}无非齐次线性方程组的系数矩阵\end{CJK} $A$ \begin{CJK}{UTF8}{mj}的秩等于\end{CJK} $3, \alpha_{1}, \alpha_{2}, \alpha_{s}$ \begin{CJK}{UTF8}{mj}是该方程组的线性无关的解\end{CJK} \begin{CJK}{UTF8}{mj}向量\end{CJK}, \begin{CJK}{UTF8}{mj}则\end{CJK} $s$ \begin{CJK}{UTF8}{mj}的最大值\end{CJK}\\
$(\mathrm{A})<5$\\
(B) $=5$\\
$(C)=6$\\
$(\mathrm{D})>6$

(6) 5 \begin{CJK}{UTF8}{mj}阶实对称矩阵的集合关于相合\end{CJK} (\begin{CJK}{UTF8}{mj}合同\end{CJK}) \begin{CJK}{UTF8}{mj}这一等价关系可分为\end{CJK}

(7) \begin{CJK}{UTF8}{mj}设\end{CJK} $A$ \begin{CJK}{UTF8}{mj}为\end{CJK} 3 \begin{CJK}{UTF8}{mj}阶矩阵\end{CJK}, \begin{CJK}{UTF8}{mj}且\end{CJK} $|A|=-\frac{1}{2}$, \begin{CJK}{UTF8}{mj}则\end{CJK} $\left|A^{-1}-2 A^{*}\right|=$

(8) \begin{CJK}{UTF8}{mj}若向量\end{CJK} $\beta$ \begin{CJK}{UTF8}{mj}可由向量组\end{CJK} $\alpha_{1}, \alpha_{2}, \cdots, \alpha_{s}$ \begin{CJK}{UTF8}{mj}线性表出\end{CJK}, \begin{CJK}{UTF8}{mj}则\end{CJK}\\
(A) \begin{CJK}{UTF8}{mj}存在一组不全为零的数\end{CJK} $k_{1}, k_{2}, \cdots, k_{s}$, \begin{CJK}{UTF8}{mj}使得\end{CJK} $\beta=k_{1} \alpha_{1}+k_{2} \alpha_{2}+\cdots+k_{s} \alpha_{s}$ ;\\
(B) \begin{CJK}{UTF8}{mj}存在一组全为零的数\end{CJK} $k_{1}, k_{2}, \cdots, k_{s}$, \begin{CJK}{UTF8}{mj}使得\end{CJK} $\beta=k_{1} \alpha_{1}+k_{2} \alpha_{2}+\cdots+k_{s} \alpha_{s}$;\\
(C) \begin{CJK}{UTF8}{mj}每个\end{CJK} $\alpha_{i}$ \begin{CJK}{UTF8}{mj}均可由\end{CJK} $\alpha_{1}, \cdots, \alpha_{i-1}, \alpha_{i+1}, \cdots, \alpha_{s}, \beta$ \begin{CJK}{UTF8}{mj}线性表出\end{CJK};\\
(D) \begin{CJK}{UTF8}{mj}向量组\end{CJK} $\alpha_{1}, \alpha_{2}, \cdots, \alpha_{s}, \beta$ \begin{CJK}{UTF8}{mj}线性相关\end{CJK}.

(9) \begin{CJK}{UTF8}{mj}设\end{CJK} $A$ \begin{CJK}{UTF8}{mj}是\end{CJK} $n$ \begin{CJK}{UTF8}{mj}阶矩阵\end{CJK}, \begin{CJK}{UTF8}{mj}则存在非零\end{CJK} $n \times m$ \begin{CJK}{UTF8}{mj}矩阵\end{CJK} $B$, \begin{CJK}{UTF8}{mj}使\end{CJK} $A B=0 \Leftrightarrow A$ \begin{CJK}{UTF8}{mj}的秩\end{CJK}

(10) \begin{CJK}{UTF8}{mj}已知\end{CJK} 3 \begin{CJK}{UTF8}{mj}阶矩阵\end{CJK} $A$ \begin{CJK}{UTF8}{mj}的\end{CJK} 3 \begin{CJK}{UTF8}{mj}个特征值为\end{CJK} $2,3,4$, \begin{CJK}{UTF8}{mj}则\end{CJK} $-A^{T}$ \begin{CJK}{UTF8}{mj}的特征多项式\end{CJK} $f(\lambda)=$

(11) \begin{CJK}{UTF8}{mj}已知\end{CJK} $A^{-1}=\left(\begin{array}{ccc}2 & -1 & 0 \\ -1 & 2 & -1 \\ 0 & -1 & 2\end{array}\right)$, \begin{CJK}{UTF8}{mj}则\end{CJK} $(2 A)^{-1}=$

(12) \begin{CJK}{UTF8}{mj}如果矩阵\end{CJK} $A$ \begin{CJK}{UTF8}{mj}与\end{CJK} $B$ \begin{CJK}{UTF8}{mj}相似\end{CJK}, \begin{CJK}{UTF8}{mj}则\end{CJK} $2 A$ \begin{CJK}{UTF8}{mj}与\end{CJK} $3 B$ \begin{CJK}{UTF8}{mj}等价\end{CJK}.

(13) \begin{CJK}{UTF8}{mj}如果\end{CJK} $A$ \begin{CJK}{UTF8}{mj}是\end{CJK} $n$ \begin{CJK}{UTF8}{mj}阶复矩阵\end{CJK}, \begin{CJK}{UTF8}{mj}则\end{CJK} $\operatorname{rank}(A)=\operatorname{rank}\left(A A^{T}\right)$.

(14) \begin{CJK}{UTF8}{mj}如果欧式空间\end{CJK} $V$ \begin{CJK}{UTF8}{mj}上的线性变换\end{CJK} $\mathscr{A}$ \begin{CJK}{UTF8}{mj}在\end{CJK} $V$ \begin{CJK}{UTF8}{mj}的任意标准正交基下的矩阵是对称矩阵\end{CJK},\begin{CJK}{UTF8}{mj}则\end{CJK} $\mathscr{A}$ \begin{CJK}{UTF8}{mj}是\end{CJK} \begin{CJK}{UTF8}{mj}对称变换\end{CJK}.

(15) \begin{CJK}{UTF8}{mj}设\end{CJK} $W_{1}, W_{2}, \cdots, W_{m}$ \begin{CJK}{UTF8}{mj}是线性空间\end{CJK} $V$ \begin{CJK}{UTF8}{mj}的子空间\end{CJK}, \begin{CJK}{UTF8}{mj}且\end{CJK} $\operatorname{dim}\left(W_{1}+W_{2}+\cdots+W_{m}\right)=\operatorname{dim} W_{1}+\operatorname{dim} W_{2}+$ $\cdots+\operatorname{dim} W_{m}$, \begin{CJK}{UTF8}{mj}则\end{CJK} $W_{1}+W_{2}+\cdots+W_{m}$ \begin{CJK}{UTF8}{mj}是直和\end{CJK}. 2. (12 \begin{CJK}{UTF8}{mj}分\end{CJK}) \begin{CJK}{UTF8}{mj}计算\end{CJK} $n$ \begin{CJK}{UTF8}{mj}阶行列式\end{CJK}
$$
D_{n}=\left|\begin{array}{ccccc}
1 & 2 & 3 & \cdots & n \\
2 & 1 & 2 & \cdots & n-1 \\
3 & 2 & 1 & \cdots & n-2 \\
\vdots & \vdots & \vdots & \ddots & \vdots \\
n & n-1 & n-2 & \cdots & 1
\end{array}\right| .
$$

\begin{enumerate}
  \setcounter{enumi}{3}
  \item (14 \begin{CJK}{UTF8}{mj}分\end{CJK}) \begin{CJK}{UTF8}{mj}求所有整数\end{CJK} $m$, \begin{CJK}{UTF8}{mj}使得\end{CJK} $x^{4}-m x^{2}+1$ \begin{CJK}{UTF8}{mj}在有理数域上可约\end{CJK}.

  \item (14 \begin{CJK}{UTF8}{mj}分\end{CJK}) \begin{CJK}{UTF8}{mj}用正交线性替换化下列二次型为标准型\end{CJK}

\end{enumerate}
$$
f\left(x_{1}, x_{2}, x_{3}\right)=2 x_{1}^{2}+2 x_{2}^{2}+2 x_{3}^{2}-2 x_{1} x_{2}-2 x_{2} x_{3}-2 x_{1} x_{3}
$$

\begin{enumerate}
  \setcounter{enumi}{5}
  \item (12 \begin{CJK}{UTF8}{mj}分\end{CJK}) \begin{CJK}{UTF8}{mj}设\end{CJK} $\alpha=\left(a_{1}, a_{2}, \cdots, a_{n}\right), \beta=\left(b_{1}, b_{2}, \cdots, b_{n}\right)$ \begin{CJK}{UTF8}{mj}是两个非零复向量\end{CJK}, \begin{CJK}{UTF8}{mj}且\end{CJK} $\sum_{i=1}^{n} a_{i} b_{i}=0$. \begin{CJK}{UTF8}{mj}令\end{CJK} $A=\alpha^{T} \beta$, \begin{CJK}{UTF8}{mj}试求\end{CJK} $A$ \begin{CJK}{UTF8}{mj}的\end{CJK} Jordan \begin{CJK}{UTF8}{mj}标准型及不变因子\end{CJK}.

  \item (14 \begin{CJK}{UTF8}{mj}分\end{CJK}) \begin{CJK}{UTF8}{mj}设\end{CJK} $A, B, C, D$ \begin{CJK}{UTF8}{mj}是\end{CJK} $n$ \begin{CJK}{UTF8}{mj}阶矩阵\end{CJK}, $G=\left(\begin{array}{cc}A & B \\ C & D\end{array}\right)$. \begin{CJK}{UTF8}{mj}如果\end{CJK} $A C=C A,|A| \neq 0$.

\end{enumerate}
(1) \begin{CJK}{UTF8}{mj}证明\end{CJK}: $|G|=|A D-C B|$;

(2) \begin{CJK}{UTF8}{mj}当\end{CJK} $|A D-C B|=0$ \begin{CJK}{UTF8}{mj}时\end{CJK}, \begin{CJK}{UTF8}{mj}证明\end{CJK}: $n \leqslant \operatorname{rank}(G)<2 n$.

\begin{enumerate}
  \setcounter{enumi}{7}
  \item (14 \begin{CJK}{UTF8}{mj}分\end{CJK}) \begin{CJK}{UTF8}{mj}若\end{CJK} $n$ \begin{CJK}{UTF8}{mj}阶矩阵\end{CJK} $A$ \begin{CJK}{UTF8}{mj}满足\end{CJK}: $A^{2}+2 A+3 E=0$.
\end{enumerate}
(1) \begin{CJK}{UTF8}{mj}证明\end{CJK}: \begin{CJK}{UTF8}{mj}对任意的实数\end{CJK} $a, A+a E$ \begin{CJK}{UTF8}{mj}可逆\end{CJK};

(2) \begin{CJK}{UTF8}{mj}求\end{CJK} $A+4 E$ \begin{CJK}{UTF8}{mj}的逆矩阵\end{CJK}.

\begin{enumerate}
  \setcounter{enumi}{8}
  \item (10 \begin{CJK}{UTF8}{mj}分\end{CJK}) \begin{CJK}{UTF8}{mj}设\end{CJK} $A$ \begin{CJK}{UTF8}{mj}是\end{CJK} $n$ \begin{CJK}{UTF8}{mj}阶非零半正定矩阵\end{CJK}, $B$ \begin{CJK}{UTF8}{mj}为\end{CJK} $n$ \begin{CJK}{UTF8}{mj}阶正定矩阵\end{CJK}, \begin{CJK}{UTF8}{mj}证明\end{CJK}:
\end{enumerate}
$$
|A+B|>|B|
$$

\section{$2.102005$ 年}
\begin{enumerate}
  \item \begin{CJK}{UTF8}{mj}填空\end{CJK}、\begin{CJK}{UTF8}{mj}是非\end{CJK}、\begin{CJK}{UTF8}{mj}选择题\end{CJK}:(\begin{CJK}{UTF8}{mj}每小题\end{CJK} 4 \begin{CJK}{UTF8}{mj}分\end{CJK}, \begin{CJK}{UTF8}{mj}共\end{CJK} 60 \begin{CJK}{UTF8}{mj}分\end{CJK})
\end{enumerate}
(1) \begin{CJK}{UTF8}{mj}设\end{CJK} 3 \begin{CJK}{UTF8}{mj}阶方阵的特征值是\end{CJK} $2,3,5$, \begin{CJK}{UTF8}{mj}则\end{CJK} $|2 A-E|=$

(2) \begin{CJK}{UTF8}{mj}如果\end{CJK} $a$ \begin{CJK}{UTF8}{mj}是\end{CJK} $f^{\prime \prime \prime}(x)$ \begin{CJK}{UTF8}{mj}的\end{CJK} 2 \begin{CJK}{UTF8}{mj}重根\end{CJK}, \begin{CJK}{UTF8}{mj}则\end{CJK} $a$ \begin{CJK}{UTF8}{mj}一定是多项式\end{CJK} $f(x)$ \begin{CJK}{UTF8}{mj}的\end{CJK} 5 \begin{CJK}{UTF8}{mj}重根\end{CJK}.

(3) \begin{CJK}{UTF8}{mj}设向量组\end{CJK} $\alpha_{1}, \alpha_{2}, \cdots, \alpha_{s}(s>2)$ \begin{CJK}{UTF8}{mj}线性相关\end{CJK}, \begin{CJK}{UTF8}{mj}且其中任意\end{CJK} $s-1$ \begin{CJK}{UTF8}{mj}个向量线性无关\end{CJK}, \begin{CJK}{UTF8}{mj}则存在全不\end{CJK} \begin{CJK}{UTF8}{mj}为\end{CJK} 0 \begin{CJK}{UTF8}{mj}的数\end{CJK} $k_{1}, k_{2}, \cdots, k_{s}$, \begin{CJK}{UTF8}{mj}使得\end{CJK} $k_{1} \alpha_{1}+k_{2} \alpha_{2}+\cdots+k_{s} \alpha_{s}=0$.

(4) \begin{CJK}{UTF8}{mj}设\end{CJK} $W_{1}$ \begin{CJK}{UTF8}{mj}与\end{CJK} $W_{2}$ \begin{CJK}{UTF8}{mj}分别是数域\end{CJK} $K$ \begin{CJK}{UTF8}{mj}上\end{CJK} 8 \begin{CJK}{UTF8}{mj}元齐次线性方程组\end{CJK} $A X=0$ \begin{CJK}{UTF8}{mj}与\end{CJK} $B X=0$ \begin{CJK}{UTF8}{mj}的解空间\end{CJK}, \begin{CJK}{UTF8}{mj}如果\end{CJK} $\operatorname{rank}(A)=3, \operatorname{rank}(B)=2$, \begin{CJK}{UTF8}{mj}且\end{CJK} $W_{1}+W_{2}=K^{8}$, \begin{CJK}{UTF8}{mj}那么\end{CJK} $\operatorname{dim}\left(W_{1} \cap W_{2}\right)=$

(5) \begin{CJK}{UTF8}{mj}实反对称矩阵的非零特征值必为\end{CJK}:\\
(A) \begin{CJK}{UTF8}{mj}正实数\end{CJK};\\
(B) \begin{CJK}{UTF8}{mj}负实数\end{CJK};\\
(C) 1 \begin{CJK}{UTF8}{mj}或\end{CJK} $-1$;\\
(D) \begin{CJK}{UTF8}{mj}纯虚数\end{CJK}.

(6) \begin{CJK}{UTF8}{mj}若三次实系数多项式\end{CJK} $f(x)$ \begin{CJK}{UTF8}{mj}恰有一个实根\end{CJK}, $\Delta$ \begin{CJK}{UTF8}{mj}为\end{CJK} $f(x)$ \begin{CJK}{UTF8}{mj}的判别式\end{CJK}, \begin{CJK}{UTF8}{mj}则\end{CJK}\\
(A) $\Delta>0$;\\
(B) $\Delta=0$;\\
(C) $\Delta<0$;\\
(D) $\Delta \notin \mathbb{R}$.

(7) 3 \begin{CJK}{UTF8}{mj}阶整系数的行列式等于\end{CJK} $-1$ \begin{CJK}{UTF8}{mj}的正交矩阵共有\end{CJK} \begin{CJK}{UTF8}{mj}个\end{CJK}.

(8) \begin{CJK}{UTF8}{mj}设\end{CJK} $\mathscr{A}$ \begin{CJK}{UTF8}{mj}是行列式等于\end{CJK} $-1$ \begin{CJK}{UTF8}{mj}的正交变换\end{CJK}, \begin{CJK}{UTF8}{mj}则\end{CJK} \begin{CJK}{UTF8}{mj}一定是\end{CJK} $\mathscr{A}$ \begin{CJK}{UTF8}{mj}的特征值\end{CJK}.

(9) \begin{CJK}{UTF8}{mj}排列\end{CJK} $j_{1} j_{2} \cdots j_{n-1} j_{n}$ \begin{CJK}{UTF8}{mj}与\end{CJK} $j_{n} j_{n-1} \cdots j_{2} j_{1}$ \begin{CJK}{UTF8}{mj}排列具有相同的奇偶性的充要条件是\end{CJK} $n \equiv$ $(\bmod 4)$

(10) \begin{CJK}{UTF8}{mj}设\end{CJK} $\gamma_{0}$ \begin{CJK}{UTF8}{mj}是数域\end{CJK} $K$ \begin{CJK}{UTF8}{mj}上非齐次线性方程组\end{CJK} $A X=B$ \begin{CJK}{UTF8}{mj}的特解\end{CJK}, $\eta_{1}, \eta_{2}, \cdots, \eta_{s}$ \begin{CJK}{UTF8}{mj}是该方程组的导出组\end{CJK} \begin{CJK}{UTF8}{mj}的基础解系\end{CJK}. \begin{CJK}{UTF8}{mj}则以下命题中错误的是\end{CJK}:\\
(A) $\gamma_{0}, \gamma_{0}-\eta_{1}, \gamma_{0}-\eta_{2}, \cdots, \gamma_{0}-\eta_{s}$ \begin{CJK}{UTF8}{mj}是\end{CJK} $A X=B$ \begin{CJK}{UTF8}{mj}的一组线性无关解向量\end{CJK};\\
(B) $A X=B$ \begin{CJK}{UTF8}{mj}的每个解均可表为\end{CJK} $\gamma_{0}, \eta_{1}, 2 \eta_{2}, \cdots, s \eta_{s}$ \begin{CJK}{UTF8}{mj}的线性组合\end{CJK};\\
(C) $2 \gamma_{0}+\eta_{1}+\eta_{2}+\cdots+\eta_{s}$ \begin{CJK}{UTF8}{mj}是\end{CJK} $A X=B$ \begin{CJK}{UTF8}{mj}的解\end{CJK};\\
(D) $A X=B$ \begin{CJK}{UTF8}{mj}的每个解均可表为\end{CJK} $\gamma_{0}, \gamma_{0}+\eta_{1}, \gamma_{0}+\eta_{2}, \cdots, \gamma_{0}+\eta_{s}$ \begin{CJK}{UTF8}{mj}的线性组合\end{CJK}.

(11)\begin{CJK}{UTF8}{mj}以下各向量组中线性无关的向量组为\end{CJK}:\\
(A) $(2,-3,4,1),(5,2,7,1),(-1,-3,5,5)$;\\
(B) $(12,0,2),(1,1,1),(3,2,1),(4,78,16)$;\\
(C) $(2,3,1,4),(3,1,2,4),(0,0,0,0)$;\\
(D) $(1,2,-3,1),(3,6,-9,3),(3,0,7,7)$.

(12) \begin{CJK}{UTF8}{mj}由标准欧几里得空间\end{CJK} $\mathbb{R}^{4}$ \begin{CJK}{UTF8}{mj}中的向量组\end{CJK} $\alpha_{1}=(1,0,1,1), \alpha_{2}=(1,-1,-1,0), \alpha_{3}=(2,0,-1,-1)$ \begin{CJK}{UTF8}{mj}张成的子空间\end{CJK} $W$ \begin{CJK}{UTF8}{mj}的一组规范正交基为\end{CJK}\\
(13) \begin{CJK}{UTF8}{mj}设\end{CJK} $V$ \begin{CJK}{UTF8}{mj}是\end{CJK} $n$ \begin{CJK}{UTF8}{mj}维欧几里得空间\end{CJK}, $W$ \begin{CJK}{UTF8}{mj}是\end{CJK} $V$ \begin{CJK}{UTF8}{mj}的子空间\end{CJK}, \begin{CJK}{UTF8}{mj}则\end{CJK} $\left(W^{\perp}\right)^{\perp}$\\
$W .$\\
(A) $\subset$;\\
(B) D;\\
(C) $=$;\\
(14) $A=\left(\begin{array}{cccc}1 & 2 & -1 & -2 \\ 1 & 1 & -1 & -1 \\ 0 & 0 & 1 & 2 \\ 0 & 0 & 1 & 1\end{array}\right)$ \begin{CJK}{UTF8}{mj}的逆矩阵\end{CJK} $A^{-1}=$

(D) $\neq$. (15) \begin{CJK}{UTF8}{mj}设\end{CJK} $\mathscr{A}$ \begin{CJK}{UTF8}{mj}为有限维线性空间\end{CJK} $V$ \begin{CJK}{UTF8}{mj}上的线性变换\end{CJK}. \begin{CJK}{UTF8}{mj}如果\end{CJK} $V \neq \operatorname{Ker}(\mathscr{A})+\operatorname{Im}(\mathscr{A})$, \begin{CJK}{UTF8}{mj}则\end{CJK} $\operatorname{Ker}(\mathscr{A}) \cap$ $\operatorname{Im}(\mathscr{A}) \neq\{0\}$.

\begin{enumerate}
  \setcounter{enumi}{2}
  \item (12 \begin{CJK}{UTF8}{mj}分\end{CJK}) \begin{CJK}{UTF8}{mj}求实三次型\end{CJK}
\end{enumerate}
$$
f\left(x_{1}, x_{2}, \cdots, x_{n}\right)=2 \sum_{i=1}^{n} x_{i}^{2}-2\left(x_{1} x_{2}+x_{2} x_{3}+\cdots+x_{n-1} x_{n}+x_{n} x_{1}\right)
$$
\begin{CJK}{UTF8}{mj}的正惯性指数\end{CJK}、\begin{CJK}{UTF8}{mj}负惯性指数\end{CJK}、\begin{CJK}{UTF8}{mj}符号差及秩\end{CJK}.

\begin{enumerate}
  \setcounter{enumi}{3}
  \item (18 \begin{CJK}{UTF8}{mj}分\end{CJK}) \begin{CJK}{UTF8}{mj}讨论\end{CJK} $b_{1}, b_{2}, \cdots, b_{n}(n \geqslant 2)$, \begin{CJK}{UTF8}{mj}满足什么条件时下列方程有解\end{CJK}, \begin{CJK}{UTF8}{mj}并求解\end{CJK}.
\end{enumerate}
$$
\left\{\begin{array}{l}
x_{1}+x_{2}=b_{1} \\
x_{2}+x_{3}=b_{2} \\
\cdots \cdots \cdots \cdots \\
x_{n-1}+x_{n}=b_{n-1} \\
x_{n}+x_{1}=b_{n}
\end{array}\right.
$$

\begin{enumerate}
  \setcounter{enumi}{4}
  \item (12 \begin{CJK}{UTF8}{mj}分\end{CJK}) \begin{CJK}{UTF8}{mj}试在有理数域\end{CJK}、\begin{CJK}{UTF8}{mj}实数域以及复数域上将\end{CJK}
\end{enumerate}
$$
f(x)=x^{9}+x^{8}+\cdots+x+1
$$
\begin{CJK}{UTF8}{mj}分解为不可约因式的乘积\end{CJK} (\begin{CJK}{UTF8}{mj}结果用根式表示\end{CJK}), \begin{CJK}{UTF8}{mj}并简述理由\end{CJK}.

\begin{enumerate}
  \setcounter{enumi}{5}
  \item (18 \begin{CJK}{UTF8}{mj}分\end{CJK}) \begin{CJK}{UTF8}{mj}已知\end{CJK} $m(\lambda)=\left(\lambda^{2}-2 \lambda+2\right)^{2}(\lambda-1)$ \begin{CJK}{UTF8}{mj}是\end{CJK} 6 \begin{CJK}{UTF8}{mj}阶方阵\end{CJK} $A$ \begin{CJK}{UTF8}{mj}的最小多项式\end{CJK}, \begin{CJK}{UTF8}{mj}且\end{CJK} $\operatorname{Tr}(A)=6$. \begin{CJK}{UTF8}{mj}试求\end{CJK}
\end{enumerate}
(1) $A$ \begin{CJK}{UTF8}{mj}的特征多项式\end{CJK} $f(\lambda)$ \begin{CJK}{UTF8}{mj}及\end{CJK} Jordan \begin{CJK}{UTF8}{mj}标准型\end{CJK};

(2) $A$ \begin{CJK}{UTF8}{mj}的伴随矩阵\end{CJK} $A^{*}$ \begin{CJK}{UTF8}{mj}的\end{CJK} Jordan \begin{CJK}{UTF8}{mj}标准型\end{CJK}.

\begin{enumerate}
  \setcounter{enumi}{6}
  \item $(10$ \begin{CJK}{UTF8}{mj}分\end{CJK}) \begin{CJK}{UTF8}{mj}设\end{CJK}
\end{enumerate}
$$
f(\lambda)=\lambda^{n}+a_{1} \lambda^{n-1}+\cdots+a_{n-1} \lambda+a_{n}
$$
\begin{CJK}{UTF8}{mj}是实对称矩阵\end{CJK} $A$ \begin{CJK}{UTF8}{mj}的特征多项式\end{CJK}. \begin{CJK}{UTF8}{mj}证明\end{CJK}: $A$ \begin{CJK}{UTF8}{mj}是负定矩阵\end{CJK} $\Leftrightarrow a_{1}, a_{2}, \cdots, a_{n}$ \begin{CJK}{UTF8}{mj}均大于\end{CJK} 0 .

\begin{enumerate}
  \setcounter{enumi}{7}
  \item (10 \begin{CJK}{UTF8}{mj}分\end{CJK}) \begin{CJK}{UTF8}{mj}证明\end{CJK}: \begin{CJK}{UTF8}{mj}如果\end{CJK} $n$ \begin{CJK}{UTF8}{mj}阶行列式\end{CJK} $D_{n}$ \begin{CJK}{UTF8}{mj}中所有元素都为\end{CJK} 1 \begin{CJK}{UTF8}{mj}或\end{CJK} $-1$, \begin{CJK}{UTF8}{mj}则当\end{CJK} $n \geqslant 3$ \begin{CJK}{UTF8}{mj}时\end{CJK},
\end{enumerate}
$$
D_{n} \leqslant(n-1)(n-1) !
$$

\begin{enumerate}
  \setcounter{enumi}{8}
  \item (10 \begin{CJK}{UTF8}{mj}分\end{CJK}) \begin{CJK}{UTF8}{mj}证明\end{CJK}: \begin{CJK}{UTF8}{mj}每个秩为\end{CJK} $r$ \begin{CJK}{UTF8}{mj}的\end{CJK} $n$ \begin{CJK}{UTF8}{mj}阶\end{CJK} $(r<n)$ \begin{CJK}{UTF8}{mj}实对称矩阵均可表示为\end{CJK} $n-r$ \begin{CJK}{UTF8}{mj}个秩为\end{CJK} $n-1$ \begin{CJK}{UTF8}{mj}的实对\end{CJK} \begin{CJK}{UTF8}{mj}称矩阵的乘积\end{CJK}.
\end{enumerate}
\section{$2.112006$ 年}
\begin{enumerate}
  \item \begin{CJK}{UTF8}{mj}填空\end{CJK}、\begin{CJK}{UTF8}{mj}是非\end{CJK}、\begin{CJK}{UTF8}{mj}选择题\end{CJK}:(\begin{CJK}{UTF8}{mj}每小题\end{CJK} 4 \begin{CJK}{UTF8}{mj}分\end{CJK}, \begin{CJK}{UTF8}{mj}共\end{CJK} 60 \begin{CJK}{UTF8}{mj}分\end{CJK})
\end{enumerate}
(1) \begin{CJK}{UTF8}{mj}设\end{CJK} $M=\left(\begin{array}{ll}a & b \\ c & d\end{array}\right)$, \begin{CJK}{UTF8}{mj}则\end{CJK} $M$ \begin{CJK}{UTF8}{mj}的伴随矩阵\end{CJK} $M^{*}$ \begin{CJK}{UTF8}{mj}的行列式的值等于\end{CJK}

(2) \begin{CJK}{UTF8}{mj}向量组\end{CJK} $\alpha_{1}=(1,1,2,3), \alpha_{2}=(1,0,7,2), \alpha_{3}=(2,2,4,6), \alpha_{4}=(0,1,5,5)$ \begin{CJK}{UTF8}{mj}的极大线性无关组为\end{CJK} (\begin{CJK}{UTF8}{mj}若有多组\end{CJK}, \begin{CJK}{UTF8}{mj}只需填㝍一组\end{CJK}).

(3) \begin{CJK}{UTF8}{mj}设\end{CJK} $A=\left(\begin{array}{cc}6 & -10 \\ 2 & -3\end{array}\right)$, \begin{CJK}{UTF8}{mj}则\end{CJK} $\operatorname{Tr}\left(A^{5}\right)=$

(4) \begin{CJK}{UTF8}{mj}四元多项式环\end{CJK} $\mathbb{C}\left[x_{1}, x_{2}, x_{3}, x_{4}\right]$ \begin{CJK}{UTF8}{mj}中有所有\end{CJK} 6 \begin{CJK}{UTF8}{mj}次齐次多项式生成的复空间的维数等于\end{CJK}

(5) \begin{CJK}{UTF8}{mj}已知\end{CJK} $A$ \begin{CJK}{UTF8}{mj}的逆矩阵等于\end{CJK} $\left(\begin{array}{ccc}2 & -1 & 2 \\ 1 & 3 & 2 \\ 1 & -1 & 4\end{array}\right)$, \begin{CJK}{UTF8}{mj}则\end{CJK} $A^{2}$ \begin{CJK}{UTF8}{mj}的逆矩阵为\end{CJK}

(6) \begin{CJK}{UTF8}{mj}如果\end{CJK} $\alpha_{1}, \alpha_{2}, \cdots, \alpha_{m}$ \begin{CJK}{UTF8}{mj}是\end{CJK} $n$ \begin{CJK}{UTF8}{mj}维欧几里德空间中的一组非零向量\end{CJK}, \begin{CJK}{UTF8}{mj}且满足\end{CJK}: $\left(\alpha_{i}, \alpha_{j}\right) \leqslant 0, \forall i \neq j$, \begin{CJK}{UTF8}{mj}则\end{CJK} $m$ \begin{CJK}{UTF8}{mj}的最大值是\end{CJK}

(7) \begin{CJK}{UTF8}{mj}设\end{CJK} $\mathscr{A}$ \begin{CJK}{UTF8}{mj}是线性空间\end{CJK} $\mathbb{R}^{2}$ \begin{CJK}{UTF8}{mj}的线性变换\end{CJK}, \begin{CJK}{UTF8}{mj}使得\end{CJK} $\mathscr{A}(1,1)=(1,-1), \mathscr{A}(3,2)=(2,1)$, \begin{CJK}{UTF8}{mj}则\end{CJK} $\mathscr{A}(4,2)$ \begin{CJK}{UTF8}{mj}等于\end{CJK}(\\
(A) $(2,4)$;\\
(B) $(-4,2)$;\\
(C) $(-2,3)$\\
(D) $(2,-3)$

(8) \begin{CJK}{UTF8}{mj}若\end{CJK} 5 \begin{CJK}{UTF8}{mj}个方程\end{CJK} 7 \begin{CJK}{UTF8}{mj}个末知量的齐次线性方程组的系数矩阵的秩为\end{CJK} 3 , \begin{CJK}{UTF8}{mj}则其线性无关解向量的最\end{CJK} \begin{CJK}{UTF8}{mj}大个数等于\end{CJK}\\
(A) 5 ;\\
(B) 4\\
(C) 3 ;\\
(D) 2 .

(9) \begin{CJK}{UTF8}{mj}特征多项式等于\end{CJK} $(\lambda-1)^{4}(\lambda-5)^{2}$ \begin{CJK}{UTF8}{mj}的两两不相似的矩阵共有\end{CJK}\\
(A) 10 \begin{CJK}{UTF8}{mj}个\end{CJK};\\
(B) 8 \begin{CJK}{UTF8}{mj}个\end{CJK};\\
(C) 6 \begin{CJK}{UTF8}{mj}个\end{CJK};\\
(D) 4 \begin{CJK}{UTF8}{mj}个\end{CJK}.

(10) \begin{CJK}{UTF8}{mj}已知\end{CJK} $3,-2,1+\sqrt{2}, 1-i$ \begin{CJK}{UTF8}{mj}是非零有理系数多项式\end{CJK} $f(x)$ \begin{CJK}{UTF8}{mj}的根\end{CJK}, \begin{CJK}{UTF8}{mj}则\end{CJK} $f(x)$ \begin{CJK}{UTF8}{mj}的次数\end{CJK}\\
(A) \begin{CJK}{UTF8}{mj}小于等于\end{CJK} 3 ;\\
(B) \begin{CJK}{UTF8}{mj}等于\end{CJK} 4 ;\\
(C) \begin{CJK}{UTF8}{mj}等于\end{CJK} 5\\
(D) \begin{CJK}{UTF8}{mj}大于\end{CJK} 5 .

(11) \begin{CJK}{UTF8}{mj}已知对任何可逆矩阵\end{CJK} $B$ \begin{CJK}{UTF8}{mj}有分解\end{CJK} $B=T_{B} \cdot P_{B}$, \begin{CJK}{UTF8}{mj}其中\end{CJK} $T_{B}$ \begin{CJK}{UTF8}{mj}是由\end{CJK} $B$ \begin{CJK}{UTF8}{mj}所确定的正交阵\end{CJK}, $P_{B}$ \begin{CJK}{UTF8}{mj}是由\end{CJK} $B$ \begin{CJK}{UTF8}{mj}所确定的上三角阵\end{CJK}, \begin{CJK}{UTF8}{mj}则对任意实可逆阵\end{CJK} $A$ \begin{CJK}{UTF8}{mj}必有分解\end{CJK}: $A=S \cdot Q$, \begin{CJK}{UTF8}{mj}这里\end{CJK} $S$ \begin{CJK}{UTF8}{mj}是正交阵\end{CJK}, $Q$ \begin{CJK}{UTF8}{mj}是下三角阵\end{CJK}. \begin{CJK}{UTF8}{mj}那\end{CJK} \begin{CJK}{UTF8}{mj}么\end{CJK} $S$ \begin{CJK}{UTF8}{mj}可取\end{CJK}\\
(A) $T_{A^{T}}^{T}$\\
(B) $T_{A^{-1}}^{T}$;\\
(C) $T_{A^{T}}$;\\
(D) $T_{\left(A^{-1}\right)^{T}}$.

(12) \begin{CJK}{UTF8}{mj}如果\end{CJK} $n$ \begin{CJK}{UTF8}{mj}个末知量\end{CJK} $n$ \begin{CJK}{UTF8}{mj}个方程的线性方程组\end{CJK} $A X=B$ \begin{CJK}{UTF8}{mj}有唯一解\end{CJK},\begin{CJK}{UTF8}{mj}则其系数行列式\end{CJK} $|A| \neq 0$. ( )

(13) \begin{CJK}{UTF8}{mj}若\end{CJK} $j_{1} j_{2} \cdots j_{n}$ \begin{CJK}{UTF8}{mj}是偶排列\end{CJK}, \begin{CJK}{UTF8}{mj}则\end{CJK} $a_{1} a_{2} \cdots a_{n}$ \begin{CJK}{UTF8}{mj}是奇排列\end{CJK}, \begin{CJK}{UTF8}{mj}其中\end{CJK} $a_{1}=j_{2}, a_{2}=j_{n}, a_{k}=j_{k}, k=$ $3, \cdots, n-1, a_{n}=j_{1}$.

(14) \begin{CJK}{UTF8}{mj}若\end{CJK} $A$ \begin{CJK}{UTF8}{mj}是实对称矩阵\end{CJK}, \begin{CJK}{UTF8}{mj}且\end{CJK} $A^{3}=0$, \begin{CJK}{UTF8}{mj}则\end{CJK} $A=0$.

(15) \begin{CJK}{UTF8}{mj}有理数域上线性空间\end{CJK} $V$ \begin{CJK}{UTF8}{mj}的变换\end{CJK} $\mathscr{A}$ \begin{CJK}{UTF8}{mj}是线性变换当且仅当对任意的\end{CJK} $u, v \in V$ \begin{CJK}{UTF8}{mj}有\end{CJK} $\mathscr{A}(u+v)=$ $\mathscr{A}(u)+\mathscr{A}(v) .$ 2. (12 \begin{CJK}{UTF8}{mj}分\end{CJK}) $a$ \begin{CJK}{UTF8}{mj}取何值时\end{CJK}, \begin{CJK}{UTF8}{mj}下列方程组有解\end{CJK}, \begin{CJK}{UTF8}{mj}并求其解\end{CJK}
$$
\left\{\begin{array}{l}
x_{1}+2 x_{2}+a x_{3}=1, \\
x_{1}+3 x_{2}+(2 a-1) x_{3}=1 \\
x_{1}+(a+3) x_{2}+a x_{3}=2 a-1
\end{array}\right.
$$

\begin{enumerate}
  \setcounter{enumi}{3}
  \item (12 \begin{CJK}{UTF8}{mj}分\end{CJK}) \begin{CJK}{UTF8}{mj}设\end{CJK} $A=\left(\begin{array}{lll}2 & 1 & 0 \\ 0 & 2 & 1 \\ 0 & 0 & 2\end{array}\right), f(x)=\sum_{n=0}^{7} x^{n}$, \begin{CJK}{UTF8}{mj}求\end{CJK} $f(A)$.

  \item (12 \begin{CJK}{UTF8}{mj}分\end{CJK}) \begin{CJK}{UTF8}{mj}用正交线性替换化二次型\end{CJK}

\end{enumerate}
$$
f\left(x_{1}, x_{2}, \cdots, x_{n}\right)=\sum_{i=1}^{n} x_{i} x_{n-i+1}
$$
\begin{CJK}{UTF8}{mj}为标准型\end{CJK}, \begin{CJK}{UTF8}{mj}并求其符号差\end{CJK}.

\begin{enumerate}
  \setcounter{enumi}{5}
  \item (12 \begin{CJK}{UTF8}{mj}分\end{CJK}) \begin{CJK}{UTF8}{mj}已知\end{CJK} 3 \begin{CJK}{UTF8}{mj}阶方阵\end{CJK} $A$ \begin{CJK}{UTF8}{mj}的特征值为\end{CJK} $1,-1,2$. \begin{CJK}{UTF8}{mj}设矩阵\end{CJK} $B=A^{3}-5 A^{2}$. \begin{CJK}{UTF8}{mj}计算\end{CJK}
\end{enumerate}
(1) $B$ \begin{CJK}{UTF8}{mj}的初等因子及\end{CJK} Jordan \begin{CJK}{UTF8}{mj}标准型\end{CJK}.

(2) \begin{CJK}{UTF8}{mj}行列式\end{CJK} $|A-5 E|,|B / 2|$ \begin{CJK}{UTF8}{mj}的值\end{CJK}.

\begin{enumerate}
  \setcounter{enumi}{6}
  \item (12 \begin{CJK}{UTF8}{mj}分\end{CJK}) \begin{CJK}{UTF8}{mj}求所有满足条件的实系数多项式\end{CJK} $f(0)=-4, f(1)=-2, f(-1)=-10, f(2)=2$ \begin{CJK}{UTF8}{mj}的实\end{CJK} \begin{CJK}{UTF8}{mj}系数多项式\end{CJK} $f(x)$.

  \item (10 \begin{CJK}{UTF8}{mj}分\end{CJK}) \begin{CJK}{UTF8}{mj}设\end{CJK} $A$ \begin{CJK}{UTF8}{mj}是实对称矩阵\end{CJK}. \begin{CJK}{UTF8}{mj}证明\end{CJK}: \begin{CJK}{UTF8}{mj}对任意正整数\end{CJK} $m$, \begin{CJK}{UTF8}{mj}存在实对称矩阵\end{CJK} $B$, \begin{CJK}{UTF8}{mj}使得\end{CJK} $A=B^{m} \Leftrightarrow A$ \begin{CJK}{UTF8}{mj}是半正定矩阵\end{CJK}.

  \item (10 \begin{CJK}{UTF8}{mj}分\end{CJK}) \begin{CJK}{UTF8}{mj}证明\end{CJK}: \begin{CJK}{UTF8}{mj}任意方阵可以表示成两个对称方阵之积\end{CJK}.

  \item $(10$ \begin{CJK}{UTF8}{mj}分\end{CJK}) \begin{CJK}{UTF8}{mj}设\end{CJK} $\mathscr{A}$ \begin{CJK}{UTF8}{mj}是\end{CJK} $n$ \begin{CJK}{UTF8}{mj}维线性空间\end{CJK} $V$ \begin{CJK}{UTF8}{mj}上的线性变换\end{CJK}. \begin{CJK}{UTF8}{mj}证明\end{CJK}:\begin{CJK}{UTF8}{mj}存在\end{CJK} $V$ \begin{CJK}{UTF8}{mj}的\end{CJK} $\mathscr{A}$-\begin{CJK}{UTF8}{mj}不变子空间\end{CJK} $W_{1}, W_{2}, \cdots, W_{m}$, \begin{CJK}{UTF8}{mj}使得\end{CJK}

\end{enumerate}
$$
V=W_{1} \oplus W_{2} \oplus \cdots \oplus W_{m}
$$
\begin{CJK}{UTF8}{mj}其中每个\end{CJK} $W_{i}$ \begin{CJK}{UTF8}{mj}的最小多项式为\end{CJK} $p_{i}(x)^{k_{i}}$, \begin{CJK}{UTF8}{mj}而\end{CJK} $p_{1}(x), p_{m}(x), \cdots, p_{m}(x)$ \begin{CJK}{UTF8}{mj}为互不相同的首\end{CJK} \begin{CJK}{UTF8}{mj}不可约多项式\end{CJK}.

\section{$2.122007$ 年}
\section{1. 填空、是非、选择题:(每小题 4 分, 共 60 分)}
(1) \begin{CJK}{UTF8}{mj}设\end{CJK} $\alpha_{1}=(2,1,-1), \alpha_{2}=(1,-1,2), \alpha_{3}=(1,-4,7), \beta_{1}=(4,-1,3), \beta_{2}=(7, a, 4), \beta_{3}=(a,-2,3)$. \begin{CJK}{UTF8}{mj}如果向量组\end{CJK} $\left\{\alpha_{1}, \alpha_{2}, \alpha_{3}\right\}$ \begin{CJK}{UTF8}{mj}与向量组\end{CJK} $\left\{\beta_{1}, \beta_{2}, \beta_{3}\right\}$ \begin{CJK}{UTF8}{mj}的秩相等\end{CJK}, \begin{CJK}{UTF8}{mj}则\end{CJK} $a=$

(2) \begin{CJK}{UTF8}{mj}设\end{CJK} $V$ \begin{CJK}{UTF8}{mj}是由数域\end{CJK} $F$ \begin{CJK}{UTF8}{mj}上全体四次三元对称多项式所生成的\end{CJK} $F$ \begin{CJK}{UTF8}{mj}上的线性空间\end{CJK}, \begin{CJK}{UTF8}{mj}则\end{CJK} $\operatorname{dim}_{F} V=$

(3) \begin{CJK}{UTF8}{mj}设\end{CJK} $A$ \begin{CJK}{UTF8}{mj}是一个\end{CJK} $n$ \begin{CJK}{UTF8}{mj}阶方阵\end{CJK}, \begin{CJK}{UTF8}{mj}满足\end{CJK} $A^{2}=A$, \begin{CJK}{UTF8}{mj}则\end{CJK} $\operatorname{rank}(A)+\operatorname{rank}(A-E)$\\
$(\mathrm{A})>n$\\
$(\mathrm{B})=n$;\\
$(\mathrm{C})<n$;\\
(D) \begin{CJK}{UTF8}{mj}无法确定\end{CJK}.

(4) \begin{CJK}{UTF8}{mj}设\end{CJK} $\mathscr{A}$ \begin{CJK}{UTF8}{mj}是数域\end{CJK} $F$ \begin{CJK}{UTF8}{mj}上\end{CJK} $n$ \begin{CJK}{UTF8}{mj}维线性空间\end{CJK} $V$ \begin{CJK}{UTF8}{mj}的一个线性变换\end{CJK}, \begin{CJK}{UTF8}{mj}则\end{CJK} $V=\mathscr{A} V \oplus \mathscr{A}^{-1}(0)$ \begin{CJK}{UTF8}{mj}的充分必要条件\end{CJK} \begin{CJK}{UTF8}{mj}是\end{CJK} $\operatorname{dim}(\mathscr{A} V)+\operatorname{dim}\left(\mathscr{A}^{-1}(0)\right)=n$.

(5) \begin{CJK}{UTF8}{mj}设\end{CJK} $A$ \begin{CJK}{UTF8}{mj}是由复数域上一个\end{CJK} $n$ \begin{CJK}{UTF8}{mj}阶方阵\end{CJK}, \begin{CJK}{UTF8}{mj}如果与\end{CJK} $A$ \begin{CJK}{UTF8}{mj}相似的矩阵只有\end{CJK} $A$ \begin{CJK}{UTF8}{mj}本身\end{CJK},\begin{CJK}{UTF8}{mj}则\end{CJK} $A$ \begin{CJK}{UTF8}{mj}一定是一个\end{CJK} \begin{CJK}{UTF8}{mj}矩阵\end{CJK}.

(6) \begin{CJK}{UTF8}{mj}已知\end{CJK} $A, B, C$ \begin{CJK}{UTF8}{mj}都是\end{CJK} $n$ \begin{CJK}{UTF8}{mj}阶方阵\end{CJK}, \begin{CJK}{UTF8}{mj}如果\end{CJK} $A B C=E$, \begin{CJK}{UTF8}{mj}则下列等式\end{CJK}
$$
B C A=E, \quad C A B=E, \quad B A C=E, \quad A C B=E, \quad C B A=E
$$
\begin{CJK}{UTF8}{mj}中一定成立的有\end{CJK} ( ) \begin{CJK}{UTF8}{mj}个\end{CJK}.\\
(A) 1 ;\\
(B) 2 ;\\
(C) 3 ;\\
(D) 4 .

(7) \begin{CJK}{UTF8}{mj}设向量组\end{CJK} $\alpha_{1}, \alpha_{2}, \cdots, \alpha_{s-1}(s \geqslant 3)$ \begin{CJK}{UTF8}{mj}线性无关\end{CJK}, $\alpha_{2}, \alpha_{3}, \cdots, \alpha_{s}$ \begin{CJK}{UTF8}{mj}线性相关\end{CJK}, \begin{CJK}{UTF8}{mj}则\end{CJK}\\
(A) $\alpha_{1}$ \begin{CJK}{UTF8}{mj}可被\end{CJK} $\alpha_{2}, \alpha_{3}, \cdots, \alpha_{s}$ \begin{CJK}{UTF8}{mj}线性表示\end{CJK}, $\alpha_{s}$ \begin{CJK}{UTF8}{mj}可被\end{CJK} $\alpha_{1}, \alpha_{2}, \cdots, \alpha_{s-1}$ \begin{CJK}{UTF8}{mj}线性表示\end{CJK};\\
(B) $\alpha_{1}$ \begin{CJK}{UTF8}{mj}可被\end{CJK} $\alpha_{2}, \alpha_{3}, \cdots, \alpha_{s}$ \begin{CJK}{UTF8}{mj}线性表示\end{CJK}, $\alpha_{s}$ \begin{CJK}{UTF8}{mj}不可被\end{CJK} $\alpha_{1}, \alpha_{2}, \cdots, \alpha_{s-1}$ \begin{CJK}{UTF8}{mj}线性表示\end{CJK};\\
(C) $\alpha_{1}$ \begin{CJK}{UTF8}{mj}不可被\end{CJK} $\alpha_{2}, \alpha_{3}, \cdots, \alpha_{s}$ \begin{CJK}{UTF8}{mj}线性表示\end{CJK}, $\alpha_{s}$ \begin{CJK}{UTF8}{mj}可被\end{CJK} $\alpha_{1}, \alpha_{2}, \cdots, \alpha_{s-1}$ \begin{CJK}{UTF8}{mj}线性表示\end{CJK};\\
(D) $\alpha_{1}$ \begin{CJK}{UTF8}{mj}不可被\end{CJK} $\alpha_{2}, \alpha_{3}, \cdots, \alpha_{s}$ \begin{CJK}{UTF8}{mj}线性表示\end{CJK}, $\alpha_{s}$ \begin{CJK}{UTF8}{mj}不可被\end{CJK} $\alpha_{1}, \alpha_{2}, \cdots, \alpha_{s-1}$ \begin{CJK}{UTF8}{mj}线性表示\end{CJK}.

(8) $n$ \begin{CJK}{UTF8}{mj}阶方阵\end{CJK} $A$ \begin{CJK}{UTF8}{mj}可对角化的充分必要条件是\end{CJK}

(9) \begin{CJK}{UTF8}{mj}矩阵\end{CJK} $A=\left(\begin{array}{lll}2 & 1 & 1 \\ 1 & 2 & 1 \\ 1 & 1 & 2\end{array}\right)$ \begin{CJK}{UTF8}{mj}的逆矩阵是\end{CJK}

(10) \begin{CJK}{UTF8}{mj}两个实对称矩阵\end{CJK} $A$ \begin{CJK}{UTF8}{mj}与\end{CJK} $B$ \begin{CJK}{UTF8}{mj}相似\end{CJK} $\Leftrightarrow A$ \begin{CJK}{UTF8}{mj}与\end{CJK} $B$ \begin{CJK}{UTF8}{mj}有相同的特征多项式\end{CJK}.

(11) \begin{CJK}{UTF8}{mj}设\end{CJK} $A$ \begin{CJK}{UTF8}{mj}是\end{CJK} $m \times n$ \begin{CJK}{UTF8}{mj}阶矩阵\end{CJK},\begin{CJK}{UTF8}{mj}下列命题正确的是\end{CJK}\\
(A) \begin{CJK}{UTF8}{mj}若\end{CJK} $\operatorname{rank}(A)=n$, \begin{CJK}{UTF8}{mj}则\end{CJK} $A X=B$ \begin{CJK}{UTF8}{mj}有唯一解\end{CJK};\\
(B) \begin{CJK}{UTF8}{mj}若\end{CJK} $\operatorname{rank}(A)<n$, \begin{CJK}{UTF8}{mj}则\end{CJK} $A X=B$ \begin{CJK}{UTF8}{mj}有无穷多解\end{CJK};\\
(C) \begin{CJK}{UTF8}{mj}若\end{CJK} $\operatorname{rank}(A, B)=m$, \begin{CJK}{UTF8}{mj}则\end{CJK} $A X=B$ \begin{CJK}{UTF8}{mj}有解\end{CJK};\\
(D) \begin{CJK}{UTF8}{mj}若\end{CJK} $\operatorname{rank}(A)=m$, \begin{CJK}{UTF8}{mj}则\end{CJK} $A X=B$ \begin{CJK}{UTF8}{mj}有解\end{CJK}.

(12) \begin{CJK}{UTF8}{mj}如果多项式\end{CJK} $f(x)=x^{3}+a x-1$ \begin{CJK}{UTF8}{mj}在有理数域上\end{CJK} $\mathbb{Q}$ \begin{CJK}{UTF8}{mj}可约\end{CJK}, \begin{CJK}{UTF8}{mj}则\end{CJK} $a=$

(13) \begin{CJK}{UTF8}{mj}当实数\end{CJK} $t=$ \begin{CJK}{UTF8}{mj}时\end{CJK}, \begin{CJK}{UTF8}{mj}多项式\end{CJK} $x^{3}-t x+2$ \begin{CJK}{UTF8}{mj}有重根\end{CJK}. (14) \begin{CJK}{UTF8}{mj}设\end{CJK} $A=\left(\begin{array}{ccc}3 & -2 & 0 \\ -2 & 2 & -2 \\ 0 & -2 & 1\end{array}\right)$, \begin{CJK}{UTF8}{mj}则使\end{CJK} $A+t E$ \begin{CJK}{UTF8}{mj}正定的实数\end{CJK} $t$ \begin{CJK}{UTF8}{mj}的取值范围是\end{CJK}

(15) \begin{CJK}{UTF8}{mj}设\end{CJK} 3 \begin{CJK}{UTF8}{mj}阶矩阵\end{CJK} $A=\left(a_{i j}\right)$ \begin{CJK}{UTF8}{mj}的特征值为\end{CJK} $1,-1,2, A_{i j}$ \begin{CJK}{UTF8}{mj}为\end{CJK} $a_{i j}$ \begin{CJK}{UTF8}{mj}的代数余子式\end{CJK}, \begin{CJK}{UTF8}{mj}则\end{CJK} $A_{11}+A_{22}+A_{33}=$

\begin{enumerate}
  \setcounter{enumi}{2}
  \item (10 \begin{CJK}{UTF8}{mj}分\end{CJK}) \begin{CJK}{UTF8}{mj}计算行列式\end{CJK}:
\end{enumerate}
$$
D_{n}=\left|\begin{array}{cccc}
1 & 1 & \cdots & 1 \\
x_{1} & x_{2} & \cdots & x_{n} \\
x_{1}^{2} & x_{2}^{2} & \cdots & x_{n}^{2} \\
\vdots & \vdots & & \vdots \\
x_{1}^{n-3} & x_{2}^{n-3} & \cdots & x_{n}^{n-3} \\
x_{1}^{n-1} & x_{2}^{n-1} & \cdots & x_{n}^{n-1} \\
x_{1}^{n} & x_{2}^{n} & \cdots & x_{n}^{n}
\end{array}\right| .
$$

\begin{enumerate}
  \setcounter{enumi}{3}
  \item (15 \begin{CJK}{UTF8}{mj}分\end{CJK}) \begin{CJK}{UTF8}{mj}试求矩阵\end{CJK} $A=\left(\begin{array}{cccc}3 & -1 & 0 & 0 \\ 1 & 1 & 0 & 0 \\ 3 & 0 & 5 & -3 \\ 4 & -1 & 3 & -1\end{array}\right)$ \begin{CJK}{UTF8}{mj}的特征多项式\end{CJK}, \begin{CJK}{UTF8}{mj}最小多项式和\end{CJK} Jordan \begin{CJK}{UTF8}{mj}标准型\end{CJK}.

  \item (10 \begin{CJK}{UTF8}{mj}分\end{CJK}) \begin{CJK}{UTF8}{mj}设\end{CJK} $A=\left(\begin{array}{lll}1 & 1 & 1 \\ 1 & 1 & 1 \\ 1 & 1 & 1\end{array}\right)$, \begin{CJK}{UTF8}{mj}求\end{CJK} $B$, \begin{CJK}{UTF8}{mj}使得\end{CJK} $B^{*}=A$.

  \item (15 \begin{CJK}{UTF8}{mj}分\end{CJK}) \begin{CJK}{UTF8}{mj}设\end{CJK} $A$ \begin{CJK}{UTF8}{mj}为\end{CJK} $n$ \begin{CJK}{UTF8}{mj}阶方阵\end{CJK},

\end{enumerate}
(1) \begin{CJK}{UTF8}{mj}证明\end{CJK}: \begin{CJK}{UTF8}{mj}如果为\end{CJK} $A$ \begin{CJK}{UTF8}{mj}实矩阵\end{CJK}, \begin{CJK}{UTF8}{mj}则非齐次线性方程组\end{CJK} $A^{T} A X=A^{T} B$ \begin{CJK}{UTF8}{mj}有解\end{CJK};

(2) \begin{CJK}{UTF8}{mj}对任意的复矩阵\end{CJK} $A$, \begin{CJK}{UTF8}{mj}非齐次线性方程组\end{CJK} $A^{T} A X=A^{T} B$ \begin{CJK}{UTF8}{mj}是否\end{CJK} \begin{CJK}{UTF8}{mj}定有解\end{CJK}?(\begin{CJK}{UTF8}{mj}请说明理由\end{CJK})

\begin{enumerate}
  \setcounter{enumi}{6}
  \item (12 \begin{CJK}{UTF8}{mj}分\end{CJK}) \begin{CJK}{UTF8}{mj}设\end{CJK} $\left(\begin{array}{ll}A & B \\ B^{T} & D\end{array}\right)$ \begin{CJK}{UTF8}{mj}为正定矩阵\end{CJK}, \begin{CJK}{UTF8}{mj}其中\end{CJK} $A$ \begin{CJK}{UTF8}{mj}为\end{CJK} $m$ \begin{CJK}{UTF8}{mj}阶矩阵\end{CJK}, $D$ \begin{CJK}{UTF8}{mj}为\end{CJK} $n$ \begin{CJK}{UTF8}{mj}阶矩阵\end{CJK}, $B$ \begin{CJK}{UTF8}{mj}为\end{CJK} $m \times n$ \begin{CJK}{UTF8}{mj}阶矩阵\end{CJK}. \begin{CJK}{UTF8}{mj}证明\end{CJK}: $A, D$ \begin{CJK}{UTF8}{mj}与\end{CJK} $D-B^{T} A^{-1} B$ \begin{CJK}{UTF8}{mj}都是正定矩阵\end{CJK}.

  \item (14 \begin{CJK}{UTF8}{mj}分\end{CJK}) \begin{CJK}{UTF8}{mj}证明\end{CJK}: \begin{CJK}{UTF8}{mj}向量组\end{CJK} $\left\{\alpha_{1}, \alpha_{2}, \cdots, \alpha_{m}\right\} 与\left\{\beta_{1}, \beta_{2}, \cdots, \beta_{m}\right\}$ \begin{CJK}{UTF8}{mj}等价\end{CJK} $\Leftrightarrow$ \begin{CJK}{UTF8}{mj}存在可逆矩阵\end{CJK} $P$, \begin{CJK}{UTF8}{mj}使得\end{CJK}

\end{enumerate}
$$
\left(\alpha_{1}, \alpha_{2}, \cdots, \alpha_{m}\right) P=\left(\beta_{1}, \beta_{2}, \cdots, \beta_{m}\right) .
$$

\begin{enumerate}
  \setcounter{enumi}{8}
  \item (14 \begin{CJK}{UTF8}{mj}分\end{CJK}) \begin{CJK}{UTF8}{mj}若\end{CJK} $A$ \begin{CJK}{UTF8}{mj}为正定矩阵\end{CJK}, \begin{CJK}{UTF8}{mj}证明\end{CJK}: \begin{CJK}{UTF8}{mj}存在唯一的正定矩阵\end{CJK} $B$, \begin{CJK}{UTF8}{mj}使得\end{CJK} $A=B^{2}$.
\end{enumerate}
\section{$2.132008$ 年}
\section{1. 填空、是非、选择题: (每小题 4 分, 共 60 分)}
(1) \begin{CJK}{UTF8}{mj}设\end{CJK} $\alpha_{1}, \alpha_{2}, \alpha_{3}$ \begin{CJK}{UTF8}{mj}是非齐次线性方程组\end{CJK} $A X=B$ \begin{CJK}{UTF8}{mj}的三个解\end{CJK},\begin{CJK}{UTF8}{mj}则下列哪个向量仍是\end{CJK} $A X=B$ \begin{CJK}{UTF8}{mj}的解\end{CJK}\\
(A) $\alpha_{1}-\alpha_{2}$\\
(B) $\alpha_{1}-2 \alpha_{2}+\alpha_{3}$\\
(C) $\frac{3}{2} \alpha_{1}+\alpha_{2}-\frac{1}{2} \alpha_{3}$\\
(D) $\alpha_{1}-\alpha_{2}+\alpha_{3}$

(2) \begin{CJK}{UTF8}{mj}设\end{CJK} $A$ \begin{CJK}{UTF8}{mj}是一个\end{CJK} $n$ \begin{CJK}{UTF8}{mj}阶方阵\end{CJK}, \begin{CJK}{UTF8}{mj}则线性空间\end{CJK} $W=L\left(E, A, A^{2}, A^{3} \cdots\right)$ \begin{CJK}{UTF8}{mj}的维数\end{CJK} $\operatorname{dim} W$ \begin{CJK}{UTF8}{mj}等于\end{CJK}\\
( )\\
(A) $A$ \begin{CJK}{UTF8}{mj}的特征多项式的次数\end{CJK}\\
(B) $A$ \begin{CJK}{UTF8}{mj}的最小多项式的次数\end{CJK}\\
(C) $A$ \begin{CJK}{UTF8}{mj}的初等因子的个数\end{CJK}\\
(D) $A$ \begin{CJK}{UTF8}{mj}的秩\end{CJK}

(3) \begin{CJK}{UTF8}{mj}每个\end{CJK} 2007 \begin{CJK}{UTF8}{mj}阶实矩阵至少有一个实特征值\end{CJK}.

(4) \begin{CJK}{UTF8}{mj}设\end{CJK} $\mathscr{A}$ \begin{CJK}{UTF8}{mj}是\end{CJK} $n$ \begin{CJK}{UTF8}{mj}维线性空间\end{CJK} $V$ \begin{CJK}{UTF8}{mj}上的线性变换\end{CJK}, \begin{CJK}{UTF8}{mj}则\end{CJK} $V=\mathscr{A} V \oplus \mathscr{A}^{-1}(0)$.

(5) \begin{CJK}{UTF8}{mj}设\end{CJK} $A$ \begin{CJK}{UTF8}{mj}为\end{CJK} 3 \begin{CJK}{UTF8}{mj}阶实对称矩阵\end{CJK}, $1,-1$ \begin{CJK}{UTF8}{mj}是\end{CJK} $A$ \begin{CJK}{UTF8}{mj}的两个特征值\end{CJK}, \begin{CJK}{UTF8}{mj}其中\end{CJK} 1 \begin{CJK}{UTF8}{mj}是二重的\end{CJK}. \begin{CJK}{UTF8}{mj}已知\end{CJK} $(1,1,1)^{T}$ \begin{CJK}{UTF8}{mj}是属于\end{CJK} \begin{CJK}{UTF8}{mj}特征值\end{CJK} 1 \begin{CJK}{UTF8}{mj}的特征向量\end{CJK}, \begin{CJK}{UTF8}{mj}则\end{CJK} \begin{CJK}{UTF8}{mj}是\end{CJK} $A$ \begin{CJK}{UTF8}{mj}的属于特征值\end{CJK} $-1$ \begin{CJK}{UTF8}{mj}的正交的特征\end{CJK} \begin{CJK}{UTF8}{mj}向量\end{CJK}.

(6) \begin{CJK}{UTF8}{mj}设\end{CJK} $\alpha_{1}=(1,-1,3), \alpha_{2}=(2,-1,1), \alpha_{3}=(-1,-1,7), \beta_{1}=(-1,0,2), \beta_{2}=(a, 1,1), \beta_{3}=$ $(4,-1, a)$. \begin{CJK}{UTF8}{mj}如果向量组\end{CJK} $\left\{\alpha_{1}, \alpha_{2}, \alpha_{3}\right\}$ \begin{CJK}{UTF8}{mj}与向量组\end{CJK} $\left\{\beta_{1}, \beta_{2}, \beta_{3}\right\}$ \begin{CJK}{UTF8}{mj}等价\end{CJK}, \begin{CJK}{UTF8}{mj}则\end{CJK} $a=$

(7) \begin{CJK}{UTF8}{mj}对任意的实矩阵\end{CJK} $A$, \begin{CJK}{UTF8}{mj}齐次线性方程组\end{CJK} $A X=0$ \begin{CJK}{UTF8}{mj}与\end{CJK} $A^{T} A X=0$ \begin{CJK}{UTF8}{mj}同解\end{CJK}.

(8) \begin{CJK}{UTF8}{mj}对于多项式\end{CJK} $f(x)$, \begin{CJK}{UTF8}{mj}下列论断正确的是\end{CJK}\\
(A) \begin{CJK}{UTF8}{mj}如果对任意的\end{CJK} $a \in \mathbb{Q}$, \begin{CJK}{UTF8}{mj}都有\end{CJK} $f(a) \in \mathbb{Q}$, \begin{CJK}{UTF8}{mj}则\end{CJK} $f(x)$ \begin{CJK}{UTF8}{mj}的系数都是有理数\end{CJK};\\
(B) \begin{CJK}{UTF8}{mj}如果对任意的\end{CJK} $a \in \mathbb{R}$, \begin{CJK}{UTF8}{mj}都有\end{CJK} $f(a) \in \mathbb{R}$, \begin{CJK}{UTF8}{mj}则\end{CJK} $f(x)$ \begin{CJK}{UTF8}{mj}的系数都是实数\end{CJK};\\
(C) \begin{CJK}{UTF8}{mj}如果对任意的\end{CJK} $a \in \mathbb{Z}$, \begin{CJK}{UTF8}{mj}都有\end{CJK} $f(a) \in \mathbb{Z}$, \begin{CJK}{UTF8}{mj}则\end{CJK} $f(x)$ \begin{CJK}{UTF8}{mj}的系数都是整数\end{CJK};\\
(D) \begin{CJK}{UTF8}{mj}如果对任意的\end{CJK} $a>0$, \begin{CJK}{UTF8}{mj}都有\end{CJK} $f(a)>0$, \begin{CJK}{UTF8}{mj}则\end{CJK} $f(x)$ \begin{CJK}{UTF8}{mj}的系数都是正数\end{CJK}.

(9) \begin{CJK}{UTF8}{mj}正交变换的属于不同的特征值的特征向量正交\end{CJK}.

(10) \begin{CJK}{UTF8}{mj}如果\end{CJK} $f(x)=x^{3}-a x+2$ \begin{CJK}{UTF8}{mj}有有理根\end{CJK}, \begin{CJK}{UTF8}{mj}则\end{CJK} $a=$

(11) $x=a$ \begin{CJK}{UTF8}{mj}为多项式\end{CJK} $f(x)$ \begin{CJK}{UTF8}{mj}的\end{CJK} $k$ \begin{CJK}{UTF8}{mj}重根的充分必要条件是\end{CJK} $x=a$ \begin{CJK}{UTF8}{mj}为\end{CJK} $f^{\prime}(x)$ \begin{CJK}{UTF8}{mj}的\end{CJK} $k-1$ \begin{CJK}{UTF8}{mj}重根\end{CJK}.

(12) \begin{CJK}{UTF8}{mj}设\end{CJK} $A$ \begin{CJK}{UTF8}{mj}是的\end{CJK} $m \times n(m \leqslant n)$ \begin{CJK}{UTF8}{mj}矩阵\end{CJK}, $B$ \begin{CJK}{UTF8}{mj}是\end{CJK} $n$ \begin{CJK}{UTF8}{mj}维列向量\end{CJK}, \begin{CJK}{UTF8}{mj}则下列命题正确的是\end{CJK}

() )\\
(A) \begin{CJK}{UTF8}{mj}当\end{CJK} $A X=0$ \begin{CJK}{UTF8}{mj}有非零解时\end{CJK}, \begin{CJK}{UTF8}{mj}则\end{CJK} $A X=B$ \begin{CJK}{UTF8}{mj}也有解\end{CJK};\\
(B) \begin{CJK}{UTF8}{mj}当\end{CJK} $A X=B$ \begin{CJK}{UTF8}{mj}有解时\end{CJK}, \begin{CJK}{UTF8}{mj}则\end{CJK} $A X=0$ \begin{CJK}{UTF8}{mj}必有无穷多解\end{CJK};\\
(C) \begin{CJK}{UTF8}{mj}当\end{CJK} $A X=0$ \begin{CJK}{UTF8}{mj}有唯一时\end{CJK}, \begin{CJK}{UTF8}{mj}则\end{CJK} $A X=B$ \begin{CJK}{UTF8}{mj}也有唯一解\end{CJK};\\
(D) \begin{CJK}{UTF8}{mj}当\end{CJK} $A X=B$ \begin{CJK}{UTF8}{mj}无解时\end{CJK}, \begin{CJK}{UTF8}{mj}则\end{CJK} $A X=0$ \begin{CJK}{UTF8}{mj}仅有零解\end{CJK}.

(13) \begin{CJK}{UTF8}{mj}已知矩阵\end{CJK} $A$ \begin{CJK}{UTF8}{mj}的特征多项式\end{CJK} $x^{3}+x^{2}-2$, \begin{CJK}{UTF8}{mj}则\end{CJK} $A+2 E$ \begin{CJK}{UTF8}{mj}的逆矩阵是\end{CJK}

(14) \begin{CJK}{UTF8}{mj}实二次型\end{CJK} $f\left(x_{1}, x_{2}, x_{3}\right)=2 x_{1}^{2}-x_{2}^{2}+3 x_{3}^{2}-2 x_{1} x_{2}+x_{1} x_{3}$ \begin{CJK}{UTF8}{mj}的正\end{CJK}、\begin{CJK}{UTF8}{mj}负惯性指数分别是\end{CJK}

(15) \begin{CJK}{UTF8}{mj}已知\end{CJK} $A$ \begin{CJK}{UTF8}{mj}是\end{CJK} $n$ \begin{CJK}{UTF8}{mj}阶方阵\end{CJK}, \begin{CJK}{UTF8}{mj}如果\end{CJK} $A^{2 n}=0$, \begin{CJK}{UTF8}{mj}则\end{CJK} $A^{n}=0$. 2. (20 \begin{CJK}{UTF8}{mj}分\end{CJK}) \begin{CJK}{UTF8}{mj}设\end{CJK} $A$ \begin{CJK}{UTF8}{mj}是数域\end{CJK} $K$ \begin{CJK}{UTF8}{mj}上的一个\end{CJK} $m \times n$ \begin{CJK}{UTF8}{mj}矩阵\end{CJK}, $B$ \begin{CJK}{UTF8}{mj}是一个\end{CJK} $m$ \begin{CJK}{UTF8}{mj}维非零列向量\end{CJK}. \begin{CJK}{UTF8}{mj}令\end{CJK}
$$
W=\left\{\alpha \in K^{n} \mid \text { 存在 } t \in K \text {, 使得 } A \alpha=t B\right\} \text {. }
$$
(1) \begin{CJK}{UTF8}{mj}证明\end{CJK}: $W$ \begin{CJK}{UTF8}{mj}关于\end{CJK} $K^{n}$ \begin{CJK}{UTF8}{mj}的加法和数乘运算构成\end{CJK} $K^{n}$ \begin{CJK}{UTF8}{mj}的一个子空间\end{CJK}.

(2) \begin{CJK}{UTF8}{mj}设线性方程组\end{CJK} $A X=B$ \begin{CJK}{UTF8}{mj}的增广矩阵的秩为\end{CJK} $r$, \begin{CJK}{UTF8}{mj}证明\end{CJK}: $\operatorname{dim} W=n-r+1$.

(3) \begin{CJK}{UTF8}{mj}对于非齐次线性方程组\end{CJK}
$$
\left\{\begin{array}{l}
2 x_{1}-x_{2}+x_{3}+3 x_{4}=-1 \\
x_{1}+2 x_{2}+3 x_{3}-x_{4}=2 \\
4 x_{1}+3 x_{2}+7 x_{3}+x_{4}=3
\end{array}\right.
$$
\begin{CJK}{UTF8}{mj}求\end{CJK} $W$ \begin{CJK}{UTF8}{mj}的一个基\end{CJK}.

\begin{enumerate}
  \setcounter{enumi}{3}
  \item (10 \begin{CJK}{UTF8}{mj}分\end{CJK}) \begin{CJK}{UTF8}{mj}试求矩阵\end{CJK} $A=\left(\begin{array}{ccc}5 & -1 & 3 \\ -8 & 3 & -6 \\ -8 & 2 & -5\end{array}\right)$ \begin{CJK}{UTF8}{mj}的\end{CJK} Jordan \begin{CJK}{UTF8}{mj}标准型\end{CJK}.

  \item (20 \begin{CJK}{UTF8}{mj}分\end{CJK}) \begin{CJK}{UTF8}{mj}设矩阵\end{CJK} $A=\left(\begin{array}{lll}2 & 1 & 1 \\ 1 & 2 & 1 \\ 1 & 1 & 2\end{array}\right)$.

\end{enumerate}
(1) \begin{CJK}{UTF8}{mj}证明\end{CJK}: $A$ \begin{CJK}{UTF8}{mj}为正定矩阵\end{CJK}.

(2) \begin{CJK}{UTF8}{mj}试求正定矩阵\end{CJK} $B$, \begin{CJK}{UTF8}{mj}使\end{CJK} $B^{2}=A$.

\begin{enumerate}
  \setcounter{enumi}{5}
  \item (10 \begin{CJK}{UTF8}{mj}分\end{CJK}) \begin{CJK}{UTF8}{mj}设\end{CJK} $x_{1}, x_{2}, \cdots, x_{s}$ \begin{CJK}{UTF8}{mj}为有理数域\end{CJK} $\mathbb{Q}$ \begin{CJK}{UTF8}{mj}上不可约多项式\end{CJK} $p(x)$ \begin{CJK}{UTF8}{mj}在复数域\end{CJK} $\mathbb{C}$ \begin{CJK}{UTF8}{mj}内的根\end{CJK}, $f(x)$ \begin{CJK}{UTF8}{mj}是\end{CJK} $\mathbb{Q}$ \begin{CJK}{UTF8}{mj}上任意多项式\end{CJK}, \begin{CJK}{UTF8}{mj}使\end{CJK} $p(x) \nmid f(x)$.
\end{enumerate}
\begin{CJK}{UTF8}{mj}证明\end{CJK}: $\exists h(x) \in \mathbb{Q}[x]$, \begin{CJK}{UTF8}{mj}使\end{CJK}
$$
h\left(x_{i}\right)=\frac{1}{f\left(x_{i}\right)}, i=1,2, \cdots, s .
$$

\begin{enumerate}
  \setcounter{enumi}{6}
  \item (20 \begin{CJK}{UTF8}{mj}分\end{CJK}) \begin{CJK}{UTF8}{mj}易知\end{CJK}, \begin{CJK}{UTF8}{mj}当\end{CJK} $a, b$ \begin{CJK}{UTF8}{mj}是不全为零的有理数时\end{CJK}, \begin{CJK}{UTF8}{mj}成立等式\end{CJK}
\end{enumerate}
$$
\frac{1}{+b \sqrt{2}}=\frac{\left|\begin{array}{cc}
1 & b \\
\sqrt{2} & a
\end{array}\right|}{\left|\begin{array}{cc}
a & b \\
2 b & a
\end{array}\right|}
$$
(1) \begin{CJK}{UTF8}{mj}证明\end{CJK}: \begin{CJK}{UTF8}{mj}当\end{CJK} $a, b, c$ \begin{CJK}{UTF8}{mj}为不全为零的有理数时\end{CJK}, \begin{CJK}{UTF8}{mj}有\end{CJK}
$$
+b \sqrt[3]{2}+c \sqrt[3]{4}=\frac{\left|\begin{array}{ccc}
1 & b & c \\
\sqrt[3]{2} & a & b \\
\sqrt[3]{4} & 2 c & a
\end{array}\right|}{\left|\begin{array}{ccc}
a & b & c \\
2 c & a & b \\
2 b & 2 c & a
\end{array}\right|}
$$
(2)\begin{CJK}{UTF8}{mj}应用上述公式\end{CJK}, \begin{CJK}{UTF8}{mj}将根式\end{CJK} $\frac{1}{1-3 \sqrt[3]{2}-2 \sqrt[3]{4}}$ \begin{CJK}{UTF8}{mj}分母有理化\end{CJK}.

(3) \begin{CJK}{UTF8}{mj}请将\end{CJK} (1) \begin{CJK}{UTF8}{mj}中的公式推广到一般的情形\end{CJK}.

\begin{enumerate}
  \setcounter{enumi}{7}
  \item (10 \begin{CJK}{UTF8}{mj}分\end{CJK}) \begin{CJK}{UTF8}{mj}设\end{CJK} $A, B$ \begin{CJK}{UTF8}{mj}是两个特征值为正数的\end{CJK} $n$ \begin{CJK}{UTF8}{mj}阶矩阵\end{CJK}. \begin{CJK}{UTF8}{mj}证明\end{CJK}: \begin{CJK}{UTF8}{mj}如果\end{CJK} $A^{2}=B^{2}$, \begin{CJK}{UTF8}{mj}则\end{CJK} $A=B$.
\end{enumerate}
\section{$2.142009$ 年}
\begin{enumerate}
  \item (10 \begin{CJK}{UTF8}{mj}分\end{CJK}) \begin{CJK}{UTF8}{mj}解下列线性方程组\end{CJK}
\end{enumerate}
$$
\left\{\begin{array}{l}
4 x_{1}+3 x_{2}+2 x_{3}+x_{4}=17 \\
3 x_{1}+2 x_{2}+x_{3}+4 x_{4}=17 \\
2 x_{1}+x_{2}+4 x_{3}+3 x_{4}=17 \\
x_{1}+4 x_{2}+3 x_{3}+2 x_{4}=17
\end{array}\right.
$$

\begin{enumerate}
  \setcounter{enumi}{2}
  \item (10 \begin{CJK}{UTF8}{mj}分\end{CJK}) \begin{CJK}{UTF8}{mj}设\end{CJK} $\mathscr{A}$ \begin{CJK}{UTF8}{mj}是\end{CJK} $n$ \begin{CJK}{UTF8}{mj}维线性空间\end{CJK} $V$ \begin{CJK}{UTF8}{mj}上的秩为\end{CJK} 1 \begin{CJK}{UTF8}{mj}的线性变换\end{CJK}, $\mathscr{B}=\mathscr{A}-\mathscr{E}($ \begin{CJK}{UTF8}{mj}其中\end{CJK} $\mathscr{E}$ \begin{CJK}{UTF8}{mj}为恒等变换\end{CJK}), \begin{CJK}{UTF8}{mj}试求线性变换\end{CJK} $\mathscr{B}$ \begin{CJK}{UTF8}{mj}的最小多项式\end{CJK}.

  \item (10 \begin{CJK}{UTF8}{mj}分\end{CJK}) \begin{CJK}{UTF8}{mj}右分块矩阵\end{CJK} $A=\left(A_{i j}\right)_{s \times t}, \operatorname{rank}\left(A_{i j}\right)=r_{i j}, i=1, \cdots, s, j=1, \cdots, t$. \begin{CJK}{UTF8}{mj}证明\end{CJK}:

\end{enumerate}
$$
\operatorname{rank}(A) \leqslant \sum_{i=1}^{s} \sum_{j=1}^{t} r_{i j}
$$

\begin{enumerate}
  \setcounter{enumi}{4}
  \item (15 \begin{CJK}{UTF8}{mj}分\end{CJK}) \begin{CJK}{UTF8}{mj}设\end{CJK} $x_{1}, x_{2}, x_{3}$ \begin{CJK}{UTF8}{mj}是多项式\end{CJK} $f(x)=x^{3}+5 x^{2}-2 x-7$ \begin{CJK}{UTF8}{mj}的根\end{CJK}, \begin{CJK}{UTF8}{mj}令\end{CJK}
\end{enumerate}
$$
s_{k}=x_{1}^{k}+x_{2}^{k}+x_{3}^{k}, \quad(k=1,2,3,4),
$$
\begin{CJK}{UTF8}{mj}试求\end{CJK} $s_{1}, s_{2}, s_{3}, s_{4}$.

\begin{enumerate}
  \setcounter{enumi}{5}
  \item (15 \begin{CJK}{UTF8}{mj}分\end{CJK}) \begin{CJK}{UTF8}{mj}我们知道\end{CJK}, \begin{CJK}{UTF8}{mj}对任意的实函数\end{CJK} $f(x)$, \begin{CJK}{UTF8}{mj}均存在唯一的偶函数\end{CJK} $g(x)$ \begin{CJK}{UTF8}{mj}和奇函数\end{CJK} $h(x)$, \begin{CJK}{UTF8}{mj}使得\end{CJK} $f(x)=g(x)+h(x) .$
\end{enumerate}
(1) \begin{CJK}{UTF8}{mj}证明\end{CJK}: \begin{CJK}{UTF8}{mj}对复数域上的任一方阵\end{CJK} $A$, \begin{CJK}{UTF8}{mj}均存在唯一的对称矩阵\end{CJK} $B$ \begin{CJK}{UTF8}{mj}和反对称矩阵\end{CJK} $C$, \begin{CJK}{UTF8}{mj}使得\end{CJK} $A=B+C$.

(2) \begin{CJK}{UTF8}{mj}根据上述两个结论\end{CJK},\begin{CJK}{UTF8}{mj}试给出有关线性空间上的线性变换的结论\end{CJK},\begin{CJK}{UTF8}{mj}使上述两个结论是其特例\end{CJK}.

\begin{enumerate}
  \setcounter{enumi}{6}
  \item (15 \begin{CJK}{UTF8}{mj}分\end{CJK}) \begin{CJK}{UTF8}{mj}设多项式\end{CJK} $f(x)=\left(x-a_{1}\right)\left(x-a_{2}\right)\left(x-a_{3}\right)\left(x-a_{4}\right)+1$, \begin{CJK}{UTF8}{mj}其中\end{CJK} $a_{1}<a_{2}<a_{3}<a_{4}$ \begin{CJK}{UTF8}{mj}都是整\end{CJK} \begin{CJK}{UTF8}{mj}数\end{CJK}. \begin{CJK}{UTF8}{mj}证明\end{CJK}: $f(x)$ \begin{CJK}{UTF8}{mj}在有理数域上可约\end{CJK} $\Leftrightarrow a_{4}-a_{1}=3$.

  \item (15 \begin{CJK}{UTF8}{mj}分\end{CJK}) \begin{CJK}{UTF8}{mj}设\end{CJK} $S$ \begin{CJK}{UTF8}{mj}是数域\end{CJK} $K$ \begin{CJK}{UTF8}{mj}上某\end{CJK} $n$ \begin{CJK}{UTF8}{mj}元的非齐次线性方程组\end{CJK} (\textit{) \begin{CJK}{UTF8}{mj}的非空解集\end{CJK}, \begin{CJK}{UTF8}{mj}且\end{CJK} (}) \begin{CJK}{UTF8}{mj}的增广矩阵的\end{CJK} \begin{CJK}{UTF8}{mj}秩为\end{CJK} $r$. \begin{CJK}{UTF8}{mj}证明\end{CJK}:

\end{enumerate}
(1) \begin{CJK}{UTF8}{mj}如果\end{CJK} $r_{0}, r_{1}, \cdots, r_{s}$ \begin{CJK}{UTF8}{mj}是\end{CJK} $S$ \begin{CJK}{UTF8}{mj}的一组线性无关的向量\end{CJK}, \begin{CJK}{UTF8}{mj}则\end{CJK} $s \leqslant n-r$.

(2) $S$ \begin{CJK}{UTF8}{mj}中存在线性无关的向量组\end{CJK} $r_{0}, r_{1}, \cdots, r_{n-r}$.

(3) \begin{CJK}{UTF8}{mj}假设\end{CJK} $r_{0}, r_{1}, \cdots, r_{n-r}$ \begin{CJK}{UTF8}{mj}是\end{CJK} $S$ \begin{CJK}{UTF8}{mj}中任意一组线性无关的向量\end{CJK}, \begin{CJK}{UTF8}{mj}则\end{CJK} $\forall \gamma \in S$, \begin{CJK}{UTF8}{mj}存在\end{CJK} $k_{0}, k_{1}, \cdots, k_{n-r}$, \begin{CJK}{UTF8}{mj}且\end{CJK} $\sum_{i=0}^{n-r} k_{i}=1$, \begin{CJK}{UTF8}{mj}使\end{CJK} $\gamma=\sum_{i=0}^{n-r} k_{i} \gamma_{i}$.

\begin{enumerate}
  \setcounter{enumi}{8}
  \item (20 \begin{CJK}{UTF8}{mj}分\end{CJK}) \begin{CJK}{UTF8}{mj}设\end{CJK} 4 \begin{CJK}{UTF8}{mj}元二次型\end{CJK}
\end{enumerate}
$$
f\left(x_{1}, x_{2}, x_{3}, x_{4}\right)=2 \sum_{i=1}^{4} x_{i}^{2}-2\left(x_{1} x_{2}+x_{2} x_{3}+x_{3} x_{4}+x_{4} x_{1}\right)
$$
\begin{CJK}{UTF8}{mj}试用正交线性替换化二次型为标准型\end{CJK}, \begin{CJK}{UTF8}{mj}并求它的正\end{CJK}、\begin{CJK}{UTF8}{mj}负惯性指数以及符号差\end{CJK}.

\begin{enumerate}
  \setcounter{enumi}{9}
  \item (20 \begin{CJK}{UTF8}{mj}分\end{CJK}) \begin{CJK}{UTF8}{mj}设\end{CJK} $A$ \begin{CJK}{UTF8}{mj}是一实方阵\end{CJK}, \begin{CJK}{UTF8}{mj}证明\end{CJK}: \begin{CJK}{UTF8}{mj}存在正交矩阵\end{CJK} $S, T$ \begin{CJK}{UTF8}{mj}以及上三角矩阵\end{CJK} $P, Q$, \begin{CJK}{UTF8}{mj}使得\end{CJK}
\end{enumerate}
$$
A=S P=Q T
$$

\begin{enumerate}
  \setcounter{enumi}{10}
  \item (20 \begin{CJK}{UTF8}{mj}分\end{CJK}) \begin{CJK}{UTF8}{mj}已知\end{CJK} $W=\mathbb{R}^{3}$ \begin{CJK}{UTF8}{mj}是\end{CJK} 3 \begin{CJK}{UTF8}{mj}维标准的欧式空间\end{CJK}.
\end{enumerate}
(1) \begin{CJK}{UTF8}{mj}设\end{CJK} $V$ \begin{CJK}{UTF8}{mj}是由\end{CJK} $W$ \begin{CJK}{UTF8}{mj}中的向量\end{CJK} $\alpha, \beta, \gamma$ \begin{CJK}{UTF8}{mj}所张成的平行只面体的体积\end{CJK}. \begin{CJK}{UTF8}{mj}证明\end{CJK}: $V=\sqrt{|G(\alpha, \beta, \gamma)|}$, \begin{CJK}{UTF8}{mj}其中\end{CJK} \begin{CJK}{UTF8}{mj}矩阵\end{CJK}
$$
G(\alpha, \beta, \gamma)=\left(\begin{array}{lll}
(\alpha, \alpha) & (\alpha, \beta) & (\alpha, \gamma) \\
(\beta, \alpha) & (\beta, \beta) & (\beta, \gamma) \\
(\gamma, \alpha) & (\gamma, \beta) & (\gamma, \gamma)
\end{array}\right)
$$
(2) \begin{CJK}{UTF8}{mj}设\end{CJK} $A B C D$ \begin{CJK}{UTF8}{mj}是一个对棱长度均相等的四面体\end{CJK}. \begin{CJK}{UTF8}{mj}假设四面体的三对对棱长度分别为\end{CJK} $4,5,6$, \begin{CJK}{UTF8}{mj}试\end{CJK} \begin{CJK}{UTF8}{mj}求该四面体的体积\end{CJK}.

\section{$2.152010$ 年}
\begin{enumerate}
  \item (18 \begin{CJK}{UTF8}{mj}分\end{CJK}) \begin{CJK}{UTF8}{mj}设\end{CJK} $A$ \begin{CJK}{UTF8}{mj}是一个秩为\end{CJK} $r$ \begin{CJK}{UTF8}{mj}的\end{CJK} $n$ \begin{CJK}{UTF8}{mj}阶方阵\end{CJK} $(0<r<n)$. \begin{CJK}{UTF8}{mj}证明\end{CJK}:
\end{enumerate}
(1) \begin{CJK}{UTF8}{mj}存在秩为\end{CJK} 1 \begin{CJK}{UTF8}{mj}的方阵\end{CJK} $B_{1}, B_{2}, \cdots, B_{r}$, \begin{CJK}{UTF8}{mj}使得\end{CJK} $A=B_{1}+B_{2}+\cdots+B_{r}$.

(2) \begin{CJK}{UTF8}{mj}存在秩为\end{CJK} $n-1$ \begin{CJK}{UTF8}{mj}的方阵\end{CJK} $C_{1}, C_{2}, \cdots, C_{n-r}$, \begin{CJK}{UTF8}{mj}使得\end{CJK} $A=C_{1} C_{2} \cdots C_{n-r}$.

(3) \begin{CJK}{UTF8}{mj}若\end{CJK} $A$ \begin{CJK}{UTF8}{mj}为对称矩阵\end{CJK}, \begin{CJK}{UTF8}{mj}则存在秩为\end{CJK} 1 \begin{CJK}{UTF8}{mj}的对称矩阵\end{CJK} $D_{1}, D_{2}, \cdots, D_{r}$, \begin{CJK}{UTF8}{mj}使得\end{CJK} $A=D_{1}+D_{2}+\cdots+D_{r}$.

\begin{enumerate}
  \setcounter{enumi}{2}
  \item (12 \begin{CJK}{UTF8}{mj}分\end{CJK}) \begin{CJK}{UTF8}{mj}设矩阵\end{CJK} $A=\left(\begin{array}{ccc}\frac{1}{2} & 1 & 1 \\ 0 & \frac{1}{3} & 1 \\ 0 & 0 & \frac{1}{5}\end{array}\right)$, \begin{CJK}{UTF8}{mj}求\end{CJK} $A^{n}(n=1,2, \cdots)$.

  \item (12 \begin{CJK}{UTF8}{mj}分\end{CJK}) \begin{CJK}{UTF8}{mj}设\end{CJK} $A$ \begin{CJK}{UTF8}{mj}是一个实系数方阵\end{CJK}.

\end{enumerate}
(1) \begin{CJK}{UTF8}{mj}举例说明\end{CJK}: \begin{CJK}{UTF8}{mj}如果\end{CJK} $A$ \begin{CJK}{UTF8}{mj}的行向量组两两正交\end{CJK}, \begin{CJK}{UTF8}{mj}它的列向量组末必两两正交\end{CJK}.

(2) \begin{CJK}{UTF8}{mj}证明\end{CJK}: \begin{CJK}{UTF8}{mj}如果\end{CJK} $A$ \begin{CJK}{UTF8}{mj}的行向量组是长度相等的正交组\end{CJK}, \begin{CJK}{UTF8}{mj}则它的列向量组也是长度相等的正交组\end{CJK}.

\begin{enumerate}
  \setcounter{enumi}{4}
  \item (20 \begin{CJK}{UTF8}{mj}分\end{CJK}) \begin{CJK}{UTF8}{mj}设\end{CJK} $V=\mathbb{R}^{n}$ \begin{CJK}{UTF8}{mj}是带有标准内积\end{CJK} $(\alpha, \beta)$ \begin{CJK}{UTF8}{mj}的\end{CJK} $n$ \begin{CJK}{UTF8}{mj}维欧式空间\end{CJK}, $\alpha$ \begin{CJK}{UTF8}{mj}是\end{CJK} $V$ \begin{CJK}{UTF8}{mj}中任意给定的非零向量\end{CJK}. \begin{CJK}{UTF8}{mj}定义\end{CJK} $V$ \begin{CJK}{UTF8}{mj}的反射变换\end{CJK} $\sigma_{\alpha}$ \begin{CJK}{UTF8}{mj}如下\end{CJK}:
\end{enumerate}
$$
\sigma_{\alpha}(\beta)=\beta-2 \frac{(\beta, \alpha)}{(\alpha, \alpha)} \alpha, \quad \beta \in V
$$
(1) \begin{CJK}{UTF8}{mj}证明\end{CJK}: \begin{CJK}{UTF8}{mj}此反射变换是正交变换\end{CJK}.

(2) \begin{CJK}{UTF8}{mj}证明\end{CJK}: \begin{CJK}{UTF8}{mj}此反射变换可对角化\end{CJK}.

(3) \begin{CJK}{UTF8}{mj}假设\end{CJK} $n=2, \alpha=(a, b)$, \begin{CJK}{UTF8}{mj}求\end{CJK} $\sigma_{\alpha}$ \begin{CJK}{UTF8}{mj}在自然基下\end{CJK} $e_{1}=(1,0), e_{2}=(0,1)$ \begin{CJK}{UTF8}{mj}的矩阵\end{CJK}.

(4) \begin{CJK}{UTF8}{mj}假设\end{CJK} $n=2$, \begin{CJK}{UTF8}{mj}证明\end{CJK}: $V$ \begin{CJK}{UTF8}{mj}的每个正交变换均可写成不超过两个反射变换的乘积\end{CJK}.

\includegraphics[max width=\textwidth]{2022_04_18_3416d289b173eb9de8c1g-298}

\begin{CJK}{UTF8}{mj}准型\end{CJK}.

\begin{enumerate}
  \setcounter{enumi}{6}
  \item (16 \begin{CJK}{UTF8}{mj}分\end{CJK}) \begin{CJK}{UTF8}{mj}设\end{CJK} $\mathbb{R}^{4}$ \begin{CJK}{UTF8}{mj}中的向量组\end{CJK}
\end{enumerate}
$$
\alpha_{1}=(1,0,0,-1), \alpha_{2}=(0,1,1,2), \alpha_{3}=(1,2,2,3) \text {, }
$$
\begin{CJK}{UTF8}{mj}它们生成的子空间为\end{CJK} $V_{1}$, \begin{CJK}{UTF8}{mj}向量组\end{CJK}
$$
\beta_{1}=(1,-1,-1,-3), \beta_{2}=(-1,1,1,1), \beta_{3}=(3,-3,-3,-7),
$$
\begin{CJK}{UTF8}{mj}它们生成的子空间为\end{CJK} $V_{2}$. \begin{CJK}{UTF8}{mj}求子空间\end{CJK} $V_{1}+V_{2}$ \begin{CJK}{UTF8}{mj}和\end{CJK} $V_{1} \cap V_{2}$ \begin{CJK}{UTF8}{mj}的基和维数\end{CJK}.

\begin{enumerate}
  \setcounter{enumi}{7}
  \item (12 \begin{CJK}{UTF8}{mj}分\end{CJK}) \begin{CJK}{UTF8}{mj}设\end{CJK} $n$ \begin{CJK}{UTF8}{mj}是正整数\end{CJK}. \begin{CJK}{UTF8}{mj}证明\end{CJK}: $x^{4}+n$ \begin{CJK}{UTF8}{mj}在有理数域上可约的充要条件是存在整数\end{CJK} $m$, \begin{CJK}{UTF8}{mj}使得\end{CJK} $n=4 m^{4}$.
\end{enumerate}
$$
\text { 阵 } A=\left(\begin{array}{ccccc}
0 & 1 & \cdots & 1 & 1 \\
1 & 0 & \cdots & 1 & 1 \\
\vdots & \vdots & \ddots & \vdots & \vdots \\
1 & 1 & \cdots & 0 & 1 \\
1 & 1 & \cdots & 1 & 0
\end{array}\right) \text {, 其中 } n>1 \text {. }
$$
(1) \begin{CJK}{UTF8}{mj}求\end{CJK} $A^{-1}$; (2) \begin{CJK}{UTF8}{mj}求对角阵\end{CJK} $D$ \begin{CJK}{UTF8}{mj}以及可逆矩阵\end{CJK} $X$, \begin{CJK}{UTF8}{mj}使得\end{CJK} $X^{-1} A X=D$.

\begin{enumerate}
  \setcounter{enumi}{9}
  \item (12 \begin{CJK}{UTF8}{mj}分\end{CJK}) \begin{CJK}{UTF8}{mj}求一个\end{CJK} 4 \begin{CJK}{UTF8}{mj}阶方阵\end{CJK} $A$, \begin{CJK}{UTF8}{mj}使\end{CJK} $A$ \begin{CJK}{UTF8}{mj}的第一行为\end{CJK} $(4,-2,9,7)$, \begin{CJK}{UTF8}{mj}且\end{CJK} $|A|=1$. \begin{CJK}{UTF8}{mj}这样的矩阵共有多少\end{CJK} \begin{CJK}{UTF8}{mj}介\end{CJK}?

  \item (20 \begin{CJK}{UTF8}{mj}分\end{CJK}) \begin{CJK}{UTF8}{mj}设有实二次函数\end{CJK}

\end{enumerate}
$$
\begin{gathered}
f\left(x_{1}, x_{2}, \cdots, x_{n}\right)=\sum_{i, j=1}^{n} a_{i j} x_{i} x_{j}+\sum_{i=1}^{n} 2 b_{i} x_{i}+c, \quad a_{i j}= \\
\left(a_{i j}\right)_{n}, D=\left(\begin{array}{cc}
A & B^{T} \\
B & c
\end{array}\right), \text { 其中 } B=\left(b_{1}, b_{2}, \cdots, b_{n}\right) .
\end{gathered}
$$
(1) \begin{CJK}{UTF8}{mj}证明\end{CJK}: $A$ \begin{CJK}{UTF8}{mj}负定时\end{CJK}, $f$ \begin{CJK}{UTF8}{mj}有最大值\end{CJK}, \begin{CJK}{UTF8}{mj}且\end{CJK} $f_{\max }=\frac{|D|}{|A|}$.

(2) \begin{CJK}{UTF8}{mj}设\end{CJK} $A$ \begin{CJK}{UTF8}{mj}负定\end{CJK}, \begin{CJK}{UTF8}{mj}试确定当\end{CJK} $x_{1}, x_{2}, \cdots, x_{n}$ \begin{CJK}{UTF8}{mj}为何值时\end{CJK}, $f$ \begin{CJK}{UTF8}{mj}取得最大值\end{CJK}, \begin{CJK}{UTF8}{mj}并说明理由\end{CJK}.

\section{$2.162011$ 年}
\begin{enumerate}
  \item (10 \begin{CJK}{UTF8}{mj}分\end{CJK}) \begin{CJK}{UTF8}{mj}求\end{CJK} $n$ \begin{CJK}{UTF8}{mj}阶矩阵\end{CJK} $M_{n}$ \begin{CJK}{UTF8}{mj}的逆矩阵\end{CJK} $(n \geqslant 3)$, \begin{CJK}{UTF8}{mj}其中\end{CJK}
\end{enumerate}
$$
M_{n}=\left(\begin{array}{cccccc}
a & 0 & 1 & 0 & \cdots & 0 \\
0 & a & 0 & 1 & \ddots & \vdots \\
\vdots & 0 & \ddots & \ddots & \ddots & 0 \\
\vdots & \vdots & \ddots & \ddots & \ddots & 1 \\
\vdots & \vdots & \vdots & \ddots & a & 0 \\
0 & 0 & \cdots & \cdots & 0 & a
\end{array}\right) \quad(a \neq 0)
$$

\begin{enumerate}
  \setcounter{enumi}{2}
  \item (15 \begin{CJK}{UTF8}{mj}分\end{CJK}) \begin{CJK}{UTF8}{mj}设\end{CJK} $K$ \begin{CJK}{UTF8}{mj}是一数域\end{CJK}, \begin{CJK}{UTF8}{mj}方阵\end{CJK} $A \in M_{m}(K), C \in M_{n}(K)$, \begin{CJK}{UTF8}{mj}如果对任意矩阵\end{CJK} $B \in M_{m \times n}(K)$ \begin{CJK}{UTF8}{mj}总有\end{CJK}
\end{enumerate}
$$
\operatorname{rank}\left(\begin{array}{cc}
A & B \\
0 & C
\end{array}\right)=\operatorname{rank}(A)+\operatorname{rank}(C)
$$
\begin{CJK}{UTF8}{mj}证明\end{CJK}: $A$ \begin{CJK}{UTF8}{mj}或\end{CJK} $C$ \begin{CJK}{UTF8}{mj}可逆\end{CJK}.

\begin{enumerate}
  \setcounter{enumi}{3}
  \item (10 \begin{CJK}{UTF8}{mj}分\end{CJK}) \begin{CJK}{UTF8}{mj}设有方阵\end{CJK} $A \in M_{m+n}(K)$, \begin{CJK}{UTF8}{mj}且\end{CJK} $a_{i j}=0,1 \leq i, j \leqslant m$ \begin{CJK}{UTF8}{mj}或\end{CJK} $m+1 \leq i, j \leq m+n$. \begin{CJK}{UTF8}{mj}证明\end{CJK}: \begin{CJK}{UTF8}{mj}若\end{CJK} $\lambda_{0}$ \begin{CJK}{UTF8}{mj}为\end{CJK} $A$ \begin{CJK}{UTF8}{mj}的特征值\end{CJK}, \begin{CJK}{UTF8}{mj}则\end{CJK} $-\lambda_{0}$ \begin{CJK}{UTF8}{mj}也为\end{CJK} $A$ \begin{CJK}{UTF8}{mj}的特征值\end{CJK}.

  \item (30 \begin{CJK}{UTF8}{mj}分\end{CJK}) \begin{CJK}{UTF8}{mj}设\end{CJK} $\mathscr{A}$ \begin{CJK}{UTF8}{mj}是数域\end{CJK} $K$ \begin{CJK}{UTF8}{mj}上有限维线性空间\end{CJK} $V$ \begin{CJK}{UTF8}{mj}上的线性变换\end{CJK}, $W$ \begin{CJK}{UTF8}{mj}是\end{CJK} $V$ \begin{CJK}{UTF8}{mj}的\end{CJK} $\mathscr{A}$-\begin{CJK}{UTF8}{mj}子空间\end{CJK}.

\end{enumerate}
(1) \begin{CJK}{UTF8}{mj}在\end{CJK} $V$ \begin{CJK}{UTF8}{mj}上定义一个二元关系\end{CJK} $\sim: u \sim v \Leftrightarrow u-v \in W$. \begin{CJK}{UTF8}{mj}证明\end{CJK}: $\sim$ \begin{CJK}{UTF8}{mj}是一个等价关系\end{CJK}.

(2) \begin{CJK}{UTF8}{mj}设\end{CJK} $V / W=\{[u] \mid u \in W\}$ \begin{CJK}{UTF8}{mj}是由\end{CJK} (1) \begin{CJK}{UTF8}{mj}中的等价关系所确定的所有等价类组成的集合\end{CJK}. \begin{CJK}{UTF8}{mj}在此集合\end{CJK} \begin{CJK}{UTF8}{mj}上定义加法及标量乘法运算如下\end{CJK}:
$$
[u]+[v]:=[u+v], \quad k[u]:=[k u], \quad k \in K, \quad u, v \in V
$$
\begin{CJK}{UTF8}{mj}证明\end{CJK}: $V / W$ \begin{CJK}{UTF8}{mj}按照这样定义的运算构成数域\end{CJK} $K$ \begin{CJK}{UTF8}{mj}上线性空间\end{CJK} (\begin{CJK}{UTF8}{mj}称为由\end{CJK} $W$ \begin{CJK}{UTF8}{mj}确定的\end{CJK} $V$ \begin{CJK}{UTF8}{mj}的商空间\end{CJK}).

(3) \begin{CJK}{UTF8}{mj}证明\end{CJK}:
$$
\operatorname{dim}(V / W)=\operatorname{dim}(V)-\operatorname{dim}(W)
$$
(4) \begin{CJK}{UTF8}{mj}定义\end{CJK} $V / W$ \begin{CJK}{UTF8}{mj}上的变换\end{CJK} $\mathscr{B}: \mathscr{B}([u])=[\mathscr{A}(u)], u \in V$. \begin{CJK}{UTF8}{mj}证明\end{CJK}: $\mathscr{B}$ \begin{CJK}{UTF8}{mj}是商空间\end{CJK} $V / W$ \begin{CJK}{UTF8}{mj}上的线性变换\end{CJK}.

(5) \begin{CJK}{UTF8}{mj}证明\end{CJK}:
$$
f_{\mathscr{A}}(\lambda)=\left.f_{\mathscr{A}}\right|_{W}(\lambda) f_{\mathscr{B}}(\lambda),
$$
\begin{CJK}{UTF8}{mj}其中\end{CJK} $f_{\mathscr{A}}(\lambda)$ \begin{CJK}{UTF8}{mj}表示线性变换\end{CJK} $\mathscr{A}$ \begin{CJK}{UTF8}{mj}的特征多项式\end{CJK}, $\left.\mathscr{A}\right|_{W}$ \begin{CJK}{UTF8}{mj}表示\end{CJK} $\mathscr{A}$ \begin{CJK}{UTF8}{mj}在\end{CJK} $W$ \begin{CJK}{UTF8}{mj}上的限制\end{CJK}.

\begin{enumerate}
  \setcounter{enumi}{5}
  \item (15 \begin{CJK}{UTF8}{mj}分\end{CJK}) \begin{CJK}{UTF8}{mj}求出所有满足条件\end{CJK} $(x-1) f(x+1)=(x+2) f(x)$ \begin{CJK}{UTF8}{mj}的非零实系数多项式\end{CJK} $f(x)$.

  \item (25 \begin{CJK}{UTF8}{mj}分\end{CJK}) \begin{CJK}{UTF8}{mj}设\end{CJK} $\alpha_{1}=(1,2,-1,0,4), \alpha_{2}=(-1,3,2,4,1), \alpha_{3}=(2,9,-1,4,13), W=L\left(\alpha_{1}, \alpha_{2}, \alpha_{3}\right)$ \begin{CJK}{UTF8}{mj}是\end{CJK} \begin{CJK}{UTF8}{mj}由这三个向量生成的数域\end{CJK} $K$ \begin{CJK}{UTF8}{mj}上的线性空间\end{CJK} $K^{5}$ \begin{CJK}{UTF8}{mj}的子空间\end{CJK}.

\end{enumerate}
(1) \begin{CJK}{UTF8}{mj}求以\end{CJK} $W$ \begin{CJK}{UTF8}{mj}作为其解空间的齐次线性方程组\end{CJK};

(2) \begin{CJK}{UTF8}{mj}求以\end{CJK} $W^{\prime}=\{\eta+\alpha \mid \alpha \in W\}$ \begin{CJK}{UTF8}{mj}为其解集的非齐次线性方程组\end{CJK}, \begin{CJK}{UTF8}{mj}其中\end{CJK} $\eta=(1,2,1,2,1)$.

\begin{enumerate}
  \setcounter{enumi}{7}
  \item (15 \begin{CJK}{UTF8}{mj}分\end{CJK}) \begin{CJK}{UTF8}{mj}证明\end{CJK}: \begin{CJK}{UTF8}{mj}三次方程\end{CJK} $x^{3}-a_{1} x^{2}+a_{2} x-a_{3}=0$ \begin{CJK}{UTF8}{mj}的三个根成等差数列的充要条件是\end{CJK}
\end{enumerate}
$$
2 a_{1}^{3}-9 a_{1} a_{2}+27 a_{3}=0 \text {. }
$$


\end{document}