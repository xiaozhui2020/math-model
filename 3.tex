\documentclass[10pt]{article}
\usepackage[utf8]{inputenc}
\usepackage[T1]{fontenc}
\usepackage{CJKutf8}
\usepackage{amsmath}
\usepackage{amsfonts}
\usepackage{amssymb}
\usepackage{mhchem}
\usepackage{stmaryrd}
\usepackage{bbold}
\usepackage{mathrsfs}
\usepackage{graphicx}
\usepackage[export]{adjustbox}
\graphicspath{ {./images/} }

\begin{document}
\section{7. 兰州大学 2015 年研究生入学考试试题数学分析}
\begin{CJK}{UTF8}{mj}李扬\end{CJK}

\begin{CJK}{UTF8}{mj}微信公众号\end{CJK}: sxkyliyang

\begin{CJK}{UTF8}{mj}一\end{CJK}. \begin{CJK}{UTF8}{mj}计算题\end{CJK} (\begin{CJK}{UTF8}{mj}每题\end{CJK} 10 \begin{CJK}{UTF8}{mj}分\end{CJK}, \begin{CJK}{UTF8}{mj}共\end{CJK} 50 \begin{CJK}{UTF8}{mj}分\end{CJK})

\begin{enumerate}
  \item \begin{CJK}{UTF8}{mj}求极限\end{CJK}
\end{enumerate}
$$
\lim _{n \rightarrow \infty} n^{4} \sqrt[n]{2}\left(1-\cos \frac{1}{n^{2}}\right)
$$

\begin{enumerate}
  \setcounter{enumi}{2}
  \item \begin{CJK}{UTF8}{mj}求不定积分\end{CJK}
\end{enumerate}
$$
\int \frac{\sin 2 x}{\sin ^{2} x+\cos x} \mathrm{~d} x
$$

\begin{enumerate}
  \setcounter{enumi}{3}
  \item \begin{CJK}{UTF8}{mj}求三重积分\end{CJK}
\end{enumerate}
$$
\iiint_{\Omega} x^{2} \mathrm{~d} x \mathrm{~d} y \mathrm{~d} z,
$$
\begin{CJK}{UTF8}{mj}其中\end{CJK} $\Omega$ \begin{CJK}{UTF8}{mj}是由抛物面\end{CJK} $z=2\left(x^{2}+y^{2}\right)$ \begin{CJK}{UTF8}{mj}和\end{CJK} $z=1+x^{2}+y^{2}$ \begin{CJK}{UTF8}{mj}所围成的区域\end{CJK}.

\begin{enumerate}
  \setcounter{enumi}{4}
  \item \begin{CJK}{UTF8}{mj}求曲面积分\end{CJK}
\end{enumerate}
$$
I=\iint_{\Sigma} \sqrt{x^{2}+y^{2}} e^{z}(\mathrm{~d} y \mathrm{~d} z+\mathrm{d} z \mathrm{~d} x+\mathrm{d} x \mathrm{~d} y),
$$
\begin{CJK}{UTF8}{mj}其中\end{CJK} $\Sigma$ \begin{CJK}{UTF8}{mj}为曲面\end{CJK} $z=\sqrt{x^{2}+y^{2}}$ \begin{CJK}{UTF8}{mj}及平面\end{CJK} $z=1, z=2$ \begin{CJK}{UTF8}{mj}所围成的立体的表面外侧\end{CJK}.

\begin{enumerate}
  \setcounter{enumi}{5}
  \item \begin{CJK}{UTF8}{mj}设\end{CJK} $|\theta|<1$, \begin{CJK}{UTF8}{mj}求积分\end{CJK}
\end{enumerate}
$$
\int_{0}^{2 \pi}\left(\ln \frac{1+\theta \cos x}{1-\theta \cos x}\right) \frac{d x}{\cos x}
$$
\begin{CJK}{UTF8}{mj}二\end{CJK}. ( 15 \begin{CJK}{UTF8}{mj}分\end{CJK}) \begin{CJK}{UTF8}{mj}证明\end{CJK}: \begin{CJK}{UTF8}{mj}当\end{CJK} $0<x<\frac{\pi}{2}$ \begin{CJK}{UTF8}{mj}时\end{CJK}, $\tan x+2 \sin x>3 x$.

\begin{CJK}{UTF8}{mj}三\end{CJK}. (10 \begin{CJK}{UTF8}{mj}分\end{CJK}) \begin{CJK}{UTF8}{mj}设\end{CJK} $f(x)$ \begin{CJK}{UTF8}{mj}在\end{CJK} $[a, b]$ \begin{CJK}{UTF8}{mj}上可微\end{CJK}, \begin{CJK}{UTF8}{mj}且\end{CJK} $0<a<b$, \begin{CJK}{UTF8}{mj}试证明\end{CJK}: \begin{CJK}{UTF8}{mj}存在\end{CJK} $\xi \in(a, b)$, \begin{CJK}{UTF8}{mj}使得\end{CJK}
$$
\frac{1}{a-b}\left|\begin{array}{cc}
a & b \\
f(a) & f(b)
\end{array}\right|=f(\xi)-\xi f^{\prime}(\xi)
$$
\begin{CJK}{UTF8}{mj}四\end{CJK}. ( 15 \begin{CJK}{UTF8}{mj}分\end{CJK}) \begin{CJK}{UTF8}{mj}设\end{CJK} $f(x)=\sum_{n=1}^{\infty} \frac{\cos n x}{n^{4}}$, \begin{CJK}{UTF8}{mj}证明\end{CJK}: $f(x)$ \begin{CJK}{UTF8}{mj}在\end{CJK} $(-\infty,+\infty)$ \begin{CJK}{UTF8}{mj}上连续\end{CJK}, \begin{CJK}{UTF8}{mj}并有连续的二阶导函数\end{CJK}.

\begin{CJK}{UTF8}{mj}五\end{CJK}. (15 \begin{CJK}{UTF8}{mj}分\end{CJK}) \begin{CJK}{UTF8}{mj}设\end{CJK} $f(x)=\frac{x+2}{x+1} \sin \frac{1}{x}, a>0$ \begin{CJK}{UTF8}{mj}为任意的正常数\end{CJK}. \begin{CJK}{UTF8}{mj}求证\end{CJK}: $f(x)$ \begin{CJK}{UTF8}{mj}在\end{CJK} $[a,+\infty)$ \begin{CJK}{UTF8}{mj}上一致连续\end{CJK}, \begin{CJK}{UTF8}{mj}在\end{CJK} $(0, a)$ \begin{CJK}{UTF8}{mj}上非一\end{CJK} \begin{CJK}{UTF8}{mj}致连续\end{CJK}.

\begin{CJK}{UTF8}{mj}六\end{CJK}. (15 \begin{CJK}{UTF8}{mj}分\end{CJK}) \begin{CJK}{UTF8}{mj}设\end{CJK} $\Omega$ \begin{CJK}{UTF8}{mj}是\end{CJK} $\mathbb{R}^{m}$ \begin{CJK}{UTF8}{mj}中包含原点的一个凸开集\end{CJK}(\begin{CJK}{UTF8}{mj}凸集指集合中任意两个点之间的连线仍然在集合中\end{CJK}), $f$ \begin{CJK}{UTF8}{mj}是定\end{CJK} \begin{CJK}{UTF8}{mj}义在\end{CJK} $\Omega$ \begin{CJK}{UTF8}{mj}上的连续可微函数\end{CJK}, $f(0)=0$. \begin{CJK}{UTF8}{mj}证明\end{CJK}: \begin{CJK}{UTF8}{mj}存在\end{CJK} $\Omega$ \begin{CJK}{UTF8}{mj}上的\end{CJK} $m$ \begin{CJK}{UTF8}{mj}个连续函数\end{CJK} $g_{1}, g_{2}, \cdots, g_{m}$ \begin{CJK}{UTF8}{mj}使成立\end{CJK}
$$
f(x)=\sum_{i=1}^{m} x_{i} g_{i}(x), \forall x=\left(x_{1}, x_{2}, \cdots, x_{m}\right) \in \Omega .
$$
\begin{CJK}{UTF8}{mj}七\end{CJK}. ( 15 \begin{CJK}{UTF8}{mj}分\end{CJK}) \begin{CJK}{UTF8}{mj}设\end{CJK} $\alpha>0$, \begin{CJK}{UTF8}{mj}给定二元函数\end{CJK} $f(x, y)=|x|^{\alpha}|y|^{\alpha}$. \begin{CJK}{UTF8}{mj}分别讨论并确定\end{CJK} $\alpha$ \begin{CJK}{UTF8}{mj}的范围使得\end{CJK}:

(1) $f(x, y)=|x|^{\alpha}|y|^{\alpha}$ \begin{CJK}{UTF8}{mj}的偏导数在\end{CJK} $\mathbb{R}^{2}$ \begin{CJK}{UTF8}{mj}上存在\end{CJK};

(2) $f(x, y)=|x|^{\alpha}|y|^{\alpha}$ \begin{CJK}{UTF8}{mj}在点\end{CJK} $(0,0)$ \begin{CJK}{UTF8}{mj}处可微\end{CJK}. $\lim _{n \rightarrow \infty} x_{n}=+\infty$ \begin{CJK}{UTF8}{mj}和\end{CJK} $\lim _{n \rightarrow \infty} f\left(x_{n}\right)=0 .$

\section{8. 兰州大学 2016 年研究生入学考试试题数学分析}
\begin{CJK}{UTF8}{mj}李扬\end{CJK}

\begin{CJK}{UTF8}{mj}微信公众号\end{CJK}: sxkyliyang

\begin{CJK}{UTF8}{mj}一\end{CJK}. \begin{CJK}{UTF8}{mj}解答\end{CJK} (\begin{CJK}{UTF8}{mj}每题\end{CJK} 10 \begin{CJK}{UTF8}{mj}分\end{CJK}, \begin{CJK}{UTF8}{mj}共\end{CJK} 50 \begin{CJK}{UTF8}{mj}分\end{CJK})

\begin{enumerate}
  \item \begin{CJK}{UTF8}{mj}求极限\end{CJK}
\end{enumerate}
$$
\lim _{x \rightarrow 0} \frac{\sin x-\tan x}{x^{3}}
$$

\begin{enumerate}
  \setcounter{enumi}{2}
  \item \begin{CJK}{UTF8}{mj}求积分\end{CJK}
\end{enumerate}
$$
I=\int_{0}^{\frac{\pi}{2}} \ln \sin x \mathrm{~d} x
$$

\begin{enumerate}
  \setcounter{enumi}{3}
  \item \begin{CJK}{UTF8}{mj}设\end{CJK} $C$ \begin{CJK}{UTF8}{mj}为抛物线\end{CJK} $x=\frac{\pi}{2} y^{2}$ \begin{CJK}{UTF8}{mj}自点\end{CJK} $(0,0)$ \begin{CJK}{UTF8}{mj}到点\end{CJK} $\left(\frac{\pi}{2}, 1\right)$ \begin{CJK}{UTF8}{mj}的弧段\end{CJK}. \begin{CJK}{UTF8}{mj}求积分\end{CJK}
\end{enumerate}
$$
I=\int_{C}\left(2 x y^{3}-y^{2} \cos x\right) \mathrm{d} x+\left(1-2 y \sin x+3 x^{2} y^{2}\right) \mathrm{d} y
$$

\begin{enumerate}
  \setcounter{enumi}{4}
  \item \begin{CJK}{UTF8}{mj}求积分\end{CJK} $I=\iint_{\Sigma} x^{2} \mathrm{~d} y \mathrm{~d} z+y^{2} \mathrm{~d} z \mathrm{~d} x+z^{2} \mathrm{~d} x \mathrm{~d} y$, \begin{CJK}{UTF8}{mj}其中\end{CJK} $\Sigma$ \begin{CJK}{UTF8}{mj}为雉面\end{CJK} $x^{2}+y^{2}=z^{2}(0 \leqslant z \leqslant h)$ \begin{CJK}{UTF8}{mj}的外侧\end{CJK}.

  \item \begin{CJK}{UTF8}{mj}讨论级数\end{CJK} $\sum_{n=2}^{\infty} \frac{(-1)^{n}}{\left[n+(-1)^{n} \sqrt{n}\right]^{p}}(p>0)$ \begin{CJK}{UTF8}{mj}的敛散性\end{CJK}(\begin{CJK}{UTF8}{mj}包含条件收敛性和绝对收敛性\end{CJK}).

\end{enumerate}
\begin{CJK}{UTF8}{mj}二\end{CJK}. ( 15 \begin{CJK}{UTF8}{mj}分\end{CJK}) \begin{CJK}{UTF8}{mj}设函数\end{CJK} $f(x, y)$ \begin{CJK}{UTF8}{mj}定义在矩形域\end{CJK} $\Omega=\{(x, y) \mid 0 \leqslant x \leqslant a, 0 \leqslant y \leqslant b\}$ \begin{CJK}{UTF8}{mj}上\end{CJK}, \begin{CJK}{UTF8}{mj}其中\end{CJK} $a, b$ \begin{CJK}{UTF8}{mj}为正常数\end{CJK}, \begin{CJK}{UTF8}{mj}在底边\end{CJK} $I_{0}=$ $\{(x, 0) \mid 0 \leqslant x \leqslant a\}$ \begin{CJK}{UTF8}{mj}上连续\end{CJK}. \begin{CJK}{UTF8}{mj}证明\end{CJK}: \begin{CJK}{UTF8}{mj}存在\end{CJK} $\delta \in(0, b)$, \begin{CJK}{UTF8}{mj}使得\end{CJK} $f(x, y)$ \begin{CJK}{UTF8}{mj}在矩形区域\end{CJK} $\Omega_{\delta}=\{(x, y) \mid 0 \leqslant x \leqslant a, 0 \leqslant y \leqslant \delta\}$ \begin{CJK}{UTF8}{mj}上有界\end{CJK}.

\begin{CJK}{UTF8}{mj}三\end{CJK}. ( 15 \begin{CJK}{UTF8}{mj}分\end{CJK}) \begin{CJK}{UTF8}{mj}证明\end{CJK}: \begin{CJK}{UTF8}{mj}对\end{CJK} $0<a<b \leqslant \frac{\pi}{2}$, \begin{CJK}{UTF8}{mj}成立不等式\end{CJK} $\frac{\sin a}{a}>\frac{\sin b}{b}$.

\begin{CJK}{UTF8}{mj}四\end{CJK}. (10 \begin{CJK}{UTF8}{mj}分\end{CJK}) \begin{CJK}{UTF8}{mj}设\end{CJK}
$$
f(x, y)= \begin{cases}x y \frac{x^{2}-y^{2}}{x^{2}+y^{2}}, & x^{2}+y^{2} \neq 0 \\ 0, & x^{2}+y^{2}=0 .\end{cases}
$$
\begin{CJK}{UTF8}{mj}证明\end{CJK}: $f(x, y)$ \begin{CJK}{UTF8}{mj}在\end{CJK} $(0,0)$ \begin{CJK}{UTF8}{mj}处可偏导且可微\end{CJK}, \begin{CJK}{UTF8}{mj}并讨论偏导数在\end{CJK} $(0,0)$ \begin{CJK}{UTF8}{mj}的连续性\end{CJK}.

\begin{CJK}{UTF8}{mj}五\end{CJK}. ( 15 \begin{CJK}{UTF8}{mj}分\end{CJK}) \begin{CJK}{UTF8}{mj}设\end{CJK} $f(x)=\sum_{n=1}^{\infty} \frac{e^{-\sqrt{n} x}}{n}$, \begin{CJK}{UTF8}{mj}证明\end{CJK}:

(1) \begin{CJK}{UTF8}{mj}函数项级数\end{CJK} $\sum_{n=1}^{\infty} \frac{e^{-\sqrt{n} x}}{n}$ \begin{CJK}{UTF8}{mj}在\end{CJK} $(0,+\infty)$ \begin{CJK}{UTF8}{mj}上不是一致收敛的\end{CJK};

(2) $f(x)$ \begin{CJK}{UTF8}{mj}在\end{CJK} $(0,+\infty)$ \begin{CJK}{UTF8}{mj}上无穷次可微\end{CJK}.

\begin{CJK}{UTF8}{mj}六\end{CJK}. (15 \begin{CJK}{UTF8}{mj}分\end{CJK}) \begin{CJK}{UTF8}{mj}证明\end{CJK}: \begin{CJK}{UTF8}{mj}含参量积分\end{CJK} $I(x)=\int_{0}^{+\infty} \frac{\sin t}{1+(x+t)^{2}} \mathrm{~d} t$ \begin{CJK}{UTF8}{mj}在\end{CJK} $x \in(-\infty,+\infty)$ \begin{CJK}{UTF8}{mj}上连续且无穷次可微\end{CJK}.

\begin{CJK}{UTF8}{mj}七\end{CJK}. (15 \begin{CJK}{UTF8}{mj}分\end{CJK}) \begin{CJK}{UTF8}{mj}设函数\end{CJK} $f(x)$ \begin{CJK}{UTF8}{mj}是定义在\end{CJK} $[0, a]$ \begin{CJK}{UTF8}{mj}上的两次连续可微函数\end{CJK}. \begin{CJK}{UTF8}{mj}证明\end{CJK}: \begin{CJK}{UTF8}{mj}存在\end{CJK} $\xi \in[0, a]$, \begin{CJK}{UTF8}{mj}使得成立\end{CJK}
$$
\int_{0}^{a} f(x) \mathrm{d} x=\frac{a}{2}[f(0)+f(a)]-\frac{1}{12} f^{\prime \prime}(\xi) a^{3} .
$$
\begin{CJK}{UTF8}{mj}八\end{CJK}. (15 \begin{CJK}{UTF8}{mj}分\end{CJK}) \begin{CJK}{UTF8}{mj}设\end{CJK} $f(x)$ \begin{CJK}{UTF8}{mj}在\end{CJK} $[0,1]$ \begin{CJK}{UTF8}{mj}上为正连续函数\end{CJK}, $m, M$ \begin{CJK}{UTF8}{mj}分别为\end{CJK} $f(x)$ \begin{CJK}{UTF8}{mj}在\end{CJK} $[0,1]$ \begin{CJK}{UTF8}{mj}上的最小值和最大值\end{CJK}. \begin{CJK}{UTF8}{mj}证明\end{CJK}:
$$
1 \leqslant \int_{0}^{1} \frac{1}{f(x)} \mathrm{d} x \int_{0}^{1} f(x) \mathrm{d} x \leqslant \frac{(m+M)^{2}}{4 m M} .
$$

\section{9. 兰州大学 2017 年研究生入学考试试题数学分析}
\begin{CJK}{UTF8}{mj}李扬\end{CJK}

\begin{CJK}{UTF8}{mj}微信公众号\end{CJK}: sxkyliyang

\begin{CJK}{UTF8}{mj}一\end{CJK}. \begin{CJK}{UTF8}{mj}解答\end{CJK} (\begin{CJK}{UTF8}{mj}每题\end{CJK} 10 \begin{CJK}{UTF8}{mj}分\end{CJK}, \begin{CJK}{UTF8}{mj}共\end{CJK} 50 \begin{CJK}{UTF8}{mj}分\end{CJK})

\begin{enumerate}
  \item \begin{CJK}{UTF8}{mj}设\end{CJK} $x_{n}=\frac{\tan \frac{\pi}{4 n}}{n+\frac{1}{n}}+\frac{\tan \frac{2 \pi}{4 n}}{n+\frac{2}{n}}+\cdots+\frac{\tan \frac{\pi}{4}}{n+1}$, \begin{CJK}{UTF8}{mj}求\end{CJK} $\lim _{n \rightarrow \infty} x_{n}$.

  \item \begin{CJK}{UTF8}{mj}求\end{CJK} $\lim _{x \rightarrow 0} \frac{x-\int_{0}^{x} e^{x^{2}} \mathrm{~d} x}{x^{2} \sin (2 x)}$.

  \item \begin{CJK}{UTF8}{mj}求幂级数\end{CJK} $\sum_{n=1}^{\infty} \frac{x^{4 n-1}}{4 n+1}$ \begin{CJK}{UTF8}{mj}的和函数\end{CJK}.

  \item \begin{CJK}{UTF8}{mj}求曲面积分\end{CJK}

\end{enumerate}
$$
\iint_{\Sigma}(x-y) \mathrm{d} x \mathrm{~d} y+(y-z) x \mathrm{~d} y \mathrm{~d} z
$$
\begin{CJK}{UTF8}{mj}其中\end{CJK} $\Sigma$ \begin{CJK}{UTF8}{mj}为柱面\end{CJK} $x^{2}+y^{2}=1$ \begin{CJK}{UTF8}{mj}及平面\end{CJK} $z=0, z=3$ \begin{CJK}{UTF8}{mj}所围成的空间闭区域\end{CJK} $\Omega$ \begin{CJK}{UTF8}{mj}的整个边界曲面的外侧\end{CJK}.

\begin{enumerate}
  \setcounter{enumi}{5}
  \item \begin{CJK}{UTF8}{mj}求曲线积分\end{CJK}
\end{enumerate}
$$
I=\int_{L}(y-z) \mathrm{d} x+(z-x) \mathrm{d} y+(x-y) \mathrm{d} z
$$
\begin{CJK}{UTF8}{mj}其中\end{CJK} $L$ \begin{CJK}{UTF8}{mj}为圆柱面\end{CJK} $x^{2}+y^{2}=a^{2}$ \begin{CJK}{UTF8}{mj}与平面\end{CJK} $\frac{x}{a}+\frac{z}{h}=1(a>0, h>0)$ \begin{CJK}{UTF8}{mj}的交线\end{CJK}, \begin{CJK}{UTF8}{mj}从\end{CJK} $x$ \begin{CJK}{UTF8}{mj}轴正向看去\end{CJK}, \begin{CJK}{UTF8}{mj}曲线是逆时\end{CJK} \begin{CJK}{UTF8}{mj}针方向\end{CJK}.

\begin{CJK}{UTF8}{mj}二\end{CJK}. ( 15 \begin{CJK}{UTF8}{mj}分\end{CJK}) \begin{CJK}{UTF8}{mj}设\end{CJK} $f(x)$ \begin{CJK}{UTF8}{mj}在\end{CJK} $(-\infty,+\infty)$ \begin{CJK}{UTF8}{mj}内有定义\end{CJK}, \begin{CJK}{UTF8}{mj}对任意\end{CJK} $\xi \in(-\infty,+\infty)$, \begin{CJK}{UTF8}{mj}存在\end{CJK} $\delta>0$ \begin{CJK}{UTF8}{mj}使得当\end{CJK} $x \in U^{0}(\xi, \delta)$ \begin{CJK}{UTF8}{mj}时\end{CJK}, \begin{CJK}{UTF8}{mj}有\end{CJK}

\begin{CJK}{UTF8}{mj}三\end{CJK}. (15 \begin{CJK}{UTF8}{mj}分\end{CJK}) \begin{CJK}{UTF8}{mj}证明\end{CJK}: \begin{CJK}{UTF8}{mj}若\end{CJK} $f(x)$ \begin{CJK}{UTF8}{mj}在\end{CJK} $[a,+\infty)$ \begin{CJK}{UTF8}{mj}上连续\end{CJK}, $\lim _{x \rightarrow+\infty} f(x)=A$, \begin{CJK}{UTF8}{mj}其中\end{CJK} $A \in(-\infty,+\infty)$. \begin{CJK}{UTF8}{mj}则\end{CJK} $f(x)$ \begin{CJK}{UTF8}{mj}在\end{CJK} $[a,+\infty)$ \begin{CJK}{UTF8}{mj}上一致\end{CJK}

\begin{CJK}{UTF8}{mj}五\end{CJK}. ( 15 \begin{CJK}{UTF8}{mj}分\end{CJK}) \begin{CJK}{UTF8}{mj}设函数\end{CJK} $f(x)$ \begin{CJK}{UTF8}{mj}在\end{CJK} $(-\infty,+\infty)$ \begin{CJK}{UTF8}{mj}上有三阶导数\end{CJK}, \begin{CJK}{UTF8}{mj}并且\end{CJK} $f(x)$ \begin{CJK}{UTF8}{mj}和\end{CJK} $f^{\prime \prime \prime}(x)$ \begin{CJK}{UTF8}{mj}在\end{CJK} $(-\infty,+\infty)$ \begin{CJK}{UTF8}{mj}上有界\end{CJK}, \begin{CJK}{UTF8}{mj}证明\end{CJK}: $f^{\prime}(x)$ \begin{CJK}{UTF8}{mj}和\end{CJK}
$$
F(x, y)=\left\{\begin{array}{lr}
\left(x^{2}+y^{2}\right) \sin \frac{1}{x^{2}+y^{2}}, & x^{2}+y^{2} \neq 0, \\
0, & x^{2}+y^{2}=0 .
\end{array}\right.
$$
$(0,0)$ \begin{CJK}{UTF8}{mj}处可微\end{CJK}.
$$
\lim _{h \rightarrow 0} \int_{a}^{b}|f(x+h)-f(x)| \mathrm{d} x=0 .
$$

\section{0. 兰州大学 2018 年研究生入学考试试题数学分析}
\begin{CJK}{UTF8}{mj}李扬\end{CJK}

\begin{CJK}{UTF8}{mj}微信公众号\end{CJK}: sxkyliyang

\begin{CJK}{UTF8}{mj}一\end{CJK}. \begin{CJK}{UTF8}{mj}计算题\end{CJK}.

\begin{enumerate}
  \item \begin{CJK}{UTF8}{mj}求定积分\end{CJK}
\end{enumerate}
$$
\int_{0}^{\pi} \frac{x \sin x}{1+\cos ^{2} x} \mathrm{~d} x
$$

\begin{enumerate}
  \setcounter{enumi}{2}
  \item \begin{CJK}{UTF8}{mj}求极限\end{CJK}
\end{enumerate}
$$
\lim _{x \rightarrow+\infty}\left[x-x^{2} \ln \left(1+\frac{1}{x}\right)\right]
$$

\begin{enumerate}
  \setcounter{enumi}{3}
  \item \begin{CJK}{UTF8}{mj}求极限\end{CJK}
\end{enumerate}
$$
\lim _{n \rightarrow \infty} \sum_{k=1}^{n} \frac{(-1)^{k}}{2 k+1}
$$

\begin{enumerate}
  \setcounter{enumi}{4}
  \item \begin{CJK}{UTF8}{mj}求曲面积分\end{CJK}
\end{enumerate}
$$
Z=\iint_{S} z \mathrm{~d} S
$$
\begin{CJK}{UTF8}{mj}其中\end{CJK}
$$
S:\left\{\begin{array}{l}
x^{2}+z^{2}=2 a z \quad(a>0) \\
z=\sqrt{x^{2}+y^{2}}
\end{array}\right.
$$

\begin{enumerate}
  \setcounter{enumi}{5}
  \item \begin{CJK}{UTF8}{mj}求曲线积分\end{CJK}
\end{enumerate}
$$
\int_{L}[\varphi(y) \cos x-\pi y] \mathrm{d} x+\left[\varphi^{\prime}(y) \sin x-\pi\right] \mathrm{d} y
$$
\begin{CJK}{UTF8}{mj}其中\end{CJK} $L:[\pi, 2] \rightarrow[3 \pi, 4], \sigma=2$.

\begin{CJK}{UTF8}{mj}二\end{CJK}. \begin{CJK}{UTF8}{mj}已知\end{CJK} $f(x)$ \begin{CJK}{UTF8}{mj}是定义在区间\end{CJK} $[a, b]$ \begin{CJK}{UTF8}{mj}上的单调递增函数\end{CJK}, \begin{CJK}{UTF8}{mj}满足\end{CJK} $f(a) \geqslant a, f(b) \leqslant b$, \begin{CJK}{UTF8}{mj}证明存在\end{CJK} $\xi \in[a, b]$ \begin{CJK}{UTF8}{mj}使得\end{CJK} $f(\xi)=\xi$.

\begin{CJK}{UTF8}{mj}三\end{CJK}. \begin{CJK}{UTF8}{mj}设函数\end{CJK} $f(x)$ \begin{CJK}{UTF8}{mj}在\end{CJK} $[0,+\infty)$ \begin{CJK}{UTF8}{mj}上一致连续\end{CJK}, \begin{CJK}{UTF8}{mj}且对任意固定的\end{CJK} $x>0$, \begin{CJK}{UTF8}{mj}都有\end{CJK} $\lim _{n \rightarrow \infty} f(x+n)=0\left(n \in \mathbb{N}^{+}\right)$, \begin{CJK}{UTF8}{mj}证明\end{CJK} $\lim _{x \rightarrow+\infty} f(x)=0 .$

\begin{CJK}{UTF8}{mj}四\end{CJK}. \begin{CJK}{UTF8}{mj}设函数\end{CJK} $f(x)$ \begin{CJK}{UTF8}{mj}在\end{CJK} $[a, b]$ \begin{CJK}{UTF8}{mj}连续\end{CJK}, \begin{CJK}{UTF8}{mj}在\end{CJK} $(a, b)$ \begin{CJK}{UTF8}{mj}存在二阶导数\end{CJK}, \begin{CJK}{UTF8}{mj}证明存在\end{CJK} $\xi \in(a, b)$ \begin{CJK}{UTF8}{mj}使得\end{CJK}
$$
f(a)-2 f\left(\frac{a+b}{2}\right)+f(b)=\frac{1}{4}(b-a)^{2} f^{\prime \prime}(\xi) .
$$
\begin{CJK}{UTF8}{mj}五\end{CJK}. \begin{CJK}{UTF8}{mj}已知\end{CJK} $a, b>0$, \begin{CJK}{UTF8}{mj}设二元函数\end{CJK}
$$
f(x, y)= \begin{cases}\frac{x y}{\sqrt{a x^{2}+b y^{2}}}, & (x, y) \neq(0,0) \\ 0, & (x, y)=(0,0)\end{cases}
$$
\begin{CJK}{UTF8}{mj}试讨论\end{CJK} $f(x, y)$ \begin{CJK}{UTF8}{mj}在原点的连续性\end{CJK}, \begin{CJK}{UTF8}{mj}可微性\end{CJK}.

\begin{CJK}{UTF8}{mj}六\end{CJK}. \begin{CJK}{UTF8}{mj}已知\end{CJK} $\alpha$ \begin{CJK}{UTF8}{mj}是实数\end{CJK}, \begin{CJK}{UTF8}{mj}试讨论反常积分\end{CJK} $\int_{0}^{+\infty} \frac{\ln (1+x)}{x^{\alpha}} \mathrm{d} x$ \begin{CJK}{UTF8}{mj}的敛散性\end{CJK}.

(1) $f(X)$ \begin{CJK}{UTF8}{mj}在\end{CJK} $\mathbb{R}^{n}$ \begin{CJK}{UTF8}{mj}有界\end{CJK};

\section{1. 南京大学 2009 年研究生入学考试试题高等代数}
\begin{CJK}{UTF8}{mj}李扬\end{CJK}

\begin{CJK}{UTF8}{mj}微信公众号\end{CJK}: sxkyliyang

\begin{CJK}{UTF8}{mj}一\end{CJK}. \begin{CJK}{UTF8}{mj}判断题\end{CJK}(\begin{CJK}{UTF8}{mj}每小题\end{CJK} 4 \begin{CJK}{UTF8}{mj}分\end{CJK}, \begin{CJK}{UTF8}{mj}共\end{CJK} 40 \begin{CJK}{UTF8}{mj}分\end{CJK})

\begin{CJK}{UTF8}{mj}判断下列陈述是否正确\end{CJK}, \begin{CJK}{UTF8}{mj}并说明理由\end{CJK}.

\begin{enumerate}
  \item \begin{CJK}{UTF8}{mj}设\end{CJK} $f(\lambda)$ \begin{CJK}{UTF8}{mj}是数域\end{CJK} $P$ \begin{CJK}{UTF8}{mj}上最高次项的系数为\end{CJK} 1 \begin{CJK}{UTF8}{mj}的\end{CJK} $n(\geqslant 1)$ \begin{CJK}{UTF8}{mj}次的多项式\end{CJK}, \begin{CJK}{UTF8}{mj}则\end{CJK} $f(\lambda)$ \begin{CJK}{UTF8}{mj}一定是某个\end{CJK} $n$ \begin{CJK}{UTF8}{mj}阶方阵\end{CJK} $A$ \begin{CJK}{UTF8}{mj}的特\end{CJK} \begin{CJK}{UTF8}{mj}征多项式\end{CJK}.

  \item \begin{CJK}{UTF8}{mj}欧式空间中保持距离的变换一定是正交变换\end{CJK}.

  \item \begin{CJK}{UTF8}{mj}设\end{CJK} $A$ \begin{CJK}{UTF8}{mj}是\end{CJK} $n$ \begin{CJK}{UTF8}{mj}阶矩阵\end{CJK}, \begin{CJK}{UTF8}{mj}则有可能存在正交矩阵\end{CJK} $T$, \begin{CJK}{UTF8}{mj}使得\end{CJK} $T^{-1} A T$ \begin{CJK}{UTF8}{mj}为对角矩阵\end{CJK}.

  \item \begin{CJK}{UTF8}{mj}设\end{CJK} $\sigma, \tau$ \begin{CJK}{UTF8}{mj}是线性空间\end{CJK} $V$ \begin{CJK}{UTF8}{mj}的两个线性变换\end{CJK}, \begin{CJK}{UTF8}{mj}则\end{CJK} $\sigma \tau-\tau \sigma$ \begin{CJK}{UTF8}{mj}一定不是恒等变换\end{CJK}.

  \item \begin{CJK}{UTF8}{mj}设\end{CJK} $m(\lambda)$ \begin{CJK}{UTF8}{mj}是\end{CJK} $n$ \begin{CJK}{UTF8}{mj}阶矩阵\end{CJK} $A$ \begin{CJK}{UTF8}{mj}的最小多项式\end{CJK}, \begin{CJK}{UTF8}{mj}则\end{CJK} $A$ \begin{CJK}{UTF8}{mj}的特征根一定是\end{CJK} $m(\lambda)$ \begin{CJK}{UTF8}{mj}的根\end{CJK}.

  \item \begin{CJK}{UTF8}{mj}如果\end{CJK} $n$ \begin{CJK}{UTF8}{mj}阶矩阵\end{CJK} $A$ \begin{CJK}{UTF8}{mj}和\end{CJK} $B$ \begin{CJK}{UTF8}{mj}的最小多项式相等并且等于它们的特征多项式\end{CJK}, \begin{CJK}{UTF8}{mj}则\end{CJK} $A$ \begin{CJK}{UTF8}{mj}和\end{CJK} $B$ \begin{CJK}{UTF8}{mj}相似\end{CJK}.

  \item \begin{CJK}{UTF8}{mj}如果\end{CJK} $n$ \begin{CJK}{UTF8}{mj}阶可逆方阵\end{CJK} $A$ \begin{CJK}{UTF8}{mj}和\end{CJK} $B$ \begin{CJK}{UTF8}{mj}的伴随矩阵相等\end{CJK}, \begin{CJK}{UTF8}{mj}则\end{CJK} $A=B$.

  \item \begin{CJK}{UTF8}{mj}如果\end{CJK} $A$ \begin{CJK}{UTF8}{mj}是\end{CJK} $n$ \begin{CJK}{UTF8}{mj}阶正定矩阵\end{CJK}, \begin{CJK}{UTF8}{mj}则\end{CJK} $A$ \begin{CJK}{UTF8}{mj}的主对角线上的元素全大于零\end{CJK}.

  \item \begin{CJK}{UTF8}{mj}设\end{CJK} $f(X)=X^{T} A X$ \begin{CJK}{UTF8}{mj}是一\end{CJK} $n$ \begin{CJK}{UTF8}{mj}元实二次型\end{CJK}, \begin{CJK}{UTF8}{mj}则\end{CJK} $f(X)=0$ \begin{CJK}{UTF8}{mj}的解集合是\end{CJK} $\mathbb{R}^{n}$ \begin{CJK}{UTF8}{mj}的子空间\end{CJK}.

\end{enumerate}
\begin{CJK}{UTF8}{mj}二\end{CJK}. \begin{CJK}{UTF8}{mj}填空题\end{CJK}(\begin{CJK}{UTF8}{mj}每小题\end{CJK} 6 \begin{CJK}{UTF8}{mj}分\end{CJK}, \begin{CJK}{UTF8}{mj}共\end{CJK} 30 \begin{CJK}{UTF8}{mj}分\end{CJK}) \begin{CJK}{UTF8}{mj}成有理数域上的矩阵\end{CJK}, \begin{CJK}{UTF8}{mj}其有理标准型为\end{CJK}

\begin{enumerate}
  \setcounter{enumi}{2}
  \item \begin{CJK}{UTF8}{mj}设\end{CJK} 4 \begin{CJK}{UTF8}{mj}级数字矩阵\end{CJK} $A$ \begin{CJK}{UTF8}{mj}的最小多项式为\end{CJK} $(\lambda+1)^{3}$, \begin{CJK}{UTF8}{mj}则\end{CJK} $A$ \begin{CJK}{UTF8}{mj}的全部不变因子为\end{CJK} \begin{CJK}{UTF8}{mj}四\end{CJK}. (15 \begin{CJK}{UTF8}{mj}分\end{CJK})
\end{enumerate}
(1) \begin{CJK}{UTF8}{mj}令\end{CJK} $\sigma$ \begin{CJK}{UTF8}{mj}为\end{CJK} $n$ \begin{CJK}{UTF8}{mj}维线性空间\end{CJK} $V$ \begin{CJK}{UTF8}{mj}的线性变换\end{CJK}, $f(\lambda)$ \begin{CJK}{UTF8}{mj}为\end{CJK} $\sigma$ \begin{CJK}{UTF8}{mj}的最小多项式\end{CJK}. \begin{CJK}{UTF8}{mj}证明\end{CJK}: \begin{CJK}{UTF8}{mj}如果\end{CJK} $f(\lambda)=g(\lambda) h(\lambda)$ \begin{CJK}{UTF8}{mj}且\end{CJK} $g(\lambda)$ \begin{CJK}{UTF8}{mj}与\end{CJK} $h(\lambda)$ \begin{CJK}{UTF8}{mj}互素\end{CJK}, \begin{CJK}{UTF8}{mj}则\end{CJK} $V=L_{1} \oplus L_{2}$, \begin{CJK}{UTF8}{mj}其中\end{CJK}
$$
L_{1}=\{\alpha \in V \mid g(\sigma) \alpha=0\}, L_{2}=\{\alpha \in V \mid h(\sigma) \alpha=0\}
$$
(2)\begin{CJK}{UTF8}{mj}设\end{CJK} 3 \begin{CJK}{UTF8}{mj}维线性空间\end{CJK} $V$ \begin{CJK}{UTF8}{mj}的线性变换\end{CJK} $\sigma$ \begin{CJK}{UTF8}{mj}在一组基\end{CJK} $e_{1}, e_{2}, e_{3}$ \begin{CJK}{UTF8}{mj}下的矩阵为\end{CJK} $\left(\begin{array}{ccc}1 & 0 & 0 \\ 1 & 2 & 1 \\ -1 & 0 & 1\end{array}\right)$. \begin{CJK}{UTF8}{mj}求\end{CJK} $\sigma$ \begin{CJK}{UTF8}{mj}的最小多项式\end{CJK} $f(\lambda)$, \begin{CJK}{UTF8}{mj}并对于\end{CJK} $f(\lambda)$ \begin{CJK}{UTF8}{mj}的一次因式方幂的分解式将\end{CJK} $V$ \begin{CJK}{UTF8}{mj}分解为直和形式\end{CJK}.

\begin{CJK}{UTF8}{mj}五\end{CJK}. (15 \begin{CJK}{UTF8}{mj}分\end{CJK}) \begin{CJK}{UTF8}{mj}设\end{CJK} $A$ \begin{CJK}{UTF8}{mj}和\end{CJK} $C$ \begin{CJK}{UTF8}{mj}都是\end{CJK} $n$ \begin{CJK}{UTF8}{mj}阶正定矩阵\end{CJK}, $B$ \begin{CJK}{UTF8}{mj}是矩阵方程\end{CJK} $A X+X A=C$ \begin{CJK}{UTF8}{mj}的唯一解\end{CJK}. \begin{CJK}{UTF8}{mj}证明\end{CJK}:

(1) $B$ \begin{CJK}{UTF8}{mj}是实对称的\end{CJK}.

(2) $B$ \begin{CJK}{UTF8}{mj}是正定的\end{CJK}.

\begin{CJK}{UTF8}{mj}六\end{CJK}. (15 \begin{CJK}{UTF8}{mj}分\end{CJK}) $n$ \begin{CJK}{UTF8}{mj}维欧式空间的两组基\end{CJK} $e_{1}, e_{2}, \cdots, e_{n}$ \begin{CJK}{UTF8}{mj}和\end{CJK} $f_{1}, f_{2}, \cdots, f_{n}$ \begin{CJK}{UTF8}{mj}称为对偶基\end{CJK}, \begin{CJK}{UTF8}{mj}如果\end{CJK}
$$
\left(e_{i}, f_{j}\right)= \begin{cases}1, & i=j \\ 0, & i \neq j\end{cases}
$$
(1) \begin{CJK}{UTF8}{mj}证明\end{CJK}: \begin{CJK}{UTF8}{mj}对\end{CJK} $V$ \begin{CJK}{UTF8}{mj}的任一组基\end{CJK} $e_{1}, e_{2}, \cdots, e_{n}$, \begin{CJK}{UTF8}{mj}其对偶基存在并且唯一确定\end{CJK}.

(2) \begin{CJK}{UTF8}{mj}设\end{CJK} $e_{1}=(1,0,0), e_{2}=(1,1,0), e_{3}=(1,1,1)$ \begin{CJK}{UTF8}{mj}为\end{CJK} $\mathbb{R}^{3}$ \begin{CJK}{UTF8}{mj}的一组基\end{CJK}, \begin{CJK}{UTF8}{mj}试求\end{CJK} $e_{1}, e_{2}, e_{3}$ \begin{CJK}{UTF8}{mj}的对偶基\end{CJK}.

\begin{CJK}{UTF8}{mj}七\end{CJK}. ( 15 \begin{CJK}{UTF8}{mj}分\end{CJK}) \begin{CJK}{UTF8}{mj}设\end{CJK} $\alpha, \beta$ \begin{CJK}{UTF8}{mj}为\end{CJK} $n$ \begin{CJK}{UTF8}{mj}维欧式空间\end{CJK} $\mathbb{R}^{n}$ \begin{CJK}{UTF8}{mj}中的两个非零列向量\end{CJK}. \begin{CJK}{UTF8}{mj}则\end{CJK} $\alpha^{T} \beta>0$ \begin{CJK}{UTF8}{mj}的充分必要条件是存在正定矩阵\end{CJK} $A$ \begin{CJK}{UTF8}{mj}使得\end{CJK} $\beta=A \alpha$.

\begin{CJK}{UTF8}{mj}八\end{CJK}. ( 10 \begin{CJK}{UTF8}{mj}分\end{CJK}) \begin{CJK}{UTF8}{mj}设\end{CJK} $A$ \begin{CJK}{UTF8}{mj}为\end{CJK} $n$ \begin{CJK}{UTF8}{mj}阶复矩阵\end{CJK}. \begin{CJK}{UTF8}{mj}证明\end{CJK}: \begin{CJK}{UTF8}{mj}存在一个\end{CJK} $n$ \begin{CJK}{UTF8}{mj}维向量\end{CJK} $\alpha$, \begin{CJK}{UTF8}{mj}使得\end{CJK} $\alpha, A \alpha, \cdots, A^{n-1} \alpha$ \begin{CJK}{UTF8}{mj}线性无关的充分必要条\end{CJK} \begin{CJK}{UTF8}{mj}件是\end{CJK} $A$ \begin{CJK}{UTF8}{mj}的每一个特征根恰有一个线性无关的特征向量\end{CJK}.

\section{2. 南京大学 2010 年研究生入学考试试题高等代数}
\begin{CJK}{UTF8}{mj}李扬\end{CJK}

\begin{CJK}{UTF8}{mj}微信公众号\end{CJK}: sxkyliyang

\begin{CJK}{UTF8}{mj}一\end{CJK}. \begin{CJK}{UTF8}{mj}判断题\end{CJK}(\begin{CJK}{UTF8}{mj}每小题\end{CJK} 2 \begin{CJK}{UTF8}{mj}分\end{CJK}, \begin{CJK}{UTF8}{mj}共\end{CJK} 40 \begin{CJK}{UTF8}{mj}分\end{CJK})

\begin{CJK}{UTF8}{mj}判断下列陈述是否正确\end{CJK}. \begin{CJK}{UTF8}{mj}若正确\end{CJK}, \begin{CJK}{UTF8}{mj}请在括号内打\end{CJK} “ $+”$; \begin{CJK}{UTF8}{mj}若错误\end{CJK}, \begin{CJK}{UTF8}{mj}请在括号内打\end{CJK} “\begin{CJK}{UTF8}{mj}一\end{CJK}”.

\begin{enumerate}
  \item \begin{CJK}{UTF8}{mj}设\end{CJK} $f(x), g(x)$ \begin{CJK}{UTF8}{mj}都是数域\end{CJK} $P$ \begin{CJK}{UTF8}{mj}上多项式\end{CJK}. \begin{CJK}{UTF8}{mj}如果\end{CJK} $f(x) \dagger g(x)$, \begin{CJK}{UTF8}{mj}则存在数域\end{CJK} $P$ \begin{CJK}{UTF8}{mj}上不可约多项式\end{CJK} $p(x)$ \begin{CJK}{UTF8}{mj}使得\end{CJK} $p(x) \mid f(x)$, \begin{CJK}{UTF8}{mj}但是\end{CJK} $p(x) \dagger g(x)$.

  \item \begin{CJK}{UTF8}{mj}设\end{CJK} $P$ \begin{CJK}{UTF8}{mj}是数域\end{CJK}, $f(x) \in P[x]$. \begin{CJK}{UTF8}{mj}如果\end{CJK} $\left(f^{\prime}(x), f^{\prime \prime}(x)\right)=1$, \begin{CJK}{UTF8}{mj}则\end{CJK} $f(x)$ \begin{CJK}{UTF8}{mj}的重因式都是\end{CJK} 2 \begin{CJK}{UTF8}{mj}重因式\end{CJK}.

  \item \begin{CJK}{UTF8}{mj}多项式\end{CJK} $x^{4}+1$ \begin{CJK}{UTF8}{mj}在实数域上不可约\end{CJK}.

  \item \begin{CJK}{UTF8}{mj}设向量组\end{CJK} $\alpha_{i}=\left(1, t_{i}, t_{i}^{2}, \cdots, t_{i}^{n-1}\right), i=1,2, \cdots, n$, \begin{CJK}{UTF8}{mj}其中\end{CJK} $t_{1}, t_{2}, \cdots, t_{n}$ \begin{CJK}{UTF8}{mj}为互不相同的数\end{CJK}, \begin{CJK}{UTF8}{mj}则任一\end{CJK} $n$ \begin{CJK}{UTF8}{mj}维\end{CJK} \begin{CJK}{UTF8}{mj}向量都可由\end{CJK} $\alpha_{1}, \alpha_{2}, \cdots, \alpha_{n}$ \begin{CJK}{UTF8}{mj}线性表出\end{CJK}.

  \item \begin{CJK}{UTF8}{mj}设\end{CJK} $\alpha_{1}, \alpha_{2}, \cdots, \alpha_{r}$ \begin{CJK}{UTF8}{mj}是\end{CJK} $n$ \begin{CJK}{UTF8}{mj}维列向量\end{CJK}, $A$ \begin{CJK}{UTF8}{mj}是\end{CJK} $n$ \begin{CJK}{UTF8}{mj}级可逆矩阵\end{CJK}, \begin{CJK}{UTF8}{mj}则\end{CJK} $\alpha_{1}, \alpha_{2}, \cdots, \alpha_{r}$ \begin{CJK}{UTF8}{mj}线性相关当且仅当\end{CJK} $A \alpha_{1}, A \alpha_{2}, \cdots, A \alpha_{r}$ \begin{CJK}{UTF8}{mj}线性相关\end{CJK}.

  \item \begin{CJK}{UTF8}{mj}设\end{CJK} $\alpha_{1}, \alpha_{2}, \cdots, \alpha_{r}$ \begin{CJK}{UTF8}{mj}都是非齐次线性方程组\end{CJK} $A X=\beta$ \begin{CJK}{UTF8}{mj}的解\end{CJK}, \begin{CJK}{UTF8}{mj}则\end{CJK} $\alpha_{1}, \alpha_{2}, \cdots, \alpha_{r}$ \begin{CJK}{UTF8}{mj}的任一线性组合还是该方\end{CJK} \begin{CJK}{UTF8}{mj}程组的解\end{CJK}.

  \item \begin{CJK}{UTF8}{mj}设\end{CJK} $A$ \begin{CJK}{UTF8}{mj}为\end{CJK} $s \times n$ \begin{CJK}{UTF8}{mj}矩阵\end{CJK}, \begin{CJK}{UTF8}{mj}则\end{CJK} $A$ \begin{CJK}{UTF8}{mj}的秩小于等于\end{CJK} $r(r \geqslant 1)$ \begin{CJK}{UTF8}{mj}当且仅当\end{CJK} $A$ \begin{CJK}{UTF8}{mj}中所有\end{CJK} $r+1$ \begin{CJK}{UTF8}{mj}级子式都是零\end{CJK}.

  \item \begin{CJK}{UTF8}{mj}设\end{CJK} $A$ \begin{CJK}{UTF8}{mj}为\end{CJK} $n$ \begin{CJK}{UTF8}{mj}级方阵\end{CJK} $(n \geqslant 2)$, \begin{CJK}{UTF8}{mj}则\end{CJK} $|A| \neq 0$ \begin{CJK}{UTF8}{mj}当且仅当\end{CJK} $A$ \begin{CJK}{UTF8}{mj}的行向量组线性无关\end{CJK}.

  \item \begin{CJK}{UTF8}{mj}两个\end{CJK} $s \times n$ \begin{CJK}{UTF8}{mj}矩阵等价当且仅当它们的秩相同\end{CJK}.

  \item \begin{CJK}{UTF8}{mj}方阵\end{CJK} $A$ \begin{CJK}{UTF8}{mj}可逆当且仅当\end{CJK} $A$ \begin{CJK}{UTF8}{mj}可表为有限个初等矩阵的乘积\end{CJK}.

  \item \begin{CJK}{UTF8}{mj}设\end{CJK} $A, B$ \begin{CJK}{UTF8}{mj}为\end{CJK} $n$ \begin{CJK}{UTF8}{mj}级方阵\end{CJK} $(n>1)$, \begin{CJK}{UTF8}{mj}则\end{CJK} $A^{3}-B^{3}=(A-B)\left(A^{2}+A B+B^{2}\right)$.

  \item \begin{CJK}{UTF8}{mj}数域\end{CJK} $P$ \begin{CJK}{UTF8}{mj}上任意方阵\end{CJK} $A$ \begin{CJK}{UTF8}{mj}与其转置\end{CJK} $A^{\prime}$ \begin{CJK}{UTF8}{mj}是相似的\end{CJK}.

  \item \begin{CJK}{UTF8}{mj}设\end{CJK} $A$ \begin{CJK}{UTF8}{mj}是不可逆实方阵\end{CJK}, \begin{CJK}{UTF8}{mj}其伴随矩阵为\end{CJK} $A^{*}$, \begin{CJK}{UTF8}{mj}并且\end{CJK} $A^{\prime}=A^{*}$, \begin{CJK}{UTF8}{mj}则\end{CJK} $A=0$.

  \item \begin{CJK}{UTF8}{mj}数域\end{CJK} $P$ \begin{CJK}{UTF8}{mj}上方阵\end{CJK} $A$ \begin{CJK}{UTF8}{mj}的特征根都是\end{CJK} $A$ \begin{CJK}{UTF8}{mj}的最小多项式的根\end{CJK}.

  \item \begin{CJK}{UTF8}{mj}欧式空间中保持向量长度不变的变换是正交变换\end{CJK}.

  \item \begin{CJK}{UTF8}{mj}如果\end{CJK} $A$ \begin{CJK}{UTF8}{mj}是\end{CJK} $n$ \begin{CJK}{UTF8}{mj}级实矩阵\end{CJK}, \begin{CJK}{UTF8}{mj}并且存在\end{CJK} $n$ \begin{CJK}{UTF8}{mj}级实矩阵\end{CJK} $C$ \begin{CJK}{UTF8}{mj}使得\end{CJK} $A=C^{\prime} C$, \begin{CJK}{UTF8}{mj}则\end{CJK} $A$ \begin{CJK}{UTF8}{mj}是正定矩阵\end{CJK}.

  \item \begin{CJK}{UTF8}{mj}欧式空间中不同基的度量矩阵是不同的\end{CJK}.

  \item \begin{CJK}{UTF8}{mj}设\end{CJK} $A$ \begin{CJK}{UTF8}{mj}是\end{CJK} $n$ \begin{CJK}{UTF8}{mj}级实对称矩阵\end{CJK}, \begin{CJK}{UTF8}{mj}则存在\end{CJK} $n$ \begin{CJK}{UTF8}{mj}级实对称矩阵\end{CJK} $B$ \begin{CJK}{UTF8}{mj}使得\end{CJK} $A=B^{2}$.

  \item \begin{CJK}{UTF8}{mj}设\end{CJK} $A$ \begin{CJK}{UTF8}{mj}是\end{CJK} 3 \begin{CJK}{UTF8}{mj}级正交矩阵\end{CJK}, \begin{CJK}{UTF8}{mj}则\end{CJK} $A$ \begin{CJK}{UTF8}{mj}的对角线元素之和为\end{CJK} $-1$ \begin{CJK}{UTF8}{mj}当且仅当\end{CJK} $-1$ \begin{CJK}{UTF8}{mj}是\end{CJK} $A$ \begin{CJK}{UTF8}{mj}的特征值\end{CJK}.

\end{enumerate}
\begin{CJK}{UTF8}{mj}二\end{CJK}. \begin{CJK}{UTF8}{mj}填空题\end{CJK}(\begin{CJK}{UTF8}{mj}每小题\end{CJK} 4 \begin{CJK}{UTF8}{mj}分\end{CJK}, \begin{CJK}{UTF8}{mj}共\end{CJK} 40 \begin{CJK}{UTF8}{mj}分\end{CJK})
$$
\text { 1. 设行列式 }\left|\begin{array}{cccc}
a_{11} & a_{12} & a_{13} & a_{14} \\
1 & 2 & 2 & 2 \\
1 & 3 & 4 & 5 \\
1 & 5 & 10 & 17
\end{array}\right| \text { 中元素 } a_{1 j} \text { 的代数余子式为 } A_{1 j}(j=1,2,3,4) \text {, 则 } A_{11}+A_{12}+A_{13}+
$$

\begin{enumerate}
  \setcounter{enumi}{2}
  \item \begin{CJK}{UTF8}{mj}设\end{CJK} $\alpha_{i}=\left(a_{i 1}, a_{i 2}, a_{i 3}, a_{i 4}\right), i=1,2,3 ; \beta_{j}=\left(a_{1 j}, a_{2 j}, a_{3 j}\right), j=1,2,3$, 4. \begin{CJK}{UTF8}{mj}如果向量组\end{CJK} $\alpha_{1}, \alpha_{2}, \alpha_{3}$ \begin{CJK}{UTF8}{mj}线性无\end{CJK} \begin{CJK}{UTF8}{mj}关\end{CJK}, \begin{CJK}{UTF8}{mj}则向量组\end{CJK} $\beta_{1}, \beta_{2}, \beta_{3}, \beta_{4}$ \begin{CJK}{UTF8}{mj}的秩为\end{CJK}

  \item \begin{CJK}{UTF8}{mj}设\end{CJK} $\alpha_{1}, \alpha_{2}, \alpha_{3}$ \begin{CJK}{UTF8}{mj}线性无关\end{CJK}, $\beta_{1}=3 \alpha_{1}+(k+1) \alpha_{2}+5 \alpha_{3}, \beta_{2}=k \alpha_{1}+\alpha_{2}+\alpha_{3}, \beta_{3}=k \alpha_{2}+4 \alpha_{3}$, \begin{CJK}{UTF8}{mj}则\end{CJK} $\beta_{1}, \beta_{2}, \beta_{3}$ \begin{CJK}{UTF8}{mj}线性相关的充要条件是\end{CJK} $k=$ 4. \begin{CJK}{UTF8}{mj}设\end{CJK} $\alpha=\left(a_{1}, a_{2}, \cdots, a_{n}\right), E$ \begin{CJK}{UTF8}{mj}是\end{CJK} $n$ \begin{CJK}{UTF8}{mj}级单位矩阵\end{CJK}, \begin{CJK}{UTF8}{mj}则\end{CJK} $\left|E+\alpha^{\prime} \alpha\right|=$

\end{enumerate}
5 . \begin{CJK}{UTF8}{mj}设\end{CJK} $A$ \begin{CJK}{UTF8}{mj}为\end{CJK} $n$ \begin{CJK}{UTF8}{mj}级方阵并且\end{CJK} $|A|=-5, A^{2}-3 A+\frac{1}{5} A A^{*}=0$, \begin{CJK}{UTF8}{mj}则\end{CJK} $A^{-1}=$

6 . \begin{CJK}{UTF8}{mj}设\end{CJK} 3 \begin{CJK}{UTF8}{mj}级方阵\end{CJK} $A$ \begin{CJK}{UTF8}{mj}的秩为\end{CJK} $2, B=\left(\begin{array}{cc}1 & 3 \\ 3 & k \\ 5 & 15\end{array}\right)$, \begin{CJK}{UTF8}{mj}并且\end{CJK} $A B=0$, \begin{CJK}{UTF8}{mj}则\end{CJK} $k=$

\begin{enumerate}
  \setcounter{enumi}{7}
  \item \begin{CJK}{UTF8}{mj}设\end{CJK} $A$ \begin{CJK}{UTF8}{mj}为\end{CJK} $n$ \begin{CJK}{UTF8}{mj}级方阵\end{CJK}, $E$ \begin{CJK}{UTF8}{mj}为\end{CJK} $n$ \begin{CJK}{UTF8}{mj}级单位矩阵\end{CJK}, $A \neq E$, \begin{CJK}{UTF8}{mj}但\end{CJK} $A^{2}=E$, \begin{CJK}{UTF8}{mj}则\end{CJK} $|A+E|=$

  \item \begin{CJK}{UTF8}{mj}矩阵方程\end{CJK} $\left(\begin{array}{lll}1 & 1 & 0 \\ 0 & 1 & 1 \\ 0 & 0 & 1\end{array}\right) X=\left(\begin{array}{ccc}2 & 1 & 1 \\ 1 & -1 & 1 \\ 1 & 1 & 1\end{array}\right)$ \begin{CJK}{UTF8}{mj}的解为\end{CJK}

  \item \begin{CJK}{UTF8}{mj}设\end{CJK} $n$ \begin{CJK}{UTF8}{mj}级方阵\end{CJK} $A$ \begin{CJK}{UTF8}{mj}的最小多项式为\end{CJK} $f(x)$ \begin{CJK}{UTF8}{mj}并且\end{CJK} $f(0) \neq 0$, \begin{CJK}{UTF8}{mj}则矩阵\end{CJK} $\left(\begin{array}{ccc}A & -A & 0 \\ A & -A & 0 \\ 0 & 0 & A\end{array}\right)$ \begin{CJK}{UTF8}{mj}的最小多项式为\end{CJK}

  \item \begin{CJK}{UTF8}{mj}设\end{CJK} $\alpha_{1}, \alpha_{2}, \cdots, \alpha_{n}$ \begin{CJK}{UTF8}{mj}是数域\end{CJK} $P$ \begin{CJK}{UTF8}{mj}上的线性空间\end{CJK} $V$ \begin{CJK}{UTF8}{mj}的一组秩为\end{CJK} $r$ \begin{CJK}{UTF8}{mj}的向量组\end{CJK}, \begin{CJK}{UTF8}{mj}则使得\end{CJK} $k_{1} \alpha_{1}+k_{2} \alpha_{2}+\cdots+k_{n} \alpha_{n}=0$ \begin{CJK}{UTF8}{mj}的\end{CJK} $n$ \begin{CJK}{UTF8}{mj}维向量\end{CJK} $\left(k_{1}, k_{2}, \cdots, k_{n}\right)$ \begin{CJK}{UTF8}{mj}的全体构成的集合是\end{CJK} $P^{n}$ \begin{CJK}{UTF8}{mj}的\end{CJK} \begin{CJK}{UTF8}{mj}维子空间\end{CJK}.

\end{enumerate}
\begin{CJK}{UTF8}{mj}三\end{CJK}. (10 \begin{CJK}{UTF8}{mj}分\end{CJK}) \begin{CJK}{UTF8}{mj}求\end{CJK} $t$ \begin{CJK}{UTF8}{mj}值使\end{CJK} $f(x)=x^{3}+t x^{2}+3 x+1$ \begin{CJK}{UTF8}{mj}有重根\end{CJK}, \begin{CJK}{UTF8}{mj}并求出重根及其重数\end{CJK}.

\begin{CJK}{UTF8}{mj}四\end{CJK}. (15 \begin{CJK}{UTF8}{mj}分\end{CJK}) \begin{CJK}{UTF8}{mj}设\end{CJK} 3 \begin{CJK}{UTF8}{mj}级实对称矩阵\end{CJK} $A$ \begin{CJK}{UTF8}{mj}的秩为\end{CJK} $2, \lambda_{1}=\lambda_{2}=6$ \begin{CJK}{UTF8}{mj}是\end{CJK} $A$ \begin{CJK}{UTF8}{mj}的二重特征值\end{CJK}, $\alpha_{1}=(1,1,0)^{\prime}, \alpha_{2}=(2,1,1)^{\prime}$ \begin{CJK}{UTF8}{mj}都是\end{CJK} $A$ \begin{CJK}{UTF8}{mj}的属于特征值\end{CJK} 6 \begin{CJK}{UTF8}{mj}的特征向量\end{CJK}.

\begin{enumerate}
  \item \begin{CJK}{UTF8}{mj}求\end{CJK} $A$ \begin{CJK}{UTF8}{mj}的另一特征值及其全部特征向量\end{CJK}.

  \item \begin{CJK}{UTF8}{mj}求矩阵\end{CJK} $A$.

\end{enumerate}
\begin{CJK}{UTF8}{mj}五\end{CJK}. ( 15 \begin{CJK}{UTF8}{mj}分\end{CJK}) \begin{CJK}{UTF8}{mj}已知二次型\end{CJK} $f\left(x_{1}, x_{2}, x_{3}\right)=2 x_{1}^{2}+3 x_{2}^{2}+3 x_{3}^{2}+2 t x_{2} x_{3}$ (\begin{CJK}{UTF8}{mj}其中\end{CJK} $t>0$ ) \begin{CJK}{UTF8}{mj}可以通过正交变换化为标准形\end{CJK} $y_{1}^{2}+2 y_{2}^{2}+5 y_{3}^{2}$. \begin{CJK}{UTF8}{mj}求\end{CJK} $t$ \begin{CJK}{UTF8}{mj}和所用的正交变换\end{CJK}.

\begin{CJK}{UTF8}{mj}六\end{CJK}. (10 \begin{CJK}{UTF8}{mj}分\end{CJK}) \begin{CJK}{UTF8}{mj}设\end{CJK} $n$ \begin{CJK}{UTF8}{mj}级方阵\end{CJK} $A$ \begin{CJK}{UTF8}{mj}满足条件\end{CJK} $A^{2}+A+E=0$, \begin{CJK}{UTF8}{mj}其中\end{CJK} $E$ \begin{CJK}{UTF8}{mj}是\end{CJK} $n$ \begin{CJK}{UTF8}{mj}级单位矩阵\end{CJK}. \begin{CJK}{UTF8}{mj}证明\end{CJK}: \begin{CJK}{UTF8}{mj}对于任意实数\end{CJK} $a, A-a E$ \begin{CJK}{UTF8}{mj}都可逆\end{CJK}.

\begin{CJK}{UTF8}{mj}七\end{CJK}. (10 \begin{CJK}{UTF8}{mj}分\end{CJK}) \begin{CJK}{UTF8}{mj}设\end{CJK} $\alpha, \beta$ \begin{CJK}{UTF8}{mj}都是实数域上的\end{CJK} $n$ \begin{CJK}{UTF8}{mj}维列向量\end{CJK}, \begin{CJK}{UTF8}{mj}并且\end{CJK} $\alpha \neq 0$. \begin{CJK}{UTF8}{mj}请构造一个\end{CJK} $n$ \begin{CJK}{UTF8}{mj}级方阵\end{CJK} $A$ \begin{CJK}{UTF8}{mj}使得\end{CJK} $A$ \begin{CJK}{UTF8}{mj}满足下面两个条件\end{CJK}:

\begin{enumerate}
  \item $A \alpha=\beta$,

  \item \begin{CJK}{UTF8}{mj}对于方程\end{CJK} $\alpha^{\prime} X=0$ \begin{CJK}{UTF8}{mj}的任意一个解\end{CJK} $X$ \begin{CJK}{UTF8}{mj}都有\end{CJK} $A X=X$.

\end{enumerate}
\begin{CJK}{UTF8}{mj}八\end{CJK}. ( 10 \begin{CJK}{UTF8}{mj}分\end{CJK}) \begin{CJK}{UTF8}{mj}设\end{CJK} $A$ \begin{CJK}{UTF8}{mj}为任意复方阵\end{CJK}. \begin{CJK}{UTF8}{mj}证明\end{CJK}: \begin{CJK}{UTF8}{mj}存在与对角矩阵相似的方阵\end{CJK} $S$ \begin{CJK}{UTF8}{mj}以及幂零方阵\end{CJK} $N$ \begin{CJK}{UTF8}{mj}使得\end{CJK} $A=S+N$ \begin{CJK}{UTF8}{mj}并且\end{CJK} $S N=N S$.

\section{3. 南京大学 2011 年研究生入学考试试题高等代数}
\begin{CJK}{UTF8}{mj}李扬\end{CJK}

\begin{CJK}{UTF8}{mj}微信公众号\end{CJK}: sxkyliyang

\begin{CJK}{UTF8}{mj}一\end{CJK}. \begin{CJK}{UTF8}{mj}判断题\end{CJK}(\begin{CJK}{UTF8}{mj}每小题\end{CJK} 2 \begin{CJK}{UTF8}{mj}分\end{CJK}, \begin{CJK}{UTF8}{mj}共\end{CJK} 40 \begin{CJK}{UTF8}{mj}分\end{CJK})

\begin{CJK}{UTF8}{mj}判断下列陈述是否正确\end{CJK}. \begin{CJK}{UTF8}{mj}若正确\end{CJK}, \begin{CJK}{UTF8}{mj}在括号内打\end{CJK} “ $+;$ \begin{CJK}{UTF8}{mj}若错误\end{CJK}, \begin{CJK}{UTF8}{mj}在括号内打\end{CJK} “-.”

\begin{enumerate}
  \item \begin{CJK}{UTF8}{mj}设\end{CJK} $A, B$ \begin{CJK}{UTF8}{mj}为\end{CJK} $n$ \begin{CJK}{UTF8}{mj}级方阵\end{CJK} $(n>1)$, \begin{CJK}{UTF8}{mj}则\end{CJK} $|A+B|=|A|+|B|$.

  \item \begin{CJK}{UTF8}{mj}设\end{CJK} $A, B$ \begin{CJK}{UTF8}{mj}为\end{CJK} $n$ \begin{CJK}{UTF8}{mj}级方阵\end{CJK} $(n>1)$, \begin{CJK}{UTF8}{mj}则\end{CJK} $A^{2}-B^{2}=(A-B)(A+B)$.

  \item \begin{CJK}{UTF8}{mj}任意\end{CJK} $n$ \begin{CJK}{UTF8}{mj}级方阵都可表为一个\end{CJK} $n$ \begin{CJK}{UTF8}{mj}级对称方阵与一个\end{CJK} $n$ \begin{CJK}{UTF8}{mj}级反对称方阵之和\end{CJK}.

  \item \begin{CJK}{UTF8}{mj}设\end{CJK} $A$ \begin{CJK}{UTF8}{mj}是实对称方阵并且\end{CJK} $A=B^{2}$, \begin{CJK}{UTF8}{mj}则\end{CJK} $B$ \begin{CJK}{UTF8}{mj}为实对称方阵\end{CJK}.

\end{enumerate}
5 . \begin{CJK}{UTF8}{mj}设\end{CJK} $A$ \begin{CJK}{UTF8}{mj}为\end{CJK} $s \times n$ \begin{CJK}{UTF8}{mj}实矩阵\end{CJK}, $A^{\prime}$ \begin{CJK}{UTF8}{mj}是\end{CJK} $A$ \begin{CJK}{UTF8}{mj}的转置\end{CJK}, \begin{CJK}{UTF8}{mj}则秩\end{CJK} $\left(A^{\prime} A\right)=$ \begin{CJK}{UTF8}{mj}秩\end{CJK} $(A)$.

\begin{enumerate}
  \setcounter{enumi}{6}
  \item \begin{CJK}{UTF8}{mj}设\end{CJK} $A, B$ \begin{CJK}{UTF8}{mj}为任意两个\end{CJK} $s \times n$ \begin{CJK}{UTF8}{mj}矩阵\end{CJK}, \begin{CJK}{UTF8}{mj}则\end{CJK} $A$ \begin{CJK}{UTF8}{mj}与\end{CJK} $B$ \begin{CJK}{UTF8}{mj}等价的充要条件是\end{CJK} $A$ \begin{CJK}{UTF8}{mj}的行向量组与\end{CJK} $B$ \begin{CJK}{UTF8}{mj}的行向量组等价\end{CJK}.

  \item \begin{CJK}{UTF8}{mj}设\end{CJK} $f\left(x_{1}, x_{2}, \cdots, x_{n}\right)$ \begin{CJK}{UTF8}{mj}与\end{CJK} $g\left(x_{1}, x_{2}, \cdots, x_{n}\right)$ \begin{CJK}{UTF8}{mj}都是实二次型\end{CJK}. \begin{CJK}{UTF8}{mj}如果它们具有相同的秩和正惯性指数\end{CJK}, \begin{CJK}{UTF8}{mj}则可经\end{CJK} \begin{CJK}{UTF8}{mj}非退化的线性替换将其中一个化为另一个\end{CJK}.

  \item \begin{CJK}{UTF8}{mj}由线性空间\end{CJK} $V$ \begin{CJK}{UTF8}{mj}中的向量\end{CJK} $\alpha_{1}, \alpha_{2} \cdots, \alpha_{n}$ \begin{CJK}{UTF8}{mj}生成的子空间是\end{CJK} $V$ \begin{CJK}{UTF8}{mj}的包含这些向量的子空间中的最小者\end{CJK}.( )

  \item \begin{CJK}{UTF8}{mj}设\end{CJK} $\alpha$ \begin{CJK}{UTF8}{mj}是有限维线性空间\end{CJK} $V$ \begin{CJK}{UTF8}{mj}中的一个非零向量\end{CJK}, \begin{CJK}{UTF8}{mj}则\end{CJK} $\alpha$ \begin{CJK}{UTF8}{mj}关于\end{CJK} $V$ \begin{CJK}{UTF8}{mj}的两组不同基的坐标是不同的\end{CJK}.

  \item \begin{CJK}{UTF8}{mj}设\end{CJK} $V_{1}, V_{2}, V_{3}$ \begin{CJK}{UTF8}{mj}是线性空间\end{CJK} $V$ \begin{CJK}{UTF8}{mj}的子空间\end{CJK}, \begin{CJK}{UTF8}{mj}则\end{CJK} $\left(V_{1}+V_{2}\right) \cap V_{3}=\left(V_{1} \cap V_{3}\right)+\left(V_{2} \cap V_{3}\right)$.

  \item \begin{CJK}{UTF8}{mj}设\end{CJK} $V$ \begin{CJK}{UTF8}{mj}是数域\end{CJK} $P$ \begin{CJK}{UTF8}{mj}上一个\end{CJK} $n$ \begin{CJK}{UTF8}{mj}维线性空间\end{CJK}, \begin{CJK}{UTF8}{mj}则\end{CJK} $V$ \begin{CJK}{UTF8}{mj}上全体线性变换也构成\end{CJK} $P$ \begin{CJK}{UTF8}{mj}上一个\end{CJK} $n$ \begin{CJK}{UTF8}{mj}维线性空间\end{CJK}. \begin{CJK}{UTF8}{mj}基使得\end{CJK} $\mathscr{A}$ \begin{CJK}{UTF8}{mj}在这组基下的矩阵是\end{CJK} $A$.

  \item $n$ \begin{CJK}{UTF8}{mj}级\end{CJK} $\lambda-$ \begin{CJK}{UTF8}{mj}矩阵\end{CJK} $A(\lambda)$ \begin{CJK}{UTF8}{mj}是可逆的当且仅当\end{CJK} $A(\lambda)$ \begin{CJK}{UTF8}{mj}的秩等于\end{CJK} $n$.

  \item \begin{CJK}{UTF8}{mj}有限维线性空间\end{CJK} $V$ \begin{CJK}{UTF8}{mj}的任意子空间都是\end{CJK} $V$ \begin{CJK}{UTF8}{mj}上某线性变换的不变子空间\end{CJK}. $(\alpha, \beta)$, \begin{CJK}{UTF8}{mj}则\end{CJK} $\sigma$ \begin{CJK}{UTF8}{mj}是\end{CJK} $V$ \begin{CJK}{UTF8}{mj}到\end{CJK} $V$ \begin{CJK}{UTF8}{mj}的同构映射\end{CJK}.

  \item \begin{CJK}{UTF8}{mj}设\end{CJK} $A$ \begin{CJK}{UTF8}{mj}是\end{CJK} $n$ \begin{CJK}{UTF8}{mj}级实对称矩阵\end{CJK}, \begin{CJK}{UTF8}{mj}则存在惟一的正交矩阵\end{CJK} $T$ \begin{CJK}{UTF8}{mj}使得\end{CJK} $T^{-1} A T$ \begin{CJK}{UTF8}{mj}为对角矩阵\end{CJK}.

\end{enumerate}
\begin{CJK}{UTF8}{mj}二\end{CJK}. \begin{CJK}{UTF8}{mj}填空题\end{CJK}(\begin{CJK}{UTF8}{mj}每小题\end{CJK} 4 \begin{CJK}{UTF8}{mj}分\end{CJK}, \begin{CJK}{UTF8}{mj}共\end{CJK} 40 \begin{CJK}{UTF8}{mj}分\end{CJK})

\begin{enumerate}
  \setcounter{enumi}{2}
  \item \begin{CJK}{UTF8}{mj}设实系数多项式\end{CJK} $f(x)=x^{3}+p x+q$ \begin{CJK}{UTF8}{mj}有一个虚根\end{CJK} $4+3 i$, \begin{CJK}{UTF8}{mj}则\end{CJK} $f(x)$ \begin{CJK}{UTF8}{mj}的其余两个根是\end{CJK}

  \item \begin{CJK}{UTF8}{mj}设\end{CJK} $f(x)=x^{4}-6 x^{2}-t x-3$, \begin{CJK}{UTF8}{mj}则当\end{CJK} $t=$ \_\begin{CJK}{UTF8}{mj}时\end{CJK}, $f(x)$ \begin{CJK}{UTF8}{mj}与\end{CJK} $f^{\prime}(x)$ \begin{CJK}{UTF8}{mj}的最大公因式是二次多项式\end{CJK}. 5. \begin{CJK}{UTF8}{mj}设行列式\end{CJK} $D=\left|\begin{array}{cccc}2 & 0 & 1 & 0 \\ 1 & 1 & 2 & 0 \\ 4 & 8 & 12 & 16 \\ 1 & 3 & 5 & 7\end{array}\right|$ \begin{CJK}{UTF8}{mj}中元素\end{CJK} $a_{i j}$ \begin{CJK}{UTF8}{mj}的代数余子式为\end{CJK} $A_{i j}(i, j=1,2,3,4)$, \begin{CJK}{UTF8}{mj}则\end{CJK} $A_{41}+2 A_{42}+$

  \item \begin{CJK}{UTF8}{mj}设\end{CJK} $n$ \begin{CJK}{UTF8}{mj}级方阵\end{CJK} $A$ \begin{CJK}{UTF8}{mj}的每一行的和为\end{CJK} 0 \begin{CJK}{UTF8}{mj}且\end{CJK} $A$ \begin{CJK}{UTF8}{mj}的秩等于\end{CJK} $n-1$, \begin{CJK}{UTF8}{mj}则齐次线性方程组\end{CJK} $A X=0$ \begin{CJK}{UTF8}{mj}的通解为\end{CJK}

  \item \begin{CJK}{UTF8}{mj}设\end{CJK} $f(x)=\left|\begin{array}{cccc}x+1 & 32 & 54 & 108 \\ 0 & x-2 & 0 & 0 \\ 0 & 72 & x+3 & 4 \\ 0 & 98 & 5 & x+4\end{array}\right|$, \begin{CJK}{UTF8}{mj}则\end{CJK} $f(x)$ \begin{CJK}{UTF8}{mj}中\end{CJK} $x^{3}$ \begin{CJK}{UTF8}{mj}的系数为\end{CJK}

  \item \begin{CJK}{UTF8}{mj}设矩阵\end{CJK} $A=\left(\begin{array}{cccc}1-a_{1}^{2} & -a_{1} a_{2} & \cdots & -a_{1} a_{n} \\ -a_{2} a_{1} & 1-a_{2}^{2} & \cdots & -a_{2} a_{n} \\ \vdots & \vdots & & \vdots \\ -a_{n} a_{1} & -a_{n} a_{2} & \cdots & -a_{n} a_{n}\end{array}\right)$, \begin{CJK}{UTF8}{mj}则\end{CJK} $|A|=$

  \item \begin{CJK}{UTF8}{mj}设\end{CJK} 4 \begin{CJK}{UTF8}{mj}级数字矩阵\end{CJK} $A$ \begin{CJK}{UTF8}{mj}的最小多项式为\end{CJK} $(\lambda+1)^{3}$, \begin{CJK}{UTF8}{mj}则\end{CJK} $A$ \begin{CJK}{UTF8}{mj}的全部行列式因子为\end{CJK}

  \item \begin{CJK}{UTF8}{mj}设\end{CJK} 3 \begin{CJK}{UTF8}{mj}维欧式空间一组基\end{CJK} $\alpha_{1}, \alpha_{2}, \alpha_{3}$ \begin{CJK}{UTF8}{mj}的度量矩阵为\end{CJK} $\left(\begin{array}{ccc}1 & 0 & -1 \\ 0 & 3 & 0 \\ -1 & 0 & 1\end{array}\right)$, \begin{CJK}{UTF8}{mj}则向量\end{CJK} $2 \alpha_{1}+3 \alpha_{2}-\alpha_{3}$ \begin{CJK}{UTF8}{mj}的长度为\end{CJK}

\end{enumerate}
\begin{CJK}{UTF8}{mj}三\end{CJK}. ( 10 \begin{CJK}{UTF8}{mj}分\end{CJK}) \begin{CJK}{UTF8}{mj}写出多项式\end{CJK} $f(x)=x^{4}+1$ \begin{CJK}{UTF8}{mj}在复数域\end{CJK}、\begin{CJK}{UTF8}{mj}实数域及有理数域上的标准分解式\end{CJK}, \begin{CJK}{UTF8}{mj}并说明理由\end{CJK}.

\begin{CJK}{UTF8}{mj}四\end{CJK}. ( 15 \begin{CJK}{UTF8}{mj}分\end{CJK}) \begin{CJK}{UTF8}{mj}设\end{CJK} $A=\left(\begin{array}{ccc}2 & 2 & -2 \\ 2 & 5 & -4 \\ -2 & -4 & 5\end{array}\right)$. \begin{CJK}{UTF8}{mj}试求一个正交矩阵\end{CJK} $T$ \begin{CJK}{UTF8}{mj}使得\end{CJK} $T^{\prime} A T=D$ \begin{CJK}{UTF8}{mj}为对角矩阵\end{CJK}, \begin{CJK}{UTF8}{mj}并写出此对角矩阵\end{CJK} $D$

\begin{CJK}{UTF8}{mj}五\end{CJK}. ( 15 \begin{CJK}{UTF8}{mj}分\end{CJK}) \begin{CJK}{UTF8}{mj}设\end{CJK} $n$ \begin{CJK}{UTF8}{mj}级方阵\end{CJK} $A=\left(\alpha_{1}, \alpha_{2}, \cdots, \alpha_{n}\right)$ \begin{CJK}{UTF8}{mj}的前\end{CJK} $n-1$ \begin{CJK}{UTF8}{mj}个列向量线性相关\end{CJK}, \begin{CJK}{UTF8}{mj}后\end{CJK} $n-1$ \begin{CJK}{UTF8}{mj}个列向量线性无关\end{CJK}, $\beta=\alpha_{1}+\alpha_{2}+\cdots+\alpha_{n} .$

\begin{enumerate}
  \item \begin{CJK}{UTF8}{mj}证明\end{CJK}: \begin{CJK}{UTF8}{mj}方程组\end{CJK} $A X=\beta$ \begin{CJK}{UTF8}{mj}必有无穷多解\end{CJK}.

  \item \begin{CJK}{UTF8}{mj}求方程组\end{CJK} $A X=\beta$ \begin{CJK}{UTF8}{mj}的通解\end{CJK}.

\end{enumerate}
\begin{CJK}{UTF8}{mj}六\end{CJK}. (15 \begin{CJK}{UTF8}{mj}分\end{CJK}) \begin{CJK}{UTF8}{mj}设\end{CJK} $A, B$ \begin{CJK}{UTF8}{mj}都是正定矩阵\end{CJK}.

\begin{enumerate}
  \item \begin{CJK}{UTF8}{mj}举例说明矩阵\end{CJK} $A B$ \begin{CJK}{UTF8}{mj}末必正定\end{CJK}.

  \item \begin{CJK}{UTF8}{mj}给出\end{CJK} $A B$ \begin{CJK}{UTF8}{mj}是正定矩阵的一个充要条件\end{CJK}, \begin{CJK}{UTF8}{mj}并加以证明\end{CJK}.

\end{enumerate}
\begin{CJK}{UTF8}{mj}七\end{CJK}. (15 \begin{CJK}{UTF8}{mj}分\end{CJK}) \begin{CJK}{UTF8}{mj}设\end{CJK} $m, n$ \begin{CJK}{UTF8}{mj}为正整数\end{CJK}, $d=(m, n)(m, n$ \begin{CJK}{UTF8}{mj}的最大公因式\end{CJK}), $a$ \begin{CJK}{UTF8}{mj}是非零复数\end{CJK}. \begin{CJK}{UTF8}{mj}证明\end{CJK}:
$$
\left(x^{m}+a^{m}, x^{n}+a^{n}\right)= \begin{cases}x^{d}+a^{d}, & \text { 当 } \frac{m}{d}, \frac{n}{d} \text { 都为奇数, } \\ 1, & \text { 当 } \frac{m}{d} \text { 或 } \frac{n}{d} \text { 为偶数. }\end{cases}
$$
\begin{CJK}{UTF8}{mj}其中\end{CJK} $\left(x^{m}+a^{m}, x^{n}+a^{n}\right)$ \begin{CJK}{UTF8}{mj}表示多项式\end{CJK} $x^{m}+a^{m}$ \begin{CJK}{UTF8}{mj}与\end{CJK} $x^{n}+a^{n}$ \begin{CJK}{UTF8}{mj}的最大公因式\end{CJK}.

\section{4. 南京大学 2012 年研究生入学考试试题高等代数}
\begin{CJK}{UTF8}{mj}李扬\end{CJK}

\begin{CJK}{UTF8}{mj}微信公众号\end{CJK}: sxkyliyang

\begin{CJK}{UTF8}{mj}一\end{CJK}. \begin{CJK}{UTF8}{mj}判断题\end{CJK}(\begin{CJK}{UTF8}{mj}每小题\end{CJK} 4 \begin{CJK}{UTF8}{mj}分\end{CJK}, \begin{CJK}{UTF8}{mj}共\end{CJK} 40 \begin{CJK}{UTF8}{mj}分\end{CJK})

\begin{CJK}{UTF8}{mj}判断下列陈述是否正确\end{CJK}. \begin{CJK}{UTF8}{mj}并说明理由\end{CJK}.

\begin{enumerate}
  \item \begin{CJK}{UTF8}{mj}设\end{CJK} $A, B$ \begin{CJK}{UTF8}{mj}是\end{CJK} $n$ \begin{CJK}{UTF8}{mj}级矩阵且\end{CJK} $A B=E$, \begin{CJK}{UTF8}{mj}则一定有\end{CJK} $B A=E$.

  \item \begin{CJK}{UTF8}{mj}在实数域中\end{CJK}, $\left(\begin{array}{cccc}1 & 0 & 0 & 0 \\ 0 & -8 & 0 & 0 \\ 0 & 0 & 3 & 0 \\ 0 & 0 & 0 & 5\end{array}\right)$ \begin{CJK}{UTF8}{mj}和\end{CJK} $\left(\begin{array}{cccc}1 & 0 & 0 & 0 \\ 0 & -8 & 0 & 0 \\ 0 & 0 & -3 & 0 \\ 0 & 0 & 0 & 5\end{array}\right)$ \begin{CJK}{UTF8}{mj}是合同的\end{CJK}.

  \item \begin{CJK}{UTF8}{mj}设\end{CJK} $U, W$ \begin{CJK}{UTF8}{mj}是有限维线性空间\end{CJK} $V$ \begin{CJK}{UTF8}{mj}的两个互不包含的子空间\end{CJK}, \begin{CJK}{UTF8}{mj}则\end{CJK} $\operatorname{dim}(U+W) \geqslant \operatorname{dim}(U \cap W)+2$, \begin{CJK}{UTF8}{mj}这里的\end{CJK} $\operatorname{dim}$ \begin{CJK}{UTF8}{mj}表示维数\end{CJK}.

  \item \begin{CJK}{UTF8}{mj}设\end{CJK} $\mathscr{A}, \mathscr{B}$ \begin{CJK}{UTF8}{mj}是有限维线性空间\end{CJK} $V$ \begin{CJK}{UTF8}{mj}的两个线性变换\end{CJK}, \begin{CJK}{UTF8}{mj}则\end{CJK} $\mathscr{A} \mathscr{B}-\mathscr{B} \mathscr{A}$ \begin{CJK}{UTF8}{mj}有可能是恒等变换\end{CJK}.

  \item \begin{CJK}{UTF8}{mj}对数域\end{CJK} $F$ \begin{CJK}{UTF8}{mj}上的任何一个首一的\end{CJK} $n$ \begin{CJK}{UTF8}{mj}次多项式\end{CJK} $f(x)$, \begin{CJK}{UTF8}{mj}都存在矩阵\end{CJK} $A$, \begin{CJK}{UTF8}{mj}使得\end{CJK} $A$ \begin{CJK}{UTF8}{mj}的特征多项式是\end{CJK} $f(x)$.

  \item \begin{CJK}{UTF8}{mj}如果\end{CJK} $n$ \begin{CJK}{UTF8}{mj}级对称矩阵\end{CJK} $A$ \begin{CJK}{UTF8}{mj}和\end{CJK} $B$ \begin{CJK}{UTF8}{mj}的最小多项式相等\end{CJK}, \begin{CJK}{UTF8}{mj}则\end{CJK} $A$ \begin{CJK}{UTF8}{mj}和\end{CJK} $B$ \begin{CJK}{UTF8}{mj}相似\end{CJK}.

  \item \begin{CJK}{UTF8}{mj}设\end{CJK} $\mathscr{A}, \mathscr{B}$ \begin{CJK}{UTF8}{mj}是线性变换\end{CJK}, \begin{CJK}{UTF8}{mj}而且\end{CJK} $\mathscr{A} \mathscr{B}-\mathscr{B} \mathscr{A}=\mathscr{E}$ (\begin{CJK}{UTF8}{mj}恒等变换\end{CJK}), \begin{CJK}{UTF8}{mj}则\end{CJK} $\mathscr{A}^{2} \mathscr{B}-\mathscr{B} \mathscr{A}^{2}=2 \mathscr{A}$.

  \item \begin{CJK}{UTF8}{mj}任意矩阵\end{CJK} $A$ \begin{CJK}{UTF8}{mj}的特征多项式的根一定是其最小多项式的根\end{CJK}.

  \item \begin{CJK}{UTF8}{mj}所有的有限维欧式空间都有标准正交基\end{CJK}.

  \item \begin{CJK}{UTF8}{mj}正交矩阵的特征值一定是\end{CJK} $\pm 1$.

\end{enumerate}
\begin{CJK}{UTF8}{mj}二\end{CJK}. \begin{CJK}{UTF8}{mj}填空题\end{CJK}(\begin{CJK}{UTF8}{mj}每小题\end{CJK} 6 \begin{CJK}{UTF8}{mj}分\end{CJK}, \begin{CJK}{UTF8}{mj}共\end{CJK} 30 \begin{CJK}{UTF8}{mj}分\end{CJK})

\begin{enumerate}
  \item \begin{CJK}{UTF8}{mj}设矩阵\end{CJK} $A=\left(\begin{array}{ccc}-2 & -4 & 12 \\ -2 & 0 & 6 \\ -2 & -2 & 8\end{array}\right)$, \begin{CJK}{UTF8}{mj}则其初等因子为\end{CJK} , Jordan \begin{CJK}{UTF8}{mj}标准型为\end{CJK}

  \item \begin{CJK}{UTF8}{mj}设实二次型\end{CJK} $f\left(x_{1}, x_{2}, x_{3}\right)=\left(x_{1}, x_{2}, x_{3}\right)\left(\begin{array}{ccc}0 & -1 & 8 \\ 5 & 0 & -2 \\ -10 & 0 & 3\end{array}\right)\left(\begin{array}{l}x_{1} \\ x_{2} \\ x_{3}\end{array}\right)$. \begin{CJK}{UTF8}{mj}则这个二次型的矩阵为\end{CJK} , \begin{CJK}{UTF8}{mj}符\end{CJK} 목\begin{CJK}{UTF8}{mj}差为\end{CJK}

  \item \begin{CJK}{UTF8}{mj}设\end{CJK} $A$ \begin{CJK}{UTF8}{mj}是\end{CJK} 4 \begin{CJK}{UTF8}{mj}级方阵\end{CJK}, \begin{CJK}{UTF8}{mj}它的特征值是\end{CJK} $1,2,3,4$. \begin{CJK}{UTF8}{mj}记\end{CJK} $A^{*}$ \begin{CJK}{UTF8}{mj}为\end{CJK} $A$ \begin{CJK}{UTF8}{mj}的伴随矩阵\end{CJK}, \begin{CJK}{UTF8}{mj}则\end{CJK} $\left|A^{*}\right|=$

  \item \begin{CJK}{UTF8}{mj}设\end{CJK} $A$ \begin{CJK}{UTF8}{mj}为\end{CJK} 3 \begin{CJK}{UTF8}{mj}级非零实方阵\end{CJK}, $A^{2}=0$, \begin{CJK}{UTF8}{mj}则\end{CJK} $A$ \begin{CJK}{UTF8}{mj}的秩\end{CJK} $=$

  \item \begin{CJK}{UTF8}{mj}设矩阵\end{CJK} $\left(\begin{array}{ccc}1 & -1 & 1 \\ 2 & 4 & -2 \\ -3 & -3 & a\end{array}\right)$ \begin{CJK}{UTF8}{mj}与\end{CJK} $\left(\begin{array}{ccc}2 & 0 & 0 \\ 0 & 2 & 0 \\ 0 & 0 & b\end{array}\right)$ \begin{CJK}{UTF8}{mj}相似\end{CJK}, \begin{CJK}{UTF8}{mj}则\end{CJK} $a=\square, b=$

\end{enumerate}
\begin{CJK}{UTF8}{mj}三\end{CJK}. (15 \begin{CJK}{UTF8}{mj}分\end{CJK}) \begin{CJK}{UTF8}{mj}求证一个秩\end{CJK} $k$ \begin{CJK}{UTF8}{mj}的\end{CJK} $n$ \begin{CJK}{UTF8}{mj}级矩阵\end{CJK} $A$ \begin{CJK}{UTF8}{mj}可以写成\end{CJK} $k$ \begin{CJK}{UTF8}{mj}个秩为\end{CJK} 1 \begin{CJK}{UTF8}{mj}的\end{CJK} $n$ \begin{CJK}{UTF8}{mj}级矩阵之和\end{CJK}.

\begin{CJK}{UTF8}{mj}四\end{CJK}. ( 15 \begin{CJK}{UTF8}{mj}分\end{CJK}) \begin{CJK}{UTF8}{mj}设\end{CJK} $A$ \begin{CJK}{UTF8}{mj}为\end{CJK} 3 \begin{CJK}{UTF8}{mj}级非零实数矩阵且\end{CJK} $A^{3}=-A$. \begin{CJK}{UTF8}{mj}求证\end{CJK} $A$ \begin{CJK}{UTF8}{mj}相似于\end{CJK} $\left(\begin{array}{ccc}0 & 0 & 0 \\ 0 & 0 & -1 \\ 0 & 1 & 0\end{array}\right)$.

\section{5. 南京大学 2013 年研究生入学考试试题高等代数}
\begin{CJK}{UTF8}{mj}李扬\end{CJK}

\begin{CJK}{UTF8}{mj}微信公众号\end{CJK}: sxkyliyang

\begin{CJK}{UTF8}{mj}一\end{CJK}. \begin{CJK}{UTF8}{mj}判断题\end{CJK}(\begin{CJK}{UTF8}{mj}每小题\end{CJK} 4 \begin{CJK}{UTF8}{mj}分\end{CJK}, \begin{CJK}{UTF8}{mj}共\end{CJK} 20 \begin{CJK}{UTF8}{mj}分\end{CJK})

\begin{CJK}{UTF8}{mj}判断下列叙述是否正确\end{CJK}, \begin{CJK}{UTF8}{mj}并说明理由\end{CJK}.

\begin{enumerate}
  \item \begin{CJK}{UTF8}{mj}若\end{CJK} $p(x)$ \begin{CJK}{UTF8}{mj}是数域\end{CJK} $P$ \begin{CJK}{UTF8}{mj}上不可约多项式\end{CJK}, $n$ \begin{CJK}{UTF8}{mj}为正整数\end{CJK}, \begin{CJK}{UTF8}{mj}则\end{CJK} $p\left(x^{n}\right)$ \begin{CJK}{UTF8}{mj}也是\end{CJK} $P$ \begin{CJK}{UTF8}{mj}上不可约多项式\end{CJK}.

  \item \begin{CJK}{UTF8}{mj}设实系数多项式\end{CJK} $f(x) \neq 0$, \begin{CJK}{UTF8}{mj}若\end{CJK} $\left(f(x), f^{\prime}(x)\right) \neq 1$, \begin{CJK}{UTF8}{mj}则\end{CJK} $f(x)$ \begin{CJK}{UTF8}{mj}在实数域上有重根\end{CJK}.

  \item \begin{CJK}{UTF8}{mj}最小多项式相同的两个\end{CJK} $n$ \begin{CJK}{UTF8}{mj}级方阵是相似的\end{CJK}.

  \item \begin{CJK}{UTF8}{mj}如果\end{CJK} $A$ \begin{CJK}{UTF8}{mj}是\end{CJK} $n$ \begin{CJK}{UTF8}{mj}级实矩阵\end{CJK}, \begin{CJK}{UTF8}{mj}并且存在\end{CJK} $n$ \begin{CJK}{UTF8}{mj}级实矩阵\end{CJK} $C$ \begin{CJK}{UTF8}{mj}使得\end{CJK} $A=C^{\prime} C$, \begin{CJK}{UTF8}{mj}则\end{CJK} $A$ \begin{CJK}{UTF8}{mj}是正定的\end{CJK}.

  \item \begin{CJK}{UTF8}{mj}设\end{CJK} $A$ \begin{CJK}{UTF8}{mj}是\end{CJK} $n$ \begin{CJK}{UTF8}{mj}级方阵\end{CJK}, \begin{CJK}{UTF8}{mj}如果\end{CJK} $A^{2}+A-6 E=0$, \begin{CJK}{UTF8}{mj}其中\end{CJK} $E$ \begin{CJK}{UTF8}{mj}为\end{CJK} $n$ \begin{CJK}{UTF8}{mj}级单位矩阵\end{CJK}, \begin{CJK}{UTF8}{mj}则\end{CJK} $A$ \begin{CJK}{UTF8}{mj}对角化\end{CJK}.

\end{enumerate}
\begin{CJK}{UTF8}{mj}二\end{CJK}. ( 20 \begin{CJK}{UTF8}{mj}分\end{CJK}) \begin{CJK}{UTF8}{mj}试用正交线性替换化二次型\end{CJK}
$$
f\left(x_{1}, x_{2}, x_{3}\right)=2 x_{1}^{2}+5 x_{2}^{2}+5 x_{3}^{2}+4 x_{1} x_{2}-4 x_{1} x_{3}-8 x_{2} x_{3}
$$
\begin{CJK}{UTF8}{mj}为标准型\end{CJK}, \begin{CJK}{UTF8}{mj}并写出所用的正交线性替换和所得的标准型\end{CJK}.

\begin{CJK}{UTF8}{mj}三\end{CJK}. ( 20 \begin{CJK}{UTF8}{mj}分\end{CJK}) \begin{CJK}{UTF8}{mj}设\end{CJK} $A=\left(\begin{array}{cccc}3 & -4 & 0 & 2 \\ 4 & -5 & -2 & 4 \\ 0 & 0 & 3 & -2 \\ 0 & 0 & 2 & -1\end{array}\right)$, \begin{CJK}{UTF8}{mj}试求\end{CJK} $A$ \begin{CJK}{UTF8}{mj}的\end{CJK} Jordan \begin{CJK}{UTF8}{mj}标准型和有理标准型\end{CJK}.

\begin{CJK}{UTF8}{mj}四\end{CJK}. ( 20 \begin{CJK}{UTF8}{mj}分\end{CJK}) \begin{CJK}{UTF8}{mj}给定\end{CJK} $n \times 2$ \begin{CJK}{UTF8}{mj}矩阵\end{CJK} $B=\left(\begin{array}{cc}1 & 1 \\ 1 & 1 \\ \vdots & \vdots \\ 1 & 1\end{array}\right)$ \begin{CJK}{UTF8}{mj}和\end{CJK} $2 \times n$ \begin{CJK}{UTF8}{mj}矩阵\end{CJK} $C=\left(\begin{array}{cccc}1 & 2 & \cdots & n \\ 1 & 2 & \cdots & n\end{array}\right)$. \begin{CJK}{UTF8}{mj}令\end{CJK} $A=E+B C$, \begin{CJK}{UTF8}{mj}其中\end{CJK} $E$ \begin{CJK}{UTF8}{mj}为\end{CJK} $n$ \begin{CJK}{UTF8}{mj}级单位矩阵\end{CJK}.

\begin{enumerate}
  \item \begin{CJK}{UTF8}{mj}求\end{CJK} $A$ \begin{CJK}{UTF8}{mj}的全部特征值\end{CJK}.

  \item \begin{CJK}{UTF8}{mj}证明\end{CJK}: $A$ \begin{CJK}{UTF8}{mj}可对角化\end{CJK}, \begin{CJK}{UTF8}{mj}并求可逆矩阵\end{CJK} $T$ \begin{CJK}{UTF8}{mj}使得\end{CJK} $T^{-1} A T$ \begin{CJK}{UTF8}{mj}为对角形\end{CJK}.

  \item \begin{CJK}{UTF8}{mj}计算行列式\end{CJK} $\left|A^{2}+A+E\right|$.

\end{enumerate}
\begin{CJK}{UTF8}{mj}五\end{CJK}. ( 20 \begin{CJK}{UTF8}{mj}分\end{CJK}) \begin{CJK}{UTF8}{mj}设\end{CJK} $\lambda_{1}, \lambda_{2}, \cdots, \lambda_{n}$ \begin{CJK}{UTF8}{mj}是\end{CJK} $n$ \begin{CJK}{UTF8}{mj}级方阵\end{CJK} $A$ \begin{CJK}{UTF8}{mj}的\end{CJK} $n$ \begin{CJK}{UTF8}{mj}个特征值\end{CJK}, $\mu_{k}=\prod_{\substack{i=1 \\ i \neq k}}^{n} \lambda_{i}, k=1,2, \cdots, n$. \begin{CJK}{UTF8}{mj}证明\end{CJK}: $\mu_{1}, \mu_{2}, \cdots, \mu_{n}$ \begin{CJK}{UTF8}{mj}是\end{CJK} $A$ \begin{CJK}{UTF8}{mj}的伴随矩阵\end{CJK} $A^{*}$ \begin{CJK}{UTF8}{mj}的\end{CJK} $n$ \begin{CJK}{UTF8}{mj}个特征值\end{CJK}.

\begin{CJK}{UTF8}{mj}六\end{CJK}. (15 \begin{CJK}{UTF8}{mj}分\end{CJK}) \begin{CJK}{UTF8}{mj}设\end{CJK} $A$ \begin{CJK}{UTF8}{mj}是\end{CJK} $n$ \begin{CJK}{UTF8}{mj}级正交矩阵且\end{CJK} $|A|=1$,
$$
f(x)=a_{0} x^{n}+a_{1} x^{n-1}+\cdots+a_{n-1} x+a_{n}
$$
\begin{CJK}{UTF8}{mj}是\end{CJK} $A$ \begin{CJK}{UTF8}{mj}的特征多项式\end{CJK}. \begin{CJK}{UTF8}{mj}证明\end{CJK}:

\begin{enumerate}
  \item \begin{CJK}{UTF8}{mj}当\end{CJK} $n$ \begin{CJK}{UTF8}{mj}为偶数时\end{CJK}, \begin{CJK}{UTF8}{mj}对任意的\end{CJK} $0 \leqslant i \leqslant n$, \begin{CJK}{UTF8}{mj}有\end{CJK} $a_{i}=a_{n-i}$.

  \item \begin{CJK}{UTF8}{mj}当\end{CJK} $n$ \begin{CJK}{UTF8}{mj}为奇数时\end{CJK}, \begin{CJK}{UTF8}{mj}对任意的\end{CJK} $0 \leqslant i \leqslant n$, \begin{CJK}{UTF8}{mj}有\end{CJK} $a_{i}=-a_{n-i}$.

\end{enumerate}
3 . \begin{CJK}{UTF8}{mj}当\end{CJK} $n=2$ \begin{CJK}{UTF8}{mj}时\end{CJK}, \begin{CJK}{UTF8}{mj}存在正交矩阵\end{CJK} $B$ \begin{CJK}{UTF8}{mj}使得\end{CJK} $A=B^{2}$.

\begin{CJK}{UTF8}{mj}七\end{CJK}. ( 15 \begin{CJK}{UTF8}{mj}分\end{CJK}) \begin{CJK}{UTF8}{mj}设\end{CJK} $V_{1}, V_{2}$ \begin{CJK}{UTF8}{mj}是\end{CJK} $n$ \begin{CJK}{UTF8}{mj}维欧式空间\end{CJK} $V$ \begin{CJK}{UTF8}{mj}的子空间\end{CJK}, \begin{CJK}{UTF8}{mj}且\end{CJK} $V_{1}$ \begin{CJK}{UTF8}{mj}的维数小于\end{CJK} $V_{2}$ \begin{CJK}{UTF8}{mj}的维数\end{CJK}. \begin{CJK}{UTF8}{mj}证明\end{CJK}: \begin{CJK}{UTF8}{mj}存在\end{CJK} $0 \neq \alpha \in V_{2}$ \begin{CJK}{UTF8}{mj}使得\end{CJK} $\alpha \perp V_{1}$.

\begin{CJK}{UTF8}{mj}八\end{CJK}. ( 20 \begin{CJK}{UTF8}{mj}分\end{CJK}) \begin{CJK}{UTF8}{mj}设\end{CJK} $A$ \begin{CJK}{UTF8}{mj}和\end{CJK} $C$ \begin{CJK}{UTF8}{mj}都是\end{CJK} $n$ \begin{CJK}{UTF8}{mj}级正定矩阵\end{CJK}, \begin{CJK}{UTF8}{mj}并且\end{CJK} $B$ \begin{CJK}{UTF8}{mj}是矩阵方程\end{CJK} $A X+X A=C$ \begin{CJK}{UTF8}{mj}的惟一解\end{CJK}. \begin{CJK}{UTF8}{mj}证明\end{CJK}: $B$ \begin{CJK}{UTF8}{mj}是正定矩阵\end{CJK}.

\section{6. 南京大学 2014 年研究生入学考试试题高等代数}
\begin{CJK}{UTF8}{mj}李扬\end{CJK}

\begin{CJK}{UTF8}{mj}微信公众号\end{CJK}: sxkyliyang

\begin{CJK}{UTF8}{mj}一\end{CJK}. \begin{CJK}{UTF8}{mj}判断题\end{CJK}(\begin{CJK}{UTF8}{mj}每小题\end{CJK} 4 \begin{CJK}{UTF8}{mj}分\end{CJK}, \begin{CJK}{UTF8}{mj}共\end{CJK} 20 \begin{CJK}{UTF8}{mj}分\end{CJK})

\begin{CJK}{UTF8}{mj}判断下列叙述是否正确\end{CJK}, \begin{CJK}{UTF8}{mj}并说明理由\end{CJK}.

\begin{enumerate}
  \item \begin{CJK}{UTF8}{mj}设\end{CJK} $f(x), g(x)$ \begin{CJK}{UTF8}{mj}都是实数域上的不可约多项式\end{CJK}, \begin{CJK}{UTF8}{mj}则\end{CJK} $f(g(x))$ \begin{CJK}{UTF8}{mj}在实数域上不可约\end{CJK}.

  \item \begin{CJK}{UTF8}{mj}设\end{CJK} $f(x)$ \begin{CJK}{UTF8}{mj}是数域\end{CJK} $P$ \begin{CJK}{UTF8}{mj}上的多项式\end{CJK}, $a \in P$, \begin{CJK}{UTF8}{mj}如果\end{CJK} $a$ \begin{CJK}{UTF8}{mj}是\end{CJK} $f^{\prime \prime \prime}(x)$ \begin{CJK}{UTF8}{mj}的\end{CJK} $k$ \begin{CJK}{UTF8}{mj}重根\end{CJK} $(k \geqslant 1)$, \begin{CJK}{UTF8}{mj}并且\end{CJK} $f(a)=0, f^{\prime}(a)=0$. \begin{CJK}{UTF8}{mj}则\end{CJK} $a$ \begin{CJK}{UTF8}{mj}是\end{CJK} $f(x)$ \begin{CJK}{UTF8}{mj}的\end{CJK} $k+3$ \begin{CJK}{UTF8}{mj}重根\end{CJK}.

  \item \begin{CJK}{UTF8}{mj}两个\end{CJK} $n$ \begin{CJK}{UTF8}{mj}级方阵是相似的当且仅当它们有相同的特征多项式\end{CJK}, \begin{CJK}{UTF8}{mj}并有相同的最小多项式\end{CJK}.

  \item \begin{CJK}{UTF8}{mj}设\end{CJK} $V_{1}, V_{2}, V_{3}$ \begin{CJK}{UTF8}{mj}是线性空间\end{CJK} $V$ \begin{CJK}{UTF8}{mj}的子空间\end{CJK}, \begin{CJK}{UTF8}{mj}则\end{CJK} $\left(V_{1}+V_{2}\right) \cap V_{3}=V_{1} \cap V_{3}+V_{2} \cap V_{3}$.

  \item \begin{CJK}{UTF8}{mj}对于有限维线性空间\end{CJK} $V$ \begin{CJK}{UTF8}{mj}的任意线性变换\end{CJK} $\mathscr{A}$, \begin{CJK}{UTF8}{mj}有\end{CJK} $V=\mathscr{A}^{-1}(0)+\mathscr{A} V$ \begin{CJK}{UTF8}{mj}当且仅当\end{CJK} $V=\mathscr{A}^{-1}(0) \oplus \mathscr{A} V$.

\end{enumerate}
\begin{CJK}{UTF8}{mj}二\end{CJK}. ( 20 \begin{CJK}{UTF8}{mj}分\end{CJK}) \begin{CJK}{UTF8}{mj}设\end{CJK} $A=\left(\begin{array}{cccc}0 & 1 & 1 & -1 \\ 1 & 0 & -1 & 1 \\ 1 & -1 & 0 & 1 \\ -1 & 1 & 1 & 0\end{array}\right)$, \begin{CJK}{UTF8}{mj}试求一正交矩阵\end{CJK} $T$, \begin{CJK}{UTF8}{mj}使\end{CJK} $T^{\prime} A T$ \begin{CJK}{UTF8}{mj}成对角形\end{CJK}.

\begin{CJK}{UTF8}{mj}三\end{CJK}. ( 20 \begin{CJK}{UTF8}{mj}分\end{CJK} $)$ \begin{CJK}{UTF8}{mj}设\end{CJK} $A=\left(\begin{array}{cccc}1 & 0 & 0 & 0 \\ 1 & 1 & 0 & 0 \\ 0 & 0 & 2 & 0 \\ 0 & 0 & 1 & 2\end{array}\right), B=\left(\begin{array}{cccc}1 & 0 & 0 & 0 \\ 0 & 2 & 0 & 0 \\ 0 & 1 & 2 & 0 \\ 0 & 0 & 0 & 1\end{array}\right), C=\left(\begin{array}{cccc}1 & -1 & 0 & 0 \\ 0 & 1 & 0 & 0 \\ 0 & 0 & 2 & -1 \\ 0 & 0 & 0 & 2\end{array}\right)$

\begin{enumerate}
  \item \begin{CJK}{UTF8}{mj}求\end{CJK} $C$ \begin{CJK}{UTF8}{mj}的行列式因子\end{CJK}, \begin{CJK}{UTF8}{mj}不变因子和初等因子\end{CJK}.

  \item \begin{CJK}{UTF8}{mj}指出\end{CJK} $A$ \begin{CJK}{UTF8}{mj}与\end{CJK} $B, B$ \begin{CJK}{UTF8}{mj}与\end{CJK} $C$ \begin{CJK}{UTF8}{mj}是否相似\end{CJK}, \begin{CJK}{UTF8}{mj}并说明理由\end{CJK}.

\end{enumerate}
\begin{CJK}{UTF8}{mj}四\end{CJK}. ( 15 \begin{CJK}{UTF8}{mj}分\end{CJK}) \begin{CJK}{UTF8}{mj}设\end{CJK} $n$ \begin{CJK}{UTF8}{mj}级实对称矩阵\end{CJK} $A$ \begin{CJK}{UTF8}{mj}的所有\end{CJK} 1 \begin{CJK}{UTF8}{mj}级主子式之和与所有\end{CJK} 2 \begin{CJK}{UTF8}{mj}级主子式之和均为\end{CJK} 0 , \begin{CJK}{UTF8}{mj}证明\end{CJK}: $A$ \begin{CJK}{UTF8}{mj}是零矩阵\end{CJK}.

\begin{CJK}{UTF8}{mj}五\end{CJK}. (15 \begin{CJK}{UTF8}{mj}分\end{CJK}) \begin{CJK}{UTF8}{mj}设\end{CJK} $A$ \begin{CJK}{UTF8}{mj}和\end{CJK} $B$ \begin{CJK}{UTF8}{mj}是\end{CJK} $n$ \begin{CJK}{UTF8}{mj}级实矩阵\end{CJK}, \begin{CJK}{UTF8}{mj}并且\end{CJK} $A B$ \begin{CJK}{UTF8}{mj}和\end{CJK} $B A$ \begin{CJK}{UTF8}{mj}都是对称矩阵\end{CJK}, \begin{CJK}{UTF8}{mj}证明\end{CJK}: \begin{CJK}{UTF8}{mj}存在正交矩阵\end{CJK} $T$, \begin{CJK}{UTF8}{mj}使得\end{CJK} $T^{-1} A B T=$ $B A$.

\begin{CJK}{UTF8}{mj}六\end{CJK}. ( 20 \begin{CJK}{UTF8}{mj}分\end{CJK}) \begin{CJK}{UTF8}{mj}设\end{CJK} $A=\left(\begin{array}{ccc}1 & -1 & 1 \\ x & 4 & y \\ -3 & -3 & 5\end{array}\right)$ \begin{CJK}{UTF8}{mj}有\end{CJK} 3 \begin{CJK}{UTF8}{mj}个线性无关的特征向量且\end{CJK} $\lambda=2$ \begin{CJK}{UTF8}{mj}是\end{CJK} $A$ \begin{CJK}{UTF8}{mj}的二重特征值\end{CJK}.

\begin{enumerate}
  \item \begin{CJK}{UTF8}{mj}求\end{CJK} $x, y$ \begin{CJK}{UTF8}{mj}的值\end{CJK}.

  \item \begin{CJK}{UTF8}{mj}求可逆矩阵\end{CJK} $P$ \begin{CJK}{UTF8}{mj}使得\end{CJK} $P^{-1} A P$ \begin{CJK}{UTF8}{mj}为对角矩阵\end{CJK}.

\end{enumerate}
\begin{CJK}{UTF8}{mj}七\end{CJK}. ( 20 \begin{CJK}{UTF8}{mj}分\end{CJK}) \begin{CJK}{UTF8}{mj}设\end{CJK} $A$ \begin{CJK}{UTF8}{mj}和\end{CJK} $B$ \begin{CJK}{UTF8}{mj}是\end{CJK} $n$ \begin{CJK}{UTF8}{mj}级实矩阵\end{CJK}. \begin{CJK}{UTF8}{mj}如果\end{CJK} $B$ \begin{CJK}{UTF8}{mj}是正定的\end{CJK}, $A-B$ \begin{CJK}{UTF8}{mj}是半正定的\end{CJK}. \begin{CJK}{UTF8}{mj}证明\end{CJK}:

\begin{enumerate}
  \item $|A-\lambda B|=0$ \begin{CJK}{UTF8}{mj}的所有根\end{CJK} $\lambda \geqslant 1$.

  \item $|A| \geqslant|B|$.

\end{enumerate}
\begin{CJK}{UTF8}{mj}八\end{CJK}. ( 20 \begin{CJK}{UTF8}{mj}分\end{CJK}) \begin{CJK}{UTF8}{mj}设整数\end{CJK} $n \geqslant 2$, \begin{CJK}{UTF8}{mj}并且\end{CJK} $a_{1}, a_{2}, \cdots, a_{n}$ \begin{CJK}{UTF8}{mj}是互不相同的整数\end{CJK}. \begin{CJK}{UTF8}{mj}证明\end{CJK}: \begin{CJK}{UTF8}{mj}多项式\end{CJK}
$$
f(x)=\left(x-a_{1}\right)\left(x-a_{2}\right) \cdots\left(x-a_{n}\right)+1
$$
\begin{CJK}{UTF8}{mj}在有理数域上不可约\end{CJK}.

\section{7. 南京大学 2015 年研究生入学考试试题高等代数}
\begin{CJK}{UTF8}{mj}李扬\end{CJK}

\begin{CJK}{UTF8}{mj}微信公众号\end{CJK}: sxkyliyang

\begin{CJK}{UTF8}{mj}一\end{CJK}. \begin{CJK}{UTF8}{mj}填空题\end{CJK}(\begin{CJK}{UTF8}{mj}每小题\end{CJK} 5 \begin{CJK}{UTF8}{mj}分\end{CJK}, \begin{CJK}{UTF8}{mj}共\end{CJK} 30 \begin{CJK}{UTF8}{mj}分\end{CJK})

\begin{enumerate}
  \item \begin{CJK}{UTF8}{mj}多项式\end{CJK} $f(x)=2 x^{3}-3 x^{2}+1$ \begin{CJK}{UTF8}{mj}的全部有理根为\end{CJK}

  \item \begin{CJK}{UTF8}{mj}设\end{CJK} $n$ \begin{CJK}{UTF8}{mj}是正整数\end{CJK}, \begin{CJK}{UTF8}{mj}多项式\end{CJK} $x^{2 n}-1$ \begin{CJK}{UTF8}{mj}在实数域上的标准分解式是\end{CJK}

  \item \begin{CJK}{UTF8}{mj}设行列式\end{CJK} $D=\left|\begin{array}{llll}2 & 0 & 1 & 3 \\ 1 & 2 & 3 & 4 \\ 4 & 3 & 2 & 1 \\ 1 & 3 & 5 & 7\end{array}\right|$ \begin{CJK}{UTF8}{mj}中第\end{CJK} $i$ \begin{CJK}{UTF8}{mj}行第\end{CJK} $j$ \begin{CJK}{UTF8}{mj}列元素\end{CJK} $a_{i j}$ \begin{CJK}{UTF8}{mj}的代数余子式为\end{CJK} $A_{i j}(i, j=1,2,3,4)$, \begin{CJK}{UTF8}{mj}则\end{CJK} $A_{41}+A_{42}+A_{43}+A_{44}=$

  \item \begin{CJK}{UTF8}{mj}设\end{CJK} $A=\left(\begin{array}{cccc}1+a_{1}^{2} & a_{1} a_{2} & \cdots & a_{1} a_{n} \\ a_{2} a_{1} & 1+a_{2}^{2} & \cdots & a_{2} a_{n} \\ \vdots & \vdots & & \vdots \\ a_{n} a_{1} & a_{n} a_{2} & \cdots & 1+a_{n}^{2}\end{array}\right)$, \begin{CJK}{UTF8}{mj}则\end{CJK} $|A|=$

\end{enumerate}
5 . \begin{CJK}{UTF8}{mj}设\end{CJK} 3 \begin{CJK}{UTF8}{mj}级方阵\end{CJK} $A$ \begin{CJK}{UTF8}{mj}的秩为\end{CJK} $2, B=\left(\begin{array}{ccc}1 & 2 & 3 \\ 2 & 4 & k \\ 3 & 6 & 9\end{array}\right)$, \begin{CJK}{UTF8}{mj}并且\end{CJK} $A B=0$, \begin{CJK}{UTF8}{mj}则\end{CJK} $k=$

\begin{enumerate}
  \setcounter{enumi}{6}
  \item \begin{CJK}{UTF8}{mj}设\end{CJK} 4 \begin{CJK}{UTF8}{mj}级数字矩阵\end{CJK} $A$ \begin{CJK}{UTF8}{mj}的最小多项式为\end{CJK} $(\lambda+1)^{3}$, \begin{CJK}{UTF8}{mj}则\end{CJK} $A$ \begin{CJK}{UTF8}{mj}的特征多项式为\end{CJK}
\end{enumerate}
\begin{CJK}{UTF8}{mj}二\end{CJK}. ( 20 \begin{CJK}{UTF8}{mj}分\end{CJK}) \begin{CJK}{UTF8}{mj}试用正交线性替换化二次型\end{CJK}
$$
f\left(x_{1}, x_{2}, x_{3}\right)=x_{1}^{2}-2 x_{2}^{2}-2 x_{3}^{2}-4 x_{1} x_{2}+4 x_{1} x_{3}+8 x_{2} x_{3}
$$
\begin{CJK}{UTF8}{mj}为标准型\end{CJK}, \begin{CJK}{UTF8}{mj}并写出所用的正交线性替换和所得的标准型\end{CJK}.

\begin{CJK}{UTF8}{mj}三\end{CJK}. ( 20 \begin{CJK}{UTF8}{mj}分\end{CJK}) \begin{CJK}{UTF8}{mj}设\end{CJK} $A=\left(\begin{array}{ccc}2 & 0 & 0 \\ a & 2 & 0 \\ b & c & -1\end{array}\right)$ \begin{CJK}{UTF8}{mj}是一个复矩阵\end{CJK}.

\begin{enumerate}
  \item \begin{CJK}{UTF8}{mj}求出\end{CJK} $A$ \begin{CJK}{UTF8}{mj}的一切可能的\end{CJK} Jordan \begin{CJK}{UTF8}{mj}标准型\end{CJK}.

  \item \begin{CJK}{UTF8}{mj}给出\end{CJK} $A$ \begin{CJK}{UTF8}{mj}相似于对角矩阵的一个充要条件\end{CJK}.

\end{enumerate}
\begin{CJK}{UTF8}{mj}四\end{CJK}. ( 20 \begin{CJK}{UTF8}{mj}分\end{CJK}) \begin{CJK}{UTF8}{mj}设\end{CJK} $A$ \begin{CJK}{UTF8}{mj}是数域\end{CJK} $P$ \begin{CJK}{UTF8}{mj}上的\end{CJK} $n$ \begin{CJK}{UTF8}{mj}级方阵\end{CJK}, \begin{CJK}{UTF8}{mj}并且\end{CJK} $A$ \begin{CJK}{UTF8}{mj}与对角矩阵相似\end{CJK}. \begin{CJK}{UTF8}{mj}证明\end{CJK}: \begin{CJK}{UTF8}{mj}对任意\end{CJK} $\lambda_{0} \in P$, \begin{CJK}{UTF8}{mj}齐次线性方程组\end{CJK} $\left(\lambda_{0} E-A\right) X=0$ \begin{CJK}{UTF8}{mj}与\end{CJK} $\left(\lambda_{0} E-A\right)^{2} X=0$ \begin{CJK}{UTF8}{mj}同解\end{CJK}. \begin{CJK}{UTF8}{mj}其中\end{CJK} $E$ \begin{CJK}{UTF8}{mj}是\end{CJK} $n$ \begin{CJK}{UTF8}{mj}级单位矩阵\end{CJK}, $X=\left(x_{1}, x_{2}, \cdots, x_{n}\right)$.

\begin{CJK}{UTF8}{mj}五\end{CJK}. ( 20 \begin{CJK}{UTF8}{mj}分\end{CJK}) \begin{CJK}{UTF8}{mj}设\end{CJK} $A$ \begin{CJK}{UTF8}{mj}是\end{CJK} $n$ \begin{CJK}{UTF8}{mj}级实可逆矩阵\end{CJK}. \begin{CJK}{UTF8}{mj}证明\end{CJK}: \begin{CJK}{UTF8}{mj}存在\end{CJK} $n$ \begin{CJK}{UTF8}{mj}级正交矩阵\end{CJK} $Q_{1}, Q_{2}$ \begin{CJK}{UTF8}{mj}使得\end{CJK} $Q_{2} A Q_{1}=\left(\begin{array}{ccccc}a_{1} & 0 & \cdots & 0 \\ 0 & a_{2} & \cdots & 0 \\ \vdots & & \vdots \\ & \vdots & & & 0 \\ & 0 & \cdots & a_{n}\end{array}\right)$, \begin{CJK}{UTF8}{mj}其中\end{CJK} $a_{i}>0, i=1,2, \cdots, n$.

\begin{CJK}{UTF8}{mj}六\end{CJK}. ( 20 \begin{CJK}{UTF8}{mj}分\end{CJK}) \begin{CJK}{UTF8}{mj}设\end{CJK} $A$ \begin{CJK}{UTF8}{mj}是可逆的\end{CJK} $n$ \begin{CJK}{UTF8}{mj}级方阵\end{CJK}, \begin{CJK}{UTF8}{mj}证明\end{CJK}: $A$ \begin{CJK}{UTF8}{mj}的伴随矩阵\end{CJK} $A^{*}$ \begin{CJK}{UTF8}{mj}可表为\end{CJK} $A$ \begin{CJK}{UTF8}{mj}的多项式\end{CJK}, \begin{CJK}{UTF8}{mj}即存在多项式\end{CJK} $g(x)$ \begin{CJK}{UTF8}{mj}使得\end{CJK} $A^{*}=g(A)$.

\begin{CJK}{UTF8}{mj}七\end{CJK}. ( 20 \begin{CJK}{UTF8}{mj}分\end{CJK}) \begin{CJK}{UTF8}{mj}设\end{CJK} $n$ \begin{CJK}{UTF8}{mj}级反对称矩阵\end{CJK} $A$ \begin{CJK}{UTF8}{mj}的元素都是整数\end{CJK}. \begin{CJK}{UTF8}{mj}证明\end{CJK}: $|A|$ \begin{CJK}{UTF8}{mj}是一个完全平方数\end{CJK}, \begin{CJK}{UTF8}{mj}即存在一个整数\end{CJK} $m$ \begin{CJK}{UTF8}{mj}使得\end{CJK} $|A|=m^{2}$.

\section{8. 南京大学 2016 年研究生入学考试试题高等代数}
\begin{CJK}{UTF8}{mj}李扬\end{CJK}

\begin{CJK}{UTF8}{mj}微信公众号\end{CJK}: sxkyliyang

\begin{CJK}{UTF8}{mj}一\end{CJK}. (20 \begin{CJK}{UTF8}{mj}分\end{CJK}) \begin{CJK}{UTF8}{mj}证明\end{CJK}: \begin{CJK}{UTF8}{mj}如果多项式\end{CJK}
$$
\left(x^{3}+x^{2}+x+1\right) \mid\left(f_{1}\left(x^{4}\right)+x f_{2}\left(x^{4}\right)+x^{2} f_{3}\left(x^{4}\right)\right)
$$
\begin{CJK}{UTF8}{mj}则对任意的\end{CJK} $1 \leqslant i \leqslant 3$, \begin{CJK}{UTF8}{mj}我们总有\end{CJK}
$$
(x-1) \mid f_{i}(x) .
$$
\begin{CJK}{UTF8}{mj}二\end{CJK}. ( 20 \begin{CJK}{UTF8}{mj}分\end{CJK}) \begin{CJK}{UTF8}{mj}求\end{CJK} $n$ \begin{CJK}{UTF8}{mj}级行列式\end{CJK}
$$
\left|\begin{array}{cccccc}
\sin \alpha & 1 & 0 & \cdots & 0 & 0 \\
1 & 2 \sin \alpha & 1 & \cdots & 0 & 0 \\
0 & 1 & 2 \sin \alpha & \cdots & 0 & 0 \\
\vdots & \vdots & \vdots & & \vdots & \vdots \\
0 & 0 & 0 & \cdots & 2 \sin \alpha & 1 \\
0 & 0 & 0 & \cdots & 1 & 2 \sin \alpha
\end{array}\right|
$$
\begin{CJK}{UTF8}{mj}三\end{CJK}. ( 15 \begin{CJK}{UTF8}{mj}分\end{CJK}) \begin{CJK}{UTF8}{mj}设\end{CJK}
$$
A=\left(\begin{array}{cccc}
-1 & -3 & 3 & -3 \\
-3 & -1 & -3 & 3 \\
3 & -3 & -1 & -3 \\
-3 & 3 & -3 & -1
\end{array}\right)
$$
\begin{CJK}{UTF8}{mj}求正交矩阵\end{CJK} $T$ \begin{CJK}{UTF8}{mj}使得\end{CJK} $T^{\prime} A T$ \begin{CJK}{UTF8}{mj}成对角矩阵\end{CJK}.

\begin{CJK}{UTF8}{mj}四\end{CJK}. ( 15 \begin{CJK}{UTF8}{mj}分\end{CJK}) \begin{CJK}{UTF8}{mj}讨论当\end{CJK} $a, b$ \begin{CJK}{UTF8}{mj}取何值时\end{CJK}, \begin{CJK}{UTF8}{mj}方程组\end{CJK}
$$
\left\{\begin{array}{l}
x_{1}+x_{2}+x_{3}+x_{4}+x_{5}=1 \\
3 x_{1}+2 x_{2}+x_{3}+x_{4}-3 x_{5}=a \\
x_{2}+2 x_{3}+2 x_{4}+6 x_{5}=3 \\
5 x_{1}+4 x_{2}+3 x_{3}+3 x_{4}-x_{5}=b
\end{array}\right.
$$
\begin{CJK}{UTF8}{mj}有解\end{CJK}, \begin{CJK}{UTF8}{mj}并在有解的情况下求一般解\end{CJK}.

\begin{CJK}{UTF8}{mj}五\end{CJK}. (20 \begin{CJK}{UTF8}{mj}分\end{CJK}) \begin{CJK}{UTF8}{mj}求矩阵\end{CJK}
$$
A=\left(\begin{array}{ccc}
-1 & -2 & 6 \\
-1 & 0 & 3 \\
-1 & -1 & 4
\end{array}\right)
$$
\begin{CJK}{UTF8}{mj}的\end{CJK} Jordan \begin{CJK}{UTF8}{mj}标准型和有理标准型\end{CJK}.

\begin{CJK}{UTF8}{mj}六\end{CJK}. ( 20 \begin{CJK}{UTF8}{mj}分\end{CJK}) \begin{CJK}{UTF8}{mj}对于全体正实数\end{CJK} $\mathbb{R}^{+}$, \begin{CJK}{UTF8}{mj}定义其上的加法和数量乘法为\end{CJK}
$$
a \oplus b=a b, k \circ a=a^{k},\left(a, b \in \mathbb{R}^{+}, k \in \mathbb{R}\right)
$$

\begin{enumerate}
  \item \begin{CJK}{UTF8}{mj}证明\end{CJK}: $\mathbb{R}^{+}$\begin{CJK}{UTF8}{mj}在上述两种运算下成为\end{CJK} $\mathbb{R}$ \begin{CJK}{UTF8}{mj}上的一个线性空间\end{CJK}.

  \item \begin{CJK}{UTF8}{mj}叙述欧式空间的定义\end{CJK}, \begin{CJK}{UTF8}{mj}并说明\end{CJK} $\mathbb{R}^{+}$\begin{CJK}{UTF8}{mj}可以成为欧式空间\end{CJK}.

\end{enumerate}
\begin{CJK}{UTF8}{mj}七\end{CJK}. ( 20 \begin{CJK}{UTF8}{mj}分\end{CJK}) \begin{CJK}{UTF8}{mj}设\end{CJK} $A$ \begin{CJK}{UTF8}{mj}为一个反对称的实矩阵\end{CJK}, \begin{CJK}{UTF8}{mj}证明\end{CJK}: \begin{CJK}{UTF8}{mj}如果\end{CJK} $A$ \begin{CJK}{UTF8}{mj}的特征值都是实数\end{CJK}, \begin{CJK}{UTF8}{mj}则\end{CJK} $A$ \begin{CJK}{UTF8}{mj}必为零矩阵\end{CJK}.

\begin{CJK}{UTF8}{mj}八\end{CJK}. ( 20 \begin{CJK}{UTF8}{mj}分\end{CJK}) \begin{CJK}{UTF8}{mj}设\end{CJK} $W$ \begin{CJK}{UTF8}{mj}为实\end{CJK} $n$ \begin{CJK}{UTF8}{mj}维向量空间\end{CJK} $\mathbb{R}^{n}$ \begin{CJK}{UTF8}{mj}的一个子空间\end{CJK}, \begin{CJK}{UTF8}{mj}且在\end{CJK} $W$ \begin{CJK}{UTF8}{mj}中每个非零向量\end{CJK} $\alpha=\left(a_{1}, \cdots, a_{n}\right)$ \begin{CJK}{UTF8}{mj}中零分量的\end{CJK} \begin{CJK}{UTF8}{mj}个数不超过\end{CJK} $r$. \begin{CJK}{UTF8}{mj}证明\end{CJK}: $\operatorname{dim} W \leqslant r+1$.

\section{9. 南京大学 2017 年研究生入学考试试题高等代数 
 李扬 
 微信公众号: sxkyliyang}
\begin{CJK}{UTF8}{mj}一\end{CJK}. (15 \begin{CJK}{UTF8}{mj}分\end{CJK}) \begin{CJK}{UTF8}{mj}设\end{CJK} $f(x)$ \begin{CJK}{UTF8}{mj}为首一多项式\end{CJK}, \begin{CJK}{UTF8}{mj}且无实根\end{CJK}. \begin{CJK}{UTF8}{mj}证明\end{CJK}: \begin{CJK}{UTF8}{mj}存在\end{CJK} $g(x), h(x)$ \begin{CJK}{UTF8}{mj}使得\end{CJK} $f(x)=g^{2}(x)+h^{2}(x)$.

\begin{CJK}{UTF8}{mj}二\end{CJK}. (15 \begin{CJK}{UTF8}{mj}分\end{CJK}) \begin{CJK}{UTF8}{mj}设\end{CJK} $D=\left|a_{i j}\right|$, \begin{CJK}{UTF8}{mj}证明\end{CJK}: \begin{CJK}{UTF8}{mj}如果有一行元素全为\end{CJK} 1 , \begin{CJK}{UTF8}{mj}则\end{CJK} $D=\sum_{1 \leqslant i, j \leqslant n} A_{i j}$.

\begin{CJK}{UTF8}{mj}三\end{CJK}. ( 15 \begin{CJK}{UTF8}{mj}分\end{CJK}) \begin{CJK}{UTF8}{mj}求一正交变换\end{CJK}, \begin{CJK}{UTF8}{mj}使\end{CJK} $x_{1}^{2}+2 x_{2}^{2}+3 x_{3}^{2}-4 x_{1} x_{2}-4 x_{2} x_{3}$ \begin{CJK}{UTF8}{mj}为标准型\end{CJK}.

\begin{CJK}{UTF8}{mj}四\end{CJK}. (15 \begin{CJK}{UTF8}{mj}分\end{CJK}) \begin{CJK}{UTF8}{mj}设\end{CJK} $V,\langle,$,$rangle 为一欧式空间, f$ \begin{CJK}{UTF8}{mj}为\end{CJK} $V$ \begin{CJK}{UTF8}{mj}上的线性函数\end{CJK}. \begin{CJK}{UTF8}{mj}证明\end{CJK}: \begin{CJK}{UTF8}{mj}存在\end{CJK} $\alpha_{0}$, \begin{CJK}{UTF8}{mj}使得\end{CJK} $\forall x \in V, f=\left\langle x, \alpha_{0}\right\rangle$.

\begin{CJK}{UTF8}{mj}五\end{CJK}. (15 \begin{CJK}{UTF8}{mj}分\end{CJK}) \begin{CJK}{UTF8}{mj}设\end{CJK} $A=\left(\begin{array}{ccc}1 & 2 & 0 \\ 0 & 2 & 0 \\ -2 & -2 & -1\end{array}\right)$, \begin{CJK}{UTF8}{mj}求\end{CJK} $A$ \begin{CJK}{UTF8}{mj}的\end{CJK} Jordan \begin{CJK}{UTF8}{mj}标准型和有理标准型\end{CJK}.

\begin{CJK}{UTF8}{mj}六\end{CJK}. (15 \begin{CJK}{UTF8}{mj}分\end{CJK}) \begin{CJK}{UTF8}{mj}设\end{CJK} $A, B$ \begin{CJK}{UTF8}{mj}为\end{CJK} $n$ \begin{CJK}{UTF8}{mj}阶实可逆矩阵\end{CJK}, \begin{CJK}{UTF8}{mj}满足\end{CJK} $A^{\prime} A=B^{\prime} B$, \begin{CJK}{UTF8}{mj}证明\end{CJK}: \begin{CJK}{UTF8}{mj}存在正交矩阵\end{CJK} $T$, \begin{CJK}{UTF8}{mj}使得\end{CJK} $A=T B$.

\begin{CJK}{UTF8}{mj}七\end{CJK}. ( 20 \begin{CJK}{UTF8}{mj}分\end{CJK}) $\sigma: P \rightarrow V, \lambda$ \begin{CJK}{UTF8}{mj}是\end{CJK} $\sigma$ \begin{CJK}{UTF8}{mj}的一个特征值\end{CJK}. \begin{CJK}{UTF8}{mj}对应特征向量为\end{CJK} $\alpha$. \begin{CJK}{UTF8}{mj}证明\end{CJK}: \begin{CJK}{UTF8}{mj}对\end{CJK} $P$ \begin{CJK}{UTF8}{mj}中\end{CJK} $k_{1}, k_{2}, \cdots, k_{n}$, \begin{CJK}{UTF8}{mj}存在\end{CJK} $V$ \begin{CJK}{UTF8}{mj}中基\end{CJK} $\eta_{1}, \eta_{2}, \cdots, \eta_{n}$, \begin{CJK}{UTF8}{mj}使得\end{CJK} $\alpha=k_{1} \eta_{1}+k_{2} \eta_{2}+\cdots+k_{n} \eta_{n}$.

\begin{CJK}{UTF8}{mj}八\end{CJK}. $\left(20\right.$ \begin{CJK}{UTF8}{mj}分\end{CJK}) \begin{CJK}{UTF8}{mj}设\end{CJK} $n \geqslant 2, \alpha_{1}, \alpha_{2}, \beta_{1}, \beta_{2} \in \mathbb{R}^{n}, V_{i}=L\left(\alpha_{i}\right), W_{i}=L\left(\beta_{i}\right)$, \begin{CJK}{UTF8}{mj}若\end{CJK} $V_{1} \neq V_{2}, W_{1} \neq W_{2}$, \begin{CJK}{UTF8}{mj}证明\end{CJK}: \begin{CJK}{UTF8}{mj}存在\end{CJK} $A \in M_{n \times n}(\mathbb{R})$, \begin{CJK}{UTF8}{mj}使得\end{CJK} $A V_{i}=W_{i}$, \begin{CJK}{UTF8}{mj}且\end{CJK} $|A|=1$.

\begin{CJK}{UTF8}{mj}九\end{CJK}. ( 20 \begin{CJK}{UTF8}{mj}分\end{CJK}) \begin{CJK}{UTF8}{mj}设\end{CJK} $A \in \mathbb{C}_{n \times n}, A$ \begin{CJK}{UTF8}{mj}的最小多项式和特征多项式相同\end{CJK}, $\mathbb{C}(A)=\left\{B \in M_{n \times n} \mid A B=B A\right\}$.

\begin{enumerate}
  \item \begin{CJK}{UTF8}{mj}证明\end{CJK}: $\mathbb{C}(A)$ \begin{CJK}{UTF8}{mj}是\end{CJK} $n$ \begin{CJK}{UTF8}{mj}维线性空间\end{CJK}.

  \item \begin{CJK}{UTF8}{mj}设\end{CJK} $X \in M_{n \times n}(\mathbb{C}), X A=A^{T} X$, \begin{CJK}{UTF8}{mj}则\end{CJK} $X$ \begin{CJK}{UTF8}{mj}是对称的\end{CJK}.

\end{enumerate}
\section{0. 南京大学 2009 年研究生入学考试试题数学分析 
 李扬 
 微信公众号: sxkyliyang}
\begin{CJK}{UTF8}{mj}一\end{CJK}. (10 \begin{CJK}{UTF8}{mj}分\end{CJK}) \begin{CJK}{UTF8}{mj}能否将\end{CJK} $(0,1)$ \begin{CJK}{UTF8}{mj}之间的有理数按照从小到大的顺序排成一列\end{CJK}? \begin{CJK}{UTF8}{mj}请说明你的理由\end{CJK}.

\begin{CJK}{UTF8}{mj}二\end{CJK}. (10 \begin{CJK}{UTF8}{mj}分\end{CJK}) \begin{CJK}{UTF8}{mj}如果级数\end{CJK} $\sum_{n=1}^{\infty} a_{n}$ \begin{CJK}{UTF8}{mj}收玫\end{CJK}, \begin{CJK}{UTF8}{mj}则级数\end{CJK} $\sum_{n=1}^{\infty} a_{n}^{2}$ \begin{CJK}{UTF8}{mj}是否也收玫\end{CJK}? \begin{CJK}{UTF8}{mj}请说明你的理由\end{CJK}.

\begin{CJK}{UTF8}{mj}三\end{CJK}. ( 20 \begin{CJK}{UTF8}{mj}分\end{CJK},\begin{CJK}{UTF8}{mj}每题\end{CJK} 10 \begin{CJK}{UTF8}{mj}分\end{CJK}) \begin{CJK}{UTF8}{mj}计算下列极限\end{CJK}:

(1) \begin{CJK}{UTF8}{mj}设\end{CJK} $a>0, b>0$. \begin{CJK}{UTF8}{mj}求极限\end{CJK} $\lim _{n \rightarrow+\infty} \sqrt[n]{a^{n}+b^{n}}$.

(2) \begin{CJK}{UTF8}{mj}求极限\end{CJK} $\lim _{x \rightarrow 0}\left(\frac{1}{x^{2}}-\frac{1}{x \sin x}\right)$.

\begin{CJK}{UTF8}{mj}四\end{CJK}. ( 20 \begin{CJK}{UTF8}{mj}分\end{CJK}) \begin{CJK}{UTF8}{mj}设\end{CJK} $f(x)$ \begin{CJK}{UTF8}{mj}为\end{CJK} $[0,1]$ \begin{CJK}{UTF8}{mj}上的可微函数\end{CJK}. \begin{CJK}{UTF8}{mj}问\end{CJK}: $f^{\prime}(x)$ \begin{CJK}{UTF8}{mj}是否在\end{CJK} $[0,1]$ \begin{CJK}{UTF8}{mj}上有界\end{CJK}? \begin{CJK}{UTF8}{mj}若答案为\end{CJK} “\begin{CJK}{UTF8}{mj}是\end{CJK}", \begin{CJK}{UTF8}{mj}请给出证明\end{CJK}; \begin{CJK}{UTF8}{mj}若答案\end{CJK} \begin{CJK}{UTF8}{mj}为\end{CJK} “\begin{CJK}{UTF8}{mj}否\end{CJK}”, \begin{CJK}{UTF8}{mj}请给出反例\end{CJK}.

\begin{CJK}{UTF8}{mj}五\end{CJK}. ( 20 \begin{CJK}{UTF8}{mj}分\end{CJK}) \begin{CJK}{UTF8}{mj}设\end{CJK} $f: \mathbb{R} \rightarrow \mathbb{R}$ \begin{CJK}{UTF8}{mj}是连续函数\end{CJK}. \begin{CJK}{UTF8}{mj}如果\end{CJK} $f$ \begin{CJK}{UTF8}{mj}有一个惟一的极值点\end{CJK}, \begin{CJK}{UTF8}{mj}证明\end{CJK}: \begin{CJK}{UTF8}{mj}这个极值点一定是最值点\end{CJK}(\begin{CJK}{UTF8}{mj}最小值或\end{CJK} \begin{CJK}{UTF8}{mj}最大值\end{CJK}).

\begin{CJK}{UTF8}{mj}六\end{CJK}. ( 15 \begin{CJK}{UTF8}{mj}分\end{CJK}) \begin{CJK}{UTF8}{mj}设函数\end{CJK} $f(x)$ \begin{CJK}{UTF8}{mj}在\end{CJK} $[0,1]$ \begin{CJK}{UTF8}{mj}区间二阶连续可微\end{CJK}, \begin{CJK}{UTF8}{mj}且\end{CJK} $f(0)=0, f(1)=1, f^{\prime \prime}(x)<0, \forall x \in[0,1]$. \begin{CJK}{UTF8}{mj}证明\end{CJK}: $f(x) \geqslant x, \forall x \in[0,1] .$

\begin{CJK}{UTF8}{mj}七\end{CJK}. ( 20 \begin{CJK}{UTF8}{mj}分\end{CJK}) \begin{CJK}{UTF8}{mj}设\end{CJK} $f(x, y)$ \begin{CJK}{UTF8}{mj}是\end{CJK} $z_{0}=\left(x_{0}, y_{0}\right) \in \mathbb{R}^{2}$ \begin{CJK}{UTF8}{mj}附近的\end{CJK} $C^{2}$ \begin{CJK}{UTF8}{mj}函数\end{CJK}, \begin{CJK}{UTF8}{mj}请计算下面的极限\end{CJK}
$$
\lim _{h \rightarrow 0} h^{-2}\left[\frac{1}{\pi h^{2}} \int_{B\left(z_{0}, h\right)} f(x, y) \mathrm{d} x \mathrm{~d} y-f\left(x_{0}, y_{0}\right)\right],
$$
\begin{CJK}{UTF8}{mj}其中\end{CJK} $B\left(z_{0}, h\right)$ \begin{CJK}{UTF8}{mj}是平面上以\end{CJK} $z_{0}$ \begin{CJK}{UTF8}{mj}为中心\end{CJK}, $h$ \begin{CJK}{UTF8}{mj}为半径的圆盘\end{CJK}.

\begin{CJK}{UTF8}{mj}八\end{CJK}. (15 \begin{CJK}{UTF8}{mj}分\end{CJK}) \begin{CJK}{UTF8}{mj}计算积分\end{CJK}
$$
\iint_{\Sigma} z \mathrm{~d} x \mathrm{~d} y+x \mathrm{~d} y \mathrm{~d} z+y \mathrm{~d} z \mathrm{~d} x
$$
\begin{CJK}{UTF8}{mj}其中\end{CJK} $\Sigma$ \begin{CJK}{UTF8}{mj}为八分之一球面\end{CJK}
$$
\Sigma=\left\{(x, y, z) \in \mathbb{R}^{3} \mid x \geqslant 0, y \geqslant 0, z \geqslant 0, x^{2}+y^{2}+z^{2}=R^{2}\right\}
$$
\begin{CJK}{UTF8}{mj}方向朝外\end{CJK}.

\begin{CJK}{UTF8}{mj}九\end{CJK}. ( 20 \begin{CJK}{UTF8}{mj}分\end{CJK}) \begin{CJK}{UTF8}{mj}设\end{CJK} $f(x)$ \begin{CJK}{UTF8}{mj}是\end{CJK} $[-\pi, \pi]$ \begin{CJK}{UTF8}{mj}上的有界变差函数\end{CJK}, \begin{CJK}{UTF8}{mj}证明\end{CJK}: \begin{CJK}{UTF8}{mj}其\end{CJK} Fourier \begin{CJK}{UTF8}{mj}系数\end{CJK} $a_{n}, b_{n}$ \begin{CJK}{UTF8}{mj}满足条件\end{CJK}
$$
a_{n}=O(n), b_{n}=O(n)
$$

\section{1. 南京大学 2010 年研究生入学考试试题数学分析}
\begin{CJK}{UTF8}{mj}李扬\end{CJK}

\begin{CJK}{UTF8}{mj}微信公众号\end{CJK}: sxkyliyang

\begin{CJK}{UTF8}{mj}一\end{CJK}. (10 \begin{CJK}{UTF8}{mj}分\end{CJK}) \begin{CJK}{UTF8}{mj}设\end{CJK} $a_{1}=1, a_{n+1}=\sqrt{1+a_{n}}$, \begin{CJK}{UTF8}{mj}说明数列\end{CJK} $\left\{a_{n}\right\}$ \begin{CJK}{UTF8}{mj}的极限存在并计算此极限\end{CJK}.

\begin{CJK}{UTF8}{mj}二\end{CJK}. (15 \begin{CJK}{UTF8}{mj}分\end{CJK}) \begin{CJK}{UTF8}{mj}求下面等式中的常数\end{CJK} $a, b$ :
$$
\left(1+\frac{1}{n}\right)^{n}=e+\frac{a}{n}+\frac{b}{n^{2}}+O\left(\frac{1}{n^{3}}\right),(n \rightarrow+\infty) .
$$
\begin{CJK}{UTF8}{mj}三\end{CJK}. (15 \begin{CJK}{UTF8}{mj}分\end{CJK}) \begin{CJK}{UTF8}{mj}计算积分\end{CJK}
$$
\int_{0}^{\frac{\pi}{2}} \frac{x^{2}}{\sin ^{2} x} d x
$$
\begin{CJK}{UTF8}{mj}四\end{CJK}. (15 \begin{CJK}{UTF8}{mj}分\end{CJK}) \begin{CJK}{UTF8}{mj}计算积分\end{CJK}
$$
\iint_{\Sigma} \frac{z}{r^{3}} \mathrm{~d} x \mathrm{~d} y+\frac{x}{r^{3}} \mathrm{~d} y \mathrm{~d} z+\frac{y}{r^{3}} \mathrm{~d} z \mathrm{~d} x
$$
\begin{CJK}{UTF8}{mj}其中\end{CJK}
$$
r=\sqrt{x^{2}+y^{2}+z^{2}}, \Sigma=\left\{(x, y, z) \in \mathbb{R}^{3}|| x|+| y|+| z \mid=1\right\}
$$
$\Sigma$ \begin{CJK}{UTF8}{mj}方向朝外\end{CJK}.

\begin{CJK}{UTF8}{mj}五\end{CJK}. (15 \begin{CJK}{UTF8}{mj}分\end{CJK}) \begin{CJK}{UTF8}{mj}设\end{CJK} $\sum_{n=1}^{\infty} a_{n}$ \begin{CJK}{UTF8}{mj}为收敛正项级数\end{CJK}, \begin{CJK}{UTF8}{mj}证明\end{CJK}: $\sum_{n=1}^{\infty} \sqrt[n]{a_{n} a_{n+1} \cdots a_{2 n-1}}$ \begin{CJK}{UTF8}{mj}也收敛\end{CJK}.

\begin{CJK}{UTF8}{mj}六\end{CJK}. ( 20 \begin{CJK}{UTF8}{mj}分\end{CJK}) \begin{CJK}{UTF8}{mj}设\end{CJK} $f(x)$ \begin{CJK}{UTF8}{mj}在\end{CJK} $[0, \pi]$ \begin{CJK}{UTF8}{mj}上连续\end{CJK}, \begin{CJK}{UTF8}{mj}在\end{CJK} $x=0$ \begin{CJK}{UTF8}{mj}处可导\end{CJK}, \begin{CJK}{UTF8}{mj}证明\end{CJK}:
$$
\lim _{n \rightarrow \infty} \int_{0}^{\pi} f(x)\left[\frac{1}{2}+\cos x+\cos 2 x+\cdots+\cos n x\right] \mathrm{d} x=\frac{\pi}{2} f(0) .
$$
\begin{CJK}{UTF8}{mj}七\end{CJK}. (20 \begin{CJK}{UTF8}{mj}分\end{CJK}) \begin{CJK}{UTF8}{mj}设\end{CJK} $f: \mathbb{R}^{n} \rightarrow \mathbb{R}$ \begin{CJK}{UTF8}{mj}是光滑函数\end{CJK}, \begin{CJK}{UTF8}{mj}且\end{CJK} $\operatorname{Hess}(f) \geqslant I_{n}$, \begin{CJK}{UTF8}{mj}其中\end{CJK}
$$
\operatorname{Hess}(f)=\left(\frac{\partial^{2} f}{\partial x_{i} \partial x_{j}}\right)_{n \times n}
$$
$I_{n}$ \begin{CJK}{UTF8}{mj}为\end{CJK} $n$ \begin{CJK}{UTF8}{mj}阶单位矩阵\end{CJK}. \begin{CJK}{UTF8}{mj}证明\end{CJK}: $\nabla f: \mathbb{R}^{n} \rightarrow \mathbb{R}^{n}$ \begin{CJK}{UTF8}{mj}可逆\end{CJK}, \begin{CJK}{UTF8}{mj}且其逆也是光滑的\end{CJK}. \begin{CJK}{UTF8}{mj}其中\end{CJK}, $\nabla f=\left(\frac{\partial f}{\partial x_{1}}, \cdots, \frac{\partial f}{\partial x_{n}}\right)$ \begin{CJK}{UTF8}{mj}表示\end{CJK} $f$ \begin{CJK}{UTF8}{mj}的梯度\end{CJK}.

\begin{CJK}{UTF8}{mj}八\end{CJK}. ( 20 \begin{CJK}{UTF8}{mj}分\end{CJK}) \begin{CJK}{UTF8}{mj}设\end{CJK} $f$ \begin{CJK}{UTF8}{mj}是区间\end{CJK} $[a, b]$ \begin{CJK}{UTF8}{mj}上的连续函数\end{CJK}, \begin{CJK}{UTF8}{mj}如果\end{CJK} $f$ \begin{CJK}{UTF8}{mj}在一个可数集之外可导\end{CJK}, \begin{CJK}{UTF8}{mj}且导数非负\end{CJK}, \begin{CJK}{UTF8}{mj}证明\end{CJK}: $f(a) \leqslant f(b)$.

\begin{CJK}{UTF8}{mj}九\end{CJK}. ( 20 \begin{CJK}{UTF8}{mj}分\end{CJK}) \begin{CJK}{UTF8}{mj}设\end{CJK} $f$ \begin{CJK}{UTF8}{mj}在\end{CJK} $[a, b]$ \begin{CJK}{UTF8}{mj}上二阶可导\end{CJK}, \begin{CJK}{UTF8}{mj}且\end{CJK} $f(a)=f(b)=0$. \begin{CJK}{UTF8}{mj}证明\end{CJK}: \begin{CJK}{UTF8}{mj}存在\end{CJK} $\xi \in[a, b]$, \begin{CJK}{UTF8}{mj}使得\end{CJK}
$$
\int_{a}^{b} f(x) \mathrm{d} x=\frac{f^{\prime \prime}(\xi)}{12}(a-b)^{3} .
$$

\section{2. 南京大学 2011 年研究生入学考试试题数学分析}
\begin{CJK}{UTF8}{mj}李扬\end{CJK}

\begin{CJK}{UTF8}{mj}微信公众号\end{CJK}: sxkyliyang

\begin{CJK}{UTF8}{mj}一\end{CJK}. (10 \begin{CJK}{UTF8}{mj}分\end{CJK}) \begin{CJK}{UTF8}{mj}如果定义在实轴上的函数\end{CJK} $f(x)$ \begin{CJK}{UTF8}{mj}满足对任意的\end{CJK} $x, y \in \mathbb{R}$ \begin{CJK}{UTF8}{mj}有\end{CJK}
$$
|f(x)-f(y)| \leqslant|x-y|^{2},
$$
\begin{CJK}{UTF8}{mj}试证\end{CJK} $f(x)$ \begin{CJK}{UTF8}{mj}是一常值函数\end{CJK}.

\begin{CJK}{UTF8}{mj}二\end{CJK}. ( 10 \begin{CJK}{UTF8}{mj}分\end{CJK}) \begin{CJK}{UTF8}{mj}设幂级数\end{CJK} $\sum_{n=0}^{\infty} a_{n} x^{n}$ \begin{CJK}{UTF8}{mj}的收敛半径为\end{CJK} 1 , \begin{CJK}{UTF8}{mj}问级数\end{CJK} $\sum_{n=0}^{\infty} a_{n}$ \begin{CJK}{UTF8}{mj}是否收敛\end{CJK}. \begin{CJK}{UTF8}{mj}请说明理由\end{CJK}.

\begin{CJK}{UTF8}{mj}三\end{CJK}. ( 20 \begin{CJK}{UTF8}{mj}分\end{CJK}) \begin{CJK}{UTF8}{mj}设\end{CJK} $f(x)$ \begin{CJK}{UTF8}{mj}是\end{CJK} $[1,+\infty)$ \begin{CJK}{UTF8}{mj}上非负单调减少函数\end{CJK}, \begin{CJK}{UTF8}{mj}令\end{CJK}
$$
a_{n}=\sum_{k=1}^{n} f(k)-\int_{1}^{n} f(x) \mathrm{d} x
$$
\begin{CJK}{UTF8}{mj}其中\end{CJK} $n$ \begin{CJK}{UTF8}{mj}是正整数\end{CJK}, \begin{CJK}{UTF8}{mj}试证明数列\end{CJK} $\left\{a_{n}\right\}$ \begin{CJK}{UTF8}{mj}收敛\end{CJK}.

\begin{CJK}{UTF8}{mj}四\end{CJK}. ( 20 \begin{CJK}{UTF8}{mj}分\end{CJK}) \begin{CJK}{UTF8}{mj}设函数\end{CJK} $f(x, y)$ \begin{CJK}{UTF8}{mj}和\end{CJK} $g(x, y)$ \begin{CJK}{UTF8}{mj}在平面开区域\end{CJK} $G$ \begin{CJK}{UTF8}{mj}上有连续一阶偏导数\end{CJK}, \begin{CJK}{UTF8}{mj}且满足对任何\end{CJK} $(x, y) \in G$,
$$
f_{x}^{\prime} g_{y}^{\prime}-f_{y}^{\prime} g_{x}^{\prime} \neq 0
$$
\begin{CJK}{UTF8}{mj}试证对\end{CJK} $G$ \begin{CJK}{UTF8}{mj}中任一紧致集\end{CJK} $K, K$ \begin{CJK}{UTF8}{mj}中同时满足\end{CJK} $f(x, y)=0$ \begin{CJK}{UTF8}{mj}和\end{CJK} $g(x, y)=0$ \begin{CJK}{UTF8}{mj}的点\end{CJK} $(x, y)$ \begin{CJK}{UTF8}{mj}只有有限多个\end{CJK}.

\begin{CJK}{UTF8}{mj}五\end{CJK}. ( 15 \begin{CJK}{UTF8}{mj}分\end{CJK}) \begin{CJK}{UTF8}{mj}方程\end{CJK} $z^{2} y-x z^{3}-1=0$ \begin{CJK}{UTF8}{mj}在\end{CJK} $(1,2,1)$ \begin{CJK}{UTF8}{mj}附近决定了隐函数\end{CJK} $z=z(x, y)$, \begin{CJK}{UTF8}{mj}求\end{CJK}
$$
\frac{\partial^{2} z}{\partial x^{2}}(1,2)
$$
\begin{CJK}{UTF8}{mj}的值\end{CJK}.

\begin{CJK}{UTF8}{mj}六\end{CJK}. (20 \begin{CJK}{UTF8}{mj}分\end{CJK}) \begin{CJK}{UTF8}{mj}讨论广义积分\end{CJK}
$$
\iint_{D} \frac{\mathrm{d} x \mathrm{~d} y}{|x|^{p}+|y|^{q}}
$$
\begin{CJK}{UTF8}{mj}的收敛性\end{CJK}, \begin{CJK}{UTF8}{mj}其中\end{CJK} $D=\{(x, y)|| x|+| y \mid \geqslant 1\}$.

\begin{CJK}{UTF8}{mj}七\end{CJK}. (20 \begin{CJK}{UTF8}{mj}分\end{CJK}) \begin{CJK}{UTF8}{mj}设\end{CJK} $f(x)$ \begin{CJK}{UTF8}{mj}在\end{CJK} $[0, \pi]$ \begin{CJK}{UTF8}{mj}上二阶连续可微\end{CJK}, \begin{CJK}{UTF8}{mj}且\end{CJK} $f(\pi)=2$, \begin{CJK}{UTF8}{mj}满足\end{CJK}
$$
\int_{0}^{\pi}\left(f(x)+f^{\prime \prime}(x)\right) \sin x \mathrm{~d} x=5
$$
\begin{CJK}{UTF8}{mj}求\end{CJK} $f(0)$.

\begin{CJK}{UTF8}{mj}八\end{CJK}. ( 15 \begin{CJK}{UTF8}{mj}分\end{CJK}) \begin{CJK}{UTF8}{mj}设\end{CJK} $\sum_{n=1}^{\infty}\left(a_{n}-a_{n-1}\right)$ \begin{CJK}{UTF8}{mj}绝对收敛\end{CJK}, $\sum_{n=1}^{\infty} b_{n}$ \begin{CJK}{UTF8}{mj}收敛\end{CJK}, \begin{CJK}{UTF8}{mj}试证\end{CJK} $\sum_{n=1}^{\infty} a_{n} b_{n}$ \begin{CJK}{UTF8}{mj}收敛\end{CJK}.

\begin{CJK}{UTF8}{mj}九\end{CJK}. ( 20 \begin{CJK}{UTF8}{mj}分\end{CJK}) \begin{CJK}{UTF8}{mj}设\end{CJK} $\Sigma$ \begin{CJK}{UTF8}{mj}是由\end{CJK} $F(x, y, z)=0$ \begin{CJK}{UTF8}{mj}确定的光滑简单闭曲面\end{CJK}, $F(x, y, z)$ \begin{CJK}{UTF8}{mj}二阶连续可微\end{CJK}, \begin{CJK}{UTF8}{mj}且\end{CJK} $\nabla F(x, y, z) \neq 0$, \begin{CJK}{UTF8}{mj}这里\end{CJK} $\nabla F$ \begin{CJK}{UTF8}{mj}是\end{CJK} $F$ \begin{CJK}{UTF8}{mj}的梯度\end{CJK}. \begin{CJK}{UTF8}{mj}如果\end{CJK} $\Omega=\{(x, y, z) \mid F(x, y, z)<0\}$ \begin{CJK}{UTF8}{mj}为曲面\end{CJK} $\Sigma$ \begin{CJK}{UTF8}{mj}所围成的区域\end{CJK}, \begin{CJK}{UTF8}{mj}计算三重积分\end{CJK}
$$
\iiint_{\Omega} \operatorname{div}\left(\frac{\nabla F}{\|\nabla F\|}\right) \mathrm{d} x \mathrm{~d} y \mathrm{~d} z
$$
\begin{CJK}{UTF8}{mj}这里\end{CJK} $\operatorname{div}(\cdot)$ \begin{CJK}{UTF8}{mj}是向量场的散度\end{CJK}.

\section{3. 南京大学 2012 年研究生入学考试试题数学分析}
\begin{CJK}{UTF8}{mj}李扬\end{CJK}

\begin{CJK}{UTF8}{mj}微信公众号\end{CJK}: sxkyliyang

\begin{CJK}{UTF8}{mj}一\end{CJK}. (10 \begin{CJK}{UTF8}{mj}分\end{CJK}) \begin{CJK}{UTF8}{mj}设\end{CJK} $x_{n}=\left(1+\frac{1}{2}\right)\left(1+\frac{1}{4}\right) \cdots\left(1+\frac{1}{2^{n}}\right)$, \begin{CJK}{UTF8}{mj}证明\end{CJK}: \begin{CJK}{UTF8}{mj}数列\end{CJK} $\left\{x_{n}\right\}$ \begin{CJK}{UTF8}{mj}收玫\end{CJK}.

\begin{CJK}{UTF8}{mj}二\end{CJK}. (15 \begin{CJK}{UTF8}{mj}分\end{CJK}) \begin{CJK}{UTF8}{mj}设\end{CJK} $\lim _{x \rightarrow a} f(x)=A$ \begin{CJK}{UTF8}{mj}及\end{CJK} $\lim _{x \rightarrow A} g(x)=B$, \begin{CJK}{UTF8}{mj}由此是否可推出\end{CJK} $\lim _{x \rightarrow a} g(f(x))=B ?$ \begin{CJK}{UTF8}{mj}若答案为\end{CJK} “\begin{CJK}{UTF8}{mj}是\end{CJK}”, \begin{CJK}{UTF8}{mj}请给出证明\end{CJK}; \begin{CJK}{UTF8}{mj}若答案为\end{CJK} “\begin{CJK}{UTF8}{mj}否\end{CJK}”, \begin{CJK}{UTF8}{mj}请给出反例\end{CJK}.

\begin{CJK}{UTF8}{mj}三\end{CJK}. (15 \begin{CJK}{UTF8}{mj}分\end{CJK}) \begin{CJK}{UTF8}{mj}设\end{CJK} $F(y)=\int_{0}^{1} \ln \sqrt{x^{2}+y^{2}} \mathrm{~d} x$, \begin{CJK}{UTF8}{mj}判断\end{CJK} $F(y)$ \begin{CJK}{UTF8}{mj}在\end{CJK} $y=0$ \begin{CJK}{UTF8}{mj}是否可导\end{CJK}.

\begin{CJK}{UTF8}{mj}四\end{CJK}. ( 15 \begin{CJK}{UTF8}{mj}分\end{CJK}) \begin{CJK}{UTF8}{mj}求由曲面\end{CJK} $x^{2}+\frac{y^{2}}{4}+\frac{z}{3}=1, x^{\frac{2}{3}}+\left(\frac{y}{2}\right)^{\frac{2}{3}}=1, z=0$ \begin{CJK}{UTF8}{mj}所围立体体积\end{CJK}.

\begin{CJK}{UTF8}{mj}五\end{CJK}. (15 \begin{CJK}{UTF8}{mj}分\end{CJK}) \begin{CJK}{UTF8}{mj}计算\end{CJK}
$$
\oint_{c}(y-z) \mathrm{d} x+(z-x) \mathrm{d} y+(x-y) \mathrm{d} z,
$$
\begin{CJK}{UTF8}{mj}式中\end{CJK} $c$ \begin{CJK}{UTF8}{mj}为椭圆\end{CJK} $x^{2}+y^{2}=1, x+\frac{z}{2}=1$, \begin{CJK}{UTF8}{mj}若从\end{CJK} $\mathrm{OX}$ \begin{CJK}{UTF8}{mj}轴正向看去\end{CJK}, \begin{CJK}{UTF8}{mj}此椭圆是依逆时针方向进行的\end{CJK}.

\begin{CJK}{UTF8}{mj}六\end{CJK}. ( 20 \begin{CJK}{UTF8}{mj}分\end{CJK}) \begin{CJK}{UTF8}{mj}若\end{CJK} $f(x)$ \begin{CJK}{UTF8}{mj}在\end{CJK} $(0,+\infty)$ \begin{CJK}{UTF8}{mj}内是二阶可导的\end{CJK}, \begin{CJK}{UTF8}{mj}且\end{CJK} $f^{\prime}(1)=f^{\prime}(2)=0$, \begin{CJK}{UTF8}{mj}证明\end{CJK}: \begin{CJK}{UTF8}{mj}存在\end{CJK} $c \in(1,2)$, \begin{CJK}{UTF8}{mj}使得\end{CJK} $\left|f^{\prime \prime}(c)\right| \geqslant$ $4|f(2)-f(1)|$.

\begin{CJK}{UTF8}{mj}七\end{CJK}. ( 20 \begin{CJK}{UTF8}{mj}分\end{CJK}) \begin{CJK}{UTF8}{mj}设\end{CJK} $f(x)$ \begin{CJK}{UTF8}{mj}在\end{CJK} $(0,1]$ \begin{CJK}{UTF8}{mj}内单调下降\end{CJK}, \begin{CJK}{UTF8}{mj}且\end{CJK} $\lim _{x \rightarrow 0^{+}} f(x)=+\infty$, \begin{CJK}{UTF8}{mj}若\end{CJK} $\int_{0}^{1} f(x) \mathrm{d} x$ \begin{CJK}{UTF8}{mj}存在\end{CJK}, \begin{CJK}{UTF8}{mj}证明\end{CJK}:
$$
\lim _{n \rightarrow \infty} \frac{1}{n} \sum_{k=1}^{n} f\left(\frac{k}{n}\right)=\int_{0}^{1} f(x) \mathrm{d} x
$$
\begin{CJK}{UTF8}{mj}八\end{CJK}. ( 20 \begin{CJK}{UTF8}{mj}分\end{CJK}) \begin{CJK}{UTF8}{mj}判断数项级数\end{CJK} $\sum_{n=1}^{\infty} \frac{(-1)^{[\sqrt{n}]}}{n}$ \begin{CJK}{UTF8}{mj}的敛散性\end{CJK}, \begin{CJK}{UTF8}{mj}这里\end{CJK} $[\cdot]$ \begin{CJK}{UTF8}{mj}表示\end{CJK}. \begin{CJK}{UTF8}{mj}的整数部分\end{CJK}.

\begin{CJK}{UTF8}{mj}九\end{CJK}. ( 20 \begin{CJK}{UTF8}{mj}分\end{CJK}) \begin{CJK}{UTF8}{mj}求级数\end{CJK} $\sum_{n=1}^{\infty} \frac{(-1)^{n+1}}{n^{2}}$ \begin{CJK}{UTF8}{mj}的和\end{CJK}.

\section{4. 南京大学 2013 年研究生入学考试试题数学分析}
\begin{CJK}{UTF8}{mj}李扬\end{CJK}

\begin{CJK}{UTF8}{mj}微信公众号\end{CJK}: sxkyliyang

\begin{CJK}{UTF8}{mj}一\end{CJK}. (20 \begin{CJK}{UTF8}{mj}分\end{CJK}) \begin{CJK}{UTF8}{mj}计算下列极限\end{CJK}:

(1) $\lim _{n \rightarrow \infty}\left(\cos \frac{1}{n}\right)^{n^{2}}$;

(2) $\lim _{n \rightarrow \infty} \frac{1}{\sqrt{n}} \sum_{k=1}^{n} \frac{1}{\sqrt{k}}$.

\begin{CJK}{UTF8}{mj}二\end{CJK}. ( 20 \begin{CJK}{UTF8}{mj}分\end{CJK}) \begin{CJK}{UTF8}{mj}计算下列积分\end{CJK}:

(1) $\int_{a}^{b}(x-a)^{2}(b-x)^{3} \mathrm{~d} x$;

(2) $\int_{0}^{+\infty} \frac{\sin ^{3} x}{x^{3}} \mathrm{~d} x$.

\begin{CJK}{UTF8}{mj}三\end{CJK}. ( 20 \begin{CJK}{UTF8}{mj}分\end{CJK}) \begin{CJK}{UTF8}{mj}在\end{CJK} $\mathbb{R}^{4}$ \begin{CJK}{UTF8}{mj}中定义如下有界区域\end{CJK} $\Omega$ :
$$
\Omega=\left\{(x, y, z, w) \in \mathbb{R}^{4}|| x|+| y \mid+\sqrt{z^{2}+w^{2}} \leqslant 1\right\}
$$
\begin{CJK}{UTF8}{mj}计算\end{CJK} $\Omega$ \begin{CJK}{UTF8}{mj}的体积\end{CJK}.

\begin{CJK}{UTF8}{mj}四\end{CJK}. ( 15 \begin{CJK}{UTF8}{mj}分\end{CJK}) \begin{CJK}{UTF8}{mj}设\end{CJK} $f$ \begin{CJK}{UTF8}{mj}在\end{CJK} $[a, b]$ \begin{CJK}{UTF8}{mj}中可导\end{CJK}, $f(a)=0$. \begin{CJK}{UTF8}{mj}如果\end{CJK} $f^{\prime} \geqslant f$, \begin{CJK}{UTF8}{mj}证明\end{CJK}: $f$ \begin{CJK}{UTF8}{mj}为单调递增函数\end{CJK}.

\begin{CJK}{UTF8}{mj}五\end{CJK}. (15 \begin{CJK}{UTF8}{mj}分\end{CJK}) \begin{CJK}{UTF8}{mj}设数列\end{CJK} $\left\{a_{n}\right\}$ \begin{CJK}{UTF8}{mj}单调递减趋于零\end{CJK}. \begin{CJK}{UTF8}{mj}证明\end{CJK}: $\sum_{n=1}^{\infty} a_{n}$ \begin{CJK}{UTF8}{mj}收敛当且仅当\end{CJK} $\sum_{k=1}^{\infty} 3^{k} a_{3^{k}}$ \begin{CJK}{UTF8}{mj}收敛\end{CJK}.

\begin{CJK}{UTF8}{mj}六\end{CJK}. (15 \begin{CJK}{UTF8}{mj}分\end{CJK}) \begin{CJK}{UTF8}{mj}设\end{CJK} $f: \mathbb{R}^{2} \rightarrow \mathbb{R}$ \begin{CJK}{UTF8}{mj}为连续函数\end{CJK}, \begin{CJK}{UTF8}{mj}且满足条件\end{CJK}
$$
f(x+1, y)=f(x, y+1)=f(x, y), \forall(x, y) \in \mathbb{R}^{2} .
$$
\begin{CJK}{UTF8}{mj}证明\end{CJK}: $f$ \begin{CJK}{UTF8}{mj}为一致连续函数\end{CJK}.

\begin{CJK}{UTF8}{mj}七\end{CJK}. (15 \begin{CJK}{UTF8}{mj}分\end{CJK}) \begin{CJK}{UTF8}{mj}证明\end{CJK}: \begin{CJK}{UTF8}{mj}在\end{CJK} $[0,+\infty)$ \begin{CJK}{UTF8}{mj}中存在唯一的连续函数\end{CJK} $y=y(x)$, \begin{CJK}{UTF8}{mj}使得\end{CJK}
$$
y^{3} e^{-y}=1-e^{-x}, \forall x>0
$$
\begin{CJK}{UTF8}{mj}并说明此时\end{CJK} $y=y(x)$ \begin{CJK}{UTF8}{mj}在\end{CJK} $(0,+\infty)$ \begin{CJK}{UTF8}{mj}中是\end{CJK} $C^{1}$ \begin{CJK}{UTF8}{mj}函数\end{CJK}.

\begin{CJK}{UTF8}{mj}八\end{CJK}. ( 15 \begin{CJK}{UTF8}{mj}分\end{CJK}) \begin{CJK}{UTF8}{mj}设\end{CJK} $\left\{a_{n}\right\}$ \begin{CJK}{UTF8}{mj}为数列\end{CJK}, $S_{n}=\sum_{i=1}^{n} a_{i}$ \begin{CJK}{UTF8}{mj}为部分和\end{CJK}.

(1) \begin{CJK}{UTF8}{mj}当\end{CJK} $\lim _{n \rightarrow \infty} a_{n}=0$ \begin{CJK}{UTF8}{mj}时\end{CJK}, \begin{CJK}{UTF8}{mj}证明\end{CJK}: $\lim _{n \rightarrow \infty} \frac{S_{n}}{n}=0$;

(2) \begin{CJK}{UTF8}{mj}设\end{CJK} $\left\{S_{n}\right\}$ \begin{CJK}{UTF8}{mj}有界\end{CJK}, $\lim _{n \rightarrow \infty}\left(a_{n+1}-a_{n}\right)=0$, \begin{CJK}{UTF8}{mj}证明\end{CJK}: $\lim _{n \rightarrow \infty} a_{n}=0$;

(3) \begin{CJK}{UTF8}{mj}当\end{CJK} $\lim _{\substack{n \rightarrow \infty \\ \text { 构造反例. }}} \frac{S_{n}}{n}=0$ \begin{CJK}{UTF8}{mj}且\end{CJK} $\lim _{n \rightarrow \infty}\left(a_{n+1}-a_{n}\right)=0$ \begin{CJK}{UTF8}{mj}时\end{CJK}, \begin{CJK}{UTF8}{mj}能否推出\end{CJK} $\lim _{n \rightarrow \infty} a_{n}=0$ ? \begin{CJK}{UTF8}{mj}如果能\end{CJK}, \begin{CJK}{UTF8}{mj}请给出证明\end{CJK}; \begin{CJK}{UTF8}{mj}如果不能\end{CJK}, \begin{CJK}{UTF8}{mj}请\end{CJK} \begin{CJK}{UTF8}{mj}构造反例\end{CJK}.

\begin{CJK}{UTF8}{mj}九\end{CJK}. ( 15 \begin{CJK}{UTF8}{mj}分\end{CJK}) \begin{CJK}{UTF8}{mj}设\end{CJK} $f$ \begin{CJK}{UTF8}{mj}为\end{CJK} $\mathbb{R}$ \begin{CJK}{UTF8}{mj}上周期为\end{CJK} 1 \begin{CJK}{UTF8}{mj}的\end{CJK} $C^{1}$ \begin{CJK}{UTF8}{mj}函数\end{CJK}. \begin{CJK}{UTF8}{mj}如果\end{CJK} $f$ \begin{CJK}{UTF8}{mj}满足以下条件\end{CJK}
$$
f(x)+f\left(x+\frac{1}{2}\right)=f(2 x), \forall x \in \mathbb{R}
$$
\begin{CJK}{UTF8}{mj}证明\end{CJK} $f$ \begin{CJK}{UTF8}{mj}恒等于零\end{CJK}.

\section{5. 南京大学 2014 年研究生入学考试试题数学分析}
\begin{CJK}{UTF8}{mj}李扬\end{CJK}

\begin{CJK}{UTF8}{mj}微信公众号\end{CJK}: sxkyliyang

\begin{CJK}{UTF8}{mj}一\end{CJK}. (20 \begin{CJK}{UTF8}{mj}分\end{CJK}) \begin{CJK}{UTF8}{mj}计算极限和积分\end{CJK}.\\
(1) $\lim _{n \rightarrow \infty}\left(n \sin \frac{1}{n}\right)^{n^{2}}$\\
(2) $\int_{0}^{\pi} \frac{\mathrm{d} x}{2+\cos x}$.

\begin{CJK}{UTF8}{mj}二\end{CJK}.(20 \begin{CJK}{UTF8}{mj}分\end{CJK}) \begin{CJK}{UTF8}{mj}设\end{CJK} $f(x)$ \begin{CJK}{UTF8}{mj}为偶函数\end{CJK}, \begin{CJK}{UTF8}{mj}在\end{CJK} $[0, \pi]$ \begin{CJK}{UTF8}{mj}中\end{CJK} $f(x)=x(\pi-x)$. \begin{CJK}{UTF8}{mj}计算\end{CJK} $f(x)$ \begin{CJK}{UTF8}{mj}的\end{CJK} Fourier \begin{CJK}{UTF8}{mj}展开式\end{CJK}, \begin{CJK}{UTF8}{mj}并用它求级数\end{CJK} $\sum_{n=1}^{+\infty} \frac{1}{n^{2}}$ \begin{CJK}{UTF8}{mj}之和\end{CJK}.

\begin{CJK}{UTF8}{mj}三\end{CJK}. ( 20 \begin{CJK}{UTF8}{mj}分\end{CJK}) \begin{CJK}{UTF8}{mj}计算积分\end{CJK}
$$
\int_{\Omega}\left(x^{2}+2 y^{2}\right) \mathrm{d} x \mathrm{~d} y \mathrm{~d} z
$$
\begin{CJK}{UTF8}{mj}其中\end{CJK} $\Omega=\left\{(x, y, z) \in \mathbb{R}^{3} \mid x^{2}+2 y^{2}+3 z^{2} \leqslant 1\right\}$.

\begin{CJK}{UTF8}{mj}四\end{CJK}. (15 \begin{CJK}{UTF8}{mj}分\end{CJK}) \begin{CJK}{UTF8}{mj}设方程\end{CJK} $\sin x-x \cos x=0$ \begin{CJK}{UTF8}{mj}在\end{CJK} $(0,+\infty)$ \begin{CJK}{UTF8}{mj}中的第\end{CJK} $n$ \begin{CJK}{UTF8}{mj}个解为\end{CJK} $x_{n}$. \begin{CJK}{UTF8}{mj}证明\end{CJK}:
$$
n \pi+\frac{\pi}{2}-\frac{1}{n \pi}<x_{n}<n \pi+\frac{\pi}{2} .
$$
\begin{CJK}{UTF8}{mj}五\end{CJK}. (15 \begin{CJK}{UTF8}{mj}分\end{CJK}) \begin{CJK}{UTF8}{mj}设\end{CJK} $f: \mathbb{R} \rightarrow \mathbb{R}$ \begin{CJK}{UTF8}{mj}为可导函数\end{CJK}, \begin{CJK}{UTF8}{mj}且\end{CJK} $f(f(x)) \equiv f(x)$. \begin{CJK}{UTF8}{mj}证明\end{CJK}: \begin{CJK}{UTF8}{mj}要么\end{CJK} $f$ \begin{CJK}{UTF8}{mj}为常值函数\end{CJK}, \begin{CJK}{UTF8}{mj}要么\end{CJK} $f(x) \equiv x$.

\begin{CJK}{UTF8}{mj}六\end{CJK}. ( 20 \begin{CJK}{UTF8}{mj}分\end{CJK}) \begin{CJK}{UTF8}{mj}设\end{CJK} $f(x, y)$ \begin{CJK}{UTF8}{mj}为\end{CJK} $[0,1] \times[0,1]$ \begin{CJK}{UTF8}{mj}中定义的二元函数\end{CJK}. \begin{CJK}{UTF8}{mj}如果对每一个固定的\end{CJK} $x \in[0,1], f(x, y)$ \begin{CJK}{UTF8}{mj}关于\end{CJK} $y$ \begin{CJK}{UTF8}{mj}在\end{CJK} $[0,1]$ \begin{CJK}{UTF8}{mj}中\end{CJK} Riemann \begin{CJK}{UTF8}{mj}可积\end{CJK}, \begin{CJK}{UTF8}{mj}且\end{CJK}
$$
\left|f\left(x_{1}, y\right)-f\left(x_{2}, y\right)\right| \leqslant\left|x_{1}-x_{2}\right|, \forall x_{1}, x_{2}, y \in[0,1]
$$
\begin{CJK}{UTF8}{mj}证明\end{CJK}: $f(x, y)$ \begin{CJK}{UTF8}{mj}为二元\end{CJK} Riemann \begin{CJK}{UTF8}{mj}可积函数\end{CJK}.

\begin{CJK}{UTF8}{mj}七\end{CJK}. ( 20 \begin{CJK}{UTF8}{mj}分\end{CJK}) \begin{CJK}{UTF8}{mj}证明\end{CJK}: \begin{CJK}{UTF8}{mj}方程\end{CJK} $x^{2}-x y+z \sin y+e^{z}=0$ \begin{CJK}{UTF8}{mj}在\end{CJK} $(1,2,0)$ \begin{CJK}{UTF8}{mj}附近决定了隐函数\end{CJK} $z=z(x, y)$. \begin{CJK}{UTF8}{mj}并计算\end{CJK}
$$
\frac{\partial z}{\partial x}(1,2), \frac{\partial z}{\partial y}(1,2), \frac{\partial^{2} z}{\partial x^{2}}(1,2), \frac{\partial^{2} z}{\partial x \partial y}(1,2), \frac{\partial^{2} z}{\partial y^{2}}(1,2)
$$
\begin{CJK}{UTF8}{mj}八\end{CJK}. ( 20 \begin{CJK}{UTF8}{mj}分\end{CJK}) \begin{CJK}{UTF8}{mj}设\end{CJK} $f(x)$ \begin{CJK}{UTF8}{mj}在\end{CJK} $[1,+\infty)$ \begin{CJK}{UTF8}{mj}中单调递减趋于\end{CJK} 0 , \begin{CJK}{UTF8}{mj}且当\end{CJK} $s>1$ \begin{CJK}{UTF8}{mj}时\end{CJK}, \begin{CJK}{UTF8}{mj}广义积分\end{CJK} $\int_{1}^{+\infty}[f(x)]^{s} \mathrm{~d} x$ \begin{CJK}{UTF8}{mj}均收敛\end{CJK}. \begin{CJK}{UTF8}{mj}证明\end{CJK}:

(1) \begin{CJK}{UTF8}{mj}极限\end{CJK} $\lim _{n \rightarrow \infty}\left\{\sum_{k=1}^{n} f(k)-\int_{1}^{n} f(x) \mathrm{d} x\right\}$ \begin{CJK}{UTF8}{mj}存在且有限\end{CJK}.

(2) \begin{CJK}{UTF8}{mj}极限\end{CJK} $\lim _{s \rightarrow 1^{+}}\left\{\sum_{k=1}^{\infty}[f(k)]^{s}-\int_{1}^{+\infty}[f(x)]^{s} \mathrm{~d} x\right\}$ \begin{CJK}{UTF8}{mj}存在且有限\end{CJK}.

\section{6. 南京大学 2015 年研究生入学考试试题数学分析}
\begin{CJK}{UTF8}{mj}李扬\end{CJK}

\begin{CJK}{UTF8}{mj}微信公众号\end{CJK}: sxkyliyang

\begin{CJK}{UTF8}{mj}一\end{CJK}. (20 \begin{CJK}{UTF8}{mj}分\end{CJK}) \begin{CJK}{UTF8}{mj}计算极限和积分\end{CJK}:

(1) $\lim _{x \rightarrow 0}\left(\frac{1}{x^{2}}-\frac{\cot x}{x}\right)$;

(2) $\int_{0}^{1} x(1-x)^{2014} \mathrm{~d} x$.

\begin{CJK}{UTF8}{mj}二\end{CJK}. ( 20 \begin{CJK}{UTF8}{mj}分\end{CJK}) \begin{CJK}{UTF8}{mj}在\end{CJK} $(0,2 \pi)$ \begin{CJK}{UTF8}{mj}中\end{CJK} $f(x)=\frac{\pi-x}{2}$. \begin{CJK}{UTF8}{mj}计算\end{CJK} $f(x)$ \begin{CJK}{UTF8}{mj}的\end{CJK} Fourier \begin{CJK}{UTF8}{mj}展开式\end{CJK}, \begin{CJK}{UTF8}{mj}并用它求级数\end{CJK} $\sum_{n=1}^{\infty} \frac{\sin n}{n}$ \begin{CJK}{UTF8}{mj}之和\end{CJK}.

\begin{CJK}{UTF8}{mj}三\end{CJK}. ( 20 \begin{CJK}{UTF8}{mj}分\end{CJK}) \begin{CJK}{UTF8}{mj}设\end{CJK} $a, b>0, q>p>0$. \begin{CJK}{UTF8}{mj}计算由椭圆\end{CJK} $\frac{x^{2}}{a^{2}}+\frac{y^{2}}{b^{2}}=1, \frac{x^{2}}{a^{2}}+\frac{y^{2}}{b^{2}}=2$ \begin{CJK}{UTF8}{mj}以及直线\end{CJK} $y=p x, y=q x$ \begin{CJK}{UTF8}{mj}在平面第一\end{CJK} \begin{CJK}{UTF8}{mj}象限所围区域的面积\end{CJK}.

\begin{CJK}{UTF8}{mj}四\end{CJK}. ( 15 \begin{CJK}{UTF8}{mj}分\end{CJK}) \begin{CJK}{UTF8}{mj}设\end{CJK} $f: \mathbb{R} \rightarrow \mathbb{R}$ \begin{CJK}{UTF8}{mj}为连续函数\end{CJK}, \begin{CJK}{UTF8}{mj}且\end{CJK} $\lim _{x \rightarrow-\infty} f(x)<0, \lim _{x \rightarrow+\infty} f(x)>0$. \begin{CJK}{UTF8}{mj}证明\end{CJK}: \begin{CJK}{UTF8}{mj}存在\end{CJK} $\xi \in \mathbb{R}$, \begin{CJK}{UTF8}{mj}使得\end{CJK} $f(\xi)=0$.

\begin{CJK}{UTF8}{mj}五\end{CJK}. ( 15 \begin{CJK}{UTF8}{mj}分\end{CJK}) \begin{CJK}{UTF8}{mj}设\end{CJK} $f: \mathbb{R} \rightarrow \mathbb{R}$ \begin{CJK}{UTF8}{mj}为可导函数\end{CJK}, \begin{CJK}{UTF8}{mj}且\end{CJK} $\lim _{x \rightarrow+\infty} \frac{f(x)}{x}=+\infty$. \begin{CJK}{UTF8}{mj}证明\end{CJK}: \begin{CJK}{UTF8}{mj}存在数列\end{CJK} $x_{n} \rightarrow+\infty$, \begin{CJK}{UTF8}{mj}使得\end{CJK} $\lim _{n \rightarrow+\infty} f^{\prime}\left(x_{n}\right)=$ $+\infty$.

\begin{CJK}{UTF8}{mj}六\end{CJK}. (20 \begin{CJK}{UTF8}{mj}分\end{CJK}) \begin{CJK}{UTF8}{mj}设\end{CJK} $a_{1}>0, a_{n+1}=n+\frac{1}{n} a_{n}$, \begin{CJK}{UTF8}{mj}请判断极限\end{CJK} $\lim _{n \rightarrow \infty} \frac{a_{n}}{n}$ \begin{CJK}{UTF8}{mj}是否存在\end{CJK}, \begin{CJK}{UTF8}{mj}如果存在\end{CJK}, \begin{CJK}{UTF8}{mj}请计算此极限\end{CJK}.

\begin{CJK}{UTF8}{mj}七\end{CJK}. ( 20 \begin{CJK}{UTF8}{mj}分\end{CJK}) \begin{CJK}{UTF8}{mj}设\end{CJK} $f: \mathbb{R}^{n} \rightarrow \mathbb{R}$ \begin{CJK}{UTF8}{mj}为可微函数\end{CJK}, \begin{CJK}{UTF8}{mj}且\end{CJK}
$$
\langle\nabla f(x), x\rangle \geqslant 0, \forall x \in \mathbb{R}^{n}
$$
\begin{CJK}{UTF8}{mj}其中\end{CJK} $\langle,$,$rangle 为 \mathbb{R}^{n}$ \begin{CJK}{UTF8}{mj}中的标准内积\end{CJK}, $\nabla f$ \begin{CJK}{UTF8}{mj}为\end{CJK} $f$ \begin{CJK}{UTF8}{mj}的梯度\end{CJK}. \begin{CJK}{UTF8}{mj}证明\end{CJK}: \begin{CJK}{UTF8}{mj}原点为\end{CJK} $f$ \begin{CJK}{UTF8}{mj}的最小值点\end{CJK}.

\begin{CJK}{UTF8}{mj}八\end{CJK}. ( 20 \begin{CJK}{UTF8}{mj}分\end{CJK}) \begin{CJK}{UTF8}{mj}设\end{CJK} $f: \mathbb{R}^{n} \rightarrow \mathbb{R}^{n}$ \begin{CJK}{UTF8}{mj}可微\end{CJK}, \begin{CJK}{UTF8}{mj}且\end{CJK} $\|J f(x)\| \leqslant \frac{1}{2}$ \begin{CJK}{UTF8}{mj}总成立\end{CJK}, \begin{CJK}{UTF8}{mj}其中\end{CJK}
$$
f(x)=\left(f_{1}(x), \cdots, f_{n}(x)\right),\|J f(x)\|=\left[\sum_{i, j=1}^{n}\left(\frac{\partial f_{i}}{\partial x_{j}}\right)^{2}\right]^{\frac{1}{2}}
$$
\begin{CJK}{UTF8}{mj}令\end{CJK} $g(x)=x+f(x)$, \begin{CJK}{UTF8}{mj}证明\end{CJK}: $g$ \begin{CJK}{UTF8}{mj}既是单射\end{CJK}, \begin{CJK}{UTF8}{mj}又是满射\end{CJK}.

\section{7. 南京大学 2016 年研究生入学考试试题数学分析}
\begin{CJK}{UTF8}{mj}李扬\end{CJK}

\begin{CJK}{UTF8}{mj}微信公众号\end{CJK}: sxkyliyang

\begin{CJK}{UTF8}{mj}一\end{CJK}. (20 \begin{CJK}{UTF8}{mj}分\end{CJK}) \begin{CJK}{UTF8}{mj}计算\end{CJK}:

(1) $\lim _{n \rightarrow \infty}\left(\frac{1}{n}+\frac{1}{n+1}+\cdots+\frac{1}{2 n}\right)$;

(2) $\int_{0}^{\frac{\pi}{2}} \frac{\mathrm{d} x}{1+\sin x}$.

\begin{CJK}{UTF8}{mj}二\end{CJK}. ( 20 \begin{CJK}{UTF8}{mj}分\end{CJK}) \begin{CJK}{UTF8}{mj}计算三重积分\end{CJK}
$$
\iiint_{\Omega}\left(x^{2}+y^{2}+z^{2}\right) \mathrm{d} x \mathrm{~d} y \mathrm{~d} z
$$
\begin{CJK}{UTF8}{mj}其中\end{CJK} $\Omega=\left\{(x, y, z) \in \mathbb{R}^{3}|| x|+| y|+| z \mid \leqslant 1\right\}$.

\begin{CJK}{UTF8}{mj}三\end{CJK}. ( 15 \begin{CJK}{UTF8}{mj}分\end{CJK}) \begin{CJK}{UTF8}{mj}设函数\end{CJK} $f: \mathbb{R} \rightarrow \mathbb{R}$ \begin{CJK}{UTF8}{mj}在每一点附近都单调递增\end{CJK}, \begin{CJK}{UTF8}{mj}即\end{CJK} $\forall x_{0} \in \mathbb{R}, \exists \delta>0$, \begin{CJK}{UTF8}{mj}使\end{CJK} $f$ \begin{CJK}{UTF8}{mj}在\end{CJK} $\left(x_{0}-\delta . x_{0}+\delta\right)$ \begin{CJK}{UTF8}{mj}中单调递\end{CJK} \begin{CJK}{UTF8}{mj}增\end{CJK}. \begin{CJK}{UTF8}{mj}证明\end{CJK}: $f$ \begin{CJK}{UTF8}{mj}在整个\end{CJK} $\mathbb{R}$ \begin{CJK}{UTF8}{mj}中单调递增\end{CJK}.

\begin{CJK}{UTF8}{mj}四\end{CJK}. ( 15 \begin{CJK}{UTF8}{mj}分\end{CJK}) \begin{CJK}{UTF8}{mj}设级数\end{CJK} $\sum_{n=1}^{\infty} \sqrt{n} a_{n}$ \begin{CJK}{UTF8}{mj}收敛\end{CJK}. \begin{CJK}{UTF8}{mj}证明\end{CJK}: \begin{CJK}{UTF8}{mj}级数\end{CJK} $\sum_{n=1}^{\infty} a_{n}$ \begin{CJK}{UTF8}{mj}也收敛\end{CJK}.

\begin{CJK}{UTF8}{mj}五\end{CJK}. ( 20 \begin{CJK}{UTF8}{mj}分\end{CJK}) \begin{CJK}{UTF8}{mj}方程\end{CJK} $x^{2}+2 y^{2}+3 z^{2}+2 x y-z=7$ \begin{CJK}{UTF8}{mj}在\end{CJK} $(1,-2,1)$ \begin{CJK}{UTF8}{mj}附近决定了隐函数\end{CJK} $z=z(x, y)$. \begin{CJK}{UTF8}{mj}计算二阶偏导数\end{CJK} $\frac{\partial^{2} z}{\partial x \partial y}(1,-2)$.

\begin{CJK}{UTF8}{mj}六\end{CJK}. ( 20 \begin{CJK}{UTF8}{mj}分\end{CJK}) \begin{CJK}{UTF8}{mj}证明\end{CJK}: \begin{CJK}{UTF8}{mj}存在常数\end{CJK} $c>0$, \begin{CJK}{UTF8}{mj}使得当\end{CJK} $f \in C^{1}[0,1]$ \begin{CJK}{UTF8}{mj}且\end{CJK} $\int_{0}^{1} f(x) \mathrm{d} x=0$ \begin{CJK}{UTF8}{mj}时成立\end{CJK}
$$
\int_{0}^{1} f^{2}(x) \mathrm{d} x \leqslant c \int_{0}^{1}\left|f^{\prime}(x)\right|^{2} \mathrm{~d} x
$$
\begin{CJK}{UTF8}{mj}七\end{CJK}. ( 20 \begin{CJK}{UTF8}{mj}分\end{CJK}) \begin{CJK}{UTF8}{mj}设\end{CJK} $A=\left(a_{i j}\right)$ \begin{CJK}{UTF8}{mj}为\end{CJK} $n$ \begin{CJK}{UTF8}{mj}阶实正定对称方阵\end{CJK}, $b_{i}(i=1,2, \cdots, n)$ \begin{CJK}{UTF8}{mj}为实数\end{CJK}. \begin{CJK}{UTF8}{mj}考虑\end{CJK} $\mathbb{R}^{n}$ \begin{CJK}{UTF8}{mj}中的函数\end{CJK}
$$
f\left(x_{1}, x_{2}, \cdots, x_{n}\right)=\sum_{i, j=1}^{n} a_{i j} x_{i} x_{j}-\sum_{i=1}^{n} b_{i} x_{i}
$$
\begin{CJK}{UTF8}{mj}证明\end{CJK}: $f$ \begin{CJK}{UTF8}{mj}在\end{CJK} $\mathbb{R}^{n}$ \begin{CJK}{UTF8}{mj}中有唯一的最小值点\end{CJK}.

\begin{CJK}{UTF8}{mj}八\end{CJK}. ( 20 \begin{CJK}{UTF8}{mj}分\end{CJK}) \begin{CJK}{UTF8}{mj}设\end{CJK} $f: \mathbb{R} \rightarrow \mathbb{R}$ \begin{CJK}{UTF8}{mj}为连续函数\end{CJK}. \begin{CJK}{UTF8}{mj}证明\end{CJK}: $f$ \begin{CJK}{UTF8}{mj}为凸函数当且仅当对任意区间\end{CJK} $[a, b] \subset \mathbb{R}$, \begin{CJK}{UTF8}{mj}均有\end{CJK}
$$
f\left(\frac{a+b}{2}\right) \leqslant \frac{1}{b-a} \int_{a}^{b} f(x) \mathrm{d} x .
$$

\section{8. 南京大学 2017 年研究生入学考试试题数学分析}
\begin{CJK}{UTF8}{mj}李扬\end{CJK}

\begin{CJK}{UTF8}{mj}微信公众号\end{CJK}: sxkyliyang

\begin{CJK}{UTF8}{mj}一\end{CJK}. \begin{CJK}{UTF8}{mj}计算极限\end{CJK}
$$
\lim _{x \rightarrow 0} \frac{\sum_{k=1}^{\infty} k x^{k}}{\sin x} .
$$
\begin{CJK}{UTF8}{mj}二\end{CJK}. \begin{CJK}{UTF8}{mj}设\end{CJK} $x \rightarrow 0$ \begin{CJK}{UTF8}{mj}时\end{CJK}, $f(x), g(x), \varphi(x) \rightarrow 0$, \begin{CJK}{UTF8}{mj}且\end{CJK} $\frac{f(x)}{g(x)} \rightarrow 1$, \begin{CJK}{UTF8}{mj}又\end{CJK} $\varphi(g(x)) \neq 0$, \begin{CJK}{UTF8}{mj}问\end{CJK}: \begin{CJK}{UTF8}{mj}是否必有\end{CJK} $\lim _{x \rightarrow 0} \frac{\varphi(f(x))}{\varphi(g(x))}=1$.

\begin{CJK}{UTF8}{mj}三\end{CJK}. \begin{CJK}{UTF8}{mj}设函数\end{CJK}
$$
f(x, y)= \begin{cases}\left(x^{2}+y^{2}\right) \sin \frac{1}{x^{2}+y^{2}}, & (x, y) \neq(0,0) \\ 0, & (x, y)=(0,0)\end{cases}
$$
(1) \begin{CJK}{UTF8}{mj}求\end{CJK} $f_{x}^{\prime}(0,0), f_{y}^{\prime}(0,0)$.

(2) \begin{CJK}{UTF8}{mj}证明\end{CJK}: $f_{x}^{\prime}(x, y), f_{y}^{\prime}(x, y)$ \begin{CJK}{UTF8}{mj}在\end{CJK} $(0,0)$ \begin{CJK}{UTF8}{mj}点的任何邻域内无界\end{CJK}.

(3) $f(x, y)$ \begin{CJK}{UTF8}{mj}在\end{CJK} $(0,0)$ \begin{CJK}{UTF8}{mj}处是否可微\end{CJK}.

\begin{CJK}{UTF8}{mj}四\end{CJK}. \begin{CJK}{UTF8}{mj}设函数\end{CJK} $f(x)=\sqrt{1-x}, x \in(-1,1)$ \begin{CJK}{UTF8}{mj}且\end{CJK} $f(x)=1+\sum_{n=1}^{\infty} C_{n} x^{n}$

(1) \begin{CJK}{UTF8}{mj}证明\end{CJK}: $C_{n} \leqslant 0, \forall n \in N$.

(2) \begin{CJK}{UTF8}{mj}求\end{CJK} $\sum_{n=1}^{\infty} C_{n}$.

\begin{CJK}{UTF8}{mj}五\end{CJK}. \begin{CJK}{UTF8}{mj}证明\end{CJK}: $\sum_{n=1}^{\infty} \frac{\sin n \theta}{n}=\frac{\pi-\theta}{2}, \theta \in(0,2 \pi)$. \begin{CJK}{UTF8}{mj}并讨论其一致连续性\end{CJK}.

\begin{CJK}{UTF8}{mj}六\end{CJK}. \begin{CJK}{UTF8}{mj}设\end{CJK} $f \in C[0,1], f(0)=1$, \begin{CJK}{UTF8}{mj}求极限\end{CJK} $\lim _{h \rightarrow 0} \int_{0}^{1} \frac{h}{h^{2}+x^{2}} f(x) \mathrm{d} x$.

\begin{CJK}{UTF8}{mj}七\end{CJK}. \begin{CJK}{UTF8}{mj}设\end{CJK} $\theta(x, y)=\arctan \frac{y}{x},(x, y) \in \mathbb{R}^{2} \backslash\{0\}$

(1) \begin{CJK}{UTF8}{mj}求\end{CJK} $\nabla \theta$.

(2) \begin{CJK}{UTF8}{mj}求曲线积分\end{CJK}
$$
\int_{\Gamma} \frac{\partial \theta}{\partial \vec{n}} \mathrm{~d} s
$$
\begin{CJK}{UTF8}{mj}其中\end{CJK} $\vec{n}$ \begin{CJK}{UTF8}{mj}是椭圆\end{CJK} $\Gamma: \frac{x^{2}}{a^{2}}+\frac{y^{2}}{b^{2}}=1$ \begin{CJK}{UTF8}{mj}上的单位法向量\end{CJK}.

\begin{CJK}{UTF8}{mj}八\end{CJK}. \begin{CJK}{UTF8}{mj}求第二型曲面积分\end{CJK}
$$
\int_{S} x z \mathrm{~d} x \mathrm{~d} y,
$$
\begin{CJK}{UTF8}{mj}其中\end{CJK} $S$ \begin{CJK}{UTF8}{mj}是八分之一球面\end{CJK}, \begin{CJK}{UTF8}{mj}即\end{CJK} $x^{2}+y^{2}+z^{2}=1, x \geqslant 0, y \geqslant 0, z \geqslant 0$ \begin{CJK}{UTF8}{mj}方向为外侧\end{CJK}.

\begin{CJK}{UTF8}{mj}九\end{CJK}. \begin{CJK}{UTF8}{mj}设\end{CJK} $\varphi(x)$ \begin{CJK}{UTF8}{mj}在\end{CJK} $[0,+\infty)$ \begin{CJK}{UTF8}{mj}上二阶可导\end{CJK}, $\varphi(0)=\varphi(1)=0$, \begin{CJK}{UTF8}{mj}且有有界函数\end{CJK} $g(x)$, \begin{CJK}{UTF8}{mj}使\end{CJK} $\varphi^{\prime \prime}(x)=e^{g(x)} \varphi(x)$, \begin{CJK}{UTF8}{mj}证明\end{CJK}:

(1) \begin{CJK}{UTF8}{mj}在\end{CJK} $[0,1]$ \begin{CJK}{UTF8}{mj}上\end{CJK}, $\varphi(x) \equiv 0$;

(2) \begin{CJK}{UTF8}{mj}在\end{CJK} $[0,+\infty)$ \begin{CJK}{UTF8}{mj}上\end{CJK}, $\varphi(x) \equiv 0$.

\begin{CJK}{UTF8}{mj}十\end{CJK}. \begin{CJK}{UTF8}{mj}设\end{CJK} $G(-,-) \in\left([0,1]^{2} ; R\right)$, \begin{CJK}{UTF8}{mj}且\end{CJK} $|G(x, y)|<1, \forall(x, y) \in[0,1]^{2}$, \begin{CJK}{UTF8}{mj}给定\end{CJK} $u(x) \in[0,1]$

(1) \begin{CJK}{UTF8}{mj}证明\end{CJK}: \begin{CJK}{UTF8}{mj}存在唯一函数\end{CJK} $\varphi(x) \in[0,1]$ \begin{CJK}{UTF8}{mj}满足\end{CJK} $\varphi(x)+\int_{0}^{1} G(x, y) \varphi(y) \mathrm{d} y=u(x)$.

(2) \begin{CJK}{UTF8}{mj}对于\end{CJK} $G(x, y)=\frac{1}{2} x y$, \begin{CJK}{UTF8}{mj}及\end{CJK} $u(x) \equiv 1$, \begin{CJK}{UTF8}{mj}求\end{CJK} $\varphi(x)$.

\section{9. 南京大学 2018 年研究生入学考试试题数学分析}
\begin{CJK}{UTF8}{mj}李扬\end{CJK}

\begin{CJK}{UTF8}{mj}微信公众号\end{CJK}: sxkyliyang

\begin{CJK}{UTF8}{mj}一\end{CJK}. \begin{CJK}{UTF8}{mj}设\end{CJK} $f(0)=0, f(x)=x^{\frac{1}{x}}(x>0)$, \begin{CJK}{UTF8}{mj}求\end{CJK} $f_{+}^{\prime}(0), f_{+}^{\prime \prime}(0)$.

\begin{CJK}{UTF8}{mj}二\end{CJK}. \begin{CJK}{UTF8}{mj}设\end{CJK} $f(x)=\frac{1}{\ln x}, x \in\left(0, \frac{1}{2}\right]$, \begin{CJK}{UTF8}{mj}讨论\end{CJK} $f(x)$ \begin{CJK}{UTF8}{mj}的一致连续性\end{CJK}, \begin{CJK}{UTF8}{mj}并判断\end{CJK} $f(x)$ \begin{CJK}{UTF8}{mj}是否\end{CJK} Holder \begin{CJK}{UTF8}{mj}连续\end{CJK}.

\begin{CJK}{UTF8}{mj}三\end{CJK}. \begin{CJK}{UTF8}{mj}设\end{CJK} $f(x)=\ln \frac{1+x}{1-x}$, \begin{CJK}{UTF8}{mj}求\end{CJK} $f(x)$ \begin{CJK}{UTF8}{mj}的\end{CJK} Taylor \begin{CJK}{UTF8}{mj}公式\end{CJK}, \begin{CJK}{UTF8}{mj}并计算\end{CJK} $\ln 2$ \begin{CJK}{UTF8}{mj}的近似值\end{CJK}.

\begin{CJK}{UTF8}{mj}四\end{CJK}. \begin{CJK}{UTF8}{mj}设\end{CJK} $f(x, y)=x^{2}+y^{2}-y^{99}$, \begin{CJK}{UTF8}{mj}求\end{CJK} $f$ \begin{CJK}{UTF8}{mj}的临界值\end{CJK}, \begin{CJK}{UTF8}{mj}并讨论临界值是否能成为局部极值或全局极值\end{CJK}.

\begin{CJK}{UTF8}{mj}五\end{CJK}. \begin{CJK}{UTF8}{mj}设\end{CJK} $f(x)=x^{2}, x \in[-\pi, \pi]$ \begin{CJK}{UTF8}{mj}是周期为\end{CJK} $2 \pi$ \begin{CJK}{UTF8}{mj}的函数\end{CJK}, \begin{CJK}{UTF8}{mj}求\end{CJK} $F(x)=\sin 100 x+f(x)$ \begin{CJK}{UTF8}{mj}的\end{CJK} Fourier \begin{CJK}{UTF8}{mj}展开\end{CJK}.

\begin{CJK}{UTF8}{mj}六\end{CJK}. \begin{CJK}{UTF8}{mj}设\end{CJK} $\Omega=\{(x, y) \mid 0 \leqslant x \leqslant 1, \phi(x) \leqslant y \leqslant \psi(x)\}$, \begin{CJK}{UTF8}{mj}且\end{CJK} $F(x, \phi(x))=0$, $F$ \begin{CJK}{UTF8}{mj}具有连续偏导数\end{CJK}, \begin{CJK}{UTF8}{mj}证明\end{CJK}:
$$
\int_{\Omega} F^{2} \leqslant \int_{\Omega}\left(\frac{\partial F}{\partial y}\right)^{2}
$$
\begin{CJK}{UTF8}{mj}七\end{CJK}. \begin{CJK}{UTF8}{mj}设\end{CJK} $B \subset \mathbb{R}^{d}, \Omega=\{x \mid\|x\| \leqslant 1\}, \Omega \subset B$ \begin{CJK}{UTF8}{mj}且\end{CJK} $|F(x)| \leqslant 1$, \begin{CJK}{UTF8}{mj}证明\end{CJK}: \begin{CJK}{UTF8}{mj}存在\end{CJK} $\xi$ \begin{CJK}{UTF8}{mj}使得\end{CJK} $\|\nabla F(\xi)\| \leqslant 2$.

\begin{CJK}{UTF8}{mj}八\end{CJK}. \begin{CJK}{UTF8}{mj}设\end{CJK} $B$ \begin{CJK}{UTF8}{mj}是\end{CJK} $\mathbb{R}^{n}$ \begin{CJK}{UTF8}{mj}的开集\end{CJK}, $\Omega$ \begin{CJK}{UTF8}{mj}是\end{CJK} $B$ \begin{CJK}{UTF8}{mj}的边界\end{CJK}, \begin{CJK}{UTF8}{mj}证明\end{CJK}:
$$
\int_{\Omega} u \frac{\partial u}{\partial n}=\int_{B} u \triangle u+\int_{B}\|\nabla u\|^{2} .
$$
\begin{CJK}{UTF8}{mj}九\end{CJK}. \begin{CJK}{UTF8}{mj}证明\end{CJK}:
$$
\lim _{n \rightarrow+\infty} \int_{0}^{1} f(x, \sin n t) \mathrm{d} x=\frac{1}{2 \pi} \int_{0}^{1} f(x, \sin y) \mathrm{d} y .
$$
\begin{CJK}{UTF8}{mj}十\end{CJK}. \begin{CJK}{UTF8}{mj}设\end{CJK} $\Omega=\{(u, v, w) \mid 0 \leqslant u \leqslant x, 0 \leqslant v \leqslant y, 0 \leqslant w \leqslant z\}, 0 \leqslant x, y, z \leqslant 1$. \begin{CJK}{UTF8}{mj}讨论方程\end{CJK}
$$
\phi(x, y, z)=x+\int_{\Omega} \phi(u, v, w)
$$
\begin{CJK}{UTF8}{mj}是否有唯一解\end{CJK}? \begin{CJK}{UTF8}{mj}有的话\end{CJK}, \begin{CJK}{UTF8}{mj}请写出具体表达式\end{CJK}.

\section{1. 南京师范大学 2008 年研究生入学考试试题高等代数 
 李扬 
 微信公众号: sxkyliyang}
\begin{CJK}{UTF8}{mj}一\end{CJK}. \begin{CJK}{UTF8}{mj}判断题\end{CJK}(\begin{CJK}{UTF8}{mj}共\end{CJK} 60 \begin{CJK}{UTF8}{mj}分\end{CJK}, \begin{CJK}{UTF8}{mj}每小题\end{CJK} 6 \begin{CJK}{UTF8}{mj}分\end{CJK}; \begin{CJK}{UTF8}{mj}若正确\end{CJK}, \begin{CJK}{UTF8}{mj}打\end{CJK} $\checkmark$ \begin{CJK}{UTF8}{mj}并给出证明\end{CJK}; \begin{CJK}{UTF8}{mj}若错误\end{CJK}, \begin{CJK}{UTF8}{mj}打\end{CJK} $\times$ \begin{CJK}{UTF8}{mj}并给出反例或说明理由\end{CJK}.)

\begin{enumerate}
  \item \begin{CJK}{UTF8}{mj}对多项式\end{CJK} $x^{8}+1$ \begin{CJK}{UTF8}{mj}来说\end{CJK}, \begin{CJK}{UTF8}{mj}不存在素数\end{CJK} $p$ \begin{CJK}{UTF8}{mj}满足艾森斯坦\end{CJK} (Eisenstein) \begin{CJK}{UTF8}{mj}判别法的条件\end{CJK}, \begin{CJK}{UTF8}{mj}故\end{CJK} $x^{8}+1$ \begin{CJK}{UTF8}{mj}不是有理数域上的\end{CJK} \begin{CJK}{UTF8}{mj}不可约多项式\end{CJK}.

  \item \begin{CJK}{UTF8}{mj}若数域\end{CJK} $P$ \begin{CJK}{UTF8}{mj}上多项式\end{CJK} $f(x)$ \begin{CJK}{UTF8}{mj}在复数域上有重根\end{CJK}, \begin{CJK}{UTF8}{mj}则在\end{CJK} $P$ \begin{CJK}{UTF8}{mj}上一定有重因式\end{CJK}.

  \item \begin{CJK}{UTF8}{mj}设向量组\end{CJK} $(I)$ \begin{CJK}{UTF8}{mj}的秩大于向量组\end{CJK} $(I I)$ \begin{CJK}{UTF8}{mj}的秩\end{CJK}, \begin{CJK}{UTF8}{mj}则\end{CJK} $(I)$ \begin{CJK}{UTF8}{mj}不能由\end{CJK} $(I I)$ \begin{CJK}{UTF8}{mj}线性表出\end{CJK}.

  \item \begin{CJK}{UTF8}{mj}设\end{CJK} $A, B$ \begin{CJK}{UTF8}{mj}都是\end{CJK} $n$ \begin{CJK}{UTF8}{mj}阶方阵\end{CJK}, $A$ \begin{CJK}{UTF8}{mj}是对角矩阵\end{CJK}, $A B=B A$, \begin{CJK}{UTF8}{mj}则\end{CJK} $B$ \begin{CJK}{UTF8}{mj}也是对角矩阵\end{CJK}.

  \item \begin{CJK}{UTF8}{mj}设\end{CJK} $A, B$ \begin{CJK}{UTF8}{mj}都是半正定矩阵\end{CJK}, \begin{CJK}{UTF8}{mj}则\end{CJK} $A B$ \begin{CJK}{UTF8}{mj}的特征值大于或者等于\end{CJK} 0 .

  \item \begin{CJK}{UTF8}{mj}设\end{CJK} $V_{i}(i=1,2, \cdots, s)$ \begin{CJK}{UTF8}{mj}是\end{CJK} $n$ \begin{CJK}{UTF8}{mj}维线性空间\end{CJK} $V$ \begin{CJK}{UTF8}{mj}的子空间\end{CJK}, $2 \leqslant s<n$, \begin{CJK}{UTF8}{mj}若\end{CJK} $V_{i} \cap V_{j}=\{0\}(i \neq j)$, \begin{CJK}{UTF8}{mj}则\end{CJK} $V_{1}+V_{2}+\cdots+V_{s}$ \begin{CJK}{UTF8}{mj}是直和\end{CJK}.

  \item \begin{CJK}{UTF8}{mj}实矩阵\end{CJK} $A \in \mathbb{R}^{m \times n}$ \begin{CJK}{UTF8}{mj}的秩为\end{CJK} $n$ \begin{CJK}{UTF8}{mj}的充要条件是对于任意的\end{CJK} $n$ \begin{CJK}{UTF8}{mj}阶实矩阵\end{CJK} $B, C$, \begin{CJK}{UTF8}{mj}由\end{CJK} $A B=A C$ \begin{CJK}{UTF8}{mj}可推得\end{CJK} $B=C$.

  \item \begin{CJK}{UTF8}{mj}设\end{CJK} $a, b$ \begin{CJK}{UTF8}{mj}属于数域\end{CJK} $P, V=\{f(x) \mid f(x) \in P[x], \partial(f(x))<n\} \cup\{0\}$, \begin{CJK}{UTF8}{mj}则\end{CJK} $V$ \begin{CJK}{UTF8}{mj}是一个线性空间\end{CJK}, \begin{CJK}{UTF8}{mj}并且\end{CJK}

\end{enumerate}
$$
\mathscr{A}: f(x) \rightarrow f(a x+b)
$$
\begin{CJK}{UTF8}{mj}是\end{CJK} $V$ \begin{CJK}{UTF8}{mj}上的一个线性变换\end{CJK}.

\begin{enumerate}
  \setcounter{enumi}{9}
  \item $\lambda$-\begin{CJK}{UTF8}{mj}矩阵\end{CJK} $A(\lambda)$ \begin{CJK}{UTF8}{mj}是可逆的当且仅当\end{CJK} $A(\lambda)$ \begin{CJK}{UTF8}{mj}的行列式\end{CJK} $|A(\lambda)| \neq 0$.

  \item \begin{CJK}{UTF8}{mj}在\end{CJK} $n$ \begin{CJK}{UTF8}{mj}维欧几里得空间中\end{CJK}, \begin{CJK}{UTF8}{mj}正交变换在一组基下的矩阵是正交矩阵\end{CJK}.

\end{enumerate}
\begin{CJK}{UTF8}{mj}二\end{CJK}. \begin{CJK}{UTF8}{mj}计算题\end{CJK}. (\begin{CJK}{UTF8}{mj}共\end{CJK} 40 \begin{CJK}{UTF8}{mj}分\end{CJK}, \begin{CJK}{UTF8}{mj}每小题\end{CJK} 10 \begin{CJK}{UTF8}{mj}分\end{CJK})

\begin{enumerate}
  \item \begin{CJK}{UTF8}{mj}设\end{CJK} $a_{i j}=\frac{\alpha_{i}^{n}-\beta_{j}^{n}}{\alpha_{i}-\beta_{j}}(i, j=1,2, \cdots, n), n$ \begin{CJK}{UTF8}{mj}阶方阵\end{CJK} $A=\left(a_{i j}\right)$, \begin{CJK}{UTF8}{mj}求\end{CJK} $A$ \begin{CJK}{UTF8}{mj}的行列式\end{CJK} $|A|$.

  \item \begin{CJK}{UTF8}{mj}求\end{CJK} $A=\left(\begin{array}{ccc}2 & 0 & 0 \\ -1 & 2 & 0 \\ 3 & 4 & -1\end{array}\right)$ \begin{CJK}{UTF8}{mj}的所有不变因子\end{CJK}, \begin{CJK}{UTF8}{mj}初等因子及若尔当\end{CJK} (Jordan) \begin{CJK}{UTF8}{mj}标准形\end{CJK}.

  \item \begin{CJK}{UTF8}{mj}设\end{CJK} $P[x]_{4}$ \begin{CJK}{UTF8}{mj}是所有次数小于\end{CJK} 4 \begin{CJK}{UTF8}{mj}的多项式和零多项式构成的线性空间\end{CJK}, \begin{CJK}{UTF8}{mj}求线性变换\end{CJK} $\mathscr{A}(f(x))=x^{2} f^{\prime \prime}(x)+f(x)+$ $f^{\prime}(x)$ \begin{CJK}{UTF8}{mj}的特征值\end{CJK}, \begin{CJK}{UTF8}{mj}求最大特征值的特征向量\end{CJK}.

  \item \begin{CJK}{UTF8}{mj}已知三维欧几里得空间\end{CJK} $V$ \begin{CJK}{UTF8}{mj}中一组基\end{CJK} $\alpha_{1}, \alpha_{2}, \alpha_{3}$, \begin{CJK}{UTF8}{mj}其度量矩阵为\end{CJK} $A=\left(\begin{array}{ccc}2 & -1 & 0 \\ -1 & 2 & 1 \\ 0 & 1 & 1\end{array}\right)$, \begin{CJK}{UTF8}{mj}求向量\end{CJK} $\beta=2 \alpha_{1}-\alpha_{3}$ \begin{CJK}{UTF8}{mj}的长度\end{CJK}.

\end{enumerate}
\begin{CJK}{UTF8}{mj}三\end{CJK}.\begin{CJK}{UTF8}{mj}证明题\end{CJK}( 1,2 \begin{CJK}{UTF8}{mj}题每小题\end{CJK} 10 \begin{CJK}{UTF8}{mj}分\end{CJK}; 3,4 \begin{CJK}{UTF8}{mj}题每小题\end{CJK} 15 \begin{CJK}{UTF8}{mj}分\end{CJK}. \begin{CJK}{UTF8}{mj}共\end{CJK} 50 \begin{CJK}{UTF8}{mj}分\end{CJK})

\begin{enumerate}
  \item \begin{CJK}{UTF8}{mj}设\end{CJK} $V$ \begin{CJK}{UTF8}{mj}是一个\end{CJK} $n$ \begin{CJK}{UTF8}{mj}维线性空间\end{CJK}, $V_{1}$ \begin{CJK}{UTF8}{mj}是一个\end{CJK} $r$ \begin{CJK}{UTF8}{mj}维子空间\end{CJK}, $r \leqslant \frac{n}{2}$, \begin{CJK}{UTF8}{mj}证明\end{CJK}: \begin{CJK}{UTF8}{mj}存在一个线性变换\end{CJK} $\mathscr{A}$, \begin{CJK}{UTF8}{mj}使得\end{CJK}
\end{enumerate}
$$
V_{1}=\mathscr{A}^{-1}(0) \subseteq \mathscr{A} V
$$

\begin{enumerate}
  \setcounter{enumi}{2}
  \item \begin{CJK}{UTF8}{mj}设分块实对称矩阵\end{CJK} $A=\left(\begin{array}{ccc}\alpha & \beta^{\prime} & 0 \\ \beta & A_{1} & \gamma \\ 0 & \gamma^{\prime} & b\end{array}\right)$, \begin{CJK}{UTF8}{mj}其中\end{CJK} $a, b \in \mathbb{R}, \beta, \gamma \in \mathbb{R}^{n}, A_{1} \in \mathbb{R}^{n \times n}$, \begin{CJK}{UTF8}{mj}证明\end{CJK}: $A$ \begin{CJK}{UTF8}{mj}正定的充要条件是\end{CJK} $a>0, b>0$ \begin{CJK}{UTF8}{mj}且矩阵\end{CJK} $A_{1}-\frac{1}{a} \beta \beta^{\prime}-\frac{1}{b} \gamma \gamma^{\prime}$ \begin{CJK}{UTF8}{mj}正定\end{CJK}.

  \item \begin{CJK}{UTF8}{mj}设\end{CJK} $\mathbb{C}[x]$ \begin{CJK}{UTF8}{mj}是由所有复系数多项式所构成的集合\end{CJK}, $A \in \mathbb{C}^{n \times n}$, \begin{CJK}{UTF8}{mj}令\end{CJK} $V=\{f(A) \mid f(x) \in \mathbb{C}[x]\}$, \begin{CJK}{UTF8}{mj}设\end{CJK} $A$ \begin{CJK}{UTF8}{mj}的最小多项式的\end{CJK} \begin{CJK}{UTF8}{mj}次数为\end{CJK} $m$, \begin{CJK}{UTF8}{mj}证明\end{CJK}:

\end{enumerate}
(1) $V$ \begin{CJK}{UTF8}{mj}是一个有限维线性空间\end{CJK};

(2) $E, A, A^{2}, \cdots, A^{m-1}$ \begin{CJK}{UTF8}{mj}构成\end{CJK} $V$ \begin{CJK}{UTF8}{mj}的一组基\end{CJK}.

\begin{enumerate}
  \setcounter{enumi}{4}
  \item \begin{CJK}{UTF8}{mj}设\end{CJK} $V$ \begin{CJK}{UTF8}{mj}是数域\end{CJK} $P$ \begin{CJK}{UTF8}{mj}上的有限维线性空间\end{CJK}, $\mathscr{A}$ \begin{CJK}{UTF8}{mj}是\end{CJK} $V$ \begin{CJK}{UTF8}{mj}上的线性变换\end{CJK}, $f(\lambda)=(\lambda-1)(\lambda-2)^{2}$ \begin{CJK}{UTF8}{mj}是\end{CJK} $\mathscr{A}$ \begin{CJK}{UTF8}{mj}的最小多项式\end{CJK}; \begin{CJK}{UTF8}{mj}再设\end{CJK} $V_{k}=\operatorname{Ker}(k \varepsilon-\mathscr{A})^{k}(k=1,2)$. \begin{CJK}{UTF8}{mj}其中\end{CJK} $\operatorname{Ker}(*)$ \begin{CJK}{UTF8}{mj}表示核空间\end{CJK}. \begin{CJK}{UTF8}{mj}证明\end{CJK}: $V=V_{1} \oplus V_{2}$.
\end{enumerate}
\section{2. 南京师范大学 2009 年研究生入学考试试题高等代数 
 李扬 
 微信公众号: sxkyliyang}
\begin{enumerate}
  \item ( 20 \begin{CJK}{UTF8}{mj}分\end{CJK}) $f(x)=x^{3}+a x^{2}+b x+c$ \begin{CJK}{UTF8}{mj}是整系数多项式\end{CJK}, \begin{CJK}{UTF8}{mj}若\end{CJK} $a, c$ \begin{CJK}{UTF8}{mj}是奇数\end{CJK}, $b$ \begin{CJK}{UTF8}{mj}是偶数\end{CJK}, \begin{CJK}{UTF8}{mj}证明\end{CJK}: $f(x)$ \begin{CJK}{UTF8}{mj}是有理数域上的不\end{CJK} \begin{CJK}{UTF8}{mj}可约多项式\end{CJK}.

  \item ( 20 \begin{CJK}{UTF8}{mj}分\end{CJK}) \begin{CJK}{UTF8}{mj}设\end{CJK} $\mathscr{A}$ \begin{CJK}{UTF8}{mj}是欧式空间\end{CJK} $V$ \begin{CJK}{UTF8}{mj}的一个正交变换\end{CJK}, $\lambda$ \begin{CJK}{UTF8}{mj}和\end{CJK} $\mu$ \begin{CJK}{UTF8}{mj}是\end{CJK} $\mathscr{A}$ \begin{CJK}{UTF8}{mj}的两个不同特征值\end{CJK}, \begin{CJK}{UTF8}{mj}设\end{CJK} $\mathscr{A}$ \begin{CJK}{UTF8}{mj}的属于\end{CJK} $\lambda$ \begin{CJK}{UTF8}{mj}的特征向量\end{CJK} \begin{CJK}{UTF8}{mj}为\end{CJK} $\alpha$, \begin{CJK}{UTF8}{mj}属于\end{CJK} $\mu$ \begin{CJK}{UTF8}{mj}的特征向量为\end{CJK} $\beta$. \begin{CJK}{UTF8}{mj}证明\end{CJK}: $\alpha$ \begin{CJK}{UTF8}{mj}与\end{CJK} $\beta$ \begin{CJK}{UTF8}{mj}是正交的\end{CJK}.

  \item ( 20 \begin{CJK}{UTF8}{mj}分\end{CJK}) \begin{CJK}{UTF8}{mj}设\end{CJK} $A, B$ \begin{CJK}{UTF8}{mj}为\end{CJK} $n$ \begin{CJK}{UTF8}{mj}级矩阵满足\end{CJK} $A^{2}+A=2 E, B^{2}=B$ \begin{CJK}{UTF8}{mj}且\end{CJK} $A B=B A$, \begin{CJK}{UTF8}{mj}证明\end{CJK}: \begin{CJK}{UTF8}{mj}存在可逆矩阵\end{CJK} $Q$ \begin{CJK}{UTF8}{mj}使得\end{CJK} $Q^{-1} A Q$ \begin{CJK}{UTF8}{mj}和\end{CJK} $Q^{-1} B Q$ \begin{CJK}{UTF8}{mj}都是对角矩阵\end{CJK}.

  \item ( 30 \begin{CJK}{UTF8}{mj}分\end{CJK}) \begin{CJK}{UTF8}{mj}求实二次型\end{CJK} $f(x)=\sum_{i=1}^{n}\left(x_{i}-\sum_{j=1}^{n} \frac{x_{j}}{n}\right)^{2}$ \begin{CJK}{UTF8}{mj}的矩阵及正负惯性指数\end{CJK}.

  \item (30 \begin{CJK}{UTF8}{mj}分\end{CJK}) \begin{CJK}{UTF8}{mj}设\end{CJK} $\mathscr{A}$ \begin{CJK}{UTF8}{mj}是有限维线性空间\end{CJK} $V$ \begin{CJK}{UTF8}{mj}的线性变换\end{CJK}, \begin{CJK}{UTF8}{mj}证明\end{CJK}: $V=\mathscr{A} V \oplus \mathscr{A}^{-1}(0)$ \begin{CJK}{UTF8}{mj}充要条件是\end{CJK} $\mathscr{A}^{2} V=\mathscr{A} V$.

  \item ( 30 \begin{CJK}{UTF8}{mj}分\end{CJK}) \begin{CJK}{UTF8}{mj}设\end{CJK} $n$ \begin{CJK}{UTF8}{mj}维线性空间上的线性变换\end{CJK} $\mathscr{A}$ \begin{CJK}{UTF8}{mj}的特征多项式为\end{CJK}

\end{enumerate}
$$
f(\lambda)=\left(\lambda-\lambda_{1}\right)^{n_{1}}\left(\lambda-\lambda_{2}\right)^{n_{2}}, \lambda_{1} \neq \lambda_{2}
$$
\begin{CJK}{UTF8}{mj}并且有\end{CJK}
$$
\begin{aligned}
&\mathscr{A} \alpha_{1}=\lambda_{1} \alpha_{1},\left(\mathscr{A}-\lambda_{1} \varepsilon\right) \alpha_{2}=\alpha_{1}, \cdots,\left(\mathscr{A}-\lambda_{1} \varepsilon\right) \alpha_{n_{1}}=\alpha_{n_{1}-1} \\
&\mathscr{A} \beta_{1}=\lambda_{2} \beta_{1},\left(\mathscr{A}-\lambda_{2} \varepsilon\right) \beta_{2}=\beta_{1}, \cdots,\left(\mathscr{A}-\lambda_{2} \varepsilon\right) \beta_{n_{2}}=\beta_{n_{2}-1}
\end{aligned}
$$
\begin{CJK}{UTF8}{mj}证明\end{CJK}: $\alpha_{1}, \alpha_{2}, \cdots, \alpha_{n_{1}}, \beta_{1}, \beta_{2}, \cdots, \beta_{n_{2}}$ \begin{CJK}{UTF8}{mj}构成整个线性空间的一组基\end{CJK}, \begin{CJK}{UTF8}{mj}并写出\end{CJK} $\mathscr{A}$ \begin{CJK}{UTF8}{mj}在这组基下的矩阵\end{CJK}.

\section{3. 南京师范大学 2010 年研究生入学考试试题高等代数 
 李扬 
 微信公众号: sxkyliyang}
\begin{enumerate}
  \item ( 15 \begin{CJK}{UTF8}{mj}分\end{CJK}) \begin{CJK}{UTF8}{mj}计算行列式\end{CJK}
\end{enumerate}
$$
D_{n}=\left|\begin{array}{cccccc}
7 & 4 & 0 & \cdots & 0 & 0 \\
3 & 7 & 4 & \cdots & 0 & 0 \\
0 & 3 & 7 & \cdots & 0 & 0 \\
\vdots & \vdots & \vdots & & \vdots & \vdots \\
0 & 0 & 0 & \cdots & 7 & 4 \\
0 & 0 & 0 & \cdots & 3 & 7
\end{array}\right|
$$

\begin{enumerate}
  \setcounter{enumi}{2}
  \item ( 15 \begin{CJK}{UTF8}{mj}分\end{CJK}) \begin{CJK}{UTF8}{mj}设整系数多项式\end{CJK} $f(x)=x^{4}+a x^{2}+b x-3$, \begin{CJK}{UTF8}{mj}记\end{CJK} $(f(x), g(x))$ \begin{CJK}{UTF8}{mj}为\end{CJK} $f(x)$ \begin{CJK}{UTF8}{mj}和\end{CJK} $g(x)$ \begin{CJK}{UTF8}{mj}的首项系数为\end{CJK} 1 \begin{CJK}{UTF8}{mj}的最大公\end{CJK} \begin{CJK}{UTF8}{mj}因式\end{CJK}, $f^{\prime}(x)$ \begin{CJK}{UTF8}{mj}为\end{CJK} $f(x)$ \begin{CJK}{UTF8}{mj}的导数\end{CJK}. \begin{CJK}{UTF8}{mj}若\end{CJK} $\frac{f(x)}{\left(f(x), f^{\prime}(x)\right)}$ \begin{CJK}{UTF8}{mj}为二次多项式\end{CJK}, \begin{CJK}{UTF8}{mj}求\end{CJK} $a^{2}+b^{2}$ \begin{CJK}{UTF8}{mj}的值\end{CJK}.

  \item (16 \begin{CJK}{UTF8}{mj}分\end{CJK}) \begin{CJK}{UTF8}{mj}设矩阵\end{CJK} $A=\left(\begin{array}{ccc}3 & 0 & 0 \\ -1 & 1 & 1 \\ 2 & 0 & 1\end{array}\right)$, \begin{CJK}{UTF8}{mj}求\end{CJK} $A$ \begin{CJK}{UTF8}{mj}的若尔当标准形和\end{CJK} $A$ \begin{CJK}{UTF8}{mj}的有理标准形\end{CJK}.

  \item ( 15 \begin{CJK}{UTF8}{mj}分\end{CJK}) \begin{CJK}{UTF8}{mj}设\end{CJK} $n$ \begin{CJK}{UTF8}{mj}级行列式\end{CJK} $D_{n}=\left|a_{i j}\right| \neq 0, A_{i j}$ \begin{CJK}{UTF8}{mj}为\end{CJK} $D_{n}$ \begin{CJK}{UTF8}{mj}中元素\end{CJK} $a_{i j}$ \begin{CJK}{UTF8}{mj}的代数余子式\end{CJK}, \begin{CJK}{UTF8}{mj}证明\end{CJK}: \begin{CJK}{UTF8}{mj}当\end{CJK} $r<n$ \begin{CJK}{UTF8}{mj}时\end{CJK}, \begin{CJK}{UTF8}{mj}线性方程组\end{CJK}

\end{enumerate}
$$
\left\{\begin{array}{l}
a_{11} x_{1}+a_{12} x_{2}+\cdots+a_{1 n} x_{n}=0 \\
a_{21} x_{1}+a_{22} x_{2}+\cdots+a_{2 n} x_{n}=0 \\
\cdots \cdots \cdots \\
a_{r 1} x_{1}+a_{r 2} x_{2}+\cdots+a_{r n} x_{n}=0
\end{array}\right.
$$
\begin{CJK}{UTF8}{mj}有一个基础解系为\end{CJK}: $\left(A_{j 1}, A_{j 2}, \cdots, A_{j n}\right), j=r+1, r+2, \cdots, n$.

\begin{enumerate}
  \setcounter{enumi}{5}
  \item ( 20 \begin{CJK}{UTF8}{mj}分\end{CJK}) \begin{CJK}{UTF8}{mj}设\end{CJK} $V$ \begin{CJK}{UTF8}{mj}是由数域\end{CJK} $F$ \begin{CJK}{UTF8}{mj}上\end{CJK} $x$ \begin{CJK}{UTF8}{mj}的次数小于\end{CJK} $n$ \begin{CJK}{UTF8}{mj}的全体多项式\end{CJK}, \begin{CJK}{UTF8}{mj}再添上零多项式构成的线性空间\end{CJK}, \begin{CJK}{UTF8}{mj}定义\end{CJK} $V$ \begin{CJK}{UTF8}{mj}上的\end{CJK} \begin{CJK}{UTF8}{mj}线性变换\end{CJK} $\mathscr{A}$, \begin{CJK}{UTF8}{mj}使\end{CJK} $\mathscr{A}(f(x))=x f^{\prime}(x)-f(x)$, \begin{CJK}{UTF8}{mj}其中\end{CJK} $f^{\prime}(x)$ \begin{CJK}{UTF8}{mj}为\end{CJK} $f(x)$ \begin{CJK}{UTF8}{mj}的导数\end{CJK}.
\end{enumerate}
(1) \begin{CJK}{UTF8}{mj}求\end{CJK} $\mathscr{A}$ \begin{CJK}{UTF8}{mj}的核\end{CJK} $\mathscr{A}^{-1}(0)$ \begin{CJK}{UTF8}{mj}与值域\end{CJK} $\mathscr{A} V$;

(2) \begin{CJK}{UTF8}{mj}证明线性空间\end{CJK} $V$ \begin{CJK}{UTF8}{mj}是\end{CJK} $\mathscr{A}^{-1}(0)$ \begin{CJK}{UTF8}{mj}与\end{CJK} $\mathscr{A} V$ \begin{CJK}{UTF8}{mj}的直和\end{CJK}.

\begin{enumerate}
  \setcounter{enumi}{6}
  \item ( 15 \begin{CJK}{UTF8}{mj}分\end{CJK}) \begin{CJK}{UTF8}{mj}设\end{CJK} $A=\left(\begin{array}{lll}2 & 4 & 2 \\ 1 & 3 & 0 \\ 1 & 2 & 1\end{array}\right)$, \begin{CJK}{UTF8}{mj}请把\end{CJK} $A$ \begin{CJK}{UTF8}{mj}分解为一个可逆矩阵\end{CJK} $B$ \begin{CJK}{UTF8}{mj}和一个幂等矩阵\end{CJK} $C$ (\begin{CJK}{UTF8}{mj}即\end{CJK} $C^{2}=C$ ) \begin{CJK}{UTF8}{mj}的乘积\end{CJK}.

  \item (14 \begin{CJK}{UTF8}{mj}分\end{CJK}) \begin{CJK}{UTF8}{mj}已知\end{CJK} $A, B$ \begin{CJK}{UTF8}{mj}都为\end{CJK} $n$ \begin{CJK}{UTF8}{mj}级正定矩阵\end{CJK}, \begin{CJK}{UTF8}{mj}证明\end{CJK}:

\end{enumerate}
(1) $A$ \begin{CJK}{UTF8}{mj}中绝对值最大的元素在主对角线上\end{CJK};

(2) $|A+B|>|A|+|B|$.

\begin{enumerate}
  \setcounter{enumi}{8}
  \item ( 15 \begin{CJK}{UTF8}{mj}分\end{CJK}) \begin{CJK}{UTF8}{mj}设\end{CJK} $A, B$ \begin{CJK}{UTF8}{mj}为复数域上的\end{CJK} $n$ \begin{CJK}{UTF8}{mj}级矩阵\end{CJK}, \begin{CJK}{UTF8}{mj}且\end{CJK} $A$ \begin{CJK}{UTF8}{mj}和\end{CJK} $B$ \begin{CJK}{UTF8}{mj}无公共特征根\end{CJK}, \begin{CJK}{UTF8}{mj}证明\end{CJK}: \begin{CJK}{UTF8}{mj}关于\end{CJK} $X$ \begin{CJK}{UTF8}{mj}的矩阵方程\end{CJK} $A X=X B$ \begin{CJK}{UTF8}{mj}只\end{CJK} \begin{CJK}{UTF8}{mj}有零解\end{CJK}.

  \item ( 15 \begin{CJK}{UTF8}{mj}分\end{CJK}) \begin{CJK}{UTF8}{mj}设复数\end{CJK} $c \neq 0$ \begin{CJK}{UTF8}{mj}为某个非零有理系数多项式的根\end{CJK}, \begin{CJK}{UTF8}{mj}记\end{CJK} $M=\{f(x) \mid f(x)$ \begin{CJK}{UTF8}{mj}为有理系数多项式\end{CJK}, $f(c)=0\}$.

\end{enumerate}
(1) \begin{CJK}{UTF8}{mj}证明\end{CJK}: $M$ \begin{CJK}{UTF8}{mj}中存在唯一的首项系数为\end{CJK} 1\begin{CJK}{UTF8}{mj}的有理数域上的不可约多项式\end{CJK} $p(x)$, \begin{CJK}{UTF8}{mj}使得对任意的\end{CJK} $f(x) \in M$ \begin{CJK}{UTF8}{mj}都有\end{CJK} $p(x) \mid f(x)$ \begin{CJK}{UTF8}{mj}成立\end{CJK};

(2) \begin{CJK}{UTF8}{mj}证明\end{CJK}: \begin{CJK}{UTF8}{mj}存在有理数域上的多项式\end{CJK} $g(x)$, \begin{CJK}{UTF8}{mj}使得\end{CJK} $g(c)=\frac{1}{c}$;

(3) \begin{CJK}{UTF8}{mj}令\end{CJK} $c=\sqrt{3}+\mathrm{i}$, \begin{CJK}{UTF8}{mj}求\end{CJK} (1) \begin{CJK}{UTF8}{mj}中的\end{CJK} $p(x)$. 10. (15 \begin{CJK}{UTF8}{mj}分\end{CJK}) \begin{CJK}{UTF8}{mj}设\end{CJK} $n$ \begin{CJK}{UTF8}{mj}级循环矩阵\end{CJK} $A=\left(\begin{array}{cccccc}a_{0} & a_{1} & a_{2} & \cdots & a_{n-2} & a_{n-1} \\ a_{n-1} & a_{0} & a_{1} & \cdots & a_{n-3} & a_{n-2} \\ a_{n-2} & a_{n-1} & a_{0} & \cdots & a_{n-4} & a_{n-3} \\ \vdots & \vdots & \vdots & & \vdots & \vdots \\ a_{2} & a_{3} & a_{4} & \cdots & a_{0} & a_{1} \\ a_{1} & a_{2} & a_{3} & \cdots & a_{n-1} & a_{0}\end{array}\right)$.

(1) \begin{CJK}{UTF8}{mj}试把\end{CJK} $A$ \begin{CJK}{UTF8}{mj}表示为一个\end{CJK} $n$ \begin{CJK}{UTF8}{mj}级可逆矩阵\end{CJK} $T$ \begin{CJK}{UTF8}{mj}的多项式\end{CJK};

(2) \begin{CJK}{UTF8}{mj}证明\end{CJK}: \begin{CJK}{UTF8}{mj}所有的\end{CJK} $n$ \begin{CJK}{UTF8}{mj}级循环矩阵在复数域上可以同时对角化\end{CJK}.

\section{4. 南京师范大学 2011 年研究生入学考试试题高等代数 
 李扬 
 微信公众号: sxkyliyang}
\begin{enumerate}
  \item ( 15 \begin{CJK}{UTF8}{mj}分\end{CJK}) \begin{CJK}{UTF8}{mj}计算行列式\end{CJK}
\end{enumerate}
$$
D=\left|\begin{array}{cccccc}
x & a & a & \cdots & a & a \\
-a & x & a & \cdots & a & a \\
-a & -a & x & \cdots & a & a \\
\vdots & \vdots & \vdots & & \vdots & \vdots \\
-a & -a & -a & \cdots & x & a \\
-a & -a & -a & \cdots & -a & x
\end{array}\right|
$$

\begin{enumerate}
  \setcounter{enumi}{2}
  \item ( 15 \begin{CJK}{UTF8}{mj}分\end{CJK}) \begin{CJK}{UTF8}{mj}设\end{CJK} $f_{1}(x), f_{2}(x)$ \begin{CJK}{UTF8}{mj}是数域\end{CJK} $P$ \begin{CJK}{UTF8}{mj}上的两个多项式\end{CJK}, \begin{CJK}{UTF8}{mj}满足\end{CJK} $\left(x^{2}+x+1\right) \mid f_{1}\left(x^{3}\right)+x f_{2}\left(x^{3}\right)$. \begin{CJK}{UTF8}{mj}证明\end{CJK}:
\end{enumerate}
$$
(x-1) \mid\left(f_{1}(x), f_{2}(x)\right) .
$$

\begin{enumerate}
  \setcounter{enumi}{3}
  \item (10 \begin{CJK}{UTF8}{mj}分\end{CJK}) \begin{CJK}{UTF8}{mj}设\end{CJK} $n$ \begin{CJK}{UTF8}{mj}级实矩阵\end{CJK} $A=\left(a_{i j}\right)$ \begin{CJK}{UTF8}{mj}满足\end{CJK}: \begin{CJK}{UTF8}{mj}对任意的\end{CJK} $1 \leqslant i, j \leqslant n$ \begin{CJK}{UTF8}{mj}且\end{CJK} $i \neq j$, \begin{CJK}{UTF8}{mj}不等式\end{CJK} $\left|a_{i i} a_{j j}\right|>\left(\sum_{k \neq i}\left|a_{i k}\right|\right)\left(\sum_{t \neq j}\left|a_{j t}\right|\right)$ \begin{CJK}{UTF8}{mj}成立\end{CJK}. \begin{CJK}{UTF8}{mj}证明\end{CJK}: $|A| \neq 0$.

  \item ( 15 \begin{CJK}{UTF8}{mj}分\end{CJK}) \begin{CJK}{UTF8}{mj}设\end{CJK} $A$ \begin{CJK}{UTF8}{mj}为\end{CJK} $n$ \begin{CJK}{UTF8}{mj}级实对称半正定矩阵\end{CJK}, $B$ \begin{CJK}{UTF8}{mj}为\end{CJK} $n$ \begin{CJK}{UTF8}{mj}级正定矩阵\end{CJK}, \begin{CJK}{UTF8}{mj}证明\end{CJK}: $|A+B| \geqslant|B|$.

  \item (15 \begin{CJK}{UTF8}{mj}分\end{CJK}) \begin{CJK}{UTF8}{mj}设\end{CJK} $A$ \begin{CJK}{UTF8}{mj}是一个\end{CJK} $n$ \begin{CJK}{UTF8}{mj}级矩阵\end{CJK}, \begin{CJK}{UTF8}{mj}证明\end{CJK}:

\end{enumerate}
(1) $A$ \begin{CJK}{UTF8}{mj}是反对称矩阵当且仅当对任一个\end{CJK} $n$ \begin{CJK}{UTF8}{mj}维向量\end{CJK} $X$, \begin{CJK}{UTF8}{mj}有\end{CJK} $X^{\prime} A X=0 ;\left(X^{\prime}\right.$ \begin{CJK}{UTF8}{mj}表示\end{CJK} $X$ \begin{CJK}{UTF8}{mj}的转置\end{CJK})

(2) \begin{CJK}{UTF8}{mj}如果\end{CJK} $A$ \begin{CJK}{UTF8}{mj}是对称矩阵\end{CJK}, \begin{CJK}{UTF8}{mj}且对任一个\end{CJK} $n$ \begin{CJK}{UTF8}{mj}维向量\end{CJK} $X$ \begin{CJK}{UTF8}{mj}有\end{CJK} $X^{\prime} A X=0$, \begin{CJK}{UTF8}{mj}那么\end{CJK} $A=0$.

\begin{enumerate}
  \setcounter{enumi}{6}
  \item ( 15 \begin{CJK}{UTF8}{mj}分\end{CJK}) \begin{CJK}{UTF8}{mj}设\end{CJK} $V_{1}$ \begin{CJK}{UTF8}{mj}与\end{CJK} $V_{2}$ \begin{CJK}{UTF8}{mj}分别是齐次方程组\end{CJK} $x_{1}+x_{2}+\cdots+x_{n}=0$ \begin{CJK}{UTF8}{mj}与\end{CJK} $x_{1}=x_{2}=\cdots=x_{n}$ \begin{CJK}{UTF8}{mj}的解空间\end{CJK}. \begin{CJK}{UTF8}{mj}证明\end{CJK}: $P^{n}=V_{1} \oplus V_{2} .$

  \item ( 20 \begin{CJK}{UTF8}{mj}分\end{CJK}) \begin{CJK}{UTF8}{mj}设三维线性空间\end{CJK} $V$ \begin{CJK}{UTF8}{mj}上的线性变换\end{CJK} $\mathscr{A}$ \begin{CJK}{UTF8}{mj}在基\end{CJK} $\varepsilon_{1}, \varepsilon_{2}, \varepsilon_{3}$ \begin{CJK}{UTF8}{mj}下的矩阵为\end{CJK} $A=\left(\begin{array}{lll}a_{11} & a_{12} & a_{13} \\ a_{21} & a_{22} & a_{23} \\ a_{31} & a_{32} & a_{33}\end{array}\right)$.

\end{enumerate}
(1) \begin{CJK}{UTF8}{mj}求\end{CJK} $\mathscr{A}$ \begin{CJK}{UTF8}{mj}在基\end{CJK} $\varepsilon_{3}, \varepsilon_{2}, \varepsilon_{1}$ \begin{CJK}{UTF8}{mj}下的矩阵\end{CJK};

(2) \begin{CJK}{UTF8}{mj}求\end{CJK} $\mathscr{A}$ \begin{CJK}{UTF8}{mj}在基\end{CJK} $\varepsilon_{1}, k \varepsilon_{2}, \varepsilon_{3}$ \begin{CJK}{UTF8}{mj}下的矩阵\end{CJK}, \begin{CJK}{UTF8}{mj}其中\end{CJK} $k \in P$ \begin{CJK}{UTF8}{mj}且\end{CJK} $k \neq 0$;

(3) \begin{CJK}{UTF8}{mj}求\end{CJK} $\mathscr{A}$ \begin{CJK}{UTF8}{mj}在基\end{CJK} $\varepsilon_{1}+\varepsilon_{2}, \varepsilon_{2}, \varepsilon_{3}$ \begin{CJK}{UTF8}{mj}下的矩阵\end{CJK}.

\begin{enumerate}
  \setcounter{enumi}{8}
  \item ( 20 \begin{CJK}{UTF8}{mj}分\end{CJK}) \begin{CJK}{UTF8}{mj}用正交线性替换化下列二次型为标准形\end{CJK}:
\end{enumerate}
$$
x_{1}^{2}-2 x_{2}^{2}-2 x_{3}^{2}-4 x_{1} x_{2}+4 x_{1} x_{3}+8 x_{2} x_{3}
$$

\begin{enumerate}
  \setcounter{enumi}{9}
  \item (15 \begin{CJK}{UTF8}{mj}分\end{CJK}) \begin{CJK}{UTF8}{mj}设\end{CJK} $A=\left(\begin{array}{ccc}-2 & 4 & 3 \\ 0 & 0 & 0 \\ -1 & 5 & 2\end{array}\right)$, \begin{CJK}{UTF8}{mj}求\end{CJK} $A^{520}+3 A^{70}-7 E$. (\begin{CJK}{UTF8}{mj}其中\end{CJK} $E$ \begin{CJK}{UTF8}{mj}为单位矩阵\end{CJK})

  \item ( 10 \begin{CJK}{UTF8}{mj}分\end{CJK}) \begin{CJK}{UTF8}{mj}证明\end{CJK}: \begin{CJK}{UTF8}{mj}任一\end{CJK} $n$ \begin{CJK}{UTF8}{mj}级方阵和它的转置矩阵相似\end{CJK}.

\end{enumerate}
\section{5. 南京师范大学 2012 年研究生入学考试试题高等代数 
 李扬 
 微信公众号: sxkyliyang}
\begin{enumerate}
  \item ( 15 \begin{CJK}{UTF8}{mj}分\end{CJK}) \begin{CJK}{UTF8}{mj}设对任意非负整数\end{CJK} $n$, \begin{CJK}{UTF8}{mj}令\end{CJK} $f_{n}(x)=x^{n+2}-(x+1)^{2 n+1}$. \begin{CJK}{UTF8}{mj}设多项式\end{CJK} $g(x)=f_{1}(x) f_{2}(x) \cdots f_{2012}(x)$, \begin{CJK}{UTF8}{mj}证明\end{CJK}:
\end{enumerate}
$$
\left(x^{2}+x+1, g(x)\right)=1
$$
\includegraphics[max width=\textwidth]{2022_04_18_a5c47c0ff534501b502eg-033}

\begin{enumerate}
  \setcounter{enumi}{3}
  \item ( 20 \begin{CJK}{UTF8}{mj}分\end{CJK}) \begin{CJK}{UTF8}{mj}解线性方程组\end{CJK}
\end{enumerate}
$$
\left\{\begin{array}{l}
x_{1}+2 x_{2}+3 x_{3}+3 x_{4}+7 x_{5}=4 \\
3 x_{1}+2 x_{2}+x_{3}+x_{4}-3 x_{5}=0 \\
x_{2}+2 x_{3}+2 x_{4}+6 x_{5}=3 \\
5 x_{1}+4 x_{2}+3 x_{3}+3 x_{4}-x_{5}=2
\end{array}\right.
$$

\begin{enumerate}
  \setcounter{enumi}{4}
  \item ( 20 \begin{CJK}{UTF8}{mj}分\end{CJK}) \begin{CJK}{UTF8}{mj}设\end{CJK} $n$ \begin{CJK}{UTF8}{mj}级实方阵\end{CJK} $A=\left(a_{i j}\right)$ \begin{CJK}{UTF8}{mj}满足\end{CJK} $A^{2}=A$, \begin{CJK}{UTF8}{mj}证明\end{CJK}: $A$ \begin{CJK}{UTF8}{mj}的秩等于\end{CJK} $a_{11}+a_{22}+\cdots+a_{n n}$.

  \item ( 20 \begin{CJK}{UTF8}{mj}分\end{CJK}) \begin{CJK}{UTF8}{mj}求二次型\end{CJK} $f\left(x_{1}, x_{2}, x_{3}\right)=5 x_{1}^{2}+x_{2}^{2}+6 x_{3}^{2}+4 x_{1} x_{2}-10 x_{1} x_{3}-6 x_{2} x_{3}$ \begin{CJK}{UTF8}{mj}的矩阵并判别该二次型是否正定\end{CJK}.

  \item (20 \begin{CJK}{UTF8}{mj}分\end{CJK}) \begin{CJK}{UTF8}{mj}设\end{CJK} $A=\left(\begin{array}{ccc}3 & 1 & 0 \\ 0 & 3 & 1 \\ 0 & 0 & 3\end{array}\right)$, \begin{CJK}{UTF8}{mj}令\end{CJK} $V=\{B \mid A B=B A, B$ \begin{CJK}{UTF8}{mj}为实方阵\end{CJK} $\}$.

\end{enumerate}
(1) \begin{CJK}{UTF8}{mj}证明\end{CJK} $V$ \begin{CJK}{UTF8}{mj}是实数域上的线性空间\end{CJK};

(2) \begin{CJK}{UTF8}{mj}求\end{CJK} $V$ \begin{CJK}{UTF8}{mj}的一组基\end{CJK}.

\begin{enumerate}
  \setcounter{enumi}{7}
  \item ( 20 \begin{CJK}{UTF8}{mj}分\end{CJK}) \begin{CJK}{UTF8}{mj}设\end{CJK} $\sigma$ \begin{CJK}{UTF8}{mj}是\end{CJK} $n$ \begin{CJK}{UTF8}{mj}维线性空间\end{CJK} $V$ \begin{CJK}{UTF8}{mj}的线性变换\end{CJK}, \begin{CJK}{UTF8}{mj}证明\end{CJK}: $\sigma(V)$ \begin{CJK}{UTF8}{mj}的一组基的原像及\end{CJK} $\sigma^{-1}(0)$ \begin{CJK}{UTF8}{mj}的一组基合起来就是\end{CJK} $V$ \begin{CJK}{UTF8}{mj}的一组基\end{CJK}.

  \item ( 10 \begin{CJK}{UTF8}{mj}分\end{CJK}) \begin{CJK}{UTF8}{mj}设\end{CJK} $n$ \begin{CJK}{UTF8}{mj}级实方阵\end{CJK} $A=\left(a_{i j}\right)$ \begin{CJK}{UTF8}{mj}满足条件\end{CJK}:

\end{enumerate}
(1) $a_{i i}>0, i=1,2, \cdots, n$;

(2) $a_{i j}<0,1 \leqslant i \neq j \leqslant n$;

(3) $\sum_{j=1}^{n} a_{i j}=0, i=1,2, \cdots, n$.

\begin{CJK}{UTF8}{mj}证明\end{CJK}: $A$ \begin{CJK}{UTF8}{mj}的秩为\end{CJK} $n-1$.

\begin{enumerate}
  \setcounter{enumi}{9}
  \item (10 \begin{CJK}{UTF8}{mj}分\end{CJK}) \begin{CJK}{UTF8}{mj}证明\end{CJK}: \begin{CJK}{UTF8}{mj}在\end{CJK} $n$ \begin{CJK}{UTF8}{mj}维欧式空间中\end{CJK}, \begin{CJK}{UTF8}{mj}至多有\end{CJK} $n+1$ \begin{CJK}{UTF8}{mj}个向量使得其中任意两个向量之间的夹角均大于\end{CJK} $90^{\circ}$.
\end{enumerate}
\section{6. 南京师范大学 2013 年研究生入学考试试题高等代数 
 李扬 
 微信公众号: sxkyliyang}
\begin{enumerate}
  \item ( 20 \begin{CJK}{UTF8}{mj}分\end{CJK}, \begin{CJK}{UTF8}{mj}每题\end{CJK} 5 \begin{CJK}{UTF8}{mj}分\end{CJK}) \begin{CJK}{UTF8}{mj}叙述题\end{CJK}:
\end{enumerate}
(1) \begin{CJK}{UTF8}{mj}艾森斯坦\end{CJK}(Eisenstein) \begin{CJK}{UTF8}{mj}判别法\end{CJK};

(2) \begin{CJK}{UTF8}{mj}克拉默\end{CJK} (Cramer) \begin{CJK}{UTF8}{mj}法则\end{CJK};

(3) \begin{CJK}{UTF8}{mj}哈密顿\end{CJK}-\begin{CJK}{UTF8}{mj}凯莱\end{CJK}(Hamilton-Caylay) \begin{CJK}{UTF8}{mj}定理\end{CJK};

(4) \begin{CJK}{UTF8}{mj}正交变换\end{CJK}.

\begin{enumerate}
  \setcounter{enumi}{2}
  \item ( 15 \begin{CJK}{UTF8}{mj}分\end{CJK}) \begin{CJK}{UTF8}{mj}设\end{CJK} $f(x)$ \begin{CJK}{UTF8}{mj}为有理数域上的非零多项式\end{CJK}, \begin{CJK}{UTF8}{mj}如果\end{CJK} $f(\sqrt{3})=0$, \begin{CJK}{UTF8}{mj}证明\end{CJK}: \begin{CJK}{UTF8}{mj}在有理数域上\end{CJK} $x^{3}-2$ \begin{CJK}{UTF8}{mj}整除\end{CJK} $f(x)$.

  \item ( 15 \begin{CJK}{UTF8}{mj}分\end{CJK}) \begin{CJK}{UTF8}{mj}设\end{CJK} $x_{1}, x_{2}, \cdots, x_{n}$ \begin{CJK}{UTF8}{mj}为\end{CJK} $n$ \begin{CJK}{UTF8}{mj}个实数\end{CJK}, \begin{CJK}{UTF8}{mj}令\end{CJK} $s_{k}=x_{1}^{k}+x_{2}^{k}+\cdots+x_{n}^{k}$. \begin{CJK}{UTF8}{mj}计算行列式\end{CJK}:

\end{enumerate}
$$
D=\left|\begin{array}{cccc}
S_{1} & S_{2} & \cdots & S_{n} \\
S_{2} & S_{3} & \cdots & S_{n+1} \\
\vdots & \vdots & & \vdots \\
S_{n} & S_{n+1} & \cdots & S_{2 n-1}
\end{array}\right|
$$

\begin{enumerate}
  \setcounter{enumi}{4}
  \item (15 \begin{CJK}{UTF8}{mj}分\end{CJK}) \begin{CJK}{UTF8}{mj}设矩阵\end{CJK} $A=\left(\begin{array}{cccc}a_{11} & a_{12} & \cdots & a_{1 n} \\ a_{21} & a_{22} & \cdots & a_{2 n} \\ \vdots & \vdots & & \vdots \\ a_{n 1} & a_{n 2} & \cdots & a_{n n}\end{array}\right)$.
\end{enumerate}
\begin{CJK}{UTF8}{mj}满足条件\end{CJK}:

(1) $a_{i i}>0, i=1,2, \cdots, n$;

(2) $a_{i j}<0, i \neq j$

(3) $a_{i 1}+a_{i 2}+\cdots+a_{i n}=0, i=1,2, \cdots, n$.

\begin{CJK}{UTF8}{mj}证明\end{CJK}: $A$ \begin{CJK}{UTF8}{mj}的秩为\end{CJK} $n-1$.

\begin{enumerate}
  \setcounter{enumi}{5}
  \item ( 15 \begin{CJK}{UTF8}{mj}分\end{CJK}) \begin{CJK}{UTF8}{mj}设矩阵\end{CJK} $A$ \begin{CJK}{UTF8}{mj}是实对称矩阵\end{CJK}, \begin{CJK}{UTF8}{mj}证明\end{CJK}: \begin{CJK}{UTF8}{mj}当实数\end{CJK} $\lambda$ \begin{CJK}{UTF8}{mj}充分大之后\end{CJK}, $\lambda E+A$ \begin{CJK}{UTF8}{mj}是正定矩阵\end{CJK}.

  \item (15 \begin{CJK}{UTF8}{mj}分\end{CJK}) \begin{CJK}{UTF8}{mj}在\end{CJK} $P^{4}$ \begin{CJK}{UTF8}{mj}中\end{CJK}, \begin{CJK}{UTF8}{mj}求由齐次线性方程组\end{CJK}

\end{enumerate}
$$
\left\{\begin{array}{l}
3 x_{1}+2 x_{2}-5 x_{3}+4 x_{4}=0 \\
3 x_{1}-x_{2}+3 x_{3}-3 x_{4}=0 \\
3 x_{1}+5 x_{2}-13 x_{3}+11 x_{4}=0
\end{array}\right.
$$
\begin{CJK}{UTF8}{mj}确定的解空间的基和维数\end{CJK}.

\begin{enumerate}
  \setcounter{enumi}{7}
  \item (15 \begin{CJK}{UTF8}{mj}分\end{CJK}) \begin{CJK}{UTF8}{mj}设\end{CJK} $A$ \begin{CJK}{UTF8}{mj}是\end{CJK} $n$ \begin{CJK}{UTF8}{mj}级实对称矩阵并且恰好有\end{CJK} $r$ \begin{CJK}{UTF8}{mj}个不同的特征值\end{CJK} $\lambda_{1}, \lambda_{2}, \cdots, \lambda_{r}$. \begin{CJK}{UTF8}{mj}证明存在矩阵\end{CJK} $A_{1}, A_{2}, \cdots, A_{r}$ \begin{CJK}{UTF8}{mj}满足条件\end{CJK}:
\end{enumerate}
(1) $A_{1}+A_{2}+\cdots+A_{r}=E_{n}$;

(2) $A_{i}^{2}=A_{i}, i=1,2, \cdots, r$;

(3) $A_{i} A_{j}=0, i \neq j$;

(4) $A=\lambda_{1} A_{1}+\lambda_{2} A_{2}+\cdots+\lambda_{r} A_{r}$.

\begin{enumerate}
  \setcounter{enumi}{8}
  \item ( 20 \begin{CJK}{UTF8}{mj}分\end{CJK}) \begin{CJK}{UTF8}{mj}设\end{CJK} $A$ \begin{CJK}{UTF8}{mj}是\end{CJK} $n$ \begin{CJK}{UTF8}{mj}级实矩阵满足\end{CJK} $A^{2}=2 A+3 E_{n}$. \begin{CJK}{UTF8}{mj}证明\end{CJK}:
\end{enumerate}
(1) $A$ \begin{CJK}{UTF8}{mj}相似于一个对角矩阵\end{CJK};

(2) $A+2 E_{n}$ \begin{CJK}{UTF8}{mj}是可逆矩阵\end{CJK}. 9. ( 20 \begin{CJK}{UTF8}{mj}分\end{CJK}) \begin{CJK}{UTF8}{mj}设矩阵\end{CJK} $A=\left(\begin{array}{ll}1 & 3 \\ 4 & 2\end{array}\right)$, \begin{CJK}{UTF8}{mj}多项式\end{CJK} $g(x)=x^{2012}+x-1$, \begin{CJK}{UTF8}{mj}计算矩阵\end{CJK} $g(A)$ \begin{CJK}{UTF8}{mj}的行列式\end{CJK}.

\section{7. 南京师范大学 2014 年研究生入学考试试题高等代数 
 李扬 
 微信公众号: sxkyliyang}
\begin{enumerate}
  \item (15 \begin{CJK}{UTF8}{mj}分\end{CJK}) \begin{CJK}{UTF8}{mj}求一个次数最低的实系数多项式\end{CJK}, \begin{CJK}{UTF8}{mj}使其被\end{CJK} $x^{2}+1$ \begin{CJK}{UTF8}{mj}除余式为\end{CJK} $x+1$, \begin{CJK}{UTF8}{mj}被\end{CJK} $x^{3}+x^{2}+1$ \begin{CJK}{UTF8}{mj}除余式为\end{CJK} $x^{2}-1$.

  \item ( 15 \begin{CJK}{UTF8}{mj}分\end{CJK}) \begin{CJK}{UTF8}{mj}设\end{CJK} $n$ \begin{CJK}{UTF8}{mj}级实矩阵\end{CJK} $A=\left(a_{i j}\right)$ \begin{CJK}{UTF8}{mj}满足\end{CJK} $|A|=1$ \begin{CJK}{UTF8}{mj}且\end{CJK} $a_{i j}+a_{j i}=0, i, j=1,2, \cdots, n$, \begin{CJK}{UTF8}{mj}对任意非零实数\end{CJK} $b$, \begin{CJK}{UTF8}{mj}计算行\end{CJK} \begin{CJK}{UTF8}{mj}列式\end{CJK}

\end{enumerate}
$$
D=\left|\begin{array}{cccc}
a_{11}+b & a_{12}+b & \cdots & a_{1 n}+b \\
a_{21}+b & a_{22}+b & \cdots & a_{2 n}+b \\
\vdots & \vdots & & \vdots \\
a_{n 1}+b & a_{n 2}+b & \cdots & a_{n n}+b
\end{array}\right|
$$

\begin{enumerate}
  \setcounter{enumi}{3}
  \item ( 25 \begin{CJK}{UTF8}{mj}分\end{CJK}) \begin{CJK}{UTF8}{mj}求矩阵\end{CJK} $A=\left(\begin{array}{ccc}-1 & -2 & 6 \\ -1 & 0 & 3 \\ -1 & -1 & 4\end{array}\right)$ \begin{CJK}{UTF8}{mj}的若尔当标准形\end{CJK} $J$, \begin{CJK}{UTF8}{mj}并求矩阵\end{CJK} $P$ \begin{CJK}{UTF8}{mj}使得\end{CJK} $P^{-1} A P=J$.

  \item (15 \begin{CJK}{UTF8}{mj}分\end{CJK}) \begin{CJK}{UTF8}{mj}设\end{CJK} $\lambda$ \begin{CJK}{UTF8}{mj}为\end{CJK} $n$ \begin{CJK}{UTF8}{mj}级实矩阵\end{CJK} $A=\left(a_{i j}\right)$ \begin{CJK}{UTF8}{mj}的一个实特征值\end{CJK}. \begin{CJK}{UTF8}{mj}证明\end{CJK}: \begin{CJK}{UTF8}{mj}存在正整数\end{CJK} $k(1 \leqslant k \leqslant n)$ \begin{CJK}{UTF8}{mj}使得\end{CJK}

\end{enumerate}
$$
\left|\lambda-a_{k k}\right| \leqslant \sum_{j \neq k}\left|a_{k j}\right| .
$$

\begin{enumerate}
  \setcounter{enumi}{5}
  \item ( 20 \begin{CJK}{UTF8}{mj}分\end{CJK}) \begin{CJK}{UTF8}{mj}设\end{CJK} $n$ \begin{CJK}{UTF8}{mj}级矩阵\end{CJK} $A$ \begin{CJK}{UTF8}{mj}和\end{CJK} $B$ \begin{CJK}{UTF8}{mj}可交换\end{CJK}. \begin{CJK}{UTF8}{mj}证明\end{CJK}: $r(A)+r(B) \geqslant r(A B)+r(A+B)$.

  \item ( 20 \begin{CJK}{UTF8}{mj}分\end{CJK}) \begin{CJK}{UTF8}{mj}证明\end{CJK}: $n$ \begin{CJK}{UTF8}{mj}维\end{CJK} $(n>2)$ \begin{CJK}{UTF8}{mj}实线性空间\end{CJK} $V$ \begin{CJK}{UTF8}{mj}的一个线性变换\end{CJK} $\sigma$ \begin{CJK}{UTF8}{mj}必有一维或二维不变子空间\end{CJK}.

  \item ( 20 \begin{CJK}{UTF8}{mj}分\end{CJK}) \begin{CJK}{UTF8}{mj}设\end{CJK} $V$ \begin{CJK}{UTF8}{mj}为有限维欧式空间\end{CJK}, $s$ \begin{CJK}{UTF8}{mj}个单位向量\end{CJK} $\alpha_{1}, \alpha_{2}, \cdots, \alpha_{s}$ \begin{CJK}{UTF8}{mj}组成\end{CJK} $V$ \begin{CJK}{UTF8}{mj}中的一个正交向量组\end{CJK}, \begin{CJK}{UTF8}{mj}使得对任意的\end{CJK} $\alpha \in V$, \begin{CJK}{UTF8}{mj}都有\end{CJK} $\sum_{i=1}^{s}\left(\alpha, \alpha_{i}\right)^{2}=|\alpha|^{2}$. \begin{CJK}{UTF8}{mj}证明\end{CJK}: $V=L\left(\alpha_{1}, \alpha_{2}, \cdots, \alpha_{s}\right)$.

  \item ( 20 \begin{CJK}{UTF8}{mj}分\end{CJK}) \begin{CJK}{UTF8}{mj}设\end{CJK} $A$ \begin{CJK}{UTF8}{mj}为\end{CJK} $n$ \begin{CJK}{UTF8}{mj}级可逆实矩阵\end{CJK}. \begin{CJK}{UTF8}{mj}证明\end{CJK}: \begin{CJK}{UTF8}{mj}存在\end{CJK} $n$ \begin{CJK}{UTF8}{mj}级正交矩阵\end{CJK} $P$ \begin{CJK}{UTF8}{mj}和\end{CJK} $Q$, \begin{CJK}{UTF8}{mj}使得\end{CJK} $P^{\prime} A Q=\left(\begin{array}{cc}\lambda_{1} & \lambda_{2} \\ \lambda_{n}\end{array}\right)$, \begin{CJK}{UTF8}{mj}其中\end{CJK} $\lambda_{i}>0$, \begin{CJK}{UTF8}{mj}且\end{CJK} $\lambda_{i}^{2}$ \begin{CJK}{UTF8}{mj}为\end{CJK} $A^{\prime} A$ \begin{CJK}{UTF8}{mj}的特征值\end{CJK} $(i=1,2, \cdots, n)$.

\end{enumerate}
\section{8. 南京师范大学 2015 年研究生入学考试试题高等代数 
 李扬 
 微信公众号: sxkyliyang}
\begin{enumerate}
  \item (10 \begin{CJK}{UTF8}{mj}分\end{CJK}) \begin{CJK}{UTF8}{mj}若方程组\end{CJK}
\end{enumerate}
$$
\left\{\begin{array}{l}
x_{1}+2 x_{2}+3 x_{3}=0 \\
2 x_{1}+3 x_{2}+5 x_{3}=0 \\
x_{1}+x_{2}+a x_{3}=0
\end{array}\right.
$$
\begin{CJK}{UTF8}{mj}与\end{CJK}
$$
\left\{\begin{array}{l}
x_{1}+b x_{2}+c x_{3}=0 \\
2 x_{1}+b^{2} x_{2}+(c+1) x_{3}=0
\end{array}\right.
$$
\begin{CJK}{UTF8}{mj}同解\end{CJK}, \begin{CJK}{UTF8}{mj}求\end{CJK} $a, b, c$ \begin{CJK}{UTF8}{mj}的值\end{CJK}.

\begin{enumerate}
  \setcounter{enumi}{2}
  \item ( 15 \begin{CJK}{UTF8}{mj}分\end{CJK}) \begin{CJK}{UTF8}{mj}设行列式\end{CJK} $D=\left|\begin{array}{cccc}a_{11} & a_{12} & \cdots & a_{1 n} \\ a_{21} & a_{22} & \cdots & a_{2 n} \\ \vdots & \vdots & & \vdots \\ a_{n 1} & a_{n 2} & \cdots & a_{n n}\end{array}\right|, n \geqslant 3$, \begin{CJK}{UTF8}{mj}令\end{CJK} $A_{i j}$ \begin{CJK}{UTF8}{mj}表示元素\end{CJK} $a_{i j}$ \begin{CJK}{UTF8}{mj}的代数余子式\end{CJK}, $1 \leqslant i, j \leqslant n$, \begin{CJK}{UTF8}{mj}证明\end{CJK}:
\end{enumerate}
$$
\begin{array}{cccc}
A_{11} & A_{12} & \cdots & A_{1, n-1} \\
A_{21} & A_{22} & \cdots & A_{2, n-1} \\
\vdots & \vdots & & \vdots \\
A_{n-1,1} & A_{n-1,2} & \cdots & A_{n-1, n-1}
\end{array} \mid=a_{n n} D^{n-2}
$$

\begin{enumerate}
  \setcounter{enumi}{3}
  \item (15 \begin{CJK}{UTF8}{mj}分\end{CJK}) \begin{CJK}{UTF8}{mj}已知多项式\end{CJK} $f(x)=x^{3}+2 x^{2}-2, g(x)=x^{2}+x-1, \alpha, \beta, \gamma$ \begin{CJK}{UTF8}{mj}为\end{CJK} $f(x)$ \begin{CJK}{UTF8}{mj}的根\end{CJK}, \begin{CJK}{UTF8}{mj}求一个整系数多项式\end{CJK} $h(x)$, \begin{CJK}{UTF8}{mj}使其以\end{CJK} $g(\alpha), g(\beta), g(\gamma)$ \begin{CJK}{UTF8}{mj}为根\end{CJK}.

  \item ( 20 \begin{CJK}{UTF8}{mj}分\end{CJK}) \begin{CJK}{UTF8}{mj}设\end{CJK} $A$ \begin{CJK}{UTF8}{mj}为反对称实矩阵\end{CJK}, $\lambda$ \begin{CJK}{UTF8}{mj}是\end{CJK} $A$ \begin{CJK}{UTF8}{mj}的一个非零特征值\end{CJK}, $\alpha+i \beta$ \begin{CJK}{UTF8}{mj}为\end{CJK} $A$ \begin{CJK}{UTF8}{mj}的属于\end{CJK} $\lambda$ \begin{CJK}{UTF8}{mj}的复特征向量\end{CJK},\begin{CJK}{UTF8}{mj}其中\end{CJK} $\alpha$ \begin{CJK}{UTF8}{mj}和\end{CJK} $\beta$ \begin{CJK}{UTF8}{mj}均为实向量\end{CJK}, \begin{CJK}{UTF8}{mj}证明\end{CJK}:

\end{enumerate}
(1) $\lambda$ \begin{CJK}{UTF8}{mj}为纯虚数\end{CJK};

(2) $\alpha$ \begin{CJK}{UTF8}{mj}和\end{CJK} $\beta$ \begin{CJK}{UTF8}{mj}的长度相等且互相正交\end{CJK}.

\begin{enumerate}
  \setcounter{enumi}{5}
  \item ( 20 \begin{CJK}{UTF8}{mj}分\end{CJK}) \begin{CJK}{UTF8}{mj}设\end{CJK} $\sigma$ \begin{CJK}{UTF8}{mj}和\end{CJK} $\tau$ \begin{CJK}{UTF8}{mj}是\end{CJK} $n$ \begin{CJK}{UTF8}{mj}维线性空间\end{CJK} $V$ \begin{CJK}{UTF8}{mj}的两个线性变换\end{CJK}, \begin{CJK}{UTF8}{mj}满足\end{CJK} $\sigma+\tau=\varepsilon$ (\begin{CJK}{UTF8}{mj}恒等变换\end{CJK}), \begin{CJK}{UTF8}{mj}且\end{CJK} $\sigma \tau=0$. \begin{CJK}{UTF8}{mj}证明\end{CJK}:
\end{enumerate}
$$
V=\sigma(V) \oplus \tau(V)
$$

\begin{enumerate}
  \setcounter{enumi}{6}
  \item ( 20 \begin{CJK}{UTF8}{mj}分\end{CJK}) \begin{CJK}{UTF8}{mj}设\end{CJK} $\sigma$ \begin{CJK}{UTF8}{mj}是线性空间\end{CJK} $V=P^{n \times n}$ \begin{CJK}{UTF8}{mj}的一个线性变换\end{CJK}, \begin{CJK}{UTF8}{mj}满足\end{CJK} $\sigma(A)=A^{\prime}$, \begin{CJK}{UTF8}{mj}其中\end{CJK} $A^{\prime}$ \begin{CJK}{UTF8}{mj}为\end{CJK} $A$ \begin{CJK}{UTF8}{mj}的转置矩阵\end{CJK}, \begin{CJK}{UTF8}{mj}求\end{CJK} $\sigma$ \begin{CJK}{UTF8}{mj}的全\end{CJK} \begin{CJK}{UTF8}{mj}部特征值及对应的特征向量\end{CJK}.

  \item ( 25 \begin{CJK}{UTF8}{mj}分\end{CJK}) \begin{CJK}{UTF8}{mj}设\end{CJK} $A$ \begin{CJK}{UTF8}{mj}是一个\end{CJK} $n$ \begin{CJK}{UTF8}{mj}阶方阵\end{CJK}, $\operatorname{tr}(A)=\sum_{i=1}^{n} a_{i i}$ \begin{CJK}{UTF8}{mj}称为矩阵\end{CJK} $A$ \begin{CJK}{UTF8}{mj}的迹\end{CJK}.

\end{enumerate}
(1) \begin{CJK}{UTF8}{mj}若\end{CJK} $f(x)=\left(x^{2}-2 x+2\right)^{2}(x-1)$ \begin{CJK}{UTF8}{mj}是\end{CJK} 6 \begin{CJK}{UTF8}{mj}阶方阵\end{CJK} $A$ \begin{CJK}{UTF8}{mj}的最小多项式\end{CJK}, \begin{CJK}{UTF8}{mj}且\end{CJK} $\operatorname{tr}(A)=6$, \begin{CJK}{UTF8}{mj}求\end{CJK} $A$ \begin{CJK}{UTF8}{mj}的若尔当标准形\end{CJK};

(2) \begin{CJK}{UTF8}{mj}若\end{CJK} $B, C$ \begin{CJK}{UTF8}{mj}均为对称半正定实矩阵\end{CJK}, \begin{CJK}{UTF8}{mj}并且\end{CJK} $\operatorname{tr}(B C)=0$, \begin{CJK}{UTF8}{mj}证明\end{CJK}: \begin{CJK}{UTF8}{mj}对任意的正整数\end{CJK} $m,(B+C)^{m}=B^{m}+C^{m}$.

\begin{enumerate}
  \setcounter{enumi}{8}
  \item ( 25 \begin{CJK}{UTF8}{mj}分\end{CJK}) \begin{CJK}{UTF8}{mj}设\end{CJK} $A$ \begin{CJK}{UTF8}{mj}是复数域上的\end{CJK} $n$ \begin{CJK}{UTF8}{mj}阶方阵\end{CJK}, $A^{n}=0$, \begin{CJK}{UTF8}{mj}且\end{CJK} $A^{n-1} \neq 0$,
\end{enumerate}
(1) \begin{CJK}{UTF8}{mj}若\end{CJK} $\lambda$ \begin{CJK}{UTF8}{mj}是\end{CJK} $A$ \begin{CJK}{UTF8}{mj}的一个特征值\end{CJK}, \begin{CJK}{UTF8}{mj}其对应的特征子空间\end{CJK} $V_{\lambda}=\{\alpha \mid A \alpha=\lambda \alpha, \alpha$ \begin{CJK}{UTF8}{mj}是复向量\end{CJK} $\}$, \begin{CJK}{UTF8}{mj}证明\end{CJK}: $V_{\lambda}$ \begin{CJK}{UTF8}{mj}的维数是\end{CJK} 1 ;

(2) \begin{CJK}{UTF8}{mj}是否存在一个复矩阵\end{CJK} $B$, \begin{CJK}{UTF8}{mj}使得\end{CJK} $B^{2}=A$ ? \begin{CJK}{UTF8}{mj}请说明理由\end{CJK}.

\section{9. 南京师范大学 2016 年研究生入学考试试题高等代数 
 李扬 
 微信公众号: sxkyliyang}
\begin{enumerate}
  \item (15 \begin{CJK}{UTF8}{mj}分\end{CJK}) \begin{CJK}{UTF8}{mj}证明高斯\end{CJK} Gauss \begin{CJK}{UTF8}{mj}引理\end{CJK}: \begin{CJK}{UTF8}{mj}两个本原多项式的乘积还是本原多项式\end{CJK}.

  \item (15 \begin{CJK}{UTF8}{mj}分\end{CJK}) \begin{CJK}{UTF8}{mj}证明数域\end{CJK} $P$ \begin{CJK}{UTF8}{mj}上的线性方程组\end{CJK} $A x=b$ \begin{CJK}{UTF8}{mj}有解的充要条件是\end{CJK}

\end{enumerate}
$$
\left\{\begin{array}{l}
A^{\prime} y=0 \\
b^{\prime} y=1
\end{array}\right.
$$
\begin{CJK}{UTF8}{mj}无解\end{CJK}, \begin{CJK}{UTF8}{mj}其中\end{CJK} $A \in P^{m \times n}, b \in P^{m}, A^{\prime}$ \begin{CJK}{UTF8}{mj}和\end{CJK} $b^{\prime}$ \begin{CJK}{UTF8}{mj}分别表示\end{CJK} $A$ \begin{CJK}{UTF8}{mj}和\end{CJK} $b$ \begin{CJK}{UTF8}{mj}的转置\end{CJK}, $x \in P^{n}$ \begin{CJK}{UTF8}{mj}和\end{CJK} $y \in P^{m}$ \begin{CJK}{UTF8}{mj}是末知量\end{CJK}.

\begin{enumerate}
  \setcounter{enumi}{3}
  \item (15 \begin{CJK}{UTF8}{mj}分\end{CJK}) \begin{CJK}{UTF8}{mj}设矩阵\end{CJK} $A, C$ \begin{CJK}{UTF8}{mj}分别为\end{CJK} $n$ \begin{CJK}{UTF8}{mj}级和\end{CJK} $m$ \begin{CJK}{UTF8}{mj}级可逆矩阵\end{CJK}, $B, D$ \begin{CJK}{UTF8}{mj}分别为\end{CJK} $n \times m$ \begin{CJK}{UTF8}{mj}和\end{CJK} $m \times n$ \begin{CJK}{UTF8}{mj}矩阵\end{CJK}, \begin{CJK}{UTF8}{mj}证明\end{CJK}:
\end{enumerate}
$$
|C| \cdot\left|A-B C^{-1} D\right|=|A| \cdot\left|C-D A^{-1} B\right| .
$$

\begin{enumerate}
  \setcounter{enumi}{4}
  \item ( 15 \begin{CJK}{UTF8}{mj}分\end{CJK}) \begin{CJK}{UTF8}{mj}设数域\end{CJK} $P$ \begin{CJK}{UTF8}{mj}上的\end{CJK} $n(n \geqslant 2)$ \begin{CJK}{UTF8}{mj}次多项式\end{CJK} $f(x)$ \begin{CJK}{UTF8}{mj}没有单因式\end{CJK}, \begin{CJK}{UTF8}{mj}证明\end{CJK}:
\end{enumerate}
$$
f^{\prime \prime}(x) \mid f(x) \text { 当且仅当 } f(x)=c(x-a)^{n} \text {, }
$$
\begin{CJK}{UTF8}{mj}其中\end{CJK} $f^{\prime \prime}(x)$ \begin{CJK}{UTF8}{mj}表二示二阶导数\end{CJK}, $a, c$ \begin{CJK}{UTF8}{mj}是数域\end{CJK} $P$ \begin{CJK}{UTF8}{mj}中的常数\end{CJK}.

\begin{enumerate}
  \setcounter{enumi}{5}
  \item ( 20 \begin{CJK}{UTF8}{mj}分\end{CJK}) \begin{CJK}{UTF8}{mj}已知\end{CJK} $s \times n$ \begin{CJK}{UTF8}{mj}实矩阵\end{CJK} $A=\left(a_{i j}\right)$ \begin{CJK}{UTF8}{mj}的秩为\end{CJK} $r$, \begin{CJK}{UTF8}{mj}求如下二次型的正惯性指数\end{CJK}.
\end{enumerate}
$$
f\left(x_{1}, x_{2}, \cdots, x_{n}\right)=\sum_{i=1}^{s}\left(a_{i 1} x_{1}+a_{i 2} x_{2}+\cdots+a_{i n 1} x_{n}\right)^{2}
$$

\begin{enumerate}
  \setcounter{enumi}{6}
  \item (20 \begin{CJK}{UTF8}{mj}分\end{CJK}) \begin{CJK}{UTF8}{mj}设\end{CJK} $V$ \begin{CJK}{UTF8}{mj}为一个有限维线性空间\end{CJK}, $\mathscr{A}$ \begin{CJK}{UTF8}{mj}是\end{CJK} $V$ \begin{CJK}{UTF8}{mj}上的线性变换\end{CJK}, \begin{CJK}{UTF8}{mj}证明\end{CJK}: $V=\mathscr{A} V \oplus \mathscr{A}^{-1}(0)$ \begin{CJK}{UTF8}{mj}当且仅当\end{CJK} $\mathscr{A}^{2} V=\mathscr{A} V$.

  \item ( 25 \begin{CJK}{UTF8}{mj}分\end{CJK}) \begin{CJK}{UTF8}{mj}设\end{CJK} $A$ \begin{CJK}{UTF8}{mj}为正定矩阵\end{CJK},

\end{enumerate}
(1) \begin{CJK}{UTF8}{mj}证明\end{CJK}: \begin{CJK}{UTF8}{mj}对任意的正整数\end{CJK} $m$, \begin{CJK}{UTF8}{mj}存在正定矩阵\end{CJK} $B$ \begin{CJK}{UTF8}{mj}使得\end{CJK} $A=B^{m}$;

(2) \begin{CJK}{UTF8}{mj}在\end{CJK} $A$ \begin{CJK}{UTF8}{mj}的特征值两两不同的情形下证明\end{CJK}: \begin{CJK}{UTF8}{mj}满足\end{CJK} $A=B^{m}$ \begin{CJK}{UTF8}{mj}的正定矩阵\end{CJK} $B$ \begin{CJK}{UTF8}{mj}是唯一确定的\end{CJK}.

\begin{enumerate}
  \setcounter{enumi}{8}
  \item ( 25 \begin{CJK}{UTF8}{mj}分\end{CJK}) \begin{CJK}{UTF8}{mj}设\end{CJK} $A$ \begin{CJK}{UTF8}{mj}为\end{CJK} $n$ \begin{CJK}{UTF8}{mj}级实对称矩阵\end{CJK}, \begin{CJK}{UTF8}{mj}记它的特征值为\end{CJK} $\lambda_{1} \leqslant \lambda_{2} \leqslant \cdots \leqslant \lambda_{n}$. \begin{CJK}{UTF8}{mj}设\end{CJK} $A$ \begin{CJK}{UTF8}{mj}的属于\end{CJK} $\lambda_{1}$ \begin{CJK}{UTF8}{mj}的一个特征向量为\end{CJK} $\mu_{1}$. \begin{CJK}{UTF8}{mj}证明\end{CJK}:
\end{enumerate}
$$
\min _{\substack{x \neq 0 \\ x \perp \mu_{1}}} \frac{x^{\prime} A x}{x^{\prime} x}=\lambda_{2}
$$

\section{0. 南京师范大学 2017 年研究生入学考试试题高等代数 
 李扬 
 微信公众号: sxkyliyang}
\begin{enumerate}
  \item (15 \begin{CJK}{UTF8}{mj}分\end{CJK}) \begin{CJK}{UTF8}{mj}设\end{CJK} $A$ \begin{CJK}{UTF8}{mj}是\end{CJK} 3 \begin{CJK}{UTF8}{mj}阶方阵\end{CJK}, \begin{CJK}{UTF8}{mj}且\end{CJK}
\end{enumerate}
$$
|A-E|=|A-2 E|=|A+E|=\lambda
$$
\begin{CJK}{UTF8}{mj}若\end{CJK} $\lambda=2$, \begin{CJK}{UTF8}{mj}求\end{CJK} $|A+3 E|$.

\begin{enumerate}
  \setcounter{enumi}{2}
  \item ( 15 \begin{CJK}{UTF8}{mj}分\end{CJK}) \begin{CJK}{UTF8}{mj}设\end{CJK} $A$ \begin{CJK}{UTF8}{mj}是\end{CJK} $n \times m$ \begin{CJK}{UTF8}{mj}阶方阵\end{CJK}, \begin{CJK}{UTF8}{mj}且\end{CJK} $\operatorname{rank}(A)=r$. \begin{CJK}{UTF8}{mj}从\end{CJK} $A$ \begin{CJK}{UTF8}{mj}中取\end{CJK} $s$ \begin{CJK}{UTF8}{mj}个列向量组成矩阵\end{CJK} $B$, \begin{CJK}{UTF8}{mj}证明\end{CJK}:
\end{enumerate}
$$
\operatorname{rank}(B) \geqslant r+s-m .
$$

\begin{enumerate}
  \setcounter{enumi}{3}
  \item (20 \begin{CJK}{UTF8}{mj}分\end{CJK}) \begin{CJK}{UTF8}{mj}讨论下面齐次线性方程组解的情况\end{CJK}
\end{enumerate}
$$
\left\{\begin{array}{l}
a x_{1}+b x_{2}+\cdots+b x_{n}=0 \\
b x_{1}+a x_{2}+\cdots+b x_{n}=0 \\
\cdots \cdots \cdots \\
b x_{1}+b x_{2}+\cdots+a x_{n}=0
\end{array}\right.
$$

\begin{enumerate}
  \setcounter{enumi}{4}
  \item ( 20 \begin{CJK}{UTF8}{mj}分\end{CJK}) \begin{CJK}{UTF8}{mj}已知矩阵\end{CJK} $A$ \begin{CJK}{UTF8}{mj}的伴随矩阵为\end{CJK} $A^{*}=\left(\begin{array}{cccc}1 & 0 & 0 & 0 \\ 0 & 1 & 0 & 0 \\ 1 & 0 & 1 & 0 \\ 0 & -3 & 0 & 8\end{array}\right)$. \begin{CJK}{UTF8}{mj}矩阵\end{CJK} $B$ \begin{CJK}{UTF8}{mj}满足\end{CJK} $A B A^{-1}=B A^{-1}+3 E$. \begin{CJK}{UTF8}{mj}求\end{CJK} $B$.

  \item ( 20 \begin{CJK}{UTF8}{mj}分\end{CJK}) \begin{CJK}{UTF8}{mj}设多项式\end{CJK}

\end{enumerate}
$$
f(x)=x^{3 m}-x^{3 n+1}+x^{3 p+2}, g(x)=x^{2}-x+1
$$
\begin{CJK}{UTF8}{mj}其中\end{CJK} $m, n, p$ \begin{CJK}{UTF8}{mj}为非负整数\end{CJK}, \begin{CJK}{UTF8}{mj}证明\end{CJK} $g(x) \mid f(x) \Leftrightarrow m, n, p$ \begin{CJK}{UTF8}{mj}有相同的奇偶性\end{CJK}.

\begin{enumerate}
  \setcounter{enumi}{6}
  \item ( 20 \begin{CJK}{UTF8}{mj}分\end{CJK}) \begin{CJK}{UTF8}{mj}证明\end{CJK}: \begin{CJK}{UTF8}{mj}若线性变换\end{CJK} $\mathscr{A}, \mathscr{B}$ \begin{CJK}{UTF8}{mj}满足\end{CJK}
\end{enumerate}
$$
\mathscr{A}^{2}=\mathscr{A}, \mathscr{B}^{2}=\mathscr{B}, \mathscr{A} \mathscr{B}=\mathscr{B} \mathscr{A}=0
$$
\begin{CJK}{UTF8}{mj}则\end{CJK} $(\mathscr{A}+\mathscr{B}) V=\mathscr{A} V \oplus \mathscr{B} V$.

\begin{enumerate}
  \setcounter{enumi}{7}
  \item ( 20 \begin{CJK}{UTF8}{mj}分\end{CJK}) \begin{CJK}{UTF8}{mj}证明\end{CJK}: $n$ \begin{CJK}{UTF8}{mj}维线性空间\end{CJK} $V$ \begin{CJK}{UTF8}{mj}的任意真子空间可表示为若干个\end{CJK} $n-1$ \begin{CJK}{UTF8}{mj}维子空间的交\end{CJK}.

  \item ( 20 \begin{CJK}{UTF8}{mj}分\end{CJK}) \begin{CJK}{UTF8}{mj}设\end{CJK} $A$ \begin{CJK}{UTF8}{mj}是\end{CJK} $n$ \begin{CJK}{UTF8}{mj}阶实对称矩阵\end{CJK}, \begin{CJK}{UTF8}{mj}二次型\end{CJK} $f(X)=X^{\prime} A X$ \begin{CJK}{UTF8}{mj}的正负惯性指数分别为\end{CJK} $p, q$, \begin{CJK}{UTF8}{mj}且\end{CJK} $p \geqslant q>0$. \begin{CJK}{UTF8}{mj}证明\end{CJK}: \begin{CJK}{UTF8}{mj}存\end{CJK} \begin{CJK}{UTF8}{mj}在\end{CJK} $q$ \begin{CJK}{UTF8}{mj}维子空间\end{CJK} $W$, \begin{CJK}{UTF8}{mj}满足\end{CJK} $\forall X_{0} \in W$, \begin{CJK}{UTF8}{mj}均有\end{CJK} $f\left(X_{0}\right)=0$.

\end{enumerate}
\section{1. 南京师范大学 2018 年研究生入学考试试题高等代数 
 李扬 
 微信公众号: sxkyliyang}
\begin{enumerate}
  \item \begin{CJK}{UTF8}{mj}叙述克拉默法则并证明\end{CJK}.

  \item \begin{CJK}{UTF8}{mj}求行列式\end{CJK}

\end{enumerate}
(1)
$$
\left|\begin{array}{cccc}
a & b & c & d \\
-b & a & -d & c \\
-c & d & a & b \\
-d & -c & -b & a
\end{array}\right| .
$$
(2) \begin{CJK}{UTF8}{mj}类似递推关系的\end{CJK}, \begin{CJK}{UTF8}{mj}如\end{CJK} $D_{n}^{2}-7 D_{n-1}+12=0$, \begin{CJK}{UTF8}{mj}具体记不清了\end{CJK}.

\begin{enumerate}
  \setcounter{enumi}{3}
  \item \begin{CJK}{UTF8}{mj}解线性方程组\end{CJK}, \begin{CJK}{UTF8}{mj}具体记不清了\end{CJK}, \begin{CJK}{UTF8}{mj}很常规的一道题\end{CJK}.

  \item \begin{CJK}{UTF8}{mj}设\end{CJK} $\eta$ \begin{CJK}{UTF8}{mj}是\end{CJK} $n$ \begin{CJK}{UTF8}{mj}维欧式空间\end{CJK} $V$ \begin{CJK}{UTF8}{mj}的一个单位向量\end{CJK}, \begin{CJK}{UTF8}{mj}定义变换\end{CJK} $\mathscr{A}$ \begin{CJK}{UTF8}{mj}为\end{CJK} $\mathscr{A} \alpha=\alpha-2(\alpha, \eta) \eta$. \begin{CJK}{UTF8}{mj}证明\end{CJK}:

\end{enumerate}
(1) $\mathscr{A}$ \begin{CJK}{UTF8}{mj}是对称变换\end{CJK};

(2) $\mathscr{A}$ \begin{CJK}{UTF8}{mj}是正交变换\end{CJK};

(3)\begin{CJK}{UTF8}{mj}忘了\end{CJK}. \begin{CJK}{UTF8}{mj}因为考试的时候没写\end{CJK}, \begin{CJK}{UTF8}{mj}跳过了\end{CJK}. \begin{CJK}{UTF8}{mj}欧氏空间没复习\end{CJK}.

\begin{enumerate}
  \setcounter{enumi}{5}
  \item \begin{CJK}{UTF8}{mj}设\end{CJK} $n$ \begin{CJK}{UTF8}{mj}阶矩阵\end{CJK} $A=\left(a_{i j}\right)_{n \times n}, a_{i i}>0, a_{i j}<0, i \neq j, \sum_{i=1}^{n} a_{i k}=0, k=1,2, \cdots, n$. \begin{CJK}{UTF8}{mj}求\end{CJK} $A$ \begin{CJK}{UTF8}{mj}的秩\end{CJK} $r(A)$.

  \item \begin{CJK}{UTF8}{mj}已知\end{CJK} $p$ \begin{CJK}{UTF8}{mj}为奇素数\end{CJK}, \begin{CJK}{UTF8}{mj}求证\end{CJK}: \begin{CJK}{UTF8}{mj}多项式\end{CJK} $f(x)=(p-1) x^{p-2}+(p-2) x^{p-3}+\cdots+2 x+1$ \begin{CJK}{UTF8}{mj}在有理数域不可约\end{CJK}.

  \item \begin{CJK}{UTF8}{mj}已知\end{CJK} $A, B$ \begin{CJK}{UTF8}{mj}为\end{CJK} $n$ \begin{CJK}{UTF8}{mj}阶矩阵\end{CJK}, $A B-B A$ \begin{CJK}{UTF8}{mj}的秩为\end{CJK} 1 , \begin{CJK}{UTF8}{mj}求证\end{CJK}: $(A B-B A)^{2}=0$.

  \item \begin{CJK}{UTF8}{mj}已知\end{CJK} $a_{1}, a_{2}, \cdots, a_{n}$ \begin{CJK}{UTF8}{mj}是两两互异的正数\end{CJK}. \begin{CJK}{UTF8}{mj}求证\end{CJK}:

\end{enumerate}
$$
A=\left(\begin{array}{cccc}
\frac{1}{a_{1}+a_{1}} & \frac{1}{a_{1}+a_{2}} & \cdots & \frac{1}{a_{1}+a_{n}} \\
\frac{1}{a_{2}+a_{1}} & \frac{1}{a_{2}+a_{2}} & \cdots & \frac{1}{a_{2}+a_{n}} \\
\vdots & \vdots & & \vdots \\
\frac{1}{a_{n}+a_{1}} & \frac{1}{a_{n}+a_{2}} & \cdots & \frac{1}{a_{n}+a_{n}}
\end{array}\right)
$$
\begin{CJK}{UTF8}{mj}为正定矩阵\end{CJK}.

\begin{enumerate}
  \setcounter{enumi}{9}
  \item \begin{CJK}{UTF8}{mj}已知\end{CJK} $A, B$ \begin{CJK}{UTF8}{mj}为\end{CJK} $n$ \begin{CJK}{UTF8}{mj}阶矩阵\end{CJK}, \begin{CJK}{UTF8}{mj}证明\end{CJK}: $(A B)^{*}=B^{*} A^{*}, A^{*}$ \begin{CJK}{UTF8}{mj}表示\end{CJK} $A$ \begin{CJK}{UTF8}{mj}的伴随矩阵\end{CJK}.
\end{enumerate}
\section{2. 南京师范大学 2008 年研究生入学考试试题数学分析 
 李扬 
 微信公众号: sxkyliyang}
\begin{enumerate}
  \item (\begin{CJK}{UTF8}{mj}每小题\end{CJK} 5 \begin{CJK}{UTF8}{mj}分\end{CJK}, \begin{CJK}{UTF8}{mj}共\end{CJK} 10 \begin{CJK}{UTF8}{mj}分\end{CJK}) \begin{CJK}{UTF8}{mj}简答题\end{CJK}: \begin{CJK}{UTF8}{mj}判断下列命题是否正确\end{CJK}, \begin{CJK}{UTF8}{mj}并简要说明理由\end{CJK}.
\end{enumerate}
(1) \begin{CJK}{UTF8}{mj}若级数\end{CJK} $\sum_{n=1}^{\infty} a_{n}$ \begin{CJK}{UTF8}{mj}收敛\end{CJK}, $b_{n} \rightarrow 1(n \rightarrow \infty)$, \begin{CJK}{UTF8}{mj}则\end{CJK} $\sum_{n=1}^{\infty} a_{n} b_{n}$ \begin{CJK}{UTF8}{mj}也收敛\end{CJK}.

(2) \begin{CJK}{UTF8}{mj}若非正常积分\end{CJK} $\int_{0}^{+\infty} f(x) \mathrm{d} x$ \begin{CJK}{UTF8}{mj}收敛\end{CJK}, \begin{CJK}{UTF8}{mj}且\end{CJK} $\int_{0}^{+\infty} f^{\prime}(x) \mathrm{d} x$ \begin{CJK}{UTF8}{mj}也收玫\end{CJK}, \begin{CJK}{UTF8}{mj}则\end{CJK} $\lim _{x \rightarrow+\infty} f(x)=0$.

\begin{enumerate}
  \setcounter{enumi}{2}
  \item (\begin{CJK}{UTF8}{mj}每小题\end{CJK} 10 \begin{CJK}{UTF8}{mj}分\end{CJK}, \begin{CJK}{UTF8}{mj}共\end{CJK} 30 \begin{CJK}{UTF8}{mj}分\end{CJK}) \begin{CJK}{UTF8}{mj}计算下列极限\end{CJK}:
\end{enumerate}
(1) $\lim _{x \rightarrow 0}\left(\frac{\cos x}{\cos 2 x}\right)^{\frac{2}{x^{2}}}$;

(2) $\lim _{n \rightarrow \infty} \sum_{k=1}^{n}\left(n^{k}-1\right)^{-\frac{1}{k}}$;

(3) $\lim _{n \rightarrow \infty} \int_{0}^{+\infty} e^{-x^{2}} \sin ^{2} n x \mathrm{~d} x$.

\begin{enumerate}
  \setcounter{enumi}{3}
  \item ( 15 \begin{CJK}{UTF8}{mj}分\end{CJK}) \begin{CJK}{UTF8}{mj}设\end{CJK} $x>0$, \begin{CJK}{UTF8}{mj}证明存在\end{CJK} $\vartheta \in(0,1)$, \begin{CJK}{UTF8}{mj}使得\end{CJK}
\end{enumerate}
$$
\int_{0}^{x} e^{t} \mathrm{~d} t=x e^{\vartheta x} \text {, 且 } \lim _{x \rightarrow+\infty} \vartheta=1 .
$$

\begin{enumerate}
  \setcounter{enumi}{4}
  \item ( 15 \begin{CJK}{UTF8}{mj}分\end{CJK}) \begin{CJK}{UTF8}{mj}计算第二型曲线积分\end{CJK}:
\end{enumerate}
$$
I=\int_{L^{+}}\left[e^{x} \sin y-b(x+y)\right] \mathrm{d} x+\left(e^{x} \cos y-a x\right) \mathrm{d} y
$$
\begin{CJK}{UTF8}{mj}其中\end{CJK} $a, b$ \begin{CJK}{UTF8}{mj}为正的常数\end{CJK}, $L^{+}$\begin{CJK}{UTF8}{mj}是从点\end{CJK} $A(2 a, 0)$ \begin{CJK}{UTF8}{mj}沿曲线\end{CJK} $y=\sqrt{2 a x-x^{2}}$ \begin{CJK}{UTF8}{mj}到点\end{CJK} $O(0,0)$ \begin{CJK}{UTF8}{mj}的弧\end{CJK}.

\begin{enumerate}
  \setcounter{enumi}{5}
  \item ( 15 \begin{CJK}{UTF8}{mj}分\end{CJK}) \begin{CJK}{UTF8}{mj}设函数\end{CJK} $f(x)$ \begin{CJK}{UTF8}{mj}在\end{CJK} $[a, b]$ \begin{CJK}{UTF8}{mj}上连续\end{CJK}, \begin{CJK}{UTF8}{mj}且\end{CJK} $f(a)=f(b)=0$, \begin{CJK}{UTF8}{mj}又设\end{CJK} $f(x)$ \begin{CJK}{UTF8}{mj}在\end{CJK} $(a, b)$ \begin{CJK}{UTF8}{mj}内存在二阶导数\end{CJK}, \begin{CJK}{UTF8}{mj}且\end{CJK} $f^{\prime \prime}(x) \leqslant 0$, \begin{CJK}{UTF8}{mj}求证在\end{CJK} $[a, b]$ \begin{CJK}{UTF8}{mj}上\end{CJK} $f(x) \geqslant 0$.

  \item (20 \begin{CJK}{UTF8}{mj}分\end{CJK}) (1) \begin{CJK}{UTF8}{mj}试列举出证明函数\end{CJK} $f(x)$ \begin{CJK}{UTF8}{mj}在区间\end{CJK} $[a, b]$ \begin{CJK}{UTF8}{mj}上为凸函数的方法\end{CJK}(\begin{CJK}{UTF8}{mj}至少两种\end{CJK});

\end{enumerate}
(2) \begin{CJK}{UTF8}{mj}设函数\end{CJK} $f(x)$ \begin{CJK}{UTF8}{mj}在区间\end{CJK} $[a, b]$ \begin{CJK}{UTF8}{mj}为递增\end{CJK}, \begin{CJK}{UTF8}{mj}证明\end{CJK}: \begin{CJK}{UTF8}{mj}对任给\end{CJK} $c \in(a, b)$, \begin{CJK}{UTF8}{mj}函数\end{CJK} $F(x)=\int_{c}^{x} f(t) \mathrm{d} t$ \begin{CJK}{UTF8}{mj}为\end{CJK} $[a, b]$ \begin{CJK}{UTF8}{mj}上的凸函数\end{CJK}.

\begin{enumerate}
  \setcounter{enumi}{7}
  \item (15 \begin{CJK}{UTF8}{mj}分\end{CJK}) \begin{CJK}{UTF8}{mj}任给\end{CJK} $x \in[0,1]$, \begin{CJK}{UTF8}{mj}试证数列\end{CJK}
\end{enumerate}
$$
a_{n}=\ln \left(1+\frac{x}{2 \ln ^{2} 2}\right)+\ln \left(1+\frac{x}{3 \ln ^{2} 3}\right)+\cdots+\ln \left(1+\frac{x}{n \ln ^{2} n}\right)
$$
\begin{CJK}{UTF8}{mj}收敛\end{CJK}.

\begin{enumerate}
  \setcounter{enumi}{8}
  \item ( 15 \begin{CJK}{UTF8}{mj}分\end{CJK}) \begin{CJK}{UTF8}{mj}将周期为\end{CJK} $2 \pi$ \begin{CJK}{UTF8}{mj}的函数\end{CJK} $f(x)=\frac{1}{4} x(2 \pi-x), x \in[0,2 \pi]$
\end{enumerate}
(1) \begin{CJK}{UTF8}{mj}展开成\end{CJK} Fourier \begin{CJK}{UTF8}{mj}级数\end{CJK};

(2) \begin{CJK}{UTF8}{mj}通过\end{CJK} Fourier \begin{CJK}{UTF8}{mj}级数的逐项积分求\end{CJK} $\sum_{n=1}^{\infty} \frac{1}{n^{4}}$ \begin{CJK}{UTF8}{mj}的值\end{CJK}.

\begin{enumerate}
  \setcounter{enumi}{9}
  \item (15 \begin{CJK}{UTF8}{mj}分\end{CJK}) \begin{CJK}{UTF8}{mj}设\end{CJK} $f(x, y)$ \begin{CJK}{UTF8}{mj}为\end{CJK} $[a, b] \times[c,+\infty)$ \begin{CJK}{UTF8}{mj}上的连续非负函数\end{CJK},
\end{enumerate}
$$
I(x)=\int_{c}^{+\infty} f(x, y) \mathrm{d} y
$$
\begin{CJK}{UTF8}{mj}在\end{CJK} $[a, b]$ \begin{CJK}{UTF8}{mj}上连续\end{CJK}, \begin{CJK}{UTF8}{mj}证明\end{CJK}: $\int_{c}^{+\infty} f(x, y) \mathrm{d} y$ \begin{CJK}{UTF8}{mj}在\end{CJK} $[a, b]$ \begin{CJK}{UTF8}{mj}上一致收敛\end{CJK}.

\section{3. 南京师范大学 2009 年研究生入学考试试题数学分析 
 李扬 
 微信公众号: sxkyliyang}
\begin{enumerate}
  \item (\begin{CJK}{UTF8}{mj}每小题\end{CJK} 8 \begin{CJK}{UTF8}{mj}分\end{CJK}, \begin{CJK}{UTF8}{mj}共\end{CJK} 16 \begin{CJK}{UTF8}{mj}分\end{CJK}) \begin{CJK}{UTF8}{mj}判断下列命题是否正确\end{CJK}? \begin{CJK}{UTF8}{mj}并简要说明理由\end{CJK}.
\end{enumerate}
(1) \begin{CJK}{UTF8}{mj}牛顿\end{CJK}-\begin{CJK}{UTF8}{mj}莱布尼茨公式可叙述为\end{CJK}: \begin{CJK}{UTF8}{mj}若\end{CJK} $f(x)$ \begin{CJK}{UTF8}{mj}在区间\end{CJK} $[a, b]$ \begin{CJK}{UTF8}{mj}上连续\end{CJK}, \begin{CJK}{UTF8}{mj}且存在原函数\end{CJK} $F(x)$, \begin{CJK}{UTF8}{mj}则\end{CJK} $f(x)$ \begin{CJK}{UTF8}{mj}在区间\end{CJK} $[a, b]$ \begin{CJK}{UTF8}{mj}上可\end{CJK} \begin{CJK}{UTF8}{mj}积\end{CJK}, \begin{CJK}{UTF8}{mj}且\end{CJK} $\int_{a}^{b} f(x) \mathrm{d} x=F(b)-F(a)$. \begin{CJK}{UTF8}{mj}现将条件减弱为\end{CJK}: $f(x)$ \begin{CJK}{UTF8}{mj}在区间\end{CJK} $[a, b]$ \begin{CJK}{UTF8}{mj}上连续\end{CJK}, \begin{CJK}{UTF8}{mj}但\end{CJK} $F(x)$ \begin{CJK}{UTF8}{mj}仅在\end{CJK} $(a, b)$ \begin{CJK}{UTF8}{mj}内为\end{CJK} $f(x)$ \begin{CJK}{UTF8}{mj}的原函数\end{CJK}, \begin{CJK}{UTF8}{mj}则原结论仍然成立\end{CJK}.

(2) \begin{CJK}{UTF8}{mj}设\end{CJK} $\int_{a}^{+\infty} f(x) \mathrm{d} x$ \begin{CJK}{UTF8}{mj}收敛\end{CJK}, \begin{CJK}{UTF8}{mj}且\end{CJK} $f(x)$ \begin{CJK}{UTF8}{mj}在\end{CJK} $[a,+\infty)$ \begin{CJK}{UTF8}{mj}连续恒正\end{CJK}, \begin{CJK}{UTF8}{mj}则\end{CJK} $\lim _{x \rightarrow+\infty} f(x)=0$.

\begin{enumerate}
  \setcounter{enumi}{2}
  \item (\begin{CJK}{UTF8}{mj}每小题\end{CJK} 8 \begin{CJK}{UTF8}{mj}分\end{CJK}, \begin{CJK}{UTF8}{mj}共\end{CJK} 24 \begin{CJK}{UTF8}{mj}分\end{CJK}) \begin{CJK}{UTF8}{mj}计算下列各题\end{CJK}:
\end{enumerate}
(1) \begin{CJK}{UTF8}{mj}讨论极限\end{CJK} $\lim _{a \rightarrow 0} \int_{-a}^{a} \frac{|x|}{a^{2}} \cos (b-x) \mathrm{d} x$ \begin{CJK}{UTF8}{mj}的存在性\end{CJK};

(2) $\lim _{n \rightarrow \infty} \frac{e^{\frac{1}{n}}-e^{\tan \frac{1}{n}}}{\frac{1}{n}-\tan \frac{1}{n}}$

(3) \begin{CJK}{UTF8}{mj}设\end{CJK} $F(x)=\int_{a}^{b} f(y)|x-y| \mathrm{d} y$, \begin{CJK}{UTF8}{mj}其中\end{CJK} $a<b, f(x)$ \begin{CJK}{UTF8}{mj}连续\end{CJK}, \begin{CJK}{UTF8}{mj}求\end{CJK} $F^{\prime \prime}(x)$.

\begin{enumerate}
  \setcounter{enumi}{3}
  \item ( 15 \begin{CJK}{UTF8}{mj}分\end{CJK}) \begin{CJK}{UTF8}{mj}设\end{CJK} $a>0$. \begin{CJK}{UTF8}{mj}我们知道\end{CJK}, \begin{CJK}{UTF8}{mj}当\end{CJK} $\alpha, \beta$ \begin{CJK}{UTF8}{mj}为有理数时\end{CJK}, \begin{CJK}{UTF8}{mj}有\end{CJK} $a^{\alpha} \cdot a^{\beta}=a^{\alpha+\beta}$. \begin{CJK}{UTF8}{mj}试证明当\end{CJK} $\alpha, \beta$ \begin{CJK}{UTF8}{mj}为无理数时\end{CJK}, \begin{CJK}{UTF8}{mj}上述等式也\end{CJK} \begin{CJK}{UTF8}{mj}成立\end{CJK}.

  \item (15 \begin{CJK}{UTF8}{mj}分\end{CJK}) \begin{CJK}{UTF8}{mj}已知\end{CJK} $b>a>0$. \begin{CJK}{UTF8}{mj}证明\end{CJK}: \begin{CJK}{UTF8}{mj}存在\end{CJK} $\xi \in(-1,1)$, \begin{CJK}{UTF8}{mj}使得\end{CJK}

\end{enumerate}
$$
\int_{a}^{b} \frac{e^{-2 x} \cos 2 x}{x} \mathrm{~d} x=\frac{\xi}{a}
$$

\begin{enumerate}
  \setcounter{enumi}{5}
  \item ( 15 \begin{CJK}{UTF8}{mj}分\end{CJK}) \begin{CJK}{UTF8}{mj}设\end{CJK} $\left\{f_{n}(x)\right\}$ \begin{CJK}{UTF8}{mj}是定义在实数集\end{CJK} $E$ \begin{CJK}{UTF8}{mj}上的函数列\end{CJK}, $f(x)$ \begin{CJK}{UTF8}{mj}在\end{CJK} $E$ \begin{CJK}{UTF8}{mj}上有定义\end{CJK}.
\end{enumerate}
(1) \begin{CJK}{UTF8}{mj}请写出\end{CJK} $\left\{f_{n}(x)\right\}$ \begin{CJK}{UTF8}{mj}在\end{CJK} $E$ \begin{CJK}{UTF8}{mj}上不一致收敛于\end{CJK} $f(x)$ \begin{CJK}{UTF8}{mj}的正面定义\end{CJK};

(2) \begin{CJK}{UTF8}{mj}若\end{CJK} $\left\{f_{n}(x)\right\}$ \begin{CJK}{UTF8}{mj}在\end{CJK} $E$ \begin{CJK}{UTF8}{mj}上一致收敛于\end{CJK} $f(x), x_{0}$ \begin{CJK}{UTF8}{mj}是\end{CJK} $E$ \begin{CJK}{UTF8}{mj}的聚点\end{CJK}, \begin{CJK}{UTF8}{mj}且\end{CJK}
$$
\lim _{x \rightarrow x_{0}} f_{n}(x)=A_{n} \quad(n=1,2, \cdots)
$$
. \begin{CJK}{UTF8}{mj}则\end{CJK} $\left\{A_{n}\right\}$ \begin{CJK}{UTF8}{mj}收敛\end{CJK}, \begin{CJK}{UTF8}{mj}且\end{CJK} $\lim _{x \rightarrow x_{0}} \lim _{n \rightarrow \infty} f_{n}(x)=\lim _{n \rightarrow \infty} \lim _{x \rightarrow x_{0}} f_{n}(x)$.

\begin{enumerate}
  \setcounter{enumi}{6}
  \item ( 15 \begin{CJK}{UTF8}{mj}分\end{CJK}) \begin{CJK}{UTF8}{mj}设\end{CJK} $f(x)$ \begin{CJK}{UTF8}{mj}在区间\end{CJK} $[a, b]$ \begin{CJK}{UTF8}{mj}上有连续的导函数\end{CJK}, $f(a)=0$. \begin{CJK}{UTF8}{mj}证明\end{CJK}:
\end{enumerate}
$$
\int_{a}^{b}\left|f(x) f^{\prime}(x)\right| \mathrm{d} x \leqslant \frac{b-a}{2} \int_{a}^{b}\left(f^{\prime}(x)\right)^{2} \mathrm{~d} x
$$

\begin{enumerate}
  \setcounter{enumi}{7}
  \item (15 \begin{CJK}{UTF8}{mj}分\end{CJK})
\end{enumerate}
$$
f(x, y)= \begin{cases}\frac{x y}{\sqrt{x^{2}+y^{2}}}, & \text { 当 }(x, y) \neq(0,0) \text { 时; } \\ 0 . & \text { 当 }(x, y)=(0,0) \text { 时. }\end{cases}
$$
\begin{CJK}{UTF8}{mj}问\end{CJK}:(1) $f(x, y)$ \begin{CJK}{UTF8}{mj}在点\end{CJK} $(0,0)$ \begin{CJK}{UTF8}{mj}是否连续\end{CJK}?

(2) $f(x, y)$ \begin{CJK}{UTF8}{mj}在点\end{CJK} $(0,0)$ \begin{CJK}{UTF8}{mj}是否可微\end{CJK}? \begin{CJK}{UTF8}{mj}请证明你的结论\end{CJK}.

\begin{enumerate}
  \setcounter{enumi}{8}
  \item (15 \begin{CJK}{UTF8}{mj}分\end{CJK}) \begin{CJK}{UTF8}{mj}计算\end{CJK}:
\end{enumerate}
$$
\iint_{S} y z \mathrm{~d} y \mathrm{~d} z+\left(x^{2}+z^{2}\right) y \mathrm{~d} z \mathrm{~d} x+x y \mathrm{~d} x \mathrm{~d} y,
$$
\begin{CJK}{UTF8}{mj}其中\end{CJK} $S$ \begin{CJK}{UTF8}{mj}为曲面\end{CJK} $4-y=x^{2}+z^{2} 上 y \geqslant 0$ \begin{CJK}{UTF8}{mj}的部分并取正侧\end{CJK}. 9. (15 \begin{CJK}{UTF8}{mj}分\end{CJK}) \begin{CJK}{UTF8}{mj}设\end{CJK} $\left\{u_{n}(x)\right\}$ \begin{CJK}{UTF8}{mj}是\end{CJK} $[a, b]$ \begin{CJK}{UTF8}{mj}上的正值递减\end{CJK} (\begin{CJK}{UTF8}{mj}即对固定的\end{CJK} $\left.x \in[a, b], u_{n+1}(x) \leqslant u_{n}(x), n=1,2, \cdots\right)$ \begin{CJK}{UTF8}{mj}且收敛于零\end{CJK} \begin{CJK}{UTF8}{mj}的函数列\end{CJK}, \begin{CJK}{UTF8}{mj}而对每个固定的\end{CJK} $n, u_{n}(x)$ \begin{CJK}{UTF8}{mj}均是\end{CJK} $[a, b]$ \begin{CJK}{UTF8}{mj}上的递增函数\end{CJK}. \begin{CJK}{UTF8}{mj}证明\end{CJK}: \begin{CJK}{UTF8}{mj}级数\end{CJK} $\sum_{n=1}^{\infty}(-1)^{n-1} u_{n}(x)$ \begin{CJK}{UTF8}{mj}在\end{CJK} $[a, b]$ \begin{CJK}{UTF8}{mj}上一致收敛\end{CJK}.

\section{4. 南京师范大学 2010 年研究生入学考试试题数学分析 
 李扬 
 微信公众号: sxkyliyang}
\begin{enumerate}
  \item ( 15 \begin{CJK}{UTF8}{mj}分\end{CJK}) \begin{CJK}{UTF8}{mj}计算极限\end{CJK}
\end{enumerate}
(1) $\lim _{x \rightarrow 0}\left(1+x^{2}\right)^{\frac{1}{x}}$.

(2) \begin{CJK}{UTF8}{mj}设\end{CJK} $x_{0}=a, x_{1}=b, x_{n}=\frac{x_{n-2}-x_{n-1}}{3}, n=2,3, \cdots$. \begin{CJK}{UTF8}{mj}证明\end{CJK} $\left\{x_{n}\right\}$ \begin{CJK}{UTF8}{mj}收敛并求极限\end{CJK}.

\begin{enumerate}
  \setcounter{enumi}{2}
  \item ( 20 \begin{CJK}{UTF8}{mj}分\end{CJK}) $(1)$ \begin{CJK}{UTF8}{mj}设\end{CJK} $f(x), g(x)$ \begin{CJK}{UTF8}{mj}是有限开区间\end{CJK} $(a, b)$ \begin{CJK}{UTF8}{mj}上的一致连续函数\end{CJK}, \begin{CJK}{UTF8}{mj}求证\end{CJK} $f(x) g(x)$ \begin{CJK}{UTF8}{mj}在\end{CJK} $(a, b)$ \begin{CJK}{UTF8}{mj}上一致连续\end{CJK}.
\end{enumerate}
(2) \begin{CJK}{UTF8}{mj}试举例说明\end{CJK}: (1) \begin{CJK}{UTF8}{mj}中开区间若无\end{CJK}"\begin{CJK}{UTF8}{mj}有限\end{CJK}" \begin{CJK}{UTF8}{mj}条件\end{CJK}, \begin{CJK}{UTF8}{mj}则结论不成立\end{CJK}.

\begin{enumerate}
  \setcounter{enumi}{3}
  \item ( 10 \begin{CJK}{UTF8}{mj}分\end{CJK}) \begin{CJK}{UTF8}{mj}计算\end{CJK}
\end{enumerate}
$$
\int_{0}^{+\infty} \frac{e^{-x}-e^{-e x}}{x} d x
$$

\begin{enumerate}
  \setcounter{enumi}{4}
  \item (15 \begin{CJK}{UTF8}{mj}分\end{CJK}) \begin{CJK}{UTF8}{mj}试用有限覆盖定理证明根的存在性定理\end{CJK}.

  \item (15 \begin{CJK}{UTF8}{mj}分\end{CJK}) \begin{CJK}{UTF8}{mj}设函数\end{CJK} $f(x)$ \begin{CJK}{UTF8}{mj}在闭区间\end{CJK} $[a, A]$ \begin{CJK}{UTF8}{mj}上连续\end{CJK}, \begin{CJK}{UTF8}{mj}证明\end{CJK}:

\end{enumerate}
$$
\lim _{h \rightarrow 0} \frac{1}{h} \int_{a}^{x}[f(t+h)-f(t)] \mathrm{d} t=f(x)-f(a)(a<x<A) .
$$

\begin{enumerate}
  \setcounter{enumi}{6}
  \item ( 15 \begin{CJK}{UTF8}{mj}分\end{CJK}) \begin{CJK}{UTF8}{mj}计算二重积分\end{CJK}
\end{enumerate}
$$
\iint_{D} x y \mathrm{~d} x \mathrm{~d} y
$$
\begin{CJK}{UTF8}{mj}其中\end{CJK} $D$ \begin{CJK}{UTF8}{mj}是由曲线\end{CJK} $x^{2}+y^{2}=x+y$ \begin{CJK}{UTF8}{mj}所围成的区域\end{CJK}.

\begin{enumerate}
  \setcounter{enumi}{7}
  \item ( 15 \begin{CJK}{UTF8}{mj}分\end{CJK}) \begin{CJK}{UTF8}{mj}证明\end{CJK}: \begin{CJK}{UTF8}{mj}函数\end{CJK} $y=\sum_{n=0}^{\infty} \frac{x^{n}}{(n !)^{2}}$ \begin{CJK}{UTF8}{mj}满足方程\end{CJK} $x y^{\prime \prime}+y^{\prime}-y=0$.

  \item ( 20 \begin{CJK}{UTF8}{mj}分\end{CJK} $)(1)$ \begin{CJK}{UTF8}{mj}设\end{CJK} $f(x)$ \begin{CJK}{UTF8}{mj}在\end{CJK} $[0,+\infty)$ \begin{CJK}{UTF8}{mj}上可微\end{CJK}, \begin{CJK}{UTF8}{mj}且\end{CJK} $0 \leqslant f^{\prime}(x) \leqslant f(x), f(0)=0$. \begin{CJK}{UTF8}{mj}试证明\end{CJK}: \begin{CJK}{UTF8}{mj}在\end{CJK} $[0,+\infty)$ \begin{CJK}{UTF8}{mj}上\end{CJK} $f(x) \equiv 0$.

\end{enumerate}
(2) \begin{CJK}{UTF8}{mj}证明\end{CJK}: \begin{CJK}{UTF8}{mj}不存在\end{CJK} $[0,+\infty)$ \begin{CJK}{UTF8}{mj}上的正值连续函数\end{CJK} $f(x)$, \begin{CJK}{UTF8}{mj}使得\end{CJK} $f^{\prime}(x)+\sqrt{f(x)} \leqslant 0$.

\begin{enumerate}
  \setcounter{enumi}{9}
  \item ( 15 \begin{CJK}{UTF8}{mj}分\end{CJK}) \begin{CJK}{UTF8}{mj}设\end{CJK} $f(x)$ \begin{CJK}{UTF8}{mj}在\end{CJK} $[a, b](b>a>0)$ \begin{CJK}{UTF8}{mj}上连续\end{CJK}, \begin{CJK}{UTF8}{mj}在\end{CJK} $(a, b)$ \begin{CJK}{UTF8}{mj}内可导\end{CJK}. \begin{CJK}{UTF8}{mj}证明\end{CJK}: \begin{CJK}{UTF8}{mj}存在\end{CJK} $\xi, \eta \in(a, b)$, \begin{CJK}{UTF8}{mj}使得\end{CJK}
\end{enumerate}
$$
f^{\prime}(\xi)=\frac{(b+a) f^{\prime}(\eta)}{2 \eta}
$$

\begin{enumerate}
  \setcounter{enumi}{10}
  \item ( 10 \begin{CJK}{UTF8}{mj}分\end{CJK}) \begin{CJK}{UTF8}{mj}设\end{CJK} $\left\{P_{n}(x)\right\}$ \begin{CJK}{UTF8}{mj}是多项式序列\end{CJK}, \begin{CJK}{UTF8}{mj}且在\end{CJK} $\mathbb{R}$ \begin{CJK}{UTF8}{mj}上一致收敛于\end{CJK} $P(x)$. \begin{CJK}{UTF8}{mj}证明\end{CJK}: $P(x)$ \begin{CJK}{UTF8}{mj}也是多项式\end{CJK}.
\end{enumerate}
\section{5. 南京师范大学 2011 年研究生入学考试试题数学分析}
\begin{CJK}{UTF8}{mj}李扬\end{CJK}

\begin{CJK}{UTF8}{mj}微信公众号\end{CJK}: sxkyliyang

\begin{enumerate}
  \item (30 \begin{CJK}{UTF8}{mj}分\end{CJK}) (1) $\lim _{x \rightarrow 1} x^{\frac{1}{1-x}}$;
\end{enumerate}
(2) \begin{CJK}{UTF8}{mj}求极限\end{CJK}
$$
\lim _{x \rightarrow 0} \frac{e-(1+x)^{\frac{1}{x}}}{x}
$$
(3) \begin{CJK}{UTF8}{mj}设\end{CJK} $f(x)$ \begin{CJK}{UTF8}{mj}在\end{CJK} $x_{0}$ \begin{CJK}{UTF8}{mj}处可导\end{CJK}, \begin{CJK}{UTF8}{mj}求\end{CJK}
$$
\lim _{h \rightarrow 0} \frac{f\left(x_{0}+h^{4}\right)-f\left(x_{0}\right)}{1-\cos \left(h^{2}\right)}
$$
(4) $\int\left(\ln \ln x+\frac{1}{\ln x}\right) \mathrm{d} x$;

(5)\begin{CJK}{UTF8}{mj}求极限\end{CJK}
$$
\lim _{x \rightarrow+\infty} \int_{x}^{x+2010} \sin t^{2} \mathrm{~d} t
$$

\begin{enumerate}
  \setcounter{enumi}{2}
  \item (20 \begin{CJK}{UTF8}{mj}分\end{CJK}) (1) \begin{CJK}{UTF8}{mj}设\end{CJK} $\lim _{n \rightarrow \infty} a_{n}=a$, \begin{CJK}{UTF8}{mj}试用极限定义证明\end{CJK}
\end{enumerate}
$$
\lim _{n \rightarrow \infty} \frac{a_{1}+a_{2}+\cdots+a_{n}}{n}=a
$$
(2) \begin{CJK}{UTF8}{mj}举例说明上述结论反之不对\end{CJK}, \begin{CJK}{UTF8}{mj}并对你的例子给出简要的说明\end{CJK}.

(3) \begin{CJK}{UTF8}{mj}设\end{CJK}
$$
\lim _{n \rightarrow \infty} \frac{a_{1}+a_{2}+\cdots+a_{n}}{n}=a
$$
\begin{CJK}{UTF8}{mj}且\end{CJK} $\lim _{n \rightarrow \infty} n\left[a_{n}-a_{n-1}\right]=0$. \begin{CJK}{UTF8}{mj}证明\end{CJK}: $\lim _{n \rightarrow \infty} a_{n}=a$.

\begin{enumerate}
  \setcounter{enumi}{3}
  \item ( 15 \begin{CJK}{UTF8}{mj}分\end{CJK}) \begin{CJK}{UTF8}{mj}设函数\end{CJK} $f$ \begin{CJK}{UTF8}{mj}连续\end{CJK}, $f^{\prime}(0)$ \begin{CJK}{UTF8}{mj}存在\end{CJK}, \begin{CJK}{UTF8}{mj}且对任何\end{CJK} $x, y$ \begin{CJK}{UTF8}{mj}都有\end{CJK} $f(x+y)=\frac{f(x)+f(y)}{1-4 f(x) f(y)}$, \begin{CJK}{UTF8}{mj}证明\end{CJK}:
\end{enumerate}
(1) $f$ \begin{CJK}{UTF8}{mj}可导\end{CJK};

(2) \begin{CJK}{UTF8}{mj}若\end{CJK} $f^{\prime}(0)=\frac{1}{2}$, \begin{CJK}{UTF8}{mj}求\end{CJK} $f(x)$.

\begin{enumerate}
  \setcounter{enumi}{4}
  \item (15 \begin{CJK}{UTF8}{mj}分\end{CJK}) \begin{CJK}{UTF8}{mj}证明\end{CJK}: \begin{CJK}{UTF8}{mj}若函数\end{CJK} $f(x)$ \begin{CJK}{UTF8}{mj}在\end{CJK} $[a,+\infty)$ \begin{CJK}{UTF8}{mj}上连续可微\end{CJK}, \begin{CJK}{UTF8}{mj}且无穷积分\end{CJK} $\int_{a}^{+\infty} f(x) \mathrm{d} x$ \begin{CJK}{UTF8}{mj}与\end{CJK} $\int_{a}^{+\infty} f^{\prime}(x) \mathrm{d} x$ \begin{CJK}{UTF8}{mj}均收敛\end{CJK}, \begin{CJK}{UTF8}{mj}则\end{CJK} $\lim _{n \rightarrow \infty} f(x)=0 .$

  \item ( 15 \begin{CJK}{UTF8}{mj}分\end{CJK}) \begin{CJK}{UTF8}{mj}设\end{CJK} $f(x)$ \begin{CJK}{UTF8}{mj}在\end{CJK} $[a, b]$ \begin{CJK}{UTF8}{mj}上连续\end{CJK}, $f^{\prime \prime}(x)$ \begin{CJK}{UTF8}{mj}在\end{CJK} $(a, b)$ \begin{CJK}{UTF8}{mj}内存在\end{CJK}, \begin{CJK}{UTF8}{mj}试证明存在\end{CJK} $\xi \in(a, b)$, \begin{CJK}{UTF8}{mj}使得\end{CJK}

\end{enumerate}
$$
f(b)+f(a)-2 f\left(\frac{a+b}{2}\right)=\frac{(b-a)^{2}}{4} f^{\prime \prime}(\xi) .
$$

\begin{enumerate}
  \setcounter{enumi}{6}
  \item ( 10 \begin{CJK}{UTF8}{mj}分\end{CJK}) \begin{CJK}{UTF8}{mj}设\end{CJK} $f(x)$ \begin{CJK}{UTF8}{mj}在\end{CJK} $[0,2]$ \begin{CJK}{UTF8}{mj}上二次可微\end{CJK}, \begin{CJK}{UTF8}{mj}且\end{CJK} $x \in[0,2]$ \begin{CJK}{UTF8}{mj}时\end{CJK} $|f(x)| \leqslant 1,\left|f^{\prime \prime}(x)\right| \leqslant 1$. \begin{CJK}{UTF8}{mj}证明\end{CJK}:
\end{enumerate}
$$
\left|f^{\prime}(x)\right| \leqslant 2, x \in[0,2] .
$$

\begin{enumerate}
  \setcounter{enumi}{7}
  \item (15 \begin{CJK}{UTF8}{mj}分\end{CJK}) \begin{CJK}{UTF8}{mj}计算反常积分\end{CJK} $I(\alpha)=\int_{0}^{+\infty} e^{-\alpha x} \frac{\sin x}{x} \mathrm{~d} x,(\alpha>0)$. \begin{CJK}{UTF8}{mj}并由此计算\end{CJK} $\int_{0}^{+\infty} \frac{\sin x}{x} \mathrm{~d} x$ \begin{CJK}{UTF8}{mj}之值\end{CJK}.

  \item (15 \begin{CJK}{UTF8}{mj}分\end{CJK}) \begin{CJK}{UTF8}{mj}计算\end{CJK}

\end{enumerate}
$$
\iint_{S}(x+y-z) \mathrm{d} y \mathrm{~d} z+(2 y+\cos (z-x)) \mathrm{d} z \mathrm{~d} x+\left(3 z+e^{x-y}\right) \mathrm{d} x \mathrm{~d} y
$$
\begin{CJK}{UTF8}{mj}其中\end{CJK} $S$ \begin{CJK}{UTF8}{mj}为曲面\end{CJK} $|x-y+z|+|y-z+x|+|z-x+y|=1$ \begin{CJK}{UTF8}{mj}的表面并取外侧\end{CJK}. 9 . (15 \begin{CJK}{UTF8}{mj}分\end{CJK}) \begin{CJK}{UTF8}{mj}设\end{CJK} $S_{n}=\sum_{k=0}^{n} a_{k}, \sigma_{n}=\frac{S_{0}+S_{1}+\cdots+S_{n-1}}{n}$.

\begin{CJK}{UTF8}{mj}证明\end{CJK}: (1) \begin{CJK}{UTF8}{mj}若\end{CJK} $\left\{\sigma_{n}\right\}$ \begin{CJK}{UTF8}{mj}收敛\end{CJK}, \begin{CJK}{UTF8}{mj}则\end{CJK} $\sigma_{n}=o(n),($ \begin{CJK}{UTF8}{mj}当\end{CJK} $n \rightarrow \infty$ \begin{CJK}{UTF8}{mj}时\end{CJK}).

(2) \begin{CJK}{UTF8}{mj}若\end{CJK} $\left\{\sigma_{n}\right\}$ \begin{CJK}{UTF8}{mj}收敛\end{CJK}, \begin{CJK}{UTF8}{mj}则\end{CJK} $f(x)=\sum_{n=0}^{\infty} a_{n} x^{n}$ \begin{CJK}{UTF8}{mj}在\end{CJK} $(-1,1)$ \begin{CJK}{UTF8}{mj}内绝对收敛\end{CJK}, \begin{CJK}{UTF8}{mj}且\end{CJK}
$$
f(x)=\left(1-x^{2}\right) \sum_{n=0}^{\infty}(n+1) \sigma_{n+1} x^{n}, x \in(-1,1)
$$

\section{6. 南京师范大学 2012 年研究生入学考试试题数学分析 
 李扬 
 微信公众号: sxkyliyang}
\begin{enumerate}
  \item (15 \begin{CJK}{UTF8}{mj}分\end{CJK}) \begin{CJK}{UTF8}{mj}计算下列各题\end{CJK}.
\end{enumerate}
(1) $\lim _{x \rightarrow+\infty}\left(x-x \ln \frac{x+1}{x}\right)$;

(2) $\lim _{n \rightarrow \infty} \frac{\left(1^{\frac{3}{2}}+2^{\frac{3}{2}}+\cdots+n^{\frac{3}{2}}\right)^{\frac{5}{3}}}{\left(1^{\frac{2}{3}}+2^{\frac{2}{3}}+\cdots+n^{\frac{2}{3}}\right)^{\frac{5}{2}}}$.

\begin{enumerate}
  \setcounter{enumi}{2}
  \item ( 15 \begin{CJK}{UTF8}{mj}分\end{CJK}) \begin{CJK}{UTF8}{mj}设\end{CJK} $f(x)$ \begin{CJK}{UTF8}{mj}在有限区间\end{CJK} $[a, b]$ \begin{CJK}{UTF8}{mj}上可微\end{CJK}, \begin{CJK}{UTF8}{mj}且满足\end{CJK} $f_{+}^{\prime}(a) f_{-}^{\prime}(b)<0$, \begin{CJK}{UTF8}{mj}则存在\end{CJK} $c \in(a, b)$, \begin{CJK}{UTF8}{mj}使得\end{CJK} $f^{\prime}(c)=0$.

  \item ( 15 \begin{CJK}{UTF8}{mj}分\end{CJK}) \begin{CJK}{UTF8}{mj}设函数\end{CJK} $f(x)$ \begin{CJK}{UTF8}{mj}在\end{CJK} $[a,+\infty)$ \begin{CJK}{UTF8}{mj}上一致连续\end{CJK}, $g(x)$ \begin{CJK}{UTF8}{mj}在\end{CJK} $[a,+\infty)$ \begin{CJK}{UTF8}{mj}上连续\end{CJK}, \begin{CJK}{UTF8}{mj}且\end{CJK} $\lim _{x \rightarrow+\infty}[f(x)-g(x)]=0$. \begin{CJK}{UTF8}{mj}证明\end{CJK}: $g(x)$ \begin{CJK}{UTF8}{mj}在\end{CJK} $[a,+\infty)$ \begin{CJK}{UTF8}{mj}上一致连续\end{CJK}.

  \item ( 15 \begin{CJK}{UTF8}{mj}分\end{CJK}) \begin{CJK}{UTF8}{mj}设数项级数\end{CJK} $\sum_{n=1}^{\infty} a_{n}$ \begin{CJK}{UTF8}{mj}收玫\end{CJK}, \begin{CJK}{UTF8}{mj}则\end{CJK}

\end{enumerate}
(1) $\lim _{n \rightarrow \infty} n a_{n}=0$;

(2) \begin{CJK}{UTF8}{mj}当数列\end{CJK} $\left\{a_{n}\right\}$ \begin{CJK}{UTF8}{mj}单调时\end{CJK},
$$
\lim _{n \rightarrow \infty} n a_{n}=0 ;
$$
(3) \begin{CJK}{UTF8}{mj}当\end{CJK} $a_{n}>0(n=1,2, \cdots)$ \begin{CJK}{UTF8}{mj}时\end{CJK},
$$
\lim _{n \rightarrow \infty} n a_{n}=0 ;
$$
\begin{CJK}{UTF8}{mj}对上述结论中正确的给予证明\end{CJK}, \begin{CJK}{UTF8}{mj}错误的给出反例\end{CJK}.

\begin{enumerate}
  \setcounter{enumi}{5}
  \item ( 15 \begin{CJK}{UTF8}{mj}分\end{CJK}) \begin{CJK}{UTF8}{mj}设函数\end{CJK} $f(x)$ \begin{CJK}{UTF8}{mj}在\end{CJK} $[0,+\infty)$ \begin{CJK}{UTF8}{mj}内递增\end{CJK}, \begin{CJK}{UTF8}{mj}对任何正数\end{CJK} $T, f(x)$ \begin{CJK}{UTF8}{mj}在\end{CJK} $[0, T]$ \begin{CJK}{UTF8}{mj}上可积\end{CJK}, \begin{CJK}{UTF8}{mj}且\end{CJK}
\end{enumerate}
$$
\left.\lim _{x \rightarrow+\infty} \frac{1}{x} \int_{0}^{x} f(t) \mathrm{d} t=C \text { ( } C \text { 为常数 }\right),
$$
\begin{CJK}{UTF8}{mj}证明\end{CJK}: $\lim _{x \rightarrow+\infty} f(x)=C$.

\begin{enumerate}
  \setcounter{enumi}{6}
  \item ( 20 \begin{CJK}{UTF8}{mj}分\end{CJK}) \begin{CJK}{UTF8}{mj}设\end{CJK} $f_{n}(x)=\sin x+\sin ^{2} x+\cdots+\sin ^{n} x$.
\end{enumerate}
\begin{CJK}{UTF8}{mj}求证\end{CJK}:

(1)\begin{CJK}{UTF8}{mj}对任意自然数\end{CJK} $n$, \begin{CJK}{UTF8}{mj}方程\end{CJK} $f_{n}(x)=1$ \begin{CJK}{UTF8}{mj}在\end{CJK} $\left(\frac{\pi}{6}, \frac{\pi}{2}\right]$ \begin{CJK}{UTF8}{mj}内有且仅有一个解\end{CJK};

(2) \begin{CJK}{UTF8}{mj}设\end{CJK} $x_{n} \in\left(\frac{\pi}{6}, \frac{\pi}{2}\right]$ \begin{CJK}{UTF8}{mj}是方程\end{CJK} $f_{n}(x)=1$ \begin{CJK}{UTF8}{mj}的解\end{CJK}, \begin{CJK}{UTF8}{mj}证明\end{CJK} $\lim _{n \rightarrow \infty} x_{n}=\frac{\pi}{6}$.

\begin{enumerate}
  \setcounter{enumi}{7}
  \item (15 \begin{CJK}{UTF8}{mj}分\end{CJK}) \begin{CJK}{UTF8}{mj}计算积分\end{CJK}
\end{enumerate}
$$
\oint_{L} \frac{(x+4 y) \mathrm{d} y+(x-y) \mathrm{d} x}{x^{2}+4 y^{2}}
$$
\begin{CJK}{UTF8}{mj}其中\end{CJK} $L$ \begin{CJK}{UTF8}{mj}为一条不经过点\end{CJK} $(0,0)$ \begin{CJK}{UTF8}{mj}任意正向的闭曲线\end{CJK}.

\begin{enumerate}
  \setcounter{enumi}{8}
  \item (10 \begin{CJK}{UTF8}{mj}分\end{CJK}) \begin{CJK}{UTF8}{mj}设数列\end{CJK} $\left\{a_{n}\right\}$ \begin{CJK}{UTF8}{mj}的极限为\end{CJK} $a$, \begin{CJK}{UTF8}{mj}证明\end{CJK} $f(x)=\sum_{n=1}^{\infty} a_{n} x^{n}$ \begin{CJK}{UTF8}{mj}在\end{CJK} $(-1,1)$ \begin{CJK}{UTF8}{mj}上有定义\end{CJK}, \begin{CJK}{UTF8}{mj}且\end{CJK} $\lim _{x \rightarrow 1^{-}}(1-x) f(x)=a$.

  \item ( 15 \begin{CJK}{UTF8}{mj}分\end{CJK}) \begin{CJK}{UTF8}{mj}设二元函数\end{CJK} $f(x, y)$ \begin{CJK}{UTF8}{mj}在正方形区域\end{CJK} $[0,1] \times[0,1]$ \begin{CJK}{UTF8}{mj}上连续\end{CJK}, \begin{CJK}{UTF8}{mj}记\end{CJK} $I=[0,1]$.

\end{enumerate}
(1) \begin{CJK}{UTF8}{mj}试比较\end{CJK} $\inf _{y \in I} \sup _{x \in I} f(x, y)$ \begin{CJK}{UTF8}{mj}与\end{CJK} $\sup _{x \in I} \inf _{y \in I} f(x, y)$ \begin{CJK}{UTF8}{mj}的大小并证明之\end{CJK};

(2) \begin{CJK}{UTF8}{mj}给出并证明使等式\end{CJK} $\inf _{y \in I} \sup _{x \in I} f(x, y)=\sup _{x \in I} \inf _{y \in I} f(x, y)$ \begin{CJK}{UTF8}{mj}成立的充分条件\end{CJK}; (\begin{CJK}{UTF8}{mj}你认为最好的\end{CJK}) 10. ( 15 \begin{CJK}{UTF8}{mj}分\end{CJK}) (1)\begin{CJK}{UTF8}{mj}证明\end{CJK} $I(x)=\int_{0}^{+\infty} \frac{y}{2+y^{x}} \mathrm{~d} y$ \begin{CJK}{UTF8}{mj}在\end{CJK} $(2,+\infty)$ \begin{CJK}{UTF8}{mj}内连续\end{CJK};

(2)\begin{CJK}{UTF8}{mj}利用欧拉积分计算\end{CJK}
$$
\int_{0}^{+\infty} \frac{x^{2}}{1+x^{4}} \mathrm{~d} x
$$
\begin{CJK}{UTF8}{mj}其中\end{CJK} $\Gamma(s) \Gamma(1-s)=\frac{\pi}{\sin \pi s} ;(0<s<1)$.

\section{7. 南京师范大学 2013 年研究生入学考试试题数学分析}
\begin{CJK}{UTF8}{mj}李扬\end{CJK}

\begin{CJK}{UTF8}{mj}微信公众号\end{CJK}: sxkyliyang

\begin{enumerate}
  \item ( 20 \begin{CJK}{UTF8}{mj}分\end{CJK}) \begin{CJK}{UTF8}{mj}计算下列各题\end{CJK}:
\end{enumerate}
(1) $\lim _{n \rightarrow \infty} \int_{0}^{\frac{\pi}{2}} \sin ^{n} x \mathrm{~d} x$;

(2) $\lim _{n \rightarrow \infty} n^{n^{2}}\left(\tan \frac{1}{n}\right)^{n^{2}}$.

\begin{enumerate}
  \setcounter{enumi}{2}
  \item (15 \begin{CJK}{UTF8}{mj}分\end{CJK}) \begin{CJK}{UTF8}{mj}叙述并证明一元函数的最大\end{CJK}, \begin{CJK}{UTF8}{mj}最小值定理\end{CJK}.

  \item (15 \begin{CJK}{UTF8}{mj}分\end{CJK}) \begin{CJK}{UTF8}{mj}判断函数\end{CJK} $\ln x$ \begin{CJK}{UTF8}{mj}在下列区间上的一致连续性\end{CJK}, \begin{CJK}{UTF8}{mj}并说明理由\end{CJK}.

\end{enumerate}
(1) $(0,1)$;

(2) $(1,+\infty)$.

\begin{enumerate}
  \setcounter{enumi}{4}
  \item ( 15 \begin{CJK}{UTF8}{mj}分\end{CJK}) \begin{CJK}{UTF8}{mj}计算积分\end{CJK}.
\end{enumerate}
$$
\iint_{S}(x+1) z^{2} \mathrm{~d} y \mathrm{~d} z+\left(x^{2} y-2\right) \mathrm{d} z \mathrm{~d} x+\left(x y+y^{2} z\right) \mathrm{d} x \mathrm{~d} y
$$
\begin{CJK}{UTF8}{mj}其中\end{CJK} $S$ \begin{CJK}{UTF8}{mj}为球面\end{CJK} $z^{2}+x^{2}+y^{2}=3^{2}$ \begin{CJK}{UTF8}{mj}的上半部分并选取外侧\end{CJK}.

\begin{enumerate}
  \setcounter{enumi}{5}
  \item ( 20 \begin{CJK}{UTF8}{mj}分\end{CJK} $)$ \begin{CJK}{UTF8}{mj}证明函数\end{CJK}
\end{enumerate}
$$
f(x)=\int_{0}^{+\infty} \frac{\cos x y}{1+y^{2}} \mathrm{~d} y
$$
\begin{CJK}{UTF8}{mj}在\end{CJK} $[0,+\infty)$ \begin{CJK}{UTF8}{mj}上连续\end{CJK}, \begin{CJK}{UTF8}{mj}在\end{CJK} $(0,+\infty)$ \begin{CJK}{UTF8}{mj}上连续可导\end{CJK}.

\begin{enumerate}
  \setcounter{enumi}{6}
  \item (15 \begin{CJK}{UTF8}{mj}分\end{CJK}) \begin{CJK}{UTF8}{mj}证明下列各题\end{CJK}.
\end{enumerate}
(1) \begin{CJK}{UTF8}{mj}设正项级数\end{CJK} $\sum_{n=1}^{\infty} u_{n}$ \begin{CJK}{UTF8}{mj}发散\end{CJK}, \begin{CJK}{UTF8}{mj}则\end{CJK} $\forall k>0$, \begin{CJK}{UTF8}{mj}正项级数\end{CJK} $\sum_{n=1}^{\infty}\left(k+\frac{1}{n^{2}}\right) u_{n}$ \begin{CJK}{UTF8}{mj}发散\end{CJK}.

(2) \begin{CJK}{UTF8}{mj}证明函数项级数\end{CJK} $\sum_{n=1}^{\infty} 3^{n} \sin \frac{1}{5^{n} x}$ \begin{CJK}{UTF8}{mj}在\end{CJK} $(0,+\infty)$ \begin{CJK}{UTF8}{mj}上收敛\end{CJK}, \begin{CJK}{UTF8}{mj}但不一致收敛\end{CJK}.

\begin{enumerate}
  \setcounter{enumi}{7}
  \item ( 15 \begin{CJK}{UTF8}{mj}分\end{CJK}) \begin{CJK}{UTF8}{mj}构造一个二元函数\end{CJK} $f(x, y)$, \begin{CJK}{UTF8}{mj}使它在原点\end{CJK} $(0,0)$ \begin{CJK}{UTF8}{mj}连续且两偏导数存在\end{CJK}, \begin{CJK}{UTF8}{mj}但在原点\end{CJK} $(0,0)$ \begin{CJK}{UTF8}{mj}不可微\end{CJK}.

  \item ( 15 \begin{CJK}{UTF8}{mj}分\end{CJK}) \begin{CJK}{UTF8}{mj}证明\end{CJK}: \begin{CJK}{UTF8}{mj}设二元函数\end{CJK} $f(x, y)$ \begin{CJK}{UTF8}{mj}在区域\end{CJK} $D=\left\{(x, y) \mid 2 x^{2}+y^{2}<6\right\}$ \begin{CJK}{UTF8}{mj}上有定义\end{CJK}, $f(0, y)$ \begin{CJK}{UTF8}{mj}在点\end{CJK} $y=0$ \begin{CJK}{UTF8}{mj}处连续\end{CJK}, \begin{CJK}{UTF8}{mj}且\end{CJK} $f_{x}(x, y)$ \begin{CJK}{UTF8}{mj}在区域\end{CJK} $D$ \begin{CJK}{UTF8}{mj}上有界\end{CJK}, \begin{CJK}{UTF8}{mj}则\end{CJK} $f(x, y)$ \begin{CJK}{UTF8}{mj}在点\end{CJK} $(0,0)$ \begin{CJK}{UTF8}{mj}连续\end{CJK}.

  \item ( 20 \begin{CJK}{UTF8}{mj}分\end{CJK}) \begin{CJK}{UTF8}{mj}证明方程\end{CJK} $3 y-3 x-\sin y=0$ \begin{CJK}{UTF8}{mj}在原点的附近能唯一地确定隐函数\end{CJK} $y=f(x)$, \begin{CJK}{UTF8}{mj}描绘隐函数\end{CJK} $y=f(x)$ \begin{CJK}{UTF8}{mj}在\end{CJK} \begin{CJK}{UTF8}{mj}原点附近的图像\end{CJK}.

\end{enumerate}
\section{8. 南京师范大学 2014 年研究生入学考试试题数学分析 
 李扬 
 微信公众号: sxkyliyang}
\begin{enumerate}
  \item (15 \begin{CJK}{UTF8}{mj}分\end{CJK}) \begin{CJK}{UTF8}{mj}设\end{CJK} $a_{0}=1, a_{n}=\sin a_{n-1}(n=1,2, \cdots)$, \begin{CJK}{UTF8}{mj}求\end{CJK}:
\end{enumerate}
(1) $\lim _{n \rightarrow \infty} a_{n}$;

(2) $\lim _{n \rightarrow \infty} \frac{1}{n a_{n}^{2}}$.

\begin{enumerate}
  \setcounter{enumi}{2}
  \item ( 15 \begin{CJK}{UTF8}{mj}分\end{CJK}) \begin{CJK}{UTF8}{mj}计算不定积分\end{CJK} $I=\int \frac{1}{\sqrt[n]{(x-a)^{n+1}(x-b)^{n-1}}} \mathrm{~d} x, a \neq b$ \begin{CJK}{UTF8}{mj}且\end{CJK} $n$ \begin{CJK}{UTF8}{mj}为正整数\end{CJK}.

  \item ( 15 \begin{CJK}{UTF8}{mj}分\end{CJK} $)$ \begin{CJK}{UTF8}{mj}若\end{CJK} $f^{\prime}(a)>0$, \begin{CJK}{UTF8}{mj}能否断定函数\end{CJK} $f$ \begin{CJK}{UTF8}{mj}在点\end{CJK} $a$ \begin{CJK}{UTF8}{mj}的某个邻域\end{CJK} $U(a ; \delta)$ \begin{CJK}{UTF8}{mj}内单调递增\end{CJK}? \begin{CJK}{UTF8}{mj}若是\end{CJK}, \begin{CJK}{UTF8}{mj}请简要证明\end{CJK}; \begin{CJK}{UTF8}{mj}若不能\end{CJK}, \begin{CJK}{UTF8}{mj}请\end{CJK} \begin{CJK}{UTF8}{mj}举例说明\end{CJK}.

  \item ( 15 \begin{CJK}{UTF8}{mj}分\end{CJK}) \begin{CJK}{UTF8}{mj}已知函数\end{CJK} $f(x)$ \begin{CJK}{UTF8}{mj}在区间\end{CJK} $(-1,1)$ \begin{CJK}{UTF8}{mj}内有二阶导数\end{CJK}, \begin{CJK}{UTF8}{mj}且\end{CJK} $f(0)=f^{\prime}(0)=0,\left|f^{\prime \prime}(x)\right| \leqslant|f(x)|+\left|f^{\prime}(x)\right|$. \begin{CJK}{UTF8}{mj}证明\end{CJK}: \begin{CJK}{UTF8}{mj}存在\end{CJK} $0<\delta<1$, \begin{CJK}{UTF8}{mj}使得在\end{CJK} $(-\delta, \delta)$ \begin{CJK}{UTF8}{mj}内\end{CJK} $f(x) \equiv 0$.

  \item ( 15 \begin{CJK}{UTF8}{mj}分\end{CJK}) \begin{CJK}{UTF8}{mj}设\end{CJK} $f(x)$ \begin{CJK}{UTF8}{mj}在\end{CJK} $[0,2 \pi]$ \begin{CJK}{UTF8}{mj}上连续\end{CJK}, \begin{CJK}{UTF8}{mj}证明\end{CJK}:

\end{enumerate}
$$
\lim _{n \rightarrow \infty} \int_{0}^{2 \pi} f(x)|\sin n x| \mathrm{d} x=\frac{2}{\pi} \int_{0}^{2 \pi} f(x) \mathrm{d} x
$$

\begin{enumerate}
  \setcounter{enumi}{6}
  \item ( 15 \begin{CJK}{UTF8}{mj}分\end{CJK}) \begin{CJK}{UTF8}{mj}设\end{CJK} $f(x)$ \begin{CJK}{UTF8}{mj}在\end{CJK} $[a,+\infty)$ \begin{CJK}{UTF8}{mj}上一致连续\end{CJK}, \begin{CJK}{UTF8}{mj}且\end{CJK} $\int_{a}^{+\infty} f(x) \mathrm{d} x$ \begin{CJK}{UTF8}{mj}收敛\end{CJK}, \begin{CJK}{UTF8}{mj}证明\end{CJK}: $\lim _{x \rightarrow+\infty} f(x)=0$.

  \item (15 \begin{CJK}{UTF8}{mj}分\end{CJK}) \begin{CJK}{UTF8}{mj}设\end{CJK} $f$ \begin{CJK}{UTF8}{mj}是以\end{CJK} $2 \pi$ \begin{CJK}{UTF8}{mj}为周期\end{CJK}, \begin{CJK}{UTF8}{mj}且具有二阶连续可微的函数\end{CJK}, $b_{n}=\frac{1}{\pi} \int_{-\pi}^{\pi} f(x) \sin n x \mathrm{~d} x, b_{n}^{\prime \prime}=\frac{1}{\pi} \int_{-\pi}^{\pi} f^{\prime \prime}(x) \sin n x \mathrm{~d} x$, \begin{CJK}{UTF8}{mj}若级数\end{CJK} $\sum_{n=1}^{\infty} b_{n}^{\prime \prime}$ \begin{CJK}{UTF8}{mj}绝对收敛\end{CJK}, \begin{CJK}{UTF8}{mj}证明\end{CJK}:

\end{enumerate}
$$
\sum_{n=1}^{\infty} \sqrt{\left|b_{n}\right|} \leqslant \frac{1}{2}\left(2+\sum_{n=1}^{\infty}\left|b_{n}^{\prime \prime}\right|\right)
$$

\begin{enumerate}
  \setcounter{enumi}{8}
  \item (15 \begin{CJK}{UTF8}{mj}分\end{CJK}) \begin{CJK}{UTF8}{mj}求积分\end{CJK} $\int_{0}^{+\infty} e^{-x} \frac{1-\cos x y}{x^{2}} \mathrm{~d} x$.

  \item (15 \begin{CJK}{UTF8}{mj}分\end{CJK}) \begin{CJK}{UTF8}{mj}设\end{CJK} $f_{x}, f_{y}$ \begin{CJK}{UTF8}{mj}在点\end{CJK} $\left(x_{0}, y_{0}\right)$ \begin{CJK}{UTF8}{mj}的某邻域内存在\end{CJK}, \begin{CJK}{UTF8}{mj}且在点\end{CJK} $\left(x_{0}, y_{0}\right)$ \begin{CJK}{UTF8}{mj}可微\end{CJK}, \begin{CJK}{UTF8}{mj}证明\end{CJK}: $f_{x y}\left(x_{0}, y_{0}\right)=f_{y x}\left(x_{0}, y_{0}\right)$.

  \item (15 \begin{CJK}{UTF8}{mj}分\end{CJK}) \begin{CJK}{UTF8}{mj}求积分\end{CJK}

\end{enumerate}
$$
\iint_{S}[f(x, y, z)+x] \mathrm{d} y \mathrm{~d} z+[f(x, y, z)+z] \mathrm{d} x \mathrm{~d} y+[2 f(x, y, z)+y] \mathrm{d} x \mathrm{~d} z
$$
\begin{CJK}{UTF8}{mj}其中\end{CJK} $f(x, y, z)$ \begin{CJK}{UTF8}{mj}为\end{CJK} $S$ \begin{CJK}{UTF8}{mj}上的连续函数\end{CJK}, $S$ \begin{CJK}{UTF8}{mj}为平面\end{CJK} $x+y+z=1$ \begin{CJK}{UTF8}{mj}在第\end{CJK} $I V$ \begin{CJK}{UTF8}{mj}卦限部分之上侧\end{CJK}.

\section{9. 南京师范大学 2015 年研究生入学考试试题数学分析}
\begin{CJK}{UTF8}{mj}李扬\end{CJK}

\begin{CJK}{UTF8}{mj}微信公众号\end{CJK}: sxkyliyang

\begin{enumerate}
  \item ( 15 \begin{CJK}{UTF8}{mj}分\end{CJK}) \begin{CJK}{UTF8}{mj}设\end{CJK} $a_{n}=\frac{1}{1^{\alpha}}+\frac{1}{2^{\alpha}}+\cdots+\frac{1}{n^{\alpha}}(n=1,2, \cdots)$, \begin{CJK}{UTF8}{mj}证明\end{CJK}:
\end{enumerate}
(1) \begin{CJK}{UTF8}{mj}当\end{CJK} $\alpha=1$ \begin{CJK}{UTF8}{mj}时\end{CJK}, \begin{CJK}{UTF8}{mj}数列\end{CJK} $\left\{a_{n}\right\}$ \begin{CJK}{UTF8}{mj}发散\end{CJK};

(2) \begin{CJK}{UTF8}{mj}当\end{CJK} $\alpha=2$ \begin{CJK}{UTF8}{mj}时\end{CJK}, \begin{CJK}{UTF8}{mj}数列\end{CJK} $\left\{a_{n}\right\}$ \begin{CJK}{UTF8}{mj}收敛\end{CJK};

(3) \begin{CJK}{UTF8}{mj}当\end{CJK} $\alpha>1$ \begin{CJK}{UTF8}{mj}时\end{CJK}, \begin{CJK}{UTF8}{mj}数列\end{CJK} $\left\{a_{n}\right\}$ \begin{CJK}{UTF8}{mj}收敛\end{CJK}.

2 . ( 15 \begin{CJK}{UTF8}{mj}分\end{CJK}) \begin{CJK}{UTF8}{mj}设\end{CJK} $a_{1}, a_{2}, \cdots, a_{k}$ \begin{CJK}{UTF8}{mj}为\end{CJK} $k$ \begin{CJK}{UTF8}{mj}个正数\end{CJK}.

(1) \begin{CJK}{UTF8}{mj}求\end{CJK}
$$
\lim _{n \rightarrow \infty}\left(\frac{a_{1}^{\frac{1}{n}}+a_{2}^{\frac{1}{n}}+\cdots+a_{k}^{\frac{1}{\frac{1}{k}}}}{k}\right)^{n} .
$$
(2) \begin{CJK}{UTF8}{mj}令\end{CJK}
$$
f(x)=\left(\frac{a_{1}^{x}+a_{2}^{x}+\cdots+a_{k}^{x}}{k}\right)^{\frac{1}{x}}
$$
\begin{CJK}{UTF8}{mj}求\end{CJK} $\lim _{x \rightarrow 0^{+}} f(x)$.

\begin{enumerate}
  \setcounter{enumi}{3}
  \item (15 \begin{CJK}{UTF8}{mj}分\end{CJK}) \begin{CJK}{UTF8}{mj}已知\end{CJK} $f(x)$ \begin{CJK}{UTF8}{mj}在\end{CJK} $x=0$ \begin{CJK}{UTF8}{mj}处可导\end{CJK},
\end{enumerate}
(1) \begin{CJK}{UTF8}{mj}若\end{CJK} $f(0)=0$, \begin{CJK}{UTF8}{mj}求\end{CJK}
$$
\lim _{x \rightarrow+\infty} \frac{f((1-\cos x) \ln (1+x))}{\left(e^{x}-1\right) \sin x^{2}}
$$
(2) \begin{CJK}{UTF8}{mj}若\end{CJK} $f(0)>0$, \begin{CJK}{UTF8}{mj}求\end{CJK}
$$
\lim _{n \rightarrow \infty}\left(\frac{f\left(\frac{1}{n}\right)}{f(0)}\right)^{n}
$$

\begin{enumerate}
  \setcounter{enumi}{4}
  \item ( 15 \begin{CJK}{UTF8}{mj}分\end{CJK}) (1) \begin{CJK}{UTF8}{mj}证明\end{CJK} $\sin \frac{1}{x}$ \begin{CJK}{UTF8}{mj}在\end{CJK} $(0,+\infty)$ \begin{CJK}{UTF8}{mj}内不一致连续\end{CJK};
\end{enumerate}
(2) $\sin \frac{1}{x}$ \begin{CJK}{UTF8}{mj}在\end{CJK} $[1,+\infty)$ \begin{CJK}{UTF8}{mj}内一致连续\end{CJK};

(3) \begin{CJK}{UTF8}{mj}设\end{CJK} $f^{x}(x)$ \begin{CJK}{UTF8}{mj}在\end{CJK} $(0,+\infty)$ \begin{CJK}{UTF8}{mj}内可导\end{CJK}, \begin{CJK}{UTF8}{mj}且\end{CJK} $\sqrt{x} f^{\prime}(x)$ \begin{CJK}{UTF8}{mj}在\end{CJK} $(0,+\infty)$ \begin{CJK}{UTF8}{mj}内有界\end{CJK}, \begin{CJK}{UTF8}{mj}证明\end{CJK}: $f(x)$ \begin{CJK}{UTF8}{mj}在\end{CJK} $(0,+\infty)$ \begin{CJK}{UTF8}{mj}内一致连续\end{CJK}.

\begin{enumerate}
  \setcounter{enumi}{5}
  \item (15 \begin{CJK}{UTF8}{mj}分\end{CJK}) \begin{CJK}{UTF8}{mj}计算不定积分\end{CJK} $\int \frac{\mathrm{d} x}{x^{3} \sqrt{x^{2}-1}}$.

  \item (15 \begin{CJK}{UTF8}{mj}分\end{CJK}) \begin{CJK}{UTF8}{mj}设\end{CJK} $f(x)$ \begin{CJK}{UTF8}{mj}是\end{CJK} $(-\infty,+\infty)$ \begin{CJK}{UTF8}{mj}内周期为\end{CJK} $T(>0)$ \begin{CJK}{UTF8}{mj}的连续函数\end{CJK}, \begin{CJK}{UTF8}{mj}证明\end{CJK}:

\end{enumerate}
(1) \begin{CJK}{UTF8}{mj}任给\end{CJK} $a \in(-\infty,+\infty)$,
$$
\int_{a}^{a+T} f(x) \mathrm{d} x=\int_{0}^{T} f(x) \mathrm{d} x
$$
(2) $\lim _{x \rightarrow+\infty} \frac{1}{x} \int_{0}^{x} f(t) \mathrm{d} t=\frac{1}{T} \int_{0}^{T} f(t) \mathrm{d} t$.

\begin{enumerate}
  \setcounter{enumi}{7}
  \item (12 \begin{CJK}{UTF8}{mj}分\end{CJK}) \begin{CJK}{UTF8}{mj}设函数\end{CJK} $f(x)$ \begin{CJK}{UTF8}{mj}在\end{CJK} $[-a, a](a>0)$ \begin{CJK}{UTF8}{mj}上连续\end{CJK}, \begin{CJK}{UTF8}{mj}且对任意\end{CJK} $x \in[-a, a], x \neq 0$, \begin{CJK}{UTF8}{mj}有\end{CJK} $|f(x)| \leqslant|x|$. \begin{CJK}{UTF8}{mj}又\end{CJK}
\end{enumerate}
$$
f_{1}(x)=f(x), f_{2}(x)=f\left(f_{1}(x)\right), \cdots, f_{n+1}(x)=f\left(f_{n}(x)\right), \cdots
$$
\begin{CJK}{UTF8}{mj}证明\end{CJK}: \begin{CJK}{UTF8}{mj}函数列\end{CJK} $\left\{f_{n}(x)\right\}$ \begin{CJK}{UTF8}{mj}在\end{CJK} $[-a, a]$ \begin{CJK}{UTF8}{mj}上一致收敛于\end{CJK} 0 .

\begin{enumerate}
  \setcounter{enumi}{8}
  \item (12 \begin{CJK}{UTF8}{mj}分\end{CJK}) \begin{CJK}{UTF8}{mj}讨论函数\end{CJK}
\end{enumerate}
$$
f(x, y)= \begin{cases}\frac{1-e^{x\left(x^{2}+y^{2}\right)}}{x^{2}+y^{2}}, & x^{2}+y^{2} \neq 0 \\ 0, & x^{2}+y^{2}=0\end{cases}
$$
\begin{CJK}{UTF8}{mj}在点\end{CJK} $(0,0)$ \begin{CJK}{UTF8}{mj}处的连续性\end{CJK}, \begin{CJK}{UTF8}{mj}可微性\end{CJK}, \begin{CJK}{UTF8}{mj}偏导数的存在性以及偏导数的连续性\end{CJK}.

\begin{enumerate}
  \setcounter{enumi}{9}
  \item (12\begin{CJK}{UTF8}{mj}分\end{CJK}) \begin{CJK}{UTF8}{mj}求函数\end{CJK} $f(x, y, z)=\ln x+\ln y+3 \ln z$ \begin{CJK}{UTF8}{mj}的最大值\end{CJK}, \begin{CJK}{UTF8}{mj}其中\end{CJK} $x^{2}+y^{2}+z^{2}=5 r^{2}(x, y, z$ \begin{CJK}{UTF8}{mj}均\end{CJK} $>0, r>0)$, \begin{CJK}{UTF8}{mj}并利\end{CJK} \begin{CJK}{UTF8}{mj}用所得结果证明不等式\end{CJK}
\end{enumerate}
$$
a b c^{3} \leqslant 27\left(\frac{a+b+c}{5}\right)^{5} \cdot(a, b, c \text { 均 }>0)
$$

\begin{enumerate}
  \setcounter{enumi}{10}
  \item (12\begin{CJK}{UTF8}{mj}分\end{CJK}) \begin{CJK}{UTF8}{mj}证明\end{CJK}: $\int_{0}^{\pi} \frac{\mathrm{d} x}{\sqrt{3+\cos x}}=\frac{1}{2 \sqrt{2}} B\left(\frac{1}{4}, \frac{1}{2}\right)$.

  \item (12\begin{CJK}{UTF8}{mj}分\end{CJK}) \begin{CJK}{UTF8}{mj}计算积分\end{CJK}

\end{enumerate}
$$
\iint_{S} \frac{x-1}{r^{3}} \mathrm{~d} y \mathrm{~d} z+\frac{y-2}{r^{3}} \mathrm{~d} z \mathrm{~d} x+\frac{z-3}{r^{3}} \mathrm{~d} x \mathrm{~d} y
$$
\begin{CJK}{UTF8}{mj}其中\end{CJK} $S$ \begin{CJK}{UTF8}{mj}为长方体\end{CJK}
$$
V=\{(x, y, z) \mid x \in[-2,2], y \in[-3,3], z \in[-4,4]\}
$$
\begin{CJK}{UTF8}{mj}的表面的外侧\end{CJK}, $r=\sqrt{(x-1)^{2}+(y-2)^{2}+(z-3)^{2}}$.

\section{0. 南京师范大学 2016 年研究生入学考试试题数学分析}
\begin{CJK}{UTF8}{mj}李扬\end{CJK}

\begin{CJK}{UTF8}{mj}微信公众号\end{CJK}: sxkyliyang

\begin{enumerate}
  \item (15 \begin{CJK}{UTF8}{mj}分\end{CJK}) \begin{CJK}{UTF8}{mj}计算下列各题\end{CJK}.
\end{enumerate}
(1) \begin{CJK}{UTF8}{mj}设\end{CJK}
$$
a_{n}=\sum_{k=1}^{n} \frac{k}{n^{2}+n+k},
$$
\begin{CJK}{UTF8}{mj}求\end{CJK} $\lim _{n \rightarrow \infty} a_{n}$.

(2) \begin{CJK}{UTF8}{mj}求\end{CJK}
$$
\lim _{m \rightarrow \infty} \lim _{n \rightarrow \infty} \frac{m}{n} \sum_{k=1}^{n} \sin \frac{k}{m n}
$$

\begin{enumerate}
  \setcounter{enumi}{2}
  \item ( 20 \begin{CJK}{UTF8}{mj}分\end{CJK}) \begin{CJK}{UTF8}{mj}完成下列各题并说明理由\end{CJK}.
\end{enumerate}
(1) \begin{CJK}{UTF8}{mj}给出函数\end{CJK} $f:(-1,1) \rightarrow \mathbb{R}, f$ \begin{CJK}{UTF8}{mj}只在一点连续\end{CJK};

(2) \begin{CJK}{UTF8}{mj}给出函数\end{CJK} $g:(-1,1) \rightarrow \mathbb{R}, g$ \begin{CJK}{UTF8}{mj}只在一点可导\end{CJK}.

\begin{enumerate}
  \setcounter{enumi}{3}
  \item (20 \begin{CJK}{UTF8}{mj}分\end{CJK}) \begin{CJK}{UTF8}{mj}证明下列各题\end{CJK}.
\end{enumerate}
(1) \begin{CJK}{UTF8}{mj}设函数\end{CJK} $h(x)$ \begin{CJK}{UTF8}{mj}在\end{CJK} $\mathbb{R}$ \begin{CJK}{UTF8}{mj}上可导\end{CJK}, \begin{CJK}{UTF8}{mj}且存在\end{CJK} $K \geqslant 0$ \begin{CJK}{UTF8}{mj}使\end{CJK} $\left|h^{\prime}(x)\right| \leqslant K, \forall x \in \mathbb{R}$, \begin{CJK}{UTF8}{mj}则\end{CJK} $h(x)$ \begin{CJK}{UTF8}{mj}在\end{CJK} $\mathbb{R}$ \begin{CJK}{UTF8}{mj}上一致连续\end{CJK}.

(2) \begin{CJK}{UTF8}{mj}设函数\end{CJK} $h(x)=x^{2}$, \begin{CJK}{UTF8}{mj}则\end{CJK} $h(x)$ \begin{CJK}{UTF8}{mj}在\end{CJK} $\mathbb{R}$ \begin{CJK}{UTF8}{mj}上不一致连续\end{CJK}.

\begin{enumerate}
  \setcounter{enumi}{4}
  \item ( 20 \begin{CJK}{UTF8}{mj}分\end{CJK}) \begin{CJK}{UTF8}{mj}完成下列各题并给出证明\end{CJK}.
\end{enumerate}
\begin{CJK}{UTF8}{mj}设\end{CJK}
$$
s(x)=\sum_{n=1}^{\infty} \frac{2^{-n x}}{n^{2}}
$$
(1) \begin{CJK}{UTF8}{mj}给出函数\end{CJK} $s(x)$ \begin{CJK}{UTF8}{mj}的连续范围\end{CJK};

(2) \begin{CJK}{UTF8}{mj}给出函数\end{CJK} $s(x)$ \begin{CJK}{UTF8}{mj}的可导范围\end{CJK}.

\begin{enumerate}
  \setcounter{enumi}{5}
  \item ( 20 \begin{CJK}{UTF8}{mj}分\end{CJK}) \begin{CJK}{UTF8}{mj}计算积分\end{CJK}.
\end{enumerate}
(1)
$$
\iint_{D} e^{v(x, y)} \mathrm{d} x \mathrm{~d} y
$$
\begin{CJK}{UTF8}{mj}其中\end{CJK} $v(x, y)=\frac{x-y}{x+y}, D$ \begin{CJK}{UTF8}{mj}是由\end{CJK} $x=0, y=0, x+y=1$ \begin{CJK}{UTF8}{mj}所围区域\end{CJK}.
$$
\int_{A B}(\sin y+y) \mathrm{d} x+x \cos y \mathrm{~d} y
$$
\begin{CJK}{UTF8}{mj}其中\end{CJK} $A B$ \begin{CJK}{UTF8}{mj}为由\end{CJK} $(0,0)$ \begin{CJK}{UTF8}{mj}到\end{CJK} $(3,0)$ \begin{CJK}{UTF8}{mj}经曲线\end{CJK} $y=x(3-x)$ \begin{CJK}{UTF8}{mj}上半部的路线\end{CJK}.

\section{1. 南京师范大学 2017 年研究生入学考试试题数学分析 
 李扬 
 微信公众号: sxkyliyang}
\begin{enumerate}
  \item (15 \begin{CJK}{UTF8}{mj}分\end{CJK}) \begin{CJK}{UTF8}{mj}计算下列各题\end{CJK}.\\
(1) $\lim _{n \rightarrow \infty} \frac{1+\sqrt{2}+\sqrt[3]{3}+\cdots+\sqrt[n]{n}}{n}$;\\
(2) $\lim _{x \rightarrow 0^{+}}(\sin x)^{\frac{1}{1+\ln x}}$;\\
(3) \begin{CJK}{UTF8}{mj}计算积分\end{CJK} $\int_{0}^{\frac{\pi}{4}} \frac{1-\sin 2 x}{1+\sin 2 x} \mathrm{~d} x$.

  \item (15 \begin{CJK}{UTF8}{mj}分\end{CJK}) \begin{CJK}{UTF8}{mj}设\end{CJK} $f(x)$ \begin{CJK}{UTF8}{mj}是\end{CJK} $[a, b]$ \begin{CJK}{UTF8}{mj}上一个非常数的连续函数\end{CJK}, $M, m$ \begin{CJK}{UTF8}{mj}分别是其最大值和最小值\end{CJK}. \begin{CJK}{UTF8}{mj}求证\end{CJK}: \begin{CJK}{UTF8}{mj}存在\end{CJK} $[\alpha, \beta] \subset[a, b]$, \begin{CJK}{UTF8}{mj}使得\end{CJK} $x \in[\alpha, \beta]$ \begin{CJK}{UTF8}{mj}时\end{CJK}, $m<f(x)<M$.

  \item ( 15 \begin{CJK}{UTF8}{mj}分\end{CJK})

\end{enumerate}
(1) \begin{CJK}{UTF8}{mj}举例说明\end{CJK}: \begin{CJK}{UTF8}{mj}有界可微函数的导函数不一定有界\end{CJK};

(2) \begin{CJK}{UTF8}{mj}证明\end{CJK}: \begin{CJK}{UTF8}{mj}有界开区间\end{CJK} $(a, b)$ \begin{CJK}{UTF8}{mj}内的无界可微函数\end{CJK} $f(x)$ \begin{CJK}{UTF8}{mj}的导函数必定无界\end{CJK}.

\begin{enumerate}
  \setcounter{enumi}{4}
  \item ( 15 \begin{CJK}{UTF8}{mj}分\end{CJK}) \begin{CJK}{UTF8}{mj}设\end{CJK} $f(x) \in C[a, b]$, \begin{CJK}{UTF8}{mj}且严格单调递减\end{CJK}. \begin{CJK}{UTF8}{mj}证明\end{CJK}: $(a+b) \int_{a}^{b} f(x) \mathrm{d} x<2 \int_{a}^{b} x f(x) \mathrm{d} x$.

  \item ( 15 \begin{CJK}{UTF8}{mj}分\end{CJK}) $f(x) \in C[0,1]$, \begin{CJK}{UTF8}{mj}且\end{CJK} $|f(x)| \geqslant 1+\frac{1}{2} \int_{0}^{x}|f(t)| \mathrm{d} t, x \in[0,1]$. \begin{CJK}{UTF8}{mj}证明\end{CJK}: $\ln |f(x)| \geqslant \frac{x^{2}}{4}, x \in[0,1]$.

  \item ( 15 \begin{CJK}{UTF8}{mj}分\end{CJK}) \begin{CJK}{UTF8}{mj}设连续函数列\end{CJK} $\left\{f_{n}(x)\right\}$ \begin{CJK}{UTF8}{mj}在\end{CJK} $[a, b]$ \begin{CJK}{UTF8}{mj}上一致收敛于\end{CJK} $f(x)$, \begin{CJK}{UTF8}{mj}而\end{CJK} $g(u)$ \begin{CJK}{UTF8}{mj}在\end{CJK} $(-\infty,+\infty)$ \begin{CJK}{UTF8}{mj}内连续\end{CJK}. \begin{CJK}{UTF8}{mj}证明\end{CJK}: $\left\{g\left(f_{n}(x)\right)\right\}$ \begin{CJK}{UTF8}{mj}在\end{CJK} $[a, b]$ \begin{CJK}{UTF8}{mj}上一致收敛于\end{CJK} $g(f(x))$.

  \item ( 15 \begin{CJK}{UTF8}{mj}分\end{CJK}) \begin{CJK}{UTF8}{mj}设\end{CJK} $f(x, y)=\left\{\begin{array}{ll}\frac{x^{2} y}{x^{4}+y^{2}}, & x^{2}+y^{2} \neq 0 ; \\ 0 . & x^{2}+y^{2}=0\end{array}\right.$, \begin{CJK}{UTF8}{mj}证明\end{CJK}:

\end{enumerate}
(1) $f(x, y)$ \begin{CJK}{UTF8}{mj}在\end{CJK} $(0,0)$ \begin{CJK}{UTF8}{mj}处任意方向的方向导数都存在\end{CJK};

(2) $f(x, y)$ \begin{CJK}{UTF8}{mj}在\end{CJK} $(0,0)$ \begin{CJK}{UTF8}{mj}处不可微\end{CJK}.

\begin{enumerate}
  \setcounter{enumi}{8}
  \item (15 \begin{CJK}{UTF8}{mj}分\end{CJK}) \begin{CJK}{UTF8}{mj}设函数\end{CJK} $f$ \begin{CJK}{UTF8}{mj}在\end{CJK} $(-\infty,+\infty)$ \begin{CJK}{UTF8}{mj}上无限次可微\end{CJK}, \begin{CJK}{UTF8}{mj}且\end{CJK} $\left\{f^{(n)}(x)\right\}$ \begin{CJK}{UTF8}{mj}在\end{CJK} $(-\infty,+\infty)$ \begin{CJK}{UTF8}{mj}上一致有界\end{CJK}, \begin{CJK}{UTF8}{mj}且存在正数列\end{CJK} $\left\{\xi_{n}\right\}$, \begin{CJK}{UTF8}{mj}使得\end{CJK} $\lim _{n} \xi_{n}=0$, \begin{CJK}{UTF8}{mj}且\end{CJK} $f\left(\xi_{n}\right)=0, n=1,2, \cdots$, \begin{CJK}{UTF8}{mj}证明\end{CJK}: $f(x) \equiv 0$.

  \item ( 15 \begin{CJK}{UTF8}{mj}分\end{CJK}) \begin{CJK}{UTF8}{mj}计算曲面积分\end{CJK}

\end{enumerate}
$$
I=\iint_{S} 4 x z \mathrm{~d} y \mathrm{~d} z-2 y z \mathrm{~d} z \mathrm{~d} x+\left(1-z^{2}\right) \mathrm{d} x \mathrm{~d} y
$$
\begin{CJK}{UTF8}{mj}其中\end{CJK} $S$ \begin{CJK}{UTF8}{mj}为曲线\end{CJK} $z=e^{y}(0 \leqslant y \leqslant a)$ \begin{CJK}{UTF8}{mj}绕\end{CJK} $z$ \begin{CJK}{UTF8}{mj}轴旋转一周生成的旋转曲面\end{CJK}, \begin{CJK}{UTF8}{mj}并取上侧\end{CJK}.

\begin{enumerate}
  \setcounter{enumi}{10}
  \item (15 \begin{CJK}{UTF8}{mj}分\end{CJK}) \begin{CJK}{UTF8}{mj}证明积分\end{CJK} $\int_{0}^{1} \frac{\sin \frac{1}{x}}{x^{p}} \mathrm{~d} x$ \begin{CJK}{UTF8}{mj}在\end{CJK} $0<p<2$ \begin{CJK}{UTF8}{mj}中非一致收敛\end{CJK}, \begin{CJK}{UTF8}{mj}但在\end{CJK} $0<p \leqslant 2-\delta(0<\delta<2)$ \begin{CJK}{UTF8}{mj}中一致收敛\end{CJK}.
\end{enumerate}
\section{1. 南开大学 2007 年研究生入学考试试题高等代数 
 李扬 
 微信公众号: sxkyliyang}
\begin{enumerate}
  \item \begin{CJK}{UTF8}{mj}计算题\end{CJK}. (\begin{CJK}{UTF8}{mj}每题\end{CJK} 12 \begin{CJK}{UTF8}{mj}分\end{CJK}, \begin{CJK}{UTF8}{mj}共\end{CJK} 60 \begin{CJK}{UTF8}{mj}分\end{CJK})
\end{enumerate}
(1) \begin{CJK}{UTF8}{mj}试求下列行列式的值\end{CJK}
$$
\left|\begin{array}{cccc}
-2 & 5 & -1 & 3 \\
1 & -9 & 13 & 7 \\
3 & -1 & 5 & 5 \\
2 & 8 & -7 & -10
\end{array}\right|
$$
(2) \begin{CJK}{UTF8}{mj}求\end{CJK} 3 \begin{CJK}{UTF8}{mj}阶实矩阵\end{CJK} $\left(\begin{array}{ccc}a & -1 & a x-y \\ 1 & a & x+a y \\ b & c & b x+c y\end{array}\right)$ \begin{CJK}{UTF8}{mj}的秩\end{CJK}.

(3) \begin{CJK}{UTF8}{mj}设\end{CJK} $A, B$ \begin{CJK}{UTF8}{mj}为\end{CJK} $n$ \begin{CJK}{UTF8}{mj}阶实正定矩阵\end{CJK}, $C$ \begin{CJK}{UTF8}{mj}为任意\end{CJK} $n$ \begin{CJK}{UTF8}{mj}阶实矩阵\end{CJK}, \begin{CJK}{UTF8}{mj}试求分块矩阵\end{CJK} $\left(\begin{array}{cc}A & C \\ -C^{\prime} & B\end{array}\right)$ \begin{CJK}{UTF8}{mj}的秩\end{CJK}.

(4) \begin{CJK}{UTF8}{mj}设\end{CJK} 3 \begin{CJK}{UTF8}{mj}阶方阵\end{CJK} $A=\left(\begin{array}{lll}6 & 3 & 2 \\ 1 & 5 & 2 \\ 1 & 1 & 3\end{array}\right)$, \begin{CJK}{UTF8}{mj}试将\end{CJK} 3 \begin{CJK}{UTF8}{mj}阶单位矩阵\end{CJK} $E_{3}$ \begin{CJK}{UTF8}{mj}写成\end{CJK} $A$ \begin{CJK}{UTF8}{mj}的多项式\end{CJK}.

(5) \begin{CJK}{UTF8}{mj}设\end{CJK} $A=\left(\begin{array}{ccc}-3 & -1 & -1 \\ 1 & -1 & -3 \\ 0 & 0 & 2\end{array}\right)$, \begin{CJK}{UTF8}{mj}求\end{CJK} $A$ \begin{CJK}{UTF8}{mj}的若尔当标准型\end{CJK} $J$, \begin{CJK}{UTF8}{mj}并求可逆矩阵\end{CJK} $T$ \begin{CJK}{UTF8}{mj}使得\end{CJK} $T^{-1} A T=J$.

\begin{enumerate}
  \setcounter{enumi}{2}
  \item ( 20 \begin{CJK}{UTF8}{mj}分\end{CJK})\begin{CJK}{UTF8}{mj}设\end{CJK} $V$ \begin{CJK}{UTF8}{mj}为数域\end{CJK} $P$ \begin{CJK}{UTF8}{mj}上的\end{CJK} $n$ \begin{CJK}{UTF8}{mj}维线性空间\end{CJK}, $V_{1}, V_{2}$ \begin{CJK}{UTF8}{mj}是\end{CJK} $V$ \begin{CJK}{UTF8}{mj}的子空间\end{CJK}, $V=V_{1} \oplus V_{2}, \mathscr{A}$ \begin{CJK}{UTF8}{mj}是\end{CJK} $V$ \begin{CJK}{UTF8}{mj}上的线性变\end{CJK} \begin{CJK}{UTF8}{mj}换\end{CJK}. \begin{CJK}{UTF8}{mj}证明\end{CJK}: $\mathscr{A}$ \begin{CJK}{UTF8}{mj}是可逆的当且仅当\end{CJK} $V=\mathscr{A}\left(V_{1}\right) \oplus \mathscr{A}\left(V_{2}\right)$.

  \item ( 20 \begin{CJK}{UTF8}{mj}分\end{CJK})\begin{CJK}{UTF8}{mj}设\end{CJK} $V$ \begin{CJK}{UTF8}{mj}为数域\end{CJK} $P$ \begin{CJK}{UTF8}{mj}上的有限维线性空间\end{CJK}, \begin{CJK}{UTF8}{mj}对于\end{CJK} $V$ \begin{CJK}{UTF8}{mj}中\end{CJK} $m$ \begin{CJK}{UTF8}{mj}个向量组成的向量组\end{CJK} $S=\left\{\alpha_{1}, \alpha_{2}, \cdots, \alpha_{m}\right\}$, \begin{CJK}{UTF8}{mj}定义\end{CJK} $P^{m \times 1}$ \begin{CJK}{UTF8}{mj}中的集合\end{CJK}

\end{enumerate}
$$
W_{S}=\left\{\left(a_{1}, a_{2}, \cdots, a_{m}\right)^{\prime} \mid a_{1} \alpha_{1}+a_{2} \alpha_{2}+\cdots+a_{m} \alpha_{m}=0, a_{i} \in P\right\}
$$
(1) \begin{CJK}{UTF8}{mj}证明\end{CJK}: $W_{S}$ \begin{CJK}{UTF8}{mj}为\end{CJK} $P^{m \times 1}$ \begin{CJK}{UTF8}{mj}的线性子空间\end{CJK};

(2) \begin{CJK}{UTF8}{mj}设\end{CJK} $S=\left\{\alpha_{1}, \alpha_{2}, \cdots, \alpha_{m}\right\}, S^{\prime}=\left\{\alpha_{1}^{\prime}, \alpha_{2}^{\prime}, \cdots, \alpha_{m}^{\prime}\right\}$ \begin{CJK}{UTF8}{mj}为两向量组\end{CJK}, \begin{CJK}{UTF8}{mj}证明\end{CJK}: \begin{CJK}{UTF8}{mj}存在\end{CJK} $V$ \begin{CJK}{UTF8}{mj}中线性变换\end{CJK} $\tau$ \begin{CJK}{UTF8}{mj}使得\end{CJK} $\tau\left(\alpha_{i}\right)=\alpha_{i}^{\prime},(i=1,2, \cdots, m)$ \begin{CJK}{UTF8}{mj}的充要条件是\end{CJK} $W_{S}=W_{S^{\prime}} .$

\begin{enumerate}
  \setcounter{enumi}{4}
  \item ( 20 \begin{CJK}{UTF8}{mj}分\end{CJK})\begin{CJK}{UTF8}{mj}设\end{CJK} $A$ \begin{CJK}{UTF8}{mj}为\end{CJK} $n$ \begin{CJK}{UTF8}{mj}阶正交矩阵\end{CJK}, \begin{CJK}{UTF8}{mj}且\end{CJK} $-1$ \begin{CJK}{UTF8}{mj}不是\end{CJK} $A$ \begin{CJK}{UTF8}{mj}的特征值\end{CJK}. \begin{CJK}{UTF8}{mj}证明\end{CJK}: $B=\left(A-E_{n}\right)\left(A+E_{n}\right)^{-1}$ \begin{CJK}{UTF8}{mj}是反对称矩阵\end{CJK} \begin{CJK}{UTF8}{mj}且\end{CJK} $A=\left(B-E_{n}\right)\left(B+E_{n}\right)^{-1}$.

  \item ( 15 \begin{CJK}{UTF8}{mj}分\end{CJK})\begin{CJK}{UTF8}{mj}设\end{CJK} $V$ \begin{CJK}{UTF8}{mj}为数域\end{CJK} $P$ \begin{CJK}{UTF8}{mj}上的有限维线性空间\end{CJK}, $\mathscr{A}$ \begin{CJK}{UTF8}{mj}为\end{CJK} $V$ \begin{CJK}{UTF8}{mj}上的线性变换\end{CJK}, \begin{CJK}{UTF8}{mj}且\end{CJK} $\mathscr{A}$ \begin{CJK}{UTF8}{mj}满足\end{CJK}

\end{enumerate}
$$
\mathscr{A}^{4}+\mathscr{A}^{3}-3 \mathscr{A}^{2}-\mathscr{A}+2 \mathscr{E}=\mathscr{O}
$$
\begin{CJK}{UTF8}{mj}其中\end{CJK} $\mathscr{E}$ \begin{CJK}{UTF8}{mj}为恒等变换\end{CJK}, \begin{CJK}{UTF8}{mj}若存在一个非零向量\end{CJK} $\alpha \in V$ \begin{CJK}{UTF8}{mj}使得\end{CJK} $\mathscr{A}^{3}(\alpha)+\mathscr{A}^{2}(\alpha)-\mathscr{A}(\alpha)=3 \alpha$, \begin{CJK}{UTF8}{mj}试问是否存在\end{CJK} $V$ \begin{CJK}{UTF8}{mj}的一组基\end{CJK}, \begin{CJK}{UTF8}{mj}使得\end{CJK} $\mathscr{A}$ \begin{CJK}{UTF8}{mj}在这组基下的矩阵为对角阵\end{CJK}? \begin{CJK}{UTF8}{mj}说明理由\end{CJK}.

\begin{enumerate}
  \setcounter{enumi}{6}
  \item ( 15 \begin{CJK}{UTF8}{mj}分\end{CJK}) \begin{CJK}{UTF8}{mj}设\end{CJK} $A$ \begin{CJK}{UTF8}{mj}为\end{CJK} $n$ \begin{CJK}{UTF8}{mj}阶正定对称矩阵\end{CJK}, $B$ \begin{CJK}{UTF8}{mj}为\end{CJK} $n$ \begin{CJK}{UTF8}{mj}阶实反对称矩阵\end{CJK}. \begin{CJK}{UTF8}{mj}证明\end{CJK}: $|A+B|>0$.
\end{enumerate}
\section{2. 南开大学 2008 年研究生入学考试试题高等代数}
\begin{CJK}{UTF8}{mj}李扬\end{CJK}

\begin{CJK}{UTF8}{mj}微信公众号\end{CJK}: sxkyliyang

\begin{CJK}{UTF8}{mj}一\end{CJK}、\begin{CJK}{UTF8}{mj}计算题\end{CJK}. (\begin{CJK}{UTF8}{mj}每小题\end{CJK} 12 \begin{CJK}{UTF8}{mj}分\end{CJK}, \begin{CJK}{UTF8}{mj}共\end{CJK} 60 \begin{CJK}{UTF8}{mj}分\end{CJK})

\begin{enumerate}
  \item \begin{CJK}{UTF8}{mj}设\end{CJK} $n$ \begin{CJK}{UTF8}{mj}阶实矩阵\end{CJK} $A=\left(a_{i j}\right)_{n \times n}$ \begin{CJK}{UTF8}{mj}满足条件\end{CJK}:
\end{enumerate}
(1) $a_{i i}>0, i=1,2, \ldots, n$;

(2) $a_{i j}<0, \quad i \neq j$

(3) $\sum_{i=1}^{n} a_{i k}=0, k=1,2, \ldots, n$.

\begin{CJK}{UTF8}{mj}试求\end{CJK} $A$ \begin{CJK}{UTF8}{mj}的秩\end{CJK}.

\begin{enumerate}
  \setcounter{enumi}{2}
  \item \begin{CJK}{UTF8}{mj}设\end{CJK} $A=\left(a_{i j}\right)_{n \times n}$ \begin{CJK}{UTF8}{mj}为数域\end{CJK} $P$ \begin{CJK}{UTF8}{mj}上的\end{CJK} $n$ \begin{CJK}{UTF8}{mj}阶方阵\end{CJK}, \begin{CJK}{UTF8}{mj}定义\end{CJK} $P^{n \times n}$ \begin{CJK}{UTF8}{mj}上的线性变换\end{CJK} $T$, \begin{CJK}{UTF8}{mj}使\end{CJK} $T(X)=A X, X \in P^{n \times n}$, \begin{CJK}{UTF8}{mj}求\end{CJK} $T$ \begin{CJK}{UTF8}{mj}的迹和行列式\end{CJK}.

  \item \begin{CJK}{UTF8}{mj}设\end{CJK} $P$ \begin{CJK}{UTF8}{mj}为数域\end{CJK} $c_{0}, c_{1}, \ldots, c_{n-1} \in P$, \begin{CJK}{UTF8}{mj}令\end{CJK}

\end{enumerate}
$$
A=\left(\begin{array}{cccccc}
0 & 0 & 0 & \cdots & 0 & -c_{0} \\
1 & 0 & 0 & \cdots & 0 & -c_{1} \\
0 & 1 & 0 & \cdots & 0 & -c_{2} \\
\vdots & \vdots & \vdots & & \vdots & \vdots \\
0 & 0 & 0 & \cdots & 0 & -c_{n-2} \\
0 & 0 & 0 & \cdots & 1 & -c_{n-1}
\end{array}\right)
$$
\begin{CJK}{UTF8}{mj}试求\end{CJK} $A$ \begin{CJK}{UTF8}{mj}的最小多项式\end{CJK}.

\begin{enumerate}
  \setcounter{enumi}{4}
  \item \begin{CJK}{UTF8}{mj}设\end{CJK} $V$ \begin{CJK}{UTF8}{mj}为数域\end{CJK} $P$ \begin{CJK}{UTF8}{mj}上的\end{CJK} 3 \begin{CJK}{UTF8}{mj}维线性空间\end{CJK}, \begin{CJK}{UTF8}{mj}已知\end{CJK} $V$ \begin{CJK}{UTF8}{mj}上的线性变换\end{CJK} $T$ \begin{CJK}{UTF8}{mj}在基\end{CJK} $\varepsilon_{1}, \varepsilon_{2}, \varepsilon_{3}$ \begin{CJK}{UTF8}{mj}下的矩阵为\end{CJK}
\end{enumerate}
$$
\left(\begin{array}{lll}
1 & 0 & -2 \\
0 & 1 & -2 \\
0 & 0 & -1
\end{array}\right)
$$
\begin{CJK}{UTF8}{mj}试求\end{CJK} $V$ \begin{CJK}{UTF8}{mj}的一组基使得\end{CJK} $T$ \begin{CJK}{UTF8}{mj}在该基下的矩阵为\end{CJK}
$$
\left(\begin{array}{rrr}
1 & 2 & -2 \\
0 & 5 & -4 \\
0 & 6 & -5
\end{array}\right)
$$

\begin{enumerate}
  \setcounter{enumi}{5}
  \item \begin{CJK}{UTF8}{mj}设\end{CJK} $n$ \begin{CJK}{UTF8}{mj}阶实矩阵\end{CJK} $P$ \begin{CJK}{UTF8}{mj}满足\end{CJK} $P^{T}=P^{2}$, \begin{CJK}{UTF8}{mj}试求出\end{CJK} $P$ \begin{CJK}{UTF8}{mj}的所有可能的特征值\end{CJK}.
\end{enumerate}
\begin{CJK}{UTF8}{mj}二\end{CJK}、 $(20$ \begin{CJK}{UTF8}{mj}分\end{CJK} $)$ \begin{CJK}{UTF8}{mj}设\end{CJK} $A_{1}, A_{2}, \cdots, A_{m}$ \begin{CJK}{UTF8}{mj}为\end{CJK} $n$ \begin{CJK}{UTF8}{mj}阶方阵\end{CJK}, \begin{CJK}{UTF8}{mj}且\end{CJK} $r\left(A_{1} A_{2} \cdots A_{m}\right)=r\left(A_{m}\right)$, \begin{CJK}{UTF8}{mj}证明\end{CJK}: \begin{CJK}{UTF8}{mj}对任意的\end{CJK} $1 \leq i, k \leq m$, \begin{CJK}{UTF8}{mj}方\end{CJK} \begin{CJK}{UTF8}{mj}程组\end{CJK} $A_{i} A_{i+1} \cdots A_{m} X=0$ \begin{CJK}{UTF8}{mj}与方程组\end{CJK} $A_{k} A_{k+1} \cdots A_{m} X=0$ \begin{CJK}{UTF8}{mj}同解\end{CJK}.

\begin{CJK}{UTF8}{mj}三\end{CJK}、 $\left(20\right.$ \begin{CJK}{UTF8}{mj}分\end{CJK}) \begin{CJK}{UTF8}{mj}设\end{CJK} $S, T$ \begin{CJK}{UTF8}{mj}为\end{CJK} $n$ \begin{CJK}{UTF8}{mj}阶实对称半正定矩阵\end{CJK}, \begin{CJK}{UTF8}{mj}证明\end{CJK} $\operatorname{det}(S+T) \geq \frac{1}{2}[\operatorname{det}(S)+\operatorname{det}(T)]$.

\begin{CJK}{UTF8}{mj}四\end{CJK}、(15 \begin{CJK}{UTF8}{mj}分\end{CJK}) \begin{CJK}{UTF8}{mj}设\end{CJK} $A, A-E_{n}$ \begin{CJK}{UTF8}{mj}均为\end{CJK} $n$ \begin{CJK}{UTF8}{mj}阶对称正定矩阵\end{CJK}, \begin{CJK}{UTF8}{mj}证明\end{CJK}: $E_{n}-A^{-1}$ \begin{CJK}{UTF8}{mj}也是正定矩阵\end{CJK}.

\begin{CJK}{UTF8}{mj}五\end{CJK}、( 15 \begin{CJK}{UTF8}{mj}分\end{CJK}) \begin{CJK}{UTF8}{mj}设\end{CJK} $f(x, y)$ \begin{CJK}{UTF8}{mj}为线性空间\end{CJK} $V$ \begin{CJK}{UTF8}{mj}上的非退化双线性函数\end{CJK}. \begin{CJK}{UTF8}{mj}证明\end{CJK}: \begin{CJK}{UTF8}{mj}对于任何\end{CJK} $g \in V^{*}$, \begin{CJK}{UTF8}{mj}存在唯一的\end{CJK} $\alpha \in V$ \begin{CJK}{UTF8}{mj}使得\end{CJK} $g(\beta)=f(\alpha, \beta), \forall \beta \in V$.

\begin{CJK}{UTF8}{mj}六\end{CJK}、 $T$ \begin{CJK}{UTF8}{mj}为欧式空间\end{CJK} $V$ \begin{CJK}{UTF8}{mj}上的线性变换\end{CJK}, $\forall x, y \in V$ \begin{CJK}{UTF8}{mj}满足\end{CJK} $(T x, y)=(x, T y)$ \begin{CJK}{UTF8}{mj}或\end{CJK} $(T x, y)=-(x, T y)$. \begin{CJK}{UTF8}{mj}证明\end{CJK}: $T$ \begin{CJK}{UTF8}{mj}为\end{CJK} \begin{CJK}{UTF8}{mj}对称变换或反对称变换\end{CJK}.

\begin{CJK}{UTF8}{mj}七\end{CJK}、\begin{CJK}{UTF8}{mj}设\end{CJK} $A, B$ \begin{CJK}{UTF8}{mj}为\end{CJK} $n$ \begin{CJK}{UTF8}{mj}阶复方阵\end{CJK}, $C=A B-B A$, \begin{CJK}{UTF8}{mj}若\end{CJK} $A C=C A$, \begin{CJK}{UTF8}{mj}则\end{CJK} $C$ \begin{CJK}{UTF8}{mj}为幂零矩阵\end{CJK}.

\section{3. 南开大学 2009 年研究生入学考试试题高等代数 
 李扬 
 微信公众号: sxkyliyang}
\begin{enumerate}
  \item (15 \begin{CJK}{UTF8}{mj}分\end{CJK}) \begin{CJK}{UTF8}{mj}线性变换\end{CJK} $\mathscr{A}$ \begin{CJK}{UTF8}{mj}在基\end{CJK} $\alpha_{1}, \alpha_{2}, \alpha_{3}, \alpha_{4}$ \begin{CJK}{UTF8}{mj}下的矩阵为\end{CJK}
\end{enumerate}
$$
A=\left(\begin{array}{cccc}
1 & 2 & 0 & 1 \\
3 & 0 & -1 & 2 \\
2 & 5 & 3 & 1 \\
1 & 2 & 1 & 3
\end{array}\right) .
$$
\begin{CJK}{UTF8}{mj}试求\end{CJK} $\mathscr{A}$ \begin{CJK}{UTF8}{mj}在基\end{CJK} $\alpha_{1}, \alpha_{1}+\alpha_{2}, \alpha_{1}+\alpha_{2}+\alpha_{3}, \alpha_{1}+\alpha_{2}+\alpha_{3}+\alpha_{4}$ \begin{CJK}{UTF8}{mj}下的矩阵\end{CJK}.

\begin{enumerate}
  \setcounter{enumi}{2}
  \item (20 \begin{CJK}{UTF8}{mj}分\end{CJK}) \begin{CJK}{UTF8}{mj}设矩阵\end{CJK}
\end{enumerate}
$$
A=\left(\begin{array}{cccc}
1 & -1 & 1 & 1 \\
-1 & -1 & 1 & -1 \\
-1 & 1 & -1 & -1 \\
-1 & -1 & 1 & 1
\end{array}\right) .
$$
\begin{CJK}{UTF8}{mj}在\end{CJK} $P^{4 \times 1}$ \begin{CJK}{UTF8}{mj}上定义线性变换\end{CJK} $\mathscr{A}$ \begin{CJK}{UTF8}{mj}使\end{CJK} $\mathscr{A} X=A X, X \in P^{4 \times 1}$, \begin{CJK}{UTF8}{mj}试求\end{CJK} $\mathscr{A}$ \begin{CJK}{UTF8}{mj}的像\end{CJK} $\operatorname{Im} \sigma$ \begin{CJK}{UTF8}{mj}与\end{CJK} $\operatorname{ker} \sigma$ \begin{CJK}{UTF8}{mj}的维数与一组基\end{CJK}.

\begin{enumerate}
  \setcounter{enumi}{3}
  \item ( 20 \begin{CJK}{UTF8}{mj}分\end{CJK}) \begin{CJK}{UTF8}{mj}试决定当实数\end{CJK} $a_{1}, a_{2}, \cdots, a_{n}$ \begin{CJK}{UTF8}{mj}满足什么条件时\end{CJK}, $n$ \begin{CJK}{UTF8}{mj}元实二次型\end{CJK}
\end{enumerate}
$$
f\left(x_{1}, x_{2}, \cdots, x_{n}\right)=\left(x_{1}+a_{2} x_{2}\right)^{2}+\left(x_{2}+a_{3} x_{3}\right)^{2}+\cdots+\left(x_{n-1}+a_{n} x_{n}\right)^{2}+\left(x_{n}+a_{1} x_{1}\right)^{2}
$$
\begin{CJK}{UTF8}{mj}是正定的\end{CJK}.

\begin{enumerate}
  \setcounter{enumi}{4}
  \item ( 20 \begin{CJK}{UTF8}{mj}分\end{CJK}) \begin{CJK}{UTF8}{mj}设\end{CJK} $\alpha \in \mathbb{R}^{n \times 1}$ \begin{CJK}{UTF8}{mj}上的标准度量下\end{CJK} $\alpha$ \begin{CJK}{UTF8}{mj}为单位向量\end{CJK}. \begin{CJK}{UTF8}{mj}证明\end{CJK}:\begin{CJK}{UTF8}{mj}必存在一个\end{CJK} $n$ \begin{CJK}{UTF8}{mj}阶实对称正交矩阵\end{CJK} $A$ \begin{CJK}{UTF8}{mj}使得\end{CJK} $\alpha$ \begin{CJK}{UTF8}{mj}为\end{CJK} $A$ \begin{CJK}{UTF8}{mj}的第一列\end{CJK}.

  \item ( 15 \begin{CJK}{UTF8}{mj}分\end{CJK}) \begin{CJK}{UTF8}{mj}设\end{CJK} $V$ \begin{CJK}{UTF8}{mj}为数域\end{CJK} $P$ \begin{CJK}{UTF8}{mj}上的\end{CJK} $n$ \begin{CJK}{UTF8}{mj}维线性空间\end{CJK}, $\mathscr{A}$ \begin{CJK}{UTF8}{mj}为\end{CJK} $V$ \begin{CJK}{UTF8}{mj}上的线性变换\end{CJK}, \begin{CJK}{UTF8}{mj}且\end{CJK} $\mathscr{A}$ \begin{CJK}{UTF8}{mj}的秩为\end{CJK} 1 . \begin{CJK}{UTF8}{mj}证明\end{CJK}: \begin{CJK}{UTF8}{mj}如果\end{CJK} $\mathscr{A}$ \begin{CJK}{UTF8}{mj}不\end{CJK} \begin{CJK}{UTF8}{mj}可对角化\end{CJK}, \begin{CJK}{UTF8}{mj}则必是幂零的\end{CJK}.

  \item ( 15 \begin{CJK}{UTF8}{mj}分\end{CJK}) \begin{CJK}{UTF8}{mj}设\end{CJK} $A, B$ \begin{CJK}{UTF8}{mj}为\end{CJK} $n$ \begin{CJK}{UTF8}{mj}阶复方阵\end{CJK}, \begin{CJK}{UTF8}{mj}证明\end{CJK}: $A B+A$ \begin{CJK}{UTF8}{mj}与\end{CJK} $B A+A$ \begin{CJK}{UTF8}{mj}有相同的特征值\end{CJK}, \begin{CJK}{UTF8}{mj}且每个特征值的重数相\end{CJK} \begin{CJK}{UTF8}{mj}同\end{CJK}.

  \item ( 15 \begin{CJK}{UTF8}{mj}分\end{CJK}) \begin{CJK}{UTF8}{mj}设\end{CJK} $A, B$ \begin{CJK}{UTF8}{mj}为\end{CJK} $n$ \begin{CJK}{UTF8}{mj}阶实正交矩阵\end{CJK}, \begin{CJK}{UTF8}{mj}且\end{CJK} $|A+B|=|A|-|B|$. \begin{CJK}{UTF8}{mj}证明\end{CJK}: $|A|=|B|$.

  \item ( 15 \begin{CJK}{UTF8}{mj}分\end{CJK}) \begin{CJK}{UTF8}{mj}设\end{CJK} $A=\left(a_{i j}\right)_{n \times n}, B=\left(b_{i j}\right)_{n \times n}$ \begin{CJK}{UTF8}{mj}为\end{CJK} $P^{n \times n}$ \begin{CJK}{UTF8}{mj}上的\end{CJK} $n$ \begin{CJK}{UTF8}{mj}级矩阵\end{CJK}, \begin{CJK}{UTF8}{mj}满足条件\end{CJK} $b_{i j}=b^{i-j} a_{i j}$, \begin{CJK}{UTF8}{mj}其中\end{CJK} $b$ \begin{CJK}{UTF8}{mj}为一非\end{CJK} \begin{CJK}{UTF8}{mj}零常数\end{CJK}, \begin{CJK}{UTF8}{mj}线性方程组\end{CJK} (I): $A X=C$ \begin{CJK}{UTF8}{mj}及\end{CJK} (II): $B X=D$. \begin{CJK}{UTF8}{mj}证明\end{CJK}: \begin{CJK}{UTF8}{mj}方程组\end{CJK} $(\mathrm{I})$ \begin{CJK}{UTF8}{mj}对任何\end{CJK} $C \in P^{n \times 1}$ \begin{CJK}{UTF8}{mj}有解当且仅\end{CJK} \begin{CJK}{UTF8}{mj}当方程组\end{CJK} $(\mathrm{II})$ \begin{CJK}{UTF8}{mj}对任何\end{CJK} $D \in P^{n \times 1}$ \begin{CJK}{UTF8}{mj}有解\end{CJK}.

  \item ( 15 \begin{CJK}{UTF8}{mj}分\end{CJK}) \begin{CJK}{UTF8}{mj}设\end{CJK} $P$ \begin{CJK}{UTF8}{mj}为数域\end{CJK}, $\mathrm{T}$ \begin{CJK}{UTF8}{mj}为\end{CJK} $P^{n \times n}$ \begin{CJK}{UTF8}{mj}上的线性变换\end{CJK}, \begin{CJK}{UTF8}{mj}满足条件\end{CJK}: \begin{CJK}{UTF8}{mj}对任何固定的\end{CJK} $A, B \in P^{n \times n}, \mathrm{~T}(\mathrm{AB})=$ $\mathrm{T}(\mathrm{A}) \mathrm{T}(\mathrm{B})$ \begin{CJK}{UTF8}{mj}和\end{CJK} $\mathrm{T}(\mathrm{AB})=\mathrm{T}(\mathrm{B}) \mathrm{T}(\mathrm{A})$ \begin{CJK}{UTF8}{mj}至少有一个成立\end{CJK}. \begin{CJK}{UTF8}{mj}证明\end{CJK}:\begin{CJK}{UTF8}{mj}或者对所有的\end{CJK} $A, B \in P^{n \times n}$ \begin{CJK}{UTF8}{mj}成立\end{CJK} $\mathrm{T}(\mathrm{AB})=$ $\mathrm{T}(\mathrm{A}) \mathrm{T}(\mathrm{B})$ \begin{CJK}{UTF8}{mj}或者对所有的\end{CJK} $A, B \in P^{n \times n}$ \begin{CJK}{UTF8}{mj}成立\end{CJK} $\mathrm{T}(\mathrm{AB})=\mathrm{T}(\mathrm{B}) \mathrm{T}(\mathrm{A})$.

\end{enumerate}
\section{4. 南开大学 2010 年研究生入学考试试题高等代数 
 李扬 
 微信公众号: sxkyliyang}
\begin{enumerate}
  \item (20 \begin{CJK}{UTF8}{mj}分\end{CJK}) \begin{CJK}{UTF8}{mj}计算下列行列式的值\end{CJK}:
\end{enumerate}
$$
\left|\begin{array}{cccc}
1+x_{1} y_{1} & x_{1} y_{2} & \cdots & x_{1} y_{n} \\
x_{2} y_{1} & 1+x_{2} y_{2} & \cdots & x_{2} y_{n} \\
\vdots & \vdots & & \vdots \\
x_{n} y_{1} & x_{n} y_{2} & \cdots & 1+x_{n} y_{n}
\end{array}\right|
$$

\begin{enumerate}
  \setcounter{enumi}{2}
  \item (20 \begin{CJK}{UTF8}{mj}分\end{CJK}) \begin{CJK}{UTF8}{mj}求解矩阵方程\end{CJK}
\end{enumerate}
$$
X\left(\begin{array}{ccc}
1 & 1 & -2 \\
0 & 1 & 3 \\
1 & 0 & 0
\end{array}\right)=\left(\begin{array}{ccc}
1 & -1 & 1 \\
0 & 1 & 0 \\
1 & 0 & 2
\end{array}\right)
$$

\begin{enumerate}
  \setcounter{enumi}{3}
  \item ( 20 \begin{CJK}{UTF8}{mj}分\end{CJK}) \begin{CJK}{UTF8}{mj}设矩阵\end{CJK}
\end{enumerate}
$$
A=\left(\begin{array}{ccc}
1 & 4 & 2 \\
0 & -3 & 4 \\
0 & 4 & 3
\end{array}\right)
$$
\begin{CJK}{UTF8}{mj}试求\end{CJK} $A^{n}$.

\begin{enumerate}
  \setcounter{enumi}{4}
  \item ( 20 \begin{CJK}{UTF8}{mj}分\end{CJK}) \begin{CJK}{UTF8}{mj}设\end{CJK} $A$ \begin{CJK}{UTF8}{mj}为\end{CJK} $n$ \begin{CJK}{UTF8}{mj}阶方阵\end{CJK}, \begin{CJK}{UTF8}{mj}已知\end{CJK} $A$ \begin{CJK}{UTF8}{mj}的特征值全为实数\end{CJK}, \begin{CJK}{UTF8}{mj}且\end{CJK} $A A^{\prime}=A^{\prime} A$. \begin{CJK}{UTF8}{mj}证明\end{CJK}: $A$ \begin{CJK}{UTF8}{mj}必为对称矩阵\end{CJK}.

  \item (15 \begin{CJK}{UTF8}{mj}分\end{CJK}) \begin{CJK}{UTF8}{mj}设\end{CJK} $A$ \begin{CJK}{UTF8}{mj}为实对称正定矩阵\end{CJK}, \begin{CJK}{UTF8}{mj}证明\end{CJK}: $A$ \begin{CJK}{UTF8}{mj}中元素之最大者必位于\end{CJK} $A$ \begin{CJK}{UTF8}{mj}的对角线上\end{CJK}.

  \item ( 15 \begin{CJK}{UTF8}{mj}分\end{CJK}) \begin{CJK}{UTF8}{mj}证明\end{CJK}: \begin{CJK}{UTF8}{mj}如果一个球面的球心坐标\end{CJK} $\left(x_{0}, y_{0}, z_{0}\right)$ \begin{CJK}{UTF8}{mj}中至少有一个是无理数\end{CJK}, \begin{CJK}{UTF8}{mj}则此球面上任何四个不在\end{CJK} \begin{CJK}{UTF8}{mj}同一平面上的点中至多有三个点使其坐标都是有理数\end{CJK}.

  \item (15 \begin{CJK}{UTF8}{mj}分\end{CJK}) \begin{CJK}{UTF8}{mj}设\end{CJK} $A, B$ \begin{CJK}{UTF8}{mj}为\end{CJK} $n$ \begin{CJK}{UTF8}{mj}阶复方阵\end{CJK}, \begin{CJK}{UTF8}{mj}且\end{CJK} $A$ \begin{CJK}{UTF8}{mj}可逆\end{CJK}, $B$ \begin{CJK}{UTF8}{mj}幂零\end{CJK}, \begin{CJK}{UTF8}{mj}且\end{CJK} $A B=B A$. \begin{CJK}{UTF8}{mj}证明\end{CJK}: $A+B$ \begin{CJK}{UTF8}{mj}为可逆矩阵\end{CJK}.

  \item ( 15 \begin{CJK}{UTF8}{mj}分\end{CJK}) \begin{CJK}{UTF8}{mj}设\end{CJK} $A$ \begin{CJK}{UTF8}{mj}为\end{CJK} $n$ \begin{CJK}{UTF8}{mj}阶实反对称矩阵\end{CJK}, \begin{CJK}{UTF8}{mj}证明\end{CJK}:

\end{enumerate}
(1) $\operatorname{det} A \geq 0$.

(2) \begin{CJK}{UTF8}{mj}如果\end{CJK} $A$ \begin{CJK}{UTF8}{mj}中元素全为整数\end{CJK}, \begin{CJK}{UTF8}{mj}则\end{CJK} $\operatorname{det} A$ \begin{CJK}{UTF8}{mj}必为某个整数的平方\end{CJK}.

\begin{enumerate}
  \setcounter{enumi}{9}
  \item ( 10 \begin{CJK}{UTF8}{mj}分\end{CJK}) \begin{CJK}{UTF8}{mj}设\end{CJK} $V$ \begin{CJK}{UTF8}{mj}为\end{CJK} $n$ \begin{CJK}{UTF8}{mj}维复线性空间\end{CJK}, End $V$ \begin{CJK}{UTF8}{mj}为\end{CJK} $V$ \begin{CJK}{UTF8}{mj}上所有线性变换构成的线性空间\end{CJK}, \begin{CJK}{UTF8}{mj}又\end{CJK} $A, B$ \begin{CJK}{UTF8}{mj}为\end{CJK} End $V$ \begin{CJK}{UTF8}{mj}的\end{CJK} \begin{CJK}{UTF8}{mj}子空间\end{CJK}, \begin{CJK}{UTF8}{mj}且\end{CJK} $A \subseteq B$. \begin{CJK}{UTF8}{mj}令\end{CJK}:
\end{enumerate}
$$
M=\{x \in \text { End } V \mid x y-y x \in A, \forall y \in B\}
$$
\begin{CJK}{UTF8}{mj}假定\end{CJK} $x_{0} \in M$ \begin{CJK}{UTF8}{mj}满足条件\end{CJK} $\operatorname{tr}\left(x_{0} y\right)=0, \forall y \in M$. \begin{CJK}{UTF8}{mj}证明\end{CJK}: $x_{0}$ \begin{CJK}{UTF8}{mj}必为幂零线性变换\end{CJK}.

\section{5. 南开大学 2011 年研究生入学考试试题高等代数 
 李扬 
 微信公众号: sxkyliyang}
\begin{enumerate}
  \item (20 \begin{CJK}{UTF8}{mj}分\end{CJK}) \begin{CJK}{UTF8}{mj}设\end{CJK} $A$ \begin{CJK}{UTF8}{mj}为秩为\end{CJK} 1 \begin{CJK}{UTF8}{mj}的\end{CJK} $n$ \begin{CJK}{UTF8}{mj}阶复方阵\end{CJK}, $A$ \begin{CJK}{UTF8}{mj}的迹\end{CJK} $\operatorname{tr}(A)=a \neq 0$, \begin{CJK}{UTF8}{mj}试求出\end{CJK} $A$ \begin{CJK}{UTF8}{mj}的所有特征值\end{CJK}(\begin{CJK}{UTF8}{mj}写出重数\end{CJK}).

  \item ( 20 \begin{CJK}{UTF8}{mj}分\end{CJK}) \begin{CJK}{UTF8}{mj}设\end{CJK} $V$ \begin{CJK}{UTF8}{mj}为\end{CJK} 4 \begin{CJK}{UTF8}{mj}维实线性空间\end{CJK}, $\varepsilon_{1}, \varepsilon_{2}, \varepsilon_{3}, \varepsilon_{4}$ \begin{CJK}{UTF8}{mj}为一组基\end{CJK}, \begin{CJK}{UTF8}{mj}已知\end{CJK} $V$ \begin{CJK}{UTF8}{mj}上线性变换\end{CJK} $T$ \begin{CJK}{UTF8}{mj}在基\end{CJK} $\varepsilon_{1}, \varepsilon_{2}, \varepsilon_{3}, \varepsilon_{4}$ \begin{CJK}{UTF8}{mj}下\end{CJK} \begin{CJK}{UTF8}{mj}的矩阵为\end{CJK}

\end{enumerate}
$$
\left(\begin{array}{cccc}
0 & 0 & -1 & -1 \\
0 & 1 & 2 & 2 \\
0 & -1 & -1 & 0 \\
0 & 0 & 0 & 1
\end{array}\right)
$$
(1) \begin{CJK}{UTF8}{mj}试求出\end{CJK} $T$ \begin{CJK}{UTF8}{mj}的特征值与特征向量\end{CJK}.

(2) \begin{CJK}{UTF8}{mj}试分别求出\end{CJK} $T$ \begin{CJK}{UTF8}{mj}的核\end{CJK} $\operatorname{ker} T$ \begin{CJK}{UTF8}{mj}与象\end{CJK} $\operatorname{Im} T$ \begin{CJK}{UTF8}{mj}的维数与一组基\end{CJK}.

\begin{enumerate}
  \setcounter{enumi}{3}
  \item ( 20 \begin{CJK}{UTF8}{mj}分\end{CJK}) \begin{CJK}{UTF8}{mj}设实矩阵\end{CJK}
\end{enumerate}
$$
A=\left(\begin{array}{llll}
1 & 2 & 1 & 1 \\
1 & 0 & 0 & 1 \\
0 & 1 & 1 & 1 \\
0 & 0 & 0 & 1
\end{array}\right)
$$
\begin{CJK}{UTF8}{mj}试将\end{CJK} $A$ \begin{CJK}{UTF8}{mj}写成一个正交矩阵\end{CJK} $Q$ \begin{CJK}{UTF8}{mj}与一个上三角矩阵\end{CJK} $T$ \begin{CJK}{UTF8}{mj}的乘积\end{CJK}.

\begin{enumerate}
  \setcounter{enumi}{4}
  \item ( 20 \begin{CJK}{UTF8}{mj}分\end{CJK}) \begin{CJK}{UTF8}{mj}设\end{CJK} $A$ \begin{CJK}{UTF8}{mj}为实反对称矩阵\end{CJK}, \begin{CJK}{UTF8}{mj}证明\end{CJK}: $E-A^{10}$ \begin{CJK}{UTF8}{mj}一定是正定矩阵\end{CJK}.

  \item ( 15 \begin{CJK}{UTF8}{mj}分\end{CJK}) \begin{CJK}{UTF8}{mj}设\end{CJK} $V$ \begin{CJK}{UTF8}{mj}为一个欧氏空间\end{CJK}. $T$ \begin{CJK}{UTF8}{mj}为\end{CJK} $V$ \begin{CJK}{UTF8}{mj}到\end{CJK} $V$ \begin{CJK}{UTF8}{mj}的一个映射\end{CJK}. \begin{CJK}{UTF8}{mj}满足条件\end{CJK}: $|T \alpha|=|\alpha|, \forall \alpha \in V$. \begin{CJK}{UTF8}{mj}试问\end{CJK} $T$ \begin{CJK}{UTF8}{mj}是\end{CJK} \begin{CJK}{UTF8}{mj}否一定是\end{CJK} $V$ \begin{CJK}{UTF8}{mj}上的正交变换\end{CJK}? \begin{CJK}{UTF8}{mj}说明理由\end{CJK}.

  \item ( 15 \begin{CJK}{UTF8}{mj}分\end{CJK}) \begin{CJK}{UTF8}{mj}设\end{CJK} $A, B$ \begin{CJK}{UTF8}{mj}为数域\end{CJK} $P$ \begin{CJK}{UTF8}{mj}上的\end{CJK} $n$ \begin{CJK}{UTF8}{mj}阶方阵\end{CJK}, \begin{CJK}{UTF8}{mj}满足方程\end{CJK} $a A^{2}+b A B+c B=0$. \begin{CJK}{UTF8}{mj}其中\end{CJK} $a, b, c$ \begin{CJK}{UTF8}{mj}为非零常数\end{CJK}, \begin{CJK}{UTF8}{mj}证\end{CJK} \begin{CJK}{UTF8}{mj}明\end{CJK}: $c E+b A$ \begin{CJK}{UTF8}{mj}为可逆矩阵\end{CJK}.

  \item ( 15 \begin{CJK}{UTF8}{mj}分\end{CJK}) \begin{CJK}{UTF8}{mj}设\end{CJK} $A, B$ \begin{CJK}{UTF8}{mj}为数域\end{CJK} $P$ \begin{CJK}{UTF8}{mj}上的\end{CJK} $n$ \begin{CJK}{UTF8}{mj}阶方阵\end{CJK}, \begin{CJK}{UTF8}{mj}且\end{CJK} $r(A)=r(A B)$, \begin{CJK}{UTF8}{mj}证明\end{CJK}: \begin{CJK}{UTF8}{mj}对任何自然数\end{CJK} $l$, \begin{CJK}{UTF8}{mj}有\end{CJK} $r\left(A^{l}\right)=r\left(B A^{l}\right)$.

  \item ( 15 \begin{CJK}{UTF8}{mj}分\end{CJK}) \begin{CJK}{UTF8}{mj}设\end{CJK} $V$ \begin{CJK}{UTF8}{mj}为复数域上的\end{CJK} $4 n$ \begin{CJK}{UTF8}{mj}维线性空间\end{CJK}, \begin{CJK}{UTF8}{mj}证明\end{CJK}: \begin{CJK}{UTF8}{mj}存在\end{CJK} $V$ \begin{CJK}{UTF8}{mj}上的线性变换\end{CJK} $T$ \begin{CJK}{UTF8}{mj}使得\end{CJK} $T^{4}=-\mathrm{id}$, \begin{CJK}{UTF8}{mj}其中\end{CJK} id \begin{CJK}{UTF8}{mj}为\end{CJK} \begin{CJK}{UTF8}{mj}恒等变换\end{CJK}, \begin{CJK}{UTF8}{mj}并且满足上述条件的线性变换必然在某组基下的矩阵为对角矩阵\end{CJK}.

  \item ( 10 \begin{CJK}{UTF8}{mj}分\end{CJK}) \begin{CJK}{UTF8}{mj}数域\end{CJK} $P$ \begin{CJK}{UTF8}{mj}上一个\end{CJK} $n$ \begin{CJK}{UTF8}{mj}阶方阵\end{CJK} $A$ \begin{CJK}{UTF8}{mj}称为幂零的\end{CJK}, \begin{CJK}{UTF8}{mj}如果存在自然数\end{CJK} $m$ \begin{CJK}{UTF8}{mj}使得\end{CJK} $A^{m}=0$. \begin{CJK}{UTF8}{mj}设\end{CJK} $A=\left(a_{i j}\right)_{n \times n}$ \begin{CJK}{UTF8}{mj}为\end{CJK} \begin{CJK}{UTF8}{mj}一个幂零方阵\end{CJK}, \begin{CJK}{UTF8}{mj}且\end{CJK} $a_{12} \neq 0, a_{13}=0, a_{22}=0, a_{23} \neq 0$. \begin{CJK}{UTF8}{mj}证明\end{CJK}: \begin{CJK}{UTF8}{mj}不存在矩阵\end{CJK} $B$ \begin{CJK}{UTF8}{mj}使得\end{CJK} $B^{n-1}=A$.

\end{enumerate}
\section{6. 南开大学 2012 年研究生入学考试试题高等代数 
 李扬 
 微信公众号: sxkyliyang}
\begin{enumerate}
  \item (20 \begin{CJK}{UTF8}{mj}分\end{CJK}) \begin{CJK}{UTF8}{mj}设\end{CJK} $P$ \begin{CJK}{UTF8}{mj}为数域\end{CJK}, \begin{CJK}{UTF8}{mj}定义\end{CJK} $n(n \geq 3)$ \begin{CJK}{UTF8}{mj}级方阵\end{CJK} $A$ \begin{CJK}{UTF8}{mj}为\end{CJK}
\end{enumerate}
$$
A=\left(\begin{array}{cccccc}
1 & 0 & 0 & \cdots & 0 & n \\
0 & 1 & 0 & \cdots & 0 & 0 \\
0 & 0 & 1 & \cdots & 0 & 0 \\
\vdots & \vdots & \vdots & & \vdots & \vdots \\
0 & 0 & 0 & \cdots & 1 & 0 \\
0 & 0 & 0 & \cdots & 0 & 1
\end{array}\right) .
$$
(1) \begin{CJK}{UTF8}{mj}设\end{CJK} $P^{n \times n}$ \begin{CJK}{UTF8}{mj}中全体与\end{CJK} $A$ \begin{CJK}{UTF8}{mj}可交换的矩阵所组成的集合为\end{CJK} $C(A)$, \begin{CJK}{UTF8}{mj}证明\end{CJK} $C(A)$ \begin{CJK}{UTF8}{mj}是\end{CJK} $P^{n \times n}$ \begin{CJK}{UTF8}{mj}的一个子空间\end{CJK}.

(2) \begin{CJK}{UTF8}{mj}试求出\end{CJK} $C(A)$ \begin{CJK}{UTF8}{mj}的维数与一组基\end{CJK}.

\begin{enumerate}
  \setcounter{enumi}{2}
  \item ( 20 \begin{CJK}{UTF8}{mj}分\end{CJK})\begin{CJK}{UTF8}{mj}设\end{CJK} $A$ \begin{CJK}{UTF8}{mj}为数域\end{CJK} $P$ \begin{CJK}{UTF8}{mj}上的\end{CJK} $n$ \begin{CJK}{UTF8}{mj}级方阵\end{CJK}, \begin{CJK}{UTF8}{mj}且有\end{CJK} $n$ \begin{CJK}{UTF8}{mj}个互异的特征值\end{CJK} $\lambda_{1}, \lambda_{2}, \cdots, \lambda_{n}$, \begin{CJK}{UTF8}{mj}定义\end{CJK} $P^{n \times n}$ \begin{CJK}{UTF8}{mj}上的线性变\end{CJK} \begin{CJK}{UTF8}{mj}换\end{CJK} $T$ \begin{CJK}{UTF8}{mj}为\end{CJK} $T(X)=A X, X \in P^{n \times 1}$, \begin{CJK}{UTF8}{mj}试求出\end{CJK} $T$ \begin{CJK}{UTF8}{mj}的所有特征值及其重数\end{CJK}.

  \item ( 20 \begin{CJK}{UTF8}{mj}分\end{CJK})\begin{CJK}{UTF8}{mj}设\end{CJK} $n$ \begin{CJK}{UTF8}{mj}级行列式\end{CJK} $d=\operatorname{det}\left(\left(a_{i j}\right)_{n \times n}\right) \neq 0, A_{i j}$ \begin{CJK}{UTF8}{mj}为\end{CJK} $a_{i j}$ \begin{CJK}{UTF8}{mj}在\end{CJK} $d$ \begin{CJK}{UTF8}{mj}中的代数余子式\end{CJK}, \begin{CJK}{UTF8}{mj}试求行列式\end{CJK}

\end{enumerate}
$$
\left|\begin{array}{cccc}
A_{11} & A_{12} & \cdots & A_{1, n-1} \\
A_{21} & A_{22} & \cdots & A_{2, n-1} \\
\vdots & \vdots & & \vdots \\
A_{n-1,1} & A_{n-1,2} & \cdots & A_{n-1 . n-1}
\end{array}\right|
$$
\begin{CJK}{UTF8}{mj}的值\end{CJK}.

\begin{enumerate}
  \setcounter{enumi}{4}
  \item ( 20 \begin{CJK}{UTF8}{mj}分\end{CJK})\begin{CJK}{UTF8}{mj}已知\end{CJK} $a_{1}, a_{2}, a_{3}, a_{4} \in \mathbb{C}$, \begin{CJK}{UTF8}{mj}试求出线性方程组\end{CJK}
\end{enumerate}
$$
\left\{\begin{array}{l}
x_{1}+a_{1} x_{2}+a_{1}^{2} x_{3}=a_{1}^{3} \\
x_{1}+a_{2} x_{2}+a_{2}^{2} x_{3}=a_{2}^{3} \\
x_{1}+a_{3} x_{2}+a_{3}^{2} x_{3}=a_{3}^{3} \\
x_{1}+a_{4} x_{2}+a_{4}^{2} x_{3}=a_{4}^{3}
\end{array}\right.
$$
\begin{CJK}{UTF8}{mj}存在解的充要条件\end{CJK}, \begin{CJK}{UTF8}{mj}并在有解时求出其通解\end{CJK}.

\begin{enumerate}
  \setcounter{enumi}{5}
  \item ( 15 \begin{CJK}{UTF8}{mj}分\end{CJK})\begin{CJK}{UTF8}{mj}已知向量组\end{CJK} $\alpha_{1}, \alpha_{2}, \cdots, \alpha_{m}$ \begin{CJK}{UTF8}{mj}线性无关\end{CJK}, \begin{CJK}{UTF8}{mj}而\end{CJK} $\alpha_{2}, \alpha_{3}, \cdots, \alpha_{m+1}$ \begin{CJK}{UTF8}{mj}线性相关\end{CJK}, \begin{CJK}{UTF8}{mj}证明\end{CJK} $\alpha_{1}$ \begin{CJK}{UTF8}{mj}不能由\end{CJK} $\alpha_{2}, \alpha_{3}, \cdots, \alpha_{m+1}$ \begin{CJK}{UTF8}{mj}线性表出\end{CJK}.

  \item \begin{CJK}{UTF8}{mj}判断下列论断是否正确\end{CJK}, \begin{CJK}{UTF8}{mj}并说明理由\end{CJK}.

\end{enumerate}
\begin{CJK}{UTF8}{mj}设\end{CJK} $\mathscr{A}, \mathscr{B}$ \begin{CJK}{UTF8}{mj}为\end{CJK} $n$ \begin{CJK}{UTF8}{mj}维实线性空间\end{CJK} $V$ \begin{CJK}{UTF8}{mj}上的两个线性变换\end{CJK}, \begin{CJK}{UTF8}{mj}且\end{CJK} $\mathscr{A} \mathscr{B}=\mathscr{B} \mathscr{A}$, \begin{CJK}{UTF8}{mj}又已知\end{CJK} $\mathscr{A}, \mathscr{B}$ \begin{CJK}{UTF8}{mj}都存在特征向量\end{CJK}, \begin{CJK}{UTF8}{mj}则\end{CJK} $\mathscr{A}, \mathscr{B}$ \begin{CJK}{UTF8}{mj}必存在公共的特征向量\end{CJK}.

\begin{enumerate}
  \setcounter{enumi}{7}
  \item ( 15 \begin{CJK}{UTF8}{mj}分\end{CJK})\begin{CJK}{UTF8}{mj}设\end{CJK} $A$ \begin{CJK}{UTF8}{mj}为\end{CJK} $n$ \begin{CJK}{UTF8}{mj}维欧氏空间\end{CJK} $V$ \begin{CJK}{UTF8}{mj}上的对称变换\end{CJK}, \begin{CJK}{UTF8}{mj}证明对\end{CJK} $V$ \begin{CJK}{UTF8}{mj}中任意的单位向量\end{CJK} $x$, \begin{CJK}{UTF8}{mj}都有\end{CJK} $|A x|^{2} \leq\left|A^{2} x\right|$.

  \item (15 \begin{CJK}{UTF8}{mj}分\end{CJK})\begin{CJK}{UTF8}{mj}证明实方阵\end{CJK} $A$ \begin{CJK}{UTF8}{mj}为反对称矩阵的充要条件是\end{CJK} $A A^{\prime}=-A^{2}$.

  \item (10 \begin{CJK}{UTF8}{mj}分\end{CJK})\begin{CJK}{UTF8}{mj}证明对任意的\end{CJK} 2 \begin{CJK}{UTF8}{mj}级复方阵\end{CJK} $A, B, C$ \begin{CJK}{UTF8}{mj}都有\end{CJK}

\end{enumerate}
$$
A(B C-C B)^{2}-(B C-C B)^{2} A=0
$$

\section{7. 南开大学 2014 年研究生入学考试试题高等代数}
\begin{CJK}{UTF8}{mj}李扬\end{CJK}

\begin{CJK}{UTF8}{mj}微信公众号\end{CJK}: sxkyliyang

\begin{enumerate}
  \item \begin{CJK}{UTF8}{mj}设\end{CJK} $n$ \begin{CJK}{UTF8}{mj}阶行列式\end{CJK}
\end{enumerate}
$$
\operatorname{det}\left(\begin{array}{cccc}
a_{11} & a_{12} & \cdots & a_{1 n} \\
a_{21} & a_{22} & \cdots & a_{2 n} \\
\vdots & \vdots & & \vdots \\
a_{n 1} & a_{n 2} & \cdots & a_{n n}
\end{array}\right)=1 .
$$
\begin{CJK}{UTF8}{mj}且满足\end{CJK} $a_{i j}=-a_{j i}, i, j=1,2, \cdots, n$. \begin{CJK}{UTF8}{mj}对任意\end{CJK} $x$, \begin{CJK}{UTF8}{mj}求\end{CJK} $n$ \begin{CJK}{UTF8}{mj}阶行列式\end{CJK}
$$
\operatorname{det}\left(\begin{array}{cccc}
a_{11}+x & a_{12}+x & \cdots & a_{1 n}+x \\
a_{21}+x & a_{22}+x & \cdots & a_{2 n}+x \\
\vdots & \vdots & & \vdots \\
a_{n 1}+x & a_{n 2}+x & \cdots & a_{n n}+x
\end{array}\right)
$$

\begin{enumerate}
  \setcounter{enumi}{2}
  \item \begin{CJK}{UTF8}{mj}在\end{CJK} $P^{4}$ \begin{CJK}{UTF8}{mj}中\end{CJK}, \begin{CJK}{UTF8}{mj}已知\end{CJK} $V_{1}=L\left(\alpha_{1}, \alpha_{2}, \alpha_{3}\right), V_{2}=L\left(\beta_{1}, \beta_{2}\right)$. \begin{CJK}{UTF8}{mj}其中\end{CJK}
\end{enumerate}
$$
\alpha_{1}=(1,1,-1,2)^{\prime}, \alpha_{2}=(2,-1,3,0)^{\prime}, \alpha_{3}=(0,-3,5,-4)^{\prime}, \beta_{1}=(1,2,2,1)^{\prime}, \beta_{2}=(4,-3,3,1)^{\prime}
$$
\begin{CJK}{UTF8}{mj}求\end{CJK} $V_{1}+V_{2}$ \begin{CJK}{UTF8}{mj}和\end{CJK} $V_{1} \cap V_{2}$ \begin{CJK}{UTF8}{mj}的维数与一组基\end{CJK}.

\begin{enumerate}
  \setcounter{enumi}{3}
  \item \begin{CJK}{UTF8}{mj}已知矩阵\end{CJK}
\end{enumerate}
$$
A=\left(\begin{array}{ccc}
1 & 1 & 0 \\
0 & 0 & 1 \\
0 & -1 & 0
\end{array}\right)
$$
(1) \begin{CJK}{UTF8}{mj}证明\end{CJK} $A^{2014}=-A^{2012}+A^{2}+E$.

(2) \begin{CJK}{UTF8}{mj}求\end{CJK} $A^{2014}$.

\begin{enumerate}
  \setcounter{enumi}{4}
  \item \begin{CJK}{UTF8}{mj}已知矩阵\end{CJK}
\end{enumerate}
$$
A=\left(\begin{array}{ccc}
0 & 0 & 0 \\
-1 & x & -4 \\
1 & -5 & 5
\end{array}\right)
$$
\begin{CJK}{UTF8}{mj}与矩阵\end{CJK}
$$
B=\left(\begin{array}{ccc}
1 & 0 & 0 \\
0 & y & 0 \\
0 & 0 & 10
\end{array}\right)
$$
\begin{CJK}{UTF8}{mj}相似\end{CJK}.

(1) \begin{CJK}{UTF8}{mj}求\end{CJK} $x, y$.

(2) \begin{CJK}{UTF8}{mj}求可逆矩阵\end{CJK} $T$ \begin{CJK}{UTF8}{mj}使\end{CJK} $T^{-1} A T=B$.

\begin{enumerate}
  \setcounter{enumi}{5}
  \item \begin{CJK}{UTF8}{mj}设\end{CJK} $A$ \begin{CJK}{UTF8}{mj}为\end{CJK} $s \times n$ \begin{CJK}{UTF8}{mj}矩阵\end{CJK}. \begin{CJK}{UTF8}{mj}证明\end{CJK}: $s-r\left(E_{s}-A A^{\prime}\right)=n-r\left(E_{n}-A A^{\prime}\right)$.

  \item \begin{CJK}{UTF8}{mj}设\end{CJK} $A$ \begin{CJK}{UTF8}{mj}为对称矩阵\end{CJK}, \begin{CJK}{UTF8}{mj}存在线性无关的向量\end{CJK} $X_{1}, X_{2}$ \begin{CJK}{UTF8}{mj}使\end{CJK} $X_{1}^{\prime} A X_{1}>0, X_{2}^{\prime} A X_{2}<0$, \begin{CJK}{UTF8}{mj}证明\end{CJK}: \begin{CJK}{UTF8}{mj}存在线性无关的向\end{CJK} \begin{CJK}{UTF8}{mj}量\end{CJK} $X_{3}, X_{4}$ \begin{CJK}{UTF8}{mj}使\end{CJK} $X_{1}, X_{2}, X_{3}, X_{4}$ \begin{CJK}{UTF8}{mj}线性相关\end{CJK}, \begin{CJK}{UTF8}{mj}且\end{CJK} $X_{3}^{\prime} A X_{3}=X_{4}^{\prime} A X_{4}=0$. 7. \begin{CJK}{UTF8}{mj}设\end{CJK} $\sigma, \tau$ \begin{CJK}{UTF8}{mj}为线性变换且\end{CJK} $\sigma$ \begin{CJK}{UTF8}{mj}有\end{CJK} $n$ \begin{CJK}{UTF8}{mj}个不同的特征值\end{CJK}. \begin{CJK}{UTF8}{mj}证明\end{CJK}:\begin{CJK}{UTF8}{mj}若\end{CJK} $\sigma \tau=\tau \sigma$, \begin{CJK}{UTF8}{mj}则\end{CJK} $\tau$ \begin{CJK}{UTF8}{mj}可由\end{CJK} $I, \sigma, \sigma^{2}, \cdots, \sigma^{n-1}$ \begin{CJK}{UTF8}{mj}线性\end{CJK} \begin{CJK}{UTF8}{mj}表出\end{CJK}, \begin{CJK}{UTF8}{mj}其中\end{CJK} $I$ \begin{CJK}{UTF8}{mj}为恒等变换\end{CJK}.

  \item \begin{CJK}{UTF8}{mj}设\end{CJK} $f(x)$ \begin{CJK}{UTF8}{mj}是\end{CJK} $n$ \begin{CJK}{UTF8}{mj}级矩阵\end{CJK} $A$ \begin{CJK}{UTF8}{mj}的特征多项式\end{CJK}. \begin{CJK}{UTF8}{mj}存在互素且次数分别为\end{CJK} $p, q$ \begin{CJK}{UTF8}{mj}的多项式\end{CJK} $g(x), h(x)$, \begin{CJK}{UTF8}{mj}满足\end{CJK} $f(x)=$ $g(x) h(x)$. \begin{CJK}{UTF8}{mj}求证\end{CJK}: $r(g(A))=q, r(h(A))=p$.

  \item $A, B$ \begin{CJK}{UTF8}{mj}都是\end{CJK} $n$ \begin{CJK}{UTF8}{mj}级反对称矩阵\end{CJK}, \begin{CJK}{UTF8}{mj}且\end{CJK} $A$ \begin{CJK}{UTF8}{mj}可逆\end{CJK}. \begin{CJK}{UTF8}{mj}求证\end{CJK}: $\left|A^{2}-B\right|>0$.

\end{enumerate}
\section{8. 南开大学 2016 年研究生入学考试试题高等代数 
 李扬 
 微信公众号: sxkyliyang}
\begin{enumerate}
  \item ( 15 \begin{CJK}{UTF8}{mj}分\end{CJK}) \begin{CJK}{UTF8}{mj}已知\end{CJK} $n$ \begin{CJK}{UTF8}{mj}阶行列式\end{CJK}
\end{enumerate}
$$
A=\left|\begin{array}{ccccc}
1 & 2 & 3 & \cdots & n \\
1 & 2 & 0 & \cdots & 0 \\
1 & 0 & 3 & \cdots & 0 \\
\vdots & \vdots & \vdots & & \vdots \\
1 & 0 & 0 & \cdots & n
\end{array}\right| .
$$
\begin{CJK}{UTF8}{mj}求\end{CJK} $A_{11}+A_{12}+\cdots+A_{1 n}$ \begin{CJK}{UTF8}{mj}的值\end{CJK}.

\begin{enumerate}
  \setcounter{enumi}{2}
  \item (20 \begin{CJK}{UTF8}{mj}分\end{CJK}) 3 \begin{CJK}{UTF8}{mj}阶矩阵\end{CJK}
\end{enumerate}
$$
A=\left(\begin{array}{ccc}
1 & 0 & 1 \\
0 & -1 & 0 \\
-1 & 1 & -1
\end{array}\right)
$$
\begin{CJK}{UTF8}{mj}定义\end{CJK} $P^{3 \times 3} \rightarrow P^{3 \times 3}$ \begin{CJK}{UTF8}{mj}的线性变换\end{CJK} $\sigma: \sigma(X)=A X$, \begin{CJK}{UTF8}{mj}其中\end{CJK} $X \in P^{3 \times 3}$. \begin{CJK}{UTF8}{mj}分别求\end{CJK} $k e r ~ \sigma$ \begin{CJK}{UTF8}{mj}与\end{CJK} $\operatorname{Im} \sigma$ \begin{CJK}{UTF8}{mj}的维数与一组\end{CJK} \begin{CJK}{UTF8}{mj}基\end{CJK}.

\begin{enumerate}
  \setcounter{enumi}{3}
  \item ( 20 \begin{CJK}{UTF8}{mj}分\end{CJK}) \begin{CJK}{UTF8}{mj}已知\end{CJK} $V$ \begin{CJK}{UTF8}{mj}是一个\end{CJK} 3 \begin{CJK}{UTF8}{mj}维线性空间\end{CJK}, \begin{CJK}{UTF8}{mj}线性变换\end{CJK} $\mathscr{A}$ \begin{CJK}{UTF8}{mj}在一组基\end{CJK} $\varepsilon_{1}, \varepsilon_{2}, \varepsilon_{3}$ \begin{CJK}{UTF8}{mj}下的矩阵为\end{CJK} $A=\left(\begin{array}{rrrrr}1 & 0 & -2 \\ 0 & 1 & -2 \\ 0 & -1\end{array}\right)$, \begin{CJK}{UTF8}{mj}在另一组基\end{CJK} $\eta_{1}, \eta_{2}, \eta_{3}$ \begin{CJK}{UTF8}{mj}下的矩阵为\end{CJK} $B=\left(\begin{array}{ccc}1 & 2 & -2 \\ 0 & 5 & -4 \\ 0 & 6 & -5\end{array}\right)$, \begin{CJK}{UTF8}{mj}求基\end{CJK} $\varepsilon_{1}, \varepsilon_{2}, \varepsilon_{3}$ \begin{CJK}{UTF8}{mj}到基\end{CJK} $\eta_{1}, \eta_{2}, \eta_{3}$ \begin{CJK}{UTF8}{mj}的过渡矩阵\end{CJK} $P$.

  \item ( 10 \begin{CJK}{UTF8}{mj}分\end{CJK}) \begin{CJK}{UTF8}{mj}设\end{CJK} $P$ \begin{CJK}{UTF8}{mj}为数域\end{CJK}, $A \in P^{m \times s}, r(A)=r$, \begin{CJK}{UTF8}{mj}证明\end{CJK}: \begin{CJK}{UTF8}{mj}对任意的正整数\end{CJK} $n$, \begin{CJK}{UTF8}{mj}存在秩为\end{CJK} $\min \{s-r, n\}$ \begin{CJK}{UTF8}{mj}的\end{CJK} $s \times n$ \begin{CJK}{UTF8}{mj}矩阵\end{CJK} $B$ \begin{CJK}{UTF8}{mj}使得对任意的\end{CJK} $C \in P^{n \times n}$, \begin{CJK}{UTF8}{mj}都有\end{CJK} $A B C=0$.

  \item (15 \begin{CJK}{UTF8}{mj}分\end{CJK}) \begin{CJK}{UTF8}{mj}设\end{CJK} $f\left(x_{1}, x_{2}, \cdots, x_{n}\right)=X^{\prime} A X$ \begin{CJK}{UTF8}{mj}与\end{CJK} $g\left(y_{1}, y_{2}, \cdots, y_{n}\right)=Y^{\prime} B Y$ \begin{CJK}{UTF8}{mj}都是实二次型\end{CJK}, \begin{CJK}{UTF8}{mj}且\end{CJK} $A B=B A$. \begin{CJK}{UTF8}{mj}证明\end{CJK}: \begin{CJK}{UTF8}{mj}存在非退化线性替换同时把\end{CJK} $f\left(x_{1}, x_{2}, \cdots, x_{n}\right)$ \begin{CJK}{UTF8}{mj}与\end{CJK} $g\left(y_{1}, y_{2}, \cdots, y_{n}\right)$ \begin{CJK}{UTF8}{mj}化为标准形\end{CJK}.

  \item ( 10 \begin{CJK}{UTF8}{mj}分\end{CJK}) $V_{1}, V_{2}$ \begin{CJK}{UTF8}{mj}为\end{CJK} $n$ \begin{CJK}{UTF8}{mj}维欧式空间\end{CJK} $V$ \begin{CJK}{UTF8}{mj}的线性子空间\end{CJK}, \begin{CJK}{UTF8}{mj}且\end{CJK} $V_{2}$ \begin{CJK}{UTF8}{mj}的维数大于\end{CJK} $V_{1}$ \begin{CJK}{UTF8}{mj}的维数\end{CJK}, \begin{CJK}{UTF8}{mj}证明\end{CJK}: $V_{2}$ \begin{CJK}{UTF8}{mj}中存在一个\end{CJK} \begin{CJK}{UTF8}{mj}向量与\end{CJK} $V_{1}$ \begin{CJK}{UTF8}{mj}中的任何一个向量都正交\end{CJK}.

  \item ( 15 \begin{CJK}{UTF8}{mj}分\end{CJK}) \begin{CJK}{UTF8}{mj}设\end{CJK} $P$ \begin{CJK}{UTF8}{mj}为数域\end{CJK}, $A=\left(a_{i j}\right)_{n \times n}$ \begin{CJK}{UTF8}{mj}为\end{CJK} $P^{n \times n}$ \begin{CJK}{UTF8}{mj}中的可逆矩阵\end{CJK}, \begin{CJK}{UTF8}{mj}已知\end{CJK} $c_{i} \in P, i=1,2, \cdots, n$, \begin{CJK}{UTF8}{mj}且记\end{CJK} $A^{-1}=B=\left(b_{i j}\right)_{n \times n}, d_{i}=\sum_{j=1}^{n} b_{i j} c_{j}, i=1,2, \cdots, n$, \begin{CJK}{UTF8}{mj}令\end{CJK} $D=\left(a_{i j}+c_{i} c_{j}\right)_{n \times n}$, \begin{CJK}{UTF8}{mj}证明\end{CJK}

\end{enumerate}
$$
\operatorname{det} D=\operatorname{det} A \cdot\left(1+\sum_{i=1}^{n} c_{i} d_{i}\right)
$$

\begin{enumerate}
  \setcounter{enumi}{8}
  \item ( 20 \begin{CJK}{UTF8}{mj}分\end{CJK}) $\mathscr{A}$ \begin{CJK}{UTF8}{mj}为\end{CJK} $n$ \begin{CJK}{UTF8}{mj}维线性空间\end{CJK} $V$ \begin{CJK}{UTF8}{mj}上的线性变换\end{CJK}, \begin{CJK}{UTF8}{mj}特征多项式为\end{CJK}
\end{enumerate}
$$
f(\lambda)=\left(\lambda-\lambda_{1}\right)^{r_{1}}\left(\lambda-\lambda_{2}\right)^{r_{2}} \cdots\left(\lambda-\lambda_{s}\right)^{r_{s}} .
$$
$\lambda_{1}, \lambda_{2}, \cdots, \lambda_{s}$ \begin{CJK}{UTF8}{mj}各不相同\end{CJK}, \begin{CJK}{UTF8}{mj}且\end{CJK} $r_{1}+r_{2}+\cdots+r_{s}=n$, \begin{CJK}{UTF8}{mj}令\end{CJK} $f_{i}(\lambda)=\frac{f(\lambda)}{\left(\lambda-\lambda_{i}\right)^{r_{i}}}$, \begin{CJK}{UTF8}{mj}证明\end{CJK}:
$$
V=V_{1} \oplus V_{2} \oplus \cdots \oplus V_{s} .
$$
\begin{CJK}{UTF8}{mj}其中\end{CJK} $V_{i}=\operatorname{Im} f_{i}(\mathscr{A})$. 9. ( 15 \begin{CJK}{UTF8}{mj}分\end{CJK}) \begin{CJK}{UTF8}{mj}设\end{CJK} $A$ \begin{CJK}{UTF8}{mj}是数域\end{CJK} $F$ \begin{CJK}{UTF8}{mj}上的\end{CJK} $n$ \begin{CJK}{UTF8}{mj}阶可逆矩阵\end{CJK}, \begin{CJK}{UTF8}{mj}把\end{CJK} $A$ \begin{CJK}{UTF8}{mj}与\end{CJK} $A^{-1}$ \begin{CJK}{UTF8}{mj}如下分块\end{CJK}:
$$
A=\left(\begin{array}{ll}
A_{11} & A_{12} \\
A_{21} & A_{22}
\end{array}\right), \quad A^{-1}=\left(\begin{array}{cc}
B_{11} & B_{12} \\
B_{21} & B_{22}
\end{array}\right)
$$
\begin{CJK}{UTF8}{mj}其中\end{CJK} $A_{11}$ \begin{CJK}{UTF8}{mj}是\end{CJK} $l \times k$ \begin{CJK}{UTF8}{mj}阶阵\end{CJK}, $B_{11}$ \begin{CJK}{UTF8}{mj}是\end{CJK} $k \times l$ \begin{CJK}{UTF8}{mj}阶阵\end{CJK}, $l, k$ \begin{CJK}{UTF8}{mj}为小于\end{CJK} $n$ \begin{CJK}{UTF8}{mj}的正整数\end{CJK}. \begin{CJK}{UTF8}{mj}用\end{CJK} $W$ \begin{CJK}{UTF8}{mj}表示\end{CJK} $A_{12} X=0$ \begin{CJK}{UTF8}{mj}的解空间\end{CJK}, $U$ \begin{CJK}{UTF8}{mj}表示\end{CJK} $B_{12} Y=0$ \begin{CJK}{UTF8}{mj}的解空间\end{CJK}, \begin{CJK}{UTF8}{mj}其中\end{CJK} $X, Y$ \begin{CJK}{UTF8}{mj}分别是\end{CJK} $(n-k) \times 1,(n-l) \times 1$ \begin{CJK}{UTF8}{mj}的列向量\end{CJK}, \begin{CJK}{UTF8}{mj}证明\end{CJK} $W \cong U$.

\begin{enumerate}
  \setcounter{enumi}{10}
  \item (10 \begin{CJK}{UTF8}{mj}分\end{CJK}) \begin{CJK}{UTF8}{mj}已知\end{CJK} $n$ \begin{CJK}{UTF8}{mj}阶矩阵\end{CJK} $A$ \begin{CJK}{UTF8}{mj}可逆\end{CJK}, $B$ \begin{CJK}{UTF8}{mj}幂零\end{CJK}, \begin{CJK}{UTF8}{mj}且\end{CJK} $A B=B A$, \begin{CJK}{UTF8}{mj}求证\end{CJK} $A+B$ \begin{CJK}{UTF8}{mj}可逆\end{CJK}.
\end{enumerate}
\section{9. 南开大学 2017 年研究生入学考试试题高等代数 
 李扬 
 微信公众号: sxkyliyang}
\begin{enumerate}
  \item ( 15 \begin{CJK}{UTF8}{mj}分\end{CJK}) \begin{CJK}{UTF8}{mj}计算\end{CJK} $n$ \begin{CJK}{UTF8}{mj}级行列式\end{CJK}
\end{enumerate}
$$
D_{n}=\left|\begin{array}{ccccc}
1 & 2 & 3 & \cdots & n \\
x & 1 & 2 & \cdots & n-1 \\
x & x & 1 & \cdots & n-2 \\
\vdots & \vdots & \vdots & & \vdots \\
x & x & x & \cdots & 1
\end{array}\right|
$$

\begin{enumerate}
  \setcounter{enumi}{2}
  \item ( 20 \begin{CJK}{UTF8}{mj}分\end{CJK}) \begin{CJK}{UTF8}{mj}设矩阵\end{CJK} $A=\left(\begin{array}{ccc}-1 & -2 & 6 \\ -1 & 0 & 3 \\ -1 & -1 & 4\end{array}\right)$, \begin{CJK}{UTF8}{mj}求\end{CJK} $A$ \begin{CJK}{UTF8}{mj}的若尔当标准形\end{CJK} $J$ \begin{CJK}{UTF8}{mj}及可逆矩阵\end{CJK} $T$, \begin{CJK}{UTF8}{mj}使得\end{CJK} $A=T J T^{-1}$.

  \item ( 20 \begin{CJK}{UTF8}{mj}分\end{CJK}) \begin{CJK}{UTF8}{mj}设\end{CJK} $A, B$ \begin{CJK}{UTF8}{mj}分别为\end{CJK} $3 \times 2$ \begin{CJK}{UTF8}{mj}和\end{CJK} $2 \times 3$ \begin{CJK}{UTF8}{mj}的实矩阵\end{CJK}, \begin{CJK}{UTF8}{mj}且\end{CJK} $A B$ \begin{CJK}{UTF8}{mj}和\end{CJK} $B A$ \begin{CJK}{UTF8}{mj}均为是对称矩阵\end{CJK}, \begin{CJK}{UTF8}{mj}已知\end{CJK}

\end{enumerate}
$$
A B=\left(\begin{array}{ccc}
8 & 2 & -2 \\
2 & 5 & 4 \\
-2 & 4 & 5
\end{array}\right)
$$
\begin{CJK}{UTF8}{mj}证明\end{CJK} $A, B$ \begin{CJK}{UTF8}{mj}的秩均为\end{CJK} 2 , \begin{CJK}{UTF8}{mj}并求\end{CJK} $B A$.

\begin{enumerate}
  \setcounter{enumi}{4}
  \item ( 15 \begin{CJK}{UTF8}{mj}分\end{CJK}) \begin{CJK}{UTF8}{mj}设\end{CJK} $a_{1}, a_{2}, \cdots, a_{n}$ \begin{CJK}{UTF8}{mj}是\end{CJK} $n$ \begin{CJK}{UTF8}{mj}个互不相同的数\end{CJK}, \begin{CJK}{UTF8}{mj}证明下列方程组有唯一解\end{CJK}, \begin{CJK}{UTF8}{mj}并求解\end{CJK}.
\end{enumerate}
$$
\left\{\begin{array}{l}
x_{1}+a_{1} x_{2}+a_{1}^{2} x_{3}+\cdots+a_{1}^{n-1} x_{n}=-a_{1}^{n} \\
x_{1}+a_{2} x_{2}+a_{2}^{2} x_{3}+\cdots+a_{2}^{n-1} x_{n}=-a_{2}^{n} \\
\cdots \cdots \\
x_{1}+a_{n} x_{2}+a_{n}^{2} x_{3}+\cdots+a_{n}^{n-1} x_{n}=-a_{n}^{n}
\end{array}\right.
$$

\begin{enumerate}
  \setcounter{enumi}{5}
  \item ( 15 \begin{CJK}{UTF8}{mj}分\end{CJK}) \begin{CJK}{UTF8}{mj}设\end{CJK} $\alpha$ \begin{CJK}{UTF8}{mj}是\end{CJK} $\mathbb{R}^{n}$ \begin{CJK}{UTF8}{mj}中标准度量下的单位向量\end{CJK}, \begin{CJK}{UTF8}{mj}即\end{CJK} $|\alpha|=1$, \begin{CJK}{UTF8}{mj}证明必存在一个\end{CJK} $n$ \begin{CJK}{UTF8}{mj}级实对称正交矩阵\end{CJK} $A$ \begin{CJK}{UTF8}{mj}使得\end{CJK} $A$ \begin{CJK}{UTF8}{mj}的第一列为\end{CJK} $\alpha$.

  \item (15 \begin{CJK}{UTF8}{mj}分\end{CJK}) \begin{CJK}{UTF8}{mj}设\end{CJK} $f(x), g(x)$ \begin{CJK}{UTF8}{mj}为数域\end{CJK} $P$ \begin{CJK}{UTF8}{mj}上两个互素的一元多项式\end{CJK}, $A$ \begin{CJK}{UTF8}{mj}是数域\end{CJK} $P$ \begin{CJK}{UTF8}{mj}上的\end{CJK} $n$ \begin{CJK}{UTF8}{mj}级矩阵\end{CJK}, \begin{CJK}{UTF8}{mj}证明\end{CJK} $f(A) g(A) X=$ 0 \begin{CJK}{UTF8}{mj}的解空间是\end{CJK} $f(A) X=0$ \begin{CJK}{UTF8}{mj}与\end{CJK} $g(A) X=0$ \begin{CJK}{UTF8}{mj}两个解空间的直和\end{CJK}.

  \item ( 10 \begin{CJK}{UTF8}{mj}分\end{CJK}) \begin{CJK}{UTF8}{mj}设\end{CJK} $V$ \begin{CJK}{UTF8}{mj}为复数域上的\end{CJK} $n$ \begin{CJK}{UTF8}{mj}维线性空间\end{CJK}, $\sigma, \tau$ \begin{CJK}{UTF8}{mj}为\end{CJK} $V$ \begin{CJK}{UTF8}{mj}中的两个线性变换\end{CJK}, \begin{CJK}{UTF8}{mj}如果\end{CJK} $\sigma \tau=\tau \sigma$, \begin{CJK}{UTF8}{mj}证明\end{CJK} $\sigma, \tau$ \begin{CJK}{UTF8}{mj}至\end{CJK} \begin{CJK}{UTF8}{mj}少有一个公共的特征向量\end{CJK}.

  \item ( 15 \begin{CJK}{UTF8}{mj}分\end{CJK}) \begin{CJK}{UTF8}{mj}设矩阵\end{CJK} $A=\left(a_{i j}\right)_{n n}, B=\left(b_{i j}\right)_{n n}, C=\left(c_{i j}\right)_{n n}$, \begin{CJK}{UTF8}{mj}其中\end{CJK} $c_{i j}=a_{i j} b_{i j}, i, j=1,2, \cdots, n$, \begin{CJK}{UTF8}{mj}如果\end{CJK} $A, B$ \begin{CJK}{UTF8}{mj}均为正定矩阵\end{CJK}, \begin{CJK}{UTF8}{mj}证明\end{CJK} $C$ \begin{CJK}{UTF8}{mj}也是正定矩阵\end{CJK}.

  \item ( 15 \begin{CJK}{UTF8}{mj}分\end{CJK}) \begin{CJK}{UTF8}{mj}设\end{CJK} $f\left(x_{1}, x_{2}, \cdots, x_{n}\right)$ \begin{CJK}{UTF8}{mj}是一个秩为\end{CJK} $n$ \begin{CJK}{UTF8}{mj}符号差为\end{CJK} $s$ \begin{CJK}{UTF8}{mj}的实二次型\end{CJK}, \begin{CJK}{UTF8}{mj}证明存在\end{CJK} $\mathbb{R}^{n}$ \begin{CJK}{UTF8}{mj}的一个\end{CJK} $\frac{1}{2}(n-|s|)$ \begin{CJK}{UTF8}{mj}维\end{CJK} \begin{CJK}{UTF8}{mj}子空间\end{CJK} $W$, \begin{CJK}{UTF8}{mj}使得对任意的\end{CJK} $\left(x_{1}, x_{2}, \cdots, x_{n}\right) \in W$, \begin{CJK}{UTF8}{mj}都有\end{CJK} $f\left(x_{1}, x_{2}, \cdots, x_{n}\right)=0$.

  \item ( 10 \begin{CJK}{UTF8}{mj}分\end{CJK}) \begin{CJK}{UTF8}{mj}设\end{CJK} $n$ \begin{CJK}{UTF8}{mj}级矩阵\end{CJK} $A=\left(a_{i j}\right)_{n n}$ \begin{CJK}{UTF8}{mj}是一个幂零矩阵\end{CJK}, \begin{CJK}{UTF8}{mj}且\end{CJK} $a_{12} \neq 0, a_{13}=1, a_{22}=0, a_{23} \neq 0$, \begin{CJK}{UTF8}{mj}证明不存在矩\end{CJK} \begin{CJK}{UTF8}{mj}阵\end{CJK} $B$ \begin{CJK}{UTF8}{mj}使得\end{CJK} $B^{n-1}=A$.

\end{enumerate}
\section{0. 南开大学 2018 年研究生入学考试试题高等代数 
 李扬 
 微信公众号: sxkyliyang}
\begin{enumerate}
  \item \begin{CJK}{UTF8}{mj}已知矩阵\end{CJK}
\end{enumerate}
$$
A=\left(\begin{array}{ccc}
3 & 1 & 0 \\
1 & -1 & 2 \\
1 & 1 & 1
\end{array}\right) .
$$
\begin{CJK}{UTF8}{mj}求\end{CJK} $A^{*}$ \begin{CJK}{UTF8}{mj}及\end{CJK} $A^{-1}$.

\begin{enumerate}
  \setcounter{enumi}{2}
  \item \begin{CJK}{UTF8}{mj}已知矩阵\end{CJK}
\end{enumerate}
$$
A=\left(\begin{array}{ccc}
3 & 0 & 8 \\
3 & -1 & 6 \\
-2 & 0 & -5
\end{array}\right) .
$$
\begin{CJK}{UTF8}{mj}求\end{CJK} $A$ \begin{CJK}{UTF8}{mj}的特征值与若尔当标准形\end{CJK}.

\begin{enumerate}
  \setcounter{enumi}{3}
  \item \begin{CJK}{UTF8}{mj}已知方程组\end{CJK}
\end{enumerate}
$$
\left\{\begin{array}{l}
a x_{1}+b x_{2}+b x_{3}=1 \\
b x_{1}+a x_{2}+b x_{3}=1 \\
b x_{1}+b x_{2}+a x_{3}=1
\end{array}\right.
$$
\begin{CJK}{UTF8}{mj}求\end{CJK} $a, b$ \begin{CJK}{UTF8}{mj}满足什么条件时\end{CJK}, \begin{CJK}{UTF8}{mj}方程组无解\end{CJK}? $a, b$ \begin{CJK}{UTF8}{mj}满足什么条件时\end{CJK}, \begin{CJK}{UTF8}{mj}方程组有解\end{CJK}? \begin{CJK}{UTF8}{mj}并在有解的情况的下求全部\end{CJK} \begin{CJK}{UTF8}{mj}解\end{CJK}.

\begin{enumerate}
  \setcounter{enumi}{4}
  \item \begin{CJK}{UTF8}{mj}已知\end{CJK} $a_{1} a_{2} \cdots a_{n} \neq 0$, \begin{CJK}{UTF8}{mj}求行列式\end{CJK}
\end{enumerate}
$$
\left|\begin{array}{cccccc}
a_{1} & 1 & 1 & \cdots & 1 & 1 \\
1 & a_{2} & 0 & \cdots & 0 & 0 \\
1 & 0 & a_{3} & \cdots & 0 & 0 \\
\vdots & \vdots & \vdots & & \vdots & \vdots \\
1 & 0 & 0 & \cdots & a_{n-1} & 0 \\
1 & 0 & 0 & \cdots & 0 & a_{n}
\end{array}\right| .
$$

\begin{enumerate}
  \setcounter{enumi}{5}
  \item \begin{CJK}{UTF8}{mj}非零向量\end{CJK} $\alpha_{1}$ \begin{CJK}{UTF8}{mj}是\end{CJK} $n$ \begin{CJK}{UTF8}{mj}级矩阵\end{CJK} $A$ \begin{CJK}{UTF8}{mj}的属于特征值\end{CJK} $\lambda$ \begin{CJK}{UTF8}{mj}的特征向量\end{CJK}, \begin{CJK}{UTF8}{mj}且向量组\end{CJK} $\alpha_{1}, \alpha_{2}, \cdots, \alpha_{m}$ \begin{CJK}{UTF8}{mj}满足\end{CJK} $(A-\lambda E) \alpha_{i+1}=$ $\alpha_{i}(i=1,2, \cdots, m-1), E$ \begin{CJK}{UTF8}{mj}为\end{CJK} $n$ \begin{CJK}{UTF8}{mj}级单位矩阵\end{CJK}, \begin{CJK}{UTF8}{mj}证明\end{CJK}: $\alpha_{1}, \alpha_{2}, \cdots, \alpha_{m}$ \begin{CJK}{UTF8}{mj}线性无关\end{CJK}.

  \item $A$ \begin{CJK}{UTF8}{mj}为\end{CJK} $n$ \begin{CJK}{UTF8}{mj}级矩阵\end{CJK}, \begin{CJK}{UTF8}{mj}已知\end{CJK} $A^{2}=A$, \begin{CJK}{UTF8}{mj}证明\end{CJK} $r(A)+r(A-E)=n$.

  \item $P^{n \times n}$ \begin{CJK}{UTF8}{mj}是数域\end{CJK} $P$ \begin{CJK}{UTF8}{mj}上所有\end{CJK} $n \times n$ \begin{CJK}{UTF8}{mj}矩阵组成的线性空间\end{CJK}, \begin{CJK}{UTF8}{mj}令\end{CJK} $V_{1}=\left\{A \in P^{n \times n} \mid A^{\prime}=A\right\}, V_{2}=\left\{A \in P^{n \times n} \mid A^{\prime}=\right.$ $-A\}$, \begin{CJK}{UTF8}{mj}证明\end{CJK}:

\end{enumerate}
(1) $V_{1}, V_{2}$ \begin{CJK}{UTF8}{mj}都是\end{CJK} $P^{n \times n}$ \begin{CJK}{UTF8}{mj}的线性子空间\end{CJK}.

(2) $P^{n \times n}=V_{1} \oplus V_{2}$.

\begin{enumerate}
  \setcounter{enumi}{8}
  \item \begin{CJK}{UTF8}{mj}实二次型\end{CJK} $f\left(x_{1}, x_{2}, \cdots, x_{n}\right)=X^{\prime} A X$, \begin{CJK}{UTF8}{mj}证明\end{CJK}:\begin{CJK}{UTF8}{mj}当\end{CJK} $x_{1}^{2}+x_{2}^{2}+\cdots+x_{n}^{2}=1$ \begin{CJK}{UTF8}{mj}时\end{CJK}, $f\left(x_{1}, x_{2}, \cdots, x_{n}\right)$ \begin{CJK}{UTF8}{mj}的最小值恰\end{CJK} \begin{CJK}{UTF8}{mj}好等于\end{CJK} $A$ \begin{CJK}{UTF8}{mj}的最小特征值\end{CJK}.

  \item $A, B$ \begin{CJK}{UTF8}{mj}为\end{CJK} $n$ \begin{CJK}{UTF8}{mj}级实对称矩阵\end{CJK}, \begin{CJK}{UTF8}{mj}且\end{CJK} $A$ \begin{CJK}{UTF8}{mj}是正定的\end{CJK}, \begin{CJK}{UTF8}{mj}证明\end{CJK}: \begin{CJK}{UTF8}{mj}存在\end{CJK} $n$ \begin{CJK}{UTF8}{mj}级可逆矩阵\end{CJK} $P$ \begin{CJK}{UTF8}{mj}使得\end{CJK} $P^{\prime} A P, P^{\prime} B P$ \begin{CJK}{UTF8}{mj}同时为对角\end{CJK} \begin{CJK}{UTF8}{mj}阵\end{CJK}.

\end{enumerate}
\section{1. 南开大学 2007 年研究生入学考试试题数学分析 
 李扬 
 微信公众号: sxkyliyang}
\begin{enumerate}
  \item \begin{CJK}{UTF8}{mj}计算题\end{CJK}. (\begin{CJK}{UTF8}{mj}每题\end{CJK} 7 \begin{CJK}{UTF8}{mj}分\end{CJK}, \begin{CJK}{UTF8}{mj}共\end{CJK} 42 \begin{CJK}{UTF8}{mj}分\end{CJK})
\end{enumerate}
(1) $\lim _{n \rightarrow \infty}\left(\frac{1}{n+1}+\frac{1}{n+2}+\cdots+\frac{1}{2 n}\right)=$

(2) $\int_{0}^{+\infty} \frac{1-e^{-t}}{t} \sin t \mathrm{~d} t=$

(3) \begin{CJK}{UTF8}{mj}函数\end{CJK} $f(x, y)=2 x^{2}+12 x y+y^{2}$ \begin{CJK}{UTF8}{mj}在闭区域\end{CJK} $\bar{D}=\left\{(x, y) \in \mathbb{R}^{2} \mid x^{2}+4 y^{2} \leq 25\right\}$ \begin{CJK}{UTF8}{mj}的最小值是\end{CJK}

(4) \begin{CJK}{UTF8}{mj}设\end{CJK} $D=\left\{(x, y) \in \mathbb{R}^{2} \mid x^{2}+y^{2} \leq 1, x \geq 0, y \geq 0\right\}$, \begin{CJK}{UTF8}{mj}则二重积分\end{CJK} $\iint_{D} \sqrt{x^{9} y^{7}} \mathrm{~d} x \mathrm{~d} y=$

(5) \begin{CJK}{UTF8}{mj}设\end{CJK} $S=\left\{(x, y, z) \in \mathbb{R}^{3} \mid x^{2 n}+y^{2 n}+z^{2 n}=1, n \in N^{+}\right\}$, \begin{CJK}{UTF8}{mj}则曲面积分\end{CJK} $\iint_{D}\left(x^{3}+y^{3}+z^{3}\right) d S$ \begin{CJK}{UTF8}{mj}的值是\end{CJK}

(6) \begin{CJK}{UTF8}{mj}设\end{CJK} $L$ \begin{CJK}{UTF8}{mj}为单位圆\end{CJK} $x^{2}+y^{2}=1$ \begin{CJK}{UTF8}{mj}的正向\end{CJK}, \begin{CJK}{UTF8}{mj}则曲线积分\end{CJK}
$$
\int_{L} \frac{e^{y}}{x^{2}+y^{2}}[(x \sin x+y \cos x) \mathrm{d} x+(y \sin x-x \cos y) \mathrm{d} y]
$$
\begin{CJK}{UTF8}{mj}的值是\end{CJK}

\begin{enumerate}
  \setcounter{enumi}{2}
  \item ( 16 \begin{CJK}{UTF8}{mj}分\end{CJK})\begin{CJK}{UTF8}{mj}设函数\end{CJK} $f(x)$ \begin{CJK}{UTF8}{mj}在\end{CJK} $[0,+\infty)$ \begin{CJK}{UTF8}{mj}上连续\end{CJK}, $f(0)<0$ \begin{CJK}{UTF8}{mj}且\end{CJK} $f^{\prime}(x)>2$ \begin{CJK}{UTF8}{mj}对\end{CJK} $x>0$ \begin{CJK}{UTF8}{mj}均成立\end{CJK}. \begin{CJK}{UTF8}{mj}证明\end{CJK}: \begin{CJK}{UTF8}{mj}方程\end{CJK} $f(x)=0$ \begin{CJK}{UTF8}{mj}在\end{CJK} $\left(0, \frac{1}{2}|f(x)|\right)$ \begin{CJK}{UTF8}{mj}中有且仅有一个根\end{CJK}.

  \item (16 \begin{CJK}{UTF8}{mj}分\end{CJK})\begin{CJK}{UTF8}{mj}设\end{CJK} $f(x)$ \begin{CJK}{UTF8}{mj}在\end{CJK} $[0,1]$ \begin{CJK}{UTF8}{mj}上连续\end{CJK}, \begin{CJK}{UTF8}{mj}证明\end{CJK}:

\end{enumerate}
$$
\lim _{n \rightarrow \infty} \frac{1}{n}\left[f\left(\frac{1}{n}\right)-f\left(\frac{2}{n}\right)+f\left(\frac{3}{n}\right)-\cdots+(-1)^{n} f\left(\frac{n-1}{n}\right)\right]=0
$$

\begin{enumerate}
  \setcounter{enumi}{4}
  \item ( 16 \begin{CJK}{UTF8}{mj}分\end{CJK})\begin{CJK}{UTF8}{mj}若正项级数\end{CJK} $\sum_{n=1}^{\infty} a_{n}$ \begin{CJK}{UTF8}{mj}收敛\end{CJK}, \begin{CJK}{UTF8}{mj}证明\end{CJK}:
\end{enumerate}
(1) $\sum_{n=1}^{\infty} a_{n}^{p}(p>1)$ \begin{CJK}{UTF8}{mj}收敛\end{CJK};

(2) $\sum_{n=1}^{\infty} \frac{\sqrt[k]{a_{n}}}{n}(k \in N, k \geq 2)$ \begin{CJK}{UTF8}{mj}收敛\end{CJK}.

\begin{enumerate}
  \setcounter{enumi}{5}
  \item ( 20 \begin{CJK}{UTF8}{mj}分\end{CJK})\begin{CJK}{UTF8}{mj}证明含参变量广义积分\end{CJK} $\int_{0}^{+\infty} t e^{-t x^{2}} \mathrm{~d} x$ \begin{CJK}{UTF8}{mj}在\end{CJK} $t \in[0,+\infty)$ \begin{CJK}{UTF8}{mj}的任何有界闭子区间上一致收敛\end{CJK}.

  \item ( 20 \begin{CJK}{UTF8}{mj}分\end{CJK})\begin{CJK}{UTF8}{mj}设\end{CJK} $f(x)$ \begin{CJK}{UTF8}{mj}在区间\end{CJK} $(0,+\infty)$ \begin{CJK}{UTF8}{mj}上连续有界\end{CJK}, \begin{CJK}{UTF8}{mj}且\end{CJK} $f(x+1) \neq f(x)$ \begin{CJK}{UTF8}{mj}对所有\end{CJK} $x>0$ \begin{CJK}{UTF8}{mj}成立\end{CJK}. \begin{CJK}{UTF8}{mj}证明\end{CJK}:

\end{enumerate}
$$
\lim _{n \rightarrow \infty}[f(n)-f(n-1)]=0
$$

\begin{enumerate}
  \setcounter{enumi}{7}
  \item ( 20 \begin{CJK}{UTF8}{mj}分\end{CJK}) \begin{CJK}{UTF8}{mj}设\end{CJK} $\Omega=\left\{x \in \mathbb{R}^{n}|| x \mid<1\right\}$, \begin{CJK}{UTF8}{mj}函数\end{CJK} $u(x)$ \begin{CJK}{UTF8}{mj}在\end{CJK} $\Omega$ \begin{CJK}{UTF8}{mj}内二阶连续可微\end{CJK}, \begin{CJK}{UTF8}{mj}在\end{CJK} $\bar{\Omega}$ \begin{CJK}{UTF8}{mj}上连续\end{CJK}, \begin{CJK}{UTF8}{mj}且在\end{CJK} $\Omega$ \begin{CJK}{UTF8}{mj}内满足\end{CJK} $\Delta u-b u=0$, \begin{CJK}{UTF8}{mj}其中\end{CJK} $\Delta=\sum_{i=1}^{n} \frac{\partial^{2}}{\partial x_{i}^{2}}$ \begin{CJK}{UTF8}{mj}为\end{CJK} Laplace \begin{CJK}{UTF8}{mj}算子\end{CJK}, $b>0$ \begin{CJK}{UTF8}{mj}为常数\end{CJK}. \begin{CJK}{UTF8}{mj}设对任意边界点\end{CJK} $x \in \partial \Omega$ \begin{CJK}{UTF8}{mj}有\end{CJK} $u(x)>0$, \begin{CJK}{UTF8}{mj}证明\end{CJK}: \begin{CJK}{UTF8}{mj}对任意\end{CJK} $x \in \bar{\Omega}$ \begin{CJK}{UTF8}{mj}有\end{CJK} $u(x)>0$.
\end{enumerate}
\section{2. 南开大学 2008 年研究生入学考试试题数学分析}
\begin{CJK}{UTF8}{mj}李扬\end{CJK}

\begin{CJK}{UTF8}{mj}微信公众号\end{CJK}: sxkyliyang

\begin{CJK}{UTF8}{mj}一\end{CJK}、\begin{CJK}{UTF8}{mj}计算题\end{CJK}.

\begin{enumerate}
  \item \begin{CJK}{UTF8}{mj}求极限\end{CJK}
\end{enumerate}
$$
\lim _{x \rightarrow \infty}\left[x-x^{2} \ln \left(1+\frac{1}{x}\right)\right] .
$$

\begin{enumerate}
  \setcounter{enumi}{2}
  \item \begin{CJK}{UTF8}{mj}求下列级数的和\end{CJK}
\end{enumerate}
$$
\sum_{n=1}^{\infty} \frac{(-1)^{n-1}}{n(n+2)} .
$$

\begin{enumerate}
  \setcounter{enumi}{3}
  \item \begin{CJK}{UTF8}{mj}已知\end{CJK} $f(x)$ \begin{CJK}{UTF8}{mj}在\end{CJK} $\mathbb{R}$ \begin{CJK}{UTF8}{mj}上连续可导\end{CJK}, \begin{CJK}{UTF8}{mj}且满足\end{CJK} $f^{\prime}(-x)=x\left(f^{\prime}(x)-1\right)$, \begin{CJK}{UTF8}{mj}求函数\end{CJK} $f(x)$.

  \item \begin{CJK}{UTF8}{mj}已知\end{CJK} $\int_{x}^{2 \ln 2} \frac{\mathrm{d} t}{\sqrt{e^{t}-1}}=\frac{\pi}{6}$, \begin{CJK}{UTF8}{mj}求\end{CJK} $x$ \begin{CJK}{UTF8}{mj}的值\end{CJK}.

  \item \begin{CJK}{UTF8}{mj}设区域\end{CJK} $D=\{(x, y) \mid 0 \leq x \leq 2,-1 \leq y \leq 1\}$, \begin{CJK}{UTF8}{mj}求积分\end{CJK} $\iint_{D} \sqrt{|x-| y||} \mathrm{d} x \mathrm{~d} y$.

\end{enumerate}
\begin{CJK}{UTF8}{mj}二\end{CJK}、\begin{CJK}{UTF8}{mj}已知\end{CJK} $x_{1} \geq-6, x_{n+1}=\sqrt{x_{n}+6}, n=1,2, \cdots$, \begin{CJK}{UTF8}{mj}证明数列\end{CJK} $\left\{x_{n}\right\}$ \begin{CJK}{UTF8}{mj}收敛\end{CJK}, \begin{CJK}{UTF8}{mj}并求其极限\end{CJK}.

\begin{CJK}{UTF8}{mj}三\end{CJK}、\begin{CJK}{UTF8}{mj}设\end{CJK} $f(x)$ \begin{CJK}{UTF8}{mj}是区间\end{CJK} $[a, b]$ \begin{CJK}{UTF8}{mj}上的连续函数\end{CJK}, \begin{CJK}{UTF8}{mj}满足对任意\end{CJK} $x \in[a, b]$, \begin{CJK}{UTF8}{mj}存在\end{CJK} $y \in[a, b]$, \begin{CJK}{UTF8}{mj}使得\end{CJK}
$$
|f(y)| \leq \frac{1}{2}|f(x)|
$$
\begin{CJK}{UTF8}{mj}证明\end{CJK}: \begin{CJK}{UTF8}{mj}存在\end{CJK} $\xi \in[a, b]$, \begin{CJK}{UTF8}{mj}使得\end{CJK} $f(\xi)=0$.

\begin{CJK}{UTF8}{mj}四\end{CJK}、\begin{CJK}{UTF8}{mj}设\end{CJK} $f(x)$ \begin{CJK}{UTF8}{mj}在\end{CJK} $[a,+\infty)$ \begin{CJK}{UTF8}{mj}一致连续\end{CJK}, \begin{CJK}{UTF8}{mj}且广义积分\end{CJK} $\int_{a}^{+\infty} f(x) \mathrm{d} x$ \begin{CJK}{UTF8}{mj}收敛\end{CJK}, \begin{CJK}{UTF8}{mj}证明\end{CJK} $\lim _{x \rightarrow+\infty} f(x)=0$.

\begin{CJK}{UTF8}{mj}五\end{CJK}、\begin{CJK}{UTF8}{mj}设\end{CJK} $f(x)$ \begin{CJK}{UTF8}{mj}在\end{CJK} $\mathbb{R}$ \begin{CJK}{UTF8}{mj}上可微\end{CJK}, \begin{CJK}{UTF8}{mj}且\end{CJK} $\forall x \in \mathbb{R}$ \begin{CJK}{UTF8}{mj}都有\end{CJK} $f(x)>0,\left|f^{\prime}(x)\right| \leq m f(x)$, \begin{CJK}{UTF8}{mj}其中\end{CJK} $0<m<1$. \begin{CJK}{UTF8}{mj}任取\end{CJK} $a_{0} \in \mathbb{R}$, \begin{CJK}{UTF8}{mj}定义\end{CJK} $a_{n}=\ln f\left(a_{n-1}\right), n=1,2, \cdots$. \begin{CJK}{UTF8}{mj}证明级数\end{CJK} $\sum_{n=1}^{\infty}\left|a_{n}-a_{n-1}\right|$ \begin{CJK}{UTF8}{mj}收敛\end{CJK}.

\begin{CJK}{UTF8}{mj}六\end{CJK}、\begin{CJK}{UTF8}{mj}已知函数项级数\end{CJK} $f(x)=\sum_{n=1}^{\infty} n e^{-n x}$.

\begin{enumerate}
  \item \begin{CJK}{UTF8}{mj}证明\end{CJK} $f(x)$ \begin{CJK}{UTF8}{mj}在\end{CJK} $(0,+\infty)$ \begin{CJK}{UTF8}{mj}收敛但不一致收敛\end{CJK}.

  \item \begin{CJK}{UTF8}{mj}证明\end{CJK} $f(x)$ \begin{CJK}{UTF8}{mj}在\end{CJK} $(0,+\infty)$ \begin{CJK}{UTF8}{mj}上任意次可微\end{CJK}.

\end{enumerate}
\begin{CJK}{UTF8}{mj}七\end{CJK}、\begin{CJK}{UTF8}{mj}作变换\end{CJK} $u=\frac{y}{x}, v=x, w=x z-y$, \begin{CJK}{UTF8}{mj}将方程\end{CJK}
$$
y \frac{\partial^{2} z}{\partial y^{2}}+2 \frac{\partial z}{\partial y}=\frac{2}{x}
$$
\begin{CJK}{UTF8}{mj}变换为\end{CJK} $w$ \begin{CJK}{UTF8}{mj}关于自变量\end{CJK} $u, v$ \begin{CJK}{UTF8}{mj}的方程\end{CJK}.

\begin{CJK}{UTF8}{mj}八\end{CJK}、\begin{CJK}{UTF8}{mj}求由雉面\end{CJK} $x^{2}+y^{2}+a z=4 a^{2}$ \begin{CJK}{UTF8}{mj}将球体\end{CJK} $x^{2}+y^{2}+z^{2} \leq 4 a z$ \begin{CJK}{UTF8}{mj}分成两部分的体积之比\end{CJK}.

\begin{CJK}{UTF8}{mj}九\end{CJK}、\begin{CJK}{UTF8}{mj}设\end{CJK} $f(x)$ \begin{CJK}{UTF8}{mj}是\end{CJK} $(0,+\infty)$ \begin{CJK}{UTF8}{mj}上具有二阶连续导数\end{CJK}, \begin{CJK}{UTF8}{mj}且\end{CJK} $f(x)>0, f^{\prime}(x) \leq 0, f^{\prime \prime}(x)$ \begin{CJK}{UTF8}{mj}有界\end{CJK}. \begin{CJK}{UTF8}{mj}证明\end{CJK}
$$
\lim _{x \rightarrow+\infty} f^{\prime}(x)=0 .
$$

\section{3. 南开大学 2009 年研究生入学考试试题数学分析 
 李扬 
 微信公众号: sxkyliyang}
\begin{enumerate}
  \item (15 \begin{CJK}{UTF8}{mj}分\end{CJK}) \begin{CJK}{UTF8}{mj}计算二重积分\end{CJK}
\end{enumerate}
$$
\iint_{D}|\cos (x+y)| \mathrm{d} x \mathrm{~d} y,
$$
\begin{CJK}{UTF8}{mj}其中\end{CJK} $D$ \begin{CJK}{UTF8}{mj}由\end{CJK} $y=x, y=0, x=\frac{\pi}{2}$ \begin{CJK}{UTF8}{mj}所围成\end{CJK}.

\begin{enumerate}
  \setcounter{enumi}{2}
  \item (15 \begin{CJK}{UTF8}{mj}分\end{CJK}) \begin{CJK}{UTF8}{mj}计算累次积分\end{CJK}
\end{enumerate}
$$
\int_{-1}^{1} \mathrm{~d} x \int_{0}^{\sqrt{1-x^{2}}} \mathrm{~d} y \int_{1}^{1+\sqrt{1-x^{2}-y^{2}}} \frac{\mathrm{d} z}{\sqrt{x^{2}+y^{2}+z^{2}}} .
$$

\begin{enumerate}
  \setcounter{enumi}{3}
  \item ( 20 \begin{CJK}{UTF8}{mj}分\end{CJK}) \begin{CJK}{UTF8}{mj}计算曲线积分\end{CJK}
\end{enumerate}
$$
\int_{L} y \mathrm{~d} x+z \mathrm{~d} y+x \mathrm{~d} z .
$$
\begin{CJK}{UTF8}{mj}其中\end{CJK} $L$ \begin{CJK}{UTF8}{mj}为\end{CJK} $\frac{x^{2}}{a^{2}}+\frac{y^{2}}{b^{2}}+\frac{z^{2}}{c^{2}}=1$ \begin{CJK}{UTF8}{mj}与\end{CJK} $\frac{x}{a}+\frac{z}{a}=1$ \begin{CJK}{UTF8}{mj}在\end{CJK} $x \geq 0, y \geq 0, z \geq 0$ \begin{CJK}{UTF8}{mj}部分的交线\end{CJK}, \begin{CJK}{UTF8}{mj}方向为从点\end{CJK} $(a, 0,0)$ \begin{CJK}{UTF8}{mj}到\end{CJK} $(0,0, c)$, \begin{CJK}{UTF8}{mj}其中\end{CJK} $a, b, c$ \begin{CJK}{UTF8}{mj}为正常数\end{CJK}.

\begin{enumerate}
  \setcounter{enumi}{4}
  \item ( 15 \begin{CJK}{UTF8}{mj}分\end{CJK}) \begin{CJK}{UTF8}{mj}求幂级数\end{CJK}
\end{enumerate}
$$
\sum_{n=0}^{\infty} \frac{2 n+1}{2^{n+1}} x^{2 n+1}
$$
\begin{CJK}{UTF8}{mj}的收敛域与和函数\end{CJK}.

\begin{enumerate}
  \setcounter{enumi}{5}
  \item ( 20 \begin{CJK}{UTF8}{mj}分\end{CJK}) \begin{CJK}{UTF8}{mj}求反常积分\end{CJK}
\end{enumerate}
$$
f(t)=\int_{1}^{+\infty} \frac{\arctan t x}{x^{2} \sqrt{x^{2}-1}} \mathrm{~d} x
$$
\begin{CJK}{UTF8}{mj}的表达式\end{CJK}.

\begin{enumerate}
  \setcounter{enumi}{6}
  \item (15 \begin{CJK}{UTF8}{mj}分\end{CJK}) \begin{CJK}{UTF8}{mj}设\end{CJK} $\int_{a}^{+\infty} f(x) \mathrm{d} x$ \begin{CJK}{UTF8}{mj}收敛\end{CJK}, $\frac{f(x)}{x}$ \begin{CJK}{UTF8}{mj}在\end{CJK} $[a,+\infty)$ \begin{CJK}{UTF8}{mj}单调递减\end{CJK}, \begin{CJK}{UTF8}{mj}试证\end{CJK}:
\end{enumerate}
$$
\lim _{x \rightarrow+\infty} x f(x)=0 .
$$

\begin{enumerate}
  \setcounter{enumi}{7}
  \item ( 15 \begin{CJK}{UTF8}{mj}分\end{CJK}) \begin{CJK}{UTF8}{mj}已知\end{CJK} $f(x)$ \begin{CJK}{UTF8}{mj}在\end{CJK} $(-1,1)$ \begin{CJK}{UTF8}{mj}内有二阶导数\end{CJK}, $f(0)=f^{\prime}(0)=0,\left|f^{\prime \prime}(x)\right|^{2} \leq\left|f(x) f^{\prime}(x)\right|$. \begin{CJK}{UTF8}{mj}证明\end{CJK}: \begin{CJK}{UTF8}{mj}存在\end{CJK} $\delta>0$, \begin{CJK}{UTF8}{mj}使在\end{CJK} $(-\delta, \delta)$ \begin{CJK}{UTF8}{mj}内有\end{CJK} $f(x) \equiv 0$.

  \item ( 20 \begin{CJK}{UTF8}{mj}分\end{CJK}) \begin{CJK}{UTF8}{mj}设\end{CJK} $f(x, y)$ \begin{CJK}{UTF8}{mj}在\end{CJK} $P_{0}$ \begin{CJK}{UTF8}{mj}的邻域\end{CJK} $U\left(P_{0}\right)$ \begin{CJK}{UTF8}{mj}内存在连续的三阶偏导数\end{CJK}, \begin{CJK}{UTF8}{mj}并且所有三阶偏导数的绝对值不超过\end{CJK} \begin{CJK}{UTF8}{mj}常数\end{CJK} $M, P_{1}$ \begin{CJK}{UTF8}{mj}与\end{CJK} $P_{2}$ \begin{CJK}{UTF8}{mj}关于\end{CJK} $P_{0}$ \begin{CJK}{UTF8}{mj}对称\end{CJK}, \begin{CJK}{UTF8}{mj}并且\end{CJK} $P_{1}, P_{2} \in U\left(P_{0}\right), P_{1}$ \begin{CJK}{UTF8}{mj}与\end{CJK} $P_{0}$ \begin{CJK}{UTF8}{mj}的距离为\end{CJK} $l, \vec{l}$ \begin{CJK}{UTF8}{mj}为由\end{CJK} $P_{0}$ \begin{CJK}{UTF8}{mj}指向\end{CJK} $P_{1}$ \begin{CJK}{UTF8}{mj}的方向\end{CJK} \begin{CJK}{UTF8}{mj}向量\end{CJK}, \begin{CJK}{UTF8}{mj}证明\end{CJK}:

\end{enumerate}
$$
\left|\frac{f\left(P_{1}\right)-f\left(P_{2}\right)}{2 l}-\frac{\partial f\left(P_{0}\right)}{\partial \vec{l}}\right| \leq \frac{\sqrt{2}}{3} M l^{2}
$$

\begin{enumerate}
  \setcounter{enumi}{9}
  \item ( 15 \begin{CJK}{UTF8}{mj}分\end{CJK}) \begin{CJK}{UTF8}{mj}证明\end{CJK}: \begin{CJK}{UTF8}{mj}若\end{CJK} $\lim _{n \rightarrow+\infty} \frac{u_{n+1}}{u_{n}}=a, u_{n}>0$, \begin{CJK}{UTF8}{mj}则\end{CJK} $\lim _{n \rightarrow+\infty} \sqrt[n]{u_{n}}=a$. \begin{CJK}{UTF8}{mj}利用这一结论分析达朗贝尔判别法与柯\end{CJK} \begin{CJK}{UTF8}{mj}西判别法二者在判别正项级数的敛散性时的关系\end{CJK}, \begin{CJK}{UTF8}{mj}可以获得怎样的经验\end{CJK}?
\end{enumerate}
\section{4. 南开大学 2010 年研究生入学考试试题数学分析 
 李扬 
 微信公众号: sxkyliyang}
\begin{enumerate}
  \item ( 15 \begin{CJK}{UTF8}{mj}分\end{CJK}) \begin{CJK}{UTF8}{mj}求极限\end{CJK}
\end{enumerate}
$$
\lim _{x \rightarrow \infty}\left[\left(x-\frac{1}{2}\right)^{2}-x^{4} \ln ^{2}\left(1+\frac{1}{x}\right)\right] .
$$

\begin{enumerate}
  \setcounter{enumi}{2}
  \item (20 \begin{CJK}{UTF8}{mj}分\end{CJK}) \begin{CJK}{UTF8}{mj}计算积分\end{CJK}
\end{enumerate}
$$
\iint_{S}(x+z) \mathrm{d} S \text {. }
$$
\begin{CJK}{UTF8}{mj}其中\end{CJK} $S$ \begin{CJK}{UTF8}{mj}是曲面\end{CJK} $x^{2}+y^{2}=2 a z(a>0)$ \begin{CJK}{UTF8}{mj}被曲面\end{CJK} $z=\sqrt{x^{2}+y^{2}}$ \begin{CJK}{UTF8}{mj}所截取的有限部分\end{CJK}.

\begin{enumerate}
  \setcounter{enumi}{3}
  \item ( 20 \begin{CJK}{UTF8}{mj}分\end{CJK}) \begin{CJK}{UTF8}{mj}计算积分\end{CJK}
\end{enumerate}
$$
\iiint_{D} x y z \mathrm{~d} x \mathrm{~d} y \mathrm{~d} z \text {. }
$$
\begin{CJK}{UTF8}{mj}其中\end{CJK} $D$ \begin{CJK}{UTF8}{mj}位于第一象限且由曲面\end{CJK} $z=p\left(x^{2}+y^{2}\right), z=q\left(x^{2}+y^{2}\right), x y=a, x y=b, y=\alpha x, y=\beta x$, $(0<p<q, 0<a<b, 0<\alpha<\beta)$ \begin{CJK}{UTF8}{mj}所围\end{CJK}.

\begin{enumerate}
  \setcounter{enumi}{4}
  \item ( 15 \begin{CJK}{UTF8}{mj}分\end{CJK}) \begin{CJK}{UTF8}{mj}求级数\end{CJK} $\sum_{n=1}^{\infty} \frac{n(n+2)}{3^{n}}$ \begin{CJK}{UTF8}{mj}的和\end{CJK}.

  \item ( 15 \begin{CJK}{UTF8}{mj}分\end{CJK}) \begin{CJK}{UTF8}{mj}讨论级数\end{CJK}

\end{enumerate}
$$
\sum_{n=2}^{\infty} \frac{(-1)^{n}}{n^{p+\frac{1}{\ln n}}}
$$
\begin{CJK}{UTF8}{mj}的敛散性和绝对收敛性\end{CJK}.

\begin{enumerate}
  \setcounter{enumi}{6}
  \item ( 20 \begin{CJK}{UTF8}{mj}分\end{CJK}) \begin{CJK}{UTF8}{mj}讨论并证明如下问题\end{CJK}:
\end{enumerate}
(1)\begin{CJK}{UTF8}{mj}设函数\end{CJK} $f(x)$ \begin{CJK}{UTF8}{mj}在闭区间\end{CJK} $[a, b]$ \begin{CJK}{UTF8}{mj}上连续\end{CJK}, \begin{CJK}{UTF8}{mj}且对任意\end{CJK} $x \in[a, b]$, \begin{CJK}{UTF8}{mj}都有\end{CJK} $f(x) \in[a, b]$. \begin{CJK}{UTF8}{mj}证明\end{CJK}: \begin{CJK}{UTF8}{mj}存在一点\end{CJK} $c \in[a, b]$ \begin{CJK}{UTF8}{mj}使得\end{CJK} $f(c)=c$.

(2) \begin{CJK}{UTF8}{mj}是否存在\end{CJK} $\mathbb{R}$ \begin{CJK}{UTF8}{mj}上的连续函数\end{CJK} $f(x)$, \begin{CJK}{UTF8}{mj}使得\end{CJK} $f(x)$ \begin{CJK}{UTF8}{mj}在有理点上取值为无理数\end{CJK}, \begin{CJK}{UTF8}{mj}在无理点上取值为有理数\end{CJK}?

\begin{enumerate}
  \setcounter{enumi}{7}
  \item (15 \begin{CJK}{UTF8}{mj}分\end{CJK}) \begin{CJK}{UTF8}{mj}设\end{CJK} $f(x)$ \begin{CJK}{UTF8}{mj}在\end{CJK} $[0,1]$ \begin{CJK}{UTF8}{mj}内二阶可导\end{CJK}, \begin{CJK}{UTF8}{mj}且\end{CJK} $f(0)=0, f(1)=3, \min _{x \in[0,1]} f(x)=-1$. \begin{CJK}{UTF8}{mj}证明\end{CJK}: \begin{CJK}{UTF8}{mj}存在一点\end{CJK} $c \in(0,1)$ \begin{CJK}{UTF8}{mj}使得\end{CJK} $f^{\prime \prime}(c) \geq 18$.

  \item (15 \begin{CJK}{UTF8}{mj}分\end{CJK}) \begin{CJK}{UTF8}{mj}已知\end{CJK} $f(x)$ \begin{CJK}{UTF8}{mj}在\end{CJK} $[a, b]$ \begin{CJK}{UTF8}{mj}内二阶连续可导\end{CJK}, \begin{CJK}{UTF8}{mj}且满足\end{CJK} $\left|\int_{a}^{b} f(x) \mathrm{d} x\right|<\int_{a}^{b}|f(x)| \mathrm{d} x$, \begin{CJK}{UTF8}{mj}记\end{CJK} $M_{1}=\max _{x \in[a, b]}\left|f^{\prime}(x)\right|$, $M_{2}=\max _{x \in[a, b]}\left|f^{\prime \prime}(x)\right|$. \begin{CJK}{UTF8}{mj}证明\end{CJK}:

\end{enumerate}
$$
\left|\int_{a}^{b} f(x) \mathrm{d} x\right| \leq \frac{M_{1}}{2}(b-a)^{2}+\frac{M_{2}}{3}(b-a)^{3}
$$

\begin{enumerate}
  \setcounter{enumi}{9}
  \item ( 15 \begin{CJK}{UTF8}{mj}分\end{CJK}) \begin{CJK}{UTF8}{mj}设\end{CJK} $f(x)$ \begin{CJK}{UTF8}{mj}在闭区间\end{CJK} $[a, b]$ \begin{CJK}{UTF8}{mj}上连续且\end{CJK} $\lim _{x \rightarrow+\infty} f(x)=A \in \mathbb{R}$, \begin{CJK}{UTF8}{mj}证明\end{CJK}:
\end{enumerate}
$$
\lim _{\alpha \rightarrow 0^{+}} \int_{0}^{+\infty} \alpha e^{-\alpha x} f(x) \mathrm{d} x=A
$$

\section{5. 南开大学 2011 年研究生入学考试试题数学分析}
\begin{CJK}{UTF8}{mj}李扬\end{CJK}

\begin{CJK}{UTF8}{mj}微信公众号\end{CJK}: sxkyliyang

\begin{CJK}{UTF8}{mj}一\end{CJK}、\begin{CJK}{UTF8}{mj}计算题\end{CJK} (\begin{CJK}{UTF8}{mj}每小题\end{CJK} 15 \begin{CJK}{UTF8}{mj}分\end{CJK}, \begin{CJK}{UTF8}{mj}共\end{CJK} 60 \begin{CJK}{UTF8}{mj}分\end{CJK})

1、\begin{CJK}{UTF8}{mj}求极限\end{CJK}
$$
\lim _{x \rightarrow 0} \frac{\cos \sqrt{2} x-e^{-x^{2}}+\frac{x^{4}}{3}}{x^{6}} .
$$
2、\begin{CJK}{UTF8}{mj}计算\end{CJK}
$$
I=\int_{L} \frac{-y \mathrm{~d} x+x \mathrm{~d} y}{4 x^{2}+y^{2}} .
$$
$L$ \begin{CJK}{UTF8}{mj}为\end{CJK} $x^{2}+y^{2}=1$, \begin{CJK}{UTF8}{mj}取逆时针方向\end{CJK}.

3、 \begin{CJK}{UTF8}{mj}计算\end{CJK}
$$
I=\iint_{s} \frac{x^{3}+y^{3}+z^{3}}{1-z} \mathrm{~d} S .
$$
$S$ \begin{CJK}{UTF8}{mj}为\end{CJK} $x^{2}+y^{2}=(1-z)^{2}, 0 \leq z \leq 1$.

4、\begin{CJK}{UTF8}{mj}求函数\end{CJK} $f(x, y)=2 x^{2}-7 y^{2}$ \begin{CJK}{UTF8}{mj}在闭区域\end{CJK} $\bar{D}=\left\{(x, y) \mid x^{2}+2 x y+4 y^{2} \leq 13\right\}$ \begin{CJK}{UTF8}{mj}的最大值和最小值\end{CJK}.

\begin{CJK}{UTF8}{mj}二\end{CJK}、 $\left(15\right.$ \begin{CJK}{UTF8}{mj}分\end{CJK}) \begin{CJK}{UTF8}{mj}设\end{CJK} $\left\{a_{n}\right\},\left\{b_{n}\right\}$ \begin{CJK}{UTF8}{mj}均为正整数数列\end{CJK}. $a_{1}=b_{1}=1, a_{n}+\sqrt{3} b_{n}=\left(a_{n-1}+\sqrt{3} b_{n-1}\right)^{2}$, \begin{CJK}{UTF8}{mj}证明数列\end{CJK} $\left\{\frac{a_{n}}{b_{n}}\right\}$ \begin{CJK}{UTF8}{mj}的极限存在\end{CJK}, \begin{CJK}{UTF8}{mj}并求该极限值\end{CJK}.

\begin{CJK}{UTF8}{mj}三\end{CJK}、 $\left(15\right.$ \begin{CJK}{UTF8}{mj}分\end{CJK}) \begin{CJK}{UTF8}{mj}设\end{CJK} $f(x)$ \begin{CJK}{UTF8}{mj}在\end{CJK} $[a, b]$ \begin{CJK}{UTF8}{mj}有连续的导函数\end{CJK}, $f\left(\frac{a+b}{2}\right)=0$, \begin{CJK}{UTF8}{mj}试证明\end{CJK}
$$
\int_{a}^{b}\left|f(x) f^{\prime}(x)\right| \mathrm{d} x \leq \frac{b-a}{4} \int_{a}^{b}\left|f^{\prime}(x)\right|^{2} \mathrm{~d} x
$$
\begin{CJK}{UTF8}{mj}四\end{CJK}、( 20 \begin{CJK}{UTF8}{mj}分\end{CJK}) \begin{CJK}{UTF8}{mj}设级数\end{CJK} $\sum_{n=2}^{\infty} \frac{a_{n}}{\ln n}$ \begin{CJK}{UTF8}{mj}收敛\end{CJK}, \begin{CJK}{UTF8}{mj}数列\end{CJK} $n a_{n}$ \begin{CJK}{UTF8}{mj}单调递减\end{CJK}, \begin{CJK}{UTF8}{mj}试证\end{CJK}:
$$
\lim _{x \rightarrow \infty} n a_{n} \ln \ln n=0
$$
\begin{CJK}{UTF8}{mj}五\end{CJK}、(20 \begin{CJK}{UTF8}{mj}分\end{CJK}) \begin{CJK}{UTF8}{mj}设\end{CJK} $p, q$ \begin{CJK}{UTF8}{mj}为实数\end{CJK}, \begin{CJK}{UTF8}{mj}试讨论广义积分\end{CJK}
$$
\int_{0}^{+\infty} \frac{e^{\sin x} \sin 2 x}{x^{p}\left(1+x^{q}\right)} d x
$$
\begin{CJK}{UTF8}{mj}何时绝对收敛\end{CJK}, \begin{CJK}{UTF8}{mj}何时条件收敛\end{CJK}, \begin{CJK}{UTF8}{mj}何时发散\end{CJK}, \begin{CJK}{UTF8}{mj}并说明理由\end{CJK}.

\begin{CJK}{UTF8}{mj}六\end{CJK}、 (20 \begin{CJK}{UTF8}{mj}分\end{CJK}) \begin{CJK}{UTF8}{mj}已知\end{CJK} $\int_{0}^{+\infty} \frac{\sin x}{x} \mathrm{~d} x=\frac{\pi}{2}$. \begin{CJK}{UTF8}{mj}设\end{CJK}
$$
F(y)=\int_{0}^{+\infty} \frac{\sin \sqrt{x} y}{x(1+x)} \mathrm{d} x, y>0 .
$$

\begin{enumerate}
  \item \begin{CJK}{UTF8}{mj}试证明\end{CJK}: $F^{\prime \prime}(y)-F(y)+\pi=0$.
\end{enumerate}
$2)$ \begin{CJK}{UTF8}{mj}求出\end{CJK} $F(y)$ \begin{CJK}{UTF8}{mj}的等函数表达式\end{CJK}.

\section{6. 南开大学 2012 年研究生入学考试试题数学分析 
 李扬 
 微信公众号: sxkyliyang}
\begin{enumerate}
  \item (15 \begin{CJK}{UTF8}{mj}分\end{CJK}) \begin{CJK}{UTF8}{mj}求极限\end{CJK}
\end{enumerate}
$$
\lim _{x \rightarrow+\infty} x^{m} \int_{0}^{\frac{1}{x}} \sin t^{2} \mathrm{~d} t .
$$
\begin{CJK}{UTF8}{mj}其中\end{CJK} $m$ \begin{CJK}{UTF8}{mj}为任意整数\end{CJK}.

\begin{enumerate}
  \setcounter{enumi}{2}
  \item ( 20 \begin{CJK}{UTF8}{mj}分\end{CJK}) \begin{CJK}{UTF8}{mj}计算积分\end{CJK}
\end{enumerate}
$$
\iint_{D} \sqrt{\left|y-x^{2}\right|} \mathrm{d} x \mathrm{~d} y .
$$
\begin{CJK}{UTF8}{mj}其中\end{CJK} $D=\{(x, y) \mid-1 \leq x \leq 1,0 \leq y \leq 1\}$.

\begin{enumerate}
  \setcounter{enumi}{3}
  \item (20 \begin{CJK}{UTF8}{mj}分\end{CJK}) \begin{CJK}{UTF8}{mj}计算积分\end{CJK}
\end{enumerate}
$$
\iint_{S} x^{2} \mathrm{~d} y \mathrm{~d} z+z \mathrm{~d} x \mathrm{~d} y
$$
\begin{CJK}{UTF8}{mj}其中\end{CJK} $S$ \begin{CJK}{UTF8}{mj}为球面\end{CJK} $x^{2}+y^{2}+(z-a)^{2}=a^{2}$ \begin{CJK}{UTF8}{mj}中满足\end{CJK} $x^{2}+y^{2} \leq a y$ \begin{CJK}{UTF8}{mj}与\end{CJK} $z \leq a$ \begin{CJK}{UTF8}{mj}那部分的下侧\end{CJK} $(a>0)$.

\begin{enumerate}
  \setcounter{enumi}{4}
  \item ( 20 \begin{CJK}{UTF8}{mj}分\end{CJK}) \begin{CJK}{UTF8}{mj}求级数\end{CJK}
\end{enumerate}
$$
\sum_{n=1}^{\infty} \frac{(-1)^{n-1}(n+2)}{n(n+1)}
$$
\begin{CJK}{UTF8}{mj}的和\end{CJK}.

\begin{enumerate}
  \setcounter{enumi}{5}
  \item (15 \begin{CJK}{UTF8}{mj}分\end{CJK}) \begin{CJK}{UTF8}{mj}讨论广义积分\end{CJK}
\end{enumerate}
$$
\int_{0}^{+\infty} \frac{\ln (1+x)}{x^{p}} \mathrm{~d} x
$$
\begin{CJK}{UTF8}{mj}的敛散性\end{CJK}, \begin{CJK}{UTF8}{mj}其中\end{CJK} $p$ \begin{CJK}{UTF8}{mj}为实数\end{CJK}.

\begin{enumerate}
  \setcounter{enumi}{6}
  \item (15 \begin{CJK}{UTF8}{mj}分\end{CJK}) \begin{CJK}{UTF8}{mj}请回答\end{CJK}: \begin{CJK}{UTF8}{mj}函数\end{CJK} $f(x)=\sin x^{2}$ \begin{CJK}{UTF8}{mj}在\end{CJK} $(-\infty,+\infty)$ \begin{CJK}{UTF8}{mj}是否一致连续\end{CJK}? \begin{CJK}{UTF8}{mj}并说明理由\end{CJK}.

  \item ( 15 \begin{CJK}{UTF8}{mj}分\end{CJK}) \begin{CJK}{UTF8}{mj}设\end{CJK} $f(x)$ \begin{CJK}{UTF8}{mj}在\end{CJK} $[0,1]$ \begin{CJK}{UTF8}{mj}可微\end{CJK}, $f(0)=0$, \begin{CJK}{UTF8}{mj}对任意\end{CJK} $x \in(0,1)$, \begin{CJK}{UTF8}{mj}有\end{CJK} $f(x) \neq 0$. \begin{CJK}{UTF8}{mj}证明\end{CJK}: \begin{CJK}{UTF8}{mj}存在\end{CJK} $\xi \in(0,1)$, \begin{CJK}{UTF8}{mj}使得\end{CJK}

\end{enumerate}
$$
2 \frac{f^{\prime}(\xi)}{f(\xi)}=\frac{f^{\prime}(1-\xi)}{f(1-\xi)} \text {. }
$$

\begin{enumerate}
  \setcounter{enumi}{8}
  \item (15 \begin{CJK}{UTF8}{mj}分\end{CJK}) \begin{CJK}{UTF8}{mj}设\end{CJK} $f(x)$ \begin{CJK}{UTF8}{mj}与\end{CJK} $g(x)$ \begin{CJK}{UTF8}{mj}在\end{CJK} $[a, b]$ \begin{CJK}{UTF8}{mj}连续\end{CJK}, \begin{CJK}{UTF8}{mj}且\end{CJK} $f(x) \geq 0, g(x)>0$. \begin{CJK}{UTF8}{mj}求极限\end{CJK}
\end{enumerate}
$$
\lim _{n \rightarrow \infty}\left(\int_{a}^{b} g(x) f^{n}(x) \mathrm{d} x\right)^{\frac{1}{n}} .
$$

\begin{enumerate}
  \setcounter{enumi}{9}
  \item ( 15 \begin{CJK}{UTF8}{mj}分\end{CJK}) \begin{CJK}{UTF8}{mj}设\end{CJK} $f(x)$ \begin{CJK}{UTF8}{mj}在\end{CJK} 0 \begin{CJK}{UTF8}{mj}的某个空心邻域有定义\end{CJK}, \begin{CJK}{UTF8}{mj}已知\end{CJK} $\lim _{x \rightarrow 0} f(x)=0$, \begin{CJK}{UTF8}{mj}且\end{CJK}
\end{enumerate}
$$
\lim _{x \rightarrow 0} \frac{f(x)-f\left(\frac{x}{2}\right)}{x}=0 .
$$
\begin{CJK}{UTF8}{mj}证明\end{CJK}:
$$
\lim _{x \rightarrow 0} \frac{f(x)}{x}=0
$$

\section{7. 南开大学 2014 年研究生入学考试试题数学分析 
 李扬 
 微信公众号: sxkyliyang}
\begin{enumerate}
  \item \begin{CJK}{UTF8}{mj}求极限\end{CJK}
\end{enumerate}
$$
\lim _{n \rightarrow \infty}(\sqrt[n]{n}-1) \sin n \ln n
$$

\begin{enumerate}
  \setcounter{enumi}{2}
  \item \begin{CJK}{UTF8}{mj}已知二元函数\end{CJK}
\end{enumerate}
$$
f(x, y)= \begin{cases}e^{-x y} \frac{\sin x}{x}, & x \neq 0 \\ 1, & x=0\end{cases}
$$
\begin{CJK}{UTF8}{mj}证明在二维平面连续\end{CJK}.

\begin{enumerate}
  \setcounter{enumi}{3}
  \item \begin{CJK}{UTF8}{mj}已知\end{CJK} $0<a<b, c>0$. \begin{CJK}{UTF8}{mj}求点\end{CJK} $(0,0, c)$ \begin{CJK}{UTF8}{mj}到曲面\end{CJK} $\frac{z}{c}=\frac{x^{2}}{a^{2}}+\frac{y^{2}}{b^{2}}$ \begin{CJK}{UTF8}{mj}的最短距离\end{CJK}.

  \item \begin{CJK}{UTF8}{mj}求曲面积分\end{CJK}

\end{enumerate}
$$
I=\iint_{S} \frac{x \mathrm{~d} y \mathrm{~d} z+y \mathrm{~d} z \mathrm{~d} x+z \mathrm{~d} x \mathrm{~d} y}{\left(a x^{2}+b y^{2}+c z^{2}\right)^{\frac{3}{2}}}
$$
\begin{CJK}{UTF8}{mj}其中\end{CJK} $S$ \begin{CJK}{UTF8}{mj}是\end{CJK} $x^{2}+y^{2}+z^{2}=1$ \begin{CJK}{UTF8}{mj}的外侧\end{CJK}, $a, b, c$ \begin{CJK}{UTF8}{mj}为正实数\end{CJK}.

\begin{enumerate}
  \setcounter{enumi}{5}
  \item \begin{CJK}{UTF8}{mj}求级数\end{CJK}
\end{enumerate}
$$
\sum_{n=0}^{\infty} \frac{(-1)^{n}}{3 n+2}
$$
\begin{CJK}{UTF8}{mj}的和\end{CJK}.

\begin{enumerate}
  \setcounter{enumi}{6}
  \item \begin{CJK}{UTF8}{mj}已知函数项级数\end{CJK}
\end{enumerate}
$$
f(x)=\sum_{n=0}^{\infty} \frac{\sin n x}{n^{x}}
$$
(1) \begin{CJK}{UTF8}{mj}证明\end{CJK} $f(x)$ \begin{CJK}{UTF8}{mj}在\end{CJK} $(0,+\infty)$ \begin{CJK}{UTF8}{mj}非一致收敛\end{CJK}.

(2) \begin{CJK}{UTF8}{mj}证明\end{CJK} $f(x)$ \begin{CJK}{UTF8}{mj}在\end{CJK} $(0,+\infty)$ \begin{CJK}{UTF8}{mj}上连续\end{CJK}.

\begin{enumerate}
  \setcounter{enumi}{7}
  \item \begin{CJK}{UTF8}{mj}已知函数\end{CJK} $f(x)$ \begin{CJK}{UTF8}{mj}在\end{CJK} $(0,+\infty)$ \begin{CJK}{UTF8}{mj}连续可导\end{CJK}.
\end{enumerate}
(1) \begin{CJK}{UTF8}{mj}若\end{CJK} $\lim _{x \rightarrow+\infty} f(x)=1, \lim _{x \rightarrow+\infty} f^{\prime \prime}(x)=0$. \begin{CJK}{UTF8}{mj}证明\end{CJK} $\lim _{x \rightarrow+\infty} f^{\prime}(x)=0$.

(2) \begin{CJK}{UTF8}{mj}构造一个函数使\end{CJK} $\lim _{x \rightarrow+\infty} f(x)=1$, \begin{CJK}{UTF8}{mj}但\end{CJK} $\lim _{x \rightarrow+\infty} f^{\prime}(x)$ \begin{CJK}{UTF8}{mj}不存在\end{CJK}.

\begin{enumerate}
  \setcounter{enumi}{8}
  \item \begin{CJK}{UTF8}{mj}已知正项级数\end{CJK} $\sum_{n=0}^{\infty} x_{n}$ \begin{CJK}{UTF8}{mj}收敛\end{CJK}, \begin{CJK}{UTF8}{mj}证明\end{CJK}: \begin{CJK}{UTF8}{mj}存在发散于正无穷的数列\end{CJK} $\left\{\theta_{n}\right\}$ \begin{CJK}{UTF8}{mj}使得\end{CJK} $\sum_{n=0}^{\infty} \theta_{n} x_{n}$ \begin{CJK}{UTF8}{mj}仍收敛\end{CJK}.

  \item \begin{CJK}{UTF8}{mj}求极限\end{CJK}

\end{enumerate}
$$
\lim _{n \rightarrow \infty} \sum_{k=1}^{n} \frac{1}{\sqrt{k(n-k+1)}}
$$

\section{8. 南开大学 2016 年研究生入学考试试题数学分析 
 李扬 
 微信公众号: sxkyliyang}
\begin{enumerate}
  \item (15 \begin{CJK}{UTF8}{mj}分\end{CJK}) \begin{CJK}{UTF8}{mj}求定积分\end{CJK} $\int_{1}^{e} x^{n} \ln x \mathrm{~d} x, n \in \mathbb{Z}$.

  \item ( 20 \begin{CJK}{UTF8}{mj}分\end{CJK}) \begin{CJK}{UTF8}{mj}求曲线积分\end{CJK}

\end{enumerate}
$$
\int_{L} x^{2}-y z \mathrm{~d} s .
$$
\begin{CJK}{UTF8}{mj}其中\end{CJK} $L$ \begin{CJK}{UTF8}{mj}为平面\end{CJK} $x+y+z=0$ \begin{CJK}{UTF8}{mj}与球面\end{CJK} $x^{2}+y^{2}+z^{2}=1$ \begin{CJK}{UTF8}{mj}的交线\end{CJK}.

\begin{enumerate}
  \setcounter{enumi}{3}
  \item ( 15 \begin{CJK}{UTF8}{mj}分\end{CJK}) \begin{CJK}{UTF8}{mj}求幂级数\end{CJK}
\end{enumerate}
$$
\sum_{n=0}^{\infty} \frac{1}{2 n+1}\left(\frac{x}{2+x}\right)^{2 n+1}
$$
\begin{CJK}{UTF8}{mj}的收敛域与和函数\end{CJK}.

\begin{enumerate}
  \setcounter{enumi}{4}
  \item (20 \begin{CJK}{UTF8}{mj}分\end{CJK}) \begin{CJK}{UTF8}{mj}求二元函数\end{CJK}
\end{enumerate}
$$
f(x, y)=9 x^{2}+6 x y+4 y^{2}-12 y
$$
\begin{CJK}{UTF8}{mj}在闭区域\end{CJK} $D=\left\{(x, y) \mid 9 x^{2}+4 y^{2} \leq 36\right\}$ \begin{CJK}{UTF8}{mj}内的最大值\end{CJK}.

\begin{enumerate}
  \setcounter{enumi}{5}
  \item ( 15 \begin{CJK}{UTF8}{mj}分\end{CJK}) \begin{CJK}{UTF8}{mj}已知函数列\end{CJK} $f_{n}(x)$ \begin{CJK}{UTF8}{mj}在区间\end{CJK} $I$ \begin{CJK}{UTF8}{mj}上一致连续\end{CJK}, \begin{CJK}{UTF8}{mj}且\end{CJK} $f_{n}(x)$ \begin{CJK}{UTF8}{mj}一致收敛到函数\end{CJK} $f(x)$, \begin{CJK}{UTF8}{mj}证明\end{CJK} $f(x)$ \begin{CJK}{UTF8}{mj}在区间\end{CJK} $I$ \begin{CJK}{UTF8}{mj}上\end{CJK} \begin{CJK}{UTF8}{mj}也一致连续\end{CJK}.

  \item ( 15 \begin{CJK}{UTF8}{mj}分\end{CJK}) $f(x)$ \begin{CJK}{UTF8}{mj}在\end{CJK} $[0,+\infty)$ \begin{CJK}{UTF8}{mj}上非负\end{CJK}, \begin{CJK}{UTF8}{mj}对任意的\end{CJK} $A>0, x f(x)$ \begin{CJK}{UTF8}{mj}在\end{CJK} $[0, A]$ \begin{CJK}{UTF8}{mj}可积\end{CJK}, \begin{CJK}{UTF8}{mj}且\end{CJK} $\int_{0}^{+\infty} f(x) \mathrm{d} x$ \begin{CJK}{UTF8}{mj}收敛\end{CJK}, \begin{CJK}{UTF8}{mj}证明\end{CJK}

\end{enumerate}
$$
\lim _{A \rightarrow+\infty} \frac{1}{A} \int_{0}^{A} x f(x) \mathrm{d} x=0
$$

\begin{enumerate}
  \setcounter{enumi}{7}
  \item ( 20 \begin{CJK}{UTF8}{mj}分\end{CJK}) \begin{CJK}{UTF8}{mj}求极限\end{CJK}
\end{enumerate}
$$
\lim _{x \rightarrow+\infty} x^{2}\left[\left(1+\frac{1}{1+x}\right)^{x+1}-\left(1+\frac{1}{x}\right)^{x}\right]
$$

\begin{enumerate}
  \setcounter{enumi}{8}
  \item ( 20 \begin{CJK}{UTF8}{mj}分\end{CJK}) \begin{CJK}{UTF8}{mj}设\end{CJK} $f(x, y)$ \begin{CJK}{UTF8}{mj}在区域\end{CJK} $D=\left\{(x, y) \mid x^{2}+y^{2} \leq 1\right\}$ \begin{CJK}{UTF8}{mj}上存在二阶偏导数\end{CJK}, \begin{CJK}{UTF8}{mj}且\end{CJK} $\frac{\partial^{2} f}{\partial x^{2}}+\frac{\partial^{2} f}{\partial y^{2}}=1$, \begin{CJK}{UTF8}{mj}求证\end{CJK}
\end{enumerate}
$$
\iint_{D}\left(x \frac{\partial f}{\partial x}+y \frac{\partial f}{\partial y}\right)=\frac{\pi}{4}
$$

\begin{enumerate}
  \setcounter{enumi}{9}
  \item ( 10 \begin{CJK}{UTF8}{mj}分\end{CJK}) \begin{CJK}{UTF8}{mj}已知定义在\end{CJK} $[0,+\infty)$ \begin{CJK}{UTF8}{mj}上的函数\end{CJK} $f(x)$ \begin{CJK}{UTF8}{mj}满足在\end{CJK} $[0,1]$ \begin{CJK}{UTF8}{mj}连续\end{CJK}, \begin{CJK}{UTF8}{mj}在\end{CJK} $(0,1)$ \begin{CJK}{UTF8}{mj}可导\end{CJK}, \begin{CJK}{UTF8}{mj}且\end{CJK} $f(x)=f(x+1)$, $f(0)=0, f^{\prime}(x)$ \begin{CJK}{UTF8}{mj}在\end{CJK} $(0,1)$ \begin{CJK}{UTF8}{mj}上单调递减\end{CJK}, \begin{CJK}{UTF8}{mj}证明对任意的\end{CJK} $x \in[0,+\infty)$ \begin{CJK}{UTF8}{mj}及\end{CJK} $n \in \mathbb{Z}^{+}$\begin{CJK}{UTF8}{mj}都有\end{CJK}
\end{enumerate}
$$
f(n x) \leq n f(x)
$$

\section{9. 南开大学 2017 年研究生入学考试试题数学分析 
 李扬 
 微信公众号: sxkyliyang}
\begin{enumerate}
  \item \begin{CJK}{UTF8}{mj}已知函数\end{CJK}
\end{enumerate}
$$
f(x)= \begin{cases}\frac{1-x^{2}}{1+x^{2}}, & x \leq 0 \\ \left(1+x^{2}\right)^{\sin x}, & x>0\end{cases}
$$
\begin{CJK}{UTF8}{mj}求\end{CJK} $f^{\prime}(x)$.

\begin{enumerate}
  \setcounter{enumi}{2}
  \item \begin{CJK}{UTF8}{mj}设区域\end{CJK} $D=\{(x, y) \mid 0 \leq x \leq 2 \pi, 0 \leq y \leq 2 \pi\}$, \begin{CJK}{UTF8}{mj}计算二重积分\end{CJK}
\end{enumerate}
$$
\iint_{D}|\sin (x-y)| \mathrm{d} x \mathrm{~d} y .
$$

\begin{enumerate}
  \setcounter{enumi}{3}
  \item $f(x, y)$ \begin{CJK}{UTF8}{mj}在\end{CJK} $\mathbb{R}^{2}$ \begin{CJK}{UTF8}{mj}上可微\end{CJK}, \begin{CJK}{UTF8}{mj}且对任意的实数\end{CJK} $t$ \begin{CJK}{UTF8}{mj}及\end{CJK} $x, y$ \begin{CJK}{UTF8}{mj}都有\end{CJK} $f(t x, t y)=t^{2} f(x, y), P_{0}(1,-2,8)$ \begin{CJK}{UTF8}{mj}是曲面\end{CJK} $z=$ $f(x, y)$ \begin{CJK}{UTF8}{mj}上一点\end{CJK}, $f_{y}^{\prime}(1,-2)=-5$, \begin{CJK}{UTF8}{mj}求曲面\end{CJK} $z=f(x, y)$ \begin{CJK}{UTF8}{mj}在点\end{CJK} $P_{0}$ \begin{CJK}{UTF8}{mj}处的切平面方程\end{CJK}.

  \item \begin{CJK}{UTF8}{mj}计算曲线积分\end{CJK}

\end{enumerate}
$$
\int_{L} \frac{y \mathrm{~d} x-x \mathrm{~d} y}{3 x^{2}+4 y^{2}}
$$
\begin{CJK}{UTF8}{mj}其中\end{CJK} $L$ \begin{CJK}{UTF8}{mj}为星形线\end{CJK} $x^{\frac{3}{2}}+y^{\frac{3}{2}}=1$, \begin{CJK}{UTF8}{mj}方向取正向\end{CJK}.

\begin{enumerate}
  \setcounter{enumi}{5}
  \item \begin{CJK}{UTF8}{mj}设函数\end{CJK} $f(x)$ \begin{CJK}{UTF8}{mj}在\end{CJK} $(0, \delta)$ \begin{CJK}{UTF8}{mj}上可导\end{CJK}, \begin{CJK}{UTF8}{mj}且\end{CJK} $\lim _{x \rightarrow 0^{+}} f(x)$ \begin{CJK}{UTF8}{mj}与\end{CJK} $\lim _{x \rightarrow 0^{+}} x f^{\prime}(x)$ \begin{CJK}{UTF8}{mj}都存在且有限\end{CJK}, \begin{CJK}{UTF8}{mj}证明\end{CJK} $\lim _{x \rightarrow 0^{+}} x f^{\prime}(x)=0$.

  \item \begin{CJK}{UTF8}{mj}设级数\end{CJK} $\sum_{n=1}^{\infty} a_{n}$ \begin{CJK}{UTF8}{mj}的部分和有界\end{CJK}, \begin{CJK}{UTF8}{mj}级数\end{CJK} $\sum_{n=1}^{\infty}\left|b_{n+1}-b_{n}\right|$ \begin{CJK}{UTF8}{mj}收敛\end{CJK}, \begin{CJK}{UTF8}{mj}且\end{CJK} $\lim _{n \rightarrow \infty} b_{n}=0$, \begin{CJK}{UTF8}{mj}证明对任意的正整数\end{CJK} $k$, \begin{CJK}{UTF8}{mj}都有\end{CJK} $\sum_{n=1}^{\infty} a_{n} b_{n}^{k}$ \begin{CJK}{UTF8}{mj}收敛\end{CJK}.

  \item \begin{CJK}{UTF8}{mj}证明函数项级数\end{CJK}

\end{enumerate}
$$
\sum_{n=1}^{\infty} \frac{x \sin \left(n^{2} x\right)}{n^{2}}
$$
\begin{CJK}{UTF8}{mj}在\end{CJK} $(-\infty,+\infty)$ \begin{CJK}{UTF8}{mj}内闭一致收敛但非一致收敛\end{CJK}.

\begin{enumerate}
  \setcounter{enumi}{8}
  \item \begin{CJK}{UTF8}{mj}设\end{CJK} $p$ \begin{CJK}{UTF8}{mj}为实数\end{CJK}, \begin{CJK}{UTF8}{mj}讨论反常积分\end{CJK}
\end{enumerate}
$$
\int_{0}^{+\infty}\left[\ln \left(1+\frac{1}{x}\right)+\frac{p}{\sqrt{4 x^{2}+7 x+1}}\right] d x
$$
\begin{CJK}{UTF8}{mj}的收敛性\end{CJK}.

\begin{enumerate}
  \setcounter{enumi}{9}
  \item \begin{CJK}{UTF8}{mj}已知数列\end{CJK}
\end{enumerate}
$$
x_{n}=\sum_{k=1}^{n} \frac{k \sin ^{2} k}{n^{2}+k \sin ^{2} k}, \quad n=1,2, \cdots
$$
\begin{CJK}{UTF8}{mj}证明\end{CJK} $\left\{x_{n}\right\}$ \begin{CJK}{UTF8}{mj}收敛\end{CJK}.

\section{0. 南开大学 2018 年研究生入学考试试题数学分析}
\begin{CJK}{UTF8}{mj}李扬\end{CJK}

\begin{CJK}{UTF8}{mj}微信公众号\end{CJK}: sxkyliyang

\begin{enumerate}
  \item \begin{CJK}{UTF8}{mj}求\end{CJK} $f(x)=4 \ln x+x^{2}-6 x$ \begin{CJK}{UTF8}{mj}的极值\end{CJK}.

  \item \begin{CJK}{UTF8}{mj}已知区域\end{CJK} $D=\{(x, y) \mid x \geq 0, y \geq 0, x+2 y \leq 1\}$, \begin{CJK}{UTF8}{mj}求二重积分\end{CJK}

\end{enumerate}
$$
\iint_{D} e^{x+2 y} \mathrm{~d} x \mathrm{~d} y
$$

\begin{enumerate}
  \setcounter{enumi}{3}
  \item \begin{CJK}{UTF8}{mj}已知二元函数\end{CJK} $f(x, y)=x \ln \left(x+\sqrt{x^{2}+y^{2}}\right)-\sqrt{x^{2}+y^{2}}$, \begin{CJK}{UTF8}{mj}证明\end{CJK}
\end{enumerate}
$$
\frac{\partial^{2} f}{\partial x^{2}}+\frac{\partial^{2} f}{\partial y^{2}}=\frac{1}{x+\sqrt{x^{2}+y^{2}}}
$$

\begin{enumerate}
  \setcounter{enumi}{4}
  \item \begin{CJK}{UTF8}{mj}求幂级数\end{CJK}
\end{enumerate}
$$
\sum_{i=1}^{\infty} \frac{(-1)^{n-1}}{n} x^{2 n}
$$
\begin{CJK}{UTF8}{mj}的收敛区间与和函数\end{CJK}.

\begin{enumerate}
  \setcounter{enumi}{5}
  \item \begin{CJK}{UTF8}{mj}已知周期为\end{CJK} $2 \pi$ \begin{CJK}{UTF8}{mj}的函数\end{CJK} $f(x)=\pi-|x|, x \in[-\pi, \pi]$.
\end{enumerate}
(1) \begin{CJK}{UTF8}{mj}求\end{CJK} $f(x)$ \begin{CJK}{UTF8}{mj}的傅里叶级数\end{CJK}.

(2) \begin{CJK}{UTF8}{mj}求级数\end{CJK} $\sum_{n=1}^{\infty} \frac{1}{(2 n-1)^{2}}$.

\begin{enumerate}
  \setcounter{enumi}{6}
  \item \begin{CJK}{UTF8}{mj}求曲线积分\end{CJK}
\end{enumerate}
$$
\int_{L}(z-y) \mathrm{d} x+(x-z) \mathrm{d} y+(y-x) \mathrm{d} z .
$$
$L$ \begin{CJK}{UTF8}{mj}为球面\end{CJK} $x^{2}+y^{2}+z^{2}=1$ \begin{CJK}{UTF8}{mj}与平面\end{CJK} $x+y+z=0$ \begin{CJK}{UTF8}{mj}的交线\end{CJK}, \begin{CJK}{UTF8}{mj}从\end{CJK} $x$ \begin{CJK}{UTF8}{mj}轴正向看\end{CJK}, $L$ \begin{CJK}{UTF8}{mj}为逆时针方向\end{CJK}.

\begin{enumerate}
  \setcounter{enumi}{7}
  \item $f(x)$ \begin{CJK}{UTF8}{mj}在\end{CJK} $[-2,2]$ \begin{CJK}{UTF8}{mj}上两次连续可导\end{CJK}, $f(-2)=f(2)$, \begin{CJK}{UTF8}{mj}且\end{CJK} $\left|f^{\prime \prime}(x)\right| \leq M$, \begin{CJK}{UTF8}{mj}证明\end{CJK} $\left|f^{\prime}(0)\right| \leq M$.

  \item $f(x)$ \begin{CJK}{UTF8}{mj}在\end{CJK} $(0,+\infty)$ \begin{CJK}{UTF8}{mj}上一致连续\end{CJK}, \begin{CJK}{UTF8}{mj}证明存在常数\end{CJK} $M>0$, \begin{CJK}{UTF8}{mj}使得对任何\end{CJK} $x>0$, \begin{CJK}{UTF8}{mj}任何\end{CJK} $h>0$, \begin{CJK}{UTF8}{mj}都有\end{CJK}

\end{enumerate}
$$
|f(x+h)-f(x)| \leq M(h+1)
$$

\begin{enumerate}
  \setcounter{enumi}{9}
  \item \begin{CJK}{UTF8}{mj}函数\end{CJK} $f(x)$ \begin{CJK}{UTF8}{mj}在\end{CJK} $[a,+\infty)$ \begin{CJK}{UTF8}{mj}上可导\end{CJK}, $g(x)$ \begin{CJK}{UTF8}{mj}在\end{CJK} $[a,+\infty)$ \begin{CJK}{UTF8}{mj}上连续且恒大于\end{CJK} 0 , \begin{CJK}{UTF8}{mj}且广义积分\end{CJK} $\int_{a}^{+\infty} g(x) \mathrm{d} x$ \begin{CJK}{UTF8}{mj}发散\end{CJK}, \begin{CJK}{UTF8}{mj}已知\end{CJK}
\end{enumerate}
$$
\lim _{x \rightarrow+\infty}\left[f(x)+\frac{f^{\prime}(x)}{g(x)}\right]=0
$$
\begin{CJK}{UTF8}{mj}证明\end{CJK}
$$
\lim _{x \rightarrow+\infty} f(x)=0 .
$$

\section{1. 厦门大学 2009 年研究生入学考试试题数学分析}
\begin{CJK}{UTF8}{mj}李扬\end{CJK}

\begin{CJK}{UTF8}{mj}微信公众号\end{CJK}: sxkyliyang

\begin{CJK}{UTF8}{mj}一\end{CJK}、\begin{CJK}{UTF8}{mj}选择题\end{CJK}(\begin{CJK}{UTF8}{mj}本题共\end{CJK} 30 \begin{CJK}{UTF8}{mj}分\end{CJK}, \begin{CJK}{UTF8}{mj}每小题\end{CJK} 6 \begin{CJK}{UTF8}{mj}分\end{CJK})

\begin{enumerate}
  \item \begin{CJK}{UTF8}{mj}设\end{CJK} $\left\{a_{n}\right\}$ \begin{CJK}{UTF8}{mj}为单调数列\end{CJK}, \begin{CJK}{UTF8}{mj}若存在一收敛子列\end{CJK} $\left\{a_{n_{j}}\right\}$, \begin{CJK}{UTF8}{mj}这时有\end{CJK} ( ).\\
A. $\lim _{n \rightarrow \infty} a_{n}=\lim _{j \rightarrow \infty} a_{n_{j}}$;\\
B. $\left\{a_{n}\right\}$ \begin{CJK}{UTF8}{mj}不一定收敛\end{CJK};\\
C. $\left\{a_{n}\right\}$ \begin{CJK}{UTF8}{mj}不一定有界\end{CJK};\\
D. \begin{CJK}{UTF8}{mj}当且仅当预先假设了\end{CJK} $\left\{a_{n}\right\}$ \begin{CJK}{UTF8}{mj}为有界数列时\end{CJK}, \begin{CJK}{UTF8}{mj}才有\end{CJK} $\mathrm{A}$ \begin{CJK}{UTF8}{mj}成立\end{CJK}.

  \item \begin{CJK}{UTF8}{mj}设\end{CJK} $f(x)$ \begin{CJK}{UTF8}{mj}为\end{CJK} $\mathbb{R}$ \begin{CJK}{UTF8}{mj}上的连续函数\end{CJK}, \begin{CJK}{UTF8}{mj}则有\end{CJK} ( ).\\
A. \begin{CJK}{UTF8}{mj}当\end{CJK} $I$ \begin{CJK}{UTF8}{mj}开区间时\end{CJK}, $f(I)$ \begin{CJK}{UTF8}{mj}必为开区间\end{CJK};\\
B. \begin{CJK}{UTF8}{mj}当\end{CJK} $I$ \begin{CJK}{UTF8}{mj}闭区间时\end{CJK}, $f(I)$ \begin{CJK}{UTF8}{mj}必为闭区间\end{CJK};\\
C. \begin{CJK}{UTF8}{mj}当\end{CJK} $f(I)$ \begin{CJK}{UTF8}{mj}开区间时\end{CJK}, $I$ \begin{CJK}{UTF8}{mj}必为开区间\end{CJK};\\
D. \begin{CJK}{UTF8}{mj}以上\end{CJK} $A, B, C$ \begin{CJK}{UTF8}{mj}都不一定成立\end{CJK}.

  \item \begin{CJK}{UTF8}{mj}设\end{CJK} $f(x)$ \begin{CJK}{UTF8}{mj}在某去心邻域\end{CJK} $U^{\circ}\left(x_{0}\right)$ \begin{CJK}{UTF8}{mj}内可导\end{CJK}, \begin{CJK}{UTF8}{mj}这时有\end{CJK} ( ).\\
A. \begin{CJK}{UTF8}{mj}若\end{CJK} $\lim _{x \rightarrow x_{0}} f^{\prime}(x)=A$ \begin{CJK}{UTF8}{mj}存在\end{CJK}, \begin{CJK}{UTF8}{mj}则\end{CJK} $f^{\prime}\left(x_{0}\right)=A$;\\
B. \begin{CJK}{UTF8}{mj}若\end{CJK} $f$ \begin{CJK}{UTF8}{mj}在\end{CJK} $x_{0}$ \begin{CJK}{UTF8}{mj}连续\end{CJK}, \begin{CJK}{UTF8}{mj}则\end{CJK} $\mathrm{A}$ \begin{CJK}{UTF8}{mj}成立\end{CJK};\\
C. \begin{CJK}{UTF8}{mj}若\end{CJK} $f^{\prime}\left(x_{0}\right)=A$ \begin{CJK}{UTF8}{mj}存在\end{CJK}, \begin{CJK}{UTF8}{mj}则\end{CJK} $\lim _{x \rightarrow x_{0}} f^{\prime}(x)=A$.\\
D. \begin{CJK}{UTF8}{mj}以上\end{CJK} $A, B, C$ \begin{CJK}{UTF8}{mj}都不一定成立\end{CJK}.

  \item \begin{CJK}{UTF8}{mj}设\end{CJK} $f(x)$ \begin{CJK}{UTF8}{mj}在\end{CJK} $[a, b]$ \begin{CJK}{UTF8}{mj}上\end{CJK}Riemann \begin{CJK}{UTF8}{mj}可积\end{CJK}, \begin{CJK}{UTF8}{mj}则有\end{CJK} ( ).\\
A. $f(x)$ \begin{CJK}{UTF8}{mj}在\end{CJK} $[a, b]$ \begin{CJK}{UTF8}{mj}上必定连续\end{CJK};\\
B. $f(x)$ \begin{CJK}{UTF8}{mj}在\end{CJK} $[a, b]$ \begin{CJK}{UTF8}{mj}上至多只有有限个间断点\end{CJK};\\
C. $f(x)$ \begin{CJK}{UTF8}{mj}可以是无界函数\end{CJK};\\
D. $f(x)$ \begin{CJK}{UTF8}{mj}在\end{CJK} $[a, b]$ \begin{CJK}{UTF8}{mj}上的连续点必定稠密\end{CJK}.

\end{enumerate}
5 . \begin{CJK}{UTF8}{mj}设\end{CJK} $\sum_{n=1}^{\infty} u_{n}$ \begin{CJK}{UTF8}{mj}为一正项级数\end{CJK}, \begin{CJK}{UTF8}{mj}这时有\end{CJK} ( ).\\
A. \begin{CJK}{UTF8}{mj}若\end{CJK} $\lim _{n \rightarrow \infty} u_{n}=0$, \begin{CJK}{UTF8}{mj}则\end{CJK} $\sum_{n=1}^{\infty} u_{n}$ \begin{CJK}{UTF8}{mj}收敛\end{CJK};\\
B. \begin{CJK}{UTF8}{mj}若\end{CJK} $\sum_{n=1}^{\infty} u_{n}$ \begin{CJK}{UTF8}{mj}收敛\end{CJK}, \begin{CJK}{UTF8}{mj}则\end{CJK} $\lim _{n \rightarrow \infty} \frac{u_{n+1}}{u_{n}}<1$;\\
C. \begin{CJK}{UTF8}{mj}若\end{CJK} $\sum_{n=1}^{\infty} u_{n}$ \begin{CJK}{UTF8}{mj}收敛\end{CJK}, \begin{CJK}{UTF8}{mj}则\end{CJK} $\lim _{n \rightarrow \infty} \sqrt[n]{u_{n}}<1$;\\
D. \begin{CJK}{UTF8}{mj}以上\end{CJK} $A, B, C$ \begin{CJK}{UTF8}{mj}都不一定成立\end{CJK}.

\section{一、解答题}
\begin{enumerate}
  \item ( 15 \begin{CJK}{UTF8}{mj}分\end{CJK}) \begin{CJK}{UTF8}{mj}设\end{CJK} $a>0, \sigma>0, a_{1}=\frac{1}{2}\left(a+\frac{\sigma}{a}\right), a_{n+1}=\frac{1}{2}\left(a_{n}+\frac{\sigma}{a_{n}}\right), n=1,2, \cdots$. \begin{CJK}{UTF8}{mj}证明\end{CJK}: \begin{CJK}{UTF8}{mj}数列\end{CJK} $\left\{a_{n}\right\}$ \begin{CJK}{UTF8}{mj}收敛\end{CJK}, \begin{CJK}{UTF8}{mj}并求\end{CJK} \begin{CJK}{UTF8}{mj}其极限\end{CJK}. 2. ( 15 \begin{CJK}{UTF8}{mj}分\end{CJK}) \begin{CJK}{UTF8}{mj}设函数\end{CJK} $f(x)$ \begin{CJK}{UTF8}{mj}在区间\end{CJK} $[a,+\infty)$ \begin{CJK}{UTF8}{mj}上连续\end{CJK}, $g(x)$ \begin{CJK}{UTF8}{mj}在区间\end{CJK} $[a,+\infty)$ \begin{CJK}{UTF8}{mj}上一致连续\end{CJK}, \begin{CJK}{UTF8}{mj}并且\end{CJK}
\end{enumerate}
$$
\lim _{x \rightarrow \infty}(f(x)-g(x))=0
$$
\begin{CJK}{UTF8}{mj}证明\end{CJK} $f(x)$ \begin{CJK}{UTF8}{mj}在区间\end{CJK} $[a,+\infty)$ \begin{CJK}{UTF8}{mj}上一致连续\end{CJK}.

\begin{enumerate}
  \setcounter{enumi}{3}
  \item ( 20 \begin{CJK}{UTF8}{mj}分\end{CJK}) \begin{CJK}{UTF8}{mj}设\end{CJK} $f(x)$ \begin{CJK}{UTF8}{mj}在\end{CJK} $(-\infty,+\infty)$ \begin{CJK}{UTF8}{mj}上有连续的二阶导数\end{CJK}, \begin{CJK}{UTF8}{mj}且\end{CJK} $f(0)=f^{\prime}(0)=0$, \begin{CJK}{UTF8}{mj}定义\end{CJK}
\end{enumerate}
$$
g(x)= \begin{cases}\frac{f(x)}{x}, & x \neq 0 \\ 0, & x=0\end{cases}
$$
\begin{CJK}{UTF8}{mj}证明\end{CJK} $g(x)$ \begin{CJK}{UTF8}{mj}在\end{CJK} $(-\infty,+\infty)$ \begin{CJK}{UTF8}{mj}上有连续的导数\end{CJK}.

\begin{enumerate}
  \setcounter{enumi}{4}
  \item (15 \begin{CJK}{UTF8}{mj}分\end{CJK}) \begin{CJK}{UTF8}{mj}求积分\end{CJK}
\end{enumerate}
$$
\iint_{S} x y^{2} \mathrm{~d} y \mathrm{~d} z+y z^{2} \mathrm{~d} z \mathrm{~d} x+z x^{2} \mathrm{~d} x \mathrm{~d} y
$$
\begin{CJK}{UTF8}{mj}其中\end{CJK} $S$ \begin{CJK}{UTF8}{mj}为圆柱面\end{CJK} $x^{2}+y^{2}=1,-1 \leq z \leq 1$ \begin{CJK}{UTF8}{mj}部分的外侧\end{CJK}.

\begin{enumerate}
  \setcounter{enumi}{5}
  \item ( 15 \begin{CJK}{UTF8}{mj}分\end{CJK}) \begin{CJK}{UTF8}{mj}证明\end{CJK}:
\end{enumerate}
(1) \begin{CJK}{UTF8}{mj}级数\end{CJK}
$$
\sum_{n=1}^{\infty} \frac{\sin ^{n} x}{2^{n}}
$$
\begin{CJK}{UTF8}{mj}在\end{CJK} $(-\infty,+\infty)$ \begin{CJK}{UTF8}{mj}一致收敛\end{CJK};

(2) \begin{CJK}{UTF8}{mj}存在\end{CJK} $\varepsilon \in\left(0, \frac{\pi}{2}\right)$, \begin{CJK}{UTF8}{mj}使得\end{CJK}
$$
\sum_{n=1}^{\infty} \frac{n \cos \varepsilon \sin ^{n-1} \varepsilon}{2^{n}}=\frac{2}{\pi}
$$

\section{2. 厦门大学 2010 年研究生入学考试试题数学分析}
\begin{CJK}{UTF8}{mj}李扬\end{CJK}

\begin{CJK}{UTF8}{mj}微信公众号\end{CJK}: sxkyliyang

\begin{CJK}{UTF8}{mj}一\end{CJK}、\begin{CJK}{UTF8}{mj}选择题\end{CJK}(\begin{CJK}{UTF8}{mj}本题共\end{CJK} 30 \begin{CJK}{UTF8}{mj}分\end{CJK}, \begin{CJK}{UTF8}{mj}每小题\end{CJK} 6 \begin{CJK}{UTF8}{mj}分\end{CJK})

\begin{enumerate}
  \item \begin{CJK}{UTF8}{mj}设函数\end{CJK} $f(x)$ \begin{CJK}{UTF8}{mj}二阶可导\end{CJK}, \begin{CJK}{UTF8}{mj}并且满足方程\end{CJK} $f^{\prime \prime}(x)+3\left[f^{\prime}(x)\right]^{2}+2 \mathrm{e}^{x} f(x)=0$, \begin{CJK}{UTF8}{mj}设\end{CJK} $x_{0}$ \begin{CJK}{UTF8}{mj}为\end{CJK} $f(x)$ \begin{CJK}{UTF8}{mj}的一个驻点并且满足\end{CJK} $f\left(x_{0}\right)<0$, \begin{CJK}{UTF8}{mj}则\end{CJK} $f(x)$ \begin{CJK}{UTF8}{mj}在\end{CJK} $x_{0}$ \begin{CJK}{UTF8}{mj}有\end{CJK} $(\quad)$.\\
A. \begin{CJK}{UTF8}{mj}取极大值\end{CJK};\\
B. \begin{CJK}{UTF8}{mj}取极小值\end{CJK};\\
C. \begin{CJK}{UTF8}{mj}不取极值\end{CJK};\\
D. \begin{CJK}{UTF8}{mj}不能确定\end{CJK}.

  \item \begin{CJK}{UTF8}{mj}函数\end{CJK} $f(x)=\ln x-\frac{x}{\mathrm{e}}+k,(k>0)$ \begin{CJK}{UTF8}{mj}在区间\end{CJK} $(0,+\infty)$ \begin{CJK}{UTF8}{mj}内零点的个数为\end{CJK} $(\quad)$.\\
A. 0\\
B. 1\\
C. 2\\
D. \begin{CJK}{UTF8}{mj}不能确定\end{CJK}

  \item \begin{CJK}{UTF8}{mj}已知当\end{CJK} $x \rightarrow 0$ \begin{CJK}{UTF8}{mj}时\end{CJK}, \begin{CJK}{UTF8}{mj}函数\end{CJK} $\mathrm{e}^{\tan x}-\mathrm{e}^{x}$ \begin{CJK}{UTF8}{mj}与\end{CJK} $x^{n}$ \begin{CJK}{UTF8}{mj}为同价无穷小量\end{CJK}, \begin{CJK}{UTF8}{mj}则\end{CJK} $n=(\quad)$.\\
A. 1\\
B. 2\\
C. 3\\
D. 4

  \item \begin{CJK}{UTF8}{mj}下列命题正确的是\end{CJK}( $)$.\\
A. \begin{CJK}{UTF8}{mj}如果\end{CJK} $f(x)$ \begin{CJK}{UTF8}{mj}在\end{CJK} $[a, b]$ \begin{CJK}{UTF8}{mj}上\end{CJK} Riemann \begin{CJK}{UTF8}{mj}可积\end{CJK}, \begin{CJK}{UTF8}{mj}且有原函数\end{CJK} $F(x)$, \begin{CJK}{UTF8}{mj}则\end{CJK} $\int_{a}^{b} f(x) \mathrm{d} x=F(b)-F(a)$;\\
B. \begin{CJK}{UTF8}{mj}如果\end{CJK} $f(x)$ \begin{CJK}{UTF8}{mj}在\end{CJK} $[a, b]$ \begin{CJK}{UTF8}{mj}上\end{CJK} Riemann \begin{CJK}{UTF8}{mj}可积\end{CJK}, \begin{CJK}{UTF8}{mj}则\end{CJK} $\int_{a}^{x} f(t) \mathrm{d} t$ \begin{CJK}{UTF8}{mj}在\end{CJK} $[a, b]$ \begin{CJK}{UTF8}{mj}上可导\end{CJK};\\
C. \begin{CJK}{UTF8}{mj}如果\end{CJK} $f^{2}(x)$ \begin{CJK}{UTF8}{mj}在\end{CJK} $[a, b]$ \begin{CJK}{UTF8}{mj}上\end{CJK} Riemann \begin{CJK}{UTF8}{mj}可积\end{CJK}, \begin{CJK}{UTF8}{mj}则\end{CJK} $|f(x)|$ \begin{CJK}{UTF8}{mj}不一定在\end{CJK} $[a, b]$ \begin{CJK}{UTF8}{mj}上\end{CJK} Riemann \begin{CJK}{UTF8}{mj}可积\end{CJK};\\
D. \begin{CJK}{UTF8}{mj}如果\end{CJK} $|f(x)|$ \begin{CJK}{UTF8}{mj}在\end{CJK} $[a, b]$ \begin{CJK}{UTF8}{mj}上\end{CJK} Riemann \begin{CJK}{UTF8}{mj}可积\end{CJK}, \begin{CJK}{UTF8}{mj}则\end{CJK} $f(x)$ \begin{CJK}{UTF8}{mj}在\end{CJK} $[a, b]$ \begin{CJK}{UTF8}{mj}上\end{CJK} Riemann \begin{CJK}{UTF8}{mj}可积\end{CJK},.

\end{enumerate}
5 . \begin{CJK}{UTF8}{mj}设\end{CJK} $a>0, f(x)$ \begin{CJK}{UTF8}{mj}在\end{CJK} $(-a, a)$ \begin{CJK}{UTF8}{mj}内满足\end{CJK} $f^{\prime \prime}(x)>0$, \begin{CJK}{UTF8}{mj}且\end{CJK} $|f(x)| \leq x^{2}, I=\int_{-a}^{a} f(x) \mathrm{d} x$, \begin{CJK}{UTF8}{mj}则必有\end{CJK} ( )\\
A. $I=0$\\
B. $I>0$\\
C. $I<0$\\
D. \begin{CJK}{UTF8}{mj}不确定\end{CJK}

\section{一、解答题}
\begin{enumerate}
  \item (10 \begin{CJK}{UTF8}{mj}分\end{CJK}) \begin{CJK}{UTF8}{mj}设\end{CJK} $f(x)=a_{1} \sin (x)+a_{2} \sin (2 x)+\cdots+a_{n} \sin (n x)$, \begin{CJK}{UTF8}{mj}且\end{CJK} $|f(x)| \leq|\sin x|, a_{i}(i=1,2, \cdots, n)$ \begin{CJK}{UTF8}{mj}为实常数\end{CJK}, \begin{CJK}{UTF8}{mj}证明\end{CJK}
\end{enumerate}
$$
\left|a_{1}+2 a_{2}+\cdots+n a_{n}\right| \leq 1
$$

\begin{enumerate}
  \setcounter{enumi}{2}
  \item ( 15 \begin{CJK}{UTF8}{mj}分\end{CJK}) \begin{CJK}{UTF8}{mj}设函数\end{CJK} $f(x)$ \begin{CJK}{UTF8}{mj}在区间\end{CJK} $(-\infty,+\infty)$ \begin{CJK}{UTF8}{mj}上一致连续\end{CJK}, $\eta>0$. \begin{CJK}{UTF8}{mj}在\end{CJK} $(-\infty,+\infty)$ \begin{CJK}{UTF8}{mj}上通过下式定义函数\end{CJK}
\end{enumerate}
$$
g(x)=\sup \{|f(y)-f(z)| y, z \in(x-\eta, x+\eta)\}
$$
\begin{CJK}{UTF8}{mj}证明\end{CJK} $g$ \begin{CJK}{UTF8}{mj}在区间\end{CJK} $(-\infty,+\infty)$ \begin{CJK}{UTF8}{mj}上一致连续\end{CJK}.

\begin{enumerate}
  \setcounter{enumi}{3}
  \item ( 15 \begin{CJK}{UTF8}{mj}分\end{CJK}) \begin{CJK}{UTF8}{mj}设函数\end{CJK} $\varphi$ \begin{CJK}{UTF8}{mj}在区间\end{CJK} $[0, a]$ \begin{CJK}{UTF8}{mj}上连续\end{CJK}, \begin{CJK}{UTF8}{mj}函数\end{CJK} $f$ \begin{CJK}{UTF8}{mj}在\end{CJK} $(-\infty,+\infty)$ \begin{CJK}{UTF8}{mj}上二阶可导\end{CJK}, \begin{CJK}{UTF8}{mj}且\end{CJK} $f^{\prime \prime}(x) \geq 0, \forall x \in(-\infty,+\infty)$. \begin{CJK}{UTF8}{mj}证明\end{CJK}
\end{enumerate}
$$
f\left(\frac{1}{a} \int_{0}^{a} \varphi(t) \mathrm{d} t\right) \leq \frac{1}{a} \int_{0}^{a} f[\varphi(t)] \mathrm{d} t
$$

\begin{enumerate}
  \setcounter{enumi}{4}
  \item ( 10 \begin{CJK}{UTF8}{mj}分\end{CJK}) \begin{CJK}{UTF8}{mj}设\end{CJK} $f_{0}(x)$ \begin{CJK}{UTF8}{mj}在区间\end{CJK} $[0, a]$ \begin{CJK}{UTF8}{mj}上连续\end{CJK}, \begin{CJK}{UTF8}{mj}令\end{CJK}
\end{enumerate}
$$
f_{n}(x)=\int_{0}^{x} f_{n-1}(t) \mathrm{d} t, n=1,2, \cdots
$$
\begin{CJK}{UTF8}{mj}证明\end{CJK} $\left\{f_{n}(x)\right\}$ \begin{CJK}{UTF8}{mj}在区间\end{CJK} $[0, a]$ \begin{CJK}{UTF8}{mj}上一致收敛于\end{CJK} 0 . 5. ( 15 \begin{CJK}{UTF8}{mj}分\end{CJK}) \begin{CJK}{UTF8}{mj}求函数\end{CJK}
$$
f(x, y)=a x^{2}+2 b x y+c y^{2}
$$
\begin{CJK}{UTF8}{mj}在\end{CJK} $x^{2}+y^{2} \leq 1$ \begin{CJK}{UTF8}{mj}上的最大值和最小值\end{CJK}.

\begin{enumerate}
  \setcounter{enumi}{6}
  \item (15 \begin{CJK}{UTF8}{mj}分\end{CJK}) \begin{CJK}{UTF8}{mj}设函数\end{CJK} $z=z(x, y)$ \begin{CJK}{UTF8}{mj}由方程\end{CJK} $F\left(x+\frac{y}{z}, y+\frac{z}{x}\right)=0$ \begin{CJK}{UTF8}{mj}所确定\end{CJK}, \begin{CJK}{UTF8}{mj}其中\end{CJK} $F$ \begin{CJK}{UTF8}{mj}有一阶连续的偏导数\end{CJK}, \begin{CJK}{UTF8}{mj}证明\end{CJK}
\end{enumerate}
$$
x \frac{\partial z}{\partial x}+y \frac{\partial z}{\partial y}=z-x y
$$

\section{3. 厦门大学 2011 年研究生入学考试试题数学分析}
\begin{CJK}{UTF8}{mj}李扬\end{CJK}

\begin{CJK}{UTF8}{mj}微信公众号\end{CJK}: sxkyliyang

\begin{CJK}{UTF8}{mj}一\end{CJK}、\begin{CJK}{UTF8}{mj}选择题\end{CJK}(\begin{CJK}{UTF8}{mj}本题共\end{CJK} 30 \begin{CJK}{UTF8}{mj}分\end{CJK}, \begin{CJK}{UTF8}{mj}每小题\end{CJK} 6 \begin{CJK}{UTF8}{mj}分\end{CJK})

\begin{enumerate}
  \item \begin{CJK}{UTF8}{mj}函数\end{CJK} $y=f(x)$ \begin{CJK}{UTF8}{mj}在点\end{CJK} $x_{0}$ \begin{CJK}{UTF8}{mj}的某个邻域内具有连续的二阶导数\end{CJK}, \begin{CJK}{UTF8}{mj}满足\end{CJK} $f^{\prime}\left(x_{0}\right)=0$, \begin{CJK}{UTF8}{mj}并且\end{CJK} $f^{\prime \prime}\left(x_{0}\right)<0$, \begin{CJK}{UTF8}{mj}则\end{CJK} $(\quad)$.\\
A. $f(x)$ \begin{CJK}{UTF8}{mj}在点\end{CJK} $x_{0}$ \begin{CJK}{UTF8}{mj}处取极大值\end{CJK};\\
B. $f(x)$ \begin{CJK}{UTF8}{mj}在点\end{CJK} $x_{0}$ \begin{CJK}{UTF8}{mj}处取极小值\end{CJK};\\
C. $\left(x_{0}, f\left(x_{0}\right)\right)$ \begin{CJK}{UTF8}{mj}为曲线\end{CJK} $y=f(x)$ \begin{CJK}{UTF8}{mj}的拐点\end{CJK};\\
D. $f(x)$ \begin{CJK}{UTF8}{mj}在\end{CJK} $x_{0}$ \begin{CJK}{UTF8}{mj}的某个邻域内单调减少\end{CJK}.

  \item \begin{CJK}{UTF8}{mj}函数\end{CJK} $f(x)=\ln x-x$ \begin{CJK}{UTF8}{mj}在区间\end{CJK} $(0,+\infty)$ \begin{CJK}{UTF8}{mj}内零点的个数为\end{CJK} ( ).\\
A. 0\\
B. 1\\
C. 2\\
D. \begin{CJK}{UTF8}{mj}不能确定\end{CJK}

  \item \begin{CJK}{UTF8}{mj}已知当\end{CJK} $x \rightarrow 0$ \begin{CJK}{UTF8}{mj}时\end{CJK}, \begin{CJK}{UTF8}{mj}函数\end{CJK} $\mathrm{e}^{\sin x}-\mathrm{e}^{x}$ \begin{CJK}{UTF8}{mj}与\end{CJK} $x^{n}$ \begin{CJK}{UTF8}{mj}为同价无穷小量\end{CJK}, \begin{CJK}{UTF8}{mj}则\end{CJK} $n=(\quad)$.\\
A. 1\\
B. 2\\
C. 3\\
D. 4

  \item \begin{CJK}{UTF8}{mj}下列命题正确的是\end{CJK} ( ).\\
A. \begin{CJK}{UTF8}{mj}如果\end{CJK} $f(x)$ \begin{CJK}{UTF8}{mj}在\end{CJK} $[a, b]$ \begin{CJK}{UTF8}{mj}上\end{CJK} Riemann \begin{CJK}{UTF8}{mj}可积\end{CJK}, \begin{CJK}{UTF8}{mj}且有原函数\end{CJK} $F(x)$, \begin{CJK}{UTF8}{mj}则\end{CJK} $\left[\int_{a}^{x} f(t) \mathrm{d} t\right]^{\prime}=f(x)$;\\
B. \begin{CJK}{UTF8}{mj}如果\end{CJK} $f(x)$ \begin{CJK}{UTF8}{mj}在\end{CJK} $[a, b]$ \begin{CJK}{UTF8}{mj}上\end{CJK} Riemann \begin{CJK}{UTF8}{mj}可积\end{CJK}, \begin{CJK}{UTF8}{mj}则\end{CJK} $|f(x)|$ \begin{CJK}{UTF8}{mj}一定在\end{CJK} $[a, b]$ \begin{CJK}{UTF8}{mj}上可积\end{CJK};\\
C. \begin{CJK}{UTF8}{mj}如果\end{CJK} $f^{2}(x)$ \begin{CJK}{UTF8}{mj}在\end{CJK} $[a, b]$ \begin{CJK}{UTF8}{mj}上\end{CJK} Riemann \begin{CJK}{UTF8}{mj}可积\end{CJK}, \begin{CJK}{UTF8}{mj}则\end{CJK} $|f(x)|$ \begin{CJK}{UTF8}{mj}一定在\end{CJK} $[a, b]$ \begin{CJK}{UTF8}{mj}上\end{CJK} Riemann \begin{CJK}{UTF8}{mj}可积\end{CJK};\\
D. \begin{CJK}{UTF8}{mj}如果\end{CJK} $|f(x)|$ \begin{CJK}{UTF8}{mj}在\end{CJK} $[a, b]$ \begin{CJK}{UTF8}{mj}上\end{CJK} Riemann \begin{CJK}{UTF8}{mj}可积\end{CJK}, \begin{CJK}{UTF8}{mj}则\end{CJK} $f(x)$ \begin{CJK}{UTF8}{mj}在\end{CJK} $[a, b]$ \begin{CJK}{UTF8}{mj}上\end{CJK} Riemann \begin{CJK}{UTF8}{mj}可积\end{CJK}, .

  \item \begin{CJK}{UTF8}{mj}设\end{CJK} $f(x)$ \begin{CJK}{UTF8}{mj}在\end{CJK} $(-1,1)$ \begin{CJK}{UTF8}{mj}内满足\end{CJK} $f^{\prime \prime}(x)>0$, \begin{CJK}{UTF8}{mj}且\end{CJK} $|f(x)| \leq x^{4}, I=\int_{-1}^{1} f(x) \mathrm{d} x$, \begin{CJK}{UTF8}{mj}则必有\end{CJK} ( ).\\
A. $I=0$\\
B. $I>0$\\
C. $I<0$\\
D. \begin{CJK}{UTF8}{mj}不确定\end{CJK}

\end{enumerate}
\section{一、解答题}
\begin{enumerate}
  \item (10 \begin{CJK}{UTF8}{mj}分\end{CJK}) $f(x)$ \begin{CJK}{UTF8}{mj}在\end{CJK} $[0,1]$ \begin{CJK}{UTF8}{mj}上二阶可导\end{CJK}, \begin{CJK}{UTF8}{mj}且有\end{CJK}
\end{enumerate}
$$
f(0)=f(1)=0, \min _{x \in[0,1]} f(x)=-1
$$
\begin{CJK}{UTF8}{mj}求证\end{CJK}: \begin{CJK}{UTF8}{mj}存在\end{CJK} $\xi \in(0,1)$, \begin{CJK}{UTF8}{mj}使得\end{CJK} $f^{\prime \prime}(\xi) \geq 8$.

\begin{enumerate}
  \setcounter{enumi}{2}
  \item (10 \begin{CJK}{UTF8}{mj}分\end{CJK})
\end{enumerate}
\begin{CJK}{UTF8}{mj}设函数\end{CJK} $f(x) \in C[0,1]$, \begin{CJK}{UTF8}{mj}证明\end{CJK}
$$
\lim _{\lambda \rightarrow+\infty} \int_{0}^{\lambda} f\left(\frac{x}{\lambda}\right) \frac{1}{1+x^{2}} \mathrm{~d} x=\frac{\pi}{2} f(0)
$$

\begin{enumerate}
  \setcounter{enumi}{3}
  \item (15 \begin{CJK}{UTF8}{mj}分\end{CJK})
\end{enumerate}
\begin{CJK}{UTF8}{mj}设\end{CJK} $f(x)$ \begin{CJK}{UTF8}{mj}在\end{CJK} $[0,1]$ \begin{CJK}{UTF8}{mj}上连续\end{CJK}, \begin{CJK}{UTF8}{mj}证明\end{CJK} $\forall t>0$, \begin{CJK}{UTF8}{mj}都有\end{CJK}
$$
\left[\int_{0}^{1} \frac{f(x)}{t^{2}+x^{2}} \mathrm{~d} x\right]^{2} \leq \frac{\pi}{2 t} \int_{0}^{1} \frac{f^{2}(x)}{t^{2}+x^{2}} \mathrm{~d} x
$$

\begin{enumerate}
  \setcounter{enumi}{4}
  \item ( 15 \begin{CJK}{UTF8}{mj}分\end{CJK})
\end{enumerate}
\begin{CJK}{UTF8}{mj}设\end{CJK} $0<\lambda<1, \lim _{n \rightarrow \infty} a_{n}=a$, \begin{CJK}{UTF8}{mj}求证\end{CJK}:
$$
\lim _{n \rightarrow \infty}\left(\lambda^{n} a_{0}+\lambda^{n-1} a_{1}+\cdots+\lambda a_{n-1}+a_{n}\right)=\frac{a}{1-\lambda}
$$

\begin{enumerate}
  \setcounter{enumi}{5}
  \item (20 \begin{CJK}{UTF8}{mj}分\end{CJK}) \begin{CJK}{UTF8}{mj}设\end{CJK} $f_{n}(x)=e^{-n x^{2}} \cos x, x \in[-1,1], n$ \begin{CJK}{UTF8}{mj}为正整数\end{CJK}. \begin{CJK}{UTF8}{mj}证明\end{CJK}:
\end{enumerate}
(1) $f_{n}$ \begin{CJK}{UTF8}{mj}在\end{CJK} $[-1,1]$ \begin{CJK}{UTF8}{mj}上不一致收敛\end{CJK};

(2) $\lim _{n \rightarrow \infty} \int_{-1}^{1} f_{n}(x) \mathrm{d} x=\int_{-1}^{1} \lim _{n \rightarrow \infty} f_{n}(x) \mathrm{d} x$.

\begin{enumerate}
  \setcounter{enumi}{6}
  \item (10 \begin{CJK}{UTF8}{mj}分\end{CJK}) \begin{CJK}{UTF8}{mj}证明有限闭区间上的连续函数能取到最小值\end{CJK}.

  \item ( 15 \begin{CJK}{UTF8}{mj}分\end{CJK}) \begin{CJK}{UTF8}{mj}设\end{CJK} $f(x)$ \begin{CJK}{UTF8}{mj}在\end{CJK} $a$ \begin{CJK}{UTF8}{mj}点可微\end{CJK}, \begin{CJK}{UTF8}{mj}且\end{CJK} $f(a) \neq 0$, \begin{CJK}{UTF8}{mj}求极限\end{CJK}

\end{enumerate}
$$
\lim _{n \rightarrow \infty}\left[\frac{f\left(a+\frac{1}{n}\right)}{f(a)}\right]^{n}
$$

\begin{enumerate}
  \setcounter{enumi}{8}
  \item ( 15 \begin{CJK}{UTF8}{mj}分\end{CJK}) \begin{CJK}{UTF8}{mj}计算\end{CJK}
\end{enumerate}
$$
\iint_{[0, \pi] \times[0,1]} y \sin (x y) \mathrm{d} x \mathrm{~d} y
$$

\begin{enumerate}
  \setcounter{enumi}{9}
  \item ( 10 \begin{CJK}{UTF8}{mj}分\end{CJK}) \begin{CJK}{UTF8}{mj}计算\end{CJK}
\end{enumerate}
$$
\lim _{n \rightarrow \infty} \sum_{k=n^{2}}^{(n+1)^{2}} \frac{1}{\sqrt{k}}
$$

\section{4. 厦门大学 2012 年研究生入学考试试题数学分析}
\begin{CJK}{UTF8}{mj}李扬\end{CJK}

\begin{CJK}{UTF8}{mj}微信公众号\end{CJK}: sxkyliyang

\begin{enumerate}
  \item ( 20 \begin{CJK}{UTF8}{mj}分\end{CJK}) \begin{CJK}{UTF8}{mj}设\end{CJK} $\left\{x_{n}\right\}$ \begin{CJK}{UTF8}{mj}为有界正实数数列\end{CJK}, \begin{CJK}{UTF8}{mj}求\end{CJK}
\end{enumerate}
$$
\lim _{n \rightarrow \infty} \frac{x_{n}}{x_{1}+x_{2}+\cdots+x_{n}}
$$

\begin{enumerate}
  \setcounter{enumi}{2}
  \item ( 20 \begin{CJK}{UTF8}{mj}分\end{CJK}) \begin{CJK}{UTF8}{mj}设\end{CJK} $g(x)$ \begin{CJK}{UTF8}{mj}满足\end{CJK} $\lim _{x \rightarrow \infty} g(x)=u_{0}, f(u)$ \begin{CJK}{UTF8}{mj}在\end{CJK} $u=u_{0}$ \begin{CJK}{UTF8}{mj}处连续\end{CJK}. \begin{CJK}{UTF8}{mj}证明\end{CJK}: $\lim _{x \rightarrow \infty} f(g(x))=f\left(u_{0}\right)$.

  \item (15 \begin{CJK}{UTF8}{mj}分\end{CJK}) \begin{CJK}{UTF8}{mj}设函数\end{CJK} $f(x)$ \begin{CJK}{UTF8}{mj}在\end{CJK} $(-\infty,+\infty)$ \begin{CJK}{UTF8}{mj}上非恒为零\end{CJK}, \begin{CJK}{UTF8}{mj}存在任意阶导数\end{CJK}, \begin{CJK}{UTF8}{mj}并且对任何\end{CJK} $x \in(-\infty,+\infty)$ \begin{CJK}{UTF8}{mj}都有\end{CJK} $\left|f^{(n)}(x)-f^{(n-1)}(x)\right| \leq \frac{1}{n^{2}}, n=1,2, \cdots$. \begin{CJK}{UTF8}{mj}证明\end{CJK}: $\lim _{n \rightarrow \infty} f^{(n)}(x)=C e^{x}$, \begin{CJK}{UTF8}{mj}其中\end{CJK} $C$ \begin{CJK}{UTF8}{mj}为常数\end{CJK}.

  \item (15 \begin{CJK}{UTF8}{mj}分\end{CJK}) \begin{CJK}{UTF8}{mj}设函数\end{CJK} $z=z(x, y)$ \begin{CJK}{UTF8}{mj}具有二阶连续偏导数\end{CJK}, \begin{CJK}{UTF8}{mj}满足方程\end{CJK}

\end{enumerate}
$$
y \frac{\partial^{2} z}{\partial y^{2}}+2 \frac{\partial z}{\partial y}=\frac{2}{x}
$$
\begin{CJK}{UTF8}{mj}求证在变换\end{CJK} $u=\frac{x}{y}, v=x, w=x z-y$ \begin{CJK}{UTF8}{mj}之下\end{CJK}, \begin{CJK}{UTF8}{mj}上述方程将变为\end{CJK} $\frac{\partial^{2} w}{\partial u^{2}}=0$.

\begin{enumerate}
  \setcounter{enumi}{5}
  \item ( 20 \begin{CJK}{UTF8}{mj}分\end{CJK}) \begin{CJK}{UTF8}{mj}函数\end{CJK} $f(x)$ \begin{CJK}{UTF8}{mj}和函数\end{CJK} $g(x)$ \begin{CJK}{UTF8}{mj}在\end{CJK} $x \geq a$ \begin{CJK}{UTF8}{mj}时可导\end{CJK}, \begin{CJK}{UTF8}{mj}并且在\end{CJK} $x \geq a$ \begin{CJK}{UTF8}{mj}时满足\end{CJK} $\left|f^{\prime}(x)\right| \leq g^{\prime}(x)$, \begin{CJK}{UTF8}{mj}求证\end{CJK}: \begin{CJK}{UTF8}{mj}当\end{CJK} $x \geq a$ \begin{CJK}{UTF8}{mj}时\end{CJK}, \begin{CJK}{UTF8}{mj}不\end{CJK} \begin{CJK}{UTF8}{mj}等式\end{CJK} $|f(x)-f(a)| \leq|g(x)-g(a)|$ \begin{CJK}{UTF8}{mj}成立\end{CJK}.

  \item ( 15 \begin{CJK}{UTF8}{mj}分\end{CJK}) \begin{CJK}{UTF8}{mj}设\end{CJK} $a_{1}, a_{2}, \cdots$ \begin{CJK}{UTF8}{mj}为正实数列\end{CJK}, \begin{CJK}{UTF8}{mj}定义\end{CJK}

\end{enumerate}
$$
s_{n}=\frac{a_{1}+a_{2}+\cdots+a_{n}}{n}, r_{n}=\frac{a_{1}^{-1}+a_{2}^{-1}+\cdots+a_{n}^{-1}}{n},
$$
\begin{CJK}{UTF8}{mj}其中\end{CJK}, $\lim _{n \rightarrow \infty} s_{n}, \lim _{n \rightarrow \infty} r_{n}$ \begin{CJK}{UTF8}{mj}均存在\end{CJK}, \begin{CJK}{UTF8}{mj}证明这两个极限之积不小于\end{CJK} $1 .$

\begin{enumerate}
  \setcounter{enumi}{7}
  \item ( 15 \begin{CJK}{UTF8}{mj}分\end{CJK}) \begin{CJK}{UTF8}{mj}设\end{CJK} $D$ \begin{CJK}{UTF8}{mj}是\end{CJK} $R^{2}$ \begin{CJK}{UTF8}{mj}中的闭圆盘\end{CJK}, $f$ \begin{CJK}{UTF8}{mj}是定义在\end{CJK} $D$ \begin{CJK}{UTF8}{mj}上的实函数\end{CJK}, \begin{CJK}{UTF8}{mj}证明若\end{CJK}
\end{enumerate}
$$
\iint_{D}(f(x, y))^{2} \mathrm{~d} x \mathrm{~d} y=0
$$
\begin{CJK}{UTF8}{mj}则\end{CJK} $f$ \begin{CJK}{UTF8}{mj}在\end{CJK} $D$ \begin{CJK}{UTF8}{mj}中的连续点上的取值为零\end{CJK}.

\begin{enumerate}
  \setcounter{enumi}{8}
  \item (15 \begin{CJK}{UTF8}{mj}分\end{CJK}) \begin{CJK}{UTF8}{mj}设\end{CJK} $\left\{n a_{n}\right\}$ \begin{CJK}{UTF8}{mj}收敛\end{CJK}, \begin{CJK}{UTF8}{mj}级数\end{CJK} $\sum_{n=1}^{\infty} n\left(a_{n}-a_{n-1}\right)$ \begin{CJK}{UTF8}{mj}收敛\end{CJK}, \begin{CJK}{UTF8}{mj}证明\end{CJK} $\sum_{n=1}^{\infty} a_{n}$ \begin{CJK}{UTF8}{mj}也收敛\end{CJK}.

  \item (15 \begin{CJK}{UTF8}{mj}分\end{CJK}) \begin{CJK}{UTF8}{mj}证明下列等式\end{CJK}

\end{enumerate}
$$
\iiint_{V} \cos (a x+b y+c z) \mathrm{d} x \mathrm{~d} y \mathrm{~d} z=\pi \int_{-1}^{1}\left(1-u^{2}\right) \cos (k u) \mathrm{d} u
$$
\begin{CJK}{UTF8}{mj}其中\end{CJK}, $V: x^{2}+y^{2}+z^{2} \leq 1$ \begin{CJK}{UTF8}{mj}为单位球\end{CJK}, $a, b, c$ \begin{CJK}{UTF8}{mj}为常数\end{CJK}.

\section{5. 厦门大学 2013 年研究生入学考试试题数学分析}
\begin{CJK}{UTF8}{mj}李扬\end{CJK}

\begin{CJK}{UTF8}{mj}微信公众号\end{CJK}: sxkyliyang

\begin{enumerate}
  \item ( 15 \begin{CJK}{UTF8}{mj}分\end{CJK}) \begin{CJK}{UTF8}{mj}设数列\end{CJK} $\left\{a_{n}\right\}$ \begin{CJK}{UTF8}{mj}单调递增\end{CJK}, \begin{CJK}{UTF8}{mj}非负\end{CJK}, \begin{CJK}{UTF8}{mj}并且\end{CJK} $\lim _{n \rightarrow \infty} a_{n}=a$, \begin{CJK}{UTF8}{mj}证明\end{CJK}:
\end{enumerate}
$$
\lim _{n \rightarrow \infty}\left(a_{1}^{n}+a_{2}^{2}+\cdots+a_{n}^{n}\right)^{\frac{1}{n}}=a .
$$

\begin{enumerate}
  \setcounter{enumi}{2}
  \item (15 \begin{CJK}{UTF8}{mj}分\end{CJK}) \begin{CJK}{UTF8}{mj}设函数\end{CJK} $f$ \begin{CJK}{UTF8}{mj}在\end{CJK} $[a, b]$ \begin{CJK}{UTF8}{mj}上为单调函数\end{CJK}, \begin{CJK}{UTF8}{mj}且\end{CJK} $f(a)>a, f(b)<b$, \begin{CJK}{UTF8}{mj}求证\end{CJK}: \begin{CJK}{UTF8}{mj}存在\end{CJK} $x_{0} \in[a, b]$, \begin{CJK}{UTF8}{mj}使得\end{CJK} $f\left(x_{0}\right)=x_{0}$.

  \item ( 15 \begin{CJK}{UTF8}{mj}分\end{CJK}) \begin{CJK}{UTF8}{mj}设函数\end{CJK} $f \in C[a, b]$, \begin{CJK}{UTF8}{mj}且\end{CJK}

\end{enumerate}
$$
\int_{0}^{1} f(x) \mathrm{d} x=\int_{0}^{1} x f(x) \mathrm{d} x=0, \int_{0}^{1} x^{2} f(x) \mathrm{d} x=1
$$
\begin{CJK}{UTF8}{mj}求证存在\end{CJK} $x_{0} \in[0,1]$, \begin{CJK}{UTF8}{mj}满足\end{CJK} $\left|f\left(x_{0}\right)\right| \geq 12$.

\begin{enumerate}
  \setcounter{enumi}{4}
  \item ( 15 \begin{CJK}{UTF8}{mj}分\end{CJK}) \begin{CJK}{UTF8}{mj}函数\end{CJK} $u=u(x, y)$ \begin{CJK}{UTF8}{mj}在整个平面上有二阶连续的偏导数\end{CJK}, \begin{CJK}{UTF8}{mj}求证\end{CJK}
\end{enumerate}
$$
\Delta u=\frac{\partial^{2} u}{\partial x^{2}}+\frac{\partial^{2} u}{\partial y^{2}}=0
$$
\begin{CJK}{UTF8}{mj}的充分必要条件是\end{CJK} $\oint_{C} \frac{\partial u}{\partial \vec{n}} \mathrm{~d} s=0$, \begin{CJK}{UTF8}{mj}其中\end{CJK} $\vec{n}$ \begin{CJK}{UTF8}{mj}为光滑封闭曲线\end{CJK} $C$ \begin{CJK}{UTF8}{mj}的单位外法向量\end{CJK}.

\begin{enumerate}
  \setcounter{enumi}{5}
  \item (15 \begin{CJK}{UTF8}{mj}分\end{CJK}) \begin{CJK}{UTF8}{mj}设函数\end{CJK} $f$ \begin{CJK}{UTF8}{mj}在区间\end{CJK} $[a, b]$ \begin{CJK}{UTF8}{mj}可导\end{CJK}, $f\left(\frac{a+b}{2}\right)=0$, \begin{CJK}{UTF8}{mj}且\end{CJK} $\left|f^{\prime}(x)\right| \leq M$, \begin{CJK}{UTF8}{mj}证明\end{CJK}:
\end{enumerate}
$$
\left|\int_{a}^{b} f(x) \mathrm{d} x\right| \leq \frac{M}{2}(b-a)^{2}
$$

\begin{enumerate}
  \setcounter{enumi}{6}
  \item ( 20 \begin{CJK}{UTF8}{mj}分\end{CJK}) \begin{CJK}{UTF8}{mj}设\end{CJK} $\left\{f_{n}\right\}$ \begin{CJK}{UTF8}{mj}为闭区间\end{CJK} $[a, b]$ \begin{CJK}{UTF8}{mj}上的一个函数列\end{CJK}, \begin{CJK}{UTF8}{mj}并且满足\end{CJK}
\end{enumerate}
(1) \begin{CJK}{UTF8}{mj}对任意\end{CJK} $z \in[a, b],\left\{f_{n}(z)\right\}$ \begin{CJK}{UTF8}{mj}是一个有界数列\end{CJK};

(2) $\forall \varepsilon>0, \exists \delta>0$, \begin{CJK}{UTF8}{mj}使得当\end{CJK} $|x-y|<\delta$ \begin{CJK}{UTF8}{mj}时\end{CJK}, \begin{CJK}{UTF8}{mj}对一切自然数\end{CJK} $n$, \begin{CJK}{UTF8}{mj}有\end{CJK} $\left|f_{n}(x)-f_{n}(y)\right|<\varepsilon$.

\begin{CJK}{UTF8}{mj}证明\end{CJK}: \begin{CJK}{UTF8}{mj}存在子列\end{CJK} $\left\{f_{n_{k}}\right\}$ \begin{CJK}{UTF8}{mj}在\end{CJK} $[a, b]$ \begin{CJK}{UTF8}{mj}上一致收敛\end{CJK}.

\begin{enumerate}
  \setcounter{enumi}{7}
  \item ( 20 \begin{CJK}{UTF8}{mj}分\end{CJK}) \begin{CJK}{UTF8}{mj}设\end{CJK} $f$ \begin{CJK}{UTF8}{mj}在区间\end{CJK} $[a, b]$ \begin{CJK}{UTF8}{mj}上非负\end{CJK}, \begin{CJK}{UTF8}{mj}连续且\end{CJK} $\lim _{n \rightarrow \infty} f(x)=0$.
\end{enumerate}
(1) \begin{CJK}{UTF8}{mj}证明\end{CJK} $f$ \begin{CJK}{UTF8}{mj}在区间\end{CJK} $[a, b]$ \begin{CJK}{UTF8}{mj}上取到最大值\end{CJK};

(2) $f$ \begin{CJK}{UTF8}{mj}在区间\end{CJK} $[a, b]$ \begin{CJK}{UTF8}{mj}上能否取到最小值\end{CJK}? (\begin{CJK}{UTF8}{mj}回答问题并说明理由\end{CJK})

\begin{enumerate}
  \setcounter{enumi}{8}
  \item ( 20 \begin{CJK}{UTF8}{mj}分\end{CJK}) \begin{CJK}{UTF8}{mj}设\end{CJK} $f$ \begin{CJK}{UTF8}{mj}在\end{CJK} $[0,+\infty)$ \begin{CJK}{UTF8}{mj}可微且有界\end{CJK}, \begin{CJK}{UTF8}{mj}证明存在\end{CJK} $\left\{x_{n}\right\}^{\infty} \subset[0,+\infty)$, \begin{CJK}{UTF8}{mj}使得\end{CJK} $x_{n} \rightarrow \infty$, \begin{CJK}{UTF8}{mj}并且\end{CJK} $f^{\prime}\left(x_{n}\right) \rightarrow 0$.

  \item (15 \begin{CJK}{UTF8}{mj}分\end{CJK}) \begin{CJK}{UTF8}{mj}计算二重积分\end{CJK} $\iint_{\Sigma} z \mathrm{~d} x \mathrm{~d} y$. \begin{CJK}{UTF8}{mj}其中\end{CJK} $\Sigma$ \begin{CJK}{UTF8}{mj}是三角形\end{CJK} $\{(x, y, z): x, y, z \geq 0, x+y+z=1\}$.

\end{enumerate}
\section{6. 厦门大学 2016 年研究生入学考试试题数学分析 
 李扬 
 微信公众号: sxkyliyang}
\begin{enumerate}
  \item ( 20 \begin{CJK}{UTF8}{mj}分\end{CJK}) \begin{CJK}{UTF8}{mj}已知\end{CJK} $f(x)$ \begin{CJK}{UTF8}{mj}在\end{CJK} $[0,+\infty)$ \begin{CJK}{UTF8}{mj}上单调递减\end{CJK}, \begin{CJK}{UTF8}{mj}且\end{CJK} $\lim _{n \rightarrow+\infty} f(x)=0$, \begin{CJK}{UTF8}{mj}证明\end{CJK}: $\sum_{n=1}^{\infty} f(n)$ \begin{CJK}{UTF8}{mj}收敛的充分必要条件是\end{CJK}
\end{enumerate}
$$
\int_{0}^{+\infty} f(x) \mathrm{d} x
$$
\begin{CJK}{UTF8}{mj}收敛\end{CJK}.

\begin{enumerate}
  \setcounter{enumi}{2}
  \item (20 \begin{CJK}{UTF8}{mj}分\end{CJK}) \begin{CJK}{UTF8}{mj}设\end{CJK} $f \in C^{1}[0,+\infty), f(0)=1, f^{\prime}(x)=\frac{1}{x^{2}+f^{2}(x)}$. \begin{CJK}{UTF8}{mj}证明\end{CJK}:
\end{enumerate}
(1)
$$
\lim _{n \rightarrow+\infty} f(x)
$$
\begin{CJK}{UTF8}{mj}存在\end{CJK};

(2)
$$
\lim _{n \rightarrow+\infty} f(x) \leq 1+\frac{\pi}{2}
$$

\begin{enumerate}
  \setcounter{enumi}{3}
  \item ( 15 \begin{CJK}{UTF8}{mj}分\end{CJK}) \begin{CJK}{UTF8}{mj}已知\end{CJK} $\lim _{n \rightarrow \infty} \frac{a_{n}}{n}=0$, \begin{CJK}{UTF8}{mj}证明\end{CJK}:
\end{enumerate}
$$
\lim _{n \rightarrow \infty} \frac{\max \left\{a_{1}, \cdots, a_{n}\right\}}{n}=0 .
$$

\begin{enumerate}
  \setcounter{enumi}{4}
  \item ( 20 \begin{CJK}{UTF8}{mj}分\end{CJK}) \begin{CJK}{UTF8}{mj}已知\end{CJK} $f(x)$ \begin{CJK}{UTF8}{mj}有界\end{CJK}, \begin{CJK}{UTF8}{mj}且在\end{CJK} $\mathbb{R}$ \begin{CJK}{UTF8}{mj}上连续\end{CJK}, \begin{CJK}{UTF8}{mj}设\end{CJK} $T>0$, \begin{CJK}{UTF8}{mj}证明\end{CJK}: \begin{CJK}{UTF8}{mj}存在数列\end{CJK} $\left\{x_{n}\right\}$, \begin{CJK}{UTF8}{mj}使得\end{CJK}
\end{enumerate}
$$
\lim _{n \rightarrow \infty} x_{n}=+\infty, \quad \lim _{n \rightarrow \infty}\left(f\left(x_{n}+T\right)-f\left(x_{n}\right)\right)=0
$$

\begin{enumerate}
  \setcounter{enumi}{5}
  \item ( 20 \begin{CJK}{UTF8}{mj}分\end{CJK}) \begin{CJK}{UTF8}{mj}设函数\end{CJK} $f$ \begin{CJK}{UTF8}{mj}在区间\end{CJK} $[a, b]$ \begin{CJK}{UTF8}{mj}上二阶可导\end{CJK}, \begin{CJK}{UTF8}{mj}且\end{CJK} $\forall x \in(a, b)$ \begin{CJK}{UTF8}{mj}有\end{CJK} $f^{\prime \prime}(x)>0$. \begin{CJK}{UTF8}{mj}证明\end{CJK}: $\forall x_{1}, x_{2} \in(a, b)$ \begin{CJK}{UTF8}{mj}有\end{CJK}
\end{enumerate}
$$
f\left(\frac{x_{1}+x_{2}}{2}\right)<\frac{1}{2}\left[f\left(x_{1}\right)+f\left(x_{2}\right)\right] .
$$

\begin{enumerate}
  \setcounter{enumi}{6}
  \item (15 \begin{CJK}{UTF8}{mj}分\end{CJK}) \begin{CJK}{UTF8}{mj}设\end{CJK} $f$ \begin{CJK}{UTF8}{mj}在\end{CJK} $[a, b]$ \begin{CJK}{UTF8}{mj}上可积\end{CJK}, \begin{CJK}{UTF8}{mj}且有\end{CJK}
\end{enumerate}
$$
\int_{0}^{x} f(t) \mathrm{d} t \geq 0, \int_{a}^{b} f(x) \mathrm{d} x=0
$$
\begin{CJK}{UTF8}{mj}证明\end{CJK}:
$$
\int_{a}^{b} x f(x) \mathrm{d} x \leq 0
$$

\begin{enumerate}
  \setcounter{enumi}{7}
  \item ( 20 \begin{CJK}{UTF8}{mj}分\end{CJK}) \begin{CJK}{UTF8}{mj}设\end{CJK} $B$ \begin{CJK}{UTF8}{mj}为单位球\end{CJK} $x^{2}+y^{2}+z^{2} \leq 1$ \begin{CJK}{UTF8}{mj}的区域\end{CJK}, $\partial B$ \begin{CJK}{UTF8}{mj}为其球面\end{CJK}. \begin{CJK}{UTF8}{mj}已知\end{CJK} $f$ \begin{CJK}{UTF8}{mj}为\end{CJK} $k$ \begin{CJK}{UTF8}{mj}次齐次函数\end{CJK}. \begin{CJK}{UTF8}{mj}即\end{CJK} $f(a x, a y, a z)=$ $a^{k} f(x, y, z)$. \begin{CJK}{UTF8}{mj}证明\end{CJK}:
\end{enumerate}
$$
\iint_{\partial B} f(x, y, z) \mathrm{d} S=\iint_{B} \Delta f \mathrm{~d} x \mathrm{~d} y \mathrm{~d} z
$$
\begin{CJK}{UTF8}{mj}其中\end{CJK}:
$$
\Delta f=\frac{\partial^{2} f}{\partial x^{2}}+\frac{\partial^{2} f}{\partial y^{2}}+\frac{\partial^{2} f}{\partial z^{2}}
$$

\begin{enumerate}
  \setcounter{enumi}{8}
  \item ( 20 \begin{CJK}{UTF8}{mj}分\end{CJK}) \begin{CJK}{UTF8}{mj}设有一张长方形纸片\end{CJK}, \begin{CJK}{UTF8}{mj}要在上面涂颜色\end{CJK}, \begin{CJK}{UTF8}{mj}长方形纸片内部涂颜色的面积为\end{CJK} $A \mathrm{~cm}^{2}$, \begin{CJK}{UTF8}{mj}边缘有空隙\end{CJK}, \begin{CJK}{UTF8}{mj}上下\end{CJK} \begin{CJK}{UTF8}{mj}边宽度之和为\end{CJK} $r \mathrm{~cm}$, \begin{CJK}{UTF8}{mj}左右宽度为\end{CJK} $h \mathrm{~cm}$. \begin{CJK}{UTF8}{mj}意思是\end{CJK}:\begin{CJK}{UTF8}{mj}在长方形纸片上给矩形求\end{CJK}: \begin{CJK}{UTF8}{mj}当长方形纸片长\end{CJK} $(y \mathrm{~cm})$ \begin{CJK}{UTF8}{mj}和宽\end{CJK} ( $x \mathrm{~cm})$ \begin{CJK}{UTF8}{mj}为多少时\end{CJK}, \begin{CJK}{UTF8}{mj}长方形面积最小\end{CJK}?
\end{enumerate}
\section{7. 厦门大学 2009 年研究生入学考试试题高等代数}
\begin{CJK}{UTF8}{mj}李扬\end{CJK}

\begin{CJK}{UTF8}{mj}微信公众号\end{CJK}: sxkyliyang

\begin{CJK}{UTF8}{mj}一\end{CJK}、\begin{CJK}{UTF8}{mj}填空题\end{CJK}(\begin{CJK}{UTF8}{mj}每小题\end{CJK} 6 \begin{CJK}{UTF8}{mj}分\end{CJK}, \begin{CJK}{UTF8}{mj}共\end{CJK} 36 \begin{CJK}{UTF8}{mj}分\end{CJK})

\begin{enumerate}
  \item \begin{CJK}{UTF8}{mj}设\end{CJK} $A$ \begin{CJK}{UTF8}{mj}为\end{CJK} $n$ \begin{CJK}{UTF8}{mj}阶可逆方阵\end{CJK}, \begin{CJK}{UTF8}{mj}则\end{CJK} $\left(A^{*}\right)^{*}=$ \begin{CJK}{UTF8}{mj}这里\end{CJK} $A^{*}$ \begin{CJK}{UTF8}{mj}是\end{CJK} $A$ \begin{CJK}{UTF8}{mj}的伴随矩阵\end{CJK}.
\end{enumerate}
\includegraphics[max width=\textwidth]{2022_04_18_a5c47c0ff534501b502eg-086}

\begin{enumerate}
  \setcounter{enumi}{3}
  \item \begin{CJK}{UTF8}{mj}设\end{CJK} $f(x), g(x)$ \begin{CJK}{UTF8}{mj}是有理系数多项式\end{CJK}, \begin{CJK}{UTF8}{mj}若\end{CJK} $f(x), g(x)$ \begin{CJK}{UTF8}{mj}在有理数域上互素\end{CJK}, \begin{CJK}{UTF8}{mj}则在复数域上\end{CJK} $f(x), g(x)$ (\begin{CJK}{UTF8}{mj}选\end{CJK} \begin{CJK}{UTF8}{mj}填\end{CJK}“\begin{CJK}{UTF8}{mj}一定\end{CJK}”\begin{CJK}{UTF8}{mj}或\end{CJK}“\begin{CJK}{UTF8}{mj}末必\end{CJK}”)\begin{CJK}{UTF8}{mj}互素\end{CJK}.

  \item \begin{CJK}{UTF8}{mj}设\end{CJK} $A=\left(\begin{array}{lll}1 & 1 & 0 \\ 0 & 1 & 1 \\ 0 & 0 & 1\end{array}\right)$, \begin{CJK}{UTF8}{mj}则与\end{CJK} $A$ \begin{CJK}{UTF8}{mj}乘法可交换的全体实矩阵构成的线性空间的维数是\end{CJK} \begin{CJK}{UTF8}{mj}一组基是\end{CJK}

  \item \begin{CJK}{UTF8}{mj}设\end{CJK} $X$ \begin{CJK}{UTF8}{mj}是\end{CJK} $n$ \begin{CJK}{UTF8}{mj}阶方阵\end{CJK} $A$ \begin{CJK}{UTF8}{mj}的属于特征值\end{CJK} $\lambda$ \begin{CJK}{UTF8}{mj}的特征向量\end{CJK}, $P$ \begin{CJK}{UTF8}{mj}是\end{CJK} $n$ \begin{CJK}{UTF8}{mj}阶可逆矩阵\end{CJK}, \begin{CJK}{UTF8}{mj}则\end{CJK} \begin{CJK}{UTF8}{mj}是\end{CJK} $P^{-1} A P$ \begin{CJK}{UTF8}{mj}的属于特征\end{CJK} \begin{CJK}{UTF8}{mj}值\end{CJK} $\lambda$ \begin{CJK}{UTF8}{mj}的特征向量\end{CJK}.

  \item \begin{CJK}{UTF8}{mj}上三角正交矩阵形如\end{CJK}

\end{enumerate}
\section{二、解答题}
\begin{enumerate}
  \item ( 20 \begin{CJK}{UTF8}{mj}分\end{CJK}) \begin{CJK}{UTF8}{mj}设\end{CJK} $n$ \begin{CJK}{UTF8}{mj}维线性空间上线性变换\end{CJK} $\mathscr{A}$ \begin{CJK}{UTF8}{mj}有\end{CJK} $n+1$ \begin{CJK}{UTF8}{mj}个特征向量\end{CJK}, \begin{CJK}{UTF8}{mj}且其中的任意\end{CJK} $n$ \begin{CJK}{UTF8}{mj}个向量都是线性无关的\end{CJK}. \begin{CJK}{UTF8}{mj}求证\end{CJK}: $\mathscr{A}$ \begin{CJK}{UTF8}{mj}为乘数变换\end{CJK}.

  \item ( 20 \begin{CJK}{UTF8}{mj}分\end{CJK}) \begin{CJK}{UTF8}{mj}设\end{CJK} $A$ \begin{CJK}{UTF8}{mj}是\end{CJK} $n$ \begin{CJK}{UTF8}{mj}阶方阵\end{CJK}, \begin{CJK}{UTF8}{mj}求证\end{CJK}:

\end{enumerate}
(1)
$$
A=B+C,
$$
\begin{CJK}{UTF8}{mj}其中\end{CJK} $\operatorname{tr} B=0, C$ \begin{CJK}{UTF8}{mj}是数量矩阵\end{CJK} $\lambda I$, \begin{CJK}{UTF8}{mj}这里\end{CJK} $I$ \begin{CJK}{UTF8}{mj}是\end{CJK} $n$ \begin{CJK}{UTF8}{mj}阶单位矩阵\end{CJK}, $\lambda$ \begin{CJK}{UTF8}{mj}是一个数\end{CJK};

(2) \begin{CJK}{UTF8}{mj}式\end{CJK} (1) \begin{CJK}{UTF8}{mj}的分解是唯一的\end{CJK}.

\begin{enumerate}
  \setcounter{enumi}{3}
  \item (20 \begin{CJK}{UTF8}{mj}分\end{CJK})
\end{enumerate}
(1)\begin{CJK}{UTF8}{mj}证明全体正实数\end{CJK} $\mathbb{R}^{+}$\begin{CJK}{UTF8}{mj}在加法定义为\end{CJK} $a \oplus b=a b$, \begin{CJK}{UTF8}{mj}数乘定义为\end{CJK} $k \circ a=a^{k}$ \begin{CJK}{UTF8}{mj}下构成实数域\end{CJK} $\mathbb{R}$ \begin{CJK}{UTF8}{mj}上线性空间\end{CJK};

(2) \begin{CJK}{UTF8}{mj}求这个空间的维数\end{CJK};

(3) \begin{CJK}{UTF8}{mj}加法定义为数的加法\end{CJK}, \begin{CJK}{UTF8}{mj}数乘定义为数的乘法\end{CJK}, $\mathbb{R}$ \begin{CJK}{UTF8}{mj}是\end{CJK} $\mathbb{R}$ \begin{CJK}{UTF8}{mj}上的线性空间\end{CJK}, \begin{CJK}{UTF8}{mj}仍记为\end{CJK} $\mathbb{R}$. \begin{CJK}{UTF8}{mj}已知\end{CJK} $\log _{a}: \mathbb{R}^{+} \rightarrow \mathbb{R}^{x}, x \mapsto$ $\log _{a} x$ \begin{CJK}{UTF8}{mj}导出了从\end{CJK} $\mathbb{R}^{+}$\begin{CJK}{UTF8}{mj}到\end{CJK} $\mathbb{R}$ \begin{CJK}{UTF8}{mj}的一一映射\end{CJK}, \begin{CJK}{UTF8}{mj}这个映射是不是线性空间的同构\end{CJK}?

(4) \begin{CJK}{UTF8}{mj}求从\end{CJK} $\mathbb{R}^{+}$\begin{CJK}{UTF8}{mj}到\end{CJK} $\mathbb{R}$ \begin{CJK}{UTF8}{mj}的同构\end{CJK} $\mathscr{A}$, \begin{CJK}{UTF8}{mj}使得\end{CJK}
$$
\mathscr{A}(2)=3
$$

\begin{enumerate}
  \setcounter{enumi}{4}
  \item ( 14 \begin{CJK}{UTF8}{mj}分\end{CJK}) \begin{CJK}{UTF8}{mj}设\end{CJK} $U, V, W$ \begin{CJK}{UTF8}{mj}是有限维线性空间\end{CJK}, $\mathscr{A}: V \rightarrow U, \mathscr{B}: W \rightarrow U$ \begin{CJK}{UTF8}{mj}是线性映射\end{CJK}.
\end{enumerate}
(1) \begin{CJK}{UTF8}{mj}求证\end{CJK}: \begin{CJK}{UTF8}{mj}存在线性映射\end{CJK} $\mathscr{C}: V \rightarrow W$ \begin{CJK}{UTF8}{mj}使得\end{CJK} $\mathscr{A}=\mathscr{B} \mathscr{C}$ \begin{CJK}{UTF8}{mj}的充分必要条件是\end{CJK} $\operatorname{Im} \mathscr{A} \subseteq \operatorname{Im} \mathscr{B}$;

(2) \begin{CJK}{UTF8}{mj}设\end{CJK}
$$
\operatorname{dim} V=\operatorname{dim} W .
$$
\begin{CJK}{UTF8}{mj}问什么条件下\end{CJK}, (1) \begin{CJK}{UTF8}{mj}中的\end{CJK} $\mathscr{C}$ \begin{CJK}{UTF8}{mj}可以取成同构映射\end{CJK}? \begin{CJK}{UTF8}{mj}并给出证明\end{CJK}.

\section{8. 厦门大学 2010 年研究生入学考试试题高等代数}
\begin{CJK}{UTF8}{mj}李扬\end{CJK}

\begin{CJK}{UTF8}{mj}微信公众号\end{CJK}: sxkyliyang

\begin{CJK}{UTF8}{mj}一\end{CJK}、\begin{CJK}{UTF8}{mj}填空题\end{CJK}(\begin{CJK}{UTF8}{mj}每小题\end{CJK} 6 \begin{CJK}{UTF8}{mj}分\end{CJK}, \begin{CJK}{UTF8}{mj}共\end{CJK} 36 \begin{CJK}{UTF8}{mj}分\end{CJK})

\begin{enumerate}
  \item \begin{CJK}{UTF8}{mj}行列式\end{CJK} $\left|\begin{array}{cccc}a_{1} & 0 & b_{1} & 0 \\ 0 & c_{1} & 0 & d_{1} \\ a_{2} & 0 & b_{2} & 0 \\ 0 & c_{2} & 0 & d_{2}\end{array}\right|=$

  \item \begin{CJK}{UTF8}{mj}所有可能的\end{CJK} 2 \begin{CJK}{UTF8}{mj}阶对称正交阵是\end{CJK}

  \item \begin{CJK}{UTF8}{mj}设向量组\end{CJK} $\alpha, \beta, \gamma$ \begin{CJK}{UTF8}{mj}以及常数\end{CJK} $k, l, m$ \begin{CJK}{UTF8}{mj}满足\end{CJK} $k \alpha+l \beta+m \gamma=0$, \begin{CJK}{UTF8}{mj}且\end{CJK} $k m \neq 0$, \begin{CJK}{UTF8}{mj}则\end{CJK} $\alpha, \beta$ \begin{CJK}{UTF8}{mj}必与\end{CJK} \begin{CJK}{UTF8}{mj}等价\end{CJK}.

  \item \begin{CJK}{UTF8}{mj}设\end{CJK} $f(x)$ \begin{CJK}{UTF8}{mj}是有理系数多项式且在有理数域上没有重因式\end{CJK}, \begin{CJK}{UTF8}{mj}则\end{CJK} $f(x)$ \begin{CJK}{UTF8}{mj}在实数域上\end{CJK} (\begin{CJK}{UTF8}{mj}选填\end{CJK}"\begin{CJK}{UTF8}{mj}一定\end{CJK}"\begin{CJK}{UTF8}{mj}或\end{CJK}"\begin{CJK}{UTF8}{mj}末\end{CJK} \begin{CJK}{UTF8}{mj}必\end{CJK}") \begin{CJK}{UTF8}{mj}没有重因式\end{CJK}.

  \item \begin{CJK}{UTF8}{mj}设\end{CJK} $n$ \begin{CJK}{UTF8}{mj}阶方阵\end{CJK} $A=\alpha^{\prime} \alpha$, \begin{CJK}{UTF8}{mj}其中\end{CJK} $\alpha=\left(a_{1}, a_{2}, \cdots, a_{n}\right) \neq 0, \alpha^{\prime}$ \begin{CJK}{UTF8}{mj}表示\end{CJK} $\alpha$ \begin{CJK}{UTF8}{mj}的转置\end{CJK}, \begin{CJK}{UTF8}{mj}则\end{CJK} $A$ \begin{CJK}{UTF8}{mj}的所有特征值是\end{CJK} (\begin{CJK}{UTF8}{mj}请\end{CJK} \begin{CJK}{UTF8}{mj}注明重数\end{CJK}).

  \item \begin{CJK}{UTF8}{mj}设\end{CJK} $V$ \begin{CJK}{UTF8}{mj}是\end{CJK} $n$ \begin{CJK}{UTF8}{mj}维线性空间\end{CJK}, $\mathscr{A}$ \begin{CJK}{UTF8}{mj}是\end{CJK} $V$ \begin{CJK}{UTF8}{mj}的线性变换\end{CJK}, \begin{CJK}{UTF8}{mj}如果\end{CJK} $\mathscr{A}$ \begin{CJK}{UTF8}{mj}的特征多项式和极小多项式相等\end{CJK}, \begin{CJK}{UTF8}{mj}则\end{CJK} $\mathscr{A}$ \begin{CJK}{UTF8}{mj}的每个特征值\end{CJK} $\lambda$ \begin{CJK}{UTF8}{mj}的特征子空间维数是\end{CJK} , \begin{CJK}{UTF8}{mj}而\end{CJK} $\mathscr{A}-\lambda \varepsilon$ \begin{CJK}{UTF8}{mj}的维数是\end{CJK} , \begin{CJK}{UTF8}{mj}其中\end{CJK} $\varepsilon$ \begin{CJK}{UTF8}{mj}是\end{CJK} $V$ \begin{CJK}{UTF8}{mj}的恒等变换\end{CJK}.

\end{enumerate}
\section{一、解答题}
\begin{enumerate}
  \item (20 \begin{CJK}{UTF8}{mj}分\end{CJK}) \begin{CJK}{UTF8}{mj}设\end{CJK} $A$ \begin{CJK}{UTF8}{mj}为\end{CJK} $n$ \begin{CJK}{UTF8}{mj}阶矩阵\end{CJK}, \begin{CJK}{UTF8}{mj}证明\end{CJK}:
\end{enumerate}
$$
\mathrm{r}(A+I)+\mathrm{r}(A-I)=n
$$
\begin{CJK}{UTF8}{mj}的充要条件是\end{CJK} $A^{2}=I$, \begin{CJK}{UTF8}{mj}其中\end{CJK} $I$ \begin{CJK}{UTF8}{mj}为\end{CJK} $n$ \begin{CJK}{UTF8}{mj}阶单位矩阵\end{CJK}.

\begin{enumerate}
  \setcounter{enumi}{2}
  \item ( 20 \begin{CJK}{UTF8}{mj}分\end{CJK}) \begin{CJK}{UTF8}{mj}设\end{CJK} $A$ \begin{CJK}{UTF8}{mj}是\end{CJK} $s \times n$ \begin{CJK}{UTF8}{mj}阶矩阵\end{CJK}, $\eta_{1}, \eta_{2}, \cdots, \eta_{r}$ \begin{CJK}{UTF8}{mj}是齐次线性方程组\end{CJK} $A X=0$ \begin{CJK}{UTF8}{mj}的一个基础解系\end{CJK}, \begin{CJK}{UTF8}{mj}记矩阵\end{CJK} $B=\left(\eta_{1}, \eta_{2}, \cdots, \eta_{r}\right)$, \begin{CJK}{UTF8}{mj}如果\end{CJK} $n \times m$ \begin{CJK}{UTF8}{mj}阶矩阵\end{CJK} $C$ \begin{CJK}{UTF8}{mj}满足\end{CJK} $A C=0$, \begin{CJK}{UTF8}{mj}证明存在唯一的矩阵\end{CJK} $G$, \begin{CJK}{UTF8}{mj}使得\end{CJK}
\end{enumerate}
$$
C=B G
$$

\begin{enumerate}
  \setcounter{enumi}{3}
  \item ( 20 \begin{CJK}{UTF8}{mj}分\end{CJK}) \begin{CJK}{UTF8}{mj}设\end{CJK} $V$ \begin{CJK}{UTF8}{mj}是数域\end{CJK} $\mathbb{F}$ \begin{CJK}{UTF8}{mj}上\end{CJK} $n$ \begin{CJK}{UTF8}{mj}维线性空间\end{CJK}, $\mathscr{A}$ \begin{CJK}{UTF8}{mj}是\end{CJK} $V$ \begin{CJK}{UTF8}{mj}上线性变换\end{CJK}. \begin{CJK}{UTF8}{mj}又设\end{CJK} $\mathbb{F}$ \begin{CJK}{UTF8}{mj}上多项式\end{CJK} $f(x), g(x)$ \begin{CJK}{UTF8}{mj}互素\end{CJK}, $\varphi=$ $f(\mathscr{A}), \psi=g(\mathscr{A})$. \begin{CJK}{UTF8}{mj}求证\end{CJK}:
\end{enumerate}
$$
\operatorname{ker} \varphi \psi=\operatorname{ker} \varphi \oplus \operatorname{ker} \psi,
$$
\begin{CJK}{UTF8}{mj}其中\end{CJK} $\operatorname{ker} \varphi$ \begin{CJK}{UTF8}{mj}是\end{CJK} $\varphi$ \begin{CJK}{UTF8}{mj}的核空间\end{CJK}, \begin{CJK}{UTF8}{mj}即\end{CJK} $\operatorname{ker} \varphi=\{\alpha \in V \mid \varphi(\alpha)=0\}$.

\begin{enumerate}
  \setcounter{enumi}{4}
  \item (14 \begin{CJK}{UTF8}{mj}分\end{CJK}) \begin{CJK}{UTF8}{mj}证明\end{CJK}: $n$ \begin{CJK}{UTF8}{mj}阶可逆对称矩阵\end{CJK} $A$ \begin{CJK}{UTF8}{mj}是正定矩阵的充要条件是对任意\end{CJK} $n$ \begin{CJK}{UTF8}{mj}阶正定矩阵\end{CJK} $B, A B$ \begin{CJK}{UTF8}{mj}的迹\end{CJK} $\operatorname{tr}(A B)$ \begin{CJK}{UTF8}{mj}均大\end{CJK} \begin{CJK}{UTF8}{mj}于\end{CJK} 0 .
\end{enumerate}
\section{9. 厦门大学 2011 年研究生入学考试试题高等代数}
\begin{CJK}{UTF8}{mj}李扬\end{CJK}

\begin{CJK}{UTF8}{mj}微信公众号\end{CJK}: sxkyliyang

\begin{CJK}{UTF8}{mj}一\end{CJK}、\begin{CJK}{UTF8}{mj}填空题\end{CJK}(\begin{CJK}{UTF8}{mj}每小题\end{CJK} 6 \begin{CJK}{UTF8}{mj}分\end{CJK}, \begin{CJK}{UTF8}{mj}共\end{CJK} 36 \begin{CJK}{UTF8}{mj}分\end{CJK})

\begin{enumerate}
  \item $x^{5}+2 x+2$ \begin{CJK}{UTF8}{mj}在实数域上是否可约\end{CJK}? $=$ (\begin{CJK}{UTF8}{mj}选填\end{CJK}“\begin{CJK}{UTF8}{mj}可约\end{CJK}”\begin{CJK}{UTF8}{mj}或\end{CJK}“\begin{CJK}{UTF8}{mj}不可约\end{CJK}").

  \item \begin{CJK}{UTF8}{mj}已知\end{CJK} $A=\left(\begin{array}{cccc}1 & -1 & 1 & -1 \\ -1 & 2 & -2 & 2 \\ 1 & -2 & 3 & -3 \\ 1 & -2 & -3 & 4\end{array}\right), M_{i j}$ \begin{CJK}{UTF8}{mj}为\end{CJK} $A$ \begin{CJK}{UTF8}{mj}的第\end{CJK} $(i, j)$ \begin{CJK}{UTF8}{mj}元素\end{CJK} $a_{i j}$ \begin{CJK}{UTF8}{mj}的余子式\end{CJK}, \begin{CJK}{UTF8}{mj}则\end{CJK} $M_{14}+M_{24}+M_{34}+M_{44}=$

  \item \begin{CJK}{UTF8}{mj}将矩阵\end{CJK} $A=\left(\begin{array}{ll}2 & 1 \\ 1 & 1\end{array}\right)$ \begin{CJK}{UTF8}{mj}表示为初等矩阵的乘积\end{CJK}

  \item \begin{CJK}{UTF8}{mj}设\end{CJK} $n$ \begin{CJK}{UTF8}{mj}阶矩阵\end{CJK} $A$ \begin{CJK}{UTF8}{mj}的各行元素之和为零\end{CJK}, \begin{CJK}{UTF8}{mj}且\end{CJK} $A$ \begin{CJK}{UTF8}{mj}的秩\end{CJK} $\mathrm{r}(A)=n-1$, \begin{CJK}{UTF8}{mj}则齐次线性方程组\end{CJK} $A X=0$ \begin{CJK}{UTF8}{mj}的所有解为\end{CJK}

  \item \begin{CJK}{UTF8}{mj}设\end{CJK} $A$ \begin{CJK}{UTF8}{mj}为\end{CJK} $n$ \begin{CJK}{UTF8}{mj}阶矩阵\end{CJK}, $A$ \begin{CJK}{UTF8}{mj}的行列式\end{CJK} $|A| \neq 0$, \begin{CJK}{UTF8}{mj}且\end{CJK} $a$ \begin{CJK}{UTF8}{mj}为\end{CJK} $A$ \begin{CJK}{UTF8}{mj}的一个特征值\end{CJK}, \begin{CJK}{UTF8}{mj}则\end{CJK} \begin{CJK}{UTF8}{mj}必是\end{CJK} $A$ \begin{CJK}{UTF8}{mj}的伴随矩阵\end{CJK} $A^{*}$ \begin{CJK}{UTF8}{mj}的一\end{CJK} \begin{CJK}{UTF8}{mj}个特征值\end{CJK}.

  \item \begin{CJK}{UTF8}{mj}设\end{CJK} $A$ \begin{CJK}{UTF8}{mj}为\end{CJK} 3 \begin{CJK}{UTF8}{mj}阶非零矩阵且\end{CJK} $A^{2}=0$, \begin{CJK}{UTF8}{mj}则\end{CJK} $A$ \begin{CJK}{UTF8}{mj}的\end{CJK} Jordan \begin{CJK}{UTF8}{mj}标准形是\end{CJK}

\end{enumerate}
\section{一、解答题}
\begin{enumerate}
  \item ( 20 \begin{CJK}{UTF8}{mj}分\end{CJK}) \begin{CJK}{UTF8}{mj}设\end{CJK} $\beta, \alpha_{1}, \alpha_{2}, \cdots, \alpha_{n}$ \begin{CJK}{UTF8}{mj}是线性空间\end{CJK} $V$ \begin{CJK}{UTF8}{mj}的向量\end{CJK}, \begin{CJK}{UTF8}{mj}且\end{CJK} $\beta$ \begin{CJK}{UTF8}{mj}可由\end{CJK} $\alpha_{1}, \alpha_{2}, \cdots, \alpha_{n}$ \begin{CJK}{UTF8}{mj}线性表示\end{CJK}. \begin{CJK}{UTF8}{mj}求证\end{CJK}:\begin{CJK}{UTF8}{mj}表示法唯\end{CJK} \begin{CJK}{UTF8}{mj}一的充要条件是\end{CJK} $\alpha_{1}, \alpha_{2}, \cdots, \alpha_{n}$ \begin{CJK}{UTF8}{mj}线性无关\end{CJK}.

  \item (15 \begin{CJK}{UTF8}{mj}分\end{CJK}) \begin{CJK}{UTF8}{mj}设\end{CJK} $A$ \begin{CJK}{UTF8}{mj}是\end{CJK} $m \times n$ \begin{CJK}{UTF8}{mj}阶实矩阵\end{CJK}, \begin{CJK}{UTF8}{mj}求证\end{CJK}: $A$ \begin{CJK}{UTF8}{mj}的秩\end{CJK}

\end{enumerate}
$$
\mathrm{r}(A)=n
$$
\begin{CJK}{UTF8}{mj}的充要条件是\end{CJK} $A^{\prime} A$ \begin{CJK}{UTF8}{mj}正定\end{CJK}, \begin{CJK}{UTF8}{mj}其中\end{CJK} $A^{\prime}$ \begin{CJK}{UTF8}{mj}表示\end{CJK} $A$ \begin{CJK}{UTF8}{mj}的转置\end{CJK}.

\begin{enumerate}
  \setcounter{enumi}{3}
  \item ( 15 \begin{CJK}{UTF8}{mj}分\end{CJK}) \begin{CJK}{UTF8}{mj}设\end{CJK} $A, B$ \begin{CJK}{UTF8}{mj}是\end{CJK} $n$ \begin{CJK}{UTF8}{mj}阶方阵\end{CJK}, \begin{CJK}{UTF8}{mj}它们秩的和小于等于\end{CJK} $n$, \begin{CJK}{UTF8}{mj}即\end{CJK}
\end{enumerate}
$$
\mathrm{r}(A)+\mathrm{r}(B) \leq n
$$
\begin{CJK}{UTF8}{mj}求证\end{CJK}:\begin{CJK}{UTF8}{mj}存在\end{CJK} $n$ \begin{CJK}{UTF8}{mj}阶非零方阵\end{CJK} $C$, \begin{CJK}{UTF8}{mj}使得\end{CJK} $A C B=0$.

\begin{enumerate}
  \setcounter{enumi}{4}
  \item (15 \begin{CJK}{UTF8}{mj}分\end{CJK}) \begin{CJK}{UTF8}{mj}设\end{CJK} $f(x)$ \begin{CJK}{UTF8}{mj}是首项系数为\end{CJK} 1 \begin{CJK}{UTF8}{mj}的\end{CJK} $n$ \begin{CJK}{UTF8}{mj}次整系数多项式\end{CJK}, $a_{1}, a_{2}, \cdots, a_{n}$ \begin{CJK}{UTF8}{mj}是\end{CJK} $n$ \begin{CJK}{UTF8}{mj}个两两不同的整数\end{CJK}, \begin{CJK}{UTF8}{mj}且\end{CJK}
\end{enumerate}
$$
f\left(a_{i}\right)=-1,(i=1,2, \cdots, n) .
$$
\begin{CJK}{UTF8}{mj}求证\end{CJK}: $f(x)$ \begin{CJK}{UTF8}{mj}在有理数域上不可约\end{CJK}.

\section{0. 厦门大学 2012 年研究生入学考试试题高等代数 
 李扬 
 微信公众号: sxkyliyang}
\begin{CJK}{UTF8}{mj}一\end{CJK}、\begin{CJK}{UTF8}{mj}填空题\end{CJK}(\begin{CJK}{UTF8}{mj}每小题\end{CJK} 6 \begin{CJK}{UTF8}{mj}分\end{CJK}, \begin{CJK}{UTF8}{mj}共\end{CJK} 48 \begin{CJK}{UTF8}{mj}分\end{CJK})

\begin{enumerate}
  \item \begin{CJK}{UTF8}{mj}设\end{CJK} $A$ \begin{CJK}{UTF8}{mj}是\end{CJK} $n$ \begin{CJK}{UTF8}{mj}阶方阵\end{CJK}, $E_{i j}$ \begin{CJK}{UTF8}{mj}表示第\end{CJK} $(i, j)$ \begin{CJK}{UTF8}{mj}元素是\end{CJK} 1 , \begin{CJK}{UTF8}{mj}其余元素是\end{CJK} 0 \begin{CJK}{UTF8}{mj}的基础矩阵\end{CJK}, \begin{CJK}{UTF8}{mj}则\end{CJK} $E_{i j} A E_{k j}=$

  \item $n$ \begin{CJK}{UTF8}{mj}阶方阵\end{CJK} $A$ \begin{CJK}{UTF8}{mj}满足\end{CJK} $A^{2}=2 A$, \begin{CJK}{UTF8}{mj}且\end{CJK} $A$ \begin{CJK}{UTF8}{mj}的秩为\end{CJK} $r$, \begin{CJK}{UTF8}{mj}则行列式\end{CJK} $|A-E|=$

  \item \begin{CJK}{UTF8}{mj}若\end{CJK} $A=\left(\begin{array}{lll}a & b & b \\ b & a & b \\ b & b & a\end{array}\right)$, \begin{CJK}{UTF8}{mj}且\end{CJK} $A$ \begin{CJK}{UTF8}{mj}的伴随矩阵\end{CJK} $A^{*}$ \begin{CJK}{UTF8}{mj}的秩为\end{CJK} 1 , \begin{CJK}{UTF8}{mj}则\end{CJK} $a, b$ \begin{CJK}{UTF8}{mj}应满足关系\end{CJK}

  \item \begin{CJK}{UTF8}{mj}设\end{CJK} $A$ \begin{CJK}{UTF8}{mj}为\end{CJK} $m \times n$ \begin{CJK}{UTF8}{mj}阶实矩阵\end{CJK}, $b$ \begin{CJK}{UTF8}{mj}是\end{CJK} $m$ \begin{CJK}{UTF8}{mj}维实列向量\end{CJK}, \begin{CJK}{UTF8}{mj}若复向量\end{CJK} $Y, Z$ \begin{CJK}{UTF8}{mj}是\end{CJK} $A X=b$ \begin{CJK}{UTF8}{mj}在复数域上的两个互异解\end{CJK}, \begin{CJK}{UTF8}{mj}则\end{CJK} $A X=0$ \begin{CJK}{UTF8}{mj}在实数域上\end{CJK} (\begin{CJK}{UTF8}{mj}选填\end{CJK}“\begin{CJK}{UTF8}{mj}必\end{CJK}"\begin{CJK}{UTF8}{mj}或\end{CJK}“\begin{CJK}{UTF8}{mj}末必\end{CJK}") \begin{CJK}{UTF8}{mj}有非零解\end{CJK}.

  \item \begin{CJK}{UTF8}{mj}设\end{CJK} $4 \times 5$ \begin{CJK}{UTF8}{mj}阶矩阵\end{CJK} $A$ \begin{CJK}{UTF8}{mj}经过一系列初等行变换后变为\end{CJK} $A=\left(\begin{array}{ccccc}1 & 2 & 0 & 0 & 3 \\ 0 & 0 & 1 & 0 & 0 \\ 0 & 0 & 0 & 1 & -1 \\ 0 & 0 & 0 & 0 & 0\end{array}\right)$, \begin{CJK}{UTF8}{mj}则\end{CJK} $A$ \begin{CJK}{UTF8}{mj}的列向量组\end{CJK} $\alpha_{1}, \alpha_{2}, \alpha_{3}, \alpha_{4}, \alpha_{5}$ \begin{CJK}{UTF8}{mj}的一个极大线性无关组是\end{CJK} ,\begin{CJK}{UTF8}{mj}其余向量用该极大线性无关组线性表示为\end{CJK}

  \item \begin{CJK}{UTF8}{mj}设\end{CJK} $\mathbb{Q}(\sqrt{2})=\{a+b \sqrt{2} \mid a, b \in \mathbb{Q}\}$, \begin{CJK}{UTF8}{mj}则\end{CJK} $\mathbb{Q}(\sqrt{2})$ \begin{CJK}{UTF8}{mj}是有理数域\end{CJK} $\mathbb{Q}$ \begin{CJK}{UTF8}{mj}上\end{CJK} \begin{CJK}{UTF8}{mj}维线性空间\end{CJK}, \begin{CJK}{UTF8}{mj}是其一组\end{CJK} \begin{CJK}{UTF8}{mj}基\end{CJK}.

  \item \begin{CJK}{UTF8}{mj}设\end{CJK} $B$ \begin{CJK}{UTF8}{mj}是\end{CJK} $n \times n$ \begin{CJK}{UTF8}{mj}阶正定矩阵\end{CJK}, $C$ \begin{CJK}{UTF8}{mj}是秩为\end{CJK} $m$ \begin{CJK}{UTF8}{mj}的\end{CJK} $n \times m$ \begin{CJK}{UTF8}{mj}阶实矩阵\end{CJK}, $n>m$, \begin{CJK}{UTF8}{mj}则\end{CJK} $A=\left(\begin{array}{cc}B & C \\ C^{\prime} & 0\end{array}\right)$ \begin{CJK}{UTF8}{mj}有\end{CJK} \begin{CJK}{UTF8}{mj}个正\end{CJK} \begin{CJK}{UTF8}{mj}特征值\end{CJK}, \begin{CJK}{UTF8}{mj}有\end{CJK} \begin{CJK}{UTF8}{mj}个负特征值\end{CJK}.

  \item \begin{CJK}{UTF8}{mj}设\end{CJK} $p$ \begin{CJK}{UTF8}{mj}是素数\end{CJK}, \begin{CJK}{UTF8}{mj}则\end{CJK} $x^{p}+p x+p$ \begin{CJK}{UTF8}{mj}和\end{CJK} $x^{2}+p$ \begin{CJK}{UTF8}{mj}的最大公因式是\end{CJK}

\end{enumerate}
\section{一、解答题}
\begin{enumerate}
  \item ( 20 \begin{CJK}{UTF8}{mj}分\end{CJK}) \begin{CJK}{UTF8}{mj}已知线性方程组\end{CJK}
\end{enumerate}
$$
\left\{\begin{array}{l}
x_{1}+x_{2}+x_{3}=1 \\
2 x_{1}+(a+2) x_{2}+(a+1) x_{3}=a+3 \\
x_{1}+2 x_{2}+a x_{3}=3
\end{array}\right.
$$
\begin{CJK}{UTF8}{mj}有无穷多解\end{CJK}, \begin{CJK}{UTF8}{mj}又设\end{CJK} $A$ \begin{CJK}{UTF8}{mj}是\end{CJK} 3 \begin{CJK}{UTF8}{mj}阶方阵\end{CJK}, $\alpha_{1}=(1, a, 0)^{\prime}, \alpha_{2}=(-a, 1,0)^{\prime}, \alpha_{3}=(0,0, a)^{\prime}$ \begin{CJK}{UTF8}{mj}为\end{CJK} $A$ \begin{CJK}{UTF8}{mj}属于特征值\end{CJK} $\lambda_{1}=$ $1, \lambda_{2}=-2, \lambda_{3}=-1$ \begin{CJK}{UTF8}{mj}的特征向量\end{CJK}.

(1) \begin{CJK}{UTF8}{mj}求矩阵\end{CJK} $A ;$

(2) \begin{CJK}{UTF8}{mj}求行列式\end{CJK} $\left|A^{*}+2 E\right|$.

\begin{enumerate}
  \setcounter{enumi}{2}
  \item (15 \begin{CJK}{UTF8}{mj}分\end{CJK}) \begin{CJK}{UTF8}{mj}设多项式\end{CJK} $f(x), g(x)$ \begin{CJK}{UTF8}{mj}互素\end{CJK}, \begin{CJK}{UTF8}{mj}证明\end{CJK}
\end{enumerate}
$$
(f(x) g(x), f(x)+g(x))=1 .
$$

\begin{enumerate}
  \setcounter{enumi}{3}
  \item (15 \begin{CJK}{UTF8}{mj}分\end{CJK}) \begin{CJK}{UTF8}{mj}设\end{CJK} $A, B$ \begin{CJK}{UTF8}{mj}分别是\end{CJK} $m \times n, n \times s$, \begin{CJK}{UTF8}{mj}且\end{CJK} $\mathrm{r}(A)=\mathrm{r}(A B)$. \begin{CJK}{UTF8}{mj}证明\end{CJK}: \begin{CJK}{UTF8}{mj}存在\end{CJK} $s \times n$ \begin{CJK}{UTF8}{mj}矩阵\end{CJK} $C$, \begin{CJK}{UTF8}{mj}使得\end{CJK}
\end{enumerate}
$$
A=A B C
$$

\begin{enumerate}
  \setcounter{enumi}{4}
  \item ( 15 \begin{CJK}{UTF8}{mj}分\end{CJK}) \begin{CJK}{UTF8}{mj}设\end{CJK} $\alpha, \beta$ \begin{CJK}{UTF8}{mj}是不同的\end{CJK} $n(>1)$ \begin{CJK}{UTF8}{mj}维单位列向量\end{CJK}, $A=\alpha \beta^{\prime} \neq 0$, \begin{CJK}{UTF8}{mj}证明\end{CJK}:
\end{enumerate}
(1) 0 \begin{CJK}{UTF8}{mj}是\end{CJK} $A$ \begin{CJK}{UTF8}{mj}的一个特征值\end{CJK};

(2) $A$ \begin{CJK}{UTF8}{mj}可对角化的充要条件的\end{CJK} $\alpha$ \begin{CJK}{UTF8}{mj}与\end{CJK} $\beta$ \begin{CJK}{UTF8}{mj}不正交\end{CJK}.

\begin{enumerate}
  \setcounter{enumi}{5}
  \item ( 15 \begin{CJK}{UTF8}{mj}分\end{CJK}) \begin{CJK}{UTF8}{mj}设\end{CJK} $V_{1}$ \begin{CJK}{UTF8}{mj}是数域\end{CJK} $\mathbb{F}$ \begin{CJK}{UTF8}{mj}上\end{CJK}
\end{enumerate}
$$
\left\{\begin{array}{l}
x_{1}+x_{2}-x_{3}=0 \\
x_{2}+x_{3}=0
\end{array}\right.
$$
\begin{CJK}{UTF8}{mj}的解空间\end{CJK}, $V_{2}$ \begin{CJK}{UTF8}{mj}是\end{CJK} $\mathbb{F}$ \begin{CJK}{UTF8}{mj}上\end{CJK} $x_{2}-x_{3}=0$ \begin{CJK}{UTF8}{mj}的解空间\end{CJK}. \begin{CJK}{UTF8}{mj}证明\end{CJK}
$$
\mathbb{F}^{3}=V_{1} \oplus V_{2}
$$

\begin{enumerate}
  \setcounter{enumi}{6}
  \item (15 \begin{CJK}{UTF8}{mj}分\end{CJK}) \begin{CJK}{UTF8}{mj}设\end{CJK} $\mathscr{A}, \mathscr{B}$ \begin{CJK}{UTF8}{mj}是\end{CJK} $n$ \begin{CJK}{UTF8}{mj}维线性空间\end{CJK} $V$ \begin{CJK}{UTF8}{mj}的线性变换\end{CJK}, \begin{CJK}{UTF8}{mj}且\end{CJK} $\mathscr{A}$ \begin{CJK}{UTF8}{mj}有\end{CJK} $n$ \begin{CJK}{UTF8}{mj}个互异的特征值\end{CJK}. \begin{CJK}{UTF8}{mj}证明\end{CJK}:
\end{enumerate}
(1) $\mathscr{A}$ \begin{CJK}{UTF8}{mj}的特征向量都是\end{CJK} $\mathscr{B}$ \begin{CJK}{UTF8}{mj}的特征向量的充要条件是\end{CJK}
$$
\mathscr{A} \mathscr{B}=\mathscr{B} \mathscr{A}
$$
(2) \begin{CJK}{UTF8}{mj}若\end{CJK} $\mathscr{A} \mathscr{B}=\mathscr{B} \mathscr{A}$, \begin{CJK}{UTF8}{mj}则\end{CJK} $\mathscr{B}$ \begin{CJK}{UTF8}{mj}是\end{CJK} $\varepsilon, \mathscr{A}, \mathscr{A}^{2}, \cdots, \mathscr{A}^{n-1}$ \begin{CJK}{UTF8}{mj}的线性组合\end{CJK}, \begin{CJK}{UTF8}{mj}其中\end{CJK} $\varepsilon$ \begin{CJK}{UTF8}{mj}是\end{CJK} $V$ \begin{CJK}{UTF8}{mj}的恒等变换\end{CJK}.

\begin{enumerate}
  \setcounter{enumi}{7}
  \item (10 \begin{CJK}{UTF8}{mj}分\end{CJK}) \begin{CJK}{UTF8}{mj}设\end{CJK} $A, A_{1}, A_{2}$ \begin{CJK}{UTF8}{mj}为\end{CJK} $n$ \begin{CJK}{UTF8}{mj}阶方阵\end{CJK}, \begin{CJK}{UTF8}{mj}满足\end{CJK} $A^{2}=A, A=A_{1}+A_{2}, \mathrm{r}(A)=\mathrm{r}\left(A_{1}\right)+\mathrm{r}\left(A_{2}\right)$, \begin{CJK}{UTF8}{mj}其中\end{CJK} $\mathrm{r}(A)$ \begin{CJK}{UTF8}{mj}表示\end{CJK} $A$ \begin{CJK}{UTF8}{mj}的\end{CJK} \begin{CJK}{UTF8}{mj}秩\end{CJK}, \begin{CJK}{UTF8}{mj}求证\end{CJK}:
\end{enumerate}
$$
\begin{gathered}
A_{i}^{2}=A_{i},(i=1,2) \\
A_{i}, A_{j}=0,(1 \leq i \neq j \leq 2)
\end{gathered}
$$

\section{1. 厦门大学 2013 年研究生入学考试试题高等代数}
\begin{CJK}{UTF8}{mj}李扬\end{CJK}

\begin{CJK}{UTF8}{mj}微信公众号\end{CJK}: sxkyliyang

\begin{CJK}{UTF8}{mj}一\end{CJK}、\begin{CJK}{UTF8}{mj}填空题\end{CJK}(\begin{CJK}{UTF8}{mj}每小题\end{CJK} 6 \begin{CJK}{UTF8}{mj}分\end{CJK}, \begin{CJK}{UTF8}{mj}共\end{CJK} 48 \begin{CJK}{UTF8}{mj}分\end{CJK})

\begin{enumerate}
  \item \begin{CJK}{UTF8}{mj}设\end{CJK} $\alpha, \beta, \gamma_{1}, \gamma_{2}, \gamma_{3}$ \begin{CJK}{UTF8}{mj}均为\end{CJK} 4 \begin{CJK}{UTF8}{mj}维列向量\end{CJK}, \begin{CJK}{UTF8}{mj}已知\end{CJK} 4 \begin{CJK}{UTF8}{mj}阶方阵\end{CJK}, $A=\left(\alpha, \gamma_{1}, \gamma_{2}, \gamma_{3}\right), B=\left(\beta, \gamma_{1}, \gamma_{2}, \gamma_{3}\right)$ \begin{CJK}{UTF8}{mj}且它们的行列式\end{CJK} \begin{CJK}{UTF8}{mj}分别为\end{CJK} $\operatorname{det} A=5, \operatorname{det} B=2$, \begin{CJK}{UTF8}{mj}则\end{CJK} $A+B$ \begin{CJK}{UTF8}{mj}的行列式\end{CJK} $\operatorname{det}(A+B)=$

  \item \begin{CJK}{UTF8}{mj}设\end{CJK} $A$ \begin{CJK}{UTF8}{mj}是\end{CJK} 3 \begin{CJK}{UTF8}{mj}阶方阵\end{CJK}, $P^{-1} A P=\left(\begin{array}{lll}1 & 0 & 0 \\ 0 & 2 & 0 \\ 0 & 0 & 3\end{array}\right)$, \begin{CJK}{UTF8}{mj}若\end{CJK} $P=\left(X_{1}, X_{2}, X_{3}\right), Q=\left(X_{1}+X_{2}, X_{2}, X_{3}\right)$, \begin{CJK}{UTF8}{mj}则\end{CJK} $Q^{-1} A Q=$

  \item \begin{CJK}{UTF8}{mj}设数域\end{CJK} $\mathbb{F}, K$ \begin{CJK}{UTF8}{mj}满足\end{CJK} $\mathbb{F} \subset K, \alpha_{1}, \alpha_{2}, \cdots, \alpha_{s}$ \begin{CJK}{UTF8}{mj}是\end{CJK} $\mathbb{F}$ \begin{CJK}{UTF8}{mj}上\end{CJK} $n$ \begin{CJK}{UTF8}{mj}维列向量\end{CJK}. \begin{CJK}{UTF8}{mj}若\end{CJK} $\alpha_{1}, \alpha_{2}, \cdots, \alpha_{s}$ \begin{CJK}{UTF8}{mj}在\end{CJK} $K$ \begin{CJK}{UTF8}{mj}上线性相关\end{CJK}, \begin{CJK}{UTF8}{mj}则\end{CJK} $\alpha_{1}, \alpha_{2}, \cdots, \alpha_{s}$ \begin{CJK}{UTF8}{mj}在\end{CJK} $\mathbb{F}$ \begin{CJK}{UTF8}{mj}上\end{CJK} (\begin{CJK}{UTF8}{mj}选填\end{CJK}“\begin{CJK}{UTF8}{mj}必\end{CJK}”\begin{CJK}{UTF8}{mj}或\end{CJK}“\begin{CJK}{UTF8}{mj}末必\end{CJK}”) \begin{CJK}{UTF8}{mj}线性相关\end{CJK}.

  \item \begin{CJK}{UTF8}{mj}设\end{CJK} $A^{*}$ \begin{CJK}{UTF8}{mj}是\end{CJK} $n(>2)$ \begin{CJK}{UTF8}{mj}阶方阵\end{CJK} $A$ \begin{CJK}{UTF8}{mj}的伴随矩阵\end{CJK}, \begin{CJK}{UTF8}{mj}若\end{CJK} $A^{*}$ \begin{CJK}{UTF8}{mj}的秩\end{CJK} $\mathrm{r}\left(A^{*}\right)=1$, \begin{CJK}{UTF8}{mj}则齐次线性方程组\end{CJK} $A X=0$ \begin{CJK}{UTF8}{mj}的解空间维数为\end{CJK}

  \item $x^{2012}+1$ \begin{CJK}{UTF8}{mj}除以\end{CJK} $(x-1)^{2}$ \begin{CJK}{UTF8}{mj}的余式为\end{CJK}

  \item \begin{CJK}{UTF8}{mj}设实数域\end{CJK} $\mathbb{R}$ \begin{CJK}{UTF8}{mj}上的\end{CJK} $n$ \begin{CJK}{UTF8}{mj}维线性空间\end{CJK} $V$ \begin{CJK}{UTF8}{mj}的线性变换\end{CJK} $\mathscr{A}$ \begin{CJK}{UTF8}{mj}满足\end{CJK} $\mathscr{A}^{3}+\mathscr{A}=0$, \begin{CJK}{UTF8}{mj}则\end{CJK} $\mathscr{A}$ \begin{CJK}{UTF8}{mj}在复数域\end{CJK} $\mathbb{C}$ \begin{CJK}{UTF8}{mj}上的特征值只可能为\end{CJK} $\mathscr{A}$ \begin{CJK}{UTF8}{mj}的迹\end{CJK} (\begin{CJK}{UTF8}{mj}即\end{CJK} $\mathscr{A}$ \begin{CJK}{UTF8}{mj}在某一组基下的矩阵的迹\end{CJK})\begin{CJK}{UTF8}{mj}为\end{CJK}

  \item \begin{CJK}{UTF8}{mj}已知方阵\end{CJK} $A$ \begin{CJK}{UTF8}{mj}的特征矩阵\end{CJK} $\lambda E-A$ \begin{CJK}{UTF8}{mj}经过初等变换化为\end{CJK} $\left(\begin{array}{ccc}1 & 0 & 0 \\ 0 & \lambda-1 & 0 \\ 0 & 0 & (\lambda-1)^{2}\end{array}\right)$, \begin{CJK}{UTF8}{mj}则\end{CJK} $A$ \begin{CJK}{UTF8}{mj}的\end{CJK} Jordan \begin{CJK}{UTF8}{mj}标准型为\end{CJK} 8. \begin{CJK}{UTF8}{mj}设实矩阵\end{CJK} $\left(\begin{array}{ll}2 & 1 \\ 1 & c\end{array}\right)$, \begin{CJK}{UTF8}{mj}若\end{CJK} 2 \begin{CJK}{UTF8}{mj}维实向量空间\end{CJK} $\mathbb{R}^{2}$ \begin{CJK}{UTF8}{mj}关于内积\end{CJK} $(\alpha, \beta)=\alpha^{\prime} A \alpha$ \begin{CJK}{UTF8}{mj}构成的欧氏空间\end{CJK}, \begin{CJK}{UTF8}{mj}则\end{CJK} $c=$

\end{enumerate}
\section{一、解答题}
\begin{enumerate}
  \item ( 20 \begin{CJK}{UTF8}{mj}分\end{CJK}) \begin{CJK}{UTF8}{mj}已知\end{CJK} $\alpha_{1}=(1,-2,1)^{T}, \alpha_{2}=(-1, a, 1)^{T}$ \begin{CJK}{UTF8}{mj}分别是三阶不可逆实对称矩阵\end{CJK} $A$ \begin{CJK}{UTF8}{mj}的属于特征值\end{CJK} $\lambda_{1}=1, \lambda_{2}=$ $-1$ \begin{CJK}{UTF8}{mj}的特征向量\end{CJK}. \begin{CJK}{UTF8}{mj}求\end{CJK}:
\end{enumerate}
(1) $A$;

(2) $A^{2012} \beta$, \begin{CJK}{UTF8}{mj}其中\end{CJK} $\beta=(1,1,1)^{T}$.

\begin{enumerate}
  \setcounter{enumi}{2}
  \item ( 20 \begin{CJK}{UTF8}{mj}分\end{CJK}) \begin{CJK}{UTF8}{mj}已知\end{CJK} $X_{1}, X_{2}, X_{3}, X_{4}$ \begin{CJK}{UTF8}{mj}是非齐次线性方程组\end{CJK} $A X=\beta \neq 0$ \begin{CJK}{UTF8}{mj}的不同解\end{CJK}. \begin{CJK}{UTF8}{mj}证明\end{CJK}:
\end{enumerate}
(1) \begin{CJK}{UTF8}{mj}若\end{CJK} $X_{1}, X_{2}, X_{3}$ \begin{CJK}{UTF8}{mj}线性相关\end{CJK}, \begin{CJK}{UTF8}{mj}则\end{CJK} $X_{1}-X_{2}, X_{1}-X_{3}$ \begin{CJK}{UTF8}{mj}必线性相关\end{CJK};

(2) \begin{CJK}{UTF8}{mj}若\end{CJK} $X_{1}, X_{2}, X_{3}, X_{4}$ \begin{CJK}{UTF8}{mj}线性无关\end{CJK}, \begin{CJK}{UTF8}{mj}则\end{CJK} $X_{1}-X_{2}, X_{1}-X_{3}, X_{1}-X_{4}$ \begin{CJK}{UTF8}{mj}必线性无关\end{CJK}, \begin{CJK}{UTF8}{mj}且是\end{CJK} $A X=0$ \begin{CJK}{UTF8}{mj}的解\end{CJK}.

\begin{enumerate}
  \setcounter{enumi}{3}
  \item (16 \begin{CJK}{UTF8}{mj}分\end{CJK}) \begin{CJK}{UTF8}{mj}设\end{CJK} $A, B, C, D$ \begin{CJK}{UTF8}{mj}为\end{CJK} $n$ \begin{CJK}{UTF8}{mj}阶方阵\end{CJK}. \begin{CJK}{UTF8}{mj}证明\end{CJK}:
\end{enumerate}
(1) \begin{CJK}{UTF8}{mj}证明\end{CJK}: \begin{CJK}{UTF8}{mj}若\end{CJK} $A$ \begin{CJK}{UTF8}{mj}可逆\end{CJK}, \begin{CJK}{UTF8}{mj}则\end{CJK} $\left|\begin{array}{cc}A & B \\ C & D\end{array}\right|=(\operatorname{det} A) \operatorname{det}\left(D-C A^{-1} B\right)$;

(2) \begin{CJK}{UTF8}{mj}若\end{CJK} $r\left(\begin{array}{cc}A & B \\ C & D\end{array}\right)=n$, \begin{CJK}{UTF8}{mj}问\end{CJK} $\left(\begin{array}{cc}\operatorname{det} A & \operatorname{det} B \\ \operatorname{det} C & \operatorname{det} D\end{array}\right)$ \begin{CJK}{UTF8}{mj}是否可逆\end{CJK}? \begin{CJK}{UTF8}{mj}若可逆\end{CJK}, \begin{CJK}{UTF8}{mj}请证明\end{CJK}: \begin{CJK}{UTF8}{mj}若不可逆\end{CJK}, \begin{CJK}{UTF8}{mj}请举反例\end{CJK}. 4. ( 16 \begin{CJK}{UTF8}{mj}分\end{CJK}) \begin{CJK}{UTF8}{mj}设\end{CJK} $F$ \begin{CJK}{UTF8}{mj}是数域\end{CJK}, $V_{1}$ \begin{CJK}{UTF8}{mj}是\end{CJK} $F$ \begin{CJK}{UTF8}{mj}上\end{CJK} $n$ \begin{CJK}{UTF8}{mj}阶上三角阵的全体\end{CJK}, $V_{2}$ \begin{CJK}{UTF8}{mj}是\end{CJK} $F$ \begin{CJK}{UTF8}{mj}上\end{CJK} $n$ \begin{CJK}{UTF8}{mj}阶反对称阵的全体\end{CJK}, \begin{CJK}{UTF8}{mj}即\end{CJK}
$$
V_{1}=\left\{A=\left(a_{i j}\right)_{n \times n} \in F^{n \times n} \mid a_{i j}=0,1 \leq j<i \leq n\right\}, V_{2}=\left\{A \in F^{n \times n} \mid A^{T}=-A\right\},
$$
\begin{CJK}{UTF8}{mj}其中\end{CJK} $A^{T}$ \begin{CJK}{UTF8}{mj}表示矩阵\end{CJK} $A$ \begin{CJK}{UTF8}{mj}的转置\end{CJK}.

(1) \begin{CJK}{UTF8}{mj}证明\end{CJK}: $F^{n \times n}$ \begin{CJK}{UTF8}{mj}中任意矩阵\end{CJK} $A$ \begin{CJK}{UTF8}{mj}均可表示为一个上三角阵和一个反对称阵的和\end{CJK};

(2) \begin{CJK}{UTF8}{mj}证明\end{CJK}: $F^{n \times n}=V_{1} \oplus V_{2}$.

\begin{enumerate}
  \setcounter{enumi}{5}
  \item (12 \begin{CJK}{UTF8}{mj}分\end{CJK}) \begin{CJK}{UTF8}{mj}设\end{CJK} $F^{2 \times 2}$ \begin{CJK}{UTF8}{mj}是数域\end{CJK} $F$ \begin{CJK}{UTF8}{mj}上的\end{CJK} 2 \begin{CJK}{UTF8}{mj}阶方阵的全体\end{CJK}, $F^{2 \times 2}$ \begin{CJK}{UTF8}{mj}的线性变换\end{CJK} $\mathscr{A}$ \begin{CJK}{UTF8}{mj}在基\end{CJK} $E_{11}, E_{12}, E_{21}, E_{22}$ \begin{CJK}{UTF8}{mj}的矩阵为\end{CJK}
\end{enumerate}
$$
A=\left(\begin{array}{llll}
2 & 0 & 2 & 0 \\
0 & 1 & 0 & 1 \\
2 & 0 & 2 & 0 \\
0 & 1 & 0 & 1
\end{array}\right)
$$
\begin{CJK}{UTF8}{mj}即\end{CJK}
$$
\mathscr{A}\left(E_{11}, E_{12}, E_{21}, E_{22}\right)=\left(E_{11}, E_{12}, E_{21}, E_{22}\right) A,
$$
\begin{CJK}{UTF8}{mj}其中\end{CJK} $E_{i j}$ \begin{CJK}{UTF8}{mj}为第\end{CJK} $(i, j)$ \begin{CJK}{UTF8}{mj}元素为\end{CJK} 1 , \begin{CJK}{UTF8}{mj}其余元素全为\end{CJK} 0 \begin{CJK}{UTF8}{mj}的\end{CJK} 2 \begin{CJK}{UTF8}{mj}阶方阵\end{CJK}.

(1) \begin{CJK}{UTF8}{mj}求\end{CJK} $\mathscr{A}$ \begin{CJK}{UTF8}{mj}的像空间\end{CJK} $\operatorname{Im} \varphi$ \begin{CJK}{UTF8}{mj}和\end{CJK} $\mathscr{A} N$ \begin{CJK}{UTF8}{mj}的核空间\end{CJK} $\operatorname{ker} \mathscr{A}$, \begin{CJK}{UTF8}{mj}要求写出过程和理由\end{CJK}.

(2) \begin{CJK}{UTF8}{mj}求\end{CJK} $\operatorname{Im} \mathscr{A}$ \begin{CJK}{UTF8}{mj}和\end{CJK} ker $\mathscr{A}$ \begin{CJK}{UTF8}{mj}的基和维数\end{CJK}.

\begin{enumerate}
  \setcounter{enumi}{6}
  \item ( 12 \begin{CJK}{UTF8}{mj}分\end{CJK}) \begin{CJK}{UTF8}{mj}设\end{CJK} $V_{1}, V_{2}$ \begin{CJK}{UTF8}{mj}是\end{CJK} $n$ \begin{CJK}{UTF8}{mj}维欧氏空间\end{CJK} $V$ \begin{CJK}{UTF8}{mj}的子空间\end{CJK}, \begin{CJK}{UTF8}{mj}且\end{CJK} $V_{1}$ \begin{CJK}{UTF8}{mj}的维数小于\end{CJK} $V_{2}$ \begin{CJK}{UTF8}{mj}的维数\end{CJK}, \begin{CJK}{UTF8}{mj}即\end{CJK} $\operatorname{dim} V_{1}<\operatorname{dim} V_{2}$. \begin{CJK}{UTF8}{mj}证明\end{CJK}: $V_{2}$ \begin{CJK}{UTF8}{mj}中必有一非零向量正交与\end{CJK} $V_{1}$ \begin{CJK}{UTF8}{mj}的一切向量\end{CJK}.

  \item (6 \begin{CJK}{UTF8}{mj}分\end{CJK}) \begin{CJK}{UTF8}{mj}设\end{CJK} $V$ \begin{CJK}{UTF8}{mj}是数域\end{CJK} $F$ \begin{CJK}{UTF8}{mj}上\end{CJK} $n$ \begin{CJK}{UTF8}{mj}阶方阵全体\end{CJK}, $V=\left\{A \mid A \in F^{n \times n}\right\}$, \begin{CJK}{UTF8}{mj}定义\end{CJK} $V$ \begin{CJK}{UTF8}{mj}的线性变换\end{CJK} $\sigma: A \mapsto 2 A-3 A^{T}$, \begin{CJK}{UTF8}{mj}其中\end{CJK} $A^{T}$ \begin{CJK}{UTF8}{mj}表示矩阵\end{CJK} $A$ \begin{CJK}{UTF8}{mj}的转置\end{CJK}.

\end{enumerate}
(1) \begin{CJK}{UTF8}{mj}求\end{CJK} $\sigma$ \begin{CJK}{UTF8}{mj}的特征多项式\end{CJK};

(2) \begin{CJK}{UTF8}{mj}证明\end{CJK} $\sigma$ \begin{CJK}{UTF8}{mj}可对角化\end{CJK}.

\section{2. 厦门大学 2014 年研究生入学考试试题高等代数 
 李扬 
 微信公众号: sxkyliyang}
\begin{CJK}{UTF8}{mj}一\end{CJK}、\begin{CJK}{UTF8}{mj}填空题\end{CJK}(\begin{CJK}{UTF8}{mj}每小题\end{CJK} 6 \begin{CJK}{UTF8}{mj}分\end{CJK}, \begin{CJK}{UTF8}{mj}共\end{CJK} 48 \begin{CJK}{UTF8}{mj}分\end{CJK})

\begin{enumerate}
  \item \begin{CJK}{UTF8}{mj}将\end{CJK} 2 \begin{CJK}{UTF8}{mj}阶矩阵\end{CJK} $\left(\begin{array}{cc}a & 0 \\ 0 & a^{-1}\end{array}\right)$ \begin{CJK}{UTF8}{mj}表示成若干个形如\end{CJK} $\left(\begin{array}{ll}1 & 0 \\ b & 1\end{array}\right)$ \begin{CJK}{UTF8}{mj}或\end{CJK} $\left(\begin{array}{ll}1 & b \\ 0 & 1\end{array}\right)$ \begin{CJK}{UTF8}{mj}的初等矩阵的乘积\end{CJK}

  \item $A, B, C, X, Y \in \mathbb{F}^{n \times n}$, \begin{CJK}{UTF8}{mj}且\end{CJK} $A X+Y B=C$, \begin{CJK}{UTF8}{mj}则\end{CJK} $\mathrm{r}\left(\begin{array}{cc}A & C \\ O & B\end{array}\right)$ \begin{CJK}{UTF8}{mj}与\end{CJK} $\mathrm{r}(A), \mathrm{r}(B)$ \begin{CJK}{UTF8}{mj}的关系是\end{CJK}

  \item \begin{CJK}{UTF8}{mj}设\end{CJK} $\alpha_{1}, \alpha_{2}, \alpha_{3}, \alpha_{4}$ \begin{CJK}{UTF8}{mj}是\end{CJK} 4 \begin{CJK}{UTF8}{mj}维非零列向量\end{CJK}, $A=\left(\alpha_{1}, \alpha_{2}, \alpha_{3}, \alpha_{4}\right)$. \begin{CJK}{UTF8}{mj}已知齐次线性方程组\end{CJK} $A X=0$ \begin{CJK}{UTF8}{mj}的通解为\end{CJK} $k(0,1,1,0)^{\prime}$, \begin{CJK}{UTF8}{mj}则\end{CJK} \begin{CJK}{UTF8}{mj}是方程组\end{CJK} $A^{*} X=0$ \begin{CJK}{UTF8}{mj}的一个基础解系\end{CJK}, \begin{CJK}{UTF8}{mj}其中\end{CJK} $A^{*}$ \begin{CJK}{UTF8}{mj}是\end{CJK} $A$ \begin{CJK}{UTF8}{mj}的伴随矩阵\end{CJK}.

  \item \begin{CJK}{UTF8}{mj}设\end{CJK} 3 \begin{CJK}{UTF8}{mj}阶矩阵\end{CJK} $A=\left(\begin{array}{ccc}1 & 2 & -2 \\ 2 & 1 & 2 \\ 3 & 0 & 4\end{array}\right), 3$ \begin{CJK}{UTF8}{mj}维列向量\end{CJK} $\alpha=\left(\begin{array}{c}a \\ 1 \\ 1\end{array}\right)$. \begin{CJK}{UTF8}{mj}已知\end{CJK} $A \alpha$ \begin{CJK}{UTF8}{mj}与\end{CJK} $\alpha$ \begin{CJK}{UTF8}{mj}线性相关\end{CJK}, \begin{CJK}{UTF8}{mj}则\end{CJK} $a=$

  \item \begin{CJK}{UTF8}{mj}设变换\end{CJK} $\mathscr{A}: a+b i \mapsto a-b i(a, b \in \mathbb{R})$. \begin{CJK}{UTF8}{mj}则\end{CJK} $\mathbb{C}$ \begin{CJK}{UTF8}{mj}作为\end{CJK} $\mathbb{R}$ \begin{CJK}{UTF8}{mj}上的线性空间\end{CJK}, $\mathscr{A}$ (\begin{CJK}{UTF8}{mj}选填\end{CJK} “\begin{CJK}{UTF8}{mj}是\end{CJK}” \begin{CJK}{UTF8}{mj}或\end{CJK} “\begin{CJK}{UTF8}{mj}不是\end{CJK}") $\mathbb{C}$ \begin{CJK}{UTF8}{mj}的\end{CJK} \begin{CJK}{UTF8}{mj}线性变换\end{CJK}; $\mathbb{C}$ \begin{CJK}{UTF8}{mj}作为\end{CJK} $\mathbb{C}$ \begin{CJK}{UTF8}{mj}上的线性空间\end{CJK}, $\mathscr{A}$ (\begin{CJK}{UTF8}{mj}选填\end{CJK}“\begin{CJK}{UTF8}{mj}是\end{CJK}"\begin{CJK}{UTF8}{mj}或\end{CJK}“\begin{CJK}{UTF8}{mj}不是\end{CJK}") $\mathbb{C}$ \begin{CJK}{UTF8}{mj}的线性变换\end{CJK}.

  \item \begin{CJK}{UTF8}{mj}设\end{CJK} $f(x), p(x)$ \begin{CJK}{UTF8}{mj}是\end{CJK} $\mathbb{F}$ \begin{CJK}{UTF8}{mj}上的多项式\end{CJK}, \begin{CJK}{UTF8}{mj}且\end{CJK} $p(x)$ \begin{CJK}{UTF8}{mj}在\end{CJK} $\mathbb{F}$ \begin{CJK}{UTF8}{mj}上不可约\end{CJK}. \begin{CJK}{UTF8}{mj}若\end{CJK} $f(x), p(x)$ \begin{CJK}{UTF8}{mj}在\end{CJK} $\mathbb{C}$ \begin{CJK}{UTF8}{mj}上有公共根\end{CJK}, \begin{CJK}{UTF8}{mj}则\end{CJK} $p(x)$ (\begin{CJK}{UTF8}{mj}选填\end{CJK}“\begin{CJK}{UTF8}{mj}必\end{CJK}” \begin{CJK}{UTF8}{mj}或\end{CJK}“\begin{CJK}{UTF8}{mj}末必\end{CJK}”) \begin{CJK}{UTF8}{mj}整除\end{CJK} $f(x)$.

  \item \begin{CJK}{UTF8}{mj}设\end{CJK} $X$ \begin{CJK}{UTF8}{mj}是\end{CJK} $n$ \begin{CJK}{UTF8}{mj}维行向量\end{CJK}, \begin{CJK}{UTF8}{mj}且\end{CJK} $X X^{\prime}=1$, \begin{CJK}{UTF8}{mj}则\end{CJK} $E-2 X^{\prime} X$ \begin{CJK}{UTF8}{mj}的所有特征值是\end{CJK} \begin{CJK}{UTF8}{mj}其中\end{CJK} $E_{n}$ \begin{CJK}{UTF8}{mj}表示\end{CJK} $n$ \begin{CJK}{UTF8}{mj}阶单位矩阵\end{CJK}.

  \item \begin{CJK}{UTF8}{mj}设实二次型\end{CJK} $f\left(x_{1}, x_{2}, x_{3}\right)=X^{\prime} A X$ \begin{CJK}{UTF8}{mj}经过正交替换\end{CJK} $X=Q Y$ \begin{CJK}{UTF8}{mj}后化为标准型\end{CJK} $f=y_{1}^{2}+y_{2}^{2}$, \begin{CJK}{UTF8}{mj}则\end{CJK} $a E+A$ \begin{CJK}{UTF8}{mj}为正定矩\end{CJK} \begin{CJK}{UTF8}{mj}阵的充分必要条件是\end{CJK} $a$ \begin{CJK}{UTF8}{mj}满足条件\end{CJK}

\end{enumerate}
\section{二、解答题}
\begin{enumerate}
  \item (12 \begin{CJK}{UTF8}{mj}分\end{CJK}) \begin{CJK}{UTF8}{mj}设矩阵\end{CJK} $A$ \begin{CJK}{UTF8}{mj}的伴随矩阵\end{CJK}
\end{enumerate}
$$
A^{*}=\left(\begin{array}{cccc}
1 & 0 & 0 & 0 \\
0 & 1 & 0 & 0 \\
1 & 0 & 1 & 0 \\
0 & -3 & 0 & 8
\end{array}\right)
$$
\begin{CJK}{UTF8}{mj}且\end{CJK} $A B A^{-1}=B A^{-1}+3 E$, \begin{CJK}{UTF8}{mj}求\end{CJK} $B$.

(1) $A$;

(2) $A^{2012} \beta$, \begin{CJK}{UTF8}{mj}其中\end{CJK} $\beta=(1,1,1)^{\prime}$.

\begin{enumerate}
  \setcounter{enumi}{2}
  \item (15 \begin{CJK}{UTF8}{mj}分\end{CJK}) \begin{CJK}{UTF8}{mj}设向量\end{CJK} $\beta$ \begin{CJK}{UTF8}{mj}可以由\end{CJK} $\alpha_{1}, \alpha_{2}, \cdots, \alpha_{s}$ \begin{CJK}{UTF8}{mj}线性表出\end{CJK}, \begin{CJK}{UTF8}{mj}但不能由\end{CJK} $\alpha_{1}, \alpha_{2}, \cdots, \alpha_{s-1}$ \begin{CJK}{UTF8}{mj}线性表出\end{CJK}. \begin{CJK}{UTF8}{mj}求证\end{CJK}:\begin{CJK}{UTF8}{mj}向量\end{CJK} $\alpha_{1}, \alpha_{2}, \cdots, \alpha_{s-1}, \alpha_{s}$ \begin{CJK}{UTF8}{mj}和向量组\end{CJK} $\alpha_{1}, \alpha_{2}, \cdots, \alpha_{s-1}, \beta$ \begin{CJK}{UTF8}{mj}等价\end{CJK}.

  \item (15 \begin{CJK}{UTF8}{mj}分\end{CJK}) \begin{CJK}{UTF8}{mj}设\end{CJK}

\end{enumerate}
$$
A=\left(\begin{array}{cc}
a_{11} & X^{T} \\
X & B
\end{array}\right)
$$
\begin{CJK}{UTF8}{mj}是实对称矩阵\end{CJK}, \begin{CJK}{UTF8}{mj}其中\end{CJK} $a_{11}<0, B$ \begin{CJK}{UTF8}{mj}是\end{CJK} $n-1$ \begin{CJK}{UTF8}{mj}阶正定矩阵\end{CJK}, \begin{CJK}{UTF8}{mj}求证\end{CJK}:

(1) $B-a_{11}^{-1} X X^{T}$ \begin{CJK}{UTF8}{mj}是正定矩阵\end{CJK};

(2) $A$ \begin{CJK}{UTF8}{mj}的符号差为\end{CJK} $n-2$. 4. ( 15 \begin{CJK}{UTF8}{mj}分\end{CJK}) \begin{CJK}{UTF8}{mj}设\end{CJK} $\mathscr{A}$ \begin{CJK}{UTF8}{mj}是欧氏空间\end{CJK} $V$ \begin{CJK}{UTF8}{mj}的正交变换\end{CJK}, $U$ \begin{CJK}{UTF8}{mj}是\end{CJK} $\mathscr{A}$-\begin{CJK}{UTF8}{mj}不变子空间\end{CJK}. \begin{CJK}{UTF8}{mj}证明\end{CJK} $U$ \begin{CJK}{UTF8}{mj}的正交补空间\end{CJK} $U^{\perp}$ \begin{CJK}{UTF8}{mj}也是\end{CJK} $\mathscr{A}-$ \begin{CJK}{UTF8}{mj}不变子\end{CJK} \begin{CJK}{UTF8}{mj}空间\end{CJK}.

\begin{enumerate}
  \setcounter{enumi}{5}
  \item (15 \begin{CJK}{UTF8}{mj}分\end{CJK}) \begin{CJK}{UTF8}{mj}设\end{CJK} $V=\{(a, b) \mid a, b \in \mathbb{R}\}$ \begin{CJK}{UTF8}{mj}中\end{CJK}, \begin{CJK}{UTF8}{mj}定义加法和数乘为\end{CJK}:
\end{enumerate}
$$
(a, b) \oplus(c, d)=(a+c, b+d+a c), k \odot(a, b)=\left(k a, k b+\frac{1}{2} k(k-1) a^{2}\right) .
$$
\begin{CJK}{UTF8}{mj}问\end{CJK} $V$ \begin{CJK}{UTF8}{mj}是否为\end{CJK} $\mathbb{R}$ \begin{CJK}{UTF8}{mj}上的线性空间\end{CJK}? \begin{CJK}{UTF8}{mj}若是\end{CJK}, \begin{CJK}{UTF8}{mj}求\end{CJK} $V$ \begin{CJK}{UTF8}{mj}的一组基和维数\end{CJK}; \begin{CJK}{UTF8}{mj}若不是\end{CJK}, \begin{CJK}{UTF8}{mj}举反例\end{CJK}.

\begin{enumerate}
  \setcounter{enumi}{6}
  \item (15 \begin{CJK}{UTF8}{mj}分\end{CJK}) \begin{CJK}{UTF8}{mj}求所有满足\end{CJK} $A^{2}=0$ \begin{CJK}{UTF8}{mj}的非零三阶矩阵\end{CJK} $A$.

  \item (15 \begin{CJK}{UTF8}{mj}分\end{CJK}) \begin{CJK}{UTF8}{mj}设\end{CJK} $f(x)$ \begin{CJK}{UTF8}{mj}是\end{CJK} $\mathbb{R}$ \begin{CJK}{UTF8}{mj}上首一多项式且无实根\end{CJK}, \begin{CJK}{UTF8}{mj}求证\end{CJK}:\begin{CJK}{UTF8}{mj}存在\end{CJK} $g(x), h(x)$, \begin{CJK}{UTF8}{mj}使得\end{CJK}

\end{enumerate}
$$
f(x)=g^{2}(x)+h^{2}(x)
$$
\begin{CJK}{UTF8}{mj}且\end{CJK} $g(x)$ \begin{CJK}{UTF8}{mj}的次数大于\end{CJK} $h(x)$ \begin{CJK}{UTF8}{mj}的次数\end{CJK}.

\section{3. 厦门大学 2015 年研究生入学考试试题高等代数}
\begin{CJK}{UTF8}{mj}李扬\end{CJK}

\begin{CJK}{UTF8}{mj}微信公众号\end{CJK}: sxkyliyang

\begin{CJK}{UTF8}{mj}一\end{CJK}、\begin{CJK}{UTF8}{mj}填空题\end{CJK}(\begin{CJK}{UTF8}{mj}每小题\end{CJK} 6 \begin{CJK}{UTF8}{mj}分\end{CJK}, \begin{CJK}{UTF8}{mj}共\end{CJK} 48 \begin{CJK}{UTF8}{mj}分\end{CJK})

\begin{enumerate}
  \item \begin{CJK}{UTF8}{mj}已知\end{CJK} $2 n$ \begin{CJK}{UTF8}{mj}阶行列式\end{CJK} $D$ \begin{CJK}{UTF8}{mj}的第一列元素及其余子式都等于\end{CJK} $a$, \begin{CJK}{UTF8}{mj}则行列式\end{CJK} $D=$

  \item \begin{CJK}{UTF8}{mj}设\end{CJK} $B=\left(\begin{array}{lll}b_{11} & b_{12} & b_{13} \\ b_{21} & b_{22} & b_{13} \\ b_{31} & b_{32} & b_{33}\end{array}\right)$ \begin{CJK}{UTF8}{mj}是\end{CJK} 3 \begin{CJK}{UTF8}{mj}阶可逆矩阵\end{CJK}, \begin{CJK}{UTF8}{mj}满足\end{CJK} $B A=\left(\begin{array}{ccc}b_{12} & 2 b_{11} & -b_{13} \\ b_{22} & 2 b_{21} & -b_{23} \\ b_{32} & 2 b_{31} & -b_{33}\end{array}\right)$, \begin{CJK}{UTF8}{mj}则\end{CJK} $A=$

  \item \begin{CJK}{UTF8}{mj}设\end{CJK} $A=\left(\begin{array}{cccc}a & 2 & -1 & 3 \\ 2 & 4 & -2 & 6 \\ -1 & -2 & a & -3\end{array}\right), B$ \begin{CJK}{UTF8}{mj}是\end{CJK} $4 \times 2$ \begin{CJK}{UTF8}{mj}的非零矩阵\end{CJK}, \begin{CJK}{UTF8}{mj}且\end{CJK} $A B=0$, \begin{CJK}{UTF8}{mj}则当\end{CJK} $a \neq 1$ \begin{CJK}{UTF8}{mj}时\end{CJK}, $B$ \begin{CJK}{UTF8}{mj}的秩\end{CJK} $r(B)=$

  \item \begin{CJK}{UTF8}{mj}设\end{CJK} $f(x)$ \begin{CJK}{UTF8}{mj}在数域\end{CJK} $\mathbb{F}$ \begin{CJK}{UTF8}{mj}上不可约\end{CJK}, \begin{CJK}{UTF8}{mj}则\end{CJK} $f(x+2)$ \begin{CJK}{UTF8}{mj}在\end{CJK} $\mathbb{F}$ \begin{CJK}{UTF8}{mj}上\end{CJK} (\begin{CJK}{UTF8}{mj}选填\end{CJK}“\begin{CJK}{UTF8}{mj}必\end{CJK}"\begin{CJK}{UTF8}{mj}或\end{CJK}"\begin{CJK}{UTF8}{mj}末必\end{CJK}") \begin{CJK}{UTF8}{mj}不可约\end{CJK}.

  \item \begin{CJK}{UTF8}{mj}设\end{CJK} $\mathscr{A}$ \begin{CJK}{UTF8}{mj}是\end{CJK} $n$ \begin{CJK}{UTF8}{mj}维线性空间\end{CJK} $V$ \begin{CJK}{UTF8}{mj}的线性变换\end{CJK}, ker $\mathscr{A}$ \begin{CJK}{UTF8}{mj}和\end{CJK} $\operatorname{Im} \mathscr{A}$ \begin{CJK}{UTF8}{mj}分别表示\end{CJK} $\mathscr{A}$ \begin{CJK}{UTF8}{mj}的像空间和核空间\end{CJK}, \begin{CJK}{UTF8}{mj}且\end{CJK} ker $\mathscr{A}=\operatorname{ker} \mathscr{A}^{2}$, \begin{CJK}{UTF8}{mj}则\end{CJK} (\begin{CJK}{UTF8}{mj}选填\end{CJK}"\begin{CJK}{UTF8}{mj}必有\end{CJK}" \begin{CJK}{UTF8}{mj}或\end{CJK}"\begin{CJK}{UTF8}{mj}末必有\end{CJK}") $\operatorname{Im} \mathscr{A}=\operatorname{Im} \mathscr{A}^{2}$.

  \item \begin{CJK}{UTF8}{mj}复数域上\end{CJK} 5 \begin{CJK}{UTF8}{mj}阶反对称矩阵全体\end{CJK}, \begin{CJK}{UTF8}{mj}作为实数域上的线性空间\end{CJK}, \begin{CJK}{UTF8}{mj}它的维数等于\end{CJK} . (\begin{CJK}{UTF8}{mj}加法和数乘定义为通\end{CJK} \begin{CJK}{UTF8}{mj}常的矩阵加法和数乘\end{CJK})

  \item \begin{CJK}{UTF8}{mj}设\end{CJK} $a$ \begin{CJK}{UTF8}{mj}是\end{CJK} $n$ \begin{CJK}{UTF8}{mj}阶方阵\end{CJK} $A$ \begin{CJK}{UTF8}{mj}的一个特征值\end{CJK}, $d_{1}(\lambda), d_{2}(\lambda), \cdots, d_{n}(\lambda)$ \begin{CJK}{UTF8}{mj}是\end{CJK} $A$ \begin{CJK}{UTF8}{mj}的不变因子\end{CJK}. \begin{CJK}{UTF8}{mj}又若\end{CJK} $d_{r}(a) \neq 0, d_{r+1}=0$, \begin{CJK}{UTF8}{mj}则\end{CJK} $a E-A$ \begin{CJK}{UTF8}{mj}的秩\end{CJK} $\mathrm{r}(a E-A)=$

  \item \begin{CJK}{UTF8}{mj}设\end{CJK} $A$ \begin{CJK}{UTF8}{mj}是实二次型\end{CJK} $f\left(x_{1}, x_{2}, \cdots, x_{n}\right)$ \begin{CJK}{UTF8}{mj}的矩阵\end{CJK}, \begin{CJK}{UTF8}{mj}且\end{CJK} $A$ \begin{CJK}{UTF8}{mj}与\end{CJK} $-A$ \begin{CJK}{UTF8}{mj}合同\end{CJK}, \begin{CJK}{UTF8}{mj}则\end{CJK} $A$ \begin{CJK}{UTF8}{mj}的符号差是\end{CJK}

\end{enumerate}
\section{二、解答题}
\begin{enumerate}
  \item (12 \begin{CJK}{UTF8}{mj}分\end{CJK}) \begin{CJK}{UTF8}{mj}设\end{CJK} $A$ \begin{CJK}{UTF8}{mj}是\end{CJK} 3 \begin{CJK}{UTF8}{mj}阶方阵\end{CJK}, $X_{1}, X_{2}$ \begin{CJK}{UTF8}{mj}分别是\end{CJK} $A$ \begin{CJK}{UTF8}{mj}的属于特征值\end{CJK} $-1,1$ \begin{CJK}{UTF8}{mj}的特征向量\end{CJK}, \begin{CJK}{UTF8}{mj}且向量\end{CJK} $X_{3}$ \begin{CJK}{UTF8}{mj}满足\end{CJK}
\end{enumerate}
$$
A X_{3}=X_{2}+X_{3} .
$$
(1) \begin{CJK}{UTF8}{mj}证明\end{CJK}: $X_{1}, X_{2}, X_{3}$ \begin{CJK}{UTF8}{mj}线性无关\end{CJK};

(2) \begin{CJK}{UTF8}{mj}令\end{CJK} $P=\left(X_{1}, X_{2}, X_{3}\right)$, \begin{CJK}{UTF8}{mj}求\end{CJK} $P^{-1} A P$.

\begin{enumerate}
  \setcounter{enumi}{2}
  \item ( 15 \begin{CJK}{UTF8}{mj}分\end{CJK}) \begin{CJK}{UTF8}{mj}设\end{CJK} 3 \begin{CJK}{UTF8}{mj}阶实矩阵\end{CJK} $A=\left(a_{i j}\right)_{3 \times 3}$ \begin{CJK}{UTF8}{mj}满足\end{CJK} $a_{33}=-1$ \begin{CJK}{UTF8}{mj}且\end{CJK} $a_{i j}=A_{i j}(i, j=1,2,3)$, \begin{CJK}{UTF8}{mj}其中\end{CJK} $A_{i j}$ \begin{CJK}{UTF8}{mj}是元素\end{CJK} $a_{i j}$ \begin{CJK}{UTF8}{mj}的代数\end{CJK} \begin{CJK}{UTF8}{mj}余子式\end{CJK}. \begin{CJK}{UTF8}{mj}求线性方程组\end{CJK}
\end{enumerate}
$$
A\left(\begin{array}{l}
x_{1} \\
x_{2} \\
x_{3}
\end{array}\right)=\left(\begin{array}{l}
0 \\
0 \\
1
\end{array}\right)
$$
\begin{CJK}{UTF8}{mj}的解\end{CJK}, \begin{CJK}{UTF8}{mj}并证明\end{CJK} (\begin{CJK}{UTF8}{mj}要求写出详细求解过程\end{CJK})

\begin{enumerate}
  \setcounter{enumi}{3}
  \item ( 15 \begin{CJK}{UTF8}{mj}分\end{CJK}) \begin{CJK}{UTF8}{mj}设\end{CJK} $A$ \begin{CJK}{UTF8}{mj}是\end{CJK} $n$ \begin{CJK}{UTF8}{mj}阶方阵\end{CJK}. \begin{CJK}{UTF8}{mj}证明\end{CJK}: \begin{CJK}{UTF8}{mj}存在矩阵\end{CJK} $B$, \begin{CJK}{UTF8}{mj}使得\end{CJK}
\end{enumerate}
$$
A B A=A, B A B=B
$$

\begin{enumerate}
  \setcounter{enumi}{4}
  \item (15 \begin{CJK}{UTF8}{mj}分\end{CJK}) \begin{CJK}{UTF8}{mj}设\end{CJK} $V$ \begin{CJK}{UTF8}{mj}是数域\end{CJK} $\mathbb{F}$ \begin{CJK}{UTF8}{mj}上的有限维线性空间\end{CJK}, $\mathscr{A}$ \begin{CJK}{UTF8}{mj}是\end{CJK} $V$ \begin{CJK}{UTF8}{mj}的线性变换\end{CJK}, $f(\lambda)=(\lambda-1)(\lambda-2)^{2}$ \begin{CJK}{UTF8}{mj}是\end{CJK} $\mathscr{A}$ \begin{CJK}{UTF8}{mj}的极小多项\end{CJK} \begin{CJK}{UTF8}{mj}式\end{CJK}: \begin{CJK}{UTF8}{mj}再设\end{CJK} $V_{k}=\operatorname{ker}(k \varepsilon-\mathscr{A})^{k}, k=1,2$, \begin{CJK}{UTF8}{mj}其中\end{CJK} $\operatorname{ker}(\mathscr{A})$ \begin{CJK}{UTF8}{mj}表示\end{CJK} $\mathscr{A}$ \begin{CJK}{UTF8}{mj}的核空间\end{CJK}, $\varepsilon$ \begin{CJK}{UTF8}{mj}表示\end{CJK} $V$ \begin{CJK}{UTF8}{mj}的恒等变换\end{CJK}. \begin{CJK}{UTF8}{mj}证明\end{CJK}
\end{enumerate}
$$
V=V_{1} \oplus V_{2}
$$

\begin{enumerate}
  \setcounter{enumi}{5}
  \item ( 15 \begin{CJK}{UTF8}{mj}分\end{CJK}) \begin{CJK}{UTF8}{mj}设\end{CJK} $\mathbb{F}^{n \times n}$ \begin{CJK}{UTF8}{mj}是数域\end{CJK} $\mathbb{F}$ \begin{CJK}{UTF8}{mj}上\end{CJK} $n$ \begin{CJK}{UTF8}{mj}阶方阵全体\end{CJK}, $V$ \begin{CJK}{UTF8}{mj}是\end{CJK} $\mathbb{F}^{n \times n}$ \begin{CJK}{UTF8}{mj}的一个非空子集且满足以下条件\end{CJK}:
\end{enumerate}
(1) $V$ \begin{CJK}{UTF8}{mj}中至少有一个非零矩阵\end{CJK};

(2) \begin{CJK}{UTF8}{mj}对\end{CJK} $V$ \begin{CJK}{UTF8}{mj}中任意方阵\end{CJK} $A, B$, \begin{CJK}{UTF8}{mj}总有\end{CJK} $A-B$ \begin{CJK}{UTF8}{mj}属于\end{CJK} $V$;

(3) \begin{CJK}{UTF8}{mj}对\end{CJK} $V$ \begin{CJK}{UTF8}{mj}中任意方阵\end{CJK} $A, \mathbb{F}^{n \times n}$ \begin{CJK}{UTF8}{mj}中任意方阵\end{CJK} $X$, \begin{CJK}{UTF8}{mj}总有\end{CJK} $A X, X A$ \begin{CJK}{UTF8}{mj}都属于\end{CJK} $V$.

\begin{CJK}{UTF8}{mj}证明\end{CJK}:
$$
V=\mathbb{F}^{n \times n}
$$

\begin{enumerate}
  \setcounter{enumi}{6}
  \item (15 \begin{CJK}{UTF8}{mj}分\end{CJK}) \begin{CJK}{UTF8}{mj}设\end{CJK} $U$ \begin{CJK}{UTF8}{mj}是\end{CJK} $n$ \begin{CJK}{UTF8}{mj}维线性空间\end{CJK} $V$ \begin{CJK}{UTF8}{mj}的\end{CJK} $r$ \begin{CJK}{UTF8}{mj}维子空间\end{CJK}, \begin{CJK}{UTF8}{mj}并且\end{CJK} $1 \leq r \leq n-1$. \begin{CJK}{UTF8}{mj}证明\end{CJK}: $U$ \begin{CJK}{UTF8}{mj}是\end{CJK} $V$ \begin{CJK}{UTF8}{mj}的若干个\end{CJK} $n-1$ \begin{CJK}{UTF8}{mj}维子空间\end{CJK} \begin{CJK}{UTF8}{mj}的交\end{CJK}.

  \item ( 15 \begin{CJK}{UTF8}{mj}分\end{CJK}) \begin{CJK}{UTF8}{mj}设\end{CJK} $A_{1}$ \begin{CJK}{UTF8}{mj}和\end{CJK} $A_{2}$ \begin{CJK}{UTF8}{mj}分别为\end{CJK} $\mathbb{C}^{n_{1} \times n_{1}}$ \begin{CJK}{UTF8}{mj}和\end{CJK} $\mathbb{C}^{n_{2} \times n_{2}}$ \begin{CJK}{UTF8}{mj}中的上三角矩阵\end{CJK}, $C \in \mathbb{C}^{n_{1} \times n_{2}}$. \begin{CJK}{UTF8}{mj}证明\end{CJK}: \begin{CJK}{UTF8}{mj}方程\end{CJK}

\end{enumerate}
$$
A_{1} B-B A_{2}=C
$$
\begin{CJK}{UTF8}{mj}有唯一解\end{CJK} $B \in \mathbb{C}^{n_{1} \times n_{2}}$ \begin{CJK}{UTF8}{mj}当且仅当\end{CJK} $A_{1}$ \begin{CJK}{UTF8}{mj}和\end{CJK} $A_{2}$ \begin{CJK}{UTF8}{mj}无公共特征值\end{CJK}.

\section{4. 厦门大学 2016 年研究生入学考试试题高等代数}
\begin{CJK}{UTF8}{mj}李扬\end{CJK}

\begin{CJK}{UTF8}{mj}微信公众号\end{CJK}: sxkyliyang

\begin{CJK}{UTF8}{mj}一\end{CJK}、\begin{CJK}{UTF8}{mj}填空题\end{CJK}(\begin{CJK}{UTF8}{mj}每小题\end{CJK} 6 \begin{CJK}{UTF8}{mj}分\end{CJK}, \begin{CJK}{UTF8}{mj}共\end{CJK} 48 \begin{CJK}{UTF8}{mj}分\end{CJK})

\begin{enumerate}
  \item \begin{CJK}{UTF8}{mj}设\end{CJK} $A$ \begin{CJK}{UTF8}{mj}为\end{CJK} $n$ \begin{CJK}{UTF8}{mj}阶方阵\end{CJK}, $B$ \begin{CJK}{UTF8}{mj}为\end{CJK} $m$ \begin{CJK}{UTF8}{mj}阶方阵\end{CJK}, \begin{CJK}{UTF8}{mj}则行列式\end{CJK} $\left|\begin{array}{cc}0 & A \\ B & 0\end{array}\right|=$

  \item \begin{CJK}{UTF8}{mj}若\end{CJK} $A$ \begin{CJK}{UTF8}{mj}为\end{CJK} $n$ \begin{CJK}{UTF8}{mj}阶方阵\end{CJK}, $A^{*}$ \begin{CJK}{UTF8}{mj}为\end{CJK} $A$ \begin{CJK}{UTF8}{mj}的伴随矩阵\end{CJK}. \begin{CJK}{UTF8}{mj}则\end{CJK} $\left(A^{*}\right)^{*}=$

  \item \begin{CJK}{UTF8}{mj}设\end{CJK} $\alpha_{1}, \alpha_{2}$ \begin{CJK}{UTF8}{mj}是非齐次线性方程组\end{CJK} $A X=\beta$ \begin{CJK}{UTF8}{mj}的两个解\end{CJK}, \begin{CJK}{UTF8}{mj}则\end{CJK} $k_{1} \alpha_{1}+k_{2} \alpha_{2}$ \begin{CJK}{UTF8}{mj}是\end{CJK} $A X=\beta$ \begin{CJK}{UTF8}{mj}的解的充分必要条件是\end{CJK} $k_{1}, k_{2}$ \begin{CJK}{UTF8}{mj}满足\end{CJK} $; l_{1} \alpha_{1}+l_{2} \alpha_{2}$ \begin{CJK}{UTF8}{mj}是\end{CJK} $A X=0$ \begin{CJK}{UTF8}{mj}的解的充分必要条件是\end{CJK} $l_{1}, l_{2}$ \begin{CJK}{UTF8}{mj}满足\end{CJK}

  \item \begin{CJK}{UTF8}{mj}设\end{CJK} $\mathscr{A}$ \begin{CJK}{UTF8}{mj}是\end{CJK} $\mathbb{R}^{1 \times 2}$ \begin{CJK}{UTF8}{mj}的线性变换\end{CJK}, \begin{CJK}{UTF8}{mj}使得\end{CJK} $\mathscr{A}((1,1))=(1,-1), \mathscr{A}((3,2))=(2,1)$, \begin{CJK}{UTF8}{mj}则\end{CJK} $\mathscr{A}((2,4))=$

  \item \begin{CJK}{UTF8}{mj}设\end{CJK} $\mathscr{A}$ \begin{CJK}{UTF8}{mj}是线性空间\end{CJK} $V$ \begin{CJK}{UTF8}{mj}到\end{CJK} $W$ \begin{CJK}{UTF8}{mj}的线性映射\end{CJK}, $\xi_{1}, \xi_{2}, \cdots, \xi_{n}$ \begin{CJK}{UTF8}{mj}及\end{CJK} $\eta_{1}, \eta_{2}, \cdots, \eta_{n}$ \begin{CJK}{UTF8}{mj}分别是\end{CJK} $V$ \begin{CJK}{UTF8}{mj}和\end{CJK} $W$ \begin{CJK}{UTF8}{mj}的一个基\end{CJK}, $A$ \begin{CJK}{UTF8}{mj}是\end{CJK} $\mathscr{A}$ \begin{CJK}{UTF8}{mj}在这两个基下的矩阵\end{CJK}, \begin{CJK}{UTF8}{mj}即\end{CJK} $\mathscr{A}\left(\xi_{1}, \xi_{2}, \cdots, \xi_{n}\right)=\left(\eta_{1}, \eta_{2}, \cdots, \eta_{n}\right) A$, \begin{CJK}{UTF8}{mj}则\end{CJK} $\mathscr{A}$ \begin{CJK}{UTF8}{mj}是满射的充分必要条件是\end{CJK} $A$ \begin{CJK}{UTF8}{mj}的秩\end{CJK} $\mathrm{r}(A)=$ $; \mathscr{A}$ \begin{CJK}{UTF8}{mj}是单射的充分必要条件是\end{CJK} $A$ \begin{CJK}{UTF8}{mj}的秩\end{CJK} $\mathrm{r}(A)=$

  \item \begin{CJK}{UTF8}{mj}已知\end{CJK} $f(x), g(x)$ \begin{CJK}{UTF8}{mj}是多项式\end{CJK}, \begin{CJK}{UTF8}{mj}若对任意实数\end{CJK} $a$, \begin{CJK}{UTF8}{mj}总有\end{CJK} $f(a)$ \begin{CJK}{UTF8}{mj}为实数\end{CJK}, \begin{CJK}{UTF8}{mj}则\end{CJK} $f(x)$ (\begin{CJK}{UTF8}{mj}选填\end{CJK}“\begin{CJK}{UTF8}{mj}必\end{CJK}"\begin{CJK}{UTF8}{mj}或\end{CJK}“\begin{CJK}{UTF8}{mj}末必\end{CJK}") \begin{CJK}{UTF8}{mj}为实\end{CJK} \begin{CJK}{UTF8}{mj}系数多项式\end{CJK}; \begin{CJK}{UTF8}{mj}若对任意的整数\end{CJK} $k$, \begin{CJK}{UTF8}{mj}总有\end{CJK} $g(k)$ \begin{CJK}{UTF8}{mj}为整数\end{CJK}, \begin{CJK}{UTF8}{mj}则\end{CJK} $g(x)$ (\begin{CJK}{UTF8}{mj}选填\end{CJK}“\begin{CJK}{UTF8}{mj}必\end{CJK}”\begin{CJK}{UTF8}{mj}或\end{CJK} “\begin{CJK}{UTF8}{mj}末必\end{CJK}”) \begin{CJK}{UTF8}{mj}为整系数多项式\end{CJK}.

  \item \begin{CJK}{UTF8}{mj}设\end{CJK} $A$ \begin{CJK}{UTF8}{mj}的特征多项式为\end{CJK} $x^{3}+x^{2}-2$, \begin{CJK}{UTF8}{mj}则\end{CJK} $(A+2 E)^{-1}=$

  \item \begin{CJK}{UTF8}{mj}设二次型\end{CJK} $f\left(x_{1}, x_{2}, x_{3}\right)=x_{1}^{2}+2 x_{2}^{2}+3 x_{3}^{2}+2 t x_{2} x_{3}$ \begin{CJK}{UTF8}{mj}是正定二次型\end{CJK}, \begin{CJK}{UTF8}{mj}则\end{CJK} $t$ \begin{CJK}{UTF8}{mj}满足\end{CJK}

\end{enumerate}
\section{二、解答题}
\begin{enumerate}
  \item (12 \begin{CJK}{UTF8}{mj}分\end{CJK}) \begin{CJK}{UTF8}{mj}设\end{CJK} 3 \begin{CJK}{UTF8}{mj}阶实对称矩阵\end{CJK} $A$ \begin{CJK}{UTF8}{mj}的特征值为\end{CJK} $2,1,1$, \begin{CJK}{UTF8}{mj}且\end{CJK} $X=(1,1,0)^{\prime}$ \begin{CJK}{UTF8}{mj}是\end{CJK} $A$ \begin{CJK}{UTF8}{mj}的属于特征值\end{CJK}. 2 \begin{CJK}{UTF8}{mj}的特征向量\end{CJK}, \begin{CJK}{UTF8}{mj}求矩阵\end{CJK} $A$.

  \item ( 15 \begin{CJK}{UTF8}{mj}分\end{CJK}) \begin{CJK}{UTF8}{mj}证明\end{CJK}: \begin{CJK}{UTF8}{mj}对任一\end{CJK} $n$ \begin{CJK}{UTF8}{mj}阶方阵\end{CJK} $A$, \begin{CJK}{UTF8}{mj}总存在可逆矩阵\end{CJK} $P$ \begin{CJK}{UTF8}{mj}和上三角矩阵\end{CJK} $U$, \begin{CJK}{UTF8}{mj}使得\end{CJK} $A=P U$, \begin{CJK}{UTF8}{mj}且\end{CJK} $P$ \begin{CJK}{UTF8}{mj}可表示为若干形\end{CJK} \begin{CJK}{UTF8}{mj}如\end{CJK}

\end{enumerate}
\includegraphics[max width=\textwidth]{2022_04_18_a5c47c0ff534501b502eg-097}

\begin{CJK}{UTF8}{mj}的矩阵的乘积\end{CJK}.

\begin{enumerate}
  \setcounter{enumi}{3}
  \item (15 \begin{CJK}{UTF8}{mj}分\end{CJK}) \begin{CJK}{UTF8}{mj}设\end{CJK} $f(x), g(x)$ \begin{CJK}{UTF8}{mj}是数域\end{CJK} $\mathbb{F}$ \begin{CJK}{UTF8}{mj}上的非零多项式\end{CJK}, $n$ \begin{CJK}{UTF8}{mj}为大于\end{CJK} 1 \begin{CJK}{UTF8}{mj}的正整数\end{CJK}. \begin{CJK}{UTF8}{mj}试给出\end{CJK}
\end{enumerate}
$$
\left(f\left(x^{n}\right), g\left(x^{n}\right)\right)=(f(x), g(x))
$$
\begin{CJK}{UTF8}{mj}的充分必要条件\end{CJK}, \begin{CJK}{UTF8}{mj}并证明之\end{CJK}. 4. ( 15 \begin{CJK}{UTF8}{mj}分\end{CJK}) \begin{CJK}{UTF8}{mj}设\end{CJK} $V$ \begin{CJK}{UTF8}{mj}是数域\end{CJK} $\mathbb{F}$ \begin{CJK}{UTF8}{mj}上的线性空间\end{CJK}, $\alpha_{1}, \alpha_{2}, \cdots, \alpha_{s}, \beta_{1}, \beta_{2}, \cdots, \beta_{s} \in V$, \begin{CJK}{UTF8}{mj}且\end{CJK}
$$
\left\{\begin{array}{l}
\beta_{1}=a_{11} \alpha_{1}+a_{21} \alpha_{2}+\cdots+a_{s 1} \alpha_{s} \\
\beta_{2}=a_{12} \alpha_{1}+a_{22} \alpha_{2}+\cdots+a_{s 2} \alpha_{s} \\
\vdots \\
\beta_{t}=a_{1 t} \alpha_{1}+a_{2 t} \alpha_{2}+\cdots+a_{s t} \alpha_{s}
\end{array}\right.
$$
\begin{CJK}{UTF8}{mj}记\end{CJK}
$$
A=\left(\begin{array}{cccc}
a_{11} & a_{12} & \cdots & a_{1 t} \\
a_{21} & a_{22} & \cdots & a_{2 t} \\
\vdots & \vdots & & \vdots \\
a_{s 1} & a_{s 2} & \cdots & a_{s t}
\end{array}\right)
$$
\begin{CJK}{UTF8}{mj}设\end{CJK} $\alpha_{1}, \alpha_{2}, \cdots, \alpha_{s}$ \begin{CJK}{UTF8}{mj}线性无关\end{CJK}, \begin{CJK}{UTF8}{mj}证明\end{CJK}: $\beta_{1}, \beta_{2}, \cdots, \beta_{s}$ \begin{CJK}{UTF8}{mj}线性无关的充分必要条件是\end{CJK} $A$ \begin{CJK}{UTF8}{mj}的列向量线性无关\end{CJK}.

\begin{enumerate}
  \setcounter{enumi}{5}
  \item ( 15 \begin{CJK}{UTF8}{mj}分\end{CJK}) \begin{CJK}{UTF8}{mj}设\end{CJK} $\mathscr{A}, \mathscr{B}$ \begin{CJK}{UTF8}{mj}是\end{CJK} $n$ \begin{CJK}{UTF8}{mj}维欧氏空间\end{CJK} $V$ \begin{CJK}{UTF8}{mj}的线性变换\end{CJK}, \begin{CJK}{UTF8}{mj}且对任意\end{CJK} $\alpha, \beta \in V$, \begin{CJK}{UTF8}{mj}总有\end{CJK}
\end{enumerate}
$$
(\mathscr{A}(\alpha), \beta)=(\alpha, \mathscr{B}(\beta))
$$
\begin{CJK}{UTF8}{mj}证明\end{CJK}:

(1) $\operatorname{ker} \mathscr{A}=(\operatorname{Im} \mathscr{B})^{\perp}$;

(2) $V=\operatorname{ker} \mathscr{A} \oplus \operatorname{Im} \mathscr{B}$.

\begin{enumerate}
  \setcounter{enumi}{6}
  \item (15 \begin{CJK}{UTF8}{mj}分\end{CJK}) \begin{CJK}{UTF8}{mj}设\end{CJK} $\mathscr{A}$ \begin{CJK}{UTF8}{mj}是数域\end{CJK} $\mathbb{F}$ \begin{CJK}{UTF8}{mj}上有限维线性空间\end{CJK} $V$ \begin{CJK}{UTF8}{mj}的线性变换\end{CJK}, \begin{CJK}{UTF8}{mj}定义\end{CJK} $\mathbb{F}$ \begin{CJK}{UTF8}{mj}上线性空间\end{CJK}
\end{enumerate}
$$
\mathbb{F}(\mathscr{A})=\left\{a_{k} \mathscr{A}^{k}+a_{k-1} \mathscr{A}^{k-1}+\cdots+a_{1} \mathscr{A}+a_{0} \mathrm{id}_{V} \mid a_{i} \in \mathbb{F}, i=0,1, \cdots, k, k \in \mathbb{Z}_{>0}\right\}
$$
\begin{CJK}{UTF8}{mj}其中\end{CJK} $\mathrm{id}_{V}$ \begin{CJK}{UTF8}{mj}为\end{CJK} $V$ \begin{CJK}{UTF8}{mj}的恒等变换\end{CJK}, $\mathbb{Z}_{>0}$ \begin{CJK}{UTF8}{mj}为正整数集\end{CJK}. \begin{CJK}{UTF8}{mj}试求\end{CJK} $\mathbb{F}(\mathscr{A})$ \begin{CJK}{UTF8}{mj}的维数\end{CJK}, \begin{CJK}{UTF8}{mj}同时给出\end{CJK} $\mathbb{F}(\mathscr{A})$ \begin{CJK}{UTF8}{mj}的一个基\end{CJK}.

\begin{enumerate}
  \setcounter{enumi}{7}
  \item ( 15 \begin{CJK}{UTF8}{mj}分\end{CJK}) \begin{CJK}{UTF8}{mj}求\end{CJK} $n$ \begin{CJK}{UTF8}{mj}阶矩阵\end{CJK}
\end{enumerate}
\includegraphics[max width=\textwidth]{2022_04_18_a5c47c0ff534501b502eg-098}

\begin{CJK}{UTF8}{mj}的\end{CJK} Jordan \begin{CJK}{UTF8}{mj}标准型\end{CJK}. (\begin{CJK}{UTF8}{mj}要求写出求解过程\end{CJK})

\section{5. 厦门大学 2017 年研究生入学考试试题高等代数}
\begin{CJK}{UTF8}{mj}李扬\end{CJK}

\begin{CJK}{UTF8}{mj}微信公众号\end{CJK}: sxkyliyang

\begin{CJK}{UTF8}{mj}一\end{CJK}、\begin{CJK}{UTF8}{mj}填空题\end{CJK}(\begin{CJK}{UTF8}{mj}每小题\end{CJK} 6 \begin{CJK}{UTF8}{mj}分\end{CJK}, \begin{CJK}{UTF8}{mj}共\end{CJK} 48 \begin{CJK}{UTF8}{mj}分\end{CJK})

\begin{enumerate}
  \item \begin{CJK}{UTF8}{mj}计算\end{CJK} $n$ \begin{CJK}{UTF8}{mj}阶行列式\end{CJK}
\end{enumerate}
$$
\left|\begin{array}{ccccc}
0 & 0 & \cdots & 0 & 1 \\
0 & 0 & \cdots & 1 & 0 \\
\vdots & \vdots & & \vdots & \vdots \\
0 & 1 & \cdots & 0 & 0 \\
1 & 0 & \cdots & 0 & 0
\end{array}\right|=
$$

\begin{enumerate}
  \setcounter{enumi}{2}
  \item \begin{CJK}{UTF8}{mj}将\end{CJK} $\left(\begin{array}{ccc}2 & 1 & 3 \\ 0 & 5 & 2 \\ -2 & 4 & 1\end{array}\right)$ \begin{CJK}{UTF8}{mj}表示为对称矩阵和反对称矩阵的和\end{CJK}

  \item \begin{CJK}{UTF8}{mj}若矩阵\end{CJK} $A=\left(\alpha_{1}, \alpha_{2}, \alpha_{3}, \alpha_{4}\right)$ \begin{CJK}{UTF8}{mj}经过初等行变换化为\end{CJK} $\left(\begin{array}{cccc}1 & 0 & 0 & 3 \\ 0 & 0 & 2 & 4 \\ 0 & -1 & 0 & 5 \\ 0 & 0 & 0 & 0\end{array}\right)$, \begin{CJK}{UTF8}{mj}那么向量组\end{CJK} $\alpha_{1}, \alpha_{2}, \alpha_{3}, \alpha_{4}$ \begin{CJK}{UTF8}{mj}的一个极\end{CJK} \begin{CJK}{UTF8}{mj}大线性无关组是\end{CJK} , \begin{CJK}{UTF8}{mj}其余向量由此极大线性无关组表示的表示式为\end{CJK}

  \item \begin{CJK}{UTF8}{mj}设\end{CJK} $n$ \begin{CJK}{UTF8}{mj}阶方阵\end{CJK} $A$ \begin{CJK}{UTF8}{mj}的秩为\end{CJK} $n-1$, \begin{CJK}{UTF8}{mj}且\end{CJK} $a_{11}$ \begin{CJK}{UTF8}{mj}的代数余子式\end{CJK} $A_{11} \neq 0$, \begin{CJK}{UTF8}{mj}则线性方程组\end{CJK} $A X=0$ \begin{CJK}{UTF8}{mj}的通解是\end{CJK}

  \item \begin{CJK}{UTF8}{mj}设\end{CJK} $p(x)$ \begin{CJK}{UTF8}{mj}是数域\end{CJK} $\mathbb{F}$ \begin{CJK}{UTF8}{mj}上的首一不可约多项式\end{CJK}, $f(x), g(x) \in \mathbb{F}[x]$, \begin{CJK}{UTF8}{mj}若\end{CJK} $p(x) \mid f(x) g(x)$, \begin{CJK}{UTF8}{mj}但\end{CJK} $p(x) \nmid f(x)$, \begin{CJK}{UTF8}{mj}则\end{CJK} $\operatorname{gcd}(p(x), g(x))=$

  \item \begin{CJK}{UTF8}{mj}设\end{CJK} $\left\{\xi_{1}, \xi_{2}, \xi_{3}\right\},\left\{\eta_{1}, \eta_{2}\right\}$ \begin{CJK}{UTF8}{mj}分别是\end{CJK} $V$ \begin{CJK}{UTF8}{mj}和\end{CJK} $U$ \begin{CJK}{UTF8}{mj}的一组基\end{CJK}, $\mathscr{A}$ \begin{CJK}{UTF8}{mj}是\end{CJK} $V$ \begin{CJK}{UTF8}{mj}到\end{CJK} $U$ \begin{CJK}{UTF8}{mj}的线性映射\end{CJK}, \begin{CJK}{UTF8}{mj}满足\end{CJK} $\mathscr{A}\left(\xi_{1}\right)=\eta_{1}+2 \eta_{2}, \mathscr{A}\left(\xi_{2}\right)=$ $\eta_{2}, \mathscr{A}\left(\xi_{3}\right)=\eta_{1}+2 \eta_{2}$, \begin{CJK}{UTF8}{mj}则\end{CJK} $\operatorname{dim}$ ker $\mathscr{A}=$ , $\operatorname{ker} \mathscr{A}=$

  \item \begin{CJK}{UTF8}{mj}设\end{CJK} $A$ \begin{CJK}{UTF8}{mj}是\end{CJK} $n$ \begin{CJK}{UTF8}{mj}阶实对称矩阵\end{CJK}, \begin{CJK}{UTF8}{mj}则在复数域上\end{CJK} $A$ \begin{CJK}{UTF8}{mj}与\end{CJK} $-A$ (\begin{CJK}{UTF8}{mj}选填\end{CJK}“\begin{CJK}{UTF8}{mj}必\end{CJK}”\begin{CJK}{UTF8}{mj}或\end{CJK}“\begin{CJK}{UTF8}{mj}末必\end{CJK}”)\begin{CJK}{UTF8}{mj}合同\end{CJK}, \begin{CJK}{UTF8}{mj}在实数域上\end{CJK} $A$ \begin{CJK}{UTF8}{mj}与\end{CJK} $-A$ (\begin{CJK}{UTF8}{mj}选填\end{CJK}“\begin{CJK}{UTF8}{mj}必\end{CJK}”\begin{CJK}{UTF8}{mj}或\end{CJK} “\begin{CJK}{UTF8}{mj}末必\end{CJK}”)\begin{CJK}{UTF8}{mj}合同\end{CJK}.

  \item 2 \begin{CJK}{UTF8}{mj}阶实对称正交阵全体按正交相似分类\end{CJK}, \begin{CJK}{UTF8}{mj}可分成\end{CJK} \begin{CJK}{UTF8}{mj}类\end{CJK}, \begin{CJK}{UTF8}{mj}每类的正交相似标准型是\end{CJK}

\end{enumerate}
\section{二、解答题}
\begin{enumerate}
  \item (12 \begin{CJK}{UTF8}{mj}分\end{CJK}) \begin{CJK}{UTF8}{mj}已知\end{CJK} 3 \begin{CJK}{UTF8}{mj}阶非零矩阵\end{CJK} $B$ \begin{CJK}{UTF8}{mj}的每个列向量都是齐次线性方程组\end{CJK}
\end{enumerate}
$$
\left\{\begin{array}{l}
x_{1}+2 x_{2}-2 x_{3}=0 \\
2 x_{1}-x_{2}+\lambda x_{3}=0 \\
3 x_{1}+x_{2}-x_{3}=0
\end{array}\right.
$$
\begin{CJK}{UTF8}{mj}的解向量\end{CJK}.

(1) \begin{CJK}{UTF8}{mj}求\end{CJK} $\lambda$ \begin{CJK}{UTF8}{mj}的值\end{CJK};

(2) \begin{CJK}{UTF8}{mj}求行列式\end{CJK} $\operatorname{det} B$.

\begin{enumerate}
  \setcounter{enumi}{2}
  \item (15 \begin{CJK}{UTF8}{mj}分\end{CJK}) \begin{CJK}{UTF8}{mj}设\end{CJK}
\end{enumerate}
$$
A=\left(\begin{array}{ccc}
1 & 0 & 0 \\
0 & -2 & 0 \\
1 & 0 & 1
\end{array}\right)
$$
$A^{*} B A=2 B A-8 E$, \begin{CJK}{UTF8}{mj}计算\end{CJK} $B$.

\begin{enumerate}
  \setcounter{enumi}{3}
  \item (15 \begin{CJK}{UTF8}{mj}分\end{CJK}) \begin{CJK}{UTF8}{mj}设\end{CJK} $f(x), g(x) \in \mathbb{F}[x]$ \begin{CJK}{UTF8}{mj}是非零多项式\end{CJK}. \begin{CJK}{UTF8}{mj}证明\end{CJK}: $(f(x), g(x))=1$ \begin{CJK}{UTF8}{mj}的充要条件是对\end{CJK} $\forall h(x) \in \mathbb{F}[x]$, \begin{CJK}{UTF8}{mj}都有\end{CJK}
\end{enumerate}
$$
(h(x) f(x), g(x))=(h(x), g(x)) .
$$

\begin{enumerate}
  \setcounter{enumi}{4}
  \item ( 15 \begin{CJK}{UTF8}{mj}分\end{CJK}) \begin{CJK}{UTF8}{mj}设\end{CJK} $A$ \begin{CJK}{UTF8}{mj}是\end{CJK} $n$ \begin{CJK}{UTF8}{mj}阶正定矩阵\end{CJK}, $\alpha$ \begin{CJK}{UTF8}{mj}是\end{CJK} $n$ \begin{CJK}{UTF8}{mj}维非零实向量\end{CJK}, \begin{CJK}{UTF8}{mj}令\end{CJK} $B=A \alpha \alpha^{\prime}$, \begin{CJK}{UTF8}{mj}求\end{CJK} $B$ \begin{CJK}{UTF8}{mj}的所有特征值和相应的特征子空\end{CJK} \begin{CJK}{UTF8}{mj}间\end{CJK}, \begin{CJK}{UTF8}{mj}并给出特特征子空间的一组基和维数\end{CJK}.

  \item ( 15 \begin{CJK}{UTF8}{mj}分\end{CJK}) \begin{CJK}{UTF8}{mj}设\end{CJK} $V_{1}$ \begin{CJK}{UTF8}{mj}和\end{CJK} $V_{2}$ \begin{CJK}{UTF8}{mj}是\end{CJK} $n$ \begin{CJK}{UTF8}{mj}维线性空间\end{CJK} $V$ \begin{CJK}{UTF8}{mj}的真子空间\end{CJK}. \begin{CJK}{UTF8}{mj}证明\end{CJK}: $V=V_{1} \oplus V_{2}$ \begin{CJK}{UTF8}{mj}的充要条件是存在\end{CJK} $V$ \begin{CJK}{UTF8}{mj}上的幂等变换\end{CJK} $\mathscr{A}$, \begin{CJK}{UTF8}{mj}使得\end{CJK} $\operatorname{Im} \mathscr{A}=V_{1}$, ker $\mathscr{A}=V_{2}$.

  \item (15 \begin{CJK}{UTF8}{mj}分\end{CJK}) \begin{CJK}{UTF8}{mj}设\end{CJK} $W$ \begin{CJK}{UTF8}{mj}是\end{CJK} $\mathbb{F}^{n \times n}$ \begin{CJK}{UTF8}{mj}中形如\end{CJK} $A B-B A$ \begin{CJK}{UTF8}{mj}的矩阵生成的子空间\end{CJK}, \begin{CJK}{UTF8}{mj}求\end{CJK} $\operatorname{dim} W$ \begin{CJK}{UTF8}{mj}并证明\end{CJK}.

  \item (15 \begin{CJK}{UTF8}{mj}分\end{CJK}) \begin{CJK}{UTF8}{mj}求证\end{CJK}: $\mathbb{C}$ \begin{CJK}{UTF8}{mj}上两个\end{CJK} $n$ \begin{CJK}{UTF8}{mj}阶方阵\end{CJK} $A, B$ \begin{CJK}{UTF8}{mj}相似的充要条件是\end{CJK} $\forall a \in \mathbb{C}, \forall k \in \mathbb{N}_{+}$, \begin{CJK}{UTF8}{mj}都有\end{CJK}

\end{enumerate}
$$
\mathrm{r}(a E-A)^{k}=\mathrm{r}(a E-B)^{k}
$$

\section{1. 山东大学 2012 年研究生入学考试试题线性代数与常微分方程 
 李扬 
 微信公众号: sxkyliyang}
\begin{enumerate}
  \item \begin{CJK}{UTF8}{mj}求常微分方程\end{CJK} $2 x \mathrm{~d} y-2 y \mathrm{~d} x=\sqrt{x^{2}+4 y^{2}} \mathrm{~d} x$.

  \item \begin{CJK}{UTF8}{mj}求\end{CJK} $y^{\prime \prime}+2 x y^{\prime}=0$ \begin{CJK}{UTF8}{mj}在\end{CJK} $y(0)=1, y^{\prime}(0)=\frac{1}{2}$ \begin{CJK}{UTF8}{mj}处的解\end{CJK}.

  \item \begin{CJK}{UTF8}{mj}求方程\end{CJK} $y^{\prime \prime}-4 y^{\prime}+y=e^{2 x}$ \begin{CJK}{UTF8}{mj}的通解\end{CJK}.

  \item \begin{CJK}{UTF8}{mj}求解\end{CJK} $X^{\prime}=A X$, \begin{CJK}{UTF8}{mj}其中\end{CJK} $\left(\begin{array}{ccc}9 & -3 & 11 \\ 3 & 1 & 3 \\ -7 & 3 & 9\end{array}\right)$.

  \item \begin{CJK}{UTF8}{mj}设\end{CJK} $A X=B$, \begin{CJK}{UTF8}{mj}其中\end{CJK} $A, X, B$ \begin{CJK}{UTF8}{mj}分别是\end{CJK} $m \times l, l \times n, m \times n$ \begin{CJK}{UTF8}{mj}阶矩阵\end{CJK}, \begin{CJK}{UTF8}{mj}证明\end{CJK} $A X=B$ \begin{CJK}{UTF8}{mj}有解等价于\end{CJK} $r(A)=r(A, B)$.

  \item \begin{CJK}{UTF8}{mj}求矩阵\end{CJK} $\left(\begin{array}{llll}1 & 1 & 0 & 0 \\ 1 & 1 & 1 & 0 \\ 0 & 1 & 1 & 1 \\ 0 & 0 & 1 & 1\end{array}\right)$ \begin{CJK}{UTF8}{mj}的逆\end{CJK}.

  \item $V$ \begin{CJK}{UTF8}{mj}为一线性空间\end{CJK}, $V_{1}, V_{2}, V_{3}$ \begin{CJK}{UTF8}{mj}为其子空间\end{CJK} ( $\operatorname{dim} V$ \begin{CJK}{UTF8}{mj}为\end{CJK} $V$ \begin{CJK}{UTF8}{mj}的维数\end{CJK}), \begin{CJK}{UTF8}{mj}问是否有\end{CJK}

\end{enumerate}
$$
\operatorname{dim}\left(V_{1}+V_{2}+V_{3}\right)=\operatorname{dim} V_{1}+\operatorname{dim} V_{2}+\operatorname{dim} V_{3}-\operatorname{dim}\left(V_{1} \cap V_{2}\right)-\operatorname{dim}\left(V_{2} \cap V_{3}\right)-\operatorname{dim}\left(V_{1} \cap V_{3}\right)+\operatorname{dim}\left(V_{1} \cap V_{2} \cap V_{3}\right)
$$

\begin{enumerate}
  \setcounter{enumi}{8}
  \item \begin{CJK}{UTF8}{mj}设\end{CJK} $\alpha_{1}, \alpha_{2}, \cdots, \alpha_{n}$ \begin{CJK}{UTF8}{mj}为\end{CJK} $n$ \begin{CJK}{UTF8}{mj}维非零向量\end{CJK}, \begin{CJK}{UTF8}{mj}且\end{CJK} $\left(\alpha, \alpha_{i}\right)>0(i=1,2, \cdots, n),\left(\alpha_{i}, \alpha_{j}\right) \leqslant 0(i \neq j)$, \begin{CJK}{UTF8}{mj}证明\end{CJK} $\alpha_{1}, \alpha_{2}, \cdots, \alpha_{n}$ \begin{CJK}{UTF8}{mj}线性无关\end{CJK}.

  \item \begin{CJK}{UTF8}{mj}设\end{CJK} $A$ \begin{CJK}{UTF8}{mj}为正定矩阵\end{CJK}, $B$ \begin{CJK}{UTF8}{mj}为半正定矩阵\end{CJK}, \begin{CJK}{UTF8}{mj}证明\end{CJK} $|A+B| \geqslant|B|$, \begin{CJK}{UTF8}{mj}且等号成立当且仅当\end{CJK} $|B|=0$.

  \item $A$ \begin{CJK}{UTF8}{mj}为\end{CJK} $n$ \begin{CJK}{UTF8}{mj}阶矩阵\end{CJK}, \begin{CJK}{UTF8}{mj}试证明存在\end{CJK} $B, C$ \begin{CJK}{UTF8}{mj}使\end{CJK} $A=B+C$, \begin{CJK}{UTF8}{mj}其中\end{CJK} $B C=C B, B$ \begin{CJK}{UTF8}{mj}与一个对角阵相似\end{CJK}, $C$ \begin{CJK}{UTF8}{mj}为幂零矩阵\end{CJK}. $\left(\exists m \in N, C^{m}=0\right)$.

\end{enumerate}
\section{2. 山东大学 2013 年研究生入学考试试题线性代数与常微分方程 
 李扬 
 微信公众号: sxkyliyang}
\begin{enumerate}
  \item \begin{CJK}{UTF8}{mj}求微分方程\end{CJK} $x y^{\prime}-y=(x+y) \ln \frac{x+y}{x}$ \begin{CJK}{UTF8}{mj}的通解\end{CJK}.

  \item \begin{CJK}{UTF8}{mj}求微分方程\end{CJK} $2 y y^{\prime}=y^{\prime 2}+y^{2}$ \begin{CJK}{UTF8}{mj}满足初始条件\end{CJK} $y(0)=1, y^{\prime}(0)=-1$ \begin{CJK}{UTF8}{mj}的通解\end{CJK}.

  \item \begin{CJK}{UTF8}{mj}求微分方程\end{CJK} $y^{\prime \prime}-2 y^{\prime}+y=x^{2} e^{x}$ \begin{CJK}{UTF8}{mj}的通解\end{CJK}.

  \item \begin{CJK}{UTF8}{mj}解微分方程组\end{CJK} $x^{\prime}=A x$, \begin{CJK}{UTF8}{mj}其中\end{CJK} $A=\left(\begin{array}{ccc}1 & -2 & 1 \\ 0 & -4 & 3 \\ 0 & -6 & 5\end{array}\right)$.

  \item \begin{CJK}{UTF8}{mj}设\end{CJK} $A, B$ \begin{CJK}{UTF8}{mj}为数域\end{CJK} $P$ \begin{CJK}{UTF8}{mj}上的\end{CJK} $n$ \begin{CJK}{UTF8}{mj}阶方阵且\end{CJK} $A^{2}=A$, \begin{CJK}{UTF8}{mj}如果\end{CJK} $A X=0$ \begin{CJK}{UTF8}{mj}与\end{CJK} $B X=0$ \begin{CJK}{UTF8}{mj}同解\end{CJK}, \begin{CJK}{UTF8}{mj}证明\end{CJK} $B=B A$.

  \item \begin{CJK}{UTF8}{mj}设\end{CJK} $A, B$ \begin{CJK}{UTF8}{mj}为\end{CJK} $n$ \begin{CJK}{UTF8}{mj}阶正定矩阵\end{CJK}, \begin{CJK}{UTF8}{mj}证明\end{CJK} $|\lambda A-B|=0$ \begin{CJK}{UTF8}{mj}的根都大于\end{CJK} 0 .

  \item \begin{CJK}{UTF8}{mj}设\end{CJK} $A$ \begin{CJK}{UTF8}{mj}为\end{CJK} $n$ \begin{CJK}{UTF8}{mj}阶复方阵且满足\end{CJK} $A^{n-1} \neq 0, A^{n}=0$, \begin{CJK}{UTF8}{mj}计算\end{CJK} $A$ \begin{CJK}{UTF8}{mj}的不变因子和\end{CJK} $A$ \begin{CJK}{UTF8}{mj}的\end{CJK} Jordan \begin{CJK}{UTF8}{mj}标准形\end{CJK}.

  \item \begin{CJK}{UTF8}{mj}设\end{CJK} $A, B$ \begin{CJK}{UTF8}{mj}为\end{CJK} $n$ \begin{CJK}{UTF8}{mj}阶复数矩阵\end{CJK}, $A B=B A$, \begin{CJK}{UTF8}{mj}证明\end{CJK} $A$ \begin{CJK}{UTF8}{mj}与\end{CJK} $B$ \begin{CJK}{UTF8}{mj}有公共的特征向量\end{CJK}.

  \item \begin{CJK}{UTF8}{mj}设\end{CJK} $A, B$ \begin{CJK}{UTF8}{mj}为\end{CJK} $n$ \begin{CJK}{UTF8}{mj}阶实对称矩阵\end{CJK}, $B$ \begin{CJK}{UTF8}{mj}可逆\end{CJK}, \begin{CJK}{UTF8}{mj}且存在\end{CJK} $n$ \begin{CJK}{UTF8}{mj}个互不相同的实数\end{CJK} $\alpha_{1}, \alpha_{2}, \cdots, \alpha_{n}$, \begin{CJK}{UTF8}{mj}满足\end{CJK} $\left|\alpha_{i} B-A\right|=0$, \begin{CJK}{UTF8}{mj}证\end{CJK} \begin{CJK}{UTF8}{mj}明存在\end{CJK} $n$ \begin{CJK}{UTF8}{mj}阶可逆实矩阵\end{CJK} $T$ \begin{CJK}{UTF8}{mj}使\end{CJK} $T^{\prime} A T, T^{\prime} B T$ \begin{CJK}{UTF8}{mj}是对角形\end{CJK}, \begin{CJK}{UTF8}{mj}其中\end{CJK} $A^{\prime}$ \begin{CJK}{UTF8}{mj}表示\end{CJK} $A$ \begin{CJK}{UTF8}{mj}的转置\end{CJK}.

  \item \begin{CJK}{UTF8}{mj}设\end{CJK} $V$ \begin{CJK}{UTF8}{mj}是数域\end{CJK} $P$ \begin{CJK}{UTF8}{mj}上的线性空间\end{CJK}, $W$ \begin{CJK}{UTF8}{mj}是\end{CJK} $V$ \begin{CJK}{UTF8}{mj}的一个子空间\end{CJK}, \begin{CJK}{UTF8}{mj}取\end{CJK} $\alpha_{1}, \alpha_{2} \in V$, \begin{CJK}{UTF8}{mj}令\end{CJK}

\end{enumerate}
$$
W_{1}=\alpha_{1}+W=\left\{\alpha_{1}+w \mid w \in W\right\}, W_{2}=\alpha_{2}+W=\left\{\alpha_{2}+w \mid w \in W\right\}
$$
\begin{CJK}{UTF8}{mj}试求\end{CJK} $W_{1}$ \begin{CJK}{UTF8}{mj}与\end{CJK} $W_{2}$ \begin{CJK}{UTF8}{mj}的交集\end{CJK}.

\section{3. 山东大学 2014 年研究生入学考试试题线性代数与常微分方程 
 李扬 
 微信公众号: sxkyliyang}
\begin{enumerate}
  \item ( 10 \begin{CJK}{UTF8}{mj}分\end{CJK}) \begin{CJK}{UTF8}{mj}求解微分方程\end{CJK} $\frac{\mathrm{d} y}{\mathrm{~d} x}=6 \frac{y}{x}-x y^{2}$.

  \item (10 \begin{CJK}{UTF8}{mj}分\end{CJK}) \begin{CJK}{UTF8}{mj}求解微分方程\end{CJK} $x \sqrt{1+y^{\prime 2}}=y^{\prime}$.

  \item ( 20 \begin{CJK}{UTF8}{mj}分\end{CJK}) \begin{CJK}{UTF8}{mj}对解微分方程组\end{CJK} $\frac{\mathrm{d} y}{\mathrm{~d} t}=A y+f(t)$, \begin{CJK}{UTF8}{mj}其中\end{CJK} $A=\left(\begin{array}{cc}1 & 2 \\ 4 & 3\end{array}\right), f(t)=\left(\begin{array}{c}e^{t} \\ 1\end{array}\right)$, \begin{CJK}{UTF8}{mj}则\end{CJK}:

\end{enumerate}
(1)\begin{CJK}{UTF8}{mj}求其相应的齐次方程组的基解矩阵\end{CJK};

( 2 ) \begin{CJK}{UTF8}{mj}求该非齐次方程组满足初值条件\end{CJK} $y(0)=\left(\begin{array}{c}-1 \\ 1\end{array}\right)$ \begin{CJK}{UTF8}{mj}的特解\end{CJK}.

\begin{enumerate}
  \setcounter{enumi}{4}
  \item ( 10 \begin{CJK}{UTF8}{mj}分\end{CJK}) \begin{CJK}{UTF8}{mj}设方程\end{CJK} $P(x, y) \mathrm{d} x+Q(x, y) \mathrm{d} y=0$ \begin{CJK}{UTF8}{mj}满足\end{CJK} $\frac{\partial P}{\partial y}-\frac{\partial Q}{\partial x}=Q f(x)-P g(y)$, \begin{CJK}{UTF8}{mj}其中\end{CJK} $f(x), g(y)$ \begin{CJK}{UTF8}{mj}分别为\end{CJK} $x, y$ \begin{CJK}{UTF8}{mj}的\end{CJK} \begin{CJK}{UTF8}{mj}连续函数\end{CJK}, \begin{CJK}{UTF8}{mj}证明\end{CJK}: \begin{CJK}{UTF8}{mj}该方程有积分因子\end{CJK} $\mu=e^{\int f(x) \mathrm{d} x+\int g(y) \mathrm{d} y}$.

  \item ( 10 \begin{CJK}{UTF8}{mj}分\end{CJK}) \begin{CJK}{UTF8}{mj}设\end{CJK} $\varphi(x)$ \begin{CJK}{UTF8}{mj}在区间\end{CJK} $(-\infty,+\infty)$ \begin{CJK}{UTF8}{mj}上连续\end{CJK}, \begin{CJK}{UTF8}{mj}试证明\end{CJK}: \begin{CJK}{UTF8}{mj}方程\end{CJK} $\frac{\mathrm{d} y}{\mathrm{~d} x}=\varphi(x) \sin y$ \begin{CJK}{UTF8}{mj}的所有解的存在区间必为\end{CJK} $(-\infty,+\infty)$.

  \item (10 \begin{CJK}{UTF8}{mj}分\end{CJK}) \begin{CJK}{UTF8}{mj}设\end{CJK} $A, B$ \begin{CJK}{UTF8}{mj}为\end{CJK} $n$ \begin{CJK}{UTF8}{mj}阶可逆矩阵且\end{CJK} $I+B A^{-1}$ \begin{CJK}{UTF8}{mj}可逆\end{CJK}, \begin{CJK}{UTF8}{mj}证明\end{CJK}:

\end{enumerate}
(1) $I+A^{-1} B$ \begin{CJK}{UTF8}{mj}可逆\end{CJK};

(2) $(A+B)^{-1}=A^{-1}\left(A^{-1}+B^{-1}\right) B^{-1}$;

(3) $\left(I+A^{-1} B\right)^{-1}=A^{-1}\left(A^{-1}+B^{-1}\right) B^{-1} A$.

\begin{enumerate}
  \setcounter{enumi}{7}
  \item ( 10 \begin{CJK}{UTF8}{mj}分\end{CJK}) \begin{CJK}{UTF8}{mj}设\end{CJK} $\lambda_{1}, \lambda_{2}, \cdots, \lambda_{n}$ \begin{CJK}{UTF8}{mj}为\end{CJK} $n$ \begin{CJK}{UTF8}{mj}阶矩阵\end{CJK} $A$ \begin{CJK}{UTF8}{mj}的特征值\end{CJK}, \begin{CJK}{UTF8}{mj}对应的特征向量为\end{CJK} $\alpha_{1}, \alpha_{2}, \cdots, \alpha_{n}$, \begin{CJK}{UTF8}{mj}设\end{CJK}
\end{enumerate}
$$
f(x)=a_{n} x^{n}+\cdots+a_{1} x+a_{0}
$$
\begin{CJK}{UTF8}{mj}是一个多项式\end{CJK}, \begin{CJK}{UTF8}{mj}求\end{CJK} $f(A)$ \begin{CJK}{UTF8}{mj}的特征值与对应的特征向量\end{CJK}.

\begin{enumerate}
  \setcounter{enumi}{8}
  \item ( 10 \begin{CJK}{UTF8}{mj}分\end{CJK}) \begin{CJK}{UTF8}{mj}若实矩阵\end{CJK} $A$ \begin{CJK}{UTF8}{mj}可对角化\end{CJK}, \begin{CJK}{UTF8}{mj}且\end{CJK} $A$ \begin{CJK}{UTF8}{mj}的特征值都是非负的\end{CJK}, \begin{CJK}{UTF8}{mj}则\end{CJK}:
\end{enumerate}
(1) \begin{CJK}{UTF8}{mj}证明\end{CJK}: \begin{CJK}{UTF8}{mj}存在一个实矩阵\end{CJK} $B$, \begin{CJK}{UTF8}{mj}使得\end{CJK} $A=B^{2}$;

$(2)$ \begin{CJK}{UTF8}{mj}设\end{CJK} $A=\left(\begin{array}{ccc}-1 & 2 & -2 \\ -2 & 3 & -1 \\ 2 & -2 & 4\end{array}\right)$, \begin{CJK}{UTF8}{mj}求\end{CJK} $B$, \begin{CJK}{UTF8}{mj}使得\end{CJK} $A=B^{2}$.

\begin{enumerate}
  \setcounter{enumi}{9}
  \item (10 \begin{CJK}{UTF8}{mj}分\end{CJK}) \begin{CJK}{UTF8}{mj}设\end{CJK} $A$ \begin{CJK}{UTF8}{mj}为\end{CJK} $m \times n$ \begin{CJK}{UTF8}{mj}阶矩阵\end{CJK}, $B$ \begin{CJK}{UTF8}{mj}为\end{CJK} $m \times 1$ \begin{CJK}{UTF8}{mj}阶矩阵\end{CJK}, \begin{CJK}{UTF8}{mj}证明\end{CJK}: \begin{CJK}{UTF8}{mj}方程组\end{CJK} $A X=B$ \begin{CJK}{UTF8}{mj}有解的充分必要条件是\end{CJK} $A^{\prime} Y=0$ \begin{CJK}{UTF8}{mj}的任一解向量\end{CJK} $Y_{0}$ \begin{CJK}{UTF8}{mj}都是\end{CJK} $B^{\prime} Y=0$ \begin{CJK}{UTF8}{mj}的解向量\end{CJK}.

  \item ( 20 \begin{CJK}{UTF8}{mj}分\end{CJK}) \begin{CJK}{UTF8}{mj}设\end{CJK} $A=\left(a_{i j}\right)_{m \times n}, V_{1}$ \begin{CJK}{UTF8}{mj}为\end{CJK} $A X=0$ \begin{CJK}{UTF8}{mj}的解空间\end{CJK}, \begin{CJK}{UTF8}{mj}令\end{CJK}

\end{enumerate}
$$
\alpha_{i}=\left(\alpha_{i 1}, \alpha_{i 2}, \cdots, \alpha_{i n}\right)^{\prime}, i=1,2, \cdots, m, V_{2}=L\left(\alpha_{1}, \alpha_{2}, \cdots, \alpha_{m}\right)
$$
\begin{CJK}{UTF8}{mj}证明\end{CJK}: $\mathbb{R}^{n}=V_{1} \oplus V_{2}$.

\begin{enumerate}
  \setcounter{enumi}{11}
  \item (10 \begin{CJK}{UTF8}{mj}分\end{CJK}) \begin{CJK}{UTF8}{mj}已知\end{CJK} $n$ \begin{CJK}{UTF8}{mj}阶方阵\end{CJK} $A$ \begin{CJK}{UTF8}{mj}满足\end{CJK} $(A-a I)(A-b I)=0$, \begin{CJK}{UTF8}{mj}其中\end{CJK} $a \neq b$, \begin{CJK}{UTF8}{mj}证明\end{CJK}: $A$ \begin{CJK}{UTF8}{mj}与对角阵相似\end{CJK}.
\end{enumerate}
\section{4. 山东大学 2015 年研究生入学考试试题线性代数与常微分方程 
 李扬 
 微信公众号: sxkyliyang}
\begin{CJK}{UTF8}{mj}一\end{CJK}、\begin{CJK}{UTF8}{mj}计算题\end{CJK} (\begin{CJK}{UTF8}{mj}共\end{CJK} 3 \begin{CJK}{UTF8}{mj}题\end{CJK}, \begin{CJK}{UTF8}{mj}第\end{CJK} 1,2 \begin{CJK}{UTF8}{mj}题每题\end{CJK} 10 \begin{CJK}{UTF8}{mj}分\end{CJK}, \begin{CJK}{UTF8}{mj}第\end{CJK} 3 \begin{CJK}{UTF8}{mj}题\end{CJK} 20 \begin{CJK}{UTF8}{mj}分\end{CJK})

\begin{enumerate}
  \item \begin{CJK}{UTF8}{mj}求解微分方程\end{CJK} $\frac{\mathrm{d} y}{\mathrm{~d} x}=\frac{2 x+y+1}{4 x+2 y-3}$.

  \item \begin{CJK}{UTF8}{mj}求解微分方程\end{CJK} $y=\left(y^{\prime}\right)^{2}-x y^{\prime}+\frac{x^{2}}{2}$.

  \item \begin{CJK}{UTF8}{mj}求解微分方程\end{CJK} $y^{\prime \prime}-4 y^{\prime}+4 y=e^{2 x}+\cos 2 x$.

\end{enumerate}
\begin{CJK}{UTF8}{mj}二\end{CJK}、\begin{CJK}{UTF8}{mj}证明题\end{CJK} (\begin{CJK}{UTF8}{mj}共\end{CJK} 2 \begin{CJK}{UTF8}{mj}题\end{CJK}, \begin{CJK}{UTF8}{mj}每题\end{CJK} 10 \begin{CJK}{UTF8}{mj}分\end{CJK})

\begin{enumerate}
  \item \begin{CJK}{UTF8}{mj}证明\end{CJK}: \begin{CJK}{UTF8}{mj}齐次方程\end{CJK} $P(x, y) \mathrm{d} x+Q(x, y) \mathrm{d} y=0$ \begin{CJK}{UTF8}{mj}有积分因子\end{CJK} $\mu=\frac{1}{x P+y Q}$.

  \item \begin{CJK}{UTF8}{mj}设方程\end{CJK} $\frac{\mathrm{d} y}{\mathrm{~d} x}=x^{2} f(y)$ \begin{CJK}{UTF8}{mj}中\end{CJK}, $f(y)$ \begin{CJK}{UTF8}{mj}在\end{CJK} $(-\infty,+\infty)$ \begin{CJK}{UTF8}{mj}上连续可微\end{CJK}, \begin{CJK}{UTF8}{mj}且\end{CJK} $y f(y)<0(y \neq 0)$. \begin{CJK}{UTF8}{mj}证明\end{CJK}: \begin{CJK}{UTF8}{mj}该方程的任一满\end{CJK} \begin{CJK}{UTF8}{mj}足初值条件\end{CJK} $y\left(x_{0}\right)=y_{0}$ \begin{CJK}{UTF8}{mj}的解\end{CJK} $y(x)$ \begin{CJK}{UTF8}{mj}必在区间\end{CJK} $\left[x_{0},+\infty\right)$ \begin{CJK}{UTF8}{mj}上存在\end{CJK}.

\end{enumerate}
\begin{CJK}{UTF8}{mj}三\end{CJK}、\begin{CJK}{UTF8}{mj}计算题\end{CJK} (\begin{CJK}{UTF8}{mj}共\end{CJK} 3 \begin{CJK}{UTF8}{mj}题\end{CJK}, \begin{CJK}{UTF8}{mj}每题\end{CJK} 10 \begin{CJK}{UTF8}{mj}分\end{CJK})

\begin{enumerate}
  \item \begin{CJK}{UTF8}{mj}求\end{CJK} $n$ \begin{CJK}{UTF8}{mj}阶矩阵\end{CJK} $A=\left(\begin{array}{ccccc}a & b & \cdots & b & b \\ b & a & \cdots & b & b \\ \vdots & \vdots & \ddots & \vdots & \vdots \\ b & b & \cdots & a & b \\ b & b & \cdots & b & a\end{array}\right)$ \begin{CJK}{UTF8}{mj}的秩\end{CJK}.

  \item \begin{CJK}{UTF8}{mj}设\end{CJK} $A=\left(\begin{array}{cccc}3 & 0 & 8 & 0 \\ 3 & -1 & 6 & 0 \\ -2 & 0 & -5 & 0 \\ 0 & 0 & 0 & 2\end{array}\right)$, \begin{CJK}{UTF8}{mj}求\end{CJK} $A$ \begin{CJK}{UTF8}{mj}的\end{CJK} Jordan \begin{CJK}{UTF8}{mj}标准型\end{CJK} $J$, \begin{CJK}{UTF8}{mj}并求出\end{CJK} $P$, \begin{CJK}{UTF8}{mj}使\end{CJK} $P^{-1} A P=J$.

  \item \begin{CJK}{UTF8}{mj}设二次型\end{CJK} $f\left(x_{1}, x_{2}, x_{3}, x_{4}\right)=X^{\prime} A X$ \begin{CJK}{UTF8}{mj}的正惯性指数为\end{CJK} 1 , \begin{CJK}{UTF8}{mj}又矩阵\end{CJK} $A$ \begin{CJK}{UTF8}{mj}满足\end{CJK} $A^{2}-2 A=3 I$, \begin{CJK}{UTF8}{mj}求此二次型的规\end{CJK} \begin{CJK}{UTF8}{mj}范型\end{CJK}.

\end{enumerate}
\begin{CJK}{UTF8}{mj}四\end{CJK}、\begin{CJK}{UTF8}{mj}证明题\end{CJK} (\begin{CJK}{UTF8}{mj}共\end{CJK} 3 \begin{CJK}{UTF8}{mj}题\end{CJK}, \begin{CJK}{UTF8}{mj}第\end{CJK} 1 \begin{CJK}{UTF8}{mj}题\end{CJK} 20 \begin{CJK}{UTF8}{mj}分\end{CJK}, \begin{CJK}{UTF8}{mj}第\end{CJK} 2 \begin{CJK}{UTF8}{mj}题\end{CJK} 10 \begin{CJK}{UTF8}{mj}分\end{CJK}, \begin{CJK}{UTF8}{mj}第\end{CJK} 3 \begin{CJK}{UTF8}{mj}题\end{CJK} 30 \begin{CJK}{UTF8}{mj}分\end{CJK})
$$
\text { 1. 设方程组 } A X=0 \text {, 其中 } A=\left(\begin{array}{cccc}
a_{11} & a_{12} & \cdots & a_{1 n} \\
a_{21} & a_{22} & \cdots & a_{2 n} \\
\vdots & \vdots & & \vdots \\
a_{n-1,1} & a_{n-1,2} & \cdots & a_{n-1, n}
\end{array}\right) \text {, 记 } M_{i}(i=1,2, \cdots, n) \text { 为 } A \text { 中划去 }
$$
\begin{CJK}{UTF8}{mj}第\end{CJK} $i$ \begin{CJK}{UTF8}{mj}列得到的\end{CJK} $n-1$ \begin{CJK}{UTF8}{mj}阶子式\end{CJK}, \begin{CJK}{UTF8}{mj}令\end{CJK} $\alpha=\left(M_{1},-M_{2}, \cdots,(-1)^{n-1} M_{n}\right)^{\prime}$, \begin{CJK}{UTF8}{mj}则\end{CJK}:

(1) \begin{CJK}{UTF8}{mj}证明\end{CJK}: $\alpha$ \begin{CJK}{UTF8}{mj}是\end{CJK} $A X=0$ \begin{CJK}{UTF8}{mj}的一个解\end{CJK};

(2) \begin{CJK}{UTF8}{mj}若\end{CJK} $r(A)=n-1$, \begin{CJK}{UTF8}{mj}则\end{CJK} $A X=0$ \begin{CJK}{UTF8}{mj}的通解为\end{CJK} $k \alpha, k \in \mathbb{R}$.

\begin{enumerate}
  \setcounter{enumi}{2}
  \item $V_{1}, V_{2}, \cdots, V_{m}$ \begin{CJK}{UTF8}{mj}是数域\end{CJK} $F$ \begin{CJK}{UTF8}{mj}上向量空间\end{CJK} $V$ \begin{CJK}{UTF8}{mj}的\end{CJK} $m$ \begin{CJK}{UTF8}{mj}个真子空间\end{CJK}, \begin{CJK}{UTF8}{mj}证明\end{CJK}: $V$ \begin{CJK}{UTF8}{mj}中必存在一个向量\end{CJK} $\alpha$, \begin{CJK}{UTF8}{mj}它不属于任\end{CJK} $-V_{i}$.

  \item \begin{CJK}{UTF8}{mj}设\end{CJK} $\alpha_{1}, \alpha_{2}, \cdots, \alpha_{n}$ \begin{CJK}{UTF8}{mj}是标准正交列向量组\end{CJK}, $k$ \begin{CJK}{UTF8}{mj}为实数\end{CJK}, \begin{CJK}{UTF8}{mj}矩阵\end{CJK} $H=I-k \alpha_{1} \alpha_{1}^{\prime}$.

\end{enumerate}
(1) \begin{CJK}{UTF8}{mj}证明\end{CJK}: $\alpha_{1}$ \begin{CJK}{UTF8}{mj}是\end{CJK} $H$ \begin{CJK}{UTF8}{mj}的特征向量\end{CJK}, \begin{CJK}{UTF8}{mj}并求出对应的特征值\end{CJK}.

( 2 ) \begin{CJK}{UTF8}{mj}证明\end{CJK}: $\alpha_{2}, \alpha_{3}, \cdots, \alpha_{n}$ \begin{CJK}{UTF8}{mj}也是\end{CJK} $H$ \begin{CJK}{UTF8}{mj}的特征向量\end{CJK}, \begin{CJK}{UTF8}{mj}并求出对应的特征值\end{CJK}.

(3)\begin{CJK}{UTF8}{mj}当\end{CJK} $k<1$ \begin{CJK}{UTF8}{mj}时\end{CJK}, \begin{CJK}{UTF8}{mj}证明\end{CJK}: $H$ \begin{CJK}{UTF8}{mj}为正定阵\end{CJK}.

\section{5. 山东大学 2016 年研究生入学考试试题线性代数与常微分方程 
 李扬 
 微信公众号: sxkyliyang}
\begin{enumerate}
  \item (10 \begin{CJK}{UTF8}{mj}分\end{CJK}) \begin{CJK}{UTF8}{mj}求方程\end{CJK} $\left(y^{\prime}\right)^{2}-2 x y^{\prime}+1=0$ \begin{CJK}{UTF8}{mj}的通解\end{CJK}.

  \item (10 \begin{CJK}{UTF8}{mj}分\end{CJK}) \begin{CJK}{UTF8}{mj}求\end{CJK} $\frac{\mathrm{d} y}{\mathrm{~d} x}=\frac{y-x+1}{y+x+5}$ \begin{CJK}{UTF8}{mj}的通解\end{CJK}.

  \item (20 \begin{CJK}{UTF8}{mj}分\end{CJK}) \begin{CJK}{UTF8}{mj}求方程组\end{CJK} $\left\{\begin{array}{l}\frac{\mathrm{d} x}{\mathrm{~d} t}=x-2 y-z ; \\ \frac{\mathrm{d} y}{\mathrm{~d} t}=y-x+z ; \\ \frac{\mathrm{d} z}{\mathrm{~d} t}=x-z .\end{array}\right.$ \begin{CJK}{UTF8}{mj}的通解\end{CJK}.

  \item (10 \begin{CJK}{UTF8}{mj}分\end{CJK}) \begin{CJK}{UTF8}{mj}证明\end{CJK}: \begin{CJK}{UTF8}{mj}方程\end{CJK} $\frac{\mathrm{d} x}{\mathrm{~d} t}=x^{2}+t^{2}$ \begin{CJK}{UTF8}{mj}任一解的存在区间都是有界的\end{CJK}.

  \item ( 10 \begin{CJK}{UTF8}{mj}分\end{CJK}) \begin{CJK}{UTF8}{mj}设\end{CJK} $f(x, y)$ \begin{CJK}{UTF8}{mj}在\end{CJK} $\left(x_{0}, y_{0}\right)$ \begin{CJK}{UTF8}{mj}的邻域内是\end{CJK} $y$ \begin{CJK}{UTF8}{mj}的不增函数\end{CJK}, \begin{CJK}{UTF8}{mj}试证\end{CJK}: \begin{CJK}{UTF8}{mj}初值\end{CJK}.\begin{CJK}{UTF8}{mj}问题\end{CJK} $\left\{\begin{array}{l}y^{\prime}=f(x, y) ; \\ y\left(x_{0}\right)=y_{0} .\end{array}\right.$ \begin{CJK}{UTF8}{mj}的右行解\end{CJK} $\left(x \geqslant x_{0}\right)$ \begin{CJK}{UTF8}{mj}至多只有一个\end{CJK}.

  \item ( 20 \begin{CJK}{UTF8}{mj}分\end{CJK}) \begin{CJK}{UTF8}{mj}设\end{CJK} $\alpha_{1}=\left(\frac{1}{2},-\frac{1}{2},-\frac{1}{2}, \frac{1}{2}\right), \alpha_{2}=\left(\frac{\sqrt{2}}{2}, \frac{\sqrt{2}}{2}, 0,0\right), \alpha_{3}=\left(0,0, \frac{\sqrt{2}}{2}, \frac{\sqrt{2}}{2}\right)$, \begin{CJK}{UTF8}{mj}则\end{CJK}:

\end{enumerate}
(1) \begin{CJK}{UTF8}{mj}求一个正交阵以\end{CJK} $\alpha_{1}, \alpha_{2}, \alpha_{3}$ \begin{CJK}{UTF8}{mj}为前\end{CJK} 3 \begin{CJK}{UTF8}{mj}列\end{CJK};

(2) \begin{CJK}{UTF8}{mj}这样的正交阵共有几个\end{CJK}?

\begin{enumerate}
  \setcounter{enumi}{7}
  \item (10 \begin{CJK}{UTF8}{mj}分\end{CJK}) \begin{CJK}{UTF8}{mj}设\end{CJK} $A$ \begin{CJK}{UTF8}{mj}为\end{CJK} $n$ \begin{CJK}{UTF8}{mj}阶可逆的反对称矩阵\end{CJK}, $\beta$ \begin{CJK}{UTF8}{mj}为\end{CJK} $n$ \begin{CJK}{UTF8}{mj}元列向量\end{CJK}, $B=\left(\begin{array}{cc}A & \beta \\ -\beta^{\prime} & 0\end{array}\right)$, \begin{CJK}{UTF8}{mj}求矩阵\end{CJK} $B$ \begin{CJK}{UTF8}{mj}的秩\end{CJK} $r(B)$.

  \item ( 30 \begin{CJK}{UTF8}{mj}分\end{CJK}) \begin{CJK}{UTF8}{mj}设\end{CJK} $f$ \begin{CJK}{UTF8}{mj}是\end{CJK} $n$ \begin{CJK}{UTF8}{mj}维向量空间\end{CJK} $V$ \begin{CJK}{UTF8}{mj}的一个线性变换\end{CJK}, \begin{CJK}{UTF8}{mj}且\end{CJK} $f$ \begin{CJK}{UTF8}{mj}是幂等的\end{CJK}, \begin{CJK}{UTF8}{mj}即\end{CJK} $f^{2}=f$. \begin{CJK}{UTF8}{mj}证明\end{CJK}:

\end{enumerate}
(1) $\operatorname{Ker} f=\{x-f(x) \mid x \in V\}$;

(2) $V=\operatorname{Ker} f \oplus \operatorname{Im} f$.

(3) \begin{CJK}{UTF8}{mj}若\end{CJK} $g$ \begin{CJK}{UTF8}{mj}是\end{CJK} $V$ \begin{CJK}{UTF8}{mj}的一个线性变换\end{CJK}, \begin{CJK}{UTF8}{mj}则\end{CJK} $\operatorname{Ker} f$ \begin{CJK}{UTF8}{mj}和\end{CJK} $\operatorname{Im} f$ \begin{CJK}{UTF8}{mj}都是\end{CJK} $g$ \begin{CJK}{UTF8}{mj}的不变子空间当且仅当\end{CJK} $f$ \begin{CJK}{UTF8}{mj}与\end{CJK} $g$ \begin{CJK}{UTF8}{mj}可交换\end{CJK}, \begin{CJK}{UTF8}{mj}即\end{CJK} $f g=g f$.

\begin{enumerate}
  \setcounter{enumi}{9}
  \item (10 \begin{CJK}{UTF8}{mj}分\end{CJK}) \begin{CJK}{UTF8}{mj}设\end{CJK} $A, B$ \begin{CJK}{UTF8}{mj}为\end{CJK} $n$ \begin{CJK}{UTF8}{mj}阶实对称阵且\end{CJK} $A$ \begin{CJK}{UTF8}{mj}是正定的\end{CJK}, \begin{CJK}{UTF8}{mj}证明\end{CJK}: \begin{CJK}{UTF8}{mj}存在\end{CJK} $n$ \begin{CJK}{UTF8}{mj}阶实可逆阵\end{CJK} $P$ \begin{CJK}{UTF8}{mj}使\end{CJK} $P^{\prime} A P$ \begin{CJK}{UTF8}{mj}与\end{CJK} $P^{\prime} B P$ \begin{CJK}{UTF8}{mj}都是对角阵\end{CJK}.

  \item (10 \begin{CJK}{UTF8}{mj}分\end{CJK}) \begin{CJK}{UTF8}{mj}设\end{CJK} $A, B$ \begin{CJK}{UTF8}{mj}为\end{CJK} $n$ \begin{CJK}{UTF8}{mj}阶正定阵\end{CJK}, \begin{CJK}{UTF8}{mj}且方程\end{CJK} $|x A-B|=0$ \begin{CJK}{UTF8}{mj}的根全是\end{CJK} 1 , \begin{CJK}{UTF8}{mj}证明\end{CJK}: $A=B$.

  \item (10 \begin{CJK}{UTF8}{mj}分\end{CJK}) \begin{CJK}{UTF8}{mj}设\end{CJK} $A$ \begin{CJK}{UTF8}{mj}是\end{CJK} $n \times s$ \begin{CJK}{UTF8}{mj}矩阵\end{CJK}, \begin{CJK}{UTF8}{mj}证明\end{CJK}: \begin{CJK}{UTF8}{mj}在实数域上\end{CJK}, $A^{\prime} A$ \begin{CJK}{UTF8}{mj}必为半正定阵\end{CJK}; \begin{CJK}{UTF8}{mj}若\end{CJK} $A$ \begin{CJK}{UTF8}{mj}的列向量线性无关\end{CJK}, \begin{CJK}{UTF8}{mj}则\end{CJK} $A^{\prime} A$ \begin{CJK}{UTF8}{mj}为正定\end{CJK} \begin{CJK}{UTF8}{mj}阵\end{CJK}.

\end{enumerate}
\section{6. 山东大学 2017 年研究生入学考试试题线性代数与常微分方程 
 李扬 
 微信公众号: sxkyliyang}
\begin{enumerate}
  \item (10 \begin{CJK}{UTF8}{mj}分\end{CJK}) \begin{CJK}{UTF8}{mj}求方程\end{CJK} $\left(x^{3}+x y^{2}\right) \mathrm{d} x+\left(x^{2} y+y^{3}\right) \mathrm{d} y=0$ \begin{CJK}{UTF8}{mj}的通解\end{CJK}.

  \item (10 \begin{CJK}{UTF8}{mj}分\end{CJK}) \begin{CJK}{UTF8}{mj}求方程\end{CJK} $x^{\prime \prime}+x=1-\frac{1}{\sin t}$ \begin{CJK}{UTF8}{mj}的通解\end{CJK}.

  \item ( 20 \begin{CJK}{UTF8}{mj}分\end{CJK} ) \begin{CJK}{UTF8}{mj}求方程组\end{CJK} $\left\{\begin{array}{l}\frac{\mathrm{d} x}{\mathrm{~d} t}+5 x+y=0 ; \\ \frac{\mathrm{d} y}{\mathrm{~d} t}-2 x+3 y=0 .\end{array} \quad\right.$ \begin{CJK}{UTF8}{mj}的通解\end{CJK}.

  \item ( 10 \begin{CJK}{UTF8}{mj}分\end{CJK}) \begin{CJK}{UTF8}{mj}设\end{CJK} $f(x, y)$ \begin{CJK}{UTF8}{mj}在整个平面上连续有界\end{CJK}, \begin{CJK}{UTF8}{mj}且\end{CJK} $\frac{\partial f(x, y)}{\partial y}$ \begin{CJK}{UTF8}{mj}也是连续的\end{CJK}. \begin{CJK}{UTF8}{mj}试证方程\end{CJK} $\frac{\mathrm{d} y}{\mathrm{~d} x}=f(x, y)$ \begin{CJK}{UTF8}{mj}的每个解\end{CJK} $y=\varphi(x)$ \begin{CJK}{UTF8}{mj}在整个实轴\end{CJK} $(-\infty,+\infty)$ \begin{CJK}{UTF8}{mj}上存在\end{CJK}.

  \item (10 \begin{CJK}{UTF8}{mj}分\end{CJK}) \begin{CJK}{UTF8}{mj}设\end{CJK} $f(t, x)$ \begin{CJK}{UTF8}{mj}在整个平面上都有定义\end{CJK}, \begin{CJK}{UTF8}{mj}连续且有界\end{CJK}, \begin{CJK}{UTF8}{mj}有关于\end{CJK} $x$ \begin{CJK}{UTF8}{mj}的连续一阶偏导数\end{CJK}, \begin{CJK}{UTF8}{mj}试证方程\end{CJK} $\frac{\mathrm{d} x}{\mathrm{~d} t}=f(t, x)$ \begin{CJK}{UTF8}{mj}的任一解均可延拓到整个区间\end{CJK} $(-\infty,+\infty)$.

  \item (10 \begin{CJK}{UTF8}{mj}分\end{CJK}) \begin{CJK}{UTF8}{mj}计算\end{CJK}: $\left(\begin{array}{cc}1 & -1 \\ 2 & 4\end{array}\right)^{2017}$.

  \item ( 10 \begin{CJK}{UTF8}{mj}分\end{CJK}) \begin{CJK}{UTF8}{mj}设\end{CJK} $A$ \begin{CJK}{UTF8}{mj}是\end{CJK} $m \times n$ \begin{CJK}{UTF8}{mj}复矩阵\end{CJK}, $X$ \begin{CJK}{UTF8}{mj}是\end{CJK} $n \times m$ \begin{CJK}{UTF8}{mj}末知数矩阵\end{CJK}. \begin{CJK}{UTF8}{mj}证明\end{CJK}: \begin{CJK}{UTF8}{mj}矩阵方程\end{CJK} $A X A=A$ \begin{CJK}{UTF8}{mj}恒有解\end{CJK}.

  \item (10 \begin{CJK}{UTF8}{mj}分\end{CJK}) \begin{CJK}{UTF8}{mj}设\end{CJK} $n$ \begin{CJK}{UTF8}{mj}阶方阵\end{CJK} $J=\left(\begin{array}{ccc}1 & \cdots & 1 \\ \vdots & & \vdots \\ 1 & \cdots & 1\end{array}\right)$, \begin{CJK}{UTF8}{mj}即每个位置元素均为\end{CJK} 1 , \begin{CJK}{UTF8}{mj}求其若尔当\end{CJK} (Jordan) \begin{CJK}{UTF8}{mj}标准形\end{CJK}.

  \item ( 20 \begin{CJK}{UTF8}{mj}分\end{CJK}) \begin{CJK}{UTF8}{mj}设\end{CJK} $A$ \begin{CJK}{UTF8}{mj}是\end{CJK} $n$ \begin{CJK}{UTF8}{mj}阶矩阵且不可逆\end{CJK}, $A^{*}$ \begin{CJK}{UTF8}{mj}为\end{CJK} $A$ \begin{CJK}{UTF8}{mj}的伴随矩阵\end{CJK}, $I$ \begin{CJK}{UTF8}{mj}为\end{CJK} $n$ \begin{CJK}{UTF8}{mj}阶单位阵\end{CJK}, $A^{*}$ \begin{CJK}{UTF8}{mj}的迹\end{CJK} $\operatorname{tr} A^{*}=a \neq 0$,

\end{enumerate}
(1) \begin{CJK}{UTF8}{mj}求矩阵\end{CJK} $A$ \begin{CJK}{UTF8}{mj}的秩\end{CJK} $r(A)$;

(2) $\left|\lambda I-A^{*}\right| .$

\begin{enumerate}
  \setcounter{enumi}{10}
  \item ( 10 \begin{CJK}{UTF8}{mj}分\end{CJK}) \begin{CJK}{UTF8}{mj}设\end{CJK} $\lambda_{1}, \lambda_{2}, \cdots, \lambda_{n}$ \begin{CJK}{UTF8}{mj}是矩阵\end{CJK} $A$ \begin{CJK}{UTF8}{mj}的\end{CJK} $n$ \begin{CJK}{UTF8}{mj}个特征值\end{CJK}, \begin{CJK}{UTF8}{mj}求矩阵\end{CJK} $B=\left(\begin{array}{cc}0 & A \\ A & 0\end{array}\right)$ \begin{CJK}{UTF8}{mj}的特征值\end{CJK}.

  \item (10 \begin{CJK}{UTF8}{mj}分\end{CJK}) \begin{CJK}{UTF8}{mj}设\end{CJK} $A, B, C$ \begin{CJK}{UTF8}{mj}为\end{CJK} $n$ \begin{CJK}{UTF8}{mj}阶方阵\end{CJK}, \begin{CJK}{UTF8}{mj}且\end{CJK} $A=B C$, \begin{CJK}{UTF8}{mj}如果\end{CJK} $B^{2}=B, C^{2}=C$, \begin{CJK}{UTF8}{mj}证明\end{CJK}: $r(E-A) \leqslant 2 n-2 r(A)$, \begin{CJK}{UTF8}{mj}其中\end{CJK} $r(A)$ \begin{CJK}{UTF8}{mj}表示矩阵\end{CJK} $A$ \begin{CJK}{UTF8}{mj}的秩\end{CJK}.

  \item ( 20 \begin{CJK}{UTF8}{mj}分\end{CJK}) \begin{CJK}{UTF8}{mj}设\end{CJK} $\varphi$ \begin{CJK}{UTF8}{mj}是线性空间\end{CJK} $V$ \begin{CJK}{UTF8}{mj}的一个线性变换\end{CJK}, \begin{CJK}{UTF8}{mj}且\end{CJK} $\varphi^{2}=\varphi$. \begin{CJK}{UTF8}{mj}令\end{CJK}

\end{enumerate}
$$
V_{1}=\{\alpha \in V \mid \varphi(\alpha)=\alpha\}, V_{0}=\{\alpha \in V \mid \varphi(\alpha)=0\}
$$
\begin{CJK}{UTF8}{mj}证明\end{CJK}:

(1) $V_{0}, V_{1}$ \begin{CJK}{UTF8}{mj}是\end{CJK} $V$ \begin{CJK}{UTF8}{mj}的线性子空间\end{CJK}.

(2) $V=V_{1} \oplus V_{0}$.

\section{7. 山东大学 2009 年研究生入学考试试题数学分析 
 李扬 
 微信公众号: sxkyliyang}
\begin{enumerate}
  \item \begin{CJK}{UTF8}{mj}设函数\end{CJK} $f(x)=\varphi(a+b x)-\varphi(a-b x)$ \begin{CJK}{UTF8}{mj}其中\end{CJK} $\varphi(x)$ \begin{CJK}{UTF8}{mj}在\end{CJK} $x=a$ \begin{CJK}{UTF8}{mj}的某个小邻域内有定义且可导\end{CJK}, \begin{CJK}{UTF8}{mj}求\end{CJK} $f^{\prime}(0)$.

  \item \begin{CJK}{UTF8}{mj}设\end{CJK} $0<x<y<\pi$, \begin{CJK}{UTF8}{mj}证明\end{CJK} $y \sin y+2 \cos y+\pi y>x \sin x+2 \cos x+\pi x$.

\end{enumerate}
3 . \begin{CJK}{UTF8}{mj}设\end{CJK} $x>0, y>0$, \begin{CJK}{UTF8}{mj}求\end{CJK} $f(x, y)=x^{2} y(4-x-y)$ \begin{CJK}{UTF8}{mj}的极值\end{CJK}.

\begin{enumerate}
  \setcounter{enumi}{4}
  \item \begin{CJK}{UTF8}{mj}设\end{CJK} $f(x)=\frac{\int_{0}^{x} \mathrm{~d} u \int_{0}^{u^{2}} \arctan (1+t) \mathrm{d} t}{x(1-\cos x)}$, \begin{CJK}{UTF8}{mj}求\end{CJK} $\lim _{x \rightarrow 0} f(x)$.

  \item \begin{CJK}{UTF8}{mj}计算\end{CJK}

\end{enumerate}
$$
\oint_{C} x \mathrm{~d} y-y \mathrm{~d} x
$$
\begin{CJK}{UTF8}{mj}其中\end{CJK} $C$ \begin{CJK}{UTF8}{mj}为椭圆\end{CJK} $(x+2 y)^{2}+(3 x+2 y)^{2}=1$, \begin{CJK}{UTF8}{mj}方向为逆时针方向\end{CJK}.

\begin{enumerate}
  \setcounter{enumi}{6}
  \item \begin{CJK}{UTF8}{mj}计算\end{CJK}
\end{enumerate}
$$
\iint_{S}(x-y) \mathrm{d} x \mathrm{~d} y+x(y-z) \mathrm{d} y \mathrm{~d} z
$$
\begin{CJK}{UTF8}{mj}其中\end{CJK} $S$ \begin{CJK}{UTF8}{mj}为柱面\end{CJK} $x^{2}+y^{2}=1$ \begin{CJK}{UTF8}{mj}及平面\end{CJK} $z=0, z=3$ \begin{CJK}{UTF8}{mj}所围成的区域\end{CJK} $\Omega$ \begin{CJK}{UTF8}{mj}的整个边界曲面外侧\end{CJK}.

\begin{enumerate}
  \setcounter{enumi}{7}
  \item \begin{CJK}{UTF8}{mj}设\end{CJK} $f(x)=\sin \sqrt{x}$, \begin{CJK}{UTF8}{mj}判断\end{CJK} $f(x)$ \begin{CJK}{UTF8}{mj}在\end{CJK} $[0,+\infty)$ \begin{CJK}{UTF8}{mj}上是否一致收敛\end{CJK}, \begin{CJK}{UTF8}{mj}并证明\end{CJK}.

  \item \begin{CJK}{UTF8}{mj}计算积分\end{CJK} $I=\iint_{D} \min \left\{x^{2} y, 2\right\} \mathrm{d} x \mathrm{~d} y$, \begin{CJK}{UTF8}{mj}其中\end{CJK} $D=\{(x, y) \mid 0 \leqslant x \leqslant 4,0 \leqslant y \leqslant 3\}$.

  \item \begin{CJK}{UTF8}{mj}计算积分\end{CJK} $I(y)=\int_{0}^{+\infty} e^{-x^{2}} \sin 2 x y \mathrm{~d} x$.

\end{enumerate}
$10 .$ \begin{CJK}{UTF8}{mj}设\end{CJK} $f(x, y)=\left\{\begin{array}{ll}\frac{x y^{2}}{x^{2}+y^{2}}, & \text { 当 } x^{2}+y^{2} \neq 0 ; \\ 0 . & \text { 当 } x^{2}+y^{2}=0 .\end{array}\right.$, \begin{CJK}{UTF8}{mj}讨论\end{CJK}:

(1) $f(x, y)$ \begin{CJK}{UTF8}{mj}的连续性\end{CJK};

(2) $f_{x}, f_{y}$ \begin{CJK}{UTF8}{mj}的存在性及连续性\end{CJK};

(3) $f(x, y)$ \begin{CJK}{UTF8}{mj}的可微性\end{CJK}.

\begin{enumerate}
  \setcounter{enumi}{11}
  \item \begin{CJK}{UTF8}{mj}设\end{CJK} $x_{0}=\sqrt{6}, x_{n+1}=\sqrt{6+x_{n}}, n=0,1,2, \cdots$, \begin{CJK}{UTF8}{mj}判断级数\end{CJK} $\sum_{n=0}^{\infty} \sqrt{3-x_{n}}$ \begin{CJK}{UTF8}{mj}的敛散性\end{CJK}.

  \item \begin{CJK}{UTF8}{mj}设\end{CJK} $f(x)$ \begin{CJK}{UTF8}{mj}在\end{CJK} $(-\infty,+\infty)$ \begin{CJK}{UTF8}{mj}有连续的一阶导数\end{CJK}, \begin{CJK}{UTF8}{mj}证明\end{CJK}:\\
(1) \begin{CJK}{UTF8}{mj}若\end{CJK} $\lim _{|x| \rightarrow+\infty} f^{\prime}(x)=\alpha>0$, \begin{CJK}{UTF8}{mj}则方程\end{CJK} $f(x)=0$ \begin{CJK}{UTF8}{mj}在\end{CJK} $(-\infty,+\infty)$ \begin{CJK}{UTF8}{mj}至少有一个实根\end{CJK};\\
(2) \begin{CJK}{UTF8}{mj}若\end{CJK} $\lim _{|x| \rightarrow+\infty} f^{\prime}(x)=0$, \begin{CJK}{UTF8}{mj}则方程\end{CJK} $f^{\prime}(x)=0$ \begin{CJK}{UTF8}{mj}在\end{CJK} $(-\infty,+\infty)$ \begin{CJK}{UTF8}{mj}至少有一个实根\end{CJK};

\end{enumerate}
\section{8. 山东大学 2010 年研究生入学考试试题数学分析}
\begin{CJK}{UTF8}{mj}李扬\end{CJK}

\begin{CJK}{UTF8}{mj}微信公众号\end{CJK}: sxkyliyang

\begin{enumerate}
  \item \begin{CJK}{UTF8}{mj}求极限\end{CJK}
\end{enumerate}
$$
\lim _{\substack{x \rightarrow 0 \\ y \rightarrow 0}}\left(x^{2}+y^{2}\right)^{x^{2} y^{2}}
$$

\begin{enumerate}
  \setcounter{enumi}{2}
  \item \begin{CJK}{UTF8}{mj}已知\end{CJK} $f(x)=\int_{0}^{x}\left[\int_{t}^{x} e^{-s^{2}} \mathrm{~d} s\right] \mathrm{d} t$, \begin{CJK}{UTF8}{mj}求\end{CJK} $f(x)$ \begin{CJK}{UTF8}{mj}及\end{CJK} $f^{\prime}(x)$.

  \item \begin{CJK}{UTF8}{mj}设\end{CJK} $f(x)$ \begin{CJK}{UTF8}{mj}在\end{CJK} $(0,1)$ \begin{CJK}{UTF8}{mj}上可微\end{CJK}, \begin{CJK}{UTF8}{mj}且\end{CJK} $\left|f^{\prime}(x)\right| \leqslant \mu$, \begin{CJK}{UTF8}{mj}问\end{CJK} $F(x)=f(\sin x)$ \begin{CJK}{UTF8}{mj}在\end{CJK} $\left(0, \frac{\pi}{2}\right)$ \begin{CJK}{UTF8}{mj}是否一致连续\end{CJK}?

  \item \begin{CJK}{UTF8}{mj}求\end{CJK} $\int_{0}^{\pi} \frac{1}{1-\sqrt{2} \cos \theta+a} \mathrm{~d} \theta,(a>1)$.

\end{enumerate}
5 . \begin{CJK}{UTF8}{mj}设\end{CJK} $\left\{\begin{array}{l}x=e^{t} \cos t \\ y=e^{t} \sin t\end{array}\right.$, \begin{CJK}{UTF8}{mj}求\end{CJK} $\frac{\mathrm{d}^{2} y}{\mathrm{~d} x^{2}}$.

\begin{enumerate}
  \setcounter{enumi}{6}
  \item \begin{CJK}{UTF8}{mj}求\end{CJK}
\end{enumerate}
$$
\iint_{S} 4 z x \mathrm{~d} y \mathrm{~d} z-2 z y \mathrm{~d} z \mathrm{~d} x+\left(1-z^{2}\right) \mathrm{d} x \mathrm{~d} y,
$$
\begin{CJK}{UTF8}{mj}其中\end{CJK} $S$ \begin{CJK}{UTF8}{mj}为\end{CJK} $z=e^{y}(0 \leqslant y \leqslant a)$ \begin{CJK}{UTF8}{mj}绕\end{CJK} $z$ \begin{CJK}{UTF8}{mj}轴旋转所形成的的旋面的下侧\end{CJK}.

\begin{enumerate}
  \setcounter{enumi}{7}
  \item \begin{CJK}{UTF8}{mj}设\end{CJK} $f(x)$ \begin{CJK}{UTF8}{mj}在\end{CJK} $[a, b]$ \begin{CJK}{UTF8}{mj}连续\end{CJK}, \begin{CJK}{UTF8}{mj}在\end{CJK} $(a, b)$ \begin{CJK}{UTF8}{mj}可导\end{CJK} $(a>0)$. \begin{CJK}{UTF8}{mj}证明\end{CJK} $\exists \varphi \in(a, b)$ \begin{CJK}{UTF8}{mj}使得\end{CJK} $f(b)-f(a)=\varphi f^{\prime}(\varphi) \ln \frac{b}{a}$.

  \item \begin{CJK}{UTF8}{mj}设\end{CJK} $f(x)=\sum_{n=1}^{\infty} \frac{x^{n}}{n^{2}}(0 \leqslant x \leqslant 1)$, \begin{CJK}{UTF8}{mj}证明当\end{CJK} $0<x<1$ \begin{CJK}{UTF8}{mj}时\end{CJK}, $f(x)+f(1-x)+\ln x \ln (1-x)=\frac{\pi^{2}}{6}$.

  \item \begin{CJK}{UTF8}{mj}证明\end{CJK}: $\sum_{n=1}^{\infty} x^{n}(1-x)^{2}$ \begin{CJK}{UTF8}{mj}在\end{CJK} $[0,1]$ \begin{CJK}{UTF8}{mj}上一致收敛\end{CJK}.

  \item \begin{CJK}{UTF8}{mj}设\end{CJK} $f(x)$ \begin{CJK}{UTF8}{mj}在\end{CJK} $[0, b]$ \begin{CJK}{UTF8}{mj}上可积\end{CJK}, \begin{CJK}{UTF8}{mj}且\end{CJK} $\lim _{x \rightarrow+\infty} f(x)=2$. \begin{CJK}{UTF8}{mj}证明\end{CJK}: $\lim _{t \rightarrow 0} t \int_{0}^{t} e^{-t x} f(x) \mathrm{d} x=2$.

  \item \begin{CJK}{UTF8}{mj}证明\end{CJK}: $\int_{0}^{1} \frac{x}{1-x} \mathrm{~d} x=\frac{\pi^{2}}{6}$.

\end{enumerate}
\section{9. 山东大学 2011 年研究生入学考试试题数学分析}
\begin{CJK}{UTF8}{mj}李扬\end{CJK}

\begin{CJK}{UTF8}{mj}微信公众号\end{CJK}: sxkyliyang

\begin{enumerate}
  \item (15 \begin{CJK}{UTF8}{mj}分\end{CJK}) \begin{CJK}{UTF8}{mj}证明\end{CJK}: $\int_{-1}^{1}\left(1-x^{2}\right)^{n} \mathrm{~d} x \geqslant \frac{4}{\sqrt[3]{n}}$.

  \item (15 \begin{CJK}{UTF8}{mj}分\end{CJK}) \begin{CJK}{UTF8}{mj}求被\end{CJK} $x^{2}+y^{2}=a x$ \begin{CJK}{UTF8}{mj}所截面积\end{CJK}.

  \item (15 \begin{CJK}{UTF8}{mj}分\end{CJK}) \begin{CJK}{UTF8}{mj}若\end{CJK} $f(x)$ \begin{CJK}{UTF8}{mj}在\end{CJK} $x_{0}$ \begin{CJK}{UTF8}{mj}处有极大值\end{CJK}, \begin{CJK}{UTF8}{mj}求证\end{CJK}: $\left.\frac{\partial f}{\partial x}\right|_{x=x_{0}}=0$.

  \item (15 \begin{CJK}{UTF8}{mj}分\end{CJK}) \begin{CJK}{UTF8}{mj}证明\end{CJK}: \begin{CJK}{UTF8}{mj}在开区间\end{CJK} $(a, b)$ \begin{CJK}{UTF8}{mj}上\end{CJK}, \begin{CJK}{UTF8}{mj}若\end{CJK} $f(x)$ \begin{CJK}{UTF8}{mj}为凸函数\end{CJK}, \begin{CJK}{UTF8}{mj}则\end{CJK} $f(x)$ \begin{CJK}{UTF8}{mj}连续\end{CJK}.

  \item (20 \begin{CJK}{UTF8}{mj}分\end{CJK}) \begin{CJK}{UTF8}{mj}已知\end{CJK} $F_{1}=1, F_{2}=2, F_{n}=F_{n-1}+F_{n-2}$, \begin{CJK}{UTF8}{mj}证明\end{CJK}: $\sum_{n=1}^{\infty} \frac{1}{F_{n}}$ \begin{CJK}{UTF8}{mj}收敛\end{CJK}.

  \item ( 20 \begin{CJK}{UTF8}{mj}分\end{CJK}) \begin{CJK}{UTF8}{mj}若函数\end{CJK} $f(x)$ \begin{CJK}{UTF8}{mj}在\end{CJK} $[0,1]$ \begin{CJK}{UTF8}{mj}上连续\end{CJK}, \begin{CJK}{UTF8}{mj}满足\end{CJK} $|f(x)| \leqslant 1$, \begin{CJK}{UTF8}{mj}且\end{CJK} $\int_{0}^{1} f(x) \mathrm{d} x=0$.

\end{enumerate}
\begin{CJK}{UTF8}{mj}求证\end{CJK}: $\forall a, b \in[0,1]$ \begin{CJK}{UTF8}{mj}有\end{CJK} $\left|\int_{a}^{b} f(x) \mathrm{d} x\right| \leqslant \frac{1}{2}$.

\begin{enumerate}
  \setcounter{enumi}{7}
  \item (20 \begin{CJK}{UTF8}{mj}分\end{CJK}) \begin{CJK}{UTF8}{mj}证明\end{CJK}: $\mathbb{R}^{m}$ \begin{CJK}{UTF8}{mj}中\end{CJK} $\rho(x, E)$ \begin{CJK}{UTF8}{mj}一致连续\end{CJK}.

  \item (15 \begin{CJK}{UTF8}{mj}分\end{CJK}) \begin{CJK}{UTF8}{mj}证明\end{CJK}: $\sum \frac{1}{P}$ \begin{CJK}{UTF8}{mj}发散\end{CJK}, $P$ \begin{CJK}{UTF8}{mj}为遍历所有质数\end{CJK}.

  \item ( 15 \begin{CJK}{UTF8}{mj}分\end{CJK}) \begin{CJK}{UTF8}{mj}已知\end{CJK} $f(x)$ \begin{CJK}{UTF8}{mj}二次可微且\end{CJK} $f^{\prime \prime}(x)$ \begin{CJK}{UTF8}{mj}有界\end{CJK}, \begin{CJK}{UTF8}{mj}证明\end{CJK}: \begin{CJK}{UTF8}{mj}若\end{CJK} $f(x) \rightarrow 0, x \rightarrow+\infty$ \begin{CJK}{UTF8}{mj}时\end{CJK}, \begin{CJK}{UTF8}{mj}则\end{CJK} $f^{\prime}(x) \rightarrow 0, x \rightarrow+\infty$.

\end{enumerate}
\section{0. 山东大学 2012 年研究生入学考试试题数学分析}
\begin{CJK}{UTF8}{mj}李扬\end{CJK}

\begin{CJK}{UTF8}{mj}微信公众号\end{CJK}: sxkyliyang

\begin{enumerate}
  \item \begin{CJK}{UTF8}{mj}设\end{CJK} $m$ \begin{CJK}{UTF8}{mj}为正整数\end{CJK}, \begin{CJK}{UTF8}{mj}求\end{CJK} $\lim _{n \rightarrow+\infty} \int_{0}^{n^{2}} x^{m} e^{-\sqrt{x}} \mathrm{~d} x$.
\end{enumerate}
2 . \begin{CJK}{UTF8}{mj}设\end{CJK} $u=f(x, y)$, \begin{CJK}{UTF8}{mj}用\end{CJK} $\left\{\begin{array}{l}x=r \cos \theta \\ y=r \sin \theta\end{array}\right.$ \begin{CJK}{UTF8}{mj}对\end{CJK} $\frac{\partial^{2} u}{\partial x^{2}}+\frac{\partial^{2} u}{\partial y^{2}}=0$ \begin{CJK}{UTF8}{mj}重新表示\end{CJK}.

\begin{enumerate}
  \setcounter{enumi}{3}
  \item \begin{CJK}{UTF8}{mj}若\end{CJK} $f(x, y)$ \begin{CJK}{UTF8}{mj}在区间\end{CJK} $[a, b]$ \begin{CJK}{UTF8}{mj}上连续\end{CJK}, \begin{CJK}{UTF8}{mj}且\end{CJK} $\int_{a}^{b} f(x) \mathrm{d} x=\int_{a}^{b} x f(x) \mathrm{d} x=0$, \begin{CJK}{UTF8}{mj}证明在\end{CJK} $(a, b)$ \begin{CJK}{UTF8}{mj}内\end{CJK} $f(x)=0$ \begin{CJK}{UTF8}{mj}至少有两个根\end{CJK}.

  \item \begin{CJK}{UTF8}{mj}求\end{CJK}

\end{enumerate}
$$
\iint_{D}\left(\frac{x}{\sqrt{x^{2}+y^{2}}} \frac{\partial f}{\partial x}+\frac{y}{\sqrt{x^{2}+y^{2}}} \frac{\partial f}{\partial y}\right) \mathrm{d} x \mathrm{~d} y
$$
\begin{CJK}{UTF8}{mj}其中\end{CJK} $D: x^{2}+y^{2} \leqslant 1$ \begin{CJK}{UTF8}{mj}且\end{CJK} $\frac{\partial^{2} f}{\partial x^{2}}+\frac{\partial^{2} f}{\partial y^{2}}=x^{2} y^{2}$.

\begin{enumerate}
  \setcounter{enumi}{5}
  \item \begin{CJK}{UTF8}{mj}设\end{CJK} $f(x)$ \begin{CJK}{UTF8}{mj}在\end{CJK} $[1,+\infty)$ \begin{CJK}{UTF8}{mj}可微\end{CJK}, $f(1)=1,\left(x^{2}+f^{2}(x)\right) f^{\prime}(x)=1$.
\end{enumerate}
(1) \begin{CJK}{UTF8}{mj}证明\end{CJK} $f(x)$ \begin{CJK}{UTF8}{mj}一致连续且\end{CJK} $f(x)<2$;

$(2)$ \begin{CJK}{UTF8}{mj}若\end{CJK} $\int_{1}^{+\infty} f(x) \mathrm{d} x$ \begin{CJK}{UTF8}{mj}收敛\end{CJK}, \begin{CJK}{UTF8}{mj}则\end{CJK} $\lim _{x \rightarrow+\infty} f(x)=0$.

\begin{enumerate}
  \setcounter{enumi}{6}
  \item \begin{CJK}{UTF8}{mj}设\end{CJK} $f(x)$ \begin{CJK}{UTF8}{mj}的一个原函数是\end{CJK} $e^{-x^{2}}$, \begin{CJK}{UTF8}{mj}求\end{CJK} $\int x f^{\prime}(x) \mathrm{d} x$.

  \item \begin{CJK}{UTF8}{mj}设\end{CJK} $f(x)$ \begin{CJK}{UTF8}{mj}单调\end{CJK}, \begin{CJK}{UTF8}{mj}问\end{CJK} $\int_{0}^{2 \pi} f(x) \sin n x \mathrm{~d} x \geqslant 0$ \begin{CJK}{UTF8}{mj}是否恒成立\end{CJK}?

  \item \begin{CJK}{UTF8}{mj}设\end{CJK} $\int_{0}^{a} x^{p} f(x) \mathrm{d} x$ \begin{CJK}{UTF8}{mj}收敛\end{CJK}, $\lim _{x \rightarrow 0^{+}} f(x)=+\infty, f(x)$ \begin{CJK}{UTF8}{mj}在\end{CJK} $(0, a]$ \begin{CJK}{UTF8}{mj}单调减少\end{CJK}, \begin{CJK}{UTF8}{mj}求证\end{CJK}: $\lim _{x \rightarrow 0^{+}} x^{p} f(x)=+\infty$.

  \item \begin{CJK}{UTF8}{mj}设\end{CJK} $f, g$ \begin{CJK}{UTF8}{mj}连续\end{CJK}, \begin{CJK}{UTF8}{mj}且\end{CJK} $\min _{a \leqslant x \leqslant b} f(x)=\min _{a \leqslant x \leqslant b} g(x)$. \begin{CJK}{UTF8}{mj}证明存在\end{CJK} $x_{0} \in[a, b]$ \begin{CJK}{UTF8}{mj}使得\end{CJK} $\lim _{x \rightarrow x_{0}} \frac{f(x)-g(x)}{x-x_{0}}=f^{\prime}\left(x_{0}\right)-g^{\prime}\left(x_{0}\right)$.

\end{enumerate}
\section{1. 山东大学 2013 年研究生入学考试试题数学分析}
\begin{CJK}{UTF8}{mj}李扬\end{CJK}

\begin{CJK}{UTF8}{mj}微信公众号\end{CJK}: sxkyliyang

\begin{enumerate}
  \item (15 \begin{CJK}{UTF8}{mj}分\end{CJK}) \begin{CJK}{UTF8}{mj}设\end{CJK} $\sum_{k=1}^{N} c_{k}=0$, \begin{CJK}{UTF8}{mj}求极限\end{CJK} $\lim _{x \rightarrow+\infty} \sum_{k=1}^{N} c_{k} \sqrt{x+k}$.

  \item ( 15 \begin{CJK}{UTF8}{mj}分\end{CJK}) \begin{CJK}{UTF8}{mj}计算\end{CJK}

\end{enumerate}
$$
\oint_{L} e^{x}(1-\cos y) \mathrm{d} x-e^{x}(y-\sin y) \mathrm{d} y
$$
$L$ \begin{CJK}{UTF8}{mj}为区域\end{CJK} $0<x<\pi, 0<y<\sin x$ \begin{CJK}{UTF8}{mj}的正向围线\end{CJK}.

\begin{enumerate}
  \setcounter{enumi}{3}
  \item ( 15 \begin{CJK}{UTF8}{mj}分\end{CJK}) \begin{CJK}{UTF8}{mj}求\end{CJK} $\iint_{S}\left(\sqrt{x^{2}+y^{2}}\right) \mathrm{d} s, S$ \begin{CJK}{UTF8}{mj}为\end{CJK} $\sqrt{x^{2}+y^{2}} \leqslant z \leqslant 1$ \begin{CJK}{UTF8}{mj}的边界\end{CJK}.

  \item (15 \begin{CJK}{UTF8}{mj}分\end{CJK}) \begin{CJK}{UTF8}{mj}求积分\end{CJK} $\int_{0}^{1} \frac{x^{b}-x^{a}}{\ln x} \mathrm{~d} x,(a, b>0)$.

  \item ( 20 \begin{CJK}{UTF8}{mj}分\end{CJK}) \begin{CJK}{UTF8}{mj}证明函数列\end{CJK} $\left\{f_{n}(x)\right\}$ \begin{CJK}{UTF8}{mj}在\end{CJK} $(a, b)$ \begin{CJK}{UTF8}{mj}内一致收敛于极限函数\end{CJK} $f(x)$ \begin{CJK}{UTF8}{mj}的充分必要条件是\end{CJK}

\end{enumerate}
$$
\lim _{n \rightarrow \infty}\left\{\sup _{x \in(a, b)}\left|r_{n}(x)\right|\right\}=0,
$$
\begin{CJK}{UTF8}{mj}其中\end{CJK} $r_{n}(x)=\left|f(x)-f_{n}(x)\right|$.

\begin{enumerate}
  \setcounter{enumi}{6}
  \item ( 20 \begin{CJK}{UTF8}{mj}分\end{CJK}) \begin{CJK}{UTF8}{mj}判别积分\end{CJK} $\int_{0}^{+\infty} x \cos \left(x^{3}\right) \mathrm{d} x$ \begin{CJK}{UTF8}{mj}的收敛性\end{CJK}.

  \item ( 20 \begin{CJK}{UTF8}{mj}分\end{CJK}) \begin{CJK}{UTF8}{mj}证明\end{CJK} $\varphi(x)=\int_{0}^{+\infty} \frac{x}{2+x^{\alpha}} \mathrm{d} x$ \begin{CJK}{UTF8}{mj}对\end{CJK} $\alpha \in(2,+\infty)$ \begin{CJK}{UTF8}{mj}是连续的\end{CJK}.

  \item (15 \begin{CJK}{UTF8}{mj}分\end{CJK}) $f(x)$ \begin{CJK}{UTF8}{mj}是\end{CJK} $[0,+\infty)$ \begin{CJK}{UTF8}{mj}上的可微函数\end{CJK}, $f(0)=0$, \begin{CJK}{UTF8}{mj}且\end{CJK} $f^{\prime}(x)>f(x), \forall x \in(0,+\infty)$. \begin{CJK}{UTF8}{mj}证明\end{CJK}: $f(x)>0, \forall x>0$.

  \item ( 15 \begin{CJK}{UTF8}{mj}分\end{CJK}) \begin{CJK}{UTF8}{mj}设数列\end{CJK} $\left\{x_{n}\right\}$ \begin{CJK}{UTF8}{mj}满足\end{CJK} $0 \leqslant x_{m+n} \leqslant x_{m}+x_{n},(n, m=1,2, \cdots)$. \begin{CJK}{UTF8}{mj}证明数列\end{CJK} $\left\{\frac{x_{n}}{n}\right\}$ \begin{CJK}{UTF8}{mj}收敛\end{CJK}.

\end{enumerate}
\section{2. 山东大学 2014 年研究生入学考试试题数学分析}
\begin{CJK}{UTF8}{mj}李扬\end{CJK}

\begin{CJK}{UTF8}{mj}微信公众号\end{CJK}: sxkyliyang

\begin{enumerate}
  \item (15 \begin{CJK}{UTF8}{mj}分\end{CJK}) \begin{CJK}{UTF8}{mj}求极限\end{CJK} $\lim _{x \rightarrow 0}\left(\frac{\sin x}{x}\right)^{\frac{1}{1-\cos x}}$.

  \item (15 \begin{CJK}{UTF8}{mj}分\end{CJK}) \begin{CJK}{UTF8}{mj}计算\end{CJK} $\int_{0}^{\pi} \frac{\sin n x}{\sin x} \mathrm{~d} x$.

  \item (10 \begin{CJK}{UTF8}{mj}分\end{CJK}) \begin{CJK}{UTF8}{mj}设函数\end{CJK} $f(x)$ \begin{CJK}{UTF8}{mj}在\end{CJK} $[0,1]$ \begin{CJK}{UTF8}{mj}上连续\end{CJK}, $f(0)=0$, \begin{CJK}{UTF8}{mj}在\end{CJK} $(0,1)$ \begin{CJK}{UTF8}{mj}中\end{CJK}, $\left|f^{\prime}(x)\right| \leqslant f(x)$, \begin{CJK}{UTF8}{mj}证明\end{CJK}: $f(x) \equiv 0$.

  \item (10 \begin{CJK}{UTF8}{mj}分\end{CJK}) \begin{CJK}{UTF8}{mj}求\end{CJK} $f(x)=\arctan x$ \begin{CJK}{UTF8}{mj}在\end{CJK} $x=0$ \begin{CJK}{UTF8}{mj}处的各阶导数\end{CJK}.

  \item ( 15 \begin{CJK}{UTF8}{mj}分\end{CJK}) \begin{CJK}{UTF8}{mj}求\end{CJK} $x>0, y>0, z>0$ \begin{CJK}{UTF8}{mj}时函数\end{CJK} $f(x, y, z)=\ln x+2 \ln y+3 \ln z$ \begin{CJK}{UTF8}{mj}在球面\end{CJK} $x^{2}+y^{2}+z^{2}=6 r^{2}$ \begin{CJK}{UTF8}{mj}上的极大\end{CJK} \begin{CJK}{UTF8}{mj}值\end{CJK}, \begin{CJK}{UTF8}{mj}证明\end{CJK}: $a, b, c$ \begin{CJK}{UTF8}{mj}为实数时\end{CJK}, $a b^{2} c^{3}<108\left(\frac{a+b+c}{6}\right)^{6}$.

  \item (15 \begin{CJK}{UTF8}{mj}分\end{CJK}) \begin{CJK}{UTF8}{mj}设空间区域\end{CJK} $\Omega$ \begin{CJK}{UTF8}{mj}由曲面\end{CJK} $z=a^{2}-x^{2}-y^{2}$ \begin{CJK}{UTF8}{mj}与平面\end{CJK} $z=0$ \begin{CJK}{UTF8}{mj}围成\end{CJK}, \begin{CJK}{UTF8}{mj}其中\end{CJK} $a$ \begin{CJK}{UTF8}{mj}为正常数\end{CJK},\begin{CJK}{UTF8}{mj}记\end{CJK} $\Omega$ \begin{CJK}{UTF8}{mj}表面的外侧为\end{CJK} $S, \Omega$ \begin{CJK}{UTF8}{mj}的体积为\end{CJK} $V$, \begin{CJK}{UTF8}{mj}求证\end{CJK}: $V=\oiint_{S} x^{2} y z^{2} \mathrm{~d} y \mathrm{~d} z-x y^{2} z^{2} \mathrm{~d} z \mathrm{~d} x+z(1+x y z) \mathrm{d} x \mathrm{~d} y$.

  \item ( 20 \begin{CJK}{UTF8}{mj}分\end{CJK} $)$ \begin{CJK}{UTF8}{mj}设二元函数\end{CJK} $f(x, y)=\left\{\begin{array}{ll}\left(x^{2}+y^{2}\right) \cos \frac{1}{\sqrt{x^{2}+y^{2}}}, & x^{2}+y^{2} \neq 0 ; \\ 0 . & x^{2}+y^{2}=0 .\end{array}\right.$, \begin{CJK}{UTF8}{mj}则\end{CJK}:

\end{enumerate}
(1) \begin{CJK}{UTF8}{mj}求\end{CJK} $f_{x}^{\prime}(0,0)$;

(2) \begin{CJK}{UTF8}{mj}证明\end{CJK}: $f_{x}^{\prime}(x, y)$ \begin{CJK}{UTF8}{mj}在\end{CJK} $(0,0)$ \begin{CJK}{UTF8}{mj}不连续\end{CJK};

(3) \begin{CJK}{UTF8}{mj}证明\end{CJK}: $f(x, y)$ \begin{CJK}{UTF8}{mj}在\end{CJK} $(0,0)$ \begin{CJK}{UTF8}{mj}处可微\end{CJK}.

\begin{enumerate}
  \setcounter{enumi}{8}
  \item (15 \begin{CJK}{UTF8}{mj}分\end{CJK}) \begin{CJK}{UTF8}{mj}设\end{CJK} $x \in[0, \pi]$, \begin{CJK}{UTF8}{mj}试求如下级数之和\end{CJK}: $\sum_{n=1}^{\infty} \frac{\sin n x}{n}$.

  \item ( 15 \begin{CJK}{UTF8}{mj}分\end{CJK}) \begin{CJK}{UTF8}{mj}将函数\end{CJK} $f(x)=\left\{\begin{array}{cc}e^{x}, & 0 \leqslant x \leqslant \frac{\pi}{2} ; \\ 0 . & -\frac{\pi}{2} \leqslant x<0 .\end{array}\right.$ \begin{CJK}{UTF8}{mj}在\end{CJK} $\left[-\frac{\pi}{2}, \frac{\pi}{2}\right]$ \begin{CJK}{UTF8}{mj}上展开为\end{CJK} Fourier \begin{CJK}{UTF8}{mj}级数\end{CJK}, \begin{CJK}{UTF8}{mj}并指出\end{CJK} Fourier \begin{CJK}{UTF8}{mj}级数所收敛\end{CJK} \begin{CJK}{UTF8}{mj}的函数\end{CJK}.

  \item ( 20 \begin{CJK}{UTF8}{mj}分\end{CJK}) \begin{CJK}{UTF8}{mj}级数\end{CJK} $\sum_{n=1}^{\infty} a_{n}$ \begin{CJK}{UTF8}{mj}收敛\end{CJK}, $\sum_{n=1}^{\infty}\left(b_{n+1}-b_{n}\right)$ \begin{CJK}{UTF8}{mj}绝对收敛\end{CJK}, \begin{CJK}{UTF8}{mj}试证\end{CJK}: \begin{CJK}{UTF8}{mj}级数\end{CJK} $\sum_{n=1}^{\infty} a_{n} b_{n}$ \begin{CJK}{UTF8}{mj}也收敛\end{CJK}.

\end{enumerate}
\section{3. 山东大学 2015 年研究生入学考试试题数学分析 
 李扬 
 微信公众号: sxkyliyang}
\begin{enumerate}
  \item (10 \begin{CJK}{UTF8}{mj}分\end{CJK}) \begin{CJK}{UTF8}{mj}设函数\end{CJK} $f(x)>0$, \begin{CJK}{UTF8}{mj}在区间\end{CJK} $[0,1]$ \begin{CJK}{UTF8}{mj}上连续\end{CJK}, \begin{CJK}{UTF8}{mj}试证\end{CJK}: $\lim _{n \rightarrow \infty} \sqrt[n]{\sum_{i=1}^{n}\left(f\left(\frac{i}{n}\right)\right)^{n} \frac{1}{n}}=\max _{0 \leqslant x \leqslant 1} f(x)$.

  \item (10 \begin{CJK}{UTF8}{mj}分\end{CJK}) \begin{CJK}{UTF8}{mj}设函数\end{CJK} $f(x)=(x-a)^{n} \varphi(x)$, \begin{CJK}{UTF8}{mj}其中函数\end{CJK} $\varphi(x)$ \begin{CJK}{UTF8}{mj}于点\end{CJK} $a$ \begin{CJK}{UTF8}{mj}的邻域内有\end{CJK} $n-1$ \begin{CJK}{UTF8}{mj}阶的连续导数\end{CJK}, \begin{CJK}{UTF8}{mj}求\end{CJK} $f^{(n)}(a)$.

  \item ( 15 \begin{CJK}{UTF8}{mj}分\end{CJK}) \begin{CJK}{UTF8}{mj}证明对于任意的函数列\end{CJK} $f_{1}(x), f_{2}(x), \cdots, f_{n}(x), \cdots\left(x_{0}<x<+\infty\right)$, \begin{CJK}{UTF8}{mj}可举出一函数\end{CJK} $f(x)$, \begin{CJK}{UTF8}{mj}当\end{CJK} $x \rightarrow+\infty$ \begin{CJK}{UTF8}{mj}时\end{CJK}, \begin{CJK}{UTF8}{mj}它比函数\end{CJK} $f_{n}(x)(n=1,2, \cdots)$ \begin{CJK}{UTF8}{mj}中的每一个都增加得较快\end{CJK}.

  \item (15 \begin{CJK}{UTF8}{mj}分\end{CJK}) \begin{CJK}{UTF8}{mj}设函数\end{CJK} $f(x)$ \begin{CJK}{UTF8}{mj}在区间\end{CJK} $[0,1]$ \begin{CJK}{UTF8}{mj}上连续\end{CJK}, \begin{CJK}{UTF8}{mj}证明\end{CJK}: $\lim _{h \rightarrow 0^{+}} \int_{0}^{1} \frac{h}{h^{2}+x^{2}} f(x) \mathrm{d} x=\frac{\pi}{2} f(0)$.

  \item (15 \begin{CJK}{UTF8}{mj}分\end{CJK}) \begin{CJK}{UTF8}{mj}计算曲面\end{CJK} $\left(x^{2}+y^{2}+z^{2}\right)^{2}=a^{2}\left(x^{2}+y^{2}-z^{2}\right)(a>0)$ \begin{CJK}{UTF8}{mj}所围之体积\end{CJK}.

  \item ( 15 \begin{CJK}{UTF8}{mj}分\end{CJK}) \begin{CJK}{UTF8}{mj}计算线积分\end{CJK} $\int_{L^{+}} x \mathrm{~d} y-y \mathrm{~d} x$, \begin{CJK}{UTF8}{mj}其中\end{CJK} $L^{+}$\begin{CJK}{UTF8}{mj}为上半球面\end{CJK} $x^{2}+y^{2}+z^{2}=1(z \geqslant 0)$ \begin{CJK}{UTF8}{mj}与柱面\end{CJK} $x^{2}+y^{2}=x$ \begin{CJK}{UTF8}{mj}的交\end{CJK} \begin{CJK}{UTF8}{mj}线\end{CJK}, \begin{CJK}{UTF8}{mj}从\end{CJK} $z$ \begin{CJK}{UTF8}{mj}轴正向往下看\end{CJK}, $L$ \begin{CJK}{UTF8}{mj}正向取逆时针方向\end{CJK}.

  \item (20 \begin{CJK}{UTF8}{mj}分\end{CJK}) \begin{CJK}{UTF8}{mj}若函数\end{CJK} $f(x, y)$ \begin{CJK}{UTF8}{mj}在某区域\end{CJK} $G$ \begin{CJK}{UTF8}{mj}内对变量\end{CJK} $x$ \begin{CJK}{UTF8}{mj}是连续的\end{CJK}, \begin{CJK}{UTF8}{mj}而关于\end{CJK} $x$ \begin{CJK}{UTF8}{mj}对变量\end{CJK} $y$ \begin{CJK}{UTF8}{mj}是一致连续的\end{CJK}, \begin{CJK}{UTF8}{mj}则此函数在\end{CJK} $G$ \begin{CJK}{UTF8}{mj}内是连续的\end{CJK}.

  \item (15 \begin{CJK}{UTF8}{mj}分\end{CJK}) \begin{CJK}{UTF8}{mj}证明\end{CJK}: $\int_{0}^{1}\left|x \sin \frac{1}{x^{2}}-\frac{1}{x} \cos \frac{1}{x^{2}}\right| \mathrm{d} x$ \begin{CJK}{UTF8}{mj}发散\end{CJK}.

  \item ( 15 \begin{CJK}{UTF8}{mj}分\end{CJK} $)$ \begin{CJK}{UTF8}{mj}已知\end{CJK} $g(0)=\frac{\sqrt{\pi}}{2}$, \begin{CJK}{UTF8}{mj}计算\end{CJK} $g(\alpha)=\int_{0}^{+\infty} e^{-x^{2}} \cos 2 \alpha x \mathrm{~d} x$.

  \item ( 20 \begin{CJK}{UTF8}{mj}分\end{CJK}) \begin{CJK}{UTF8}{mj}证明\end{CJK}: \begin{CJK}{UTF8}{mj}级数\end{CJK} $\sum_{n=1}^{\infty}(1-x) \frac{x^{n}}{1-x^{2 n}} \sin n x$ \begin{CJK}{UTF8}{mj}在\end{CJK} $\left(\frac{1}{2}, 1\right)$ \begin{CJK}{UTF8}{mj}内一致收敛\end{CJK}.

\end{enumerate}
\section{4. 山东大学 2016 年研究生入学考试试题数学分析 
 李扬 
 微信公众号: sxkyliyang}
\begin{enumerate}
  \item (10 \begin{CJK}{UTF8}{mj}分\end{CJK}) \begin{CJK}{UTF8}{mj}设\end{CJK}
\end{enumerate}
$$
S_{n}=\frac{\sum_{k=0}^{n} \ln C_{n}^{k}}{n^{2}} \text { 其中 }\left(C_{n}^{k}=\frac{n(n-1) \cdots(n-k+1)}{1 \cdot 2 \cdots k}\right) \text {, }
$$
\begin{CJK}{UTF8}{mj}求\end{CJK} $\lim _{n \rightarrow \infty} S_{n}$.

\begin{enumerate}
  \setcounter{enumi}{2}
  \item (10 \begin{CJK}{UTF8}{mj}分\end{CJK}) \begin{CJK}{UTF8}{mj}计算积分\end{CJK} $\int_{0}^{\pi} \frac{\sin x}{1+\cos ^{2} x} \mathrm{~d} x$.

  \item (10 \begin{CJK}{UTF8}{mj}分\end{CJK}) \begin{CJK}{UTF8}{mj}设\end{CJK} $f(x)$ \begin{CJK}{UTF8}{mj}与\end{CJK} $g(x)$ \begin{CJK}{UTF8}{mj}为似序的\end{CJK}, \begin{CJK}{UTF8}{mj}即对\end{CJK} $\forall x, y$ \begin{CJK}{UTF8}{mj}都成立\end{CJK} $(f(x)-f(y))(g(x)-g(y)) \geqslant 0$, \begin{CJK}{UTF8}{mj}证明\end{CJK}:

\end{enumerate}
$$
\int_{a}^{b} f(x) \mathrm{d} x \int_{a}^{b} g(x) \mathrm{d} x \leqslant(b-a) \int_{a}^{b} f(x) g(x) \mathrm{d} x
$$

\begin{enumerate}
  \setcounter{enumi}{4}
  \item (10 \begin{CJK}{UTF8}{mj}分\end{CJK}) \begin{CJK}{UTF8}{mj}设\end{CJK} $f(x+h)=f(x)+h f^{\prime}(x)+\cdots+\frac{h^{n}}{n !} f^{(n)}(x+\theta h)(0<\theta<1)$, \begin{CJK}{UTF8}{mj}且\end{CJK} $f^{(n+1)}(x) \neq 0$. \begin{CJK}{UTF8}{mj}试证\end{CJK}: $\lim _{n \rightarrow 0} \theta=\frac{1}{n+1}$.

  \item ( 10 \begin{CJK}{UTF8}{mj}分\end{CJK}) \begin{CJK}{UTF8}{mj}设\end{CJK} $f(0)=0, f(x)$ \begin{CJK}{UTF8}{mj}在\end{CJK} $[0,+\infty)$ \begin{CJK}{UTF8}{mj}为非负的严格凸函数\end{CJK}, $F(x)=\frac{f(x)}{x}(x>0)$. \begin{CJK}{UTF8}{mj}试证\end{CJK}: $f(x), F(x)$ \begin{CJK}{UTF8}{mj}为严\end{CJK} \begin{CJK}{UTF8}{mj}格递增的\end{CJK}.

  \item (15 \begin{CJK}{UTF8}{mj}分\end{CJK}) \begin{CJK}{UTF8}{mj}求\end{CJK} $I=\int_{0}^{\pi}\left(\frac{\sin \theta}{1+\cos \theta}\right)^{\alpha-1} \frac{\mathrm{d} \theta}{1+k \cos \theta}(0<k<1)$.

  \item ( 15 \begin{CJK}{UTF8}{mj}分\end{CJK}) \begin{CJK}{UTF8}{mj}试作一个函数\end{CJK} $f(x, y)$, \begin{CJK}{UTF8}{mj}使得当\end{CJK} $x \rightarrow+\infty, y \rightarrow+\infty$ \begin{CJK}{UTF8}{mj}时\end{CJK}, \begin{CJK}{UTF8}{mj}成立\end{CJK}:

\end{enumerate}
(1) \begin{CJK}{UTF8}{mj}两个累次极限存在而重极限不存在\end{CJK};

(2) \begin{CJK}{UTF8}{mj}两个累次极限不存在而重极限存在\end{CJK};

(3) \begin{CJK}{UTF8}{mj}两个累次极限与重极限都不存在\end{CJK}.

\begin{enumerate}
  \setcounter{enumi}{8}
  \item (20 \begin{CJK}{UTF8}{mj}分\end{CJK}) \begin{CJK}{UTF8}{mj}试证\end{CJK}:
\end{enumerate}
$$
\left|\iint_{S} f(m x+n y+p z) \mathrm{d} s\right| \leqslant 4 \pi M,
$$
\begin{CJK}{UTF8}{mj}其中\end{CJK} $m^{2}+n^{2}+p^{2}=1, m, n, p$ \begin{CJK}{UTF8}{mj}为常数\end{CJK}, $f(t)$ \begin{CJK}{UTF8}{mj}在\end{CJK} $|t| \leqslant 1$ \begin{CJK}{UTF8}{mj}时为连续可微函数\end{CJK}, $f(-1)=f(1)=0$, $M=\max _{-1 \leqslant t \leqslant 1}\left\{\left|f^{\prime}(t)\right|\right\}, S: x^{2}+y^{2}+z^{2}=1 .$

\begin{enumerate}
  \setcounter{enumi}{9}
  \item ( 15 \begin{CJK}{UTF8}{mj}分\end{CJK}) \begin{CJK}{UTF8}{mj}设\end{CJK} $0<x_{1}<\pi, x_{n}=\sin x_{n-1}(n=1,2, \cdots)$, \begin{CJK}{UTF8}{mj}证明\end{CJK}:
\end{enumerate}
(1) $\lim _{n \rightarrow \infty} x_{n}=0$;

( 2 ) \begin{CJK}{UTF8}{mj}级数\end{CJK} $\sum_{n=0}^{\infty} x_{n}^{p}$, \begin{CJK}{UTF8}{mj}当\end{CJK} $p>2$ \begin{CJK}{UTF8}{mj}时收敛\end{CJK}, \begin{CJK}{UTF8}{mj}当\end{CJK} $p \leqslant 2$ \begin{CJK}{UTF8}{mj}时发散\end{CJK}.

\begin{enumerate}
  \setcounter{enumi}{10}
  \item (15 \begin{CJK}{UTF8}{mj}分\end{CJK}) \begin{CJK}{UTF8}{mj}设\end{CJK} $f(x)=\sum_{n=1}^{\infty}(-1)^{n+1} \frac{e^{-n x}}{n}$, \begin{CJK}{UTF8}{mj}求\end{CJK}:
\end{enumerate}
(1) $f$ \begin{CJK}{UTF8}{mj}的连续范围\end{CJK};

(2) $f$ \begin{CJK}{UTF8}{mj}的可导范围\end{CJK}.

\begin{enumerate}
  \setcounter{enumi}{11}
  \item (20 \begin{CJK}{UTF8}{mj}分\end{CJK}) \begin{CJK}{UTF8}{mj}设\end{CJK} $f(x)$ \begin{CJK}{UTF8}{mj}是周期为\end{CJK} $2 \pi$ \begin{CJK}{UTF8}{mj}的函数\end{CJK}, \begin{CJK}{UTF8}{mj}且\end{CJK} $f(x)=x,-\pi<x<\pi$, \begin{CJK}{UTF8}{mj}求\end{CJK} $f(x)$ \begin{CJK}{UTF8}{mj}与\end{CJK} $|f(x)|$ \begin{CJK}{UTF8}{mj}的\end{CJK} Fourier \begin{CJK}{UTF8}{mj}展开式\end{CJK}, \begin{CJK}{UTF8}{mj}它们的\end{CJK} Fourier \begin{CJK}{UTF8}{mj}级数是否一致收敛\end{CJK} (\begin{CJK}{UTF8}{mj}给出证明\end{CJK})?
\end{enumerate}
\section{5. 山东大学 2017 年研究生入学考试试题数学分析}
\begin{CJK}{UTF8}{mj}李扬\end{CJK}

\begin{CJK}{UTF8}{mj}微信公众号\end{CJK}: sxkyliyang

\begin{enumerate}
  \item (15 \begin{CJK}{UTF8}{mj}分\end{CJK}) \begin{CJK}{UTF8}{mj}求极限\end{CJK} $\lim _{n \rightarrow \infty} \prod_{k=1}^{n}\left(1+\frac{k}{n^{2}}\right)$.

  \item ( 15 \begin{CJK}{UTF8}{mj}分\end{CJK}) \begin{CJK}{UTF8}{mj}证明函数\end{CJK} $f(x)=\frac{|\sin x|}{x}$ \begin{CJK}{UTF8}{mj}在区间\end{CJK} $(-1,0)$ \begin{CJK}{UTF8}{mj}上是一致收敛的\end{CJK}, \begin{CJK}{UTF8}{mj}但在区间\end{CJK} $J=\{x|0<| x \mid<1\}$ \begin{CJK}{UTF8}{mj}上并非一致\end{CJK} \begin{CJK}{UTF8}{mj}连续\end{CJK}.

  \item (10 \begin{CJK}{UTF8}{mj}分\end{CJK}) \begin{CJK}{UTF8}{mj}证明不连续的函数\end{CJK} $f(x)=\operatorname{sgn}\left(\sin \frac{\pi}{x}\right)$ \begin{CJK}{UTF8}{mj}于区间\end{CJK} $[0,1]$ \begin{CJK}{UTF8}{mj}上可积分\end{CJK}.

  \item ( 10 \begin{CJK}{UTF8}{mj}分\end{CJK}) \begin{CJK}{UTF8}{mj}证明函数\end{CJK} $f(x)=x^{2}\left|\cos \frac{\pi}{x}\right|(x \neq 0)$ \begin{CJK}{UTF8}{mj}及\end{CJK} $f(0)=0$ \begin{CJK}{UTF8}{mj}在点\end{CJK} $x=0$ \begin{CJK}{UTF8}{mj}的任何邻域上有不可微分的点\end{CJK}, \begin{CJK}{UTF8}{mj}但在\end{CJK} $x=0$ \begin{CJK}{UTF8}{mj}这点是可微分的\end{CJK}.

  \item ( 15 \begin{CJK}{UTF8}{mj}分\end{CJK}) \begin{CJK}{UTF8}{mj}证明不等式\end{CJK} $y x^{y}(1-x)<e^{-1}$, \begin{CJK}{UTF8}{mj}其中\end{CJK} $(x, y) \in D=\{(x, y) \mid 0<x<1, y>0\}$.

  \item (15 \begin{CJK}{UTF8}{mj}分\end{CJK}) \begin{CJK}{UTF8}{mj}计算\end{CJK}

\end{enumerate}
$$
\int_{L} y^{2} \mathrm{~d} x+z^{2} \mathrm{~d} y+x^{2} \mathrm{~d} z
$$
\begin{CJK}{UTF8}{mj}其中\end{CJK} $L$ \begin{CJK}{UTF8}{mj}是曲线\end{CJK} $\left\{\begin{array}{l}x^{2}+y^{2}+z^{2}=a^{2} \\ x^{2}+y^{2}=a x\end{array} \quad(z \geqslant 0, a>0)\right.$, \begin{CJK}{UTF8}{mj}若从\end{CJK} $x$ \begin{CJK}{UTF8}{mj}轴正向看去\end{CJK}, \begin{CJK}{UTF8}{mj}此曲线是沿逆时针方向运行的\end{CJK}.

\begin{enumerate}
  \setcounter{enumi}{7}
  \item ( 20 \begin{CJK}{UTF8}{mj}分\end{CJK}) \begin{CJK}{UTF8}{mj}设函数\end{CJK} $f(x, y)=\varphi(|x y|)$, \begin{CJK}{UTF8}{mj}其中\end{CJK} $\varphi(0)=0$, \begin{CJK}{UTF8}{mj}且在点\end{CJK} $u=0$ \begin{CJK}{UTF8}{mj}的近旁满足\end{CJK} $|\varphi(u)| \leqslant u^{2}$, \begin{CJK}{UTF8}{mj}试证\end{CJK}: \begin{CJK}{UTF8}{mj}函数\end{CJK} $f(x, y)$ \begin{CJK}{UTF8}{mj}在点\end{CJK} $(0,0)$ \begin{CJK}{UTF8}{mj}处可微\end{CJK}.

  \item (15 \begin{CJK}{UTF8}{mj}分\end{CJK}) \begin{CJK}{UTF8}{mj}计算积分\end{CJK}

\end{enumerate}
$$
I(a)=\int_{0}^{\frac{\pi}{2}}\left(\ln \frac{1+a \cos x}{1-a \cos x}\right) \frac{\mathrm{d} x}{\cos x}
$$
\begin{CJK}{UTF8}{mj}其中\end{CJK} $(|a|<1)$.

\begin{enumerate}
  \setcounter{enumi}{9}
  \item ( 15 \begin{CJK}{UTF8}{mj}分\end{CJK}) \begin{CJK}{UTF8}{mj}设\end{CJK} $f(x)$ \begin{CJK}{UTF8}{mj}对一切\end{CJK} $b(0<b<+\infty)$ \begin{CJK}{UTF8}{mj}在\end{CJK} $[0, b]$ \begin{CJK}{UTF8}{mj}上可积\end{CJK}, \begin{CJK}{UTF8}{mj}且\end{CJK} $\lim _{x \rightarrow+\infty} f(x)=\alpha$, \begin{CJK}{UTF8}{mj}证明\end{CJK}:
\end{enumerate}
$$
\lim _{t \rightarrow 0^{+}} t \int_{0}^{+\infty} e^{-t x} f(x) \mathrm{d} x=\alpha .
$$

\begin{enumerate}
  \setcounter{enumi}{10}
  \item ( 20 \begin{CJK}{UTF8}{mj}分\end{CJK}) \begin{CJK}{UTF8}{mj}设\end{CJK} $x \in[0, \pi]$, \begin{CJK}{UTF8}{mj}试求如下级数之和\end{CJK} $\sum_{n=1}^{\infty} \frac{\sin n x}{n}$.
\end{enumerate}
\section{1. 上海大学 2009 年研究生入学考试试题数学分析}
\begin{CJK}{UTF8}{mj}李扬\end{CJK}

\begin{CJK}{UTF8}{mj}微信公众号\end{CJK}: sxkyliyang

\begin{enumerate}
  \item $\lim _{n \rightarrow \infty} a_{n}=0$, \begin{CJK}{UTF8}{mj}求证\end{CJK}
\end{enumerate}
$$
\lim _{n \rightarrow \infty} \frac{a_{1}+2 a_{2}+\cdots+n a_{n}}{n^{2}}=0
$$

\begin{enumerate}
  \setcounter{enumi}{2}
  \item \begin{CJK}{UTF8}{mj}叙述一致连续定义\end{CJK}. \begin{CJK}{UTF8}{mj}问\end{CJK}
\end{enumerate}
$$
g(x)=\cos ^{2} x+\cos x^{2}
$$
\begin{CJK}{UTF8}{mj}是否是周期函数\end{CJK}? \begin{CJK}{UTF8}{mj}证之\end{CJK}.

\begin{enumerate}
  \setcounter{enumi}{3}
  \item $f(x)$ \begin{CJK}{UTF8}{mj}在\end{CJK} $[1,+\infty)$ \begin{CJK}{UTF8}{mj}可导\end{CJK}, $f(1)=1$ \begin{CJK}{UTF8}{mj}且\end{CJK}
\end{enumerate}
$$
f^{\prime}(x)=\frac{1}{x^{2}+f^{2}(x)}
$$
\begin{CJK}{UTF8}{mj}证\end{CJK} $\lim _{x \rightarrow+\infty} f(x)$ \begin{CJK}{UTF8}{mj}存在且极限小于\end{CJK} $1+\frac{\pi}{4}$.

\begin{enumerate}
  \setcounter{enumi}{4}
  \item \begin{CJK}{UTF8}{mj}计算\end{CJK}
\end{enumerate}
$$
I=\int_{0}^{\frac{1}{2}} \frac{\sin x}{x} \mathrm{~d} x
$$
\begin{CJK}{UTF8}{mj}误差\end{CJK} $<0.0005$.

\begin{enumerate}
  \setcounter{enumi}{5}
  \item $f(x) \in \mathrm{C}(0,+\infty), f(1)=3$, \begin{CJK}{UTF8}{mj}当\end{CJK} $x, y>0$,
\end{enumerate}
$$
\int_{1}^{x y} f(t) \mathrm{d} t=x \int_{1}^{y} f(t) \mathrm{d} t+y \int_{1}^{x} f(t) \mathrm{d} t
$$
\begin{CJK}{UTF8}{mj}求\end{CJK} $f(x)$.

\begin{enumerate}
  \setcounter{enumi}{6}
  \item $f(x)$ \begin{CJK}{UTF8}{mj}在\end{CJK} $[a, b]$ \begin{CJK}{UTF8}{mj}可积\end{CJK}. $\int_{a}^{b} f(x) \mathrm{d} x \neq 0$, \begin{CJK}{UTF8}{mj}是否存在\end{CJK} $[\alpha, \beta] \subseteq[a, b]$, \begin{CJK}{UTF8}{mj}使\end{CJK} $[\alpha, \beta]$ \begin{CJK}{UTF8}{mj}上\end{CJK} $f(x)$ \begin{CJK}{UTF8}{mj}为恒正或者恒负\end{CJK}. \begin{CJK}{UTF8}{mj}证之\end{CJK}.

  \item $\left\{x_{n}\right\}$ \begin{CJK}{UTF8}{mj}在\end{CJK} $\lim _{n \rightarrow+\infty} x_{n}=0$ \begin{CJK}{UTF8}{mj}的条件下\end{CJK}, \begin{CJK}{UTF8}{mj}试问\end{CJK}

\end{enumerate}
$$
\sum_{n=1}^{\infty}(-1)^{n} x_{n}
$$
\begin{CJK}{UTF8}{mj}收敛吗\end{CJK}? \begin{CJK}{UTF8}{mj}证之\end{CJK}.

\begin{enumerate}
  \setcounter{enumi}{8}
  \item $f(x)$ \begin{CJK}{UTF8}{mj}在\end{CJK} $[1,+\infty)$ \begin{CJK}{UTF8}{mj}单减连续可微\end{CJK}, $\lim _{x \rightarrow+\infty} f(x)=0$, \begin{CJK}{UTF8}{mj}证明\end{CJK}: \begin{CJK}{UTF8}{mj}当\end{CJK}
\end{enumerate}
$$
\int_{1}^{+\infty} x f(x) \mathrm{d} x
$$
\begin{CJK}{UTF8}{mj}收敛\end{CJK}, \begin{CJK}{UTF8}{mj}则\end{CJK} $\lim _{x \rightarrow+\infty} x f(x)=0$.

\begin{enumerate}
  \setcounter{enumi}{9}
  \item \begin{CJK}{UTF8}{mj}证明\end{CJK}: $f_{n}(x)=x^{n}, n=1,2, \cdots$ \begin{CJK}{UTF8}{mj}在\end{CJK} $[0,1)$ \begin{CJK}{UTF8}{mj}非一致收敛\end{CJK}, \begin{CJK}{UTF8}{mj}但\end{CJK} $g_{n}(x)=x^{n} S(x), n=1,2, \cdots$ \begin{CJK}{UTF8}{mj}在\end{CJK} $[0,1)$ \begin{CJK}{UTF8}{mj}上一致收敛\end{CJK}, \begin{CJK}{UTF8}{mj}其中\end{CJK} $S(x)$ \begin{CJK}{UTF8}{mj}在\end{CJK} $[0,1)$ \begin{CJK}{UTF8}{mj}上连续且\end{CJK} $S(1)=0$.

  \item $f(x) \in \mathrm{C}[0,1]$, \begin{CJK}{UTF8}{mj}证明\end{CJK}:

\end{enumerate}
$$
\lim _{n \rightarrow+\infty}(n+1) \int_{0}^{1} x^{n} f(x) \mathrm{d} x=f(1) .
$$

\begin{enumerate}
  \setcounter{enumi}{11}
  \item $a<b, S=\left\{\varphi(x): \varphi(x)\right.$ \begin{CJK}{UTF8}{mj}在\end{CJK} $[a, b]$ \begin{CJK}{UTF8}{mj}上可积且\end{CJK} $\left.\int_{a}^{b} \varphi(x) \mathrm{d} x=0\right\}, f(x)$ \begin{CJK}{UTF8}{mj}在\end{CJK} $[a, b]$ \begin{CJK}{UTF8}{mj}上连续\end{CJK}, \begin{CJK}{UTF8}{mj}若\end{CJK}
\end{enumerate}
$$
\int_{a}^{b} f(x) \varphi(x) \mathrm{d} x=0
$$
\begin{CJK}{UTF8}{mj}对任意\end{CJK} $\varphi(x) \in S$ \begin{CJK}{UTF8}{mj}成立\end{CJK}, \begin{CJK}{UTF8}{mj}试证\end{CJK} $f(x)$ \begin{CJK}{UTF8}{mj}恒为常数\end{CJK}. 12. \begin{CJK}{UTF8}{mj}在\end{CJK} $\sqrt{x}+\sqrt{y}+\sqrt{z}=\sqrt{a}(x>0, y>0, z>0, a>0)$ \begin{CJK}{UTF8}{mj}上任取一点做切平面\end{CJK}, \begin{CJK}{UTF8}{mj}求该切平面截三坐标轴所得三线\end{CJK} \begin{CJK}{UTF8}{mj}段长度之和\end{CJK}.

\begin{enumerate}
  \setcounter{enumi}{13}
  \item \begin{CJK}{UTF8}{mj}中心在原点的\end{CJK} $A x^{2}+B y^{2}+C z^{2}+2 D x y+2 E y z+2 F x z=1$ \begin{CJK}{UTF8}{mj}的长半轴\end{CJK} $l$ \begin{CJK}{UTF8}{mj}是下行列式的最大实根\end{CJK}
\end{enumerate}
$$
\left|\begin{array}{ccc}
A-\frac{1}{l^{2}} & D & F \\
D & B-\frac{1}{l^{2}} & E \\
F & E & C-\frac{1}{l^{2}}
\end{array}\right| .
$$

\begin{enumerate}
  \setcounter{enumi}{14}
  \item $L$ \begin{CJK}{UTF8}{mj}是从\end{CJK} $A(-1,0)$ \begin{CJK}{UTF8}{mj}经过\end{CJK} $x^{2}+y^{2}=1(y \geq 0)$ \begin{CJK}{UTF8}{mj}到\end{CJK} $B(1,0)$ \begin{CJK}{UTF8}{mj}的线段\end{CJK}, \begin{CJK}{UTF8}{mj}求\end{CJK}:
\end{enumerate}
$$
I=\int \frac{x-y}{x^{2}+a y^{2}} \mathrm{~d} x+\frac{x+a y}{x^{2}+a y^{2}} \mathrm{~d} y, a>0 .
$$

\begin{enumerate}
  \setcounter{enumi}{15}
  \item \begin{CJK}{UTF8}{mj}将\end{CJK} $f(x)=(x-1)^{2}$ \begin{CJK}{UTF8}{mj}在\end{CJK} $[0,1)$ \begin{CJK}{UTF8}{mj}上展开成余弦级数\end{CJK}, \begin{CJK}{UTF8}{mj}并证明\end{CJK}
\end{enumerate}
$$
\frac{\pi^{2}}{6}=1+\frac{1}{2^{2}}+\frac{1}{3^{2}}+\cdots+\frac{1}{n^{2}}+\cdots
$$

\section{2. 上海大学 2010 年研究生入学考试试题数学分析}
\begin{CJK}{UTF8}{mj}李扬\end{CJK}

\begin{CJK}{UTF8}{mj}微信公众号\end{CJK}: sxkyliyang

\begin{enumerate}
  \item \begin{CJK}{UTF8}{mj}计算极限\end{CJK}
\end{enumerate}
(1)
$$
I_{1}=\lim _{n \rightarrow \infty}\left(\frac{1}{n+\sqrt{1}}+\frac{1}{n+\sqrt{2}}+\cdots+\frac{1}{n+\sqrt{n}}\right)
$$
$(2)$
$$
I_{2}=\lim _{x \rightarrow 0}(1+x-\sin x)^{\frac{1}{x^{3}}}
$$

\begin{enumerate}
  \setcounter{enumi}{2}
  \item \begin{CJK}{UTF8}{mj}假设数列\end{CJK} $\left\{x_{n}\right\}$ \begin{CJK}{UTF8}{mj}是一个无界数列\end{CJK}, \begin{CJK}{UTF8}{mj}而且数列\end{CJK} $\left\{\frac{1}{x_{n}}\right\}$ \begin{CJK}{UTF8}{mj}不收敛\end{CJK}, \begin{CJK}{UTF8}{mj}证明存在\end{CJK} $\left\{x_{n}\right\}$ \begin{CJK}{UTF8}{mj}的两个子列\end{CJK} $\left\{x_{n_{k}}^{(1)}\right\}$ \begin{CJK}{UTF8}{mj}和\end{CJK} $\left\{x_{n_{k}}^{(2)}\right\}$ \begin{CJK}{UTF8}{mj}使得\end{CJK} $\left\{x_{n_{k}}^{(1)}\right\}$ \begin{CJK}{UTF8}{mj}收敛\end{CJK}, \begin{CJK}{UTF8}{mj}而\end{CJK} $\left\{x_{n_{k}}^{(2)}\right\}$ \begin{CJK}{UTF8}{mj}是无穷大量\end{CJK}.

  \item \begin{CJK}{UTF8}{mj}如果\end{CJK}

\end{enumerate}
$$
\lim _{x \rightarrow \infty}\left(\sqrt[3]{\delta x^{3}+3 x^{2}+2010 x \sin x+4}-k x-b\right)=0
$$
(\begin{CJK}{UTF8}{mj}其中\end{CJK} $\delta$ \begin{CJK}{UTF8}{mj}为常数\end{CJK}). \begin{CJK}{UTF8}{mj}请计算出这里的常数\end{CJK} $k, b$. (\begin{CJK}{UTF8}{mj}用\end{CJK} $\delta$ \begin{CJK}{UTF8}{mj}表示\end{CJK})

\begin{enumerate}
  \setcounter{enumi}{4}
  \item \begin{CJK}{UTF8}{mj}假设\end{CJK} $f(x)$ \begin{CJK}{UTF8}{mj}是闭区间\end{CJK} $[a, b]$ \begin{CJK}{UTF8}{mj}上连续函数\end{CJK}, $f(a)=f(b)$, \begin{CJK}{UTF8}{mj}在端点处有单侧导数\end{CJK}, \begin{CJK}{UTF8}{mj}且\end{CJK} $f^{\prime}(a) \cdot f^{\prime}(b)>0$. \begin{CJK}{UTF8}{mj}证明\end{CJK}: \begin{CJK}{UTF8}{mj}存在一\end{CJK} \begin{CJK}{UTF8}{mj}点\end{CJK} $\zeta \in(a, b)$, \begin{CJK}{UTF8}{mj}使得\end{CJK}
\end{enumerate}
$$
f(a)=f(\zeta)
$$

\begin{enumerate}
  \setcounter{enumi}{5}
  \item \begin{CJK}{UTF8}{mj}假设\end{CJK} $f(x)$ \begin{CJK}{UTF8}{mj}在区间\end{CJK} $[0,1]$ \begin{CJK}{UTF8}{mj}内可导\end{CJK} (\begin{CJK}{UTF8}{mj}端点处存在单侧导数\end{CJK})
\end{enumerate}
(1) \begin{CJK}{UTF8}{mj}问导函数\end{CJK} $f^{\prime}(x)$ \begin{CJK}{UTF8}{mj}是否在区间\end{CJK} $[0,1]$ \begin{CJK}{UTF8}{mj}内连续\end{CJK}? \begin{CJK}{UTF8}{mj}若连续请证明\end{CJK}, \begin{CJK}{UTF8}{mj}不连续举出反例\end{CJK}.

(2) \begin{CJK}{UTF8}{mj}进一步如果\end{CJK} $f^{\prime}(x) \geq 0$, \begin{CJK}{UTF8}{mj}问\end{CJK} $f(x)$ \begin{CJK}{UTF8}{mj}是否在区间\end{CJK} $[0,1]$ \begin{CJK}{UTF8}{mj}内严格单调递增\end{CJK}? \begin{CJK}{UTF8}{mj}证明你的结论\end{CJK}.

\begin{enumerate}
  \setcounter{enumi}{6}
  \item \begin{CJK}{UTF8}{mj}假设\end{CJK} $f(x)$ \begin{CJK}{UTF8}{mj}是区间内\end{CJK} $(0,+\infty)$ \begin{CJK}{UTF8}{mj}连续\end{CJK}, \begin{CJK}{UTF8}{mj}而且\end{CJK}
\end{enumerate}
$$
f(x)=\ln x-\int_{1}^{\mathrm{e}^{2}} f(x) \mathrm{d} x
$$
\begin{CJK}{UTF8}{mj}请计算\end{CJK} $\int_{1}^{\mathrm{e}^{2}} f(x) \mathrm{d} x$ \begin{CJK}{UTF8}{mj}以及\end{CJK} $f(x)$.

\begin{enumerate}
  \setcounter{enumi}{7}
  \item \begin{CJK}{UTF8}{mj}假设\end{CJK} $f(x)$ \begin{CJK}{UTF8}{mj}在原点附近无穷阶可导\end{CJK}, \begin{CJK}{UTF8}{mj}而且\end{CJK} $f(0)=0, \lim _{x \rightarrow 0} \frac{f(x)}{x}=0$, \begin{CJK}{UTF8}{mj}证明幂级数\end{CJK} $\sum_{n=1}^{\infty} a_{n} x^{n}$ \begin{CJK}{UTF8}{mj}的收敛半径\end{CJK} $R \geq 1$. (\begin{CJK}{UTF8}{mj}题目不全\end{CJK})

  \item \begin{CJK}{UTF8}{mj}假设\end{CJK} $f(x)$ \begin{CJK}{UTF8}{mj}是\end{CJK} $[1,+\infty)$ \begin{CJK}{UTF8}{mj}上的连续函数\end{CJK}, \begin{CJK}{UTF8}{mj}而且\end{CJK} $\lim _{x \rightarrow \infty} f(x)=0$

\end{enumerate}
(1) \begin{CJK}{UTF8}{mj}如果\end{CJK}
$$
\int_{1}^{+\infty} f^{2}(x) \mathrm{d} x
$$
\begin{CJK}{UTF8}{mj}收敛\end{CJK}, \begin{CJK}{UTF8}{mj}问\end{CJK} $\int_{1}^{+\infty} f(x) \mathrm{d} x$ \begin{CJK}{UTF8}{mj}是否也收敛\end{CJK}?

(2) \begin{CJK}{UTF8}{mj}反过来如果\end{CJK}
$$
\int_{1}^{+\infty} f(x) \mathrm{d} x
$$
\begin{CJK}{UTF8}{mj}收敛\end{CJK}, \begin{CJK}{UTF8}{mj}问\end{CJK} $\int_{1}^{+\infty} f^{2}(x) \mathrm{d} x$ \begin{CJK}{UTF8}{mj}是否还是收敛的\end{CJK}? \begin{CJK}{UTF8}{mj}如果结论成立则必须证明\end{CJK}; \begin{CJK}{UTF8}{mj}如果结论不成立请举出反例\end{CJK}. 9. \begin{CJK}{UTF8}{mj}请写出函数列\end{CJK} $\left\{f_{n}(x)\right\}$ \begin{CJK}{UTF8}{mj}在区间\end{CJK} $I$ \begin{CJK}{UTF8}{mj}内一致有界的定义\end{CJK}, \begin{CJK}{UTF8}{mj}其次对于\end{CJK} $[0,1]$ \begin{CJK}{UTF8}{mj}的连续函数列\end{CJK} $\left\{f_{n}(x)\right\}$, \begin{CJK}{UTF8}{mj}如果一致收敛到\end{CJK} $f(x)$, \begin{CJK}{UTF8}{mj}证明这个函数列\end{CJK} $\left\{f_{n}(x)\right\}$ \begin{CJK}{UTF8}{mj}在\end{CJK} $[0,1]$ \begin{CJK}{UTF8}{mj}上是一致有界的\end{CJK}.

\begin{enumerate}
  \setcounter{enumi}{10}
  \item \begin{CJK}{UTF8}{mj}假设函数\end{CJK}
\end{enumerate}
$$
f(x, y)= \begin{cases}x y \sin \frac{1}{\sqrt{x^{2}+y^{2}}}, & x^{2}+y^{2} \neq 0 \\ 0, & x^{2}+y^{2}=0\end{cases}
$$
\begin{CJK}{UTF8}{mj}试通过详细过程讨论\end{CJK}:

(1) \begin{CJK}{UTF8}{mj}函数\end{CJK} $f(x, y)$ \begin{CJK}{UTF8}{mj}在原点\end{CJK} $(0,0)$ \begin{CJK}{UTF8}{mj}连续吗\end{CJK}?

(2) \begin{CJK}{UTF8}{mj}函数\end{CJK} $f(x, y)$ \begin{CJK}{UTF8}{mj}在原点\end{CJK} $(0,0)$ \begin{CJK}{UTF8}{mj}的偏导数\end{CJK} $f_{x}(0,0), f_{y}(0,0)$ \begin{CJK}{UTF8}{mj}是多少\end{CJK}?

(3) \begin{CJK}{UTF8}{mj}函数\end{CJK} $f(x, y)$ \begin{CJK}{UTF8}{mj}在原点\end{CJK} $(0,0)$ \begin{CJK}{UTF8}{mj}可微吗\end{CJK}?

\begin{enumerate}
  \setcounter{enumi}{11}
  \item \begin{CJK}{UTF8}{mj}记曲面\end{CJK}
\end{enumerate}
$$
S: x^{2}-3 y^{2}-z=1
$$
(1) \begin{CJK}{UTF8}{mj}详细验证是否有\end{CJK}: $p_{0}(1,2,-12) \in S$

(2) \begin{CJK}{UTF8}{mj}问过点\end{CJK} $p_{0}$ \begin{CJK}{UTF8}{mj}处是否有平面与曲面\end{CJK} $S$ \begin{CJK}{UTF8}{mj}相切\end{CJK}? \begin{CJK}{UTF8}{mj}若有请求出该切平面\end{CJK} $\pi$ \begin{CJK}{UTF8}{mj}和过\end{CJK} $p_{0}$ \begin{CJK}{UTF8}{mj}点\end{CJK} $\pi$ \begin{CJK}{UTF8}{mj}的法线方程\end{CJK}; \begin{CJK}{UTF8}{mj}若没有也请证\end{CJK} \begin{CJK}{UTF8}{mj}明\end{CJK}.

\begin{enumerate}
  \setcounter{enumi}{12}
  \item \begin{CJK}{UTF8}{mj}讨论广义积分\end{CJK}
\end{enumerate}
$$
I(y)=\int_{0}^{+\infty} \mathrm{e}^{-x y} \frac{\sin x}{x} \mathrm{~d} x
$$
\begin{CJK}{UTF8}{mj}的收敛性\end{CJK}, \begin{CJK}{UTF8}{mj}并由此计算\end{CJK} Dirichlet \begin{CJK}{UTF8}{mj}积分\end{CJK}
$$
I(0)=\int_{0}^{+\infty} \frac{\sin x}{x} \mathrm{~d} x .
$$

\begin{enumerate}
  \setcounter{enumi}{13}
  \item \begin{CJK}{UTF8}{mj}计算曲线积分\end{CJK}
\end{enumerate}
$$
I=\oint_{L} \frac{x \mathrm{~d} y-y \mathrm{~d} x}{x^{2}+y^{2}}
$$
\begin{CJK}{UTF8}{mj}其中曲线\end{CJK} $L:|x|+|y|=1$ \begin{CJK}{UTF8}{mj}沿着逆时针方向\end{CJK}.

\begin{enumerate}
  \setcounter{enumi}{14}
  \item \begin{CJK}{UTF8}{mj}求立体\end{CJK}
\end{enumerate}
$$
V:\left(x^{2}+y^{2}+z^{2}+8\right)^{2} \leq 36\left(x^{2}+y^{2}\right)
$$
\begin{CJK}{UTF8}{mj}的体积\end{CJK}

\begin{enumerate}
  \setcounter{enumi}{15}
  \item \begin{CJK}{UTF8}{mj}假设\end{CJK} $f(x)$ \begin{CJK}{UTF8}{mj}是以\end{CJK} $2 \pi$ \begin{CJK}{UTF8}{mj}为周期的函数\end{CJK}, \begin{CJK}{UTF8}{mj}且具有二阶连续的导数\end{CJK}, \begin{CJK}{UTF8}{mj}请证明\end{CJK} $f(x)$ \begin{CJK}{UTF8}{mj}的\end{CJK} Fourier \begin{CJK}{UTF8}{mj}级数在整个实数轴\end{CJK} $\mathbb{R}$ \begin{CJK}{UTF8}{mj}上一\end{CJK} \begin{CJK}{UTF8}{mj}致收敛于\end{CJK} $f(x)$ \begin{CJK}{UTF8}{mj}本身\end{CJK}.
\end{enumerate}
\section{3. 上海大学 2011 年研究生入学考试试题数学分析}
\begin{CJK}{UTF8}{mj}李扬\end{CJK}

\begin{CJK}{UTF8}{mj}微信公众号\end{CJK}: sxkyliyang

\begin{enumerate}
  \item \begin{CJK}{UTF8}{mj}设\end{CJK} $\lim _{n \rightarrow \infty} a_{n}=a$, \begin{CJK}{UTF8}{mj}证明\end{CJK}:\\
(1) $\lim _{n \rightarrow \infty} \frac{\left[n a_{n}\right]}{n}=a$;\\
(2) \begin{CJK}{UTF8}{mj}若\end{CJK} $a>0, a_{n}>0$, \begin{CJK}{UTF8}{mj}则\end{CJK} $\lim _{n \rightarrow \infty} \sqrt[n]{a_{n}}=1$.

  \item \begin{CJK}{UTF8}{mj}设\end{CJK} $f$ \begin{CJK}{UTF8}{mj}在\end{CJK} $[a, b]$ \begin{CJK}{UTF8}{mj}上连续\end{CJK}, $x_{1}, x_{2}, \cdots, x_{n} \in[a, b]$, \begin{CJK}{UTF8}{mj}证明\end{CJK}: \begin{CJK}{UTF8}{mj}存在一点\end{CJK} $\xi \in[a, b]$, \begin{CJK}{UTF8}{mj}使得\end{CJK}:

\end{enumerate}
$$
f(\xi)=\frac{1}{n}\left[f\left(x_{1}\right)+f\left(x_{2}\right)+\cdots+f\left(x_{n}\right)\right]
$$

\begin{enumerate}
  \setcounter{enumi}{3}
  \item \begin{CJK}{UTF8}{mj}设\end{CJK}
\end{enumerate}
$$
f(x, y)= \begin{cases}\frac{x y}{x^{2}+y^{2}}, & (x, y) \neq 0 \\ 0, & (x, y)=0\end{cases}
$$
\begin{CJK}{UTF8}{mj}问\end{CJK} $f(x, y)$ \begin{CJK}{UTF8}{mj}在\end{CJK} $(0,0)$ \begin{CJK}{UTF8}{mj}点连续吗\end{CJK}? \begin{CJK}{UTF8}{mj}并计算\end{CJK} $f_{x}(0,0), f_{y}(0,0)$.

\begin{enumerate}
  \setcounter{enumi}{4}
  \item \begin{CJK}{UTF8}{mj}证明\end{CJK}: \begin{CJK}{UTF8}{mj}函数\end{CJK}
\end{enumerate}
$$
f(x)=\frac{1}{x}
$$
\begin{CJK}{UTF8}{mj}在区间\end{CJK} $(0,1)$ \begin{CJK}{UTF8}{mj}上非一致连续\end{CJK}.

\begin{enumerate}
  \setcounter{enumi}{5}
  \item \begin{CJK}{UTF8}{mj}求由方程\end{CJK}
\end{enumerate}
$$
\mathrm{e}^{x+y}-x y-\mathrm{e}=0
$$
\begin{CJK}{UTF8}{mj}所确定的隐函数曲线在点\end{CJK} $(0,1)$ \begin{CJK}{UTF8}{mj}处的切线方程\end{CJK}.

\begin{enumerate}
  \setcounter{enumi}{6}
  \item \begin{CJK}{UTF8}{mj}判断正项级数\end{CJK}
\end{enumerate}
$$
\sum_{n=1}^{\infty}\left(\mathrm{e}^{\frac{1}{n^{2}}}-\cos \frac{\pi}{n}\right)
$$
\begin{CJK}{UTF8}{mj}的敛散性\end{CJK}.

\begin{enumerate}
  \setcounter{enumi}{7}
  \item \begin{CJK}{UTF8}{mj}假设广义积分\end{CJK}
\end{enumerate}
$$
\int_{0}^{+\infty} x f^{\prime}(x) \mathrm{d} x
$$
\begin{CJK}{UTF8}{mj}收敛\end{CJK}, \begin{CJK}{UTF8}{mj}且当\end{CJK} $x \rightarrow+\infty$ \begin{CJK}{UTF8}{mj}时\end{CJK}, $f(x)$ \begin{CJK}{UTF8}{mj}递减趋于\end{CJK} 0 , \begin{CJK}{UTF8}{mj}证明\end{CJK} $\lim _{x \rightarrow+\infty} x f(x)=0$.

\begin{enumerate}
  \setcounter{enumi}{8}
  \item \begin{CJK}{UTF8}{mj}把函数\end{CJK}
\end{enumerate}
$$
f(x)=\frac{\pi}{2}-x
$$
\begin{CJK}{UTF8}{mj}在\end{CJK} $[0, \pi]$ \begin{CJK}{UTF8}{mj}上展开成余弦级数\end{CJK}.

\begin{enumerate}
  \setcounter{enumi}{9}
  \item \begin{CJK}{UTF8}{mj}求曲线\end{CJK}
\end{enumerate}
$$
y=\frac{(x-1)^{2}}{3(x+1)}
$$
\begin{CJK}{UTF8}{mj}的渐近线方程\end{CJK}.

\begin{enumerate}
  \setcounter{enumi}{10}
  \item \begin{CJK}{UTF8}{mj}考察函数\end{CJK}
\end{enumerate}
$$
f(x, y)= \begin{cases}x y \sin \frac{1}{x^{2}+y^{2}}, & x^{2}+y^{2} \neq 0 \\ 0, & x^{2}+y^{2}=0\end{cases}
$$
\begin{CJK}{UTF8}{mj}在点\end{CJK} $(0,0)$ \begin{CJK}{UTF8}{mj}的可微性\end{CJK}. 11. \begin{CJK}{UTF8}{mj}叙述格林公式\end{CJK}, \begin{CJK}{UTF8}{mj}并计算\end{CJK}
$$
\int_{L} \frac{x \mathrm{~d} y-y \mathrm{~d} x}{x^{2}+y^{2}} .
$$
\begin{CJK}{UTF8}{mj}其中\end{CJK} $L$ \begin{CJK}{UTF8}{mj}为一条不经过原点的简单闭曲线\end{CJK}, \begin{CJK}{UTF8}{mj}方向为逆时针方向\end{CJK}.

\begin{enumerate}
  \setcounter{enumi}{12}
  \item \begin{CJK}{UTF8}{mj}设函数\end{CJK} $f(x)$ \begin{CJK}{UTF8}{mj}在\end{CJK} $[0,1]$ \begin{CJK}{UTF8}{mj}上连续\end{CJK}, \begin{CJK}{UTF8}{mj}在\end{CJK} $(0,1)$ \begin{CJK}{UTF8}{mj}上可导\end{CJK}, \begin{CJK}{UTF8}{mj}且\end{CJK} $f(0)=0, f(1)=1$. \begin{CJK}{UTF8}{mj}证明在\end{CJK} $(0,1)$ \begin{CJK}{UTF8}{mj}内存在不同的\end{CJK} $\lambda, \mu$ \begin{CJK}{UTF8}{mj}使得\end{CJK}
\end{enumerate}
$$
f^{\prime}(\lambda)\left(f^{\prime}(\mu)+1\right)=2
$$

\begin{enumerate}
  \setcounter{enumi}{13}
  \item \begin{CJK}{UTF8}{mj}计算曲面\end{CJK} $x^{2}+y^{2}+z^{2}=a^{2}$ \begin{CJK}{UTF8}{mj}包含在曲面\end{CJK}
\end{enumerate}
$$
\frac{x^{2}}{a^{2}}+\frac{y^{2}}{b^{2}}=1(0<b \leq a)
$$
\begin{CJK}{UTF8}{mj}内部的面积\end{CJK}.

\begin{enumerate}
  \setcounter{enumi}{14}
  \item \begin{CJK}{UTF8}{mj}证明\end{CJK}: \begin{CJK}{UTF8}{mj}当\end{CJK} $|x| \leq 1,|y| \leq 1$ \begin{CJK}{UTF8}{mj}时\end{CJK}, \begin{CJK}{UTF8}{mj}有不等式\end{CJK}
\end{enumerate}
$$
\left(x^{2}-y^{2}\right)^{2} \leq 2+y^{2}-x^{2}
$$

\begin{enumerate}
  \setcounter{enumi}{15}
  \item \begin{CJK}{UTF8}{mj}讨论级数\end{CJK}
\end{enumerate}
$$
\sum_{n=1}^{\infty}(-1)^{n} \frac{x^{2 n+1}}{2 n+1}
$$
\begin{CJK}{UTF8}{mj}在\end{CJK} $D=(-1,1)$ \begin{CJK}{UTF8}{mj}上的一致收敛性\end{CJK}.

\section{4. 上海大学 2012 年研究生入学考试试题数学分析 
 李扬 
 微信公众号: sxkyliyang}
(\begin{CJK}{UTF8}{mj}这份真题准确性不高\end{CJK}, \begin{CJK}{UTF8}{mj}题目有错的\end{CJK}, \begin{CJK}{UTF8}{mj}会做就做\end{CJK}, \begin{CJK}{UTF8}{mj}不会做的话问问老师\end{CJK}, \begin{CJK}{UTF8}{mj}是不是有错的\end{CJK}.)

\begin{enumerate}
  \item $\lim _{n \rightarrow \infty} a_{n}=a$, \begin{CJK}{UTF8}{mj}求证\end{CJK}
\end{enumerate}
$$
\lim _{n \rightarrow \infty} \frac{a_{1}+2^{k} a_{2}+\cdots+n^{k} a_{n}}{n^{k+1}}=\frac{a}{k+1} .
$$

\begin{enumerate}
  \setcounter{enumi}{2}
  \item $f(x) \in \mathrm{C}(0,1]$, \begin{CJK}{UTF8}{mj}有\end{CJK} $f(x)$ \begin{CJK}{UTF8}{mj}在\end{CJK} $(0,1]$ \begin{CJK}{UTF8}{mj}可导且\end{CJK}
\end{enumerate}
$$
x^{\frac{2}{3}} f(x)
$$
\begin{CJK}{UTF8}{mj}有界\end{CJK}, \begin{CJK}{UTF8}{mj}对\end{CJK} $\forall x \in(0,1]$, \begin{CJK}{UTF8}{mj}求证\end{CJK} $f(x)$ \begin{CJK}{UTF8}{mj}在\end{CJK} $(0,1]$ \begin{CJK}{UTF8}{mj}上一致连续\end{CJK}.

\begin{enumerate}
  \setcounter{enumi}{3}
  \item $f(x) \in[0,+\infty)$, \begin{CJK}{UTF8}{mj}对\end{CJK} $\forall x \in[0,+\infty), \lim _{x \rightarrow+\infty} f(x+a)=a$. \begin{CJK}{UTF8}{mj}求证\end{CJK}:
\end{enumerate}
$$
\lim _{x \rightarrow+\infty} f(x)=0 .
$$
$4 .$
$$
y=\frac{x^{2}}{4}-\frac{\ln x}{2}
$$
$1 \leq x \leq \mathrm{e}$, \begin{CJK}{UTF8}{mj}求该曲线弧长\end{CJK}.

\begin{enumerate}
  \setcounter{enumi}{5}
  \item $f(x) \in[0,1)$,
\end{enumerate}
$$
f(x)=\arctan x+\int_{0}^{1} f(x) \mathrm{d} x .
$$
\begin{CJK}{UTF8}{mj}求\end{CJK} $f(x)$.

\begin{enumerate}
  \setcounter{enumi}{6}
  \item \begin{CJK}{UTF8}{mj}求\end{CJK}
\end{enumerate}
$$
\sum_{n=0}^{\infty} \frac{(-1)^{n}}{2 n+1}
$$
\begin{CJK}{UTF8}{mj}的和\end{CJK}.

\begin{enumerate}
  \setcounter{enumi}{7}
  \item $\sum a_{n}^{2} \cdot b_{n}$ \begin{CJK}{UTF8}{mj}收敛\end{CJK}, \begin{CJK}{UTF8}{mj}证\end{CJK}
\end{enumerate}
$$
\sum a_{n} \cdot b_{n}
$$
\begin{CJK}{UTF8}{mj}收敛\end{CJK}.

\begin{enumerate}
  \setcounter{enumi}{8}
  \item $a_{n}>0, n \geq 1, \sum a_{n}^{2}$ \begin{CJK}{UTF8}{mj}发散\end{CJK}, \begin{CJK}{UTF8}{mj}是否存在\end{CJK} $\left\{b_{n}\right\}$, \begin{CJK}{UTF8}{mj}使得\end{CJK} $\left\{b_{n}^{2}\right\}$ \begin{CJK}{UTF8}{mj}收敛\end{CJK}, \begin{CJK}{UTF8}{mj}而\end{CJK} $\left\{\sqrt{a_{n} b_{n}}\right\}$ \begin{CJK}{UTF8}{mj}发散\end{CJK}.

  \item $f(x)$ \begin{CJK}{UTF8}{mj}是\end{CJK} $[1,+\infty)$ \begin{CJK}{UTF8}{mj}上连续可微的函数\end{CJK}, $\int_{0}^{1}\left|f^{\prime}(x)\right| \mathrm{d} x$ \begin{CJK}{UTF8}{mj}收敛\end{CJK}, \begin{CJK}{UTF8}{mj}如果\end{CJK} $\int_{1}^{+\infty} f(x) \mathrm{d} x$ \begin{CJK}{UTF8}{mj}收敛\end{CJK}, \begin{CJK}{UTF8}{mj}求证\end{CJK}:

\end{enumerate}
$$
\sum_{n=1}^{\infty} f(n)
$$
\begin{CJK}{UTF8}{mj}收敛\end{CJK}.

\begin{enumerate}
  \setcounter{enumi}{10}
  \item \begin{CJK}{UTF8}{mj}幂级数\end{CJK}
\end{enumerate}
$$
\sum_{n=0}^{\infty} a_{n} x^{n}
$$
\begin{CJK}{UTF8}{mj}在\end{CJK} $\mathbb{R}$ \begin{CJK}{UTF8}{mj}上一致收敛\end{CJK}, \begin{CJK}{UTF8}{mj}问其和函数的零点集是否有限\end{CJK}? $11 .$
$$
\int_{-\frac{\pi}{2}}^{\frac{\pi}{2}} f(x) \mathrm{d} x=\int_{-\frac{\pi}{2}}^{\frac{\pi}{2}} f(x) \sin x \mathrm{~d} x=0
$$
\begin{CJK}{UTF8}{mj}求证\end{CJK} $f(x)$ \begin{CJK}{UTF8}{mj}在\end{CJK} $\left[-\frac{\pi}{2}, \frac{\pi}{2}\right]$ \begin{CJK}{UTF8}{mj}上有两个根\end{CJK}.

\begin{enumerate}
  \setcounter{enumi}{12}
  \item $f(x)$ \begin{CJK}{UTF8}{mj}在\end{CJK} $[0,1]$ \begin{CJK}{UTF8}{mj}上连续且\end{CJK} $f(x)>0$, \begin{CJK}{UTF8}{mj}求\end{CJK}
\end{enumerate}
$$
I(y)=\int_{0}^{1} \frac{y f(x)}{x^{2}+y^{2}} \mathrm{~d} x
$$
\begin{CJK}{UTF8}{mj}的连续性\end{CJK}.

\begin{enumerate}
  \setcounter{enumi}{13}
  \item $f(x, y, z)$ \begin{CJK}{UTF8}{mj}有连续的偏导数\end{CJK}, \begin{CJK}{UTF8}{mj}求证\end{CJK}: \begin{CJK}{UTF8}{mj}存在\end{CJK} $\mathbb{R}$ \begin{CJK}{UTF8}{mj}上连续函数\end{CJK} $x(t), y(t), z(t)$ \begin{CJK}{UTF8}{mj}满足\end{CJK}:
\end{enumerate}
$$
\forall t_{1} \neq t_{2},\left(x\left(t_{1}\right), y\left(t_{1}\right), z\left(t_{1}\right)\right) \neq\left(x\left(t_{2}\right), y\left(t_{2}\right), z\left(t_{2}\right)\right)
$$
\begin{CJK}{UTF8}{mj}使得\end{CJK} $f(x(t), y(t), z(t)) \equiv$ \begin{CJK}{UTF8}{mj}常数\end{CJK}.

\begin{enumerate}
  \setcounter{enumi}{14}
  \item $S: x^{2}+y^{2}+z^{2}=1$ \begin{CJK}{UTF8}{mj}方向外侧\end{CJK}, \begin{CJK}{UTF8}{mj}求\end{CJK}
\end{enumerate}
$$
\iint_{S} \frac{x \mathrm{~d} y \mathrm{~d} z+y \mathrm{~d} x \mathrm{~d} z+z \mathrm{~d} x \mathrm{~d} y}{\left(x^{2}+2 y^{2}+3 z^{2}\right)^{\frac{3}{2}}}
$$

\begin{enumerate}
  \setcounter{enumi}{15}
  \item \begin{CJK}{UTF8}{mj}求\end{CJK} $f(x)=(x-1)^{2}$ \begin{CJK}{UTF8}{mj}在\end{CJK} $[0,1)$ \begin{CJK}{UTF8}{mj}上展开余项\end{CJK}, \begin{CJK}{UTF8}{mj}求证\end{CJK}:
\end{enumerate}
$$
\frac{\pi^{2}}{8}=\sum_{n=1}^{\infty} \frac{1}{(2 n+1)^{2}}
$$

\section{5. 上海大学 2013 年研究生入学考试试题数学分析}
\begin{CJK}{UTF8}{mj}李扬\end{CJK}

\begin{CJK}{UTF8}{mj}微信公众号\end{CJK}: sxkyliyang

\begin{enumerate}
  \item \begin{CJK}{UTF8}{mj}设函数\end{CJK}
\end{enumerate}
$$
f(x)=\left\{\begin{array}{ll}
\frac{x}{2}+x^{2} \sin \frac{1}{x}, & x \neq 0 \\
0, & x=0
\end{array} .\right.
$$
\begin{CJK}{UTF8}{mj}求\end{CJK} $f^{\prime}(x)$ \begin{CJK}{UTF8}{mj}并讨论一下\end{CJK} $f^{\prime \prime}(0)$ \begin{CJK}{UTF8}{mj}的存在性\end{CJK}.

\begin{enumerate}
  \setcounter{enumi}{2}
  \item \begin{CJK}{UTF8}{mj}计算\end{CJK}:
\end{enumerate}
$$
I=\int \frac{1}{\sqrt{1+\mathrm{e}^{2 x}}} \mathrm{~d} x
$$

\begin{enumerate}
  \setcounter{enumi}{3}
  \item \begin{CJK}{UTF8}{mj}计算\end{CJK}:
\end{enumerate}
$$
\sum_{n=1}^{\infty} \frac{n}{(n+1) !}
$$
(\begin{CJK}{UTF8}{mj}复旦版\end{CJK} 106 \begin{CJK}{UTF8}{mj}页原题\end{CJK})

\begin{enumerate}
  \setcounter{enumi}{4}
  \item \begin{CJK}{UTF8}{mj}求\end{CJK}
\end{enumerate}
$$
f(x, y)=x^{2}+y^{2}
$$
\begin{CJK}{UTF8}{mj}在\end{CJK} $x+y=1$ \begin{CJK}{UTF8}{mj}的条件极值\end{CJK}.

\begin{enumerate}
  \setcounter{enumi}{5}
  \item \begin{CJK}{UTF8}{mj}已知公式\end{CJK} $F=\frac{k m_{1} m_{2}}{r^{2}}$ \begin{CJK}{UTF8}{mj}为两个物体之间吸引力大小的计算公式\end{CJK}, \begin{CJK}{UTF8}{mj}现在有一水平圆形导线\end{CJK}
\end{enumerate}
$$
L: x^{2}+y^{2}=R^{2} .
$$
\begin{CJK}{UTF8}{mj}圆心为\end{CJK} $O$ \begin{CJK}{UTF8}{mj}点\end{CJK}, \begin{CJK}{UTF8}{mj}半径为\end{CJK} $R$, \begin{CJK}{UTF8}{mj}导线的密度函数为\end{CJK} $\rho(x, y)=|x| \sqrt{x^{2}+y^{2}}$, \begin{CJK}{UTF8}{mj}在\end{CJK} $A(0,0, R)$ \begin{CJK}{UTF8}{mj}处有一质量为\end{CJK} 1 \begin{CJK}{UTF8}{mj}的质点\end{CJK}, \begin{CJK}{UTF8}{mj}求\end{CJK} \begin{CJK}{UTF8}{mj}导线\end{CJK} $L$ \begin{CJK}{UTF8}{mj}对质点\end{CJK} $A$ \begin{CJK}{UTF8}{mj}的引力\end{CJK}.

\begin{enumerate}
  \setcounter{enumi}{6}
  \item \begin{CJK}{UTF8}{mj}求\end{CJK}:
\end{enumerate}
$$
\iint_{S} y z \mathrm{~d} z \mathrm{~d} x .
$$
$S$ \begin{CJK}{UTF8}{mj}为球面\end{CJK} $x^{2}+y^{2}+z^{2}=1$ \begin{CJK}{UTF8}{mj}的上半部分\end{CJK}, \begin{CJK}{UTF8}{mj}取外侧为正方向\end{CJK}.

\begin{enumerate}
  \setcounter{enumi}{7}
  \item (1) \begin{CJK}{UTF8}{mj}试计算\end{CJK}
\end{enumerate}
$$
2 x^{2}+3 y^{2}+z^{2}=9
$$
\begin{CJK}{UTF8}{mj}在点\end{CJK} $A(1,-1,2)$ \begin{CJK}{UTF8}{mj}处的法向量\end{CJK} $\overrightarrow{n_{1}}$.

(2) \begin{CJK}{UTF8}{mj}试计算\end{CJK}
$$
3 x^{2}+y^{2}-z^{2}=0
$$
\begin{CJK}{UTF8}{mj}在点\end{CJK} $A(1,-1,2)$ \begin{CJK}{UTF8}{mj}处的法向量\end{CJK} $\overrightarrow{n_{2}}$.

(3) \begin{CJK}{UTF8}{mj}试计算曲线\end{CJK}
$$
\left\{\begin{array}{l}
2 x^{2}+3 y^{2}+z^{2}=9 \\
3 x^{2}+y^{2}-z^{2}=0
\end{array}\right.
$$
\begin{CJK}{UTF8}{mj}在点\end{CJK} $A(1,-1,2)$ \begin{CJK}{UTF8}{mj}处的切向量\end{CJK}.

\begin{enumerate}
  \setcounter{enumi}{8}
  \item \begin{CJK}{UTF8}{mj}设\end{CJK} $A, B$ \begin{CJK}{UTF8}{mj}都是非空有界数集\end{CJK}, \begin{CJK}{UTF8}{mj}定义数集\end{CJK} $A+B=\{z \mid z=x+y, z \in A, y \in B\}$, \begin{CJK}{UTF8}{mj}证明\end{CJK}:
\end{enumerate}
$$
\sup (A+B)=\sup A+\sup B
$$

\begin{enumerate}
  \setcounter{enumi}{9}
  \item \begin{CJK}{UTF8}{mj}证明\end{CJK} $y=x^{2}$ \begin{CJK}{UTF8}{mj}在\end{CJK} $[1,+\infty)$ \begin{CJK}{UTF8}{mj}上不一致收敛\end{CJK}.

  \item \begin{CJK}{UTF8}{mj}已知\end{CJK} $f(x)$ \begin{CJK}{UTF8}{mj}在区间\end{CJK} $[a, b]$ \begin{CJK}{UTF8}{mj}内有连续导函数\end{CJK}, \begin{CJK}{UTF8}{mj}定义二元函数\end{CJK}

\end{enumerate}
$$
F(x, y)=\left\{\begin{array}{ll}
\frac{f(x)-f(y)}{x-y}, & x \neq y \\
f^{\prime}(x), & x=y
\end{array} .\right.
$$
\begin{CJK}{UTF8}{mj}证明\end{CJK} $F(x, y)$ \begin{CJK}{UTF8}{mj}在区域\end{CJK} $D=(a, b) \times(a, b)$ \begin{CJK}{UTF8}{mj}内连续\end{CJK}.

\begin{enumerate}
  \setcounter{enumi}{11}
  \item \begin{CJK}{UTF8}{mj}二元函数\end{CJK} $f(x, y)$ \begin{CJK}{UTF8}{mj}在闭区域\end{CJK} $D=[a, b] \times[c, d]$ \begin{CJK}{UTF8}{mj}上连续\end{CJK}, \begin{CJK}{UTF8}{mj}用有限覆盖定理证明\end{CJK} $f(x, y)$ \begin{CJK}{UTF8}{mj}在\end{CJK} $D$ \begin{CJK}{UTF8}{mj}上有界\end{CJK}.

  \item $f(x, y)$ \begin{CJK}{UTF8}{mj}为连续函数且\end{CJK} $f(x, y)=f(y, x)$, \begin{CJK}{UTF8}{mj}证明\end{CJK}

\end{enumerate}
$$
\int_{0}^{1} \mathrm{~d} x \int_{0}^{x} f(x, y) \mathrm{d} y=\int_{0}^{1} \mathrm{~d} x \int_{0}^{x} f(1-x, 1-y) \mathrm{d} y
$$

\section{6. 上海大学 2014 年研究生入学考试试题数学分析}
\begin{CJK}{UTF8}{mj}李扬\end{CJK}

\begin{CJK}{UTF8}{mj}微信公众号\end{CJK}: sxkyliyang

\begin{enumerate}
  \item \begin{CJK}{UTF8}{mj}叙述\end{CJK} $\lim _{x \rightarrow x_{0}} f(x)=A$ \begin{CJK}{UTF8}{mj}的\end{CJK} $\varepsilon-\delta$ \begin{CJK}{UTF8}{mj}定义\end{CJK}, \begin{CJK}{UTF8}{mj}并用该定义证明\end{CJK}
\end{enumerate}
$$
\lim _{x \rightarrow 1} \frac{x+1}{2 x^{3}-2 x+1}=2
$$
2 . \begin{CJK}{UTF8}{mj}设\end{CJK} $n f(n) \rightarrow 0(n \rightarrow \infty)$, \begin{CJK}{UTF8}{mj}求\end{CJK}
$$
\lim _{n \rightarrow \infty}\left(1+\frac{1}{n}+f(n)\right)^{n}
$$

\begin{enumerate}
  \setcounter{enumi}{3}
  \item \begin{CJK}{UTF8}{mj}求函数\end{CJK}
\end{enumerate}
$$
f(x)=\sqrt{x^{2}+x+1}
$$
\begin{CJK}{UTF8}{mj}的所有渐近线\end{CJK}.

\begin{enumerate}
  \setcounter{enumi}{4}
  \item \begin{CJK}{UTF8}{mj}讨论函数\end{CJK} $\cos \sqrt{x}, \cos x^{2}$ \begin{CJK}{UTF8}{mj}在\end{CJK} $[0,+\infty)$ \begin{CJK}{UTF8}{mj}上的一致连续性\end{CJK}.

  \item \begin{CJK}{UTF8}{mj}设\end{CJK} $f(x)$ \begin{CJK}{UTF8}{mj}在区间\end{CJK} $I$ \begin{CJK}{UTF8}{mj}上有界\end{CJK}, \begin{CJK}{UTF8}{mj}记\end{CJK} $M=\sup _{x \in I} f(x), m=\inf _{x \in I} f(x)$. \begin{CJK}{UTF8}{mj}证明\end{CJK}:

\end{enumerate}
$$
\sup _{x^{\prime}, x^{\prime \prime} \in I}\left|f\left(x^{\prime}\right)-f\left(x^{\prime \prime}\right)\right|=M-m .
$$

\begin{enumerate}
  \setcounter{enumi}{6}
  \item \begin{CJK}{UTF8}{mj}设函数\end{CJK} $f(x)$ \begin{CJK}{UTF8}{mj}在区间\end{CJK} $[a, b]$ \begin{CJK}{UTF8}{mj}上可导\end{CJK}, \begin{CJK}{UTF8}{mj}且\end{CJK} $f^{\prime}(x)$ \begin{CJK}{UTF8}{mj}在区间\end{CJK} $[a, b]$ \begin{CJK}{UTF8}{mj}上单调递增\end{CJK}, \begin{CJK}{UTF8}{mj}证明\end{CJK}:
\end{enumerate}
$$
f\left(\frac{a+b}{2}\right) \leq \frac{1}{b-a} \int_{a}^{b} f(x) \mathrm{d} x .
$$
\begin{CJK}{UTF8}{mj}问若\end{CJK} $f(x)$ \begin{CJK}{UTF8}{mj}在区间\end{CJK} $[a, b]$ \begin{CJK}{UTF8}{mj}上连续且为凸函数\end{CJK}, \begin{CJK}{UTF8}{mj}结论是否成立\end{CJK}? \begin{CJK}{UTF8}{mj}证明你的结果\end{CJK}.

\begin{enumerate}
  \setcounter{enumi}{7}
  \item \begin{CJK}{UTF8}{mj}计算曲线段\end{CJK}
\end{enumerate}
$$
l: f(x)=\ln \cos x, 0 \leq x \leq \frac{\pi}{4}
$$
\begin{CJK}{UTF8}{mj}的弧长\end{CJK}.

\begin{enumerate}
  \setcounter{enumi}{8}
  \item \begin{CJK}{UTF8}{mj}设函数\end{CJK} $f(x)$ \begin{CJK}{UTF8}{mj}在区间\end{CJK} $[0,1]$ \begin{CJK}{UTF8}{mj}上连续\end{CJK}, \begin{CJK}{UTF8}{mj}在\end{CJK} $(0,1)$ \begin{CJK}{UTF8}{mj}内可导\end{CJK}, \begin{CJK}{UTF8}{mj}且\end{CJK} $f(0)=0, f(1)=1$, \begin{CJK}{UTF8}{mj}证明在\end{CJK} $(0,1)$ \begin{CJK}{UTF8}{mj}内存在不同的\end{CJK} $\lambda$, $\mu$, \begin{CJK}{UTF8}{mj}使得\end{CJK}
\end{enumerate}
$$
f^{\prime}(\lambda)\left(f^{\prime}(\mu)+1\right)=2 .
$$

\begin{enumerate}
  \setcounter{enumi}{9}
  \item \begin{CJK}{UTF8}{mj}假设广义积分\end{CJK} $\int_{0}^{+\infty} x f^{\prime}(x) \mathrm{d} x$ \begin{CJK}{UTF8}{mj}收敛\end{CJK}, \begin{CJK}{UTF8}{mj}且当\end{CJK} $x \rightarrow+\infty$ \begin{CJK}{UTF8}{mj}时\end{CJK}, $f(x)$ \begin{CJK}{UTF8}{mj}递减趋于\end{CJK} 0 , \begin{CJK}{UTF8}{mj}证明\end{CJK}
\end{enumerate}
$$
\lim _{x \rightarrow+\infty} x f(x)=0 .
$$
10 . \begin{CJK}{UTF8}{mj}设\end{CJK}
$$
f(x, y)= \begin{cases}\frac{x y^{3}}{x^{2}+y^{4}}, & x^{2}+y^{2} \neq 0 \\ 0, & x^{2}+y^{2}=0\end{cases}
$$
\begin{CJK}{UTF8}{mj}试讨论函数\end{CJK} $f(x, y)$ \begin{CJK}{UTF8}{mj}在原点\end{CJK} $O(0,0)$ \begin{CJK}{UTF8}{mj}的可微性\end{CJK}.

\begin{enumerate}
  \setcounter{enumi}{11}
  \item \begin{CJK}{UTF8}{mj}抛物面\end{CJK}
\end{enumerate}
$$
z=x^{2}+y^{2}
$$
\begin{CJK}{UTF8}{mj}被平面\end{CJK} $x+y+z=1$ \begin{CJK}{UTF8}{mj}截成一椭圆\end{CJK}, \begin{CJK}{UTF8}{mj}求原点到该椭圆的最长距离和最短距离\end{CJK}. 12. \begin{CJK}{UTF8}{mj}证明函数列\end{CJK}
$$
f_{n}(x)=x^{n},(n=1,2, \cdots)
$$
\begin{CJK}{UTF8}{mj}在\end{CJK} $[0,1]$ \begin{CJK}{UTF8}{mj}上非一致收敛\end{CJK}, \begin{CJK}{UTF8}{mj}但函数列\end{CJK}
$$
g_{n}(x)=x^{n}(1-x),(n=1,2, \cdots)
$$
\begin{CJK}{UTF8}{mj}在\end{CJK} $[0,1]$ \begin{CJK}{UTF8}{mj}上一致收敛\end{CJK}.

\begin{enumerate}
  \setcounter{enumi}{13}
  \item \begin{CJK}{UTF8}{mj}计算级数\end{CJK}
\end{enumerate}
$$
\sum_{n=0}^{\infty} \frac{(n+1)^{2}}{2^{n}}
$$
\begin{CJK}{UTF8}{mj}的和\end{CJK}.

\begin{enumerate}
  \setcounter{enumi}{14}
  \item \begin{CJK}{UTF8}{mj}计算积分\end{CJK}
\end{enumerate}
$$
I=\oint_{L} \frac{x \mathrm{~d} y-y \mathrm{~d} x}{9 x^{2}+y^{2}}
$$
$L$ \begin{CJK}{UTF8}{mj}是圆周\end{CJK} $(x-1)^{2}+y^{2}=2$, \begin{CJK}{UTF8}{mj}方向为逆时针\end{CJK}.

\begin{enumerate}
  \setcounter{enumi}{15}
  \item \begin{CJK}{UTF8}{mj}把函数\end{CJK} $f(x)=x$ \begin{CJK}{UTF8}{mj}在\end{CJK} $(0,2 \pi)$ \begin{CJK}{UTF8}{mj}内展开成正弦级数\end{CJK}, \begin{CJK}{UTF8}{mj}并计算\end{CJK}
\end{enumerate}
$$
I=1-\frac{1}{3}+\frac{1}{5}-\cdots+(-1)^{n} \frac{1}{2 n-1}+\cdots
$$

\section{7. 上海大学 2015 年研究生入学考试试题数学分析}
\begin{CJK}{UTF8}{mj}李扬\end{CJK}

\begin{CJK}{UTF8}{mj}微信公众号\end{CJK}: sxkyliyang

\begin{enumerate}
  \item \begin{CJK}{UTF8}{mj}设\end{CJK}
\end{enumerate}
$$
y_{n}=\frac{x_{1}+2 x_{2}+\cdots+n x_{n}}{n(n+1)} .
$$
\begin{CJK}{UTF8}{mj}若\end{CJK} $\lim _{n \rightarrow \infty} x_{n}=a$, \begin{CJK}{UTF8}{mj}证明\end{CJK}:\\
(1) \begin{CJK}{UTF8}{mj}当\end{CJK} $a$ \begin{CJK}{UTF8}{mj}为有限数时\end{CJK}, $\lim _{n \rightarrow \infty} y_{n}=\frac{a}{2}$.\\
(2) \begin{CJK}{UTF8}{mj}当\end{CJK} $a=+\infty$ \begin{CJK}{UTF8}{mj}时\end{CJK}, $\lim _{n \rightarrow \infty} y_{n}=+\infty$.

\begin{enumerate}
  \setcounter{enumi}{2}
  \item \begin{CJK}{UTF8}{mj}求\end{CJK}
\end{enumerate}
$$
\lim _{x \rightarrow 0} \frac{\sin x}{|x|}
$$

\begin{enumerate}
  \setcounter{enumi}{3}
  \item \begin{CJK}{UTF8}{mj}求\end{CJK}
\end{enumerate}
$$
\int_{0}^{+\infty} e^{-x^{2}} d x
$$

\begin{enumerate}
  \setcounter{enumi}{4}
  \item \begin{CJK}{UTF8}{mj}应用\end{CJK} $\frac{\mathrm{e}^{x}-1}{x}$ \begin{CJK}{UTF8}{mj}在\end{CJK} $x=0$ \begin{CJK}{UTF8}{mj}的幂级数展开\end{CJK}, \begin{CJK}{UTF8}{mj}证明\end{CJK}:
\end{enumerate}
$$
\sum_{n=1}^{\infty} \frac{n}{(n+1) !}=1
$$
$5 .$
$$
f(x, y)= \begin{cases}\frac{2 x y^{3}}{x^{2}+y^{4}}, & x^{2}+y^{2} \neq 0 \\ 0, & x^{2}+y^{2}=0\end{cases}
$$
\begin{CJK}{UTF8}{mj}在\end{CJK} $(0,0)$ \begin{CJK}{UTF8}{mj}点的连续性\end{CJK}, \begin{CJK}{UTF8}{mj}可微性\end{CJK}.

\begin{enumerate}
  \setcounter{enumi}{6}
  \item \begin{CJK}{UTF8}{mj}一个证明等式\end{CJK}, \begin{CJK}{UTF8}{mj}一个证明不等式\end{CJK}. \begin{CJK}{UTF8}{mj}用\end{CJK} Taylor \begin{CJK}{UTF8}{mj}展开\end{CJK}.

  \item \begin{CJK}{UTF8}{mj}计算级数\end{CJK}.

  \item \begin{CJK}{UTF8}{mj}利用格林公式\end{CJK}, \begin{CJK}{UTF8}{mj}第二类曲线积分\end{CJK}.

  \item \begin{CJK}{UTF8}{mj}傅里叶级数展开\end{CJK}.

\end{enumerate}
\section{8. 上海大学 2016 年研究生入学考试试题数学分析}
\begin{CJK}{UTF8}{mj}李扬\end{CJK}

\begin{CJK}{UTF8}{mj}微信公众号\end{CJK}: sxkyliyang

$1 .$
$$
\lim _{x \rightarrow 0} \frac{\ln \left(1+\frac{f(x)}{x}\right)}{\mathrm{e}^{x}-1}=1
$$
\begin{CJK}{UTF8}{mj}求\end{CJK} $\lim _{x \rightarrow 0} \frac{f(x)}{x^{2}}$.

\begin{enumerate}
  \setcounter{enumi}{2}
  \item \begin{CJK}{UTF8}{mj}叙述函数极限的\end{CJK} $\varepsilon-N$ \begin{CJK}{UTF8}{mj}定义\end{CJK}, \begin{CJK}{UTF8}{mj}并且用定义证明\end{CJK}:
\end{enumerate}
$$
\lim _{n \rightarrow \infty} \frac{n^{2}+2}{3 n^{2}-5}=\frac{1}{3}
$$

\begin{enumerate}
  \setcounter{enumi}{3}
  \item \begin{CJK}{UTF8}{mj}求\end{CJK}
\end{enumerate}
$$
f(x)=\sqrt{x^{2}+x+2}
$$
\begin{CJK}{UTF8}{mj}的渐近线\end{CJK}.

\begin{enumerate}
  \setcounter{enumi}{4}
  \item \begin{CJK}{UTF8}{mj}设\end{CJK} $E$ \begin{CJK}{UTF8}{mj}为有界数集\end{CJK}, $m$ \begin{CJK}{UTF8}{mj}为其下确界\end{CJK}, \begin{CJK}{UTF8}{mj}且\end{CJK} $m>1$. \begin{CJK}{UTF8}{mj}设任意\end{CJK} $x, y \in E$, \begin{CJK}{UTF8}{mj}且\end{CJK} $x<y$ \begin{CJK}{UTF8}{mj}则有\end{CJK} $\frac{y}{x} \in E$, \begin{CJK}{UTF8}{mj}求证\end{CJK}: $m \in E$.

  \item $f(x)$ \begin{CJK}{UTF8}{mj}为二次可微函数\end{CJK}, \begin{CJK}{UTF8}{mj}对于\end{CJK} $a<c<b$, \begin{CJK}{UTF8}{mj}求证\end{CJK}: \begin{CJK}{UTF8}{mj}存在\end{CJK} $\xi$, \begin{CJK}{UTF8}{mj}使得\end{CJK}

\end{enumerate}
$$
\frac{f(b)-f(a)}{b-a}-\frac{f(c)-f(a)}{c-a}=\frac{1}{2}(b-c) f^{\prime}(\xi) .
$$

\begin{enumerate}
  \setcounter{enumi}{6}
  \item $f^{\prime}(x)+f^{\prime 2}(x)=x$, \begin{CJK}{UTF8}{mj}求证\end{CJK}: \begin{CJK}{UTF8}{mj}是否在\end{CJK} $x=0$ \begin{CJK}{UTF8}{mj}处取得极值\end{CJK}.

  \item \begin{CJK}{UTF8}{mj}讨论\end{CJK}: $\sqrt[4]{x}, x^{4}$ \begin{CJK}{UTF8}{mj}在\end{CJK} $[0,+\infty)$ \begin{CJK}{UTF8}{mj}的一致连续性\end{CJK}.

  \item \begin{CJK}{UTF8}{mj}求\end{CJK}

\end{enumerate}
$$
\oint_{l} \frac{x \mathrm{~d} y-y \mathrm{~d} x}{4 x^{2}+y^{2}}
$$
\begin{CJK}{UTF8}{mj}其中\end{CJK} $l$ \begin{CJK}{UTF8}{mj}为\end{CJK} $x^{2}+y^{2}=1$ \begin{CJK}{UTF8}{mj}所围\end{CJK}.

9 . \begin{CJK}{UTF8}{mj}设\end{CJK} $f(x)$ \begin{CJK}{UTF8}{mj}在\end{CJK} $[0,+\infty)$ \begin{CJK}{UTF8}{mj}上一致连续\end{CJK}, \begin{CJK}{UTF8}{mj}且\end{CJK} $\int_{0}^{+\infty} f(x) \mathrm{d} x$ \begin{CJK}{UTF8}{mj}收敛\end{CJK}, \begin{CJK}{UTF8}{mj}求证\end{CJK}:
$$
\lim _{x \rightarrow+\infty} f(x)=0 .
$$

\begin{enumerate}
  \setcounter{enumi}{10}
  \item \begin{CJK}{UTF8}{mj}证明\end{CJK}:
\end{enumerate}
$$
\sum_{i=0}^{n-1} \frac{1}{n} f\left(x+\frac{i}{n}\right)=f_{n}(x)
$$
\begin{CJK}{UTF8}{mj}一致收敛\end{CJK}.

\begin{enumerate}
  \setcounter{enumi}{11}
  \item \begin{CJK}{UTF8}{mj}若\end{CJK}: $\lim _{x \rightarrow 0} \frac{f^{\prime}(x)}{x}+f(x)=0$, \begin{CJK}{UTF8}{mj}求证\end{CJK}:
\end{enumerate}
$$
\lim _{x \rightarrow 0} f(x)=0 .
$$

\begin{enumerate}
  \setcounter{enumi}{12}
  \item \begin{CJK}{UTF8}{mj}求和\end{CJK}:
\end{enumerate}
$$
\sum_{n=0}^{\infty} \frac{(-1)^{n}}{3 n+1}
$$

\begin{enumerate}
  \setcounter{enumi}{13}
  \item \begin{CJK}{UTF8}{mj}求弧长\end{CJK}:
\end{enumerate}
$$
f(x)=\frac{x^{4}}{4}-\frac{1}{8 x^{2}}
$$
$14 .$
$$
f(x, y)= \begin{cases}\frac{x y}{|x|+y^{2}}, & x^{2}+y^{2} \neq 0 \\ 0, & x^{2}+y^{2}=0\end{cases}
$$
\begin{CJK}{UTF8}{mj}讨论在\end{CJK} $(0,0)$ \begin{CJK}{UTF8}{mj}点的可微性\end{CJK}.

\begin{enumerate}
  \setcounter{enumi}{15}
  \item \begin{CJK}{UTF8}{mj}末知\end{CJK}.
\end{enumerate}
\section{9. 上海大学 2017 年研究生入学考试试题数学分析}
\begin{CJK}{UTF8}{mj}李扬\end{CJK}

\begin{CJK}{UTF8}{mj}微信公众号\end{CJK}: sxkyliyang

\begin{enumerate}
  \item \begin{CJK}{UTF8}{mj}叙述函数极限的\end{CJK} $\varepsilon-N$ \begin{CJK}{UTF8}{mj}定义\end{CJK}, \begin{CJK}{UTF8}{mj}并且用定义证明\end{CJK}:
\end{enumerate}
$$
\lim _{n \rightarrow \infty} \frac{n^{2}+2}{3 n^{2}-5}=\frac{1}{3} .
$$

\begin{enumerate}
  \setcounter{enumi}{2}
  \item \begin{CJK}{UTF8}{mj}叙述\end{CJK} $\lim _{x \rightarrow x_{0}} f(x)=A$ \begin{CJK}{UTF8}{mj}的\end{CJK} $\varepsilon-\delta$ \begin{CJK}{UTF8}{mj}定义\end{CJK}, \begin{CJK}{UTF8}{mj}并用该定义证明\end{CJK}
\end{enumerate}
$$
\lim _{x \rightarrow 1} \frac{x+1}{2 x^{3}-2 x+1}=2 .
$$

\begin{enumerate}
  \setcounter{enumi}{3}
  \item $\left\{x_{n}\right\}$ \begin{CJK}{UTF8}{mj}有上界\end{CJK} $a$, \begin{CJK}{UTF8}{mj}证明存在一个单调递增数列\end{CJK} $\left\{x_{n}\right\}$, \begin{CJK}{UTF8}{mj}使得\end{CJK}
\end{enumerate}
$$
\lim _{n \rightarrow \infty} x_{n}=a .
$$
$4 .$
$$
f(x)=\frac{\mathrm{e}^{x}+\mathrm{e}^{-x}}{2}
$$
\begin{CJK}{UTF8}{mj}求\end{CJK} $f(x)$ \begin{CJK}{UTF8}{mj}的所有渐近线\end{CJK}.

\begin{enumerate}
  \setcounter{enumi}{5}
  \item \begin{CJK}{UTF8}{mj}证明\end{CJK} $\sin x^{2}$ \begin{CJK}{UTF8}{mj}和\end{CJK} $\sin ^{2} x$ \begin{CJK}{UTF8}{mj}在\end{CJK} $[0,+\infty)$ \begin{CJK}{UTF8}{mj}上的一致连续性\end{CJK}.

  \item $f(x)$ \begin{CJK}{UTF8}{mj}在\end{CJK} $[0,1]$ \begin{CJK}{UTF8}{mj}上连续\end{CJK}, \begin{CJK}{UTF8}{mj}在\end{CJK} $[0,1]$ \begin{CJK}{UTF8}{mj}上可导\end{CJK}, \begin{CJK}{UTF8}{mj}且\end{CJK} $f(0)=0, f(1)=1$, \begin{CJK}{UTF8}{mj}证\end{CJK}: \begin{CJK}{UTF8}{mj}存在不同的\end{CJK} $x_{1}, x_{2}, x_{3}$ \begin{CJK}{UTF8}{mj}使得\end{CJK}

\end{enumerate}
$$
\frac{1}{f^{\prime}\left(x_{1}\right)}+\frac{1}{f^{\prime}\left(x_{2}\right)}+\frac{1}{f^{\prime}\left(x_{3}\right)}=3 \text {. }
$$

\begin{enumerate}
  \setcounter{enumi}{7}
  \item $f(x)$ \begin{CJK}{UTF8}{mj}在\end{CJK} $[0,1]$ \begin{CJK}{UTF8}{mj}上可导\end{CJK}, \begin{CJK}{UTF8}{mj}且\end{CJK} $f(0)=0$,
\end{enumerate}
$$
f(x, y)=y f(x)+x f(y)+x y\left(x^{2}+x+2\right) .
$$
\begin{CJK}{UTF8}{mj}求\end{CJK} $f(x)$. $\left(f(x, y)=y f(x)+x f(y)+x y\left(x^{2}+x+2\right)\right.$ \begin{CJK}{UTF8}{mj}这个式子可能写错\end{CJK}, \begin{CJK}{UTF8}{mj}和某年的真题类似\end{CJK}. \begin{CJK}{UTF8}{mj}先对\end{CJK} $x$ \begin{CJK}{UTF8}{mj}求导\end{CJK}, \begin{CJK}{UTF8}{mj}然\end{CJK} \begin{CJK}{UTF8}{mj}后令\end{CJK} $x=1$, \begin{CJK}{UTF8}{mj}再两边积分可以得到\end{CJK}.)
$$
f(x, y)= \begin{cases}\frac{x y}{x^{2}+y^{2}}, & x^{2}+y^{2} \neq 0 \\ 0, & x^{2}+y^{2}=0\end{cases}
$$
\begin{CJK}{UTF8}{mj}求\end{CJK} $f(x, y)$ \begin{CJK}{UTF8}{mj}在\end{CJK} $(0,0)$ \begin{CJK}{UTF8}{mj}处的可微性\end{CJK}.

\begin{enumerate}
  \setcounter{enumi}{9}
  \item $f(x, y, z, t)=0, g(y, z, t)=0, h(z, t)=0$, \begin{CJK}{UTF8}{mj}求\end{CJK} $\frac{\mathrm{d} y}{\mathrm{~d} z}, \frac{\mathrm{d} z}{\mathrm{~d} x}$ (\begin{CJK}{UTF8}{mj}题目有点忘了\end{CJK})
\end{enumerate}
$10 .$
$$
f_{n}(x)=(x-1)^{n} s(x) .
$$
\begin{CJK}{UTF8}{mj}证明\end{CJK}: $f_{n}(x)$ \begin{CJK}{UTF8}{mj}在\end{CJK} $[1,2]$ \begin{CJK}{UTF8}{mj}上一致收敛的充要条件是\end{CJK} $s(2)=0$.

\begin{enumerate}
  \setcounter{enumi}{11}
  \item \begin{CJK}{UTF8}{mj}计算积分\end{CJK}
\end{enumerate}
$$
I=\oint_{L} \frac{x \mathrm{~d} y-y \mathrm{~d} x}{4 x^{2}+y^{2}}
$$
$L$ \begin{CJK}{UTF8}{mj}是由\end{CJK} $A(0,1), B(-1,-1), C(1,-1)$ \begin{CJK}{UTF8}{mj}围成的封闭曲线\end{CJK}. 12. \begin{CJK}{UTF8}{mj}计算级数\end{CJK}
$$
\sum_{n=0}^{\infty} \frac{(n+1) !}{n^{2}-1}
$$
(\begin{CJK}{UTF8}{mj}题目有点忘了\end{CJK})

$13 .$
$$
f(x)= \begin{cases}\pi-x, & x \in[-\pi, 0] \\ \pi+x, & x \in[0, \pi]\end{cases}
$$
\begin{CJK}{UTF8}{mj}展成傅立叶级数\end{CJK}, \begin{CJK}{UTF8}{mj}并计算\end{CJK} $1-\frac{1}{3}+\frac{1}{5}-\frac{1}{7}+\cdots$.

\begin{enumerate}
  \setcounter{enumi}{14}
  \item \begin{CJK}{UTF8}{mj}末知\end{CJK}

  \item \begin{CJK}{UTF8}{mj}末知\end{CJK}

\end{enumerate}
\section{0. 上海大学 2018 年研究生入学考试试题数学分析}
\begin{CJK}{UTF8}{mj}李扬\end{CJK}

\begin{CJK}{UTF8}{mj}微信公众号\end{CJK}: sxkyliyang

\begin{enumerate}
  \item \begin{CJK}{UTF8}{mj}叙述数列极限的\end{CJK} $\epsilon-N$ \begin{CJK}{UTF8}{mj}定义\end{CJK}, \begin{CJK}{UTF8}{mj}并用该定义证明\end{CJK}
\end{enumerate}
$$
\lim _{n \rightarrow \infty} \frac{n^{2}+n+1}{3 n^{2}-4 n+1}=\frac{1}{3} .
$$

\begin{enumerate}
  \setcounter{enumi}{2}
  \item \begin{CJK}{UTF8}{mj}叙述\end{CJK} $\lim _{x \rightarrow x_{0}} f(x)=A$ \begin{CJK}{UTF8}{mj}的\end{CJK} $\epsilon-\delta$ \begin{CJK}{UTF8}{mj}定义\end{CJK}, \begin{CJK}{UTF8}{mj}并用该定义证明\end{CJK}
\end{enumerate}
$$
\lim _{x \rightarrow 2} \frac{x^{2}-4}{x^{2}+x-6}=\frac{4}{5} .
$$

\begin{enumerate}
  \setcounter{enumi}{3}
  \item \begin{CJK}{UTF8}{mj}叙述\end{CJK} $f(x)$ \begin{CJK}{UTF8}{mj}在区间\end{CJK} $I$ \begin{CJK}{UTF8}{mj}上一致收敛的定义\end{CJK}, \begin{CJK}{UTF8}{mj}并讨论函数\end{CJK} $\ln x$ \begin{CJK}{UTF8}{mj}分别在区间\end{CJK} $(0,1]$ \begin{CJK}{UTF8}{mj}与\end{CJK} $[1,+\infty)$ \begin{CJK}{UTF8}{mj}上的一致收敛性\end{CJK}.

  \item \begin{CJK}{UTF8}{mj}求函数\end{CJK} $f(x)=\sqrt{x^{2}+2 x+5}$ \begin{CJK}{UTF8}{mj}的所有渐近线\end{CJK}.

  \item \begin{CJK}{UTF8}{mj}设\end{CJK} $f(x), g(x)$ \begin{CJK}{UTF8}{mj}均是定义在区间\end{CJK} $I$ \begin{CJK}{UTF8}{mj}上的有界函数\end{CJK}. \begin{CJK}{UTF8}{mj}证明\end{CJK}

\end{enumerate}
$$
\sup _{x \in I}\{f(x)+g(x)\} \leq \sup _{x \in I}\{f(x)\}+\sup _{x \in I}\{g(x)\} .
$$

\begin{enumerate}
  \setcounter{enumi}{6}
  \item \begin{CJK}{UTF8}{mj}设\end{CJK} $f(x)$ \begin{CJK}{UTF8}{mj}在\end{CJK} $[0,1]$ \begin{CJK}{UTF8}{mj}上有连续导数\end{CJK}, \begin{CJK}{UTF8}{mj}在\end{CJK} $(0,1)$ \begin{CJK}{UTF8}{mj}上二阶可导且\end{CJK} $f(0)=f(1)$. \begin{CJK}{UTF8}{mj}证明\end{CJK}: \begin{CJK}{UTF8}{mj}在\end{CJK} $(0,1)$ \begin{CJK}{UTF8}{mj}中至少存在一个\end{CJK} $\xi$ \begin{CJK}{UTF8}{mj}使得\end{CJK}
\end{enumerate}
$$
f^{\prime \prime}(\xi)=\frac{3 f^{\prime}(\xi)}{1-\xi}
$$

\begin{enumerate}
  \setcounter{enumi}{7}
  \item \begin{CJK}{UTF8}{mj}假设\end{CJK} $\int_{0}^{+\infty} f(x) \mathrm{d} x$ \begin{CJK}{UTF8}{mj}收敛\end{CJK}, \begin{CJK}{UTF8}{mj}且\end{CJK} $f(x)$ \begin{CJK}{UTF8}{mj}在\end{CJK} $[0, \infty)$ \begin{CJK}{UTF8}{mj}上一致连续\end{CJK}, \begin{CJK}{UTF8}{mj}证明\end{CJK} $\lim _{x \rightarrow+\infty} f(x)=0$.

  \item \begin{CJK}{UTF8}{mj}设函数\end{CJK} $f(x)$ \begin{CJK}{UTF8}{mj}在\end{CJK} $[0,1]$ \begin{CJK}{UTF8}{mj}连续且\end{CJK} $\int_{0}^{1} f(x) \mathrm{d} x=\int_{0}^{1} x f(x) \mathrm{d} x=0$. \begin{CJK}{UTF8}{mj}证明\end{CJK}: \begin{CJK}{UTF8}{mj}在\end{CJK} $(0,1)$ \begin{CJK}{UTF8}{mj}中至少存在两个不同的点\end{CJK} $x_{1}, x_{2}$ \begin{CJK}{UTF8}{mj}使得\end{CJK} $f\left(x_{1}\right)=f\left(x_{2}\right)=0$.

  \item \begin{CJK}{UTF8}{mj}设\end{CJK}

\end{enumerate}
$$
f(x, y)= \begin{cases}\frac{x y^{2}}{\sqrt{x^{2}+y^{2}}}, & x^{2}+y^{2} \neq 0 \\ 0, & x^{2}+y^{2}=0\end{cases}
$$
\begin{CJK}{UTF8}{mj}讨论函数\end{CJK} $f(x, y)$ \begin{CJK}{UTF8}{mj}在原点\end{CJK} $(0,0)$ \begin{CJK}{UTF8}{mj}处的可微性\end{CJK}.

\begin{enumerate}
  \setcounter{enumi}{10}
  \item \begin{CJK}{UTF8}{mj}证明\end{CJK}: \begin{CJK}{UTF8}{mj}曲面\end{CJK} $F\left(\frac{z}{y}, \frac{x}{z}, \frac{y}{x}\right)=0$ \begin{CJK}{UTF8}{mj}的所有切平面都经过某一定点\end{CJK}, \begin{CJK}{UTF8}{mj}其中函数\end{CJK} $F$ \begin{CJK}{UTF8}{mj}有连续的偏导数\end{CJK}.

  \item \begin{CJK}{UTF8}{mj}叙述函数列\end{CJK} $\left\{f_{n}(x)\right\}$ \begin{CJK}{UTF8}{mj}在区间\end{CJK} $I$ \begin{CJK}{UTF8}{mj}上一致收敛于\end{CJK} $f(x)$ \begin{CJK}{UTF8}{mj}的定义\end{CJK}, \begin{CJK}{UTF8}{mj}并讨论函数列\end{CJK} $\{\sqrt[n]{x}\}$ \begin{CJK}{UTF8}{mj}在\end{CJK} $\left[0, \frac{1}{2}\right]$ \begin{CJK}{UTF8}{mj}与\end{CJK} $\left[\frac{1}{2}, 1\right]$ \begin{CJK}{UTF8}{mj}上的一致收\end{CJK} \begin{CJK}{UTF8}{mj}敛性\end{CJK}.

  \item \begin{CJK}{UTF8}{mj}计算\end{CJK}

\end{enumerate}
$$
\sum_{n=1}^{\infty} \frac{n^{2}+1}{n 2^{n}}
$$
\begin{CJK}{UTF8}{mj}的和\end{CJK}.

\begin{enumerate}
  \setcounter{enumi}{13}
  \item \begin{CJK}{UTF8}{mj}计算积分\end{CJK}
\end{enumerate}
$$
I=\oint_{L} \frac{x \mathrm{~d} y-y \mathrm{~d} x}{4 x^{2}+3 y^{2}},
$$
\begin{CJK}{UTF8}{mj}其中\end{CJK} $L$ \begin{CJK}{UTF8}{mj}是顶点为\end{CJK} $(1,1),(2,0),(1,-1),(-1,-1),(-2,0),(-1,1)$ \begin{CJK}{UTF8}{mj}的六边形\end{CJK}, \begin{CJK}{UTF8}{mj}方向为逆时针\end{CJK}.

\begin{enumerate}
  \setcounter{enumi}{14}
  \item \begin{CJK}{UTF8}{mj}将函数\end{CJK} $f(x)=x(\pi-x), 0 \leq x \leq \pi$ \begin{CJK}{UTF8}{mj}展开成正弦级数\end{CJK}, \begin{CJK}{UTF8}{mj}并证明\end{CJK}
\end{enumerate}
$$
\sum_{n=1}^{\infty}(-1)^{n-1} \frac{1}{(2 n-1)^{3}}=\frac{\pi^{3}}{32}
$$

\section{1. 上海大学 2009 年研究生入学考试试题高等代数 
 李扬 
 微信公众号: sxkyliyang}
\begin{enumerate}
  \item \begin{CJK}{UTF8}{mj}填空\end{CJK}
\end{enumerate}
(1) $\mathbb{F}^{3 \times 3}$ \begin{CJK}{UTF8}{mj}为\end{CJK} $\mathbb{F}$ \begin{CJK}{UTF8}{mj}上所有三阶矩阵组成的集合\end{CJK}, \begin{CJK}{UTF8}{mj}令\end{CJK} $V=\left\{A \mid A \in \mathbb{F}^{3 \times 3}\right\}$ (\begin{CJK}{UTF8}{mj}其中\end{CJK} $\operatorname{tr}(A)=0$ \begin{CJK}{UTF8}{mj}且\end{CJK} $A$ \begin{CJK}{UTF8}{mj}为上三角矩阵\end{CJK}), \begin{CJK}{UTF8}{mj}则\end{CJK} $\operatorname{dim} V=$

(2) $f(x), g(x)$ \begin{CJK}{UTF8}{mj}为\end{CJK} $\mathbb{F}$ \begin{CJK}{UTF8}{mj}上多项式\end{CJK}, \begin{CJK}{UTF8}{mj}且在复数域上无公共根\end{CJK}, \begin{CJK}{UTF8}{mj}则\end{CJK} $f(x), f(x)+g(x)$ \begin{CJK}{UTF8}{mj}在\end{CJK} $\mathbb{F}$ \begin{CJK}{UTF8}{mj}上的首项系数为\end{CJK} 1 \begin{CJK}{UTF8}{mj}的最大\end{CJK} \begin{CJK}{UTF8}{mj}公因式为\end{CJK}

(3) \begin{CJK}{UTF8}{mj}设\end{CJK} $A$ \begin{CJK}{UTF8}{mj}是\end{CJK} $n$ \begin{CJK}{UTF8}{mj}阶矩阵\end{CJK}, \begin{CJK}{UTF8}{mj}则\end{CJK} $\left|\begin{array}{ccc}A & 2 A & 3 A \\ 4 A & 5 A & 6 A \\ 7 A & 8 A & 9 A\end{array}\right|=$

(4) $A$ \begin{CJK}{UTF8}{mj}为\end{CJK} 3 \begin{CJK}{UTF8}{mj}阶对称矩阵\end{CJK}, $1,2,3$ \begin{CJK}{UTF8}{mj}为其特征值\end{CJK}, \begin{CJK}{UTF8}{mj}则\end{CJK} $A$ \begin{CJK}{UTF8}{mj}的伴随矩阵\end{CJK} $A^{*}$ \begin{CJK}{UTF8}{mj}与对角矩阵\end{CJK} \begin{CJK}{UTF8}{mj}相似\end{CJK}.

\begin{enumerate}
  \setcounter{enumi}{2}
  \item \begin{CJK}{UTF8}{mj}求\end{CJK}
\end{enumerate}
$$
A=\left|\begin{array}{ccccc}
1 & 1 & \cdots & 1 & a \\
x_{1} & x_{2} & \cdots & x_{n-1} & x_{n} \\
\vdots & \vdots & & \vdots & \vdots \\
x_{1}^{n-1} & x_{2}^{n-1} & \cdots & x_{n-1}^{n-1} & x_{n}^{n-1}
\end{array}\right| .
$$

\begin{enumerate}
  \setcounter{enumi}{3}
  \item $V$ \begin{CJK}{UTF8}{mj}为\end{CJK} $\mathbb{F}$ \begin{CJK}{UTF8}{mj}上\end{CJK} $n$ \begin{CJK}{UTF8}{mj}维线性空间\end{CJK}, \begin{CJK}{UTF8}{mj}且\end{CJK} $U, W$ \begin{CJK}{UTF8}{mj}为\end{CJK} $V$ \begin{CJK}{UTF8}{mj}的子空间\end{CJK}, \begin{CJK}{UTF8}{mj}证明\end{CJK}:
\end{enumerate}
$$
\operatorname{dim}(U+W)+\operatorname{dim}(U \cap W)=\operatorname{dim} W+\operatorname{dim} U
$$

\begin{enumerate}
  \setcounter{enumi}{4}
  \item $\mathbb{R}$ \begin{CJK}{UTF8}{mj}为实数域\end{CJK}, $\mathscr{A}$ \begin{CJK}{UTF8}{mj}为\end{CJK} $\mathbb{R}^{3}$ \begin{CJK}{UTF8}{mj}上的线性变换\end{CJK}, \begin{CJK}{UTF8}{mj}且\end{CJK} $\mathscr{A}$ \begin{CJK}{UTF8}{mj}在\end{CJK} $\mathbb{R}^{3}$ \begin{CJK}{UTF8}{mj}基\end{CJK}: $e_{1}=(0,0,1), e_{2}=(0,1,0), e_{3}=(1,0,0)$ \begin{CJK}{UTF8}{mj}下的矩阵是\end{CJK}
\end{enumerate}
$$
\left(\begin{array}{lll}
3 & 0 & 0 \\
0 & 3 & 1 \\
0 & 0 & 3
\end{array}\right) \text {. }
$$
\begin{CJK}{UTF8}{mj}证明\end{CJK}:

(1) \begin{CJK}{UTF8}{mj}若\end{CJK} $W_{1}=L\left(e_{1}, e_{2}\right)$, \begin{CJK}{UTF8}{mj}则\end{CJK} $W_{1}$ \begin{CJK}{UTF8}{mj}是\end{CJK} $\mathscr{A}$ \begin{CJK}{UTF8}{mj}的不变子空间\end{CJK}.

(2) \begin{CJK}{UTF8}{mj}不存在\end{CJK} $\mathscr{A}$ \begin{CJK}{UTF8}{mj}的不变子空间\end{CJK} $W_{2}$, \begin{CJK}{UTF8}{mj}使\end{CJK} $\mathbb{R}^{3}=W_{1} \oplus W_{2}$.

\begin{enumerate}
  \setcounter{enumi}{5}
  \item \begin{CJK}{UTF8}{mj}已知\end{CJK}
\end{enumerate}
$$
\begin{gathered}
f(x)=x^{3}+x^{2}+x+1 \\
g(x)=x^{4 n}+x^{4 m+1}+x^{4 k+2}+x^{4 l+3}
\end{gathered}
$$
$(n, m, k, l$ \begin{CJK}{UTF8}{mj}为正整数\end{CJK} $)$, \begin{CJK}{UTF8}{mj}证明\end{CJK}: $f(x)$ \begin{CJK}{UTF8}{mj}整除\end{CJK} $g(x)$.

\begin{enumerate}
  \setcounter{enumi}{6}
  \item \begin{CJK}{UTF8}{mj}已知\end{CJK} $n$ \begin{CJK}{UTF8}{mj}阶矩阵\end{CJK} $A, A$ \begin{CJK}{UTF8}{mj}满足\end{CJK} $A^{n-1}=0$, \begin{CJK}{UTF8}{mj}而\end{CJK} $A^{n-2} \neq 0$, \begin{CJK}{UTF8}{mj}则称\end{CJK} $A$ \begin{CJK}{UTF8}{mj}的幂零指数为\end{CJK} $n-1$, \begin{CJK}{UTF8}{mj}证明\end{CJK}: \begin{CJK}{UTF8}{mj}幂零指数为\end{CJK} $n-1$ \begin{CJK}{UTF8}{mj}的矩\end{CJK} \begin{CJK}{UTF8}{mj}阵都相似\end{CJK}.

  \item \begin{CJK}{UTF8}{mj}设\end{CJK} $A$ \begin{CJK}{UTF8}{mj}是\end{CJK} $n$ \begin{CJK}{UTF8}{mj}阶矩阵\end{CJK}, \begin{CJK}{UTF8}{mj}证明\end{CJK}:

\end{enumerate}
$$
A^{2}=2 I-A \Leftrightarrow \mathrm{r}(A+2 I)+\mathrm{r}(A-I)=n
$$

\begin{enumerate}
  \setcounter{enumi}{8}
  \item $\mathbb{F}$ \begin{CJK}{UTF8}{mj}上齐次方程组\end{CJK} $X_{1 \times n} A_{n \times m}=O_{1 \times m}(*)$, \begin{CJK}{UTF8}{mj}令\end{CJK} $C=\left(A_{n \times m} I_{n}\right)$, \begin{CJK}{UTF8}{mj}对\end{CJK} $C$ \begin{CJK}{UTF8}{mj}做一系列的初等变换化为\end{CJK} $\left(\left(\begin{array}{c}D_{r} \\ 0\end{array}\right) P\right)$, \begin{CJK}{UTF8}{mj}其中\end{CJK} $D_{r}$ \begin{CJK}{UTF8}{mj}为一行满秩\end{CJK}, $r=r(A), P$ \begin{CJK}{UTF8}{mj}为\end{CJK} $n$ \begin{CJK}{UTF8}{mj}阶可逆方阵\end{CJK}. \begin{CJK}{UTF8}{mj}证明\end{CJK}: $P$ \begin{CJK}{UTF8}{mj}的最后\end{CJK} $n-r$ \begin{CJK}{UTF8}{mj}行即为\end{CJK} \begin{CJK}{UTF8}{mj}方程组\end{CJK} $(*)$ \begin{CJK}{UTF8}{mj}的一个基础解系\end{CJK}. 9. $n$ \begin{CJK}{UTF8}{mj}阶实对称矩阵\end{CJK} $A, B$ \begin{CJK}{UTF8}{mj}的特征值都是正数\end{CJK}, $C$ \begin{CJK}{UTF8}{mj}为正定矩阵\end{CJK}, $A$ \begin{CJK}{UTF8}{mj}的特征向量都是\end{CJK} $B$ \begin{CJK}{UTF8}{mj}的特征向量\end{CJK}. \begin{CJK}{UTF8}{mj}证明\end{CJK}:\\
(1) $A B$ \begin{CJK}{UTF8}{mj}为正定阵\end{CJK};\\
(2) $\operatorname{tr}(A B C) \geq 0$.
\end{enumerate}
\section{2. 上海大学 2010 年研究生入学考试试题高等代数 
 李扬 
 微信公众号: sxkyliyang}
\begin{enumerate}
  \item (\begin{CJK}{UTF8}{mj}每题\end{CJK} 4 \begin{CJK}{UTF8}{mj}分\end{CJK}, \begin{CJK}{UTF8}{mj}共\end{CJK} 20 \begin{CJK}{UTF8}{mj}分\end{CJK}) \begin{CJK}{UTF8}{mj}填空题\end{CJK}
\end{enumerate}
(1) \begin{CJK}{UTF8}{mj}设\end{CJK} $A$ \begin{CJK}{UTF8}{mj}为\end{CJK} $n$ \begin{CJK}{UTF8}{mj}阶方阵\end{CJK}, \begin{CJK}{UTF8}{mj}且\end{CJK} $|A|=2$, \begin{CJK}{UTF8}{mj}则\end{CJK} $\left|A^{*} A-I\right|=$ ,$\left|A^{*}\right|=$

(2) \begin{CJK}{UTF8}{mj}设\end{CJK} $t$ \begin{CJK}{UTF8}{mj}为整数\end{CJK}, \begin{CJK}{UTF8}{mj}且\end{CJK} $A=\left[\begin{array}{ccc}1 & 2 & 1 \\ 2 & t & 2 \\ 1 & 2 & 7-t\end{array}\right]$ \begin{CJK}{UTF8}{mj}为正定矩阵\end{CJK}, \begin{CJK}{UTF8}{mj}则\end{CJK} $t=$

(3) \begin{CJK}{UTF8}{mj}设\end{CJK} $V$ \begin{CJK}{UTF8}{mj}为\end{CJK} 3 \begin{CJK}{UTF8}{mj}维欧氏空间\end{CJK}, $\varepsilon_{1}, \varepsilon_{2}, \varepsilon_{3}$ \begin{CJK}{UTF8}{mj}为\end{CJK} $V$ \begin{CJK}{UTF8}{mj}的标准正交基\end{CJK}, \begin{CJK}{UTF8}{mj}如果基\end{CJK} $\alpha_{1}=\varepsilon_{1}+\varepsilon_{2}+\varepsilon_{3}, \alpha_{2}=\varepsilon_{1}+\varepsilon_{2}, \alpha_{3}=\varepsilon_{1}$, \begin{CJK}{UTF8}{mj}则基\end{CJK} $\alpha_{1}, \alpha_{2}, \alpha_{3}$ \begin{CJK}{UTF8}{mj}的度量矩阵为\end{CJK}

(4) \begin{CJK}{UTF8}{mj}已知\end{CJK} $n$ \begin{CJK}{UTF8}{mj}阶方阵\end{CJK} $A$ \begin{CJK}{UTF8}{mj}的秩\end{CJK} $\mathrm{r}(A)=n-2, \alpha_{1}=[1,2,3]^{T}, \alpha_{2}=[1,1,1]^{T}, \alpha_{3}=[2,3,2]^{T}$ \begin{CJK}{UTF8}{mj}为非齐次线性方程组\end{CJK} $A X=b$ \begin{CJK}{UTF8}{mj}的解\end{CJK}, \begin{CJK}{UTF8}{mj}则\end{CJK} $A X=b$ \begin{CJK}{UTF8}{mj}的通解为\end{CJK}

(5) \begin{CJK}{UTF8}{mj}设\end{CJK} $A=\left[\begin{array}{lll}1 & 0 & 0 \\ 0 & 2 & 0 \\ 0 & 0 & 3\end{array}\right], V=\left\{B \in \mathbb{F}^{3 \times 3} \mid A B=B A\right\}$, \begin{CJK}{UTF8}{mj}则\end{CJK} $V$ \begin{CJK}{UTF8}{mj}为\end{CJK} $\mathbb{F}$ \begin{CJK}{UTF8}{mj}上\end{CJK} \begin{CJK}{UTF8}{mj}维线性空间\end{CJK}, \begin{CJK}{UTF8}{mj}基为\end{CJK}

\begin{enumerate}
  \setcounter{enumi}{2}
  \item (\begin{CJK}{UTF8}{mj}每题\end{CJK} 3 \begin{CJK}{UTF8}{mj}分\end{CJK}, \begin{CJK}{UTF8}{mj}共\end{CJK} 30 \begin{CJK}{UTF8}{mj}分\end{CJK}) \begin{CJK}{UTF8}{mj}多项选择题\end{CJK}
\end{enumerate}
(1) \begin{CJK}{UTF8}{mj}设矩阵\end{CJK} $A$ \begin{CJK}{UTF8}{mj}与矩阵\end{CJK} $B$ \begin{CJK}{UTF8}{mj}相似\end{CJK}, \begin{CJK}{UTF8}{mj}则有\end{CJK} ( )\\
A. $A$ \begin{CJK}{UTF8}{mj}与\end{CJK} $B$ \begin{CJK}{UTF8}{mj}有相同的特征值\end{CJK};\\
B. $A$ \begin{CJK}{UTF8}{mj}与\end{CJK} $B$ \begin{CJK}{UTF8}{mj}有相同的特征向量\end{CJK};\\
C. $A$ \begin{CJK}{UTF8}{mj}与\end{CJK} $B$ \begin{CJK}{UTF8}{mj}有相同的特征多项式\end{CJK};\\
D. $A$ \begin{CJK}{UTF8}{mj}与\end{CJK} $B$ \begin{CJK}{UTF8}{mj}有相同的行列式\end{CJK}.

(2) \begin{CJK}{UTF8}{mj}设\end{CJK} $f(x)$ \begin{CJK}{UTF8}{mj}为有理系数多项式\end{CJK}, \begin{CJK}{UTF8}{mj}且没有有理根\end{CJK}, \begin{CJK}{UTF8}{mj}下列结论正确的有\end{CJK} ( )\\
A. $f(x)$ \begin{CJK}{UTF8}{mj}在有理系数域上不可约\end{CJK};\\
B. \begin{CJK}{UTF8}{mj}如果\end{CJK} $f(x)$ \begin{CJK}{UTF8}{mj}次数小于等于\end{CJK} 3 , \begin{CJK}{UTF8}{mj}则\end{CJK} $f(x)$ \begin{CJK}{UTF8}{mj}在有理系数域上不可约\end{CJK};\\
C. $f(x)$ \begin{CJK}{UTF8}{mj}在复数域上不可约\end{CJK};\\
D. \begin{CJK}{UTF8}{mj}不能确定\end{CJK} $f(x)$ \begin{CJK}{UTF8}{mj}在有理系数域上是否可约\end{CJK}.

(3) \begin{CJK}{UTF8}{mj}设\end{CJK} $A$ \begin{CJK}{UTF8}{mj}为\end{CJK} $n \times m$ \begin{CJK}{UTF8}{mj}阶矩阵\end{CJK}, $B$ \begin{CJK}{UTF8}{mj}为方阵\end{CJK}, \begin{CJK}{UTF8}{mj}且\end{CJK} $A B=0$, \begin{CJK}{UTF8}{mj}则\end{CJK} ( )\\
A. \begin{CJK}{UTF8}{mj}当\end{CJK} $A$ \begin{CJK}{UTF8}{mj}为非零矩阵时\end{CJK}, $B=0$;\\
B. $\mathrm{r}(A)+\mathrm{r}(B) \leq n$;\\
C. $B$ \begin{CJK}{UTF8}{mj}的列向量为线性方程组\end{CJK} $A X=0$ \begin{CJK}{UTF8}{mj}的解\end{CJK};\\
D. $|B|=0$.

(4) \begin{CJK}{UTF8}{mj}设\end{CJK} $A, B$ \begin{CJK}{UTF8}{mj}分别为\end{CJK} $n \times m$ \begin{CJK}{UTF8}{mj}阶与\end{CJK} $m \times n$ \begin{CJK}{UTF8}{mj}矩阵\end{CJK}, \begin{CJK}{UTF8}{mj}下列结论正确的有\end{CJK} ( )\\
A. \begin{CJK}{UTF8}{mj}当\end{CJK} $n>m$ \begin{CJK}{UTF8}{mj}时\end{CJK}, $|A B|=0$;\\
B. \begin{CJK}{UTF8}{mj}当\end{CJK} $n<m$ \begin{CJK}{UTF8}{mj}时\end{CJK}, $A B$ \begin{CJK}{UTF8}{mj}可逆的充分必要条件是\end{CJK} $\mathrm{r}(A)=\mathrm{r}(B)=n$;\\
C. $\mathrm{r}(A B) \leq \min \{\mathrm{r}(A), \mathrm{r}(B)\}$;\\
D. \begin{CJK}{UTF8}{mj}如果\end{CJK} $A B$ \begin{CJK}{UTF8}{mj}可逆\end{CJK}, \begin{CJK}{UTF8}{mj}则\end{CJK} $B A$ \begin{CJK}{UTF8}{mj}可逆\end{CJK}.

(5) \begin{CJK}{UTF8}{mj}设\end{CJK} $A$ \begin{CJK}{UTF8}{mj}为\end{CJK} $n$ \begin{CJK}{UTF8}{mj}阶方阵\end{CJK}, \begin{CJK}{UTF8}{mj}下列结论正确的有\end{CJK}()\\
A. $A$ \begin{CJK}{UTF8}{mj}的行向量组线性相关的充分必要条件是\end{CJK} $|A|=0$;\\
B. \begin{CJK}{UTF8}{mj}线性方程组\end{CJK} $A X=b$ \begin{CJK}{UTF8}{mj}有无穷多组解的充分必要条件是\end{CJK} $|A|=0$;\\
C. $\left|A^{*}\right|=0$ \begin{CJK}{UTF8}{mj}的充分必要条件是\end{CJK} $|A|=0$;\\
D. \begin{CJK}{UTF8}{mj}以上结论都正确\end{CJK}. (6)\begin{CJK}{UTF8}{mj}设\end{CJK} $\mathscr{A}$ \begin{CJK}{UTF8}{mj}为\end{CJK} $n$ \begin{CJK}{UTF8}{mj}维线性空间\end{CJK} $V$ \begin{CJK}{UTF8}{mj}上线性变换\end{CJK}, \begin{CJK}{UTF8}{mj}则\end{CJK} ( )\\
A. $\mathscr{A}$ \begin{CJK}{UTF8}{mj}可逆的充分必要条件是\end{CJK} $\operatorname{Im} \mathscr{A}=V$;\\
B. $\operatorname{Im} \mathscr{A}+\operatorname{ker} \mathscr{A}=V$;\\
C. $\operatorname{dim} \operatorname{Im} \mathscr{A}+\operatorname{dim} \operatorname{ker} \mathscr{A}=n$;\\
D. $\operatorname{Im} \mathscr{A} \cap \operatorname{ker} \mathscr{A}=\{0\}$.

(7) \begin{CJK}{UTF8}{mj}设\end{CJK} $V$ \begin{CJK}{UTF8}{mj}为\end{CJK} $n$ \begin{CJK}{UTF8}{mj}维欧氏空间\end{CJK}, \begin{CJK}{UTF8}{mj}则\end{CJK} ( )\\
A. $A$ \begin{CJK}{UTF8}{mj}中存在非零正交向量组\end{CJK} $\alpha_{1}, \alpha_{2}, \cdots, \alpha_{n+1}$;\\
B. $V$ \begin{CJK}{UTF8}{mj}的任意一个基的度量矩阵是正定矩阵\end{CJK};\\
C. \begin{CJK}{UTF8}{mj}如果\end{CJK} $W$ \begin{CJK}{UTF8}{mj}是\end{CJK} $V$ \begin{CJK}{UTF8}{mj}的子空间\end{CJK}, \begin{CJK}{UTF8}{mj}则\end{CJK} $W$ \begin{CJK}{UTF8}{mj}的正交补\end{CJK} $W^{\perp}$ \begin{CJK}{UTF8}{mj}不唯一\end{CJK};\\
D. $V$ \begin{CJK}{UTF8}{mj}中标准正交基的过渡矩阵是正交阵\end{CJK}.

(8) \begin{CJK}{UTF8}{mj}设\end{CJK} $A$ \begin{CJK}{UTF8}{mj}为\end{CJK} $n$ \begin{CJK}{UTF8}{mj}阶实对称矩阵\end{CJK}, \begin{CJK}{UTF8}{mj}下列结论正确的有\end{CJK} ( )\\
A. $A$ \begin{CJK}{UTF8}{mj}的特征值都是实数\end{CJK};\\
B. $A$ \begin{CJK}{UTF8}{mj}的不同特征值下的实特征向量正交\end{CJK};\\
C. \begin{CJK}{UTF8}{mj}如果\end{CJK} $C$ \begin{CJK}{UTF8}{mj}是实可逆矩阵\end{CJK}, \begin{CJK}{UTF8}{mj}使\end{CJK} $C^{T} A C$ \begin{CJK}{UTF8}{mj}为对角矩阵\end{CJK}, \begin{CJK}{UTF8}{mj}则\end{CJK} $C^{-1} A C$ \begin{CJK}{UTF8}{mj}为对角矩阵\end{CJK};\\
D. \begin{CJK}{UTF8}{mj}与\end{CJK} $A$ \begin{CJK}{UTF8}{mj}合同的实矩阵\end{CJK} $B$ \begin{CJK}{UTF8}{mj}一定与\end{CJK} $A$ \begin{CJK}{UTF8}{mj}相似\end{CJK}.

(9) \begin{CJK}{UTF8}{mj}设\end{CJK} $\mathscr{A}$ \begin{CJK}{UTF8}{mj}为\end{CJK} $n$ \begin{CJK}{UTF8}{mj}维线性空间\end{CJK} $V$ \begin{CJK}{UTF8}{mj}上线性变换\end{CJK}, $W_{1}, W_{2}, \cdots, W_{n}$ \begin{CJK}{UTF8}{mj}为\end{CJK} $V$ \begin{CJK}{UTF8}{mj}的\end{CJK} 1 \begin{CJK}{UTF8}{mj}维子空间\end{CJK}, \begin{CJK}{UTF8}{mj}且为\end{CJK} $\mathscr{A}$ \begin{CJK}{UTF8}{mj}的不变子空间\end{CJK}, \begin{CJK}{UTF8}{mj}如\end{CJK} \begin{CJK}{UTF8}{mj}果\end{CJK} $V=W_{1}+W_{2}+\cdots+W_{n}$, \begin{CJK}{UTF8}{mj}则一定存在\end{CJK} $V$ \begin{CJK}{UTF8}{mj}中一个基\end{CJK}, \begin{CJK}{UTF8}{mj}使得\end{CJK} $\mathscr{A}$ \begin{CJK}{UTF8}{mj}在此基下矩阵为\end{CJK}()\\
A. \begin{CJK}{UTF8}{mj}对角矩阵\end{CJK};\\
B. \begin{CJK}{UTF8}{mj}反对称矩阵\end{CJK};\\
C. \begin{CJK}{UTF8}{mj}可逆矩阵\end{CJK};\\
D. \begin{CJK}{UTF8}{mj}对称矩阵\end{CJK}.

(10) \begin{CJK}{UTF8}{mj}设\end{CJK} $V$ \begin{CJK}{UTF8}{mj}为\end{CJK} $n$ \begin{CJK}{UTF8}{mj}维线性空间\end{CJK}, $\alpha_{1}, \alpha_{2}, \cdots, \alpha_{n} ; \beta_{1}, \beta_{2}, \cdots, \beta_{n} \in V$, \begin{CJK}{UTF8}{mj}且\end{CJK} $\left[\alpha_{1}, \alpha_{2}, \cdots, \alpha_{n}\right]=\left[\beta_{1}, \beta_{2}, \cdots, \beta_{n}\right] A$ \begin{CJK}{UTF8}{mj}其\end{CJK} \begin{CJK}{UTF8}{mj}中\end{CJK} $A$ \begin{CJK}{UTF8}{mj}为\end{CJK} $n$ \begin{CJK}{UTF8}{mj}阶方阵\end{CJK}, \begin{CJK}{UTF8}{mj}下列结论正确的有\end{CJK} ( )\\
A. \begin{CJK}{UTF8}{mj}当\end{CJK} $\alpha_{1}, \alpha_{2}, \cdots, \alpha_{n}$ \begin{CJK}{UTF8}{mj}为\end{CJK} $V$ \begin{CJK}{UTF8}{mj}的基时\end{CJK}, $\beta_{1}, \beta_{2}, \cdots, \beta_{n}$ \begin{CJK}{UTF8}{mj}为\end{CJK} $V$ \begin{CJK}{UTF8}{mj}的基\end{CJK};\\
B. \begin{CJK}{UTF8}{mj}当\end{CJK} $\beta_{1}, \beta_{2}, \cdots, \beta_{n}$ \begin{CJK}{UTF8}{mj}为\end{CJK} $V$ \begin{CJK}{UTF8}{mj}的基时\end{CJK}, $\alpha_{1}, \alpha_{2}, \cdots, \alpha_{n}$ \begin{CJK}{UTF8}{mj}为\end{CJK} $V$ \begin{CJK}{UTF8}{mj}的基\end{CJK};\\
C. \begin{CJK}{UTF8}{mj}当\end{CJK} $\beta_{1}, \beta_{2}, \cdots, \beta_{n}$ \begin{CJK}{UTF8}{mj}为\end{CJK} $V$ \begin{CJK}{UTF8}{mj}的基\end{CJK}, \begin{CJK}{UTF8}{mj}且\end{CJK} $A$ \begin{CJK}{UTF8}{mj}可逆时\end{CJK}, $\alpha_{1}, \alpha_{2}, \cdots, \alpha_{n}$ \begin{CJK}{UTF8}{mj}为\end{CJK} $V$ \begin{CJK}{UTF8}{mj}的基\end{CJK};\\
D. \begin{CJK}{UTF8}{mj}只有当\end{CJK} $\alpha_{1}, \alpha_{2}, \cdots, \alpha_{n}$ \begin{CJK}{UTF8}{mj}为\end{CJK} $V$ \begin{CJK}{UTF8}{mj}的基\end{CJK}, \begin{CJK}{UTF8}{mj}且\end{CJK} $A$ \begin{CJK}{UTF8}{mj}可逆时\end{CJK}, $\beta_{1}, \beta_{2}, \cdots, \beta_{n}$ \begin{CJK}{UTF8}{mj}为\end{CJK} $V$ \begin{CJK}{UTF8}{mj}的基\end{CJK}.

\begin{enumerate}
  \setcounter{enumi}{3}
  \item (15 \begin{CJK}{UTF8}{mj}分\end{CJK}) \begin{CJK}{UTF8}{mj}叙述并证明\end{CJK} Eisenstein \begin{CJK}{UTF8}{mj}判别法\end{CJK}.

  \item (15 \begin{CJK}{UTF8}{mj}分\end{CJK}) \begin{CJK}{UTF8}{mj}设矩阵\end{CJK} $A=\left[\alpha_{1}+\beta_{1}, \alpha_{2}+\beta_{2}, \cdots, \alpha_{n}+\beta_{n}\right], \operatorname{rank}\left(\alpha_{1}, \cdots, \alpha_{n}\right)=1$, \begin{CJK}{UTF8}{mj}且\end{CJK} $\alpha_{i}, \beta_{i}(i=1,2, \cdots, n)$ \begin{CJK}{UTF8}{mj}为\end{CJK} $n$ \begin{CJK}{UTF8}{mj}维\end{CJK} \begin{CJK}{UTF8}{mj}列向量\end{CJK}.

\end{enumerate}
(1) \begin{CJK}{UTF8}{mj}求证\end{CJK}:
$$
|A|=\left|\left(\beta_{1}, \cdots, \beta_{n}\right)\right|+\sum_{i=1}^{n}\left|\left(\beta_{1}, \cdots, \beta_{i-1}, \alpha_{i}, \beta_{i+1}, \cdots, \beta_{n}\right)\right|
$$
(2) \begin{CJK}{UTF8}{mj}利用\end{CJK} (1) \begin{CJK}{UTF8}{mj}计算行列式\end{CJK}
$$
\left|\begin{array}{ccccc}
x_{1} & a & a & \cdots & a \\
a & x_{2} & a & \cdots & a \\
a & a & x_{3} & \cdots & a \\
\vdots & \vdots & \vdots & & \vdots \\
a & a & a & \cdots & x_{n}
\end{array}\right|
$$

\begin{enumerate}
  \setcounter{enumi}{5}
  \item ( 10 \begin{CJK}{UTF8}{mj}分\end{CJK}) \begin{CJK}{UTF8}{mj}设\end{CJK} $\mathbb{F}^{4}$ \begin{CJK}{UTF8}{mj}上两个线性空间\end{CJK}
\end{enumerate}
$$
\begin{gathered}
W_{1}=\left\{\left[x_{1}, x_{2}, x_{3}, x_{4}\right] \mid x_{1}+x_{2}+2 x_{3}+2 x_{4}=0, x_{1}+2 x_{2}+3 x_{3}+3 x_{4}=0, x_{i} \in \mathbb{F}\right\}, \\
W_{2}=\left\{\left[x_{1}, x_{2}, x_{3}, x_{4}\right] \mid 3 x_{1}+4 x_{2}+5 x_{3}+x_{4}=0, x_{i} \in \mathbb{F}\right\}
\end{gathered}
$$
\begin{CJK}{UTF8}{mj}求\end{CJK} $W_{1} \cap W_{2}$ \begin{CJK}{UTF8}{mj}和\end{CJK} $W_{1}+W_{2}$ \begin{CJK}{UTF8}{mj}的基与维数\end{CJK}. 6. (15 \begin{CJK}{UTF8}{mj}分\end{CJK}) \begin{CJK}{UTF8}{mj}设\end{CJK} $\mathbb{F}^{2 \times 2}$ \begin{CJK}{UTF8}{mj}的两组基分别为\end{CJK}
$$
\begin{aligned}
&\alpha_{1}=\left[\begin{array}{ll}
1 & 0 \\
0 & 0
\end{array}\right], \alpha_{2}=\left[\begin{array}{ll}
0 & 1 \\
0 & 0
\end{array}\right], \alpha_{3}=\left[\begin{array}{ll}
0 & 0 \\
1 & 0
\end{array}\right], \alpha_{4}=\left[\begin{array}{ll}
0 & 0 \\
0 & 1
\end{array}\right] \\
&\beta_{1}=\left[\begin{array}{ll}
1 & 1 \\
1 & 1
\end{array}\right], \beta_{2}=\left[\begin{array}{ll}
1 & 2 \\
2 & 4
\end{array}\right], \beta_{3}=\left[\begin{array}{ll}
2 & 3 \\
4 & 6
\end{array}\right], \beta_{4}=\left[\begin{array}{ll}
3 & 4 \\
4 & 7
\end{array}\right]
\end{aligned}
$$
(1) \begin{CJK}{UTF8}{mj}求基\end{CJK} $\beta_{1}, \beta_{2}, \beta_{3}, \beta_{4}$ \begin{CJK}{UTF8}{mj}到基\end{CJK} $\alpha_{1}, \alpha_{2}, \alpha_{3}, \alpha_{4}$ \begin{CJK}{UTF8}{mj}的过渡矩阵\end{CJK};

(2) \begin{CJK}{UTF8}{mj}设\end{CJK} $\gamma=\left[\begin{array}{ll}a & b \\ c & d\end{array}\right]$, \begin{CJK}{UTF8}{mj}分别求\end{CJK} $\gamma$ \begin{CJK}{UTF8}{mj}在这两个基下坐标向量\end{CJK}.

\begin{enumerate}
  \setcounter{enumi}{7}
  \item (15 \begin{CJK}{UTF8}{mj}分\end{CJK}) \begin{CJK}{UTF8}{mj}设\end{CJK}
\end{enumerate}
$$
A=\left[\begin{array}{ccc}
10 & 7 & 7 \\
7 & 10 & 7 \\
7 & 7 & 10
\end{array}\right]
$$
\begin{CJK}{UTF8}{mj}求\end{CJK} $A$ \begin{CJK}{UTF8}{mj}的特征值与特征向量\end{CJK}, \begin{CJK}{UTF8}{mj}并求正定矩阵\end{CJK} $B$ \begin{CJK}{UTF8}{mj}使得\end{CJK} $B^{2}-I=A$.

\begin{enumerate}
  \setcounter{enumi}{8}
  \item (10 \begin{CJK}{UTF8}{mj}分\end{CJK}) \begin{CJK}{UTF8}{mj}设\end{CJK} $A, B$ \begin{CJK}{UTF8}{mj}为\end{CJK} $n$ \begin{CJK}{UTF8}{mj}阶复方阵\end{CJK}, \begin{CJK}{UTF8}{mj}且\end{CJK} $A B-B A=A$.
\end{enumerate}
(1) \begin{CJK}{UTF8}{mj}求证\end{CJK} $\operatorname{tr}(A)=0$;

(2) \begin{CJK}{UTF8}{mj}如果\end{CJK} $n=2$, \begin{CJK}{UTF8}{mj}求证\end{CJK} $A^{2}=0$.

\begin{enumerate}
  \setcounter{enumi}{9}
  \item ( 10 \begin{CJK}{UTF8}{mj}分\end{CJK}) \begin{CJK}{UTF8}{mj}设\end{CJK} $V$ \begin{CJK}{UTF8}{mj}为\end{CJK} $n$ \begin{CJK}{UTF8}{mj}维欧氏空间\end{CJK}, $\mathscr{A}, \mathscr{B}$ \begin{CJK}{UTF8}{mj}为\end{CJK} $V$ \begin{CJK}{UTF8}{mj}到\end{CJK} $V$ \begin{CJK}{UTF8}{mj}的映射\end{CJK}, $k$ \begin{CJK}{UTF8}{mj}为固定非零实数\end{CJK}, \begin{CJK}{UTF8}{mj}且\end{CJK} $\forall \alpha, \beta \in V$ \begin{CJK}{UTF8}{mj}有\end{CJK}
\end{enumerate}
$$
(\mathscr{A} \alpha, \beta)=k(\alpha, \mathscr{B} \beta)
$$
(1) \begin{CJK}{UTF8}{mj}求证\end{CJK} $\mathscr{A}, \mathscr{B}$ \begin{CJK}{UTF8}{mj}为\end{CJK} $V$ \begin{CJK}{UTF8}{mj}上线性变换\end{CJK};

(2) \begin{CJK}{UTF8}{mj}如果\end{CJK} $\mathscr{A}=\mathscr{B}$, \begin{CJK}{UTF8}{mj}且\end{CJK} $\mathscr{A}$ \begin{CJK}{UTF8}{mj}为非零映射\end{CJK}, \begin{CJK}{UTF8}{mj}求证\end{CJK} $k=1$ \begin{CJK}{UTF8}{mj}或者\end{CJK} $k=-1$.

\begin{enumerate}
  \setcounter{enumi}{10}
  \item (10 \begin{CJK}{UTF8}{mj}分\end{CJK}) \begin{CJK}{UTF8}{mj}设\end{CJK} $A$ \begin{CJK}{UTF8}{mj}为\end{CJK} $n$ \begin{CJK}{UTF8}{mj}阶实对称矩阵\end{CJK}, $\alpha=\left[a_{1}, a_{2}, \cdots, a_{n}\right]^{T}$ \begin{CJK}{UTF8}{mj}为\end{CJK} $n$ \begin{CJK}{UTF8}{mj}维实列向量\end{CJK}.
\end{enumerate}
(1) \begin{CJK}{UTF8}{mj}证明\end{CJK} $I-\alpha \alpha^{T}$ \begin{CJK}{UTF8}{mj}的特征值为\end{CJK} $1\left(n-1\right.$ \begin{CJK}{UTF8}{mj}重\end{CJK}) \begin{CJK}{UTF8}{mj}与\end{CJK} $1-\sum_{i=1}^{n} a_{i}^{2}$;

(2) \begin{CJK}{UTF8}{mj}证明\end{CJK} $A-\alpha \alpha^{T}$ \begin{CJK}{UTF8}{mj}正定的充分必要条件是\end{CJK} $A$ \begin{CJK}{UTF8}{mj}正定\end{CJK}, \begin{CJK}{UTF8}{mj}而且\end{CJK} $\alpha^{T} A^{-1} \alpha<1$.

\section{3. 上海大学 2011 年研究生入学考试试题高等代数 
 李扬 
 微信公众号: sxkyliyang}
\begin{enumerate}
  \item (\begin{CJK}{UTF8}{mj}每题\end{CJK} 4 \begin{CJK}{UTF8}{mj}分\end{CJK}, \begin{CJK}{UTF8}{mj}共\end{CJK} 20 \begin{CJK}{UTF8}{mj}分\end{CJK})\begin{CJK}{UTF8}{mj}填空题\end{CJK}
\end{enumerate}
(1) \begin{CJK}{UTF8}{mj}多项式\end{CJK} $f(x)=x^{3}-2 x-4$ \begin{CJK}{UTF8}{mj}的有理根是\end{CJK}

(2) \begin{CJK}{UTF8}{mj}设\end{CJK} $A$ \begin{CJK}{UTF8}{mj}是\end{CJK} $m \times 4$ \begin{CJK}{UTF8}{mj}矩阵\end{CJK}, \begin{CJK}{UTF8}{mj}且\end{CJK} $A$ \begin{CJK}{UTF8}{mj}中有个三列向量线性无关\end{CJK}, \begin{CJK}{UTF8}{mj}如果线性方程组\end{CJK} $A X=b$ \begin{CJK}{UTF8}{mj}有解\end{CJK} $\alpha=[1,2,3,4]^{T}$, $\beta=[1,1,1,1]^{T}$, \begin{CJK}{UTF8}{mj}则\end{CJK} $A X=b$ \begin{CJK}{UTF8}{mj}的通解是\end{CJK}

(3) \begin{CJK}{UTF8}{mj}设同阶矩阵\end{CJK} $A, B$ \begin{CJK}{UTF8}{mj}中元素都是整数\end{CJK}, \begin{CJK}{UTF8}{mj}如果\end{CJK} $A^{2} B+A=I$, \begin{CJK}{UTF8}{mj}则\end{CJK} $(\operatorname{det} A)^{2}=$

(4) \begin{CJK}{UTF8}{mj}四维线性空间\end{CJK} $V$ \begin{CJK}{UTF8}{mj}上线性变换\end{CJK} $\mathscr{A}$ \begin{CJK}{UTF8}{mj}的最小多项式是\end{CJK} $x(x-1)$, \begin{CJK}{UTF8}{mj}值域维数是\end{CJK} 2 , \begin{CJK}{UTF8}{mj}则存在\end{CJK} $V$ \begin{CJK}{UTF8}{mj}上的一组基\end{CJK}, \begin{CJK}{UTF8}{mj}使得\end{CJK} $\mathscr{A}$ \begin{CJK}{UTF8}{mj}在此组基下矩阵是对角阵\end{CJK} $A=$

(5) \begin{CJK}{UTF8}{mj}设\end{CJK} $A=\left[\begin{array}{lll}0 & 1 & 0 \\ 0 & 0 & 1 \\ 0 & 0 & 0\end{array}\right], V=\left\{B \in \mathbb{F}^{3 \times 3} \mid A B=B A\right\}$, \begin{CJK}{UTF8}{mj}则\end{CJK} $V$ \begin{CJK}{UTF8}{mj}为\end{CJK} $\mathbb{F}$ \begin{CJK}{UTF8}{mj}上\end{CJK} \begin{CJK}{UTF8}{mj}维线性空间\end{CJK}, \begin{CJK}{UTF8}{mj}基为\end{CJK}

\begin{enumerate}
  \setcounter{enumi}{2}
  \item (\begin{CJK}{UTF8}{mj}每题\end{CJK} 5 \begin{CJK}{UTF8}{mj}分\end{CJK}, \begin{CJK}{UTF8}{mj}共\end{CJK} 20 \begin{CJK}{UTF8}{mj}分\end{CJK}) \begin{CJK}{UTF8}{mj}是非题\end{CJK} (\begin{CJK}{UTF8}{mj}正确的在括号内添\end{CJK}“\begin{CJK}{UTF8}{mj}正确\end{CJK}”, \begin{CJK}{UTF8}{mj}错误的在括号内添\end{CJK} “\begin{CJK}{UTF8}{mj}错误\end{CJK}”, \begin{CJK}{UTF8}{mj}如果不正确\end{CJK}, \begin{CJK}{UTF8}{mj}请举出反\end{CJK} \begin{CJK}{UTF8}{mj}例\end{CJK}.)
\end{enumerate}
(1) \begin{CJK}{UTF8}{mj}设\end{CJK} $A, B$ \begin{CJK}{UTF8}{mj}是\end{CJK} $n$ \begin{CJK}{UTF8}{mj}阶矩阵\end{CJK}, \begin{CJK}{UTF8}{mj}则\end{CJK} $A B=B A$. ( )

(2) \begin{CJK}{UTF8}{mj}设有理系数多项式\end{CJK} $f(x)$ \begin{CJK}{UTF8}{mj}在有理数域上无重根\end{CJK}, \begin{CJK}{UTF8}{mj}则\end{CJK} $f(x)$ \begin{CJK}{UTF8}{mj}在有理数域上无重因式\end{CJK}. ( )

(3) \begin{CJK}{UTF8}{mj}设\end{CJK} $\mathscr{A}$ \begin{CJK}{UTF8}{mj}是\end{CJK} $n$ \begin{CJK}{UTF8}{mj}维线性空间\end{CJK} $V$ \begin{CJK}{UTF8}{mj}上的线性变换\end{CJK}, \begin{CJK}{UTF8}{mj}则其核\end{CJK} $\operatorname{ker} \mathscr{A}$ \begin{CJK}{UTF8}{mj}与值域\end{CJK} $\operatorname{Im} \mathscr{A}$ \begin{CJK}{UTF8}{mj}满足\end{CJK} $\operatorname{ker} \mathscr{A} \oplus \operatorname{Im} \mathscr{A}=V$. ( )

(4) \begin{CJK}{UTF8}{mj}设\end{CJK} $A, B$ \begin{CJK}{UTF8}{mj}是\end{CJK} $n$ \begin{CJK}{UTF8}{mj}阶实矩阵\end{CJK}, \begin{CJK}{UTF8}{mj}如果\end{CJK} $A, B$ \begin{CJK}{UTF8}{mj}相似\end{CJK}, \begin{CJK}{UTF8}{mj}则\end{CJK} $A, B$ \begin{CJK}{UTF8}{mj}合同\end{CJK}. ( )

\begin{enumerate}
  \setcounter{enumi}{3}
  \item (20 \begin{CJK}{UTF8}{mj}分\end{CJK}) \begin{CJK}{UTF8}{mj}设\end{CJK} $A, B$ \begin{CJK}{UTF8}{mj}是\end{CJK} $n$ \begin{CJK}{UTF8}{mj}阶方阵\end{CJK}, $A+B, A-B$ \begin{CJK}{UTF8}{mj}可逆\end{CJK}, \begin{CJK}{UTF8}{mj}求证\end{CJK}
\end{enumerate}
$$
C=\left[\begin{array}{ll}
A & B \\
B & A
\end{array}\right]
$$
\begin{CJK}{UTF8}{mj}可逆\end{CJK}, \begin{CJK}{UTF8}{mj}进一步如果\end{CJK}
$$
A=\left[\begin{array}{ll}
1 & 0 \\
0 & 1
\end{array}\right], \quad B=\left[\begin{array}{ll}
2 & 0 \\
0 & 2
\end{array}\right]
$$
\begin{CJK}{UTF8}{mj}求\end{CJK} $C^{-1}$.

\begin{enumerate}
  \setcounter{enumi}{4}
  \item (20 \begin{CJK}{UTF8}{mj}分\end{CJK}) (1) \begin{CJK}{UTF8}{mj}设\end{CJK} $X, Y \in \mathbb{F}^{n}, A \in \mathbb{F}^{n \times n}$, \begin{CJK}{UTF8}{mj}求证\end{CJK}
\end{enumerate}
$$
\operatorname{det}\left(A+X Y^{T}\right)=\operatorname{det} A+Y^{T} A^{*} X
$$
(2) \begin{CJK}{UTF8}{mj}利用\end{CJK} (1) \begin{CJK}{UTF8}{mj}的结论证明\end{CJK}: \begin{CJK}{UTF8}{mj}如果\end{CJK} $n$ \begin{CJK}{UTF8}{mj}阶矩阵\end{CJK} $A$ \begin{CJK}{UTF8}{mj}的行列式为\end{CJK} $1, \operatorname{det}(A+J)=2$, \begin{CJK}{UTF8}{mj}其中\end{CJK} $J$ \begin{CJK}{UTF8}{mj}为\end{CJK} $n$ \begin{CJK}{UTF8}{mj}阶矩阵\end{CJK}, \begin{CJK}{UTF8}{mj}且矩阵中元\end{CJK} \begin{CJK}{UTF8}{mj}素都是\end{CJK} 1 , \begin{CJK}{UTF8}{mj}则\end{CJK} $A^{*}$ \begin{CJK}{UTF8}{mj}所有元素之和为\end{CJK} 1 .

\begin{enumerate}
  \setcounter{enumi}{5}
  \item (15 \begin{CJK}{UTF8}{mj}分\end{CJK}) \begin{CJK}{UTF8}{mj}设\end{CJK} $\varepsilon_{1}, \varepsilon_{2}, \varepsilon_{3}, \varepsilon_{4}$ \begin{CJK}{UTF8}{mj}是\end{CJK} 4 \begin{CJK}{UTF8}{mj}维线性空间\end{CJK} $V$ \begin{CJK}{UTF8}{mj}的一组基\end{CJK}, \begin{CJK}{UTF8}{mj}一线性变换\end{CJK} $\mathscr{A}$ \begin{CJK}{UTF8}{mj}在这组基下的矩阵为\end{CJK}
\end{enumerate}
$$
\left[\begin{array}{cccc}
1 & 0 & 2 & 1 \\
-1 & 2 & 1 & 3 \\
2 & 2 & 7 & 6 \\
2 & -2 & 1 & -2
\end{array}\right]
$$
\begin{CJK}{UTF8}{mj}求\end{CJK} $\mathscr{A}$ \begin{CJK}{UTF8}{mj}的核的基与值域的基\end{CJK}. 6. ( 15 \begin{CJK}{UTF8}{mj}分\end{CJK}) \begin{CJK}{UTF8}{mj}设\end{CJK} $A$ \begin{CJK}{UTF8}{mj}是\end{CJK} 3 \begin{CJK}{UTF8}{mj}阶实对称矩阵\end{CJK}, \begin{CJK}{UTF8}{mj}而且\end{CJK} $\operatorname{det}(A)=4$, \begin{CJK}{UTF8}{mj}特征值为\end{CJK} $1,1, \lambda$, \begin{CJK}{UTF8}{mj}如果\end{CJK} $\left[\begin{array}{c}1 \\ -1 \\ 0\end{array}\right],\left[\begin{array}{c}1 \\ 0 \\ -1\end{array}\right]$ \begin{CJK}{UTF8}{mj}为\end{CJK} $A$ \begin{CJK}{UTF8}{mj}特征向量\end{CJK}, \begin{CJK}{UTF8}{mj}求\end{CJK} $A$.

\begin{enumerate}
  \setcounter{enumi}{7}
  \item ( 10 \begin{CJK}{UTF8}{mj}分\end{CJK}) \begin{CJK}{UTF8}{mj}设\end{CJK}
\end{enumerate}
$$
\begin{gathered}
f(x)=\sum_{t=0}^{n} a_{t} x^{t} \\
g(x)=\sum_{t=0}^{n} a_{n-t} x^{t} \in \mathbb{F}[x] .
\end{gathered}
$$
\begin{CJK}{UTF8}{mj}求证\end{CJK} $f(x)$ \begin{CJK}{UTF8}{mj}不可约当且仅当\end{CJK} $g(x)$ \begin{CJK}{UTF8}{mj}不可约\end{CJK}. \begin{CJK}{UTF8}{mj}利用此结论说明\end{CJK} $2 x^{n}+2 x^{n-1}+\cdots+2 x+1$ \begin{CJK}{UTF8}{mj}在有理数域上不可约\end{CJK}.

\begin{enumerate}
  \setcounter{enumi}{8}
  \item ( 10 \begin{CJK}{UTF8}{mj}分\end{CJK}) \begin{CJK}{UTF8}{mj}设\end{CJK} $V$ \begin{CJK}{UTF8}{mj}是\end{CJK} $n$ \begin{CJK}{UTF8}{mj}维欧式空间\end{CJK}, $e_{1}, e_{2}, \cdots, e_{n}$ \begin{CJK}{UTF8}{mj}是\end{CJK} $V$ \begin{CJK}{UTF8}{mj}的一组基\end{CJK}, \begin{CJK}{UTF8}{mj}求证\end{CJK} $e_{1}, e_{2}, \cdots, e_{n}$ \begin{CJK}{UTF8}{mj}的度量矩阵\end{CJK} $A$ \begin{CJK}{UTF8}{mj}是正定矩\end{CJK} \begin{CJK}{UTF8}{mj}阵\end{CJK}.

  \item ( 10 \begin{CJK}{UTF8}{mj}分\end{CJK}) \begin{CJK}{UTF8}{mj}设\end{CJK} $A, B$ \begin{CJK}{UTF8}{mj}是数域\end{CJK} $\mathbb{F}$ \begin{CJK}{UTF8}{mj}上\end{CJK} $n$ \begin{CJK}{UTF8}{mj}阶方阵\end{CJK}, \begin{CJK}{UTF8}{mj}且\end{CJK} $A B=0, A+B$ \begin{CJK}{UTF8}{mj}可逆\end{CJK}. \begin{CJK}{UTF8}{mj}如果\end{CJK} $V=\left\{\alpha \in \mathbb{F}^{n} \mid A \alpha=0\right\}, W=$ $\{A \alpha \mid \alpha \in \mathbb{F}\}$, \begin{CJK}{UTF8}{mj}求证\end{CJK}:

\end{enumerate}
$$
V \oplus W=\mathbb{F}^{n}
$$

\begin{enumerate}
  \setcounter{enumi}{10}
  \item (10 \begin{CJK}{UTF8}{mj}分\end{CJK}) \begin{CJK}{UTF8}{mj}设\end{CJK} $A, B$ \begin{CJK}{UTF8}{mj}分别为\end{CJK} $3 \times 2,2 \times 3$ \begin{CJK}{UTF8}{mj}实矩阵\end{CJK}, \begin{CJK}{UTF8}{mj}且\end{CJK}
\end{enumerate}
$$
A B=\left[\begin{array}{ccc}
1 & 1 & 1 \\
-2 & 0 & -6 \\
0 & 1 & -2
\end{array}\right]
$$
\begin{CJK}{UTF8}{mj}求证\end{CJK}: $B A$ \begin{CJK}{UTF8}{mj}与矩阵\end{CJK} $\left[\begin{array}{ll}0 & -6 \\ 1 & -1\end{array}\right]$ \begin{CJK}{UTF8}{mj}在复数域上相似\end{CJK}, \begin{CJK}{UTF8}{mj}进一步问\end{CJK} $B A$ \begin{CJK}{UTF8}{mj}与矩阵\end{CJK} $\left[\begin{array}{cc}0 & -6 \\ 1 & -1\end{array}\right]$ \begin{CJK}{UTF8}{mj}在实数域上相似吗\end{CJK}?

\section{4. 上海大学 2012 年研究生入学考试试题高等代数 $(\mathrm{A})$ 
 李扬 
 微信公众号: sxkyliyang}
\begin{enumerate}
  \item \begin{CJK}{UTF8}{mj}试证\end{CJK}:\begin{CJK}{UTF8}{mj}设向量\end{CJK} (1) $\alpha_{1}, \alpha_{2}, \cdots, \alpha_{r}$ \begin{CJK}{UTF8}{mj}线性无关\end{CJK}, \begin{CJK}{UTF8}{mj}并且可由向量组\end{CJK} $(2) \beta_{1}, \beta_{2}, \cdots, \beta_{s}$ \begin{CJK}{UTF8}{mj}线性表出\end{CJK}, \begin{CJK}{UTF8}{mj}则\end{CJK}
\end{enumerate}
(1) $r \leq s$

(2) \begin{CJK}{UTF8}{mj}适当的排列向量组\end{CJK} (2) \begin{CJK}{UTF8}{mj}中向量的次序\end{CJK}, \begin{CJK}{UTF8}{mj}使\end{CJK} (1) \begin{CJK}{UTF8}{mj}替换\end{CJK} (2) \begin{CJK}{UTF8}{mj}中前\end{CJK} $r$ \begin{CJK}{UTF8}{mj}个向量后得到的向量组\end{CJK} (3) $\alpha_{1}, \alpha_{2}, \cdots, \alpha_{r}, \beta_{r+1}, \cdots, \beta_{s}$ \begin{CJK}{UTF8}{mj}与向量组\end{CJK} (2) \begin{CJK}{UTF8}{mj}等价\end{CJK}.

\begin{enumerate}
  \setcounter{enumi}{2}
  \item (1) \begin{CJK}{UTF8}{mj}设\end{CJK} $A, B, C, D$ \begin{CJK}{UTF8}{mj}都是\end{CJK} $n$ \begin{CJK}{UTF8}{mj}阶方阵\end{CJK}, \begin{CJK}{UTF8}{mj}且\end{CJK} $|A| \neq 0, A C=C A$, \begin{CJK}{UTF8}{mj}试证明\end{CJK}:
\end{enumerate}
$$
\left|\begin{array}{ll}
A & B \\
C & D
\end{array}\right|=|A D-C B| .
$$
(2) \begin{CJK}{UTF8}{mj}试证明\end{CJK}: \begin{CJK}{UTF8}{mj}如果两个\end{CJK} $r$ \begin{CJK}{UTF8}{mj}阶矩阵\end{CJK} $A$ \begin{CJK}{UTF8}{mj}和\end{CJK} $C$ \begin{CJK}{UTF8}{mj}的行向量组分别构成同一个齐次线性方程组的基础解系\end{CJK}, \begin{CJK}{UTF8}{mj}则必定存在\end{CJK} \begin{CJK}{UTF8}{mj}一个\end{CJK} $r$ \begin{CJK}{UTF8}{mj}阶满秩矩阵\end{CJK} $B$, \begin{CJK}{UTF8}{mj}使得\end{CJK}
$$
A=B C
$$

\begin{enumerate}
  \setcounter{enumi}{3}
  \item (1) \begin{CJK}{UTF8}{mj}设\end{CJK} $A$ \begin{CJK}{UTF8}{mj}是\end{CJK} $n$ \begin{CJK}{UTF8}{mj}阶方阵\end{CJK}, $n \geq 3$, \begin{CJK}{UTF8}{mj}试证明\end{CJK}:
\end{enumerate}
$$
\left(A^{*}\right)^{*}=|A|^{n-2} A .
$$
(2) \begin{CJK}{UTF8}{mj}设\end{CJK} $A$ \begin{CJK}{UTF8}{mj}是一个\end{CJK} $n \times n$ \begin{CJK}{UTF8}{mj}矩阵\end{CJK}, $A^{2}=A$, \begin{CJK}{UTF8}{mj}试证\end{CJK}: $A$ \begin{CJK}{UTF8}{mj}相似与一个对角矩阵\end{CJK}
$$
\left(\begin{array}{llllll}
1 & & & & & \\
& \ddots & & & & \\
& & 1 & & & \\
& & & 0 & & \\
& & & & \ddots & \\
& & & & & 0
\end{array}\right) \text {. }
$$

\begin{enumerate}
  \setcounter{enumi}{4}
  \item \begin{CJK}{UTF8}{mj}试证\end{CJK}: \begin{CJK}{UTF8}{mj}如果已知既约分数\end{CJK} $\frac{p}{q}$ \begin{CJK}{UTF8}{mj}是整系数多项式\end{CJK} $f(x)$ \begin{CJK}{UTF8}{mj}的根\end{CJK}, \begin{CJK}{UTF8}{mj}则\end{CJK} $(q-p)|f(1),(q+p)| f(-1)$, \begin{CJK}{UTF8}{mj}且对任意整数\end{CJK} $m$, \begin{CJK}{UTF8}{mj}有\end{CJK}
\end{enumerate}
$$
(m q-p) \mid f(m)
$$

\begin{enumerate}
  \setcounter{enumi}{5}
  \item \begin{CJK}{UTF8}{mj}设\end{CJK} $A$ \begin{CJK}{UTF8}{mj}是\end{CJK} $n$ \begin{CJK}{UTF8}{mj}阶实对称矩阵\end{CJK}, \begin{CJK}{UTF8}{mj}且\end{CJK} $A^{2}=E$, \begin{CJK}{UTF8}{mj}试证明\end{CJK}: \begin{CJK}{UTF8}{mj}存在正交矩阵\end{CJK} $T$, \begin{CJK}{UTF8}{mj}使得\end{CJK}
\end{enumerate}
$$
T^{-1} A T=\left(\begin{array}{cc}
E_{r} & 0 \\
0 & -E_{n-r}
\end{array}\right)
$$

\section{5. 上海大学 2012 年研究生入学考试试题高等代数 $(B)$ 
 李扬 
 微信公众号: sxkyliyang}
\begin{enumerate}
  \item \begin{CJK}{UTF8}{mj}填空题\end{CJK}
\end{enumerate}
(1) \begin{CJK}{UTF8}{mj}多项式\end{CJK} $f(x)=x^{3}-2 x-4$ \begin{CJK}{UTF8}{mj}与\end{CJK} $g(x)=x^{3}+x^{2}-2$ \begin{CJK}{UTF8}{mj}的最大公因式是\end{CJK}:

(2) \begin{CJK}{UTF8}{mj}设\end{CJK} $A$ \begin{CJK}{UTF8}{mj}为\end{CJK} $n$ \begin{CJK}{UTF8}{mj}阶非可逆矩阵\end{CJK}, \begin{CJK}{UTF8}{mj}且\end{CJK} $A^{*}$ \begin{CJK}{UTF8}{mj}的第一列向量为\end{CJK} $\alpha \neq 0$, \begin{CJK}{UTF8}{mj}如果线性方程组\end{CJK} $A x=b$ \begin{CJK}{UTF8}{mj}有解\end{CJK} $\beta$, \begin{CJK}{UTF8}{mj}则线性方程组\end{CJK} $A x=b$ \begin{CJK}{UTF8}{mj}的通解为\end{CJK}:

(3) \begin{CJK}{UTF8}{mj}设三阶实对称矩阵\end{CJK} $A$ \begin{CJK}{UTF8}{mj}的特征值分别为\end{CJK} $a, a, b(a \neq b)$, \begin{CJK}{UTF8}{mj}如果\end{CJK} $(1,1,1),(1,0,1)$ \begin{CJK}{UTF8}{mj}为\end{CJK} $A$ \begin{CJK}{UTF8}{mj}的对应于特征值\end{CJK} $a$ \begin{CJK}{UTF8}{mj}的\end{CJK} \begin{CJK}{UTF8}{mj}特征向量\end{CJK}, \begin{CJK}{UTF8}{mj}则矩阵\end{CJK} $A$ \begin{CJK}{UTF8}{mj}对应于特征值\end{CJK} $b$ \begin{CJK}{UTF8}{mj}的特征向量为\end{CJK}:

(4) \begin{CJK}{UTF8}{mj}五维复线性空间\end{CJK} $V$ \begin{CJK}{UTF8}{mj}上的线性变换\end{CJK} $\mathscr{A}$ \begin{CJK}{UTF8}{mj}的最小多项式为\end{CJK} $x(x-1)^{2}$, \begin{CJK}{UTF8}{mj}值域维数为\end{CJK} 4 , \begin{CJK}{UTF8}{mj}则存在\end{CJK} $V$ \begin{CJK}{UTF8}{mj}的一组基\end{CJK}, \begin{CJK}{UTF8}{mj}使\end{CJK} \begin{CJK}{UTF8}{mj}得\end{CJK} $\mathscr{A}$ \begin{CJK}{UTF8}{mj}在此组基下的矩阵是\end{CJK} Jordan \begin{CJK}{UTF8}{mj}矩阵为\end{CJK}:

(5) \begin{CJK}{UTF8}{mj}设\end{CJK} $A=\left(\begin{array}{lll}1 & 1 & 0 \\ 0 & 1 & 1 \\ 0 & 0 & 1\end{array}\right), V=\{f(A) \mid f(x) \in \mathbb{F}(x)\}$, \begin{CJK}{UTF8}{mj}则\end{CJK} $V$ \begin{CJK}{UTF8}{mj}为\end{CJK} $\mathbb{F}$ \begin{CJK}{UTF8}{mj}上\end{CJK} \begin{CJK}{UTF8}{mj}维线性空间\end{CJK}, \begin{CJK}{UTF8}{mj}基为\end{CJK}:

\begin{enumerate}
  \setcounter{enumi}{2}
  \item \begin{CJK}{UTF8}{mj}是非题\end{CJK} (\begin{CJK}{UTF8}{mj}正确的在括号内添\end{CJK} “\begin{CJK}{UTF8}{mj}正确\end{CJK}”, \begin{CJK}{UTF8}{mj}错误的在括号内添\end{CJK} “\begin{CJK}{UTF8}{mj}错误\end{CJK}”, \begin{CJK}{UTF8}{mj}如果不正确\end{CJK}, \begin{CJK}{UTF8}{mj}请举出反例\end{CJK}.)
\end{enumerate}
(1) \begin{CJK}{UTF8}{mj}存在\end{CJK} $n$ \begin{CJK}{UTF8}{mj}阶矩阵\end{CJK} $A, B$, \begin{CJK}{UTF8}{mj}使得\end{CJK} $A B-B A=I$. ( )

(2) \begin{CJK}{UTF8}{mj}设数域\end{CJK} $\mathbb{F}$ \begin{CJK}{UTF8}{mj}上多项式\end{CJK} $f(x)|h(x), g(x)| h(x)$, \begin{CJK}{UTF8}{mj}则\end{CJK} $f(x) g(x) \mid h(x)$. ( )

(3) \begin{CJK}{UTF8}{mj}设\end{CJK} $\mathscr{A}$ \begin{CJK}{UTF8}{mj}是\end{CJK} $n$ \begin{CJK}{UTF8}{mj}维线性空间\end{CJK} $V$ \begin{CJK}{UTF8}{mj}上线性变换\end{CJK}, \begin{CJK}{UTF8}{mj}其核\end{CJK} ker $\mathscr{A}$ \begin{CJK}{UTF8}{mj}与值域\end{CJK} $\operatorname{Im} \mathscr{A}$, \begin{CJK}{UTF8}{mj}必定满足\end{CJK} $\operatorname{ker} \mathscr{A} \cap \operatorname{Im} \mathscr{A}=\{0\}$. ( )

(4) \begin{CJK}{UTF8}{mj}设\end{CJK} $A, B$ \begin{CJK}{UTF8}{mj}为非零矩阵\end{CJK}, \begin{CJK}{UTF8}{mj}且齐次线性方程组\end{CJK} $A x=0$, \begin{CJK}{UTF8}{mj}与\end{CJK} $B x=0$ \begin{CJK}{UTF8}{mj}同解\end{CJK}, \begin{CJK}{UTF8}{mj}则\end{CJK} $A, B$ \begin{CJK}{UTF8}{mj}的行向量组等价\end{CJK}. ( )

(5) \begin{CJK}{UTF8}{mj}设\end{CJK} $A, B$ \begin{CJK}{UTF8}{mj}为同阶复矩阵\end{CJK}, \begin{CJK}{UTF8}{mj}则\end{CJK} $A, B$ \begin{CJK}{UTF8}{mj}在负复数域上可以同时对角化\end{CJK}. ( )

\begin{enumerate}
  \setcounter{enumi}{3}
  \item (1) $A$ \begin{CJK}{UTF8}{mj}为\end{CJK} $n \times m$ \begin{CJK}{UTF8}{mj}阶矩阵\end{CJK}, $B$ \begin{CJK}{UTF8}{mj}为\end{CJK} $m \times n$ \begin{CJK}{UTF8}{mj}阶矩阵\end{CJK}, \begin{CJK}{UTF8}{mj}其中\end{CJK} $n \geq m$, \begin{CJK}{UTF8}{mj}证明\end{CJK}:
\end{enumerate}
$$
\left|\lambda I_{n}-A B\right|=\lambda^{n-m}\left|\lambda I_{m}-B A\right|
$$
(2) \begin{CJK}{UTF8}{mj}求\end{CJK}
$$
\left|\begin{array}{cccc}
a_{1}+x_{1}+1 & a_{1}+x_{2} & \cdots & a_{1}+x_{n} \\
a_{2}+x_{1} & a_{2}+x_{2}+1 & \cdots & a_{2}+x_{n} \\
\vdots & \vdots & & \vdots \\
a_{n}+x_{1} & a_{n}+x_{2} & \cdots & a_{n}+x_{n}+1
\end{array}\right|
$$

\begin{enumerate}
  \setcounter{enumi}{4}
  \item \begin{CJK}{UTF8}{mj}设\end{CJK}
\end{enumerate}
$$
A=\left(\begin{array}{lllll}
1 & 2 & 1 & 2 & 1 \\
2 & 5 & 2 & 5 & 2 \\
3 & 7 & 3 & 8 & 8
\end{array}\right), \quad B=\left(\begin{array}{lllll}
1 & 3 & 3 & 3 & 1 \\
3 & 7 & 3 & 7 & 3 \\
4 & 9 & 4 & 9 & 4
\end{array}\right)
$$
$A X=0$ \begin{CJK}{UTF8}{mj}的解\end{CJK} \begin{CJK}{UTF8}{mj}空间\end{CJK} $V_{1}, B X=0$ \begin{CJK}{UTF8}{mj}的解空间\end{CJK} $V_{2}$, \begin{CJK}{UTF8}{mj}求\end{CJK} $V_{1} \cap V_{2}, V_{1}+V_{2}$ \begin{CJK}{UTF8}{mj}的一组基\end{CJK}.

\begin{enumerate}
  \setcounter{enumi}{5}
  \item $A$ \begin{CJK}{UTF8}{mj}为\end{CJK}
\end{enumerate}
$$
\left(\begin{array}{lll}
2 & a & b \\
a & 2 & 1 \\
b & 1 & 2
\end{array}\right) \text {. }
$$
\begin{CJK}{UTF8}{mj}其中\end{CJK} $a, b$ \begin{CJK}{UTF8}{mj}为常数\end{CJK}, $A$ \begin{CJK}{UTF8}{mj}的特征多项式\end{CJK} $f(x)=x^{3}-6 x^{2}+9 x-4$, \begin{CJK}{UTF8}{mj}求\end{CJK} $A$ \begin{CJK}{UTF8}{mj}的特征向量\end{CJK}. 6. \begin{CJK}{UTF8}{mj}设\end{CJK} $f(x)$ \begin{CJK}{UTF8}{mj}为数域\end{CJK} $\mathbb{F}$ \begin{CJK}{UTF8}{mj}上不可约多项式\end{CJK}, \begin{CJK}{UTF8}{mj}如果\end{CJK} $\alpha \in \mathbb{C}$ \begin{CJK}{UTF8}{mj}是\end{CJK} $f(x)$ \begin{CJK}{UTF8}{mj}的根\end{CJK}, \begin{CJK}{UTF8}{mj}则\end{CJK}
$$
F(\alpha)=\{g(\alpha) \mid g(x) \in \mathbb{F}(x)\}
$$
\begin{CJK}{UTF8}{mj}是数域\end{CJK}, \begin{CJK}{UTF8}{mj}且\end{CJK} $F(\alpha)$ \begin{CJK}{UTF8}{mj}作为数域\end{CJK} $\mathbb{F}$ \begin{CJK}{UTF8}{mj}上线性空间的维数是\end{CJK} $f(x)$ \begin{CJK}{UTF8}{mj}的次数\end{CJK}.

\begin{enumerate}
  \setcounter{enumi}{7}
  \item \begin{CJK}{UTF8}{mj}设\end{CJK} $V$ \begin{CJK}{UTF8}{mj}是\end{CJK} $n$ \begin{CJK}{UTF8}{mj}维欧氏空间\end{CJK}, $\mathscr{A}$ \begin{CJK}{UTF8}{mj}是\end{CJK} $V$ \begin{CJK}{UTF8}{mj}上对称变换\end{CJK}, \begin{CJK}{UTF8}{mj}求证\end{CJK}: \begin{CJK}{UTF8}{mj}存在标准正交基\end{CJK} $\varepsilon_{1}, \varepsilon_{2}, \cdots, \varepsilon_{n}$, \begin{CJK}{UTF8}{mj}使得\end{CJK} $\mathscr{A}$ \begin{CJK}{UTF8}{mj}在此组基下的矩\end{CJK} \begin{CJK}{UTF8}{mj}阵为对角矩阵\end{CJK}.

  \item $A$ \begin{CJK}{UTF8}{mj}为\end{CJK} $n$ \begin{CJK}{UTF8}{mj}阶实矩阵\end{CJK}, $A+A^{\prime}$ \begin{CJK}{UTF8}{mj}为正定矩阵\end{CJK}, \begin{CJK}{UTF8}{mj}证明\end{CJK} $|A|>0$.

  \item \begin{CJK}{UTF8}{mj}设\end{CJK} $A=\left(\begin{array}{l}A_{1} \\ A_{2}\end{array}\right)$ \begin{CJK}{UTF8}{mj}为数域\end{CJK} $\mathbb{F}$ \begin{CJK}{UTF8}{mj}上\end{CJK} $n$ \begin{CJK}{UTF8}{mj}阶矩阵\end{CJK}, $A_{1}, A_{2}$ \begin{CJK}{UTF8}{mj}分别为\end{CJK} $k \times n,(n-k) \times n$ \begin{CJK}{UTF8}{mj}矩阵\end{CJK}, $A_{1} X=0$ \begin{CJK}{UTF8}{mj}的解空间为\end{CJK} $V_{1}$, $A_{2} X=0$ \begin{CJK}{UTF8}{mj}的解空间\end{CJK} $V_{2}$, \begin{CJK}{UTF8}{mj}证明\end{CJK}

\end{enumerate}
$$
\mathbb{F}^{n}=V_{1} \oplus V_{2} \Leftrightarrow \mathrm{r}(A)=\mathrm{r}\left(A_{1}\right)+\mathrm{r}\left(A_{2}\right)
$$

\section{6. 上海大学 2013 年研究生入学考试试题高等代数 
 李扬 
 微信公众号: sxkyliyang}
\begin{enumerate}
  \item \begin{CJK}{UTF8}{mj}填空题\end{CJK} 5 \begin{CJK}{UTF8}{mj}个\end{CJK}

  \item \begin{CJK}{UTF8}{mj}判断题\end{CJK} (\begin{CJK}{UTF8}{mj}对的答对\end{CJK}, \begin{CJK}{UTF8}{mj}错的举反例\end{CJK}) 5 \begin{CJK}{UTF8}{mj}个\end{CJK}

  \item \begin{CJK}{UTF8}{mj}设\end{CJK} $\alpha_{1}=\left(\begin{array}{l}1 \\ 1 \\ 1 \\ 1\end{array}\right), \alpha_{2}=\left(\begin{array}{l}2 \\ 2 \\ 1 \\ 2\end{array}\right), \alpha_{3}=\left(\begin{array}{l}3 \\ 3 \\ 2 \\ 3\end{array}\right), \alpha_{4}=\left(\begin{array}{l}2 \\ 0 \\ 1 \\ 1\end{array}\right), \alpha_{5}=\left(\begin{array}{l}3 \\ 1 \\ 1 \\ 2\end{array}\right)$, \begin{CJK}{UTF8}{mj}求此向量组的极大无关组\end{CJK}, \begin{CJK}{UTF8}{mj}并将其它向量用此\end{CJK} \begin{CJK}{UTF8}{mj}向量组的极大无关组表示出来\end{CJK}.

  \item \begin{CJK}{UTF8}{mj}已知矩阵\end{CJK}

\end{enumerate}
$$
A=\left(\begin{array}{lll}
3 & 1 & 1 \\
0 & 3 & 1 \\
0 & 0 & 3
\end{array}\right), \quad B=\left(\begin{array}{cc}
A & 0 \\
A-I & A
\end{array}\right)
$$
\begin{CJK}{UTF8}{mj}如果\end{CJK} $C B=4 C+2 I$, \begin{CJK}{UTF8}{mj}求\end{CJK} $C$.

\begin{enumerate}
  \setcounter{enumi}{5}
  \item \begin{CJK}{UTF8}{mj}设矩阵\end{CJK}
\end{enumerate}
$$
A=\left(\begin{array}{llll}
1 & 1 & 0 & 0 \\
0 & 1 & 1 & 0 \\
0 & 0 & 1 & 1 \\
0 & 0 & 0 & 1
\end{array}\right)
$$
\begin{CJK}{UTF8}{mj}求\end{CJK} $A^{n}$.

\begin{enumerate}
  \setcounter{enumi}{6}
  \item \begin{CJK}{UTF8}{mj}设\end{CJK}
\end{enumerate}
$$
A=\left(\begin{array}{lll}
2 & a & b \\
a & 2 & 1 \\
b & 1 & 2
\end{array}\right)
$$
\begin{CJK}{UTF8}{mj}与\end{CJK}
$$
B=\left(\begin{array}{lll}
1 & 0 & 0 \\
0 & 1 & 0 \\
0 & 0 & 4
\end{array}\right)
$$
\begin{CJK}{UTF8}{mj}相似\end{CJK}, ( $a, b$ \begin{CJK}{UTF8}{mj}为复数\end{CJK}), \begin{CJK}{UTF8}{mj}求\end{CJK} $A$ \begin{CJK}{UTF8}{mj}的特征向量\end{CJK}.

\begin{enumerate}
  \setcounter{enumi}{7}
  \item \begin{CJK}{UTF8}{mj}设\end{CJK} $n$ \begin{CJK}{UTF8}{mj}阶矩阵\end{CJK} $A, B, C, D$ \begin{CJK}{UTF8}{mj}满足\end{CJK} $A C=C A$, \begin{CJK}{UTF8}{mj}求证\end{CJK}
\end{enumerate}
$$
B=\left|\begin{array}{ll}
A & B \\
C & D
\end{array}\right|=|A D-C B|
$$

\begin{enumerate}
  \setcounter{enumi}{8}
  \item \begin{CJK}{UTF8}{mj}设\end{CJK} $A=\left(a_{i j}\right)_{n \times n}$, \begin{CJK}{UTF8}{mj}且\end{CJK}
\end{enumerate}
$$
\left|a_{i i}-1\right|>\sum_{i \neq j}\left|a_{i j}\right|(i=1,2, \cdots, n)
$$
\begin{CJK}{UTF8}{mj}求证\end{CJK}: 1 \begin{CJK}{UTF8}{mj}不是\end{CJK} $A$ \begin{CJK}{UTF8}{mj}的特征值\end{CJK}.

\begin{enumerate}
  \setcounter{enumi}{9}
  \item \begin{CJK}{UTF8}{mj}叙述并且证明实二次型惯性定理\end{CJK}. 10. \begin{CJK}{UTF8}{mj}设\end{CJK} $A$ \begin{CJK}{UTF8}{mj}为\end{CJK} $n$ \begin{CJK}{UTF8}{mj}阶实对称矩阵\end{CJK}, $C$ \begin{CJK}{UTF8}{mj}为\end{CJK} $n$ \begin{CJK}{UTF8}{mj}阶实矩阵\end{CJK}, \begin{CJK}{UTF8}{mj}求证\end{CJK}
\end{enumerate}
$$
B=\left|\begin{array}{ll}
A & C^{\prime} \\
C & 0
\end{array}\right|
$$
\begin{CJK}{UTF8}{mj}为半正定阵的充要条件是\end{CJK} $A$ \begin{CJK}{UTF8}{mj}半正定且\end{CJK} $C=0$.

\section{7. 上海大学 2014 年研究生入学考试试题高等代数 
 李扬 
 微信公众号: sxkyliyang}
\begin{enumerate}
  \item (\begin{CJK}{UTF8}{mj}每题\end{CJK} 5 \begin{CJK}{UTF8}{mj}分\end{CJK}, \begin{CJK}{UTF8}{mj}共\end{CJK} 25 \begin{CJK}{UTF8}{mj}分\end{CJK}) \begin{CJK}{UTF8}{mj}填空题\end{CJK}
\end{enumerate}
(1) \begin{CJK}{UTF8}{mj}使\end{CJK} $f(x)=x^{3}-3^{2}+t x-1$ \begin{CJK}{UTF8}{mj}有三重根的\end{CJK} $t$ \begin{CJK}{UTF8}{mj}的值是\end{CJK}:

(2) \begin{CJK}{UTF8}{mj}设\end{CJK} $\beta_{1}=\alpha_{2}+\alpha_{3}+\cdots+\alpha_{r}, \beta_{2}=\alpha_{1}+\alpha_{3}+\cdots+\alpha_{r}, \cdots, \beta_{r-1}=\alpha_{1}+\cdots+\alpha_{r-2}+\alpha_{r}, \beta_{r}=\alpha_{1}+\alpha_{2}+\cdots+\alpha_{r-1}$. \begin{CJK}{UTF8}{mj}则\end{CJK} $\alpha_{1}, \cdots, \alpha_{r}$ \begin{CJK}{UTF8}{mj}之秩\end{CJK} $s$ \begin{CJK}{UTF8}{mj}与\end{CJK} $\beta_{1}, \cdots, \beta_{r}$ \begin{CJK}{UTF8}{mj}之秩\end{CJK} $t$ \begin{CJK}{UTF8}{mj}的关系是\end{CJK}

(3) \begin{CJK}{UTF8}{mj}设同阶矩阵\end{CJK} $A, B$ \begin{CJK}{UTF8}{mj}中元素都是整数\end{CJK}, \begin{CJK}{UTF8}{mj}如果\end{CJK} $A^{2} B+A=I$, \begin{CJK}{UTF8}{mj}则\end{CJK} $(\operatorname{det} A)^{2}=$

(4) \begin{CJK}{UTF8}{mj}设\end{CJK} $\alpha_{1}, \alpha_{2}, \alpha_{3}$ \begin{CJK}{UTF8}{mj}是\end{CJK} 3 \begin{CJK}{UTF8}{mj}维向量空间\end{CJK} $\mathbb{R}^{3}$ \begin{CJK}{UTF8}{mj}的一组基\end{CJK}, \begin{CJK}{UTF8}{mj}则由基\end{CJK} $\alpha_{1}, \frac{1}{2} \alpha_{2}, \frac{1}{3} \alpha_{3}$ \begin{CJK}{UTF8}{mj}到基\end{CJK} $\alpha_{1}+\alpha_{2}, \alpha_{2}+\alpha_{3}, \alpha_{3}+\alpha_{1}$ \begin{CJK}{UTF8}{mj}的\end{CJK} \begin{CJK}{UTF8}{mj}过渡矩阵是\end{CJK}

(5) \begin{CJK}{UTF8}{mj}设\end{CJK} $A \in \mathbb{C}^{n \times n}$. \begin{CJK}{UTF8}{mj}若\end{CJK} $A$ \begin{CJK}{UTF8}{mj}是酉矩阵\end{CJK}, \begin{CJK}{UTF8}{mj}则\end{CJK}

\begin{enumerate}
  \setcounter{enumi}{2}
  \item (\begin{CJK}{UTF8}{mj}每题\end{CJK} 5 \begin{CJK}{UTF8}{mj}分\end{CJK}, \begin{CJK}{UTF8}{mj}共\end{CJK} 25 \begin{CJK}{UTF8}{mj}分\end{CJK}) \begin{CJK}{UTF8}{mj}是非题\end{CJK} (\begin{CJK}{UTF8}{mj}正确的在括号内添\end{CJK}“\begin{CJK}{UTF8}{mj}正确\end{CJK}”, \begin{CJK}{UTF8}{mj}错误的在括号内添\end{CJK}“\begin{CJK}{UTF8}{mj}错误\end{CJK}”, \begin{CJK}{UTF8}{mj}如果不正确\end{CJK}, \begin{CJK}{UTF8}{mj}请举出反\end{CJK} \begin{CJK}{UTF8}{mj}例\end{CJK}.)
\end{enumerate}
(1) \begin{CJK}{UTF8}{mj}设\end{CJK} $A, B$ \begin{CJK}{UTF8}{mj}为\end{CJK} $n$ \begin{CJK}{UTF8}{mj}阶矩阵\end{CJK}, \begin{CJK}{UTF8}{mj}则\end{CJK} $A B=B A$. ( )

(2) \begin{CJK}{UTF8}{mj}数域\end{CJK} $P$ \begin{CJK}{UTF8}{mj}上任何多项式的次数都大于或等于零\end{CJK}. ( )

(3) \begin{CJK}{UTF8}{mj}设\end{CJK} $A \in P^{n \times n}$, \begin{CJK}{UTF8}{mj}则\end{CJK} $A$ \begin{CJK}{UTF8}{mj}的最小多项式是\end{CJK} $A$ \begin{CJK}{UTF8}{mj}的最后个不变因子\end{CJK}. ( )

(4) \begin{CJK}{UTF8}{mj}设\end{CJK} $\mathscr{A}$ \begin{CJK}{UTF8}{mj}是\end{CJK} $n$ \begin{CJK}{UTF8}{mj}维线性空间\end{CJK} $V$ \begin{CJK}{UTF8}{mj}线性变换\end{CJK}, \begin{CJK}{UTF8}{mj}则其核\end{CJK} ker $\mathscr{A}$ \begin{CJK}{UTF8}{mj}与值域\end{CJK} $\operatorname{Im} \mathscr{A}$ \begin{CJK}{UTF8}{mj}满足\end{CJK}: $\operatorname{ker} \mathscr{A} \oplus \operatorname{Im} \mathscr{A}=V$. ( )

(5) $A$ \begin{CJK}{UTF8}{mj}的某个某一个零化多项式无重根\end{CJK}, \begin{CJK}{UTF8}{mj}则\end{CJK} $A$ \begin{CJK}{UTF8}{mj}相似于对角阵\end{CJK}. ( )

\begin{enumerate}
  \setcounter{enumi}{3}
  \item ( 15 \begin{CJK}{UTF8}{mj}分\end{CJK}) \begin{CJK}{UTF8}{mj}设\end{CJK}
\end{enumerate}
$$
A=\left[\begin{array}{lll}
1 & \alpha & \beta \\
0 & 1 & \alpha \\
0 & 0 & 1
\end{array}\right]
$$
\begin{CJK}{UTF8}{mj}试求\end{CJK} $A^{2}, A^{3}$, \begin{CJK}{UTF8}{mj}并进而求\end{CJK} $A^{n}$.

\begin{enumerate}
  \setcounter{enumi}{4}
  \item (15 \begin{CJK}{UTF8}{mj}分\end{CJK}) \begin{CJK}{UTF8}{mj}设\end{CJK} $A, B$ \begin{CJK}{UTF8}{mj}都是\end{CJK} $n$ \begin{CJK}{UTF8}{mj}阶方阵\end{CJK}, $E$ \begin{CJK}{UTF8}{mj}是单位矩阵\end{CJK}.
\end{enumerate}
(1) \begin{CJK}{UTF8}{mj}证明\end{CJK} $|E-A B|=0$ \begin{CJK}{UTF8}{mj}的充要条件是\end{CJK} $|E-B A|=0$.

(2) \begin{CJK}{UTF8}{mj}已知\end{CJK} $E-A B$ \begin{CJK}{UTF8}{mj}可逆且\end{CJK} $C=(E-A B)^{-1}$, \begin{CJK}{UTF8}{mj}求\end{CJK} $(E-B A)^{-1}$.

\begin{enumerate}
  \setcounter{enumi}{5}
  \item (15 \begin{CJK}{UTF8}{mj}分\end{CJK}) \begin{CJK}{UTF8}{mj}若矩阵\end{CJK}
\end{enumerate}
$$
A=\left[\begin{array}{ccc}
5 & 7 & -5 \\
0 & 4 & 1 \\
2 & 8 & -3
\end{array}\right]
$$
(1) \begin{CJK}{UTF8}{mj}求\end{CJK} $A$ \begin{CJK}{UTF8}{mj}的特征值和特征向量\end{CJK};

(2) \begin{CJK}{UTF8}{mj}将\end{CJK} $A$ \begin{CJK}{UTF8}{mj}经相似变换化成对角阵\end{CJK}.

\begin{enumerate}
  \setcounter{enumi}{6}
  \item ( 15 \begin{CJK}{UTF8}{mj}分\end{CJK}) $P$ \begin{CJK}{UTF8}{mj}是数域\end{CJK}, $P^{n \times n}$ \begin{CJK}{UTF8}{mj}关于矩阵加法和数乘矩阵构成线性空间\end{CJK}, $V_{1}=\left\{A \mid A \in P^{n \times n}, A^{\prime}=A\right\}$.
\end{enumerate}
(1) \begin{CJK}{UTF8}{mj}证明\end{CJK}: $V$ \begin{CJK}{UTF8}{mj}是\end{CJK} $P^{n \times n}$ \begin{CJK}{UTF8}{mj}的子空间\end{CJK};

(2)\begin{CJK}{UTF8}{mj}求\end{CJK} $P^{n \times n}$ \begin{CJK}{UTF8}{mj}的子空间\end{CJK} $V_{2}$, \begin{CJK}{UTF8}{mj}使\end{CJK} $P^{n \times n}=V_{1} \oplus V_{2}$.

\begin{enumerate}
  \setcounter{enumi}{7}
  \item ( 10 \begin{CJK}{UTF8}{mj}分\end{CJK}) \begin{CJK}{UTF8}{mj}证明多项式\end{CJK} $f(x)=x^{5}-5 x+1$ \begin{CJK}{UTF8}{mj}在有理数域上不可约\end{CJK}. 8. (15 \begin{CJK}{UTF8}{mj}分\end{CJK}) \begin{CJK}{UTF8}{mj}设\end{CJK} $V$ \begin{CJK}{UTF8}{mj}是全体实\end{CJK} $2 \times 2$ \begin{CJK}{UTF8}{mj}矩阵所构成的实线性空间\end{CJK}, $A=\left[\begin{array}{ll}a & b \\ c & d\end{array}\right] \in V$, \begin{CJK}{UTF8}{mj}定义\end{CJK} $V$ \begin{CJK}{UTF8}{mj}的变换\end{CJK} $\mathscr{A} x=A x, \forall x \in V$.
\end{enumerate}
(1) \begin{CJK}{UTF8}{mj}证明\end{CJK}: \begin{CJK}{UTF8}{mj}变换\end{CJK} $\mathscr{A}$ \begin{CJK}{UTF8}{mj}是线性的\end{CJK}.

(2) \begin{CJK}{UTF8}{mj}当\end{CJK} $A=\left[\begin{array}{cc}1 & 2 \\ -2 & -4\end{array}\right]$ \begin{CJK}{UTF8}{mj}时\end{CJK}, \begin{CJK}{UTF8}{mj}求\end{CJK} $\mathscr{A}$ \begin{CJK}{UTF8}{mj}的核和值城及它们的一组基\end{CJK}.

\begin{enumerate}
  \setcounter{enumi}{9}
  \item (15 \begin{CJK}{UTF8}{mj}分\end{CJK}) \begin{CJK}{UTF8}{mj}设\end{CJK} $A, C$ \begin{CJK}{UTF8}{mj}是\end{CJK} $n$ \begin{CJK}{UTF8}{mj}阶实对称矩阵\end{CJK}, \begin{CJK}{UTF8}{mj}且矩阵方程\end{CJK} $A X-X A=C$ \begin{CJK}{UTF8}{mj}存在唯一解\end{CJK} $B$.
\end{enumerate}
(1) \begin{CJK}{UTF8}{mj}证明\end{CJK}: $B$ \begin{CJK}{UTF8}{mj}为实对称矩阵\end{CJK}.

(2) \begin{CJK}{UTF8}{mj}如果\end{CJK} $A, C$ \begin{CJK}{UTF8}{mj}为正定矩阵\end{CJK}, \begin{CJK}{UTF8}{mj}求证\end{CJK}: $B$ \begin{CJK}{UTF8}{mj}为正定矩阵\end{CJK}.

\section{8. 上海大学 2015 年研究生入学考试试题高等代数 
 李扬 
 微信公众号: sxkyliyang}
\begin{enumerate}
  \item \begin{CJK}{UTF8}{mj}填空题\end{CJK} (\begin{CJK}{UTF8}{mj}与\end{CJK} 2011 \begin{CJK}{UTF8}{mj}年题型基本一样\end{CJK})
\end{enumerate}
(1) \begin{CJK}{UTF8}{mj}多项式\end{CJK} $f(x)=x^{3}-2 x-4$ \begin{CJK}{UTF8}{mj}的有理根是\end{CJK}

(2) \begin{CJK}{UTF8}{mj}设\end{CJK} $A$ \begin{CJK}{UTF8}{mj}是\end{CJK} $m \times 4$ \begin{CJK}{UTF8}{mj}矩阵\end{CJK}, \begin{CJK}{UTF8}{mj}且\end{CJK} $A$ \begin{CJK}{UTF8}{mj}中有个三列向量线性无关\end{CJK}, \begin{CJK}{UTF8}{mj}如果线性方程组\end{CJK} $A X=b$ \begin{CJK}{UTF8}{mj}有解\end{CJK} $\alpha=[1,2,3,4]^{T}$, $\beta=[1,1,1,1]^{T}$, \begin{CJK}{UTF8}{mj}则\end{CJK} $A X=b$ \begin{CJK}{UTF8}{mj}的通解是\end{CJK}

(3) \begin{CJK}{UTF8}{mj}设同阶矩阵\end{CJK} $A, B$ \begin{CJK}{UTF8}{mj}中元素都是整数\end{CJK}, \begin{CJK}{UTF8}{mj}如果\end{CJK} $A^{2} B+A=I$, \begin{CJK}{UTF8}{mj}则\end{CJK} $(\operatorname{det} A)^{2}=$

(4)\begin{CJK}{UTF8}{mj}四维线性空间\end{CJK} $V$ \begin{CJK}{UTF8}{mj}上线性变换\end{CJK} $\mathscr{A}$ \begin{CJK}{UTF8}{mj}的最小多项式是\end{CJK} $x(x-1)$, \begin{CJK}{UTF8}{mj}值域维数是\end{CJK} 2 , \begin{CJK}{UTF8}{mj}则存在\end{CJK} $V$ \begin{CJK}{UTF8}{mj}上的一组基\end{CJK}, \begin{CJK}{UTF8}{mj}使得\end{CJK} $\mathscr{A}$ \begin{CJK}{UTF8}{mj}在此组基下矩阵是对角阵\end{CJK} $A=$

(5) \begin{CJK}{UTF8}{mj}设\end{CJK} $A=\left[\begin{array}{lll}0 & 1 & 0 \\ 0 & 0 & 1 \\ 0 & 0 & 0\end{array}\right], V=\left\{B \in \mathbb{F}^{3 \times 3} \mid A B=B A\right\}$, \begin{CJK}{UTF8}{mj}则\end{CJK} $V$ \begin{CJK}{UTF8}{mj}为\end{CJK} $\mathbb{F}$ \begin{CJK}{UTF8}{mj}上\end{CJK} \begin{CJK}{UTF8}{mj}维线性空间\end{CJK}, \begin{CJK}{UTF8}{mj}基为\end{CJK}

\begin{enumerate}
  \setcounter{enumi}{2}
  \item \begin{CJK}{UTF8}{mj}判断题\end{CJK}
\end{enumerate}
(1) \begin{CJK}{UTF8}{mj}设有理系数多项式\end{CJK} $f(x)$ \begin{CJK}{UTF8}{mj}在有理数域上无重根\end{CJK}, \begin{CJK}{UTF8}{mj}则\end{CJK} $f(x)$ \begin{CJK}{UTF8}{mj}在有理数域上无重因式\end{CJK}. ( )

(2) \begin{CJK}{UTF8}{mj}如果矩阵\end{CJK} $A$ \begin{CJK}{UTF8}{mj}与\end{CJK} $B$ \begin{CJK}{UTF8}{mj}相似\end{CJK}, \begin{CJK}{UTF8}{mj}则\end{CJK} $2 A$ \begin{CJK}{UTF8}{mj}与\end{CJK} $3 B$ \begin{CJK}{UTF8}{mj}等价\end{CJK}. ( )

(3) \begin{CJK}{UTF8}{mj}对任意实矩阵\end{CJK} $A$, \begin{CJK}{UTF8}{mj}齐次线性方程组\end{CJK} $A X=0$ \begin{CJK}{UTF8}{mj}与齐次线性方程组\end{CJK} $A^{\prime} A X=0$ \begin{CJK}{UTF8}{mj}的解相同\end{CJK}. ( )

(4) \begin{CJK}{UTF8}{mj}设\end{CJK} $\mathscr{A}$ \begin{CJK}{UTF8}{mj}为\end{CJK} $n$ \begin{CJK}{UTF8}{mj}维线性空间\end{CJK} $V$ \begin{CJK}{UTF8}{mj}的线性变换\end{CJK}, $\operatorname{ker} \mathscr{A}, \operatorname{Im} \mathscr{A}$ \begin{CJK}{UTF8}{mj}分别为\end{CJK} $\mathscr{A}$ \begin{CJK}{UTF8}{mj}的核与值域\end{CJK}, \begin{CJK}{UTF8}{mj}则\end{CJK} $V=\operatorname{ker} \mathscr{A} \oplus \operatorname{Im} \mathscr{A}$ \begin{CJK}{UTF8}{mj}的充\end{CJK} \begin{CJK}{UTF8}{mj}要条件是\end{CJK} $\{0\}=\operatorname{ker} \mathscr{A} \cap \operatorname{Im} \mathscr{A}$. ( )

(5) \begin{CJK}{UTF8}{mj}设向量\end{CJK} $\alpha_{1}, \alpha_{2}, \cdots, \alpha_{r}$ \begin{CJK}{UTF8}{mj}可由向量组\end{CJK} $\beta_{1}, \beta_{2}, \cdots, \beta_{t}$ \begin{CJK}{UTF8}{mj}线性表出且向量组\end{CJK} $\beta_{1}, \beta_{2}, \cdots, \beta_{t}$ \begin{CJK}{UTF8}{mj}线性无关\end{CJK}, \begin{CJK}{UTF8}{mj}则两\end{CJK} \begin{CJK}{UTF8}{mj}个向量组等价的充要条件是\end{CJK} $\mathrm{r}\left(\alpha_{1}, \alpha_{2}, \cdots, \alpha_{r}\right)=t$. ( )

\begin{enumerate}
  \setcounter{enumi}{3}
  \item \begin{CJK}{UTF8}{mj}设\end{CJK}
\end{enumerate}
$$
A=\left(\begin{array}{lll}
1 & 2 & 3 \\
0 & 1 & 4 \\
0 & 0 & 1
\end{array}\right)
$$
\begin{CJK}{UTF8}{mj}求\end{CJK} $A^{n}$.

\begin{enumerate}
  \setcounter{enumi}{4}
  \item \begin{CJK}{UTF8}{mj}设\end{CJK} $A$ \begin{CJK}{UTF8}{mj}是\end{CJK} $n$ \begin{CJK}{UTF8}{mj}阶实对称矩阵\end{CJK}, \begin{CJK}{UTF8}{mj}求证存在正交矩阵\end{CJK} $P$, \begin{CJK}{UTF8}{mj}使得\end{CJK} $P^{-1} A P$ \begin{CJK}{UTF8}{mj}为对角矩阵\end{CJK}.

  \item \begin{CJK}{UTF8}{mj}计算行列式\end{CJK}

\end{enumerate}
$$
\left|\begin{array}{ccccc}
a+x_{1} & a+x_{1}^{2} & \cdots & a+x_{1}^{n-1} & a+x_{1}^{n} \\
a+x_{2} & a+x_{2}^{2} & \cdots & a+x_{2}^{n-1} & a+x_{2}^{n} \\
\vdots & \vdots & & \vdots & \vdots \\
a+x_{n} & a+x_{n}^{2} & \cdots & a+x_{n}^{n-1} & a+x_{n}^{n}
\end{array}\right|
$$

\begin{enumerate}
  \setcounter{enumi}{6}
  \item \begin{CJK}{UTF8}{mj}设\end{CJK} $B, C$ \begin{CJK}{UTF8}{mj}分别为\end{CJK} $n \times m, m \times n$ \begin{CJK}{UTF8}{mj}阶矩阵\end{CJK}, \begin{CJK}{UTF8}{mj}求证\end{CJK}:
\end{enumerate}
$$
\operatorname{rank}\left(I_{n}-B C\right)+m=n+\operatorname{rank}\left(I_{m}-C B\right)
$$

\begin{enumerate}
  \setcounter{enumi}{7}
  \item $\mathbb{F}$ \begin{CJK}{UTF8}{mj}上齐次方程组\end{CJK} $X_{1 \times n} A_{n \times m}=O_{1 \times m}(*)$, \begin{CJK}{UTF8}{mj}令\end{CJK} $C=\left(A_{n \times m} \quad I_{n}\right)$, \begin{CJK}{UTF8}{mj}对\end{CJK} $C$ \begin{CJK}{UTF8}{mj}做一系列的初等变换化为\end{CJK} $\left(\left(\begin{array}{c}D_{r} \\ 0\end{array}\right) P\right)$, \begin{CJK}{UTF8}{mj}其中\end{CJK} $D_{r}$ \begin{CJK}{UTF8}{mj}为一行满秩\end{CJK}, $r=r(A), P$ \begin{CJK}{UTF8}{mj}为\end{CJK} $n$ \begin{CJK}{UTF8}{mj}阶可逆方阵\end{CJK}. \begin{CJK}{UTF8}{mj}证明\end{CJK}: $P$ \begin{CJK}{UTF8}{mj}的最后\end{CJK} $n-r$ \begin{CJK}{UTF8}{mj}行即为\end{CJK} \begin{CJK}{UTF8}{mj}方程组\end{CJK} $(*)$ \begin{CJK}{UTF8}{mj}的一个基础解系\end{CJK}. 8. \begin{CJK}{UTF8}{mj}设\end{CJK}
\end{enumerate}
$$
\begin{gathered}
f(x)=x^{3}+x^{2}+x+1 \\
g(x)=x^{8 n}+x^{8 m+2}+x^{4 k+1}+x^{12 l+3}
\end{gathered}
$$
$(n, m, k, l$ \begin{CJK}{UTF8}{mj}为正整数\end{CJK}), \begin{CJK}{UTF8}{mj}求证\end{CJK}: $f(x) \mid g(x)$.

\begin{enumerate}
  \setcounter{enumi}{9}
  \item \begin{CJK}{UTF8}{mj}设\end{CJK} $A, B, C, D$ \begin{CJK}{UTF8}{mj}为\end{CJK} $n$ \begin{CJK}{UTF8}{mj}阶实对称矩阵\end{CJK}, \begin{CJK}{UTF8}{mj}且两两交换\end{CJK}, \begin{CJK}{UTF8}{mj}并满足\end{CJK} $A C+B D=I$, \begin{CJK}{UTF8}{mj}如果\end{CJK} $V=\left\{x \in \mathbb{R}^{n \times n} \mid A B X=0\right\}$, $V_{1}=\left\{x \in \mathbb{R}^{n \times n} \mid B X=0\right\}, V_{2}=\left\{x \in \mathbb{R}^{n \times n} \mid A X=0\right\}$, \begin{CJK}{UTF8}{mj}求证\end{CJK}:
\end{enumerate}
$$
V=V_{1} \oplus V_{2}
$$

\begin{enumerate}
  \setcounter{enumi}{10}
  \item \begin{CJK}{UTF8}{mj}设\end{CJK} $A$ \begin{CJK}{UTF8}{mj}为\end{CJK} $n$ \begin{CJK}{UTF8}{mj}阶正定矩阵\end{CJK}, $B$ \begin{CJK}{UTF8}{mj}为\end{CJK} $n$ \begin{CJK}{UTF8}{mj}阶实对称矩阵\end{CJK}, \begin{CJK}{UTF8}{mj}且\end{CJK} $|\lambda A+B|=0$ \begin{CJK}{UTF8}{mj}的根\end{CJK} $\lambda$ \begin{CJK}{UTF8}{mj}都大于\end{CJK} 0 , \begin{CJK}{UTF8}{mj}求证\end{CJK} $B$ \begin{CJK}{UTF8}{mj}为负定矩阵\end{CJK}.
\end{enumerate}
\section{9. 上海大学 2016 年研究生入学考试试题高等代数 
 李扬 
 微信公众号: sxkyliyang}
\begin{enumerate}
  \item \begin{CJK}{UTF8}{mj}填空题\end{CJK}
\end{enumerate}
(1) \begin{CJK}{UTF8}{mj}已知矩阵\end{CJK} $A$, \begin{CJK}{UTF8}{mj}求\end{CJK} $A^{4}+A^{3}+A+I$.

(2) $x^{3}+a x+1$ \begin{CJK}{UTF8}{mj}在有理数域上有有理根\end{CJK}, \begin{CJK}{UTF8}{mj}求\end{CJK} $a$.

(3) \begin{CJK}{UTF8}{mj}含参数\end{CJK} $t$ \begin{CJK}{UTF8}{mj}的矩阵正定\end{CJK}, \begin{CJK}{UTF8}{mj}求\end{CJK} $t$ \begin{CJK}{UTF8}{mj}的范围\end{CJK}.

(4) \begin{CJK}{UTF8}{mj}告诉你矩阵\end{CJK} $A$, \begin{CJK}{UTF8}{mj}求\end{CJK} $A^{*}$ \begin{CJK}{UTF8}{mj}与\end{CJK}

(5) $\mathscr{A}\left(x_{1}, x_{2}, x_{3}\right)=\left(\begin{array}{c}2 x_{1}+x_{2}+x_{3} \\ x_{1}+2 x_{2}-x_{3} \\ x_{3}\end{array}\right)$, \begin{CJK}{UTF8}{mj}求线性变换\end{CJK} $\mathscr{A}$ \begin{CJK}{UTF8}{mj}在一组标准正交基下的矩阵\end{CJK}.

\begin{enumerate}
  \setcounter{enumi}{2}
  \item \begin{CJK}{UTF8}{mj}是非题\end{CJK} (\begin{CJK}{UTF8}{mj}正确的在括号内添\end{CJK} “\begin{CJK}{UTF8}{mj}正确\end{CJK}”, \begin{CJK}{UTF8}{mj}错误的在括号内添\end{CJK} “\begin{CJK}{UTF8}{mj}错误\end{CJK}”, \begin{CJK}{UTF8}{mj}如果不正确\end{CJK}, \begin{CJK}{UTF8}{mj}请举出反例\end{CJK}.)
\end{enumerate}
(1) \begin{CJK}{UTF8}{mj}在复数域上\end{CJK}, $n$ \begin{CJK}{UTF8}{mj}阶矩阵\end{CJK} $A$ \begin{CJK}{UTF8}{mj}的最小多项式\end{CJK} $(x-1)^{2}(x-2)$, \begin{CJK}{UTF8}{mj}则\end{CJK} $A$ \begin{CJK}{UTF8}{mj}不可对角化\end{CJK}.

(2) \begin{CJK}{UTF8}{mj}线性方程组\end{CJK} $A_{m \times n} X=b$ \begin{CJK}{UTF8}{mj}有唯一解\end{CJK}, \begin{CJK}{UTF8}{mj}则\end{CJK} $X=\left(A^{\prime} A\right)^{-1} A^{\prime} b$.

(3) $A$ \begin{CJK}{UTF8}{mj}是\end{CJK} $n$ \begin{CJK}{UTF8}{mj}阶半正定矩阵\end{CJK}, $a_{22}=0$, \begin{CJK}{UTF8}{mj}则\end{CJK} $a_{2 j}=0(j \neq 2)$.

(4) $f(x), g(x), h(x)$ \begin{CJK}{UTF8}{mj}在复数域上\end{CJK}, \begin{CJK}{UTF8}{mj}若\end{CJK} $f(x) \mid g(x) h(x)$, \begin{CJK}{UTF8}{mj}则\end{CJK} $f(x) \mid g(x)$ \begin{CJK}{UTF8}{mj}或\end{CJK} $f(x) \mid h(x)$.

(5) $V_{1}$ \begin{CJK}{UTF8}{mj}是子空间\end{CJK}, \begin{CJK}{UTF8}{mj}则\end{CJK} $V_{1}$ \begin{CJK}{UTF8}{mj}有唯一的补空间和正交补空间\end{CJK}.

\begin{enumerate}
  \setcounter{enumi}{3}
  \item \begin{CJK}{UTF8}{mj}已知\end{CJK}
\end{enumerate}
$$
A=\left(\begin{array}{ccccc}
0 & 2 & 0 & 0 & 0 \\
1 & -2 & 0 & 0 & 0 \\
0 & 0 & -2 & 2 & 3 \\
0 & 0 & 1 & 0 & 4 \\
0 & 0 & 2 & 4 & 4
\end{array}\right)
$$
\begin{CJK}{UTF8}{mj}且\end{CJK} $A X=-3 X+I$, \begin{CJK}{UTF8}{mj}求\end{CJK} $X$.

\begin{enumerate}
  \setcounter{enumi}{4}
  \item \begin{CJK}{UTF8}{mj}正交变换\end{CJK} $X=P Y$ \begin{CJK}{UTF8}{mj}将\end{CJK} $f\left(x_{1}, x_{2}, x_{3}\right)=3 x_{1}^{2}+a x_{2}^{2}+3 x_{3}^{2}+2 b x_{1} x_{2}+2 x_{1} x_{3}+2 x_{2} x_{3}$ \begin{CJK}{UTF8}{mj}化为\end{CJK}
\end{enumerate}
$$
f\left(y_{1}, y_{2}, y_{3}\right)=2 y_{1}^{2}+2 y_{2}^{2}+5 y_{3}^{2} .
$$
\begin{CJK}{UTF8}{mj}求\end{CJK} $a, b$ \begin{CJK}{UTF8}{mj}及正交矩阵\end{CJK} $P$.

\begin{enumerate}
  \setcounter{enumi}{5}
  \item \begin{CJK}{UTF8}{mj}设\end{CJK} $A$ \begin{CJK}{UTF8}{mj}是\end{CJK} 3 \begin{CJK}{UTF8}{mj}阶实矩阵\end{CJK}, $(a, b, c)$ \begin{CJK}{UTF8}{mj}是\end{CJK} $A$ \begin{CJK}{UTF8}{mj}的第一行元素\end{CJK}, \begin{CJK}{UTF8}{mj}且\end{CJK} $a, b, c$ \begin{CJK}{UTF8}{mj}不全为\end{CJK} 0 ,
\end{enumerate}
$$
B=\left(\begin{array}{ccc}
1 & 2 & 3 \\
2 & 4 & 6 \\
3 & 6 & 9
\end{array}\right)
$$
$A B=0$, \begin{CJK}{UTF8}{mj}求\end{CJK} $A X=0$ \begin{CJK}{UTF8}{mj}的通解\end{CJK}.

\begin{enumerate}
  \setcounter{enumi}{6}
  \item $A$ \begin{CJK}{UTF8}{mj}是实数域上\end{CJK} $n$ \begin{CJK}{UTF8}{mj}阶矩阵\end{CJK}, $V_{1}=\left\{x \in \mathbb{R}^{n} \mid A X=0\right\}, V_{2}=\left\{A x \mid x \in \mathbb{R}^{n}\right\}$, \begin{CJK}{UTF8}{mj}求证\end{CJK} $V_{1}$, $V_{2}$ \begin{CJK}{UTF8}{mj}是实数域上的线性空间\end{CJK}, \begin{CJK}{UTF8}{mj}且\end{CJK}
\end{enumerate}
$$
\operatorname{dim} V_{1}+\operatorname{dim} V_{2}=n .
$$

\begin{enumerate}
  \setcounter{enumi}{7}
  \item $f(x)$ \begin{CJK}{UTF8}{mj}是次数\end{CJK} $n(n \geq 2)$ \begin{CJK}{UTF8}{mj}的系数为正整数的一个多项式\end{CJK}, \begin{CJK}{UTF8}{mj}且根为实数\end{CJK}, \begin{CJK}{UTF8}{mj}对任意正整数\end{CJK} $m, f(m)$ \begin{CJK}{UTF8}{mj}都为素数\end{CJK}, \begin{CJK}{UTF8}{mj}证明\end{CJK}: $f(x)$ \begin{CJK}{UTF8}{mj}在有理数域\end{CJK} $\mathbb{Q}$ \begin{CJK}{UTF8}{mj}上不可约\end{CJK}. 8. $A$ \begin{CJK}{UTF8}{mj}称作是幂零矩阵\end{CJK}, \begin{CJK}{UTF8}{mj}如果存在正整数\end{CJK} $k$, \begin{CJK}{UTF8}{mj}使得\end{CJK} $A^{k}=0$. \begin{CJK}{UTF8}{mj}若\end{CJK} $\mathrm{r}(A)=n-1$
\end{enumerate}
(1) \begin{CJK}{UTF8}{mj}求证\end{CJK}: $A$ \begin{CJK}{UTF8}{mj}相似于\end{CJK}
$$
\left[\begin{array}{ccccc}
0 & 0 & \cdots & 0 & 0 \\
1 & 0 & \cdots & 0 & 0 \\
0 & 1 & \cdots & 0 & 0 \\
\vdots & \vdots & & \vdots & \vdots \\
0 & 0 & \cdots & 1 & 0
\end{array}\right]
$$
(2) $V=\left\{B \in \mathbb{C}^{n \times n} \mid A B=B A\right\}$, \begin{CJK}{UTF8}{mj}求证\end{CJK}:
$$
\operatorname{dim} V=n
$$

\begin{enumerate}
  \setcounter{enumi}{9}
  \item $A, B$ \begin{CJK}{UTF8}{mj}是\end{CJK} $n$ \begin{CJK}{UTF8}{mj}阶矩阵\end{CJK}, $A, B$ \begin{CJK}{UTF8}{mj}的特征多项式互素\end{CJK}, \begin{CJK}{UTF8}{mj}求证矩阵方程\end{CJK} $A X=X B$ \begin{CJK}{UTF8}{mj}只有零解\end{CJK}.
\end{enumerate}
\section{0. 上海大学 2017 年研究生入学考试试题高等代数 
 李扬 
 微信公众号: sxkyliyang}
\begin{enumerate}
  \item \begin{CJK}{UTF8}{mj}是非题\end{CJK} (\begin{CJK}{UTF8}{mj}正确的在括号内添\end{CJK}“\begin{CJK}{UTF8}{mj}正确\end{CJK}”, \begin{CJK}{UTF8}{mj}错误的在括号内添\end{CJK} “\begin{CJK}{UTF8}{mj}错误\end{CJK}”, \begin{CJK}{UTF8}{mj}如果不正确\end{CJK}, \begin{CJK}{UTF8}{mj}请举出反例\end{CJK}.)\\
(1) $A$ \begin{CJK}{UTF8}{mj}可逆的充要条件是\end{CJK} $A^{*}$ \begin{CJK}{UTF8}{mj}可逆\end{CJK}. ( )\\
(2) $A$ \begin{CJK}{UTF8}{mj}可对角化的充要条件是\end{CJK} $A$ \begin{CJK}{UTF8}{mj}的特征多项式无重根\end{CJK}.\\
(3) $A, B$ \begin{CJK}{UTF8}{mj}相似的充要条件是\end{CJK} $A, B$ \begin{CJK}{UTF8}{mj}等价\end{CJK}. ( )\\
(4)\begin{CJK}{UTF8}{mj}设\end{CJK} $A, B$ \begin{CJK}{UTF8}{mj}正定\end{CJK}, \begin{CJK}{UTF8}{mj}则\end{CJK} $\left(\begin{array}{cc}A & A \\ A & A+B\end{array}\right)^{n}$ \begin{CJK}{UTF8}{mj}正定\end{CJK}. ( )

  \item $A$ \begin{CJK}{UTF8}{mj}与\end{CJK}

\end{enumerate}
$$
\left(\begin{array}{lll}
2 & 1 & 0 \\
0 & 2 & 0 \\
0 & 0 & 2
\end{array}\right)
$$
\begin{CJK}{UTF8}{mj}相似\end{CJK}, \begin{CJK}{UTF8}{mj}求\end{CJK} $A$ \begin{CJK}{UTF8}{mj}的最小多项式\end{CJK}.

\begin{enumerate}
  \setcounter{enumi}{3}
  \item \begin{CJK}{UTF8}{mj}求\end{CJK}
\end{enumerate}
$$
x^{3}+x-2
$$
\begin{CJK}{UTF8}{mj}在有理数域上的二次不可约因式\end{CJK}.

$4 .$
$$
A=\left[\begin{array}{lll}
3 & 1 & 1 \\
1 & 3 & 1 \\
1 & 1 & t
\end{array}\right]
$$
$A$ \begin{CJK}{UTF8}{mj}实对称\end{CJK}, $|A|>0$, \begin{CJK}{UTF8}{mj}求\end{CJK} $X^{\prime} A X$ \begin{CJK}{UTF8}{mj}的规范形\end{CJK}.

$5 .$
$$
\mathscr{A}\left(\begin{array}{l}
x_{1} \\
x_{2} \\
x_{3}
\end{array}\right)=\left(\begin{array}{c}
2 x_{1}+x_{2}+x_{3} \\
x_{1}+2 x_{2}+x_{3} \\
3 x_{1}+4 x_{2}+2 x_{3}
\end{array}\right)
$$
\begin{CJK}{UTF8}{mj}求\end{CJK} $\mathscr{A}$ \begin{CJK}{UTF8}{mj}核的维数\end{CJK}, \begin{CJK}{UTF8}{mj}值域的维数\end{CJK}.

\begin{enumerate}
  \setcounter{enumi}{6}
  \item \begin{CJK}{UTF8}{mj}求\end{CJK}
\end{enumerate}
$$
\left(\begin{array}{lll}
1 & 1 & 1 \\
0 & 1 & 1 \\
0 & 0 & 1
\end{array}\right)^{n}
$$

\begin{enumerate}
  \setcounter{enumi}{7}
  \item \begin{CJK}{UTF8}{mj}设\end{CJK} $A, B$ \begin{CJK}{UTF8}{mj}是\end{CJK} $n$ \begin{CJK}{UTF8}{mj}维多项式\end{CJK}, \begin{CJK}{UTF8}{mj}且\end{CJK} $\mathrm{r}(A)+\mathrm{r}(B)=n, V_{A}=\left\{x \in \mathbb{R}^{n} \mid A X=0\right\}, V_{B}=\left\{x \in \mathbb{R}^{n} \mid B X=0\right\}$, \begin{CJK}{UTF8}{mj}证明\end{CJK}:
\end{enumerate}
$$
\operatorname{dim}\left(V_{A}+V_{B}\right)=n
$$

\begin{enumerate}
  \setcounter{enumi}{8}
  \item \begin{CJK}{UTF8}{mj}已知方程组\end{CJK}
\end{enumerate}
$$
\text { I }\left\{\begin{array}{l}
x_{1}+x_{2}-2 x_{4}=-6 \\
6 x_{1}+x_{2}-x_{3}-5 x_{4}=-11 \\
3 x_{1}-x_{2}-x_{3}=3
\end{array}\right.
$$
$$
\text { II }\left\{\begin{array}{l}
x_{1}+(a+b) x_{2}-2 x_{3}-3 x_{4}=-16 \\
b x_{1}-x_{3}-2 x_{4}=-11 \\
x_{3}-2 x_{4}=c+1
\end{array}\right.
$$
(1) \begin{CJK}{UTF8}{mj}求\end{CJK} I \begin{CJK}{UTF8}{mj}的通解\end{CJK};

(2) \begin{CJK}{UTF8}{mj}若方程组\end{CJK} II \begin{CJK}{UTF8}{mj}的增广矩阵秩为\end{CJK} 2 , \begin{CJK}{UTF8}{mj}求\end{CJK} $a, b, c$;

(3) \begin{CJK}{UTF8}{mj}求\end{CJK} II \begin{CJK}{UTF8}{mj}的通解\end{CJK}.

\begin{enumerate}
  \setcounter{enumi}{9}
  \item \begin{CJK}{UTF8}{mj}求\end{CJK} $z_{1}=(1,1,1,1), z_{2}=(1,2,2,1), z_{3}=(2,3,3,2), z_{4}=(5,7,4,5)$ \begin{CJK}{UTF8}{mj}的秩和极大无关组\end{CJK}, \begin{CJK}{UTF8}{mj}并将其他向量用此极\end{CJK} \begin{CJK}{UTF8}{mj}大无关组表示出来\end{CJK}.

  \item $A$ \begin{CJK}{UTF8}{mj}实对称\end{CJK}, $\operatorname{tr}(A)=6$, \begin{CJK}{UTF8}{mj}特征值为\end{CJK} $1, \lambda$, \begin{CJK}{UTF8}{mj}对应特征向量是\end{CJK} $\left(\begin{array}{c}1 \\ -1 \\ 0\end{array}\right)\left(\begin{array}{c}1 \\ 0 \\ -1\end{array}\right)$, \begin{CJK}{UTF8}{mj}求\end{CJK} $A$.

\end{enumerate}
$11 .$
$$
B=\left(\begin{array}{ccc}
b_{1} & \cdots & b_{n} \\
\vdots & \ddots & \vdots \\
b_{1} & \cdots & b_{n}
\end{array}\right), A=B^{\prime} B
$$
\begin{CJK}{UTF8}{mj}求\end{CJK} $|\lambda E-A|$.

\begin{enumerate}
  \setcounter{enumi}{12}
  \item \begin{CJK}{UTF8}{mj}叙述并证明艾森斯坦判别法\end{CJK}.
\end{enumerate}
$13 .$
$$
A=\left(\begin{array}{cccc}
a_{1} & a_{2} & \cdots & a_{n} \\
a_{n} & a_{1} & \cdots & a_{n-1} \\
\vdots & \vdots & & \vdots \\
a_{2} & a_{3} & \cdots & a_{1}
\end{array}\right)
$$
\begin{CJK}{UTF8}{mj}证明\end{CJK}: \begin{CJK}{UTF8}{mj}是\end{CJK} $M_{n}(\mathbb{F})$ \begin{CJK}{UTF8}{mj}的一个子空间\end{CJK}, \begin{CJK}{UTF8}{mj}求它的基和维数\end{CJK}.

\begin{enumerate}
  \setcounter{enumi}{14}
  \item $A$ \begin{CJK}{UTF8}{mj}是复数域上一矩阵\end{CJK}, $A^{3}+A^{2}-2 E=0$ \begin{CJK}{UTF8}{mj}的充要条件是\end{CJK}
\end{enumerate}
$$
\mathrm{r}(A-E)+\mathrm{r}\left(A^{2}+2 A+2 E\right)=n .
$$

\section{1. 上海大学 2018 年研究生入学考试试题高等代数(回忆版)}
\begin{CJK}{UTF8}{mj}李扬\end{CJK}

\begin{CJK}{UTF8}{mj}微信公众号\end{CJK}: sxkyliyang

\section{一. 填空题}
\begin{enumerate}
  \item $B_{2 \times 3} A_{3 \times 2}$ \begin{CJK}{UTF8}{mj}的特征多项式为\end{CJK} $(\lambda-1)(\lambda-2)$, \begin{CJK}{UTF8}{mj}求\end{CJK} $A_{3 \times 2} B_{2 \times 3}$ \begin{CJK}{UTF8}{mj}与\end{CJK} \begin{CJK}{UTF8}{mj}相似\end{CJK}.

  \item $f(x)$ \begin{CJK}{UTF8}{mj}无重因式的充分必要条件\end{CJK}

  \item $A$ \begin{CJK}{UTF8}{mj}是实对称可逆矩阵\end{CJK}, \begin{CJK}{UTF8}{mj}求\end{CJK} $X^{T} A^{2} X$ \begin{CJK}{UTF8}{mj}的规范型\end{CJK}

  \item $\mathscr{A}$ \begin{CJK}{UTF8}{mj}是三维空间\end{CJK} $\mathbb{R}^{3}$ \begin{CJK}{UTF8}{mj}投影到\end{CJK} $x o y$ \begin{CJK}{UTF8}{mj}上的投影\end{CJK}, \begin{CJK}{UTF8}{mj}求值域\end{CJK} \begin{CJK}{UTF8}{mj}和核\end{CJK}

  \item $\left(\begin{array}{ccc}0 & 1 & 2 \\ -1 & 0 & 3 \\ 0 & 0 & -1\end{array}\right)^{n}=$

\end{enumerate}
\begin{CJK}{UTF8}{mj}二\end{CJK}. \begin{CJK}{UTF8}{mj}是非题\end{CJK} (\begin{CJK}{UTF8}{mj}正确的请证明\end{CJK}, \begin{CJK}{UTF8}{mj}错误的举出反例\end{CJK})

\begin{enumerate}
  \item $A^{2}=0$ \begin{CJK}{UTF8}{mj}的充分必要条件是\end{CJK} $A=0$.

  \item \begin{CJK}{UTF8}{mj}如果\end{CJK} $V_{1}, V_{2} \in V, V_{1}+V_{2}=V$, \begin{CJK}{UTF8}{mj}则\end{CJK} $\operatorname{dim} V_{1}+\operatorname{dim} V_{2}=\operatorname{dim} V$.

  \item \begin{CJK}{UTF8}{mj}可逆矩阵的乘积还是逆矩阵\end{CJK}.

  \item $A, B$ \begin{CJK}{UTF8}{mj}是正定矩阵\end{CJK}, \begin{CJK}{UTF8}{mj}则\end{CJK} $\left(\begin{array}{cc}A & A \\ A & A-B\end{array}\right)$ \begin{CJK}{UTF8}{mj}也是正定矩阵\end{CJK}.

  \item $A^{2}=A$, \begin{CJK}{UTF8}{mj}则\end{CJK} $\mathrm{r}(A)+\mathrm{r}(A-I)=n$.

\end{enumerate}
\begin{CJK}{UTF8}{mj}三\end{CJK}. \begin{CJK}{UTF8}{mj}计算和证明\end{CJK}

\begin{enumerate}
  \item \begin{CJK}{UTF8}{mj}求解方程组\end{CJK}.

  \item \begin{CJK}{UTF8}{mj}常数\end{CJK} $a, b, b<0$,

\end{enumerate}
$$
A=\left(\begin{array}{l}
b \\
a
\end{array}\right),|A|=-6
$$
\begin{CJK}{UTF8}{mj}求可逆矩阵\end{CJK} $P$, \begin{CJK}{UTF8}{mj}使得\end{CJK} $P^{-1} A P$ \begin{CJK}{UTF8}{mj}可对角化\end{CJK}.

\begin{enumerate}
  \setcounter{enumi}{3}
  \item (\begin{CJK}{UTF8}{mj}往年有很多类似的题\end{CJK}) $A$ \begin{CJK}{UTF8}{mj}和\end{CJK} $B$ \begin{CJK}{UTF8}{mj}为整数矩阵\end{CJK}, $B^{2}=I-A B$
\end{enumerate}
(1) \begin{CJK}{UTF8}{mj}证明\end{CJK}: $|A+B|^{2}=1$.

(2) \begin{CJK}{UTF8}{mj}若\end{CJK} $B^{-1}=()$, \begin{CJK}{UTF8}{mj}求\end{CJK} $(A+2 B)^{-1}$.

\begin{enumerate}
  \setcounter{enumi}{4}
  \item (\begin{CJK}{UTF8}{mj}忘记哪一年了\end{CJK}, \begin{CJK}{UTF8}{mj}和这道题几乎一致\end{CJK}) \begin{CJK}{UTF8}{mj}矩阵\end{CJK} $A$ \begin{CJK}{UTF8}{mj}的一行之和为\end{CJK} $a$
\end{enumerate}
(1) \begin{CJK}{UTF8}{mj}证明\end{CJK}: $\sum_{j=1}^{n}=a^{-1}|A|,(i=1,2,3, \cdots)$, \begin{CJK}{UTF8}{mj}其中\end{CJK} $A_{i j}$ \begin{CJK}{UTF8}{mj}是\end{CJK} $A$ \begin{CJK}{UTF8}{mj}的代数余子式\end{CJK}.

(2) \begin{CJK}{UTF8}{mj}若\end{CJK} $a_{i j}$ \begin{CJK}{UTF8}{mj}是整数\end{CJK}, \begin{CJK}{UTF8}{mj}证明\end{CJK} $a$ \begin{CJK}{UTF8}{mj}整除\end{CJK} $|A|$.

\begin{enumerate}
  \setcounter{enumi}{5}
  \item \begin{CJK}{UTF8}{mj}叙述并证明二次型惯性定理\end{CJK}.

  \item $V$ \begin{CJK}{UTF8}{mj}是非零向量\end{CJK}, $A$ \begin{CJK}{UTF8}{mj}是\end{CJK} $n$ \begin{CJK}{UTF8}{mj}阶可逆矩阵\end{CJK}, \begin{CJK}{UTF8}{mj}若\end{CJK} $f(A) V=0$, \begin{CJK}{UTF8}{mj}则\end{CJK} $f(x)$ \begin{CJK}{UTF8}{mj}称为\end{CJK} $(A, V)$ \begin{CJK}{UTF8}{mj}的零化多项式\end{CJK}.

\end{enumerate}
(2) $m(x)$ \begin{CJK}{UTF8}{mj}是\end{CJK} $r$ \begin{CJK}{UTF8}{mj}次的最小多项式\end{CJK}, \begin{CJK}{UTF8}{mj}证\end{CJK}: $V, A V, A^{2} V, \cdots, A^{r-1} V$ \begin{CJK}{UTF8}{mj}线性无关\end{CJK}. (3) $A$ \begin{CJK}{UTF8}{mj}与\end{CJK} $\left(\begin{array}{cccccc}0 & & & & & -a_{0}^{n-1} \\ 1 & 0 & & & & -a_{1}^{n-1} \\ & 1 & \ddots & & & \vdots \\ & & \ddots & 0 & & \vdots \\ & & & 1 & 0 & -a_{n}^{n-1}\end{array}\right)$

\begin{enumerate}
  \setcounter{enumi}{7}
  \item $S=\left\{A_{1}, A_{2}, \cdots, A_{n}\right\}$ \begin{CJK}{UTF8}{mj}是复逆矩阵\end{CJK}, $S=\sum_{r=1}^{n} A_{r}$ \begin{CJK}{UTF8}{mj}证明\end{CJK}
\end{enumerate}
(2) $S=0 \Leftrightarrow \operatorname{tr}(S)=0$.

\section{1. 上海交通大学 2010 年硕士研究生入学考试试题 (828 高等代数) 
 李扬 
 微信公众号: sxkyliyang}
\begin{CJK}{UTF8}{mj}一\end{CJK}、 (20 \begin{CJK}{UTF8}{mj}分\end{CJK}) \begin{CJK}{UTF8}{mj}计算行列式\end{CJK}

$\begin{array}{ll}\text { (1) } D_{n+1}= & \left|\begin{array}{ccccc}a_{1}^{n} & a_{1}^{n-1} b_{1} & \cdots & a_{1} b_{1}^{n-1} & b_{1}^{n} \\ a_{2}^{n} & a_{2}^{n-1} b_{2} & \cdots & a_{2} b_{2}^{n-1} & b_{2}^{n} \\ \vdots & \vdots & & \vdots & \vdots \\ a_{n+1}^{n} & a_{n+1}^{n-1} b_{n+1} & \cdots & a_{n+1} b_{n+1}^{n-1} & b_{n+1}^{n}\end{array}\right| \\ \text { (2) } D_{n}=\left|\begin{array}{ccccc}1+a_{1}+b_{1} & a_{1}+b_{2} & \cdots & a_{1}+b_{n} \\ a_{2}+b_{1} & 1+a_{2}+b_{2} & \cdots & a_{2}+b_{n} \\ \vdots & \vdots & & \vdots \\ a_{n}+b_{1} & a_{n}+b_{2} & \cdots & 1+a_{n}+b_{n}\end{array}\right|\end{array}$

\begin{CJK}{UTF8}{mj}二\end{CJK}、 $a, b$ \begin{CJK}{UTF8}{mj}为何值时\end{CJK}, \begin{CJK}{UTF8}{mj}下列方程组\end{CJK}
$$
\left\{\begin{array}{c}
a x_{1}+x_{2}+x_{3}=1 \\
2 x_{1}+x_{2}+b x_{3}=3 \\
2 x_{1}+x_{2}+3 b x_{3}=1
\end{array}\right.
$$
\begin{CJK}{UTF8}{mj}无解\end{CJK}, \begin{CJK}{UTF8}{mj}有唯一解\end{CJK}, \begin{CJK}{UTF8}{mj}有无穷多解\end{CJK}? \begin{CJK}{UTF8}{mj}并在有无穷多解时写出方程组的通解\end{CJK}.

\begin{CJK}{UTF8}{mj}三\end{CJK}、\begin{CJK}{UTF8}{mj}求多项式\end{CJK} $f(x)=x^{3}-6 x^{2}+15 x-14$ \begin{CJK}{UTF8}{mj}的全部复根\end{CJK}.

\begin{CJK}{UTF8}{mj}四\end{CJK}、\begin{CJK}{UTF8}{mj}证明存在多项式\end{CJK} $f(x)$ \begin{CJK}{UTF8}{mj}满足\end{CJK}
$$
(x-1)^{n}\left|(f(x)+1),(x+1)^{n}\right|(f(x)-1) .
$$
\begin{CJK}{UTF8}{mj}五\end{CJK}、\begin{CJK}{UTF8}{mj}设\end{CJK} $V$ \begin{CJK}{UTF8}{mj}为数域\end{CJK} $F$ \begin{CJK}{UTF8}{mj}上的\end{CJK} $n$ \begin{CJK}{UTF8}{mj}维线性空间\end{CJK}, $\mathscr{A}$ \begin{CJK}{UTF8}{mj}为\end{CJK} $V$ \begin{CJK}{UTF8}{mj}上的线性变换满足\end{CJK} $\mathscr{A}^{3}-2 \mathscr{A}^{2}-\mathscr{A}=-2 i d$, \begin{CJK}{UTF8}{mj}其中\end{CJK} $i d$ \begin{CJK}{UTF8}{mj}为\end{CJK} $V$ \begin{CJK}{UTF8}{mj}上恒\end{CJK} \begin{CJK}{UTF8}{mj}等变换\end{CJK}.

(1) $\mathscr{A}$ \begin{CJK}{UTF8}{mj}是否可对角化\end{CJK}, \begin{CJK}{UTF8}{mj}若是\end{CJK}, \begin{CJK}{UTF8}{mj}请证明\end{CJK}.

(2) \begin{CJK}{UTF8}{mj}令\end{CJK} $V_{1}=\{(\mathscr{A}-2 i d) v \mid v \in V\}, V_{2}=\left\{\left(\mathscr{A}^{2}-i d\right) v \mid v \in V\right\}$. \begin{CJK}{UTF8}{mj}证明\end{CJK}: $V=V_{1} \oplus V_{2}$.

\begin{CJK}{UTF8}{mj}六\end{CJK}、\begin{CJK}{UTF8}{mj}设\end{CJK} $n$ \begin{CJK}{UTF8}{mj}阶方阵\end{CJK} $A$ \begin{CJK}{UTF8}{mj}可逆\end{CJK}, $\alpha=\left(\alpha_{1}, \alpha_{2}, \cdots, \alpha_{n}\right)^{\prime}$, \begin{CJK}{UTF8}{mj}试证\end{CJK}:
$$
\left|A-\alpha \alpha^{\prime}\right|=\left(1-\alpha^{\prime} A^{-1} \alpha\right)|A| .
$$
\begin{CJK}{UTF8}{mj}七\end{CJK}、\begin{CJK}{UTF8}{mj}设\end{CJK} $m \times r$ \begin{CJK}{UTF8}{mj}阶矩阵\end{CJK} $A=\left(a_{i j}\right)$ \begin{CJK}{UTF8}{mj}是列满秩的\end{CJK}, $A^{*}$ \begin{CJK}{UTF8}{mj}为\end{CJK} $A$ \begin{CJK}{UTF8}{mj}的共轭转置\end{CJK}, \begin{CJK}{UTF8}{mj}证明\end{CJK}: $A^{*} A$ \begin{CJK}{UTF8}{mj}的行列式大于\end{CJK} 0

\begin{CJK}{UTF8}{mj}八\end{CJK}、\begin{CJK}{UTF8}{mj}求西矩阵\end{CJK} $P$, \begin{CJK}{UTF8}{mj}使\end{CJK} $P^{*} A P$ \begin{CJK}{UTF8}{mj}为对角矩阵\end{CJK}, \begin{CJK}{UTF8}{mj}其中\end{CJK} $A=\left(\begin{array}{ccc}-1 & i & 0 \\ -i & 0 & -i \\ 0 & i & -1\end{array}\right)$.

\begin{CJK}{UTF8}{mj}九\end{CJK}、\begin{CJK}{UTF8}{mj}设\end{CJK} $\mathscr{A}$ \begin{CJK}{UTF8}{mj}为\end{CJK} $n$ \begin{CJK}{UTF8}{mj}维复线性空间\end{CJK} $V$ \begin{CJK}{UTF8}{mj}上的线性变换\end{CJK}, \begin{CJK}{UTF8}{mj}证明对任意正整数\end{CJK} $r, 1 \leq r \leq n$, \begin{CJK}{UTF8}{mj}在\end{CJK} $V$ \begin{CJK}{UTF8}{mj}中均存在一个\end{CJK} $r$ \begin{CJK}{UTF8}{mj}维\end{CJK} $\mathscr{A}-$ \begin{CJK}{UTF8}{mj}不\end{CJK} \begin{CJK}{UTF8}{mj}变子空间\end{CJK}.

\begin{CJK}{UTF8}{mj}十\end{CJK}、\begin{CJK}{UTF8}{mj}设\end{CJK} $\mathscr{A}_{1}, \cdots, \mathscr{A}_{s}$ \begin{CJK}{UTF8}{mj}为数域\end{CJK} $F$ \begin{CJK}{UTF8}{mj}上\end{CJK} $n$ \begin{CJK}{UTF8}{mj}维线性空间\end{CJK} $V$ \begin{CJK}{UTF8}{mj}上的两两不同的线性变换\end{CJK}. \begin{CJK}{UTF8}{mj}证明\end{CJK}: \begin{CJK}{UTF8}{mj}存在\end{CJK} $v \in V$, \begin{CJK}{UTF8}{mj}满足\end{CJK} $\mathscr{A}_{i} v \neq \mathscr{A}_{j} v$, \begin{CJK}{UTF8}{mj}对任意\end{CJK} $i \neq j$.

\begin{CJK}{UTF8}{mj}十一\end{CJK}、\begin{CJK}{UTF8}{mj}设\end{CJK} $\mathscr{A}$ \begin{CJK}{UTF8}{mj}是\end{CJK} $n$ \begin{CJK}{UTF8}{mj}维欧氏空间中的正交变换\end{CJK}, $V_{1}$ \begin{CJK}{UTF8}{mj}是\end{CJK} $V$ \begin{CJK}{UTF8}{mj}的\end{CJK} $\mathscr{A}$ - \begin{CJK}{UTF8}{mj}不变子空间\end{CJK}. \begin{CJK}{UTF8}{mj}证明\end{CJK}: $V_{1}$ \begin{CJK}{UTF8}{mj}的正交补也是\end{CJK} $V$ \begin{CJK}{UTF8}{mj}的\end{CJK} $\mathscr{A}-$ \begin{CJK}{UTF8}{mj}不变\end{CJK} \begin{CJK}{UTF8}{mj}子空间\end{CJK}

\begin{CJK}{UTF8}{mj}十二\end{CJK}、\begin{CJK}{UTF8}{mj}设\end{CJK} $A, B$ \begin{CJK}{UTF8}{mj}均为\end{CJK} $n$ \begin{CJK}{UTF8}{mj}阶实对称阵\end{CJK}, \begin{CJK}{UTF8}{mj}证明\end{CJK}: $A B$ \begin{CJK}{UTF8}{mj}的特征值都大于零\end{CJK}.

\section{2. 上海交通大学 2011 年硕士研究生入学考试试题 (828 高等代数)}
\begin{CJK}{UTF8}{mj}李扬\end{CJK}

\begin{CJK}{UTF8}{mj}微信公众号\end{CJK}: sxkyliyang

\begin{CJK}{UTF8}{mj}一\end{CJK}、 (12 \begin{CJK}{UTF8}{mj}分\end{CJK}) \begin{CJK}{UTF8}{mj}判断多项式\end{CJK} $x^{4}+3 x^{3}+6 x^{2}+6 x+4$ \begin{CJK}{UTF8}{mj}在\end{CJK} $\mathbb{Q}$ \begin{CJK}{UTF8}{mj}上是否可约\end{CJK}, \begin{CJK}{UTF8}{mj}并说明理由\end{CJK}

\begin{CJK}{UTF8}{mj}二\end{CJK}、 (10 \begin{CJK}{UTF8}{mj}分\end{CJK}) \begin{CJK}{UTF8}{mj}用初等对称多项式表示出\end{CJK} $n$ \begin{CJK}{UTF8}{mj}元对称多项式\end{CJK} $\sum_{i=1}^{n} x_{i}^{2}$

\begin{CJK}{UTF8}{mj}三\end{CJK}、 (20 \begin{CJK}{UTF8}{mj}分\end{CJK}) \begin{CJK}{UTF8}{mj}计算行列式\end{CJK} $\left|\begin{array}{cccc}1+\sin \left(2 \theta_{1}\right) & \sin \left(\theta_{1}+\theta_{2}\right) & \cdots & \sin \left(\theta_{1}+\theta_{n}\right) \\ \sin \left(\theta_{2}+\theta_{1}\right) & 1+\sin \left(2 \theta_{2}\right) & \cdots & \sin \left(\theta_{2}+\theta_{n}\right) \\ \vdots & \vdots & & \vdots \\ \sin \left(\theta_{n}+\theta_{2}\right) & \sin \left(\theta_{n}+\theta_{2}\right) & \cdots & 1+\sin \left(2 \theta_{n}\right)\end{array}\right|$

(1) $\left|\begin{array}{ccccc}a_{1} & a_{2} & a_{3} & \cdots & a_{n} \\ 2 a_{n} & a_{1} & a_{2} & \cdots & a_{n-1} \\ 2 a_{n-1} & 2 a_{n} & a_{1} & \cdots & a_{n-2} \\ \vdots & \vdots & \vdots & & \vdots \\ 2 a_{2} & 2 a_{3} & 2 a_{4} & \cdots & a_{1}\end{array}\right|$

(2)

\begin{CJK}{UTF8}{mj}四\end{CJK}、 $(14$ \begin{CJK}{UTF8}{mj}分\end{CJK}) $\lambda$ \begin{CJK}{UTF8}{mj}为何值\end{CJK}, \begin{CJK}{UTF8}{mj}下列方程组\end{CJK}
$$
\left\{\begin{array}{c}
(\lambda+1) x_{1}+2 x_{2}+3 x_{3}=1 \\
x_{1}+(\lambda+2) x_{2}+3 x_{3}=1 \\
x_{1}+2 x_{2}+(\lambda+3) x_{3}=1
\end{array}\right.
$$
\begin{CJK}{UTF8}{mj}无解\end{CJK}, \begin{CJK}{UTF8}{mj}有唯一解\end{CJK}, \begin{CJK}{UTF8}{mj}有无穷多解\end{CJK}? \begin{CJK}{UTF8}{mj}并在有解时写出方程的解\end{CJK}.

\begin{CJK}{UTF8}{mj}五\end{CJK}、(10 \begin{CJK}{UTF8}{mj}分\end{CJK}) $A$ \begin{CJK}{UTF8}{mj}为\end{CJK} $n$ \begin{CJK}{UTF8}{mj}阶方阵\end{CJK}, \begin{CJK}{UTF8}{mj}其秩为\end{CJK} $l$, \begin{CJK}{UTF8}{mj}若矩阵方程\end{CJK} $A X=\beta$ ( $\beta$ \begin{CJK}{UTF8}{mj}为\end{CJK} $n \times 1$ \begin{CJK}{UTF8}{mj}阶向量\end{CJK}) \begin{CJK}{UTF8}{mj}有解\end{CJK}. \begin{CJK}{UTF8}{mj}求其所有解张成的线性空\end{CJK} \begin{CJK}{UTF8}{mj}间的维数\end{CJK}.

\begin{CJK}{UTF8}{mj}六\end{CJK}、(12 \begin{CJK}{UTF8}{mj}分\end{CJK}) $V$ \begin{CJK}{UTF8}{mj}是数域\end{CJK} $F$ \begin{CJK}{UTF8}{mj}上的\end{CJK} $n$ \begin{CJK}{UTF8}{mj}维线性空间\end{CJK}, $\mathscr{A}$ \begin{CJK}{UTF8}{mj}为\end{CJK} $V$ \begin{CJK}{UTF8}{mj}上的线性变换\end{CJK}. \begin{CJK}{UTF8}{mj}若\end{CJK} $\mathscr{A}$ \begin{CJK}{UTF8}{mj}在\end{CJK} $V$ \begin{CJK}{UTF8}{mj}的某组基下矩阵为对角形\end{CJK}, \begin{CJK}{UTF8}{mj}证\end{CJK} \begin{CJK}{UTF8}{mj}明\end{CJK}: $\mathscr{A}$ \begin{CJK}{UTF8}{mj}的任一不变子空间\end{CJK} $V^{\prime}$ \begin{CJK}{UTF8}{mj}也存在某组基\end{CJK}, \begin{CJK}{UTF8}{mj}使\end{CJK} $\mathscr{A} \mid V^{\prime}$ \begin{CJK}{UTF8}{mj}在该组基下的矩阵为对角阵\end{CJK}.

\begin{CJK}{UTF8}{mj}七\end{CJK}、 $(12$ \begin{CJK}{UTF8}{mj}分\end{CJK} $)$ \begin{CJK}{UTF8}{mj}一矩阵\end{CJK} $P$ \begin{CJK}{UTF8}{mj}称为西阵\end{CJK}, \begin{CJK}{UTF8}{mj}若\end{CJK} $P P^{*}=E, P^{*}$ \begin{CJK}{UTF8}{mj}为\end{CJK} $P$ \begin{CJK}{UTF8}{mj}的共轭转置\end{CJK}, \begin{CJK}{UTF8}{mj}求西阵\end{CJK} $P$, \begin{CJK}{UTF8}{mj}使\end{CJK} $P^{*} A P$ \begin{CJK}{UTF8}{mj}为对角阵\end{CJK}, \begin{CJK}{UTF8}{mj}其中\end{CJK}
$$
A=\left(\begin{array}{ccc}
0 & 0 & 3 \\
0 & 0 & 4 \\
-3 & -4 & 0
\end{array}\right)
$$
\begin{CJK}{UTF8}{mj}八\end{CJK}、(12 \begin{CJK}{UTF8}{mj}分\end{CJK}) $V$ \begin{CJK}{UTF8}{mj}为代数闭域\end{CJK} $F$ \begin{CJK}{UTF8}{mj}上的\end{CJK} $n$ \begin{CJK}{UTF8}{mj}维线性空间\end{CJK}, $\mathscr{A}$ \begin{CJK}{UTF8}{mj}为\end{CJK} $V$ \begin{CJK}{UTF8}{mj}上的线性变换\end{CJK}. \begin{CJK}{UTF8}{mj}证明\end{CJK}: \begin{CJK}{UTF8}{mj}存在唯一的\end{CJK} $V$ \begin{CJK}{UTF8}{mj}上的线性变换\end{CJK} $\mathscr{B}$ \begin{CJK}{UTF8}{mj}和\end{CJK} $\mathscr{C}$, \begin{CJK}{UTF8}{mj}满足\end{CJK} $\mathscr{A}=\mathscr{B}+\mathscr{C}, \mathscr{B}$ \begin{CJK}{UTF8}{mj}是可对角化的\end{CJK}, $\mathscr{C}$ \begin{CJK}{UTF8}{mj}是幂零的\end{CJK}, \begin{CJK}{UTF8}{mj}且\end{CJK} $\mathscr{B}$ \begin{CJK}{UTF8}{mj}与\end{CJK} $\mathscr{C}$ \begin{CJK}{UTF8}{mj}可交换\end{CJK}.

\begin{CJK}{UTF8}{mj}九\end{CJK}、 (12 \begin{CJK}{UTF8}{mj}分\end{CJK}) \begin{CJK}{UTF8}{mj}证\end{CJK}: \begin{CJK}{UTF8}{mj}任意正交矩阵都可以表示为两个实对称矩阵的乘积\end{CJK}.

\begin{CJK}{UTF8}{mj}十\end{CJK}、 (12 \begin{CJK}{UTF8}{mj}分\end{CJK}) $H$ \begin{CJK}{UTF8}{mj}为\end{CJK} Hermite \begin{CJK}{UTF8}{mj}阵\end{CJK}, \begin{CJK}{UTF8}{mj}证明\end{CJK}: $H$ \begin{CJK}{UTF8}{mj}正定当且仅当\end{CJK} $H$ \begin{CJK}{UTF8}{mj}的所有主子式均为正实数\end{CJK}.

\begin{CJK}{UTF8}{mj}十一\end{CJK}、 (12 \begin{CJK}{UTF8}{mj}分\end{CJK}) $A$ \begin{CJK}{UTF8}{mj}为复数域上\end{CJK} $n$ \begin{CJK}{UTF8}{mj}阶方阵\end{CJK}, \begin{CJK}{UTF8}{mj}且\end{CJK} $A^{k}=0, A^{k-1} \neq 0$, \begin{CJK}{UTF8}{mj}秩\end{CJK} $\left(A^{i}\right)=a_{i}(i=1,2, \cdots, k-1)$, \begin{CJK}{UTF8}{mj}试给出\end{CJK} $A$ \begin{CJK}{UTF8}{mj}的\end{CJK} Jordan \begin{CJK}{UTF8}{mj}标准形\end{CJK}.

\begin{CJK}{UTF8}{mj}十二\end{CJK}、 (12 \begin{CJK}{UTF8}{mj}分\end{CJK}) \begin{CJK}{UTF8}{mj}设\end{CJK} $\mathscr{A}$ \begin{CJK}{UTF8}{mj}为\end{CJK} Euclid \begin{CJK}{UTF8}{mj}空间\end{CJK} $V$ \begin{CJK}{UTF8}{mj}上的正规变换\end{CJK}, \begin{CJK}{UTF8}{mj}即\end{CJK} $\mathscr{A} \mathscr{A}^{*}=\mathscr{A}^{*} \mathscr{A}, U$ \begin{CJK}{UTF8}{mj}是\end{CJK} $\mathscr{A}-$ \begin{CJK}{UTF8}{mj}不变子空间\end{CJK}, \begin{CJK}{UTF8}{mj}证明\end{CJK} $U$ \begin{CJK}{UTF8}{mj}也是\end{CJK} $\mathscr{A}^{*}-$

\section{3. 上海交通大学 2013 年硕士研究生入学考试试题 (828 高等代数) 
 李扬 
 微信公众号: sxkyliyang}
\begin{CJK}{UTF8}{mj}一\end{CJK}、\begin{CJK}{UTF8}{mj}判断题\end{CJK} ( 10 \begin{CJK}{UTF8}{mj}分\end{CJK})

\begin{enumerate}
  \item \begin{CJK}{UTF8}{mj}设\end{CJK} $A, B$ \begin{CJK}{UTF8}{mj}为\end{CJK} $n$ \begin{CJK}{UTF8}{mj}级方阵\end{CJK} $(n>1)$. \begin{CJK}{UTF8}{mj}则\end{CJK} $A^{2}-B^{2}=(A-B)(A+B)$.

  \item \begin{CJK}{UTF8}{mj}若\end{CJK} $n$ \begin{CJK}{UTF8}{mj}维向量\end{CJK} $b$ \begin{CJK}{UTF8}{mj}不在由\end{CJK} $A$ \begin{CJK}{UTF8}{mj}的列向量生成的列向量空间中\end{CJK}, \begin{CJK}{UTF8}{mj}则方程组\end{CJK} $A x=b$ \begin{CJK}{UTF8}{mj}无解\end{CJK}.

  \item \begin{CJK}{UTF8}{mj}若矩阵\end{CJK} $C_{m \times n}$ \begin{CJK}{UTF8}{mj}为列独立阵\end{CJK}, \begin{CJK}{UTF8}{mj}则存在可逆阵\end{CJK} $P$ \begin{CJK}{UTF8}{mj}使得\end{CJK} $C=P\left[\begin{array}{c}I_{n} \\ O\end{array}\right]$.

  \item \begin{CJK}{UTF8}{mj}设\end{CJK} $f(X)=X^{\prime} A X$ \begin{CJK}{UTF8}{mj}是一\end{CJK} $n$ \begin{CJK}{UTF8}{mj}元实二次型\end{CJK}, \begin{CJK}{UTF8}{mj}则\end{CJK} $f(X)=0$ \begin{CJK}{UTF8}{mj}的解集合是\end{CJK} $R^{n}$ \begin{CJK}{UTF8}{mj}的子空间\end{CJK}.

  \item \begin{CJK}{UTF8}{mj}设\end{CJK} $V_{1}, V_{2}, W$ \begin{CJK}{UTF8}{mj}都是欧氏空间\end{CJK} $V$ \begin{CJK}{UTF8}{mj}的子空间\end{CJK}, \begin{CJK}{UTF8}{mj}如果\end{CJK} $V_{1} \oplus W=V=V_{2} \oplus W$ \begin{CJK}{UTF8}{mj}且\end{CJK} $V_{1} \perp W$, \begin{CJK}{UTF8}{mj}则\end{CJK} $V_{1}=V_{2}$.

\end{enumerate}
\begin{CJK}{UTF8}{mj}二\end{CJK}、 (20\begin{CJK}{UTF8}{mj}分\end{CJK})

\begin{enumerate}
  \item \begin{CJK}{UTF8}{mj}求出下述行列式所表示的一元多项式\end{CJK} $f(x)$ \begin{CJK}{UTF8}{mj}的最高次幂项\end{CJK}:
\end{enumerate}
$$
f(x)=\left(\begin{array}{cccccc}
a_{1} & a_{2} & a_{3} & a_{4} & \cdots & a_{n} \\
a_{n} & a_{1} & a_{2} & a_{3} & \cdots & a_{n-1} \\
a_{n-1} & x & a_{1} & a_{2} & \cdots & a_{n-2} \\
\vdots & \vdots & \vdots & \vdots & \vdots & \vdots \\
a_{3} & x & \cdots & x & a_{1} & a_{2} \\
a_{2} & x & \cdots & x & x & a_{1}
\end{array}\right)
$$
\begin{CJK}{UTF8}{mj}其中\end{CJK}: $a_{1}, a_{2}, \cdots, a_{n}$ \begin{CJK}{UTF8}{mj}为数域\end{CJK} $P$ \begin{CJK}{UTF8}{mj}中的数\end{CJK}.

\begin{enumerate}
  \setcounter{enumi}{2}
  \item \begin{CJK}{UTF8}{mj}将二次型\end{CJK} $x_{1}^{2}+2 x_{2}^{2}+3 x_{3}^{2}+4 x_{1} x_{2}+2 x_{1} x_{3}+2 x_{2} x_{3}$ \begin{CJK}{UTF8}{mj}化为标准型\end{CJK}, \begin{CJK}{UTF8}{mj}并给出所用的线性替换\end{CJK}.
\end{enumerate}
\begin{CJK}{UTF8}{mj}三\end{CJK}、(15 \begin{CJK}{UTF8}{mj}分\end{CJK}) \begin{CJK}{UTF8}{mj}讨论线性方程组\end{CJK} $A x=b$ \begin{CJK}{UTF8}{mj}的可解性\end{CJK}, \begin{CJK}{UTF8}{mj}其中\end{CJK}
$$
A=\left(\begin{array}{cccc}
1 & 1 & 1 & 1 \\
2 & 1 & 3 & -3 \\
3 & 2 & c & -2 \\
4 & 3 & 5 & a
\end{array}\right) . b=\left(\begin{array}{c}
1 \\
-1 \\
0 \\
1
\end{array}\right)
$$
\begin{CJK}{UTF8}{mj}在有无穷多解时求通解\end{CJK}(\begin{CJK}{UTF8}{mj}要求用向量形式表示\end{CJK})

\begin{CJK}{UTF8}{mj}四\end{CJK}、(10 \begin{CJK}{UTF8}{mj}分\end{CJK})\begin{CJK}{UTF8}{mj}设线性空间\end{CJK} $V$ \begin{CJK}{UTF8}{mj}上的线性变换\end{CJK} $\mathscr{A}$ \begin{CJK}{UTF8}{mj}满足\end{CJK} $\mathscr{A}^{2}=\mathscr{A}$, \begin{CJK}{UTF8}{mj}证明\end{CJK}:

(1) $\beta \in \operatorname{Im} \mathscr{A}$ \begin{CJK}{UTF8}{mj}当且仅当\end{CJK} $\mathscr{A} \beta=\beta$ :

(2) $V=\operatorname{Im} \mathscr{A} \oplus \operatorname{ker} \mathscr{A}$.

\begin{CJK}{UTF8}{mj}五\end{CJK}、(20 \begin{CJK}{UTF8}{mj}分\end{CJK})

(1) \begin{CJK}{UTF8}{mj}在\end{CJK} $R^{2}$ \begin{CJK}{UTF8}{mj}中内积定义为\end{CJK}
$$
\langle x, y\rangle=4 x_{1} y_{1}+x_{2} y_{2} .
$$
\begin{CJK}{UTF8}{mj}其中\end{CJK} $x=\left(x_{1}, x_{2}\right), y=\left(y_{1}, y_{2}\right)^{\prime} \in R^{2}$. \begin{CJK}{UTF8}{mj}令\end{CJK} $S=\{x:\|x\|=1\},\|\|$ \begin{CJK}{UTF8}{mj}表示向量的长度\end{CJK}, \begin{CJK}{UTF8}{mj}说明\end{CJK} $S$ \begin{CJK}{UTF8}{mj}是什么形状的图\end{CJK} \begin{CJK}{UTF8}{mj}形\end{CJK}, \begin{CJK}{UTF8}{mj}并画出草图\end{CJK}.

(2) \begin{CJK}{UTF8}{mj}令\end{CJK}
$$
W=\left\{\left[\begin{array}{ll}
a & b \\
c & d
\end{array}\right]: 2 a-b+3 c+d=0, \quad a, b, c, d \in \mathbb{R}\right\}
$$
\begin{CJK}{UTF8}{mj}证明\end{CJK} $W$ \begin{CJK}{UTF8}{mj}关于矩阵的加法和数乘成为\end{CJK} $\mathbb{R}$ \begin{CJK}{UTF8}{mj}上的线性空间\end{CJK}, \begin{CJK}{UTF8}{mj}并求出\end{CJK} $W$ \begin{CJK}{UTF8}{mj}的维数\end{CJK}, \begin{CJK}{UTF8}{mj}给出\end{CJK} $W$ \begin{CJK}{UTF8}{mj}的一组基\end{CJK}.

\begin{CJK}{UTF8}{mj}六\end{CJK}、 ( 15 \begin{CJK}{UTF8}{mj}分\end{CJK}) \begin{CJK}{UTF8}{mj}求证\end{CJK}: \begin{CJK}{UTF8}{mj}任一复矩阵\end{CJK} $A$ \begin{CJK}{UTF8}{mj}均可分解为\end{CJK} $A=B+C$, \begin{CJK}{UTF8}{mj}其中\end{CJK} $C$ \begin{CJK}{UTF8}{mj}为幂零阵\end{CJK} (\begin{CJK}{UTF8}{mj}即有正整数\end{CJK} $k$, \begin{CJK}{UTF8}{mj}使\end{CJK} $\left.C^{k}=0\right), B$ \begin{CJK}{UTF8}{mj}相似\end{CJK} \begin{CJK}{UTF8}{mj}于对角形\end{CJK}, \begin{CJK}{UTF8}{mj}且\end{CJK} $B C=C B$. \begin{CJK}{UTF8}{mj}七\end{CJK}、 $(10$ \begin{CJK}{UTF8}{mj}分\end{CJK} $)$ \begin{CJK}{UTF8}{mj}设\end{CJK} $f(x), g(x), h(x)$ \begin{CJK}{UTF8}{mj}为实系数多项式\end{CJK}, $h(x)$ \begin{CJK}{UTF8}{mj}首项系数为\end{CJK} 1 , \begin{CJK}{UTF8}{mj}求证\end{CJK}:
$$
(f(x) h(x), g(x) h(x))=(f(x), g(x)) h(x)
$$
\begin{CJK}{UTF8}{mj}八\end{CJK}、 $(20$ \begin{CJK}{UTF8}{mj}分\end{CJK} $)$ \begin{CJK}{UTF8}{mj}令\end{CJK}
$$
A=\left(\begin{array}{ccccc}
1 & 0 & -1 & 2 & 1 \\
-1 & 1 & 3 & -1 & 0 \\
-2 & 1 & 4 & -1 & 3 \\
3 & -1 & -5 & 1 & -6
\end{array}\right)
$$
(1) \begin{CJK}{UTF8}{mj}求\end{CJK} $5 \times 5$ \begin{CJK}{UTF8}{mj}阶秩为\end{CJK} 2 \begin{CJK}{UTF8}{mj}的矩阵\end{CJK} $M$, \begin{CJK}{UTF8}{mj}使得\end{CJK} $A M=O$ (\begin{CJK}{UTF8}{mj}零矩阵\end{CJK})

(2) \begin{CJK}{UTF8}{mj}假如\end{CJK} $B$ \begin{CJK}{UTF8}{mj}是满足\end{CJK} $A B=O$ \begin{CJK}{UTF8}{mj}的\end{CJK} $5 \times 5$ \begin{CJK}{UTF8}{mj}阶矩阵\end{CJK}, \begin{CJK}{UTF8}{mj}证明\end{CJK}: \begin{CJK}{UTF8}{mj}秩\end{CJK} $\operatorname{rank}(B) \leq 2$.

\begin{CJK}{UTF8}{mj}九\end{CJK}、 ( 15 \begin{CJK}{UTF8}{mj}分\end{CJK}) \begin{CJK}{UTF8}{mj}设\end{CJK} $A \neq 0$ \begin{CJK}{UTF8}{mj}是\end{CJK} $m \times n$ \begin{CJK}{UTF8}{mj}矩阵\end{CJK}, $\beta=\left(b_{1}, b_{2}, \cdots, b_{m}\right)^{\prime}$, \begin{CJK}{UTF8}{mj}证明线性方程组\end{CJK} $A X=\beta$ \begin{CJK}{UTF8}{mj}有解的充要条件是\end{CJK}:\begin{CJK}{UTF8}{mj}齐次\end{CJK} \begin{CJK}{UTF8}{mj}线性方程组\end{CJK} $A^{\prime} Y=0$ \begin{CJK}{UTF8}{mj}的每一个解\end{CJK} $v=\left(v_{1}, v_{2}, \cdots, v_{m}\right)^{\prime}$ \begin{CJK}{UTF8}{mj}都满足\end{CJK} $v^{\prime} \beta=0$, \begin{CJK}{UTF8}{mj}即\end{CJK} $\beta$ \begin{CJK}{UTF8}{mj}与\end{CJK} $A^{\prime} Y=0$ \begin{CJK}{UTF8}{mj}的解空间正交\end{CJK}.

\begin{CJK}{UTF8}{mj}十\end{CJK}、 ( 15 \begin{CJK}{UTF8}{mj}分\end{CJK}) \begin{CJK}{UTF8}{mj}用\end{CJK} $R$ \begin{CJK}{UTF8}{mj}表示实数域\end{CJK}, \begin{CJK}{UTF8}{mj}定义\end{CJK} $R^{n}$ \begin{CJK}{UTF8}{mj}到\end{CJK} $R$ \begin{CJK}{UTF8}{mj}的映射\end{CJK} $f$ \begin{CJK}{UTF8}{mj}如下\end{CJK}:
$$
f(X)=\left|x_{1}\right|+\cdots+\left|x_{r}\right|-\left|x_{r+1}\right|-\cdots-\left|x_{r+s}\right|, \forall X=\left(x_{1}, x_{2}, \cdots, x_{n}\right)^{\prime} \in R^{n}
$$
\begin{CJK}{UTF8}{mj}其中\end{CJK} $r \geq s \geq 0$. \begin{CJK}{UTF8}{mj}证明\end{CJK}:

(1) \begin{CJK}{UTF8}{mj}存在\end{CJK} $R^{n}$ \begin{CJK}{UTF8}{mj}的一个\end{CJK} $n-r$ \begin{CJK}{UTF8}{mj}维子空间\end{CJK} $W$, \begin{CJK}{UTF8}{mj}使得\end{CJK} $f(X)=0, \forall X \in W$

(2) \begin{CJK}{UTF8}{mj}若\end{CJK} $W_{1}, W_{2}$ \begin{CJK}{UTF8}{mj}是\end{CJK} $R^{n}$ \begin{CJK}{UTF8}{mj}的两个\end{CJK} $n-r$ \begin{CJK}{UTF8}{mj}维子空间\end{CJK}, \begin{CJK}{UTF8}{mj}且满足\end{CJK}
$$
f(X)=0, \forall X \in W_{1} \cup W_{2},
$$
\begin{CJK}{UTF8}{mj}则一定有\end{CJK} $\operatorname{dim}\left(W_{1} \cap W_{2}\right) \geq n-(r+s)$.

\section{4. 上海交通大学 2014 年硕士研究生入学考试试题 (828 高等代数) 
 李扬 
 微信公众号: sxkyliyang}
\begin{CJK}{UTF8}{mj}一\end{CJK}、 (10 \begin{CJK}{UTF8}{mj}分\end{CJK}) \begin{CJK}{UTF8}{mj}计算下列行列式的值\end{CJK}.
$$
\left|\begin{array}{cccc}
1 & 1 & 1 & 1 \\
a & b & c & d \\
a^{2} & b^{2} & c^{2} & d^{2} \\
a^{4} & b^{4} & c^{4} & d^{4}
\end{array}\right|
$$
\begin{CJK}{UTF8}{mj}二\end{CJK}、 $(10$ \begin{CJK}{UTF8}{mj}分\end{CJK} $)$ \begin{CJK}{UTF8}{mj}设\end{CJK} $R[x]$ \begin{CJK}{UTF8}{mj}为次数小于等于\end{CJK} 2 \begin{CJK}{UTF8}{mj}的实系数多项式全体\end{CJK}, \begin{CJK}{UTF8}{mj}令\end{CJK} $f_{1}=1, f_{2}=x-1, f_{3}=(x-2)(x-1)$. \begin{CJK}{UTF8}{mj}试证\end{CJK} $f_{1}, f_{2}, f_{3}$ \begin{CJK}{UTF8}{mj}是\end{CJK} $R[x]$ \begin{CJK}{UTF8}{mj}的一组基\end{CJK}.

\begin{CJK}{UTF8}{mj}三\end{CJK}、 (10 \begin{CJK}{UTF8}{mj}分\end{CJK}) \begin{CJK}{UTF8}{mj}设\end{CJK} $A$ \begin{CJK}{UTF8}{mj}是各阶顺序主子式均不为零的\end{CJK} $n$ \begin{CJK}{UTF8}{mj}阶矩阵\end{CJK}, \begin{CJK}{UTF8}{mj}证明\end{CJK}: \begin{CJK}{UTF8}{mj}存在下三角矩阵\end{CJK} $B$ \begin{CJK}{UTF8}{mj}与上三角矩阵\end{CJK} $C$, \begin{CJK}{UTF8}{mj}使\end{CJK} $A=B C$.

\begin{CJK}{UTF8}{mj}四\end{CJK}、(15 \begin{CJK}{UTF8}{mj}分\end{CJK}) \begin{CJK}{UTF8}{mj}设\end{CJK} $V$ \begin{CJK}{UTF8}{mj}是实数域上所有\end{CJK} $n$ \begin{CJK}{UTF8}{mj}阶对称矩阵所构成的线性空间\end{CJK}, \begin{CJK}{UTF8}{mj}对于任意\end{CJK} $A, B \in V$, \begin{CJK}{UTF8}{mj}定义\end{CJK} $(A, B)=\operatorname{tr}(A B)$, \begin{CJK}{UTF8}{mj}其中\end{CJK} $\operatorname{tr}(A B)$ \begin{CJK}{UTF8}{mj}表示矩阵\end{CJK} $A B$ \begin{CJK}{UTF8}{mj}的主对角线上数的和\end{CJK}.

(1) \begin{CJK}{UTF8}{mj}证明\end{CJK} $V$ \begin{CJK}{UTF8}{mj}构成一欧氏空间\end{CJK};

(2) \begin{CJK}{UTF8}{mj}求子空间\end{CJK} $S=\{A \mid \operatorname{tr}(A)=0\}$ \begin{CJK}{UTF8}{mj}的维数和一组基\end{CJK};

(3) \begin{CJK}{UTF8}{mj}求\end{CJK} $S$ \begin{CJK}{UTF8}{mj}的正交补的一组基和维数\end{CJK}.

\begin{CJK}{UTF8}{mj}五\end{CJK}、(15 \begin{CJK}{UTF8}{mj}分\end{CJK}) \begin{CJK}{UTF8}{mj}设\end{CJK} $A \in P^{n \times n}$, \begin{CJK}{UTF8}{mj}定义由\end{CJK} $P^{n \times n}$ \begin{CJK}{UTF8}{mj}到自身的映射\end{CJK} $T$ \begin{CJK}{UTF8}{mj}为\end{CJK}: \begin{CJK}{UTF8}{mj}对任意\end{CJK} $X \in P^{n \times n}$, \begin{CJK}{UTF8}{mj}有\end{CJK} $T(X)=X A$.

(1) \begin{CJK}{UTF8}{mj}简要证明\end{CJK} $T$ \begin{CJK}{UTF8}{mj}是线性变换\end{CJK};

(2) \begin{CJK}{UTF8}{mj}试问\end{CJK} $T$ \begin{CJK}{UTF8}{mj}是否可逆\end{CJK}? \begin{CJK}{UTF8}{mj}为什么\end{CJK}

(3) \begin{CJK}{UTF8}{mj}已知\end{CJK} $A=\left(\begin{array}{cc}\alpha & \beta \\ \gamma & \delta\end{array}\right), n=2$, \begin{CJK}{UTF8}{mj}求\end{CJK} $T$ \begin{CJK}{UTF8}{mj}在基\end{CJK} $\left\{E_{11}, E_{12}, E_{21}, E_{22}\right\}$ \begin{CJK}{UTF8}{mj}下的矩阵\end{CJK}. \begin{CJK}{UTF8}{mj}其中\end{CJK}: $E_{i j}, i=1,2 ; j=1,2$ \begin{CJK}{UTF8}{mj}为\end{CJK} $(i, j)$ \begin{CJK}{UTF8}{mj}元是\end{CJK} 1 , \begin{CJK}{UTF8}{mj}其余元是\end{CJK} 0 \begin{CJK}{UTF8}{mj}的二阶方阵\end{CJK}.

\begin{CJK}{UTF8}{mj}六\end{CJK}、 (15 \begin{CJK}{UTF8}{mj}分\end{CJK}) \begin{CJK}{UTF8}{mj}求证\end{CJK}: \begin{CJK}{UTF8}{mj}实对称矩阵\end{CJK} $A$ \begin{CJK}{UTF8}{mj}的所有特征值位于区间\end{CJK} $[a, b]$ \begin{CJK}{UTF8}{mj}上的充分必要条件是\end{CJK}: \begin{CJK}{UTF8}{mj}实对称矩阵\end{CJK} $A-\lambda_{0} I$ \begin{CJK}{UTF8}{mj}对任意\end{CJK} $\lambda_{0}<a$ \begin{CJK}{UTF8}{mj}是正定的\end{CJK}, \begin{CJK}{UTF8}{mj}而对任意\end{CJK} $\lambda_{0}>b$ \begin{CJK}{UTF8}{mj}是负定的\end{CJK}.

\begin{CJK}{UTF8}{mj}七\end{CJK}、 (15 \begin{CJK}{UTF8}{mj}分\end{CJK}) \begin{CJK}{UTF8}{mj}设矩阵\end{CJK} $A=\left(\begin{array}{ccc}1 & 5 & 5 \\ 0 & 4 & 3 \\ 0 & a & 2\end{array}\right)$ \begin{CJK}{UTF8}{mj}有一个二重特征值\end{CJK}

(1) \begin{CJK}{UTF8}{mj}试求\end{CJK} $A$ \begin{CJK}{UTF8}{mj}的最小多项式与\end{CJK} Jordan \begin{CJK}{UTF8}{mj}标准形\end{CJK};

(2) \begin{CJK}{UTF8}{mj}确定\end{CJK} $A$ \begin{CJK}{UTF8}{mj}相似于对角矩阵的充分必要条件\end{CJK}.

\begin{CJK}{UTF8}{mj}八\end{CJK}、 ( 15 \begin{CJK}{UTF8}{mj}分\end{CJK}) \begin{CJK}{UTF8}{mj}求证\end{CJK}:\begin{CJK}{UTF8}{mj}设\end{CJK} $A$ \begin{CJK}{UTF8}{mj}是\end{CJK} $n$ \begin{CJK}{UTF8}{mj}阶实对称矩阵\end{CJK}, \begin{CJK}{UTF8}{mj}存在一正数\end{CJK} $c$ \begin{CJK}{UTF8}{mj}使对任一\end{CJK} $n$ \begin{CJK}{UTF8}{mj}维向量\end{CJK} $X$ \begin{CJK}{UTF8}{mj}都有\end{CJK} $\left|X^{\prime} A X\right| \geq c X^{\prime} X$.

\begin{CJK}{UTF8}{mj}九\end{CJK}、 ( 15 \begin{CJK}{UTF8}{mj}分\end{CJK}) \begin{CJK}{UTF8}{mj}设\end{CJK} $n$ \begin{CJK}{UTF8}{mj}级方阵\end{CJK} $A=\left(\alpha_{1}, \alpha_{2}, \cdots, \alpha_{n}\right)$ \begin{CJK}{UTF8}{mj}的前\end{CJK} $n-1$ \begin{CJK}{UTF8}{mj}个列向量线性相关\end{CJK}, \begin{CJK}{UTF8}{mj}后\end{CJK} $n-1$ \begin{CJK}{UTF8}{mj}个向量线性无关\end{CJK}, $\beta=\alpha_{1}+\alpha_{2}+\cdots+\alpha_{n} .$

(1) \begin{CJK}{UTF8}{mj}证明\end{CJK}: \begin{CJK}{UTF8}{mj}方程组\end{CJK} $A X=\beta$ \begin{CJK}{UTF8}{mj}必有无穷多解\end{CJK}.

(2) \begin{CJK}{UTF8}{mj}求方程组\end{CJK} $A X=\beta$ \begin{CJK}{UTF8}{mj}的通解\end{CJK}.

\begin{CJK}{UTF8}{mj}十\end{CJK}、 (15 \begin{CJK}{UTF8}{mj}分\end{CJK}) $n$ \begin{CJK}{UTF8}{mj}维欧氏空间的两组基\end{CJK} $e_{1}, e_{2}, \cdots, e_{n}$ \begin{CJK}{UTF8}{mj}和\end{CJK} $f_{1}, f_{2}, \cdots, f_{n}$ \begin{CJK}{UTF8}{mj}称为对偶基\end{CJK}, \begin{CJK}{UTF8}{mj}如果\end{CJK}
$$
\left(e_{i}, f_{j}\right)= \begin{cases}1, & i=j \\ 0, & i \neq j\end{cases}
$$
(1) \begin{CJK}{UTF8}{mj}证明\end{CJK}: \begin{CJK}{UTF8}{mj}对\end{CJK} $V$ \begin{CJK}{UTF8}{mj}的任一组基\end{CJK} $e_{1}, e_{2}, \cdots, e_{n}$, \begin{CJK}{UTF8}{mj}其对偶基存在并且唯一确定\end{CJK}. \begin{CJK}{UTF8}{mj}与\end{CJK} $V_{1}$ \begin{CJK}{UTF8}{mj}中任一个向量正交\end{CJK}.

\section{5. 上海交通大学 2015 年硕士研究生入学考试试题 (828 高等代数) 
 李扬 
 微信公众号: sxkyliyang}
\begin{CJK}{UTF8}{mj}一\end{CJK}、 ( 15 \begin{CJK}{UTF8}{mj}分\end{CJK}) \begin{CJK}{UTF8}{mj}计算\end{CJK} $n(n>1)$ \begin{CJK}{UTF8}{mj}阶行列式的值\end{CJK}:
$$
\left|\begin{array}{ccccc}
n & S_{1} & S_{2} & \cdots & S_{n-1} \\
S_{1} & S_{2} & S_{3} & \cdots & S_{n} \\
S_{2} & S_{3} & S_{4} & \cdots & S_{n+1} \\
\vdots & \vdots & \vdots & & \vdots \\
S_{n-1} & S_{n} & S_{n+1} & \cdots & S_{2 n-2}
\end{array}\right|,
$$
\begin{CJK}{UTF8}{mj}其中\end{CJK} $S_{k}=x_{1}^{k}+x_{2}^{k}+\cdots+x_{n}^{k}$.

\begin{CJK}{UTF8}{mj}二\end{CJK}、 (15 \begin{CJK}{UTF8}{mj}分\end{CJK}) $\lambda, \mu$ \begin{CJK}{UTF8}{mj}为何值时\end{CJK}, \begin{CJK}{UTF8}{mj}方程组\end{CJK}
$$
\left\{\begin{array}{c}
\lambda x_{1}+x_{2}+x_{3}=4 \\
x_{1}+\mu x_{2}+x_{3}=3 \\
x_{1}+2 \mu x_{2}+x_{3}=4
\end{array}\right.
$$
\begin{CJK}{UTF8}{mj}有唯一解\end{CJK}? \begin{CJK}{UTF8}{mj}无解\end{CJK}? \begin{CJK}{UTF8}{mj}有无穷解\end{CJK}? \begin{CJK}{UTF8}{mj}无穷解时并求其全部解\end{CJK}.

\begin{CJK}{UTF8}{mj}三\end{CJK}、 $\left(15\right.$ \begin{CJK}{UTF8}{mj}分\end{CJK}) \begin{CJK}{UTF8}{mj}假设\end{CJK} $f(x)=\left|\begin{array}{ccc}1 & 1 & 1 \\ 2-x & 2-x^{2} & 2 x^{3}-1 \\ 2 x^{2}-1 & 3 x^{3}-1 & 4 x^{3}-1\end{array}\right|$

(1) \begin{CJK}{UTF8}{mj}证明\end{CJK}: \begin{CJK}{UTF8}{mj}存在实数\end{CJK} $c(0<c<1)$, \begin{CJK}{UTF8}{mj}使得\end{CJK} $f^{\prime}(c)=0$. \begin{CJK}{UTF8}{mj}这里\end{CJK} $f^{\prime}(x)$ \begin{CJK}{UTF8}{mj}为\end{CJK} $f(x)$ \begin{CJK}{UTF8}{mj}的导函数\end{CJK};

(2) \begin{CJK}{UTF8}{mj}在\end{CJK} $Q[x]$ \begin{CJK}{UTF8}{mj}中将\end{CJK} $f(x)$ \begin{CJK}{UTF8}{mj}分解为不可约因式之积\end{CJK}.

\begin{CJK}{UTF8}{mj}四\end{CJK}、 (15 \begin{CJK}{UTF8}{mj}分\end{CJK}) \begin{CJK}{UTF8}{mj}设\end{CJK} $A$ \begin{CJK}{UTF8}{mj}是实矩阵\end{CJK}, $A^{\prime}$ \begin{CJK}{UTF8}{mj}是\end{CJK} $A$ \begin{CJK}{UTF8}{mj}的转置矩阵\end{CJK}. \begin{CJK}{UTF8}{mj}求证\end{CJK}: $A A^{\prime}$ \begin{CJK}{UTF8}{mj}与\end{CJK} $A$ \begin{CJK}{UTF8}{mj}的秩相等\end{CJK}, \begin{CJK}{UTF8}{mj}且当\end{CJK} $A$ \begin{CJK}{UTF8}{mj}满秩时\end{CJK} $A A^{\prime}$ \begin{CJK}{UTF8}{mj}是正定的\end{CJK}.

\begin{CJK}{UTF8}{mj}五\end{CJK}、(15 \begin{CJK}{UTF8}{mj}分\end{CJK}) \begin{CJK}{UTF8}{mj}设\end{CJK} $V$ \begin{CJK}{UTF8}{mj}是数域\end{CJK} $P$ \begin{CJK}{UTF8}{mj}上\end{CJK} $n$ \begin{CJK}{UTF8}{mj}维线性空间\end{CJK}, $\alpha_{1}, \alpha_{2}, \alpha_{3}, \alpha_{4} \in V, W=L\left(\alpha_{1}, \alpha_{2}, \alpha_{3}, \alpha_{4}\right)$, \begin{CJK}{UTF8}{mj}又有\end{CJK} $\beta_{1}, \beta_{2} \in W$ \begin{CJK}{UTF8}{mj}且\end{CJK} $\beta_{1}, \beta_{2}$ \begin{CJK}{UTF8}{mj}线性无关\end{CJK}.

\begin{CJK}{UTF8}{mj}求证\end{CJK}: \begin{CJK}{UTF8}{mj}可用\end{CJK} $\beta_{1}, \beta_{2}$ \begin{CJK}{UTF8}{mj}替换\end{CJK} $\alpha_{1}, \alpha_{2}, \alpha_{3}, \alpha_{4}$ \begin{CJK}{UTF8}{mj}中的两个向量\end{CJK} $\alpha_{i_{1}}, \alpha_{i_{2}}$, \begin{CJK}{UTF8}{mj}使得剩下的两个向量\end{CJK} $\alpha_{i_{3}}, \alpha_{i_{4}}$ \begin{CJK}{UTF8}{mj}与\end{CJK} $\beta_{1}, \beta_{2}$ \begin{CJK}{UTF8}{mj}仍然\end{CJK} \begin{CJK}{UTF8}{mj}生成子空间\end{CJK} $W$, \begin{CJK}{UTF8}{mj}也即\end{CJK} $W=L\left(\beta_{1}, \beta_{2}, \alpha_{i_{3}}, \alpha_{i_{4}}\right)$.

\begin{CJK}{UTF8}{mj}六\end{CJK}、 (15 \begin{CJK}{UTF8}{mj}分\end{CJK}) \begin{CJK}{UTF8}{mj}求证\end{CJK}: \begin{CJK}{UTF8}{mj}任一复矩阵\end{CJK} $A$ \begin{CJK}{UTF8}{mj}均可分解为\end{CJK} $A=B+C$, \begin{CJK}{UTF8}{mj}其中\end{CJK} $C$ \begin{CJK}{UTF8}{mj}为幂零阵\end{CJK}(\begin{CJK}{UTF8}{mj}即有正整数\end{CJK} $k$, \begin{CJK}{UTF8}{mj}使\end{CJK} $\left.C^{k}=0\right), B$ \begin{CJK}{UTF8}{mj}相似\end{CJK} \begin{CJK}{UTF8}{mj}于对角形\end{CJK}, \begin{CJK}{UTF8}{mj}且\end{CJK} $B C=C B$.

\begin{CJK}{UTF8}{mj}七\end{CJK}、 (15 \begin{CJK}{UTF8}{mj}分\end{CJK}) \begin{CJK}{UTF8}{mj}设\end{CJK} $A, B$ \begin{CJK}{UTF8}{mj}是\end{CJK} $n$ \begin{CJK}{UTF8}{mj}阶非零矩阵\end{CJK}, \begin{CJK}{UTF8}{mj}且\end{CJK} $A^{2}=A, B^{2}=B, A B=B A=0$. \begin{CJK}{UTF8}{mj}证明\end{CJK}:

(1) 0,1 \begin{CJK}{UTF8}{mj}必是\end{CJK} $A, B$ \begin{CJK}{UTF8}{mj}的特征值\end{CJK};

(2) \begin{CJK}{UTF8}{mj}若\end{CJK} $x$ \begin{CJK}{UTF8}{mj}是\end{CJK} $A$ \begin{CJK}{UTF8}{mj}的属于特征值\end{CJK} 1 \begin{CJK}{UTF8}{mj}的特征向量\end{CJK}, \begin{CJK}{UTF8}{mj}则\end{CJK} $x$ \begin{CJK}{UTF8}{mj}也是\end{CJK} $B$ \begin{CJK}{UTF8}{mj}的属于特征值\end{CJK} 0 \begin{CJK}{UTF8}{mj}的特征向量\end{CJK}.

\begin{CJK}{UTF8}{mj}八\end{CJK}、 ( 30 \begin{CJK}{UTF8}{mj}分\end{CJK}) \begin{CJK}{UTF8}{mj}设\end{CJK} $F_{n}[x]$ \begin{CJK}{UTF8}{mj}是数域\end{CJK} $F$ \begin{CJK}{UTF8}{mj}上次数\end{CJK} $<n$ \begin{CJK}{UTF8}{mj}的全体多项式构成的线性空间\end{CJK}. $F_{n}[x]$ \begin{CJK}{UTF8}{mj}上线性变换\end{CJK} $D$ \begin{CJK}{UTF8}{mj}将每个多项式\end{CJK} $f(x)$ \begin{CJK}{UTF8}{mj}映到其导数\end{CJK} $f^{\prime}(x)$.

(1) \begin{CJK}{UTF8}{mj}求\end{CJK} $D$ \begin{CJK}{UTF8}{mj}的特征多项式和最小多项式\end{CJK};

(2) \begin{CJK}{UTF8}{mj}找出\end{CJK} $F_{n}[x]$ \begin{CJK}{UTF8}{mj}的一组基\end{CJK}, \begin{CJK}{UTF8}{mj}使\end{CJK} $D$ \begin{CJK}{UTF8}{mj}在这组基下的矩阵是若当标准形\end{CJK};

(3) \begin{CJK}{UTF8}{mj}设\end{CJK} $I$ \begin{CJK}{UTF8}{mj}是\end{CJK} $F_{n}[x]$ \begin{CJK}{UTF8}{mj}上的单位变换\end{CJK}, $A=I+\sum_{k=1}^{n-1} \frac{D^{k}}{k !}$. \begin{CJK}{UTF8}{mj}求证\end{CJK} $A$ \begin{CJK}{UTF8}{mj}是\end{CJK} $F_{n}[x]$ \begin{CJK}{UTF8}{mj}上的可逆变换\end{CJK},\begin{CJK}{UTF8}{mj}并求出\end{CJK} $A$ \begin{CJK}{UTF8}{mj}的逆\end{CJK}.

$N(A) \perp R\left(A^{H}\right)$ \begin{CJK}{UTF8}{mj}且\end{CJK} $C^{n}=N(A) \oplus R\left(A^{H}\right)$ \begin{CJK}{UTF8}{mj}其中\end{CJK}: $C$ \begin{CJK}{UTF8}{mj}为复数域\end{CJK}, $N(A)$ \begin{CJK}{UTF8}{mj}为\end{CJK} $A$ \begin{CJK}{UTF8}{mj}的零空间\end{CJK}(\begin{CJK}{UTF8}{mj}即核\end{CJK}), $R(A)$ \begin{CJK}{UTF8}{mj}为\end{CJK} $A$ \begin{CJK}{UTF8}{mj}的象\end{CJK} 6. \begin{CJK}{UTF8}{mj}上海交通大学\end{CJK} 2018 \begin{CJK}{UTF8}{mj}年硕士研究生入学考试试题\end{CJK} (828 \begin{CJK}{UTF8}{mj}高等代数\end{CJK})

\begin{CJK}{UTF8}{mj}李扬\end{CJK}

\begin{CJK}{UTF8}{mj}微信公众号\end{CJK}: sxkyliyang

\begin{CJK}{UTF8}{mj}一\end{CJK}、 \begin{CJK}{UTF8}{mj}证明\end{CJK}
$$
f(x)=1+x+\frac{x^{2}}{2 !}+\cdots+\frac{x^{n}}{n !}
$$
\begin{CJK}{UTF8}{mj}在有理数域上不可约\end{CJK}.

\begin{CJK}{UTF8}{mj}二\end{CJK}、 $A=\alpha \alpha^{\prime}$, \begin{CJK}{UTF8}{mj}其中\end{CJK} $\alpha$ \begin{CJK}{UTF8}{mj}是一个\end{CJK} $n$ \begin{CJK}{UTF8}{mj}维列向量\end{CJK}, \begin{CJK}{UTF8}{mj}且\end{CJK}
$$
\alpha^{\prime} \alpha=1, \quad B=E+A+A^{2}+\cdots+A^{n},
$$
\begin{CJK}{UTF8}{mj}证明\end{CJK} $B$ \begin{CJK}{UTF8}{mj}可逆\end{CJK}, \begin{CJK}{UTF8}{mj}并求出\end{CJK} $B$ \begin{CJK}{UTF8}{mj}的逆\end{CJK}.

\begin{CJK}{UTF8}{mj}三\end{CJK}, $A$ \begin{CJK}{UTF8}{mj}是\end{CJK} $n$ \begin{CJK}{UTF8}{mj}阶矩阵\end{CJK}, $\operatorname{rank}(A)=n-1$, \begin{CJK}{UTF8}{mj}证明\end{CJK} $A^{*}$ \begin{CJK}{UTF8}{mj}可以表示成\end{CJK} $A$ \begin{CJK}{UTF8}{mj}的多项式\end{CJK}.

\begin{CJK}{UTF8}{mj}四\end{CJK}、 $f(x)$ \begin{CJK}{UTF8}{mj}与\end{CJK} $g(x)$ \begin{CJK}{UTF8}{mj}互素\end{CJK},
$$
f(M) g(M) X=0, \quad f(M) X=0, \quad g(M) X=0
$$
\begin{CJK}{UTF8}{mj}的解空间分别是\end{CJK} $W, W_{1}, W_{2}$, \begin{CJK}{UTF8}{mj}证明\end{CJK}: $W=W_{1} \oplus W_{2}$.

\begin{CJK}{UTF8}{mj}五\end{CJK}、 $A=\left(a_{i j}\right)_{n \times n}$ \begin{CJK}{UTF8}{mj}是一个\end{CJK} $n$ \begin{CJK}{UTF8}{mj}阶可逆矩阵\end{CJK}, $B=\left(a_{i j}\right)_{r \times n}(r \leq n)$, \begin{CJK}{UTF8}{mj}求\end{CJK} $B X=0$ \begin{CJK}{UTF8}{mj}的基础解系\end{CJK}.

\begin{CJK}{UTF8}{mj}六\end{CJK}、(1) \begin{CJK}{UTF8}{mj}证明在复数域上\end{CJK}, \begin{CJK}{UTF8}{mj}有\end{CJK}
$$
A^{2}=-E \Leftrightarrow \operatorname{rank}(A+i E)+\operatorname{rank}(A-i E)=n .
$$
(2) \begin{CJK}{UTF8}{mj}证明复数域上的矩阵\end{CJK} $A$ \begin{CJK}{UTF8}{mj}若满足\end{CJK} $A^{2}=-E$, \begin{CJK}{UTF8}{mj}则\end{CJK} $A$ \begin{CJK}{UTF8}{mj}可对角化\end{CJK}, \begin{CJK}{UTF8}{mj}并求出与它相似的对角矩阵\end{CJK}.

\begin{CJK}{UTF8}{mj}七\end{CJK}、 $A, B$ \begin{CJK}{UTF8}{mj}是\end{CJK} $n$ \begin{CJK}{UTF8}{mj}阶矩阵\end{CJK}, \begin{CJK}{UTF8}{mj}且\end{CJK} $A B=B A$, \begin{CJK}{UTF8}{mj}若\end{CJK} $A$ \begin{CJK}{UTF8}{mj}有\end{CJK} $r$ \begin{CJK}{UTF8}{mj}个互不相同的特征值\end{CJK}, \begin{CJK}{UTF8}{mj}则\end{CJK} $A, B$ \begin{CJK}{UTF8}{mj}至少有\end{CJK} $r$ \begin{CJK}{UTF8}{mj}个公共且线性无关的特\end{CJK} \begin{CJK}{UTF8}{mj}征向量\end{CJK}.

\begin{CJK}{UTF8}{mj}八\end{CJK}、 $A, B$ \begin{CJK}{UTF8}{mj}都是实对称矩阵\end{CJK}, \begin{CJK}{UTF8}{mj}证明\end{CJK} $A$ \begin{CJK}{UTF8}{mj}是正定矩阵的充要条件是\end{CJK}: \begin{CJK}{UTF8}{mj}对于任意一个正定矩阵\end{CJK} $B$, \begin{CJK}{UTF8}{mj}都有\end{CJK} $\operatorname{tr}(A B)>0$.

\begin{CJK}{UTF8}{mj}九\end{CJK}、\begin{CJK}{UTF8}{mj}对于任一实可逆矩阵\end{CJK} $A$, \begin{CJK}{UTF8}{mj}都存在正交矩阵\end{CJK} $Q_{1}, Q_{2}$, \begin{CJK}{UTF8}{mj}使得\end{CJK}
$$
Q_{1} A Q_{2}=\left[\begin{array}{lll}
\lambda_{1} & & \\
& \ddots & \\
& & \lambda_{n}
\end{array}\right]
$$
\begin{CJK}{UTF8}{mj}其中\end{CJK} $\lambda_{n} \geq \lambda_{n-1} \geq \cdots \geq \lambda_{1} \geq 0$, \begin{CJK}{UTF8}{mj}且\end{CJK} $\lambda_{1}^{2}, \cdots, \lambda_{n}^{2}$ \begin{CJK}{UTF8}{mj}都是\end{CJK} $A^{\prime} A$ \begin{CJK}{UTF8}{mj}的特征值\end{CJK}.

\section{7. 上海交通大学 2010 年硕士研究生入学考试试题( 614 数学分析) 
 李扬 
 微信公众号: sxkyliyang}
\begin{CJK}{UTF8}{mj}一\end{CJK}、(\begin{CJK}{UTF8}{mj}每小题\end{CJK} 6 \begin{CJK}{UTF8}{mj}分\end{CJK}, \begin{CJK}{UTF8}{mj}共\end{CJK} 24 \begin{CJK}{UTF8}{mj}分\end{CJK}) \begin{CJK}{UTF8}{mj}判断下列命题的真伪\end{CJK}(\begin{CJK}{UTF8}{mj}正确的命题请简要证明\end{CJK}, \begin{CJK}{UTF8}{mj}错误的命题举出反例\end{CJK})

\begin{enumerate}
  \item \begin{CJK}{UTF8}{mj}数列\end{CJK} $\left\{x_{n}\right\}$ \begin{CJK}{UTF8}{mj}的任何子列都不收敛\end{CJK}, \begin{CJK}{UTF8}{mj}则对任意实数\end{CJK} $a$, \begin{CJK}{UTF8}{mj}存在\end{CJK} $a$ \begin{CJK}{UTF8}{mj}的某邻域使得该邻域内最多只含有\end{CJK} $\left\{x_{n}\right\}$ \begin{CJK}{UTF8}{mj}的有限\end{CJK} I\begin{CJK}{UTF8}{mj}页\end{CJK}.

  \item \begin{CJK}{UTF8}{mj}若\end{CJK} $f(x)$ \begin{CJK}{UTF8}{mj}在开区间\end{CJK} $(a, b)$ \begin{CJK}{UTF8}{mj}内连续且有界\end{CJK}, \begin{CJK}{UTF8}{mj}则\end{CJK} $f(x)$ \begin{CJK}{UTF8}{mj}在\end{CJK} $(a, b)$ \begin{CJK}{UTF8}{mj}内必一致连续\end{CJK}.

  \item \begin{CJK}{UTF8}{mj}若\end{CJK} $f(x)$ \begin{CJK}{UTF8}{mj}在\end{CJK} $(a, b)$ \begin{CJK}{UTF8}{mj}内可导\end{CJK}, \begin{CJK}{UTF8}{mj}则\end{CJK} $\lim _{x \rightarrow a^{+}} f(x)=\infty$, \begin{CJK}{UTF8}{mj}则必定有\end{CJK} $\lim _{x \rightarrow a^{+}} f^{\prime}(x)=\infty$.

  \item \begin{CJK}{UTF8}{mj}若幂级数\end{CJK} $\sum_{n=1}^{\infty} a_{n} x^{n}$ \begin{CJK}{UTF8}{mj}收敛半径为\end{CJK} $r(r>0)$, \begin{CJK}{UTF8}{mj}则\end{CJK}

\end{enumerate}
$$
r=\lim _{n \rightarrow \infty}\left|\frac{a_{n}}{a_{n+1}}\right|
$$
\begin{CJK}{UTF8}{mj}二\end{CJK}、(\begin{CJK}{UTF8}{mj}每小题\end{CJK} 10 \begin{CJK}{UTF8}{mj}分\end{CJK}, \begin{CJK}{UTF8}{mj}共\end{CJK} 50 \begin{CJK}{UTF8}{mj}分\end{CJK}) \begin{CJK}{UTF8}{mj}计算题\end{CJK}

\begin{enumerate}
  \setcounter{enumi}{5}
  \item $\lim _{n \rightarrow \infty} \frac{\sqrt{n+\sqrt{n}}-\sqrt{n}}{\sqrt[n]{3^{n}+5^{n}+7^{n}}}$

  \item \begin{CJK}{UTF8}{mj}设\end{CJK} $a_{i}, i=1,2, \cdots, l$, \begin{CJK}{UTF8}{mj}为\end{CJK} $l$ \begin{CJK}{UTF8}{mj}个实数\end{CJK}, \begin{CJK}{UTF8}{mj}求\end{CJK} $\lim _{x \rightarrow+\infty}\left(\sqrt[l]{\left(x+a_{1}\right)\left(x+a_{2}\right) \cdots\left(x+a_{l}\right)}-x\right)$.

  \item \begin{CJK}{UTF8}{mj}设函数\end{CJK} $z=f(x, y)$ \begin{CJK}{UTF8}{mj}满足方程\end{CJK} $F(u, v)=0$, \begin{CJK}{UTF8}{mj}其中\end{CJK} $a, b$ \begin{CJK}{UTF8}{mj}为常数\end{CJK}, $u=x+a z, v=y+b z$. \begin{CJK}{UTF8}{mj}若\end{CJK} $F$ \begin{CJK}{UTF8}{mj}可微且\end{CJK} $a F_{u}+b F_{v} \neq 0$, \begin{CJK}{UTF8}{mj}求积分\end{CJK}:

\end{enumerate}
$$
\iint_{x^{2}+y^{2} \leq 1} e^{-\left(x^{2}+y^{2}\right)}\left(a \frac{\partial z}{\partial x}+b \frac{\partial z}{\partial y}\right) \mathrm{d} x \mathrm{~d} y
$$

\begin{enumerate}
  \setcounter{enumi}{8}
  \item \begin{CJK}{UTF8}{mj}试求函数\end{CJK}
\end{enumerate}
$$
I(y)=\int_{0}^{+\infty} \frac{\cos x y^{2}}{x^{p}} \mathrm{~d} x
$$
\begin{CJK}{UTF8}{mj}的连续区间\end{CJK}. $(0<p<1)$

\begin{enumerate}
  \setcounter{enumi}{9}
  \item \begin{CJK}{UTF8}{mj}曲面积分\end{CJK}
\end{enumerate}
$$
\iint_{\Sigma} \frac{x \mathrm{~d} y \mathrm{~d} z+y \mathrm{~d} z \mathrm{~d} x+z \mathrm{~d} x \mathrm{~d} y}{\left(a x^{2}+b y^{2}+c z^{2}\right)^{\frac{3}{2}}}, \Sigma: x^{2}+y^{2}+z^{2}=1
$$
\begin{CJK}{UTF8}{mj}方向取外侧\end{CJK}, $(a, b, c>0)$.

\begin{CJK}{UTF8}{mj}三\end{CJK}、 \begin{CJK}{UTF8}{mj}证明题\end{CJK}

\begin{enumerate}
  \setcounter{enumi}{10}
  \item $f(x)$ \begin{CJK}{UTF8}{mj}在\end{CJK} $[a, b]$ \begin{CJK}{UTF8}{mj}上有定义且恒正\end{CJK}. \begin{CJK}{UTF8}{mj}若对每一\end{CJK} $x \in[a, b], \lim _{y \rightarrow x} f(y)=C_{x}>0$, \begin{CJK}{UTF8}{mj}试证\end{CJK}: $f(x)$ \begin{CJK}{UTF8}{mj}在\end{CJK} $[a, b]$ \begin{CJK}{UTF8}{mj}上有正下界\end{CJK}.

  \item $a_{n}>0, \sum_{n=1}^{\infty} \frac{1}{a_{n}}$ \begin{CJK}{UTF8}{mj}发散\end{CJK}, \begin{CJK}{UTF8}{mj}证明\end{CJK}: $\sum_{n=1}^{\infty} \frac{1}{a_{n}+1}$ \begin{CJK}{UTF8}{mj}发散\end{CJK}.

  \item \begin{CJK}{UTF8}{mj}研究函数\end{CJK}

\end{enumerate}
$$
f(x)=\sum_{n=2}^{\infty} \frac{1}{n^{2}+\sin x}
$$
\begin{CJK}{UTF8}{mj}在\end{CJK} $[0,+\infty)$ \begin{CJK}{UTF8}{mj}上的连续性\end{CJK}, \begin{CJK}{UTF8}{mj}一致连续性\end{CJK}, \begin{CJK}{UTF8}{mj}可微性\end{CJK}.

\begin{enumerate}
  \setcounter{enumi}{13}
  \item \begin{CJK}{UTF8}{mj}证明\end{CJK}:
\end{enumerate}
$$
f(x)=x e^{-x^{2}} \int_{0}^{x} e^{x^{2}} \mathrm{~d} x
$$
\begin{CJK}{UTF8}{mj}在\end{CJK} $[0,+\infty)$ \begin{CJK}{UTF8}{mj}上一致连续\end{CJK}.

\begin{enumerate}
  \setcounter{enumi}{14}
  \item \begin{CJK}{UTF8}{mj}已知\end{CJK} $f(x)$ \begin{CJK}{UTF8}{mj}在\end{CJK} $(-1,1)$ \begin{CJK}{UTF8}{mj}内有二阶导数\end{CJK}, \begin{CJK}{UTF8}{mj}且\end{CJK}
\end{enumerate}
$$
f(0)=f^{\prime}(0)=0,\left|f^{\prime \prime}(x)\right| \leq|f(x)|+\left|f^{\prime}(x)\right|,
$$
\begin{CJK}{UTF8}{mj}证明\end{CJK}: $\exists \sigma>0$, \begin{CJK}{UTF8}{mj}使得\end{CJK} $(-\sigma, \sigma)$ \begin{CJK}{UTF8}{mj}内\end{CJK} $f(x) \equiv 0$

\section{8. 上海交通大学 2011 年硕士研究生入学考试试题 ( 614 数学分析) 
 李扬 
 微信公众号: sxkyliyang}
\begin{CJK}{UTF8}{mj}一\end{CJK}、\begin{CJK}{UTF8}{mj}判断\end{CJK}(\begin{CJK}{UTF8}{mj}正确的命题简要证明\end{CJK}, \begin{CJK}{UTF8}{mj}错误的举反例\end{CJK})

\begin{enumerate}
  \item \begin{CJK}{UTF8}{mj}数列\end{CJK} $\left\{x_{n}\right\}$ \begin{CJK}{UTF8}{mj}的任何子列都有收敛子列\end{CJK}, \begin{CJK}{UTF8}{mj}则\end{CJK} $\left\{x_{n}\right\}$ \begin{CJK}{UTF8}{mj}必为有界数列\end{CJK}.

  \item \begin{CJK}{UTF8}{mj}若\end{CJK} $f(x)$ \begin{CJK}{UTF8}{mj}在\end{CJK} $x_{0}$ \begin{CJK}{UTF8}{mj}处可导\end{CJK}, \begin{CJK}{UTF8}{mj}则\end{CJK} $f(x)$ \begin{CJK}{UTF8}{mj}在\end{CJK} $x_{0}$ \begin{CJK}{UTF8}{mj}的某邻域\end{CJK} $U\left(x_{0}\right)$ \begin{CJK}{UTF8}{mj}内必定连续\end{CJK}.

  \item \begin{CJK}{UTF8}{mj}设\end{CJK} $f(x)$ \begin{CJK}{UTF8}{mj}在\end{CJK} $[a, b]$ \begin{CJK}{UTF8}{mj}上可导\end{CJK}, \begin{CJK}{UTF8}{mj}且\end{CJK} $f(a) f(b)<0, f^{\prime}(x)>-f(x)$, \begin{CJK}{UTF8}{mj}则\end{CJK} $f(x)$ \begin{CJK}{UTF8}{mj}在\end{CJK} $[a, b]$ \begin{CJK}{UTF8}{mj}上仅有\end{CJK} 1 \begin{CJK}{UTF8}{mj}个零点\end{CJK}.

  \item \begin{CJK}{UTF8}{mj}级数\end{CJK} $\sum_{n=1}^{\infty} \ln \left(1+\frac{(-1)^{n}}{\sqrt{n}}\right)$ \begin{CJK}{UTF8}{mj}为收敛级数\end{CJK}.

\end{enumerate}
\begin{CJK}{UTF8}{mj}一\end{CJK}、 \begin{CJK}{UTF8}{mj}计算题\end{CJK} $(8 \times 5)$

\begin{enumerate}
  \setcounter{enumi}{5}
  \item $\lim _{x \rightarrow 0}\left(\frac{2 e+e^{\frac{1}{x}}}{1+e^{\frac{2}{x}}}+\frac{e-(1+x)^{\frac{1}{x}}}{\ln \left(1+\left|\frac{x}{2}\right|\right)}\right)$.
\end{enumerate}
6 . \begin{CJK}{UTF8}{mj}设\end{CJK}
$$
f_{n}(x)=x^{n} \ln x, n=0,1,2, \cdots
$$
(1) \begin{CJK}{UTF8}{mj}证明\end{CJK}: $\frac{f_{n}^{(n)}(x)}{n !}=\frac{f_{n-1}^{(n-1)}(x)}{(n-1) !}+\frac{1}{n}, n=1,2, \cdots$.

(2) \begin{CJK}{UTF8}{mj}计算\end{CJK} $\lim _{n \rightarrow \infty} \frac{f_{n}^{(n)}\left(\frac{1}{n}\right)}{n !}$.

\begin{enumerate}
  \setcounter{enumi}{7}
  \item \begin{CJK}{UTF8}{mj}设\end{CJK}
\end{enumerate}
$$
I(y)=\int_{0}^{+\infty} \frac{1-e^{-x y}}{x e^{x}} \mathrm{~d} x, y \in[0,+\infty)
$$
\begin{CJK}{UTF8}{mj}其中\end{CJK} $x=0$ \begin{CJK}{UTF8}{mj}不是奇点\end{CJK}, \begin{CJK}{UTF8}{mj}求\end{CJK} $I(y)$ \begin{CJK}{UTF8}{mj}表达式\end{CJK}.

\begin{enumerate}
  \setcounter{enumi}{8}
  \item \begin{CJK}{UTF8}{mj}计算曲线积分\end{CJK}
\end{enumerate}
$$
I=\int_{L} \frac{(3 y-x) \mathrm{d} x+(y-3 x) \mathrm{d} y}{(x+y)^{3}},
$$
\begin{CJK}{UTF8}{mj}其中\end{CJK} $L$ : \begin{CJK}{UTF8}{mj}由\end{CJK} $A\left(\frac{\pi}{2}, 0\right)$ \begin{CJK}{UTF8}{mj}沿\end{CJK} $y=\frac{\pi}{2} \cos x$ \begin{CJK}{UTF8}{mj}到\end{CJK} $B\left(0, \frac{\pi}{2}\right)$ \begin{CJK}{UTF8}{mj}弧段\end{CJK}.

\begin{enumerate}
  \setcounter{enumi}{9}
  \item $\iiint_{V} \frac{1}{x^{2}+y^{2}} \mathrm{~d} x \mathrm{~d} y \mathrm{~d} z, V:$ \begin{CJK}{UTF8}{mj}由平面\end{CJK} $x=1, x=2, z=0, y=x$ \begin{CJK}{UTF8}{mj}与\end{CJK} $y=z$ \begin{CJK}{UTF8}{mj}围成\end{CJK}.
\end{enumerate}
\begin{CJK}{UTF8}{mj}三\end{CJK}、 \begin{CJK}{UTF8}{mj}证明题\end{CJK}

\begin{enumerate}
  \setcounter{enumi}{10}
  \item \begin{CJK}{UTF8}{mj}设\end{CJK} $f(x)$ \begin{CJK}{UTF8}{mj}在\end{CJK} $x_{0}$ \begin{CJK}{UTF8}{mj}某一邻域\end{CJK} $U\left(x_{0}\right)$ \begin{CJK}{UTF8}{mj}内有界\end{CJK}, \begin{CJK}{UTF8}{mj}令\end{CJK}
\end{enumerate}
$$
m(\sigma)=\inf f\left(U\left(x_{0}, \sigma\right)\right), M(\sigma)=\sup f\left(U\left(x_{0}, \sigma\right)\right) .
$$
\begin{CJK}{UTF8}{mj}证明\end{CJK}:

(1) $\lim _{\sigma \rightarrow 0^{+}}(M(\sigma)-m(\sigma))$ \begin{CJK}{UTF8}{mj}存在\end{CJK}.

(2) $f(x)$ \begin{CJK}{UTF8}{mj}在\end{CJK} $x_{0}$ \begin{CJK}{UTF8}{mj}连续\end{CJK} $\Leftrightarrow \lim _{\sigma \rightarrow 0^{+}}(M(\sigma)-m(\sigma))=0$.

\begin{enumerate}
  \setcounter{enumi}{11}
  \item \begin{CJK}{UTF8}{mj}设\end{CJK} $\int_{a}^{+\infty} f(x) \mathrm{d} x$ \begin{CJK}{UTF8}{mj}收敛\end{CJK}, \begin{CJK}{UTF8}{mj}且\end{CJK} $f(x)$ \begin{CJK}{UTF8}{mj}在\end{CJK} $[a,+\infty)$ \begin{CJK}{UTF8}{mj}上一致连续\end{CJK}, \begin{CJK}{UTF8}{mj}证\end{CJK}: $\lim _{x \rightarrow+\infty} f(x)=0$.

  \item \begin{CJK}{UTF8}{mj}设正项数列\end{CJK} $\left\{a_{n}\right\}$ \begin{CJK}{UTF8}{mj}单调增加\end{CJK}, \begin{CJK}{UTF8}{mj}证\end{CJK}: \begin{CJK}{UTF8}{mj}级数\end{CJK} $\sum_{n=1}^{\infty}\left(1-\frac{a_{n}}{a_{n+1}}\right)$ \begin{CJK}{UTF8}{mj}当\end{CJK} $\left\{a_{n}\right\}$ \begin{CJK}{UTF8}{mj}有界时收敛\end{CJK}, \begin{CJK}{UTF8}{mj}当\end{CJK} $\left\{a_{n}\right\}$ \begin{CJK}{UTF8}{mj}无界时发散\end{CJK}.

  \item \begin{CJK}{UTF8}{mj}研究\end{CJK}

\end{enumerate}
$$
f(x)=\sum_{n=1}^{\infty} \frac{x^{n}}{n^{2} \ln (1+n)}
$$
\begin{CJK}{UTF8}{mj}在区间\end{CJK} $[-1,1)$ \begin{CJK}{UTF8}{mj}上的连续性和可微性\end{CJK}.

\begin{enumerate}
  \setcounter{enumi}{14}
  \item \begin{CJK}{UTF8}{mj}设\end{CJK} $f(x)$ \begin{CJK}{UTF8}{mj}在\end{CJK} $[a, b]$ \begin{CJK}{UTF8}{mj}上可导\end{CJK}, \begin{CJK}{UTF8}{mj}且存在\end{CJK} $c: a<c<b$, \begin{CJK}{UTF8}{mj}使\end{CJK} $f^{\prime}(c)=0$, \begin{CJK}{UTF8}{mj}则存在\end{CJK} $\xi \in(a, b)$, \begin{CJK}{UTF8}{mj}使得\end{CJK}
\end{enumerate}
$$
f^{\prime}(\xi)=(f(\xi)-f(a))(b-a) .
$$

\begin{enumerate}
  \setcounter{enumi}{15}
  \item \begin{CJK}{UTF8}{mj}设\end{CJK} $f(x)$ \begin{CJK}{UTF8}{mj}在\end{CJK} $[a, b)$ \begin{CJK}{UTF8}{mj}上连续\end{CJK}, \begin{CJK}{UTF8}{mj}且无上界\end{CJK}, \begin{CJK}{UTF8}{mj}证\end{CJK}: \begin{CJK}{UTF8}{mj}若对任何区间\end{CJK} $(\alpha, \beta) \subset[a, b), f(x)$ \begin{CJK}{UTF8}{mj}在\end{CJK} $(\alpha, \beta)$ \begin{CJK}{UTF8}{mj}内不能取得最小值\end{CJK}, \begin{CJK}{UTF8}{mj}则\end{CJK} $f(x)$ \begin{CJK}{UTF8}{mj}的值域为区间\end{CJK} $[f(a),+\infty)$.
\end{enumerate}
\section{9. 上海交通大学 2012 年硕士研究生入学考试试题 ( 614 数学分析) 
 李扬 
 微信公众号: sxkyliyang}
\begin{CJK}{UTF8}{mj}一\end{CJK}、(\begin{CJK}{UTF8}{mj}每小题\end{CJK} 6 \begin{CJK}{UTF8}{mj}分\end{CJK}, \begin{CJK}{UTF8}{mj}共\end{CJK} 30 \begin{CJK}{UTF8}{mj}分\end{CJK}) \begin{CJK}{UTF8}{mj}判断下列命题的真伪\end{CJK}(\begin{CJK}{UTF8}{mj}正确的命题请简要证明\end{CJK}, \begin{CJK}{UTF8}{mj}错误的命题举出反例\end{CJK})

\begin{enumerate}
  \item \begin{CJK}{UTF8}{mj}设\end{CJK} $\left\{a_{n}\right\}$ \begin{CJK}{UTF8}{mj}是一个数列\end{CJK}, \begin{CJK}{UTF8}{mj}若在任一子列\end{CJK} $\left\{a_{n_{k}}\right\}$ \begin{CJK}{UTF8}{mj}中均存在收玫子列\end{CJK} $\left\{a_{n_{k_{l}}}\right\}$, \begin{CJK}{UTF8}{mj}则\end{CJK} $\left\{a_{n}\right\}$ \begin{CJK}{UTF8}{mj}必为收敛列\end{CJK}.

  \item $\{\sqrt[n]{n}\}(n=1,2,3, \cdots)$ \begin{CJK}{UTF8}{mj}的最大项为\end{CJK} $\sqrt{5}$.

  \item \begin{CJK}{UTF8}{mj}函数列\end{CJK} $\left\{\frac{n x}{1+n x}\right\}$ \begin{CJK}{UTF8}{mj}在\end{CJK} $(0,1)$ \begin{CJK}{UTF8}{mj}一致收敛\end{CJK}.

  \item \begin{CJK}{UTF8}{mj}若级数\end{CJK} $\sum_{n=1}^{\infty} u_{n}^{2}$ \begin{CJK}{UTF8}{mj}和\end{CJK} $\sum_{n=1}^{\infty} v_{n}^{2}$ \begin{CJK}{UTF8}{mj}都收敛\end{CJK}, \begin{CJK}{UTF8}{mj}则级数\end{CJK} $\sum_{n=1}^{\infty}\left(u_{n}+v_{n}\right)^{2}$ \begin{CJK}{UTF8}{mj}也收敛\end{CJK}.

  \item \begin{CJK}{UTF8}{mj}若函数列\end{CJK} $\left\{f_{n}(x)\right\}$ \begin{CJK}{UTF8}{mj}在区间\end{CJK} $(a, c]$ \begin{CJK}{UTF8}{mj}和\end{CJK} $[c, b)$ \begin{CJK}{UTF8}{mj}上一致收敛\end{CJK}, \begin{CJK}{UTF8}{mj}那么\end{CJK} $\left\{f_{n}(x)\right\}$ \begin{CJK}{UTF8}{mj}在\end{CJK} $(a, b)$ \begin{CJK}{UTF8}{mj}上一致收敛\end{CJK}.

\end{enumerate}
\begin{CJK}{UTF8}{mj}二\end{CJK}、 (10 \begin{CJK}{UTF8}{mj}分\end{CJK}) \begin{CJK}{UTF8}{mj}计算\end{CJK} $\lim _{n \rightarrow \infty} \int_{0}^{1}\left(1-x^{2}\right)^{n} \mathrm{~d} x$.

\begin{CJK}{UTF8}{mj}三\end{CJK}、 ( 10 \begin{CJK}{UTF8}{mj}分\end{CJK}) \begin{CJK}{UTF8}{mj}已知方程\end{CJK}
$$
\frac{a_{1}}{x-\lambda_{1}}+\frac{a_{2}}{x-\lambda_{2}}+\frac{a_{3}}{x-\lambda_{3}}=0,
$$
\begin{CJK}{UTF8}{mj}其中\end{CJK} $a_{1}, a_{2}, a_{3}>0, \lambda_{1}<\lambda_{2}<\lambda_{3}$. \begin{CJK}{UTF8}{mj}证明此方程在区间\end{CJK} $\left(\lambda_{1}, \lambda_{2}\right)$ \begin{CJK}{UTF8}{mj}和\end{CJK} $\left(\lambda_{2}, \lambda_{3}\right)$ \begin{CJK}{UTF8}{mj}中各有一根\end{CJK}.

\begin{CJK}{UTF8}{mj}四\end{CJK}、(10 \begin{CJK}{UTF8}{mj}分\end{CJK}) \begin{CJK}{UTF8}{mj}求幂函数\end{CJK} $\sum_{n=1}^{\infty} \frac{x^{n}}{n(n+1)}$ \begin{CJK}{UTF8}{mj}的和函数\end{CJK}, \begin{CJK}{UTF8}{mj}并指出其定义域\end{CJK}.

\begin{CJK}{UTF8}{mj}五\end{CJK}、(15 \begin{CJK}{UTF8}{mj}分\end{CJK}) \begin{CJK}{UTF8}{mj}设连续可微函数\end{CJK} $z=f(x, y)$ \begin{CJK}{UTF8}{mj}由方程\end{CJK} $F(x z-y, x-y z)=0$ (\begin{CJK}{UTF8}{mj}其中\end{CJK} $F(u, v)=0$ \begin{CJK}{UTF8}{mj}有连续的偏导数\end{CJK}) \begin{CJK}{UTF8}{mj}唯一\end{CJK} \begin{CJK}{UTF8}{mj}确定\end{CJK}, $L$ \begin{CJK}{UTF8}{mj}为正向单位圆周\end{CJK}. \begin{CJK}{UTF8}{mj}试求\end{CJK}:
$$
I=\oint_{L}\left(x z^{2}+2 y z\right) \mathrm{d} y-\left(2 x z+y z^{2}\right) \mathrm{d} x
$$
\begin{CJK}{UTF8}{mj}六\end{CJK}、 $\left(15\right.$ \begin{CJK}{UTF8}{mj}分\end{CJK}) \begin{CJK}{UTF8}{mj}设\end{CJK} $f(x)=\left(x-x_{0}\right)^{n} \varphi(x)$, \begin{CJK}{UTF8}{mj}其中\end{CJK} $n$ \begin{CJK}{UTF8}{mj}为正整数\end{CJK}, $\varphi(x)$ \begin{CJK}{UTF8}{mj}在\end{CJK} $x_{0}$ \begin{CJK}{UTF8}{mj}连续且\end{CJK} $\varphi\left(x_{0}\right) \neq 0$, \begin{CJK}{UTF8}{mj}讨论\end{CJK} $f(x)$ \begin{CJK}{UTF8}{mj}在\end{CJK} $x_{0}$ \begin{CJK}{UTF8}{mj}点处能否\end{CJK} \begin{CJK}{UTF8}{mj}取极值\end{CJK}?

\begin{CJK}{UTF8}{mj}七\end{CJK}、 (10 \begin{CJK}{UTF8}{mj}分\end{CJK}) \begin{CJK}{UTF8}{mj}对于正项数列\end{CJK} $\left\{a_{n}\right\}$, \begin{CJK}{UTF8}{mj}如果有\end{CJK} $\lim _{n \rightarrow \infty} \frac{a_{n+1}}{a_{n}}=a, a>0$, \begin{CJK}{UTF8}{mj}证明必有\end{CJK} $\lim _{n \rightarrow \infty} \sqrt[n]{a_{n}}=a$.

\begin{CJK}{UTF8}{mj}八\end{CJK}、 (15 \begin{CJK}{UTF8}{mj}分\end{CJK}) \begin{CJK}{UTF8}{mj}讨论级数\end{CJK}
$$
\sum_{n=1}^{\infty} \frac{(-1)^{n}}{\left(1+x^{2}\right)^{n}}
$$
\begin{CJK}{UTF8}{mj}在\end{CJK} $(-\infty,+\infty)$ \begin{CJK}{UTF8}{mj}上的收敛性和一致收敛性\end{CJK}.

\begin{CJK}{UTF8}{mj}九\end{CJK}、 ( 15 \begin{CJK}{UTF8}{mj}分\end{CJK}) \begin{CJK}{UTF8}{mj}设函数\end{CJK} $f(x)$ \begin{CJK}{UTF8}{mj}在\end{CJK} $[0,1]$ \begin{CJK}{UTF8}{mj}上连续\end{CJK}, \begin{CJK}{UTF8}{mj}在\end{CJK} $(0,1)$ \begin{CJK}{UTF8}{mj}内二阶可导\end{CJK}, \begin{CJK}{UTF8}{mj}过点\end{CJK} $A(0, f(0))$ \begin{CJK}{UTF8}{mj}与点\end{CJK} $B(1, f(1))$ \begin{CJK}{UTF8}{mj}的直线与曲线\end{CJK} $y=f(x)$ \begin{CJK}{UTF8}{mj}相交于点\end{CJK} $C(c, f(c))$, \begin{CJK}{UTF8}{mj}其中\end{CJK} $0<c<1$, \begin{CJK}{UTF8}{mj}证明\end{CJK}: \begin{CJK}{UTF8}{mj}在\end{CJK} $(0,1)$ \begin{CJK}{UTF8}{mj}内至少存在一点\end{CJK} $\xi$, \begin{CJK}{UTF8}{mj}使\end{CJK} $f^{\prime \prime}(\xi)=0$.

\begin{CJK}{UTF8}{mj}十\end{CJK}、 (20 \begin{CJK}{UTF8}{mj}分\end{CJK}) \begin{CJK}{UTF8}{mj}设函数\end{CJK} $f \in C[0,1]$, \begin{CJK}{UTF8}{mj}记\end{CJK}
$$
I_{n}=\int_{0}^{1} f\left(t^{n}\right) \mathrm{d} t(n \geq 1)
$$
\begin{CJK}{UTF8}{mj}证明\end{CJK}:

(1) $\lim _{n \rightarrow \infty} I_{n}$ \begin{CJK}{UTF8}{mj}存在\end{CJK}, \begin{CJK}{UTF8}{mj}并且等于\end{CJK} $f(0)$;

(2) \begin{CJK}{UTF8}{mj}若\end{CJK} $f^{\prime}(0)$ \begin{CJK}{UTF8}{mj}存在\end{CJK}, \begin{CJK}{UTF8}{mj}则\end{CJK}
$$
I_{n}=f(0)+\frac{1}{n} \int_{0}^{1} \frac{f(t)-f(0)}{t} \mathrm{~d} t+o\left(\frac{1}{n}\right)
$$

\section{0. 上海交通大学 2013 年硕士研究生入学考试试题( 614 数学分析) 
 李扬 
 微信公众号: sxkyliyang}
\begin{CJK}{UTF8}{mj}一\end{CJK}、 \begin{CJK}{UTF8}{mj}填空题\end{CJK} 6 \begin{CJK}{UTF8}{mj}个\end{CJK}

(1) $\lim _{x \rightarrow \infty} \frac{2 x-1}{x^{3} \sin \frac{1}{x^{2}}}$

(2) \begin{CJK}{UTF8}{mj}求\end{CJK} $\sum_{n=0}^{\infty} \frac{n+1}{n !}\left(\frac{1}{2}\right)^{n}$

(3) $y=\frac{x^{2}+1}{x^{2}-1}$, \begin{CJK}{UTF8}{mj}求\end{CJK} $y^{(n)}$.

(4) \begin{CJK}{UTF8}{mj}求\end{CJK} $\int \frac{1}{\sin ^{6} x+\cos ^{6} x} \mathrm{~d} x$

(5) $\frac{d}{\mathrm{~d} x} \int_{0}^{x} t f\left(x^{2}-t^{2}\right) \mathrm{d} t$

(6) $f(x)$ \begin{CJK}{UTF8}{mj}在\end{CJK} $(-\infty,+\infty)$ \begin{CJK}{UTF8}{mj}连续\end{CJK}, $\int_{a}^{a+b} f(x) \mathrm{d} x$ \begin{CJK}{UTF8}{mj}与\end{CJK} $a$ \begin{CJK}{UTF8}{mj}无关\end{CJK}, \begin{CJK}{UTF8}{mj}且\end{CJK} $f(1)=1$, \begin{CJK}{UTF8}{mj}则\end{CJK} $f(x)=$

\begin{CJK}{UTF8}{mj}二\end{CJK}、 \begin{CJK}{UTF8}{mj}计算题\end{CJK}

\begin{enumerate}
  \item \begin{CJK}{UTF8}{mj}已知\end{CJK}
\end{enumerate}
$$
\lim _{x \rightarrow 0} \frac{f(x)}{x}=1, f^{\prime \prime}(x)>0,
$$
\begin{CJK}{UTF8}{mj}求证\end{CJK} $f(x) \geq x$.

\begin{enumerate}
  \setcounter{enumi}{2}
  \item \begin{CJK}{UTF8}{mj}求\end{CJK} $4 x^{2}+y^{2}+z^{2}=1$ \begin{CJK}{UTF8}{mj}所围体积\end{CJK}

  \item \begin{CJK}{UTF8}{mj}证明\end{CJK} Riemann \begin{CJK}{UTF8}{mj}函数可积\end{CJK}

  \item \begin{CJK}{UTF8}{mj}证明\end{CJK}

\end{enumerate}
$$
\sum_{n=1}^{\infty} \frac{\cos n x}{n^{2}+1}
$$
\begin{CJK}{UTF8}{mj}在\end{CJK} $(0,2 \pi)$ \begin{CJK}{UTF8}{mj}内连续可导\end{CJK}

\begin{enumerate}
  \setcounter{enumi}{5}
  \item $f \in C[0, \infty), f(0)=0,|f(x)| \leq M(0 \leq x<\infty)$. \begin{CJK}{UTF8}{mj}若\end{CJK} $\int_{0}^{+\infty} g(x) \mathrm{d} x$ \begin{CJK}{UTF8}{mj}绝对收敛\end{CJK}, \begin{CJK}{UTF8}{mj}则\end{CJK}
\end{enumerate}
$$
\lim _{\alpha \rightarrow 0} \int_{0}^{+\infty}\left(\frac{\alpha}{x}\right) g(x) \mathrm{d} x=0 .
$$

\begin{enumerate}
  \setcounter{enumi}{6}
  \item \begin{CJK}{UTF8}{mj}求\end{CJK}
\end{enumerate}
$$
\iint_{S} x^{3} \mathrm{~d} y \mathrm{~d} z+y^{3} \mathrm{~d} z \mathrm{~d} x+z^{3} \mathrm{~d} x \mathrm{~d} y
$$
\begin{CJK}{UTF8}{mj}其中\end{CJK} $S$ \begin{CJK}{UTF8}{mj}为球面\end{CJK} $x^{2}+y^{2}+z^{2}=R^{2}$, \begin{CJK}{UTF8}{mj}取内侧\end{CJK}.

\begin{enumerate}
  \setcounter{enumi}{7}
  \item \begin{CJK}{UTF8}{mj}已知\end{CJK}
\end{enumerate}
$$
f(x, y)=\left\{\begin{array}{cc}
(x+y)^{2} \sin \frac{1}{x^{2}+y^{2}}, & x^{2}+y^{2} \neq 0 \\
0, & x^{2}+y^{2}=0
\end{array}\right.
$$
\begin{CJK}{UTF8}{mj}求其偏导数及考察其连续性\end{CJK}.

\begin{enumerate}
  \setcounter{enumi}{8}
  \item $a, b>0$, \begin{CJK}{UTF8}{mj}证明\end{CJK}
\end{enumerate}
$$
\int_{0}^{+\infty} f\left(\left(a x-\frac{b}{x}\right)^{2}\right) \mathrm{d} x=\frac{1}{a} \int_{0}^{+\infty} f\left(x^{2}\right) \mathrm{d} x
$$

\section{1. 上海交通大学 2014 年硕士研究生入学考试试题( 614 数学分析) 
 李扬 
 微信公众号: sxkyliyang}
\begin{CJK}{UTF8}{mj}一\end{CJK}、 (\begin{CJK}{UTF8}{mj}每小题\end{CJK} 6 \begin{CJK}{UTF8}{mj}分\end{CJK}, \begin{CJK}{UTF8}{mj}共\end{CJK} 24 \begin{CJK}{UTF8}{mj}分\end{CJK}) \begin{CJK}{UTF8}{mj}判断下列命题的真伪\end{CJK} (\begin{CJK}{UTF8}{mj}正确的命题请简要证明\end{CJK}, \begin{CJK}{UTF8}{mj}错误的命题举出反例\end{CJK})

\begin{enumerate}
  \item \begin{CJK}{UTF8}{mj}设\end{CJK} $f(x)$ \begin{CJK}{UTF8}{mj}在\end{CJK} $[a, b]$ \begin{CJK}{UTF8}{mj}上有界且有原函数\end{CJK}, $g \in C([a, b])$, \begin{CJK}{UTF8}{mj}则\end{CJK} $f(x) \cdot g(x)$ \begin{CJK}{UTF8}{mj}在\end{CJK} $[a, b]$ \begin{CJK}{UTF8}{mj}上有原函数\end{CJK}.

  \item \begin{CJK}{UTF8}{mj}若函数\end{CJK} $f(x)$ \begin{CJK}{UTF8}{mj}和\end{CJK} $g(x)$ \begin{CJK}{UTF8}{mj}在\end{CJK} $[a,+\infty)$ \begin{CJK}{UTF8}{mj}上一致连续\end{CJK}, \begin{CJK}{UTF8}{mj}则\end{CJK} $f(x) g(x)$ \begin{CJK}{UTF8}{mj}在\end{CJK} $[a,+\infty)$ \begin{CJK}{UTF8}{mj}上一致连续\end{CJK}.

  \item \begin{CJK}{UTF8}{mj}设定义在\end{CJK} $R^{2}$ \begin{CJK}{UTF8}{mj}上的二元函数\end{CJK} $f(x, y)$ \begin{CJK}{UTF8}{mj}关于\end{CJK} $x, y$ \begin{CJK}{UTF8}{mj}的偏导数均恒为\end{CJK} 0 , \begin{CJK}{UTF8}{mj}则\end{CJK} $f$ \begin{CJK}{UTF8}{mj}为常值函数\end{CJK}.

  \item $f$ \begin{CJK}{UTF8}{mj}在点\end{CJK} $x_{0} \in(a, b)$ \begin{CJK}{UTF8}{mj}可导的充要条件是\end{CJK} $f$ \begin{CJK}{UTF8}{mj}在点\end{CJK} $x_{0}$ \begin{CJK}{UTF8}{mj}既左可导又右可导\end{CJK}.

\end{enumerate}
\begin{CJK}{UTF8}{mj}二\end{CJK}、 (\begin{CJK}{UTF8}{mj}每小题\end{CJK} 8 \begin{CJK}{UTF8}{mj}分\end{CJK}, \begin{CJK}{UTF8}{mj}共\end{CJK} 40 \begin{CJK}{UTF8}{mj}分\end{CJK}) \begin{CJK}{UTF8}{mj}计算题\end{CJK}

\begin{enumerate}
  \item $\lim _{x \rightarrow+\infty}[\sqrt{(x+a)(x+b)}-x]$.
\end{enumerate}
2 . \begin{CJK}{UTF8}{mj}设\end{CJK} $x-\frac{1}{x}=t$, \begin{CJK}{UTF8}{mj}求积分\end{CJK} $\int_{-1}^{1} \frac{1+x^{2}}{1+x^{4}} \mathrm{~d} x$.

\begin{enumerate}
  \setcounter{enumi}{3}
  \item \begin{CJK}{UTF8}{mj}设\end{CJK} $D$ \begin{CJK}{UTF8}{mj}为平面曲线\end{CJK} $x y=1, x y=3, y^{2}=x, y^{2}=3 x$ \begin{CJK}{UTF8}{mj}所围成的有界闭区域\end{CJK}, \begin{CJK}{UTF8}{mj}求积分\end{CJK}
\end{enumerate}
$$
\iint_{D} \frac{3 x \mathrm{~d} x \mathrm{~d} y}{y^{2}+x y^{3}}
$$

\begin{enumerate}
  \setcounter{enumi}{4}
  \item \begin{CJK}{UTF8}{mj}计算\end{CJK} $\sum_{m=1}^{\infty} \sum_{n=1}^{\infty} \frac{m^{2} n}{3^{m}\left(n 3^{m}+m 3^{n}\right)}$.

  \item \begin{CJK}{UTF8}{mj}计算曲线积分\end{CJK}

\end{enumerate}
$$
I=\int_{L^{*}}\left(x^{2}-y z\right) \mathrm{d} x+\left(y^{2}-x z\right) \mathrm{d} y+\left(z^{2}-x y\right) \mathrm{d} x
$$
\begin{CJK}{UTF8}{mj}其中\end{CJK} $L^{*}$ \begin{CJK}{UTF8}{mj}是从点\end{CJK} $A(1,0,0)$ \begin{CJK}{UTF8}{mj}至\end{CJK} $B(1,0,2)$ \begin{CJK}{UTF8}{mj}的光滑曲线\end{CJK}.

\begin{CJK}{UTF8}{mj}三\end{CJK}、 $($ \begin{CJK}{UTF8}{mj}共\end{CJK} 86 \begin{CJK}{UTF8}{mj}分\end{CJK}) \begin{CJK}{UTF8}{mj}证明题\end{CJK}

\begin{enumerate}
  \item ( 15 \begin{CJK}{UTF8}{mj}分\end{CJK}) \begin{CJK}{UTF8}{mj}设\end{CJK} $f_{n}(x)=\cos x+\cos ^{2} x+\cdots+\cos ^{n} x$.
\end{enumerate}
(1) $x \in\left(0, \frac{\pi}{2}\right]$ \begin{CJK}{UTF8}{mj}时\end{CJK}, \begin{CJK}{UTF8}{mj}求\end{CJK} $\lim _{n \rightarrow \infty} f_{n}(x)$, \begin{CJK}{UTF8}{mj}并讨论\end{CJK} $\left\{f_{n}(x)\right\}$ \begin{CJK}{UTF8}{mj}在\end{CJK} $\left(0, \frac{\pi}{2}\right]$ \begin{CJK}{UTF8}{mj}上的一致收敛性\end{CJK};

(2) \begin{CJK}{UTF8}{mj}证明\end{CJK}: \begin{CJK}{UTF8}{mj}对任一自然数\end{CJK} $n$, \begin{CJK}{UTF8}{mj}方程\end{CJK} $f_{n}(x)=1$ \begin{CJK}{UTF8}{mj}在\end{CJK} $\left[0, \frac{\pi}{3}\right)$ \begin{CJK}{UTF8}{mj}内有且仅有一个根\end{CJK};

(3) \begin{CJK}{UTF8}{mj}若\end{CJK} $\left[0, \frac{\pi}{3}\right)$ \begin{CJK}{UTF8}{mj}是\end{CJK} $f_{n}(x)=1$ \begin{CJK}{UTF8}{mj}的根\end{CJK}, \begin{CJK}{UTF8}{mj}求\end{CJK} $\lim _{n \rightarrow \infty} x_{n}$.

\begin{enumerate}
  \setcounter{enumi}{2}
  \item (15 \begin{CJK}{UTF8}{mj}分\end{CJK}) \begin{CJK}{UTF8}{mj}设函数\end{CJK} $f \in C^{\infty}(R)$, \begin{CJK}{UTF8}{mj}满足\end{CJK} $f(0) f^{\prime}(0) \geq 0$, \begin{CJK}{UTF8}{mj}并且\end{CJK} $f(x) \rightarrow 0(x \rightarrow \infty)$, \begin{CJK}{UTF8}{mj}证明\end{CJK}: \begin{CJK}{UTF8}{mj}存在一个严格递增的无穷\end{CJK} \begin{CJK}{UTF8}{mj}数列\end{CJK} $0 \leq x_{1}<x_{2}<\cdots$, \begin{CJK}{UTF8}{mj}使得\end{CJK} $f^{n}\left(x_{n}\right)=0(n \geq 1)$.

  \item ( 12 \begin{CJK}{UTF8}{mj}分\end{CJK}) \begin{CJK}{UTF8}{mj}证明级数\end{CJK} $\sum_{n=1}^{\infty} \frac{\sqrt[n]{n}-1}{n^{\alpha}}$ \begin{CJK}{UTF8}{mj}在\end{CJK} $\alpha>0$ \begin{CJK}{UTF8}{mj}时收敛\end{CJK}, \begin{CJK}{UTF8}{mj}在\end{CJK} $\alpha \leq 0$ \begin{CJK}{UTF8}{mj}时发散\end{CJK}.

  \item ( 14 \begin{CJK}{UTF8}{mj}分\end{CJK}) \begin{CJK}{UTF8}{mj}设\end{CJK}

\end{enumerate}
$$
f(x)=\left\{\begin{array}{cc}
|x|, & x \neq 0 \\
1, & x=0
\end{array}\right.
$$
\begin{CJK}{UTF8}{mj}证明\end{CJK}: \begin{CJK}{UTF8}{mj}不存在一个函数以\end{CJK} $f(x)$ \begin{CJK}{UTF8}{mj}为其导函数\end{CJK}.

\begin{enumerate}
  \setcounter{enumi}{5}
  \item (15 \begin{CJK}{UTF8}{mj}分\end{CJK}) \begin{CJK}{UTF8}{mj}设函数\end{CJK} $u=u(x, y, z)$ \begin{CJK}{UTF8}{mj}由下面方程组给出\end{CJK}:
\end{enumerate}
$$
\frac{x^{2}}{a^{2}+u}+\frac{y^{2}}{b^{2}+u}+\frac{z^{2}}{c^{2}+u}=1
$$
\begin{CJK}{UTF8}{mj}证明\end{CJK}: $|\operatorname{grad} u|^{2}=2 A \cdot(\operatorname{grad} u)$. \begin{CJK}{UTF8}{mj}其中\end{CJK} $\operatorname{grad}$ \begin{CJK}{UTF8}{mj}为梯度\end{CJK}, $A=(x, y, z)$.

\begin{enumerate}
  \setcounter{enumi}{6}
  \item ( 15 \begin{CJK}{UTF8}{mj}分\end{CJK}) \begin{CJK}{UTF8}{mj}设实函数\end{CJK} $f \in C^{1}[0, \pi]$, \begin{CJK}{UTF8}{mj}令\end{CJK}
\end{enumerate}
$$
C_{n}=\int_{0}^{\pi} f(x) \cos (n x) \mathrm{d} x, n=1,2,3, \cdots
$$
\begin{CJK}{UTF8}{mj}证明\end{CJK}: $\sum_{n=1}^{+\infty}\left|C_{n}\right|<+\infty$.

\section{2. 上海交通大学 2015 年硕士研究生入学考试试题( 614 数学分析) 
 李扬 
 微信公众号: sxkyliyang}
\begin{CJK}{UTF8}{mj}一\end{CJK}、(\begin{CJK}{UTF8}{mj}每小题\end{CJK} 6 \begin{CJK}{UTF8}{mj}分\end{CJK}, \begin{CJK}{UTF8}{mj}共\end{CJK} 24 \begin{CJK}{UTF8}{mj}分\end{CJK}) \begin{CJK}{UTF8}{mj}判断下列命题的真伪\end{CJK}(\begin{CJK}{UTF8}{mj}正确的命题请简要证明\end{CJK}, \begin{CJK}{UTF8}{mj}错误的命题举出反例\end{CJK})

\begin{enumerate}
  \item \begin{CJK}{UTF8}{mj}设\end{CJK} $f(x)$ \begin{CJK}{UTF8}{mj}在\end{CJK} $[a, b]$ \begin{CJK}{UTF8}{mj}上有界\end{CJK}, \begin{CJK}{UTF8}{mj}若对任意的\end{CJK} $\sigma>0, f(x)$ \begin{CJK}{UTF8}{mj}在\end{CJK} $[a+\sigma, b]$ \begin{CJK}{UTF8}{mj}上可积\end{CJK}, \begin{CJK}{UTF8}{mj}则\end{CJK} $f(x)$ \begin{CJK}{UTF8}{mj}在\end{CJK} $[a, b]$ \begin{CJK}{UTF8}{mj}上可积\end{CJK}.

  \item \begin{CJK}{UTF8}{mj}若函数\end{CJK} $f(x)$ \begin{CJK}{UTF8}{mj}在\end{CJK} $R$ \begin{CJK}{UTF8}{mj}上连续且有界\end{CJK}, \begin{CJK}{UTF8}{mj}则\end{CJK} $f(x)$ \begin{CJK}{UTF8}{mj}在\end{CJK} $R$ \begin{CJK}{UTF8}{mj}上必一致连续\end{CJK}.

  \item \begin{CJK}{UTF8}{mj}对\end{CJK} $[a, b]$ \begin{CJK}{UTF8}{mj}上的连续函数\end{CJK} $f$, \begin{CJK}{UTF8}{mj}积分\end{CJK} $\int_{a}^{b} f(x) \mathrm{d} x$ \begin{CJK}{UTF8}{mj}为正当且仅当\end{CJK} $f$ \begin{CJK}{UTF8}{mj}在\end{CJK} $[a, b]$ \begin{CJK}{UTF8}{mj}上恒为正\end{CJK}.

  \item \begin{CJK}{UTF8}{mj}可积函数的复合函数为可积函数\end{CJK}.

\end{enumerate}
\begin{CJK}{UTF8}{mj}二\end{CJK}、 (\begin{CJK}{UTF8}{mj}每小题\end{CJK} 8\begin{CJK}{UTF8}{mj}分\end{CJK}, \begin{CJK}{UTF8}{mj}共\end{CJK} 40 \begin{CJK}{UTF8}{mj}分\end{CJK}) \begin{CJK}{UTF8}{mj}计算题\end{CJK}

\begin{enumerate}
  \item $\lim _{n \rightarrow \infty}\left(1-\frac{1}{2^{2}}\right)\left(1-\frac{1}{3^{2}}\right) \cdots\left(1-\frac{1}{n^{2}}\right)$.

  \item \begin{CJK}{UTF8}{mj}设\end{CJK}

\end{enumerate}
$$
f\left(x^{2}-1\right)=\ln \left(x^{2} /\left(x^{2}-2\right)\right), f[\varphi(x)]=\ln x,
$$
\begin{CJK}{UTF8}{mj}求\end{CJK} $\int \varphi(x) \mathrm{d} x$.

\begin{enumerate}
  \setcounter{enumi}{3}
  \item \begin{CJK}{UTF8}{mj}设\end{CJK} $D$ \begin{CJK}{UTF8}{mj}是\end{CJK} $y=x^{3}, y=1, x=-1$ \begin{CJK}{UTF8}{mj}围成\end{CJK}, $f \in C\left(R^{1}\right)$, \begin{CJK}{UTF8}{mj}求\end{CJK} $I=\iint_{D} x y f\left(x^{2}+y^{2}\right) \mathrm{d} x \mathrm{~d} y$.

  \item \begin{CJK}{UTF8}{mj}设\end{CJK} $u=x^{2}+y^{2}+z$, \begin{CJK}{UTF8}{mj}其中\end{CJK} $z=f(x, y)$ \begin{CJK}{UTF8}{mj}是由方程\end{CJK} $x^{3}+y^{3}+z^{3}=-3 z$ \begin{CJK}{UTF8}{mj}所确定的隐函数\end{CJK}, \begin{CJK}{UTF8}{mj}求\end{CJK} $u_{x}$ \begin{CJK}{UTF8}{mj}和\end{CJK} $u_{x y}$.

  \item \begin{CJK}{UTF8}{mj}求\end{CJK}

\end{enumerate}
$$
\int_{C} \frac{x}{y} \mathrm{~d} x+\frac{1}{y-a} \mathrm{~d} y
$$
\begin{CJK}{UTF8}{mj}其中\end{CJK} $C$ \begin{CJK}{UTF8}{mj}是旋轮线\end{CJK} $x=a(t-\sin t), y=a(1-\cos t)$ \begin{CJK}{UTF8}{mj}对应于\end{CJK} $t=\frac{\pi}{6}$ \begin{CJK}{UTF8}{mj}到\end{CJK} $t=\frac{\pi}{3}$ \begin{CJK}{UTF8}{mj}的一段\end{CJK}.

\begin{CJK}{UTF8}{mj}三\end{CJK}、 (\begin{CJK}{UTF8}{mj}共\end{CJK} 86 \begin{CJK}{UTF8}{mj}分\end{CJK}) \begin{CJK}{UTF8}{mj}证明题\end{CJK}

\begin{enumerate}
  \item (14 \begin{CJK}{UTF8}{mj}分\end{CJK}) \begin{CJK}{UTF8}{mj}设\end{CJK} $f(x)$ \begin{CJK}{UTF8}{mj}在\end{CJK} $[1,+\infty)$ \begin{CJK}{UTF8}{mj}上连续且单调减少\end{CJK}, \begin{CJK}{UTF8}{mj}且\end{CJK} $\int_{1}^{+\infty} f(x) \mathrm{d} x$ \begin{CJK}{UTF8}{mj}收敛\end{CJK}, \begin{CJK}{UTF8}{mj}证明\end{CJK}:
\end{enumerate}
(1) $\lim _{x \rightarrow+\infty} x f(x)=0$;

(2) $g(x)=x f(x)$ \begin{CJK}{UTF8}{mj}在\end{CJK} $[1,+\infty)$ \begin{CJK}{UTF8}{mj}上一致连续\end{CJK}.

\begin{enumerate}
  \setcounter{enumi}{2}
  \item ( 15 \begin{CJK}{UTF8}{mj}分\end{CJK}) \begin{CJK}{UTF8}{mj}设\end{CJK} $\left\{x_{n}\right\}$ \begin{CJK}{UTF8}{mj}为非负递增数列\end{CJK}, \begin{CJK}{UTF8}{mj}对\end{CJK} $\forall m, n \in N$, \begin{CJK}{UTF8}{mj}有\end{CJK} $x_{m n} \geq m x_{n}$, \begin{CJK}{UTF8}{mj}证明\end{CJK}: \begin{CJK}{UTF8}{mj}存在\end{CJK} $\left\{\frac{x_{n}}{n}\right\}$ \begin{CJK}{UTF8}{mj}的子列\end{CJK} $\left\{\frac{x_{n_{k}}}{n_{k}}\right\}$, \begin{CJK}{UTF8}{mj}使得\end{CJK}
\end{enumerate}
$$
\lim _{k \rightarrow \infty} \frac{x_{n}}{n}=\sup _{n}\left\{\frac{x_{n_{k}}}{n_{k}}\right\}
$$

\begin{enumerate}
  \setcounter{enumi}{3}
  \item (12 \begin{CJK}{UTF8}{mj}分\end{CJK}) \begin{CJK}{UTF8}{mj}设函数\end{CJK} $f$ \begin{CJK}{UTF8}{mj}在\end{CJK} $[-\pi, \pi]$ \begin{CJK}{UTF8}{mj}可积\end{CJK}, $a_{n}, b_{n}$ \begin{CJK}{UTF8}{mj}为\end{CJK} $f$ \begin{CJK}{UTF8}{mj}在\end{CJK} $[-\pi, \pi]$ \begin{CJK}{UTF8}{mj}上的\end{CJK} Fourier \begin{CJK}{UTF8}{mj}系数\end{CJK}, \begin{CJK}{UTF8}{mj}证明\end{CJK}: $\sum_{n=1}^{\infty} a_{n} b_{n}$ \begin{CJK}{UTF8}{mj}收敛\end{CJK}.

  \item ( 15 \begin{CJK}{UTF8}{mj}分\end{CJK}) \begin{CJK}{UTF8}{mj}设\end{CJK} $f$ \begin{CJK}{UTF8}{mj}在\end{CJK} $[0,1]$ \begin{CJK}{UTF8}{mj}上可微\end{CJK}, \begin{CJK}{UTF8}{mj}且使得\end{CJK}

\end{enumerate}
$$
\left\{x \in[0,1] \mid f(x)=0=f^{\prime}(x)\right\}=\emptyset .
$$
\begin{CJK}{UTF8}{mj}证明\end{CJK}: $f$ \begin{CJK}{UTF8}{mj}在\end{CJK} $[0,1]$ \begin{CJK}{UTF8}{mj}中只有有限个零点\end{CJK}.

\begin{enumerate}
  \setcounter{enumi}{5}
  \item (15 \begin{CJK}{UTF8}{mj}分\end{CJK}) \begin{CJK}{UTF8}{mj}设二元函数\end{CJK} $f(x, y)$ \begin{CJK}{UTF8}{mj}的两个混合偏导数\end{CJK} $f_{x y}(x, y), f_{y x}(x, y)$ \begin{CJK}{UTF8}{mj}在\end{CJK} $(0,0)$ \begin{CJK}{UTF8}{mj}附近存在\end{CJK}, \begin{CJK}{UTF8}{mj}且\end{CJK} $f_{x y}(x, y)$ \begin{CJK}{UTF8}{mj}在\end{CJK} $(0,0)$ \begin{CJK}{UTF8}{mj}处连续\end{CJK}. \begin{CJK}{UTF8}{mj}证明\end{CJK}:
\end{enumerate}
$$
f_{x y}(0,0)=f_{y x}(0,0) .
$$

\begin{enumerate}
  \setcounter{enumi}{6}
  \item ( 15 \begin{CJK}{UTF8}{mj}分\end{CJK}) \begin{CJK}{UTF8}{mj}设\end{CJK} $f(x)$ \begin{CJK}{UTF8}{mj}在\end{CJK} $R$ \begin{CJK}{UTF8}{mj}上有二阶导函数\end{CJK}, $f(x), f^{\prime}(x), f^{\prime \prime}(x)$ \begin{CJK}{UTF8}{mj}都大于\end{CJK} 0 , \begin{CJK}{UTF8}{mj}假设存在正数\end{CJK} $a, b$ \begin{CJK}{UTF8}{mj}使得\end{CJK}
\end{enumerate}
$$
f^{\prime \prime}(x) \leq a f(x)+b f^{\prime}(x)
$$
\begin{CJK}{UTF8}{mj}对一切\end{CJK} $x \in R$ \begin{CJK}{UTF8}{mj}成立\end{CJK}, \begin{CJK}{UTF8}{mj}证明\end{CJK}:

(1) \begin{CJK}{UTF8}{mj}求证\end{CJK}: $\lim _{x \rightarrow-\infty} f^{\prime}(x)=0$.

(2) \begin{CJK}{UTF8}{mj}求证\end{CJK}: \begin{CJK}{UTF8}{mj}存在常数\end{CJK} $c$ \begin{CJK}{UTF8}{mj}使得\end{CJK} $f^{\prime}(x) \leq c f(x)$.

(3) \begin{CJK}{UTF8}{mj}求使上面不等式成立的最小常数\end{CJK} $c$.

\section{1. 首都师范大学 2009 年研究生入学考试试题数学分析}
\begin{CJK}{UTF8}{mj}李扬\end{CJK}

\begin{CJK}{UTF8}{mj}微信公众号\end{CJK}: sxkyliyang

\begin{CJK}{UTF8}{mj}一\end{CJK}、(\begin{CJK}{UTF8}{mj}本题共\end{CJK} 15 \begin{CJK}{UTF8}{mj}分\end{CJK})

\begin{enumerate}
  \item \begin{CJK}{UTF8}{mj}求极限\end{CJK}
\end{enumerate}
$$
\lim _{n \rightarrow \infty}(\sqrt{n+\sqrt{n}}-\sqrt{n}) .
$$

\begin{enumerate}
  \setcounter{enumi}{2}
  \item \begin{CJK}{UTF8}{mj}证明函数\end{CJK} $y=\sin \frac{1}{x}$ \begin{CJK}{UTF8}{mj}当\end{CJK} $x \rightarrow 0$ \begin{CJK}{UTF8}{mj}时极限不存在\end{CJK}.
\end{enumerate}
\begin{CJK}{UTF8}{mj}二\end{CJK}、 (\begin{CJK}{UTF8}{mj}本题共\end{CJK} 15 \begin{CJK}{UTF8}{mj}分\end{CJK})

\begin{CJK}{UTF8}{mj}设\end{CJK} $f(x)$ \begin{CJK}{UTF8}{mj}在\end{CJK} $[a,+\infty)$ \begin{CJK}{UTF8}{mj}中连续\end{CJK}, \begin{CJK}{UTF8}{mj}并且存在极限\end{CJK} $\lim _{x \rightarrow \infty} f(x)=l, l>f(a)$, \begin{CJK}{UTF8}{mj}对任意的\end{CJK} $\eta \in(f(a), l)$, \begin{CJK}{UTF8}{mj}证明\end{CJK}:

(1) \begin{CJK}{UTF8}{mj}存在\end{CJK} $c_{1} \in(a,+\infty)$, \begin{CJK}{UTF8}{mj}使得\end{CJK} $f\left(c_{1}\right)>\eta$;

(2) \begin{CJK}{UTF8}{mj}存在\end{CJK} $c_{2} \in(a,+\infty)$, \begin{CJK}{UTF8}{mj}使得\end{CJK} $f\left(c_{2}\right)=\eta$.

\begin{CJK}{UTF8}{mj}三\end{CJK}、 (\begin{CJK}{UTF8}{mj}本题共\end{CJK} 15 \begin{CJK}{UTF8}{mj}分\end{CJK})

\begin{enumerate}
  \item \begin{CJK}{UTF8}{mj}设\end{CJK} $f(x), g(x)$ \begin{CJK}{UTF8}{mj}在\end{CJK} $[a, b]$ \begin{CJK}{UTF8}{mj}上连续\end{CJK}, \begin{CJK}{UTF8}{mj}在\end{CJK} $(a, b)$ \begin{CJK}{UTF8}{mj}中可导\end{CJK}, $f^{\prime}(x)>g^{\prime}(x), x \in(a, b)$, \begin{CJK}{UTF8}{mj}又设\end{CJK} $f(a)=g(a)$. \begin{CJK}{UTF8}{mj}证明\end{CJK}:
\end{enumerate}
$$
f(x)>g(x), a<x \leqslant b
$$

\begin{enumerate}
  \setcounter{enumi}{2}
  \item \begin{CJK}{UTF8}{mj}证明\end{CJK}:
\end{enumerate}
$$
\mathrm{e}^{x}>1+x+\frac{x^{2}}{2}
$$
\begin{CJK}{UTF8}{mj}在\end{CJK} $(0,+\infty)$ \begin{CJK}{UTF8}{mj}上成立\end{CJK}.

\begin{CJK}{UTF8}{mj}四\end{CJK}、 (\begin{CJK}{UTF8}{mj}本题共\end{CJK} 15 \begin{CJK}{UTF8}{mj}分\end{CJK})

\begin{enumerate}
  \item \begin{CJK}{UTF8}{mj}运用一元函数的微分计算\end{CJK} $\sqrt[5]{32.16}$ \begin{CJK}{UTF8}{mj}的近似值\end{CJK};

  \item \begin{CJK}{UTF8}{mj}运用泰勒公式计算上面求得的近似值的绝对误差\end{CJK}.

\end{enumerate}
\begin{CJK}{UTF8}{mj}五\end{CJK}、 (\begin{CJK}{UTF8}{mj}本题共\end{CJK} 15 \begin{CJK}{UTF8}{mj}分\end{CJK})

\begin{CJK}{UTF8}{mj}设\end{CJK} $f(x)$ \begin{CJK}{UTF8}{mj}在\end{CJK} $(-\infty,+\infty)$ \begin{CJK}{UTF8}{mj}上连续\end{CJK}, \begin{CJK}{UTF8}{mj}且以\end{CJK} $T$ \begin{CJK}{UTF8}{mj}为周期\end{CJK}, \begin{CJK}{UTF8}{mj}证明\end{CJK}:

(1) \begin{CJK}{UTF8}{mj}函数\end{CJK}
$$
F(x)=\frac{x}{T} \int_{0}^{T} f(t) \mathrm{d} t-\int_{0}^{x} f(t) \mathrm{d} t
$$
\begin{CJK}{UTF8}{mj}也是以\end{CJK} $T$ \begin{CJK}{UTF8}{mj}为周期的函数\end{CJK};

(2) \begin{CJK}{UTF8}{mj}证明\end{CJK}
$$
\lim _{x \rightarrow+\infty} \frac{1}{x} \int_{0}^{x} f(t) \mathrm{d} t=\frac{1}{T} \int_{0}^{T} f(t) \mathrm{d} t
$$
\begin{CJK}{UTF8}{mj}六\end{CJK}、(\begin{CJK}{UTF8}{mj}本题共\end{CJK} 15 \begin{CJK}{UTF8}{mj}分\end{CJK})

\begin{CJK}{UTF8}{mj}设\end{CJK} $y=f(x)$ \begin{CJK}{UTF8}{mj}在\end{CJK} $[a, b]$ \begin{CJK}{UTF8}{mj}上连续\end{CJK}, \begin{CJK}{UTF8}{mj}且不是常函数\end{CJK}. \begin{CJK}{UTF8}{mj}证明\end{CJK}:

(1) \begin{CJK}{UTF8}{mj}设\end{CJK} $M$ \begin{CJK}{UTF8}{mj}和\end{CJK} $m$ \begin{CJK}{UTF8}{mj}分别为\end{CJK} $f$ \begin{CJK}{UTF8}{mj}在\end{CJK} $[a, b]$ \begin{CJK}{UTF8}{mj}中的最大值和最小值\end{CJK}, \begin{CJK}{UTF8}{mj}则\end{CJK}
$$
m(b-a)<\int_{a}^{b} f(x) \mathrm{d} x<M(b-a)
$$
(2) \begin{CJK}{UTF8}{mj}存在\end{CJK} $c \in(a, b)$, \begin{CJK}{UTF8}{mj}使得\end{CJK}
$$
f(c)=\frac{1}{b-a} \int_{a}^{b} f(x) \mathrm{d} x .
$$
\begin{CJK}{UTF8}{mj}七\end{CJK}、 (\begin{CJK}{UTF8}{mj}本题共\end{CJK} 15 \begin{CJK}{UTF8}{mj}分\end{CJK})

\begin{enumerate}
  \item \begin{CJK}{UTF8}{mj}证明函数项级数\end{CJK}
\end{enumerate}
$$
\sum_{n=1}^{\infty}\left[n x \mathrm{e}^{-n x}-(n-1) x \mathrm{e}^{(n-1) x}\right]
$$
\begin{CJK}{UTF8}{mj}在\end{CJK} $[0,1]$ \begin{CJK}{UTF8}{mj}上收敛且和函数连续\end{CJK};

\begin{enumerate}
  \setcounter{enumi}{2}
  \item \begin{CJK}{UTF8}{mj}证明上述函数项级数在\end{CJK} $[0,1]$ \begin{CJK}{UTF8}{mj}上不一致收敛\end{CJK}.
\end{enumerate}
\begin{CJK}{UTF8}{mj}八\end{CJK}、 (\begin{CJK}{UTF8}{mj}本题共\end{CJK} 10 \begin{CJK}{UTF8}{mj}分\end{CJK})

\begin{CJK}{UTF8}{mj}设\end{CJK} $f(u, v)$ \begin{CJK}{UTF8}{mj}有一阶连续偏导数\end{CJK}, \begin{CJK}{UTF8}{mj}又设\end{CJK} $y=y(x)$ \begin{CJK}{UTF8}{mj}是方程\end{CJK} $f\left(x y^{2}, x+y\right)=0$ \begin{CJK}{UTF8}{mj}所确定的隐函数\end{CJK}.

(1) \begin{CJK}{UTF8}{mj}求\end{CJK} $\frac{\mathrm{d} y}{\mathrm{~d} x}$;

(2) \begin{CJK}{UTF8}{mj}当\end{CJK} $f(u, v)=u \mathrm{e}^{v}+v$ \begin{CJK}{UTF8}{mj}时\end{CJK}, \begin{CJK}{UTF8}{mj}求\end{CJK} $\frac{\mathrm{d} y}{\mathrm{~d} x}$.

\begin{CJK}{UTF8}{mj}九\end{CJK}、 (\begin{CJK}{UTF8}{mj}本题共\end{CJK} 10 \begin{CJK}{UTF8}{mj}分\end{CJK})

\begin{CJK}{UTF8}{mj}设\end{CJK} $\varphi(x, y)$ \begin{CJK}{UTF8}{mj}在\end{CJK} $(0,0)$ \begin{CJK}{UTF8}{mj}点连续\end{CJK}, \begin{CJK}{UTF8}{mj}证明函数\end{CJK}
$$
f(x, y)=(a x+b y) \varphi(x, y)
$$
\begin{CJK}{UTF8}{mj}在\end{CJK} $(0,0)$ \begin{CJK}{UTF8}{mj}点可微\end{CJK}, \begin{CJK}{UTF8}{mj}且\end{CJK} $\mathrm{d} f(0,0)=\varphi(0,0)(a \Delta x+b \Delta y)$, \begin{CJK}{UTF8}{mj}这里的\end{CJK} $a$ \begin{CJK}{UTF8}{mj}与\end{CJK} $b$ \begin{CJK}{UTF8}{mj}为常数\end{CJK}.

\begin{CJK}{UTF8}{mj}十\end{CJK}、 (\begin{CJK}{UTF8}{mj}本题共\end{CJK} 10 \begin{CJK}{UTF8}{mj}分\end{CJK})

\begin{CJK}{UTF8}{mj}求第二型曲面积分\end{CJK}
$$
\iint_{\Sigma} x \mathrm{~d} y \mathrm{~d} z+y \mathrm{~d} z \mathrm{~d} x+z \mathrm{~d} x \mathrm{~d} y,
$$
\begin{CJK}{UTF8}{mj}其中\end{CJK} $\Sigma$ \begin{CJK}{UTF8}{mj}是球面\end{CJK} $x^{2}+y^{2}+z^{2}=1$, \begin{CJK}{UTF8}{mj}方向取外侧\end{CJK}.

\begin{CJK}{UTF8}{mj}十一\end{CJK}、(\begin{CJK}{UTF8}{mj}本题共\end{CJK} 15 \begin{CJK}{UTF8}{mj}分\end{CJK})

\begin{CJK}{UTF8}{mj}设可微函数列\end{CJK} $\left\{f_{n}\right\}$ \begin{CJK}{UTF8}{mj}在\end{CJK} $[a, b]$ \begin{CJK}{UTF8}{mj}上收敛\end{CJK}, \begin{CJK}{UTF8}{mj}而且存在常数\end{CJK} $M>0$ \begin{CJK}{UTF8}{mj}使得\end{CJK}
$$
\left|f_{n}^{\prime}(x)\right| \leq M, x \in[a, b], n=1,2, \cdots
$$
\begin{CJK}{UTF8}{mj}证明\end{CJK} $\left\{f_{n}\right\}$ \begin{CJK}{UTF8}{mj}在\end{CJK} $[a, b]$ \begin{CJK}{UTF8}{mj}上一致收敛\end{CJK}.

\section{2. 首都师范大学 2010 年研究生入学考试试题数学分析}
\begin{CJK}{UTF8}{mj}李扬\end{CJK}

\begin{CJK}{UTF8}{mj}微信公众号\end{CJK}: sxkyliyang

\begin{CJK}{UTF8}{mj}一\end{CJK}、 (\begin{CJK}{UTF8}{mj}本题共\end{CJK} 15 \begin{CJK}{UTF8}{mj}分\end{CJK})

\begin{enumerate}
  \item \begin{CJK}{UTF8}{mj}求极限\end{CJK}
\end{enumerate}
$$
\lim _{n \rightarrow \infty}(\sqrt{n+2}-2 \sqrt{n+1}+\sqrt{n})
$$

\begin{enumerate}
  \setcounter{enumi}{2}
  \item \begin{CJK}{UTF8}{mj}求极限\end{CJK}
\end{enumerate}
$$
\lim _{x \rightarrow 0}(\sqrt{1+x}-x)^{\frac{1}{x}}
$$
\begin{CJK}{UTF8}{mj}二\end{CJK}、 (\begin{CJK}{UTF8}{mj}本题共\end{CJK} 15 \begin{CJK}{UTF8}{mj}分\end{CJK})

\begin{enumerate}
  \item \begin{CJK}{UTF8}{mj}求\end{CJK} $\left(\frac{x}{1-x}\right)^{2}$ \begin{CJK}{UTF8}{mj}在\end{CJK} $x=0$ \begin{CJK}{UTF8}{mj}点处的带皮亚诺\end{CJK} (Peano) \begin{CJK}{UTF8}{mj}型余项的二阶泰勒展开式\end{CJK};

  \item \begin{CJK}{UTF8}{mj}设当\end{CJK} $|x|<1$ \begin{CJK}{UTF8}{mj}时\end{CJK},

\end{enumerate}
$$
f(x)=1+\frac{x}{1-x}+\frac{1}{2}\left(\frac{x}{1-x}\right)^{2}+o\left(x^{2}\right)(x \rightarrow 0),
$$
\begin{CJK}{UTF8}{mj}求\end{CJK} $f(x)$ \begin{CJK}{UTF8}{mj}在\end{CJK} $x=0$ \begin{CJK}{UTF8}{mj}点处的带皮亚诺型余项的二阶泰勒展开式\end{CJK}, \begin{CJK}{UTF8}{mj}这里\end{CJK} $o\left(x^{2}\right)(x \rightarrow 0)$ \begin{CJK}{UTF8}{mj}表示当\end{CJK} $x \rightarrow 0$ \begin{CJK}{UTF8}{mj}时比\end{CJK} $x^{2}$ \begin{CJK}{UTF8}{mj}高阶\end{CJK} \begin{CJK}{UTF8}{mj}的无穷小量\end{CJK}.

\begin{CJK}{UTF8}{mj}三\end{CJK}、 (\begin{CJK}{UTF8}{mj}本题共\end{CJK} 10 \begin{CJK}{UTF8}{mj}分\end{CJK})

\begin{CJK}{UTF8}{mj}设\end{CJK} $f(x)$ \begin{CJK}{UTF8}{mj}在区间\end{CJK} $[0,1]$ \begin{CJK}{UTF8}{mj}上二阶可导\end{CJK}, \begin{CJK}{UTF8}{mj}且\end{CJK} $f^{\prime \prime}(x) \leq 0, x \in[0,1]$. \begin{CJK}{UTF8}{mj}证明\end{CJK}:
$$
f(x) \leq f\left(\frac{1}{3}\right)+f^{\prime}\left(\frac{1}{3}\right)\left(x-\frac{1}{3}\right)
$$
(2)
$$
\int_{0}^{1} f\left(x^{2}\right) \mathrm{d} x \leq f\left(\frac{1}{3}\right)
$$
\begin{CJK}{UTF8}{mj}四\end{CJK}、 (\begin{CJK}{UTF8}{mj}本题共\end{CJK} 15 \begin{CJK}{UTF8}{mj}分\end{CJK})

\begin{CJK}{UTF8}{mj}设\end{CJK} $f(x)$ \begin{CJK}{UTF8}{mj}在\end{CJK} $(-\infty,+\infty)$ \begin{CJK}{UTF8}{mj}上连续\end{CJK}, \begin{CJK}{UTF8}{mj}令\end{CJK}
$$
f_{n}(x)=\frac{n}{2} \int_{-\frac{1}{n}}^{\frac{1}{n}} f(x+t) \mathrm{d} t, n=1,2, \cdots
$$
\begin{CJK}{UTF8}{mj}证明在任意区间\end{CJK} $[a, b]$ \begin{CJK}{UTF8}{mj}中\end{CJK}, \begin{CJK}{UTF8}{mj}函数列\end{CJK} $\left\{f_{n}(x)\right\}$ \begin{CJK}{UTF8}{mj}一致收敛于\end{CJK} $f(x)$.

\begin{CJK}{UTF8}{mj}五\end{CJK}、 (\begin{CJK}{UTF8}{mj}本题共\end{CJK} 15 \begin{CJK}{UTF8}{mj}分\end{CJK})

\begin{enumerate}
  \item \begin{CJK}{UTF8}{mj}证明函数\end{CJK}
\end{enumerate}
$$
f(x)=\sqrt{x} \ln x
$$
\begin{CJK}{UTF8}{mj}在\end{CJK} $[0,+\infty)$ \begin{CJK}{UTF8}{mj}中一致连续\end{CJK};

\begin{enumerate}
  \setcounter{enumi}{2}
  \item \begin{CJK}{UTF8}{mj}证明函数\end{CJK}
\end{enumerate}
$$
f(x)=x \ln x
$$
\begin{CJK}{UTF8}{mj}在\end{CJK} $[0,+\infty)$ \begin{CJK}{UTF8}{mj}中不一致连续\end{CJK}.

\begin{CJK}{UTF8}{mj}六\end{CJK}、(\begin{CJK}{UTF8}{mj}本题共\end{CJK} 15 \begin{CJK}{UTF8}{mj}分\end{CJK})

\begin{enumerate}
  \item \begin{CJK}{UTF8}{mj}若级数\end{CJK} $\sum_{n=1}^{+\infty} a_{n}$ \begin{CJK}{UTF8}{mj}绝对收敛\end{CJK}, \begin{CJK}{UTF8}{mj}数列\end{CJK} $\left\{b_{n}\right\}$ \begin{CJK}{UTF8}{mj}有界\end{CJK}, \begin{CJK}{UTF8}{mj}则级数\end{CJK} $\sum_{n=1}^{+\infty} a_{n} b_{n}$ \begin{CJK}{UTF8}{mj}绝对收敛\end{CJK}.

  \item \begin{CJK}{UTF8}{mj}试问\end{CJK}: \begin{CJK}{UTF8}{mj}若级数\end{CJK} $\sum_{n=1}^{+\infty} a_{n}$ \begin{CJK}{UTF8}{mj}条件收敛\end{CJK}, \begin{CJK}{UTF8}{mj}上述结论成立么\end{CJK}? \begin{CJK}{UTF8}{mj}若回答肯定\end{CJK}, \begin{CJK}{UTF8}{mj}证明之\end{CJK}; \begin{CJK}{UTF8}{mj}若回答否定\end{CJK}, \begin{CJK}{UTF8}{mj}举出反例\end{CJK}. \begin{CJK}{UTF8}{mj}七\end{CJK}、 (\begin{CJK}{UTF8}{mj}本题共\end{CJK} 15 \begin{CJK}{UTF8}{mj}分\end{CJK})

  \item \begin{CJK}{UTF8}{mj}证明级数\end{CJK}

\end{enumerate}
$$
\sum_{n=1}^{+\infty}\left(\left(1+\frac{1}{n}\right)^{n+1}-\mathrm{e}\right)
$$
\begin{CJK}{UTF8}{mj}发散\end{CJK};

\begin{enumerate}
  \setcounter{enumi}{2}
  \item \begin{CJK}{UTF8}{mj}证明级数\end{CJK}
\end{enumerate}
$$
\sum_{n=1}^{+\infty} \ln \cos \frac{1}{n}
$$
\begin{CJK}{UTF8}{mj}收敛\end{CJK}.

\begin{CJK}{UTF8}{mj}八\end{CJK}、 (\begin{CJK}{UTF8}{mj}本题共\end{CJK} 15 \begin{CJK}{UTF8}{mj}分\end{CJK})
$$
f(x, y, z)=(x+y+z) \mathrm{e}^{-x^{2}-y^{2}-z^{2}}
$$
\begin{CJK}{UTF8}{mj}证明\end{CJK}:

(1) $\lim _{x^{2}+y^{2}+z^{2} \rightarrow+\infty} f(x, y, z)=0$;

(2) $f(x, y, z)$ \begin{CJK}{UTF8}{mj}在\end{CJK} $\mathbb{R}^{3}$ \begin{CJK}{UTF8}{mj}上有最大值和最小值\end{CJK}, \begin{CJK}{UTF8}{mj}并求之\end{CJK}.

\begin{CJK}{UTF8}{mj}九\end{CJK}、 (\begin{CJK}{UTF8}{mj}本题共\end{CJK} 15 \begin{CJK}{UTF8}{mj}分\end{CJK})

\begin{CJK}{UTF8}{mj}设函数\end{CJK} $x=x(z)$ \begin{CJK}{UTF8}{mj}及\end{CJK} $y=y(z)$ \begin{CJK}{UTF8}{mj}是方程组\end{CJK}
$$
\left\{\begin{array}{l}
x^{2}+y^{2}=\frac{1}{2} z^{2} \\
x+y+z=2
\end{array}\right.
$$
\begin{CJK}{UTF8}{mj}所确定的隐函数\end{CJK}.

(1) \begin{CJK}{UTF8}{mj}求当\end{CJK} $x=1, y=-1, z=2$ \begin{CJK}{UTF8}{mj}时\end{CJK}, $\frac{\mathrm{d} x}{\mathrm{~d} z}$ \begin{CJK}{UTF8}{mj}及\end{CJK} $\frac{\mathrm{d} y}{\mathrm{~d} z}$ \begin{CJK}{UTF8}{mj}的值\end{CJK};

(2) \begin{CJK}{UTF8}{mj}求当\end{CJK} $x=1, y=-1, z=2$ \begin{CJK}{UTF8}{mj}时\end{CJK}, $\frac{\mathrm{d}^{2} x}{\mathrm{~d} z^{2}}$ \begin{CJK}{UTF8}{mj}及\end{CJK} $\frac{\mathrm{d}^{2} y}{\mathrm{~d} z^{2}}$ \begin{CJK}{UTF8}{mj}的值\end{CJK}.

\begin{CJK}{UTF8}{mj}十\end{CJK}、 (\begin{CJK}{UTF8}{mj}本题共\end{CJK} 15 \begin{CJK}{UTF8}{mj}分\end{CJK})

\begin{CJK}{UTF8}{mj}计算二型曲面积分\end{CJK}:
$$
\iint_{S} x^{2} \mathrm{~d} y \mathrm{~d} z+y^{2} \mathrm{~d} z \mathrm{~d} x+z^{2} \mathrm{~d} x \mathrm{~d} y
$$
\begin{CJK}{UTF8}{mj}其中\end{CJK} $S$ \begin{CJK}{UTF8}{mj}是球面\end{CJK} $(x-a)^{2}+(y-b)^{2}+(z-c)^{2}=r^{2}$ \begin{CJK}{UTF8}{mj}并取外侧为正向\end{CJK}.

\begin{CJK}{UTF8}{mj}十一\end{CJK}、 (\begin{CJK}{UTF8}{mj}本题共\end{CJK} 10 \begin{CJK}{UTF8}{mj}分\end{CJK})

\begin{CJK}{UTF8}{mj}设函数\end{CJK} $f(x)$ \begin{CJK}{UTF8}{mj}在\end{CJK} $(-\infty,+\infty)$ \begin{CJK}{UTF8}{mj}上二阶可导\end{CJK}, \begin{CJK}{UTF8}{mj}若\end{CJK} $f(x)$ \begin{CJK}{UTF8}{mj}有界\end{CJK}, \begin{CJK}{UTF8}{mj}证明\end{CJK} $f^{\prime \prime}(x)$ \begin{CJK}{UTF8}{mj}必有零点\end{CJK}.

\section{3. 首都师范大学 2011 年研究生入学考试试题数学分析}
\begin{CJK}{UTF8}{mj}李扬\end{CJK}

\begin{CJK}{UTF8}{mj}微信公众号\end{CJK}: sxkyliyang

\begin{CJK}{UTF8}{mj}一\end{CJK}、(\begin{CJK}{UTF8}{mj}本题共\end{CJK} 15 \begin{CJK}{UTF8}{mj}分\end{CJK})

\begin{enumerate}
  \item \begin{CJK}{UTF8}{mj}求极限\end{CJK}
\end{enumerate}
$$
\lim _{x \rightarrow 0}(\cos x)^{\frac{1}{x^{2}}}
$$

\begin{enumerate}
  \setcounter{enumi}{2}
  \item \begin{CJK}{UTF8}{mj}求极限\end{CJK}
\end{enumerate}
$$
\lim _{n \rightarrow+\infty} \frac{1 !+2 !+\cdots+n !}{n !}
$$
\begin{CJK}{UTF8}{mj}二\end{CJK}、 (\begin{CJK}{UTF8}{mj}本题共\end{CJK} 15 \begin{CJK}{UTF8}{mj}分\end{CJK})

\begin{CJK}{UTF8}{mj}判断下列函数在指定区间上是否一致收玫\end{CJK}, \begin{CJK}{UTF8}{mj}并给出证明\end{CJK}.

(1) \begin{CJK}{UTF8}{mj}在区间\end{CJK} $(0,1)$ \begin{CJK}{UTF8}{mj}上的函数\end{CJK} $\sqrt{x} \sin \frac{1}{x} ;$

(2) \begin{CJK}{UTF8}{mj}在区间\end{CJK} $[0,+\infty)$ \begin{CJK}{UTF8}{mj}上的函数\end{CJK} $x \sin x$;

\begin{CJK}{UTF8}{mj}三\end{CJK}、 (\begin{CJK}{UTF8}{mj}本题共\end{CJK} 10 \begin{CJK}{UTF8}{mj}分\end{CJK})

\begin{enumerate}
  \item \begin{CJK}{UTF8}{mj}设\end{CJK} $f(x)$ \begin{CJK}{UTF8}{mj}是区间\end{CJK} $[0,1]$ \begin{CJK}{UTF8}{mj}上的递减连续函数\end{CJK}, \begin{CJK}{UTF8}{mj}证明\end{CJK}: \begin{CJK}{UTF8}{mj}对于任意的\end{CJK} $a \in(0,1)$, \begin{CJK}{UTF8}{mj}有\end{CJK}
\end{enumerate}
$$
\int_{0}^{1} f(x) \mathrm{d} x \leq \frac{1}{a} \int_{0}^{a} f(x) \mathrm{d} x
$$

\begin{enumerate}
  \setcounter{enumi}{2}
  \item \begin{CJK}{UTF8}{mj}设\end{CJK} $f(x, y)$ \begin{CJK}{UTF8}{mj}为连续函数\end{CJK}, \begin{CJK}{UTF8}{mj}求\end{CJK}
\end{enumerate}
$$
\lim _{r \rightarrow 0} \frac{1}{\pi r^{2}} \iint_{x^{2}+y^{2}<r^{2}} f(x, y) \mathrm{d} x \mathrm{~d} y .
$$
\begin{CJK}{UTF8}{mj}四\end{CJK}、(\begin{CJK}{UTF8}{mj}本题共\end{CJK} 15 \begin{CJK}{UTF8}{mj}分\end{CJK})

\begin{CJK}{UTF8}{mj}设二元函数\end{CJK} $f$ \begin{CJK}{UTF8}{mj}在\end{CJK} $x^{2}+y^{2} \neq 0$ \begin{CJK}{UTF8}{mj}时为\end{CJK}
$$
f(x, y)=\frac{x^{2} y}{\sqrt{x^{4}+y^{2}}},
$$
\begin{CJK}{UTF8}{mj}并且\end{CJK} $f(0,0)=0$, \begin{CJK}{UTF8}{mj}试讨论函数\end{CJK} $f(x, y)$ \begin{CJK}{UTF8}{mj}在原点\end{CJK} $(0,0)$ \begin{CJK}{UTF8}{mj}处的连续性\end{CJK}, \begin{CJK}{UTF8}{mj}偏导数存在性及可微性\end{CJK}.

\begin{CJK}{UTF8}{mj}五\end{CJK}、(\begin{CJK}{UTF8}{mj}本题共\end{CJK} 10 \begin{CJK}{UTF8}{mj}分\end{CJK})

\begin{CJK}{UTF8}{mj}设函数\end{CJK} $f(x)$ \begin{CJK}{UTF8}{mj}在\end{CJK} $[a,+\infty)$ \begin{CJK}{UTF8}{mj}上连续\end{CJK}, \begin{CJK}{UTF8}{mj}在\end{CJK} $(a,+\infty)$ \begin{CJK}{UTF8}{mj}上可微\end{CJK}, \begin{CJK}{UTF8}{mj}且\end{CJK} $\lim _{x \rightarrow+\infty} f(x)=f(a)$, \begin{CJK}{UTF8}{mj}证明\end{CJK}: \begin{CJK}{UTF8}{mj}存在\end{CJK} $\xi>a$, \begin{CJK}{UTF8}{mj}使得\end{CJK}
$$
f^{\prime}(\xi)=0 .
$$
\begin{CJK}{UTF8}{mj}六\end{CJK}、(\begin{CJK}{UTF8}{mj}本题共\end{CJK} 15 \begin{CJK}{UTF8}{mj}分\end{CJK})

(1) \begin{CJK}{UTF8}{mj}证明函数项级数\end{CJK}
$$
\sum_{n=2}^{\infty} \ln \left(1+\frac{x}{n \ln ^{3} n}\right)
$$
\begin{CJK}{UTF8}{mj}在\end{CJK} $[-1,1]$ \begin{CJK}{UTF8}{mj}上是一致收敛\end{CJK}.

(2) \begin{CJK}{UTF8}{mj}证明级数\end{CJK}
$$
\sum_{n=1}^{\infty} \frac{(-1)^{n-1}}{n} e^{-n x}(0<x<+\infty)
$$
\begin{CJK}{UTF8}{mj}的和函数\end{CJK} $S(x)$ \begin{CJK}{UTF8}{mj}在区间\end{CJK} $(0,+\infty)$ \begin{CJK}{UTF8}{mj}上有连续的导数\end{CJK}. \begin{CJK}{UTF8}{mj}七\end{CJK}、 (\begin{CJK}{UTF8}{mj}本题共\end{CJK} 15 \begin{CJK}{UTF8}{mj}分\end{CJK})

\begin{CJK}{UTF8}{mj}求积分\end{CJK}
$$
I=\iint_{S} y z \mathrm{~d} z \mathrm{~d} x+z x \mathrm{~d} x \mathrm{~d} y
$$
\begin{CJK}{UTF8}{mj}其中\end{CJK} $S$ \begin{CJK}{UTF8}{mj}为上半球面\end{CJK} $z=\sqrt{R^{2}-x^{2}-y^{2}}$ \begin{CJK}{UTF8}{mj}的上侧\end{CJK}.

\begin{CJK}{UTF8}{mj}八\end{CJK}、 (\begin{CJK}{UTF8}{mj}本题共\end{CJK} 10 \begin{CJK}{UTF8}{mj}分\end{CJK})

\begin{CJK}{UTF8}{mj}讨论方程\end{CJK}
$$
y-x e^{x}-\varepsilon \sin y=0(0<\varepsilon<1)
$$
\begin{CJK}{UTF8}{mj}在\end{CJK} $(0,0)$ \begin{CJK}{UTF8}{mj}点处隐函数\end{CJK} $y=f(x)$ \begin{CJK}{UTF8}{mj}的存在性\end{CJK}, \begin{CJK}{UTF8}{mj}并求该隐函数在\end{CJK} $x=0$ \begin{CJK}{UTF8}{mj}处的导数\end{CJK}.

\begin{CJK}{UTF8}{mj}九\end{CJK}、 (\begin{CJK}{UTF8}{mj}本题共\end{CJK} 15 \begin{CJK}{UTF8}{mj}分\end{CJK})

\begin{CJK}{UTF8}{mj}设函数\end{CJK} $f(x)$ \begin{CJK}{UTF8}{mj}在\end{CJK} $(0,+\infty)$ \begin{CJK}{UTF8}{mj}上二阶可微\end{CJK}, \begin{CJK}{UTF8}{mj}并且已知\end{CJK}
$$
M_{0}=\sup \{|f(x)|: x \in(0,+\infty)\} \text { 与 } M_{2}=\sup \left\{\left|f^{\prime \prime}(x)\right|: x \in(0,+\infty)\right\} \text { 为有限数. }
$$
\begin{CJK}{UTF8}{mj}证明\end{CJK}: \begin{CJK}{UTF8}{mj}若令\end{CJK} $M_{1}=\sup \left\{\left|f^{\prime}(x)\right|: x \in(0,+\infty)\right\}$. \begin{CJK}{UTF8}{mj}证明\end{CJK}: $M_{1} \leq 2 \sqrt{M_{0} M_{2}}$.

\begin{CJK}{UTF8}{mj}十\end{CJK}、 (\begin{CJK}{UTF8}{mj}本题共\end{CJK} 15 \begin{CJK}{UTF8}{mj}分\end{CJK})

\begin{CJK}{UTF8}{mj}设\end{CJK} $f(x)=\sum_{i, j=1}^{3} a_{i j} x_{i} x_{j}$, \begin{CJK}{UTF8}{mj}其中\end{CJK} $A=\left(a_{i j}\right)$ \begin{CJK}{UTF8}{mj}是\end{CJK} 3 \begin{CJK}{UTF8}{mj}阶正定对称矩阵\end{CJK}, \begin{CJK}{UTF8}{mj}求\end{CJK}
$$
I=\iiint_{f(x) \leq 1} e^{\sqrt{f(x)}} \mathrm{d} x_{1} \mathrm{~d} x_{2} \mathrm{~d} x_{3}
$$
\begin{CJK}{UTF8}{mj}十一\end{CJK}、 (\begin{CJK}{UTF8}{mj}本题共\end{CJK} 10 \begin{CJK}{UTF8}{mj}分\end{CJK})

\begin{CJK}{UTF8}{mj}设\end{CJK}
$$
f(t)=\int_{-\infty}^{+\infty} e^{-x^{2}} \cos (x t) \mathrm{d} x
$$
\begin{CJK}{UTF8}{mj}求\end{CJK} $f(t)$ \begin{CJK}{UTF8}{mj}的表达式\end{CJK}.

\section{4. 首都师范大学 2012 年研究生入学考试试题数学分析}
\begin{CJK}{UTF8}{mj}李扬\end{CJK}

\begin{CJK}{UTF8}{mj}微信公众号\end{CJK}: sxkyliyang

\begin{CJK}{UTF8}{mj}一\end{CJK}、 (\begin{CJK}{UTF8}{mj}本题共\end{CJK} 18 \begin{CJK}{UTF8}{mj}分\end{CJK})

\begin{enumerate}
  \item \begin{CJK}{UTF8}{mj}求极限\end{CJK}
\end{enumerate}
$$
\lim _{x \rightarrow 0}(\cos x)^{\frac{1}{\sin ^{2} x}}
$$

\begin{enumerate}
  \setcounter{enumi}{2}
  \item \begin{CJK}{UTF8}{mj}求极限\end{CJK}
\end{enumerate}
$$
\lim _{x \rightarrow 0} \frac{\sqrt[5]{1+3 x^{4}}-\sqrt{1-2 x}}{\sqrt[3]{1+x}-\sqrt{1+x}}
$$

\begin{enumerate}
  \setcounter{enumi}{3}
  \item \begin{CJK}{UTF8}{mj}求极限\end{CJK}
\end{enumerate}
$$
\lim _{n \rightarrow+\infty} \ln \sqrt[n]{\left(1+\frac{1}{n}\right)\left(1+\frac{2}{n}\right) \cdots\left(1+\frac{n}{n}\right)}
$$
\begin{CJK}{UTF8}{mj}二\end{CJK}、 (\begin{CJK}{UTF8}{mj}本题共\end{CJK} 15 \begin{CJK}{UTF8}{mj}分\end{CJK})

\begin{CJK}{UTF8}{mj}试证明\end{CJK} $f(x)=x^{-2}$ \begin{CJK}{UTF8}{mj}在\end{CJK} $(0,1)$ \begin{CJK}{UTF8}{mj}上不一致连续\end{CJK}, \begin{CJK}{UTF8}{mj}但是对任何\end{CJK} $0<\delta<1, f(x)$ \begin{CJK}{UTF8}{mj}在\end{CJK} $[\delta, 1)$ \begin{CJK}{UTF8}{mj}上一致连续\end{CJK}.

\begin{CJK}{UTF8}{mj}三\end{CJK}、 (\begin{CJK}{UTF8}{mj}每小题\end{CJK} 8 \begin{CJK}{UTF8}{mj}分\end{CJK}, \begin{CJK}{UTF8}{mj}共\end{CJK} 24 \begin{CJK}{UTF8}{mj}分\end{CJK})

\begin{enumerate}
  \item \begin{CJK}{UTF8}{mj}判断函数序列\end{CJK}
\end{enumerate}
$$
f_{n}(x)=\frac{n x}{1+n^{2} x^{2}}
$$
\begin{CJK}{UTF8}{mj}在\end{CJK} $x \in[0,1]$ \begin{CJK}{UTF8}{mj}是否一致收敛\end{CJK};

\begin{enumerate}
  \setcounter{enumi}{2}
  \item \begin{CJK}{UTF8}{mj}判断级数\end{CJK}
\end{enumerate}
$$
\sum_{n=1}^{+\infty}(-1)^{n} \frac{\ln n}{\sqrt{n}}
$$
\begin{CJK}{UTF8}{mj}的条件收敛性\end{CJK};

3 . \begin{CJK}{UTF8}{mj}设\end{CJK} $a>0$, \begin{CJK}{UTF8}{mj}判断使\end{CJK}
$$
\sum_{n=1}^{+\infty} \frac{1}{a^{\ln n}}
$$
\begin{CJK}{UTF8}{mj}收敛的\end{CJK} $a$ \begin{CJK}{UTF8}{mj}的取值范围\end{CJK}.

\begin{CJK}{UTF8}{mj}四\end{CJK}、(\begin{CJK}{UTF8}{mj}本题共\end{CJK} 15 \begin{CJK}{UTF8}{mj}分\end{CJK}, \begin{CJK}{UTF8}{mj}第\end{CJK} (1) \begin{CJK}{UTF8}{mj}题\end{CJK} 8 \begin{CJK}{UTF8}{mj}分\end{CJK}, \begin{CJK}{UTF8}{mj}第\end{CJK} (2) \begin{CJK}{UTF8}{mj}题\end{CJK} 7 \begin{CJK}{UTF8}{mj}分\end{CJK})

\begin{CJK}{UTF8}{mj}设函数\end{CJK} $f(x)$ \begin{CJK}{UTF8}{mj}在区间\end{CJK} $[0,1]$ \begin{CJK}{UTF8}{mj}上二阶可导\end{CJK}, \begin{CJK}{UTF8}{mj}且\end{CJK} $\forall x \in[0,1], f^{\prime \prime}(x) \geq 0$, \begin{CJK}{UTF8}{mj}证明\end{CJK}:

(1)
$$
f(x) \geq f\left(\frac{1}{4}\right)+f^{\prime}\left(\frac{1}{4}\right)\left(x-\frac{1}{4}\right)
$$
(2)
$$
\int_{0}^{1} f\left(x^{3}\right) \mathrm{d} x \geq f\left(\frac{1}{4}\right)
$$
\begin{CJK}{UTF8}{mj}五\end{CJK}、(\begin{CJK}{UTF8}{mj}本题共\end{CJK} 15 \begin{CJK}{UTF8}{mj}分\end{CJK})

ख
$$
x^{2}+y^{2}+z^{2}=a^{2}
$$
\begin{CJK}{UTF8}{mj}包含在柱面\end{CJK} $\frac{x^{2}}{a^{2}}+\frac{y^{2}}{b^{2}}=1(b \leq a)$ \begin{CJK}{UTF8}{mj}内的那部分面积\end{CJK}. \begin{CJK}{UTF8}{mj}六\end{CJK}、(\begin{CJK}{UTF8}{mj}本题共\end{CJK} 12 \begin{CJK}{UTF8}{mj}分\end{CJK})

\begin{CJK}{UTF8}{mj}设\end{CJK} $f$ \begin{CJK}{UTF8}{mj}是区间\end{CJK} $[a, b]$ \begin{CJK}{UTF8}{mj}上的连续函数\end{CJK}, $a \leq x_{1}<x_{2}<\cdots<x_{n} \leq b$, \begin{CJK}{UTF8}{mj}证明\end{CJK}: \begin{CJK}{UTF8}{mj}存在\end{CJK} $\xi \in\left[x_{1}, x_{n}\right]$, \begin{CJK}{UTF8}{mj}使得\end{CJK}
$$
f(\xi)=\frac{f\left(x_{1}\right)+f\left(x_{2}\right)+\cdots+f\left(x_{n}\right)}{n} .
$$
\begin{CJK}{UTF8}{mj}七\end{CJK}、 (\begin{CJK}{UTF8}{mj}車匙\end{CJK} 15 \begin{CJK}{UTF8}{mj}分\end{CJK})

\begin{CJK}{UTF8}{mj}设\end{CJK}
$$
f(x, y)= \begin{cases}x y \sin \frac{1}{x^{2}+y^{2}}, & x^{2}+y^{2} \neq 0 \\ 0, & x^{2}+y^{2}=0\end{cases}
$$
\begin{CJK}{UTF8}{mj}证明\end{CJK}:

(1) $f_{x}(0,0), f_{y}(0,0)$ \begin{CJK}{UTF8}{mj}都存在\end{CJK};

(2) $f_{x}(0,0), f_{y}(0,0)$ \begin{CJK}{UTF8}{mj}在\end{CJK} $(0,0)$ \begin{CJK}{UTF8}{mj}点不连续\end{CJK};

(3) $f(x, y)$ \begin{CJK}{UTF8}{mj}在\end{CJK} $(0,0)$ \begin{CJK}{UTF8}{mj}点可微\end{CJK}.

\begin{CJK}{UTF8}{mj}八\end{CJK}、 (\begin{CJK}{UTF8}{mj}本题共\end{CJK} 12 \begin{CJK}{UTF8}{mj}分\end{CJK})

\begin{CJK}{UTF8}{mj}设\end{CJK} $f_{1}(x)$ \begin{CJK}{UTF8}{mj}在区间\end{CJK} $[a, b]$ \begin{CJK}{UTF8}{mj}上\end{CJK} Riemann \begin{CJK}{UTF8}{mj}可积\end{CJK}, \begin{CJK}{UTF8}{mj}令\end{CJK}
$$
f_{n+1}(x)=\int_{a}^{x} f_{n}(t) \mathrm{d} t, n=1,2, \cdots
$$
\begin{CJK}{UTF8}{mj}证明\end{CJK}: $\left\{f_{n}(x)\right\}$ \begin{CJK}{UTF8}{mj}在\end{CJK} $[a, b]$ \begin{CJK}{UTF8}{mj}上一致收敛于零\end{CJK}.

\begin{CJK}{UTF8}{mj}九\end{CJK}、 (\begin{CJK}{UTF8}{mj}本题共\end{CJK} 12 \begin{CJK}{UTF8}{mj}分\end{CJK})

\begin{CJK}{UTF8}{mj}若函数\end{CJK} $f$ \begin{CJK}{UTF8}{mj}在区间\end{CJK} $[a, b]$ \begin{CJK}{UTF8}{mj}上连续\end{CJK}, \begin{CJK}{UTF8}{mj}且对每一个\end{CJK} $x \in[a, b]$ \begin{CJK}{UTF8}{mj}都存在\end{CJK} $y \in[a, b]$ \begin{CJK}{UTF8}{mj}使得\end{CJK} $|f(y)| \leq \frac{1}{2}|f(x)|$. \begin{CJK}{UTF8}{mj}证明\end{CJK}: \begin{CJK}{UTF8}{mj}函数\end{CJK} $f$ \begin{CJK}{UTF8}{mj}在\end{CJK} $[a, b]$ \begin{CJK}{UTF8}{mj}中有零点\end{CJK}.

\begin{CJK}{UTF8}{mj}十\end{CJK}、 (\begin{CJK}{UTF8}{mj}本题共\end{CJK} 12 \begin{CJK}{UTF8}{mj}分\end{CJK})

\begin{CJK}{UTF8}{mj}设\end{CJK} $f(x)=\sum_{n=0}^{\infty} a_{n} x^{n}$ \begin{CJK}{UTF8}{mj}和\end{CJK} $g(x)=\sum_{n=0}^{\infty} b_{n} x^{n}$ \begin{CJK}{UTF8}{mj}在\end{CJK} $0 \leq x<1$ \begin{CJK}{UTF8}{mj}时收敛\end{CJK}, \begin{CJK}{UTF8}{mj}且\end{CJK} $b_{n}>0, \lim _{x \rightarrow 1^{-}} g(x)=\infty$. \begin{CJK}{UTF8}{mj}证明\end{CJK}:

(1) \begin{CJK}{UTF8}{mj}若\end{CJK} $\lim _{n \rightarrow \infty} \frac{a_{n}}{b_{n}}=0$, \begin{CJK}{UTF8}{mj}则\end{CJK}
$$
\lim _{x \rightarrow 1^{-}} \frac{f(x)}{g(x)}=0
$$
(2) \begin{CJK}{UTF8}{mj}若\end{CJK} $\lim _{n \rightarrow \infty} \frac{a_{n}}{b_{n}}=1$, \begin{CJK}{UTF8}{mj}则\end{CJK}
$$
\lim _{x \rightarrow 1^{-}} \frac{f(x)}{g(x)}=1
$$

\section{5. 首都师范大学 2013 年研究生入学考试试题数学分析}
\begin{CJK}{UTF8}{mj}李扬\end{CJK}

\begin{CJK}{UTF8}{mj}微信公众号\end{CJK}: sxkyliyang

\begin{CJK}{UTF8}{mj}一\end{CJK}、(\begin{CJK}{UTF8}{mj}每小题\end{CJK} 5 \begin{CJK}{UTF8}{mj}分\end{CJK}, \begin{CJK}{UTF8}{mj}本题共\end{CJK} 15 \begin{CJK}{UTF8}{mj}分\end{CJK})

\begin{enumerate}
  \item \begin{CJK}{UTF8}{mj}求极限\end{CJK}
\end{enumerate}
$$
\lim _{x \rightarrow 0^{+}}\left(\frac{\sin x}{x}\right)^{\frac{1}{x^{2}}}
$$

\begin{enumerate}
  \setcounter{enumi}{2}
  \item \begin{CJK}{UTF8}{mj}求极限\end{CJK}
\end{enumerate}
$$
\lim _{n \rightarrow+\infty} \frac{3^{n}}{n !}
$$

\begin{enumerate}
  \setcounter{enumi}{3}
  \item \begin{CJK}{UTF8}{mj}求极限\end{CJK}
\end{enumerate}
$$
\lim _{x \rightarrow 0} \frac{\tan x-\sin x}{\sin x^{3}}
$$
\begin{CJK}{UTF8}{mj}二\end{CJK}、 (\begin{CJK}{UTF8}{mj}本题共\end{CJK} 15 \begin{CJK}{UTF8}{mj}分\end{CJK})

\begin{CJK}{UTF8}{mj}求函数\end{CJK}
$$
f(x)= \begin{cases}x^{2} \sin \frac{1}{x}, & x \neq 0 \\ 0, & x=0\end{cases}
$$
\begin{CJK}{UTF8}{mj}的导函数\end{CJK}, \begin{CJK}{UTF8}{mj}并讨论导函数的连续性\end{CJK}.

\begin{CJK}{UTF8}{mj}三\end{CJK}、 (\begin{CJK}{UTF8}{mj}本题\end{CJK} 15 \begin{CJK}{UTF8}{mj}分\end{CJK})

\begin{CJK}{UTF8}{mj}设函数\end{CJK} $f: \mathbb{R} \rightarrow \mathbb{R}$ \begin{CJK}{UTF8}{mj}连续\end{CJK}, \begin{CJK}{UTF8}{mj}极限\end{CJK} $\lim _{|x| \rightarrow+\infty} f(x)$ \begin{CJK}{UTF8}{mj}存在并有限\end{CJK}. \begin{CJK}{UTF8}{mj}证明\end{CJK}: $f$ \begin{CJK}{UTF8}{mj}在\end{CJK} $\mathbb{R}$ \begin{CJK}{UTF8}{mj}上一致连续\end{CJK}.

\begin{CJK}{UTF8}{mj}四\end{CJK}、 (\begin{CJK}{UTF8}{mj}本题共\end{CJK} 15 \begin{CJK}{UTF8}{mj}分\end{CJK})

\begin{CJK}{UTF8}{mj}判断函数\end{CJK}
$$
f(x)= \begin{cases}\frac{x y}{\sqrt{x^{2}+y^{2}}}, & x^{2}+y^{2} \neq 0 ; \\ 0, & x^{2}+y^{2}=0 .\end{cases}
$$
\begin{CJK}{UTF8}{mj}在原点\end{CJK} $(0,0)$ \begin{CJK}{UTF8}{mj}处是否连续\end{CJK}, \begin{CJK}{UTF8}{mj}偏导数是否存在\end{CJK}, \begin{CJK}{UTF8}{mj}是否可导\end{CJK}.

\begin{CJK}{UTF8}{mj}五\end{CJK}、(\begin{CJK}{UTF8}{mj}本题共\end{CJK} 15 \begin{CJK}{UTF8}{mj}分\end{CJK}, \begin{CJK}{UTF8}{mj}第\end{CJK} (1) \begin{CJK}{UTF8}{mj}题\end{CJK} 7 \begin{CJK}{UTF8}{mj}分\end{CJK}, \begin{CJK}{UTF8}{mj}第\end{CJK} (2) \begin{CJK}{UTF8}{mj}题\end{CJK} 8 \begin{CJK}{UTF8}{mj}分\end{CJK})

\begin{CJK}{UTF8}{mj}判断下列级数的敛散性\end{CJK}:

(1)
$$
\sum_{n=1}^{\infty}(\sqrt[n]{n}-1)^{n}
$$
(2)
$$
\sum_{n=1}^{\infty} n\left(1-\cos \frac{1}{n}\right)
$$
\begin{CJK}{UTF8}{mj}六\end{CJK}、 (\begin{CJK}{UTF8}{mj}本题共\end{CJK} 15 \begin{CJK}{UTF8}{mj}分\end{CJK})

\begin{CJK}{UTF8}{mj}求三重积分\end{CJK}
$$
\iiint_{\Omega} y(x+z) \mathrm{d} x \mathrm{~d} y \mathrm{~d} z
$$
\begin{CJK}{UTF8}{mj}其中\end{CJK} $\Omega$ \begin{CJK}{UTF8}{mj}为由平面\end{CJK} $y=0, z=0, x+z=\frac{\pi}{2}$ \begin{CJK}{UTF8}{mj}及曲面\end{CJK} $y=\sqrt{x}$ \begin{CJK}{UTF8}{mj}所围的区域\end{CJK}. \begin{CJK}{UTF8}{mj}七\end{CJK}、 (\begin{CJK}{UTF8}{mj}本题共\end{CJK} 15 \begin{CJK}{UTF8}{mj}分\end{CJK})

\begin{CJK}{UTF8}{mj}设\end{CJK} $S$ \begin{CJK}{UTF8}{mj}为非空数集\end{CJK}, \begin{CJK}{UTF8}{mj}有上界\end{CJK}, \begin{CJK}{UTF8}{mj}并且上确界\end{CJK} $\sup S=a \notin S$. \begin{CJK}{UTF8}{mj}证明\end{CJK}: \begin{CJK}{UTF8}{mj}存在严格单调递增数列\end{CJK} $\left\{x_{n}\right\}_{n=1}^{\infty} \subset S$, \begin{CJK}{UTF8}{mj}使得\end{CJK}
$$
\lim _{n \rightarrow \infty} x_{n}=a .
$$
\begin{CJK}{UTF8}{mj}八\end{CJK}、 (\begin{CJK}{UTF8}{mj}本题共\end{CJK} 15 \begin{CJK}{UTF8}{mj}分\end{CJK})

\begin{CJK}{UTF8}{mj}证明级数\end{CJK}
$$
\sum_{n=1}^{\infty} \frac{\sin n x}{n}
$$
\begin{CJK}{UTF8}{mj}在\end{CJK} $(0,2 \pi)$ \begin{CJK}{UTF8}{mj}上内闭一致收敛\end{CJK}.

\begin{CJK}{UTF8}{mj}九\end{CJK}、 (\begin{CJK}{UTF8}{mj}本题共\end{CJK} 15 \begin{CJK}{UTF8}{mj}分\end{CJK})

\begin{CJK}{UTF8}{mj}设\end{CJK} $f$ \begin{CJK}{UTF8}{mj}在\end{CJK} $[0,+\infty)$ \begin{CJK}{UTF8}{mj}上连续可微\end{CJK}, \begin{CJK}{UTF8}{mj}且严格单调递增\end{CJK}, $f(0)=0, a \geq 0, b \geq 0, g(y)$ \begin{CJK}{UTF8}{mj}是\end{CJK} $f(x)$ \begin{CJK}{UTF8}{mj}的反函数\end{CJK}. \begin{CJK}{UTF8}{mj}证明\end{CJK}:

(1) \begin{CJK}{UTF8}{mj}不等式\end{CJK}
$$
a b \leq \int_{0}^{a} f(x) \mathrm{d} x+\int_{0}^{b} g(y) \mathrm{d} y
$$
\begin{CJK}{UTF8}{mj}当且仅当\end{CJK} $b=f(a)$ \begin{CJK}{UTF8}{mj}等号成立\end{CJK};

(2) \begin{CJK}{UTF8}{mj}当\end{CJK} $\alpha \geq 1, \beta \geq 1$ \begin{CJK}{UTF8}{mj}时\end{CJK}, \begin{CJK}{UTF8}{mj}有\end{CJK}
$$
\alpha \beta \leq \mathrm{e}^{\alpha-1}+\beta \ln \beta
$$
\begin{CJK}{UTF8}{mj}成立\end{CJK}.

\begin{CJK}{UTF8}{mj}十\end{CJK}、 (\begin{CJK}{UTF8}{mj}車题共\end{CJK} 15 \begin{CJK}{UTF8}{mj}分\end{CJK})

\begin{CJK}{UTF8}{mj}设级数\end{CJK} $\sum_{n=0}^{\infty} a_{n}$ \begin{CJK}{UTF8}{mj}的部分和数列为\end{CJK} $S_{n}=a_{0}+a_{1}+\cdots+a_{n}$,
$$
\sigma_{n}=\frac{S_{0}+S_{1}+\cdots+S_{n-1}}{n}, n=0,1, \cdots
$$
\begin{CJK}{UTF8}{mj}且有极限\end{CJK} $\lim _{n \rightarrow \infty} \sigma_{n}=S$. \begin{CJK}{UTF8}{mj}证明\end{CJK}:

(1) $\lim _{n \rightarrow \infty} \frac{a_{n}}{n}=0$;

(2) $\sum_{n=0}^{\infty} a_{n} x^{n}$ \begin{CJK}{UTF8}{mj}在\end{CJK} $(-1,1)$ \begin{CJK}{UTF8}{mj}上内闭一致收敛\end{CJK};

(3) \begin{CJK}{UTF8}{mj}在\end{CJK} $(-1,1)$ \begin{CJK}{UTF8}{mj}上成立\end{CJK}
$$
\sum_{n=0}^{\infty} a_{n} x^{n}=(1-x)^{2} \sum_{n=0}^{\infty}(n+1) \sigma_{n+1} x^{n}
$$

\section{6. 首都师范大学 2014 年研究生入学考试试题数学分析}
\begin{CJK}{UTF8}{mj}李扬\end{CJK}

\begin{CJK}{UTF8}{mj}微信公众号\end{CJK}: sxkyliyang

\begin{CJK}{UTF8}{mj}一\end{CJK}、(\begin{CJK}{UTF8}{mj}每小题\end{CJK} 10 \begin{CJK}{UTF8}{mj}分\end{CJK}, \begin{CJK}{UTF8}{mj}本题共\end{CJK} 40 \begin{CJK}{UTF8}{mj}分\end{CJK})

\begin{enumerate}
  \item \begin{CJK}{UTF8}{mj}求极限\end{CJK}
\end{enumerate}
$$
\lim _{n \rightarrow+\infty}\left(\frac{3 n+5}{3 n+1}\right)^{n}
$$

\begin{enumerate}
  \setcounter{enumi}{2}
  \item \begin{CJK}{UTF8}{mj}求极限\end{CJK}
\end{enumerate}
$$
\lim _{x \rightarrow 1}(2-x)^{\frac{\tan \pi x}{2}}
$$

\begin{enumerate}
  \setcounter{enumi}{3}
  \item \begin{CJK}{UTF8}{mj}求积分\end{CJK}
\end{enumerate}
$$
\iiint_{V}\left(x^{2}+y^{2}+2 z\right) \mathrm{d} x \mathrm{~d} y \mathrm{~d} z
$$
\begin{CJK}{UTF8}{mj}其中\end{CJK} $V$ \begin{CJK}{UTF8}{mj}表示三维空间中的区域\end{CJK} $\left\{(x, y, z): \sqrt{x^{2}+y^{2}} \leq z \leq 1\right\}$;

\begin{enumerate}
  \setcounter{enumi}{4}
  \item \begin{CJK}{UTF8}{mj}求积分\end{CJK}
\end{enumerate}
$$
\iint_{S}\left(x^{2}+y^{2}\right) \mathrm{d} S,(\text { 第一型曲面积分 }),
$$
\begin{CJK}{UTF8}{mj}其中\end{CJK} $S$ \begin{CJK}{UTF8}{mj}表示三维空间中的立体区域\end{CJK} $\left\{(x, y, z): \sqrt{x^{2}+y^{2}} \leq z \leq 1\right\}$ \begin{CJK}{UTF8}{mj}的边界曲面\end{CJK}.

\begin{CJK}{UTF8}{mj}二\end{CJK}、 (\begin{CJK}{UTF8}{mj}本题共\end{CJK} 10 \begin{CJK}{UTF8}{mj}分\end{CJK})

\begin{CJK}{UTF8}{mj}设\end{CJK} $D$ \begin{CJK}{UTF8}{mj}为不包含原点的单连通闭区域\end{CJK}, \begin{CJK}{UTF8}{mj}其边界曲线\end{CJK} $L$ \begin{CJK}{UTF8}{mj}分段光滑\end{CJK}, \begin{CJK}{UTF8}{mj}试证明\end{CJK}
$$
\oint_{L} \frac{x \mathrm{~d} y-y \mathrm{~d} x}{x^{2}+y^{2}}=0
$$
\begin{CJK}{UTF8}{mj}进一步\end{CJK}, \begin{CJK}{UTF8}{mj}如果\end{CJK} $l$ \begin{CJK}{UTF8}{mj}表示以原点为中心\end{CJK}, $a>0$ \begin{CJK}{UTF8}{mj}为半径的圆周\end{CJK}, \begin{CJK}{UTF8}{mj}试计算\end{CJK}
$$
\oint_{L} \frac{x \mathrm{~d} y-y \mathrm{~d} x}{x^{2}+y^{2}}
$$
\begin{CJK}{UTF8}{mj}三\end{CJK}、 (\begin{CJK}{UTF8}{mj}本题\end{CJK} 15 \begin{CJK}{UTF8}{mj}分\end{CJK}, \begin{CJK}{UTF8}{mj}每小题\end{CJK} 5 \begin{CJK}{UTF8}{mj}分\end{CJK})
$$
f(x, y)= \begin{cases}\left(x^{2}+y^{2}\right) \sin \frac{1}{x^{2}+y^{2}}, & x^{2}+y^{2} \neq 0 \\ 0, & x^{2}+y^{2}=0 .\end{cases}
$$
\begin{CJK}{UTF8}{mj}请回答下列问题\end{CJK}, \begin{CJK}{UTF8}{mj}并说明理由\end{CJK}:

(1) $f(x, y)$ \begin{CJK}{UTF8}{mj}在\end{CJK} $(0,0)$ \begin{CJK}{UTF8}{mj}点是否连续\end{CJK}?

(2) \begin{CJK}{UTF8}{mj}偏导数\end{CJK} $f_{x}^{\prime}(0,0), f_{y}^{\prime}(0,0)$ \begin{CJK}{UTF8}{mj}是否存在\end{CJK}?

(3) $f(x, y)$ \begin{CJK}{UTF8}{mj}在\end{CJK} $(0,0)$ \begin{CJK}{UTF8}{mj}点是否可微\end{CJK}?

\begin{CJK}{UTF8}{mj}四\end{CJK}、 (\begin{CJK}{UTF8}{mj}本题共\end{CJK} 10 \begin{CJK}{UTF8}{mj}分\end{CJK})

\begin{CJK}{UTF8}{mj}求幂级数\end{CJK}
$$
\sum_{n=1}^{\infty} \frac{\left[3+(-1)^{n}\right]^{n}}{n} x^{n}
$$
\begin{CJK}{UTF8}{mj}的收敛半径及其在收敛区间上的和函数\end{CJK}. \begin{CJK}{UTF8}{mj}五\end{CJK}、 (\begin{CJK}{UTF8}{mj}本题共\end{CJK} 10 \begin{CJK}{UTF8}{mj}分\end{CJK})

\begin{CJK}{UTF8}{mj}证明函数项级数\end{CJK}
$$
\sum_{n=0}^{\infty} x^{n}(1-x)^{2}
$$
\begin{CJK}{UTF8}{mj}在\end{CJK} $[0,1]$ \begin{CJK}{UTF8}{mj}区间上一致收敛\end{CJK}.

\begin{CJK}{UTF8}{mj}六\end{CJK}、(\begin{CJK}{UTF8}{mj}本题共\end{CJK} 15 \begin{CJK}{UTF8}{mj}分\end{CJK}, \begin{CJK}{UTF8}{mj}第\end{CJK} 1 \begin{CJK}{UTF8}{mj}小题\end{CJK} 10 \begin{CJK}{UTF8}{mj}分\end{CJK}, \begin{CJK}{UTF8}{mj}第\end{CJK} 2 \begin{CJK}{UTF8}{mj}小题\end{CJK} 5 \begin{CJK}{UTF8}{mj}分\end{CJK})

\begin{enumerate}
  \item \begin{CJK}{UTF8}{mj}设\end{CJK} $p>0$, \begin{CJK}{UTF8}{mj}讨论无穷积分\end{CJK}
\end{enumerate}
$$
\int_{1}^{+\infty} \frac{\sin x}{x^{p}} \mathrm{~d} x
$$
\begin{CJK}{UTF8}{mj}的收敛性\end{CJK};

\begin{enumerate}
  \setcounter{enumi}{2}
  \item \begin{CJK}{UTF8}{mj}证明无穷积分\end{CJK}
\end{enumerate}
$$
\int_{1}^{+\infty} x \sin x^{4} \mathrm{~d} x
$$
\begin{CJK}{UTF8}{mj}条件收敛\end{CJK}.

\begin{CJK}{UTF8}{mj}七\end{CJK}、 (\begin{CJK}{UTF8}{mj}本题共\end{CJK} 10 \begin{CJK}{UTF8}{mj}分\end{CJK})

\begin{CJK}{UTF8}{mj}设\end{CJK} $f$ \begin{CJK}{UTF8}{mj}在\end{CJK} $(a,+\infty)$ \begin{CJK}{UTF8}{mj}上可导\end{CJK}, $\lim _{x \rightarrow+\infty} f^{\prime}(x)$ \begin{CJK}{UTF8}{mj}存在且有限\end{CJK}, \begin{CJK}{UTF8}{mj}求证\end{CJK}: $\lim _{x \rightarrow+\infty} f^{\prime}(x)=0$ \begin{CJK}{UTF8}{mj}当且仅当\end{CJK}
$$
\lim _{x \rightarrow+\infty} \frac{f(x)}{x}=0 .
$$
\begin{CJK}{UTF8}{mj}八\end{CJK}、 (\begin{CJK}{UTF8}{mj}本题共\end{CJK} 15 \begin{CJK}{UTF8}{mj}分\end{CJK})

\begin{CJK}{UTF8}{mj}设\end{CJK} $y=f(x)$ \begin{CJK}{UTF8}{mj}由方程\end{CJK}
$$
y-\frac{1}{2} \sin y=2 x
$$
\begin{CJK}{UTF8}{mj}所确定的隐函数\end{CJK}, \begin{CJK}{UTF8}{mj}求证\end{CJK} $y=f(x)$ \begin{CJK}{UTF8}{mj}是\end{CJK} $(-\infty,+\infty)$ \begin{CJK}{UTF8}{mj}上的严格递增函数\end{CJK}, \begin{CJK}{UTF8}{mj}并求出积分\end{CJK} $\int_{0}^{\pi} f(x) \mathrm{d} x$ \begin{CJK}{UTF8}{mj}的值\end{CJK}.

\begin{CJK}{UTF8}{mj}九\end{CJK}、 (\begin{CJK}{UTF8}{mj}本题共\end{CJK} 15 \begin{CJK}{UTF8}{mj}分\end{CJK})

\begin{CJK}{UTF8}{mj}设\end{CJK} $f$ \begin{CJK}{UTF8}{mj}在闭区间\end{CJK} $[a, b]$ \begin{CJK}{UTF8}{mj}上二阶可导\end{CJK}, \begin{CJK}{UTF8}{mj}且\end{CJK} $f^{\prime}(x) \geq 0$. \begin{CJK}{UTF8}{mj}试证明\end{CJK}:
$$
f\left(\frac{a+b}{2}\right) \leq \frac{1}{b-a} \int_{a}^{b} f(x) \mathrm{d} x \leq \frac{f(a)+f(b)}{2}
$$
\begin{CJK}{UTF8}{mj}十\end{CJK}、 (\begin{CJK}{UTF8}{mj}本题共\end{CJK} 10 \begin{CJK}{UTF8}{mj}分\end{CJK}, \begin{CJK}{UTF8}{mj}每小题\end{CJK} 5 \begin{CJK}{UTF8}{mj}分\end{CJK})

\begin{enumerate}
  \item \begin{CJK}{UTF8}{mj}证明\end{CJK}:\begin{CJK}{UTF8}{mj}函数\end{CJK} $f(x)=\sqrt{x}$ \begin{CJK}{UTF8}{mj}在\end{CJK} $[1,+\infty)$ \begin{CJK}{UTF8}{mj}上一致连续\end{CJK};

  \item \begin{CJK}{UTF8}{mj}证明\end{CJK}: \begin{CJK}{UTF8}{mj}若函数\end{CJK} $f$ \begin{CJK}{UTF8}{mj}在区间\end{CJK} $(0,1)$ \begin{CJK}{UTF8}{mj}中无界\end{CJK}, \begin{CJK}{UTF8}{mj}则\end{CJK} $f$ \begin{CJK}{UTF8}{mj}在区间\end{CJK} $(0,1)$ \begin{CJK}{UTF8}{mj}上不一致连续\end{CJK}.

\end{enumerate}
\section{7. 首都师范大学 2015 年研究生入学考试试题数学分析}
\begin{CJK}{UTF8}{mj}李扬\end{CJK}

\begin{CJK}{UTF8}{mj}微信公众号\end{CJK}: sxkyliyang

\begin{CJK}{UTF8}{mj}一\end{CJK}、 (\begin{CJK}{UTF8}{mj}本题\end{CJK} 20 \begin{CJK}{UTF8}{mj}分\end{CJK})

\begin{enumerate}
  \item \begin{CJK}{UTF8}{mj}求极限\end{CJK}
\end{enumerate}
$$
\lim _{x \rightarrow 0}\left(1+\sin ^{2} x\right)^{\mathrm{e}^{\frac{1}{1-1-x}}}
$$
2 . \begin{CJK}{UTF8}{mj}设\end{CJK} $a_{n}=1+\frac{1}{3}+\frac{1}{3^{2}}+\cdots+\frac{1}{3^{n}}$, \begin{CJK}{UTF8}{mj}求\end{CJK}
$$
\lim _{n \rightarrow \infty} \frac{a_{1}+a_{2}+\cdots a_{n}}{n}
$$
\begin{CJK}{UTF8}{mj}二\end{CJK}、 (\begin{CJK}{UTF8}{mj}本题共\end{CJK} 10 \begin{CJK}{UTF8}{mj}分\end{CJK})

\begin{CJK}{UTF8}{mj}判断无穷级数\end{CJK}
$$
\sum_{n=1}^{\infty}\left[\frac{1}{n}-\ln \left(\frac{n+1}{n}\right)\right]
$$
\begin{CJK}{UTF8}{mj}的敛散性\end{CJK}.

\begin{CJK}{UTF8}{mj}三\end{CJK}、 (\begin{CJK}{UTF8}{mj}本题\end{CJK} 30 \begin{CJK}{UTF8}{mj}分\end{CJK})

\begin{CJK}{UTF8}{mj}设\end{CJK}
$$
a_{n}=\frac{(-2)^{n}}{n(n+1)}, n=1,2, \cdots
$$
(1) \begin{CJK}{UTF8}{mj}求幂级数\end{CJK} $\sum_{n=1}^{\infty} a_{n} x^{n}$ \begin{CJK}{UTF8}{mj}的收敛域\end{CJK};

(2) \begin{CJK}{UTF8}{mj}求幂级数\end{CJK} $\sum_{n=1}^{\infty} a_{n} x^{n}$ \begin{CJK}{UTF8}{mj}的和函数\end{CJK};

(3) \begin{CJK}{UTF8}{mj}判断幂级数\end{CJK} $\sum_{n=1}^{\infty} a_{n} x^{n}$ \begin{CJK}{UTF8}{mj}在其收敛域上是否一致收敛\end{CJK}.

\begin{CJK}{UTF8}{mj}四\end{CJK}、 (\begin{CJK}{UTF8}{mj}本题共\end{CJK} 10 \begin{CJK}{UTF8}{mj}分\end{CJK})

\begin{CJK}{UTF8}{mj}求三维空间中的球体\end{CJK}
$$
x^{2}+y^{2}+z^{2} \leq 1
$$
\begin{CJK}{UTF8}{mj}被圆柱面\end{CJK} $x^{2}+y^{2}=x$ \begin{CJK}{UTF8}{mj}所截下的部分的体积\end{CJK}.

\begin{CJK}{UTF8}{mj}五\end{CJK}、 (\begin{CJK}{UTF8}{mj}本题共\end{CJK} 10 \begin{CJK}{UTF8}{mj}分\end{CJK})

\begin{CJK}{UTF8}{mj}求第一型曲面积分\end{CJK}
$$
\iint_{S} \sqrt{1-x^{2}-y^{2}} \mathrm{~d} S,
$$
\begin{CJK}{UTF8}{mj}其中\end{CJK} $S$ \begin{CJK}{UTF8}{mj}表示三维空间中的球面\end{CJK} $x^{2}+y^{2}+z^{2}=1$ \begin{CJK}{UTF8}{mj}被圆柱面\end{CJK} $x^{2}+y^{2}=x$ \begin{CJK}{UTF8}{mj}所截下的部分\end{CJK}.

\begin{CJK}{UTF8}{mj}六\end{CJK}、(\begin{CJK}{UTF8}{mj}本题共\end{CJK} 10 \begin{CJK}{UTF8}{mj}分\end{CJK})

\begin{CJK}{UTF8}{mj}设\end{CJK} $l$ \begin{CJK}{UTF8}{mj}为平面上连接点\end{CJK} $A\left(x_{1}, y_{1}\right)$ \begin{CJK}{UTF8}{mj}和点\end{CJK} $B\left(x_{2}, y_{2}\right)$ \begin{CJK}{UTF8}{mj}的一条有向光滑曲线\end{CJK}, $\varphi(x)$ \begin{CJK}{UTF8}{mj}为直线上具有一阶连续导数的函数\end{CJK}. \begin{CJK}{UTF8}{mj}试求第二型曲线积分\end{CJK}
$$
\int_{l} \varphi^{\prime}(x) \mathrm{e}^{y} \mathrm{~d} x+\varphi(x) \mathrm{e}^{y} \mathrm{~d} y
$$
\begin{CJK}{UTF8}{mj}七\end{CJK}、 (\begin{CJK}{UTF8}{mj}本题共\end{CJK} 10 \begin{CJK}{UTF8}{mj}分\end{CJK})

\begin{CJK}{UTF8}{mj}证明下列各题\end{CJK}:

(1) \begin{CJK}{UTF8}{mj}设\end{CJK} $f(x)$ \begin{CJK}{UTF8}{mj}是定义在\end{CJK} $(-\infty,+\infty)$ \begin{CJK}{UTF8}{mj}上的函数\end{CJK}, \begin{CJK}{UTF8}{mj}且对\end{CJK} $\forall x, y \in(-\infty,+\infty)$,
$$
|f(x)-f(y)| \leqslant(x-y)^{2} .
$$
\begin{CJK}{UTF8}{mj}求证\end{CJK}: $f(x)$ \begin{CJK}{UTF8}{mj}是\end{CJK} $(-\infty,+\infty)$ \begin{CJK}{UTF8}{mj}上的常值函数\end{CJK};

(2) \begin{CJK}{UTF8}{mj}设\end{CJK} $f(x)$ \begin{CJK}{UTF8}{mj}在闭区间\end{CJK} $[a, b]$ \begin{CJK}{UTF8}{mj}上具有连续一阶导数\end{CJK}, $f(a)=f(b)=0, \int_{a}^{b} f^{2}(x) \mathrm{d} x=1$. \begin{CJK}{UTF8}{mj}求证\end{CJK}:
$$
\int_{a}^{b} x f(x) f^{\prime}(x) \mathrm{d} x=-\frac{1}{2}
$$
\begin{CJK}{UTF8}{mj}八\end{CJK}、 (\begin{CJK}{UTF8}{mj}本题共\end{CJK} 20 \begin{CJK}{UTF8}{mj}分\end{CJK})

\begin{CJK}{UTF8}{mj}设无穷积分\end{CJK} $\int_{a}^{+\infty} f(x) \mathrm{d} x$ \begin{CJK}{UTF8}{mj}收敛\end{CJK}, \begin{CJK}{UTF8}{mj}试证\end{CJK}:

(1) \begin{CJK}{UTF8}{mj}若存在有限极限\end{CJK} $\lim _{x \rightarrow+\infty} f(x)=A$, \begin{CJK}{UTF8}{mj}则\end{CJK} $A=0$;

(2) \begin{CJK}{UTF8}{mj}若\end{CJK} $f(x)$ \begin{CJK}{UTF8}{mj}在\end{CJK} $[a,+\infty)$ \begin{CJK}{UTF8}{mj}上单调\end{CJK}, \begin{CJK}{UTF8}{mj}则\end{CJK} $\lim _{x \rightarrow+\infty} x f(x)=0$.

\begin{CJK}{UTF8}{mj}九\end{CJK}、 (\begin{CJK}{UTF8}{mj}本题共\end{CJK} 10 \begin{CJK}{UTF8}{mj}分\end{CJK})

\begin{CJK}{UTF8}{mj}设函数\end{CJK} $f(x, y)$ \begin{CJK}{UTF8}{mj}在\end{CJK}
$$
B(0,1)=\left\{(x, y) \in \mathbb{R}^{2} \mid x^{2}+y^{2}<1\right\}
$$
\begin{CJK}{UTF8}{mj}内连续\end{CJK}, \begin{CJK}{UTF8}{mj}且对\end{CJK} $\forall \varepsilon>0, \exists \delta>0$. \begin{CJK}{UTF8}{mj}对\end{CJK} $\forall(x, y) \in \mathbb{R}^{2}$, \begin{CJK}{UTF8}{mj}若\end{CJK} $1-\delta<\sqrt{x^{2}+y^{2}}<1$, \begin{CJK}{UTF8}{mj}都有\end{CJK} $|f(x, y)|<\varepsilon$, \begin{CJK}{UTF8}{mj}证明函数\end{CJK} $f(x, y)$ \begin{CJK}{UTF8}{mj}在\end{CJK} $B(0,1)$ \begin{CJK}{UTF8}{mj}一致连续\end{CJK}.

\begin{CJK}{UTF8}{mj}十\end{CJK}、 (\begin{CJK}{UTF8}{mj}本题共\end{CJK} 10 \begin{CJK}{UTF8}{mj}分\end{CJK})

\begin{CJK}{UTF8}{mj}设函数\end{CJK} $f(x)$ \begin{CJK}{UTF8}{mj}在\end{CJK} $[0,1]$ \begin{CJK}{UTF8}{mj}上二阶可导\end{CJK}, $f(0)=f(1)=0, f^{\prime}(0) f^{\prime}(1)>0$, \begin{CJK}{UTF8}{mj}求证\end{CJK}: \begin{CJK}{UTF8}{mj}存在\end{CJK} $\xi \in(0,1)$, \begin{CJK}{UTF8}{mj}使得\end{CJK}
$$
f^{\prime \prime}(\xi)=0
$$

\section{8. 首都师范大学 2017 年研究生入学考试试题数学分析}
\begin{CJK}{UTF8}{mj}李扬\end{CJK}

\begin{CJK}{UTF8}{mj}微信公众号\end{CJK}: sxkyliyang

\begin{CJK}{UTF8}{mj}一\end{CJK}、\begin{CJK}{UTF8}{mj}求极限\end{CJK}

(1)
$$
\lim _{x \rightarrow \infty}\left[(x+2) e^{\frac{1}{x}}-x\right] .
$$
(2)
$$
\lim _{x \rightarrow 0} \frac{\int_{0}^{x} t^{2}(t-\sin t) \mathrm{d} t}{\int_{0}^{x^{2}} t^{2} \mathrm{~d} t}
$$
\begin{CJK}{UTF8}{mj}二\end{CJK}、\begin{CJK}{UTF8}{mj}求函数\end{CJK} $f(x)$ \begin{CJK}{UTF8}{mj}的定义域及间断点并判断类型\end{CJK}
$$
f(x)=\frac{|x|(x-1)}{x\left(x^{2}-1\right)}
$$
\begin{CJK}{UTF8}{mj}三\end{CJK}、\begin{CJK}{UTF8}{mj}设\end{CJK} $z=z(x, y)$ \begin{CJK}{UTF8}{mj}是由方程\end{CJK}
$$
z^{3}-3 x y z=a^{3}(a \neq 0)
$$
\begin{CJK}{UTF8}{mj}确定的隐函数\end{CJK}, \begin{CJK}{UTF8}{mj}求\end{CJK} $\frac{\partial^{2} z}{\partial x \partial y}$ \begin{CJK}{UTF8}{mj}在原点\end{CJK} $(0,0)$ \begin{CJK}{UTF8}{mj}处的值\end{CJK}.

\begin{CJK}{UTF8}{mj}四\end{CJK}、\begin{CJK}{UTF8}{mj}设函数\end{CJK} $f(x), g(x)$ \begin{CJK}{UTF8}{mj}在区间\end{CJK} $[a, b]$ \begin{CJK}{UTF8}{mj}上二阶可导且\end{CJK} $g^{\prime \prime}(x) \neq 0, f(a)=f(b)=0, g(a)=g(b)=0$.

(1) \begin{CJK}{UTF8}{mj}求证在\end{CJK} $(a, b)$ \begin{CJK}{UTF8}{mj}内\end{CJK} $g(x) \neq 0$.

$(2)$ \begin{CJK}{UTF8}{mj}在\end{CJK} $(a, b)$ \begin{CJK}{UTF8}{mj}内至少存在一点\end{CJK} $\xi$, \begin{CJK}{UTF8}{mj}使得\end{CJK}
$$
\frac{f(\xi)}{g(\xi)}=\frac{f^{\prime \prime}(\xi)}{g^{\prime \prime}(\xi)}
$$
\begin{CJK}{UTF8}{mj}五\end{CJK}、\begin{CJK}{UTF8}{mj}求幂级数\end{CJK}
$$
\sum_{n=1}^{\infty} \frac{x^{n+1}}{n(n+1)}
$$
\begin{CJK}{UTF8}{mj}的收敛域与和函数\end{CJK}.

\begin{CJK}{UTF8}{mj}六\end{CJK}、\begin{CJK}{UTF8}{mj}设\end{CJK}
$$
f(x, y)= \begin{cases}\frac{|x|^{\alpha} y}{x^{2}+y^{2}}, & (x, y) \neq(0,0) \\ 0, & (x, y)=(0,0)\end{cases}
$$
\begin{CJK}{UTF8}{mj}其中\end{CJK} $\alpha>0$

(1) \begin{CJK}{UTF8}{mj}写出\end{CJK} $f(x, y)$ \begin{CJK}{UTF8}{mj}的偏导函数\end{CJK} (\begin{CJK}{UTF8}{mj}无需写过程\end{CJK}) \begin{CJK}{UTF8}{mj}并讨论偏导函数在原点\end{CJK} $(0,0)$ \begin{CJK}{UTF8}{mj}处的连续性\end{CJK}.

(2) \begin{CJK}{UTF8}{mj}讨论\end{CJK} $f(x, y)$ \begin{CJK}{UTF8}{mj}在\end{CJK} $(0,0)$ \begin{CJK}{UTF8}{mj}点处的可微性\end{CJK}.

\begin{CJK}{UTF8}{mj}七\end{CJK}、\begin{CJK}{UTF8}{mj}确定常数\end{CJK} $\lambda$ \begin{CJK}{UTF8}{mj}使得在右半平面\end{CJK} $x>0$ \begin{CJK}{UTF8}{mj}上的向量值函数\end{CJK}
$$
A(x, y)=\left(2 x y\left(x^{4}+y^{2}\right)^{\lambda},-x^{2}\left(x^{4}+y^{2}\right)^{\lambda}\right)
$$
\begin{CJK}{UTF8}{mj}为某二元函数\end{CJK} $u(x, y)$ \begin{CJK}{UTF8}{mj}的梯度\end{CJK}, \begin{CJK}{UTF8}{mj}并求\end{CJK} $u(x, y)$.

\begin{CJK}{UTF8}{mj}八\end{CJK}、\begin{CJK}{UTF8}{mj}设\end{CJK}
$$
\vec{F}(x, y, z)=\left(z \arctan y^{2}, z^{3} \ln \left(x^{2}+1\right), z\right)
$$
\begin{CJK}{UTF8}{mj}求\end{CJK} $\vec{F}$ \begin{CJK}{UTF8}{mj}通过抛物面\end{CJK} $x^{2}+y^{2}+z=2$ \begin{CJK}{UTF8}{mj}位于平面\end{CJK} $z=1$ \begin{CJK}{UTF8}{mj}的上方那一块流向上侧的流量\end{CJK}. \begin{CJK}{UTF8}{mj}九\end{CJK}、\begin{CJK}{UTF8}{mj}设\end{CJK} $f$ \begin{CJK}{UTF8}{mj}在区间\end{CJK} $[0,1]$ \begin{CJK}{UTF8}{mj}上连续\end{CJK}, \begin{CJK}{UTF8}{mj}试证明\end{CJK}
$$
\lim _{n \rightarrow \infty} n \int_{0}^{1} x^{n} f(x) \mathrm{d} x=f(1)
$$
\begin{CJK}{UTF8}{mj}十\end{CJK}、\begin{CJK}{UTF8}{mj}设\end{CJK} $\int_{0}^{+\infty} f(x, y) \mathrm{d} y$ \begin{CJK}{UTF8}{mj}在\end{CJK} $[0,+\infty)$ \begin{CJK}{UTF8}{mj}上一致收敛于\end{CJK} $F(x)$, \begin{CJK}{UTF8}{mj}且对于任意区间\end{CJK} $[a, b] \subset[a,+\infty)$, \begin{CJK}{UTF8}{mj}极限\end{CJK} $\lim _{x \rightarrow+\infty} f(x, t)=\psi(t)$ \begin{CJK}{UTF8}{mj}关于\end{CJK} $t$ \begin{CJK}{UTF8}{mj}在\end{CJK} $[a, b]$ \begin{CJK}{UTF8}{mj}上一致地成立\end{CJK}, \begin{CJK}{UTF8}{mj}即对任意\end{CJK} $\varepsilon>0$, \begin{CJK}{UTF8}{mj}存在\end{CJK} $M>0$, \begin{CJK}{UTF8}{mj}当\end{CJK} $x>M$ \begin{CJK}{UTF8}{mj}时\end{CJK}, $|f(x, t)-\psi(t)|<\varepsilon$ \begin{CJK}{UTF8}{mj}对一切\end{CJK} $t \in(a, b)$ \begin{CJK}{UTF8}{mj}成立\end{CJK}, \begin{CJK}{UTF8}{mj}试证明\end{CJK}:

(1)
$$
\int_{0}^{+\infty} \psi(t) \mathrm{d} t
$$
\begin{CJK}{UTF8}{mj}收敛\end{CJK}.
$$
\lim _{x \rightarrow+\infty} F(x)=\int_{0}^{+\infty} \psi(t) \mathrm{d} t
$$

\section{9. 首都师范大学 2009 年研究生入学考试试题高等代数 
 李扬 
 微信公众号: sxkyliyang}
\begin{enumerate}
  \item (15 \begin{CJK}{UTF8}{mj}分\end{CJK}) \begin{CJK}{UTF8}{mj}设\end{CJK} $f(x), g(x) \in \mathbb{Q}[x]$ \begin{CJK}{UTF8}{mj}为有理数域\end{CJK} $\mathbb{Q}$ \begin{CJK}{UTF8}{mj}上的两个多项式\end{CJK}, $m$ \begin{CJK}{UTF8}{mj}为一个正整数\end{CJK}, \begin{CJK}{UTF8}{mj}证明\end{CJK}: $f(x)^{m} \mid g(x)^{m}$ \begin{CJK}{UTF8}{mj}当且仅当\end{CJK}
\end{enumerate}
$$
f(x) \mid g(x)
$$

\begin{enumerate}
  \setcounter{enumi}{2}
  \item ( 15 \begin{CJK}{UTF8}{mj}分\end{CJK}) \begin{CJK}{UTF8}{mj}计算行列式\end{CJK}
\end{enumerate}
$$
D_{n}=\left|\begin{array}{cccc}
1+a_{1} & 1 & \cdots & 1 \\
1 & 1+a_{2} & \cdots & 1 \\
\vdots & \vdots & & \vdots \\
1 & 1 & \cdots & 1+a_{n}
\end{array}\right|
$$
\begin{CJK}{UTF8}{mj}其中\end{CJK} $a_{i} \neq 0,1 \leq i \leq n$.

\begin{enumerate}
  \setcounter{enumi}{3}
  \item (15 \begin{CJK}{UTF8}{mj}分\end{CJK}) \begin{CJK}{UTF8}{mj}证明\end{CJK}: \begin{CJK}{UTF8}{mj}线性方程组\end{CJK}
\end{enumerate}
$$
\left\{\begin{array}{l}
a_{11} x_{1}+a_{12} x_{2}+\cdots+a_{1 n} x_{n}=b_{1} \\
a_{21} x_{1}+a_{22} x_{2}+\cdots+a_{2 n} x_{n}=b_{2} \\
\vdots \\
a_{m 1} x_{1}+a_{m 2} x_{2}+\cdots+a_{m n} x_{n}=b_{m}
\end{array}\right.
$$
\begin{CJK}{UTF8}{mj}有解的充要条件是方程组的系数矩阵与增广矩阵具有相同的秩\end{CJK}.

\begin{enumerate}
  \setcounter{enumi}{4}
  \item ( 15 \begin{CJK}{UTF8}{mj}分\end{CJK}) \begin{CJK}{UTF8}{mj}已知实数域\end{CJK} $\mathbb{R}$ \begin{CJK}{UTF8}{mj}上的\end{CJK} 3 \begin{CJK}{UTF8}{mj}阶方阵\end{CJK}
\end{enumerate}
$$
A=\left(\begin{array}{ccc}
5 & 0 & 0 \\
-2 & 9 & -2 \\
1 & -2 & 6
\end{array}\right)
$$
\begin{CJK}{UTF8}{mj}求一个实可逆矩阵\end{CJK} $T$ \begin{CJK}{UTF8}{mj}使得\end{CJK} $T^{-1} A T$ \begin{CJK}{UTF8}{mj}成为对角矩阵\end{CJK}, \begin{CJK}{UTF8}{mj}其中\end{CJK} $T^{-1}$ \begin{CJK}{UTF8}{mj}表示矩阵\end{CJK} $T$ \begin{CJK}{UTF8}{mj}的逆矩阵\end{CJK}.

\begin{enumerate}
  \setcounter{enumi}{5}
  \item (15 \begin{CJK}{UTF8}{mj}分\end{CJK}) \begin{CJK}{UTF8}{mj}试求\end{CJK} $\lambda$ \begin{CJK}{UTF8}{mj}取什么值得时候\end{CJK}, \begin{CJK}{UTF8}{mj}下列二次型\end{CJK}
\end{enumerate}
$$
5 x_{1}^{2}+x_{2}^{2}+\lambda x_{3}^{2}+4 x_{1} x_{2}-2 x_{1} x_{3}-2 x_{2} x_{3}
$$
\begin{CJK}{UTF8}{mj}是正定二次型\end{CJK}.

\begin{enumerate}
  \setcounter{enumi}{6}
  \item (15 \begin{CJK}{UTF8}{mj}分\end{CJK}) \begin{CJK}{UTF8}{mj}设\end{CJK} $A, B$ \begin{CJK}{UTF8}{mj}分别为数域\end{CJK} $K$ \begin{CJK}{UTF8}{mj}上的\end{CJK} $m \times n, n \times m$ \begin{CJK}{UTF8}{mj}矩阵\end{CJK}, $E_{m}$ \begin{CJK}{UTF8}{mj}为\end{CJK} $m$ \begin{CJK}{UTF8}{mj}阶单位矩阵\end{CJK}, $A B=E_{m}, m \leq n$. \begin{CJK}{UTF8}{mj}证明\end{CJK}: $B$ \begin{CJK}{UTF8}{mj}的列向量可以扩充成\end{CJK} $K$ \begin{CJK}{UTF8}{mj}上\end{CJK} $n$ \begin{CJK}{UTF8}{mj}维列向量空间\end{CJK} $K^{n}$ \begin{CJK}{UTF8}{mj}的一组基\end{CJK}.

  \item ( 15 \begin{CJK}{UTF8}{mj}分\end{CJK}) \begin{CJK}{UTF8}{mj}设\end{CJK} $V_{1}, V_{2}$ \begin{CJK}{UTF8}{mj}为有限维向量空间\end{CJK} $V$ \begin{CJK}{UTF8}{mj}的两个线性子空间\end{CJK}, $V_{1}+V_{2}, V_{1} \cap V_{2}$ \begin{CJK}{UTF8}{mj}分别是它们的和空间与交空间\end{CJK}, $\operatorname{dim} V_{1}, \operatorname{dim} V_{2}, \operatorname{dim}\left(V_{1}+V_{2}\right), \operatorname{dim}\left(V_{1} \cap V_{2}\right)$ \begin{CJK}{UTF8}{mj}分别为\end{CJK} $V_{1}, V_{2}, V_{1}+V_{2}, V_{1} \cap V_{2}$ \begin{CJK}{UTF8}{mj}的维数\end{CJK}, \begin{CJK}{UTF8}{mj}证明\end{CJK}:

\end{enumerate}
$$
\operatorname{dim} V_{1}+\operatorname{dim} V_{2}=\operatorname{dim}\left(V_{1}+V_{2}\right)+\operatorname{dim}\left(V_{1} \cap V_{2}\right)
$$

\begin{enumerate}
  \setcounter{enumi}{8}
  \item (15 \begin{CJK}{UTF8}{mj}分\end{CJK}) \begin{CJK}{UTF8}{mj}把\end{CJK} 3 \begin{CJK}{UTF8}{mj}维单位向量\end{CJK}
\end{enumerate}
$$
\gamma_{1}=\frac{1}{\sqrt{3}}(1,1,1)
$$
\begin{CJK}{UTF8}{mj}扩充为\end{CJK} 3 \begin{CJK}{UTF8}{mj}维欧式空间\end{CJK} $\mathbb{R}^{3}$ \begin{CJK}{UTF8}{mj}的标准正交基\end{CJK}. 9. (15 \begin{CJK}{UTF8}{mj}分\end{CJK}) \begin{CJK}{UTF8}{mj}记\end{CJK} $X=\mathbb{R}^{n}$ \begin{CJK}{UTF8}{mj}为实数域\end{CJK} $\mathbb{R}$ \begin{CJK}{UTF8}{mj}上\end{CJK} $n$ \begin{CJK}{UTF8}{mj}维标准欧式空间\end{CJK}, $A$ \begin{CJK}{UTF8}{mj}为实数域\end{CJK} $\mathbb{R}$ \begin{CJK}{UTF8}{mj}上的一个\end{CJK} $n$ \begin{CJK}{UTF8}{mj}阶方阵\end{CJK},
$$
\begin{gathered}
V=\left\{\xi \mid \xi \in X, A^{T} \xi=0\right\} \\
W=\{A \xi \mid \xi \in X\}
\end{gathered}
$$
\begin{CJK}{UTF8}{mj}其中\end{CJK} $A^{T}$ \begin{CJK}{UTF8}{mj}表示矩阵\end{CJK} $A$ \begin{CJK}{UTF8}{mj}的转置矩阵\end{CJK}, \begin{CJK}{UTF8}{mj}证明\end{CJK}: $X=V \oplus W$.

\begin{enumerate}
  \setcounter{enumi}{10}
  \item ( 15 \begin{CJK}{UTF8}{mj}分\end{CJK}) \begin{CJK}{UTF8}{mj}设\end{CJK} $A$ \begin{CJK}{UTF8}{mj}为实数域\end{CJK} $\mathbb{R}$ \begin{CJK}{UTF8}{mj}上的一个\end{CJK} $n$ \begin{CJK}{UTF8}{mj}阶方阵\end{CJK}, \begin{CJK}{UTF8}{mj}满足\end{CJK} $A A^{T}=A^{T} A$, \begin{CJK}{UTF8}{mj}其中\end{CJK} $A^{T}$ \begin{CJK}{UTF8}{mj}表示矩阵\end{CJK} $A$ \begin{CJK}{UTF8}{mj}的转置矩阵\end{CJK}.
\end{enumerate}
(1) \begin{CJK}{UTF8}{mj}设\end{CJK} $\lambda \in \mathbb{R}$ \begin{CJK}{UTF8}{mj}为\end{CJK} $A$ \begin{CJK}{UTF8}{mj}的一个特征值\end{CJK}, \begin{CJK}{UTF8}{mj}证明\end{CJK}: $\lambda$ \begin{CJK}{UTF8}{mj}也是\end{CJK} $A^{T}$ \begin{CJK}{UTF8}{mj}的特征值\end{CJK};

(2) \begin{CJK}{UTF8}{mj}证明\end{CJK}: \begin{CJK}{UTF8}{mj}如果\end{CJK} $A$ \begin{CJK}{UTF8}{mj}的所有特征值都是实数\end{CJK}, \begin{CJK}{UTF8}{mj}则\end{CJK} $A$ \begin{CJK}{UTF8}{mj}是一个对称矩阵\end{CJK}.

\section{0. 首都师范大学 2010 年研究生入学考试试题高等代数 
 李扬 
 微信公众号: sxkyliyang}
\begin{enumerate}
  \item (15 \begin{CJK}{UTF8}{mj}分\end{CJK}) \begin{CJK}{UTF8}{mj}计算行列式\end{CJK}
\end{enumerate}
$$
D=\left|\begin{array}{llll}
a & b & c & d \\
b & a & c & d \\
c & b & a & d \\
b & d & c & a
\end{array}\right|
$$

\begin{enumerate}
  \setcounter{enumi}{2}
  \item (15 \begin{CJK}{UTF8}{mj}分\end{CJK}) \begin{CJK}{UTF8}{mj}设\end{CJK} $\lambda$ \begin{CJK}{UTF8}{mj}为复数\end{CJK}, \begin{CJK}{UTF8}{mj}计算矩阵\end{CJK}
\end{enumerate}
$$
D=\left(\begin{array}{llll}
\lambda & 1 & 0 & 0 \\
0 & \lambda & 1 & 0 \\
0 & 0 & \lambda & 1 \\
0 & 0 & 0 & \lambda
\end{array}\right)^{17}
$$

\begin{enumerate}
  \setcounter{enumi}{3}
  \item (15 \begin{CJK}{UTF8}{mj}分\end{CJK}) \begin{CJK}{UTF8}{mj}设\end{CJK} $A$ \begin{CJK}{UTF8}{mj}为\end{CJK} $n$ \begin{CJK}{UTF8}{mj}阶矩阵\end{CJK} (\begin{CJK}{UTF8}{mj}不一定是实矩阵\end{CJK}), \begin{CJK}{UTF8}{mj}证明\end{CJK}:
\end{enumerate}
$$
\mathrm{r}\left(A A^{\prime}\right) \geq 2 \mathrm{r}(A)-n
$$
\begin{CJK}{UTF8}{mj}并给出等号成立的一个充分条件\end{CJK} (\begin{CJK}{UTF8}{mj}不必证明这个条件\end{CJK}).

\begin{enumerate}
  \setcounter{enumi}{4}
  \item ( 15 \begin{CJK}{UTF8}{mj}分\end{CJK}) \begin{CJK}{UTF8}{mj}设方阵\end{CJK} $A$ \begin{CJK}{UTF8}{mj}的特征多项式为\end{CJK} $f(\lambda)=(\lambda-1)^{2}(\lambda+1)$, \begin{CJK}{UTF8}{mj}求矩阵\end{CJK}
\end{enumerate}
$$
B=A^{3}+2 A^{2}+3 A
$$
\begin{CJK}{UTF8}{mj}的行列式\end{CJK}.

\begin{enumerate}
  \setcounter{enumi}{5}
  \item (15 \begin{CJK}{UTF8}{mj}分\end{CJK}) \begin{CJK}{UTF8}{mj}设\end{CJK} 2 \begin{CJK}{UTF8}{mj}阶方阵\end{CJK} $A$ \begin{CJK}{UTF8}{mj}中所有元都是正实数\end{CJK}, \begin{CJK}{UTF8}{mj}证明\end{CJK}: $A$ \begin{CJK}{UTF8}{mj}有实特征向量\end{CJK} (\begin{CJK}{UTF8}{mj}即每个分量都是实数的特征向量\end{CJK}).

  \item (15 \begin{CJK}{UTF8}{mj}分\end{CJK}) \begin{CJK}{UTF8}{mj}设\end{CJK} $V$ \begin{CJK}{UTF8}{mj}为所有\end{CJK} $n$ \begin{CJK}{UTF8}{mj}阶实对称方阵组成的实线性空间\end{CJK}, \begin{CJK}{UTF8}{mj}计算\end{CJK} $V$ \begin{CJK}{UTF8}{mj}的维数\end{CJK}.

  \item (15 \begin{CJK}{UTF8}{mj}分\end{CJK}) \begin{CJK}{UTF8}{mj}设\end{CJK} $B$ \begin{CJK}{UTF8}{mj}为\end{CJK} $n$ \begin{CJK}{UTF8}{mj}阶可逆矩阵\end{CJK},

\end{enumerate}
$$
A=\left(\begin{array}{cc}
B & 0 \\
0 & 0
\end{array}\right)
$$
\begin{CJK}{UTF8}{mj}为\end{CJK} $m$ \begin{CJK}{UTF8}{mj}阶矩阵\end{CJK}, $V$ \begin{CJK}{UTF8}{mj}为\end{CJK} $m$ \begin{CJK}{UTF8}{mj}维线性空间\end{CJK}, $\mathscr{A}$ \begin{CJK}{UTF8}{mj}为\end{CJK} $V$ \begin{CJK}{UTF8}{mj}上的线性变换\end{CJK}, \begin{CJK}{UTF8}{mj}对应矩阵为\end{CJK} $A$. \begin{CJK}{UTF8}{mj}证明\end{CJK}:
$$
\mathscr{A}^{2} V=\mathscr{A} V, \text { ker } \mathscr{A}=\operatorname{ker} \mathscr{A}^{2}, V=\mathscr{A} V \oplus \operatorname{ker} \mathscr{A}
$$

\begin{enumerate}
  \setcounter{enumi}{8}
  \item (15 \begin{CJK}{UTF8}{mj}分\end{CJK}) \begin{CJK}{UTF8}{mj}设\end{CJK} $V=\mathbb{R}^{4}$ \begin{CJK}{UTF8}{mj}是实数域\end{CJK} $\mathbb{R}$ \begin{CJK}{UTF8}{mj}上通常的\end{CJK} 4 \begin{CJK}{UTF8}{mj}维欧式空间\end{CJK}, $\varepsilon_{1}=\left(\frac{1}{2}, \frac{1}{2}, \frac{1}{2}, \frac{1}{2}\right)$ \begin{CJK}{UTF8}{mj}和\end{CJK} $\varepsilon_{2}=\left(\frac{1}{2}, \frac{-1}{2}, \frac{1}{2}, \frac{-1}{2}\right)$, \begin{CJK}{UTF8}{mj}求\end{CJK} $V$ \begin{CJK}{UTF8}{mj}中向量\end{CJK} $\varepsilon_{3}, \varepsilon_{4}$ \begin{CJK}{UTF8}{mj}使得\end{CJK} $\varepsilon_{1}, \varepsilon_{2}, \varepsilon_{3}, \varepsilon_{4}$ \begin{CJK}{UTF8}{mj}为\end{CJK} $V$ \begin{CJK}{UTF8}{mj}的一组标准正交基\end{CJK}.

  \item (15 \begin{CJK}{UTF8}{mj}分\end{CJK}) \begin{CJK}{UTF8}{mj}设\end{CJK} $A$ \begin{CJK}{UTF8}{mj}是行列式为\end{CJK} $-1$ \begin{CJK}{UTF8}{mj}的正交矩阵\end{CJK}, \begin{CJK}{UTF8}{mj}证明\end{CJK}: $-1$ \begin{CJK}{UTF8}{mj}是\end{CJK} $A$ \begin{CJK}{UTF8}{mj}的一个特征值\end{CJK}.

  \item (15 \begin{CJK}{UTF8}{mj}分\end{CJK}) \begin{CJK}{UTF8}{mj}设\end{CJK} $A, B$ \begin{CJK}{UTF8}{mj}为\end{CJK} $n$ \begin{CJK}{UTF8}{mj}阶半正定矩阵\end{CJK}, \begin{CJK}{UTF8}{mj}证明\end{CJK}: $A B$ \begin{CJK}{UTF8}{mj}的特征值全是非负实数\end{CJK}.

\end{enumerate}
\section{1. 首都师范大学 2011 年研究生入学考试试题高等代数 
 李扬 
 微信公众号: sxkyliyang}
\begin{enumerate}
  \item (15 \begin{CJK}{UTF8}{mj}分\end{CJK}) \begin{CJK}{UTF8}{mj}设\end{CJK}
\end{enumerate}
$$
A=\left(\begin{array}{ccccc}
1 & -1 & 2 & 1 & 0 \\
0 & 2 & -2 & -2 & 1 \\
0 & -1 & -1 & 1 & 1 \\
1 & 1 & 0 & 1 & -1
\end{array}\right)
$$
(1) $A$ \begin{CJK}{UTF8}{mj}的秩是多少\end{CJK};

(2) \begin{CJK}{UTF8}{mj}说明你对\end{CJK} (1) \begin{CJK}{UTF8}{mj}的答案的理由\end{CJK};

(3) \begin{CJK}{UTF8}{mj}求出\end{CJK} $A$ \begin{CJK}{UTF8}{mj}的列向量的一个极大无关组\end{CJK}.

\begin{enumerate}
  \setcounter{enumi}{2}
  \item (15 \begin{CJK}{UTF8}{mj}分\end{CJK}) \begin{CJK}{UTF8}{mj}设\end{CJK} $f(x), g(x)$ \begin{CJK}{UTF8}{mj}是实系数多项式\end{CJK}, \begin{CJK}{UTF8}{mj}且\end{CJK}
\end{enumerate}
$$
\left(x^{2}+2\right) f(x)-\left(x^{3}+1\right) g(x)=1 .
$$
\begin{CJK}{UTF8}{mj}求\end{CJK}:

(1) \begin{CJK}{UTF8}{mj}求\end{CJK} $f(x), g(x)$ \begin{CJK}{UTF8}{mj}的最大公因式\end{CJK} $(f(x), g(x))$;

(2) $f(x), g(x)$ \begin{CJK}{UTF8}{mj}都是非零的\end{CJK}, \begin{CJK}{UTF8}{mj}而且对于任意实系数多项式\end{CJK} $h(x)$ \begin{CJK}{UTF8}{mj}都存在实系数多项式\end{CJK} $p(x), q(x)$ \begin{CJK}{UTF8}{mj}使得\end{CJK}
$$
h(x)=p(x) f(x)+q(x) g(x) .
$$

\begin{enumerate}
  \setcounter{enumi}{3}
  \item ( 15 \begin{CJK}{UTF8}{mj}分\end{CJK}) \begin{CJK}{UTF8}{mj}设\end{CJK}
\end{enumerate}
$$
A=\left(\begin{array}{cccc}
a_{11} & a_{12} & \cdots & a_{1 n} \\
a_{21} & a_{22} & \cdots & a_{2 n} \\
\vdots & \vdots & & \vdots \\
a_{n 1} & a_{n 2} & \cdots & a_{n n}
\end{array}\right)_{n \times n}
$$
\begin{CJK}{UTF8}{mj}为实矩阵\end{CJK}, $V=\mathbb{R}^{n}, W=\mathbb{R}^{m}$ \begin{CJK}{UTF8}{mj}分别为通常的\end{CJK} $n, m$ \begin{CJK}{UTF8}{mj}维实列向量空间\end{CJK}. \begin{CJK}{UTF8}{mj}映射\end{CJK} $f: V \rightarrow W$ \begin{CJK}{UTF8}{mj}定义为\end{CJK}: \begin{CJK}{UTF8}{mj}任给\end{CJK} $X=\left(x_{1}, x_{2}, \cdots, x_{n}\right)^{T} \in V$, \begin{CJK}{UTF8}{mj}有\end{CJK} $f(X)=\left(y_{1}, y_{2}, \cdots, y_{n}\right)^{T}$, \begin{CJK}{UTF8}{mj}其中\end{CJK} $y_{i}=\sum_{i=1}^{m} a_{i} x_{i}, i=1,2, \cdots, m$, \begin{CJK}{UTF8}{mj}记\end{CJK}
$$
\operatorname{Im} f=\{f(X) \mid X \in V\}
$$
$$
\operatorname{ker} f=\{X \in V \mid f(X)=0\}
$$
\begin{CJK}{UTF8}{mj}非别是\end{CJK} $f$ \begin{CJK}{UTF8}{mj}的象和核\end{CJK}, \begin{CJK}{UTF8}{mj}证明\end{CJK}:

(1) \begin{CJK}{UTF8}{mj}对\end{CJK} $b=\left(b_{1}, b_{2}, \cdots, b_{n}\right)^{T} \in W$, \begin{CJK}{UTF8}{mj}方程\end{CJK} $A X=b$ \begin{CJK}{UTF8}{mj}有解的充要条件是\end{CJK} $b \in \operatorname{Im} f$;

(2) Im $f$ \begin{CJK}{UTF8}{mj}是\end{CJK} $W$ \begin{CJK}{UTF8}{mj}的子空间\end{CJK}, \begin{CJK}{UTF8}{mj}则\end{CJK} ker $f$ \begin{CJK}{UTF8}{mj}是\end{CJK} $V$ \begin{CJK}{UTF8}{mj}的子空间\end{CJK};

(3) $\operatorname{dim} \operatorname{ker} f+\operatorname{dim} \operatorname{Im} f=n$, \begin{CJK}{UTF8}{mj}其中\end{CJK} $\operatorname{dim}$ \begin{CJK}{UTF8}{mj}是维数\end{CJK}.

\begin{enumerate}
  \setcounter{enumi}{4}
  \item ( 15 \begin{CJK}{UTF8}{mj}分\end{CJK}) \begin{CJK}{UTF8}{mj}设\end{CJK}
\end{enumerate}
$$
A=\left(\begin{array}{lll}
1 & 2 & 2 \\
2 & 1 & 2 \\
2 & 2 & 1
\end{array}\right)
$$
(1) \begin{CJK}{UTF8}{mj}求\end{CJK} $A$ \begin{CJK}{UTF8}{mj}的特征多项式\end{CJK};

(2) \begin{CJK}{UTF8}{mj}求\end{CJK} $A$ \begin{CJK}{UTF8}{mj}的特征根及重数\end{CJK};

(3) \begin{CJK}{UTF8}{mj}若\end{CJK} $f(x)=\left(x^{2}-1\right)(x+1)(x-5)+2$, \begin{CJK}{UTF8}{mj}计算\end{CJK} $A$ \begin{CJK}{UTF8}{mj}的多项式\end{CJK} $f(A)$.

\begin{enumerate}
  \setcounter{enumi}{5}
  \item ( 15 \begin{CJK}{UTF8}{mj}分\end{CJK})\begin{CJK}{UTF8}{mj}计算行列式\end{CJK}
\end{enumerate}
$$
D=\left|\begin{array}{ccccc}
1+a_{1} & 2 & \cdots & n-1 & n \\
1 & 2+a_{2} & \cdots & n-1 & n \\
\vdots & \vdots & & \vdots & \vdots \\
1 & 2 & \cdots & 1+a_{n-1} & n \\
1 & 2 & \cdots & n-1 & 1+a_{n}
\end{array}\right|
$$
\begin{CJK}{UTF8}{mj}其中每一个\end{CJK} $a_{i} \neq 0$, \begin{CJK}{UTF8}{mj}若有\end{CJK} $a_{i}=0$ \begin{CJK}{UTF8}{mj}讨论结论\end{CJK}.

\begin{enumerate}
  \setcounter{enumi}{6}
  \item (15 \begin{CJK}{UTF8}{mj}分\end{CJK}) \begin{CJK}{UTF8}{mj}解线性方程组\end{CJK}
\end{enumerate}
$$
\left\{\begin{array}{l}
2 x_{1}+x_{2}-5 x_{3}+x_{4}+6 x_{5}=8 \\
x_{1}-3 x_{2}-6 x_{4}+4 x_{5}=9 \\
2 x_{2}-x_{3}+2 x_{4}-x_{5}=-5 \\
x_{1}+4 x_{2}-7 x_{3}+6 x_{4}+4 x_{5}=0 \\
3 x_{1}-6 x_{3}-3 x_{4}+9 x_{5}=12 \\
2 x_{1}+3 x_{2}-8 x_{3}+2 x_{4}+7 x_{5}=4
\end{array}\right.
$$

\begin{enumerate}
  \setcounter{enumi}{7}
  \item (15 \begin{CJK}{UTF8}{mj}分\end{CJK}) \begin{CJK}{UTF8}{mj}设\end{CJK} $A, B$ \begin{CJK}{UTF8}{mj}是\end{CJK} $n$ \begin{CJK}{UTF8}{mj}阶正定矩阵\end{CJK}, \begin{CJK}{UTF8}{mj}证明\end{CJK}:
\end{enumerate}
$$
A^{2}-A+E+B^{-1}+B
$$
\begin{CJK}{UTF8}{mj}是正定矩阵\end{CJK}.

\begin{enumerate}
  \setcounter{enumi}{8}
  \item (15 \begin{CJK}{UTF8}{mj}分\end{CJK}) \begin{CJK}{UTF8}{mj}设\end{CJK} $T$ \begin{CJK}{UTF8}{mj}为\end{CJK} $n$ \begin{CJK}{UTF8}{mj}阶实方阵\end{CJK}, $E$ \begin{CJK}{UTF8}{mj}为\end{CJK} $n$ \begin{CJK}{UTF8}{mj}阶单位矩阵\end{CJK}, $T^{2}=E$, \begin{CJK}{UTF8}{mj}证明\end{CJK}: \begin{CJK}{UTF8}{mj}任给\end{CJK} $n$ \begin{CJK}{UTF8}{mj}维实列向量\end{CJK} $\alpha$ \begin{CJK}{UTF8}{mj}都存在唯一的\end{CJK} $n$ \begin{CJK}{UTF8}{mj}维实\end{CJK} \begin{CJK}{UTF8}{mj}列向量\end{CJK} $\beta, \gamma$ \begin{CJK}{UTF8}{mj}使得\end{CJK} $\alpha=\beta+\gamma$ \begin{CJK}{UTF8}{mj}且\end{CJK} $T \beta=\beta, T \gamma=-\gamma$.

  \item ( 15 \begin{CJK}{UTF8}{mj}分\end{CJK}) \begin{CJK}{UTF8}{mj}设\end{CJK} $V$ \begin{CJK}{UTF8}{mj}为\end{CJK} $n$ \begin{CJK}{UTF8}{mj}维欧式空间\end{CJK}, $\alpha_{1}, \cdots, \alpha_{n}, \alpha_{n+1}$ \begin{CJK}{UTF8}{mj}是\end{CJK} $V$ \begin{CJK}{UTF8}{mj}中的\end{CJK} $n+1$ \begin{CJK}{UTF8}{mj}个非零向量\end{CJK}, \begin{CJK}{UTF8}{mj}证明\end{CJK}: $V$ \begin{CJK}{UTF8}{mj}中存在非零向量\end{CJK} $\alpha$ \begin{CJK}{UTF8}{mj}与\end{CJK} $\alpha_{1}, \cdots, \alpha_{n}, \alpha_{n+1}$ \begin{CJK}{UTF8}{mj}都不正交\end{CJK}.

  \item (15 \begin{CJK}{UTF8}{mj}分\end{CJK}) \begin{CJK}{UTF8}{mj}设\end{CJK} $A$ \begin{CJK}{UTF8}{mj}为\end{CJK} $n$ \begin{CJK}{UTF8}{mj}阶非零实矩阵\end{CJK}, \begin{CJK}{UTF8}{mj}且存在实数\end{CJK} $c, d$ \begin{CJK}{UTF8}{mj}使得\end{CJK}

\end{enumerate}
$$
A^{2}=c A, c d \neq-1
$$
\begin{CJK}{UTF8}{mj}证明\end{CJK}: $E+d A$ \begin{CJK}{UTF8}{mj}可逆\end{CJK}, \begin{CJK}{UTF8}{mj}其中\end{CJK} $E$ \begin{CJK}{UTF8}{mj}是\end{CJK} $n$ \begin{CJK}{UTF8}{mj}阶单位矩阵\end{CJK}.

\section{2. 首都师范大学 2012 年研究生入学考试试题高等代数 
 李扬 
 微信公众号: sxkyliyang}
\begin{enumerate}
  \item (15 \begin{CJK}{UTF8}{mj}分\end{CJK}) \begin{CJK}{UTF8}{mj}求方程组\end{CJK}
\end{enumerate}
$$
\left\{\begin{array}{l}
4 x+y+11 z=-5 \\
-x+8 y+10 z=5
\end{array}\right.
$$
\begin{CJK}{UTF8}{mj}的通解\end{CJK}.

\begin{enumerate}
  \setcounter{enumi}{2}
  \item (15 \begin{CJK}{UTF8}{mj}分\end{CJK}) \begin{CJK}{UTF8}{mj}设\end{CJK} $A$ \begin{CJK}{UTF8}{mj}为\end{CJK} $m \times n$ \begin{CJK}{UTF8}{mj}矩阵\end{CJK}, $\beta$ \begin{CJK}{UTF8}{mj}是\end{CJK} $m$ \begin{CJK}{UTF8}{mj}维列向量\end{CJK}, \begin{CJK}{UTF8}{mj}证明\end{CJK}: \begin{CJK}{UTF8}{mj}方程组\end{CJK} $A X=\beta$ \begin{CJK}{UTF8}{mj}有解当且仅当方程组\end{CJK} $A^{\prime} Y=0$ \begin{CJK}{UTF8}{mj}的解都\end{CJK} \begin{CJK}{UTF8}{mj}是方程\end{CJK} $\beta^{\prime} Y=0$ \begin{CJK}{UTF8}{mj}的解\end{CJK} ( $A^{\prime}$ \begin{CJK}{UTF8}{mj}为\end{CJK} $A$ \begin{CJK}{UTF8}{mj}的转置矩阵\end{CJK}).

  \item ( 15 \begin{CJK}{UTF8}{mj}分\end{CJK}) \begin{CJK}{UTF8}{mj}证明行列式\end{CJK}

\end{enumerate}
$$
A=\left|\begin{array}{cccc}
a^{2} & a b & a b & b^{2} \\
a c & a d & b c & b d \\
a c & b c & a d & b d \\
c^{2} & c d & c d & d^{2}
\end{array}\right|=(a d-b c)^{4}
$$

\begin{enumerate}
  \setcounter{enumi}{4}
  \item (15 \begin{CJK}{UTF8}{mj}分\end{CJK}) \begin{CJK}{UTF8}{mj}设\end{CJK} $f(x), g(x)$ \begin{CJK}{UTF8}{mj}为有理系数非零多项式\end{CJK}, \begin{CJK}{UTF8}{mj}其中\end{CJK} $f(x)$ \begin{CJK}{UTF8}{mj}是不可约的\end{CJK} (\begin{CJK}{UTF8}{mj}即不能分解为两个较低有理系数多项\end{CJK} \begin{CJK}{UTF8}{mj}式的积\end{CJK}). \begin{CJK}{UTF8}{mj}假设存在复数\end{CJK} $\alpha$ \begin{CJK}{UTF8}{mj}使得\end{CJK} $f(\alpha)=g(\alpha)=0$, \begin{CJK}{UTF8}{mj}证明\end{CJK}:
\end{enumerate}
$$
f(x) \mid g(x)
$$

\begin{enumerate}
  \setcounter{enumi}{5}
  \item ( 15 \begin{CJK}{UTF8}{mj}分\end{CJK}) \begin{CJK}{UTF8}{mj}设\end{CJK} $V$ \begin{CJK}{UTF8}{mj}为\end{CJK} 7 \begin{CJK}{UTF8}{mj}维实线性空间\end{CJK}, $W \subset V$ \begin{CJK}{UTF8}{mj}为\end{CJK} 4 \begin{CJK}{UTF8}{mj}维线性子空间\end{CJK}, \begin{CJK}{UTF8}{mj}记\end{CJK} $\operatorname{End}(V)$ \begin{CJK}{UTF8}{mj}为\end{CJK} $V$ \begin{CJK}{UTF8}{mj}到自身的所有线性映射组成\end{CJK} \begin{CJK}{UTF8}{mj}的线性空间\end{CJK}, \begin{CJK}{UTF8}{mj}令\end{CJK}
\end{enumerate}
$$
M=\{f \in \operatorname{End}(V) \mid F(W) \subset W\},
$$
\begin{CJK}{UTF8}{mj}说明\end{CJK} $M$ \begin{CJK}{UTF8}{mj}是\end{CJK} $\operatorname{End}(V)$ \begin{CJK}{UTF8}{mj}的线性子空间\end{CJK}, \begin{CJK}{UTF8}{mj}并给出\end{CJK} $M$ \begin{CJK}{UTF8}{mj}的维数\end{CJK}.

\begin{enumerate}
  \setcounter{enumi}{6}
  \item (15 \begin{CJK}{UTF8}{mj}分\end{CJK}) \begin{CJK}{UTF8}{mj}一个复方阵\end{CJK} $T$ \begin{CJK}{UTF8}{mj}称为是幂零的\end{CJK}, \begin{CJK}{UTF8}{mj}说明存在正整数\end{CJK} $m$, \begin{CJK}{UTF8}{mj}使得\end{CJK} $T^{m}=0$, \begin{CJK}{UTF8}{mj}设\end{CJK} $T$ \begin{CJK}{UTF8}{mj}为\end{CJK} $n$ \begin{CJK}{UTF8}{mj}阶复方阵\end{CJK}, \begin{CJK}{UTF8}{mj}证明\end{CJK} $T$ \begin{CJK}{UTF8}{mj}为幂零\end{CJK} \begin{CJK}{UTF8}{mj}阵当且仅当\end{CJK} $T$ \begin{CJK}{UTF8}{mj}的特征多项式\end{CJK}
\end{enumerate}
$$
\chi_{T}(x)=x^{n} .
$$

\begin{enumerate}
  \setcounter{enumi}{7}
  \item (15 \begin{CJK}{UTF8}{mj}分\end{CJK}) \begin{CJK}{UTF8}{mj}设\end{CJK} $A, B$ \begin{CJK}{UTF8}{mj}为\end{CJK} $n$ \begin{CJK}{UTF8}{mj}阶半正定实对称方阵\end{CJK}, \begin{CJK}{UTF8}{mj}问\end{CJK} $A+B$ \begin{CJK}{UTF8}{mj}是否为半正定的\end{CJK}? (\begin{CJK}{UTF8}{mj}若肯定给出理由\end{CJK}, \begin{CJK}{UTF8}{mj}若否定给出反例\end{CJK})

  \item ( 15 \begin{CJK}{UTF8}{mj}分\end{CJK}) \begin{CJK}{UTF8}{mj}在\end{CJK} 3 \begin{CJK}{UTF8}{mj}维实向量空间中\end{CJK}, \begin{CJK}{UTF8}{mj}定义\end{CJK} $x=\left(x_{1}, x_{2}, x_{3}\right)^{T}$ \begin{CJK}{UTF8}{mj}与\end{CJK} $y=\left(y_{1}, y_{2}, y_{3}\right)$ \begin{CJK}{UTF8}{mj}的内积如下\end{CJK}:

\end{enumerate}
$$
(x, y)=\left(x_{1}, x_{2}, x_{3}\right)\left(\begin{array}{ccc}
1 & 1 & 1 \\
1 & 2 & 1 \\
1 & 1 & 4
\end{array}\right)\left(\begin{array}{l}
y_{1} \\
y_{2} \\
y_{3}
\end{array}\right) \text {. }
$$
\begin{CJK}{UTF8}{mj}这样定义了一个欧式空间\end{CJK}, \begin{CJK}{UTF8}{mj}求这个欧式空间中的包含\end{CJK} $e_{1}=\left(x_{1}, x_{2}, x_{3}\right)^{T}$ \begin{CJK}{UTF8}{mj}在内的一组标准正交基\end{CJK} $\left\{e_{1}, e_{2}, e_{3}\right\}$.

\begin{enumerate}
  \setcounter{enumi}{9}
  \item (15 \begin{CJK}{UTF8}{mj}分\end{CJK}) \begin{CJK}{UTF8}{mj}设\end{CJK} $V$ \begin{CJK}{UTF8}{mj}为有理数域上的线性空间\end{CJK}, $A$ \begin{CJK}{UTF8}{mj}为\end{CJK} $V$ \begin{CJK}{UTF8}{mj}的一个非零线性变换\end{CJK}, \begin{CJK}{UTF8}{mj}且\end{CJK}
\end{enumerate}
$$
A^{4}=4 A^{2}-2 A,
$$
\begin{CJK}{UTF8}{mj}证明\end{CJK}: $V=A V \oplus A^{-1}(0)$, \begin{CJK}{UTF8}{mj}且\end{CJK} $A$ \begin{CJK}{UTF8}{mj}有一个\end{CJK} 3 \begin{CJK}{UTF8}{mj}维不变子空间\end{CJK}.

\begin{enumerate}
  \setcounter{enumi}{10}
  \item ( 15 \begin{CJK}{UTF8}{mj}分\end{CJK}) \begin{CJK}{UTF8}{mj}设\end{CJK} $T$ \begin{CJK}{UTF8}{mj}为\end{CJK} $n$ \begin{CJK}{UTF8}{mj}阶实对称方阵\end{CJK}, \begin{CJK}{UTF8}{mj}同时也是正交阵\end{CJK}. \begin{CJK}{UTF8}{mj}证明\end{CJK}: \begin{CJK}{UTF8}{mj}存在整数\end{CJK} $m(0 \leq m \leq n)$ \begin{CJK}{UTF8}{mj}使得\end{CJK}
\end{enumerate}
$$
\operatorname{tr}(T)=n-2 m,
$$
\begin{CJK}{UTF8}{mj}此外若\end{CJK} $m=0$ \begin{CJK}{UTF8}{mj}则\end{CJK} $T=I$ (\begin{CJK}{UTF8}{mj}单位方阵\end{CJK}). \begin{CJK}{UTF8}{mj}而若\end{CJK} $m=n$ \begin{CJK}{UTF8}{mj}则\end{CJK} $T=-I$. (\begin{CJK}{UTF8}{mj}注\end{CJK}: $\operatorname{tr}(T)$ \begin{CJK}{UTF8}{mj}为矩阵\end{CJK} $T$ \begin{CJK}{UTF8}{mj}的迹\end{CJK}, \begin{CJK}{UTF8}{mj}即\end{CJK} $T$ \begin{CJK}{UTF8}{mj}的主对角线上元\end{CJK} \begin{CJK}{UTF8}{mj}素之和\end{CJK}).

\section{3. 首都师范大学 2013 年研究生入学考试试题高等代数 
 李扬 
 微信公众号: sxkyliyang}
\begin{enumerate}
  \item (15 \begin{CJK}{UTF8}{mj}分\end{CJK}) \begin{CJK}{UTF8}{mj}求出次数最低的首项系数为\end{CJK} 1 \begin{CJK}{UTF8}{mj}的实系数多项式\end{CJK} $f(x)$ \begin{CJK}{UTF8}{mj}使得\end{CJK} $f(0)=7, f(1)=14, f(2)=35, f(3)=76$.

  \item ( 15 \begin{CJK}{UTF8}{mj}分\end{CJK}) \begin{CJK}{UTF8}{mj}求下面行列式的值\end{CJK} (\begin{CJK}{UTF8}{mj}其中每个\end{CJK} $a_{i} \neq 1$ )

\end{enumerate}
$$
D_{n}=\left|\begin{array}{ccccc}
1+a_{1} & 2 & 2 & \cdots & 2 \\
2 & 1+a_{2} & 2 & \cdots & 2 \\
2 & 2 & 1+a_{3} & \cdots & 2 \\
\vdots & \vdots & \vdots & & \vdots \\
2 & 2 & 2 & \cdots & 1+a_{n}
\end{array}\right|
$$

\begin{enumerate}
  \setcounter{enumi}{3}
  \item (15 \begin{CJK}{UTF8}{mj}分\end{CJK}) \begin{CJK}{UTF8}{mj}证明可逆对称方阵的逆矩阵也是对称的\end{CJK}.

  \item (15 \begin{CJK}{UTF8}{mj}分\end{CJK}) \begin{CJK}{UTF8}{mj}设\end{CJK} $A, B$ \begin{CJK}{UTF8}{mj}都是\end{CJK} $n \times n$ \begin{CJK}{UTF8}{mj}矩阵\end{CJK}, $E$ \begin{CJK}{UTF8}{mj}是\end{CJK} $n \times n$ \begin{CJK}{UTF8}{mj}的单位矩阵\end{CJK}. \begin{CJK}{UTF8}{mj}证明\end{CJK}

\end{enumerate}
$$
A B-B A \neq E
$$

\begin{enumerate}
  \setcounter{enumi}{5}
  \item (15 \begin{CJK}{UTF8}{mj}分\end{CJK}) \begin{CJK}{UTF8}{mj}设\end{CJK} $\mathscr{A}$ \begin{CJK}{UTF8}{mj}是有限维线性空间\end{CJK} $V$ \begin{CJK}{UTF8}{mj}上的线性变换\end{CJK}, \begin{CJK}{UTF8}{mj}已知存在正整数\end{CJK} $k$ \begin{CJK}{UTF8}{mj}使得\end{CJK} $\operatorname{ker}\left(\mathscr{A}^{k}\right)=\operatorname{ker}\left(\mathscr{A}^{k+1}\right)$. \begin{CJK}{UTF8}{mj}证明\end{CJK}:
\end{enumerate}
$$
\operatorname{ker}\left(\mathscr{A}^{k}\right) \cap \operatorname{Im}\left(\mathscr{A}^{k}\right)=0,
$$
(\begin{CJK}{UTF8}{mj}其中\end{CJK} $\operatorname{ker}\left(\mathscr{A}^{k}\right), \operatorname{Im}\left(\mathscr{A}^{k}\right)$ \begin{CJK}{UTF8}{mj}分别为线性变换\end{CJK} $\mathscr{A}^{k}$ \begin{CJK}{UTF8}{mj}的核空间和象空间\end{CJK}).

\begin{enumerate}
  \setcounter{enumi}{6}
  \item ( 15 \begin{CJK}{UTF8}{mj}分\end{CJK}) \begin{CJK}{UTF8}{mj}设\end{CJK} $\mathscr{A}, \mathscr{B}$ \begin{CJK}{UTF8}{mj}为\end{CJK} $n$ \begin{CJK}{UTF8}{mj}维线性空间\end{CJK} $V$ \begin{CJK}{UTF8}{mj}上的线性变换\end{CJK}, \begin{CJK}{UTF8}{mj}使得\end{CJK} $(\mathscr{B} \mathscr{A})^{2}=\varepsilon$ (\begin{CJK}{UTF8}{mj}其中\end{CJK} $\varepsilon$ \begin{CJK}{UTF8}{mj}是\end{CJK} $V$ \begin{CJK}{UTF8}{mj}上的恒等变换\end{CJK}), \begin{CJK}{UTF8}{mj}令\end{CJK} $\mathscr{C}=\varepsilon-\mathscr{A} \mathscr{B}$, \begin{CJK}{UTF8}{mj}证明\end{CJK}:
\end{enumerate}
$$
\operatorname{ker}(\mathscr{C})=\operatorname{Im}(\mathscr{C}-2 \varepsilon)
$$

\begin{enumerate}
  \setcounter{enumi}{7}
  \item (15 \begin{CJK}{UTF8}{mj}分\end{CJK}) \begin{CJK}{UTF8}{mj}设\end{CJK} $A$ \begin{CJK}{UTF8}{mj}为\end{CJK} $n$ \begin{CJK}{UTF8}{mj}阶实方阵\end{CJK}, \begin{CJK}{UTF8}{mj}它的每行各数的和都等于\end{CJK} 2 , \begin{CJK}{UTF8}{mj}证明\end{CJK}:
\end{enumerate}
$$
\operatorname{det}(A-2 E)=0 .
$$
$(\operatorname{det}(A-2 E)=|A-2 E|$, \begin{CJK}{UTF8}{mj}即行列式\end{CJK}. $)$

\begin{enumerate}
  \setcounter{enumi}{8}
  \item (15 \begin{CJK}{UTF8}{mj}分\end{CJK}) \begin{CJK}{UTF8}{mj}设\end{CJK} $a, b, c, d$ \begin{CJK}{UTF8}{mj}为非零实数\end{CJK}. \begin{CJK}{UTF8}{mj}求矩阵\end{CJK}
\end{enumerate}
$$
A=\left(\begin{array}{cccc}
a & b & c & d \\
-b & a & -d & c \\
-c & d & a & -b \\
-d & -c & b & a
\end{array}\right)
$$
\begin{CJK}{UTF8}{mj}的逆矩阵\end{CJK}.

\begin{enumerate}
  \setcounter{enumi}{9}
  \item (15 \begin{CJK}{UTF8}{mj}分\end{CJK}) \begin{CJK}{UTF8}{mj}设\end{CJK} $A$ \begin{CJK}{UTF8}{mj}为\end{CJK} $m \times n$ \begin{CJK}{UTF8}{mj}实矩阵\end{CJK} $(m \neq n), E$ \begin{CJK}{UTF8}{mj}是\end{CJK} $n \times n$ \begin{CJK}{UTF8}{mj}单位矩阵\end{CJK}. \begin{CJK}{UTF8}{mj}证明\end{CJK}:
\end{enumerate}
$$
E+A^{T} A
$$
\begin{CJK}{UTF8}{mj}是正定对称阵\end{CJK}. ( $A^{T}$ \begin{CJK}{UTF8}{mj}为\end{CJK} $A$ \begin{CJK}{UTF8}{mj}的转置矩阵\end{CJK})

\begin{enumerate}
  \setcounter{enumi}{10}
  \item (15 \begin{CJK}{UTF8}{mj}分\end{CJK}) \begin{CJK}{UTF8}{mj}求方阵\end{CJK}
\end{enumerate}
$$
\left(\begin{array}{ccccc}
0 & 1 & 0 & 0 & 0 \\
0 & 0 & 1 & 0 & 0 \\
0 & 0 & 0 & 1 & 0 \\
0 & 0 & 0 & 0 & 1 \\
1 & -5 & 10 & -10 & 5
\end{array}\right)
$$
\begin{CJK}{UTF8}{mj}的若尔当标准型\end{CJK}.

\section{4. 首都师范大学 2014 年研究生入学考试试题高等代数 
 李扬 
 微信公众号: sxkyliyang}
\begin{enumerate}
  \item (15 \begin{CJK}{UTF8}{mj}分\end{CJK}) \begin{CJK}{UTF8}{mj}求一个\end{CJK} 3 \begin{CJK}{UTF8}{mj}次多项式\end{CJK} $f(x)$, \begin{CJK}{UTF8}{mj}使得\end{CJK} $f(x)$ \begin{CJK}{UTF8}{mj}除以\end{CJK} $x^{2}+1$ \begin{CJK}{UTF8}{mj}的余式是\end{CJK} $3 x+4$, \begin{CJK}{UTF8}{mj}除以\end{CJK} $x^{2}+x+1$ \begin{CJK}{UTF8}{mj}的余式是\end{CJK} $3 x+5$.

  \item (15 \begin{CJK}{UTF8}{mj}分\end{CJK}) \begin{CJK}{UTF8}{mj}求行列式\end{CJK}

\end{enumerate}
$$
D_{n}=\left|\begin{array}{cccccc}
1-a_{1} & a_{2} & 0 & \cdots & 0 & 0 \\
-1 & 1-a_{2} & a_{3} & \cdots & 0 & 0 \\
0 & -1 & 1-a_{3} & \cdots & 0 & 0 \\
\vdots & \vdots & \vdots & & \vdots & \vdots \\
0 & 0 & 0 & \cdots & 1-a_{n-1} & a_{n} \\
0 & 0 & 0 & \cdots & -1 & 1-a_{n}
\end{array}\right|
$$

\begin{enumerate}
  \setcounter{enumi}{3}
  \item ( 15 \begin{CJK}{UTF8}{mj}分\end{CJK}) \begin{CJK}{UTF8}{mj}求下列\end{CJK} 100 \begin{CJK}{UTF8}{mj}个变元的方程组\end{CJK}
\end{enumerate}
$$
\left\{\begin{array}{l}
x_{1}-x_{2}+x_{3}=0 \\
x_{2}-x_{3}+x_{4}=0 \\
x_{3}-x_{4}+x_{5}=0 \\
\vdots \\
x_{98}-x_{99}+x_{100}=0 \\
x_{99}-x_{100}+1=0
\end{array}\right.
$$
\begin{CJK}{UTF8}{mj}的通解\end{CJK}.

\begin{enumerate}
  \setcounter{enumi}{4}
  \item ( 15 \begin{CJK}{UTF8}{mj}分\end{CJK}) \begin{CJK}{UTF8}{mj}记\end{CJK} $M$ \begin{CJK}{UTF8}{mj}为\end{CJK} 2 \begin{CJK}{UTF8}{mj}阶实方阵组成的线性空间\end{CJK}, $B=\left(\begin{array}{cc}b_{1} & b_{2} \\ b_{3} & b_{4}\end{array}\right) \in M$. \begin{CJK}{UTF8}{mj}定义映射\end{CJK} $f: M \rightarrow M$ \begin{CJK}{UTF8}{mj}为\end{CJK} $f(A)=$ $A B-B A(\forall A \in M)$, \begin{CJK}{UTF8}{mj}验证\end{CJK} $f$ \begin{CJK}{UTF8}{mj}是线性映射\end{CJK}, \begin{CJK}{UTF8}{mj}并写出\end{CJK} $f$ \begin{CJK}{UTF8}{mj}的基\end{CJK}
\end{enumerate}
$$
\left\{\left(\begin{array}{ll}
1 & 0 \\
0 & 0
\end{array}\right),\left(\begin{array}{ll}
0 & 1 \\
0 & 0
\end{array}\right),\left(\begin{array}{ll}
0 & 0 \\
1 & 0
\end{array}\right),\left(\begin{array}{ll}
0 & 0 \\
0 & 1
\end{array}\right)\right\} .
$$
\begin{CJK}{UTF8}{mj}下的矩阵\end{CJK}.

\begin{enumerate}
  \setcounter{enumi}{5}
  \item (15 \begin{CJK}{UTF8}{mj}分\end{CJK}) \begin{CJK}{UTF8}{mj}设\end{CJK}
\end{enumerate}
$$
f\left(x_{1}, x_{2}, x_{3}\right)=\left(x_{1}, x_{2}, x_{3}\right) A\left(\begin{array}{c}
x_{1} \\
x_{2} \\
x_{3}
\end{array}\right) .
$$
\begin{CJK}{UTF8}{mj}为三元实二次型\end{CJK} ( $A$ \begin{CJK}{UTF8}{mj}为三元实对称方阵\end{CJK}), $B$ \begin{CJK}{UTF8}{mj}为\end{CJK} 3 \begin{CJK}{UTF8}{mj}维欧几里得空间\end{CJK} $V$ \begin{CJK}{UTF8}{mj}中的单位球面\end{CJK}
$$
\left\{\left(a_{1}, a_{2}, a_{3}\right)^{T} \mid a_{1}^{2}+a_{2}^{2}+a_{3}^{2}=1\right\} .
$$
\begin{CJK}{UTF8}{mj}设\end{CJK} $f$ \begin{CJK}{UTF8}{mj}在\end{CJK} $B$ \begin{CJK}{UTF8}{mj}上的最大值为\end{CJK} $\lambda$, \begin{CJK}{UTF8}{mj}而最大值在点\end{CJK} $v \in B$ \begin{CJK}{UTF8}{mj}处达到\end{CJK}. \begin{CJK}{UTF8}{mj}证明\end{CJK}: \begin{CJK}{UTF8}{mj}若向量\end{CJK} $w$ \begin{CJK}{UTF8}{mj}与\end{CJK} $v$ \begin{CJK}{UTF8}{mj}正交\end{CJK}, \begin{CJK}{UTF8}{mj}则\end{CJK} $A w$ \begin{CJK}{UTF8}{mj}与\end{CJK} $v$ \begin{CJK}{UTF8}{mj}正交\end{CJK}.

\begin{enumerate}
  \setcounter{enumi}{6}
  \item (15 \begin{CJK}{UTF8}{mj}分\end{CJK}) \begin{CJK}{UTF8}{mj}设\end{CJK} $A$ \begin{CJK}{UTF8}{mj}为半正定对称方阵\end{CJK}, \begin{CJK}{UTF8}{mj}证明\end{CJK}:
\end{enumerate}
$$
\operatorname{tr}(A) \geq 0
$$
\begin{CJK}{UTF8}{mj}且等号仅当\end{CJK} $A=0$ \begin{CJK}{UTF8}{mj}时成立\end{CJK}. 7. (15 \begin{CJK}{UTF8}{mj}分\end{CJK}) \begin{CJK}{UTF8}{mj}设\end{CJK} $\mathscr{A}, \mathscr{B}$ \begin{CJK}{UTF8}{mj}为\end{CJK} $n$ \begin{CJK}{UTF8}{mj}维实向量空间\end{CJK} $V$ \begin{CJK}{UTF8}{mj}到自身的线性映射\end{CJK}, \begin{CJK}{UTF8}{mj}且\end{CJK} $\mathscr{A}^{2}=\mathscr{A}$, \begin{CJK}{UTF8}{mj}记\end{CJK} $V_{1}=\operatorname{ker}(\mathscr{A})(\mathscr{A}$ \begin{CJK}{UTF8}{mj}的核\end{CJK}), $V_{2}=\operatorname{Im}(\mathscr{A})(\mathscr{A}$ \begin{CJK}{UTF8}{mj}的像\end{CJK}), \begin{CJK}{UTF8}{mj}证明\end{CJK}:
$$
\mathscr{A} \mathscr{B}=\mathscr{B} \mathscr{A}
$$
\begin{CJK}{UTF8}{mj}当且仅当\end{CJK} $\mathscr{B}\left(V_{1}\right) \subset V_{1}$, \begin{CJK}{UTF8}{mj}且\end{CJK} $\mathscr{B}\left(V_{2}\right) \subset V_{2}$.

\begin{enumerate}
  \setcounter{enumi}{8}
  \item (15 \begin{CJK}{UTF8}{mj}分\end{CJK}) \begin{CJK}{UTF8}{mj}设\end{CJK} $A$ \begin{CJK}{UTF8}{mj}为\end{CJK} $7 \times 8$ \begin{CJK}{UTF8}{mj}实矩阵\end{CJK}, \begin{CJK}{UTF8}{mj}记\end{CJK} $A^{T}$ \begin{CJK}{UTF8}{mj}为\end{CJK} $A$ \begin{CJK}{UTF8}{mj}的转置矩阵\end{CJK}, \begin{CJK}{UTF8}{mj}证明\end{CJK}: $A^{T} A$ \begin{CJK}{UTF8}{mj}是半正定对称阵\end{CJK}, \begin{CJK}{UTF8}{mj}但不是正定的\end{CJK}.

  \item ( 15 \begin{CJK}{UTF8}{mj}分\end{CJK}) \begin{CJK}{UTF8}{mj}设\end{CJK} $A$ \begin{CJK}{UTF8}{mj}为\end{CJK} $n$ \begin{CJK}{UTF8}{mj}阶实方阵\end{CJK}, $f(x)$ \begin{CJK}{UTF8}{mj}为实系数多项式\end{CJK}, \begin{CJK}{UTF8}{mj}证明\end{CJK} $f(A)$ \begin{CJK}{UTF8}{mj}为可逆阵当且仅当\end{CJK}

\end{enumerate}
$$
\operatorname{ged}\left(f(x), \chi_{A}(x)\right)=1
$$
(\begin{CJK}{UTF8}{mj}其中\end{CJK} $\chi_{A}(x)$ \begin{CJK}{UTF8}{mj}为\end{CJK} $A$ \begin{CJK}{UTF8}{mj}的特征多项式\end{CJK}.)

\begin{enumerate}
  \setcounter{enumi}{10}
  \item (15 \begin{CJK}{UTF8}{mj}分\end{CJK}) \begin{CJK}{UTF8}{mj}设\end{CJK} 3 \begin{CJK}{UTF8}{mj}阶正定对称实方阵\end{CJK} $A$ \begin{CJK}{UTF8}{mj}的特征值为\end{CJK} $0,3,3$, \begin{CJK}{UTF8}{mj}而\end{CJK} $(1,1,1)^{T}$ \begin{CJK}{UTF8}{mj}是特征值\end{CJK} 6 \begin{CJK}{UTF8}{mj}的特征向量\end{CJK}, \begin{CJK}{UTF8}{mj}求\end{CJK} $A$.
\end{enumerate}
\section{5. 首都师范大学 2015 年研究生入学考试试题高等代数 
 李扬 
 微信公众号: sxkyliyang}
\begin{enumerate}
  \item (15 \begin{CJK}{UTF8}{mj}分\end{CJK}) \begin{CJK}{UTF8}{mj}设\end{CJK} $f(x), g(x)$ \begin{CJK}{UTF8}{mj}为两个互素的多项式\end{CJK}, \begin{CJK}{UTF8}{mj}次数分别为\end{CJK} $m, n>0$. \begin{CJK}{UTF8}{mj}证明\end{CJK}: \begin{CJK}{UTF8}{mj}存在唯一的次数小于\end{CJK} $n$ \begin{CJK}{UTF8}{mj}的多项式\end{CJK} $u(x)$ \begin{CJK}{UTF8}{mj}及唯一的次数小于\end{CJK} $m$ \begin{CJK}{UTF8}{mj}的多项式\end{CJK} $v(x)$, \begin{CJK}{UTF8}{mj}使得\end{CJK}
\end{enumerate}
$$
u(x) f(x)+v(x) g(x)=1 .
$$

\begin{enumerate}
  \setcounter{enumi}{2}
  \item (15 \begin{CJK}{UTF8}{mj}分\end{CJK}) \begin{CJK}{UTF8}{mj}求行列式\end{CJK}
\end{enumerate}
$$
\left|\begin{array}{cccc}
1 & 1 & 1 & 1 \\
a & b & c & d \\
a^{2} & b^{2} & c^{2} & d^{2} \\
a^{4} & b^{4} & c^{4} & d^{4}
\end{array}\right|
$$

\begin{enumerate}
  \setcounter{enumi}{3}
  \item ( 15 \begin{CJK}{UTF8}{mj}分\end{CJK}) \begin{CJK}{UTF8}{mj}记\end{CJK} $M_{5}(\mathbb{R})$ \begin{CJK}{UTF8}{mj}为所有\end{CJK} 5 \begin{CJK}{UTF8}{mj}阶实方阵组成的线性空间\end{CJK}. \begin{CJK}{UTF8}{mj}令\end{CJK}
\end{enumerate}
$$
T=\left(\begin{array}{lllll}
0 & 1 & 0 & 0 & 0 \\
0 & 0 & 1 & 0 & 0 \\
0 & 0 & 0 & 1 & 0 \\
0 & 0 & 0 & 0 & 1 \\
0 & 0 & 0 & 0 & 0
\end{array}\right) .
$$
\begin{CJK}{UTF8}{mj}令\end{CJK} $S=\left\{A \in M_{5}(\mathbb{R}) \mid A T=T A\right\}$. \begin{CJK}{UTF8}{mj}证明\end{CJK}: $S$ \begin{CJK}{UTF8}{mj}是\end{CJK} $M_{5}(\mathbb{R})$ \begin{CJK}{UTF8}{mj}的线性子空间\end{CJK}, \begin{CJK}{UTF8}{mj}并计算它的维数\end{CJK}.

\begin{enumerate}
  \setcounter{enumi}{4}
  \item (15 \begin{CJK}{UTF8}{mj}分\end{CJK}) \begin{CJK}{UTF8}{mj}设\end{CJK} $a_{1}, \cdots, a_{n}$ \begin{CJK}{UTF8}{mj}为\end{CJK} $n$ \begin{CJK}{UTF8}{mj}个不同的数\end{CJK}, \begin{CJK}{UTF8}{mj}证明对任意\end{CJK} $n$ \begin{CJK}{UTF8}{mj}个数\end{CJK} $b_{1}, \cdots, b_{n}$, \begin{CJK}{UTF8}{mj}存在唯一次数小于\end{CJK} $n-1$ \begin{CJK}{UTF8}{mj}的多项式\end{CJK} $f(x)=c_{1}+c_{2} x+\cdots+c_{n} x^{n-1}$, \begin{CJK}{UTF8}{mj}使得\end{CJK}
\end{enumerate}
$$
f\left(a_{i}\right)=b_{i}(1 \leq i \leq n)
$$

\begin{enumerate}
  \setcounter{enumi}{5}
  \item ( 15 \begin{CJK}{UTF8}{mj}分\end{CJK}) \begin{CJK}{UTF8}{mj}设\end{CJK} $v_{1}, \cdots, v_{m}$ \begin{CJK}{UTF8}{mj}为线性相关的\end{CJK} $n$ \begin{CJK}{UTF8}{mj}维实向量\end{CJK}, \begin{CJK}{UTF8}{mj}其中每个分量都是有理数\end{CJK}, \begin{CJK}{UTF8}{mj}证明\end{CJK}: \begin{CJK}{UTF8}{mj}存在不全为零的有理数\end{CJK} $r_{1}, \cdots, r_{m}$ \begin{CJK}{UTF8}{mj}使得\end{CJK}
\end{enumerate}
$$
r_{1} v_{1}+\cdots+r_{m} v_{m}=0 .
$$

\begin{enumerate}
  \setcounter{enumi}{6}
  \item (15 \begin{CJK}{UTF8}{mj}分\end{CJK}) \begin{CJK}{UTF8}{mj}设\end{CJK} $A, B, C, D$ \begin{CJK}{UTF8}{mj}为\end{CJK} $n$ \begin{CJK}{UTF8}{mj}阶方阵\end{CJK}, \begin{CJK}{UTF8}{mj}其中\end{CJK} $D$ \begin{CJK}{UTF8}{mj}为可逆阵\end{CJK}, \begin{CJK}{UTF8}{mj}且\end{CJK} $C D=D C$. \begin{CJK}{UTF8}{mj}证明\end{CJK}:
\end{enumerate}
$$
\operatorname{det}\left(\begin{array}{cc}
A & B \\
C & D
\end{array}\right)=\operatorname{det}(A D-B C) .
$$

\begin{enumerate}
  \setcounter{enumi}{7}
  \item (15 \begin{CJK}{UTF8}{mj}分\end{CJK}) \begin{CJK}{UTF8}{mj}设\end{CJK} $n$ \begin{CJK}{UTF8}{mj}阶复方阵\end{CJK} $A$ \begin{CJK}{UTF8}{mj}的所有特征值为\end{CJK} $\lambda_{1}, \lambda_{2}, \cdots, \lambda_{r}, m$ \begin{CJK}{UTF8}{mj}为正整数\end{CJK}. \begin{CJK}{UTF8}{mj}证明\end{CJK} $A^{m}$ \begin{CJK}{UTF8}{mj}的所有特征值为\end{CJK}
\end{enumerate}
$$
\lambda_{1}^{m}, \lambda_{2}^{m}, \cdots, \lambda_{r}^{m}
$$

\begin{enumerate}
  \setcounter{enumi}{8}
  \item ( 15 \begin{CJK}{UTF8}{mj}分\end{CJK}) \begin{CJK}{UTF8}{mj}设\end{CJK} $A, B$ \begin{CJK}{UTF8}{mj}为\end{CJK} $n$ \begin{CJK}{UTF8}{mj}阶幂零方阵\end{CJK}. \begin{CJK}{UTF8}{mj}问\end{CJK} $A B$ \begin{CJK}{UTF8}{mj}是否为幂零阵\end{CJK}. (\begin{CJK}{UTF8}{mj}若肯定则给出证明\end{CJK}, \begin{CJK}{UTF8}{mj}若否定则给出反例\end{CJK}, \begin{CJK}{UTF8}{mj}仅回\end{CJK} \begin{CJK}{UTF8}{mj}答\end{CJK}" \begin{CJK}{UTF8}{mj}是\end{CJK}" \begin{CJK}{UTF8}{mj}或\end{CJK}"\begin{CJK}{UTF8}{mj}否\end{CJK}"\begin{CJK}{UTF8}{mj}无分\end{CJK}.)

  \item ( 15 \begin{CJK}{UTF8}{mj}分\end{CJK}) \begin{CJK}{UTF8}{mj}三维欧式空间\end{CJK} $V=\mathbb{R}^{3}$ \begin{CJK}{UTF8}{mj}到自身的一个映射\end{CJK} $\phi: V \rightarrow V$ \begin{CJK}{UTF8}{mj}称为运动\end{CJK}, \begin{CJK}{UTF8}{mj}如果它是一个正交变换与一个平移\end{CJK} \begin{CJK}{UTF8}{mj}的合成\end{CJK}, \begin{CJK}{UTF8}{mj}即存在一个正交变换\end{CJK} $\mathscr{A}: V \rightarrow V$ \begin{CJK}{UTF8}{mj}及一个向量\end{CJK} $v_{0} \in V$ \begin{CJK}{UTF8}{mj}使得对任意\end{CJK} $v \in V$ \begin{CJK}{UTF8}{mj}有\end{CJK}

\end{enumerate}
$$
\phi(v)=\mathscr{A} v+v_{0} .
$$
\begin{CJK}{UTF8}{mj}给出一个运动\end{CJK} $\phi$, \begin{CJK}{UTF8}{mj}使得\end{CJK} $\phi(0) \neq 0$ \begin{CJK}{UTF8}{mj}且\end{CJK} $\phi^{5}=\mathrm{id}_{v}$ (\begin{CJK}{UTF8}{mj}这里\end{CJK} $0 \in V$ \begin{CJK}{UTF8}{mj}为零向量\end{CJK}, $\phi^{5}$ \begin{CJK}{UTF8}{mj}为\end{CJK} 5 \begin{CJK}{UTF8}{mj}个\end{CJK} $\phi$ \begin{CJK}{UTF8}{mj}的合成\end{CJK}, $\operatorname{id}_{v}$ \begin{CJK}{UTF8}{mj}为\end{CJK} $V$ \begin{CJK}{UTF8}{mj}到自身的单\end{CJK} \begin{CJK}{UTF8}{mj}位映射\end{CJK}.)

\begin{enumerate}
  \setcounter{enumi}{10}
  \item (15 \begin{CJK}{UTF8}{mj}分\end{CJK}) \begin{CJK}{UTF8}{mj}设\end{CJK} $T$ \begin{CJK}{UTF8}{mj}为\end{CJK} 2 \begin{CJK}{UTF8}{mj}阶实正交阵\end{CJK}, \begin{CJK}{UTF8}{mj}且不是单位阵\end{CJK}. \begin{CJK}{UTF8}{mj}证明\end{CJK}: $\operatorname{det}(T)=-1$ \begin{CJK}{UTF8}{mj}当且仅当存在\end{CJK} 2 \begin{CJK}{UTF8}{mj}元实向量\end{CJK} $v \neq 0$ \begin{CJK}{UTF8}{mj}使得\end{CJK}
\end{enumerate}
$$
T v=v .
$$

\section{6. 首都师范大学 2016 年研究生入学考试试题高等代数 
 李扬 
 微信公众号: sxkyliyang}
\begin{enumerate}
  \item \begin{CJK}{UTF8}{mj}求行列式\end{CJK}
\end{enumerate}
$$
\left|\begin{array}{cccc}
a^{2} & a b & a b & b^{2} \\
a c & a d & b c & b d \\
a c & b c & a d & b d \\
c^{2} & c d & c d & d^{2}
\end{array}\right|
$$

\begin{enumerate}
  \setcounter{enumi}{2}
  \item \begin{CJK}{UTF8}{mj}设矩阵\end{CJK}
\end{enumerate}
$$
T=\left(\begin{array}{llll}
a & 0 & 0 & 0 \\
1 & a & 0 & 0 \\
0 & 1 & a & 0 \\
0 & 0 & 1 & a
\end{array}\right)
$$
\begin{CJK}{UTF8}{mj}而多项式\end{CJK}
$$
f(x)=a_{0}+a_{1} x+a_{2} x^{2} .
$$
\begin{CJK}{UTF8}{mj}求\end{CJK} $f(T)$.

\begin{enumerate}
  \setcounter{enumi}{3}
  \item \begin{CJK}{UTF8}{mj}设\end{CJK} $v_{1}, v_{2}, v_{3}, v_{4}, v_{5}$ \begin{CJK}{UTF8}{mj}为任意\end{CJK} 5 \begin{CJK}{UTF8}{mj}个\end{CJK} 3 \begin{CJK}{UTF8}{mj}维向量\end{CJK}. \begin{CJK}{UTF8}{mj}令\end{CJK}
\end{enumerate}
$$
W=\left\{\left(a_{1}, a_{2}, a_{3}, a_{4}, a_{5}\right) \in \mathbb{R}^{5} \mid a_{1} v_{1}+a_{2} v_{2}+a_{3} v_{3}+a_{4} v_{4}+a_{5} v_{5}=0\right\}
$$
\begin{CJK}{UTF8}{mj}证明\end{CJK} $W$ \begin{CJK}{UTF8}{mj}是\end{CJK} $\mathbb{R}^{5}$ \begin{CJK}{UTF8}{mj}的线性子空间\end{CJK}, \begin{CJK}{UTF8}{mj}并给出\end{CJK} $W$ \begin{CJK}{UTF8}{mj}的所有可能的维数\end{CJK}.

\begin{enumerate}
  \setcounter{enumi}{4}
  \item \begin{CJK}{UTF8}{mj}对于一个\end{CJK} $n \times n$ \begin{CJK}{UTF8}{mj}多项式矩阵\end{CJK}
\end{enumerate}
$$
A(t)=\left(\begin{array}{ccc}
a_{11}(t) & \cdots & a_{1 n}(t) \\
\vdots & & \vdots \\
a_{n 1}(t) & \cdots & a_{n n}(t)
\end{array}\right),
$$
\begin{CJK}{UTF8}{mj}定义其导数为\end{CJK}
$$
\frac{\mathrm{d}}{\mathrm{d} t} A(t)=\left(\begin{array}{ccc}
\frac{\mathrm{d}}{\mathrm{d} t} a_{11}(t) & \cdots & \frac{\mathrm{d}}{\mathrm{d} t} a_{1 n}(t) \\
\vdots & & \vdots \\
\frac{\mathrm{d}}{\mathrm{d} t} a_{n 1}(t) & \cdots & \frac{\mathrm{d}}{\mathrm{d} t} a_{n n}(t)
\end{array}\right) .
$$
\begin{CJK}{UTF8}{mj}证明对于任意两个\end{CJK} $n \times n$ \begin{CJK}{UTF8}{mj}多项式矩阵\end{CJK} $A(t), B(t)$, \begin{CJK}{UTF8}{mj}有乘积导数公式\end{CJK}
$$
\frac{\mathrm{d}}{\mathrm{d} t}(A(t) B(t))=\frac{\mathrm{d}}{\mathrm{d} t}(A(t)) B(t)+A(t) \frac{\mathrm{d}}{\mathrm{d} t} B(t) .
$$

\begin{enumerate}
  \setcounter{enumi}{5}
  \item \begin{CJK}{UTF8}{mj}设\end{CJK} $m \times n$ \begin{CJK}{UTF8}{mj}矩阵\end{CJK} $A$ \begin{CJK}{UTF8}{mj}的秩为\end{CJK} $r$, \begin{CJK}{UTF8}{mj}取其中\end{CJK} $s$ \begin{CJK}{UTF8}{mj}行组成一个\end{CJK} $s \times n$ \begin{CJK}{UTF8}{mj}矩阵\end{CJK} $B, B$ \begin{CJK}{UTF8}{mj}的秩至少为多少\end{CJK}, \begin{CJK}{UTF8}{mj}证之\end{CJK}.

  \item \begin{CJK}{UTF8}{mj}求齐次线性方程组\end{CJK}

\end{enumerate}
$$
\left\{\begin{array}{l}
2 x_{1}+x_{2}-x_{3}+x_{4}-3 x_{5}=0 \\
x_{1}+x_{2}-x_{3}-x_{5}=0 .
\end{array}\right.
$$
\begin{CJK}{UTF8}{mj}的解空间\end{CJK} (\begin{CJK}{UTF8}{mj}作为\end{CJK} $\mathbb{R}^{5}$ \begin{CJK}{UTF8}{mj}的子空间\end{CJK}) \begin{CJK}{UTF8}{mj}的一组标准正交基\end{CJK}.

\begin{enumerate}
  \setcounter{enumi}{7}
  \item \begin{CJK}{UTF8}{mj}矩梿\end{CJK}
\end{enumerate}
$$
\left(\begin{array}{lll}
1 & 1 & 0 \\
1 & 0 & 1 \\
0 & 1 & 1
\end{array}\right)
$$
\begin{CJK}{UTF8}{mj}和矩阵\end{CJK}
$$
\left(\begin{array}{lll}
0 & 0 & 1 \\
0 & 2 & 0 \\
1 & 0 & 0
\end{array}\right)
$$
\includegraphics[max width=\textwidth]{2022_04_18_a5c47c0ff534501b502eg-197}

\begin{enumerate}
  \setcounter{enumi}{8}
  \item \begin{CJK}{UTF8}{mj}设\end{CJK} $A$ \begin{CJK}{UTF8}{mj}为\end{CJK} $n$ \begin{CJK}{UTF8}{mj}阶复方阵\end{CJK}, $A^{2}+A+I=0$ ( $I$ \begin{CJK}{UTF8}{mj}为\end{CJK} $n$ \begin{CJK}{UTF8}{mj}阶单位方阵\end{CJK}), \begin{CJK}{UTF8}{mj}证明\end{CJK}
\end{enumerate}
$$
\operatorname{det}(A) \neq 0
$$

\begin{enumerate}
  \setcounter{enumi}{9}
  \item \begin{CJK}{UTF8}{mj}设\end{CJK} $\mathscr{A}$ \begin{CJK}{UTF8}{mj}为\end{CJK} $n$ \begin{CJK}{UTF8}{mj}维向量空间\end{CJK} $V$ \begin{CJK}{UTF8}{mj}到自身的线性变换\end{CJK}, \begin{CJK}{UTF8}{mj}满足\end{CJK} $\mathscr{A}^{3}=\mathscr{A}$. \begin{CJK}{UTF8}{mj}证明任意向量\end{CJK} $v \in V$ \begin{CJK}{UTF8}{mj}可以唯一地分解为\end{CJK}
\end{enumerate}
$$
v=v_{1}+v_{0}+v_{-1}
$$
\begin{CJK}{UTF8}{mj}其中\end{CJK} $v_{1}, v_{0}, v_{-1}$ \begin{CJK}{UTF8}{mj}分别满足\end{CJK} $\mathscr{A} v_{1}=v_{1}, \mathscr{A} v_{0}=0, \mathscr{A} v_{-1}=-v_{-1}$.

\begin{enumerate}
  \setcounter{enumi}{10}
  \item \begin{CJK}{UTF8}{mj}设\end{CJK} $\alpha, \beta, \gamma$ \begin{CJK}{UTF8}{mj}为三维欧式空间\end{CJK} $V$ \begin{CJK}{UTF8}{mj}的一组标准正交基\end{CJK}, \begin{CJK}{UTF8}{mj}求\end{CJK} $V$ \begin{CJK}{UTF8}{mj}的一个正交变换\end{CJK} $\mathscr{A}$, \begin{CJK}{UTF8}{mj}使得\end{CJK}
\end{enumerate}
$$
\left\{\begin{array}{l}
\mathscr{A}(\alpha)=\frac{2}{3} \alpha+\frac{2}{3} \beta-\frac{1}{3} \gamma \\
\mathscr{A}(\beta)=\frac{2}{3} \alpha-\frac{1}{3} \beta+\frac{2}{3} \gamma
\end{array}\right.
$$

\section{7. 首都师范大学 2017 年研究生入学考试试题高等代数 
 李扬 
 微信公众号: sxkyliyang}
\begin{enumerate}
  \item \begin{CJK}{UTF8}{mj}求有理系数一次多项式\end{CJK} $f(x), g(x)$, \begin{CJK}{UTF8}{mj}使得\end{CJK}
\end{enumerate}
$$
x+1=\left(x^{2}+x+1\right) f(x)+\left(x^{2}-x+1\right) g(x) .
$$

\begin{enumerate}
  \setcounter{enumi}{2}
  \item \begin{CJK}{UTF8}{mj}求行列式\end{CJK}
\end{enumerate}
$$
\left|\begin{array}{cccc}
\frac{1}{1} & \frac{1}{3} & \frac{1}{5} & \frac{1}{7} \\
\frac{1}{2} & \frac{1}{4} & \frac{1}{6} & \frac{1}{8} \\
\frac{1}{3} & \frac{1}{5} & \frac{1}{7} & \frac{1}{9} \\
\frac{1}{4} & \frac{1}{6} & \frac{1}{8} & \frac{1}{10}
\end{array}\right|
$$

\begin{enumerate}
  \setcounter{enumi}{3}
  \item \begin{CJK}{UTF8}{mj}讨论下面线性方程组有解的条件\end{CJK}, \begin{CJK}{UTF8}{mj}并写出一般解\end{CJK}.
\end{enumerate}
$$
\left\{\begin{array}{l}
x_{1}+x_{2}+x_{3}=a_{1} \\
x_{2}+x_{3}+x_{4}=a_{2} \\
x_{3}+x_{4}+x_{5}=a_{3} \\
x_{4}+x_{5}+x_{6}=a_{4} \\
x_{5}+x_{6}+x_{7}=a_{5} \\
x_{6}+x_{7}+x_{8}=a_{6} \\
x_{7}+x_{8}+x_{9}=a_{7} \\
x_{8}+x_{9}+x_{10}=a_{8} \\
x_{9}+x_{10}+x_{2}=a_{9}
\end{array}\right.
$$

\begin{enumerate}
  \setcounter{enumi}{4}
  \item \begin{CJK}{UTF8}{mj}已知\end{CJK}
\end{enumerate}
$$
A=\left(\begin{array}{ccc}
4 & 6 & 0 \\
-3 & -5 & 0 \\
-3 & -6 & 0
\end{array}\right)
$$
\begin{CJK}{UTF8}{mj}求\end{CJK} $A^{100}$.

\begin{enumerate}
  \setcounter{enumi}{5}
  \item \begin{CJK}{UTF8}{mj}设\end{CJK} 4 \begin{CJK}{UTF8}{mj}维欧式空间\end{CJK} $\mathbb{R}^{4}$ \begin{CJK}{UTF8}{mj}中\end{CJK}
\end{enumerate}
$$
V=\{(a, b,-a,-b) \mid a, b \in \mathbb{R}\} .
$$
\begin{CJK}{UTF8}{mj}证明\end{CJK} $V$ \begin{CJK}{UTF8}{mj}是一个线性子空间\end{CJK}, \begin{CJK}{UTF8}{mj}求出\end{CJK} $V$ \begin{CJK}{UTF8}{mj}的一组标准正交基\end{CJK}, \begin{CJK}{UTF8}{mj}并求\end{CJK} $V$ \begin{CJK}{UTF8}{mj}在\end{CJK} $\mathbb{R}^{4}$ \begin{CJK}{UTF8}{mj}中的正交补空间\end{CJK}.

\begin{enumerate}
  \setcounter{enumi}{6}
  \item \begin{CJK}{UTF8}{mj}设\end{CJK} $n$ \begin{CJK}{UTF8}{mj}阶实方阵\end{CJK} $P$ \begin{CJK}{UTF8}{mj}既是正定对称阵又是正交阵\end{CJK}, \begin{CJK}{UTF8}{mj}证明\end{CJK} $P$ \begin{CJK}{UTF8}{mj}是单位阵\end{CJK}.

  \item \begin{CJK}{UTF8}{mj}设\end{CJK} $n$ \begin{CJK}{UTF8}{mj}阶非零方阵\end{CJK} $A$ \begin{CJK}{UTF8}{mj}满足\end{CJK} $A^{n}=0$, \begin{CJK}{UTF8}{mj}证明\end{CJK} $A$ \begin{CJK}{UTF8}{mj}不可对角化\end{CJK}.

  \item \begin{CJK}{UTF8}{mj}设线性方程组\end{CJK} $A X=b(b \neq 0)$ \begin{CJK}{UTF8}{mj}的解向量中有一组极大线性无关向量\end{CJK} $\alpha_{1}, \alpha_{2}, \cdots, \alpha_{s}$. \begin{CJK}{UTF8}{mj}令\end{CJK} $V$ \begin{CJK}{UTF8}{mj}为\end{CJK} $\alpha_{1}, \alpha_{2}, \cdots, \alpha_{s}$ \begin{CJK}{UTF8}{mj}生成的向量空间\end{CJK}, \begin{CJK}{UTF8}{mj}给出\end{CJK} $V$ \begin{CJK}{UTF8}{mj}中向量是\end{CJK} $A X=b$ \begin{CJK}{UTF8}{mj}的解的充要条件\end{CJK}.

  \item \begin{CJK}{UTF8}{mj}设\end{CJK} $A, B$ \begin{CJK}{UTF8}{mj}是\end{CJK} $n$ \begin{CJK}{UTF8}{mj}阶实方阵\end{CJK}, $A$ \begin{CJK}{UTF8}{mj}为对称的\end{CJK}, \begin{CJK}{UTF8}{mj}且\end{CJK} $A B+B^{T} A$ \begin{CJK}{UTF8}{mj}正定\end{CJK}, \begin{CJK}{UTF8}{mj}证明\end{CJK} $A$ \begin{CJK}{UTF8}{mj}可逆\end{CJK}.

  \item \begin{CJK}{UTF8}{mj}设\end{CJK} $n$ \begin{CJK}{UTF8}{mj}阶复方阵\end{CJK} $A$ \begin{CJK}{UTF8}{mj}有\end{CJK} $n$ \begin{CJK}{UTF8}{mj}个不同特征值\end{CJK}, $n$ \begin{CJK}{UTF8}{mj}阶复方阵\end{CJK} $B$ \begin{CJK}{UTF8}{mj}满足\end{CJK} $A B=B A$, \begin{CJK}{UTF8}{mj}证明存在多项式\end{CJK} $f(x)$, \begin{CJK}{UTF8}{mj}使得\end{CJK}

\end{enumerate}
$$
B=f(A)
$$

\section{1. 四川大学 2009 年研究生入学考试试题高等代数}
\begin{CJK}{UTF8}{mj}李扬\end{CJK}

\begin{CJK}{UTF8}{mj}微信公众号\end{CJK}: sxkyliyang

\begin{CJK}{UTF8}{mj}一\end{CJK}、\begin{CJK}{UTF8}{mj}解答下列各题\end{CJK}.

\begin{enumerate}
  \item (5 \begin{CJK}{UTF8}{mj}分\end{CJK}) \begin{CJK}{UTF8}{mj}设\end{CJK} $f(x)$ \begin{CJK}{UTF8}{mj}是有理数域上的\end{CJK} 2008 \begin{CJK}{UTF8}{mj}次多项式\end{CJK}, \begin{CJK}{UTF8}{mj}证明\end{CJK}: $\sqrt[2009]{2}$ \begin{CJK}{UTF8}{mj}不可能是\end{CJK} $f(x)$ \begin{CJK}{UTF8}{mj}的根\end{CJK}.

  \item (10 \begin{CJK}{UTF8}{mj}分\end{CJK}) \begin{CJK}{UTF8}{mj}用代数学基本定理证明\end{CJK}: \begin{CJK}{UTF8}{mj}实数域\end{CJK} $\mathbb{R}$ \begin{CJK}{UTF8}{mj}上的任意不可约多项式只能是一次多项式或满足\end{CJK} $b^{2}-4 a c<0$ \begin{CJK}{UTF8}{mj}的二次多项式\end{CJK}: $a x^{2}+b x+c$.

  \item ( 5 \begin{CJK}{UTF8}{mj}分\end{CJK}) \begin{CJK}{UTF8}{mj}不用\end{CJK} Hamilton-Calay \begin{CJK}{UTF8}{mj}定理证明\end{CJK}: \begin{CJK}{UTF8}{mj}对数域\end{CJK} $F$ \begin{CJK}{UTF8}{mj}上的\end{CJK} $n$ \begin{CJK}{UTF8}{mj}阶矩阵\end{CJK} $A$, \begin{CJK}{UTF8}{mj}存在\end{CJK} $F$ \begin{CJK}{UTF8}{mj}上的多项式\end{CJK} $f(x)$ \begin{CJK}{UTF8}{mj}使得\end{CJK} $f(A)=0 .$

  \item ( 10 \begin{CJK}{UTF8}{mj}分\end{CJK}) \begin{CJK}{UTF8}{mj}设\end{CJK} $f(x)=3 x^{3}+2 x+1, \alpha_{1}, \alpha_{2}, \alpha_{3}$ \begin{CJK}{UTF8}{mj}是\end{CJK} $f(x)$ \begin{CJK}{UTF8}{mj}的全部根\end{CJK}, \begin{CJK}{UTF8}{mj}求下式的值\end{CJK}

\end{enumerate}
$$
\left(\alpha_{1}^{2}+\alpha_{2} \alpha_{3}\right)\left(\alpha_{2}^{2}+\alpha_{1} \alpha_{3}\right)\left(\alpha_{3}^{2}+\alpha_{1} \alpha_{2}\right)
$$
\begin{CJK}{UTF8}{mj}二\end{CJK}、\begin{CJK}{UTF8}{mj}解答下列各题\end{CJK}.

\begin{enumerate}
  \item (10 \begin{CJK}{UTF8}{mj}分\end{CJK}) \begin{CJK}{UTF8}{mj}叙述并证明线性方程组的\end{CJK} Gramer \begin{CJK}{UTF8}{mj}法则\end{CJK}.

  \item ( 5 \begin{CJK}{UTF8}{mj}分\end{CJK}) $F, K$ \begin{CJK}{UTF8}{mj}是数域\end{CJK}, \begin{CJK}{UTF8}{mj}且\end{CJK} $F \subseteq K$, \begin{CJK}{UTF8}{mj}设\end{CJK} $A X=\beta$ \begin{CJK}{UTF8}{mj}是数域\end{CJK} $F$ \begin{CJK}{UTF8}{mj}上的线性方程组\end{CJK}, \begin{CJK}{UTF8}{mj}证明\end{CJK}: $A X=\beta$ \begin{CJK}{UTF8}{mj}在\end{CJK} $F$ \begin{CJK}{UTF8}{mj}中有解\end{CJK} \begin{CJK}{UTF8}{mj}当且仅当它在\end{CJK} $K$ \begin{CJK}{UTF8}{mj}中有解\end{CJK}.

  \item \begin{CJK}{UTF8}{mj}设\end{CJK}

\end{enumerate}
$$
A=\left(\begin{array}{ccc}
2 & -2 & 2 \\
-2 & -1 & 4 \\
2 & 4 & -1
\end{array}\right)
$$
(1) ( 5 \begin{CJK}{UTF8}{mj}分\end{CJK}) \begin{CJK}{UTF8}{mj}在任意数域\end{CJK} $F$ \begin{CJK}{UTF8}{mj}上\end{CJK}, $A$ \begin{CJK}{UTF8}{mj}能否相似于对角矩阵\end{CJK}? \begin{CJK}{UTF8}{mj}说明理由\end{CJK}.

(2) (5 \begin{CJK}{UTF8}{mj}分\end{CJK}) \begin{CJK}{UTF8}{mj}求\end{CJK} $A$ \begin{CJK}{UTF8}{mj}的最小多项式\end{CJK}.

(3) (5 \begin{CJK}{UTF8}{mj}分\end{CJK}) \begin{CJK}{UTF8}{mj}设\end{CJK} $f(X)=X^{\prime} A X$, \begin{CJK}{UTF8}{mj}其中\end{CJK} $X=\left(x_{1}, x_{2}, \cdots, x_{n}\right)^{\prime}$ \begin{CJK}{UTF8}{mj}是列向量\end{CJK}. \begin{CJK}{UTF8}{mj}求\end{CJK} $f(x)$ \begin{CJK}{UTF8}{mj}的一个标准形\end{CJK}.

\begin{enumerate}
  \setcounter{enumi}{4}
  \item (10 \begin{CJK}{UTF8}{mj}分\end{CJK}) \begin{CJK}{UTF8}{mj}证明\end{CJK}: \begin{CJK}{UTF8}{mj}在任意数域\end{CJK} $F$ \begin{CJK}{UTF8}{mj}上\end{CJK}, \begin{CJK}{UTF8}{mj}矩阵\end{CJK} $A=\left(\begin{array}{ccc}2 & -1 & 0 \\ 1 & 0 & 0 \\ -1 & 1 & 1\end{array}\right)$ \begin{CJK}{UTF8}{mj}与\end{CJK} $B=\left(\begin{array}{ccc}1 & 0 & 0 \\ 1 & 1 & 0 \\ 0 & 1 & 1\end{array}\right)$ \begin{CJK}{UTF8}{mj}不相似\end{CJK}.

  \item ( 5 \begin{CJK}{UTF8}{mj}分\end{CJK}) \begin{CJK}{UTF8}{mj}设\end{CJK} $A$ \begin{CJK}{UTF8}{mj}是\end{CJK} $n$ \begin{CJK}{UTF8}{mj}级实对称矩阵\end{CJK}, \begin{CJK}{UTF8}{mj}证明\end{CJK}: $A$ \begin{CJK}{UTF8}{mj}是正定矩阵的充要条件是对任意的整数\end{CJK} $k, A^{k}$ \begin{CJK}{UTF8}{mj}是正定矩阵\end{CJK}.

\end{enumerate}
\begin{CJK}{UTF8}{mj}三\end{CJK}、( 15 \begin{CJK}{UTF8}{mj}分\end{CJK}) \begin{CJK}{UTF8}{mj}设\end{CJK} $M_{n}(F)$ \begin{CJK}{UTF8}{mj}是数域\end{CJK} $F$ \begin{CJK}{UTF8}{mj}上全体\end{CJK} $n$ \begin{CJK}{UTF8}{mj}阶方阵组成的集合\end{CJK}, \begin{CJK}{UTF8}{mj}对任意的可逆矩阵\end{CJK} $A \in M_{n}(F)$, \begin{CJK}{UTF8}{mj}定义集合\end{CJK} $T_{A}=\left\{X \in M_{n}(F) \mid A^{-1} X A=X\right\}$, \begin{CJK}{UTF8}{mj}设\end{CJK}
$$
V=\bigcap_{A \in M_{n}(F):|A| \neq 0} T_{A} .
$$
\begin{CJK}{UTF8}{mj}即\end{CJK} $V$ \begin{CJK}{UTF8}{mj}是所有可逆矩阵构造出来的\end{CJK} $T_{A}$ \begin{CJK}{UTF8}{mj}的交\end{CJK}, \begin{CJK}{UTF8}{mj}求\end{CJK} $\operatorname{dim} V$ \begin{CJK}{UTF8}{mj}和\end{CJK} $V$ \begin{CJK}{UTF8}{mj}的一组基\end{CJK}.

\begin{CJK}{UTF8}{mj}四\end{CJK}、\begin{CJK}{UTF8}{mj}设\end{CJK} $M_{2 r+1}(F)$ \begin{CJK}{UTF8}{mj}是数域\end{CJK} $F$ \begin{CJK}{UTF8}{mj}上全体\end{CJK} $2 r+1$ \begin{CJK}{UTF8}{mj}阶方阵组成的集合\end{CJK}, \begin{CJK}{UTF8}{mj}设分块矩阵\end{CJK}
$$
M=\left(\begin{array}{ccc}
2 & 0 & 0 \\
0 & 0 & E_{r} \\
0 & E_{r} & 0
\end{array}\right)
$$
\begin{CJK}{UTF8}{mj}其中\end{CJK} $E_{r}$ \begin{CJK}{UTF8}{mj}为\end{CJK} $r$ \begin{CJK}{UTF8}{mj}阶单位矩阵\end{CJK}. \begin{CJK}{UTF8}{mj}记\end{CJK} $B=\left\{X \in M_{2 r+1}(F) \mid X^{\prime} M+M X=0\right\}$, \begin{CJK}{UTF8}{mj}其中\end{CJK} $X^{\prime}$ \begin{CJK}{UTF8}{mj}表示\end{CJK} $X$ \begin{CJK}{UTF8}{mj}的转置\end{CJK}, \begin{CJK}{UTF8}{mj}对任意的\end{CJK} $X \in M_{2 r+1}(F)$, \begin{CJK}{UTF8}{mj}设\end{CJK} $e^{X}=\sum_{k=0}^{\infty} \frac{X^{k}}{k !}$, \begin{CJK}{UTF8}{mj}已知\end{CJK} $e^{X} \in M_{2 r+1}(F)$.

\begin{enumerate}
  \item (15 \begin{CJK}{UTF8}{mj}分\end{CJK}) \begin{CJK}{UTF8}{mj}求\end{CJK} $B$ \begin{CJK}{UTF8}{mj}的维数和一组基\end{CJK}. 2. ( 15 \begin{CJK}{UTF8}{mj}分\end{CJK}) \begin{CJK}{UTF8}{mj}证明\end{CJK}: \begin{CJK}{UTF8}{mj}对任意的\end{CJK} $X \in B$, \begin{CJK}{UTF8}{mj}都有\end{CJK} $\operatorname{det}\left(e^{X}\right)=1$.

  \item (10 \begin{CJK}{UTF8}{mj}分\end{CJK}) \begin{CJK}{UTF8}{mj}设列向量空间\end{CJK} $F^{2 r+1}$ \begin{CJK}{UTF8}{mj}上的一个双线性形\end{CJK} $(,$, \begin{CJK}{UTF8}{mj}在它的基\end{CJK} $\varepsilon_{i}=(0, \cdots, 1, \cdots, 0)^{\prime}, i=1,2, \cdots, 2 r+1$. \begin{CJK}{UTF8}{mj}下的度量矩阵为上述的\end{CJK} $M$. \begin{CJK}{UTF8}{mj}证明\end{CJK}: \begin{CJK}{UTF8}{mj}对任意的\end{CJK} $X \in B$ \begin{CJK}{UTF8}{mj}及列向量\end{CJK} $\alpha, \beta \in F^{2 r+1}$ \begin{CJK}{UTF8}{mj}都有\end{CJK} $\left(e^{X} \alpha, e^{X} \beta\right)=(\alpha, \beta)$.

\end{enumerate}
\begin{CJK}{UTF8}{mj}五\end{CJK}、(20 \begin{CJK}{UTF8}{mj}分\end{CJK}) \begin{CJK}{UTF8}{mj}证明\end{CJK}: \begin{CJK}{UTF8}{mj}数域\end{CJK} $F$ \begin{CJK}{UTF8}{mj}上的任意一个\end{CJK} $n$ \begin{CJK}{UTF8}{mj}元多项式都是线性多项式\end{CJK}(\begin{CJK}{UTF8}{mj}即一次齐次多项式\end{CJK})\begin{CJK}{UTF8}{mj}的幂的线性组合\end{CJK}.

\section{2. 四川大学 2010 年研究生入学考试试题高等代数 
 李扬 
 微信公众号: sxkyliyang}
\begin{CJK}{UTF8}{mj}一\end{CJK}、 (\begin{CJK}{UTF8}{mj}每小题满分\end{CJK} 10 \begin{CJK}{UTF8}{mj}分\end{CJK}, \begin{CJK}{UTF8}{mj}共\end{CJK} 50 \begin{CJK}{UTF8}{mj}分\end{CJK}) \begin{CJK}{UTF8}{mj}设\end{CJK} $A$ \begin{CJK}{UTF8}{mj}为实数域\end{CJK} $\mathbb{R}$ \begin{CJK}{UTF8}{mj}上的\end{CJK} $n$ \begin{CJK}{UTF8}{mj}阶实对称矩阵\end{CJK}, \begin{CJK}{UTF8}{mj}解答下列各题\end{CJK}.

\begin{enumerate}
  \item \begin{CJK}{UTF8}{mj}证明\end{CJK}: \begin{CJK}{UTF8}{mj}矩阵\end{CJK} $\sqrt{-1} E_{n}+A$ \begin{CJK}{UTF8}{mj}可逆\end{CJK}, \begin{CJK}{UTF8}{mj}这里\end{CJK} $E_{n}$ \begin{CJK}{UTF8}{mj}是\end{CJK} $n$ \begin{CJK}{UTF8}{mj}阶单位矩阵\end{CJK}.

  \item \begin{CJK}{UTF8}{mj}设函数\end{CJK} $f: \mathbb{R}^{n} \times \mathbb{R}^{n} \rightarrow \mathbb{R}$ \begin{CJK}{UTF8}{mj}为\end{CJK}: $f(X, Y)=X^{\prime} A Y, X, Y \in \mathbb{R}^{n}$. \begin{CJK}{UTF8}{mj}证明\end{CJK}: $f$ \begin{CJK}{UTF8}{mj}不是零函数当且仅当存在\end{CJK} $X_{0} \in \mathbb{R}^{n}$ \begin{CJK}{UTF8}{mj}使得\end{CJK} $f\left(X_{0}, X_{0}\right) \neq 0$.

  \item \begin{CJK}{UTF8}{mj}设\end{CJK} $f(x)=\left|x E_{n}-A\right|$ \begin{CJK}{UTF8}{mj}是\end{CJK} $A$ \begin{CJK}{UTF8}{mj}的特征多项式\end{CJK}, \begin{CJK}{UTF8}{mj}设\end{CJK} $f^{\prime}(x)$ \begin{CJK}{UTF8}{mj}为\end{CJK} $f(x)$ \begin{CJK}{UTF8}{mj}的导数且\end{CJK} $f^{\prime}(x) \mid f(x)$. \begin{CJK}{UTF8}{mj}证明\end{CJK}: $A$ \begin{CJK}{UTF8}{mj}是数量矩阵\end{CJK}.

  \item \begin{CJK}{UTF8}{mj}设\end{CJK} $A$ \begin{CJK}{UTF8}{mj}的秩\end{CJK} $r(A)=r$, \begin{CJK}{UTF8}{mj}设\end{CJK} $V=\left\{X \in \mathbb{R}^{n} \mid X^{\prime} A X=0\right\}$. \begin{CJK}{UTF8}{mj}证明\end{CJK}: $V$ \begin{CJK}{UTF8}{mj}包含\end{CJK} $\mathbb{R}^{n}$ \begin{CJK}{UTF8}{mj}的一个维数为\end{CJK} $n-r$ \begin{CJK}{UTF8}{mj}的子空间\end{CJK}. $V$ \begin{CJK}{UTF8}{mj}是\end{CJK} $\mathbb{R}^{n}$ \begin{CJK}{UTF8}{mj}的子空间吗\end{CJK}? \begin{CJK}{UTF8}{mj}说明你的理由\end{CJK}.

  \item \begin{CJK}{UTF8}{mj}进一步假设\end{CJK} $A$ \begin{CJK}{UTF8}{mj}正定\end{CJK}, \begin{CJK}{UTF8}{mj}而\end{CJK} $B$ \begin{CJK}{UTF8}{mj}是一个负定的\end{CJK} $n$ \begin{CJK}{UTF8}{mj}阶矩阵\end{CJK}, \begin{CJK}{UTF8}{mj}证明\end{CJK}: \begin{CJK}{UTF8}{mj}如果\end{CJK} $A C=C B$, \begin{CJK}{UTF8}{mj}那么必然有\end{CJK} $C=0$.

\end{enumerate}
\begin{CJK}{UTF8}{mj}二\end{CJK}、(\begin{CJK}{UTF8}{mj}每小题满分\end{CJK} 10 \begin{CJK}{UTF8}{mj}分\end{CJK}, \begin{CJK}{UTF8}{mj}共\end{CJK} 50 \begin{CJK}{UTF8}{mj}分\end{CJK})\begin{CJK}{UTF8}{mj}设\end{CJK} $A$ \begin{CJK}{UTF8}{mj}为数域\end{CJK} $F$ \begin{CJK}{UTF8}{mj}上的\end{CJK} $n$ \begin{CJK}{UTF8}{mj}阶方阵\end{CJK}, \begin{CJK}{UTF8}{mj}它的秩为\end{CJK} $r$, \begin{CJK}{UTF8}{mj}解答下列各题\end{CJK}.

\begin{enumerate}
  \item \begin{CJK}{UTF8}{mj}设\end{CJK} $E_{r}$ \begin{CJK}{UTF8}{mj}是\end{CJK} $r$ \begin{CJK}{UTF8}{mj}阶单位阵\end{CJK}, \begin{CJK}{UTF8}{mj}写出\end{CJK} “\begin{CJK}{UTF8}{mj}存在可逆矩阵\end{CJK} $P$ \begin{CJK}{UTF8}{mj}使得\end{CJK} $P A=\left(\begin{array}{cc}E_{r} & 0 \\ 0 & 0\end{array}\right)$ " \begin{CJK}{UTF8}{mj}的一个充分必要条件\end{CJK}, \begin{CJK}{UTF8}{mj}并证明你\end{CJK} \begin{CJK}{UTF8}{mj}的结论\end{CJK}.

  \item \begin{CJK}{UTF8}{mj}设\end{CJK} $\alpha_{1}, \alpha_{2}, \cdots, \alpha_{n}$ \begin{CJK}{UTF8}{mj}是\end{CJK} $F^{n}$ \begin{CJK}{UTF8}{mj}的一个基\end{CJK}. \begin{CJK}{UTF8}{mj}令\end{CJK} $\left(\beta_{1}, \beta_{2}, \cdots, \beta_{n}\right)=\left(\alpha_{1}, \alpha_{2}, \cdots, \alpha_{n}\right) A$. \begin{CJK}{UTF8}{mj}求向量组\end{CJK} $\beta_{1}, \beta_{2}, \cdots, \beta_{n}$ \begin{CJK}{UTF8}{mj}的秩\end{CJK}, \begin{CJK}{UTF8}{mj}并给出它的一个极大无关组\end{CJK}.

  \item \begin{CJK}{UTF8}{mj}设\end{CJK} $P(A)$ \begin{CJK}{UTF8}{mj}是满足\end{CJK} $f(A)=0$ \begin{CJK}{UTF8}{mj}的\end{CJK} $F$ \begin{CJK}{UTF8}{mj}上的所有多项式\end{CJK} $f(x)$ \begin{CJK}{UTF8}{mj}组成的集合\end{CJK}. \begin{CJK}{UTF8}{mj}证明\end{CJK}: $P(A)$ \begin{CJK}{UTF8}{mj}是\end{CJK} $F$ \begin{CJK}{UTF8}{mj}上的无穷维线性空\end{CJK} \begin{CJK}{UTF8}{mj}间\end{CJK}. \begin{CJK}{UTF8}{mj}并且\end{CJK}, \begin{CJK}{UTF8}{mj}如果\end{CJK} $g(x) \in P(A)$ \begin{CJK}{UTF8}{mj}的次数大于\end{CJK} $n$, \begin{CJK}{UTF8}{mj}那么\end{CJK} $g(x)$ \begin{CJK}{UTF8}{mj}在\end{CJK} $F$ \begin{CJK}{UTF8}{mj}上是可约的\end{CJK}.

  \item \begin{CJK}{UTF8}{mj}设\end{CJK} $\lambda_{1}, \cdots, \lambda_{n}$ \begin{CJK}{UTF8}{mj}是\end{CJK} $A$ \begin{CJK}{UTF8}{mj}的全部复特征值\end{CJK}, \begin{CJK}{UTF8}{mj}证明\end{CJK}: \begin{CJK}{UTF8}{mj}对任意非负整数\end{CJK} $k$, \begin{CJK}{UTF8}{mj}数\end{CJK} $s_{k}=\sum_{i=1}^{n} \lambda_{i}^{k}$ \begin{CJK}{UTF8}{mj}属于\end{CJK} $F$.

  \item \begin{CJK}{UTF8}{mj}设\end{CJK} $V$ \begin{CJK}{UTF8}{mj}是\end{CJK} $F$ \begin{CJK}{UTF8}{mj}上的线性空间\end{CJK}, $\varepsilon_{1}, \cdots, \varepsilon_{n}$ \begin{CJK}{UTF8}{mj}是\end{CJK} $V$ \begin{CJK}{UTF8}{mj}的一个基\end{CJK}. \begin{CJK}{UTF8}{mj}设\end{CJK} $V$ \begin{CJK}{UTF8}{mj}上的一个线性变换\end{CJK} $\mathscr{A}$ \begin{CJK}{UTF8}{mj}满足\end{CJK}

\end{enumerate}
$$
\mathscr{A}\left(\varepsilon_{1}, \cdots, \varepsilon_{n}\right)=\left(\varepsilon_{1}, \cdots, \varepsilon_{n}\right) A \text {, 且 } \mathscr{A}^{2}=\mathscr{A} \text {. }
$$
\begin{CJK}{UTF8}{mj}证明\end{CJK}: $\operatorname{ker} \mathscr{A}+\operatorname{Im} \mathscr{A}$ \begin{CJK}{UTF8}{mj}是直和\end{CJK}, \begin{CJK}{UTF8}{mj}这里\end{CJK}, $\operatorname{ker} \mathscr{A}$ \begin{CJK}{UTF8}{mj}和\end{CJK} $\operatorname{Im} \mathscr{A}$ \begin{CJK}{UTF8}{mj}分别是\end{CJK} $\mathscr{A}$ \begin{CJK}{UTF8}{mj}的核和像\end{CJK}.

\begin{CJK}{UTF8}{mj}三\end{CJK}、\begin{CJK}{UTF8}{mj}设\end{CJK} $A X=\beta$ \begin{CJK}{UTF8}{mj}是数域\end{CJK} $F$ \begin{CJK}{UTF8}{mj}上的一个\end{CJK} $n$ \begin{CJK}{UTF8}{mj}元线性方程组\end{CJK}, \begin{CJK}{UTF8}{mj}其系数矩阵\end{CJK} $A$ \begin{CJK}{UTF8}{mj}的秩\end{CJK} $r(A)=r$. \begin{CJK}{UTF8}{mj}设\end{CJK} $S$ \begin{CJK}{UTF8}{mj}为它的解集\end{CJK}.

\begin{enumerate}
  \item (5 \begin{CJK}{UTF8}{mj}分\end{CJK}) \begin{CJK}{UTF8}{mj}给出\end{CJK} " $S$ \begin{CJK}{UTF8}{mj}是\end{CJK} $F^{n}$ \begin{CJK}{UTF8}{mj}的子空间\end{CJK}" \begin{CJK}{UTF8}{mj}的充分必要条件\end{CJK}, \begin{CJK}{UTF8}{mj}并证明你的结论\end{CJK}.

  \item (10 \begin{CJK}{UTF8}{mj}分\end{CJK}) \begin{CJK}{UTF8}{mj}假设\end{CJK} $S$ \begin{CJK}{UTF8}{mj}不是空集且不是\end{CJK} $F^{n}$ \begin{CJK}{UTF8}{mj}的子空间\end{CJK}. \begin{CJK}{UTF8}{mj}求\end{CJK} $S$ \begin{CJK}{UTF8}{mj}的秩\end{CJK}, \begin{CJK}{UTF8}{mj}并给出它的一个极大无关组\end{CJK}.

\end{enumerate}
\begin{CJK}{UTF8}{mj}四\end{CJK}、\begin{CJK}{UTF8}{mj}设\end{CJK} $A=\left(\begin{array}{ccc}2 & -1 & 0 \\ -1 & 2 & -1 \\ 0 & -1 & 2\end{array}\right)$, \begin{CJK}{UTF8}{mj}设\end{CJK} $C(A)$ \begin{CJK}{UTF8}{mj}是所有与\end{CJK} $A$ \begin{CJK}{UTF8}{mj}可交换的实矩阵组成的集合\end{CJK}.

\begin{enumerate}
  \item ( 5 \begin{CJK}{UTF8}{mj}分\end{CJK}) \begin{CJK}{UTF8}{mj}证明\end{CJK}: $C(A)$ \begin{CJK}{UTF8}{mj}是实数域\end{CJK} $\mathbb{R}$ \begin{CJK}{UTF8}{mj}上的线性空间\end{CJK}.
\end{enumerate}
\section{3. 四川大学 2011 年研究生入学考试试题高等代数 
 李扬 
 微信公众号: sxkyliyang}
\begin{CJK}{UTF8}{mj}一\end{CJK}、\begin{CJK}{UTF8}{mj}解答下列各题\end{CJK}.

\begin{enumerate}
  \item ( 5 \begin{CJK}{UTF8}{mj}分\end{CJK}) \begin{CJK}{UTF8}{mj}设\end{CJK} $V$ \begin{CJK}{UTF8}{mj}是数域\end{CJK} $F$ \begin{CJK}{UTF8}{mj}上的线性空间\end{CJK}, $\alpha_{1}, \alpha_{2}, \cdots, \alpha_{s} \in V$. \begin{CJK}{UTF8}{mj}令\end{CJK} $W=\left\{\sum_{i=1}^{s} k_{i} \alpha_{i} \mid k_{i} \in F\right\}$. \begin{CJK}{UTF8}{mj}证明\end{CJK}: $W$ \begin{CJK}{UTF8}{mj}是\end{CJK} $V$ \begin{CJK}{UTF8}{mj}的子空间\end{CJK}(\begin{CJK}{UTF8}{mj}称为由\end{CJK} $\alpha_{1}, \alpha_{2}, \cdots, \alpha_{s}$ \begin{CJK}{UTF8}{mj}生成的子空间\end{CJK}).

  \item ( 15 \begin{CJK}{UTF8}{mj}分\end{CJK}) \begin{CJK}{UTF8}{mj}设\end{CJK} $M_{2}(F)$ \begin{CJK}{UTF8}{mj}是数域\end{CJK} $F$ \begin{CJK}{UTF8}{mj}上的\end{CJK} 2 \begin{CJK}{UTF8}{mj}阶方阵组成的线性空间\end{CJK}, \begin{CJK}{UTF8}{mj}设\end{CJK} $V$ \begin{CJK}{UTF8}{mj}是由如下的\end{CJK} 4 \begin{CJK}{UTF8}{mj}个矩阵生成\end{CJK} $M_{2}(F)$ \begin{CJK}{UTF8}{mj}的\end{CJK} \begin{CJK}{UTF8}{mj}子空间\end{CJK}:

\end{enumerate}
$$
A_{1}=\left(\begin{array}{cc}
-1 & 4 \\
2 & 0
\end{array}\right), A_{2}=\left(\begin{array}{cc}
5 & 1 \\
0 & 3
\end{array}\right), A_{3}=\left(\begin{array}{cc}
3 & -2 \\
-1 & 4
\end{array}\right), A_{4}=\left(\begin{array}{cc}
-2 & 9 \\
4 & -5
\end{array}\right)
$$
(1) \begin{CJK}{UTF8}{mj}求\end{CJK} $\operatorname{dim} V$ \begin{CJK}{UTF8}{mj}并写出\end{CJK} $V$ \begin{CJK}{UTF8}{mj}的一组基\end{CJK}.

(2) \begin{CJK}{UTF8}{mj}设映射\end{CJK} $f: V \rightarrow F$ \begin{CJK}{UTF8}{mj}为\end{CJK}: $f(A)=\operatorname{tr}(A)$, \begin{CJK}{UTF8}{mj}其中\end{CJK} $\operatorname{tr}(A)$ \begin{CJK}{UTF8}{mj}表示矩阵\end{CJK} $A$ \begin{CJK}{UTF8}{mj}的迹\end{CJK}. \begin{CJK}{UTF8}{mj}求\end{CJK} $\operatorname{dim} \operatorname{ker} f$ \begin{CJK}{UTF8}{mj}并写出\end{CJK} $\operatorname{ker} f$ \begin{CJK}{UTF8}{mj}的一\end{CJK} \begin{CJK}{UTF8}{mj}组基\end{CJK}.

\begin{CJK}{UTF8}{mj}二\end{CJK}、\begin{CJK}{UTF8}{mj}设\end{CJK} $F, K$ \begin{CJK}{UTF8}{mj}都是数域且\end{CJK} $F \subseteq K$ :

\begin{enumerate}
  \item ( 5 \begin{CJK}{UTF8}{mj}分\end{CJK}) \begin{CJK}{UTF8}{mj}设\end{CJK} $\alpha_{1}, \alpha_{2}, \cdots, \alpha_{s}$ \begin{CJK}{UTF8}{mj}是\end{CJK} $F$ \begin{CJK}{UTF8}{mj}上的\end{CJK} $n$ \begin{CJK}{UTF8}{mj}维列向量\end{CJK}. \begin{CJK}{UTF8}{mj}证明\end{CJK}: $\alpha_{1}, \alpha_{2}, \cdots, \alpha_{s}$ \begin{CJK}{UTF8}{mj}在\end{CJK} $F$ \begin{CJK}{UTF8}{mj}上线性相关当且仅当\end{CJK} $\alpha_{1}, \alpha_{2}, \cdots, \alpha_{s}$ \begin{CJK}{UTF8}{mj}在\end{CJK} $K$ \begin{CJK}{UTF8}{mj}上线性相关\end{CJK}.

  \item (5 \begin{CJK}{UTF8}{mj}分\end{CJK}) \begin{CJK}{UTF8}{mj}设\end{CJK} $A, B$ \begin{CJK}{UTF8}{mj}是\end{CJK} $F$ \begin{CJK}{UTF8}{mj}上的\end{CJK} $n$ \begin{CJK}{UTF8}{mj}阶方阵\end{CJK}. \begin{CJK}{UTF8}{mj}证明\end{CJK}: $A, B$ \begin{CJK}{UTF8}{mj}在\end{CJK} $F$ \begin{CJK}{UTF8}{mj}上相似当且仅当\end{CJK} $A, B$ \begin{CJK}{UTF8}{mj}在\end{CJK} $K$ \begin{CJK}{UTF8}{mj}上相似\end{CJK}.

  \item ( 5 \begin{CJK}{UTF8}{mj}分\end{CJK}) \begin{CJK}{UTF8}{mj}设\end{CJK} $F$ \begin{CJK}{UTF8}{mj}上的\end{CJK} $n$ \begin{CJK}{UTF8}{mj}次多项式\end{CJK} $f(x)$ \begin{CJK}{UTF8}{mj}在\end{CJK} $K$ \begin{CJK}{UTF8}{mj}上有\end{CJK} $n$ \begin{CJK}{UTF8}{mj}个根\end{CJK} $x_{1}, x_{2}, \cdots, x_{n}$. \begin{CJK}{UTF8}{mj}证明\end{CJK}: $\prod_{1 \leq i<j \leq n}\left(x_{i}-x_{j}\right)^{2} \in F$.

  \item ( 5 \begin{CJK}{UTF8}{mj}分\end{CJK}) \begin{CJK}{UTF8}{mj}证明\end{CJK}:\begin{CJK}{UTF8}{mj}关于数的加法和乘法\end{CJK} $K$ \begin{CJK}{UTF8}{mj}是\end{CJK} $F$ \begin{CJK}{UTF8}{mj}上的线性空间\end{CJK}.

\end{enumerate}
\begin{CJK}{UTF8}{mj}三\end{CJK}、 (20 \begin{CJK}{UTF8}{mj}分\end{CJK}) \begin{CJK}{UTF8}{mj}给定任意的可逆矩阵\end{CJK} $A$. \begin{CJK}{UTF8}{mj}请说出\end{CJK} 4 \begin{CJK}{UTF8}{mj}种求\end{CJK} $A^{-1}$ \begin{CJK}{UTF8}{mj}的方法\end{CJK}(\begin{CJK}{UTF8}{mj}使用计算机程序的方法除外\end{CJK}), \begin{CJK}{UTF8}{mj}并简要说明理由\end{CJK}.

\begin{CJK}{UTF8}{mj}四\end{CJK}、\begin{CJK}{UTF8}{mj}设\end{CJK} $f(x)=x_{p-1}+x_{p-2}+\cdots+x+1, p$ \begin{CJK}{UTF8}{mj}是素数\end{CJK}.

\begin{enumerate}
  \item (10 \begin{CJK}{UTF8}{mj}分\end{CJK}) \begin{CJK}{UTF8}{mj}证明\end{CJK}: $f(x)$ \begin{CJK}{UTF8}{mj}在有理数域\end{CJK} $\mathbb{Q}$ \begin{CJK}{UTF8}{mj}上不可约\end{CJK}.

  \item (10 \begin{CJK}{UTF8}{mj}分\end{CJK}) \begin{CJK}{UTF8}{mj}令\end{CJK} $\mathscr{M}=\left\{A \in M_{n}(\mathbb{C}) \mid f(A)=0\right\}$, \begin{CJK}{UTF8}{mj}其中\end{CJK} $M_{n}(\mathbb{C})$ \begin{CJK}{UTF8}{mj}是全体\end{CJK} $n$ \begin{CJK}{UTF8}{mj}阶复矩阵组成的集合\end{CJK}. \begin{CJK}{UTF8}{mj}把\end{CJK} $\mathscr{M}$ \begin{CJK}{UTF8}{mj}中的矩阵\end{CJK} \begin{CJK}{UTF8}{mj}按相似关系分类\end{CJK}, \begin{CJK}{UTF8}{mj}即\end{CJK}, $A, B$ \begin{CJK}{UTF8}{mj}属于同一类当且仅当存在可逆的复矩阵\end{CJK} $C$ \begin{CJK}{UTF8}{mj}使得\end{CJK} $A=C B C^{-1}$. \begin{CJK}{UTF8}{mj}问\end{CJK} $\mathscr{M}$ \begin{CJK}{UTF8}{mj}中的全\end{CJK} \begin{CJK}{UTF8}{mj}部矩阵可以分成几类\end{CJK}? \begin{CJK}{UTF8}{mj}说明理由\end{CJK}.

\end{enumerate}
\begin{CJK}{UTF8}{mj}五\end{CJK}、\begin{CJK}{UTF8}{mj}设\end{CJK} $V$ \begin{CJK}{UTF8}{mj}是数域\end{CJK} $F$ \begin{CJK}{UTF8}{mj}上的\end{CJK} $n$ \begin{CJK}{UTF8}{mj}维线性空间\end{CJK}, $\operatorname{End}(V)$ \begin{CJK}{UTF8}{mj}表示\end{CJK} $V$ \begin{CJK}{UTF8}{mj}上的全体线性变换组成的线性空间\end{CJK}.

\begin{enumerate}
  \item ( 10 \begin{CJK}{UTF8}{mj}分\end{CJK}) \begin{CJK}{UTF8}{mj}求\end{CJK} $\operatorname{dim} \operatorname{End}(V)$ \begin{CJK}{UTF8}{mj}并写出\end{CJK} $\operatorname{End}(V)$ \begin{CJK}{UTF8}{mj}的一组基\end{CJK}.

  \item (10 \begin{CJK}{UTF8}{mj}分\end{CJK}) \begin{CJK}{UTF8}{mj}设\end{CJK} $\mathscr{A} \in \operatorname{End}(V)$, \begin{CJK}{UTF8}{mj}设\end{CJK} $\mathscr{A}$ \begin{CJK}{UTF8}{mj}的特征多项式为\end{CJK} $f(x)$. \begin{CJK}{UTF8}{mj}证明\end{CJK}: \begin{CJK}{UTF8}{mj}如果\end{CJK} $V$ \begin{CJK}{UTF8}{mj}可以分解为非平凡的\end{CJK} $\mathscr{A}$ \begin{CJK}{UTF8}{mj}不变子空\end{CJK} \begin{CJK}{UTF8}{mj}间的直和\end{CJK}, \begin{CJK}{UTF8}{mj}那么\end{CJK}, $f(x)$ \begin{CJK}{UTF8}{mj}在\end{CJK} $F$ \begin{CJK}{UTF8}{mj}上可约\end{CJK}. \begin{CJK}{UTF8}{mj}问\end{CJK}:\begin{CJK}{UTF8}{mj}此结论的逆命题是否成立\end{CJK}? \begin{CJK}{UTF8}{mj}说明理由\end{CJK}.

\end{enumerate}
\begin{CJK}{UTF8}{mj}六\end{CJK}、\begin{CJK}{UTF8}{mj}设\end{CJK} $V$ \begin{CJK}{UTF8}{mj}是\end{CJK} $n$ \begin{CJK}{UTF8}{mj}维欧式空间\end{CJK}, \begin{CJK}{UTF8}{mj}内积为\end{CJK} $(,$, .

\begin{enumerate}
  \item (10 \begin{CJK}{UTF8}{mj}分\end{CJK}) \begin{CJK}{UTF8}{mj}设\end{CJK} $\alpha_{1}, \alpha_{2}, \cdots, \alpha_{s}$ \begin{CJK}{UTF8}{mj}是\end{CJK} $V$ \begin{CJK}{UTF8}{mj}中的一个线性无关的向量组\end{CJK}. \begin{CJK}{UTF8}{mj}证明如下的\end{CJK} Schmidt \begin{CJK}{UTF8}{mj}正交化定理\end{CJK}:\begin{CJK}{UTF8}{mj}存\end{CJK} \begin{CJK}{UTF8}{mj}在\end{CJK} $V$ \begin{CJK}{UTF8}{mj}中的一个两两正交的向量组\end{CJK} $\beta_{1}, \beta_{2}, \cdots, \beta_{s}$ \begin{CJK}{UTF8}{mj}满足\end{CJK}:\begin{CJK}{UTF8}{mj}对任意\end{CJK} $1 \leq k \leq s$ \begin{CJK}{UTF8}{mj}有\end{CJK}, $\alpha_{1}, \alpha_{2}, \cdots, \alpha_{k}$ \begin{CJK}{UTF8}{mj}与\end{CJK} $\beta_{1}, \beta_{2}, \cdots, \beta_{k}$ \begin{CJK}{UTF8}{mj}等价\end{CJK}. \begin{CJK}{UTF8}{mj}七\end{CJK}、 ( 15 \begin{CJK}{UTF8}{mj}分\end{CJK}) \begin{CJK}{UTF8}{mj}设\end{CJK} $A, B$ \begin{CJK}{UTF8}{mj}是\end{CJK} 2 \begin{CJK}{UTF8}{mj}阶实矩阵\end{CJK}, \begin{CJK}{UTF8}{mj}满足\end{CJK} $A B+B A=0$ \begin{CJK}{UTF8}{mj}且\end{CJK} $A^{2}=B^{2}=E$, \begin{CJK}{UTF8}{mj}这里\end{CJK} $E$ \begin{CJK}{UTF8}{mj}是单位阵\end{CJK}. \begin{CJK}{UTF8}{mj}证明\end{CJK}: \begin{CJK}{UTF8}{mj}存在可逆实\end{CJK} \begin{CJK}{UTF8}{mj}矩阵\end{CJK} $T$ \begin{CJK}{UTF8}{mj}使得\end{CJK} $T^{-1} A T=\left(\begin{array}{cc}1 & 0 \\ 0 & -1\end{array}\right)$ \begin{CJK}{UTF8}{mj}且\end{CJK} $T^{-1} B T=\left(\begin{array}{ll}0 & 1 \\ 1 & 0\end{array}\right)$.
\end{enumerate}
\begin{CJK}{UTF8}{mj}八\end{CJK}、 $\left(15\right.$ \begin{CJK}{UTF8}{mj}分\end{CJK}) \begin{CJK}{UTF8}{mj}求实矩阵\end{CJK} $X$ \begin{CJK}{UTF8}{mj}使得\end{CJK} $X^{4}=\left(\begin{array}{lll}3 & 0 & 0 \\ 0 & 3 & 1 \\ 0 & 0 & 0\end{array}\right)$.

\section{4. 四川大学 2012 年研究生入学考试试题高等代数}
\begin{CJK}{UTF8}{mj}李扬\end{CJK}

\begin{CJK}{UTF8}{mj}微信公众号\end{CJK}: sxkyliyang

\begin{CJK}{UTF8}{mj}一\end{CJK}、\begin{CJK}{UTF8}{mj}设\end{CJK} $f(x)=\sum_{i=0}^{n} a_{i} x^{i}$ \begin{CJK}{UTF8}{mj}是数域\end{CJK} $F$ \begin{CJK}{UTF8}{mj}上的\end{CJK} $n$ \begin{CJK}{UTF8}{mj}次多项式\end{CJK}, $n>0$.

\begin{enumerate}
  \item ( 10 \begin{CJK}{UTF8}{mj}分\end{CJK}) \begin{CJK}{UTF8}{mj}设\end{CJK} $c \in F$. \begin{CJK}{UTF8}{mj}证明\end{CJK}: \begin{CJK}{UTF8}{mj}存在唯一的\end{CJK} $b_{i} \in F$ \begin{CJK}{UTF8}{mj}使得\end{CJK} $f(x)=\sum_{i=0}^{n} b_{i}(x-c)^{i}$, \begin{CJK}{UTF8}{mj}并写出\end{CJK} $b_{i}$ \begin{CJK}{UTF8}{mj}的表达式\end{CJK}.

  \item (10 \begin{CJK}{UTF8}{mj}分\end{CJK}) \begin{CJK}{UTF8}{mj}设\end{CJK} $f(x)$ \begin{CJK}{UTF8}{mj}在\end{CJK} $F$ \begin{CJK}{UTF8}{mj}上不可约\end{CJK}, $\alpha$ \begin{CJK}{UTF8}{mj}是\end{CJK} $f(x)$ \begin{CJK}{UTF8}{mj}的一个复根\end{CJK}. \begin{CJK}{UTF8}{mj}证明\end{CJK}: \begin{CJK}{UTF8}{mj}集合\end{CJK} $K=\{g(\alpha) \mid g(x)$ \begin{CJK}{UTF8}{mj}是\end{CJK} $F$ \begin{CJK}{UTF8}{mj}上的多项式\end{CJK} $\}$ \begin{CJK}{UTF8}{mj}是\end{CJK} \begin{CJK}{UTF8}{mj}一个数域\end{CJK}, \begin{CJK}{UTF8}{mj}且\end{CJK} $f(x)$ \begin{CJK}{UTF8}{mj}在\end{CJK} $K$ \begin{CJK}{UTF8}{mj}上可约\end{CJK}.

  \item ( 5 \begin{CJK}{UTF8}{mj}分\end{CJK}) \begin{CJK}{UTF8}{mj}设\end{CJK} $\alpha_{1}, \cdots, \alpha_{n}$ \begin{CJK}{UTF8}{mj}是\end{CJK} $f(x)$ \begin{CJK}{UTF8}{mj}的全部复根\end{CJK}. \begin{CJK}{UTF8}{mj}证明\end{CJK}: \begin{CJK}{UTF8}{mj}存在\end{CJK} $F$ \begin{CJK}{UTF8}{mj}上的\end{CJK} $n$ \begin{CJK}{UTF8}{mj}次多项式\end{CJK} $h(x)$, \begin{CJK}{UTF8}{mj}其全部复根为\end{CJK} $\sum_{j=1}^{n} \alpha_{j}^{k}$, $k=1,2, \cdots, n$.

\end{enumerate}
\begin{CJK}{UTF8}{mj}二\end{CJK}、 \begin{CJK}{UTF8}{mj}设\end{CJK} $A$ \begin{CJK}{UTF8}{mj}是数域\end{CJK} $F$ \begin{CJK}{UTF8}{mj}上的\end{CJK} $m \times n$ \begin{CJK}{UTF8}{mj}型矩阵\end{CJK}.

\begin{enumerate}
  \item (5 \begin{CJK}{UTF8}{mj}分\end{CJK}) \begin{CJK}{UTF8}{mj}问\end{CJK}: $A$ \begin{CJK}{UTF8}{mj}应该满足什么条件\end{CJK}, \begin{CJK}{UTF8}{mj}使得对任意\end{CJK} $\beta \in F^{m}$, \begin{CJK}{UTF8}{mj}线性方程组\end{CJK} $A X=\beta$ \begin{CJK}{UTF8}{mj}都有解\end{CJK}? \begin{CJK}{UTF8}{mj}说明理由\end{CJK}.

  \item (10 \begin{CJK}{UTF8}{mj}分\end{CJK}) \begin{CJK}{UTF8}{mj}设\end{CJK} $F=\mathbb{R}$ \begin{CJK}{UTF8}{mj}是实数域\end{CJK}. \begin{CJK}{UTF8}{mj}证明\end{CJK}: \begin{CJK}{UTF8}{mj}对任意\end{CJK} $m$ \begin{CJK}{UTF8}{mj}为实向量\end{CJK} $\beta$, \begin{CJK}{UTF8}{mj}线性方程组\end{CJK} $A^{\prime} A X=A^{\prime} \beta$ \begin{CJK}{UTF8}{mj}都有解\end{CJK}, \begin{CJK}{UTF8}{mj}其中\end{CJK}, $A^{\prime}$ \begin{CJK}{UTF8}{mj}表\end{CJK} \begin{CJK}{UTF8}{mj}示\end{CJK} $A$ \begin{CJK}{UTF8}{mj}的转置\end{CJK}.

  \item ( 10 \begin{CJK}{UTF8}{mj}分\end{CJK}) \begin{CJK}{UTF8}{mj}设\end{CJK} $B$ \begin{CJK}{UTF8}{mj}也是数域\end{CJK} $F$ \begin{CJK}{UTF8}{mj}上的\end{CJK} $m \times n$ \begin{CJK}{UTF8}{mj}型矩阵\end{CJK}, $M_{m \times n}(F) 、 M_{n \times m}(F)$ \begin{CJK}{UTF8}{mj}分别是\end{CJK} $F$ \begin{CJK}{UTF8}{mj}上的所有\end{CJK} $m \times n$ \begin{CJK}{UTF8}{mj}型矩\end{CJK} \begin{CJK}{UTF8}{mj}阵\end{CJK}, $n \times m$ \begin{CJK}{UTF8}{mj}型矩阵组成的线性空间\end{CJK}. \begin{CJK}{UTF8}{mj}证明\end{CJK}: \begin{CJK}{UTF8}{mj}当\end{CJK} $m \neq n$ \begin{CJK}{UTF8}{mj}时\end{CJK}, \begin{CJK}{UTF8}{mj}由\end{CJK}

\end{enumerate}
$$
f(X)=A X B, \quad X \in M_{n \times m}
$$
\begin{CJK}{UTF8}{mj}给出的从\end{CJK} $M_{n \times m}(F)$ \begin{CJK}{UTF8}{mj}到\end{CJK} $M_{m \times n}(F)$ \begin{CJK}{UTF8}{mj}的线性映射\end{CJK} $f$ \begin{CJK}{UTF8}{mj}不是可逆的\end{CJK}.

\begin{CJK}{UTF8}{mj}三\end{CJK}、\begin{CJK}{UTF8}{mj}解答下列各题\end{CJK}.

\begin{enumerate}
  \item ( 5 \begin{CJK}{UTF8}{mj}分\end{CJK}) \begin{CJK}{UTF8}{mj}证明\end{CJK}: \begin{CJK}{UTF8}{mj}反对称实矩阵的特征值只能是\end{CJK} 0 \begin{CJK}{UTF8}{mj}或纯虚数\end{CJK}.

  \item (10 \begin{CJK}{UTF8}{mj}分\end{CJK}) \begin{CJK}{UTF8}{mj}证明\end{CJK}: \begin{CJK}{UTF8}{mj}数域\end{CJK} $F$ \begin{CJK}{UTF8}{mj}上的有限维空间上的线性变换只有有限多个特征值\end{CJK}. \begin{CJK}{UTF8}{mj}举一个有无穷多个特征值的\end{CJK} \begin{CJK}{UTF8}{mj}线性变换的例子\end{CJK}.

  \item (10 \begin{CJK}{UTF8}{mj}分\end{CJK}) \begin{CJK}{UTF8}{mj}设\end{CJK} $A$ \begin{CJK}{UTF8}{mj}是对角元为\end{CJK} $a_{1}, \cdots, a_{n}$ \begin{CJK}{UTF8}{mj}的上三角阵\end{CJK}, $B$ \begin{CJK}{UTF8}{mj}是对角元为\end{CJK} $b_{1}, \cdots, b_{n}$ \begin{CJK}{UTF8}{mj}的上三角阵\end{CJK}, \begin{CJK}{UTF8}{mj}且\end{CJK} $B$ \begin{CJK}{UTF8}{mj}可逆\end{CJK}. \begin{CJK}{UTF8}{mj}求矩阵\end{CJK} $\left(\begin{array}{cc}A & B \\ B & A\end{array}\right)$ \begin{CJK}{UTF8}{mj}的特征值\end{CJK}.

\end{enumerate}
\begin{CJK}{UTF8}{mj}四\end{CJK}、\begin{CJK}{UTF8}{mj}解答下列各题\end{CJK}.

\begin{enumerate}
  \item (10 \begin{CJK}{UTF8}{mj}分\end{CJK}) \begin{CJK}{UTF8}{mj}设\end{CJK} $A, B$ \begin{CJK}{UTF8}{mj}是\end{CJK} $n$ \begin{CJK}{UTF8}{mj}阶实矩阵\end{CJK}, \begin{CJK}{UTF8}{mj}且存在\end{CJK} $n$ \begin{CJK}{UTF8}{mj}阶复矩阵\end{CJK} $C$ \begin{CJK}{UTF8}{mj}使得\end{CJK} $C^{-1} A C=B$. \begin{CJK}{UTF8}{mj}问\end{CJK}: \begin{CJK}{UTF8}{mj}是否一定存在\end{CJK} $n$ \begin{CJK}{UTF8}{mj}阶实矩阵\end{CJK} $D$ \begin{CJK}{UTF8}{mj}使得\end{CJK} $D^{-1} A D=B$ ? \begin{CJK}{UTF8}{mj}如果不一定存在\end{CJK}, \begin{CJK}{UTF8}{mj}说明理由\end{CJK}; \begin{CJK}{UTF8}{mj}如果一定存在\end{CJK}, \begin{CJK}{UTF8}{mj}利用\end{CJK} $A, B, C$ \begin{CJK}{UTF8}{mj}求出这样的\end{CJK} $D$.

  \item ( 15 \begin{CJK}{UTF8}{mj}分\end{CJK}) \begin{CJK}{UTF8}{mj}设\end{CJK} $A$ \begin{CJK}{UTF8}{mj}是\end{CJK} 2 \begin{CJK}{UTF8}{mj}阶整数矩阵且存在正整数\end{CJK} $m$ \begin{CJK}{UTF8}{mj}使得\end{CJK} $A^{m}$ \begin{CJK}{UTF8}{mj}是单位阵\end{CJK}. \begin{CJK}{UTF8}{mj}证明\end{CJK}: $A^{12}$ \begin{CJK}{UTF8}{mj}是单位阵\end{CJK}.

\end{enumerate}
\begin{CJK}{UTF8}{mj}五\end{CJK}、\begin{CJK}{UTF8}{mj}设\end{CJK} $T: V \rightarrow W$ \begin{CJK}{UTF8}{mj}是数域\end{CJK} $F$ \begin{CJK}{UTF8}{mj}上的线性空间之间的线性映射\end{CJK}.

\begin{enumerate}
  \item (10 \begin{CJK}{UTF8}{mj}分\end{CJK}) \begin{CJK}{UTF8}{mj}证明\end{CJK}: $T$ \begin{CJK}{UTF8}{mj}是单射当且仅当对任意满足\end{CJK} $T S=0$ \begin{CJK}{UTF8}{mj}的线性映射\end{CJK} $S: U \rightarrow V$, \begin{CJK}{UTF8}{mj}都有\end{CJK} $S=0 ; T$ \begin{CJK}{UTF8}{mj}是满射当且\end{CJK} \begin{CJK}{UTF8}{mj}仅当对任意满足\end{CJK} $R T=0$ \begin{CJK}{UTF8}{mj}的线性映射\end{CJK} $R: W \rightarrow X$, \begin{CJK}{UTF8}{mj}都有\end{CJK} $R=0$.
\end{enumerate}
$$
\operatorname{dim} T^{-1}\left(W_{1}\right) \geq \operatorname{dim} V-\operatorname{dim} W+\operatorname{dim} W_{1}
$$
\begin{CJK}{UTF8}{mj}六\end{CJK}、\begin{CJK}{UTF8}{mj}解答下列各题\end{CJK}. 1. ( 10 \begin{CJK}{UTF8}{mj}分\end{CJK}) \begin{CJK}{UTF8}{mj}设\end{CJK} $A$ \begin{CJK}{UTF8}{mj}是\end{CJK} $n$ \begin{CJK}{UTF8}{mj}阶实对称矩阵\end{CJK}. \begin{CJK}{UTF8}{mj}证明\end{CJK}: $A$ \begin{CJK}{UTF8}{mj}是半正定矩阵当且仅当对任意\end{CJK} $n$ \begin{CJK}{UTF8}{mj}阶半正定矩阵\end{CJK} $B$ \begin{CJK}{UTF8}{mj}都有\end{CJK} $\operatorname{tr}(\mathrm{AB}) \geq 0$, \begin{CJK}{UTF8}{mj}这里\end{CJK} $\operatorname{tr}$ \begin{CJK}{UTF8}{mj}表示矩阵的迹\end{CJK}.

\begin{enumerate}
  \setcounter{enumi}{2}
  \item ( 15 \begin{CJK}{UTF8}{mj}分\end{CJK}) \begin{CJK}{UTF8}{mj}设\end{CJK} $V$ \begin{CJK}{UTF8}{mj}是复数域\end{CJK} $\mathbb{C}$ \begin{CJK}{UTF8}{mj}上的有限维线性空间\end{CJK}, $H$ \begin{CJK}{UTF8}{mj}是由\end{CJK} $V$ \begin{CJK}{UTF8}{mj}上的两两可交换且可对角化的线性变换组成\end{CJK} \begin{CJK}{UTF8}{mj}的线性空间\end{CJK}. \begin{CJK}{UTF8}{mj}证明\end{CJK}: \begin{CJK}{UTF8}{mj}存在若干线性函数\end{CJK} $\alpha_{i}: H \rightarrow \mathbb{C}$ \begin{CJK}{UTF8}{mj}使得有如下的子空间直和分解\end{CJK}:
\end{enumerate}
$$
V=\bigoplus_{i=1}^{m} V_{i}
$$
\begin{CJK}{UTF8}{mj}其中\end{CJK} $V_{i}=\left\{v \in V \mid h(v)=\alpha_{i}(h) v, \forall h \in H\right\}$.

\section{5. 四川大学 2013 年研究生入学考试试题高等代数}
\begin{CJK}{UTF8}{mj}李扬\end{CJK}

\begin{CJK}{UTF8}{mj}微信公众号\end{CJK}: sxkyliyang

\begin{CJK}{UTF8}{mj}一\end{CJK}、 \begin{CJK}{UTF8}{mj}解答下列各题\end{CJK}.

\begin{enumerate}
  \item ( 5 \begin{CJK}{UTF8}{mj}分\end{CJK}) \begin{CJK}{UTF8}{mj}叙述代数基本定理\end{CJK}.

  \item (10 \begin{CJK}{UTF8}{mj}分\end{CJK}) \begin{CJK}{UTF8}{mj}写出实数域上的所有不可约多项式\end{CJK}, \begin{CJK}{UTF8}{mj}并说明理由\end{CJK}.

  \item ( 10 \begin{CJK}{UTF8}{mj}分\end{CJK}) \begin{CJK}{UTF8}{mj}设\end{CJK} $\alpha_{1}, \cdots, \alpha_{n}$ \begin{CJK}{UTF8}{mj}是数域\end{CJK} $F$ \begin{CJK}{UTF8}{mj}上的多项式\end{CJK} $f(x)$ \begin{CJK}{UTF8}{mj}的全部复根\end{CJK}. \begin{CJK}{UTF8}{mj}证明\end{CJK}: \begin{CJK}{UTF8}{mj}如果存在\end{CJK} $i \neq j$ \begin{CJK}{UTF8}{mj}使得\end{CJK} $\alpha_{i}=\alpha_{j}$, \begin{CJK}{UTF8}{mj}那\end{CJK} \begin{CJK}{UTF8}{mj}么\end{CJK} $f(x)$ \begin{CJK}{UTF8}{mj}在\end{CJK} $F$ \begin{CJK}{UTF8}{mj}上是可约的\end{CJK}.

  \item ( 5 \begin{CJK}{UTF8}{mj}分\end{CJK}) \begin{CJK}{UTF8}{mj}设\end{CJK} $A$ \begin{CJK}{UTF8}{mj}是数域\end{CJK} $F$ \begin{CJK}{UTF8}{mj}上的\end{CJK} $n$ \begin{CJK}{UTF8}{mj}阶方阵\end{CJK}, $V=\{f(A) \mid f(x) \in F[x]\}$. \begin{CJK}{UTF8}{mj}证明\end{CJK}: $\operatorname{dim} V=1$ \begin{CJK}{UTF8}{mj}当且仅当\end{CJK} $A$ \begin{CJK}{UTF8}{mj}是数量矩\end{CJK} \begin{CJK}{UTF8}{mj}阵\end{CJK}.

\end{enumerate}
\begin{CJK}{UTF8}{mj}二\end{CJK}、\begin{CJK}{UTF8}{mj}设\end{CJK} $A$ \begin{CJK}{UTF8}{mj}是数域\end{CJK} $F$ \begin{CJK}{UTF8}{mj}上的\end{CJK} $m \times n$ \begin{CJK}{UTF8}{mj}型矩阵\end{CJK}.

\begin{enumerate}
  \item (5 \begin{CJK}{UTF8}{mj}分\end{CJK}) \begin{CJK}{UTF8}{mj}设\end{CJK} $F$ \begin{CJK}{UTF8}{mj}是实数域\end{CJK} $\mathbb{R}$ \begin{CJK}{UTF8}{mj}且\end{CJK} $A$ \begin{CJK}{UTF8}{mj}的秩为\end{CJK} $r$. \begin{CJK}{UTF8}{mj}求线性空间\end{CJK} $V=\left\{X \in \mathbb{R}^{n} \mid A^{\prime} A X=0\right\}$ \begin{CJK}{UTF8}{mj}的维数\end{CJK}, \begin{CJK}{UTF8}{mj}这里\end{CJK}, $A^{\prime}$ \begin{CJK}{UTF8}{mj}表示\end{CJK} $A$ \begin{CJK}{UTF8}{mj}的转置\end{CJK}.

  \item ( 10 \begin{CJK}{UTF8}{mj}分\end{CJK}) \begin{CJK}{UTF8}{mj}设\end{CJK} $f: F^{n} \rightarrow F^{m}$ \begin{CJK}{UTF8}{mj}是映射\end{CJK} $X \mapsto A X$. \begin{CJK}{UTF8}{mj}证明\end{CJK}: $f$ \begin{CJK}{UTF8}{mj}是单射当且仅当\end{CJK} $A$ \begin{CJK}{UTF8}{mj}的列向量线性无关\end{CJK}; $f$ \begin{CJK}{UTF8}{mj}是满射当\end{CJK} \begin{CJK}{UTF8}{mj}且仅当\end{CJK} $A$ \begin{CJK}{UTF8}{mj}的行向量线性无关\end{CJK}.

  \item (15 \begin{CJK}{UTF8}{mj}分\end{CJK}) \begin{CJK}{UTF8}{mj}一个矩阵的秩定义为它的行向量组的秩\end{CJK}(\begin{CJK}{UTF8}{mj}与它的列向量组的秩相等\end{CJK}). \begin{CJK}{UTF8}{mj}证明如下的定理\end{CJK}: $A$ \begin{CJK}{UTF8}{mj}的秩为\end{CJK} $r$ \begin{CJK}{UTF8}{mj}当且仅当\end{CJK} $A$ \begin{CJK}{UTF8}{mj}至少有一个\end{CJK} $r$ \begin{CJK}{UTF8}{mj}阶子式不为\end{CJK} 0 , \begin{CJK}{UTF8}{mj}而所有\end{CJK} $r+1$ \begin{CJK}{UTF8}{mj}阶子式都为\end{CJK} 0 .

\end{enumerate}
\begin{CJK}{UTF8}{mj}三\end{CJK}、\begin{CJK}{UTF8}{mj}解答下列各题\end{CJK}.

\begin{enumerate}
  \item ( 10 \begin{CJK}{UTF8}{mj}分\end{CJK}) \begin{CJK}{UTF8}{mj}设\end{CJK} $F$ \begin{CJK}{UTF8}{mj}是数域\end{CJK}, $\alpha_{1}, \cdots, \alpha_{m} \in F^{n}$ \begin{CJK}{UTF8}{mj}是线性无关的列向量\end{CJK}, $A$ \begin{CJK}{UTF8}{mj}是\end{CJK} $F$ \begin{CJK}{UTF8}{mj}上的\end{CJK} $m$ \begin{CJK}{UTF8}{mj}阶方阵\end{CJK}. \begin{CJK}{UTF8}{mj}令\end{CJK}
\end{enumerate}
$$
\left(\beta_{1}, \cdots, \beta_{m}\right)=\left(\alpha_{1}, \cdots, \alpha_{m}\right) A .
$$
\begin{CJK}{UTF8}{mj}证明\end{CJK}: $\beta_{1}, \cdots, \beta_{m}$ \begin{CJK}{UTF8}{mj}生成的\end{CJK} $F^{n}$ \begin{CJK}{UTF8}{mj}的子空间\end{CJK} $V$ \begin{CJK}{UTF8}{mj}的维数\end{CJK} $\operatorname{dim} V$ \begin{CJK}{UTF8}{mj}等于\end{CJK} $A$ \begin{CJK}{UTF8}{mj}的秩\end{CJK} $r(A)$.

\begin{enumerate}
  \setcounter{enumi}{2}
  \item (10 \begin{CJK}{UTF8}{mj}分\end{CJK}) \begin{CJK}{UTF8}{mj}设\end{CJK} $A$ \begin{CJK}{UTF8}{mj}是\end{CJK} $m \times n$ \begin{CJK}{UTF8}{mj}型实矩阵\end{CJK}. \begin{CJK}{UTF8}{mj}证明\end{CJK}: \begin{CJK}{UTF8}{mj}存在\end{CJK} $n$ \begin{CJK}{UTF8}{mj}阶半正定\end{CJK} $B$ \begin{CJK}{UTF8}{mj}使得\end{CJK} $A^{\prime} A=B^{2}$, \begin{CJK}{UTF8}{mj}这里\end{CJK} $A^{\prime}$ \begin{CJK}{UTF8}{mj}表示\end{CJK} $A$ \begin{CJK}{UTF8}{mj}的转置\end{CJK}.

  \item ( 10 \begin{CJK}{UTF8}{mj}分\end{CJK}) \begin{CJK}{UTF8}{mj}设\end{CJK}

\end{enumerate}
$$
A=\left(\begin{array}{lll}
1 & 1 & 1 \\
1 & 1 & 1 \\
1 & 1 & 1
\end{array}\right)
$$
\begin{CJK}{UTF8}{mj}求一个正交矩阵\end{CJK} $T$ \begin{CJK}{UTF8}{mj}使得\end{CJK} $T^{-1} A T$ \begin{CJK}{UTF8}{mj}是对角阵\end{CJK}.

\begin{CJK}{UTF8}{mj}四\end{CJK}、\begin{CJK}{UTF8}{mj}设\end{CJK} $A$ \begin{CJK}{UTF8}{mj}是数域\end{CJK} $F$ \begin{CJK}{UTF8}{mj}上的\end{CJK} $n$ \begin{CJK}{UTF8}{mj}阶方阵\end{CJK}, $A$ \begin{CJK}{UTF8}{mj}的特征多项式在\end{CJK} $F$ \begin{CJK}{UTF8}{mj}上不可约\end{CJK}. \begin{CJK}{UTF8}{mj}设\end{CJK} $M_{n}(\mathbb{C})$ \begin{CJK}{UTF8}{mj}是所有\end{CJK} $n$ \begin{CJK}{UTF8}{mj}阶复方阵组成的线性空\end{CJK} \begin{CJK}{UTF8}{mj}间\end{CJK}。

\begin{enumerate}
  \item (5 \begin{CJK}{UTF8}{mj}分\end{CJK}) \begin{CJK}{UTF8}{mj}证明\end{CJK}: $A$ \begin{CJK}{UTF8}{mj}是可逆矩阵\end{CJK}.

  \item ( 15 \begin{CJK}{UTF8}{mj}分\end{CJK}) \begin{CJK}{UTF8}{mj}设\end{CJK} $\sigma_{A}$ \begin{CJK}{UTF8}{mj}是\end{CJK} $M_{n}(\mathbb{C})$ \begin{CJK}{UTF8}{mj}上的如下线性变换\end{CJK}:

\end{enumerate}
$$
\sigma_{A}(X)=A^{-1} X-X A^{-1}, X \in M_{n}(\mathbb{C}) .
$$
\begin{CJK}{UTF8}{mj}记\end{CJK} $\sigma_{A}$ \begin{CJK}{UTF8}{mj}的核为\end{CJK} $\operatorname{Ker} \sigma_{A}, \sigma_{A}$ \begin{CJK}{UTF8}{mj}的像为\end{CJK} $\operatorname{Im} \sigma_{A}$. \begin{CJK}{UTF8}{mj}求\end{CJK} $\operatorname{Ker} \sigma_{A} \cap \operatorname{Im} \sigma_{A}$ \begin{CJK}{UTF8}{mj}的维数\end{CJK}.

\begin{CJK}{UTF8}{mj}五\end{CJK}、 (20 \begin{CJK}{UTF8}{mj}分\end{CJK}) \begin{CJK}{UTF8}{mj}设\end{CJK} $A=\left(a_{i j}\right)$ \begin{CJK}{UTF8}{mj}是\end{CJK} $n$ \begin{CJK}{UTF8}{mj}阶复矩阵\end{CJK}. \begin{CJK}{UTF8}{mj}令\end{CJK} $\operatorname{Re}(A)=\left(b_{i j}\right)$ \begin{CJK}{UTF8}{mj}是\end{CJK} $n$ \begin{CJK}{UTF8}{mj}阶实矩阵\end{CJK}, \begin{CJK}{UTF8}{mj}其中\end{CJK} $b_{i j}$ \begin{CJK}{UTF8}{mj}是\end{CJK} $a_{i j}$ \begin{CJK}{UTF8}{mj}的实部\end{CJK}; \begin{CJK}{UTF8}{mj}令\end{CJK} $\operatorname{Im}(A)=\left(c_{i j}\right)$, \begin{CJK}{UTF8}{mj}其中\end{CJK} $c_{i j}$ \begin{CJK}{UTF8}{mj}是\end{CJK} $a_{i j}$ \begin{CJK}{UTF8}{mj}的虚部\end{CJK}, $1 \leq i, j \leq n$.

\begin{CJK}{UTF8}{mj}令\end{CJK}
$$
B=\left(\begin{array}{cc}
\operatorname{Re}(A) & -\operatorname{Im}(A) \\
\operatorname{Im}(A) & \operatorname{Re}(A)
\end{array}\right)
$$
\begin{CJK}{UTF8}{mj}证明\end{CJK}: $\operatorname{det}(B)=|\operatorname{det}(A)|^{2}$, \begin{CJK}{UTF8}{mj}这里\end{CJK}, det \begin{CJK}{UTF8}{mj}表示行列式\end{CJK}, $|\cdot|$ \begin{CJK}{UTF8}{mj}表示复数的模\end{CJK}. \begin{CJK}{UTF8}{mj}六\end{CJK}、\begin{CJK}{UTF8}{mj}设\end{CJK} $M_{n}(\mathbb{R})$ \begin{CJK}{UTF8}{mj}是所有\end{CJK} $n$ \begin{CJK}{UTF8}{mj}阶实方阵组成的线性空间\end{CJK},\begin{CJK}{UTF8}{mj}设\end{CJK} $(-,-)$ \begin{CJK}{UTF8}{mj}是\end{CJK} $M_{n}(\mathbb{R})$ \begin{CJK}{UTF8}{mj}上的双线性型\end{CJK}, \begin{CJK}{UTF8}{mj}其定义为\end{CJK}: \begin{CJK}{UTF8}{mj}对任意\end{CJK} $X, Y \in M_{n}(\mathbb{R}),(X, Y)=\operatorname{tr}(\mathrm{XY})$, \begin{CJK}{UTF8}{mj}其中\end{CJK} $\operatorname{tr}$ \begin{CJK}{UTF8}{mj}表示方阵的迹\end{CJK}.

\begin{enumerate}
  \item ( 10 \begin{CJK}{UTF8}{mj}分\end{CJK}) \begin{CJK}{UTF8}{mj}任取\end{CJK} $M_{n}(\mathbb{R})$ \begin{CJK}{UTF8}{mj}的一组基\end{CJK}: $A_{1}, \cdots, A_{n^{2}}$. \begin{CJK}{UTF8}{mj}证明\end{CJK}: \begin{CJK}{UTF8}{mj}存在\end{CJK} $M_{n}(\mathbb{R})$ \begin{CJK}{UTF8}{mj}的唯一的一组基\end{CJK}: $B_{1}, \cdots, B_{n^{2}}$ \begin{CJK}{UTF8}{mj}使得\end{CJK}
\end{enumerate}
$$
\left(A_{i}, B_{j}\right)=\delta_{i j}= \begin{cases}1, & i=j \\ 0, & i \neq j\end{cases}
$$
\begin{CJK}{UTF8}{mj}称\end{CJK} $B_{1}, \cdots, B_{n^{2}}$ \begin{CJK}{UTF8}{mj}是\end{CJK} $A_{1}, \cdots, A_{n^{2}}$ \begin{CJK}{UTF8}{mj}的对偶基\end{CJK}.

\begin{enumerate}
  \setcounter{enumi}{2}
  \item ( 10 \begin{CJK}{UTF8}{mj}分\end{CJK}) \begin{CJK}{UTF8}{mj}对于\end{CJK} $M_{n}(\mathbb{R})$ \begin{CJK}{UTF8}{mj}的基\end{CJK} $A_{1}, \cdots, A_{n^{2}}$ \begin{CJK}{UTF8}{mj}和基\end{CJK} $A_{1}^{\prime}, \cdots, A_{n^{2}}^{\prime}$, \begin{CJK}{UTF8}{mj}设它们的由\end{CJK} 1 \begin{CJK}{UTF8}{mj}给出的对偶基分别为\end{CJK} $B_{1}, \cdots, B_{n^{2}}$ \begin{CJK}{UTF8}{mj}和\end{CJK} $B_{1}^{\prime}, \cdots, B_{n^{2}}^{\prime}$, \begin{CJK}{UTF8}{mj}证明\end{CJK}: $\sum_{i=1}^{n^{2}} A_{i} B_{i}=\sum_{i=1}^{n^{2}} A_{i}^{\prime} B_{i}^{\prime}$.
\end{enumerate}
\section{6. 四川大学 2014 年研究生入学考试试题高等代数 
 李扬 
 微信公众号: sxkyliyang}
\begin{CJK}{UTF8}{mj}一\end{CJK}、 \begin{CJK}{UTF8}{mj}解答下列各题\end{CJK}.

\begin{enumerate}
  \item (10 \begin{CJK}{UTF8}{mj}分\end{CJK}) \begin{CJK}{UTF8}{mj}设\end{CJK} $F$ \begin{CJK}{UTF8}{mj}是数域\end{CJK}, $a_{1}, a_{2}, \cdots, a_{n+1} \in F$ \begin{CJK}{UTF8}{mj}互不相同\end{CJK}. \begin{CJK}{UTF8}{mj}证明如下的结论\end{CJK}: \begin{CJK}{UTF8}{mj}对任意\end{CJK} $b_{1}, b_{2}, \cdots, b_{n+1} \in F$, \begin{CJK}{UTF8}{mj}存在唯一的\end{CJK} $F$ \begin{CJK}{UTF8}{mj}上的次数不超过\end{CJK} $n$ \begin{CJK}{UTF8}{mj}的多项式\end{CJK} $f(x)$ \begin{CJK}{UTF8}{mj}使得\end{CJK} $f\left(a_{i}\right)=b_{i}, 1 \leq i \leq n+1$.

  \item ( 10 \begin{CJK}{UTF8}{mj}分\end{CJK}) \begin{CJK}{UTF8}{mj}设\end{CJK} $A$ \begin{CJK}{UTF8}{mj}是数域\end{CJK} $F$ \begin{CJK}{UTF8}{mj}上的\end{CJK} $n$ \begin{CJK}{UTF8}{mj}阶方阵\end{CJK}, \begin{CJK}{UTF8}{mj}其全部复特征值为\end{CJK} $\lambda_{1}, \lambda_{2}, \cdots, \lambda_{n}$. \begin{CJK}{UTF8}{mj}设\end{CJK} $g(x)$ \begin{CJK}{UTF8}{mj}是\end{CJK} $F$ \begin{CJK}{UTF8}{mj}上的多\end{CJK} \begin{CJK}{UTF8}{mj}项式\end{CJK}, \begin{CJK}{UTF8}{mj}满足\end{CJK} $g\left(\lambda_{i}\right) \neq 0,1 \leq i \leq n$. \begin{CJK}{UTF8}{mj}证明\end{CJK}: $g(A)$ \begin{CJK}{UTF8}{mj}可逆\end{CJK}, \begin{CJK}{UTF8}{mj}且存在\end{CJK} $F$ \begin{CJK}{UTF8}{mj}上的次数小于\end{CJK} $n$ \begin{CJK}{UTF8}{mj}的多项式\end{CJK} $h(x)$ \begin{CJK}{UTF8}{mj}使得\end{CJK} $(g(A))^{-1}=h(A) .$

  \item ( 5 \begin{CJK}{UTF8}{mj}分\end{CJK}) \begin{CJK}{UTF8}{mj}设\end{CJK} $u(x), v(x)$ \begin{CJK}{UTF8}{mj}分别是\end{CJK} $n, m$ \begin{CJK}{UTF8}{mj}次整系数多项式\end{CJK} $(n>m \geq 1)$, \begin{CJK}{UTF8}{mj}且\end{CJK} $v(x)$ \begin{CJK}{UTF8}{mj}的首项系数为\end{CJK} 1 . \begin{CJK}{UTF8}{mj}证明\end{CJK}: \begin{CJK}{UTF8}{mj}存在唯\end{CJK} \begin{CJK}{UTF8}{mj}一的整系数多项式\end{CJK} $q(x), r(x)$ \begin{CJK}{UTF8}{mj}使得\end{CJK} $u(x)=q(x) v(x)+r(x)$, \begin{CJK}{UTF8}{mj}其中\end{CJK}, $r(x)=0$ \begin{CJK}{UTF8}{mj}或\end{CJK} $r(x)$ \begin{CJK}{UTF8}{mj}的次数小于\end{CJK} $m$.

\end{enumerate}
\begin{CJK}{UTF8}{mj}二\end{CJK}、\begin{CJK}{UTF8}{mj}设\end{CJK} $M_{n}$ \begin{CJK}{UTF8}{mj}是复数域\end{CJK} $\mathbb{C}$ \begin{CJK}{UTF8}{mj}上的所有\end{CJK} $n$ \begin{CJK}{UTF8}{mj}阶方阵组成的集合\end{CJK}. \begin{CJK}{UTF8}{mj}用\end{CJK} $r(B)$ \begin{CJK}{UTF8}{mj}表示任意矩阵\end{CJK} $B$ \begin{CJK}{UTF8}{mj}的秩\end{CJK}. \begin{CJK}{UTF8}{mj}设\end{CJK} $A \in M_{n}$.

\begin{enumerate}
  \item ( 5 \begin{CJK}{UTF8}{mj}分\end{CJK}) \begin{CJK}{UTF8}{mj}设\end{CJK} $\alpha_{i}$ \begin{CJK}{UTF8}{mj}是\end{CJK} $A$ \begin{CJK}{UTF8}{mj}的第\end{CJK} $i$ \begin{CJK}{UTF8}{mj}个列向量\end{CJK}; $A_{i}$ \begin{CJK}{UTF8}{mj}是\end{CJK} $A$ \begin{CJK}{UTF8}{mj}划去的第\end{CJK} $i$ \begin{CJK}{UTF8}{mj}列而得到的矩阵\end{CJK} $(1 \leq i \leq n)$. \begin{CJK}{UTF8}{mj}证明\end{CJK}: $A$ \begin{CJK}{UTF8}{mj}不可逆当\end{CJK} \begin{CJK}{UTF8}{mj}且仅当存在\end{CJK} $j$ \begin{CJK}{UTF8}{mj}使得方程组\end{CJK} $A_{j} X=\alpha_{j}$ \begin{CJK}{UTF8}{mj}有解\end{CJK}.
\end{enumerate}
2 . $(20$ \begin{CJK}{UTF8}{mj}分\end{CJK} $)$ \begin{CJK}{UTF8}{mj}设\end{CJK} $r(A)=r\left(A^{2}\right)=r$. \begin{CJK}{UTF8}{mj}设\end{CJK}
$$
U=\left\{Y \in M_{n} \mid \text { 存在 } X \in M_{n} \text { 使得 } A X=Y\right\}, V=\left\{Z \in M_{n} \mid A Z=0\right\} \text {. }
$$
\begin{CJK}{UTF8}{mj}分别求线性空间\end{CJK} $U, V$ \begin{CJK}{UTF8}{mj}的维数\end{CJK} $\operatorname{dim} U$ \begin{CJK}{UTF8}{mj}和\end{CJK} $\operatorname{dim} V$. \begin{CJK}{UTF8}{mj}问\end{CJK}: $M_{n}=U \oplus V$ \begin{CJK}{UTF8}{mj}是否成立\end{CJK}? \begin{CJK}{UTF8}{mj}说明理由\end{CJK}.

\begin{CJK}{UTF8}{mj}三\end{CJK}、\begin{CJK}{UTF8}{mj}设数域\end{CJK} $F$ \begin{CJK}{UTF8}{mj}上的\end{CJK} 4 \begin{CJK}{UTF8}{mj}阶方阵\end{CJK} $A$ \begin{CJK}{UTF8}{mj}的特征多项式为\end{CJK} $f(x)=\operatorname{det}\left(x E_{4}-A\right)=x^{4}+2 x^{3}-3 x^{2}-4 x+4$, \begin{CJK}{UTF8}{mj}其中\end{CJK} $E_{4}$ \begin{CJK}{UTF8}{mj}是单位\end{CJK} \begin{CJK}{UTF8}{mj}阵\end{CJK}.

\begin{enumerate}
  \item (10 \begin{CJK}{UTF8}{mj}分\end{CJK}) \begin{CJK}{UTF8}{mj}对满足上述条件的矩阵\end{CJK} $A$ \begin{CJK}{UTF8}{mj}在相似意义下进行分类\end{CJK}, \begin{CJK}{UTF8}{mj}可以分为几类\end{CJK}? \begin{CJK}{UTF8}{mj}说明理由\end{CJK}.

  \item (10 \begin{CJK}{UTF8}{mj}分\end{CJK}) \begin{CJK}{UTF8}{mj}证明\end{CJK}: 4 \begin{CJK}{UTF8}{mj}维向量空间\end{CJK} $F^{4}$ \begin{CJK}{UTF8}{mj}不能分解为\end{CJK} $A$ \begin{CJK}{UTF8}{mj}的特征子空间的直和当且仅当\end{CJK} $A^{2}, A, E_{4}$ \begin{CJK}{UTF8}{mj}线性无关\end{CJK}.

  \item (5 \begin{CJK}{UTF8}{mj}分\end{CJK}) \begin{CJK}{UTF8}{mj}当\end{CJK} $A^{2}+a A+b E_{4}=0$ \begin{CJK}{UTF8}{mj}时求\end{CJK} $a, b$ \begin{CJK}{UTF8}{mj}的值\end{CJK}.

\end{enumerate}
\begin{CJK}{UTF8}{mj}四\end{CJK}、\begin{CJK}{UTF8}{mj}设\end{CJK} $f: V \times V \rightarrow F$ \begin{CJK}{UTF8}{mj}是数域\end{CJK} $F$ \begin{CJK}{UTF8}{mj}上的线性空间\end{CJK} $V$ \begin{CJK}{UTF8}{mj}上的对称双线性型\end{CJK}, $\operatorname{dim} V=n$.

\begin{enumerate}
  \item ( 10 \begin{CJK}{UTF8}{mj}分\end{CJK}) \begin{CJK}{UTF8}{mj}证明\end{CJK}: \begin{CJK}{UTF8}{mj}存在\end{CJK} $V$ \begin{CJK}{UTF8}{mj}的基\end{CJK} $\alpha_{1}, \alpha_{2}, \cdots, \alpha_{n}$ \begin{CJK}{UTF8}{mj}使得当\end{CJK} $i \neq j$ \begin{CJK}{UTF8}{mj}时有\end{CJK} $f\left(\alpha_{i}, \alpha_{j}\right)=0$.

  \item ( 15 \begin{CJK}{UTF8}{mj}分\end{CJK}) \begin{CJK}{UTF8}{mj}设\end{CJK} $f$ \begin{CJK}{UTF8}{mj}非退化\end{CJK}, \begin{CJK}{UTF8}{mj}即对任意\end{CJK} $0 \neq \alpha \in V$ \begin{CJK}{UTF8}{mj}都存在\end{CJK} $\beta \in V$ \begin{CJK}{UTF8}{mj}使得\end{CJK} $f(\alpha, \beta) \neq 0$. \begin{CJK}{UTF8}{mj}设\end{CJK} $\mathscr{A}$ \begin{CJK}{UTF8}{mj}是\end{CJK} $V$ \begin{CJK}{UTF8}{mj}上的线性变换\end{CJK}, \begin{CJK}{UTF8}{mj}满足\end{CJK}: \begin{CJK}{UTF8}{mj}对任意\end{CJK} $\alpha, \beta \in V$ \begin{CJK}{UTF8}{mj}都有\end{CJK} $f(\mathscr{A} \alpha, \mathscr{A} \beta)=f(\alpha, \beta)$. \begin{CJK}{UTF8}{mj}证明\end{CJK}: $\mathscr{A}$ \begin{CJK}{UTF8}{mj}是可逆的\end{CJK}, \begin{CJK}{UTF8}{mj}并求出\end{CJK} $\mathscr{A}$ \begin{CJK}{UTF8}{mj}在\end{CJK} $V$ \begin{CJK}{UTF8}{mj}的基下的矩阵的\end{CJK}

  \item (5 \begin{CJK}{UTF8}{mj}分\end{CJK}) \begin{CJK}{UTF8}{mj}设\end{CJK} $A$ \begin{CJK}{UTF8}{mj}可逆\end{CJK}, $B=0$ \begin{CJK}{UTF8}{mj}且\end{CJK} (*) \begin{CJK}{UTF8}{mj}有一个解是可逆矩阵\end{CJK}. \begin{CJK}{UTF8}{mj}证明\end{CJK}: $n$ \begin{CJK}{UTF8}{mj}必然是偶数\end{CJK}.

  \item ( 10 \begin{CJK}{UTF8}{mj}分\end{CJK}) \begin{CJK}{UTF8}{mj}设\end{CJK} $B \neq 0$ \begin{CJK}{UTF8}{mj}且\end{CJK} $(\star)$ \begin{CJK}{UTF8}{mj}有解\end{CJK}. \begin{CJK}{UTF8}{mj}证明\end{CJK}: \begin{CJK}{UTF8}{mj}存在\end{CJK} $(\star)$ \begin{CJK}{UTF8}{mj}的解\end{CJK} $X_{1}, X_{2}, \cdots, X_{s}$ \begin{CJK}{UTF8}{mj}使得对\end{CJK} $(\star)$ \begin{CJK}{UTF8}{mj}的任意解\end{CJK} $X$ \begin{CJK}{UTF8}{mj}都有\end{CJK}:

\end{enumerate}
\begin{CJK}{UTF8}{mj}六\end{CJK}、\begin{CJK}{UTF8}{mj}设\end{CJK} $V$ \begin{CJK}{UTF8}{mj}是数域\end{CJK} $F$ \begin{CJK}{UTF8}{mj}上的\end{CJK} $n$ \begin{CJK}{UTF8}{mj}维线性空间\end{CJK}, $\operatorname{End}(V)$ \begin{CJK}{UTF8}{mj}是\end{CJK} $V$ \begin{CJK}{UTF8}{mj}的所有线性变换组成的集合\end{CJK}.

\section{7. 四川大学 2015 年研究生入学考试试题高等代数}
\begin{CJK}{UTF8}{mj}李扬\end{CJK}

\begin{CJK}{UTF8}{mj}微信公众号\end{CJK}: sxkyliyang

\begin{CJK}{UTF8}{mj}一\end{CJK}、\begin{CJK}{UTF8}{mj}设\end{CJK}
$$
A=\left(\begin{array}{ccc}
\frac{1}{2} & -\frac{5}{2} & \frac{1}{2} \\
\frac{1}{2} & -\frac{1}{2} & \frac{1}{2} \\
0 & 2 & 0
\end{array}\right) .
$$

\begin{enumerate}
  \item ( 10 \begin{CJK}{UTF8}{mj}分\end{CJK}) \begin{CJK}{UTF8}{mj}求\end{CJK} $A$ \begin{CJK}{UTF8}{mj}的特征多项式\end{CJK}、\begin{CJK}{UTF8}{mj}极小多项式和\end{CJK} Jordan \begin{CJK}{UTF8}{mj}标准型\end{CJK}.

  \item (10 \begin{CJK}{UTF8}{mj}分\end{CJK}) \begin{CJK}{UTF8}{mj}设\end{CJK} $V$ \begin{CJK}{UTF8}{mj}是所有与\end{CJK} $A$ \begin{CJK}{UTF8}{mj}可交换的三阶实方阵组成的线性空间\end{CJK}. \begin{CJK}{UTF8}{mj}求\end{CJK} $V$ \begin{CJK}{UTF8}{mj}的维数并写出它的一组基\end{CJK}.

  \item ( 5 \begin{CJK}{UTF8}{mj}分\end{CJK}) \begin{CJK}{UTF8}{mj}是否存在矩阵\end{CJK} $X$ \begin{CJK}{UTF8}{mj}使得\end{CJK} $X^{2}=A$ ? \begin{CJK}{UTF8}{mj}说明理由\end{CJK}.

\end{enumerate}
\begin{CJK}{UTF8}{mj}二\end{CJK}、\begin{CJK}{UTF8}{mj}设\end{CJK} $M_{n}$ \begin{CJK}{UTF8}{mj}是数域\end{CJK} $F$ \begin{CJK}{UTF8}{mj}上的所有\end{CJK} $n$ \begin{CJK}{UTF8}{mj}阶方阵组成的集合\end{CJK}, \begin{CJK}{UTF8}{mj}在\end{CJK} $M_{n}$ \begin{CJK}{UTF8}{mj}上定义二元运算\end{CJK} $[-,-]$ \begin{CJK}{UTF8}{mj}如下\end{CJK}: $[X, Y]=X Y-Y X$, \begin{CJK}{UTF8}{mj}任\end{CJK} \begin{CJK}{UTF8}{mj}意\end{CJK} $X, Y \in M_{n}$.

\begin{enumerate}
  \item (5 \begin{CJK}{UTF8}{mj}分\end{CJK}) \begin{CJK}{UTF8}{mj}证明\end{CJK}: \begin{CJK}{UTF8}{mj}对任意\end{CJK} $A, B, C \in M_{n}$ \begin{CJK}{UTF8}{mj}都有\end{CJK}
\end{enumerate}
$$
[A,[B, C]]+[B,[C, A]]+[C,[A, B]]=0
$$

\begin{enumerate}
  \setcounter{enumi}{2}
  \item (10 \begin{CJK}{UTF8}{mj}分\end{CJK}) \begin{CJK}{UTF8}{mj}设\end{CJK} $A \in M_{n}$ \begin{CJK}{UTF8}{mj}满足\end{CJK}: \begin{CJK}{UTF8}{mj}对任意可逆矩阵\end{CJK} $B \in M_{n}$ \begin{CJK}{UTF8}{mj}都有\end{CJK} $[A, B]=0$, \begin{CJK}{UTF8}{mj}证明\end{CJK}: $A$ \begin{CJK}{UTF8}{mj}是数量矩阵\end{CJK}.

  \item (10 \begin{CJK}{UTF8}{mj}分\end{CJK}) \begin{CJK}{UTF8}{mj}设\end{CJK} $A \in M_{n}$ \begin{CJK}{UTF8}{mj}满足\end{CJK}: \begin{CJK}{UTF8}{mj}存在\end{CJK} $B \in M_{n}$ \begin{CJK}{UTF8}{mj}使得\end{CJK} $[A, B]=A$, \begin{CJK}{UTF8}{mj}证明\end{CJK}: $A$ \begin{CJK}{UTF8}{mj}必然是幂零矩阵\end{CJK}, \begin{CJK}{UTF8}{mj}即存在正整数\end{CJK} $k$ \begin{CJK}{UTF8}{mj}使\end{CJK} \begin{CJK}{UTF8}{mj}得\end{CJK} $A^{k}=0$.

\end{enumerate}
\begin{CJK}{UTF8}{mj}三\end{CJK}、\begin{CJK}{UTF8}{mj}解答下列个题\end{CJK}.

\begin{enumerate}
  \item (15 \begin{CJK}{UTF8}{mj}分\end{CJK}) \begin{CJK}{UTF8}{mj}分别举出满足如下条件的二阶实方阵的例子\end{CJK}:
\end{enumerate}
(1) $A$ \begin{CJK}{UTF8}{mj}与\end{CJK} $B$ \begin{CJK}{UTF8}{mj}在实数域上是相似的\end{CJK}, \begin{CJK}{UTF8}{mj}但在实数域上不是合同的\end{CJK}.

(2) $A$ \begin{CJK}{UTF8}{mj}与\end{CJK} $B$ \begin{CJK}{UTF8}{mj}在实数域上是合同的\end{CJK}, \begin{CJK}{UTF8}{mj}但在实数域上不是相似的\end{CJK}.

(3) $A, B$ \begin{CJK}{UTF8}{mj}都不是正交阵\end{CJK}, \begin{CJK}{UTF8}{mj}但\end{CJK} $A B$ \begin{CJK}{UTF8}{mj}是不为单位阵的正交阵\end{CJK}.

\begin{enumerate}
  \setcounter{enumi}{2}
  \item (10 \begin{CJK}{UTF8}{mj}分\end{CJK}) \begin{CJK}{UTF8}{mj}证明如下的定理\end{CJK}: \begin{CJK}{UTF8}{mj}设\end{CJK} $A$ \begin{CJK}{UTF8}{mj}是实对称矩阵\end{CJK}, \begin{CJK}{UTF8}{mj}则\end{CJK} $A$ \begin{CJK}{UTF8}{mj}是正定矩阵的充要条件是\end{CJK} $A$ \begin{CJK}{UTF8}{mj}的顺序主子式全部大\end{CJK} \begin{CJK}{UTF8}{mj}于\end{CJK} 0 .
\end{enumerate}
\begin{CJK}{UTF8}{mj}四\end{CJK}、\begin{CJK}{UTF8}{mj}设\end{CJK}
$$
A=\left(\begin{array}{ccc}
7 & -2 & 0 \\
-2 & 6 & -2 \\
0 & -2 & 5
\end{array}\right)
$$

\begin{enumerate}
  \item (15 \begin{CJK}{UTF8}{mj}分\end{CJK}) \begin{CJK}{UTF8}{mj}设\end{CJK} $f(X)=X^{\prime} A X$ \begin{CJK}{UTF8}{mj}是二次型\end{CJK}. \begin{CJK}{UTF8}{mj}求一个正交变换\end{CJK} $X=T Y$ \begin{CJK}{UTF8}{mj}把\end{CJK} $f(X)$ \begin{CJK}{UTF8}{mj}化为标准型\end{CJK}.

  \item (10 \begin{CJK}{UTF8}{mj}分\end{CJK}) \begin{CJK}{UTF8}{mj}在实向量空间\end{CJK} $\mathbb{R}^{3}$ \begin{CJK}{UTF8}{mj}上定义二元函数\end{CJK} $(,$, \begin{CJK}{UTF8}{mj}为\end{CJK}: $(X, Y)=X^{\prime} A Y, X, Y \in \mathbb{R}^{3}$. \begin{CJK}{UTF8}{mj}证明\end{CJK}: (, ) \begin{CJK}{UTF8}{mj}是\end{CJK} $\mathbb{R}^{3}$ \begin{CJK}{UTF8}{mj}上的一个内积\end{CJK}, \begin{CJK}{UTF8}{mj}并写出\end{CJK} $\mathbb{R}^{3}$ \begin{CJK}{UTF8}{mj}的一个关于这个内积的一组标准正交基\end{CJK}.

\end{enumerate}
\begin{CJK}{UTF8}{mj}五\end{CJK}、\begin{CJK}{UTF8}{mj}设\end{CJK} $V$ \begin{CJK}{UTF8}{mj}是数域\end{CJK} $F$ \begin{CJK}{UTF8}{mj}上的线性空间\end{CJK}, $V^{\star}$ \begin{CJK}{UTF8}{mj}是\end{CJK} $V$ \begin{CJK}{UTF8}{mj}的对偶空间\end{CJK}, $T$ \begin{CJK}{UTF8}{mj}是\end{CJK} $V$ \begin{CJK}{UTF8}{mj}上的线性变换\end{CJK}.

\begin{enumerate}
  \item ( 10 \begin{CJK}{UTF8}{mj}分\end{CJK}) \begin{CJK}{UTF8}{mj}对任意\end{CJK} $f \in V^{\star}$, \begin{CJK}{UTF8}{mj}定义\end{CJK} $V$ \begin{CJK}{UTF8}{mj}上的函数\end{CJK} $f^{\star}: V \rightarrow F$ \begin{CJK}{UTF8}{mj}为\end{CJK}: $f^{\star}(\alpha)=f(T(\alpha)), \alpha \in V$. \begin{CJK}{UTF8}{mj}证明\end{CJK}: \begin{CJK}{UTF8}{mj}映射\end{CJK} $T^{\star}: f \mapsto f^{\star}$ \begin{CJK}{UTF8}{mj}是\end{CJK} $V^{\star}$ \begin{CJK}{UTF8}{mj}上的线性变换\end{CJK}.

  \item (10 \begin{CJK}{UTF8}{mj}分\end{CJK}) \begin{CJK}{UTF8}{mj}设\end{CJK} $\alpha_{1}, \alpha_{2}, \cdots, \alpha_{n}$ \begin{CJK}{UTF8}{mj}是\end{CJK} $V$ \begin{CJK}{UTF8}{mj}的一组基\end{CJK}, \begin{CJK}{UTF8}{mj}且\end{CJK} $T$ \begin{CJK}{UTF8}{mj}在这个基下的矩阵为\end{CJK} $A$. \begin{CJK}{UTF8}{mj}求\end{CJK} 1 \begin{CJK}{UTF8}{mj}中的线性变换\end{CJK} $T^{\star}$ \begin{CJK}{UTF8}{mj}在对偶\end{CJK} \begin{CJK}{UTF8}{mj}基\end{CJK} $\alpha_{1}^{\star}, \alpha_{2}^{\star}, \cdots, \alpha_{n}^{\star}$ \begin{CJK}{UTF8}{mj}下的矩阵\end{CJK}. \begin{CJK}{UTF8}{mj}六\end{CJK}、\begin{CJK}{UTF8}{mj}设\end{CJK} $V$ \begin{CJK}{UTF8}{mj}是数域\end{CJK} $F$ \begin{CJK}{UTF8}{mj}上的\end{CJK} $n$ \begin{CJK}{UTF8}{mj}维线性空间\end{CJK}, $T$ \begin{CJK}{UTF8}{mj}是\end{CJK} $V$ \begin{CJK}{UTF8}{mj}上的线性变换\end{CJK}, $\lambda_{1}, \lambda_{2}, \cdots, \lambda_{n} \in F$ \begin{CJK}{UTF8}{mj}互不相同\end{CJK}, \begin{CJK}{UTF8}{mj}且都不是\end{CJK} $T$ \begin{CJK}{UTF8}{mj}的特\end{CJK} \begin{CJK}{UTF8}{mj}征值\end{CJK}; $I$ \begin{CJK}{UTF8}{mj}是\end{CJK} $V$ \begin{CJK}{UTF8}{mj}上的恒等变换\end{CJK}.

  \item ( 5 \begin{CJK}{UTF8}{mj}分\end{CJK}) \begin{CJK}{UTF8}{mj}证明\end{CJK}: \begin{CJK}{UTF8}{mj}对每个\end{CJK} $1 \leq i \leq n, T-\lambda_{i} I$ \begin{CJK}{UTF8}{mj}都是\end{CJK} $V$ \begin{CJK}{UTF8}{mj}上的可逆线性变换\end{CJK}.

  \item ( 20 \begin{CJK}{UTF8}{mj}分\end{CJK}) \begin{CJK}{UTF8}{mj}证明\end{CJK}: \begin{CJK}{UTF8}{mj}存在\end{CJK} $a_{1}, a_{2}, \cdots, a_{n} \in F$ \begin{CJK}{UTF8}{mj}使得\end{CJK}

\end{enumerate}
$$
\sum_{k=1}^{n} a_{k}\left(T-\lambda_{k} I\right)^{-1}=I
$$

\section{8. 四川大学 2016 年研究生入学考试试题高等代数}
\begin{CJK}{UTF8}{mj}李扬\end{CJK}

\begin{CJK}{UTF8}{mj}微信公众号\end{CJK}: sxkyliyang

\begin{CJK}{UTF8}{mj}一\end{CJK}、\begin{CJK}{UTF8}{mj}解答下列各题\end{CJK}.

\begin{enumerate}
  \item (10 \begin{CJK}{UTF8}{mj}分\end{CJK}) \begin{CJK}{UTF8}{mj}证明如下的定理\end{CJK}(Eisenstein \begin{CJK}{UTF8}{mj}判别法\end{CJK}): \begin{CJK}{UTF8}{mj}设\end{CJK} $f(x)=\sum_{i=0}^{n} a_{i} x^{i}$ \begin{CJK}{UTF8}{mj}是正次数的整系数多项式\end{CJK}, \begin{CJK}{UTF8}{mj}如果存在\end{CJK} \begin{CJK}{UTF8}{mj}满足如下条件的素数\end{CJK} $p$ :
\end{enumerate}
i. $p \nmid a_{n}$;

ii. $p \mid a_{i}(0 \leq i \leq n-1)$;

iii. $p^{2} \nmid a_{0}$,

\begin{CJK}{UTF8}{mj}则\end{CJK} $f(x)$ \begin{CJK}{UTF8}{mj}在有理数域上不可约\end{CJK}.

\begin{enumerate}
  \setcounter{enumi}{2}
  \item ( 5 \begin{CJK}{UTF8}{mj}分\end{CJK}) \begin{CJK}{UTF8}{mj}设\end{CJK} $p$ \begin{CJK}{UTF8}{mj}是素数\end{CJK}, \begin{CJK}{UTF8}{mj}证明\end{CJK}: \begin{CJK}{UTF8}{mj}对任意正整数\end{CJK} $n \geq 2$, \begin{CJK}{UTF8}{mj}多项式\end{CJK} $x^{n}-p x+p$ \begin{CJK}{UTF8}{mj}的全部\end{CJK} $n$ \begin{CJK}{UTF8}{mj}个复根互不相同\end{CJK}.

  \item ( 5 \begin{CJK}{UTF8}{mj}分\end{CJK}) \begin{CJK}{UTF8}{mj}设\end{CJK} $W$ \begin{CJK}{UTF8}{mj}是数域\end{CJK} $F$ \begin{CJK}{UTF8}{mj}上的\end{CJK} $n$ \begin{CJK}{UTF8}{mj}元三次齐次对称多项式和零多项式组成的线性空间\end{CJK}. \begin{CJK}{UTF8}{mj}求\end{CJK} $\operatorname{dim} W$ \begin{CJK}{UTF8}{mj}并写出它\end{CJK} \begin{CJK}{UTF8}{mj}的一组基\end{CJK}.

  \item ( 5 \begin{CJK}{UTF8}{mj}分\end{CJK}) \begin{CJK}{UTF8}{mj}设\end{CJK} $F$ \begin{CJK}{UTF8}{mj}是数域\end{CJK}, $F[x]$ \begin{CJK}{UTF8}{mj}是\end{CJK} $F$ \begin{CJK}{UTF8}{mj}上的一元多项式组成的集合\end{CJK}, $n$ \begin{CJK}{UTF8}{mj}是正整数\end{CJK}. \begin{CJK}{UTF8}{mj}设\end{CJK} $a \in F$, \begin{CJK}{UTF8}{mj}记\end{CJK}

\end{enumerate}
$$
V=\{f(x) \in F[x] \mid \partial(f(x)) \leq n, f(a)=0\} \cup\{0\}
$$
\begin{CJK}{UTF8}{mj}求\end{CJK} $\operatorname{dim} V$ \begin{CJK}{UTF8}{mj}并写出它的一组基\end{CJK}, \begin{CJK}{UTF8}{mj}这里\end{CJK}, $\partial(f(x))$ \begin{CJK}{UTF8}{mj}表示多项式\end{CJK} $f(x)$ \begin{CJK}{UTF8}{mj}的次数\end{CJK}.

\begin{CJK}{UTF8}{mj}二\end{CJK}、\begin{CJK}{UTF8}{mj}设\end{CJK} $A, B$ \begin{CJK}{UTF8}{mj}是数域\end{CJK} $F$ \begin{CJK}{UTF8}{mj}上的\end{CJK} $m \times n$ \begin{CJK}{UTF8}{mj}型矩阵\end{CJK}.

\begin{enumerate}
  \item (10 \begin{CJK}{UTF8}{mj}分\end{CJK}) \begin{CJK}{UTF8}{mj}证明\end{CJK}: \begin{CJK}{UTF8}{mj}齐次线性方程组\end{CJK} $A X=0$ \begin{CJK}{UTF8}{mj}与\end{CJK} $B X=0$ \begin{CJK}{UTF8}{mj}同解的充分必要条件是\end{CJK} $A$ \begin{CJK}{UTF8}{mj}的行向量组与\end{CJK} $B$ \begin{CJK}{UTF8}{mj}的行向\end{CJK} \begin{CJK}{UTF8}{mj}量组等价\end{CJK}, \begin{CJK}{UTF8}{mj}即\end{CJK} $A$ \begin{CJK}{UTF8}{mj}的每个行向量都可由\end{CJK} $B$ \begin{CJK}{UTF8}{mj}的行向量组线性表出\end{CJK}, \begin{CJK}{UTF8}{mj}且\end{CJK} $B$ \begin{CJK}{UTF8}{mj}的每个行向量都可由\end{CJK} $A$ \begin{CJK}{UTF8}{mj}的行向量组\end{CJK} \begin{CJK}{UTF8}{mj}线性表出\end{CJK}.

  \item (5 \begin{CJK}{UTF8}{mj}分\end{CJK}) \begin{CJK}{UTF8}{mj}举例说明\end{CJK}: \begin{CJK}{UTF8}{mj}当\end{CJK} $A$ \begin{CJK}{UTF8}{mj}的列向量组与\end{CJK} $B$ \begin{CJK}{UTF8}{mj}的列向量组等价时\end{CJK}, \begin{CJK}{UTF8}{mj}齐次线性方程组\end{CJK} $A X=0$ \begin{CJK}{UTF8}{mj}与\end{CJK} $B X=0$ \begin{CJK}{UTF8}{mj}可以\end{CJK} \begin{CJK}{UTF8}{mj}不同解\end{CJK}.

  \item (10 \begin{CJK}{UTF8}{mj}分\end{CJK}) \begin{CJK}{UTF8}{mj}设\end{CJK} $B \neq 0$ \begin{CJK}{UTF8}{mj}且矩阵方程\end{CJK} $A Y=B$ \begin{CJK}{UTF8}{mj}有解\end{CJK}, \begin{CJK}{UTF8}{mj}其解集记为\end{CJK} $W$. \begin{CJK}{UTF8}{mj}证明\end{CJK}: \begin{CJK}{UTF8}{mj}存在\end{CJK} $F$ \begin{CJK}{UTF8}{mj}上的\end{CJK} $n$ \begin{CJK}{UTF8}{mj}阶方阵\end{CJK} $Y_{1}, Y_{2}, \cdots, Y_{s}$ \begin{CJK}{UTF8}{mj}使得对任意\end{CJK} $Y \in W$ \begin{CJK}{UTF8}{mj}都有\end{CJK} $Y=\sum_{i=1}^{s} a_{i} Y_{i}$ \begin{CJK}{UTF8}{mj}且\end{CJK} $\sum_{i=1}^{s} a_{i}=1$.

\end{enumerate}
\begin{CJK}{UTF8}{mj}三\end{CJK}、\begin{CJK}{UTF8}{mj}设\end{CJK} $A$ \begin{CJK}{UTF8}{mj}是数域\end{CJK} $F$ \begin{CJK}{UTF8}{mj}上的\end{CJK} $n$ \begin{CJK}{UTF8}{mj}阶方阵\end{CJK}, \begin{CJK}{UTF8}{mj}秩为\end{CJK} $r$.

\begin{enumerate}
  \item ( 10 \begin{CJK}{UTF8}{mj}分\end{CJK}) \begin{CJK}{UTF8}{mj}设\end{CJK} $A^{2}=2 A$. \begin{CJK}{UTF8}{mj}求行列式\end{CJK} $\operatorname{det}\left(A-3 I_{n}\right)$, \begin{CJK}{UTF8}{mj}其中\end{CJK} $I_{n}$ \begin{CJK}{UTF8}{mj}是\end{CJK} $n$ \begin{CJK}{UTF8}{mj}阶单位阵\end{CJK}.

  \item ( 15 \begin{CJK}{UTF8}{mj}分\end{CJK}) \begin{CJK}{UTF8}{mj}证明\end{CJK}: $A^{2}=0$ \begin{CJK}{UTF8}{mj}当且仅当\end{CJK} $A$ \begin{CJK}{UTF8}{mj}在\end{CJK} $F$ \begin{CJK}{UTF8}{mj}上相似于分块矩阵\end{CJK} $\left(\begin{array}{cc}0_{r} & 0 \\ I_{r} & 0 \\ 0 & 0\end{array}\right)$, \begin{CJK}{UTF8}{mj}其中\end{CJK}, $0_{r}, I_{r}$ \begin{CJK}{UTF8}{mj}分别是\end{CJK} $r$ \begin{CJK}{UTF8}{mj}阶零矩\end{CJK} \begin{CJK}{UTF8}{mj}阵和\end{CJK} $r$ \begin{CJK}{UTF8}{mj}阶单位阵\end{CJK}.

  \item (10 \begin{CJK}{UTF8}{mj}分\end{CJK}) \begin{CJK}{UTF8}{mj}设\end{CJK} $\lambda_{1}, \lambda_{2}, \cdots, \lambda_{k} \in F$ \begin{CJK}{UTF8}{mj}是\end{CJK} $\mathscr{A}$ \begin{CJK}{UTF8}{mj}的互不相同的特征值\end{CJK}, $\alpha_{1}, \alpha_{2}, \cdots, \alpha_{k}$ \begin{CJK}{UTF8}{mj}分别是\end{CJK} $\mathscr{A}$ \begin{CJK}{UTF8}{mj}的属于\end{CJK} $\lambda_{1}, \lambda_{2}, \cdots, \lambda_{k}$ \begin{CJK}{UTF8}{mj}的特征向量\end{CJK}. \begin{CJK}{UTF8}{mj}证明\end{CJK}: \begin{CJK}{UTF8}{mj}如果\end{CJK} $\mathscr{A}$ \begin{CJK}{UTF8}{mj}的一个不变子空间\end{CJK} $W$ \begin{CJK}{UTF8}{mj}满足\end{CJK} $\alpha_{1}+\cdots+\alpha_{k} \in W$, \begin{CJK}{UTF8}{mj}则\end{CJK} $\operatorname{dim} W \geq k .$

  \item (5 \begin{CJK}{UTF8}{mj}分\end{CJK}) \begin{CJK}{UTF8}{mj}设\end{CJK} $C(\mathscr{A})$ \begin{CJK}{UTF8}{mj}是\end{CJK} $V$ \begin{CJK}{UTF8}{mj}的所有与\end{CJK} $\mathscr{A}$ \begin{CJK}{UTF8}{mj}可交换的线性变换组成的集合\end{CJK}. \begin{CJK}{UTF8}{mj}求\end{CJK} $\mathscr{A}$ \begin{CJK}{UTF8}{mj}使得\end{CJK} $\operatorname{dim} C(\mathscr{A})$ \begin{CJK}{UTF8}{mj}最大\end{CJK}. 4. ( 5 \begin{CJK}{UTF8}{mj}分\end{CJK}) \begin{CJK}{UTF8}{mj}设\end{CJK} $F$ \begin{CJK}{UTF8}{mj}是复数域且\end{CJK} $\mathscr{A}$ \begin{CJK}{UTF8}{mj}可逆\end{CJK}. \begin{CJK}{UTF8}{mj}证明\end{CJK}: \begin{CJK}{UTF8}{mj}存在\end{CJK} $V$ \begin{CJK}{UTF8}{mj}上的线性变换\end{CJK} $\mathscr{B}$ \begin{CJK}{UTF8}{mj}使得\end{CJK} $\mathscr{B}^{2}=\mathscr{A}$.

\end{enumerate}
\begin{CJK}{UTF8}{mj}五\end{CJK}、\begin{CJK}{UTF8}{mj}设\end{CJK} $V$ \begin{CJK}{UTF8}{mj}是\end{CJK} $n$ \begin{CJK}{UTF8}{mj}维欧式空间\end{CJK}.

\begin{enumerate}
  \item (10 \begin{CJK}{UTF8}{mj}分\end{CJK}) \begin{CJK}{UTF8}{mj}设\end{CJK} $\mathscr{A}, \mathscr{B}$ \begin{CJK}{UTF8}{mj}是\end{CJK} $V$ \begin{CJK}{UTF8}{mj}上的对称变换\end{CJK}, \begin{CJK}{UTF8}{mj}满足\end{CJK} $\mathscr{A} \mathscr{B}=\mathscr{B} \mathscr{A}$. \begin{CJK}{UTF8}{mj}证明\end{CJK}: \begin{CJK}{UTF8}{mj}存在\end{CJK} $V$ \begin{CJK}{UTF8}{mj}的一组基\end{CJK}, \begin{CJK}{UTF8}{mj}使得\end{CJK} $\mathscr{A}, \mathscr{B}$ \begin{CJK}{UTF8}{mj}在这个基\end{CJK} \begin{CJK}{UTF8}{mj}下的矩阵都是对角阵\end{CJK}.

  \item (10 \begin{CJK}{UTF8}{mj}分\end{CJK}) \begin{CJK}{UTF8}{mj}证明\end{CJK}: $V$ \begin{CJK}{UTF8}{mj}上的任意正交变换\end{CJK} $\mathscr{T}$ \begin{CJK}{UTF8}{mj}的任意特征值\end{CJK} $\lambda$ \begin{CJK}{UTF8}{mj}都满足\end{CJK}: $|\lambda|=1$.

  \item ( 5 \begin{CJK}{UTF8}{mj}分\end{CJK}) \begin{CJK}{UTF8}{mj}是否存在\end{CJK} $V$ \begin{CJK}{UTF8}{mj}上的正交变换\end{CJK} $\mathscr{T}_{1}, \mathscr{T}_{2}$ \begin{CJK}{UTF8}{mj}使得\end{CJK} $\mathscr{T}_{1}+\mathscr{T}_{2}$ \begin{CJK}{UTF8}{mj}是\end{CJK} $V$ \begin{CJK}{UTF8}{mj}上的正交变换\end{CJK}? \begin{CJK}{UTF8}{mj}说明理由\end{CJK}.

\end{enumerate}
\begin{CJK}{UTF8}{mj}六\end{CJK}、\begin{CJK}{UTF8}{mj}设\end{CJK} $V$ \begin{CJK}{UTF8}{mj}是数域\end{CJK} $F$ \begin{CJK}{UTF8}{mj}上的\end{CJK} $n$ \begin{CJK}{UTF8}{mj}维线性空间\end{CJK}, $(,$, \begin{CJK}{UTF8}{mj}是\end{CJK} $V$ \begin{CJK}{UTF8}{mj}上的一个非退化的双线性型\end{CJK}, $V^{*}$ \begin{CJK}{UTF8}{mj}是\end{CJK} $V$ \begin{CJK}{UTF8}{mj}的对偶空间\end{CJK}.

\begin{enumerate}
  \item ( 5 \begin{CJK}{UTF8}{mj}分\end{CJK}) $V$ \begin{CJK}{UTF8}{mj}的子集\end{CJK} $W=\{\alpha \in V \mid(\alpha, \alpha)=0\}$ \begin{CJK}{UTF8}{mj}是否是\end{CJK} $V$ \begin{CJK}{UTF8}{mj}的子空间\end{CJK}? \begin{CJK}{UTF8}{mj}说明理由\end{CJK}.

  \item (20 \begin{CJK}{UTF8}{mj}分\end{CJK}) \begin{CJK}{UTF8}{mj}证明\end{CJK}: \begin{CJK}{UTF8}{mj}对任意\end{CJK} $f \in V^{*}$, \begin{CJK}{UTF8}{mj}都存在唯一的\end{CJK} $\alpha \in V$, \begin{CJK}{UTF8}{mj}使得\end{CJK} $f(\beta)=(\alpha, \beta)$ \begin{CJK}{UTF8}{mj}对任意\end{CJK} $\beta \in V$ \begin{CJK}{UTF8}{mj}都成立\end{CJK}.

\end{enumerate}
\section{9. 四川大学 2017 年研究生入学考试试题高等代数}
\begin{CJK}{UTF8}{mj}李扬\end{CJK}

\begin{CJK}{UTF8}{mj}微信公众号\end{CJK}: sxkyliyang

\begin{CJK}{UTF8}{mj}一\end{CJK}、\begin{CJK}{UTF8}{mj}设\end{CJK} $f_{n}(x)=\sum_{i=0}^{n-1} x^{i}, n \geq 1$ \begin{CJK}{UTF8}{mj}是整数\end{CJK}. \begin{CJK}{UTF8}{mj}解答下列各题\end{CJK}.

\begin{enumerate}
  \item ( 5 \begin{CJK}{UTF8}{mj}分\end{CJK}) \begin{CJK}{UTF8}{mj}设\end{CJK} $\alpha_{1}, \alpha_{2}, \cdots, \alpha_{n-1}$ \begin{CJK}{UTF8}{mj}是\end{CJK} $f_{n}(x)$ \begin{CJK}{UTF8}{mj}的全部复根\end{CJK} $(n \geq 2), k$ \begin{CJK}{UTF8}{mj}是正整数\end{CJK}. \begin{CJK}{UTF8}{mj}求\end{CJK} $\sum_{j=1}^{n-1} \alpha_{j}^{n k}$ \begin{CJK}{UTF8}{mj}的值\end{CJK}.

  \item (10 \begin{CJK}{UTF8}{mj}分\end{CJK}) \begin{CJK}{UTF8}{mj}给出\end{CJK} $f_{n}(x) \mid f_{m}(x)$ \begin{CJK}{UTF8}{mj}的一个充分必要条件\end{CJK}, \begin{CJK}{UTF8}{mj}并证明你的结论\end{CJK}.

  \item ( 5 \begin{CJK}{UTF8}{mj}分\end{CJK}) \begin{CJK}{UTF8}{mj}设\end{CJK} $A$ \begin{CJK}{UTF8}{mj}是数域\end{CJK} $F$ \begin{CJK}{UTF8}{mj}上的\end{CJK} $p$ \begin{CJK}{UTF8}{mj}阶方阵且存在\end{CJK} $q$ \begin{CJK}{UTF8}{mj}使得\end{CJK} $f_{q}(A)=0$. \begin{CJK}{UTF8}{mj}证明\end{CJK}: $A$ \begin{CJK}{UTF8}{mj}在复数域上可对角化\end{CJK}, \begin{CJK}{UTF8}{mj}即存在可\end{CJK} \begin{CJK}{UTF8}{mj}逆的复方阵\end{CJK} $B$ \begin{CJK}{UTF8}{mj}使得\end{CJK} $B^{-1} A B$ \begin{CJK}{UTF8}{mj}为对角阵\end{CJK}.

  \item (5 \begin{CJK}{UTF8}{mj}分\end{CJK}) \begin{CJK}{UTF8}{mj}设\end{CJK} $C$ \begin{CJK}{UTF8}{mj}是\end{CJK} $r$ \begin{CJK}{UTF8}{mj}阶方阵\end{CJK}, \begin{CJK}{UTF8}{mj}其\end{CJK} $(i, j)$-\begin{CJK}{UTF8}{mj}元为\end{CJK}: $\int_{0}^{1} f_{i}(x) f_{j}(x) \mathrm{d} x$. \begin{CJK}{UTF8}{mj}问\end{CJK} $C$ \begin{CJK}{UTF8}{mj}是否是正定阵\end{CJK}, \begin{CJK}{UTF8}{mj}说明理由\end{CJK}.

\end{enumerate}
\section{二、解答下列各题.}
\begin{enumerate}
  \item ( 15 \begin{CJK}{UTF8}{mj}分\end{CJK}) \begin{CJK}{UTF8}{mj}设\end{CJK} $A X=\beta$ \begin{CJK}{UTF8}{mj}是\end{CJK} 4 \begin{CJK}{UTF8}{mj}元线性方程组\end{CJK}, $A$ \begin{CJK}{UTF8}{mj}的秩为\end{CJK} 2 . \begin{CJK}{UTF8}{mj}已知\end{CJK} $A X=\beta$ \begin{CJK}{UTF8}{mj}有四个解\end{CJK} $\alpha_{1}, \alpha_{2}, \alpha_{3}, \alpha_{4}$, \begin{CJK}{UTF8}{mj}且满足\end{CJK}:
\end{enumerate}
$$
\left\{\begin{array}{l}
\alpha_{1}+\alpha_{2}=(1,5,4,-15)^{\prime} \\
\alpha_{2}+\alpha_{3}=(1,4,3,-13)^{\prime} \\
2 \alpha_{3}-\alpha_{4}=(2,3,0,-3)^{\prime}
\end{array}\right.
$$
\begin{CJK}{UTF8}{mj}其中\end{CJK}, $(a, b, c, d)^{\prime}$ \begin{CJK}{UTF8}{mj}表示行向量的转置\end{CJK}. \begin{CJK}{UTF8}{mj}求\end{CJK} $A X=\beta$ \begin{CJK}{UTF8}{mj}的通解\end{CJK}.

\begin{enumerate}
  \setcounter{enumi}{2}
  \item (10 \begin{CJK}{UTF8}{mj}分\end{CJK}) \begin{CJK}{UTF8}{mj}设数域\end{CJK} $F$ \begin{CJK}{UTF8}{mj}上的\end{CJK} $m \times n$ \begin{CJK}{UTF8}{mj}型矩阵\end{CJK} $M$ \begin{CJK}{UTF8}{mj}和\end{CJK} $p \times q$ \begin{CJK}{UTF8}{mj}型矩阵\end{CJK} $N$ \begin{CJK}{UTF8}{mj}的秩分别为\end{CJK} $n, p$. \begin{CJK}{UTF8}{mj}证明\end{CJK}: \begin{CJK}{UTF8}{mj}矩阵方程\end{CJK} $M Y N=0$ \begin{CJK}{UTF8}{mj}只有零解\end{CJK} $Y=0$.
\end{enumerate}
\begin{CJK}{UTF8}{mj}三\end{CJK}、\begin{CJK}{UTF8}{mj}设\end{CJK} $A$ \begin{CJK}{UTF8}{mj}是数域\end{CJK} $F$ \begin{CJK}{UTF8}{mj}上的特征值全为\end{CJK} 0 \begin{CJK}{UTF8}{mj}的三阶方阵\end{CJK}.

\begin{enumerate}
  \item (5 \begin{CJK}{UTF8}{mj}分\end{CJK}) \begin{CJK}{UTF8}{mj}写出\end{CJK} $A$ \begin{CJK}{UTF8}{mj}的所有可能的\end{CJK} Jordan \begin{CJK}{UTF8}{mj}标准型\end{CJK}.

  \item ( 5 \begin{CJK}{UTF8}{mj}分\end{CJK}) \begin{CJK}{UTF8}{mj}设\end{CJK} $F$ \begin{CJK}{UTF8}{mj}上的多项式\end{CJK} $f(x)$ \begin{CJK}{UTF8}{mj}满足\end{CJK} $f(0) \neq 0$. \begin{CJK}{UTF8}{mj}证明\end{CJK}: $f(A)$ \begin{CJK}{UTF8}{mj}可逆\end{CJK}, \begin{CJK}{UTF8}{mj}且\end{CJK} $(f(A))^{-1}$ \begin{CJK}{UTF8}{mj}是\end{CJK} $A$ \begin{CJK}{UTF8}{mj}的多项式\end{CJK}.

  \item $\left(15\right.$ \begin{CJK}{UTF8}{mj}分\end{CJK}) \begin{CJK}{UTF8}{mj}设\end{CJK} $g(x)=x^{11}-x^{5}-x^{4}+x^{3}+x-3$. \begin{CJK}{UTF8}{mj}求行列式\end{CJK} $\operatorname{det}(g(A))$.

\end{enumerate}
\begin{CJK}{UTF8}{mj}四\end{CJK}、\begin{CJK}{UTF8}{mj}设\end{CJK} $\mathscr{A}$ \begin{CJK}{UTF8}{mj}是数域\end{CJK} $F$ \begin{CJK}{UTF8}{mj}上的线性空间\end{CJK} $V$ \begin{CJK}{UTF8}{mj}上的线性变换\end{CJK}, $V$ \begin{CJK}{UTF8}{mj}可能是无穷维的\end{CJK}.

\begin{enumerate}
  \item (5 \begin{CJK}{UTF8}{mj}分\end{CJK}) \begin{CJK}{UTF8}{mj}如何利用\end{CJK} $\mathscr{A}$ \begin{CJK}{UTF8}{mj}构造\end{CJK} $V$ \begin{CJK}{UTF8}{mj}的对偶空间\end{CJK} $V^{\star}$ \begin{CJK}{UTF8}{mj}上的一个线性变换\end{CJK}?

  \item (10 \begin{CJK}{UTF8}{mj}分\end{CJK}) \begin{CJK}{UTF8}{mj}设\end{CJK} $\mathscr{A}$ \begin{CJK}{UTF8}{mj}满足\end{CJK} $\mathscr{A}^{2}=2 \mathscr{A}$. \begin{CJK}{UTF8}{mj}证明\end{CJK}: $V=\operatorname{ker} \mathscr{A} \oplus \operatorname{ker}(\mathscr{A}-2 \mathscr{E})$, \begin{CJK}{UTF8}{mj}其中\end{CJK}, $\mathscr{E}$ \begin{CJK}{UTF8}{mj}是\end{CJK} $V$ \begin{CJK}{UTF8}{mj}上的恒等变换\end{CJK}.

  \item ( 10 \begin{CJK}{UTF8}{mj}分\end{CJK}) \begin{CJK}{UTF8}{mj}设\end{CJK} $\operatorname{dim} V=n$. \begin{CJK}{UTF8}{mj}证明\end{CJK}: $\mathscr{A}$ \begin{CJK}{UTF8}{mj}有\end{CJK} $r$ \begin{CJK}{UTF8}{mj}维不变子空间\end{CJK} $W$ \begin{CJK}{UTF8}{mj}当且仅当存在\end{CJK} $V$ \begin{CJK}{UTF8}{mj}的一组基\end{CJK}, \begin{CJK}{UTF8}{mj}使得\end{CJK} $\mathscr{A}$ \begin{CJK}{UTF8}{mj}在这个基下\end{CJK} \begin{CJK}{UTF8}{mj}的矩阵是形如\end{CJK} $A=\left(\begin{array}{cc}B & C \\ 0 & D\end{array}\right)$ \begin{CJK}{UTF8}{mj}的分块矩阵\end{CJK}, \begin{CJK}{UTF8}{mj}其中\end{CJK} $B$ \begin{CJK}{UTF8}{mj}是\end{CJK} $r$ \begin{CJK}{UTF8}{mj}阶方阵\end{CJK}.

\end{enumerate}
\begin{CJK}{UTF8}{mj}五\end{CJK}、\begin{CJK}{UTF8}{mj}设\end{CJK} $A$ \begin{CJK}{UTF8}{mj}是\end{CJK} $n$ \begin{CJK}{UTF8}{mj}阶实对称矩阵\end{CJK}, $n>1$.

\begin{enumerate}
  \item ( 10 \begin{CJK}{UTF8}{mj}分\end{CJK}) \begin{CJK}{UTF8}{mj}设对任意实向量\end{CJK} $\alpha \in \mathbb{R}^{n}$ \begin{CJK}{UTF8}{mj}都有\end{CJK} $\alpha^{\prime} A \alpha \geq 0$. \begin{CJK}{UTF8}{mj}证明\end{CJK}: \begin{CJK}{UTF8}{mj}集合\end{CJK}
\end{enumerate}
$$
W=\left\{X \in \mathbb{R}^{n} \mid X^{\prime} A X=0\right\}
$$
\begin{CJK}{UTF8}{mj}是\end{CJK} $\mathbb{R}^{n}$ \begin{CJK}{UTF8}{mj}的维数为\end{CJK} $n-r(A)$ \begin{CJK}{UTF8}{mj}的子空间\end{CJK}, \begin{CJK}{UTF8}{mj}其中\end{CJK} $r(A)$ \begin{CJK}{UTF8}{mj}是\end{CJK} $A$ \begin{CJK}{UTF8}{mj}的秩\end{CJK}.

\begin{enumerate}
  \setcounter{enumi}{2}
  \item (15 \begin{CJK}{UTF8}{mj}分\end{CJK}) \begin{CJK}{UTF8}{mj}设\end{CJK} $A$ \begin{CJK}{UTF8}{mj}的元全为\end{CJK} 1 . \begin{CJK}{UTF8}{mj}求正交阵\end{CJK} $T$ \begin{CJK}{UTF8}{mj}使得\end{CJK} $T^{-1} A T$ \begin{CJK}{UTF8}{mj}是对角阵\end{CJK}.

  \item (10 \begin{CJK}{UTF8}{mj}分\end{CJK}) \begin{CJK}{UTF8}{mj}设\end{CJK} $F$ \begin{CJK}{UTF8}{mj}上的方阵\end{CJK} $B$ \begin{CJK}{UTF8}{mj}满足\end{CJK} $A B=B A$ \begin{CJK}{UTF8}{mj}且\end{CJK} $B$ \begin{CJK}{UTF8}{mj}在\end{CJK} $F$ \begin{CJK}{UTF8}{mj}上相似于一个\end{CJK} Jordan \begin{CJK}{UTF8}{mj}阵\end{CJK}. \begin{CJK}{UTF8}{mj}证明\end{CJK}: $A B$ \begin{CJK}{UTF8}{mj}在\end{CJK} $F$ \begin{CJK}{UTF8}{mj}上也相\end{CJK} \begin{CJK}{UTF8}{mj}似于一个\end{CJK} Jordan \begin{CJK}{UTF8}{mj}阵\end{CJK}.

\end{enumerate}
\section{0. 四川大学 2018 年研究生入学考试试题高等代数 
 李扬 
 微信公众号: sxkyliyang}
\begin{CJK}{UTF8}{mj}一\end{CJK}、\begin{CJK}{UTF8}{mj}设\end{CJK} $f(x)$ \begin{CJK}{UTF8}{mj}是首项系数为\end{CJK} 1 \begin{CJK}{UTF8}{mj}的三次实系数多项式\end{CJK}, $f^{\prime}(x)$ \begin{CJK}{UTF8}{mj}表示\end{CJK} $f(x)$ \begin{CJK}{UTF8}{mj}的导数\end{CJK}, $\left(f(x), f^{\prime}(x)\right)$ \begin{CJK}{UTF8}{mj}表示\end{CJK} $f(x)$ \begin{CJK}{UTF8}{mj}和\end{CJK} $f^{\prime}(x)$ \begin{CJK}{UTF8}{mj}的首\end{CJK} \begin{CJK}{UTF8}{mj}项系数为\end{CJK} 1 \begin{CJK}{UTF8}{mj}的最大公因式\end{CJK}.

\begin{enumerate}
  \item ( 5 \begin{CJK}{UTF8}{mj}分\end{CJK}) \begin{CJK}{UTF8}{mj}证明\end{CJK}: \begin{CJK}{UTF8}{mj}如果\end{CJK} $f(x)$ \begin{CJK}{UTF8}{mj}有重根\end{CJK}, \begin{CJK}{UTF8}{mj}那么\end{CJK} $f(x)$ \begin{CJK}{UTF8}{mj}的根都是实根\end{CJK}.

  \item (5 \begin{CJK}{UTF8}{mj}分\end{CJK}) \begin{CJK}{UTF8}{mj}设\end{CJK} $\frac{f(x)}{\left(f(x), f^{\prime}(x)\right)}=(x-1)(x-2)$. \begin{CJK}{UTF8}{mj}写出所有以\end{CJK} $f(x)$ \begin{CJK}{UTF8}{mj}为特征多项式的\end{CJK} Jordan \begin{CJK}{UTF8}{mj}阵\end{CJK}.

  \item ( 5 \begin{CJK}{UTF8}{mj}分\end{CJK}) \begin{CJK}{UTF8}{mj}设\end{CJK} $\alpha$ \begin{CJK}{UTF8}{mj}是\end{CJK} $f(x)$ \begin{CJK}{UTF8}{mj}的一个复根\end{CJK}, \begin{CJK}{UTF8}{mj}令\end{CJK} $V=\{y \mid y=g(\alpha), g(x)$ \begin{CJK}{UTF8}{mj}是实系数多项式\end{CJK} $\}$. \begin{CJK}{UTF8}{mj}证明\end{CJK}: $V$ \begin{CJK}{UTF8}{mj}是实数域上的\end{CJK} \begin{CJK}{UTF8}{mj}线性空间\end{CJK}, \begin{CJK}{UTF8}{mj}并求它的维数\end{CJK} $\operatorname{dim} V$.

  \item ( 5 \begin{CJK}{UTF8}{mj}分\end{CJK}) \begin{CJK}{UTF8}{mj}设\end{CJK} $\alpha_{1}, \alpha_{2}, \alpha_{3}$ \begin{CJK}{UTF8}{mj}是\end{CJK} $f(x)$ \begin{CJK}{UTF8}{mj}的一个复根\end{CJK}, \begin{CJK}{UTF8}{mj}设\end{CJK} $s_{k}=\alpha_{1}^{k}+\alpha_{2}^{k}+\alpha_{3}^{k}$. \begin{CJK}{UTF8}{mj}设\end{CJK} $s_{1}=s_{2}=0, s_{3}=1$, \begin{CJK}{UTF8}{mj}求\end{CJK} $f(x)$ \begin{CJK}{UTF8}{mj}的表\end{CJK} \begin{CJK}{UTF8}{mj}达式\end{CJK}.

  \item ( 5 \begin{CJK}{UTF8}{mj}分\end{CJK}) \begin{CJK}{UTF8}{mj}设\end{CJK} $f(x)$ \begin{CJK}{UTF8}{mj}的全部根都在单位圆上\end{CJK}, \begin{CJK}{UTF8}{mj}证明\end{CJK}: \begin{CJK}{UTF8}{mj}多项式\end{CJK} $g(x)=2 x f^{\prime}(x)-3 f(x)$ \begin{CJK}{UTF8}{mj}的全部复根也都在单位圆\end{CJK} \begin{CJK}{UTF8}{mj}上\end{CJK}.

\end{enumerate}
\begin{CJK}{UTF8}{mj}二\end{CJK}、\begin{CJK}{UTF8}{mj}设\end{CJK} $A X=\beta$ \begin{CJK}{UTF8}{mj}是数域\end{CJK} $F$ \begin{CJK}{UTF8}{mj}上的非齐次线性方程组\end{CJK}.

\begin{enumerate}
  \item ( 5 \begin{CJK}{UTF8}{mj}分\end{CJK}) \begin{CJK}{UTF8}{mj}设\end{CJK} $A X=\beta$ \begin{CJK}{UTF8}{mj}有无穷多个解\end{CJK}, \begin{CJK}{UTF8}{mj}证明\end{CJK}: \begin{CJK}{UTF8}{mj}存在它的解\end{CJK} $r_{1}, \cdots, r_{s}$, \begin{CJK}{UTF8}{mj}使得它的任意解都是\end{CJK} $r_{1}, \cdots, r_{s}$ \begin{CJK}{UTF8}{mj}的线\end{CJK} \begin{CJK}{UTF8}{mj}性组合\end{CJK}.

  \item (10 \begin{CJK}{UTF8}{mj}分\end{CJK}) \begin{CJK}{UTF8}{mj}设\end{CJK} $A X=\beta$ \begin{CJK}{UTF8}{mj}有无穷多个解\end{CJK}, \begin{CJK}{UTF8}{mj}且它的任意解都可以由向量组\end{CJK} $(1,-1,0,4)^{\prime},(0,3,7,2)^{\prime},(3,0,7,14)^{\prime}$ \begin{CJK}{UTF8}{mj}线性表出\end{CJK}. \begin{CJK}{UTF8}{mj}求矩阵\end{CJK} $A$ \begin{CJK}{UTF8}{mj}的秩\end{CJK} $r(A)$.

  \item (10 \begin{CJK}{UTF8}{mj}分\end{CJK}) \begin{CJK}{UTF8}{mj}设矩阵\end{CJK} $A$ \begin{CJK}{UTF8}{mj}的秩为\end{CJK} $r(A)=3$, \begin{CJK}{UTF8}{mj}已知\end{CJK} $\alpha_{1}, \alpha_{2}, \alpha_{3}$ \begin{CJK}{UTF8}{mj}是\end{CJK} $A X=\beta$ \begin{CJK}{UTF8}{mj}的两两不同的解且满足\end{CJK}:

\end{enumerate}
$$
\alpha_{1}+\alpha_{2}+\alpha_{3}=(3,0,0,0)^{\prime}, 2 \alpha_{1}+3 \alpha_{2}=(5,2,2,0)^{\prime}
$$
\begin{CJK}{UTF8}{mj}求\end{CJK} $A X=\beta$ \begin{CJK}{UTF8}{mj}的通解\end{CJK}.

\begin{CJK}{UTF8}{mj}三\end{CJK}、\begin{CJK}{UTF8}{mj}对任意矩阵\end{CJK} $M$, \begin{CJK}{UTF8}{mj}用\end{CJK} $r(M)$ \begin{CJK}{UTF8}{mj}表示\end{CJK} $M$ \begin{CJK}{UTF8}{mj}的秩\end{CJK}, $M^{\prime}$ \begin{CJK}{UTF8}{mj}表示\end{CJK} $M$ \begin{CJK}{UTF8}{mj}的转置\end{CJK}, \begin{CJK}{UTF8}{mj}设\end{CJK} $A$ \begin{CJK}{UTF8}{mj}是数域\end{CJK} $F$ \begin{CJK}{UTF8}{mj}上的\end{CJK} $s \times n$ \begin{CJK}{UTF8}{mj}型实矩阵\end{CJK}, $F^{m}$ \begin{CJK}{UTF8}{mj}是\end{CJK} $F$ \begin{CJK}{UTF8}{mj}上的\end{CJK} $m$ \begin{CJK}{UTF8}{mj}维列向量组成的线性空间\end{CJK}.

\begin{enumerate}
  \item ( 10 \begin{CJK}{UTF8}{mj}分\end{CJK}) \begin{CJK}{UTF8}{mj}设\end{CJK}
\end{enumerate}
$$
V_{1}=\left\{X \in F^{n} \mid A^{\prime} A X=X\right\}, \quad V_{2}=\left\{Y \in F^{s} \mid A A^{\prime} Y=Y\right\}
$$
\begin{CJK}{UTF8}{mj}证明\end{CJK}: $V_{1}, V_{2}$ \begin{CJK}{UTF8}{mj}分别是\end{CJK} $F^{n}, F^{s}$ \begin{CJK}{UTF8}{mj}的子空间\end{CJK}, \begin{CJK}{UTF8}{mj}且有线性空间同构\end{CJK}: $V_{1} \cong V_{2}$.

\begin{enumerate}
  \setcounter{enumi}{2}
  \item ( 15 \begin{CJK}{UTF8}{mj}分\end{CJK}) \begin{CJK}{UTF8}{mj}设\end{CJK} $F=\mathbb{R}$ \begin{CJK}{UTF8}{mj}是实数域\end{CJK}, \begin{CJK}{UTF8}{mj}证明\end{CJK}: $r\left(A^{\prime} A\right)=r(A)$, \begin{CJK}{UTF8}{mj}并由此证明\end{CJK}: \begin{CJK}{UTF8}{mj}对\end{CJK} $\forall \beta \in \mathbb{R}^{n}$, \begin{CJK}{UTF8}{mj}线性方程组\end{CJK} $A^{\prime} A X=A^{\prime} \beta$ \begin{CJK}{UTF8}{mj}都有解\end{CJK}.
\end{enumerate}
\begin{CJK}{UTF8}{mj}四\end{CJK}、\begin{CJK}{UTF8}{mj}设\end{CJK} $\mathscr{A}$ \begin{CJK}{UTF8}{mj}是复数域\end{CJK} $\mathbb{C}$ \begin{CJK}{UTF8}{mj}上的\end{CJK} $n$ \begin{CJK}{UTF8}{mj}维线性空间\end{CJK} $V$ \begin{CJK}{UTF8}{mj}上的一个线性变换\end{CJK} $(n \geq 3)$, \begin{CJK}{UTF8}{mj}且\end{CJK} $\mathscr{A}$ \begin{CJK}{UTF8}{mj}在\end{CJK} $V$ \begin{CJK}{UTF8}{mj}的某个基下的矩阵为\end{CJK}:
$$
D=\left(\begin{array}{ccccccc}
0 & 0 & 0 & \cdots & 0 & 0 & 6 \\
1 & 0 & 0 & \cdots & 0 & 0 & 6 \\
0 & 1 & 0 & \cdots & 0 & 0 & -12 \\
0 & 0 & 1 & \cdots & 0 & 0 & 0 \\
\vdots & \vdots & \vdots & & \vdots & \vdots & \vdots \\
0 & 0 & 0 & \cdots & 1 & 0 & 0 \\
0 & 0 & 0 & \cdots & 0 & 1 & 0
\end{array}\right)
$$

\begin{enumerate}
  \setcounter{enumi}{2}
  \item (5 \begin{CJK}{UTF8}{mj}分\end{CJK}) $f(\lambda)$ \begin{CJK}{UTF8}{mj}在有理数域\end{CJK} $\mathbb{Q}$ \begin{CJK}{UTF8}{mj}上是否可约\end{CJK}? \begin{CJK}{UTF8}{mj}说明理由\end{CJK}.

  \item (5 \begin{CJK}{UTF8}{mj}分\end{CJK}) \begin{CJK}{UTF8}{mj}设\end{CJK} $W$ \begin{CJK}{UTF8}{mj}是\end{CJK} $V$ \begin{CJK}{UTF8}{mj}上的所有与\end{CJK} $\mathscr{A}$ \begin{CJK}{UTF8}{mj}可交换的线性变换组成的集合\end{CJK}, \begin{CJK}{UTF8}{mj}证明\end{CJK}: $W$ \begin{CJK}{UTF8}{mj}是\end{CJK} $V$ \begin{CJK}{UTF8}{mj}的子空间并求它的维\end{CJK} \begin{CJK}{UTF8}{mj}数\end{CJK}.

  \item ( 5 \begin{CJK}{UTF8}{mj}分\end{CJK}) \begin{CJK}{UTF8}{mj}求\end{CJK} $\mathscr{A}$ \begin{CJK}{UTF8}{mj}的所有不变子空间的个数\end{CJK}.

\end{enumerate}
\begin{CJK}{UTF8}{mj}五\end{CJK}、\begin{CJK}{UTF8}{mj}设\end{CJK} $A$ \begin{CJK}{UTF8}{mj}是\end{CJK} $n$ \begin{CJK}{UTF8}{mj}阶实对称矩阵\end{CJK},

\begin{enumerate}
  \item ( 15 \begin{CJK}{UTF8}{mj}分\end{CJK}) \begin{CJK}{UTF8}{mj}证明\end{CJK}: $A$ \begin{CJK}{UTF8}{mj}是正定的当且仅当\end{CJK} $A$ \begin{CJK}{UTF8}{mj}是某个欧式空间的内积的度量阵\end{CJK}.

  \item ( 5 \begin{CJK}{UTF8}{mj}分\end{CJK}) \begin{CJK}{UTF8}{mj}设\end{CJK} $A$ \begin{CJK}{UTF8}{mj}是正交阵\end{CJK}, \begin{CJK}{UTF8}{mj}证明\end{CJK}: $A$ \begin{CJK}{UTF8}{mj}的极小多项式的次数不超过\end{CJK} 2 .

  \item (5 \begin{CJK}{UTF8}{mj}分\end{CJK}) \begin{CJK}{UTF8}{mj}设\end{CJK} $B$ \begin{CJK}{UTF8}{mj}是\end{CJK} $n$ \begin{CJK}{UTF8}{mj}阶实方阵\end{CJK}, \begin{CJK}{UTF8}{mj}且\end{CJK} $B B^{\prime}+A^{2}=0$, \begin{CJK}{UTF8}{mj}证明\end{CJK}: $A=B=0$.

\end{enumerate}
\begin{CJK}{UTF8}{mj}六\end{CJK}、\begin{CJK}{UTF8}{mj}设\end{CJK} $V$ \begin{CJK}{UTF8}{mj}是欧式空间\end{CJK}, \begin{CJK}{UTF8}{mj}其内积为\end{CJK} $(-,-)$. \begin{CJK}{UTF8}{mj}设\end{CJK} $\alpha_{1}, \alpha_{2}, \cdots, \alpha_{n}$ \begin{CJK}{UTF8}{mj}是\end{CJK} $V$ \begin{CJK}{UTF8}{mj}的一组基\end{CJK}.

\begin{enumerate}
  \item ( 10 \begin{CJK}{UTF8}{mj}分\end{CJK}) \begin{CJK}{UTF8}{mj}证明\end{CJK}: \begin{CJK}{UTF8}{mj}存在\end{CJK} $V$ \begin{CJK}{UTF8}{mj}的唯一的一组基\end{CJK} $\beta_{1}, \beta_{2}, \cdots, \beta_{n}$ \begin{CJK}{UTF8}{mj}使得\end{CJK} $\left(\alpha_{i}, \beta_{j}\right)=\delta_{i j}, 1 \leq i, j \leq n$, \begin{CJK}{UTF8}{mj}其中\end{CJK} $\delta_{i i}=1(1 \leq i \leq n)$ \begin{CJK}{UTF8}{mj}且\end{CJK} $\delta_{i j}=0(1 \leq i \neq j \leq n) .$

  \item ( 15 \begin{CJK}{UTF8}{mj}分\end{CJK}) \begin{CJK}{UTF8}{mj}已知\end{CJK} $\left(\alpha_{i}, \alpha_{j}\right) \leq 0,1 \leq i \neq j \leq n$. \begin{CJK}{UTF8}{mj}设\end{CJK} $\alpha=\sum_{i=1}^{n} k_{i} \alpha_{i} \in V$ \begin{CJK}{UTF8}{mj}满足\end{CJK} $\left(\alpha, \alpha_{i}\right) \geq 0,1 \leq i \leq n$. \begin{CJK}{UTF8}{mj}证明\end{CJK}: $k_{i} \geq 0,1 \leq i \leq n .$

\end{enumerate}
\section{1. 四川大学 2009 年研究生入学考试试题数学分析}
\begin{CJK}{UTF8}{mj}李扬\end{CJK}

\begin{CJK}{UTF8}{mj}微信公众号\end{CJK}: sxkyliyang

\begin{CJK}{UTF8}{mj}一\end{CJK}、 (\begin{CJK}{UTF8}{mj}每题\end{CJK} 7 \begin{CJK}{UTF8}{mj}分\end{CJK}, \begin{CJK}{UTF8}{mj}共\end{CJK} 28 \begin{CJK}{UTF8}{mj}分\end{CJK}) \begin{CJK}{UTF8}{mj}求下列极限\end{CJK}.

\begin{enumerate}
  \item $\lim _{n \rightarrow \infty} \frac{1}{n^{2}} \sum_{k=0}^{n} \ln C_{n}^{k}$.

  \item $\lim _{n \rightarrow \infty} \sin ^{2}\left(\pi \sqrt{n^{2}+n}\right)$.

  \item $\lim _{x \rightarrow 0^{+}} \frac{\int_{0}^{x^{2}} \sin ^{\frac{3}{2}} t \mathrm{~d} t}{\int_{0}^{x} t(t-\sin t) \mathrm{d} t}$.

  \item $\lim _{x \rightarrow 0} \frac{e^{x}-1-x}{\sqrt{1-x}-\cos \sqrt{x}}$.

\end{enumerate}
\begin{CJK}{UTF8}{mj}二\end{CJK}、 (\begin{CJK}{UTF8}{mj}每题\end{CJK} 10 \begin{CJK}{UTF8}{mj}分\end{CJK}, \begin{CJK}{UTF8}{mj}共\end{CJK} 40 \begin{CJK}{UTF8}{mj}分\end{CJK}) \begin{CJK}{UTF8}{mj}计算下列积分\end{CJK}.

\begin{enumerate}
  \item $\iint_{D}\left|\frac{x+y}{\sqrt{2}}-x^{2}-y^{2}\right| \mathrm{d} x \mathrm{~d} y$, \begin{CJK}{UTF8}{mj}其中\end{CJK}, $D=\left\{(x, y) \in R^{2}: x^{2}+y^{2} \leq 1\right\}$.

  \item $\int_{L} y z \mathrm{~d} s$, \begin{CJK}{UTF8}{mj}其中曲线\end{CJK} $L$ \begin{CJK}{UTF8}{mj}是球面\end{CJK} $x^{2}+y^{2}+z^{2}=a^{2}$ \begin{CJK}{UTF8}{mj}与平面\end{CJK} $x+y+y=1$ \begin{CJK}{UTF8}{mj}的交线\end{CJK}.

  \item \begin{CJK}{UTF8}{mj}设\end{CJK} $f(x)$ \begin{CJK}{UTF8}{mj}在\end{CJK} $(-\infty,+\infty)$ \begin{CJK}{UTF8}{mj}内有连续导函数\end{CJK}, \begin{CJK}{UTF8}{mj}求积分\end{CJK}

\end{enumerate}
$$
\int_{L} \frac{1+y^{2} f(x y)}{y} \mathrm{~d} x+\frac{x}{y^{2}}\left[y^{2} f(x y)-1\right] \mathrm{d} y
$$
\begin{CJK}{UTF8}{mj}其中\end{CJK} $L$ \begin{CJK}{UTF8}{mj}是从点\end{CJK} $A\left(3, \frac{2}{3}\right)$ \begin{CJK}{UTF8}{mj}到点\end{CJK} $B(1,2)$ \begin{CJK}{UTF8}{mj}的直线段\end{CJK}.

\begin{enumerate}
  \setcounter{enumi}{4}
  \item $\iint_{S} \frac{x \mathrm{~d} y \mathrm{~d} z+y \mathrm{~d} z \mathrm{~d} x+z \mathrm{~d} x \mathrm{~d} y}{\sqrt{\left(x^{2}+y^{2}+z^{2}\right)^{3}}}$, \begin{CJK}{UTF8}{mj}其中\end{CJK} $S$ \begin{CJK}{UTF8}{mj}是抛物面\end{CJK} $1-\frac{z}{7}=\frac{(x-2)^{2}}{25}+\frac{(y-1)^{2}}{16}(z \geq 0)$ \begin{CJK}{UTF8}{mj}的上侧\end{CJK}.
\end{enumerate}
\begin{CJK}{UTF8}{mj}三\end{CJK}、 (\begin{CJK}{UTF8}{mj}本题\end{CJK} 10 \begin{CJK}{UTF8}{mj}分\end{CJK}) \begin{CJK}{UTF8}{mj}设\end{CJK} $z=f(x, y)$ \begin{CJK}{UTF8}{mj}在有界闭区域\end{CJK} $D$ \begin{CJK}{UTF8}{mj}内有二阶连续的偏导数\end{CJK}, \begin{CJK}{UTF8}{mj}且\end{CJK} $f_{x x}^{\prime \prime}+f_{y y}^{\prime \prime}=0, f_{x y}^{\prime \prime} \neq 0$. \begin{CJK}{UTF8}{mj}证明\end{CJK}: $z=f(x, y)$ \begin{CJK}{UTF8}{mj}的最大值和最小值只能在区域\end{CJK} $D$ \begin{CJK}{UTF8}{mj}的边界上取得\end{CJK}.

\begin{CJK}{UTF8}{mj}四\end{CJK}、 (\begin{CJK}{UTF8}{mj}本题\end{CJK} 12 \begin{CJK}{UTF8}{mj}分\end{CJK}) \begin{CJK}{UTF8}{mj}证明\end{CJK}: \begin{CJK}{UTF8}{mj}在变换\end{CJK} $u=\frac{x}{y}, v=x, w=x z-y$ \begin{CJK}{UTF8}{mj}之下\end{CJK}, \begin{CJK}{UTF8}{mj}方程\end{CJK} $y \frac{\partial^{2} z}{\partial y^{2}}+2 \frac{\partial z}{\partial y}=\frac{2}{x}$ \begin{CJK}{UTF8}{mj}可变为\end{CJK} $\frac{\partial^{2} w}{\partial u^{2}}=0$.

\begin{CJK}{UTF8}{mj}五\end{CJK}、(\begin{CJK}{UTF8}{mj}本题\end{CJK} 12 \begin{CJK}{UTF8}{mj}分\end{CJK}) \begin{CJK}{UTF8}{mj}证明\end{CJK}: $\sum_{n=1}^{\infty}(1-x) \frac{x^{n}}{1-x^{2 n}} \sin n x$ \begin{CJK}{UTF8}{mj}在\end{CJK} $\left(\frac{1}{2}, 1\right)$ \begin{CJK}{UTF8}{mj}上一致收敛\end{CJK}.

\begin{CJK}{UTF8}{mj}六\end{CJK}、(\begin{CJK}{UTF8}{mj}本题\end{CJK} 12 \begin{CJK}{UTF8}{mj}分\end{CJK}) \begin{CJK}{UTF8}{mj}设\end{CJK} $f(x)$ \begin{CJK}{UTF8}{mj}在\end{CJK} $[a, b]$ \begin{CJK}{UTF8}{mj}上连续\end{CJK}, \begin{CJK}{UTF8}{mj}且\end{CJK} $f(a)=f(b)$, \begin{CJK}{UTF8}{mj}证明\end{CJK}: $\forall n \in \mathbb{Z}^{+}, \exists \xi \in(a, b)$, \begin{CJK}{UTF8}{mj}使得\end{CJK}
$$
f\left(\xi+\frac{b-a}{n}\right)=f(\xi)
$$
\begin{CJK}{UTF8}{mj}七\end{CJK}、 (\begin{CJK}{UTF8}{mj}本题\end{CJK} 12 \begin{CJK}{UTF8}{mj}分\end{CJK}) \begin{CJK}{UTF8}{mj}设\end{CJK} $f(x)$ \begin{CJK}{UTF8}{mj}在\end{CJK} $[0,1]$ \begin{CJK}{UTF8}{mj}上连续\end{CJK}, \begin{CJK}{UTF8}{mj}在\end{CJK} $(0,1)$ \begin{CJK}{UTF8}{mj}可导\end{CJK}, \begin{CJK}{UTF8}{mj}且\end{CJK} $f^{\prime}(x)>0, f(0)=0$, \begin{CJK}{UTF8}{mj}证明\end{CJK}: \begin{CJK}{UTF8}{mj}存在\end{CJK} $\xi, \eta \in(0,1)$, \begin{CJK}{UTF8}{mj}使得\end{CJK} $\xi+\eta=1$, \begin{CJK}{UTF8}{mj}且\end{CJK} $\frac{f^{\prime}(\xi)}{f(\xi)}=\frac{f^{\prime}(\eta)}{f(\eta)}$.

\begin{CJK}{UTF8}{mj}八\end{CJK}、 (\begin{CJK}{UTF8}{mj}本题\end{CJK} 12 \begin{CJK}{UTF8}{mj}分\end{CJK}) \begin{CJK}{UTF8}{mj}设函数\end{CJK} $f(x)$ \begin{CJK}{UTF8}{mj}在\end{CJK} $[a, b]$ \begin{CJK}{UTF8}{mj}上连续可导\end{CJK}, \begin{CJK}{UTF8}{mj}证明\end{CJK}:
$$
\int_{a}^{b}|f(x)| \mathrm{d} x \leq \max \left\{(b-a) \int_{a}^{b}\left|f^{\prime}(x)\right| \mathrm{d} x,\left|\int_{a}^{b} f(x) \mathrm{d} x\right|\right\}
$$
\begin{CJK}{UTF8}{mj}九\end{CJK}、 (\begin{CJK}{UTF8}{mj}本题\end{CJK} 12 \begin{CJK}{UTF8}{mj}分\end{CJK}) \begin{CJK}{UTF8}{mj}设对任意实数\end{CJK} $A>0$, \begin{CJK}{UTF8}{mj}函数\end{CJK} $f(x)$ \begin{CJK}{UTF8}{mj}在\end{CJK} $[0, A]$ \begin{CJK}{UTF8}{mj}上可积\end{CJK}, \begin{CJK}{UTF8}{mj}且\end{CJK} $\lim _{x \rightarrow+\infty} f(x)=B(B$ \begin{CJK}{UTF8}{mj}有限\end{CJK}). \begin{CJK}{UTF8}{mj}证明\end{CJK}:
$$
\lim _{t \rightarrow 0^{+}} t \int_{0}^{+\infty} e^{-t x} f(x) \mathrm{d} x=B .
$$

\section{2. 四川大学 2010 年研究生入学考试试题数学分析}
\begin{CJK}{UTF8}{mj}李扬\end{CJK}

\begin{CJK}{UTF8}{mj}微信公众号\end{CJK}: sxkyliyang

\begin{CJK}{UTF8}{mj}一\end{CJK}、 \begin{CJK}{UTF8}{mj}计算下列极限\end{CJK}(\begin{CJK}{UTF8}{mj}每题\end{CJK} 7 \begin{CJK}{UTF8}{mj}分\end{CJK}, \begin{CJK}{UTF8}{mj}共\end{CJK} 28 \begin{CJK}{UTF8}{mj}分\end{CJK}).

\begin{enumerate}
  \item $\lim _{x \rightarrow 0} \frac{\sqrt{\cos x}-\sqrt[3]{\cos x}}{\sin ^{2} x}$.

  \item $\lim _{n \rightarrow \infty}\left(\frac{1^{p}+2^{p}+\cdots+n^{p}}{n^{p}}-\frac{n}{p+1}\right),(p \in N, p \geq 1)$.

  \item $\lim _{x \rightarrow+\infty} \frac{x^{2 \ln x}-x}{(\ln x)^{x}+x}$.

  \item $\lim _{n \rightarrow \infty} \sqrt[n]{1+a^{n}+\sin ^{2} n}, a>0 .$

\end{enumerate}
\begin{CJK}{UTF8}{mj}二\end{CJK}、 \begin{CJK}{UTF8}{mj}计算下列积分\end{CJK} (\begin{CJK}{UTF8}{mj}每题\end{CJK} 8 \begin{CJK}{UTF8}{mj}分\end{CJK}, \begin{CJK}{UTF8}{mj}共\end{CJK} 40 \begin{CJK}{UTF8}{mj}分\end{CJK}).

\begin{enumerate}
  \item $\int \frac{\sin x \cos ^{3} x}{1+\cos ^{2} x} \mathrm{~d} x$.

  \item $\oint_{L} \frac{(x-y) \mathrm{d} x+(x+4 y) \mathrm{d} y}{x^{2}+4 y^{2}}$, \begin{CJK}{UTF8}{mj}其中\end{CJK} $L$ \begin{CJK}{UTF8}{mj}为单位圆周\end{CJK} $x^{2}+y^{2}=1$, \begin{CJK}{UTF8}{mj}取逆时针方向\end{CJK}.

  \item $\iint_{\Sigma}(2 x+z) \mathrm{d} y \mathrm{~d} z+z \mathrm{~d} x \mathrm{~d} y$, \begin{CJK}{UTF8}{mj}其中\end{CJK} $\Sigma$ \begin{CJK}{UTF8}{mj}是曲面\end{CJK} $z=x^{2}+y^{2}(0 \leq z \leq 1)$ \begin{CJK}{UTF8}{mj}取上侧\end{CJK}.

  \item $g(\alpha)=\int_{1}^{+\infty} \frac{\arctan \alpha x}{x^{2} \sqrt{x^{2}-1}} \mathrm{~d} x$.

\end{enumerate}
5 . \begin{CJK}{UTF8}{mj}设\end{CJK} $f(x)=\int_{1}^{x} \frac{\sin t}{t} \mathrm{~d} t$, \begin{CJK}{UTF8}{mj}求\end{CJK} $\int_{0}^{1} x f(x) \mathrm{d} x$.

\begin{CJK}{UTF8}{mj}三\end{CJK}、(\begin{CJK}{UTF8}{mj}本题\end{CJK} 12 \begin{CJK}{UTF8}{mj}分\end{CJK}) $f(x)$ \begin{CJK}{UTF8}{mj}在\end{CJK} $[0,+\infty)$ \begin{CJK}{UTF8}{mj}连续\end{CJK}, \begin{CJK}{UTF8}{mj}且\end{CJK} $\lim _{x \rightarrow+\infty}[f(x)+\sin x]=0$. \begin{CJK}{UTF8}{mj}证明\end{CJK}: $f(x)$ \begin{CJK}{UTF8}{mj}在\end{CJK} $[0,+\infty)$ \begin{CJK}{UTF8}{mj}上一致连续\end{CJK}.

\begin{CJK}{UTF8}{mj}四\end{CJK}、(\begin{CJK}{UTF8}{mj}本题\end{CJK} 10 \begin{CJK}{UTF8}{mj}分\end{CJK}) \begin{CJK}{UTF8}{mj}令\end{CJK} $u=f(z)$, \begin{CJK}{UTF8}{mj}其中\end{CJK} $z=z(x, y)$ \begin{CJK}{UTF8}{mj}是由方程\end{CJK} $z=x+y \varphi(z)$ \begin{CJK}{UTF8}{mj}确定的隐函数\end{CJK}, \begin{CJK}{UTF8}{mj}且\end{CJK} $f(z)$ \begin{CJK}{UTF8}{mj}和\end{CJK} $\varphi(z)$ \begin{CJK}{UTF8}{mj}是任意阶\end{CJK} \begin{CJK}{UTF8}{mj}可微函数\end{CJK}. \begin{CJK}{UTF8}{mj}证明\end{CJK}: $\frac{\partial^{n} u}{\partial y^{n}}=\frac{\partial^{n-1}}{\partial x^{n-1}}\left((\varphi(z))^{n} \frac{\partial u}{\partial x}\right)$.

\begin{CJK}{UTF8}{mj}五\end{CJK}、(\begin{CJK}{UTF8}{mj}本题\end{CJK} 15 \begin{CJK}{UTF8}{mj}分\end{CJK}) \begin{CJK}{UTF8}{mj}证明\end{CJK}: \begin{CJK}{UTF8}{mj}如果函数\end{CJK} $f(x)$ \begin{CJK}{UTF8}{mj}在\end{CJK} $(0,+\infty)$ \begin{CJK}{UTF8}{mj}内可微\end{CJK}, \begin{CJK}{UTF8}{mj}且\end{CJK} $\lim _{x \rightarrow+\infty} f^{\prime}(x)=0$. \begin{CJK}{UTF8}{mj}则\end{CJK} $\lim _{x \rightarrow+\infty} \frac{f(x)}{x}=0$.

\begin{CJK}{UTF8}{mj}六\end{CJK}、(\begin{CJK}{UTF8}{mj}本题\end{CJK} 10 \begin{CJK}{UTF8}{mj}分\end{CJK}) \begin{CJK}{UTF8}{mj}设\end{CJK} $f(x)$ \begin{CJK}{UTF8}{mj}在\end{CJK} $[a, b]$ \begin{CJK}{UTF8}{mj}内可导且\end{CJK} $f(a)=0$. \begin{CJK}{UTF8}{mj}证明\end{CJK}:
$$
M^{2} \leq(b-a) \int_{a}^{b}\left(f^{\prime}(x)\right)^{2} \mathrm{~d} x
$$
\begin{CJK}{UTF8}{mj}其中\end{CJK}, $M=\sup _{a \leq x \leq b}\{|f(x)|\}$.

\begin{CJK}{UTF8}{mj}七\end{CJK}、 (\begin{CJK}{UTF8}{mj}本题\end{CJK} 20 \begin{CJK}{UTF8}{mj}分\end{CJK}) \begin{CJK}{UTF8}{mj}设\end{CJK} $f_{n}(x)=n^{\alpha} x e^{-n x}, n \in N$. \begin{CJK}{UTF8}{mj}当参数\end{CJK} $\alpha$ \begin{CJK}{UTF8}{mj}为何值时\end{CJK}

\begin{enumerate}
  \item \begin{CJK}{UTF8}{mj}函数列\end{CJK} $f_{n}(x)$ \begin{CJK}{UTF8}{mj}在\end{CJK} $[0,1]$ \begin{CJK}{UTF8}{mj}上收敛\end{CJK};

  \item \begin{CJK}{UTF8}{mj}函数列\end{CJK} $f_{n}(x)$ \begin{CJK}{UTF8}{mj}在\end{CJK} $[0,1]$ \begin{CJK}{UTF8}{mj}上一致收敛\end{CJK};

  \item $\int_{0}^{1} \lim _{n \rightarrow \infty} f_{n}(x) \mathrm{d} x=\lim _{n \rightarrow \infty} \int_{0}^{1} f_{n}(x) \mathrm{d} x$.

\end{enumerate}
\begin{CJK}{UTF8}{mj}八\end{CJK}、 (\begin{CJK}{UTF8}{mj}本题\end{CJK} 15 \begin{CJK}{UTF8}{mj}分\end{CJK}) \begin{CJK}{UTF8}{mj}证明\end{CJK}:
$$
\iiint_{\Omega} \frac{\mathrm{d} x \mathrm{~d} y \mathrm{~d} z}{r}=\frac{1}{2} \iint_{\partial \Omega} \cos (\vec{r}, \vec{n}) \mathrm{d} S .
$$
\begin{CJK}{UTF8}{mj}其中\end{CJK} $\Omega$ \begin{CJK}{UTF8}{mj}为\end{CJK} $\mathbb{R}^{3}$ \begin{CJK}{UTF8}{mj}中的单连通区域\end{CJK}, $\partial \Omega$ \begin{CJK}{UTF8}{mj}为其光滑边界曲面\end{CJK}, $\vec{n}$ \begin{CJK}{UTF8}{mj}为曲面\end{CJK} $\partial \Omega$ \begin{CJK}{UTF8}{mj}在点\end{CJK} $(x, y, z)$ \begin{CJK}{UTF8}{mj}的单位外法矢量\end{CJK}, $r=\sqrt{(\xi-x)^{2}+(\eta-y)^{2}+(\zeta-z)^{2}}, \vec{r}=(x-\xi) \mathrm{i}+(y-\eta) \mathrm{j}+(z-\zeta) \mathrm{k}$ \begin{CJK}{UTF8}{mj}为连接空间中的点\end{CJK} $(\xi, \eta, \zeta)$ \begin{CJK}{UTF8}{mj}到点\end{CJK} $(x, y, z)$ \begin{CJK}{UTF8}{mj}的矢量\end{CJK}.

\section{3. 四川大学 2011 年研究生入学考试试题数学分析}
\begin{CJK}{UTF8}{mj}李扬\end{CJK}

\begin{CJK}{UTF8}{mj}微信公众号\end{CJK}: sxkyliyang

\begin{CJK}{UTF8}{mj}一\end{CJK}、 \begin{CJK}{UTF8}{mj}计算下列极限\end{CJK}(\begin{CJK}{UTF8}{mj}每题\end{CJK} 7 \begin{CJK}{UTF8}{mj}分\end{CJK}, \begin{CJK}{UTF8}{mj}共\end{CJK} 28 \begin{CJK}{UTF8}{mj}分\end{CJK}).

\begin{enumerate}
  \item $\lim _{n \rightarrow \infty} \sqrt{n+\sqrt{n+2 \sqrt{n}}}-\sqrt{n}$.

  \item $\lim _{n \rightarrow \infty} \sum_{k=1}^{2 n} \frac{1}{n+k}$.

  \item \begin{CJK}{UTF8}{mj}已知\end{CJK} $\lim _{x \rightarrow \infty}\left(1+\frac{1}{x}\right)^{a x}=\lim _{x \rightarrow 0} \arccos \frac{\sqrt{x+1}-1}{\sin x}$, \begin{CJK}{UTF8}{mj}求\end{CJK} $a$.

  \item $\lim _{x \rightarrow 0}\left(e^{x}+x^{2}+3 \sin x-1\right)^{\frac{1}{2 x}}$

\end{enumerate}
\begin{CJK}{UTF8}{mj}二\end{CJK}、\begin{CJK}{UTF8}{mj}计算积分\end{CJK} (\begin{CJK}{UTF8}{mj}每题\end{CJK} 8 \begin{CJK}{UTF8}{mj}分\end{CJK}, \begin{CJK}{UTF8}{mj}共\end{CJK} 48 \begin{CJK}{UTF8}{mj}分\end{CJK}).

\begin{enumerate}
  \item \begin{CJK}{UTF8}{mj}求\end{CJK} $\int \cos (\ln x) \mathrm{d} x$.

  \item \begin{CJK}{UTF8}{mj}求\end{CJK} $\int_{0}^{+\infty} \frac{1}{x^{4}+1} \mathrm{~d} x$.

  \item \begin{CJK}{UTF8}{mj}计算积分\end{CJK} $I=\int_{L}|y| \mathrm{d} s$, \begin{CJK}{UTF8}{mj}其中\end{CJK} $L$ \begin{CJK}{UTF8}{mj}为球面\end{CJK} $x^{2}+y^{2}+z^{2}=2$ \begin{CJK}{UTF8}{mj}与平面\end{CJK} $x=y$ \begin{CJK}{UTF8}{mj}的交线\end{CJK}.

  \item \begin{CJK}{UTF8}{mj}计算曲面积分\end{CJK} $I=\iint_{\Sigma}(x+y+y)^{2} \mathrm{~d} S$, \begin{CJK}{UTF8}{mj}其中\end{CJK} $\Sigma$ \begin{CJK}{UTF8}{mj}是球面\end{CJK} $x^{2}+y^{2}+z^{2}=R^{2}$.

  \item \begin{CJK}{UTF8}{mj}设\end{CJK} $f(x)$ \begin{CJK}{UTF8}{mj}在\end{CJK} $(-\infty,+\infty)$ \begin{CJK}{UTF8}{mj}上有连续导数\end{CJK}, \begin{CJK}{UTF8}{mj}计算积分\end{CJK}

\end{enumerate}
$$
I=\int_{L} \frac{1+y^{2} f(x y)}{y} \mathrm{~d} x+\frac{x}{y^{2}}\left[y^{2} f(x y)-1\right] \mathrm{d} y,
$$
\begin{CJK}{UTF8}{mj}其中\end{CJK} $L$ \begin{CJK}{UTF8}{mj}为上半平面\end{CJK} $(y>0)$ \begin{CJK}{UTF8}{mj}内以\end{CJK} $(2,3)$ \begin{CJK}{UTF8}{mj}为起点\end{CJK}, $(3,2)$ \begin{CJK}{UTF8}{mj}为终点的有向分段光滑曲线\end{CJK}.

\begin{enumerate}
  \setcounter{enumi}{6}
  \item \begin{CJK}{UTF8}{mj}计算\end{CJK} $I=\iint_{\Sigma} \frac{x \mathrm{~d} y \mathrm{~d} z+z^{2} \mathrm{~d} x \mathrm{~d} y}{\sqrt{x^{2}+y^{2}+z^{2}}}$, \begin{CJK}{UTF8}{mj}其中\end{CJK} $\Sigma$ \begin{CJK}{UTF8}{mj}为下半球面\end{CJK} $z=-\sqrt{1-x^{2}-y^{2}}$ \begin{CJK}{UTF8}{mj}的上侧\end{CJK}.
\end{enumerate}
\begin{CJK}{UTF8}{mj}三\end{CJK}、(\begin{CJK}{UTF8}{mj}本题\end{CJK} 10 \begin{CJK}{UTF8}{mj}分\end{CJK}) \begin{CJK}{UTF8}{mj}设函数\end{CJK} $z=f(x, y)$ \begin{CJK}{UTF8}{mj}具有二阶连续偏导数\end{CJK}, \begin{CJK}{UTF8}{mj}且\end{CJK} $f_{y} \neq 0$. \begin{CJK}{UTF8}{mj}证明\end{CJK}: \begin{CJK}{UTF8}{mj}对任意实数\end{CJK} $c, f(x, y)=c$ \begin{CJK}{UTF8}{mj}为一条直\end{CJK} \begin{CJK}{UTF8}{mj}线的充要条件是\end{CJK}
$$
\left(f_{y}\right)^{2} f_{x x}-2 f_{x} f_{y} f_{x y}+\left(f_{x}\right)^{2} f_{y y}=0 .
$$
\begin{CJK}{UTF8}{mj}四\end{CJK}、 (\begin{CJK}{UTF8}{mj}本题\end{CJK} 12 \begin{CJK}{UTF8}{mj}分\end{CJK}) \begin{CJK}{UTF8}{mj}函数\end{CJK} $x \sin \frac{1}{x}$ \begin{CJK}{UTF8}{mj}和\end{CJK} $\sin \frac{1}{x}$ \begin{CJK}{UTF8}{mj}在\end{CJK} $(0,+\infty)$ \begin{CJK}{UTF8}{mj}上是否一致连续\end{CJK}, \begin{CJK}{UTF8}{mj}并给出证明\end{CJK}.

\begin{CJK}{UTF8}{mj}五\end{CJK}、(\begin{CJK}{UTF8}{mj}本题\end{CJK} 12 \begin{CJK}{UTF8}{mj}分\end{CJK}) \begin{CJK}{UTF8}{mj}设偶函数\end{CJK} $f(x)$ \begin{CJK}{UTF8}{mj}的二阶导数\end{CJK} $f^{\prime \prime}(x)$ \begin{CJK}{UTF8}{mj}在\end{CJK} $x=0$ \begin{CJK}{UTF8}{mj}的某邻域内连续\end{CJK}, \begin{CJK}{UTF8}{mj}且\end{CJK} $f(0)=1, f^{\prime \prime}(0)=2$. \begin{CJK}{UTF8}{mj}证明\end{CJK}: \begin{CJK}{UTF8}{mj}级数\end{CJK} $\sum_{n=1}^{\infty}\left[f\left(\frac{1}{n}\right)-1\right]$ \begin{CJK}{UTF8}{mj}绝对收敛\end{CJK}.

\begin{CJK}{UTF8}{mj}六\end{CJK}、(\begin{CJK}{UTF8}{mj}本题\end{CJK} 10 \begin{CJK}{UTF8}{mj}分\end{CJK}) \begin{CJK}{UTF8}{mj}函数\end{CJK} $f:[0,1] \rightarrow(0,1)$ \begin{CJK}{UTF8}{mj}在\end{CJK} $[0,1]$ \begin{CJK}{UTF8}{mj}内可导\end{CJK}, \begin{CJK}{UTF8}{mj}且\end{CJK} $f^{\prime}(x) \neq 1$. \begin{CJK}{UTF8}{mj}证明\end{CJK}: \begin{CJK}{UTF8}{mj}方程\end{CJK} $f(x)=x$ \begin{CJK}{UTF8}{mj}在\end{CJK} $(0,1)$ \begin{CJK}{UTF8}{mj}内存在唯一\end{CJK} \begin{CJK}{UTF8}{mj}的实根\end{CJK}.

\begin{CJK}{UTF8}{mj}七\end{CJK}、 (\begin{CJK}{UTF8}{mj}本题\end{CJK} 15 \begin{CJK}{UTF8}{mj}分\end{CJK}) \begin{CJK}{UTF8}{mj}设\end{CJK} $f(x)$ \begin{CJK}{UTF8}{mj}在\end{CJK} $[0,1]$ \begin{CJK}{UTF8}{mj}上可积\end{CJK}, \begin{CJK}{UTF8}{mj}在\end{CJK} $x=1$ \begin{CJK}{UTF8}{mj}连续\end{CJK}, \begin{CJK}{UTF8}{mj}证明\end{CJK}:
$$
\lim _{n \rightarrow \infty} n \int_{0}^{1} x^{n} f(x) \mathrm{d} x=f(1) .
$$
\begin{CJK}{UTF8}{mj}八\end{CJK}、 (\begin{CJK}{UTF8}{mj}本题\end{CJK} 15 \begin{CJK}{UTF8}{mj}分\end{CJK}) \begin{CJK}{UTF8}{mj}设函数\end{CJK} $f(x, y)$ \begin{CJK}{UTF8}{mj}在区域\end{CJK} $D: x^{2}+y^{2} \leq 1$ \begin{CJK}{UTF8}{mj}上有二阶连续偏导数\end{CJK}, \begin{CJK}{UTF8}{mj}且\end{CJK} $\frac{\partial^{2} f}{\partial x^{2}}+\frac{\partial^{2} f}{\partial y^{2}}=e^{-\left(x^{2}+y^{2}\right)}$. \begin{CJK}{UTF8}{mj}证明\end{CJK}:
$$
\iint_{D}\left(x \frac{\partial f}{\partial x}+y \frac{\partial f}{\partial y}\right) \mathrm{d} x \mathrm{~d} y=\frac{\pi}{2 e}
$$

\section{4. 四川大学 2012 年研究生入学考试试题数学分析}
\begin{CJK}{UTF8}{mj}李扬\end{CJK}

\begin{CJK}{UTF8}{mj}微信公众号\end{CJK}: sxkyliyang

\begin{CJK}{UTF8}{mj}一\end{CJK}、 \begin{CJK}{UTF8}{mj}极限问题\end{CJK}(\begin{CJK}{UTF8}{mj}每题\end{CJK} 8 \begin{CJK}{UTF8}{mj}分\end{CJK}, \begin{CJK}{UTF8}{mj}共\end{CJK} 32 \begin{CJK}{UTF8}{mj}分\end{CJK}).

\begin{enumerate}
  \item \begin{CJK}{UTF8}{mj}设集合\end{CJK} $A \neq \varnothing, \alpha=\sup A, \alpha \notin A$. \begin{CJK}{UTF8}{mj}证明\end{CJK}: $A$ \begin{CJK}{UTF8}{mj}中存在严格单调递增数列\end{CJK} $\left\{x_{n}\right\}$, \begin{CJK}{UTF8}{mj}满足\end{CJK} $\lim _{n \rightarrow \infty} x_{n}=\alpha$.

  \item \begin{CJK}{UTF8}{mj}设\end{CJK} $x_{0}=a, x_{1}=b(0<a<b)$, \begin{CJK}{UTF8}{mj}且\end{CJK} $x_{n+1}=\sqrt{x_{n} x_{n-1}},(n \geq 1)$. \begin{CJK}{UTF8}{mj}证明\end{CJK}: $\left\{x_{n}\right\}$ \begin{CJK}{UTF8}{mj}收敛\end{CJK}, \begin{CJK}{UTF8}{mj}并求\end{CJK} $\lim _{n \rightarrow \infty} x_{n}$.

  \item \begin{CJK}{UTF8}{mj}求\end{CJK} $\lim _{x \rightarrow 0} \frac{e^{x^{2}}-x \sin x-1}{x^{4}}$.

  \item \begin{CJK}{UTF8}{mj}求\end{CJK} $\lim _{x \rightarrow 0} \frac{\sqrt{\cos x}-\sqrt[3]{\cos x}}{\ln \left(x^{2}+1\right)}$.

\end{enumerate}
\begin{CJK}{UTF8}{mj}二\end{CJK}、 \begin{CJK}{UTF8}{mj}计算积分\end{CJK} (\begin{CJK}{UTF8}{mj}每题\end{CJK} 8 \begin{CJK}{UTF8}{mj}分\end{CJK}, \begin{CJK}{UTF8}{mj}共\end{CJK} 40 \begin{CJK}{UTF8}{mj}分\end{CJK}).

\begin{enumerate}
  \item \begin{CJK}{UTF8}{mj}求\end{CJK} $\int_{0}^{1} \frac{x^{2011}-x^{1005}}{\ln x} \mathrm{~d} x$.

  \item \begin{CJK}{UTF8}{mj}设\end{CJK} $f(x)$ \begin{CJK}{UTF8}{mj}在\end{CJK} $[0,1]$ \begin{CJK}{UTF8}{mj}上可积\end{CJK}, \begin{CJK}{UTF8}{mj}且满足\end{CJK} $x^{2}(\ln x)^{2}-f(x)=\int_{0}^{1} f(x) \mathrm{d} x$, \begin{CJK}{UTF8}{mj}求\end{CJK} $\int_{0}^{1} f(x) \mathrm{d} x$ \begin{CJK}{UTF8}{mj}的值\end{CJK}.

  \item \begin{CJK}{UTF8}{mj}计算\end{CJK} $\int_{L}\left(x^{2}+2 y+z\right) \mathrm{d} s$, \begin{CJK}{UTF8}{mj}其中\end{CJK} $L$ \begin{CJK}{UTF8}{mj}为球面\end{CJK} $x^{2}+y^{2}+z^{2}=1$ \begin{CJK}{UTF8}{mj}与平面\end{CJK} $x+y+z=0$ \begin{CJK}{UTF8}{mj}的交线\end{CJK}.

  \item \begin{CJK}{UTF8}{mj}计算\end{CJK} $\int_{L} \frac{x \mathrm{~d} y-y \mathrm{~d} x}{x^{2}+2 y^{2}}$, \begin{CJK}{UTF8}{mj}其中\end{CJK} $L$ \begin{CJK}{UTF8}{mj}为圆周\end{CJK} $(x-2)^{2}+y^{2}=r^{2}(r>0, r \neq 2)$, \begin{CJK}{UTF8}{mj}取逆时针方向\end{CJK}.

  \item \begin{CJK}{UTF8}{mj}计算\end{CJK} $\iint_{S}(x+2 y) \mathrm{d} y \mathrm{~d} z+(y+z) \mathrm{d} z \mathrm{~d} x+(z+2) \mathrm{d} x \mathrm{~d} y$, \begin{CJK}{UTF8}{mj}其中\end{CJK} $S$ \begin{CJK}{UTF8}{mj}为柇球面\end{CJK} $\frac{x^{2}}{a^{2}}+\frac{y^{2}}{b^{2}}+\frac{z^{2}}{c^{2}}=1$ \begin{CJK}{UTF8}{mj}的上半部分\end{CJK}, \begin{CJK}{UTF8}{mj}其方向为下侧\end{CJK}.

\end{enumerate}
\begin{CJK}{UTF8}{mj}三\end{CJK}、(15 \begin{CJK}{UTF8}{mj}分\end{CJK})\begin{CJK}{UTF8}{mj}设正项级数\end{CJK} $\sum_{n=1}^{\infty} a_{n}$ \begin{CJK}{UTF8}{mj}发散\end{CJK}, \begin{CJK}{UTF8}{mj}且\end{CJK} $S_{n}=\sum_{k=1}^{n} a_{k}$, \begin{CJK}{UTF8}{mj}讨论\end{CJK} $\sum_{n=1}^{\infty} \frac{a_{n}}{S_{n}^{\sigma}}$ \begin{CJK}{UTF8}{mj}的敛散性\end{CJK}, \begin{CJK}{UTF8}{mj}其中\end{CJK} $\alpha>0$.

\begin{CJK}{UTF8}{mj}四\end{CJK}、(15 \begin{CJK}{UTF8}{mj}分\end{CJK}) \begin{CJK}{UTF8}{mj}讨论函数\end{CJK}
$$
f(x, y)= \begin{cases}(x+y)^{2} \sin \frac{1}{x^{2}+y^{2}}, & (x, y) \neq(0,0) \\ 0, & (x, y)=(0,0)\end{cases}
$$
\begin{CJK}{UTF8}{mj}的偏导数\end{CJK} $f_{x}, f_{y}$ \begin{CJK}{UTF8}{mj}在原点的连续性和\end{CJK} $f$ \begin{CJK}{UTF8}{mj}在原点的可微性\end{CJK}.

\begin{CJK}{UTF8}{mj}五\end{CJK}、 $\left(15\right.$ \begin{CJK}{UTF8}{mj}分\end{CJK}) \begin{CJK}{UTF8}{mj}设\end{CJK} $f(x)$ \begin{CJK}{UTF8}{mj}在\end{CJK} $(0,2)$ \begin{CJK}{UTF8}{mj}上二阶可导\end{CJK}, $f^{\prime \prime}(1)>0$. \begin{CJK}{UTF8}{mj}证明\end{CJK}: \begin{CJK}{UTF8}{mj}存在\end{CJK} $x_{1}, x_{2} \in(0,2)$, \begin{CJK}{UTF8}{mj}使得\end{CJK}
$$
f^{\prime}(1)=\frac{f\left(x_{2}\right)-f\left(x_{1}\right)}{x_{2}-x_{1}}
$$
\begin{CJK}{UTF8}{mj}六\end{CJK}、 ( 12 \begin{CJK}{UTF8}{mj}分\end{CJK}) \begin{CJK}{UTF8}{mj}设连续函数\end{CJK} $f: \mathbb{R} \rightarrow \mathbb{R}$ \begin{CJK}{UTF8}{mj}在所有无理数处取有理数值\end{CJK}, \begin{CJK}{UTF8}{mj}且\end{CJK} $f(0)=1$, \begin{CJK}{UTF8}{mj}求\end{CJK} $f(x)$.

\begin{CJK}{UTF8}{mj}七\end{CJK}、 (\begin{CJK}{UTF8}{mj}每小题\end{CJK} 7 \begin{CJK}{UTF8}{mj}分\end{CJK}, \begin{CJK}{UTF8}{mj}共\end{CJK} 21 \begin{CJK}{UTF8}{mj}分\end{CJK}) \begin{CJK}{UTF8}{mj}设\end{CJK} $f(x)=\int_{1}^{+\infty} \frac{\sin x t}{t\left(1+t^{2}\right)} \mathrm{d} t, x \in(-\infty,+\infty)$.

\begin{enumerate}
  \item \begin{CJK}{UTF8}{mj}证明\end{CJK}: \begin{CJK}{UTF8}{mj}积分\end{CJK} $\int_{1}^{+\infty} \frac{\sin x t}{t\left(1+t^{2}\right)} \mathrm{d} t$ \begin{CJK}{UTF8}{mj}关于\end{CJK} $x$ \begin{CJK}{UTF8}{mj}在\end{CJK} $(-\infty,+\infty)$ \begin{CJK}{UTF8}{mj}上一致收敛\end{CJK}.

  \item \begin{CJK}{UTF8}{mj}证明\end{CJK}: $\lim _{x \rightarrow+\infty} f(x)=0$.

  \item \begin{CJK}{UTF8}{mj}证明\end{CJK}: $f(x)$ \begin{CJK}{UTF8}{mj}在\end{CJK} $(-\infty,+\infty)$ \begin{CJK}{UTF8}{mj}上一致连续\end{CJK}.

\end{enumerate}
\section{5. 四川大学 2013 年研究生入学考试试题数学分析}
\begin{CJK}{UTF8}{mj}李扬\end{CJK}

\begin{CJK}{UTF8}{mj}微信公众号\end{CJK}: sxkyliyang

\begin{CJK}{UTF8}{mj}一\end{CJK}、 \begin{CJK}{UTF8}{mj}计算\end{CJK} (80 \begin{CJK}{UTF8}{mj}分\end{CJK}, \begin{CJK}{UTF8}{mj}每小题\end{CJK} 10 \begin{CJK}{UTF8}{mj}分\end{CJK})

\begin{enumerate}
  \item \begin{CJK}{UTF8}{mj}设\end{CJK} $m$ \begin{CJK}{UTF8}{mj}为正整数\end{CJK}, \begin{CJK}{UTF8}{mj}求\end{CJK} $\lim _{n \rightarrow \infty}\left(\frac{1}{m} \sum_{k=1}^{m} \sqrt[n]{k}\right)^{n}$.

  \item \begin{CJK}{UTF8}{mj}求\end{CJK} $\lim _{x \rightarrow \infty} \int_{x}^{x+1} \frac{\sin t^{2}}{t+\cos t} \mathrm{~d} t$.

  \item \begin{CJK}{UTF8}{mj}求\end{CJK} $\lim _{x \rightarrow 0} \frac{(2 x-\sin 2 x) \arcsin x}{e^{-\frac{x^{2}}{2}}-\cos x}$.

  \item \begin{CJK}{UTF8}{mj}求\end{CJK} $\int_{0}^{\pi} \frac{2 \cos x}{\sin x+\cos x} \mathrm{~d} x$.

  \item \begin{CJK}{UTF8}{mj}求球体\end{CJK} $x^{2}+y^{2}+z^{2} \leq 1$ \begin{CJK}{UTF8}{mj}被柱面\end{CJK} $x^{2}+y^{2}=x$ \begin{CJK}{UTF8}{mj}所截出部分的体积\end{CJK}.

  \item \begin{CJK}{UTF8}{mj}求流速为\end{CJK} $\vec{v}=(x, y, z)$ \begin{CJK}{UTF8}{mj}的不可压缩流体单位时间内穿过圆雉体\end{CJK} $x^{2}+y^{2} \leq z^{2}(0 \leq z \leq h)$ \begin{CJK}{UTF8}{mj}表面的流量\end{CJK}, \begin{CJK}{UTF8}{mj}表\end{CJK} \begin{CJK}{UTF8}{mj}面法向量朝外\end{CJK}.

  \item \begin{CJK}{UTF8}{mj}求\end{CJK} $\int_{L} \frac{x \mathrm{~d} y-y \mathrm{~d} x}{x^{2}+2 y^{2}}$, \begin{CJK}{UTF8}{mj}其中\end{CJK} $L$ \begin{CJK}{UTF8}{mj}为不过原点的简单闭曲线\end{CJK}, $L$ \begin{CJK}{UTF8}{mj}取顺时针方向\end{CJK}.

  \item \begin{CJK}{UTF8}{mj}求幂级数\end{CJK} $\sum_{n=1}^{\infty}(n+1)(x+1)^{n}$ \begin{CJK}{UTF8}{mj}的收敛域与和函数\end{CJK}.

\end{enumerate}
\begin{CJK}{UTF8}{mj}二\end{CJK}、 ( 20 \begin{CJK}{UTF8}{mj}分\end{CJK}, \begin{CJK}{UTF8}{mj}每小题\end{CJK} 5 \begin{CJK}{UTF8}{mj}分\end{CJK}) \begin{CJK}{UTF8}{mj}判断下列命题是否正确\end{CJK}. \begin{CJK}{UTF8}{mj}若正确\end{CJK}, \begin{CJK}{UTF8}{mj}给出证明\end{CJK}; \begin{CJK}{UTF8}{mj}若不正确\end{CJK}, \begin{CJK}{UTF8}{mj}举出反例\end{CJK}.

\begin{enumerate}
  \item \begin{CJK}{UTF8}{mj}对任意\end{CJK} $\varepsilon>0, f$ \begin{CJK}{UTF8}{mj}在\end{CJK} $[a+\varepsilon, b-\varepsilon]$ \begin{CJK}{UTF8}{mj}上连续\end{CJK}, \begin{CJK}{UTF8}{mj}则\end{CJK} $f$ \begin{CJK}{UTF8}{mj}在\end{CJK} $(a, b)$ \begin{CJK}{UTF8}{mj}上连续\end{CJK}.

  \item \begin{CJK}{UTF8}{mj}函数\end{CJK} $f(x)$ \begin{CJK}{UTF8}{mj}在\end{CJK} $\mathbb{R}$ \begin{CJK}{UTF8}{mj}上可导\end{CJK}, \begin{CJK}{UTF8}{mj}则\end{CJK} $f^{\prime}(x)$ \begin{CJK}{UTF8}{mj}在\end{CJK} $\mathbb{R}$ \begin{CJK}{UTF8}{mj}上连续\end{CJK}.

  \item $f, g$ \begin{CJK}{UTF8}{mj}为\end{CJK} $\mathbb{R}$ \begin{CJK}{UTF8}{mj}上的连续函数\end{CJK}, \begin{CJK}{UTF8}{mj}则\end{CJK} $f(\min \{g(x), 1\})$ \begin{CJK}{UTF8}{mj}关于\end{CJK} $x$ \begin{CJK}{UTF8}{mj}在\end{CJK} $\mathbb{R}$ \begin{CJK}{UTF8}{mj}上一致连续\end{CJK}.

  \item $f: \mathbb{R}^{2} \rightarrow \mathbb{R}^{2}$ \begin{CJK}{UTF8}{mj}可微\end{CJK}, $x, y \in \mathbb{R}^{2}$, \begin{CJK}{UTF8}{mj}则存在\end{CJK} $\theta \in(0,1)$ \begin{CJK}{UTF8}{mj}使得\end{CJK}

\end{enumerate}
$$
f(y)-f(x)=f^{\prime}(x+\theta(y-x))(y-x) .
$$
\begin{CJK}{UTF8}{mj}三\end{CJK}、 $(10$ \begin{CJK}{UTF8}{mj}分\end{CJK} $)$ \begin{CJK}{UTF8}{mj}若正项级数\end{CJK} $\sum_{n=1}^{\infty} a_{n}$ \begin{CJK}{UTF8}{mj}收敛\end{CJK}, \begin{CJK}{UTF8}{mj}且数列\end{CJK} $\left\{a_{n}\right\}$ \begin{CJK}{UTF8}{mj}单调\end{CJK}, \begin{CJK}{UTF8}{mj}证明\end{CJK}: $\lim _{n \rightarrow \infty} n a_{n}=0$.

\begin{CJK}{UTF8}{mj}四\end{CJK}、(15 \begin{CJK}{UTF8}{mj}分\end{CJK}) \begin{CJK}{UTF8}{mj}讨论积分\end{CJK} $\int_{0}^{1} \frac{1}{x^{p}|\ln x|^{q}} \mathrm{~d} x$ \begin{CJK}{UTF8}{mj}的敛散性\end{CJK}, \begin{CJK}{UTF8}{mj}其中\end{CJK} $p, q \in(0,+\infty)$.

\begin{CJK}{UTF8}{mj}五\end{CJK}、( 10 \begin{CJK}{UTF8}{mj}分\end{CJK}) \begin{CJK}{UTF8}{mj}设\end{CJK} $f_{x}(x, y)$ \begin{CJK}{UTF8}{mj}在\end{CJK} $(0,0)$ \begin{CJK}{UTF8}{mj}处连续\end{CJK}, $f_{y}(x, y)$ \begin{CJK}{UTF8}{mj}在\end{CJK} $(0,0)$ \begin{CJK}{UTF8}{mj}处存在\end{CJK}, \begin{CJK}{UTF8}{mj}证明\end{CJK}: $f(x, y)$ \begin{CJK}{UTF8}{mj}在\end{CJK} $(0,0)$ \begin{CJK}{UTF8}{mj}处可微\end{CJK}.

\begin{CJK}{UTF8}{mj}六\end{CJK}、 (15 \begin{CJK}{UTF8}{mj}分\end{CJK}) \begin{CJK}{UTF8}{mj}函数\end{CJK} $f, g$ \begin{CJK}{UTF8}{mj}在\end{CJK} $(-1,1)$ \begin{CJK}{UTF8}{mj}上可导\end{CJK}, \begin{CJK}{UTF8}{mj}且对任意的\end{CJK} $x \in(-1,1)$ \begin{CJK}{UTF8}{mj}有\end{CJK} $g^{\prime}(x) \neq 0$. \begin{CJK}{UTF8}{mj}若\end{CJK}
$$
\lim _{x \rightarrow 0} g(x)=\infty, \quad \lim _{x \rightarrow 0} \frac{f^{\prime}(x)}{g^{\prime}(x)}=1,
$$
\begin{CJK}{UTF8}{mj}证明\end{CJK}: \begin{CJK}{UTF8}{mj}当\end{CJK} $x \rightarrow 0$ \begin{CJK}{UTF8}{mj}时\end{CJK}, $f(x)$ \begin{CJK}{UTF8}{mj}和\end{CJK} $g(x)$ \begin{CJK}{UTF8}{mj}是等价无穷大量\end{CJK}.

\section{6. 四川大学 2014 年研究生入学考试试题数学分析}
\begin{CJK}{UTF8}{mj}李扬\end{CJK}

\begin{CJK}{UTF8}{mj}微信公众号\end{CJK}: sxkyliyang

\begin{CJK}{UTF8}{mj}一\end{CJK}、\begin{CJK}{UTF8}{mj}计算\end{CJK}(\begin{CJK}{UTF8}{mj}每小题\end{CJK} 10 \begin{CJK}{UTF8}{mj}分\end{CJK}, \begin{CJK}{UTF8}{mj}共\end{CJK} 70 \begin{CJK}{UTF8}{mj}分\end{CJK}).

\begin{enumerate}
  \item \begin{CJK}{UTF8}{mj}求极限\end{CJK} $\lim _{n \rightarrow \infty} \prod_{k=1}^{n} \frac{4 k-3}{4 k}$.

  \item \begin{CJK}{UTF8}{mj}计算\end{CJK} $\lim _{n \rightarrow \infty} \sqrt[n]{n !} \ln \left(1+\frac{1}{n}\right)$.

  \item \begin{CJK}{UTF8}{mj}对任意\end{CJK} $A>0, f(x)$ \begin{CJK}{UTF8}{mj}在\end{CJK} $[0, A]$ \begin{CJK}{UTF8}{mj}上可积\end{CJK}, \begin{CJK}{UTF8}{mj}且\end{CJK} $\lim _{x \rightarrow+\infty} f(x)=1$, \begin{CJK}{UTF8}{mj}求\end{CJK} $\lim _{T \rightarrow+\infty} \frac{1}{T} \int_{0}^{T} f(x) \mathrm{d} x$.

  \item \begin{CJK}{UTF8}{mj}设\end{CJK} $z=z(x, y)$ \begin{CJK}{UTF8}{mj}由方程\end{CJK} $e^{-x y}-2 z+e^{z}=0$ \begin{CJK}{UTF8}{mj}确定\end{CJK}, \begin{CJK}{UTF8}{mj}求\end{CJK} $\frac{\partial^{2} z}{\partial x \partial y}$.

  \item \begin{CJK}{UTF8}{mj}求椭球面\end{CJK} $\frac{x^{2}}{3}+\frac{y^{2}}{4}+\frac{z^{2}}{9}=1(x>0, y>0, z>0)$ \begin{CJK}{UTF8}{mj}的切平面与三坐标平面所围成的几何体的最小体积\end{CJK}.

  \item \begin{CJK}{UTF8}{mj}设\end{CJK} $f(t)$ \begin{CJK}{UTF8}{mj}连续\end{CJK}, $f(t) \sim t^{2}(t \rightarrow 0), F(t)=\iint_{x^{2}+y^{2} \leq 1} f\left(x^{2}+y^{2}\right) \mathrm{d} x \mathrm{~d} y(t \geq 0)$, \begin{CJK}{UTF8}{mj}求\end{CJK} $F^{\prime \prime}(0+)$.

  \item \begin{CJK}{UTF8}{mj}计算\end{CJK}

\end{enumerate}
$$
\oint_{L}\left(y^{2}+z^{2}\right) \mathrm{d} x+\left(z^{2}+x^{2}\right) \mathrm{d} y+\left(x^{2}+y^{2}\right) \mathrm{d} z
$$
\begin{CJK}{UTF8}{mj}其中\end{CJK} $L$ \begin{CJK}{UTF8}{mj}是曲面\end{CJK} $x^{2}+y^{2}+z^{2}=4 x$ \begin{CJK}{UTF8}{mj}与\end{CJK} $x^{2}+y^{2}=2 x$ \begin{CJK}{UTF8}{mj}的交线\end{CJK} $z \geq 0$ \begin{CJK}{UTF8}{mj}的部分\end{CJK}, \begin{CJK}{UTF8}{mj}曲线的方向规定从\end{CJK} $z$ \begin{CJK}{UTF8}{mj}轴正向看\end{CJK} $L$ \begin{CJK}{UTF8}{mj}为逆时针方向\end{CJK}.

\begin{CJK}{UTF8}{mj}二\end{CJK}、 (10 \begin{CJK}{UTF8}{mj}分\end{CJK}) \begin{CJK}{UTF8}{mj}设\end{CJK} $f(x)$ \begin{CJK}{UTF8}{mj}在\end{CJK} $[a, b]$ \begin{CJK}{UTF8}{mj}上连续\end{CJK}, \begin{CJK}{UTF8}{mj}且对任意的\end{CJK} $x \in[a, b], f(x) \neq 0$. \begin{CJK}{UTF8}{mj}用定义证明\end{CJK} $\frac{1}{f(x)}$ \begin{CJK}{UTF8}{mj}在\end{CJK} $[a, b]$ \begin{CJK}{UTF8}{mj}上一致连续\end{CJK}.

\begin{CJK}{UTF8}{mj}三\end{CJK}、 $\left(10\right.$ \begin{CJK}{UTF8}{mj}分\end{CJK}) $f(x)$ \begin{CJK}{UTF8}{mj}在\end{CJK} $[0,1]$ \begin{CJK}{UTF8}{mj}上可导\end{CJK}, $f(1)=\int_{0}^{1} f(x) e^{1-x^{2}} \mathrm{~d} x$, \begin{CJK}{UTF8}{mj}证明\end{CJK}: \begin{CJK}{UTF8}{mj}存在\end{CJK} $a \in(0,1)$ \begin{CJK}{UTF8}{mj}使得\end{CJK} $f^{\prime}(a)=2 a f(a)$.

\begin{CJK}{UTF8}{mj}四\end{CJK}、 $(10$ \begin{CJK}{UTF8}{mj}分\end{CJK} $)$ \begin{CJK}{UTF8}{mj}令\end{CJK} $f(x)=\frac{1}{x}, a_{2 n-1}=f(n), a_{2 n}=\int_{n}^{n+1} f(x) \mathrm{d} x$, \begin{CJK}{UTF8}{mj}讨论\end{CJK} $\sum_{n=1}^{\infty}(-1)^{n} a_{n}$ \begin{CJK}{UTF8}{mj}的敛散性\end{CJK}(\begin{CJK}{UTF8}{mj}包括收敛\end{CJK}、\begin{CJK}{UTF8}{mj}绝对收敛\end{CJK} \begin{CJK}{UTF8}{mj}和条件收敛\end{CJK}).

\begin{CJK}{UTF8}{mj}五\end{CJK}、(10 \begin{CJK}{UTF8}{mj}分\end{CJK}) \begin{CJK}{UTF8}{mj}证明\end{CJK}: $\int_{0}^{+\infty} \frac{\cos x^{2}}{1+x^{y}} \mathrm{~d} x$ \begin{CJK}{UTF8}{mj}当\end{CJK} $y \in[0,+\infty)$ \begin{CJK}{UTF8}{mj}时一致收敛\end{CJK}.

\begin{CJK}{UTF8}{mj}六\end{CJK}、(10 \begin{CJK}{UTF8}{mj}分\end{CJK}) \begin{CJK}{UTF8}{mj}设\end{CJK} $f(t)$ \begin{CJK}{UTF8}{mj}在\end{CJK} $[0,1]$ \begin{CJK}{UTF8}{mj}连续\end{CJK}, \begin{CJK}{UTF8}{mj}证明\end{CJK}: $\iint_{S} f\left(x^{2}+y^{2}\right) \mathrm{d} S=2 \pi \int_{-1}^{1} f\left(1-t^{2}\right) \mathrm{d} t$, \begin{CJK}{UTF8}{mj}其中\end{CJK} $S$ \begin{CJK}{UTF8}{mj}为球面\end{CJK} $x^{2}+y^{2}+z^{2}=1$.

\begin{CJK}{UTF8}{mj}七\end{CJK}、 (15 \begin{CJK}{UTF8}{mj}分\end{CJK}) \begin{CJK}{UTF8}{mj}设\end{CJK} $f(x)$ \begin{CJK}{UTF8}{mj}和\end{CJK} $g(x)$ \begin{CJK}{UTF8}{mj}在\end{CJK} $x_{0}$ \begin{CJK}{UTF8}{mj}点附近恒正\end{CJK}, \begin{CJK}{UTF8}{mj}判断下面命题\end{CJK} $(A)$ \begin{CJK}{UTF8}{mj}及其逆命题是否成立\end{CJK}. \begin{CJK}{UTF8}{mj}若成立\end{CJK}, \begin{CJK}{UTF8}{mj}给出证明\end{CJK}; \begin{CJK}{UTF8}{mj}若不\end{CJK} \begin{CJK}{UTF8}{mj}成立\end{CJK}, \begin{CJK}{UTF8}{mj}举例说明\end{CJK}.

$(A)$ \begin{CJK}{UTF8}{mj}当\end{CJK} $x \rightarrow x_{0}$ \begin{CJK}{UTF8}{mj}时\end{CJK}, \begin{CJK}{UTF8}{mj}若\end{CJK} $f(x)$ \begin{CJK}{UTF8}{mj}和\end{CJK} $g(x)$ \begin{CJK}{UTF8}{mj}为等价无穷小量\end{CJK}, \begin{CJK}{UTF8}{mj}则\end{CJK} $\ln f(x)$ \begin{CJK}{UTF8}{mj}和\end{CJK} $\ln g(x)$ \begin{CJK}{UTF8}{mj}为等价无穷大量\end{CJK}.

\begin{CJK}{UTF8}{mj}八\end{CJK}、 (15 \begin{CJK}{UTF8}{mj}分\end{CJK}) \begin{CJK}{UTF8}{mj}叙述数列的\end{CJK} Cauchy \begin{CJK}{UTF8}{mj}收敛原理\end{CJK}, \begin{CJK}{UTF8}{mj}并用确界存在定理证明之\end{CJK}.

\section{7. 四川大学 2015 年研究生入学考试试题数学分析}
\begin{CJK}{UTF8}{mj}李扬\end{CJK}

\begin{CJK}{UTF8}{mj}微信公众号\end{CJK}: sxkyliyang

\begin{CJK}{UTF8}{mj}一\end{CJK}、 \begin{CJK}{UTF8}{mj}计算\end{CJK}(\begin{CJK}{UTF8}{mj}每小题\end{CJK} 8 \begin{CJK}{UTF8}{mj}分\end{CJK}, \begin{CJK}{UTF8}{mj}共\end{CJK} 72 \begin{CJK}{UTF8}{mj}分\end{CJK})

\begin{enumerate}
  \item \begin{CJK}{UTF8}{mj}求\end{CJK} $\lim _{x \rightarrow 0}\left(\frac{\cos x}{\cos 2 x}\right)^{x^{-2}}$.

  \item \begin{CJK}{UTF8}{mj}设\end{CJK} $x_{0}=1, x_{1}=2, x_{n+1}=\frac{x_{n}+x_{n+1}}{2}, n \geq 1$, \begin{CJK}{UTF8}{mj}求极限\end{CJK} $\left\{x_{n}\right\}$ \begin{CJK}{UTF8}{mj}的极限\end{CJK}.

  \item \begin{CJK}{UTF8}{mj}求\end{CJK} $\lim _{\substack{x \rightarrow 0 \\ y \rightarrow 0}} \frac{\cos x+\cos y-2}{x^{2}+y^{2}}$.

  \item \begin{CJK}{UTF8}{mj}设\end{CJK} $f$ \begin{CJK}{UTF8}{mj}是\end{CJK} $\mathbb{R}$ \begin{CJK}{UTF8}{mj}上周期为\end{CJK} $T$ \begin{CJK}{UTF8}{mj}的连续函数\end{CJK}, \begin{CJK}{UTF8}{mj}且\end{CJK} $\int_{0}^{T} f(t) \mathrm{d} t=a$, \begin{CJK}{UTF8}{mj}求\end{CJK} $\lim _{x \rightarrow+\infty} \frac{1}{x} \int_{0}^{x} f(t) \mathrm{d} t$.

  \item \begin{CJK}{UTF8}{mj}求级数\end{CJK} $\sum_{n=1}^{\infty} \frac{n^{2}+1}{2^{n}}$ \begin{CJK}{UTF8}{mj}之和\end{CJK}.

  \item \begin{CJK}{UTF8}{mj}求原点到曲线\end{CJK}

\end{enumerate}
$$
\left\{\begin{array}{l}
z=x^{2}+y^{2} \\
x+y+z=1
\end{array}\right.
$$
\begin{CJK}{UTF8}{mj}的最短距离\end{CJK}.

\begin{enumerate}
  \setcounter{enumi}{7}
  \item \begin{CJK}{UTF8}{mj}求积分\end{CJK} $\iint_{|x|+|y| \leq 1}\left(x^{2}-y^{2}\right)^{3} \mathrm{~d} x \mathrm{~d} y$.

  \item \begin{CJK}{UTF8}{mj}求积分\end{CJK} $\oint_{L} \frac{y^{2} \mathrm{~d} x-x^{2} \mathrm{~d} y}{x^{3}+y^{3}}, L: x^{2}+y^{2}=1$, \begin{CJK}{UTF8}{mj}沿逆时针方向\end{CJK}.

  \item \begin{CJK}{UTF8}{mj}求积分\end{CJK} $\iint_{S} z^{2} \mathrm{~d} x \mathrm{~d} y$, \begin{CJK}{UTF8}{mj}其中\end{CJK} $S$ \begin{CJK}{UTF8}{mj}是椭球面\end{CJK} $\frac{x^{2}}{4}+\frac{y^{2}}{9}+\frac{z^{2}}{16}=1$ \begin{CJK}{UTF8}{mj}在\end{CJK} $z>0$ \begin{CJK}{UTF8}{mj}的部分\end{CJK}, \begin{CJK}{UTF8}{mj}取外法线方向\end{CJK}.

\end{enumerate}
\begin{CJK}{UTF8}{mj}二\end{CJK}、( 16 \begin{CJK}{UTF8}{mj}分\end{CJK}, \begin{CJK}{UTF8}{mj}每小题\end{CJK} 8 \begin{CJK}{UTF8}{mj}分\end{CJK}) \begin{CJK}{UTF8}{mj}判断下列命题是否正确\end{CJK}. \begin{CJK}{UTF8}{mj}若正确\end{CJK}, \begin{CJK}{UTF8}{mj}给出证明\end{CJK}; \begin{CJK}{UTF8}{mj}若不正确\end{CJK}, \begin{CJK}{UTF8}{mj}举出反例\end{CJK}.

\begin{enumerate}
  \item \begin{CJK}{UTF8}{mj}函数\end{CJK} $f(x)$ \begin{CJK}{UTF8}{mj}在\end{CJK} $\mathbb{R}$ \begin{CJK}{UTF8}{mj}上可导\end{CJK}, \begin{CJK}{UTF8}{mj}则\end{CJK} $f^{\prime}(x)$ \begin{CJK}{UTF8}{mj}在\end{CJK} $\mathbb{R}$ \begin{CJK}{UTF8}{mj}上连续\end{CJK}.

  \item \begin{CJK}{UTF8}{mj}若\end{CJK} $\lim _{\substack{x \rightarrow x_{0} \\ y \rightarrow y_{0}}} f(x, y)$ \begin{CJK}{UTF8}{mj}存在\end{CJK}, \begin{CJK}{UTF8}{mj}则\end{CJK} $\lim _{x \rightarrow x_{0}} f(x, y)$ \begin{CJK}{UTF8}{mj}和\end{CJK} $\lim _{y \rightarrow y_{0}} f(x, y)$ \begin{CJK}{UTF8}{mj}都存在\end{CJK}.

\end{enumerate}
\begin{CJK}{UTF8}{mj}三\end{CJK}、 $\left(10\right.$ \begin{CJK}{UTF8}{mj}分\end{CJK}) \begin{CJK}{UTF8}{mj}设\end{CJK} $f(x)$ \begin{CJK}{UTF8}{mj}在\end{CJK} $[0,+\infty)$ \begin{CJK}{UTF8}{mj}上导数有界\end{CJK}, \begin{CJK}{UTF8}{mj}且\end{CJK} $\int_{0}^{+\infty} f(x) \mathrm{d} x$ \begin{CJK}{UTF8}{mj}收敛\end{CJK}, \begin{CJK}{UTF8}{mj}证明\end{CJK}: $\lim _{x \rightarrow+\infty} f(x)=0$.

\begin{CJK}{UTF8}{mj}四\end{CJK}、 (10 \begin{CJK}{UTF8}{mj}分\end{CJK}) \begin{CJK}{UTF8}{mj}设\end{CJK} $a>0, b>0$, \begin{CJK}{UTF8}{mj}证明\end{CJK}: $\frac{a^{n}+b^{n}}{2} \geq\left(\frac{a+b}{2}\right)^{n}$.

\begin{CJK}{UTF8}{mj}五\end{CJK}、(12 \begin{CJK}{UTF8}{mj}分\end{CJK}) $f(x)$ \begin{CJK}{UTF8}{mj}在\end{CJK} $[0,1]$ \begin{CJK}{UTF8}{mj}可导\end{CJK}, \begin{CJK}{UTF8}{mj}且\end{CJK} $f(0)=0, f(1)=1$, \begin{CJK}{UTF8}{mj}证明\end{CJK}: \begin{CJK}{UTF8}{mj}存在\end{CJK} $a, b \in(0,1)$ \begin{CJK}{UTF8}{mj}使得\end{CJK} $\frac{1}{f^{\prime}(a)}+\frac{1}{f^{\prime}(b)}=2$.

\begin{CJK}{UTF8}{mj}六\end{CJK}、 (15 \begin{CJK}{UTF8}{mj}分\end{CJK}) \begin{CJK}{UTF8}{mj}设函数列\end{CJK} $f_{n}(x)$ \begin{CJK}{UTF8}{mj}在\end{CJK} $[a, b]$ \begin{CJK}{UTF8}{mj}上可导\end{CJK}, \begin{CJK}{UTF8}{mj}且存在常数\end{CJK} $M$ \begin{CJK}{UTF8}{mj}使得对任意\end{CJK} $n$ \begin{CJK}{UTF8}{mj}和\end{CJK} $x \in[a, b]$ \begin{CJK}{UTF8}{mj}有\end{CJK} $\left|f_{n}^{\prime}(x)\right| \leq M$, \begin{CJK}{UTF8}{mj}证\end{CJK} \begin{CJK}{UTF8}{mj}明\end{CJK}: \begin{CJK}{UTF8}{mj}如果对任意\end{CJK} $x \in[a, b]$ \begin{CJK}{UTF8}{mj}数列\end{CJK} $\left\{f_{n}(x)\right\}$ \begin{CJK}{UTF8}{mj}收玫\end{CJK}, \begin{CJK}{UTF8}{mj}则函数列\end{CJK} $\left\{f_{n}(x)\right\}$ \begin{CJK}{UTF8}{mj}在\end{CJK} $[a, b]$ \begin{CJK}{UTF8}{mj}上一致收玫\end{CJK}.

\begin{CJK}{UTF8}{mj}七\end{CJK}、 $(15$ \begin{CJK}{UTF8}{mj}分\end{CJK} $)$ \begin{CJK}{UTF8}{mj}设\end{CJK} $f(x)$ \begin{CJK}{UTF8}{mj}在\end{CJK} $(0,1)$ \begin{CJK}{UTF8}{mj}连续\end{CJK}, \begin{CJK}{UTF8}{mj}且\end{CJK} $(0,1)$ \begin{CJK}{UTF8}{mj}中两序列\end{CJK} $\left\{a_{n}\right\},\left\{b_{n}\right\}$ \begin{CJK}{UTF8}{mj}使得\end{CJK}
$$
\lim _{n \rightarrow \infty} f\left(a_{n}\right)=a<b=\lim _{n \rightarrow \infty} f\left(b_{n}\right) .
$$
\begin{CJK}{UTF8}{mj}证明\end{CJK}: \begin{CJK}{UTF8}{mj}对任意\end{CJK} $c \in(a, b),(0,1)$ \begin{CJK}{UTF8}{mj}中存在序列\end{CJK} $\left\{c_{n}\right\}$ \begin{CJK}{UTF8}{mj}使得\end{CJK} $\lim _{n \rightarrow \infty} f\left(c_{n}\right)=c$.

\section{8. 四川大学 2016 年研究生入学考试试题数学分析}
\begin{CJK}{UTF8}{mj}李扬\end{CJK}

\begin{CJK}{UTF8}{mj}微信公众号\end{CJK}: sxkyliyang

\begin{CJK}{UTF8}{mj}一\end{CJK}、\begin{CJK}{UTF8}{mj}计算\end{CJK}(\begin{CJK}{UTF8}{mj}每小题\end{CJK} 10 \begin{CJK}{UTF8}{mj}分\end{CJK}, \begin{CJK}{UTF8}{mj}共\end{CJK} 70 \begin{CJK}{UTF8}{mj}分\end{CJK})

\begin{enumerate}
  \item \begin{CJK}{UTF8}{mj}求\end{CJK} $\lim _{x \rightarrow 0} \frac{\tan (\tan x)-\sin (\sin x)}{\tan x-\sin x}$.

  \item \begin{CJK}{UTF8}{mj}求\end{CJK} $\lim _{x \rightarrow 0} \frac{\ln (1+x)^{\frac{1}{x}}-1}{x}$.

  \item \begin{CJK}{UTF8}{mj}求\end{CJK} $\lim _{n \rightarrow \infty} \int_{0}^{\frac{\pi}{2}} \sin ^{n} x \mathrm{~d} x$.

  \item \begin{CJK}{UTF8}{mj}设\end{CJK} $u(x, y)$ \begin{CJK}{UTF8}{mj}具有连续二阶偏导数\end{CJK}, \begin{CJK}{UTF8}{mj}且\end{CJK} $u_{x x}(x, y)=u_{y y}(x, y)$. \begin{CJK}{UTF8}{mj}又\end{CJK} $u(x, 2 x)=x, u_{x}(x, 2 x)=x^{2}$. \begin{CJK}{UTF8}{mj}求\end{CJK} $u_{x x}(x, 2 x), u_{x y}(x, 2 x) .$

  \item \begin{CJK}{UTF8}{mj}求\end{CJK} $\oint_{L} \frac{x \mathrm{~d} y-y \mathrm{~d} x}{3 x^{2}+4 y^{2}}$, \begin{CJK}{UTF8}{mj}其中\end{CJK} $L$ \begin{CJK}{UTF8}{mj}为椭圆\end{CJK} $2 x^{2}+3 y^{2}=1$, \begin{CJK}{UTF8}{mj}方向沿逆时针方向\end{CJK}.

  \item \begin{CJK}{UTF8}{mj}求\end{CJK} $\iint_{S}|z| \mathrm{d} S$, \begin{CJK}{UTF8}{mj}其中\end{CJK} $S$ \begin{CJK}{UTF8}{mj}为柱体\end{CJK} $x^{2}+y^{2} \leq x$ \begin{CJK}{UTF8}{mj}被球体\end{CJK} $x^{2}+y^{2}+z^{2} \leq 1$ \begin{CJK}{UTF8}{mj}截取部分\end{CJK}.

  \item \begin{CJK}{UTF8}{mj}求\end{CJK} $\iiint_{\Omega}(x+z) \mathrm{d} V$, \begin{CJK}{UTF8}{mj}其中\end{CJK} $\Omega$ \begin{CJK}{UTF8}{mj}是由曲面\end{CJK} $z=\sqrt{x^{2}+y^{2}}$ \begin{CJK}{UTF8}{mj}与\end{CJK} $z=\sqrt{1-x^{2}-y^{2}}$ \begin{CJK}{UTF8}{mj}所围成的区域\end{CJK}.

\end{enumerate}
\begin{CJK}{UTF8}{mj}二\end{CJK}、 (10 \begin{CJK}{UTF8}{mj}分\end{CJK}) \begin{CJK}{UTF8}{mj}设\end{CJK} $f(x)$ \begin{CJK}{UTF8}{mj}在\end{CJK} $\mathbb{R}$ \begin{CJK}{UTF8}{mj}上一致连续\end{CJK}, \begin{CJK}{UTF8}{mj}证明\end{CJK}: \begin{CJK}{UTF8}{mj}存在非负实数\end{CJK} $a, b$ \begin{CJK}{UTF8}{mj}使得\end{CJK}
$$
|f(x)| \leq a|x|+b, \forall x \in \mathbb{R}
$$
\begin{CJK}{UTF8}{mj}三\end{CJK}、 $\left(10\right.$ \begin{CJK}{UTF8}{mj}分\end{CJK}) \begin{CJK}{UTF8}{mj}设函数\end{CJK} $f(x)$ \begin{CJK}{UTF8}{mj}在\end{CJK} $[0,1]$ \begin{CJK}{UTF8}{mj}上可导\end{CJK}, $f(0)=0$, \begin{CJK}{UTF8}{mj}且\end{CJK} $\left|f^{\prime}(x)\right| \leq|f(x)|, \forall x \in[0,1]$, \begin{CJK}{UTF8}{mj}证明\end{CJK}: $f(x)=0, \forall x \in[0,1]$.

\begin{CJK}{UTF8}{mj}四\end{CJK}、 (10 \begin{CJK}{UTF8}{mj}分\end{CJK}) \begin{CJK}{UTF8}{mj}设\end{CJK} $\left\{a_{n}\right\}$ \begin{CJK}{UTF8}{mj}为递减正数列\end{CJK}, \begin{CJK}{UTF8}{mj}且级数\end{CJK} $\sum_{n=1}^{\infty} a_{n} \sin n x$ \begin{CJK}{UTF8}{mj}在\end{CJK} $\mathbb{R}$ \begin{CJK}{UTF8}{mj}中一致收敛\end{CJK}, \begin{CJK}{UTF8}{mj}证明\end{CJK}: $\lim _{n \rightarrow \infty} n a_{n}=0$.

\begin{CJK}{UTF8}{mj}五\end{CJK}、 $(10$ \begin{CJK}{UTF8}{mj}分\end{CJK}) \begin{CJK}{UTF8}{mj}设\end{CJK} $f(x)$ \begin{CJK}{UTF8}{mj}在\end{CJK} $[0,1]$ \begin{CJK}{UTF8}{mj}上单调增加\end{CJK}, \begin{CJK}{UTF8}{mj}且\end{CJK} $f([0,1]) \subset[0,1]$, \begin{CJK}{UTF8}{mj}证明\end{CJK}: \begin{CJK}{UTF8}{mj}存在\end{CJK} $a \in[0,1]$ \begin{CJK}{UTF8}{mj}使得\end{CJK} $f(a)=a$.

\begin{CJK}{UTF8}{mj}六\end{CJK}、 $\left(10\right.$ \begin{CJK}{UTF8}{mj}分\end{CJK}) \begin{CJK}{UTF8}{mj}设\end{CJK} $(x, y) \in(0,1) \times(0,+\infty)$, \begin{CJK}{UTF8}{mj}证明\end{CJK}: $y x^{y}(1-x)<e^{-1}$.

\begin{CJK}{UTF8}{mj}七\end{CJK}、 (15 \begin{CJK}{UTF8}{mj}分\end{CJK}) \begin{CJK}{UTF8}{mj}证明\end{CJK}: \begin{CJK}{UTF8}{mj}级数\end{CJK} $\sum_{n=1}^{\infty} \frac{\sin n x}{n}$ \begin{CJK}{UTF8}{mj}在\end{CJK} $(0, \pi)$ \begin{CJK}{UTF8}{mj}不一致收敛\end{CJK}, \begin{CJK}{UTF8}{mj}但内闭一致收敛\end{CJK}.

\begin{CJK}{UTF8}{mj}八\end{CJK}、 $(15$ \begin{CJK}{UTF8}{mj}分\end{CJK}) \begin{CJK}{UTF8}{mj}设函数\end{CJK} $z=g(y)$ \begin{CJK}{UTF8}{mj}在\end{CJK} $[A, B]$ \begin{CJK}{UTF8}{mj}上连续\end{CJK},\begin{CJK}{UTF8}{mj}函数\end{CJK} $y=f(x)$ \begin{CJK}{UTF8}{mj}在\end{CJK} $[a, b]$ \begin{CJK}{UTF8}{mj}上可积\end{CJK}, \begin{CJK}{UTF8}{mj}且\end{CJK} $f([a, b]) \subset[A, B]$. \begin{CJK}{UTF8}{mj}证明\end{CJK}:

\begin{enumerate}
  \item \begin{CJK}{UTF8}{mj}复合函数\end{CJK} $g(f(x))$ \begin{CJK}{UTF8}{mj}在\end{CJK} $[a, b]$ \begin{CJK}{UTF8}{mj}上可积\end{CJK}.

  \item \begin{CJK}{UTF8}{mj}若上述条件中\end{CJK}“ $z=g(y)$ \begin{CJK}{UTF8}{mj}在\end{CJK} $[A, B]$ \begin{CJK}{UTF8}{mj}上连续\end{CJK}” \begin{CJK}{UTF8}{mj}改为\end{CJK} “ $z=g(y)$ \begin{CJK}{UTF8}{mj}在\end{CJK} $[A, B]$ \begin{CJK}{UTF8}{mj}上可积\end{CJK}”, \begin{CJK}{UTF8}{mj}其余条件不变\end{CJK}, \begin{CJK}{UTF8}{mj}复合函数\end{CJK} $g(f(x))$ \begin{CJK}{UTF8}{mj}在\end{CJK} $[a, b]$ \begin{CJK}{UTF8}{mj}上是否可积\end{CJK}? \begin{CJK}{UTF8}{mj}试证明你的结论\end{CJK}.

\end{enumerate}
\section{9. 四川大学 2017 年研究生入学考试试题数学分析}
\begin{CJK}{UTF8}{mj}李扬\end{CJK}

\begin{CJK}{UTF8}{mj}微信公众号\end{CJK}: sxkyliyang

\begin{CJK}{UTF8}{mj}一\end{CJK}、\begin{CJK}{UTF8}{mj}计算\end{CJK}(\begin{CJK}{UTF8}{mj}每小题\end{CJK} 10 \begin{CJK}{UTF8}{mj}分\end{CJK}, \begin{CJK}{UTF8}{mj}共\end{CJK} 70 \begin{CJK}{UTF8}{mj}分\end{CJK}).

\begin{enumerate}
  \item \begin{CJK}{UTF8}{mj}设\end{CJK} $a \in(0,1)$, \begin{CJK}{UTF8}{mj}求\end{CJK} $\lim _{n \rightarrow \infty}\left[(n+1)^{a}-n^{a}\right]$.

  \item \begin{CJK}{UTF8}{mj}求\end{CJK} $\lim _{x \rightarrow \infty}\left[\ln \frac{x+\sqrt{x^{2}+1}}{x+\sqrt{x^{2}-1}} \cdot \ln ^{-2}\left(\frac{x+1}{x-1}\right)\right]$.

  \item \begin{CJK}{UTF8}{mj}设\end{CJK} $f(x)=x^{8} \arctan x$, \begin{CJK}{UTF8}{mj}求\end{CJK} $f^{(n)}(0)$.

  \item \begin{CJK}{UTF8}{mj}求\end{CJK} $\int \max \{1,|x|\} \mathrm{d} x$.

\end{enumerate}
5 . \begin{CJK}{UTF8}{mj}设\end{CJK} $D$ \begin{CJK}{UTF8}{mj}是由曲线\end{CJK} $\left(\frac{x}{a}+\frac{y}{b}\right)^{3}=x y$ \begin{CJK}{UTF8}{mj}围成的区域\end{CJK}, \begin{CJK}{UTF8}{mj}其中\end{CJK} $a>0, b>0$, \begin{CJK}{UTF8}{mj}求\end{CJK} $D$ \begin{CJK}{UTF8}{mj}的面积\end{CJK}.

\begin{enumerate}
  \setcounter{enumi}{6}
  \item \begin{CJK}{UTF8}{mj}求\end{CJK} $\oint_{S} \frac{x \mathrm{~d} y-y \mathrm{~d} x}{3 x^{2}+4 y^{2}}$, \begin{CJK}{UTF8}{mj}其中\end{CJK} $S$ \begin{CJK}{UTF8}{mj}是椭圆\end{CJK} $2 x^{2}+3 y^{2}=1$, \begin{CJK}{UTF8}{mj}方向沿逆时针方向\end{CJK}.

  \item \begin{CJK}{UTF8}{mj}求\end{CJK} $\iint_{S} f(x, y, z) \mathrm{d} S$, \begin{CJK}{UTF8}{mj}其中\end{CJK} $S$ \begin{CJK}{UTF8}{mj}是球面\end{CJK} $x^{2}+y^{2}+z^{2}=1$,

\end{enumerate}
$$
f(x, y, z)= \begin{cases}\sqrt{x^{2}+y^{2}}, & \text { 当 } 0 \leq z \leq \sqrt{x^{2}+y^{2}} \text { 时; } \\ 0, & \text { 当 } z<0 \text { 或 } z>\sqrt{x^{2}+y^{2}} \text { 时. }\end{cases}
$$
\begin{CJK}{UTF8}{mj}二\end{CJK}、 $\left(12\right.$ \begin{CJK}{UTF8}{mj}分\end{CJK}) \begin{CJK}{UTF8}{mj}证明\end{CJK}: $f(x)=\frac{|\sin x|}{x}$ \begin{CJK}{UTF8}{mj}在\end{CJK} $(-1,0)$ \begin{CJK}{UTF8}{mj}和\end{CJK} $(0,1)$ \begin{CJK}{UTF8}{mj}上都一致连续\end{CJK}, \begin{CJK}{UTF8}{mj}但在\end{CJK} $(-1,0) \cup(0,1)$ \begin{CJK}{UTF8}{mj}上不一致连续\end{CJK}.

\begin{CJK}{UTF8}{mj}三\end{CJK}、 $\left(10\right.$ \begin{CJK}{UTF8}{mj}分\end{CJK}) \begin{CJK}{UTF8}{mj}设\end{CJK} $f(x)$ \begin{CJK}{UTF8}{mj}在实数\end{CJK} $\mathbb{R}$ \begin{CJK}{UTF8}{mj}上有界且二次可导\end{CJK}, \begin{CJK}{UTF8}{mj}证明\end{CJK}: \begin{CJK}{UTF8}{mj}存在\end{CJK} $x_{0} \in \mathbb{R}$ \begin{CJK}{UTF8}{mj}使得\end{CJK} $f^{\prime \prime}\left(x_{0}\right)=0$.

\begin{CJK}{UTF8}{mj}四\end{CJK}、 (10 \begin{CJK}{UTF8}{mj}分\end{CJK}) \begin{CJK}{UTF8}{mj}设\end{CJK} $f(x)$ \begin{CJK}{UTF8}{mj}在\end{CJK} $[a, b]$ \begin{CJK}{UTF8}{mj}上可积\end{CJK}, \begin{CJK}{UTF8}{mj}证明\end{CJK}:
$$
\lim _{\alpha \rightarrow+\infty} \int_{a}^{b} f(x) \sin \alpha x \mathrm{~d} x=0
$$
\begin{CJK}{UTF8}{mj}五\end{CJK}、(10 \begin{CJK}{UTF8}{mj}分\end{CJK}) \begin{CJK}{UTF8}{mj}证明\end{CJK}: $\sum_{n=0}^{\infty} x^{n}(1-x)$ \begin{CJK}{UTF8}{mj}在\end{CJK} $[0,1]$ \begin{CJK}{UTF8}{mj}上收敛但不一致收敛\end{CJK}.

\begin{CJK}{UTF8}{mj}六\end{CJK}、(12 \begin{CJK}{UTF8}{mj}分\end{CJK}) \begin{CJK}{UTF8}{mj}求\end{CJK} $a, b$ \begin{CJK}{UTF8}{mj}的值\end{CJK}, \begin{CJK}{UTF8}{mj}使得椭圆\end{CJK} $\frac{x^{2}}{a^{2}}+\frac{y^{2}}{b^{2}}=1$ \begin{CJK}{UTF8}{mj}包含圆\end{CJK} $(x-1)^{2}+y^{2}=1$, \begin{CJK}{UTF8}{mj}且面积最小\end{CJK}.

\begin{CJK}{UTF8}{mj}七\end{CJK}、 (14 \begin{CJK}{UTF8}{mj}分\end{CJK}) \begin{CJK}{UTF8}{mj}举例说明\end{CJK}: \begin{CJK}{UTF8}{mj}二元函数的\end{CJK}“\begin{CJK}{UTF8}{mj}两个累次极限存在\end{CJK}”\begin{CJK}{UTF8}{mj}与\end{CJK}“\begin{CJK}{UTF8}{mj}二重极限存在\end{CJK}”\begin{CJK}{UTF8}{mj}互不蕴涵\end{CJK}.

\begin{CJK}{UTF8}{mj}八\end{CJK}、 $(12$ \begin{CJK}{UTF8}{mj}分\end{CJK}) \begin{CJK}{UTF8}{mj}函数\end{CJK} $f$ \begin{CJK}{UTF8}{mj}在\end{CJK} $(0,1)$ \begin{CJK}{UTF8}{mj}上存在第一类间断点\end{CJK}, \begin{CJK}{UTF8}{mj}证明\end{CJK}: $f$ \begin{CJK}{UTF8}{mj}在\end{CJK} $(0,1)$ \begin{CJK}{UTF8}{mj}上没有原函数\end{CJK}.

\section{0. 四川大学 2018 年研究生入学考试试题数学分析}
\begin{CJK}{UTF8}{mj}李扬\end{CJK}

\begin{CJK}{UTF8}{mj}微信公众号\end{CJK}: sxkyliyang

\begin{CJK}{UTF8}{mj}一\end{CJK}、\begin{CJK}{UTF8}{mj}计算\end{CJK}(\begin{CJK}{UTF8}{mj}每小题\end{CJK} 10 \begin{CJK}{UTF8}{mj}分\end{CJK}, \begin{CJK}{UTF8}{mj}共\end{CJK} 40 \begin{CJK}{UTF8}{mj}分\end{CJK}).

\begin{enumerate}
  \item \begin{CJK}{UTF8}{mj}设\end{CJK} $f(x)=|\ln | x||$, \begin{CJK}{UTF8}{mj}求\end{CJK} $f^{\prime}(x)$.

  \item \begin{CJK}{UTF8}{mj}求积分\end{CJK} $\int_{0}^{\frac{\pi}{2}} \frac{\mathrm{d} x}{1+\tan x}$.

  \item \begin{CJK}{UTF8}{mj}计算\end{CJK} $\oint_{L}\left(z+y^{2}\right) \mathrm{d} s$, \begin{CJK}{UTF8}{mj}其中\end{CJK} $L$ \begin{CJK}{UTF8}{mj}为球面\end{CJK} $x^{2}+y^{2}+z^{2}=1$ \begin{CJK}{UTF8}{mj}与平面\end{CJK} $x+y+z=0$ \begin{CJK}{UTF8}{mj}的交线\end{CJK}.

  \item \begin{CJK}{UTF8}{mj}计算\end{CJK}

\end{enumerate}
$$
\iint_{S}(x+y-z) \mathrm{d} y \mathrm{~d} z+(2 y+\sin (x+z)) \mathrm{d} z \mathrm{~d} x+\left(3 z+e^{x+y}\right) \mathrm{d} x \mathrm{~d} y .
$$
\begin{CJK}{UTF8}{mj}其中\end{CJK} $S$ \begin{CJK}{UTF8}{mj}是曲面\end{CJK} $|x-y+z|+|y-z+x|+|z-x+y|=1$ \begin{CJK}{UTF8}{mj}的外表面\end{CJK}.

\begin{CJK}{UTF8}{mj}二\end{CJK}、 (10 \begin{CJK}{UTF8}{mj}分\end{CJK}) \begin{CJK}{UTF8}{mj}设\end{CJK} $x_{0}=\sqrt{7}, x_{1}=\sqrt{7-\sqrt{7}}, x_{n+2}=\sqrt{7-\sqrt{7+x_{n}}}, n=0,1, \cdots$. \begin{CJK}{UTF8}{mj}证明\end{CJK}: \begin{CJK}{UTF8}{mj}数列\end{CJK} $\left\{x_{n}\right\}$ \begin{CJK}{UTF8}{mj}收敛并求极限\end{CJK}.

\begin{CJK}{UTF8}{mj}三\end{CJK}、 $\left(10\right.$ \begin{CJK}{UTF8}{mj}分\end{CJK}) \begin{CJK}{UTF8}{mj}证明\end{CJK}: \begin{CJK}{UTF8}{mj}对任意正整数\end{CJK} $n, x+x^{2}+\cdots+x^{n}=1$ \begin{CJK}{UTF8}{mj}在\end{CJK} $[0,1]$ \begin{CJK}{UTF8}{mj}存在唯一根\end{CJK} $x_{n}$, \begin{CJK}{UTF8}{mj}并求\end{CJK} $\lim _{n \rightarrow \infty} x_{n}$.

\begin{CJK}{UTF8}{mj}四\end{CJK}、 $(10$ \begin{CJK}{UTF8}{mj}分\end{CJK}) \begin{CJK}{UTF8}{mj}函数\end{CJK} $f(x)$ \begin{CJK}{UTF8}{mj}在\end{CJK} $[a, b]$ \begin{CJK}{UTF8}{mj}上单调增加\end{CJK}, \begin{CJK}{UTF8}{mj}且\end{CJK} $f(a) \geq a, f(b) \leq b$. \begin{CJK}{UTF8}{mj}证明\end{CJK}: \begin{CJK}{UTF8}{mj}存在\end{CJK} $c \in[a, b]$, \begin{CJK}{UTF8}{mj}使得\end{CJK} $f(c)=c$.

\begin{CJK}{UTF8}{mj}五\end{CJK}、 ( 10 \begin{CJK}{UTF8}{mj}分\end{CJK}) \begin{CJK}{UTF8}{mj}函数\end{CJK} $f$ \begin{CJK}{UTF8}{mj}在\end{CJK} $[a, b]$ \begin{CJK}{UTF8}{mj}上可导\end{CJK}, \begin{CJK}{UTF8}{mj}且在所有零点处导数不等于\end{CJK} 0 , \begin{CJK}{UTF8}{mj}证明\end{CJK}: $f$ \begin{CJK}{UTF8}{mj}在\end{CJK} $[a, b]$ \begin{CJK}{UTF8}{mj}上只有有限个零点\end{CJK}.

\begin{CJK}{UTF8}{mj}六\end{CJK}、 $\left(10\right.$ \begin{CJK}{UTF8}{mj}分\end{CJK}) \begin{CJK}{UTF8}{mj}设\end{CJK} $f(x)=\sum_{n=1}^{\infty} \frac{x^{n}}{n^{2} \ln (1+n)}$, \begin{CJK}{UTF8}{mj}证明\end{CJK}: $f(x)$ \begin{CJK}{UTF8}{mj}在\end{CJK} $[-1,1]$ \begin{CJK}{UTF8}{mj}上连续\end{CJK}, \begin{CJK}{UTF8}{mj}在\end{CJK} $[-1,1)$ \begin{CJK}{UTF8}{mj}上可导\end{CJK}.

\begin{CJK}{UTF8}{mj}七\end{CJK}、(12 \begin{CJK}{UTF8}{mj}分\end{CJK}) \begin{CJK}{UTF8}{mj}设\end{CJK} $a, b \in \mathbb{R}$, \begin{CJK}{UTF8}{mj}讨论积分\end{CJK} $\int_{0}^{+\infty} x^{a} \sin x^{b} \mathrm{~d} x$ \begin{CJK}{UTF8}{mj}的敛散性\end{CJK}(\begin{CJK}{UTF8}{mj}包括条件收敛和绝对收敛\end{CJK}).

\begin{CJK}{UTF8}{mj}八\end{CJK}、 $(12$ \begin{CJK}{UTF8}{mj}分\end{CJK} $)$ \begin{CJK}{UTF8}{mj}设\end{CJK} $f_{x}, f_{y}$ \begin{CJK}{UTF8}{mj}在\end{CJK} $(0,0)$ \begin{CJK}{UTF8}{mj}点附近存在\end{CJK}, \begin{CJK}{UTF8}{mj}且在\end{CJK} $(0,0)$ \begin{CJK}{UTF8}{mj}点可微\end{CJK}, \begin{CJK}{UTF8}{mj}证明\end{CJK}: $f_{x y}(0,0)=f_{y x}(0,0)$.

\begin{CJK}{UTF8}{mj}九\end{CJK}、 (12 \begin{CJK}{UTF8}{mj}分\end{CJK}) \begin{CJK}{UTF8}{mj}证明\end{CJK}: $x^{2}+y^{2} \leq 4 e^{x+y-2}$, \begin{CJK}{UTF8}{mj}其中\end{CJK} $x \geq 0, y \geq 0$.

\begin{CJK}{UTF8}{mj}十\end{CJK}、(24 \begin{CJK}{UTF8}{mj}分\end{CJK}) Riemann \begin{CJK}{UTF8}{mj}函数\end{CJK} $R: \mathbb{R} \rightarrow \mathbb{R}$ \begin{CJK}{UTF8}{mj}定义为\end{CJK}
$$
R(x)= \begin{cases}\frac{1}{p}, & x=\frac{q}{p}\left(p \in \mathrm{N}^{+}, q \in \mathrm{Z}, p, q \text { 互质 }\right), \\ 0, & x \text { 为无理数. }\end{cases}
$$
\begin{CJK}{UTF8}{mj}证明\end{CJK}:

\begin{enumerate}
  \item $R(x)$ \begin{CJK}{UTF8}{mj}在任意一点\end{CJK} $x_{0} \in \mathbb{R}$ \begin{CJK}{UTF8}{mj}处有极限\end{CJK}, \begin{CJK}{UTF8}{mj}在所有无理数点连续\end{CJK}, \begin{CJK}{UTF8}{mj}所有有理数点为可去间断点\end{CJK}.

  \item $R(x)$ \begin{CJK}{UTF8}{mj}在\end{CJK} $\mathbb{R}$ \begin{CJK}{UTF8}{mj}上处处不可导\end{CJK}.

  \item $R(x)$ \begin{CJK}{UTF8}{mj}在\end{CJK} $[0,1]$ \begin{CJK}{UTF8}{mj}上可积\end{CJK}.

\end{enumerate}
\section{1. 天津大学 2009 年研究生入学考试试题高等代数}
\begin{CJK}{UTF8}{mj}李扬\end{CJK}

\begin{CJK}{UTF8}{mj}微信公众号\end{CJK}: sxkyliyang

\begin{CJK}{UTF8}{mj}一\end{CJK}. \begin{CJK}{UTF8}{mj}填空题\end{CJK}.

\begin{enumerate}
  \item \begin{CJK}{UTF8}{mj}设\end{CJK} $f(x)=\left|\begin{array}{lll}a_{1}+x & b_{1}+x & c_{1}+x \\ a_{2}+x & b_{2}+x & c_{2}+x \\ a_{3}+x & b_{3}+x & c_{3}+x\end{array}\right|$, \begin{CJK}{UTF8}{mj}则\end{CJK} $f(x)$ \begin{CJK}{UTF8}{mj}最多是\end{CJK} $\quad$ \begin{CJK}{UTF8}{mj}次多项式\end{CJK}.

  \item \begin{CJK}{UTF8}{mj}设\end{CJK} $\alpha_{1}=(6,-1,1)^{\prime}, \alpha_{2}=(-7,4,2)^{\prime}$ \begin{CJK}{UTF8}{mj}是线性方程组\end{CJK} $\left\{\begin{array}{l}a_{1} x_{1}+a_{2} x_{2}+a_{3} x_{3}=a_{4} ; \\ x_{1}+3 x_{2}-2 x_{3}=1 ; \\ 2 x_{1}+5 x_{2}+x_{3}=8 .\end{array}\right.$ \begin{CJK}{UTF8}{mj}的两个解\end{CJK}, \begin{CJK}{UTF8}{mj}那么此方程组的\end{CJK} \begin{CJK}{UTF8}{mj}通解为\end{CJK}

  \item \begin{CJK}{UTF8}{mj}设\end{CJK} $A=\left(\begin{array}{llll}0 & 0 & 1 & 1 \\ 0 & 0 & 2 & 3 \\ 2 & 5 & 0 & 0 \\ 3 & 7 & 0 & 0\end{array}\right)$, \begin{CJK}{UTF8}{mj}则\end{CJK} $A^{-1}=$

  \item \begin{CJK}{UTF8}{mj}设\end{CJK} $\mathbb{R}^{3}$ \begin{CJK}{UTF8}{mj}两个基为\end{CJK} $(\mathrm{I}): \alpha_{1}=(1,1,1)^{\prime}, \alpha_{2}=(1,0,-1)^{\prime}, \alpha_{3}=(1,0,1)^{\prime} ;(\mathrm{II}): \beta_{1}=(1,2,1)^{\prime}, \beta_{2}=(2,3,4)^{\prime}, \beta_{3}=$ $(3,4,3)^{\prime}$, \begin{CJK}{UTF8}{mj}则基\end{CJK} (I) \begin{CJK}{UTF8}{mj}到基\end{CJK} (II) \begin{CJK}{UTF8}{mj}的过渡矩阵为\end{CJK} \begin{CJK}{UTF8}{mj}向量\end{CJK} $\beta=\beta_{1}+2 \beta_{2}+3 \beta_{3}$ \begin{CJK}{UTF8}{mj}在基\end{CJK} (I) \begin{CJK}{UTF8}{mj}下的坐标为\end{CJK}

\end{enumerate}
5 . \begin{CJK}{UTF8}{mj}设\end{CJK} $x_{1}, x_{2}, x_{3}$ \begin{CJK}{UTF8}{mj}为矩阵\end{CJK} $A$ \begin{CJK}{UTF8}{mj}的分别属于特征值\end{CJK} $\lambda_{1}=1, \lambda_{2}=2, \lambda_{3}=3$ \begin{CJK}{UTF8}{mj}的特征向量\end{CJK}, \begin{CJK}{UTF8}{mj}而\end{CJK} $S=\left(3 x_{2}, x_{3},-2 x_{1}\right)$, \begin{CJK}{UTF8}{mj}则\end{CJK} $S^{-1} A^{-1} S=$

\begin{enumerate}
  \setcounter{enumi}{6}
  \item \begin{CJK}{UTF8}{mj}设\end{CJK} $A=\left(\begin{array}{ccc}-1 & 0 & 0 \\ 0 & 1 & 3 \\ 0 & 0 & 1\end{array}\right)$, \begin{CJK}{UTF8}{mj}则\end{CJK} $A^{100}$

  \item \begin{CJK}{UTF8}{mj}设\end{CJK} 3 \begin{CJK}{UTF8}{mj}阶方阵\end{CJK} $A$ \begin{CJK}{UTF8}{mj}的特征值\end{CJK} $-1,1,2, A^{*}$ \begin{CJK}{UTF8}{mj}为\end{CJK} $A$ \begin{CJK}{UTF8}{mj}的伴随矩阵\end{CJK}, \begin{CJK}{UTF8}{mj}则\end{CJK} $B=3 A^{2}+A^{-1}+A^{*}-2 A$ \begin{CJK}{UTF8}{mj}的特征值为\end{CJK}

  \item \begin{CJK}{UTF8}{mj}设\end{CJK} $A=\left(\begin{array}{ccc}3 & a & 2 \\ a & 3 & -2 \\ 2 & -2 & 8\end{array}\right)$ \begin{CJK}{UTF8}{mj}是正定矩阵\end{CJK}, $a$ \begin{CJK}{UTF8}{mj}是正整数\end{CJK}, \begin{CJK}{UTF8}{mj}则\end{CJK} $a=$

  \item \begin{CJK}{UTF8}{mj}设\end{CJK} $\alpha_{1}, \alpha_{2}, \alpha_{3}$ \begin{CJK}{UTF8}{mj}是欧式空间\end{CJK} $V$ \begin{CJK}{UTF8}{mj}的一个基\end{CJK}, \begin{CJK}{UTF8}{mj}其度量矩阵为\end{CJK} $A=\left(\begin{array}{ccc}2 & 2 & -2 \\ 2 & 5 & -4 \\ -2 & -4 & 5\end{array}\right), \beta=\alpha_{1}+\alpha_{2}+\alpha_{3}$, \begin{CJK}{UTF8}{mj}则\end{CJK}

  \item \begin{CJK}{UTF8}{mj}设\end{CJK} $n$ \begin{CJK}{UTF8}{mj}元实二次型\end{CJK} $f$ \begin{CJK}{UTF8}{mj}的秩为\end{CJK} $n$, \begin{CJK}{UTF8}{mj}且\end{CJK} $f$ \begin{CJK}{UTF8}{mj}与\end{CJK} $-f$ \begin{CJK}{UTF8}{mj}有相同的规范型\end{CJK}, \begin{CJK}{UTF8}{mj}则\end{CJK} $f$ \begin{CJK}{UTF8}{mj}的正惯性指数为\end{CJK}

\end{enumerate}
$$
(A, B)=\operatorname{tr}\left(A B^{T}\right), \forall A, B \in \mathbb{R}^{n \times n}
$$
\begin{CJK}{UTF8}{mj}四\end{CJK}. (1) \begin{CJK}{UTF8}{mj}试证存在\end{CJK} 2 \begin{CJK}{UTF8}{mj}阶矩阵\end{CJK} $A, B$ \begin{CJK}{UTF8}{mj}使得\end{CJK}
$$
A B-B A=\left(\begin{array}{cc}
1 & 0 \\
0 & -1
\end{array}\right)
$$
(2) \begin{CJK}{UTF8}{mj}试证不存在矩阵空间\end{CJK} $\mathbb{R}^{2 \times 2}$ \begin{CJK}{UTF8}{mj}上的线性变换\end{CJK} $\mathscr{A}$ \begin{CJK}{UTF8}{mj}满足\end{CJK}
$$
\forall A, B \in \mathbb{R}^{2 \times 2}, \mathscr{A}(A B)=\mathscr{A}(A) \mathscr{B}(B) \text {, 且 } \mathscr{A}\left(\begin{array}{cc}
1 & 0 \\
0 & -1
\end{array}\right)=\left(\begin{array}{ll}
1 & 0 \\
0 & 1
\end{array}\right) \text {. }
$$
\begin{CJK}{UTF8}{mj}五\end{CJK}. \begin{CJK}{UTF8}{mj}设\end{CJK} $\mathrm{r}\left(A_{m \times n}\right)=r \neq 0$, \begin{CJK}{UTF8}{mj}求证\end{CJK}:\begin{CJK}{UTF8}{mj}必有矩阵\end{CJK} $B_{m \times r}, C_{r \times n}$, \begin{CJK}{UTF8}{mj}使得\end{CJK}
$$
A=B C, \mathrm{r}(B)=\mathrm{r}(C)
$$
\begin{CJK}{UTF8}{mj}六\end{CJK}. \begin{CJK}{UTF8}{mj}设矩阵\end{CJK} $A_{s \times n}$ \begin{CJK}{UTF8}{mj}的秩为\end{CJK} $r$, \begin{CJK}{UTF8}{mj}线性方程组\end{CJK} $A x=b(\mathrm{I})$ (\begin{CJK}{UTF8}{mj}其中\end{CJK} $x, b$ \begin{CJK}{UTF8}{mj}为列向量\end{CJK}, $\left.b \neq 0\right)$ \begin{CJK}{UTF8}{mj}有特解\end{CJK} $\xi_{0}$, \begin{CJK}{UTF8}{mj}导出方程组\end{CJK} $A x=0$ \begin{CJK}{UTF8}{mj}的\end{CJK} \begin{CJK}{UTF8}{mj}一个基础解系为\end{CJK} $\xi_{1}, \xi_{2}, \cdots, \xi_{n-r}$, \begin{CJK}{UTF8}{mj}试证明\end{CJK}:

(1) \begin{CJK}{UTF8}{mj}向量\end{CJK} $\eta_{0}=\xi_{0}, \eta_{1}=\xi_{0}+\xi_{1}, \cdots, \eta_{n-r}=\xi_{0}+\xi_{n-r}$ \begin{CJK}{UTF8}{mj}是方程组\end{CJK} (I) \begin{CJK}{UTF8}{mj}的线性无关解向量\end{CJK}.

(2) $\eta_{0}, \eta_{1}, \cdots, \eta_{n-r}$ \begin{CJK}{UTF8}{mj}的一切线性组合\end{CJK} $k_{0} \eta_{0}+k_{1} \eta_{1}+\cdots+k_{n-r} \eta_{n-r}$ (\begin{CJK}{UTF8}{mj}其中\end{CJK} $k_{0}+k_{1}+\cdots+k_{n-r}=1$ ) \begin{CJK}{UTF8}{mj}是方程\end{CJK} \begin{CJK}{UTF8}{mj}组\end{CJK} (I) \begin{CJK}{UTF8}{mj}的全部解\end{CJK}.

\begin{CJK}{UTF8}{mj}七\end{CJK}. (1) \begin{CJK}{UTF8}{mj}设\end{CJK} $A, B$ \begin{CJK}{UTF8}{mj}都是\end{CJK} $n$ \begin{CJK}{UTF8}{mj}阶正定矩阵\end{CJK}, \begin{CJK}{UTF8}{mj}求证\end{CJK}: $A+B$ \begin{CJK}{UTF8}{mj}是正定矩阵\end{CJK}.

(2) \begin{CJK}{UTF8}{mj}设\end{CJK} $A, B$ \begin{CJK}{UTF8}{mj}都是\end{CJK} $n$ \begin{CJK}{UTF8}{mj}阶实对称阵\end{CJK}, \begin{CJK}{UTF8}{mj}且\end{CJK} $A$ \begin{CJK}{UTF8}{mj}的特征值大于\end{CJK} $a, B$ \begin{CJK}{UTF8}{mj}的特征值小于\end{CJK} $b$, \begin{CJK}{UTF8}{mj}求证\end{CJK}: $A-B$ \begin{CJK}{UTF8}{mj}的特征值大于\end{CJK} $a-b$.

\begin{CJK}{UTF8}{mj}八\end{CJK}. \begin{CJK}{UTF8}{mj}设\end{CJK} $M \in P^{n \times n}$, \begin{CJK}{UTF8}{mj}其中\end{CJK} $P$ \begin{CJK}{UTF8}{mj}为数域\end{CJK}, $f(x) \in P[x], g(x) \in P[x]$, \begin{CJK}{UTF8}{mj}且\end{CJK} $(f(x), g(x))=1, A=f(M), B=g(M), W, W_{1}, W_{2}$ \begin{CJK}{UTF8}{mj}分别为方程组\end{CJK} $A B x=0, A x=0, B x=0$ \begin{CJK}{UTF8}{mj}的解空间\end{CJK}, \begin{CJK}{UTF8}{mj}试证\end{CJK}:

\includegraphics[max width=\textwidth]{2022_04_18_a5c47c0ff534501b502eg-227}

\section{2. 天津大学 2010 年研究生入学考试试题高等代数}
\begin{CJK}{UTF8}{mj}李扬\end{CJK}

\begin{CJK}{UTF8}{mj}微信公众号\end{CJK}: sxkyliyang

-. \begin{CJK}{UTF8}{mj}填空题\end{CJK}.

\begin{enumerate}
  \item \begin{CJK}{UTF8}{mj}设\end{CJK} $x^{2}+2 x+1 \mid x^{4}+a x^{2}+b$, \begin{CJK}{UTF8}{mj}则\end{CJK} $a=$ $b=$

  \item \begin{CJK}{UTF8}{mj}设\end{CJK} $A, B$ \begin{CJK}{UTF8}{mj}均为\end{CJK} $n$ \begin{CJK}{UTF8}{mj}阶方阵\end{CJK} $(n \geqslant 2), A^{*}, B^{*}$ \begin{CJK}{UTF8}{mj}分别为\end{CJK} $A, B$ \begin{CJK}{UTF8}{mj}的伴随矩阵\end{CJK}, \begin{CJK}{UTF8}{mj}且\end{CJK} $|A|=2,|B|=-3$, \begin{CJK}{UTF8}{mj}则\end{CJK} $\mid A^{-1} B^{*}-$ $A^{*} B^{-1} \mid=$

  \item \begin{CJK}{UTF8}{mj}设\end{CJK} $\alpha$ \begin{CJK}{UTF8}{mj}是\end{CJK} 3 \begin{CJK}{UTF8}{mj}元列向量\end{CJK}, $\alpha^{\prime}$ \begin{CJK}{UTF8}{mj}是\end{CJK} $\alpha$ \begin{CJK}{UTF8}{mj}的转置\end{CJK}, \begin{CJK}{UTF8}{mj}若\end{CJK} $\alpha \alpha^{\prime}=\left(\begin{array}{ccc}1 & -1 & 1 \\ -1 & 1 & -1 \\ 1 & -1 & 1\end{array}\right)$, \begin{CJK}{UTF8}{mj}则\end{CJK} $\alpha^{\prime} \alpha=$

  \item (\begin{CJK}{UTF8}{mj}有问题\end{CJK})\begin{CJK}{UTF8}{mj}设\end{CJK} $4 \times 4$ \begin{CJK}{UTF8}{mj}齐次方程组\end{CJK} $A x=0$ \begin{CJK}{UTF8}{mj}的解空间的维数为\end{CJK} 1 , \begin{CJK}{UTF8}{mj}则\end{CJK} $A x=0$ \begin{CJK}{UTF8}{mj}的解空间的维数为\end{CJK}

  \item \begin{CJK}{UTF8}{mj}设\end{CJK} $n$ \begin{CJK}{UTF8}{mj}元向量空间\end{CJK} $P^{n}$ \begin{CJK}{UTF8}{mj}的子空间\end{CJK}

\end{enumerate}
$$
\begin{aligned}
&W_{1}=\left\{\left(x_{1}, x_{2}, \cdots, x_{n-1}, 0\right)^{\prime} \mid x_{i} \in P, i=1,2, \cdots, n-1\right\} \\
&W_{2}=\left\{\left(x_{1}, x_{2}, \cdots, x_{n-1}, x_{n}\right)^{\prime} \in P^{n} \mid x_{1}+x_{2}+\cdots+x_{n}=0\right\}
\end{aligned}
$$
\begin{CJK}{UTF8}{mj}则子空间\end{CJK} $W_{1} \cap W_{2}$ \begin{CJK}{UTF8}{mj}的维数为\end{CJK}

\begin{enumerate}
  \setcounter{enumi}{6}
  \item \begin{CJK}{UTF8}{mj}设实对称矩阵\end{CJK} $A$ \begin{CJK}{UTF8}{mj}与\end{CJK} $B=\left(\begin{array}{ccc}1 & 0 & 0 \\ 0 & -1 & 2 \\ 0 & 2 & 2\end{array}\right)$ \begin{CJK}{UTF8}{mj}合同\end{CJK}, \begin{CJK}{UTF8}{mj}则一次型\end{CJK} $x^{\prime} A x$ \begin{CJK}{UTF8}{mj}的规范型为\end{CJK}

  \item \begin{CJK}{UTF8}{mj}设矩阵\end{CJK} $A=\left(\begin{array}{ccc}-2 & a & 3 \\ 2 & -3 & 2 \\ -1 & 2 & 2\end{array}\right)$ \begin{CJK}{UTF8}{mj}与\end{CJK} $B=\left(\begin{array}{lll}1 & 0 & 0 \\ 0 & 1 & 0 \\ 0 & 0 & b\end{array}\right)$ \begin{CJK}{UTF8}{mj}相似\end{CJK}, \begin{CJK}{UTF8}{mj}则\end{CJK} $a=$

  \item \begin{CJK}{UTF8}{mj}设\end{CJK} $(\mathrm{I})=\left\{\alpha_{1}, \alpha_{2}, \alpha_{3}\right\},(\mathrm{II})=\left\{\beta_{1}, \beta_{2}, \beta_{3}\right\}$ \begin{CJK}{UTF8}{mj}是\end{CJK} 3 \begin{CJK}{UTF8}{mj}维线性空间\end{CJK} $V$ \begin{CJK}{UTF8}{mj}的两个基\end{CJK}, \begin{CJK}{UTF8}{mj}它们在\end{CJK} $V$ \begin{CJK}{UTF8}{mj}的对偶空间\end{CJK} $V^{*}$ \begin{CJK}{UTF8}{mj}中的对偶基分\end{CJK}

\end{enumerate}
(1) \begin{CJK}{UTF8}{mj}求证\end{CJK} $A+2 E_{n}$ \begin{CJK}{UTF8}{mj}是可逆矩阵\end{CJK}. 2. \begin{CJK}{UTF8}{mj}在\end{CJK} $\mathbb{R}^{4}$ \begin{CJK}{UTF8}{mj}中\end{CJK}, \begin{CJK}{UTF8}{mj}设\end{CJK} $W$ \begin{CJK}{UTF8}{mj}是由向量组\end{CJK}
$$
\alpha_{1}=\left(\begin{array}{l}
3 \\
2 \\
1 \\
1
\end{array}\right), \alpha_{2}=\left(\begin{array}{c}
4 \\
5 \\
-1 \\
6
\end{array}\right), \alpha_{3}=\left(\begin{array}{c}
? \\
3 \\
-2 \\
5
\end{array}\right), \alpha_{4}=\left(\begin{array}{c}
6 \\
1 \\
5 \\
-3
\end{array}\right), \alpha_{5}=\left(\begin{array}{c}
-2 \\
4 \\
-6 \\
9
\end{array}\right)
$$
\begin{CJK}{UTF8}{mj}生成的子空间\end{CJK}, \begin{CJK}{UTF8}{mj}求证\end{CJK} $\left\{\alpha_{1}, \alpha_{2}, \alpha_{4}\right\}$ \begin{CJK}{UTF8}{mj}是\end{CJK} $W$ \begin{CJK}{UTF8}{mj}的一个基\end{CJK}, \begin{CJK}{UTF8}{mj}并求\end{CJK} $\alpha_{3}, \alpha_{5}$ \begin{CJK}{UTF8}{mj}分别在该基下的坐标\end{CJK}

\begin{enumerate}
  \setcounter{enumi}{3}
  \item \begin{CJK}{UTF8}{mj}设\end{CJK} $V$ \begin{CJK}{UTF8}{mj}是数域\end{CJK} $P$ \begin{CJK}{UTF8}{mj}上的线性空间\end{CJK}.\\
(1) \begin{CJK}{UTF8}{mj}设\end{CJK} $W=L\left(\alpha_{1}, \alpha_{2}, \cdots, \alpha_{s}\right)$, \begin{CJK}{UTF8}{mj}求证\end{CJK} $\operatorname{dim} W=\mathrm{r}\left\{\alpha_{1}, \alpha_{2}, \cdots, \alpha_{s}\right\}$.\\
(2) \begin{CJK}{UTF8}{mj}若向量组\end{CJK} $\left\{\alpha_{1}, \alpha_{2}, \cdots, \alpha_{s}\right\}$ \begin{CJK}{UTF8}{mj}线性无关\end{CJK}, $W_{1}=L\left(\alpha_{1}+\alpha_{2}, \alpha_{2}+\alpha_{3}, \cdots, \alpha_{s-1}+\alpha_{s}, \alpha_{s}+\alpha_{1}\right)$, \begin{CJK}{UTF8}{mj}求\end{CJK} $\operatorname{dim} W_{1}$.

  \item \begin{CJK}{UTF8}{mj}设\end{CJK} $\alpha_{1}, \alpha_{2}$ \begin{CJK}{UTF8}{mj}是\end{CJK} 2 \begin{CJK}{UTF8}{mj}阶实对称矩阵\end{CJK} $A$ \begin{CJK}{UTF8}{mj}的分别属于特征值\end{CJK} $\lambda_{1}, \lambda_{2}$ \begin{CJK}{UTF8}{mj}的单位特征向量\end{CJK}, $\lambda_{1} \neq \lambda_{2}$. \begin{CJK}{UTF8}{mj}定义\end{CJK} $\mathbb{R}^{2 \times 2}$ \begin{CJK}{UTF8}{mj}的线性变换\end{CJK} A \begin{CJK}{UTF8}{mj}如下\end{CJK}:

\end{enumerate}
$$
\mathscr{A}(X)=A X+X A, \forall X \in \mathbb{R}^{2 \times 2} .
$$
(1) \begin{CJK}{UTF8}{mj}求证\end{CJK} $\left\{\alpha_{1} \alpha_{1}^{\prime}, \alpha_{1} \alpha_{2}^{\prime}, \alpha_{2} \alpha_{1}^{\prime}, \alpha_{2} \alpha_{2}^{\prime}\right\}$ \begin{CJK}{UTF8}{mj}是\end{CJK} $\mathbb{R}^{2 \times 2}$ \begin{CJK}{UTF8}{mj}的一个基\end{CJK}.

(2) \begin{CJK}{UTF8}{mj}求\end{CJK} $\mathscr{A}$ \begin{CJK}{UTF8}{mj}在基\end{CJK} $\left\{\alpha_{1} \alpha_{1}^{\prime}, \alpha_{1} \alpha_{2}^{\prime}, \alpha_{2} \alpha_{1}^{\prime}, \alpha_{2} \alpha_{2}^{\prime}\right\}$ \begin{CJK}{UTF8}{mj}下的矩阵\end{CJK}.

(3) \begin{CJK}{UTF8}{mj}求证\end{CJK} $\operatorname{tr}(\mathscr{A})=4 \operatorname{tr}(A)$, \begin{CJK}{UTF8}{mj}其中\end{CJK} $\operatorname{tr}(\mathscr{A}), \operatorname{tr}(A)$ \begin{CJK}{UTF8}{mj}分别为\end{CJK} $\mathscr{A}, A$ \begin{CJK}{UTF8}{mj}的迹\end{CJK}.

\begin{enumerate}
  \setcounter{enumi}{5}
  \item \begin{CJK}{UTF8}{mj}设\end{CJK}
\end{enumerate}
$$
A=\left(\begin{array}{ccc}
6 & 6 & -15 \\
1 & 5 & -5 \\
1 & 2 & -2
\end{array}\right), B=\left(\begin{array}{ccc}
37 & -20 & -4 \\
34 & -17 & -4 \\
119 & -70 & -11
\end{array}\right)
$$
(1) \begin{CJK}{UTF8}{mj}求\end{CJK} $A$ \begin{CJK}{UTF8}{mj}的不变因子组\end{CJK}.\\
(2) \begin{CJK}{UTF8}{mj}求\end{CJK} $B$ \begin{CJK}{UTF8}{mj}的若当标准型\end{CJK}.\\
(3) \begin{CJK}{UTF8}{mj}判断\end{CJK} $A, B$ \begin{CJK}{UTF8}{mj}是否相似\end{CJK}.

\begin{CJK}{UTF8}{mj}三\end{CJK}. \begin{CJK}{UTF8}{mj}证明题\end{CJK}.

\begin{enumerate}
  \item \begin{CJK}{UTF8}{mj}设\end{CJK} $A=\left(a_{i j}\right)_{m \times n}$ \begin{CJK}{UTF8}{mj}的元素全是整数\end{CJK}, \begin{CJK}{UTF8}{mj}求证\end{CJK} $\frac{1}{2}$ \begin{CJK}{UTF8}{mj}不是\end{CJK} $A$ \begin{CJK}{UTF8}{mj}的特征值\end{CJK}.
\end{enumerate}
\section{3. 天津大学 2011 年研究生入学考试试题高等代数}
\begin{CJK}{UTF8}{mj}李扬\end{CJK}

\begin{CJK}{UTF8}{mj}微信公众号\end{CJK}: sxkyliyang

\begin{CJK}{UTF8}{mj}一\end{CJK}. \begin{CJK}{UTF8}{mj}填空题\end{CJK}.

\begin{enumerate}
  \item $\left|\begin{array}{ccc}246 & 427 & 327 \\ 814 & 643 & 543 \\ -312 & 721 & 621\end{array}\right|=\underline{\square}$.

  \item \begin{CJK}{UTF8}{mj}已知方阵\end{CJK} $A=\operatorname{diag}\{1,2,3\}$, \begin{CJK}{UTF8}{mj}有\end{CJK} $f(A)=\operatorname{diag}\{0,3,8\}$, \begin{CJK}{UTF8}{mj}则\end{CJK} $f(x)=$

  \item \begin{CJK}{UTF8}{mj}已知\end{CJK} $A=\left(\begin{array}{lll}1 & 0 & 1 \\ 0 & 1 & 0 \\ 1 & 0 & 1\end{array}\right), A^{*}$ \begin{CJK}{UTF8}{mj}是\end{CJK} $A$ \begin{CJK}{UTF8}{mj}的伴随矩阵\end{CJK}, \begin{CJK}{UTF8}{mj}则\end{CJK} $A^{*} x=0$ \begin{CJK}{UTF8}{mj}的通解是\end{CJK}

  \item \begin{CJK}{UTF8}{mj}已知\end{CJK} $A=\left(a_{i j}\right)$ \begin{CJK}{UTF8}{mj}是正交矩阵\end{CJK}, $a_{11}=-1$, \begin{CJK}{UTF8}{mj}且\end{CJK} $A x=(1,0,0, \cdots, 0)^{\prime}$, \begin{CJK}{UTF8}{mj}则\end{CJK} $x=$

  \item \begin{CJK}{UTF8}{mj}已知\end{CJK} $\alpha$ \begin{CJK}{UTF8}{mj}是三维列向量\end{CJK}, \begin{CJK}{UTF8}{mj}满足\end{CJK} $\alpha^{\prime} \alpha=9$, \begin{CJK}{UTF8}{mj}则三阶矩阵\end{CJK} $\alpha \alpha^{\prime}$ \begin{CJK}{UTF8}{mj}的所有特征值和是\end{CJK}

  \item $A=\left(\begin{array}{ccc}2 & -2 & 0 \\ -2 & 1 & -1 \\ 0 & -1 & -4\end{array}\right)$ \begin{CJK}{UTF8}{mj}的合同标准型是\end{CJK}

  \item \begin{CJK}{UTF8}{mj}已知\end{CJK} $A, B$ \begin{CJK}{UTF8}{mj}都是正定矩阵\end{CJK}, \begin{CJK}{UTF8}{mj}则下列矩阵不一定是正定矩阵的是\end{CJK}\\
A. $A B$\\
B. $A B A$\\
C. $A^{2}-A+B+B^{2}$\\
D. $\left(\begin{array}{cc}A & 2 A \\ 2 A & 5 A+3 B\end{array}\right)$

  \item \begin{CJK}{UTF8}{mj}已知二次型\end{CJK} $f\left(x_{1}, x_{2}, x_{3}\right)=t x_{1}^{2}+t x_{2}^{2}+(2-t) x_{3}^{2}+2 x_{1} x_{2}$ \begin{CJK}{UTF8}{mj}是正定的\end{CJK}, \begin{CJK}{UTF8}{mj}则\end{CJK} $t$ \begin{CJK}{UTF8}{mj}的取值范围是\end{CJK}

  \item \begin{CJK}{UTF8}{mj}实矩阵\end{CJK} $A=\left(\begin{array}{ccc}-1 & 1 & 1 \\ 0 & 0 & 1 \\ 0 & 0 & 1\end{array}\right)$, \begin{CJK}{UTF8}{mj}则\end{CJK} $A^{2}$ \begin{CJK}{UTF8}{mj}比\end{CJK} $A$ \begin{CJK}{UTF8}{mj}增加的特征向量是\end{CJK}

\end{enumerate}
$$
\cos (n x)=\left(\begin{array}{cccccc}
\cos x & 1 & \cdots & \cdots & \cdots & 0 \\
1 & 2 \cos x & 1 & 0 & \cdots & 0 \\
0 & 1 & 2 \cos x & 1 & \cdots & 0 \\
\vdots & \vdots & \vdots & & & \vdots \\
0 & 0 & 0 & \cdots & 1 & 2 \cos x
\end{array}\right)
$$
(1) $2 \cos (n \pi)=f(2 \cos x)$, \begin{CJK}{UTF8}{mj}其中\end{CJK} $f(x)$ \begin{CJK}{UTF8}{mj}首项系数为\end{CJK} 1 \begin{CJK}{UTF8}{mj}的整系数一元多项式\end{CJK}.

(2) \begin{CJK}{UTF8}{mj}正整数\end{CJK} $m, n$ \begin{CJK}{UTF8}{mj}互素\end{CJK}, \begin{CJK}{UTF8}{mj}且\end{CJK} $m<n, \cos \frac{m}{n} \pi \in Q$, \begin{CJK}{UTF8}{mj}则\end{CJK} $\frac{m}{n} \in\left\{\frac{1}{3}, \frac{1}{2}, \frac{2}{3}\right\}$. \begin{CJK}{UTF8}{mj}间\end{CJK}, \begin{CJK}{UTF8}{mj}证明\end{CJK}: (2) $W_{g} \cap W_{f}=W_{m}$.

(3) $W_{g}+W_{f}=W_{d}$.

(4) $W_{g}+W_{f}=P[x]_{n} \Leftrightarrow f(x)$ \begin{CJK}{UTF8}{mj}与\end{CJK} $g(x)$ \begin{CJK}{UTF8}{mj}互素\end{CJK}.

\begin{enumerate}
  \setcounter{enumi}{3}
  \item \begin{CJK}{UTF8}{mj}已知\end{CJK} 4 \begin{CJK}{UTF8}{mj}维线性空间\end{CJK} $V$ \begin{CJK}{UTF8}{mj}上的变换\end{CJK} $\mathscr{A}$ \begin{CJK}{UTF8}{mj}满足\end{CJK} $A^{2}=-E$, \begin{CJK}{UTF8}{mj}证明\end{CJK} $\mathscr{A}$ \begin{CJK}{UTF8}{mj}在线性空间\end{CJK} $V$ \begin{CJK}{UTF8}{mj}上总有某个基使得\end{CJK} $\mathscr{A}$ \begin{CJK}{UTF8}{mj}在该基下的\end{CJK} \begin{CJK}{UTF8}{mj}矩阵是\end{CJK}
\end{enumerate}
$$
\left(\begin{array}{cccc}
0 & -1 & 0 & 0 \\
1 & 0 & 0 & 0 \\
0 & 0 & 0 & -1 \\
0 & 0 & 1 & 0
\end{array}\right)
$$
\begin{CJK}{UTF8}{mj}三\end{CJK}. 1. \begin{CJK}{UTF8}{mj}已知\end{CJK}
$$
\left\{\begin{array}{l}
a_{n}=3 a_{n-1}+b_{n-1}+2^{n-1} \\
b_{n}=2 a_{n-1}+b_{n-1}+2^{n}
\end{array}\right.
$$
\begin{CJK}{UTF8}{mj}且\end{CJK} $a_{0}=-1, b_{0}=3$, \begin{CJK}{UTF8}{mj}记\end{CJK} $\xi=\left(\begin{array}{l}a_{n} \\ b_{n} \\ 2^{n}\end{array}\right)$

(1) \begin{CJK}{UTF8}{mj}求\end{CJK} $\xi_{n}$ \begin{CJK}{UTF8}{mj}的递推公式\end{CJK}.

(2) \begin{CJK}{UTF8}{mj}计算\end{CJK} $\xi_{n}$ \begin{CJK}{UTF8}{mj}的通项公式\end{CJK}.

\begin{enumerate}
  \setcounter{enumi}{2}
  \item \begin{CJK}{UTF8}{mj}已知\end{CJK}
\end{enumerate}
$$
J_{t}\left(\lambda_{0}\right)=\left(\begin{array}{ccccc}
\lambda_{0} & 0 & 0 & \cdots & 0 \\
1 & \lambda_{0} & 0 & \cdots & 0 \\
\vdots & \ddots & \ddots & \ddots & \vdots \\
0 & \cdots & 1 & \lambda_{0} & 0 \\
0 & \cdots & 0 & 1 & \lambda_{0}
\end{array}\right),
$$
\begin{CJK}{UTF8}{mj}计算\end{CJK} $\left[J_{t}\left(\lambda_{0}\right)\right]^{2}$ \begin{CJK}{UTF8}{mj}若当标准型\end{CJK}.

\begin{enumerate}
  \setcounter{enumi}{3}
  \item \begin{CJK}{UTF8}{mj}已知\end{CJK} $\mathscr{A}$ \begin{CJK}{UTF8}{mj}是欧式空间\end{CJK} $V=W \oplus W^{\perp}$ \begin{CJK}{UTF8}{mj}上的线性变换\end{CJK}, $W$ \begin{CJK}{UTF8}{mj}与\end{CJK} $W^{\perp}$ \begin{CJK}{UTF8}{mj}为\end{CJK} $\mathscr{A}$-\begin{CJK}{UTF8}{mj}子空间\end{CJK}, $\mathscr{A} \mid W$ \begin{CJK}{UTF8}{mj}为对称变换\end{CJK}, $\mathscr{A} \mid W^{\perp}$ \begin{CJK}{UTF8}{mj}为反\end{CJK} \begin{CJK}{UTF8}{mj}对称变换\end{CJK}, \begin{CJK}{UTF8}{mj}证明\end{CJK} $\mathscr{A}$ \begin{CJK}{UTF8}{mj}的像空间\end{CJK} $\mathscr{A} V$ \begin{CJK}{UTF8}{mj}是\end{CJK} $\mathscr{A}^{-1}(0)$ \begin{CJK}{UTF8}{mj}的正交补\end{CJK}.
\end{enumerate}
\section{4. 天津大学 2012 年研究生入学考试试题高等代数}
\begin{CJK}{UTF8}{mj}李扬\end{CJK}

\begin{CJK}{UTF8}{mj}微信公众号\end{CJK}: sxkyliyang

-. \begin{CJK}{UTF8}{mj}填空题\end{CJK}.

\begin{enumerate}
  \item \begin{CJK}{UTF8}{mj}关于数域说法正确的是\end{CJK}
\end{enumerate}
A. \begin{CJK}{UTF8}{mj}数域有无穷多个\end{CJK}.

B. \begin{CJK}{UTF8}{mj}每个数域都有无穷个集合\end{CJK}.

C. \begin{CJK}{UTF8}{mj}含\end{CJK} $i$ \begin{CJK}{UTF8}{mj}的数域都是复数域\end{CJK}.

\begin{enumerate}
  \setcounter{enumi}{2}
  \item $x_{n}$ \begin{CJK}{UTF8}{mj}除\end{CJK} $(x-2)(x+3)$ \begin{CJK}{UTF8}{mj}的余式为\end{CJK} ( $x_{n}$ \begin{CJK}{UTF8}{mj}记不清了\end{CJK}, \begin{CJK}{UTF8}{mj}题型正确\end{CJK} $)$

  \item $A=\left(\begin{array}{ll}1 & 2 \\ 1 & 2\end{array}\right), B=\left(\begin{array}{ll}3 & 4 \\ 3 & 4\end{array}\right), A C=B C$, \begin{CJK}{UTF8}{mj}则\end{CJK} $C=$

  \item \begin{CJK}{UTF8}{mj}求\end{CJK} $\left(\begin{array}{ll}0 & A \\ B & C\end{array}\right)^{-1}=$

  \item \begin{CJK}{UTF8}{mj}已知\end{CJK} $A, B,\left(\begin{array}{l}A \\ B\end{array}\right)$ \begin{CJK}{UTF8}{mj}的秩分别为\end{CJK} $r_{1}, r_{2}, r_{3} . W_{1}=\{\xi \mid A \xi=0\}, W_{2}=\{\eta \mid B \eta=0\}, W_{3}=\{(\xi, \eta) \mid A \xi=0, B \eta=0\}$, \begin{CJK}{UTF8}{mj}则\end{CJK} $\operatorname{dim} W_{3}=$

  \item $A=\left(a_{i j}\right)_{3 \times 3},|A+2 E|=0, a_{11}+a_{12}+a_{13}=a_{21}+a_{22}+a_{23}=a_{31}+a_{32}+a_{33}=1, a_{11}+a_{22}+a_{33}=2$, \begin{CJK}{UTF8}{mj}求\end{CJK} $A$ \begin{CJK}{UTF8}{mj}的所有特征根\end{CJK}

  \item \begin{CJK}{UTF8}{mj}下列属于实系数多项式空间\end{CJK} $P[x]$ \begin{CJK}{UTF8}{mj}内积的是\end{CJK}\\
A. $(f, g)=2012$\\
B. $(f, g)=\int_{0}^{1} f(x) g(x) \mathrm{d} x$.\\
C. $(f, g)=\int_{0}^{1} f(x) g(x) \mathrm{d} x-\int_{-1}^{0} f(x) g(x) \mathrm{d} x$\\
D. $(f, g)=2 \int_{0}^{1} f(x) g(x) \mathrm{d} x+\int_{-1}^{0} f(x) g(x) \mathrm{d} x$.

  \item (\begin{CJK}{UTF8}{mj}有问题\end{CJK}) $A$ \begin{CJK}{UTF8}{mj}与\end{CJK} $B$ \begin{CJK}{UTF8}{mj}相似且为正交阵\end{CJK}, $A, B$ \begin{CJK}{UTF8}{mj}在正交基下的矩阵为\end{CJK} $\frac{1}{7} \ldots\left(\begin{array}{cccc}1 & 0 & 0 \\ 0 & \cos \theta & \sin \theta \\ 0 & \sin \theta & \cos \theta\end{array}\right)$, \begin{CJK}{UTF8}{mj}和\end{CJK} $\theta \in\left[\begin{array}{l}0,2] \text {, }\end{array}\right.$

\end{enumerate}
$$
\left|\begin{array}{cc}
A & t^{2} B \\
B & A
\end{array}\right|=|A-t B||A+t B|
$$
\begin{CJK}{UTF8}{mj}已知\end{CJK} $|A-2 B|=x,|A+2 B|=y$. \begin{CJK}{UTF8}{mj}求\end{CJK} $\left|\begin{array}{cc}A & 4 B \\ B & A\end{array}\right|$.
$$
\left(\begin{array}{l}
a_{1} \\
b_{1}
\end{array}\right) \oplus\left(\begin{array}{l}
a_{2} \\
b_{2}
\end{array}\right)=\left(\begin{array}{l}
a_{1}+a_{2} \\
b_{1}+b_{2}
\end{array}\right), k\left(\begin{array}{l}
a_{1} \\
b_{1}
\end{array}\right)=\left(\begin{array}{c}
a_{1}^{k} \\
k b_{1}
\end{array}\right)
$$
(1)\begin{CJK}{UTF8}{mj}加法和数乘构成线性空间\end{CJK}.

(2) \begin{CJK}{UTF8}{mj}求\end{CJK} $V$ \begin{CJK}{UTF8}{mj}的一组基\end{CJK}. 3. \begin{CJK}{UTF8}{mj}已知\end{CJK} $\lambda_{0}$ \begin{CJK}{UTF8}{mj}为\end{CJK} $n$ \begin{CJK}{UTF8}{mj}级矩阵\end{CJK} $A$ \begin{CJK}{UTF8}{mj}的\end{CJK} $k$ \begin{CJK}{UTF8}{mj}重根\end{CJK}, \begin{CJK}{UTF8}{mj}证明\end{CJK} $\left(A-\lambda_{0} E\right)^{k}$ \begin{CJK}{UTF8}{mj}的秩为\end{CJK} $n-k$.

\begin{enumerate}
  \setcounter{enumi}{4}
  \item \begin{CJK}{UTF8}{mj}已知\end{CJK} $A \in \mathbb{C}^{n \times n}$ \begin{CJK}{UTF8}{mj}可对角化\end{CJK}, \begin{CJK}{UTF8}{mj}对\end{CJK} $\forall X \in \mathbb{C}^{n \times n}$, \begin{CJK}{UTF8}{mj}线性变换\end{CJK} $\mathscr{A}$, \begin{CJK}{UTF8}{mj}使得\end{CJK}(\begin{CJK}{UTF8}{mj}后面还有\end{CJK}, \begin{CJK}{UTF8}{mj}但是看不清楚\end{CJK})
\end{enumerate}
\begin{CJK}{UTF8}{mj}三\end{CJK}. \begin{CJK}{UTF8}{mj}已知\end{CJK} $A$ \begin{CJK}{UTF8}{mj}为对称矩阵\end{CJK},
$$
A=\frac{1}{2}\left(\begin{array}{ll}
31 & 33 \\
33 & 31
\end{array}\right)
$$
\begin{CJK}{UTF8}{mj}求一实对称矩阵\end{CJK} $B$, \begin{CJK}{UTF8}{mj}使得\end{CJK} $A=B^{5}$.

\begin{CJK}{UTF8}{mj}四\end{CJK}. \begin{CJK}{UTF8}{mj}讨论是否存在一个元素都为有理数的\end{CJK} $n$ \begin{CJK}{UTF8}{mj}级矩阵\end{CJK} $A$ \begin{CJK}{UTF8}{mj}使得\end{CJK} $A^{2}=2 E$, \begin{CJK}{UTF8}{mj}并说明理由\end{CJK}.

\section{5. 天津大学 2013 年研究生入学考试试题高等代数}
\begin{CJK}{UTF8}{mj}李扬\end{CJK}

\begin{CJK}{UTF8}{mj}微信公众号\end{CJK}: sxkyliyang

\begin{CJK}{UTF8}{mj}一\end{CJK}. \begin{CJK}{UTF8}{mj}填空题\end{CJK}.

\begin{enumerate}
  \item $\left|\begin{array}{lll}a_{11} & a_{12} & a_{13} \\ a_{21} & a_{22} & a_{23} \\ a_{31} & a_{32} & a_{33}\end{array}\right|$ \begin{CJK}{UTF8}{mj}的第二行全体元素的余子式之和等于行列式\end{CJK}

  \item \begin{CJK}{UTF8}{mj}若\end{CJK} $\left|A_{n \times n}\right|=-\frac{1}{2},\left|B_{n \times n}\right|=-\frac{1}{3}$, \begin{CJK}{UTF8}{mj}则\end{CJK} $\left|\begin{array}{cc}A-B & 2 A \\ 2 A+B & 4 A\end{array}\right|=$

  \item \begin{CJK}{UTF8}{mj}设\end{CJK} $A^{3}=0$, \begin{CJK}{UTF8}{mj}则\end{CJK} $\left(A-E_{n}\right)^{-2}=$

  \item \begin{CJK}{UTF8}{mj}设\end{CJK} $A B=0$, \begin{CJK}{UTF8}{mj}则\end{CJK}

\end{enumerate}
A. $A=0$ \begin{CJK}{UTF8}{mj}或\end{CJK} $B=0$.

B. \begin{CJK}{UTF8}{mj}当\end{CJK} $A \neq 0$ \begin{CJK}{UTF8}{mj}时\end{CJK}, $B$ \begin{CJK}{UTF8}{mj}的列向量线性相关\end{CJK}.

C. \begin{CJK}{UTF8}{mj}当\end{CJK} $B \neq 0$ \begin{CJK}{UTF8}{mj}时\end{CJK}, $A$ \begin{CJK}{UTF8}{mj}的列向量线性相关\end{CJK}.

D. $B$ \begin{CJK}{UTF8}{mj}的列向量的极大无关组是齐次线性方程组\end{CJK} $A X=0$ \begin{CJK}{UTF8}{mj}的一个基础解系\end{CJK}.

\begin{enumerate}
  \setcounter{enumi}{5}
  \item \begin{CJK}{UTF8}{mj}若\end{CJK} 3 \begin{CJK}{UTF8}{mj}维列向量\end{CJK} $(\mathrm{I})=\left\{\alpha_{1}, \alpha_{2}\right\},(\mathrm{II})=\left\{\beta_{1}, \beta_{2}\right\}$ \begin{CJK}{UTF8}{mj}都线性无关\end{CJK}, \begin{CJK}{UTF8}{mj}下列正确的是\end{CJK}
\end{enumerate}
A. $\left\{\alpha_{1}, \alpha_{2}, \beta_{1}, \beta_{2}\right\}$ \begin{CJK}{UTF8}{mj}线性相关\end{CJK}.

B. \begin{CJK}{UTF8}{mj}存在非零向量既与\end{CJK} (I) \begin{CJK}{UTF8}{mj}线性相关\end{CJK}, \begin{CJK}{UTF8}{mj}又与\end{CJK} (II) \begin{CJK}{UTF8}{mj}线性相关\end{CJK}.

C. \begin{CJK}{UTF8}{mj}任何\end{CJK} 3 \begin{CJK}{UTF8}{mj}维列向量可由\end{CJK} $\alpha_{1}, \alpha_{2}, \beta_{1}, \beta_{2}$ \begin{CJK}{UTF8}{mj}线性表示\end{CJK}.

D. \begin{CJK}{UTF8}{mj}只有零向量既与\end{CJK} (I) \begin{CJK}{UTF8}{mj}线性相关\end{CJK}, \begin{CJK}{UTF8}{mj}又与\end{CJK} (II) \begin{CJK}{UTF8}{mj}线性相关\end{CJK}.

\begin{enumerate}
  \setcounter{enumi}{6}
  \item $A=\left(\begin{array}{ccc}0 & 5 & b \\ 1 & -2 & c \\ 2 & a & -1\end{array}\right)$ \begin{CJK}{UTF8}{mj}为实矩阵\end{CJK}, $D=\left(\begin{array}{ccc}d_{1} & 0 & 0 \\ 0 & d_{2} & 0 \\ 0 & 0 & d_{3}\end{array}\right)$ \begin{CJK}{UTF8}{mj}的主对角元全为正\end{CJK}, $A D$ \begin{CJK}{UTF8}{mj}为正交矩阵\end{CJK}, \begin{CJK}{UTF8}{mj}则\end{CJK} $d_{1}, d_{2}, d_{3}$ \begin{CJK}{UTF8}{mj}分别\end{CJK}
\end{enumerate}
$$
f(x)=x^{6}+2 x^{5}-x^{4}-4 x^{3}-4 x^{2}-4 x-2 .
$$
\begin{CJK}{UTF8}{mj}求\end{CJK} $f(x)$ \begin{CJK}{UTF8}{mj}的所有有理根\end{CJK}, \begin{CJK}{UTF8}{mj}并求\end{CJK} $f(x) \in \mathbb{Q}[x]$ \begin{CJK}{UTF8}{mj}的标准解析式\end{CJK}.
$$
\left\{\alpha_{1}=\left(\begin{array}{l}
1 \\
4 \\
2 \\
0
\end{array}\right), \alpha_{2}=\left(\begin{array}{l}
1 \\
2 \\
0 \\
2
\end{array}\right), \alpha_{3}=\left(\begin{array}{c}
0 \\
1 \\
a \\
-1
\end{array}\right)\right\}
$$
$与$
$$
\left\{\beta_{1}=\left(\begin{array}{l}
2 \\
7 \\
3 \\
1
\end{array}\right), \beta_{2}=\left(\begin{array}{l}
1 \\
3 \\
1 \\
1
\end{array}\right), \beta_{3}=\left(\begin{array}{c}
3 \\
10 \\
4 \\
b
\end{array}\right)\right\}
$$
\begin{CJK}{UTF8}{mj}线性等价\end{CJK}, \begin{CJK}{UTF8}{mj}求\end{CJK} $a, b$ \begin{CJK}{UTF8}{mj}的值\end{CJK}.

\begin{enumerate}
  \setcounter{enumi}{3}
  \item \begin{CJK}{UTF8}{mj}设\end{CJK} $A_{3 \times 3}=\left(\alpha_{1}, \alpha_{2}, \alpha_{3}\right)$, \begin{CJK}{UTF8}{mj}非齐次线性方程组\end{CJK} $A X=\beta$ \begin{CJK}{UTF8}{mj}的通解为\end{CJK} $\left(\begin{array}{c}1 \\ 2 \\ -1\end{array}\right)+k\left(\begin{array}{c}1 \\ -2 \\ 3\end{array}\right)$. \begin{CJK}{UTF8}{mj}令\end{CJK} $B_{3 \times 4}=\left(\alpha_{1}, \alpha_{2}, \alpha_{3}, \alpha_{3}+\right.$ $\beta)$,
\end{enumerate}
(1) \begin{CJK}{UTF8}{mj}求\end{CJK} $A$ \begin{CJK}{UTF8}{mj}的秩\end{CJK}, \begin{CJK}{UTF8}{mj}并用\end{CJK} $\left\{\alpha_{1}, \alpha_{2}, \alpha_{3}\right\}$ \begin{CJK}{UTF8}{mj}表示\end{CJK} $\beta$.

(2) \begin{CJK}{UTF8}{mj}求\end{CJK} $B$ \begin{CJK}{UTF8}{mj}的秩及线性方程组\end{CJK} $B Y=\alpha_{1}-\alpha_{2}$ \begin{CJK}{UTF8}{mj}的通解\end{CJK}.

\begin{CJK}{UTF8}{mj}三\end{CJK}. \begin{CJK}{UTF8}{mj}解答题\end{CJK}.

\begin{enumerate}
  \item \begin{CJK}{UTF8}{mj}设\end{CJK}
\end{enumerate}
$$
J=\left(\begin{array}{lll}
0 & 1 & 0 \\
0 & 0 & 1 \\
0 & 0 & 0
\end{array}\right), K=\left(\begin{array}{lll}
0 & 0 & 1 \\
0 & 1 & 0 \\
1 & 0 & 0
\end{array}\right)
$$
\begin{CJK}{UTF8}{mj}而\end{CJK} $W=\left\{A+B \mid A, B \in \mathbb{R}^{3 \times 3}, A J=J A, B K=K B\right\}$.

(1) \begin{CJK}{UTF8}{mj}求证\end{CJK} $W$ \begin{CJK}{UTF8}{mj}是线性空间\end{CJK} $\mathbb{R}^{3 \times 3}$ \begin{CJK}{UTF8}{mj}的一个子空间\end{CJK}.

(2) \begin{CJK}{UTF8}{mj}求\end{CJK} $\operatorname{dim}(W)$.

\begin{enumerate}
  \setcounter{enumi}{2}
  \item \begin{CJK}{UTF8}{mj}设数列\end{CJK} $\left\{a_{n}\right\},\left\{b_{n}\right\}$ \begin{CJK}{UTF8}{mj}满足\end{CJK}
\end{enumerate}
$$
\left\{\begin{array}{l}
a_{n}=3 a_{n-1}+b_{n-1}+2^{n-1} ; \\
b_{n}=2 a_{n-1}+4 b_{n-1}+2^{n} .
\end{array} \quad(n=1,2, \cdots)\right.
$$
\begin{CJK}{UTF8}{mj}且\end{CJK} $a_{0}=-1, b_{0}=3$. \begin{CJK}{UTF8}{mj}记\end{CJK} $\xi_{n}=\left(\begin{array}{l}a_{n} \\ b_{n} \\ 2^{n}\end{array}\right)$

(2) \begin{CJK}{UTF8}{mj}求两数列的通项公式\end{CJK}.

\begin{enumerate}
  \setcounter{enumi}{3}
  \item \begin{CJK}{UTF8}{mj}设\end{CJK}
\end{enumerate}
$$
A=\left(\begin{array}{ccc}
3 & 1 & -1 \\
2 & 2 & -1 \\
2 & 2 & 0
\end{array}\right)
$$
(1) \begin{CJK}{UTF8}{mj}求\end{CJK} $A$ \begin{CJK}{UTF8}{mj}的\end{CJK} Jordan \begin{CJK}{UTF8}{mj}标准形\end{CJK}.

(2) \begin{CJK}{UTF8}{mj}求次数尽可能低的多项式\end{CJK} $\varphi(x)$, \begin{CJK}{UTF8}{mj}使\end{CJK} $A^{-1}=\varphi(A)$.
$$
\varphi\left(\alpha_{1}+\alpha_{2}, \beta_{1}+\beta_{2}\right)=20 \varphi_{1}\left(\alpha_{1}, \beta_{1}\right)+13 \varphi_{2}\left(\alpha_{2}, \beta_{2}\right) \sqrt{b^{2}-4 a c}, \forall \alpha_{1}, \beta_{1} \in W_{1}, \alpha_{2}, \beta_{2} \in W_{2}
$$
(1) \begin{CJK}{UTF8}{mj}求证\end{CJK}: $\varphi$ \begin{CJK}{UTF8}{mj}是\end{CJK} $V$ \begin{CJK}{UTF8}{mj}的一个内积\end{CJK}.

\begin{CJK}{UTF8}{mj}四\end{CJK}. \begin{CJK}{UTF8}{mj}证明题\end{CJK}. 1. \begin{CJK}{UTF8}{mj}设\end{CJK} $V$ \begin{CJK}{UTF8}{mj}为数域\end{CJK} $\mathbb{F}$ \begin{CJK}{UTF8}{mj}上的线性空间\end{CJK}, $V$ \begin{CJK}{UTF8}{mj}的线性变换\end{CJK} $\mathscr{A}$ \begin{CJK}{UTF8}{mj}的特征多项式为\end{CJK} $P^{2}(\lambda)$, \begin{CJK}{UTF8}{mj}其中\end{CJK} $P(\lambda) \in \mathbb{F}(\lambda)$ \begin{CJK}{UTF8}{mj}的\end{CJK} $k$ \begin{CJK}{UTF8}{mj}次不可约多项\end{CJK} \begin{CJK}{UTF8}{mj}式\end{CJK}, $W$ \begin{CJK}{UTF8}{mj}为\end{CJK} $\mathscr{A}$ \begin{CJK}{UTF8}{mj}的一个非平凡不变子空间\end{CJK}, \begin{CJK}{UTF8}{mj}求证\end{CJK}: $\operatorname{dim}(W)$ \begin{CJK}{UTF8}{mj}必为\end{CJK} $k$ \begin{CJK}{UTF8}{mj}或\end{CJK} $2 k$.

\begin{enumerate}
  \setcounter{enumi}{2}
  \item \begin{CJK}{UTF8}{mj}设\end{CJK} $A$ \begin{CJK}{UTF8}{mj}为\end{CJK} $n$ \begin{CJK}{UTF8}{mj}阶正定矩阵\end{CJK}, $B$ \begin{CJK}{UTF8}{mj}为\end{CJK} $n$ \begin{CJK}{UTF8}{mj}阶方阵\end{CJK}, \begin{CJK}{UTF8}{mj}求证\end{CJK}: \begin{CJK}{UTF8}{mj}矩阵方程\end{CJK} $A X+X A=B$ \begin{CJK}{UTF8}{mj}有唯一解\end{CJK}.
\end{enumerate}
\section{6. 天津大学 2014 年研究生入学考试试题高等代数}
\begin{CJK}{UTF8}{mj}李扬\end{CJK}

\begin{CJK}{UTF8}{mj}微信公众号\end{CJK}: sxkyliyang

\begin{CJK}{UTF8}{mj}一\end{CJK}. \begin{CJK}{UTF8}{mj}填空题\end{CJK}.

\begin{enumerate}
  \item \begin{CJK}{UTF8}{mj}设\end{CJK} $A$ \begin{CJK}{UTF8}{mj}是奇数阶方阵\end{CJK}, $A^{*}$ \begin{CJK}{UTF8}{mj}为其伴随矩阵\end{CJK}, \begin{CJK}{UTF8}{mj}且\end{CJK} $|A|=\frac{1}{3}$, \begin{CJK}{UTF8}{mj}则\end{CJK} $\left|\left(\frac{A}{2}\right)^{-1}-15 A^{*}\right|=$
\end{enumerate}
$2 .\left(\begin{array}{ll}0 & 1 \\ 1 & 0\end{array}\right)^{2015}\left(\begin{array}{lll}1 & 2 & 3 \\ 4 & 5 & 6\end{array}\right)\left(\begin{array}{ccc}1 & 0 & 1 \\ 0 & 1 & -1 \\ 0 & 0 & 1\end{array}\right)^{2014}=$

\begin{enumerate}
  \setcounter{enumi}{3}
  \item \begin{CJK}{UTF8}{mj}设实矩阵\end{CJK} $M=\left(\begin{array}{l}A \\ B\end{array}\right), A B^{\prime}=0$, \begin{CJK}{UTF8}{mj}则下列不一定对的是\end{CJK}\\
A. $\left|M M^{\prime}\right|=\left|A A^{\prime}\right|\left|B B^{\prime}\right|$.\\
B. $\mathrm{r}\left(M M^{\prime}\right)=\mathrm{r}\left(A A^{\prime}\right)+\mathrm{r}\left(B B^{\prime}\right)$.\\
C. $\mathrm{r}(M)=\mathrm{r}(A)+\mathrm{r}(B)$.\\
D. $M$ \begin{CJK}{UTF8}{mj}可逆\end{CJK} $\Leftrightarrow A, B$ \begin{CJK}{UTF8}{mj}列满秩\end{CJK}.

  \item \begin{CJK}{UTF8}{mj}设\end{CJK} $m \times n$ \begin{CJK}{UTF8}{mj}非齐次线性方程组\end{CJK} $A x=\beta$ \begin{CJK}{UTF8}{mj}有唯一解\end{CJK} $x_{0}, \tilde{A}$ \begin{CJK}{UTF8}{mj}为其增广矩阵\end{CJK}, \begin{CJK}{UTF8}{mj}则下列说法不一定正确的是\end{CJK}\\
A. $A$ \begin{CJK}{UTF8}{mj}列向量组线性无关\end{CJK}.\\
B. \begin{CJK}{UTF8}{mj}行向量组线性无关\end{CJK}.\\
C. \begin{CJK}{UTF8}{mj}矩阵方程\end{CJK} $A x=\tilde{A}$ \begin{CJK}{UTF8}{mj}有唯一解\end{CJK} $\left(E_{n}, x_{0}\right)$.\\
D. \begin{CJK}{UTF8}{mj}必有\end{CJK} $n \times m$ \begin{CJK}{UTF8}{mj}阵\end{CJK} $U$, \begin{CJK}{UTF8}{mj}使\end{CJK} $U A=E_{n}$.

  \item \begin{CJK}{UTF8}{mj}四阶方阵\end{CJK} $A$ \begin{CJK}{UTF8}{mj}满足\end{CJK} $|2 A-E|=|3 A-E|=|4 A-E|=|5 A-E|=0$, \begin{CJK}{UTF8}{mj}则\end{CJK} $\left|A^{-1}-E\right|=$

  \item \begin{CJK}{UTF8}{mj}设\end{CJK} $\mathrm{r}\left(\begin{array}{lll}a_{1} & \cdots & a_{n} \\ b_{1} & \cdots & b_{n}\end{array}\right)=2, a_{1} x_{1}+\cdots+a_{n} x_{n}=0$ \begin{CJK}{UTF8}{mj}的解空间\end{CJK} $W_{1}, b_{1} x_{1}+\cdots+b_{n} x_{n}=0$ \begin{CJK}{UTF8}{mj}的解空间\end{CJK} $W_{2}$, \begin{CJK}{UTF8}{mj}则\end{CJK} $\operatorname{dim}\left(W_{1}+W_{2}\right)=$

\end{enumerate}
$$
v(x) g(x)+u(x) f(x)=1
$$
(2) $\frac{1}{3+2 \sqrt[3]{2}+\sqrt[3]{4}}$ \begin{CJK}{UTF8}{mj}分母有理化\end{CJK}.
$$
A=\left(\begin{array}{cccc}
0 & 0 & 0 & 1 \\
0 & 1 & 1 & 0 \\
0 & 2 & 3 & 0 \\
-1 & 0 & 0 & 0
\end{array}\right), B=\left(\begin{array}{cccc}
1 & 1 & 1 & 1 \\
1 & 2 & 4 & 8 \\
1 & 3 & 9 & 27 \\
1 & 4 & 16 & 64
\end{array}\right)
$$
(2) \begin{CJK}{UTF8}{mj}求行列式\end{CJK} $\left|\left(2 A+A^{-1}\right) B\right|$. 3. \begin{CJK}{UTF8}{mj}已知\end{CJK}
$$
A=\left(\begin{array}{llll}
1 & 2 & 1 & 2 \\
1 & 3 & 2 & 1 \\
1 & 1 & 3 & 1
\end{array}\right)
$$
\begin{CJK}{UTF8}{mj}求满足\end{CJK} $A X=E_{3}$ \begin{CJK}{UTF8}{mj}的所有\end{CJK} $X$.

\begin{enumerate}
  \setcounter{enumi}{4}
  \item \begin{CJK}{UTF8}{mj}求\end{CJK}
\end{enumerate}
$$
A=\left(\begin{array}{cccc}
2 & 0 & 1 & 0 \\
0 & 1 & 0 & 0 \\
0 & 0 & 1 & 0 \\
-1 & 0 & 0 & 1
\end{array}\right)
$$
\begin{CJK}{UTF8}{mj}的若尔当标准形与最小多项式\end{CJK}.

\begin{CJK}{UTF8}{mj}三\end{CJK}. 1. \begin{CJK}{UTF8}{mj}线性空间\end{CJK} $V=C(-\infty,+\infty)$ \begin{CJK}{UTF8}{mj}上定义\end{CJK} $\mathscr{A}$,
$$
\mathscr{A}[f(x)]=f(x+1), \forall f(x) \in V
$$
(1) \begin{CJK}{UTF8}{mj}求证\end{CJK} $\mathscr{A}$ \begin{CJK}{UTF8}{mj}是\end{CJK} $V$ \begin{CJK}{UTF8}{mj}的一个线性变换\end{CJK}.

(2)\begin{CJK}{UTF8}{mj}设\end{CJK} $W=L\left(\mathrm{e}^{x}, x \mathrm{e}^{x}, \mathrm{e}^{2 x}\right)$, \begin{CJK}{UTF8}{mj}求证\end{CJK} $W$ \begin{CJK}{UTF8}{mj}是\end{CJK} $\mathscr{A}$ \begin{CJK}{UTF8}{mj}的不变子空间\end{CJK}.

(3) \begin{CJK}{UTF8}{mj}设\end{CJK} $I$ \begin{CJK}{UTF8}{mj}为\end{CJK} $V$ \begin{CJK}{UTF8}{mj}的单位变换\end{CJK}, \begin{CJK}{UTF8}{mj}求\end{CJK} $\left.(\mathscr{A}-\mathrm{e} I)^{2}\right|_{W}$ \begin{CJK}{UTF8}{mj}的核与值域\end{CJK}.

\begin{enumerate}
  \setcounter{enumi}{2}
  \item \begin{CJK}{UTF8}{mj}设\end{CJK} $(\mathrm{I})=\left\{\alpha_{1}, \alpha_{2}, \alpha_{3}\right\}$ \begin{CJK}{UTF8}{mj}为欧式空间\end{CJK} $V$ \begin{CJK}{UTF8}{mj}的标准正交基\end{CJK}, $\mathscr{A}$ \begin{CJK}{UTF8}{mj}为\end{CJK} $V$ \begin{CJK}{UTF8}{mj}的线性变换\end{CJK}, \begin{CJK}{UTF8}{mj}且\end{CJK} $\mathscr{A} \alpha_{1}=\alpha_{2}+\alpha_{3}, \mathscr{A} \alpha_{2}=$ $\alpha_{1}+\alpha_{3}, \mathscr{A}\left(\alpha_{1}+\alpha_{2}+\alpha_{3}\right)=2\left(\alpha_{1}+\alpha_{2}+\alpha_{3}\right) .$
\end{enumerate}
(1) \begin{CJK}{UTF8}{mj}求\end{CJK} $\mathscr{A}$ \begin{CJK}{UTF8}{mj}在基\end{CJK} (I) \begin{CJK}{UTF8}{mj}下的矩阵\end{CJK} $A$, \begin{CJK}{UTF8}{mj}说明\end{CJK} $\mathscr{A}$ \begin{CJK}{UTF8}{mj}为\end{CJK} $V$ \begin{CJK}{UTF8}{mj}的对称变换\end{CJK}.

(2) \begin{CJK}{UTF8}{mj}求\end{CJK} $V$ \begin{CJK}{UTF8}{mj}的标准正交基\end{CJK} (II), \begin{CJK}{UTF8}{mj}使\end{CJK} $\mathscr{A}$ \begin{CJK}{UTF8}{mj}在基\end{CJK} (II) \begin{CJK}{UTF8}{mj}下矩阵为对角阵\end{CJK}.

\begin{CJK}{UTF8}{mj}四\end{CJK}. 1. $n$ \begin{CJK}{UTF8}{mj}元齐次线性方程组\end{CJK} $A X=0$ \begin{CJK}{UTF8}{mj}有基础解系\end{CJK} $(\mathrm{I})=\left\{x_{1}, x_{2}, \cdots, x_{s}\right\} ;\left(A-E_{n}\right) Y=0$ \begin{CJK}{UTF8}{mj}有基础解系\end{CJK} (II) $=$ $\left\{y_{1}, y_{2}, \cdots, y_{t}\right\}$; \begin{CJK}{UTF8}{mj}求证\end{CJK}: \begin{CJK}{UTF8}{mj}方程组\end{CJK} $\left(A^{2}-A\right) Z=0$ \begin{CJK}{UTF8}{mj}有基础解系\end{CJK}
$$
(\mathrm{III})=\left\{x_{1}, x_{2}, \cdots, x_{s}, y_{1}, y_{2}, \cdots, y_{t}\right\} .
$$
$$
f(\lambda)=\lambda^{s}\left(\lambda^{2}+a_{1}^{2}\right)^{k_{1}} \cdots\left(\lambda^{2}+a_{t}^{2}\right)^{k_{t}}
$$

\section{7. 天津大学 2015 年研究生入学考试试题高等代数}
\begin{CJK}{UTF8}{mj}李扬\end{CJK}

\begin{CJK}{UTF8}{mj}微信公众号\end{CJK}: sxkyliyang

\begin{enumerate}
  \item $A$ \begin{CJK}{UTF8}{mj}为一方阵\end{CJK}, \begin{CJK}{UTF8}{mj}证明存在可逆矩阵\end{CJK} $B$ \begin{CJK}{UTF8}{mj}及幂等阵\end{CJK} $C\left(C^{2}=C\right)$, \begin{CJK}{UTF8}{mj}使\end{CJK} $A=B C$.

  \item $V$ \begin{CJK}{UTF8}{mj}为欧式空间\end{CJK}, $\mathscr{A}$ \begin{CJK}{UTF8}{mj}为正交变换\end{CJK}, $I$ \begin{CJK}{UTF8}{mj}为单位变换\end{CJK}, \begin{CJK}{UTF8}{mj}证明\end{CJK}

\end{enumerate}
$$
(\mathscr{A}-I) V=(\mathscr{A}-I)^{2} V .
$$

\begin{enumerate}
  \setcounter{enumi}{3}
  \item \begin{CJK}{UTF8}{mj}设\end{CJK} $A \in \mathbb{C}^{5 \times 5}, \mathrm{r}(A)=3, A^{3}=0, W=\left\{B \in \mathbb{C}^{5 \times 5} \mid A B=B A\right\}$.
\end{enumerate}
(1) \begin{CJK}{UTF8}{mj}求\end{CJK} $A$ \begin{CJK}{UTF8}{mj}的若尔当标准形\end{CJK} $J$.

(2) $W^{\prime}=\left\{X \in \mathbb{C}^{5 \times 5} \mid J X=X J\right\}$, \begin{CJK}{UTF8}{mj}证明\end{CJK} $W$ \begin{CJK}{UTF8}{mj}与\end{CJK} $W^{\prime}$ \begin{CJK}{UTF8}{mj}同构\end{CJK}.

(3) \begin{CJK}{UTF8}{mj}求\end{CJK} $W$ \begin{CJK}{UTF8}{mj}维数\end{CJK}.

\begin{enumerate}
  \setcounter{enumi}{4}
  \item \begin{CJK}{UTF8}{mj}已知\end{CJK} $A-E_{n}$ \begin{CJK}{UTF8}{mj}为正定矩阵\end{CJK}, \begin{CJK}{UTF8}{mj}证明\end{CJK} $E-A^{-1}$ \begin{CJK}{UTF8}{mj}也为正定矩阵\end{CJK}.

  \item $A$ \begin{CJK}{UTF8}{mj}为\end{CJK} 3 \begin{CJK}{UTF8}{mj}阶对称矩阵\end{CJK}, \begin{CJK}{UTF8}{mj}特征值\end{CJK} $\lambda_{1}=1, \lambda_{2}=2, \lambda_{3}=2$, \begin{CJK}{UTF8}{mj}其中\end{CJK} $\xi=(1,1,1)^{\prime}$ \begin{CJK}{UTF8}{mj}为\end{CJK} $\lambda_{1}=1$ \begin{CJK}{UTF8}{mj}的特征向量\end{CJK}.

\end{enumerate}
(1) $A$ \begin{CJK}{UTF8}{mj}是否唯一\end{CJK}, \begin{CJK}{UTF8}{mj}说明理由\end{CJK}.

(2) \begin{CJK}{UTF8}{mj}计算\end{CJK} $B=A^{2}-4 E_{3}$.

\begin{enumerate}
  \setcounter{enumi}{6}
  \item \begin{CJK}{UTF8}{mj}有理数域上\end{CJK} $p, q$ \begin{CJK}{UTF8}{mj}为何值\end{CJK}, \begin{CJK}{UTF8}{mj}非齐次线性方程组\end{CJK}
\end{enumerate}
\begin{CJK}{UTF8}{mj}有解并求通解\end{CJK}
$$
\left\{\begin{array}{l}
x_{1}+x_{2}-2 x_{3}+3 x_{4}=0 \\
2 x_{1}+x_{2}-6 x_{3}+4 x_{4}=-1 \\
3 x_{1}+2 x_{2}+p x_{3}+7 x_{4}=-1 \\
x_{1}-x_{2}-6 x_{3}-x_{4}=q
\end{array}\right.
$$

\begin{enumerate}
  \setcounter{enumi}{7}
  \item \begin{CJK}{UTF8}{mj}求\end{CJK}
\end{enumerate}
$$
f(x)=x^{5}-5 x^{4}+8 x^{3}-5 x^{2}+4 x-4
$$
\begin{CJK}{UTF8}{mj}的所有有理根\end{CJK}, \begin{CJK}{UTF8}{mj}并写出标准分解式\end{CJK}.
$$
M=\left(\begin{array}{ll}
0 & A \\
B & C
\end{array}\right)
$$
(1) \begin{CJK}{UTF8}{mj}求\end{CJK} $M^{-1}$.

\begin{enumerate}
  \setcounter{enumi}{9}
  \item $f(x), g(x)$ \begin{CJK}{UTF8}{mj}互素\end{CJK}, $f(A) x_{n \times 1}=0, g(A) x_{n \times 1}=0, f(A) g(A) x_{n \times 1}=0$ \begin{CJK}{UTF8}{mj}解空间\end{CJK} $W_{1}, W_{2}, W_{3}$, \begin{CJK}{UTF8}{mj}证明\end{CJK}
\end{enumerate}
$$
W_{3}=W_{1} \oplus W_{2} .
$$

\begin{enumerate}
  \setcounter{enumi}{10}
  \item (\begin{CJK}{UTF8}{mj}有问题\end{CJK}) $A$ \begin{CJK}{UTF8}{mj}为\end{CJK} $\mathbb{R}^{7 \times 7},\left(A^{2}+E_{7}\right)\left(A^{3}-2 E_{7}\right)=0$, \begin{CJK}{UTF8}{mj}求\end{CJK}
\end{enumerate}
(1) $A$ \begin{CJK}{UTF8}{mj}的最小多项式\end{CJK}, \begin{CJK}{UTF8}{mj}特征多项式\end{CJK}.

(2) \begin{CJK}{UTF8}{mj}求\end{CJK} $\lambda E_{7}-A$ \begin{CJK}{UTF8}{mj}的不变因子组\end{CJK}. 11. \begin{CJK}{UTF8}{mj}已知\end{CJK}
$$
A=\left(\begin{array}{cccc}
1 & 2 & 3 & 4 \\
4 & 3 & 2 & 1 \\
0 & 5 & 0 & 6 \\
a_{41} & a_{42} & a_{43} & a_{44}
\end{array}\right)
$$
\begin{CJK}{UTF8}{mj}求\end{CJK} $A_{41}+A_{42}+A_{43}+A_{44}$ \begin{CJK}{UTF8}{mj}及\end{CJK} $M_{41}+M_{42}+M_{43}+M_{44}$.

\begin{enumerate}
  \setcounter{enumi}{12}
  \item \begin{CJK}{UTF8}{mj}设\end{CJK} $A=\left(a_{i j}\right)$ \begin{CJK}{UTF8}{mj}是\end{CJK} $2 n$ \begin{CJK}{UTF8}{mj}阶可逆的反对称矩阵\end{CJK}, \begin{CJK}{UTF8}{mj}方阵\end{CJK} $A_{\lambda}=\left(a_{i j}+\lambda\right)$, \begin{CJK}{UTF8}{mj}求证\end{CJK}
\end{enumerate}
(1) $A^{*}$ \begin{CJK}{UTF8}{mj}也是反对称矩阵\end{CJK}.

(2) $\operatorname{det} A_{\lambda}=\operatorname{det} A$.

\begin{enumerate}
  \setcounter{enumi}{13}
  \item \begin{CJK}{UTF8}{mj}设\end{CJK}
\end{enumerate}
$$
\begin{aligned}
&A_{1}=\left(\begin{array}{ll}
1 & 2 \\
3 & 1
\end{array}\right), A_{2}=\left(\begin{array}{cc}
-1 & 5 \\
4 & 6
\end{array}\right), A_{3}=\left(\begin{array}{cc}
-2 & 3 \\
1 & 5
\end{array}\right) \\
&A_{4}=\left(\begin{array}{cc}
5 & 1 \\
6 & -3
\end{array}\right), A_{5}=\left(\begin{array}{cc}
6 & -4 \\
2 & -9
\end{array}\right), A_{6}=\left(\begin{array}{cc}
-2 & 5 \\
3 & 6
\end{array}\right)
\end{aligned}
$$
\begin{CJK}{UTF8}{mj}求向量组\end{CJK} $(\mathrm{I})=\left(A_{1}, A_{2}, A_{3}, A_{4}, A_{5}, A_{6}\right)$ \begin{CJK}{UTF8}{mj}的一个极大无关组\end{CJK} $\left(\mathrm{I}^{\prime}\right)$ \begin{CJK}{UTF8}{mj}并由\end{CJK} $\left(\mathrm{I}^{\prime}\right)$ \begin{CJK}{UTF8}{mj}线性表示\end{CJK} $(\mathrm{I})$ \begin{CJK}{UTF8}{mj}中其它向量\end{CJK}.

\begin{enumerate}
  \setcounter{enumi}{14}
  \item \begin{CJK}{UTF8}{mj}设向量组\end{CJK} $(\mathrm{I})=\left(\overrightarrow{O A_{1}}, \overrightarrow{O A_{2}}\right)$ \begin{CJK}{UTF8}{mj}和\end{CJK} $(\mathrm{II})=\left(\overrightarrow{O B_{1}}, \overrightarrow{O B_{2}}\right)$ \begin{CJK}{UTF8}{mj}为\end{CJK} 3 \begin{CJK}{UTF8}{mj}维几何向量空间\end{CJK} $V$ \begin{CJK}{UTF8}{mj}中的两个线性无关向量组\end{CJK}, \begin{CJK}{UTF8}{mj}求证必\end{CJK} \begin{CJK}{UTF8}{mj}存在\end{CJK} $V$ \begin{CJK}{UTF8}{mj}中的非零几何向量\end{CJK} $\overrightarrow{O M}$, \begin{CJK}{UTF8}{mj}使得\end{CJK} $\overrightarrow{O M}$ \begin{CJK}{UTF8}{mj}既可由\end{CJK} $(\mathrm{I})$ \begin{CJK}{UTF8}{mj}线性表示\end{CJK}, \begin{CJK}{UTF8}{mj}也可由\end{CJK} $(\mathrm{II})$ \begin{CJK}{UTF8}{mj}线性表示\end{CJK}.

  \item \begin{CJK}{UTF8}{mj}设\end{CJK} $A, B, A-B$ \begin{CJK}{UTF8}{mj}为\end{CJK} $n$ \begin{CJK}{UTF8}{mj}阶正定矩阵\end{CJK}, \begin{CJK}{UTF8}{mj}求证\end{CJK} $B^{-1}-A^{-1}$ \begin{CJK}{UTF8}{mj}也是正定矩阵\end{CJK}.

\end{enumerate}
\section{8. 天津大学 2016 年研究生入学考试试题高等代数}
\begin{CJK}{UTF8}{mj}李扬\end{CJK}

\begin{CJK}{UTF8}{mj}微信公众号\end{CJK}: sxkyliyang

\begin{CJK}{UTF8}{mj}题型\end{CJK}: \begin{CJK}{UTF8}{mj}全是大题\end{CJK}, \begin{CJK}{UTF8}{mj}大概\end{CJK} 13,14 \begin{CJK}{UTF8}{mj}道\end{CJK}.

\begin{enumerate}
  \item \begin{CJK}{UTF8}{mj}第一章有关多项式整除的证明题\end{CJK}.

  \item \begin{CJK}{UTF8}{mj}计算\end{CJK} $n$ \begin{CJK}{UTF8}{mj}阶行列式\end{CJK}

\end{enumerate}
$$
D_{n}=\left|\begin{array}{ccccc}
x & y & \cdots & y & y \\
z & x & \cdots & y & y \\
\vdots & \vdots & & \vdots & \vdots \\
z & z & \cdots & x & y \\
z & z & \cdots & z & x
\end{array}\right|
$$

\begin{enumerate}
  \setcounter{enumi}{3}
  \item \begin{CJK}{UTF8}{mj}设\end{CJK} $\alpha_{1}, \alpha_{2}$ \begin{CJK}{UTF8}{mj}是\end{CJK} 2 \begin{CJK}{UTF8}{mj}阶实对称矩阵\end{CJK} $A$ \begin{CJK}{UTF8}{mj}的分别属于特征值\end{CJK} $\lambda_{1}, \lambda_{2}$ \begin{CJK}{UTF8}{mj}的单位特征向量\end{CJK}, $\lambda_{1} \neq \lambda_{2}$, \begin{CJK}{UTF8}{mj}定义\end{CJK} $\mathbb{R}^{2 \times 2}$ \begin{CJK}{UTF8}{mj}的线性变换\end{CJK} $\mathscr{A}$ \begin{CJK}{UTF8}{mj}如下\end{CJK}:
\end{enumerate}
$$
\mathscr{A}(X)=A X+X A, \forall X \in \mathbb{R}^{2 \times 2}
$$
(1) \begin{CJK}{UTF8}{mj}求证\end{CJK} $\left\{\alpha_{1} \alpha_{1}^{\prime}, \alpha_{1} \alpha_{2}^{\prime}, \alpha_{2} \alpha_{1}^{\prime}, \alpha_{2} \alpha_{2}^{\prime}\right\}$ \begin{CJK}{UTF8}{mj}是\end{CJK} $\mathbb{R}^{2 \times 2}$ \begin{CJK}{UTF8}{mj}的一个基\end{CJK}.

(2) \begin{CJK}{UTF8}{mj}求\end{CJK} $\mathscr{A}$ \begin{CJK}{UTF8}{mj}在基\end{CJK} $\left\{\alpha_{1} \alpha_{1}^{\prime}, \alpha_{1} \alpha_{2}^{\prime}, \alpha_{2} \alpha_{1}^{\prime}, \alpha_{2} \alpha_{2}^{\prime}\right\}$ \begin{CJK}{UTF8}{mj}下的矩阵\end{CJK}.

(3) \begin{CJK}{UTF8}{mj}求证\end{CJK} $\operatorname{tr}(\mathscr{A})=4 \operatorname{tr}(A)$.

\begin{enumerate}
  \setcounter{enumi}{4}
  \item \begin{CJK}{UTF8}{mj}求一矩阵的\end{CJK} Jordan \begin{CJK}{UTF8}{mj}标准形\end{CJK}.
\end{enumerate}
\includegraphics[max width=\textwidth]{2022_04_18_a5c47c0ff534501b502eg-241}

\section{9. 天津大学 2017 年研究生入学考试试题高等代数}
\begin{CJK}{UTF8}{mj}李扬\end{CJK}

\begin{CJK}{UTF8}{mj}微信公众号\end{CJK}: sxkyliyang

\begin{CJK}{UTF8}{mj}一\end{CJK}. \begin{CJK}{UTF8}{mj}求首\end{CJK} 1 \begin{CJK}{UTF8}{mj}的\end{CJK} 3 \begin{CJK}{UTF8}{mj}次多项式\end{CJK} $f(x)$, \begin{CJK}{UTF8}{mj}除以\end{CJK} $x-1$ \begin{CJK}{UTF8}{mj}余\end{CJK} 1 , \begin{CJK}{UTF8}{mj}除以\end{CJK} $x-2$ \begin{CJK}{UTF8}{mj}余\end{CJK} 2 , \begin{CJK}{UTF8}{mj}除以\end{CJK} $x-3$ \begin{CJK}{UTF8}{mj}余\end{CJK} 3 .

\begin{CJK}{UTF8}{mj}二\end{CJK}. $A=\left(a_{i j}\right)(i, j=1,2, \cdots, n)$, \begin{CJK}{UTF8}{mj}第一列元素均为\end{CJK} 1 , \begin{CJK}{UTF8}{mj}且\end{CJK} $a_{i j}=a_{i-1, j}+a_{i, j-1},(i, j=2,3, \cdots, n)$.

(1) \begin{CJK}{UTF8}{mj}将\end{CJK} $A$ \begin{CJK}{UTF8}{mj}作初等行变换\end{CJK}, \begin{CJK}{UTF8}{mj}将\end{CJK} $A$ \begin{CJK}{UTF8}{mj}化为上三角形矩阵\end{CJK}.

(2) \begin{CJK}{UTF8}{mj}求\end{CJK} $A$ \begin{CJK}{UTF8}{mj}的行列式\end{CJK}.

\begin{CJK}{UTF8}{mj}三\end{CJK}. \begin{CJK}{UTF8}{mj}定义线性变换\end{CJK}
$$
\mathscr{A} X=M^{\prime} X M, M=\left(\begin{array}{ll}
1 & 2 \\
0 & 3
\end{array}\right), \forall X \in V
$$
(1) \begin{CJK}{UTF8}{mj}证明\end{CJK} $\mathscr{A}$ \begin{CJK}{UTF8}{mj}为\end{CJK} $V$ \begin{CJK}{UTF8}{mj}上的线性变换\end{CJK}.

(2) \begin{CJK}{UTF8}{mj}求\end{CJK} $\mathscr{A}$ \begin{CJK}{UTF8}{mj}在基\end{CJK} $(\mathrm{I})=\left\{\left(\begin{array}{ll}0 & 1 \\ 0 & 0\end{array}\right),\left(\begin{array}{ll}0 & 1 \\ 1 & 0\end{array}\right),\left(\begin{array}{ll}0 & 0 \\ 0 & 1\end{array}\right)\right\}$ \begin{CJK}{UTF8}{mj}下的矩阵\end{CJK}.

\begin{CJK}{UTF8}{mj}四\end{CJK}. \begin{CJK}{UTF8}{mj}设\end{CJK}
$$
\alpha_{1}=\left(\begin{array}{l}
1 \\
2 \\
1 \\
2
\end{array}\right), \alpha_{2}=\left(\begin{array}{l}
1 \\
2 \\
5 \\
6
\end{array}\right), \alpha_{3}=\left(\begin{array}{l}
4 \\
5 \\
1 \\
6
\end{array}\right), \alpha_{4}=\left(\begin{array}{l}
1 \\
1 \\
2 \\
2
\end{array}\right), \xi=\left(\begin{array}{l}
3 \\
6 \\
6 \\
t
\end{array}\right)
$$
\begin{CJK}{UTF8}{mj}且\end{CJK} $W=L\left(\alpha_{1}, \alpha_{2}, \alpha_{3}, \alpha_{4}\right)$.

(1) \begin{CJK}{UTF8}{mj}证明\end{CJK} $\alpha_{1}, \alpha_{2}, \alpha_{3}$ \begin{CJK}{UTF8}{mj}可作为\end{CJK} $W$ \begin{CJK}{UTF8}{mj}的一组基\end{CJK}

(1) \begin{CJK}{UTF8}{mj}求\end{CJK} $A$ \begin{CJK}{UTF8}{mj}的全部特征向量\end{CJK}.

(2) \begin{CJK}{UTF8}{mj}求\end{CJK} $A$.

\begin{CJK}{UTF8}{mj}六\end{CJK}. \begin{CJK}{UTF8}{mj}已知\end{CJK}
$$
A=\left(\begin{array}{cc}
0 & A_{1} \\
A_{2} & 0
\end{array}\right), A_{1}=\left(\begin{array}{ll}
1 & 1 \\
1 & 1 \\
2 & 3
\end{array}\right), A_{2}=\left(\begin{array}{ccc}
1 & 1 & 2 \\
1 & -1 & 2
\end{array}\right) \text {. }
$$
\begin{CJK}{UTF8}{mj}求\end{CJK} $A^{2 m}$, \begin{CJK}{UTF8}{mj}其中\end{CJK} $m$ \begin{CJK}{UTF8}{mj}为正整数\end{CJK}.
$$
W=\left\{X \in \mathbb{F}^{n \times 2} \mid A X=0\right\} .
$$
\begin{CJK}{UTF8}{mj}证明\end{CJK} $W$ \begin{CJK}{UTF8}{mj}是\end{CJK} $\mathbb{F}^{n \times 2}$ \begin{CJK}{UTF8}{mj}的子空间并且\end{CJK} $W$ \begin{CJK}{UTF8}{mj}的维数为\end{CJK} $2(n-r)$.

\begin{CJK}{UTF8}{mj}八\end{CJK}. $A$ \begin{CJK}{UTF8}{mj}为实对称矩阵\end{CJK},
$$
f_{A}(\lambda)=\left|\lambda E_{n}-A\right|=\lambda^{n}+C_{n-1} \lambda^{n-1}+\cdots+C_{1} \lambda+C_{0}
$$
\begin{CJK}{UTF8}{mj}证明\end{CJK} $A$ \begin{CJK}{UTF8}{mj}为正定矩阵\end{CJK} $\Leftrightarrow C_{n-i}$ \begin{CJK}{UTF8}{mj}与\end{CJK} $(-1)^{i}$ \begin{CJK}{UTF8}{mj}同号\end{CJK}.

\begin{CJK}{UTF8}{mj}九\end{CJK}. \begin{CJK}{UTF8}{mj}在\end{CJK} $V$ \begin{CJK}{UTF8}{mj}中定义内积\end{CJK},
$$
(f(x), g(x))=\int_{0}^{\pi} f(x) g(x) \mathrm{d} x .
$$

\section{0. 天津大学 2009 年研究生入学考试试题数学分析}
\begin{CJK}{UTF8}{mj}李扬\end{CJK}

\begin{CJK}{UTF8}{mj}微信公众号\end{CJK}: sxkyliyang

\begin{CJK}{UTF8}{mj}一\end{CJK}. \begin{CJK}{UTF8}{mj}填空题\end{CJK}.

\begin{enumerate}
  \item \begin{CJK}{UTF8}{mj}极限\end{CJK} $\lim _{n \rightarrow+\infty}\left(\sqrt[n]{n^{2}+1}-1\right) \sin \frac{n \pi}{2}=$

  \item \begin{CJK}{UTF8}{mj}已知函数\end{CJK} $f(x)=\left\{\begin{array}{ll}x^{m} \sin \frac{1}{x}, & x \neq 0 ; \\ 0, & x=0 .\end{array}\right.$ \begin{CJK}{UTF8}{mj}在\end{CJK} $x=0$ \begin{CJK}{UTF8}{mj}附近一阶导数连续\end{CJK}, \begin{CJK}{UTF8}{mj}则\end{CJK} $m=$

  \item \begin{CJK}{UTF8}{mj}设\end{CJK} $\left\{\begin{array}{l}x=\mathrm{e}^{2 t} \cos ^{2} t ; \\ y=\mathrm{e}^{2 t} \sin ^{2} t\end{array}\right.$ \begin{CJK}{UTF8}{mj}求\end{CJK} $\frac{\mathrm{d} y}{\mathrm{~d} x}=$

  \item \begin{CJK}{UTF8}{mj}极限\end{CJK} $\lim _{x \rightarrow 0} \frac{x \mathrm{e}^{x}-\ln (1+x)}{x \tan x \cos ^{4}(5 x)}=$

  \item $\int\left(2^{x}-3^{x}\right) \mathrm{d} x=$

  \item \begin{CJK}{UTF8}{mj}设函数\end{CJK} $f(x)=\frac{\pi-x}{2}$ \begin{CJK}{UTF8}{mj}在\end{CJK} $(0,2 \pi)$ \begin{CJK}{UTF8}{mj}内展开\end{CJK} $f(x)$ \begin{CJK}{UTF8}{mj}的\end{CJK} Fourier \begin{CJK}{UTF8}{mj}级数为\end{CJK}

  \item \begin{CJK}{UTF8}{mj}函数\end{CJK} $f(x)=(x-1) \sqrt[3]{x^{2}}$ \begin{CJK}{UTF8}{mj}的极大值为\end{CJK}

  \item \begin{CJK}{UTF8}{mj}摆线\end{CJK} $x=a(t-\sin t), y=a(1-\cos t)(0 \leqslant t \leqslant 2 \pi)$ \begin{CJK}{UTF8}{mj}的一拱弧长为\end{CJK}

  \item $\int_{0}^{1} \mathrm{~d} y \int_{y}^{1}\left(\mathrm{e}^{-x^{2}}+\mathrm{e}^{x} \sin x\right) \mathrm{d} x=$

  \item \begin{CJK}{UTF8}{mj}若\end{CJK} $\Omega$ \begin{CJK}{UTF8}{mj}为\end{CJK} $x^{2}+y^{2}+z^{2}=z$ \begin{CJK}{UTF8}{mj}所围成的区域\end{CJK}, \begin{CJK}{UTF8}{mj}则三重积分\end{CJK} $\iiint_{\Omega} \sqrt{x^{2}+y^{2}+z^{2}} \mathrm{~d} x \mathrm{~d} y \mathrm{~d} z=$

\end{enumerate}
\begin{CJK}{UTF8}{mj}二\end{CJK}. \begin{CJK}{UTF8}{mj}计算\end{CJK}
$$
I=\iint_{\Sigma} 4 z x \mathrm{~d} y \mathrm{~d} z-2 z y \mathrm{~d} z \mathrm{~d} x+\left(1-z^{2}\right) \mathrm{d} x \mathrm{~d} y
$$
\begin{CJK}{UTF8}{mj}其中\end{CJK} $\Sigma$ \begin{CJK}{UTF8}{mj}是由曲线\end{CJK} $\left\{\begin{array}{l}z=\mathrm{e}^{y} ; \\ x=0 .\end{array} \quad(0 \leqslant y \leqslant a)\right.$ \begin{CJK}{UTF8}{mj}绕\end{CJK} $z$ \begin{CJK}{UTF8}{mj}轴旋转而成的曲面的下侧\end{CJK}.
$$
f(x)= \begin{cases}\left(x^{2}+y^{2}\right) \cos \frac{1}{\sqrt{x^{2}+y^{2}}}, & x^{2}+y^{2} \neq 0 \\ 0, & x^{2}+y^{2}=0\end{cases}
$$
(1) \begin{CJK}{UTF8}{mj}求\end{CJK} $f_{x}^{\prime}(0,0), f_{y}^{\prime}(0,0)$.

(3) \begin{CJK}{UTF8}{mj}证明\end{CJK} $f(x, y)$ \begin{CJK}{UTF8}{mj}在\end{CJK} $(0,0)$ \begin{CJK}{UTF8}{mj}处可微分\end{CJK}.
$$
F^{\prime \prime \prime}(\xi)=0 .
$$
$$
\sum_{n=1}^{\infty}(-1)^{n} \frac{\mathrm{e}^{x^{2}}+\sqrt{n}}{n^{2}}
$$
\begin{CJK}{UTF8}{mj}在任何有限区间上一致收敛\end{CJK}, \begin{CJK}{UTF8}{mj}在任何一点\end{CJK} $x_{0}$ \begin{CJK}{UTF8}{mj}处不绝对收敛\end{CJK}. \begin{CJK}{UTF8}{mj}六\end{CJK}. \begin{CJK}{UTF8}{mj}设\end{CJK} $\lim _{n \rightarrow \infty} x_{n}=0, \lim _{n \rightarrow \infty} y_{n}=0$, \begin{CJK}{UTF8}{mj}且存在正常数\end{CJK} $M$, \begin{CJK}{UTF8}{mj}使得\end{CJK} $\left|y_{1}\right|+\left|y_{2}\right|+\cdots+\left|y_{n}\right| \leqslant M$ \begin{CJK}{UTF8}{mj}对一切自然数\end{CJK} $n$ \begin{CJK}{UTF8}{mj}成立\end{CJK}, \begin{CJK}{UTF8}{mj}令\end{CJK} $z_{n}=x_{1} y_{n}+x_{2} y_{n-1}+\cdots+x_{n} y_{1}$, \begin{CJK}{UTF8}{mj}证明\end{CJK}:
$$
\lim _{n \rightarrow \infty} z_{n}=0 .
$$
\begin{CJK}{UTF8}{mj}七\end{CJK}. \begin{CJK}{UTF8}{mj}设\end{CJK} $0<x<y<1$ \begin{CJK}{UTF8}{mj}或\end{CJK} $1<x<y$, \begin{CJK}{UTF8}{mj}求证\end{CJK}:
$$
\frac{y}{x}>\frac{y^{x}}{x^{y}}
$$
\begin{CJK}{UTF8}{mj}八\end{CJK}. \begin{CJK}{UTF8}{mj}已知在\end{CJK} $x>-1$ \begin{CJK}{UTF8}{mj}上定义的可微函数\end{CJK} $f(x)$ \begin{CJK}{UTF8}{mj}满足条件\end{CJK}
$$
f^{\prime}(x)+f(x)-\frac{1}{x+1} \int_{0}^{x} f(t) \mathrm{d} t=0, f(0) \equiv 1
$$
(1) \begin{CJK}{UTF8}{mj}求\end{CJK} $f^{\prime}(x)$.

(2) \begin{CJK}{UTF8}{mj}证明当\end{CJK} $x \geqslant 0$ \begin{CJK}{UTF8}{mj}时\end{CJK}, $\mathrm{e}^{-x} \leqslant f(x) \leqslant 1$.

\begin{CJK}{UTF8}{mj}九\end{CJK}. \begin{CJK}{UTF8}{mj}设\end{CJK} $a>0$, \begin{CJK}{UTF8}{mj}函数\end{CJK} $f(x)$ \begin{CJK}{UTF8}{mj}在\end{CJK} $[a,+\infty)$ \begin{CJK}{UTF8}{mj}上满足利普希兹条件\end{CJK}, \begin{CJK}{UTF8}{mj}对任意的\end{CJK} $x, y \in[a,+\infty)$, \begin{CJK}{UTF8}{mj}都有\end{CJK}
$$
|f(x)-f(y)| \leqslant L|x-y|
$$
\begin{CJK}{UTF8}{mj}其中\end{CJK} $L$ \begin{CJK}{UTF8}{mj}为常数\end{CJK}, \begin{CJK}{UTF8}{mj}求证\end{CJK} $\frac{f(x)}{x}$ \begin{CJK}{UTF8}{mj}在\end{CJK} $[a,+\infty)$ \begin{CJK}{UTF8}{mj}上一致收敛\end{CJK}.

\begin{CJK}{UTF8}{mj}十\end{CJK}. \begin{CJK}{UTF8}{mj}若\end{CJK} $f(x)$ \begin{CJK}{UTF8}{mj}是\end{CJK} $[a,+\infty)$ \begin{CJK}{UTF8}{mj}上的单调函数\end{CJK}, \begin{CJK}{UTF8}{mj}且\end{CJK} $\int_{a}^{+\infty} f(x) \mathrm{d} x$ \begin{CJK}{UTF8}{mj}收敛\end{CJK}, \begin{CJK}{UTF8}{mj}求证\end{CJK}

\includegraphics[max width=\textwidth]{2022_04_18_a5c47c0ff534501b502eg-244}

\section{1. 天津大学 2010 年研究生入学考试试题数学分析}
\begin{CJK}{UTF8}{mj}李扬\end{CJK}

\begin{CJK}{UTF8}{mj}微信公众号\end{CJK}: sxkyliyang

\begin{CJK}{UTF8}{mj}一\end{CJK}. \begin{CJK}{UTF8}{mj}填空题\end{CJK}.

\begin{enumerate}
  \item \begin{CJK}{UTF8}{mj}极限\end{CJK} $\lim _{n \rightarrow \infty}\left(\frac{1}{n^{2}+n+1}+\frac{2}{n^{2}+n+1}+\cdots+\frac{n}{n^{2}+n+1}\right)=$

  \item \begin{CJK}{UTF8}{mj}曲线\end{CJK} $y=x^{3}$ \begin{CJK}{UTF8}{mj}与\end{CJK} $y=\frac{16}{x}(x>0)$ \begin{CJK}{UTF8}{mj}在交点处对应切线的夹角为\end{CJK}

  \item \begin{CJK}{UTF8}{mj}由方程\end{CJK} $y^{2}+2 \ln y-x^{4}=0$ \begin{CJK}{UTF8}{mj}所确定的隐函数\end{CJK} $y=y(x)$ \begin{CJK}{UTF8}{mj}的二阶导数\end{CJK} $\frac{\mathrm{d}^{2} y}{\mathrm{~d} x^{2}}=$

  \item \begin{CJK}{UTF8}{mj}设\end{CJK} $F^{\prime}(x)=\frac{1}{\sqrt{1-x^{2}}}$, \begin{CJK}{UTF8}{mj}且\end{CJK} $F(1)=\frac{3 \pi}{2}$, \begin{CJK}{UTF8}{mj}则\end{CJK} $F(x)=$

  \item $\lim _{x \rightarrow 0} \frac{1}{b x-2 \sin x} \int_{0}^{x} \frac{t^{2}}{\sqrt{a+t^{2}}} \mathrm{~d} t=3$, \begin{CJK}{UTF8}{mj}则\end{CJK} $a=$ $b=$

  \item \begin{CJK}{UTF8}{mj}级数\end{CJK} $\sum_{n=1}^{\infty} \frac{1}{3^{n}} x^{2 n}$ \begin{CJK}{UTF8}{mj}的收敛半径为\end{CJK}

  \item \begin{CJK}{UTF8}{mj}曲线\end{CJK} $x=3 t, y=-t^{2}, z=\frac{1}{3} t^{3}$ \begin{CJK}{UTF8}{mj}上与平面\end{CJK} $x-2 y+z=7$ \begin{CJK}{UTF8}{mj}平行的切线方程为\end{CJK}

  \item \begin{CJK}{UTF8}{mj}设\end{CJK} $D$ \begin{CJK}{UTF8}{mj}为曲线\end{CJK} $x^{2}+y^{2}=x+y$ \begin{CJK}{UTF8}{mj}所围成的区域\end{CJK}, \begin{CJK}{UTF8}{mj}则二重积分\end{CJK} $\iint_{D}(x+y) \mathrm{d} x \mathrm{~d} y=$

  \item \begin{CJK}{UTF8}{mj}曲线\end{CJK} $y=\ln \left(1-x^{2}\right), 0 \leqslant x \leqslant \frac{1}{2}$ \begin{CJK}{UTF8}{mj}的弧长为\end{CJK}

  \item \begin{CJK}{UTF8}{mj}极限\end{CJK} $\lim _{x \rightarrow 0} \frac{x-x \cos \sqrt{x}}{1-\sqrt{\cos x}}=$

\end{enumerate}
\begin{CJK}{UTF8}{mj}二\end{CJK}. \begin{CJK}{UTF8}{mj}计算积分\end{CJK}

\includegraphics[max width=\textwidth]{2022_04_18_a5c47c0ff534501b502eg-245}

\begin{CJK}{UTF8}{mj}其中\end{CJK} $C$ \begin{CJK}{UTF8}{mj}为从\end{CJK} $(1,0)$ \begin{CJK}{UTF8}{mj}到\end{CJK} $(-1,0)$ \begin{CJK}{UTF8}{mj}的半圆\end{CJK} $y=\sqrt{1-x^{2}}(-1 \leqslant x \leqslant 1)$.

\begin{CJK}{UTF8}{mj}三\end{CJK}. \begin{CJK}{UTF8}{mj}求\end{CJK}
$$
z=2 x^{2}+y^{2}-8 x-2 y+9
$$
\begin{CJK}{UTF8}{mj}四\end{CJK}. \begin{CJK}{UTF8}{mj}证明\end{CJK}: \begin{CJK}{UTF8}{mj}当\end{CJK} $x \in\left(0, \frac{\pi}{2}\right)$ \begin{CJK}{UTF8}{mj}时\end{CJK}, $\sin x>x-\frac{x^{3}}{6}$.

\begin{CJK}{UTF8}{mj}五\end{CJK}. \begin{CJK}{UTF8}{mj}试证\end{CJK}: \begin{CJK}{UTF8}{mj}无穷级数\end{CJK}
$$
\sum_{n=1}^{\infty} \frac{n^{2}}{2+n^{4} x}
$$
$$
f^{\prime \prime}(\xi) \geqslant 8
$$
$$
\lim _{n \rightarrow \infty} \frac{a_{n}}{\sqrt{2 n}}=1
$$
\begin{CJK}{UTF8}{mj}八\end{CJK}. \begin{CJK}{UTF8}{mj}设\end{CJK} $f(x)$ \begin{CJK}{UTF8}{mj}在\end{CJK} $[a, b]$ \begin{CJK}{UTF8}{mj}上连续\end{CJK}, \begin{CJK}{UTF8}{mj}且\end{CJK} $\int_{a}^{b} f(x) \mathrm{d} x=0, \int_{a}^{b} x f(x) \mathrm{d} x=0$. \begin{CJK}{UTF8}{mj}证明\end{CJK}: \begin{CJK}{UTF8}{mj}至少存在互不相同的两点\end{CJK} $x_{1}, x_{2} \in(a, b)$, \begin{CJK}{UTF8}{mj}使得\end{CJK}
$$
f\left(x_{1}\right)=f\left(x_{2}\right)=0
$$
\begin{CJK}{UTF8}{mj}九\end{CJK}. \begin{CJK}{UTF8}{mj}设函数\end{CJK} $f$ \begin{CJK}{UTF8}{mj}在\end{CJK} $[a, b]$ \begin{CJK}{UTF8}{mj}上递增\end{CJK}, \begin{CJK}{UTF8}{mj}且\end{CJK} $f(a) \geqslant a, f(b) \leqslant b$, \begin{CJK}{UTF8}{mj}求证存在点\end{CJK} $\xi \in[a, b]$, \begin{CJK}{UTF8}{mj}使得\end{CJK}
$$
f(\xi)=\xi
$$
\begin{CJK}{UTF8}{mj}十\end{CJK}. \begin{CJK}{UTF8}{mj}设\end{CJK} $x_{n}>0(n=1,2, \cdots)$, \begin{CJK}{UTF8}{mj}求证\end{CJK}
$$
\lim _{n \rightarrow \infty}\left(\frac{x_{1}+x_{n+1}}{x_{n}}\right)^{n} \geqslant \mathrm{e}
$$

\section{2. 天津大学 2011 年研究生入学考试试题数学分析}
\begin{CJK}{UTF8}{mj}李扬\end{CJK}

\begin{CJK}{UTF8}{mj}微信公众号\end{CJK}: sxkyliyang

\begin{CJK}{UTF8}{mj}一\end{CJK}. \begin{CJK}{UTF8}{mj}填空题\end{CJK}.

\begin{enumerate}
  \item \begin{CJK}{UTF8}{mj}求\end{CJK} $u=x y z$ \begin{CJK}{UTF8}{mj}在点\end{CJK} $(0,0,0)$ \begin{CJK}{UTF8}{mj}沿方向向量\end{CJK} $(2,-1,3)$ \begin{CJK}{UTF8}{mj}的方向导数\end{CJK}

  \item \begin{CJK}{UTF8}{mj}计算\end{CJK} $\lim _{x \rightarrow 0} \frac{\tan x(\sin 5 x-\sin 3 x)}{x \sin 2 x \cdot \cos ^{3} 4 x}=$

\end{enumerate}
3 . \begin{CJK}{UTF8}{mj}计算\end{CJK} $\sum \frac{\ln (n+1)}{n+1} x^{n+1}$ \begin{CJK}{UTF8}{mj}的收敛域是\end{CJK}

\begin{enumerate}
  \setcounter{enumi}{4}
  \item \begin{CJK}{UTF8}{mj}计算\end{CJK} $x(t)=\frac{t}{1+t}, y(t)=\frac{1+t}{t}, z(t)=t^{2}$ \begin{CJK}{UTF8}{mj}在点\end{CJK} $t=1$ \begin{CJK}{UTF8}{mj}处切线方程\end{CJK}

  \item \begin{CJK}{UTF8}{mj}计算\end{CJK} $f(x)=(1+x)^{2}$ \begin{CJK}{UTF8}{mj}在\end{CJK} $x=0$ \begin{CJK}{UTF8}{mj}处的皮亚诺型余项展开式\end{CJK}

  \item \begin{CJK}{UTF8}{mj}计算隐函数\end{CJK} $x \sin y+y \mathrm{e}^{x}+2=0$ \begin{CJK}{UTF8}{mj}所确定的函数\end{CJK} $y=f(x)$ \begin{CJK}{UTF8}{mj}在点\end{CJK} $x=0$ \begin{CJK}{UTF8}{mj}处的导数\end{CJK}

\end{enumerate}
\includegraphics[max width=\textwidth]{2022_04_18_a5c47c0ff534501b502eg-247}

\begin{enumerate}
  \setcounter{enumi}{8}
  \item \begin{CJK}{UTF8}{mj}计算\end{CJK} $y=x^{4}-2 x^{2}$ \begin{CJK}{UTF8}{mj}的所有极值点\end{CJK}

  \item \begin{CJK}{UTF8}{mj}计算\end{CJK} $\int \frac{2^{x+1}-5^{x-1}}{10^{x}} \mathrm{~d} x=$

  \item \begin{CJK}{UTF8}{mj}计算\end{CJK} $\lim _{\substack{x \rightarrow 0 \\ y \rightarrow 0}}\left(x^{2}+y^{2}\right)^{x^{2} y^{2}}=$

\end{enumerate}
\begin{CJK}{UTF8}{mj}二\end{CJK}. \begin{CJK}{UTF8}{mj}证明二元函数\end{CJK}
$$
f(x, y)= \begin{cases}\left(x^{2}+y^{2}\right) \sin \frac{1}{x^{2}+y^{2}}, & (x, y) \neq 0 ; \\ 0, & (x, y)=0 .\end{cases}
$$
\begin{CJK}{UTF8}{mj}在点\end{CJK} $(0,0)$ \begin{CJK}{UTF8}{mj}处的偏导数存在但不连续\end{CJK}, \begin{CJK}{UTF8}{mj}并且在\end{CJK} $(0,0)$ \begin{CJK}{UTF8}{mj}的任何邻域内无界\end{CJK}, \begin{CJK}{UTF8}{mj}在\end{CJK} $(0,0)$ \begin{CJK}{UTF8}{mj}处可微\end{CJK}.
$$
\lim _{n \rightarrow \infty} \frac{a_{1}+2 a_{2}+\cdots+n a_{n}}{n}=0 .
$$
\begin{CJK}{UTF8}{mj}四\end{CJK}. \begin{CJK}{UTF8}{mj}求证\end{CJK} $\lim _{x \rightarrow x^{+}} f(x)=+\infty$ \begin{CJK}{UTF8}{mj}的充分必要条件是对任意的从右边收敛于\end{CJK} $x_{0}$ \begin{CJK}{UTF8}{mj}的数列\end{CJK} $\left\{x_{n}\right\}$ \begin{CJK}{UTF8}{mj}有\end{CJK}
$$
\lim _{n \rightarrow+\infty} f\left(x_{n}\right)=+\infty
$$
$$
\int_{-\infty}^{+\infty}[f(x+a)-f(x)] \mathrm{d} x
$$
\begin{CJK}{UTF8}{mj}存在\end{CJK}, \begin{CJK}{UTF8}{mj}并求其值\end{CJK}.
$$
f_{n}(x)=\sum_{k=0}^{n-1} \frac{1}{n} f\left(x+\frac{k}{n}\right)
$$
\begin{CJK}{UTF8}{mj}八\end{CJK}. \begin{CJK}{UTF8}{mj}已知\end{CJK}

(1) $f(x)$ \begin{CJK}{UTF8}{mj}在\end{CJK} $\left(x_{0}-\delta, x_{0}+\delta\right)$ \begin{CJK}{UTF8}{mj}存在\end{CJK} $n$ \begin{CJK}{UTF8}{mj}阶函数\end{CJK}.

(2) \begin{CJK}{UTF8}{mj}当\end{CJK} $k=2,3, \cdots, n-1$ \begin{CJK}{UTF8}{mj}时\end{CJK}, $f^{(k)}\left(x_{0}\right)=0$, \begin{CJK}{UTF8}{mj}但\end{CJK} $f^{(n)}\left(x_{0}\right) \neq 0$.

(3) \begin{CJK}{UTF8}{mj}对\end{CJK} $\frac{f\left(x_{0}+h\right)-f\left(x_{0}\right)}{h}=f^{\prime}\left(x_{0}+h \theta(h)\right)$, \begin{CJK}{UTF8}{mj}其中\end{CJK} $0<\theta<1$.

\begin{CJK}{UTF8}{mj}证明\end{CJK}:
$$
\lim _{h \rightarrow \infty} \theta(h)=\sqrt[n-1]{\frac{1}{n}}
$$
\begin{CJK}{UTF8}{mj}九\end{CJK}. \begin{CJK}{UTF8}{mj}已知\end{CJK} $x_{1}=\frac{1}{2}, x_{n+1}=x_{n}^{2}+x_{n-1}$, \begin{CJK}{UTF8}{mj}证明\end{CJK}
$$
\lim _{n \rightarrow \infty}\left(\frac{1}{1+x_{1}}+\frac{1}{1+x_{2}}+\cdots+\frac{1}{1+x_{n}}\right)=2 .
$$

\section{3. 天津大学 2012 年研究生入学考试试题数学分析}
\begin{CJK}{UTF8}{mj}李扬\end{CJK}

\begin{CJK}{UTF8}{mj}微信公众号\end{CJK}: sxkyliyang

-. \begin{CJK}{UTF8}{mj}填空题\end{CJK}.

\begin{enumerate}
  \item \begin{CJK}{UTF8}{mj}已知\end{CJK} $\left.D=\{(x, y)) \mid x^{2}+y^{2} \leqslant 1\right\}$, \begin{CJK}{UTF8}{mj}内求\end{CJK} $\iint_{D}\left(x^{2}+\frac{x \cos y}{y^{2}}\right) \mathrm{d} x \mathrm{~d} y=$

  \item $\sum_{n=0}^{\infty} \frac{n^{n+1}}{n+1} x^{n+1}$ \begin{CJK}{UTF8}{mj}的收敛域\end{CJK}

  \item $\lim _{n \rightarrow \infty}\left[\left(1+\frac{1}{n}\right)\left(1+\frac{2}{n}\right) \cdots\left(1+\frac{n}{n}\right)\right]^{\frac{1}{n}}=$

  \item $f(x)=\left\{\begin{array}{ll}x \sin \frac{1}{x}, & x \neq 0 ; \\ 0, & x=0 .\end{array}\right.$ \begin{CJK}{UTF8}{mj}求\end{CJK} $f^{\prime}(0)=$

\end{enumerate}
\begin{CJK}{UTF8}{mj}二\end{CJK}. 1. \begin{CJK}{UTF8}{mj}已知\end{CJK} $\frac{f(x)}{x}$ \begin{CJK}{UTF8}{mj}为减函数\end{CJK}, $x \in(0,+\infty)$, \begin{CJK}{UTF8}{mj}证明\end{CJK} $\forall a>0, b>0$,
$$
f(a+b) \leqslant f(a) f(b)
$$

\begin{enumerate}
  \setcounter{enumi}{2}
  \item (\begin{CJK}{UTF8}{mj}有问题\end{CJK}) \begin{CJK}{UTF8}{mj}已知二元函数\end{CJK} $f(x, y)$ \begin{CJK}{UTF8}{mj}在平面上连续\end{CJK}, $A, B$ \begin{CJK}{UTF8}{mj}是某圆上直径的两端点\end{CJK}, $u=\max \sqrt{\left(\frac{\partial f}{\partial x}\right)^{2}\left(\frac{\partial f}{\partial y}\right)^{2}}$, \begin{CJK}{UTF8}{mj}证明\end{CJK}
\end{enumerate}
$$
\frac{\partial f}{\partial x}+\frac{\partial f}{\partial y} \leqslant u|A B| .
$$

\begin{enumerate}
  \setcounter{enumi}{3}
  \item \begin{CJK}{UTF8}{mj}使\end{CJK}
\end{enumerate}
\begin{CJK}{UTF8}{mj}最小\end{CJK}, \begin{CJK}{UTF8}{mj}求\end{CJK} $a, b$ \begin{CJK}{UTF8}{mj}的值\end{CJK}

\includegraphics[max width=\textwidth]{2022_04_18_a5c47c0ff534501b502eg-249}

\begin{CJK}{UTF8}{mj}三\end{CJK}. 1. \begin{CJK}{UTF8}{mj}已知\end{CJK} $I(\alpha)=\int_{0}^{\alpha} \frac{\varphi(x)}{(\alpha-x)} \mathrm{d} x, \varphi(x)$ \begin{CJK}{UTF8}{mj}有连续的导数\end{CJK}, \begin{CJK}{UTF8}{mj}证明\end{CJK}
$$
I^{\prime}(\alpha)=\frac{\varphi^{\prime}(0)}{\alpha}+\int_{0}^{\alpha} \frac{\varphi^{\prime}(x)}{(\alpha-x)} \mathrm{d} x
$$

\begin{enumerate}
  \setcounter{enumi}{2}
  \item (\begin{CJK}{UTF8}{mj}少条件\end{CJK})\begin{CJK}{UTF8}{mj}已知\end{CJK}
\end{enumerate}
$$
f_{n}(x)=\sqrt{x f_{n-1}(x)},
$$
\begin{CJK}{UTF8}{mj}证明\end{CJK}

(1) $f_{n}(x)$ \begin{CJK}{UTF8}{mj}收敛且连续\end{CJK}.

(2) \begin{CJK}{UTF8}{mj}证明一致连续\end{CJK}.

\begin{enumerate}
  \setcounter{enumi}{3}
  \item $0<x_{1}<1, x_{n+1}=\sin x_{n}$, \begin{CJK}{UTF8}{mj}证明\end{CJK}
\end{enumerate}
$$
\lim _{n \rightarrow \infty} \sqrt{n} x_{n}=\sqrt{3} .
$$

\section{4. 天津大学 2013 年研究生入学考试试题数学分析}
\begin{CJK}{UTF8}{mj}李扬\end{CJK}

\begin{CJK}{UTF8}{mj}微信公众号\end{CJK}: sxkyliyang

-. \begin{CJK}{UTF8}{mj}填空题\end{CJK}.

\begin{enumerate}
  \item \begin{CJK}{UTF8}{mj}三元函数\end{CJK} $u=x y^{2}+y z^{2}$ \begin{CJK}{UTF8}{mj}在点\end{CJK} $(-2,1,1)$ \begin{CJK}{UTF8}{mj}的梯度为\end{CJK}

  \item $\lim _{x \rightarrow 1} \frac{1-x^{2}}{1-2 \sin ^{2}\left(\frac{\pi}{4} x\right)}=$

  \item $\int \ln ^{2} x \mathrm{~d} x=$

  \item \begin{CJK}{UTF8}{mj}函数\end{CJK} $z=\tan \left(x^{2}+y^{2}\right)$ \begin{CJK}{UTF8}{mj}的不连续点构成的集合为\end{CJK}

\end{enumerate}
5 . \begin{CJK}{UTF8}{mj}设\end{CJK} $a \in \mathbb{R}$, \begin{CJK}{UTF8}{mj}则函数\end{CJK} $\sum_{n=1}^{+\infty} 4^{n}(x+a)^{4 n}$ \begin{CJK}{UTF8}{mj}的收敛区间为\end{CJK}

\begin{enumerate}
  \setcounter{enumi}{6}
  \item \begin{CJK}{UTF8}{mj}函数\end{CJK} $(x-1)^{3} \sqrt{x^{2}}$ \begin{CJK}{UTF8}{mj}的单调减区间\end{CJK}

  \item \begin{CJK}{UTF8}{mj}设\end{CJK} $y, z$ \begin{CJK}{UTF8}{mj}关于\end{CJK} $x$ \begin{CJK}{UTF8}{mj}的隐函数由方程组\end{CJK} $\left\{\begin{array}{l}x^{2}+y^{2}-z=0 ; \\ x^{2}+2 y^{2}+3 z=0 .\end{array}\right.$ \begin{CJK}{UTF8}{mj}确定\end{CJK}, \begin{CJK}{UTF8}{mj}则\end{CJK} $\frac{\mathrm{d} y}{\mathrm{~d} x}=$

  \item \begin{CJK}{UTF8}{mj}曲面\end{CJK} $\mathrm{e}^{z}-z+x y=3$ \begin{CJK}{UTF8}{mj}在点\end{CJK} $(2,1,0)$ \begin{CJK}{UTF8}{mj}的单位法向量\end{CJK}

  \item $\int_{0}^{+\infty} 3^{-2 y^{2}} \mathrm{~d} y=$

  \item \begin{CJK}{UTF8}{mj}设\end{CJK} $D=\left\{(x, y) \mid x^{2}+y^{2} \leqslant 1\right\}$, \begin{CJK}{UTF8}{mj}则\end{CJK} $\iint_{D} \sin (x+y) \mathrm{d} x \mathrm{~d} y=$

\end{enumerate}
\begin{CJK}{UTF8}{mj}二\end{CJK}. \begin{CJK}{UTF8}{mj}计算第二类曲面积分\end{CJK}

\includegraphics[max width=\textwidth]{2022_04_18_a5c47c0ff534501b502eg-250}

\begin{CJK}{UTF8}{mj}其中\end{CJK} $\Sigma: x^{2}+y^{2}+z^{2}=1$, \begin{CJK}{UTF8}{mj}外侧为正\end{CJK}:

\begin{CJK}{UTF8}{mj}三\end{CJK}. \begin{CJK}{UTF8}{mj}求曲面\end{CJK} $x+y=1$ \begin{CJK}{UTF8}{mj}和\end{CJK} $x^{2}+y^{2}+2 z^{2}=1$ \begin{CJK}{UTF8}{mj}的交线上距离原点最近的点\end{CJK}.

\begin{CJK}{UTF8}{mj}四\end{CJK}. \begin{CJK}{UTF8}{mj}设\end{CJK}
$$
f(x, y)=|x| \sin (x y)
$$
\begin{CJK}{UTF8}{mj}试讨论\end{CJK} $f$ \begin{CJK}{UTF8}{mj}在\end{CJK} $(0,0)$ \begin{CJK}{UTF8}{mj}处的可微性\end{CJK}.

\begin{CJK}{UTF8}{mj}五\end{CJK}. \begin{CJK}{UTF8}{mj}证明\end{CJK}
$$
S(x)=\sum_{n=1}^{+\infty} \frac{\sin n x}{n^{4}}
$$
\begin{CJK}{UTF8}{mj}在\end{CJK} $(-\infty,+\infty)$ \begin{CJK}{UTF8}{mj}有二阶连续导数\end{CJK}.

\begin{CJK}{UTF8}{mj}六\end{CJK}. \begin{CJK}{UTF8}{mj}设\end{CJK} $p>\frac{1}{2}$, \begin{CJK}{UTF8}{mj}证明如果正项级数\end{CJK} $\sum_{n=1}^{+\infty} u_{n}$ \begin{CJK}{UTF8}{mj}收敛\end{CJK}, \begin{CJK}{UTF8}{mj}则\end{CJK}
$$
\sum_{n=1}^{+\infty} \frac{\sqrt{u_{n}}}{n^{p}}
$$
\begin{CJK}{UTF8}{mj}也收敛\end{CJK}.

\begin{CJK}{UTF8}{mj}七\end{CJK}. \begin{CJK}{UTF8}{mj}设\end{CJK} $f(x)$ \begin{CJK}{UTF8}{mj}在\end{CJK} $[a, b]$ \begin{CJK}{UTF8}{mj}上连续且在\end{CJK} $(a, b)$ \begin{CJK}{UTF8}{mj}内只有一个极大值点和一个极小值点\end{CJK}, \begin{CJK}{UTF8}{mj}求证\end{CJK} $f(x)$ \begin{CJK}{UTF8}{mj}在\end{CJK} $(a, b)$ \begin{CJK}{UTF8}{mj}内极大值必大于\end{CJK} \begin{CJK}{UTF8}{mj}极小值\end{CJK}. \begin{CJK}{UTF8}{mj}八\end{CJK}. \begin{CJK}{UTF8}{mj}试证明不存在定义在\end{CJK} $[0,2]$ \begin{CJK}{UTF8}{mj}上的连续可微函数\end{CJK} $f(x)$, \begin{CJK}{UTF8}{mj}使得\end{CJK} $f(0)=f(2)=1,\left|f^{\prime}(x)\right| \leqslant 1,\left|\int_{0}^{2} f(x) \mathrm{d} x\right| 1$.

\begin{CJK}{UTF8}{mj}九\end{CJK}. \begin{CJK}{UTF8}{mj}设函数\end{CJK} $f(x)$ \begin{CJK}{UTF8}{mj}在\end{CJK} $(0,+\infty)$ \begin{CJK}{UTF8}{mj}上可导\end{CJK}, \begin{CJK}{UTF8}{mj}且\end{CJK} $\lim _{x \rightarrow+\infty}\left(f(x)+f^{\prime}(x)\right)=0$, \begin{CJK}{UTF8}{mj}证明\end{CJK} $\lim _{x \rightarrow+\infty} f(x)=0$.

\begin{CJK}{UTF8}{mj}十\end{CJK}. \begin{CJK}{UTF8}{mj}设函数\end{CJK} $f(x)$ \begin{CJK}{UTF8}{mj}在\end{CJK} $(-1,1)$ \begin{CJK}{UTF8}{mj}上二阶导数存在且连续\end{CJK}, $f(0)=f^{\prime}(0)=0, f^{\prime \prime}(0) \neq 0$, \begin{CJK}{UTF8}{mj}对于\end{CJK} $x \in(-1,1), u$ \begin{CJK}{UTF8}{mj}为\end{CJK} $f(x)$ \begin{CJK}{UTF8}{mj}在\end{CJK} $(x, f(x))$ \begin{CJK}{UTF8}{mj}处的切线与\end{CJK} $x$ \begin{CJK}{UTF8}{mj}轴的交点的横坐标\end{CJK}, \begin{CJK}{UTF8}{mj}求\end{CJK}
$$
\lim _{x \rightarrow 0} \frac{u f(x)}{x f(u)}
$$

\section{5. 天津大学 2014 年研究生入学考试试题数学分析}
\begin{CJK}{UTF8}{mj}李扬\end{CJK}

\begin{CJK}{UTF8}{mj}微信公众号\end{CJK}: sxkyliyang

\begin{CJK}{UTF8}{mj}一\end{CJK}. \begin{CJK}{UTF8}{mj}填空题\end{CJK}.

\begin{enumerate}
  \item \begin{CJK}{UTF8}{mj}设\end{CJK} $f(x, y)=\left\{\begin{array}{ll}\frac{x y^{2}}{x^{2}+y^{2}}, & x^{2}+y^{2} \neq 0 ; \\ 0, & x^{2}+y^{2}=0 .\end{array}\right.$ \begin{CJK}{UTF8}{mj}则\end{CJK} $\frac{\partial f}{\partial x}(0,0)=$

  \item $\lim _{n \rightarrow+\infty}\left(1+\frac{2}{n}+\frac{3}{n^{2}}\right)^{n}=$

  \item $\int x^{x}(1+\ln x) \mathrm{d} x=$

  \item $\lim _{x \rightarrow 0} \frac{x-\arctan x}{x^{2} \arctan x}=$

  \item \begin{CJK}{UTF8}{mj}向量场\end{CJK} $\bar{A}(x, y, z)=2 x y^{2} \vec{i}+x y \vec{j}+z^{3} \vec{k}$ \begin{CJK}{UTF8}{mj}散度为\end{CJK}

  \item \begin{CJK}{UTF8}{mj}设\end{CJK} $L=\left\{(x, y) \mid \frac{x^{2}}{4}+y^{2}=1\right\}$ \begin{CJK}{UTF8}{mj}逆时针为正\end{CJK}, \begin{CJK}{UTF8}{mj}则\end{CJK} $\int_{L} x \mathrm{~d} y=$

  \item $\lim _{x \rightarrow 1} \lim _{y \rightarrow 0^{+}}\left(\frac{\left(\mathrm{e}^{y}-1\right)^{x}}{(2 \sin y)^{x}}\right)=$

  \item $\int_{0}^{1} x^{4}(1-x)^{3} \mathrm{~d} x=$

  \item \begin{CJK}{UTF8}{mj}级数\end{CJK} $\sum_{n=1}^{\infty} \frac{x^{n}}{3^{n}+2^{n}}$ \begin{CJK}{UTF8}{mj}的收敛半径\end{CJK} $R=$

  \item \begin{CJK}{UTF8}{mj}设\end{CJK} $f(u)$ \begin{CJK}{UTF8}{mj}具有连续导数\end{CJK}, \begin{CJK}{UTF8}{mj}且\end{CJK} $f(0)=0$, \begin{CJK}{UTF8}{mj}则\end{CJK} $\lim _{t \rightarrow 0} \frac{1}{\pi t^{4}} \iiint_{x^{2}+y^{2}+z^{2} \leqslant t^{2}} f\left(\sqrt{x^{2}+y^{2}+z^{2}}\right) \mathrm{d} v=$

\end{enumerate}
\begin{CJK}{UTF8}{mj}二\end{CJK}. \begin{CJK}{UTF8}{mj}求函数\end{CJK}

-\begin{CJK}{UTF8}{mj}求函数\end{CJK}
$$
\sum_{i=1}^{n} a_{i} x_{i}=1\left(a_{i}>0, i=1, \cdots, n\right) \text { 之下的最小值. }
$$
\begin{CJK}{UTF8}{mj}三\end{CJK}. \begin{CJK}{UTF8}{mj}计算\end{CJK}
$$
\int_{0}^{\pi} \frac{x \sin x}{1+\cos ^{2} x} \mathrm{~d} x
$$
\begin{CJK}{UTF8}{mj}四\end{CJK}. \begin{CJK}{UTF8}{mj}试作出函数\end{CJK}
$$
y=\int_{0}^{1} t|t-x| \mathrm{d} t
$$
\begin{CJK}{UTF8}{mj}在\end{CJK} $(-\infty,+\infty)$ \begin{CJK}{UTF8}{mj}上的图形\end{CJK}.

\begin{CJK}{UTF8}{mj}五\end{CJK}. \begin{CJK}{UTF8}{mj}设\end{CJK} $\left\{u_{n}\right\}$ \begin{CJK}{UTF8}{mj}是单调递减的正数且\end{CJK} $\sum_{n=1}^{\infty} u_{n}$ \begin{CJK}{UTF8}{mj}收敛\end{CJK}, \begin{CJK}{UTF8}{mj}证\end{CJK}
$$
\lim _{n \rightarrow \infty} n u_{n}=0 .
$$
\begin{CJK}{UTF8}{mj}六\end{CJK}. \begin{CJK}{UTF8}{mj}设\end{CJK} $c>0$, \begin{CJK}{UTF8}{mj}任取\end{CJK} $x_{0} \in\left(0, \frac{1}{c}\right), x_{1}=x_{0}\left(2-c x_{0}\right), \cdots, x_{n+1}=x_{n}\left(2-c x_{n}\right)$, \begin{CJK}{UTF8}{mj}证明\end{CJK} $\left\{x_{n}\right\}$ \begin{CJK}{UTF8}{mj}极限存在并求极限\end{CJK}. \begin{CJK}{UTF8}{mj}七\end{CJK}. \begin{CJK}{UTF8}{mj}设\end{CJK} $f_{0}(x), \cdots, f_{n}(x), \cdots$ \begin{CJK}{UTF8}{mj}在区间\end{CJK} $I$ \begin{CJK}{UTF8}{mj}上有定义\end{CJK}, \begin{CJK}{UTF8}{mj}且满足\end{CJK} $\left|f_{0}(x)\right| \leqslant M, \sum_{n=0}^{m}\left|f_{n}(x)-f_{n+1}(x)\right| \leqslant M(m=$ $0,1,2, \cdots), M$ \begin{CJK}{UTF8}{mj}为常数\end{CJK}, \begin{CJK}{UTF8}{mj}证明\end{CJK}: \begin{CJK}{UTF8}{mj}若级数\end{CJK} $\sum_{n=0}^{+\infty} b_{n}$ \begin{CJK}{UTF8}{mj}收敛\end{CJK}, \begin{CJK}{UTF8}{mj}则级数\end{CJK} $\sum_{n=0}^{+\infty} b_{n} f_{n}(x)$ \begin{CJK}{UTF8}{mj}在区间\end{CJK} $I$ \begin{CJK}{UTF8}{mj}上一致收敛\end{CJK}.

\begin{CJK}{UTF8}{mj}八\end{CJK}. \begin{CJK}{UTF8}{mj}设\end{CJK} $f(x)$ \begin{CJK}{UTF8}{mj}在\end{CJK} $[a, b]$ \begin{CJK}{UTF8}{mj}上可微\end{CJK}, \begin{CJK}{UTF8}{mj}证明\end{CJK}: $f^{\prime}(x)$ \begin{CJK}{UTF8}{mj}在\end{CJK} $[a, b]$ \begin{CJK}{UTF8}{mj}上连续的充要条件是\end{CJK} $f(x)$ \begin{CJK}{UTF8}{mj}在\end{CJK} $[a, b]$ \begin{CJK}{UTF8}{mj}上一致可微\end{CJK}, \begin{CJK}{UTF8}{mj}即对任意\end{CJK} $\varepsilon>0, \exists \delta>0$, \begin{CJK}{UTF8}{mj}当\end{CJK} $0<|h|<\delta$ \begin{CJK}{UTF8}{mj}时有\end{CJK}
$$
\left|\frac{f(x+h)-f(x)}{h}-f^{\prime}(x)\right|<\varepsilon
$$
\begin{CJK}{UTF8}{mj}对一切\end{CJK} $x \in[a, b]$ \begin{CJK}{UTF8}{mj}成立\end{CJK}.

\begin{CJK}{UTF8}{mj}九\end{CJK}. \begin{CJK}{UTF8}{mj}设\end{CJK} $f(x)$ \begin{CJK}{UTF8}{mj}在\end{CJK} $[a,+\infty)$ \begin{CJK}{UTF8}{mj}上连续\end{CJK}, \begin{CJK}{UTF8}{mj}且\end{CJK}
$$
\lim _{x \rightarrow+\infty}[f(x)-c x-d]=0
$$
$c, d$ \begin{CJK}{UTF8}{mj}为常数\end{CJK}, \begin{CJK}{UTF8}{mj}证明\end{CJK} $f(x)$ \begin{CJK}{UTF8}{mj}在\end{CJK} $[a,+\infty)$ \begin{CJK}{UTF8}{mj}上一致连续\end{CJK}.

\section{6. 天津大学 2015 年研究生入学考试试题数学分析}
\begin{CJK}{UTF8}{mj}李扬\end{CJK}

\begin{CJK}{UTF8}{mj}微信公众号\end{CJK}: sxkyliyang

\begin{CJK}{UTF8}{mj}一\end{CJK}. \begin{CJK}{UTF8}{mj}填空题\end{CJK}.

\begin{enumerate}
  \item $\lim _{x \rightarrow+\infty}\left[x-x^{2} \ln \left(1+\frac{1}{x}\right)\right]=$

  \item $\int(\tan x+1)^{2} \mathrm{e}^{2 x} \mathrm{~d} x=$

  \item $\int_{-\infty}^{+\infty} \mathrm{e}^{k|x|}=1$, \begin{CJK}{UTF8}{mj}求\end{CJK} $k=$

  \item $\int_{0}^{x} t f\left(x^{2}-t^{2}\right) \mathrm{d} t=$

  \item $\lim _{n \rightarrow \infty}\left(\frac{1}{n^{2}+1^{2}}+\frac{2}{n^{2}+2^{2}}+\cdots+\frac{n}{n^{2}+n^{2}}\right)=$

  \item \begin{CJK}{UTF8}{mj}求\end{CJK} $\sum_{n=1}^{\infty}\left(1+\frac{1}{2}+\cdots+\frac{1}{n}\right) x^{n}$ \begin{CJK}{UTF8}{mj}收敛域\end{CJK}

  \item \begin{CJK}{UTF8}{mj}向量场的一道题\end{CJK}, \begin{CJK}{UTF8}{mj}几种场的定义看一下\end{CJK}.

  \item \begin{CJK}{UTF8}{mj}已知\end{CJK} $x=0, y=0, x+y=1$, \begin{CJK}{UTF8}{mj}求\end{CJK} $\iint_{D}\left(\frac{x-y}{x+y}\right)^{2} \mathrm{~d} x \mathrm{~d} y=$

\end{enumerate}
\begin{CJK}{UTF8}{mj}二\end{CJK}. \begin{CJK}{UTF8}{mj}计算与证明题\end{CJK}.

\begin{enumerate}
  \item \begin{CJK}{UTF8}{mj}已知\end{CJK} $f(x, y)$ \begin{CJK}{UTF8}{mj}有二阶偏导\end{CJK}, $D=\left\{(x, y) \mid x^{2}+y^{2} \leqslant 1\right\}, f_{x x}^{\prime \prime}+f_{y y}^{\prime \prime}=\mathrm{e}^{-\left(x^{2}+y^{2}\right)}$, \begin{CJK}{UTF8}{mj}计算\end{CJK}
\end{enumerate}
$$
I=\iint_{D}\left(x f_{x}^{\prime}+y f_{y}^{\prime}\right) \mathrm{d} x \mathrm{~d} y
$$

\begin{enumerate}
  \setcounter{enumi}{2}
  \item $f(x)$ \begin{CJK}{UTF8}{mj}连续且不为常数\end{CJK}, \begin{CJK}{UTF8}{mj}证存在\end{CJK} $\xi \in(a, b)$, \begin{CJK}{UTF8}{mj}使\end{CJK} $f(x)$ \begin{CJK}{UTF8}{mj}在\end{CJK} $(a, b)$ \begin{CJK}{UTF8}{mj}内不取极值\end{CJK}.

  \item \begin{CJK}{UTF8}{mj}已知\end{CJK}

\end{enumerate}
$$
f(x)=-\frac{1}{2}\left(1+\mathrm{e}^{-1}\right)+\int_{-1}^{1}|x-t| \mathrm{e}^{-t^{2}} \mathrm{~d} t
$$
\begin{CJK}{UTF8}{mj}讨论\end{CJK} $f(x)=0$ \begin{CJK}{UTF8}{mj}在\end{CJK} $[-1,1]$ \begin{CJK}{UTF8}{mj}上的实根个数\end{CJK}.

\begin{enumerate}
  \setcounter{enumi}{4}
  \item \begin{CJK}{UTF8}{mj}讨论\end{CJK}
\end{enumerate}
$$
\int_{1}^{+\infty} \frac{\mathrm{d} x}{x(1+\ln x)}
$$
\begin{CJK}{UTF8}{mj}的敛散性\end{CJK}.

\begin{enumerate}
  \setcounter{enumi}{5}
  \item \begin{CJK}{UTF8}{mj}求由曲面\end{CJK} $x^{2}+y^{2}=z$ \begin{CJK}{UTF8}{mj}与\end{CJK} $z=2-\sqrt{x^{2}+y^{2}}$ \begin{CJK}{UTF8}{mj}所围成的立体表面积\end{CJK}.

  \item $z=z(x, y)$ \begin{CJK}{UTF8}{mj}是由\end{CJK}

\end{enumerate}
$$
(x+y)^{2}+(y+z)^{2}+(z+x)^{2}=3
$$
\begin{CJK}{UTF8}{mj}所确定的隐函数\end{CJK}, \begin{CJK}{UTF8}{mj}讨论\end{CJK} $z=z(x, y)$ \begin{CJK}{UTF8}{mj}的极值\end{CJK}.

\begin{enumerate}
  \setcounter{enumi}{7}
  \item \begin{CJK}{UTF8}{mj}已知\end{CJK} $z=f\left(\sqrt{x^{2}+y^{2}}\right), \frac{\partial^{2} z}{\partial x^{2}}+\frac{\partial^{2} z}{\partial y^{2}}=0$, \begin{CJK}{UTF8}{mj}求证\end{CJK}
\end{enumerate}
$$
f^{\prime \prime}(u)+\frac{f^{\prime}(u)}{u}=0
$$

\begin{enumerate}
  \setcounter{enumi}{8}
  \item \begin{CJK}{UTF8}{mj}已知\end{CJK} $f^{\prime \prime}\left(x_{0}\right)=\cdots=f^{(n-1)}\left(x_{0}\right)=0, f^{(n)}\left(x_{0}\right) \neq 0$, \begin{CJK}{UTF8}{mj}当\end{CJK} $0<|h|<\delta$ \begin{CJK}{UTF8}{mj}时\end{CJK}, \begin{CJK}{UTF8}{mj}有\end{CJK} $f\left(x_{0}+h\right)-f\left(x_{0}\right)=h f^{\prime}\left(x_{0}+\theta(h)\right)$, \begin{CJK}{UTF8}{mj}证\end{CJK}
\end{enumerate}
$$
\lim _{h \rightarrow 0} \theta=\sqrt[n-1]{\frac{1}{n}}
$$

\begin{enumerate}
  \setcounter{enumi}{9}
  \item \begin{CJK}{UTF8}{mj}在\end{CJK} $[0,1]$ \begin{CJK}{UTF8}{mj}上有\end{CJK} $f(x)=\sum_{n=1}^{\infty} \frac{x^{n}}{n^{2}}$, \begin{CJK}{UTF8}{mj}证\end{CJK}
\end{enumerate}
$$
f(x)+f(1-x)+\ln x \ln (1-x)=\frac{\pi}{6} .
$$

\section{7. 天津大学 2016 年研究生入学考试试题数学分析}
\begin{CJK}{UTF8}{mj}李扬\end{CJK}

\begin{CJK}{UTF8}{mj}微信公众号\end{CJK}: sxkyliyang

\begin{CJK}{UTF8}{mj}题型\end{CJK}: 10 \begin{CJK}{UTF8}{mj}个填空题\end{CJK}, \begin{CJK}{UTF8}{mj}共\end{CJK} 50 \begin{CJK}{UTF8}{mj}分\end{CJK}, 10 \begin{CJK}{UTF8}{mj}个左右的大题\end{CJK}.

\begin{CJK}{UTF8}{mj}一\end{CJK}. \begin{CJK}{UTF8}{mj}填空题\end{CJK}.

\begin{enumerate}
  \item $z=f\left(x^{2}+y^{2}, x+y\right), \frac{\partial^{2} z}{\partial x \partial y}=$

  \item $L: x^{2}+y^{2}=a^{2}, \int_{L} x \mathrm{~d} S=$

  \item $D:|x|+|y| \leqslant 1, \iint_{D}(x+y) \mathrm{d} x \mathrm{~d} y=$

  \item $\int_{0}^{+\infty} x^{4} \mathrm{e}^{-x^{2}} \mathrm{~d} x=$

  \item \begin{CJK}{UTF8}{mj}旋度场的题\end{CJK} (\begin{CJK}{UTF8}{mj}记住梯度\end{CJK}, \begin{CJK}{UTF8}{mj}散度\end{CJK}, \begin{CJK}{UTF8}{mj}旋转场的这几个公式即可\end{CJK}).

\end{enumerate}
\begin{CJK}{UTF8}{mj}二\end{CJK}. 1. \begin{CJK}{UTF8}{mj}设\end{CJK}
$$
f(x, y)= \begin{cases}\frac{x^{2} y^{2}}{\left(x^{2}+y^{2}\right)^{\frac{3}{2}}}, & x^{2}+y^{2} \neq 0 \\ 0, & x^{2}+y^{2}=0 .\end{cases}
$$
\begin{CJK}{UTF8}{mj}求证\end{CJK}: \begin{CJK}{UTF8}{mj}在\end{CJK} $(0,0)$ \begin{CJK}{UTF8}{mj}处\end{CJK} $f(x, y)$ \begin{CJK}{UTF8}{mj}连续但不可微\end{CJK} (\begin{CJK}{UTF8}{mj}题型一样\end{CJK}, \begin{CJK}{UTF8}{mj}类似\end{CJK}).

\begin{enumerate}
  \setcounter{enumi}{2}
  \item \begin{CJK}{UTF8}{mj}已知\end{CJK} $A=\left(a_{i j}\right)_{n \times n}$ \begin{CJK}{UTF8}{mj}的特征值\end{CJK} $\lambda_{n} \geqslant \lambda_{n-1} \geqslant \cdot \cdot \lambda_{1}$, \begin{CJK}{UTF8}{mj}求\end{CJK}
\end{enumerate}
\includegraphics[max width=\textwidth]{2022_04_18_a5c47c0ff534501b502eg-256}

\begin{CJK}{UTF8}{mj}在\end{CJK} $\sum_{j=1}^{n}\left|x_{j}\right| \leqslant 1$ \begin{CJK}{UTF8}{mj}的条件下的最大值和最小值\end{CJK}.

\begin{enumerate}
  \setcounter{enumi}{3}
  \item \begin{CJK}{UTF8}{mj}用闭区间套定理证明有界数列必有收敛子列\end{CJK}.

  \item $f(x)$ \begin{CJK}{UTF8}{mj}在\end{CJK} $[0,3]$ \begin{CJK}{UTF8}{mj}上连续\end{CJK}, $(0,3)$ \begin{CJK}{UTF8}{mj}内可导\end{CJK}, $f(0)+f(1)+f(2)=3, f(3)=1$, \begin{CJK}{UTF8}{mj}证明\end{CJK} $\exists \xi \in(0,3)$, \begin{CJK}{UTF8}{mj}使得\end{CJK}

\end{enumerate}
$$
f^{\prime}(\xi)=0 .
$$

\begin{enumerate}
  \setcounter{enumi}{5}
  \item (\begin{CJK}{UTF8}{mj}有问题\end{CJK}) $f(x)$ \begin{CJK}{UTF8}{mj}在\end{CJK} $[0,1]$ \begin{CJK}{UTF8}{mj}上连续\end{CJK}, $f(1)=0$, \begin{CJK}{UTF8}{mj}证明\end{CJK} $f(x)$ \begin{CJK}{UTF8}{mj}在\end{CJK} $(0,1)$ \begin{CJK}{UTF8}{mj}内一致收敛\end{CJK}.

  \item \begin{CJK}{UTF8}{mj}证明\end{CJK}

\end{enumerate}
$$
\int_{0}^{+\infty} \frac{\sqrt{x} \cos x}{x+2015} \mathrm{~d} x
$$
\begin{CJK}{UTF8}{mj}的敛散性\end{CJK}.

\begin{enumerate}
  \setcounter{enumi}{7}
  \item \begin{CJK}{UTF8}{mj}已知\end{CJK} $f(x)$ \begin{CJK}{UTF8}{mj}为连续函数\end{CJK}, \begin{CJK}{UTF8}{mj}证明极限\end{CJK} $\frac{2}{\pi} \lim _{n \rightarrow \infty} \int_{0}^{1} \frac{n}{n^{2} x^{2}+1} f(x) \mathrm{d} x=f(0)$.
\end{enumerate}
\section{8. 天津大学 2017 年研究生入学考试试题数学分析}
\begin{CJK}{UTF8}{mj}李扬\end{CJK}

\begin{CJK}{UTF8}{mj}微信公众号\end{CJK}: sxkyliyang

-. \begin{CJK}{UTF8}{mj}填空题\end{CJK}.

\begin{enumerate}
  \item $\int_{-1}^{1} \mathrm{~d} x \int_{|x|}^{1} \mathrm{e}^{y^{2}} \mathrm{~d} y=$

  \item $\alpha \in(0,1)$, \begin{CJK}{UTF8}{mj}利用贝塔函数表示\end{CJK} $\int_{0}^{\frac{\pi}{2}} \tan ^{2} x \mathrm{~d} x$,

  \item $p>0, \lim _{n \rightarrow+\infty} n^{p} a_{n} \ln \left(1+\frac{1}{n}\right)=1, \sum_{n=1}^{+\infty} a_{n}$ \begin{CJK}{UTF8}{mj}收敛\end{CJK}, \begin{CJK}{UTF8}{mj}求\end{CJK} $p$ \begin{CJK}{UTF8}{mj}的取值范围\end{CJK}

  \item \begin{CJK}{UTF8}{mj}设\end{CJK} $z=x y z$, \begin{CJK}{UTF8}{mj}求\end{CJK} $z$ \begin{CJK}{UTF8}{mj}在\end{CJK} $(1,-1,1)$ \begin{CJK}{UTF8}{mj}处向上的单位法向量\end{CJK}

  \item \begin{CJK}{UTF8}{mj}简单的求旋度场的题\end{CJK} (\begin{CJK}{UTF8}{mj}利用公式\end{CJK}).

\end{enumerate}
\begin{CJK}{UTF8}{mj}二\end{CJK}. $f(x)=x+1, x \in(0, \pi)$ \begin{CJK}{UTF8}{mj}展成正弦函数\end{CJK}.

\begin{CJK}{UTF8}{mj}三\end{CJK}. \begin{CJK}{UTF8}{mj}计算\end{CJK}
$$
\iint_{D} \frac{1}{x y} \mathrm{~d} x \mathrm{~d} y,
$$
$D$ \begin{CJK}{UTF8}{mj}为\end{CJK} $x y=1, x y=4, \frac{y}{x}=1, \frac{y}{x}=2$ \begin{CJK}{UTF8}{mj}围成的区域\end{CJK}.

\begin{CJK}{UTF8}{mj}四\end{CJK}. $f(x)$ \begin{CJK}{UTF8}{mj}在\end{CJK} $[0,1]$ \begin{CJK}{UTF8}{mj}上连续\end{CJK}, \begin{CJK}{UTF8}{mj}在\end{CJK} $(0,1)$ \begin{CJK}{UTF8}{mj}上可导\end{CJK}, $f(0)=f(1)=0$ \begin{CJK}{UTF8}{mj}且\end{CJK} $\left|f^{\prime \prime}(x)\right| \leqslant 1$. \begin{CJK}{UTF8}{mj}证明\end{CJK}: \begin{CJK}{UTF8}{mj}对\end{CJK} $\forall x \in[0,1],|f(x)| \leqslant$ $\frac{1}{4},\left|f^{\prime}(x)\right| \leqslant 1 .$

\begin{CJK}{UTF8}{mj}五\end{CJK}. $f(x)$ \begin{CJK}{UTF8}{mj}在\end{CJK} $[a, b]$ \begin{CJK}{UTF8}{mj}上有定义\end{CJK}, \begin{CJK}{UTF8}{mj}证明\end{CJK}: $f(x)$ \begin{CJK}{UTF8}{mj}在\end{CJK} $(a, b)$ \begin{CJK}{UTF8}{mj}上一致连续当且仅当存在\end{CJK} $[a, b]$ \begin{CJK}{UTF8}{mj}上的连续函数\end{CJK} $F(x)$, \begin{CJK}{UTF8}{mj}当\end{CJK} $x \in(a, b)$ \begin{CJK}{UTF8}{mj}时有\end{CJK}
$$
f(x)=F(x) .
$$
\begin{CJK}{UTF8}{mj}六\end{CJK}. \begin{CJK}{UTF8}{mj}设\end{CJK} $f(x)$ \begin{CJK}{UTF8}{mj}在\end{CJK} $[a, b]$ \begin{CJK}{UTF8}{mj}上连续\end{CJK}, \begin{CJK}{UTF8}{mj}证明\end{CJK}: \begin{CJK}{UTF8}{mj}对\end{CJK} $\forall n \in \mathbb{N}_{+}$, \begin{CJK}{UTF8}{mj}总存在\end{CJK} $x_{n} \in(a, b)$, \begin{CJK}{UTF8}{mj}使得\end{CJK}
$$
f\left(x_{n}\right)=f\left(x_{n}+\frac{b-a}{n}\right) .
$$
\begin{CJK}{UTF8}{mj}七\end{CJK}. \begin{CJK}{UTF8}{mj}证明\end{CJK}: \begin{CJK}{UTF8}{mj}若二元函数\end{CJK} $u(x, y)$ \begin{CJK}{UTF8}{mj}满足\end{CJK} $y \frac{\partial u}{\partial x}+x \frac{\partial u}{\partial y}=0$, \begin{CJK}{UTF8}{mj}存在一元函数\end{CJK} $f$, \begin{CJK}{UTF8}{mj}使得\end{CJK}
$$
u(x, y)=f\left(x^{2}-y^{2}\right)
$$
\begin{CJK}{UTF8}{mj}八\end{CJK}. \begin{CJK}{UTF8}{mj}证明\end{CJK}: $f(x)$ \begin{CJK}{UTF8}{mj}在\end{CJK} $[a, b]$ \begin{CJK}{UTF8}{mj}上连续\end{CJK}, $(a, b)$ \begin{CJK}{UTF8}{mj}内可导\end{CJK}, $a<c<b$, \begin{CJK}{UTF8}{mj}则存在\end{CJK} $\xi \in(a, b)$, \begin{CJK}{UTF8}{mj}使得\end{CJK}
$$
\operatorname{det}\left(\begin{array}{ccc}
1 & 1 & 1 \\
a & b & c \\
f(a) & f(b) & f(c)
\end{array}\right)=\frac{1}{2} f^{\prime \prime}(\xi)(c-a)(b-a)(c-b)
$$
\begin{CJK}{UTF8}{mj}九\end{CJK}. \begin{CJK}{UTF8}{mj}证明数列收玫的题目\end{CJK}, \begin{CJK}{UTF8}{mj}用到\end{CJK} stolz \begin{CJK}{UTF8}{mj}公式\end{CJK}.

\section{1. 武汉大学 2009 年研究生入学考试试题数学分析}
\begin{CJK}{UTF8}{mj}李扬\end{CJK}

\begin{CJK}{UTF8}{mj}微信公众号\end{CJK}: sxkyliyang

\begin{CJK}{UTF8}{mj}一\end{CJK}、\begin{CJK}{UTF8}{mj}计算题\end{CJK}(\begin{CJK}{UTF8}{mj}本题共\end{CJK} 5 \begin{CJK}{UTF8}{mj}小题\end{CJK}, \begin{CJK}{UTF8}{mj}每小题\end{CJK} 8 \begin{CJK}{UTF8}{mj}分\end{CJK}, \begin{CJK}{UTF8}{mj}共\end{CJK} 40 \begin{CJK}{UTF8}{mj}分\end{CJK})

$1 .$
$$
\lim _{n \rightarrow \infty}\left(\frac{1}{1+2}+\frac{1}{1+2+3}+\cdots+\frac{1}{1+2+\cdots+n}\right)
$$
$2 .$
$$
\lim _{x \rightarrow 0} \frac{\int_{0}^{x}(x-t) \sin t^{2} \mathrm{~d} t}{x \int_{0}^{x} \sin t^{2} \mathrm{~d} t}
$$

\begin{enumerate}
  \setcounter{enumi}{3}
  \item \begin{CJK}{UTF8}{mj}设\end{CJK}
\end{enumerate}
$$
F(x)=\frac{1}{x} \int_{0}^{x} \frac{\sin t}{t} \mathrm{~d} t
$$
\begin{CJK}{UTF8}{mj}求\end{CJK} $F^{(4)}(0), F^{(9)}(0)$;

\begin{enumerate}
  \setcounter{enumi}{4}
  \item \begin{CJK}{UTF8}{mj}设\end{CJK} $x y z e^{x+y+z}=1$, \begin{CJK}{UTF8}{mj}求\end{CJK} $\frac{\partial z}{\partial x}, \frac{\partial z}{\partial y}, \frac{\partial^{2} z}{\partial x^{2}}, \frac{\partial^{2} z}{\partial x \partial y}$;

  \item \begin{CJK}{UTF8}{mj}求积分\end{CJK}

\end{enumerate}
$$
\iint_{D} \ln \frac{x^{3}}{y} \mathrm{~d} x \mathrm{~d} y
$$
\begin{CJK}{UTF8}{mj}其中\end{CJK} $D$ \begin{CJK}{UTF8}{mj}是由\end{CJK} $y=x, y=1, x=2$ \begin{CJK}{UTF8}{mj}围成的三角形\end{CJK}.

\begin{CJK}{UTF8}{mj}二\end{CJK}、 (\begin{CJK}{UTF8}{mj}本题\end{CJK} 12 \begin{CJK}{UTF8}{mj}分\end{CJK})

\begin{CJK}{UTF8}{mj}设\end{CJK} $\left\{O_{\alpha}\right\}$ \begin{CJK}{UTF8}{mj}是有界闭区间\end{CJK} $[a, b]$ \begin{CJK}{UTF8}{mj}的一个开覆盖\end{CJK}.

\begin{enumerate}
  \item \begin{CJK}{UTF8}{mj}证明\end{CJK}: $\exists \delta>0, \forall x_{1}, x_{2} \in[a, b]$, \begin{CJK}{UTF8}{mj}只要\end{CJK} $\left|x_{1}-x_{2}\right|<\delta$, \begin{CJK}{UTF8}{mj}就存在\end{CJK} $O \in\left\{O_{\alpha}\right\}$, \begin{CJK}{UTF8}{mj}使得\end{CJK} $x_{1}, x_{2} \in O$;

  \item \begin{CJK}{UTF8}{mj}举例说明\end{CJK}: \begin{CJK}{UTF8}{mj}开区间的开覆盖可能没有这个性质\end{CJK}.

\end{enumerate}
\begin{CJK}{UTF8}{mj}三\end{CJK}、 (\begin{CJK}{UTF8}{mj}本题\end{CJK} 12 \begin{CJK}{UTF8}{mj}分\end{CJK})

\begin{CJK}{UTF8}{mj}设\end{CJK} $f(x)$ \begin{CJK}{UTF8}{mj}在\end{CJK} $(0, a)$ \begin{CJK}{UTF8}{mj}上可微\end{CJK}, \begin{CJK}{UTF8}{mj}且\end{CJK} $f(a-0)=+\infty$, \begin{CJK}{UTF8}{mj}求证\end{CJK}: $f^{\prime}(x)$ \begin{CJK}{UTF8}{mj}在\end{CJK} $x=a$ \begin{CJK}{UTF8}{mj}的右侧无上界\end{CJK}.

\begin{CJK}{UTF8}{mj}四\end{CJK}、(\begin{CJK}{UTF8}{mj}本题\end{CJK} 14 \begin{CJK}{UTF8}{mj}分\end{CJK})

\begin{CJK}{UTF8}{mj}设\end{CJK} $D: x^{2}+y^{2} \leqslant y, x \geqslant 0, f$ \begin{CJK}{UTF8}{mj}连续\end{CJK},
$$
f(x, y)=\sqrt{1-x^{2}-y^{2}}-\frac{8}{\pi} \iint_{D} f(x, y) \mathrm{d} \delta .
$$
\begin{CJK}{UTF8}{mj}求\end{CJK} $f(x, y)$.

\begin{CJK}{UTF8}{mj}五\end{CJK}、(\begin{CJK}{UTF8}{mj}本题\end{CJK} 14 \begin{CJK}{UTF8}{mj}分\end{CJK})

\begin{CJK}{UTF8}{mj}设\end{CJK} $f(x)$ \begin{CJK}{UTF8}{mj}是\end{CJK} $\left\{(x, y) \mid x^{2}+y^{2} \leqslant 1\right\}$ \begin{CJK}{UTF8}{mj}上二次连续可微函数\end{CJK}, \begin{CJK}{UTF8}{mj}且满足\end{CJK}
$$
\frac{\partial^{2} f}{\partial x^{2}}+\frac{\partial^{2} f}{\partial y^{2}}=\left(x^{2}+y^{2}\right)^{2}
$$
\begin{CJK}{UTF8}{mj}试求积分\end{CJK}
$$
\iint_{x^{2}+y^{2} \leqslant 1}\left(\frac{x}{\sqrt{x^{2}+y^{2}}} \frac{\partial f}{\partial x}+\frac{y}{\sqrt{x^{2}+y^{2}}} \frac{\partial f}{\partial y}\right) \mathrm{d} x \mathrm{~d} y
$$
\begin{CJK}{UTF8}{mj}六\end{CJK}、(\begin{CJK}{UTF8}{mj}本题\end{CJK} 14 \begin{CJK}{UTF8}{mj}分\end{CJK})

\begin{CJK}{UTF8}{mj}设\end{CJK} $\lim _{n \rightarrow \infty} a_{n}=+\infty$, \begin{CJK}{UTF8}{mj}证明\end{CJK}:
$$
\lim _{n \rightarrow \infty} \frac{1}{n} \sum_{k=1}^{n} a_{k}=+\infty
$$
\begin{CJK}{UTF8}{mj}七\end{CJK}、 (\begin{CJK}{UTF8}{mj}本题\end{CJK} 14 \begin{CJK}{UTF8}{mj}分\end{CJK})

\begin{CJK}{UTF8}{mj}设二元函数\end{CJK}
$$
f(x, y)= \begin{cases}\left(x^{2}+y^{2}\right) \sin \frac{1}{\sqrt{x^{2}+y^{2}}}, & x^{2}+y^{2} \neq 0 \\ 0, & x^{2}+y^{2}=0\end{cases}
$$

\begin{enumerate}
  \item \begin{CJK}{UTF8}{mj}求\end{CJK} $f_{x}(0,0), f_{y}(0,0)$;

  \item \begin{CJK}{UTF8}{mj}证明\end{CJK}: $f_{x}(0,0), f_{y}(0,0)$ \begin{CJK}{UTF8}{mj}在\end{CJK} $(0,0)$ \begin{CJK}{UTF8}{mj}不连续\end{CJK};

  \item \begin{CJK}{UTF8}{mj}证明\end{CJK}: $f(x, y)$ \begin{CJK}{UTF8}{mj}在\end{CJK} $(0,0)$ \begin{CJK}{UTF8}{mj}可微\end{CJK}, \begin{CJK}{UTF8}{mj}并求\end{CJK} $\mathrm{d} f(0,0)$.

\end{enumerate}
\begin{CJK}{UTF8}{mj}八\end{CJK}、 (\begin{CJK}{UTF8}{mj}本题\end{CJK} 15 \begin{CJK}{UTF8}{mj}分\end{CJK})

\begin{CJK}{UTF8}{mj}设\end{CJK} $z(x, y)$ \begin{CJK}{UTF8}{mj}连续二阶可微\end{CJK}, \begin{CJK}{UTF8}{mj}对微分方程\end{CJK}
$$
\frac{1}{\left(x^{2}+y^{2}\right)^{2}}\left(\frac{\partial^{2} z}{\partial x^{2}}+2 \frac{\partial^{2} z}{\partial x \partial y}+\frac{\partial^{2} z}{\partial y^{2}}\right)-\frac{1}{\left(x^{2}+y^{2}\right)^{3}}\left(\frac{\partial z}{\partial x}+\frac{\partial z}{\partial y}\right)=0
$$
\begin{CJK}{UTF8}{mj}作变量代换\end{CJK} $u=x y, v=x-y$.

\begin{enumerate}
  \item \begin{CJK}{UTF8}{mj}求代换后的方程\end{CJK};

  \item \begin{CJK}{UTF8}{mj}指出变量代换失效的点集\end{CJK}, \begin{CJK}{UTF8}{mj}并说明失效的理由\end{CJK}, \begin{CJK}{UTF8}{mj}代换在失效点集上会产生什么现象\end{CJK}.

\end{enumerate}
\begin{CJK}{UTF8}{mj}九\end{CJK}、 (\begin{CJK}{UTF8}{mj}本题\end{CJK} 15 \begin{CJK}{UTF8}{mj}分\end{CJK})

\begin{CJK}{UTF8}{mj}设\end{CJK} $u_{n}(x)=\frac{1}{n^{3}} \ln \left(1+n^{3} x\right), n=1,2, \cdots$, \begin{CJK}{UTF8}{mj}记\end{CJK} $S(x)=\sum_{n=1}^{\infty} u_{n}(x)$.

\begin{enumerate}
  \item \begin{CJK}{UTF8}{mj}求证\end{CJK}: $\sum_{n=1}^{\infty} u_{n}(x)$ \begin{CJK}{UTF8}{mj}在有界区间\end{CJK} $[0, b]$ \begin{CJK}{UTF8}{mj}上是一致收敛的\end{CJK}, \begin{CJK}{UTF8}{mj}在\end{CJK} $(0,+\infty)$ \begin{CJK}{UTF8}{mj}上不是一致收敛的\end{CJK};

  \item \begin{CJK}{UTF8}{mj}讨论\end{CJK} $S(x)$ \begin{CJK}{UTF8}{mj}的可微性\end{CJK}.

\end{enumerate}
\section{2. 武汉大学 2010 年研究生入学考试试题数学分析}
\begin{CJK}{UTF8}{mj}李扬\end{CJK}

\begin{CJK}{UTF8}{mj}微信公众号\end{CJK}: sxkyliyang

\begin{CJK}{UTF8}{mj}一\end{CJK}、 \begin{CJK}{UTF8}{mj}计算题\end{CJK}(\begin{CJK}{UTF8}{mj}本题共\end{CJK} 5 \begin{CJK}{UTF8}{mj}小题\end{CJK}, \begin{CJK}{UTF8}{mj}每小题\end{CJK} 10 \begin{CJK}{UTF8}{mj}分\end{CJK}, \begin{CJK}{UTF8}{mj}共\end{CJK} 50 \begin{CJK}{UTF8}{mj}分\end{CJK})

\begin{enumerate}
  \item \begin{CJK}{UTF8}{mj}计算极限\end{CJK}:
\end{enumerate}
$$
\lim _{x \rightarrow 0} \frac{\ln (1+x)^{\frac{1}{x}}-1}{x}
$$

\begin{enumerate}
  \setcounter{enumi}{2}
  \item \begin{CJK}{UTF8}{mj}计算极限\end{CJK}:
\end{enumerate}
$$
\lim _{n \rightarrow \infty}\left(\frac{2^{\frac{1}{n}}}{n+\frac{1}{1}}+\frac{2^{\frac{2}{n}}}{n+\frac{1}{2}}+\cdots+\frac{2^{\frac{n}{n}}}{n+\frac{1}{n}}\right) ;
$$

\begin{enumerate}
  \setcounter{enumi}{3}
  \item \begin{CJK}{UTF8}{mj}计算不定积分\end{CJK}:
\end{enumerate}
$$
\int \frac{\mathrm{d} x}{1+\tan x}
$$

\begin{enumerate}
  \setcounter{enumi}{4}
  \item \begin{CJK}{UTF8}{mj}计算\end{CJK} $F^{\prime}(\alpha)$, \begin{CJK}{UTF8}{mj}其中\end{CJK}:
\end{enumerate}
$$
F(\alpha)=\int_{0}^{e^{\alpha}} \mathrm{d} x \int_{x-2 \alpha}^{x+3 \alpha} \cos \left(x^{2}+y^{2}+z^{2}\right) \mathrm{d} y
$$

\begin{enumerate}
  \setcounter{enumi}{5}
  \item \begin{CJK}{UTF8}{mj}计算三重积分\end{CJK}
\end{enumerate}
$$
\iiint_{V} e^{x} y^{2} z^{3} \mathrm{~d} x \mathrm{~d} y \mathrm{~d} z
$$
\begin{CJK}{UTF8}{mj}其中\end{CJK} $V$ \begin{CJK}{UTF8}{mj}是由曲面\end{CJK} $z=x y, y=x, z=0, x=1$ \begin{CJK}{UTF8}{mj}所围成\end{CJK}.

\begin{CJK}{UTF8}{mj}二\end{CJK}、 $($ \begin{CJK}{UTF8}{mj}本题\end{CJK} 14 \begin{CJK}{UTF8}{mj}分\end{CJK})

\begin{CJK}{UTF8}{mj}设\end{CJK} $a>0, x_{1}=\sqrt{a}, x_{n+1}=\sqrt{a+x_{n}}, n=1,2, \cdots$. \begin{CJK}{UTF8}{mj}证明\end{CJK}: $\left\{x_{n}\right\}$ \begin{CJK}{UTF8}{mj}收敛\end{CJK}, \begin{CJK}{UTF8}{mj}并求\end{CJK} $\lim _{n \rightarrow+\infty} x_{n}$.

\begin{CJK}{UTF8}{mj}三\end{CJK}、 (\begin{CJK}{UTF8}{mj}本题\end{CJK} 14 \begin{CJK}{UTF8}{mj}分\end{CJK})

\begin{CJK}{UTF8}{mj}设\end{CJK} $f(x)$ \begin{CJK}{UTF8}{mj}在区间\end{CJK} $[0,2]$ \begin{CJK}{UTF8}{mj}上可微\end{CJK}, \begin{CJK}{UTF8}{mj}且\end{CJK} $f(2)=\int_{0}^{\frac{1}{2}} x f(x) \mathrm{d} x$, \begin{CJK}{UTF8}{mj}求证\end{CJK}: \begin{CJK}{UTF8}{mj}存在\end{CJK} $\xi \in(0,2)$, \begin{CJK}{UTF8}{mj}使得\end{CJK} $f(\xi)+\xi f^{\prime}(\xi)=0$.

\begin{CJK}{UTF8}{mj}四\end{CJK}、(\begin{CJK}{UTF8}{mj}本题\end{CJK} 14 \begin{CJK}{UTF8}{mj}分\end{CJK})

\begin{CJK}{UTF8}{mj}设\end{CJK} $v=v(x, y)$ \begin{CJK}{UTF8}{mj}有连续的一阶偏导数\end{CJK}, $u(x, y)=x v+y \varphi(v)+\phi(v)$, \begin{CJK}{UTF8}{mj}其中\end{CJK} $\varphi, \phi$ \begin{CJK}{UTF8}{mj}可微\end{CJK}, \begin{CJK}{UTF8}{mj}且\end{CJK} $x+y \varphi^{\prime}(v)+\phi^{\prime}(v)=0$. \begin{CJK}{UTF8}{mj}证明\end{CJK}:
$$
\frac{\partial^{2} u}{\partial x^{2}} \cdot \frac{\partial^{2} u}{\partial y^{2}}-\left(\frac{\partial^{2} u}{\partial x \partial y}\right)^{2}=0
$$
\begin{CJK}{UTF8}{mj}五\end{CJK}、(\begin{CJK}{UTF8}{mj}本题\end{CJK} 14 \begin{CJK}{UTF8}{mj}分\end{CJK})

\begin{CJK}{UTF8}{mj}求曲面\end{CJK} $x^{2}+y^{2}=a z$ \begin{CJK}{UTF8}{mj}和\end{CJK} $z=2 a-\sqrt{x^{2}+y^{2}}(a>0)$ \begin{CJK}{UTF8}{mj}所围立体的表面积\end{CJK}.

\begin{CJK}{UTF8}{mj}六\end{CJK}、(\begin{CJK}{UTF8}{mj}本题\end{CJK} 14 \begin{CJK}{UTF8}{mj}分\end{CJK})

\begin{CJK}{UTF8}{mj}求证\end{CJK}: \begin{CJK}{UTF8}{mj}级数\end{CJK}
$$
\sum_{n=1}^{\infty} \frac{\ln (1+n x)}{n x^{n}}
$$

\begin{enumerate}
  \item \begin{CJK}{UTF8}{mj}在区间\end{CJK} $[1+a,+\infty]$ \begin{CJK}{UTF8}{mj}上一致收敛\end{CJK} (\begin{CJK}{UTF8}{mj}这里\end{CJK} $a>0)$;

  \item \begin{CJK}{UTF8}{mj}在\end{CJK} $[1,+\infty)$ \begin{CJK}{UTF8}{mj}内连续\end{CJK}. \begin{CJK}{UTF8}{mj}七\end{CJK}、 (\begin{CJK}{UTF8}{mj}本题\end{CJK} 14 \begin{CJK}{UTF8}{mj}分\end{CJK})

\end{enumerate}
\begin{CJK}{UTF8}{mj}设\end{CJK}
$$
f(y)=\int_{0}^{+\infty} e^{-x^{2}} \cos x y \mathrm{~d} x
$$

\begin{enumerate}
  \item \begin{CJK}{UTF8}{mj}求\end{CJK} $f(y)$ \begin{CJK}{UTF8}{mj}的定义域\end{CJK};

  \item \begin{CJK}{UTF8}{mj}求证\end{CJK} $f(y)$ \begin{CJK}{UTF8}{mj}有任意阶连续的导数\end{CJK};

  \item \begin{CJK}{UTF8}{mj}求\end{CJK} $f(y)$.

\end{enumerate}
\begin{CJK}{UTF8}{mj}八\end{CJK}、 (\begin{CJK}{UTF8}{mj}本题\end{CJK} 14 \begin{CJK}{UTF8}{mj}分\end{CJK})

\begin{CJK}{UTF8}{mj}设\end{CJK} $\Sigma$ \begin{CJK}{UTF8}{mj}为\end{CJK}
$$
1-\frac{z}{5}=\frac{(x-3)^{2}}{16}+\frac{(y-2)^{2}}{9}(z \geqslant 0)
$$
\begin{CJK}{UTF8}{mj}的上侧\end{CJK}, \begin{CJK}{UTF8}{mj}求证\end{CJK}:
$$
\iint_{\Sigma} \frac{x \mathrm{~d} y \mathrm{~d} z+y \mathrm{~d} z \mathrm{~d} x+z \mathrm{~d} x \mathrm{~d} y}{\sqrt{\left(x^{2}+y^{2}+z^{2}\right)^{3}}}=2 \pi .
$$
\begin{CJK}{UTF8}{mj}九\end{CJK}、 (\begin{CJK}{UTF8}{mj}本题\end{CJK} 12 \begin{CJK}{UTF8}{mj}分\end{CJK})

\begin{enumerate}
  \item \begin{CJK}{UTF8}{mj}证明\end{CJK}: $f(x)=\sqrt{x}$ \begin{CJK}{UTF8}{mj}在\end{CJK} $(0,+\infty)$ \begin{CJK}{UTF8}{mj}上一致连续\end{CJK};

  \item \begin{CJK}{UTF8}{mj}讨论\end{CJK} $f(x)=\sqrt{x}$ \begin{CJK}{UTF8}{mj}在\end{CJK} $(0,+\infty)$ \begin{CJK}{UTF8}{mj}上是否\end{CJK} Lipschitz \begin{CJK}{UTF8}{mj}连续\end{CJK}, \begin{CJK}{UTF8}{mj}即存在常数\end{CJK} $L>0$, \begin{CJK}{UTF8}{mj}使得\end{CJK}

\end{enumerate}
$$
\left|f\left(x_{2}\right)-f\left(x_{1}\right)\right| \leqslant L\left|x_{2}-x_{1}\right|, \forall x_{1}, x_{2} \in(0,+\infty)
$$

\section{3. 武汉大学 2011 年研究生入学考试试题数学分析}
\begin{CJK}{UTF8}{mj}李扬\end{CJK}

\begin{CJK}{UTF8}{mj}微信公众号\end{CJK}: sxkyliyang

\begin{CJK}{UTF8}{mj}一\end{CJK}、 \begin{CJK}{UTF8}{mj}计算题\end{CJK}(\begin{CJK}{UTF8}{mj}本题共\end{CJK} 5 \begin{CJK}{UTF8}{mj}小题\end{CJK}, \begin{CJK}{UTF8}{mj}每小题\end{CJK} 10 \begin{CJK}{UTF8}{mj}分\end{CJK}, \begin{CJK}{UTF8}{mj}共\end{CJK} 50 \begin{CJK}{UTF8}{mj}分\end{CJK})

\begin{enumerate}
  \item \begin{CJK}{UTF8}{mj}计算极限\end{CJK}:
\end{enumerate}
$$
\lim _{n \rightarrow \infty} \frac{\sqrt[n]{n !}}{n^{\alpha}}(\alpha>0)
$$

\begin{enumerate}
  \setcounter{enumi}{2}
  \item \begin{CJK}{UTF8}{mj}计算极限\end{CJK}:
\end{enumerate}
$$
\lim _{x \rightarrow+\infty} \frac{1-\cos \sqrt{\tan x-\sin x}}{\sqrt[3]{1+x^{3}}-\sqrt[3]{1-x^{3}}}
$$

\begin{enumerate}
  \setcounter{enumi}{3}
  \item \begin{CJK}{UTF8}{mj}计算不定积分\end{CJK}:
\end{enumerate}
$$
\int \sqrt{1+\cos x} \mathrm{~d} x
$$

\begin{enumerate}
  \setcounter{enumi}{4}
  \item \begin{CJK}{UTF8}{mj}计算\end{CJK} $F_{x}^{\prime \prime}(x, y), f(x)$ \begin{CJK}{UTF8}{mj}为可微函数\end{CJK}, \begin{CJK}{UTF8}{mj}其中\end{CJK}:
\end{enumerate}
$$
F(x, y)=\int_{\frac{y}{x}}^{x y}(x z-y) f(z) \mathrm{d} z
$$

\begin{enumerate}
  \setcounter{enumi}{5}
  \item \begin{CJK}{UTF8}{mj}计算二重积分\end{CJK}
\end{enumerate}
$$
\iint_{D}\left|x-y^{2}\right| \mathrm{d} x \mathrm{~d} y
$$
\begin{CJK}{UTF8}{mj}其中\end{CJK} $D:\{(x, y) \mid-1 \leqslant x \leqslant 1,0 \leqslant y \leqslant 1\}$.

\begin{CJK}{UTF8}{mj}二\end{CJK}、 (\begin{CJK}{UTF8}{mj}本题\end{CJK} 12 \begin{CJK}{UTF8}{mj}分\end{CJK})

\begin{CJK}{UTF8}{mj}已知\end{CJK} $f(x), g(x)$ \begin{CJK}{UTF8}{mj}在\end{CJK} $[a, b]$ \begin{CJK}{UTF8}{mj}上连续\end{CJK}, \begin{CJK}{UTF8}{mj}在\end{CJK} $(a, b)$ \begin{CJK}{UTF8}{mj}内可微\end{CJK}, \begin{CJK}{UTF8}{mj}且\end{CJK} $g^{\prime}(x)$ \begin{CJK}{UTF8}{mj}在\end{CJK} $(a, b)$ \begin{CJK}{UTF8}{mj}上无零点\end{CJK}, \begin{CJK}{UTF8}{mj}证明\end{CJK}: $\exists \xi \in(a, b)$, \begin{CJK}{UTF8}{mj}使得\end{CJK}:
$$
\frac{f^{\prime}(\xi)}{g^{\prime}(\xi)}=\frac{f(b)-f(\xi)}{g(\xi)-g(a)}
$$
\begin{CJK}{UTF8}{mj}三\end{CJK}、 (\begin{CJK}{UTF8}{mj}本题\end{CJK} 14 \begin{CJK}{UTF8}{mj}分\end{CJK})

\begin{CJK}{UTF8}{mj}已知数列\end{CJK} $\left\{a_{n}\right\}$ \begin{CJK}{UTF8}{mj}非负单调递增\end{CJK}, \begin{CJK}{UTF8}{mj}且\end{CJK} $\lim _{n \rightarrow \infty} b_{n}=b$, \begin{CJK}{UTF8}{mj}证明\end{CJK}:
$$
\lim _{n \rightarrow \infty} \frac{a_{1} b_{n}+a_{2} b_{n-1}+\cdots+a_{n} b_{1}}{a_{1}+a_{2}+\cdots+a_{n}}=b .
$$
\begin{CJK}{UTF8}{mj}四\end{CJK}、(\begin{CJK}{UTF8}{mj}本题\end{CJK} 14 \begin{CJK}{UTF8}{mj}分\end{CJK})

\begin{CJK}{UTF8}{mj}已知\end{CJK} $f(x)$ \begin{CJK}{UTF8}{mj}为\end{CJK} $[-\pi, \pi]$ \begin{CJK}{UTF8}{mj}上的凸函数\end{CJK}, $f^{\prime}(x)$ \begin{CJK}{UTF8}{mj}有界\end{CJK}, \begin{CJK}{UTF8}{mj}证明\end{CJK}:

$1 .$
$$
a_{2 n}=\frac{1}{\pi} \int_{-\pi}^{\pi} f(x) \cos 2 n x \mathrm{~d} x \geqslant 0
$$
$2 .$
$$
a_{2 n+1}=\frac{1}{\pi} \int_{-\pi}^{\pi} f(x) \cos 2(n+1) x \mathrm{~d} x \leqslant 0 .
$$
\begin{CJK}{UTF8}{mj}五\end{CJK}、(\begin{CJK}{UTF8}{mj}本题\end{CJK} 16 \begin{CJK}{UTF8}{mj}分\end{CJK})

\begin{CJK}{UTF8}{mj}已知\end{CJK} $u_{n}(x), n=1,2, \cdots$ \begin{CJK}{UTF8}{mj}都在\end{CJK} $R$ \begin{CJK}{UTF8}{mj}上一致收敛\end{CJK},

\begin{enumerate}
  \item \begin{CJK}{UTF8}{mj}若\end{CJK} $\sum_{n=1}^{\infty} u_{n}(x)$ \begin{CJK}{UTF8}{mj}一致收敛于\end{CJK} $S(x)$, \begin{CJK}{UTF8}{mj}证明\end{CJK} $S(x)$ \begin{CJK}{UTF8}{mj}在\end{CJK} $R$ \begin{CJK}{UTF8}{mj}上一致连续\end{CJK}; 2. \begin{CJK}{UTF8}{mj}若\end{CJK} $\sum_{n=1}^{\infty} u_{n}(x)$ \begin{CJK}{UTF8}{mj}在\end{CJK} $R$ \begin{CJK}{UTF8}{mj}上逐点收敛于\end{CJK} $S(x)$, \begin{CJK}{UTF8}{mj}上述结论是否成立\end{CJK}.
\end{enumerate}
\begin{CJK}{UTF8}{mj}六\end{CJK}、(\begin{CJK}{UTF8}{mj}本题\end{CJK} 14 \begin{CJK}{UTF8}{mj}分\end{CJK})

\begin{CJK}{UTF8}{mj}已知\end{CJK} $f(x, y)$ \begin{CJK}{UTF8}{mj}定义在\end{CJK} $[a, b] \times(c, d]$ \begin{CJK}{UTF8}{mj}上\end{CJK}, \begin{CJK}{UTF8}{mj}且在\end{CJK} $y=c$ \begin{CJK}{UTF8}{mj}附近无界\end{CJK}.

\begin{enumerate}
  \item \begin{CJK}{UTF8}{mj}叙述含参变量反常积分\end{CJK} $\int_{c}^{d} f(x, y) \mathrm{d} y$ \begin{CJK}{UTF8}{mj}一致收敛的定义与\end{CJK} Cauchy \begin{CJK}{UTF8}{mj}准则\end{CJK};

  \item \begin{CJK}{UTF8}{mj}若\end{CJK} $\int_{c}^{d}|f(x, y)| \mathrm{d} y$ \begin{CJK}{UTF8}{mj}在区间\end{CJK} $[a, b]$ \begin{CJK}{UTF8}{mj}上一致收敛\end{CJK}, \begin{CJK}{UTF8}{mj}且\end{CJK} $g(x, y)$ \begin{CJK}{UTF8}{mj}在\end{CJK} $[a, b] \times[c, d]$ \begin{CJK}{UTF8}{mj}上连续\end{CJK}, \begin{CJK}{UTF8}{mj}证明\end{CJK}: $\int_{c}^{d} f(x, y) g(x, y) \mathrm{d} y$ \begin{CJK}{UTF8}{mj}在\end{CJK} $[a, b]$ \begin{CJK}{UTF8}{mj}上一致收敛\end{CJK}.

\end{enumerate}
\begin{CJK}{UTF8}{mj}七\end{CJK}、 (\begin{CJK}{UTF8}{mj}本题\end{CJK} 16 \begin{CJK}{UTF8}{mj}分\end{CJK})

\begin{CJK}{UTF8}{mj}已知\end{CJK} $f(u)$ \begin{CJK}{UTF8}{mj}有二阶的连续的导数\end{CJK},
$$
g(x, y, z)=f\left(\frac{x}{z}\right) \mp f\left(\frac{y}{z}\right) \mp z\left(f\left(\frac{z}{x}+\frac{z}{y}\right)\right) .
$$

\begin{enumerate}
  \item \begin{CJK}{UTF8}{mj}求\end{CJK}
\end{enumerate}
$$
x^{2} \frac{\partial^{2} g}{\partial x^{2}}+y^{2} \frac{\partial^{2} g}{\partial y^{2}}-z^{2} \frac{\partial^{2} g}{\partial z^{2}}
$$

\begin{enumerate}
  \setcounter{enumi}{2}
  \item \begin{CJK}{UTF8}{mj}计算三重积分\end{CJK}
\end{enumerate}
$$
\iiint_{\Omega}\left(x^{2} \frac{\partial^{2} g}{\partial x^{2}}+y^{2} \frac{\partial^{2} g}{\partial y^{2}}-z^{2} \frac{\partial^{2} g}{\partial z^{2}}\right) \mathrm{d} x \mathrm{~d} y \mathrm{~d} z
$$
\begin{CJK}{UTF8}{mj}其中\end{CJK} $\Omega$ \begin{CJK}{UTF8}{mj}由平面\end{CJK} $z=a_{i} x, z=b_{i} y$ \begin{CJK}{UTF8}{mj}和曲面\end{CJK} $x y z=c_{i}(i=1,2)$ \begin{CJK}{UTF8}{mj}且\end{CJK} $a_{1}<a_{2}, b_{1}<b_{2}, c_{1}<c_{2}$ \begin{CJK}{UTF8}{mj}围成\end{CJK}.

\begin{CJK}{UTF8}{mj}八\end{CJK}、 (\begin{CJK}{UTF8}{mj}本题\end{CJK} 14 \begin{CJK}{UTF8}{mj}分\end{CJK})

\begin{CJK}{UTF8}{mj}已知区域\end{CJK} $D:\left\{(x, y) \mid x^{2}+y^{2} \leqslant 1\right\}, u(x, y)$ \begin{CJK}{UTF8}{mj}在\end{CJK} $D$ \begin{CJK}{UTF8}{mj}上连续\end{CJK}, \begin{CJK}{UTF8}{mj}且\end{CJK} $\nabla u=\cos \left(\pi\left(x^{2}+y^{2}\right)\right)$, \begin{CJK}{UTF8}{mj}求证\end{CJK}:
$$
\iint_{D}\left(x \frac{\partial u}{\partial x}+y \frac{\partial u}{\partial y}\right) \mathrm{d} x \mathrm{~d} y=\frac{1}{\pi}
$$

\section{4. 武汉大学 2012 年研究生入学考试试题数学分析}
\begin{CJK}{UTF8}{mj}李扬\end{CJK}

\begin{CJK}{UTF8}{mj}微信公众号\end{CJK}: sxkyliyang

\begin{CJK}{UTF8}{mj}一\end{CJK}、\begin{CJK}{UTF8}{mj}计算题\end{CJK}(\begin{CJK}{UTF8}{mj}本题共\end{CJK} 5 \begin{CJK}{UTF8}{mj}小题\end{CJK}, \begin{CJK}{UTF8}{mj}每小题\end{CJK} 8 \begin{CJK}{UTF8}{mj}分\end{CJK}, \begin{CJK}{UTF8}{mj}共\end{CJK} 40 \begin{CJK}{UTF8}{mj}分\end{CJK})

\begin{enumerate}
  \item \begin{CJK}{UTF8}{mj}计算极限\end{CJK}:
\end{enumerate}
$$
\lim _{n \rightarrow \infty}\left(1-\frac{1}{1+3}\right)\left(1-\frac{1}{1+3+5}\right) \cdots\left(1-\frac{1}{1+3+\cdots+(2 n+1)}\right)
$$
$2 .$
$$
\lim _{x \rightarrow 0} \frac{1}{x^{4}} \int_{0}^{x}(x-t) \sin \left(t^{2}\right) \mathrm{d} t
$$

\begin{enumerate}
  \setcounter{enumi}{3}
  \item \begin{CJK}{UTF8}{mj}设\end{CJK}
\end{enumerate}
$$
F(x)=\frac{1}{x} \int_{0}^{x} \cos \left(t^{2}\right) \mathrm{d} t
$$
\begin{CJK}{UTF8}{mj}求\end{CJK} $F^{(8)}(0), F^{(10)}(0)$;

\begin{enumerate}
  \setcounter{enumi}{4}
  \item \begin{CJK}{UTF8}{mj}设\end{CJK} $z=f(x y, x+y)$, \begin{CJK}{UTF8}{mj}其中\end{CJK} $f$ \begin{CJK}{UTF8}{mj}有二阶连续偏导数\end{CJK}, \begin{CJK}{UTF8}{mj}求\end{CJK} $z_{x x}, z_{x y}$.

  \item $\iint_{D} \ln \frac{y}{x^{2}}$, \begin{CJK}{UTF8}{mj}其中\end{CJK} $D$ \begin{CJK}{UTF8}{mj}是由\end{CJK} $y=x, y=1, x=2$ \begin{CJK}{UTF8}{mj}所围成的三角形\end{CJK}.

\end{enumerate}
\begin{CJK}{UTF8}{mj}二\end{CJK}、 (\begin{CJK}{UTF8}{mj}本题\end{CJK} 12 \begin{CJK}{UTF8}{mj}分\end{CJK})

\begin{CJK}{UTF8}{mj}设\end{CJK} $a_{n+1} \geqslant a_{n}, n=1,2, \cdots$, \begin{CJK}{UTF8}{mj}且\end{CJK} $\lim _{n \rightarrow \infty} a_{n}=a>0$, \begin{CJK}{UTF8}{mj}求\end{CJK}
$$
\lim _{n \rightarrow \infty} \sqrt[n]{\sum_{k=1}^{n} a_{k}^{n}}
$$
\begin{CJK}{UTF8}{mj}三\end{CJK}、 (\begin{CJK}{UTF8}{mj}本题\end{CJK} 12 \begin{CJK}{UTF8}{mj}分\end{CJK})

\begin{CJK}{UTF8}{mj}设\end{CJK} $f(x)$ \begin{CJK}{UTF8}{mj}在有限区间上\end{CJK} $(a, b)$ \begin{CJK}{UTF8}{mj}上可微\end{CJK}, \begin{CJK}{UTF8}{mj}且无界\end{CJK}. \begin{CJK}{UTF8}{mj}求证\end{CJK}: $f^{\prime}(x)$ \begin{CJK}{UTF8}{mj}在\end{CJK} $(a, b)$ \begin{CJK}{UTF8}{mj}上也一定无界\end{CJK}.

\begin{CJK}{UTF8}{mj}四\end{CJK}、(\begin{CJK}{UTF8}{mj}本题\end{CJK} 14 \begin{CJK}{UTF8}{mj}分\end{CJK})

\begin{CJK}{UTF8}{mj}设\end{CJK} $f(x)$ \begin{CJK}{UTF8}{mj}与\end{CJK} $g(x)$ \begin{CJK}{UTF8}{mj}在\end{CJK} $(a, b)$ \begin{CJK}{UTF8}{mj}上连续\end{CJK}, \begin{CJK}{UTF8}{mj}且同为单调不减\end{CJK} (\begin{CJK}{UTF8}{mj}或同为单调不增\end{CJK})\begin{CJK}{UTF8}{mj}函数\end{CJK}, \begin{CJK}{UTF8}{mj}证明\end{CJK}:
$$
(b-a) \int_{a}^{b} f(x) g(x) \mathrm{d} x \geqslant \int_{a}^{b} f(x) \mathrm{d} x \cdot \int_{a}^{b} g(x) \mathrm{d} x
$$
\begin{CJK}{UTF8}{mj}五\end{CJK}、 (\begin{CJK}{UTF8}{mj}本题\end{CJK} 14 \begin{CJK}{UTF8}{mj}分\end{CJK})

\begin{CJK}{UTF8}{mj}设二元函数\end{CJK}

\includegraphics[max width=\textwidth]{2022_04_18_a5c47c0ff534501b502eg-264}

\begin{CJK}{UTF8}{mj}讨论\end{CJK}:

\begin{enumerate}
  \item $f_{x}(0,0), f_{y}(0,0)$ \begin{CJK}{UTF8}{mj}是否存在\end{CJK};

  \item $f_{x}(0,0), f_{y}(0,0)$ \begin{CJK}{UTF8}{mj}在\end{CJK} $(0,0)$ \begin{CJK}{UTF8}{mj}点处是否连续\end{CJK};

  \item $f(x, y)$ \begin{CJK}{UTF8}{mj}在\end{CJK} $(0,0)$ \begin{CJK}{UTF8}{mj}处是否可微\end{CJK}.

\end{enumerate}
\begin{CJK}{UTF8}{mj}六\end{CJK}、 (\begin{CJK}{UTF8}{mj}本题\end{CJK} 14 \begin{CJK}{UTF8}{mj}分\end{CJK})
$$
\frac{x z}{y}=u, \sqrt{x^{2}+y^{2}}+\sqrt{y^{2}+z^{2}}=v, \sqrt{x^{2}+y^{2}}-\sqrt{y^{2}+z^{2}}=w
$$
\begin{CJK}{UTF8}{mj}是三个分别以\end{CJK} $u, v, w$ \begin{CJK}{UTF8}{mj}为参数的单参数曲面族\end{CJK}.

\begin{CJK}{UTF8}{mj}证明\end{CJK}: \begin{CJK}{UTF8}{mj}过同一点的三曲面族的三个曲面是两两正交的\end{CJK}.

\begin{CJK}{UTF8}{mj}七\end{CJK}、 (\begin{CJK}{UTF8}{mj}本题\end{CJK} 14 \begin{CJK}{UTF8}{mj}分\end{CJK})

\begin{CJK}{UTF8}{mj}设\end{CJK} $f: \mathbb{R} \rightarrow \mathbb{R}$ \begin{CJK}{UTF8}{mj}有连续导数\end{CJK}, \begin{CJK}{UTF8}{mj}且\end{CJK} $f^{\prime}\left(x_{0}\right) \neq 0$. \begin{CJK}{UTF8}{mj}阐明\end{CJK}: \begin{CJK}{UTF8}{mj}坐标变换\end{CJK}
$$
\left\{\begin{array}{l}
u=f(x) \\
v=x f(x)-y
\end{array}\right.
$$
\begin{CJK}{UTF8}{mj}在\end{CJK} $\left(x_{0}, y_{0}\right)$ \begin{CJK}{UTF8}{mj}附近是局部可逆的\end{CJK}, \begin{CJK}{UTF8}{mj}且其逆具有形式\end{CJK}
$$
\left\{\begin{array}{l}
x=g(u) \\
y=u g(u)-v
\end{array}\right.
$$
\begin{CJK}{UTF8}{mj}八\end{CJK}、 (\begin{CJK}{UTF8}{mj}本题\end{CJK} 15 \begin{CJK}{UTF8}{mj}分\end{CJK})

\begin{CJK}{UTF8}{mj}设\end{CJK} $\Omega \subset \mathbb{R}^{3}$ \begin{CJK}{UTF8}{mj}是有界区域\end{CJK}, \begin{CJK}{UTF8}{mj}其边界\end{CJK} $\partial \Omega$ \begin{CJK}{UTF8}{mj}是光滑曲面\end{CJK}, \begin{CJK}{UTF8}{mj}函数在\end{CJK} $\bar{\Omega}$ \begin{CJK}{UTF8}{mj}上连续\end{CJK}, \begin{CJK}{UTF8}{mj}在\end{CJK} $\Omega$ \begin{CJK}{UTF8}{mj}内有二阶连续偏导数\end{CJK}, \begin{CJK}{UTF8}{mj}且满足\end{CJK}
$$
\begin{cases}\Delta u \equiv u_{x x}+u_{y y}+u_{z z}=\lambda u, & (x, y, z) \in \Omega \\ u(x, y, z)=0, & (x, y, z) \in \partial \Omega\end{cases}
$$
\begin{CJK}{UTF8}{mj}其中\end{CJK} $\lambda$ \begin{CJK}{UTF8}{mj}是常数\end{CJK}. \begin{CJK}{UTF8}{mj}求证\end{CJK}:
$$
\iiint_{\Omega}|\nabla u|^{2} \mathrm{~d} x \mathrm{~d} y \mathrm{~d} z+\lambda \iiint_{\Omega} u^{2} \mathrm{~d} x \mathrm{~d} y \mathrm{~d} z=0 .
$$
\begin{CJK}{UTF8}{mj}九\end{CJK}、 (\begin{CJK}{UTF8}{mj}本题\end{CJK} 15 \begin{CJK}{UTF8}{mj}分\end{CJK})
$$
u(x, y)=\frac{1}{y^{3}} \ln \left(1+y^{3} x\right)
$$
\begin{CJK}{UTF8}{mj}记\end{CJK}
$$
S(x)=\int_{1}^{+\infty} u(x, y) \mathrm{d} y
$$

\begin{enumerate}
  \item \begin{CJK}{UTF8}{mj}求\end{CJK} $S(x)$ \begin{CJK}{UTF8}{mj}的定义域\end{CJK};

  \item \begin{CJK}{UTF8}{mj}求证\end{CJK}: $S(x)$ \begin{CJK}{UTF8}{mj}在有界区间\end{CJK} $[0, b]$ \begin{CJK}{UTF8}{mj}上是一致收敛的\end{CJK}, \begin{CJK}{UTF8}{mj}在\end{CJK} $(0,+\infty)$ \begin{CJK}{UTF8}{mj}上不是一致收敛的\end{CJK};

  \item \begin{CJK}{UTF8}{mj}讨论\end{CJK} $S(x)$ \begin{CJK}{UTF8}{mj}的可微性\end{CJK}.

\end{enumerate}
\section{5. 武汉大学 2013 年研究生入学考试试题数学分析}
\begin{CJK}{UTF8}{mj}李扬\end{CJK}

\begin{CJK}{UTF8}{mj}微信公众号\end{CJK}: sxkyliyang

\begin{CJK}{UTF8}{mj}一\end{CJK}、\begin{CJK}{UTF8}{mj}计算题\end{CJK}(\begin{CJK}{UTF8}{mj}本题共\end{CJK} 5 \begin{CJK}{UTF8}{mj}小题\end{CJK}, \begin{CJK}{UTF8}{mj}每小题\end{CJK} 8 \begin{CJK}{UTF8}{mj}分\end{CJK}, \begin{CJK}{UTF8}{mj}共\end{CJK} 40 \begin{CJK}{UTF8}{mj}分\end{CJK})

1 . \begin{CJK}{UTF8}{mj}求\end{CJK}
$$
\lim _{x \rightarrow 0} \frac{\sqrt[n]{1+x}-1}{\ln (1+x)},
$$
\begin{CJK}{UTF8}{mj}其中\end{CJK} $n$ \begin{CJK}{UTF8}{mj}是正整数\end{CJK};

\begin{enumerate}
  \setcounter{enumi}{2}
  \item \begin{CJK}{UTF8}{mj}求\end{CJK}
\end{enumerate}
$$
\int \frac{x \ln \left(x+\sqrt{1+x^{2}}\right)}{\left(1+x^{2}\right)^{2}} \mathrm{~d} x
$$

\begin{enumerate}
  \setcounter{enumi}{3}
  \item \begin{CJK}{UTF8}{mj}求\end{CJK}
\end{enumerate}
$$
\int_{0}^{\pi / 2} \sqrt{1-2 \sin 2 x} d x
$$

\begin{enumerate}
  \setcounter{enumi}{4}
  \item \begin{CJK}{UTF8}{mj}设\end{CJK} $y=\arcsin x$, \begin{CJK}{UTF8}{mj}求\end{CJK} $y^{n}(0)$;

  \item \begin{CJK}{UTF8}{mj}设\end{CJK}

\end{enumerate}
$$
S_{n}=\frac{1}{n^{2}} \sum_{k=1}^{n} \sqrt{(n x+k)(n x+k-1)}
$$
\begin{CJK}{UTF8}{mj}其中\end{CJK} $x>0$, \begin{CJK}{UTF8}{mj}求\end{CJK} $\lim _{n \rightarrow \infty} S_{n}$.

\begin{CJK}{UTF8}{mj}二\end{CJK}、 (\begin{CJK}{UTF8}{mj}本题\end{CJK} 12 \begin{CJK}{UTF8}{mj}分\end{CJK})

\begin{CJK}{UTF8}{mj}设\end{CJK} $a>0, x_{1}=\sqrt{a}, x_{n+1}=\sqrt{a x_{n}}, n \in \mathbb{N}_{+}$. \begin{CJK}{UTF8}{mj}证明\end{CJK}: $\left\{x_{n}\right\}$ \begin{CJK}{UTF8}{mj}收敛\end{CJK}, \begin{CJK}{UTF8}{mj}并求\end{CJK} $\lim _{n \rightarrow \infty} x_{n}$.

\begin{CJK}{UTF8}{mj}三\end{CJK}、 (\begin{CJK}{UTF8}{mj}本题\end{CJK} 12 \begin{CJK}{UTF8}{mj}分\end{CJK})

\begin{CJK}{UTF8}{mj}证明\end{CJK}: \begin{CJK}{UTF8}{mj}反常积分\end{CJK}
$$
\int_{0}^{+\infty} \frac{\mathrm{d} x}{(1+x)^{2}\left(1+x^{\alpha}\right)}
$$
\begin{CJK}{UTF8}{mj}与\end{CJK} $\alpha$ \begin{CJK}{UTF8}{mj}无关\end{CJK}, \begin{CJK}{UTF8}{mj}并求其值\end{CJK}.

\begin{CJK}{UTF8}{mj}四\end{CJK}、(\begin{CJK}{UTF8}{mj}本题\end{CJK} 12 \begin{CJK}{UTF8}{mj}分\end{CJK})

\begin{CJK}{UTF8}{mj}设函数\end{CJK} $f(x)$ \begin{CJK}{UTF8}{mj}在\end{CJK} $[0,2]$ \begin{CJK}{UTF8}{mj}上连续\end{CJK}, \begin{CJK}{UTF8}{mj}且\end{CJK} $f(0)=f(2)$, \begin{CJK}{UTF8}{mj}证明\end{CJK}: \begin{CJK}{UTF8}{mj}存在\end{CJK} $\xi \in[0,1]$, \begin{CJK}{UTF8}{mj}使得\end{CJK} $f(\xi)=f(\xi+1)$.

\begin{CJK}{UTF8}{mj}五\end{CJK}、(\begin{CJK}{UTF8}{mj}本题\end{CJK} 14 \begin{CJK}{UTF8}{mj}分\end{CJK})

\begin{CJK}{UTF8}{mj}已知函数\end{CJK} $z=z(x, y)$ \begin{CJK}{UTF8}{mj}二阶连续可微\end{CJK}, \begin{CJK}{UTF8}{mj}满足偏微分方程\end{CJK}
$$
\frac{1}{(x+y)^{2}}\left(\frac{\partial^{2} z}{\partial x^{2}}+2 \frac{\partial^{2} z}{\partial x \partial y}+\frac{\partial^{2} z}{\partial y^{2}}\right)-\frac{1}{(x+y)^{3}}\left(\frac{\partial z}{\partial x}+\frac{\partial z}{\partial y}\right)=0
$$
\begin{CJK}{UTF8}{mj}令\end{CJK} $u=x y, v=x-y$, \begin{CJK}{UTF8}{mj}求证\end{CJK}:
$$
\frac{\partial^{2} z}{\partial u^{2}}+\frac{1}{v^{2}+4 u} \frac{\partial z}{\partial u}=0
$$
\begin{CJK}{UTF8}{mj}六\end{CJK}、 (\begin{CJK}{UTF8}{mj}本题\end{CJK} 14 \begin{CJK}{UTF8}{mj}分\end{CJK})

\begin{CJK}{UTF8}{mj}设\end{CJK} $D: 0 \leqslant x \leqslant 1,0 \leqslant y \leqslant 1$.

\begin{enumerate}
  \item \begin{CJK}{UTF8}{mj}计算积分\end{CJK}
\end{enumerate}
$$
A=\iint_{D}\left|x y-\frac{1}{4}\right| \mathrm{d} x \mathrm{~d} y
$$

\begin{enumerate}
  \setcounter{enumi}{2}
  \item \begin{CJK}{UTF8}{mj}设\end{CJK} $z=f(x, y)$ \begin{CJK}{UTF8}{mj}在\end{CJK} $D$ \begin{CJK}{UTF8}{mj}上连续\end{CJK}, \begin{CJK}{UTF8}{mj}并满足\end{CJK}
\end{enumerate}
$$
\iint_{D} f(x, y) \mathrm{d} x \mathrm{~d} y=0, \quad \iint_{D} x y f(x, y) \mathrm{d} x \mathrm{~d} y=1
$$
\begin{CJK}{UTF8}{mj}求证\end{CJK}: $\exists\left(x^{*}, y^{*}\right) \in D$ \begin{CJK}{UTF8}{mj}使得\end{CJK} $\left|f\left(x^{*}, y^{*}\right)\right| \geqslant \frac{1}{A}$.

\begin{CJK}{UTF8}{mj}七\end{CJK}、(\begin{CJK}{UTF8}{mj}本题\end{CJK} 15 \begin{CJK}{UTF8}{mj}分\end{CJK})

\begin{CJK}{UTF8}{mj}求曲面\end{CJK}
$$
\frac{x^{2}}{a^{2}}+\frac{y^{2}}{b^{2}}+\frac{z^{2}}{c^{2}}=1
$$
\begin{CJK}{UTF8}{mj}的切平面\end{CJK} $\pi$, \begin{CJK}{UTF8}{mj}使得它在第一卦限的部分与三个坐标平面所围的四面体的体积最小\end{CJK}, \begin{CJK}{UTF8}{mj}并求这个最小的体积\end{CJK}.

\begin{CJK}{UTF8}{mj}八\end{CJK}、 (\begin{CJK}{UTF8}{mj}本题\end{CJK} 15 \begin{CJK}{UTF8}{mj}分\end{CJK})

\begin{CJK}{UTF8}{mj}设\end{CJK}
$$
f(y)=\int_{0}^{+\infty} x e^{-x^{2}} \cos x y \mathrm{~d} x,-\infty<y<+\infty
$$

\begin{enumerate}
  \item \begin{CJK}{UTF8}{mj}求证\end{CJK} $f(y)$ \begin{CJK}{UTF8}{mj}有任意阶连续的导数\end{CJK};

  \item \begin{CJK}{UTF8}{mj}求\end{CJK} $f(y)$ \begin{CJK}{UTF8}{mj}的麦克劳林级数\end{CJK}.

\end{enumerate}
\begin{CJK}{UTF8}{mj}九\end{CJK}、 (\begin{CJK}{UTF8}{mj}本题\end{CJK} 15 \begin{CJK}{UTF8}{mj}分\end{CJK})

\begin{CJK}{UTF8}{mj}求第一类曲面积分\end{CJK}
$$
I=\iint_{\Sigma}\left(x^{2}+y^{2}+z^{2}\right)^{-\frac{3}{2}}\left(\frac{x^{2}}{a^{4}}+\frac{y^{2}}{b^{4}}+\frac{z^{2}}{c^{4}}\right)^{-\frac{1}{2}} \mathrm{~d} S .
$$
\begin{CJK}{UTF8}{mj}其中\end{CJK} $\Sigma$ \begin{CJK}{UTF8}{mj}为椭球面\end{CJK}: $\frac{x^{2}}{a^{2}}+\frac{y^{2}}{b^{2}}+\frac{z^{2}}{c^{2}}=1(a, b, c>0)$.

\section{6. 武汉大学 2014 年研究生入学考试试题数学分析}
\begin{CJK}{UTF8}{mj}李扬\end{CJK}

\begin{CJK}{UTF8}{mj}微信公众号\end{CJK}: sxkyliyang

\begin{CJK}{UTF8}{mj}一\end{CJK}、 \begin{CJK}{UTF8}{mj}计算题\end{CJK}(\begin{CJK}{UTF8}{mj}本题共\end{CJK} 5 \begin{CJK}{UTF8}{mj}小题\end{CJK}, \begin{CJK}{UTF8}{mj}每小题\end{CJK} 12 \begin{CJK}{UTF8}{mj}分\end{CJK}, \begin{CJK}{UTF8}{mj}共\end{CJK} 60 \begin{CJK}{UTF8}{mj}分\end{CJK})

\begin{enumerate}
  \item \begin{CJK}{UTF8}{mj}求积分\end{CJK}
\end{enumerate}
$$
\int_{0}^{1}(x \ln x)^{n} \mathrm{~d} x
$$

\begin{enumerate}
  \setcounter{enumi}{2}
  \item \begin{CJK}{UTF8}{mj}求极限\end{CJK}
\end{enumerate}
$$
\lim _{x \rightarrow 1^{-}}(1-x)^{3} \sum_{n=1}^{\infty} n^{2} x^{n}
$$

\begin{enumerate}
  \setcounter{enumi}{3}
  \item \begin{CJK}{UTF8}{mj}求极限\end{CJK}
\end{enumerate}
$$
\lim _{x \rightarrow \infty} n\left[\left(1+\frac{1}{n}\right)^{n}-e\right]
$$

\begin{enumerate}
  \setcounter{enumi}{4}
  \item \begin{CJK}{UTF8}{mj}求不定积分\end{CJK}
\end{enumerate}
$$
\int \frac{1+\sin x}{1-\cos x} e^{-x} \mathrm{~d} x
$$

\begin{enumerate}
  \setcounter{enumi}{5}
  \item \begin{CJK}{UTF8}{mj}求\end{CJK} $\int_{c} x y \mathrm{~d} s$, \begin{CJK}{UTF8}{mj}其中\end{CJK} $c$ \begin{CJK}{UTF8}{mj}是\end{CJK} $x^{2}+y^{2}+z^{2}=9$ \begin{CJK}{UTF8}{mj}与\end{CJK} $x+y+z=0$ \begin{CJK}{UTF8}{mj}的交线\end{CJK}.
\end{enumerate}
\begin{CJK}{UTF8}{mj}二\end{CJK}、 (\begin{CJK}{UTF8}{mj}本题\end{CJK} 12 \begin{CJK}{UTF8}{mj}分\end{CJK})

$f(x)$ \begin{CJK}{UTF8}{mj}在\end{CJK} $[-1,1]$ \begin{CJK}{UTF8}{mj}上有二阶连续导数\end{CJK}, \begin{CJK}{UTF8}{mj}且\end{CJK} $f(0)=0$, \begin{CJK}{UTF8}{mj}证明\end{CJK}: \begin{CJK}{UTF8}{mj}存在\end{CJK} $\xi \in[-1,1]$, \begin{CJK}{UTF8}{mj}使得\end{CJK}
$$
f^{\prime \prime}(\xi)=3 \int_{-1}^{1} f(x) \mathrm{d} x
$$
\begin{CJK}{UTF8}{mj}三\end{CJK}、 (\begin{CJK}{UTF8}{mj}本题\end{CJK} 12 \begin{CJK}{UTF8}{mj}分\end{CJK})

\begin{CJK}{UTF8}{mj}设\end{CJK}
$$
x_{n}=f\left(\frac{1}{n^{2}}\right)+f\left(\frac{2}{n^{2}}\right)+\cdots+f\left(\frac{n}{n^{2}}\right),
$$
$f(x)$ \begin{CJK}{UTF8}{mj}在零点附近可微\end{CJK}, \begin{CJK}{UTF8}{mj}且\end{CJK} $f(0)=0, f^{\prime}(0)=1$, \begin{CJK}{UTF8}{mj}证明\end{CJK} $\left\{x_{n}\right\}$ \begin{CJK}{UTF8}{mj}存在\end{CJK}, \begin{CJK}{UTF8}{mj}并求\end{CJK} $x_{n}$ \begin{CJK}{UTF8}{mj}的极限\end{CJK}.

\begin{CJK}{UTF8}{mj}四\end{CJK}、(\begin{CJK}{UTF8}{mj}本题\end{CJK} 12 \begin{CJK}{UTF8}{mj}分\end{CJK})

$z$ \begin{CJK}{UTF8}{mj}关于\end{CJK} $x, y$ \begin{CJK}{UTF8}{mj}的函数\end{CJK}, \begin{CJK}{UTF8}{mj}用\end{CJK}
$$
\left\{\begin{array}{l}
u=x+2 y \\
v=x+a y
\end{array}\right.
$$
\begin{CJK}{UTF8}{mj}变换\end{CJK}
$$
2 \frac{\partial^{2} z}{\partial x^{2}}+\frac{\partial^{2} z}{\partial x \partial y}-\frac{\partial^{2} z}{\partial y^{2}}=0
$$
\begin{CJK}{UTF8}{mj}变换为\end{CJK}
$$
\frac{\partial^{2} z}{\partial u \partial v}=0
$$
\begin{CJK}{UTF8}{mj}求\end{CJK} $a$.

\begin{CJK}{UTF8}{mj}五\end{CJK}、 (\begin{CJK}{UTF8}{mj}本题\end{CJK} 14 \begin{CJK}{UTF8}{mj}分\end{CJK}) \begin{CJK}{UTF8}{mj}椭圆\end{CJK} $3 x^{2}+y^{2}=1$ \begin{CJK}{UTF8}{mj}绕\end{CJK} $y$ \begin{CJK}{UTF8}{mj}旋转得到\end{CJK} $\Omega,(u, v, w)$ \begin{CJK}{UTF8}{mj}是\end{CJK} $\Omega$ \begin{CJK}{UTF8}{mj}球面上的法向量的方向余弦\end{CJK}, $\Sigma$ \begin{CJK}{UTF8}{mj}为\end{CJK} $\Omega$ \begin{CJK}{UTF8}{mj}的上半部分\end{CJK}, \begin{CJK}{UTF8}{mj}方向向\end{CJK} \begin{CJK}{UTF8}{mj}上\end{CJK}, \begin{CJK}{UTF8}{mj}计算\end{CJK}
$$
\iint_{\Sigma} z(x u+y v+2 w z) \mathrm{d} S .
$$
\begin{CJK}{UTF8}{mj}六\end{CJK}、(\begin{CJK}{UTF8}{mj}本题\end{CJK} 14 \begin{CJK}{UTF8}{mj}分\end{CJK})

\begin{CJK}{UTF8}{mj}设\end{CJK}
$$
I=\int_{0}^{+\infty} \frac{1-e^{-\alpha x}}{x e^{x}} \mathrm{~d} x, \alpha>-1
$$

\begin{enumerate}
  \item \begin{CJK}{UTF8}{mj}证明\end{CJK} $I$ \begin{CJK}{UTF8}{mj}在\end{CJK} $\alpha>-1$ \begin{CJK}{UTF8}{mj}上收敛\end{CJK};

  \item \begin{CJK}{UTF8}{mj}计算\end{CJK} $I$ \begin{CJK}{UTF8}{mj}的值\end{CJK}.

\end{enumerate}
\begin{CJK}{UTF8}{mj}七\end{CJK}、 (\begin{CJK}{UTF8}{mj}本题\end{CJK} 15 \begin{CJK}{UTF8}{mj}分\end{CJK})

\begin{CJK}{UTF8}{mj}有函数项级数\end{CJK}
$$
\sum_{n=1}^{+\infty} \frac{n^{n+2}}{(1+n x)^{n}}
$$

\begin{enumerate}
  \item \begin{CJK}{UTF8}{mj}证明级数在\end{CJK} $(1,+\infty)$ \begin{CJK}{UTF8}{mj}上收敛\end{CJK};

  \item \begin{CJK}{UTF8}{mj}级数在\end{CJK} $(1,+\infty)$ \begin{CJK}{UTF8}{mj}上非一致收敛\end{CJK}, \begin{CJK}{UTF8}{mj}但在\end{CJK} $(1,+\infty)$ \begin{CJK}{UTF8}{mj}上连续\end{CJK}.

\end{enumerate}
\section{7. 武汉大学 2015 年研究生入学考试试题数学分析}
\begin{CJK}{UTF8}{mj}李扬\end{CJK}

\begin{CJK}{UTF8}{mj}微信公众号\end{CJK}: sxkyliyang

\begin{CJK}{UTF8}{mj}一\end{CJK}、\begin{CJK}{UTF8}{mj}计算题\end{CJK}(\begin{CJK}{UTF8}{mj}本题共\end{CJK} 20 \begin{CJK}{UTF8}{mj}分\end{CJK})

\begin{enumerate}
  \item \begin{CJK}{UTF8}{mj}求极限\end{CJK}
\end{enumerate}
$$
\lim _{x \rightarrow 1} \frac{\left(x^{n}-1\right)\left(x^{n-1}-1\right) \cdots\left(x^{n-k+1}-1\right)}{\left(x^{1}-1\right)\left(x^{2}-1\right) \cdots\left(x^{k}-1\right)} ;
$$

\begin{enumerate}
  \setcounter{enumi}{2}
  \item \begin{CJK}{UTF8}{mj}求极限\end{CJK}
\end{enumerate}
$$
\lim _{x \rightarrow 0} \frac{\sqrt[n]{\cos \alpha x}-\sqrt[m]{\cos \beta x}}{\sin ^{2} x}
$$
\begin{CJK}{UTF8}{mj}其中\end{CJK} $m, n$ \begin{CJK}{UTF8}{mj}为正整数\end{CJK};

\begin{enumerate}
  \setcounter{enumi}{3}
  \item \begin{CJK}{UTF8}{mj}求极限\end{CJK}
\end{enumerate}
$$
\lim _{n \rightarrow \infty} \sum_{k=1}^{n}\left(\sqrt{1+\frac{k^{2}}{n^{3}}}-1\right)
$$

\begin{enumerate}
  \setcounter{enumi}{4}
  \item \begin{CJK}{UTF8}{mj}设\end{CJK} $0<x_{n} \leqslant x_{n+1}+\frac{1}{n^{2}}$, \begin{CJK}{UTF8}{mj}讨论极限\end{CJK} $\lim _{n \rightarrow \infty} x_{n}$ \begin{CJK}{UTF8}{mj}的存在性\end{CJK}.
\end{enumerate}
\begin{CJK}{UTF8}{mj}二\end{CJK}、 (\begin{CJK}{UTF8}{mj}本题\end{CJK} 20 \begin{CJK}{UTF8}{mj}分\end{CJK})

\begin{CJK}{UTF8}{mj}给定曲面\end{CJK}
$$
F\left((x-a)(x-c)^{-1},(y-b)(y-c)^{-1}\right)=0(a, b, c \text { 为常数 })
$$
, \begin{CJK}{UTF8}{mj}其中\end{CJK} $u=F(s, t)$ \begin{CJK}{UTF8}{mj}二阶连续可微\end{CJK}, \begin{CJK}{UTF8}{mj}梯度处处不为\end{CJK} 0 . \begin{CJK}{UTF8}{mj}证明\end{CJK}:

\begin{enumerate}
  \item \begin{CJK}{UTF8}{mj}曲面的切平面过一定点\end{CJK};

  \item \begin{CJK}{UTF8}{mj}函数\end{CJK} $z=z(x, y)$ \begin{CJK}{UTF8}{mj}满足\end{CJK}

\end{enumerate}
$$
\frac{\partial^{2} z}{\partial x^{2}} \frac{\partial^{2} z}{\partial y^{2}}-\left(\frac{\partial^{2} z}{\partial x \partial y}\right)^{2}=0
$$
\begin{CJK}{UTF8}{mj}三\end{CJK}、 (\begin{CJK}{UTF8}{mj}本题\end{CJK} 20 \begin{CJK}{UTF8}{mj}分\end{CJK})

\begin{CJK}{UTF8}{mj}设\end{CJK} $a_{n}>0$,
$$
\lim _{n \rightarrow \infty} n\left(\frac{a_{n}}{a_{n+1}}-1\right)=\lambda>0
$$
\begin{CJK}{UTF8}{mj}证明\end{CJK}: $\sum_{n=1}^{\infty}(-1)^{n-1} a_{n}$ \begin{CJK}{UTF8}{mj}收敛\end{CJK}.

\begin{CJK}{UTF8}{mj}四\end{CJK}、 (\begin{CJK}{UTF8}{mj}本题\end{CJK} 15 \begin{CJK}{UTF8}{mj}分\end{CJK})

\begin{CJK}{UTF8}{mj}求极限\end{CJK}
$$
\lim _{t \rightarrow+\infty}\left\{e^{-t} \int_{0}^{t} \int_{0}^{t} \frac{e^{x}-e^{y}}{x-y} \mathrm{~d} x \mathrm{~d} y\right\}
$$
\begin{CJK}{UTF8}{mj}或证明此极限不存在\end{CJK}.

\begin{CJK}{UTF8}{mj}五\end{CJK}、(\begin{CJK}{UTF8}{mj}本题\end{CJK} 30 \begin{CJK}{UTF8}{mj}分\end{CJK})

\begin{enumerate}
  \item \begin{CJK}{UTF8}{mj}求积分\end{CJK}
\end{enumerate}
$$
\iint_{D}|\cos (x+y)| \mathrm{d} x \mathrm{~d} y
$$
\begin{CJK}{UTF8}{mj}其中\end{CJK} $D: 0 \leqslant x \leqslant \pi, 0 \leqslant y \leqslant \pi$.

\begin{enumerate}
  \setcounter{enumi}{2}
  \item \begin{CJK}{UTF8}{mj}设\end{CJK} $0<\alpha<1$, \begin{CJK}{UTF8}{mj}求积分\end{CJK} $\int_{0}^{1} f\left(t^{\alpha}\right) \mathrm{d} t$ \begin{CJK}{UTF8}{mj}的上确界\end{CJK}, \begin{CJK}{UTF8}{mj}其中连续函数\end{CJK} $f$ \begin{CJK}{UTF8}{mj}满足\end{CJK}: $\int_{0}^{1}|f(t)| \mathrm{d} t \leqslant 1$. \begin{CJK}{UTF8}{mj}六\end{CJK}、 (\begin{CJK}{UTF8}{mj}本题\end{CJK} 25 \begin{CJK}{UTF8}{mj}分\end{CJK})
\end{enumerate}
\begin{CJK}{UTF8}{mj}设\end{CJK}
$$
f(t)=\int_{1}^{+\infty} \frac{\cos x t}{1+x^{2}} \mathrm{~d} x
$$
\begin{CJK}{UTF8}{mj}证明\end{CJK}:

\begin{enumerate}
  \item \begin{CJK}{UTF8}{mj}积分在在\end{CJK} $(-\infty,+\infty)$ \begin{CJK}{UTF8}{mj}上一致收敛\end{CJK};

  \item $\lim _{t \rightarrow \infty} f(t)=0$;

  \item $f(t)$ \begin{CJK}{UTF8}{mj}在\end{CJK} $(-\infty,+\infty)$ \begin{CJK}{UTF8}{mj}上一致收敛\end{CJK};

  \item $\int_{0}^{\pi} f(t) \sin t \mathrm{~d} t \leqslant 0$

  \item $\exists \xi \in[0, \pi]$, \begin{CJK}{UTF8}{mj}使得\end{CJK} $f(\xi)=0$.

\end{enumerate}
\section{8. 武汉大学 2017 年研究生入学考试试题数学分析}
\begin{CJK}{UTF8}{mj}李扬\end{CJK}

\begin{CJK}{UTF8}{mj}微信公众号\end{CJK}: sxkyliyang

\begin{CJK}{UTF8}{mj}一\end{CJK}、 \begin{CJK}{UTF8}{mj}计算题\end{CJK}

\begin{enumerate}
  \item \begin{CJK}{UTF8}{mj}求极限\end{CJK}
\end{enumerate}
$$
\lim _{x \rightarrow 0} \frac{\sqrt[n]{\cos \alpha x}-\sqrt[m]{\cos \beta x}}{\sin ^{2} x}
$$

\begin{enumerate}
  \setcounter{enumi}{2}
  \item \begin{CJK}{UTF8}{mj}求极限\end{CJK}
\end{enumerate}
$$
\lim _{n \rightarrow \infty} \sum_{k=1}^{n}\left(e^{\frac{k^{2}}{n^{3}}}-1\right)
$$

\begin{enumerate}
  \setcounter{enumi}{3}
  \item $a>0, x_{1}=\sqrt{a}, x_{n+1}=\sqrt{a x_{n}}(n=1,2,3, \cdots)$, \begin{CJK}{UTF8}{mj}求\end{CJK} $\lim _{n \rightarrow \infty} x_{n}$.
\end{enumerate}
\begin{CJK}{UTF8}{mj}二\end{CJK}、\begin{CJK}{UTF8}{mj}已知\end{CJK} $y=f(x, t)$, \begin{CJK}{UTF8}{mj}其中\end{CJK} $t$ \begin{CJK}{UTF8}{mj}是由方程\end{CJK} $F(x, y, t)=0$ \begin{CJK}{UTF8}{mj}确定的隐函数\end{CJK}, \begin{CJK}{UTF8}{mj}求\end{CJK} $\frac{\mathrm{d} y}{\mathrm{~d} x}$.

\begin{CJK}{UTF8}{mj}三\end{CJK}、\begin{CJK}{UTF8}{mj}已知\end{CJK} $f(x)$ \begin{CJK}{UTF8}{mj}在\end{CJK} $[0,+\infty)$ \begin{CJK}{UTF8}{mj}上有非负的二阶导函数\end{CJK}.

\begin{enumerate}
  \item \begin{CJK}{UTF8}{mj}证明\end{CJK}:
\end{enumerate}
$$
\frac{f(x)-f(x-h)}{h} \leqslant f^{\prime}(x) \leqslant \frac{f(x+h)-f(x)}{h}, 0<h<x
$$

\begin{enumerate}
  \setcounter{enumi}{2}
  \item $f(x)$ \begin{CJK}{UTF8}{mj}在满足什么条件时\end{CJK}, \begin{CJK}{UTF8}{mj}有\end{CJK} $\lim _{x \rightarrow+\infty} f^{\prime}(x)=0$.
\end{enumerate}
\begin{CJK}{UTF8}{mj}四\end{CJK}、 $f(x)$ \begin{CJK}{UTF8}{mj}在\end{CJK} $[0,1]$ \begin{CJK}{UTF8}{mj}上有二阶连续导数\end{CJK}, $f(0)=f(1)=0, \min _{x \in[0,1]} f(x)=-\frac{1}{8}$, \begin{CJK}{UTF8}{mj}求证\end{CJK}: $\max _{x \in(0,1)} f^{\prime \prime}(x) \geqslant 1$.

\begin{CJK}{UTF8}{mj}五\end{CJK}、1. \begin{CJK}{UTF8}{mj}计算\end{CJK}
$$
\iint_{D}\left|x^{2}+y^{2}-2 y\right| \mathrm{d} x \mathrm{~d} y
$$
\begin{CJK}{UTF8}{mj}其中\end{CJK}: $D=\left\{(x, y) \mid x^{2}+y^{2} \leqslant 4\right\}$;

\begin{enumerate}
  \setcounter{enumi}{2}
  \item \begin{CJK}{UTF8}{mj}一个与路径无关的第二型曲线积分\end{CJK};

  \item \begin{CJK}{UTF8}{mj}求\end{CJK}

\end{enumerate}
$$
\iint_{\Sigma} \frac{x \mathrm{~d} y \mathrm{~d} z+y \mathrm{~d} z \mathrm{~d} x+z \mathrm{~d} x \mathrm{~d} y}{\sqrt{\left(x^{2}+y^{2}+z^{2}\right)^{3}}},
$$
\begin{CJK}{UTF8}{mj}其中\end{CJK} $\Sigma$ \begin{CJK}{UTF8}{mj}是\end{CJK}
$$
1-z=\frac{(x-1)^{2}}{4}+\frac{(y-1)^{2}}{9}
$$
\begin{CJK}{UTF8}{mj}的上半部分的上侧\end{CJK}.

\begin{CJK}{UTF8}{mj}六\end{CJK}、 $u(x, y)$ \begin{CJK}{UTF8}{mj}在开区域\end{CJK} $\Omega$ \begin{CJK}{UTF8}{mj}内有连续的二阶偏导数\end{CJK}, \begin{CJK}{UTF8}{mj}证明\end{CJK}:
$$
\frac{\partial^{2} u}{\partial x^{2}}+\frac{\partial^{2} u}{\partial y^{2}}=0
$$
\begin{CJK}{UTF8}{mj}在\end{CJK} $\Omega$ \begin{CJK}{UTF8}{mj}内处处成立当且仅当\end{CJK} $\int_{\partial B} \frac{\partial u}{\partial \vec{n}} \mathrm{~d} s=0$ \begin{CJK}{UTF8}{mj}对任意圆盘\end{CJK} $B \subseteq \Omega$ \begin{CJK}{UTF8}{mj}都成立\end{CJK}.

\begin{CJK}{UTF8}{mj}七\end{CJK}、 $f(x)$ \begin{CJK}{UTF8}{mj}在\end{CJK} $(-\infty,+\infty)$ \begin{CJK}{UTF8}{mj}上连续\end{CJK}, \begin{CJK}{UTF8}{mj}且恒有\end{CJK} $f(x+1)=f(x+\pi)$, \begin{CJK}{UTF8}{mj}求证\end{CJK}: $f(x)$ \begin{CJK}{UTF8}{mj}恒为常数\end{CJK}.

\begin{CJK}{UTF8}{mj}八\end{CJK}、 $f(x)$ \begin{CJK}{UTF8}{mj}在\end{CJK} $[0,1]$ \begin{CJK}{UTF8}{mj}上连续\end{CJK}, \begin{CJK}{UTF8}{mj}且不为常数\end{CJK}, \begin{CJK}{UTF8}{mj}证明\end{CJK}: $f(x)$ \begin{CJK}{UTF8}{mj}在\end{CJK} $[0,1]$ \begin{CJK}{UTF8}{mj}内有非极值点\end{CJK}.

\section{9. 武汉大学 2018 年研究生入学考试试题数学分析}
\begin{CJK}{UTF8}{mj}李扬\end{CJK}

\begin{CJK}{UTF8}{mj}微信公众号\end{CJK}: sxkyliyang

\begin{CJK}{UTF8}{mj}一\end{CJK}、\begin{CJK}{UTF8}{mj}计算题\end{CJK}(\begin{CJK}{UTF8}{mj}本题\end{CJK} 30 \begin{CJK}{UTF8}{mj}分\end{CJK}, \begin{CJK}{UTF8}{mj}每小题\end{CJK} 10 \begin{CJK}{UTF8}{mj}分\end{CJK})

\begin{enumerate}
  \item \begin{CJK}{UTF8}{mj}计算极限\end{CJK}
\end{enumerate}
$$
\lim _{n \rightarrow \infty} \sum_{k=n^{2}}^{(n+1)^{2}} \frac{1}{\sqrt{k}}
$$

\begin{enumerate}
  \setcounter{enumi}{2}
  \item \begin{CJK}{UTF8}{mj}计算极限\end{CJK}
\end{enumerate}
$$
\lim _{n \rightarrow \infty} \frac{\int_{0}^{\pi} \sin ^{n} x \cos ^{6} x d x}{\int_{0}^{\pi} \sin ^{n} x d x}
$$

\begin{enumerate}
  \setcounter{enumi}{3}
  \item \begin{CJK}{UTF8}{mj}已知\end{CJK} $x_{n+1}=\ln \left(1+x_{n}\right)$, \begin{CJK}{UTF8}{mj}且\end{CJK} $x_{1}>0$, \begin{CJK}{UTF8}{mj}计算\end{CJK} $\lim _{n \rightarrow \infty} n x_{n}$.
\end{enumerate}
\begin{CJK}{UTF8}{mj}二\end{CJK}、\begin{CJK}{UTF8}{mj}设\end{CJK} $f(x), f_{1}(x)$ \begin{CJK}{UTF8}{mj}在\end{CJK} $[a, b]$ \begin{CJK}{UTF8}{mj}区间上连续\end{CJK}, $f_{n+1}(x)=f(x)+\int_{a}^{x} \sin \left\{f_{n}(t)\right\} \mathrm{d}$. \begin{CJK}{UTF8}{mj}证明\end{CJK}: $\left\{f_{n}\right\}$ \begin{CJK}{UTF8}{mj}在\end{CJK} $[a, b]$ \begin{CJK}{UTF8}{mj}一致收敛\end{CJK}.

\begin{CJK}{UTF8}{mj}三\end{CJK}、\begin{CJK}{UTF8}{mj}设\end{CJK}
$$
f(x)= \begin{cases}e^{-\frac{1}{x^{2}}}, & x \neq 0 \\ 0, & x=0\end{cases}
$$
\begin{CJK}{UTF8}{mj}证明\end{CJK} $f(x)$ \begin{CJK}{UTF8}{mj}在\end{CJK} $x=0$ \begin{CJK}{UTF8}{mj}处任意阶导数存在\end{CJK}.

\begin{CJK}{UTF8}{mj}四\end{CJK}、\begin{CJK}{UTF8}{mj}已知\end{CJK} $\left(x_{1}, x_{2}, x_{3}\right) \in R^{3}$, \begin{CJK}{UTF8}{mj}其中\end{CJK} $u=\frac{1}{|x|},|x|=\sqrt{x_{1}^{2}+x_{2}^{2}+x_{3}^{2}}$, \begin{CJK}{UTF8}{mj}计算\end{CJK}
$$
\iint_{S} \frac{\partial^{u}}{\partial x_{i} \partial x_{j}} \mathrm{~d} S, i, j=1,2,3
$$
\begin{CJK}{UTF8}{mj}其中\end{CJK} $S: x_{1}^{2}+x_{2}^{2}+x_{3}^{2}=R^{2}$.

\begin{CJK}{UTF8}{mj}五\end{CJK}、\begin{CJK}{UTF8}{mj}讨论求解方程\end{CJK} $f(x)$ \begin{CJK}{UTF8}{mj}牛顿切线法\end{CJK}.

\begin{enumerate}
  \item \begin{CJK}{UTF8}{mj}推导牛顿切线法迭代公式\end{CJK};

  \item \begin{CJK}{UTF8}{mj}在适当条件下\end{CJK}, \begin{CJK}{UTF8}{mj}证明牛顿迭代法收敛\end{CJK}.

\end{enumerate}
\begin{CJK}{UTF8}{mj}六\end{CJK}、\begin{CJK}{UTF8}{mj}求极限\end{CJK}
$$
\lim _{n \rightarrow \infty}\left(n A-\sum_{k=1}^{n} f\left(\frac{k}{n}\right)\right)=B
$$
\begin{CJK}{UTF8}{mj}存在时\end{CJK}, $A, B$ \begin{CJK}{UTF8}{mj}的值\end{CJK}.

\begin{CJK}{UTF8}{mj}七\end{CJK}、\begin{CJK}{UTF8}{mj}设\end{CJK} $u_{i}=u_{i}\left(x_{1}, x_{2}\right), i=1,2$, \begin{CJK}{UTF8}{mj}且关于每个变量为周期为\end{CJK} 1 \begin{CJK}{UTF8}{mj}的连续可微函数\end{CJK}, \begin{CJK}{UTF8}{mj}求\end{CJK}
$$
\iint_{0 \leqslant x_{1}, x_{2} \leqslant 1} \operatorname{det}\left(\delta_{i j}+\frac{\partial u_{i}}{\partial x_{j}}\right) \mathrm{d} x_{1} \mathrm{~d} x_{2},
$$
\begin{CJK}{UTF8}{mj}其中\end{CJK} $\operatorname{det}\left(\delta_{i j}+\frac{\partial u_{i}}{\partial x_{j}}\right)$ \begin{CJK}{UTF8}{mj}是映射\end{CJK} $x \mapsto\left(x_{1}+u_{1}, x_{2}+u_{2}\right)$ \begin{CJK}{UTF8}{mj}的雅可比行列式\end{CJK}.

\begin{CJK}{UTF8}{mj}八\end{CJK}、\begin{CJK}{UTF8}{mj}设\end{CJK} $f(x)$ \begin{CJK}{UTF8}{mj}在\end{CJK} $[a, b]$ \begin{CJK}{UTF8}{mj}上\end{CJK} Riemann \begin{CJK}{UTF8}{mj}可积\end{CJK}, $\varphi(x)$ \begin{CJK}{UTF8}{mj}是周期为\end{CJK} $T$ \begin{CJK}{UTF8}{mj}的连续函数\end{CJK}, \begin{CJK}{UTF8}{mj}证明\end{CJK}:

\begin{enumerate}
  \item \begin{CJK}{UTF8}{mj}存在阶梯函数\end{CJK} $g_{\varepsilon}(x)$ \begin{CJK}{UTF8}{mj}使得\end{CJK}
\end{enumerate}
$$
\int_{a}^{b}\left|f(x)-g_{\varepsilon}(x)\right| \mathrm{d} x<\frac{\varepsilon}{2}
$$

\begin{enumerate}
  \setcounter{enumi}{2}
  \item \begin{CJK}{UTF8}{mj}计算\end{CJK}:
\end{enumerate}
$$
\lim _{n \rightarrow \infty} \int_{a}^{b} \varphi(n x) \mathrm{d} x
$$

\begin{enumerate}
  \setcounter{enumi}{3}
  \item \begin{CJK}{UTF8}{mj}证明\end{CJK}:
\end{enumerate}
$$
\lim _{n \rightarrow \infty} \int_{a}^{b} f(x) \varphi(n x) \mathrm{d} x=\frac{1}{T} \int_{0}^{T} \varphi(x) \mathrm{d} x \int_{a}^{b} f(x) \mathrm{d} x
$$

\begin{enumerate}
  \setcounter{enumi}{4}
  \item \begin{CJK}{UTF8}{mj}计算\end{CJK}:
\end{enumerate}
$$
\lim _{n \rightarrow \infty} \frac{1}{\ln n} \int_{0}^{T} \frac{\varphi(n x)}{x} \mathrm{~d} x \text {, 其中函数 } \frac{\varphi(n x)}{x} \text { 收敛. }
$$

\section{0. 武汉大学 2009 年研究生入学考试试题高等代数}
\begin{CJK}{UTF8}{mj}李扬\end{CJK}

\begin{CJK}{UTF8}{mj}微信公众号\end{CJK}: sxkyliyang

\begin{CJK}{UTF8}{mj}一\end{CJK}、 (15 \begin{CJK}{UTF8}{mj}分\end{CJK}) \begin{CJK}{UTF8}{mj}计算\end{CJK} $n$ \begin{CJK}{UTF8}{mj}阶行列式\end{CJK}
$$
D=\left|\begin{array}{cccc}
a_{1}^{2}-\mu & a_{1} a_{2} & \cdots & a_{1} a_{n} \\
a_{2} a_{1} & a_{2}^{2}-\mu & \cdots & a_{2} a_{n} \\
\vdots & \vdots & & \vdots \\
a_{n} a_{1} & a_{n} a_{2} & \cdots & a_{n}^{2}-\mu
\end{array}\right|(\mu \neq 0) . .
$$
\begin{CJK}{UTF8}{mj}二\end{CJK}、 $(16$ \begin{CJK}{UTF8}{mj}分\end{CJK} $)$ \begin{CJK}{UTF8}{mj}设\end{CJK} $A=\left(\alpha_{1}, \alpha_{2}, \cdots, \alpha_{n}\right)$ \begin{CJK}{UTF8}{mj}与\end{CJK} $B=\left(\beta_{1}, \beta_{2}, \cdots, \beta_{n}\right)$ \begin{CJK}{UTF8}{mj}都是\end{CJK} $m \times n$ \begin{CJK}{UTF8}{mj}矩阵\end{CJK}, \begin{CJK}{UTF8}{mj}且满足\end{CJK} $r(A)<r(B)=n$. \begin{CJK}{UTF8}{mj}对于下\end{CJK} \begin{CJK}{UTF8}{mj}述\end{CJK} 4 \begin{CJK}{UTF8}{mj}个选项\end{CJK}, \begin{CJK}{UTF8}{mj}若正确则给予证明\end{CJK}, \begin{CJK}{UTF8}{mj}若不正确请给出反例\end{CJK}.\\
(A) \begin{CJK}{UTF8}{mj}向量组\end{CJK} $\alpha_{1}, \alpha_{2}, \cdots, \alpha_{n}, \beta_{1}, \beta_{2}, \cdots, \beta_{n}$ \begin{CJK}{UTF8}{mj}必线性相关\end{CJK};\\
(B) \begin{CJK}{UTF8}{mj}向量组\end{CJK} $\alpha_{1}, \alpha_{2}, \cdots, \alpha_{n}, \beta_{1}, \beta_{2}, \cdots, \beta_{n}$ \begin{CJK}{UTF8}{mj}必线性无关\end{CJK};\\
(C) \begin{CJK}{UTF8}{mj}向量组\end{CJK} $\alpha_{1}+\beta_{1}, \alpha_{2}+\beta_{2}, \cdots, \alpha_{n}+\beta_{n}, \alpha_{1}-\beta_{1}, \alpha_{2}-\beta_{2}, \cdots, \alpha_{n}-\beta_{n}$ \begin{CJK}{UTF8}{mj}必线性相关\end{CJK};\\
(D)\begin{CJK}{UTF8}{mj}向量组\end{CJK} $\alpha_{1}+\beta_{1}, \alpha_{2}+\beta_{2}, \cdots, \alpha_{n}+\beta_{n}, \alpha_{1}-\beta_{1}, \alpha_{2}-\beta_{2}, \cdots, \alpha_{n}-\beta_{n}$ \begin{CJK}{UTF8}{mj}必线性无关\end{CJK};

\begin{CJK}{UTF8}{mj}三\end{CJK}、 (15 \begin{CJK}{UTF8}{mj}分\end{CJK}) \begin{CJK}{UTF8}{mj}已知矩阵\end{CJK}
$$
A=\left(\begin{array}{lll}
1 & 0 & 3 \\
1 & 4 & 5 \\
0 & 0 & 2
\end{array}\right), B=\left(\begin{array}{lll}
1 & 2 & 1 \\
3 & 5 & a \\
2 & 5 & 7
\end{array}\right)
$$
\begin{CJK}{UTF8}{mj}且矩阵\end{CJK} $Q$ \begin{CJK}{UTF8}{mj}满足\end{CJK} $A Q A^{*}=B, r(Q)=2$, \begin{CJK}{UTF8}{mj}其中\end{CJK} $A^{*}$ \begin{CJK}{UTF8}{mj}是\end{CJK} $A$ \begin{CJK}{UTF8}{mj}的伴随矩阵\end{CJK}, \begin{CJK}{UTF8}{mj}试确定\end{CJK} $a$ \begin{CJK}{UTF8}{mj}的值\end{CJK}.

\begin{CJK}{UTF8}{mj}四\end{CJK}、(15 \begin{CJK}{UTF8}{mj}分\end{CJK}) \begin{CJK}{UTF8}{mj}设\end{CJK} $A, B$ \begin{CJK}{UTF8}{mj}是数域\end{CJK} $\mathbb{K}$ \begin{CJK}{UTF8}{mj}上的\end{CJK} $n$ \begin{CJK}{UTF8}{mj}阶方阵\end{CJK}, $x$ \begin{CJK}{UTF8}{mj}是末知量\end{CJK} $x_{1}, x_{2}, \cdots, x_{n}$ \begin{CJK}{UTF8}{mj}所构成的\end{CJK} $n \times 1$ \begin{CJK}{UTF8}{mj}矩阵\end{CJK}. \begin{CJK}{UTF8}{mj}已知齐次线性方\end{CJK} \begin{CJK}{UTF8}{mj}程组\end{CJK} $A x=0$ \begin{CJK}{UTF8}{mj}和\end{CJK} $B x=0$ \begin{CJK}{UTF8}{mj}分别有\end{CJK} $l, m$ \begin{CJK}{UTF8}{mj}个线性无关的解向量\end{CJK}, \begin{CJK}{UTF8}{mj}这里\end{CJK} $l \geqslant, m \geqslant 0$. \begin{CJK}{UTF8}{mj}证明\end{CJK}:

\begin{enumerate}
  \item \begin{CJK}{UTF8}{mj}方程组\end{CJK} $A B x=0$ \begin{CJK}{UTF8}{mj}至少有\end{CJK} $\max (l, m)$ \begin{CJK}{UTF8}{mj}个线性无关的解向量\end{CJK};
\end{enumerate}
2 . \begin{CJK}{UTF8}{mj}如果\end{CJK} $l+m>n$, \begin{CJK}{UTF8}{mj}那么\end{CJK} $(A+B) x=0$ \begin{CJK}{UTF8}{mj}必有非零解\end{CJK};

\begin{enumerate}
  \setcounter{enumi}{3}
  \item \begin{CJK}{UTF8}{mj}如果\end{CJK} $A x=0$ \begin{CJK}{UTF8}{mj}与\end{CJK} $B x=0$ \begin{CJK}{UTF8}{mj}无公共的非零解向量\end{CJK}, \begin{CJK}{UTF8}{mj}且\end{CJK} $l+m=n$, \begin{CJK}{UTF8}{mj}那么\end{CJK} $\mathbb{K}^{n}$ \begin{CJK}{UTF8}{mj}中任意一向量\end{CJK} $\alpha$ \begin{CJK}{UTF8}{mj}都可唯一地表示成\end{CJK} $\alpha=\beta+\gamma$, \begin{CJK}{UTF8}{mj}这里\end{CJK} $\beta, \gamma$ \begin{CJK}{UTF8}{mj}分别是\end{CJK} $A x=0$ \begin{CJK}{UTF8}{mj}与\end{CJK} $B x=0$ \begin{CJK}{UTF8}{mj}的解向量\end{CJK}.
\end{enumerate}
\begin{CJK}{UTF8}{mj}五\end{CJK}、 $(15$ \begin{CJK}{UTF8}{mj}分\end{CJK} $)$ \begin{CJK}{UTF8}{mj}设\end{CJK} $A=\left(a_{i j}\right)$ \begin{CJK}{UTF8}{mj}是\end{CJK} $n$ \begin{CJK}{UTF8}{mj}阶实矩阵\end{CJK}, \begin{CJK}{UTF8}{mj}且满足\end{CJK}
$$
(I) \sum_{j=1}^{n} a_{i j}=1, \forall i=1,2, \cdots, n ;(I I) a_{i j} \geqslant 0, \forall i=1,2, \cdots, n
$$
\begin{CJK}{UTF8}{mj}证明\end{CJK}:

\begin{enumerate}
  \item \begin{CJK}{UTF8}{mj}存在\end{CJK} $n \times s$ \begin{CJK}{UTF8}{mj}矩阵\end{CJK} $B \neq 0$, \begin{CJK}{UTF8}{mj}使得\end{CJK} $A B=B$, \begin{CJK}{UTF8}{mj}其中\end{CJK} $s>1$;

  \item $r(E-A)<n$, \begin{CJK}{UTF8}{mj}其中\end{CJK} $E$ \begin{CJK}{UTF8}{mj}为\end{CJK} $n$ \begin{CJK}{UTF8}{mj}阶单位矩阵\end{CJK};

  \item \begin{CJK}{UTF8}{mj}对于\end{CJK} $\lambda \in \mathbb{R}$, \begin{CJK}{UTF8}{mj}存在\end{CJK} $n$ \begin{CJK}{UTF8}{mj}维实的列向量\end{CJK} $\xi \neq 0$, \begin{CJK}{UTF8}{mj}使得\end{CJK} $A \xi=\xi$, \begin{CJK}{UTF8}{mj}则\end{CJK} $|\lambda| \leqslant 1$.

\end{enumerate}
\begin{CJK}{UTF8}{mj}六\end{CJK}、(16 \begin{CJK}{UTF8}{mj}分\end{CJK})\begin{CJK}{UTF8}{mj}设\end{CJK} $A, B$ \begin{CJK}{UTF8}{mj}都是\end{CJK} $n$ \begin{CJK}{UTF8}{mj}阶实矩阵\end{CJK}, \begin{CJK}{UTF8}{mj}且\end{CJK} $A$ \begin{CJK}{UTF8}{mj}与\end{CJK} $A-B^{T} A B$ \begin{CJK}{UTF8}{mj}都是正定矩阵\end{CJK}, \begin{CJK}{UTF8}{mj}证明\end{CJK}:

\begin{enumerate}
  \item $\operatorname{det}(E+A)>1$, \begin{CJK}{UTF8}{mj}其实\end{CJK} $E$ \begin{CJK}{UTF8}{mj}为\end{CJK} $n$ \begin{CJK}{UTF8}{mj}阶单位矩阵\end{CJK};

  \item \begin{CJK}{UTF8}{mj}如果\end{CJK} $\lambda$ \begin{CJK}{UTF8}{mj}是\end{CJK} $B$ \begin{CJK}{UTF8}{mj}的特征值\end{CJK}, \begin{CJK}{UTF8}{mj}那么\end{CJK} $|\lambda|<1$. \begin{CJK}{UTF8}{mj}七\end{CJK}、 $(14$ \begin{CJK}{UTF8}{mj}分\end{CJK})

  \item \begin{CJK}{UTF8}{mj}设\end{CJK} $n$ \begin{CJK}{UTF8}{mj}阶矩阵\end{CJK} $A$ \begin{CJK}{UTF8}{mj}和\end{CJK} $B$ \begin{CJK}{UTF8}{mj}有相同的特征多项式及最小多项式\end{CJK}, \begin{CJK}{UTF8}{mj}问\end{CJK} $A$ \begin{CJK}{UTF8}{mj}与\end{CJK} $B$ \begin{CJK}{UTF8}{mj}是否相似\end{CJK}? \begin{CJK}{UTF8}{mj}若是\end{CJK}, \begin{CJK}{UTF8}{mj}则给孕证明\end{CJK}; \begin{CJK}{UTF8}{mj}若不是\end{CJK}, \begin{CJK}{UTF8}{mj}则给出反例\end{CJK};

  \item \begin{CJK}{UTF8}{mj}设\end{CJK} $A, B \in M_{3}$ ( $\left.\mathbb{C}\right)$ \begin{CJK}{UTF8}{mj}都只有一个特征值\end{CJK} $\lambda_{0}$. \begin{CJK}{UTF8}{mj}证明\end{CJK}: $A$ \begin{CJK}{UTF8}{mj}与\end{CJK} $B$ \begin{CJK}{UTF8}{mj}相似的充分必要条件是\end{CJK} $\operatorname{dim} V_{\lambda_{0}}(A)=\operatorname{dim} V_{\lambda_{0}}(B)$, \begin{CJK}{UTF8}{mj}这里\end{CJK} $V_{\lambda_{0}(A)}, V_{\lambda_{0}(B)}$ \begin{CJK}{UTF8}{mj}分别表示\end{CJK} $A, B$ \begin{CJK}{UTF8}{mj}的属于\end{CJK} $\lambda_{0}$ \begin{CJK}{UTF8}{mj}的特征子空间\end{CJK}.

\end{enumerate}
\begin{CJK}{UTF8}{mj}八\end{CJK}、 $(14$ \begin{CJK}{UTF8}{mj}分\end{CJK}) \begin{CJK}{UTF8}{mj}设\end{CJK} $\sigma$ \begin{CJK}{UTF8}{mj}是欧式空间\end{CJK} $V$ \begin{CJK}{UTF8}{mj}的线性空间\end{CJK}, $\tau$ \begin{CJK}{UTF8}{mj}是\end{CJK} $V$ \begin{CJK}{UTF8}{mj}的一个变换\end{CJK}, \begin{CJK}{UTF8}{mj}且\end{CJK} $\forall \alpha, \beta \in V$ \begin{CJK}{UTF8}{mj}都有\end{CJK} $(\sigma(\alpha), \beta)=(\alpha, \tau(\beta))$. \begin{CJK}{UTF8}{mj}证明\end{CJK}:

\begin{enumerate}
  \item $\tau$ \begin{CJK}{UTF8}{mj}是\end{CJK} $V$ \begin{CJK}{UTF8}{mj}的一个线性变换\end{CJK};

  \item $\tau$ \begin{CJK}{UTF8}{mj}的值域\end{CJK} $\operatorname{Im} \tau$ \begin{CJK}{UTF8}{mj}等于\end{CJK} $\sigma$ \begin{CJK}{UTF8}{mj}的核\end{CJK} $\operatorname{ker} \sigma$ \begin{CJK}{UTF8}{mj}的正交补\end{CJK}.

\end{enumerate}
\begin{CJK}{UTF8}{mj}九\end{CJK}、 (15 \begin{CJK}{UTF8}{mj}分\end{CJK}) \begin{CJK}{UTF8}{mj}已知线性空间\end{CJK} $M_{2}(\mathbb{K})$ \begin{CJK}{UTF8}{mj}的线性变换\end{CJK}
$$
A(X)=B^{T} X-X^{T} B, \forall X \in M_{2}(\mathbb{K})
$$
\begin{CJK}{UTF8}{mj}其中\end{CJK} $B=\left(\begin{array}{ll}1 & 1 \\ 0 & 1\end{array}\right)$, \begin{CJK}{UTF8}{mj}和线性子空间\end{CJK}
$$
W=\left\{\left(\begin{array}{ll}
x_{11} & x_{12} \\
x_{21} & x_{22}
\end{array}\right) \mid x_{11}+x_{21}=0, x_{i j} \in \mathbb{K}\right\}
$$

\begin{enumerate}
  \item \begin{CJK}{UTF8}{mj}求\end{CJK} $W$ \begin{CJK}{UTF8}{mj}的一个基\end{CJK};

  \item \begin{CJK}{UTF8}{mj}证明\end{CJK}: $W$ \begin{CJK}{UTF8}{mj}是\end{CJK} $A$ \begin{CJK}{UTF8}{mj}的不变子空间\end{CJK};

  \item \begin{CJK}{UTF8}{mj}将\end{CJK} $A$ \begin{CJK}{UTF8}{mj}看成\end{CJK} $W$ \begin{CJK}{UTF8}{mj}上的线性变换\end{CJK}, \begin{CJK}{UTF8}{mj}求\end{CJK} $W$ \begin{CJK}{UTF8}{mj}的一个基\end{CJK}, \begin{CJK}{UTF8}{mj}使得\end{CJK} $A$ \begin{CJK}{UTF8}{mj}在该基下的矩阵为对角矩阵\end{CJK}.

\end{enumerate}
\begin{CJK}{UTF8}{mj}十\end{CJK}、 ( 20 \begin{CJK}{UTF8}{mj}分\end{CJK}) \begin{CJK}{UTF8}{mj}设\end{CJK} $f: \mathbb{R}^{n \times n} \mapsto \mathbb{R}$ \begin{CJK}{UTF8}{mj}是由\end{CJK} $\mathbb{R}^{n \times n}$ \begin{CJK}{UTF8}{mj}到实数域\end{CJK} $\mathbb{R}$ \begin{CJK}{UTF8}{mj}的线性映射\end{CJK}.

\begin{enumerate}
  \item \begin{CJK}{UTF8}{mj}给出\end{CJK} $\mathbb{R}^{n \times n}$ \begin{CJK}{UTF8}{mj}的一个基\end{CJK}, \begin{CJK}{UTF8}{mj}使得\end{CJK} $\mathbb{R}^{n \times n}$ \begin{CJK}{UTF8}{mj}中的任一矩阵\end{CJK} $A$ \begin{CJK}{UTF8}{mj}在这个基下的坐标恰好就是\end{CJK} $A$ \begin{CJK}{UTF8}{mj}的元素\end{CJK};

  \item \begin{CJK}{UTF8}{mj}证明\end{CJK}: \begin{CJK}{UTF8}{mj}存在唯一的\end{CJK} $C \in \mathbb{R}^{n \times n}$, \begin{CJK}{UTF8}{mj}使得\end{CJK} $f(A)=\operatorname{Tr}(A C), \forall A \in \mathbb{R}^{n \times n}$;

  \item \begin{CJK}{UTF8}{mj}证明\end{CJK}: \begin{CJK}{UTF8}{mj}若\end{CJK} $\forall A, B \in \mathbb{R}^{n \times n}, f(A B)=f(B A)$, \begin{CJK}{UTF8}{mj}则存在\end{CJK} $\lambda \in \mathbb{R}$ \begin{CJK}{UTF8}{mj}使得\end{CJK}

\end{enumerate}
$$
f(A)=\lambda \operatorname{Tr}(A), \forall A \in \mathbb{R}^{n \times n}
$$
\begin{CJK}{UTF8}{mj}这里\end{CJK} $\operatorname{Tr}(A)$ \begin{CJK}{UTF8}{mj}是矩阵\end{CJK} $A$ \begin{CJK}{UTF8}{mj}的迹\end{CJK}, \begin{CJK}{UTF8}{mj}即\end{CJK} $A=\left(a_{i j}\right)$ \begin{CJK}{UTF8}{mj}的对角元之和\end{CJK} $\operatorname{Tr}(A)=\sum_{i=1}^{n} a_{i i}$.

\section{1. 武汉大学 2010 年研究生入学考试试题高等代数}
\begin{CJK}{UTF8}{mj}李扬\end{CJK}

\begin{CJK}{UTF8}{mj}微信公众号\end{CJK}: sxkyliyang

\begin{CJK}{UTF8}{mj}一\end{CJK}、 (15 \begin{CJK}{UTF8}{mj}分\end{CJK}) \begin{CJK}{UTF8}{mj}已知矩阵\end{CJK}
$$
A=\left(\begin{array}{lll}
1 & 1 & 1 \\
0 & 2 & 0 \\
1 & 0 & 3
\end{array}\right), B=\left(\begin{array}{lll}
1 & 2 & 3 \\
2 & 3 & 1 \\
3 & 1 & 2
\end{array}\right),
$$
\begin{CJK}{UTF8}{mj}试求\end{CJK} $X$, \begin{CJK}{UTF8}{mj}使得\end{CJK} $A^{T} X-3 X=B$.

\begin{CJK}{UTF8}{mj}二\end{CJK}、 (16 \begin{CJK}{UTF8}{mj}分\end{CJK}) \begin{CJK}{UTF8}{mj}设\end{CJK} $A, B$ \begin{CJK}{UTF8}{mj}是\end{CJK} $n$ \begin{CJK}{UTF8}{mj}阶矩阵\end{CJK} $(n \geqslant 2), A^{*}$ \begin{CJK}{UTF8}{mj}与\end{CJK} $B^{*}$ \begin{CJK}{UTF8}{mj}分别是\end{CJK} $A, B$ \begin{CJK}{UTF8}{mj}的伴随矩阵\end{CJK}. \begin{CJK}{UTF8}{mj}已知\end{CJK} $B$ \begin{CJK}{UTF8}{mj}是交换\end{CJK} $A$ \begin{CJK}{UTF8}{mj}的第\end{CJK} 1 \begin{CJK}{UTF8}{mj}行与第\end{CJK} 2 \begin{CJK}{UTF8}{mj}行得到的矩阵\end{CJK}. \begin{CJK}{UTF8}{mj}对于下述四项\end{CJK}, \begin{CJK}{UTF8}{mj}若正确则给予证明\end{CJK}; \begin{CJK}{UTF8}{mj}若不正确请给出反例\end{CJK}.\\
(A) \begin{CJK}{UTF8}{mj}交换\end{CJK} $A^{*}$ \begin{CJK}{UTF8}{mj}的第\end{CJK} 1 \begin{CJK}{UTF8}{mj}列与第\end{CJK} 2 \begin{CJK}{UTF8}{mj}列得到\end{CJK} $B^{*}$;\\
(B) \begin{CJK}{UTF8}{mj}交换\end{CJK} $A^{*}$ \begin{CJK}{UTF8}{mj}的第\end{CJK} 1 \begin{CJK}{UTF8}{mj}行与第\end{CJK} 2 \begin{CJK}{UTF8}{mj}行得到\end{CJK} $B^{*}$;\\
(C) \begin{CJK}{UTF8}{mj}交换\end{CJK} $A^{*}$ \begin{CJK}{UTF8}{mj}的第\end{CJK} 1 \begin{CJK}{UTF8}{mj}列与第\end{CJK} 2 \begin{CJK}{UTF8}{mj}列得到\end{CJK} $-B^{*}$;\\
(D) \begin{CJK}{UTF8}{mj}交换\end{CJK} $A^{*}$ \begin{CJK}{UTF8}{mj}的第\end{CJK} 1 \begin{CJK}{UTF8}{mj}行与第\end{CJK} 2 \begin{CJK}{UTF8}{mj}行得到\end{CJK} $-B^{*}$;

\begin{CJK}{UTF8}{mj}三\end{CJK}、 ( 15 \begin{CJK}{UTF8}{mj}分\end{CJK}) \begin{CJK}{UTF8}{mj}设\end{CJK} $A$ \begin{CJK}{UTF8}{mj}是\end{CJK} $n$ \begin{CJK}{UTF8}{mj}阶反对称矩阵\end{CJK}, $b$ \begin{CJK}{UTF8}{mj}是\end{CJK} $n$ \begin{CJK}{UTF8}{mj}维列向量\end{CJK}, $r(A)=r(A, b)$. \begin{CJK}{UTF8}{mj}求证\end{CJK}:
$$
r\left(\begin{array}{cc}
A & b \\
-b^{T} & 0
\end{array}\right)=r(A)
$$
\begin{CJK}{UTF8}{mj}四\end{CJK}、(15 \begin{CJK}{UTF8}{mj}分\end{CJK}) \begin{CJK}{UTF8}{mj}设\end{CJK} $M_{2}(\mathbb{R})$ \begin{CJK}{UTF8}{mj}表示实数域\end{CJK} $\mathbb{R}$ \begin{CJK}{UTF8}{mj}上全体二阶方阵构成的线性空间\end{CJK}, \begin{CJK}{UTF8}{mj}矩阵\end{CJK}
$$
\eta_{1}=\left(\begin{array}{ll}
1 & 0 \\
0 & 0
\end{array}\right), \eta_{2}=\left(\begin{array}{ll}
1 & 1 \\
0 & 0
\end{array}\right), \eta_{3}=\left(\begin{array}{ll}
1 & 1 \\
1 & 0
\end{array}\right), \eta_{4}=\left(\begin{array}{cc}
1 & 1 \\
1 & 1
\end{array}\right)
$$
\begin{CJK}{UTF8}{mj}是\end{CJK} $M_{2}(\mathbb{R})$ \begin{CJK}{UTF8}{mj}的一个基\end{CJK}, \begin{CJK}{UTF8}{mj}又设\end{CJK}
$$
\xi_{1}=\left(\begin{array}{ll}
1 & 0 \\
3 & 0
\end{array}\right), \xi_{2}=\left(\begin{array}{ll}
1 & 1 \\
3 & 3
\end{array}\right), \xi_{3}=\left(\begin{array}{ll}
3 & 1 \\
7 & 3
\end{array}\right), \xi_{4}=\left(\begin{array}{ll}
3 & 3 \\
7 & 7
\end{array}\right)
$$
\begin{CJK}{UTF8}{mj}已知\end{CJK} $\sigma$ \begin{CJK}{UTF8}{mj}是\end{CJK} $M_{2}(\mathbb{R})$ \begin{CJK}{UTF8}{mj}的一个线性变换\end{CJK}, $\sigma\left(\eta_{i}\right)=\xi_{i}(i=1,2,3,4)$.

\begin{enumerate}
  \item \begin{CJK}{UTF8}{mj}求\end{CJK} $\sigma\left(\xi_{1}\right), \sigma\left(\xi_{2}\right), \sigma\left(\xi_{3}\right), \sigma\left(\xi_{4}\right)$;

  \item \begin{CJK}{UTF8}{mj}问\end{CJK} $\sigma\left(\xi_{1}\right), \sigma\left(\xi_{2}\right), \sigma\left(\xi_{3}\right), \sigma\left(\xi_{4}\right)$ \begin{CJK}{UTF8}{mj}能否构成\end{CJK} $M_{2}(\mathbb{R})$ \begin{CJK}{UTF8}{mj}的一个基\end{CJK}? \begin{CJK}{UTF8}{mj}请阐述理由\end{CJK}.

\end{enumerate}
\begin{CJK}{UTF8}{mj}五\end{CJK}、(15 \begin{CJK}{UTF8}{mj}分\end{CJK}) \begin{CJK}{UTF8}{mj}设\end{CJK} 3 \begin{CJK}{UTF8}{mj}阶实对称矩阵\end{CJK} $A$ \begin{CJK}{UTF8}{mj}的各行元素之和均为\end{CJK} 3 , \begin{CJK}{UTF8}{mj}向量\end{CJK} $\alpha_{1}=(-1,2,-1)^{T}, \alpha_{2}=(0,-1,-1)^{T}$ \begin{CJK}{UTF8}{mj}是线性方\end{CJK} \begin{CJK}{UTF8}{mj}程组\end{CJK} $A x=0$ \begin{CJK}{UTF8}{mj}的两个解\end{CJK}.

\begin{enumerate}
  \item \begin{CJK}{UTF8}{mj}求\end{CJK} $A$ \begin{CJK}{UTF8}{mj}的特征值与特征向量\end{CJK};

  \item \begin{CJK}{UTF8}{mj}求正交矩阵\end{CJK} $Q$ \begin{CJK}{UTF8}{mj}和对角矩阵\end{CJK} $D$, \begin{CJK}{UTF8}{mj}使得\end{CJK} $Q^{T} A Q=D$;

  \item \begin{CJK}{UTF8}{mj}求行列式\end{CJK}

\end{enumerate}
$$
\left|\left(\frac{2}{3} B^{2}\right)^{-1}+\frac{4}{9} B^{*}+B\right|
$$
\begin{CJK}{UTF8}{mj}其中\end{CJK} $B$ \begin{CJK}{UTF8}{mj}是\end{CJK} $A-\frac{3}{2} E$ \begin{CJK}{UTF8}{mj}的相似矩阵\end{CJK}, $B^{*}$ \begin{CJK}{UTF8}{mj}是\end{CJK} $B$ \begin{CJK}{UTF8}{mj}的伴随矩阵\end{CJK}. \begin{CJK}{UTF8}{mj}六\end{CJK}、 (16 \begin{CJK}{UTF8}{mj}分\end{CJK})\begin{CJK}{UTF8}{mj}已知非齐次线性方程组\end{CJK}
$$
(I):\left\{\begin{array}{l}
x_{1}+x_{2}+x_{3}+x_{4}=-1 \\
3 x_{1}+2 x_{2}+4 x_{3}-x_{4}=0 \\
5 x_{1}+3 x_{2}+7 x_{3}-3 x_{4}=1 \\
a x_{1}+x_{2}+5 x_{3}+b x_{4}=3
\end{array}\right.
$$
\begin{CJK}{UTF8}{mj}有三个线性无关的解\end{CJK}.

\begin{enumerate}
  \item \begin{CJK}{UTF8}{mj}记方程组\end{CJK} $(I)$ \begin{CJK}{UTF8}{mj}的系数矩阵为\end{CJK} $A$, \begin{CJK}{UTF8}{mj}证明\end{CJK}: $r(A)=2$;

  \item \begin{CJK}{UTF8}{mj}求\end{CJK} $a, b$ \begin{CJK}{UTF8}{mj}的值\end{CJK};

  \item \begin{CJK}{UTF8}{mj}求方程组\end{CJK} (I) \begin{CJK}{UTF8}{mj}的通解\end{CJK}.

\end{enumerate}
\begin{CJK}{UTF8}{mj}七\end{CJK}、(14 \begin{CJK}{UTF8}{mj}分\end{CJK})\begin{CJK}{UTF8}{mj}设\end{CJK} $V$ \begin{CJK}{UTF8}{mj}是数域\end{CJK} $\mathbb{F}$ \begin{CJK}{UTF8}{mj}上的\end{CJK} $n$ \begin{CJK}{UTF8}{mj}维线性空间\end{CJK}, $\sigma$ \begin{CJK}{UTF8}{mj}是\end{CJK} $V$ \begin{CJK}{UTF8}{mj}的线性变换\end{CJK}, \begin{CJK}{UTF8}{mj}且存在\end{CJK} $\xi \in V$ \begin{CJK}{UTF8}{mj}使得\end{CJK} $\xi, \sigma(\xi), \cdots, \sigma^{n-1}(\xi)$ \begin{CJK}{UTF8}{mj}构\end{CJK} \begin{CJK}{UTF8}{mj}成\end{CJK} $V$ \begin{CJK}{UTF8}{mj}的一个基\end{CJK}, \begin{CJK}{UTF8}{mj}试求\end{CJK} $\sigma$ \begin{CJK}{UTF8}{mj}的特征多项式和最小多项式\end{CJK}.

\begin{CJK}{UTF8}{mj}八\end{CJK}、 $\left(14\right.$ \begin{CJK}{UTF8}{mj}分\end{CJK}) \begin{CJK}{UTF8}{mj}设\end{CJK} $f$ \begin{CJK}{UTF8}{mj}是平面\end{CJK} $\mathbb{R}^{2}$ \begin{CJK}{UTF8}{mj}上的线性变换\end{CJK}, \begin{CJK}{UTF8}{mj}使得\end{CJK}

\begin{enumerate}
  \item \begin{CJK}{UTF8}{mj}点\end{CJK} $(1,0)$ \begin{CJK}{UTF8}{mj}的像位于第\end{CJK} 4 \begin{CJK}{UTF8}{mj}象限\end{CJK};

  \item \begin{CJK}{UTF8}{mj}点\end{CJK} $(0,1)$ \begin{CJK}{UTF8}{mj}的像位于第\end{CJK} 2 \begin{CJK}{UTF8}{mj}象限\end{CJK};

  \item \begin{CJK}{UTF8}{mj}点\end{CJK} $(1,1)$ \begin{CJK}{UTF8}{mj}的像位于第\end{CJK} 1 \begin{CJK}{UTF8}{mj}象限\end{CJK}.

\end{enumerate}
\begin{CJK}{UTF8}{mj}证明\end{CJK}: $f$ \begin{CJK}{UTF8}{mj}是可逆变换\end{CJK}, \begin{CJK}{UTF8}{mj}且\end{CJK} $f^{-1}$ \begin{CJK}{UTF8}{mj}把第\end{CJK} 1 \begin{CJK}{UTF8}{mj}象限内的任意点都映射到底\end{CJK} 1 \begin{CJK}{UTF8}{mj}象限内\end{CJK}.

\begin{CJK}{UTF8}{mj}九\end{CJK}、 ( 15 \begin{CJK}{UTF8}{mj}分\end{CJK}) \begin{CJK}{UTF8}{mj}设\end{CJK} $A$ \begin{CJK}{UTF8}{mj}为\end{CJK} $n$ \begin{CJK}{UTF8}{mj}阶矩阵\end{CJK}, \begin{CJK}{UTF8}{mj}证明下述命题相互等价\end{CJK}:

\begin{enumerate}
  \item $r(A)=r\left(A^{2}\right)$;

  \item \begin{CJK}{UTF8}{mj}存在可逆矩阵\end{CJK} $P$ \begin{CJK}{UTF8}{mj}与\end{CJK} $B$, \begin{CJK}{UTF8}{mj}使得\end{CJK} $A=P\left(\begin{array}{cc}B & O \\ O & O\end{array}\right) P^{-1}$, \begin{CJK}{UTF8}{mj}其中\end{CJK} $O$ \begin{CJK}{UTF8}{mj}是零矩阵\end{CJK};

  \item \begin{CJK}{UTF8}{mj}存在可逆矩阵\end{CJK} $C$, \begin{CJK}{UTF8}{mj}使得\end{CJK} $A=A^{2} C$.

\end{enumerate}
\begin{CJK}{UTF8}{mj}十\end{CJK}、(15 \begin{CJK}{UTF8}{mj}分\end{CJK}) \begin{CJK}{UTF8}{mj}设\end{CJK} $V$ \begin{CJK}{UTF8}{mj}是数域\end{CJK} $\mathbb{F}$ \begin{CJK}{UTF8}{mj}上的\end{CJK} $n$ \begin{CJK}{UTF8}{mj}维线性空间\end{CJK}, $W_{1}, W_{2}, \cdots, W_{s}$ \begin{CJK}{UTF8}{mj}为\end{CJK} $V$ \begin{CJK}{UTF8}{mj}的\end{CJK} $s$ \begin{CJK}{UTF8}{mj}个子空间\end{CJK},
$$
W=W_{1} \cup W_{2} \cup \cdots \cup W_{s}
$$
\begin{CJK}{UTF8}{mj}证明\end{CJK}: $W$ \begin{CJK}{UTF8}{mj}为\end{CJK} $V$ \begin{CJK}{UTF8}{mj}的子空间的充分必要条件是存在某个\end{CJK} $i$ \begin{CJK}{UTF8}{mj}使得\end{CJK} $W=W_{i}$.

\section{2. 武汉大学 2011 年研究生入学考试试题高等代数 
 李扬 
 微信公众号: sxkyliyang}
\begin{CJK}{UTF8}{mj}一\end{CJK}、 (16 \begin{CJK}{UTF8}{mj}分\end{CJK}) \begin{CJK}{UTF8}{mj}设\end{CJK} $A, B$ \begin{CJK}{UTF8}{mj}为\end{CJK} $n$ \begin{CJK}{UTF8}{mj}阶矩阵\end{CJK}, $A^{*}, B^{*}$ \begin{CJK}{UTF8}{mj}分别为\end{CJK} $A, B$ \begin{CJK}{UTF8}{mj}的伴随矩阵\end{CJK}, \begin{CJK}{UTF8}{mj}又设分块矩阵\end{CJK} $M=\left(\begin{array}{cc}A & 0 \\ 0 & B\end{array}\right)$. \begin{CJK}{UTF8}{mj}对于下述\end{CJK} \begin{CJK}{UTF8}{mj}四个选项\end{CJK}, \begin{CJK}{UTF8}{mj}若正确则给子证明\end{CJK}; \begin{CJK}{UTF8}{mj}若不正确请给出反例\end{CJK}.\\
(A) $M^{*}=\left(\begin{array}{cc}|A| A^{*} & 0 \\ 0 & |B| B^{*}\end{array}\right)$;\\
(B) $M^{*}=\left(\begin{array}{cc}|B| B^{*} & 0 \\ 0 & |A| A^{*}\end{array}\right)$;\\
$(\mathrm{C}) M^{*}=\left(\begin{array}{cc}|A| B^{*} & 0 \\ 0 & |B| A^{*}\end{array}\right)$\\
$(\mathrm{D}) M^{*}=\left(\begin{array}{cc}|B| A^{*} & 0 \\ 0 & |A| B^{*}\end{array}\right)$

\begin{CJK}{UTF8}{mj}二\end{CJK}、 (15 \begin{CJK}{UTF8}{mj}分\end{CJK}) \begin{CJK}{UTF8}{mj}设\end{CJK} $n$ \begin{CJK}{UTF8}{mj}元线性方程组\end{CJK} $A x=b$, \begin{CJK}{UTF8}{mj}其中\end{CJK}
$$
r\left(\begin{array}{cccc}
2 a & 1 & & \\
a^{2} & 2 a & \ddots & \\
& \ddots & \ddots & 1 \\
& & a^{2} & 2 a
\end{array}\right), x=\left(\begin{array}{c}
x_{1} \\
x_{2} \\
\vdots \\
x_{n}
\end{array}\right), b=\left(\begin{array}{c}
1 \\
0 \\
\vdots \\
0
\end{array}\right)
$$

\begin{enumerate}
  \item \begin{CJK}{UTF8}{mj}证明行列式\end{CJK} $|A|=(n+1) a^{n}$;

  \item \begin{CJK}{UTF8}{mj}当\end{CJK} $a$ \begin{CJK}{UTF8}{mj}为何值时\end{CJK}, \begin{CJK}{UTF8}{mj}该方程组有唯一解\end{CJK}, \begin{CJK}{UTF8}{mj}并求\end{CJK} $x_{1}$;

  \item \begin{CJK}{UTF8}{mj}当\end{CJK} $a$ \begin{CJK}{UTF8}{mj}为何值时\end{CJK}, \begin{CJK}{UTF8}{mj}该方程组有无穷多解\end{CJK}, \begin{CJK}{UTF8}{mj}并求通解\end{CJK}.

\end{enumerate}
\begin{CJK}{UTF8}{mj}三\end{CJK}、( 15 \begin{CJK}{UTF8}{mj}分\end{CJK}) \begin{CJK}{UTF8}{mj}设在\end{CJK} $n$ \begin{CJK}{UTF8}{mj}维欧式空间\end{CJK} $V$, \begin{CJK}{UTF8}{mj}向量\end{CJK} $\alpha, \beta$ \begin{CJK}{UTF8}{mj}的内积记为\end{CJK} $(\alpha, \beta) ; T$ \begin{CJK}{UTF8}{mj}为\end{CJK} $V$ \begin{CJK}{UTF8}{mj}的线性变换\end{CJK}, \begin{CJK}{UTF8}{mj}对于\end{CJK} $\alpha, \beta \in V$ : \begin{CJK}{UTF8}{mj}定义二元\end{CJK} \begin{CJK}{UTF8}{mj}函数\end{CJK} $f(\alpha, \beta)=(T(\alpha), T(\beta))$. \begin{CJK}{UTF8}{mj}问\end{CJK} $f(\alpha, \beta)$ \begin{CJK}{UTF8}{mj}是否为\end{CJK} $V$ \begin{CJK}{UTF8}{mj}的内积\end{CJK}? \begin{CJK}{UTF8}{mj}请阐述理由\end{CJK}.

\begin{CJK}{UTF8}{mj}四\end{CJK}、 (14 \begin{CJK}{UTF8}{mj}分\end{CJK}) \begin{CJK}{UTF8}{mj}设\end{CJK} $A$ \begin{CJK}{UTF8}{mj}是\end{CJK} $n$ \begin{CJK}{UTF8}{mj}阶矩阵\end{CJK}, \begin{CJK}{UTF8}{mj}证明\end{CJK}:

\begin{enumerate}
  \item \begin{CJK}{UTF8}{mj}如果\end{CJK} $A^{k-1} \alpha \neq 0$, \begin{CJK}{UTF8}{mj}但\end{CJK} $A^{k} \alpha=0$, \begin{CJK}{UTF8}{mj}那么\end{CJK} $\alpha, A \alpha, \cdots, A^{k-1} \alpha(k>0)$ \begin{CJK}{UTF8}{mj}线性无关\end{CJK};

  \item $r\left(A^{n}\right)=r\left(A^{n+1}\right)$.

\end{enumerate}
\begin{CJK}{UTF8}{mj}五\end{CJK}、(15 \begin{CJK}{UTF8}{mj}分\end{CJK})\begin{CJK}{UTF8}{mj}已知\end{CJK} $m$ \begin{CJK}{UTF8}{mj}个向量\end{CJK} $\alpha_{1}, \alpha_{2}, \cdots, \alpha_{m}$ \begin{CJK}{UTF8}{mj}线性相关\end{CJK}, \begin{CJK}{UTF8}{mj}但其中任意\end{CJK} $m-1$ \begin{CJK}{UTF8}{mj}个向量都线性无关\end{CJK}. \begin{CJK}{UTF8}{mj}证明\end{CJK}:

\begin{enumerate}
  \item \begin{CJK}{UTF8}{mj}若\end{CJK} $k_{1} \alpha+k_{2} \alpha_{2}+\cdots+k_{m} \alpha_{m}=0$, \begin{CJK}{UTF8}{mj}则系数\end{CJK} $k_{1}, k_{2}, \cdots, k_{m}$ \begin{CJK}{UTF8}{mj}或者全为零\end{CJK}, \begin{CJK}{UTF8}{mj}或全不为零\end{CJK};

  \item \begin{CJK}{UTF8}{mj}若\end{CJK} $a_{1} \alpha+a_{2} \alpha_{2}+\cdots+a_{m} \alpha_{m}=0$, \begin{CJK}{UTF8}{mj}且\end{CJK} $b_{1} \alpha+b_{2} \alpha_{2}+\cdots+b_{m} \alpha_{m}=0$, \begin{CJK}{UTF8}{mj}其中\end{CJK} $b_{1} \neq 0$, \begin{CJK}{UTF8}{mj}则\end{CJK}

\end{enumerate}
$$
\frac{a_{1}}{b_{1}}=\frac{a_{2}}{b_{2}}=\cdots=\frac{a_{m}}{b_{m}}
$$
\begin{CJK}{UTF8}{mj}六\end{CJK}、(15 \begin{CJK}{UTF8}{mj}分\end{CJK})\begin{CJK}{UTF8}{mj}设\end{CJK} $n$ \begin{CJK}{UTF8}{mj}元实二次型\end{CJK} $f(x)=x^{T} A x$ \begin{CJK}{UTF8}{mj}的秩为\end{CJK} $n$, \begin{CJK}{UTF8}{mj}正\end{CJK}、\begin{CJK}{UTF8}{mj}负惯性指数分别为\end{CJK} $p, q$, \begin{CJK}{UTF8}{mj}且\end{CJK} $p \geqslant q>0$.

\begin{enumerate}
  \item \begin{CJK}{UTF8}{mj}证明存在\end{CJK} $\mathbb{R}^{n}$ \begin{CJK}{UTF8}{mj}的一个\end{CJK} $q$ \begin{CJK}{UTF8}{mj}维子空间\end{CJK} $W$, \begin{CJK}{UTF8}{mj}使\end{CJK} $\forall x_{0} \in W, f\left(x_{0}\right)=0$;

  \item \begin{CJK}{UTF8}{mj}令\end{CJK} $T=\left\{x \in \mathbb{R}^{n} \mid f(x)=0\right\}$, \begin{CJK}{UTF8}{mj}问\end{CJK} $T$ \begin{CJK}{UTF8}{mj}与\end{CJK} $W$ \begin{CJK}{UTF8}{mj}是否相等\end{CJK}? \begin{CJK}{UTF8}{mj}为什么\end{CJK}?

\end{enumerate}
\begin{CJK}{UTF8}{mj}七\end{CJK}、 ( 14 \begin{CJK}{UTF8}{mj}分\end{CJK}) \begin{CJK}{UTF8}{mj}设\end{CJK} $A$ \begin{CJK}{UTF8}{mj}是\end{CJK} $n$ \begin{CJK}{UTF8}{mj}阶实矩阵\end{CJK}, \begin{CJK}{UTF8}{mj}且\end{CJK} $\forall \alpha(\neq 0) \in \mathbb{R}^{n \times 1}$ \begin{CJK}{UTF8}{mj}均有\end{CJK} $\alpha^{T} A \alpha>0$, \begin{CJK}{UTF8}{mj}求证\end{CJK}: $\operatorname{det} A>0$. \begin{CJK}{UTF8}{mj}八\end{CJK}、 $\left(16\right.$ \begin{CJK}{UTF8}{mj}分\end{CJK}) \begin{CJK}{UTF8}{mj}设\end{CJK} $V$ \begin{CJK}{UTF8}{mj}是\end{CJK} $n$ \begin{CJK}{UTF8}{mj}维线性空间\end{CJK}, $V_{1}, V_{2}, \cdots, V_{s}(s>1)$ \begin{CJK}{UTF8}{mj}是\end{CJK} $V$ \begin{CJK}{UTF8}{mj}的子空间\end{CJK}, \begin{CJK}{UTF8}{mj}且\end{CJK}
$$
V=V_{1} \oplus V_{2} \oplus \cdots \oplus V_{s}
$$
\begin{CJK}{UTF8}{mj}证明\end{CJK}: \begin{CJK}{UTF8}{mj}存在\end{CJK} $V$ \begin{CJK}{UTF8}{mj}上的线性变换\end{CJK} $\sigma_{1}, \sigma_{2}, \cdots, \sigma_{s}$, \begin{CJK}{UTF8}{mj}使得\end{CJK}

\begin{enumerate}
  \item $\sigma_{i}^{2}=\sigma_{i}(1 \leqslant i \leqslant s)$;

  \item $\sigma_{i} \sigma_{j}=0(1 \leqslant i, j \leqslant s$ and $i \neq j)$;

  \item $\sigma_{1}+\sigma_{2}+\cdots+\sigma_{s}=\varepsilon$ \begin{CJK}{UTF8}{mj}是恒等变换\end{CJK};

  \item $\sigma_{i}(V)=V_{i}(1 \leqslant i \leqslant s)$.

\end{enumerate}
\begin{CJK}{UTF8}{mj}九\end{CJK}、 $(15$ \begin{CJK}{UTF8}{mj}分\end{CJK})\begin{CJK}{UTF8}{mj}设\end{CJK} 6 \begin{CJK}{UTF8}{mj}阶实矩阵\end{CJK}
$$
A=\left(\begin{array}{cccccc}
a & -b & & & & \\
b & a & 1 & & & \\
& & a & -b & & \\
& & b & a & 1 & \\
& & & & a & -b \\
& & & & b & a
\end{array}\right)
$$
\begin{CJK}{UTF8}{mj}其中\end{CJK} $b \neq 0$, \begin{CJK}{UTF8}{mj}试求\end{CJK} $A$ \begin{CJK}{UTF8}{mj}的不变因子和初等因子\end{CJK}, \begin{CJK}{UTF8}{mj}并写出\end{CJK} $A$ \begin{CJK}{UTF8}{mj}的\end{CJK} Jordan \begin{CJK}{UTF8}{mj}标准形\end{CJK}.

\begin{CJK}{UTF8}{mj}十\end{CJK}、 (15 \begin{CJK}{UTF8}{mj}分\end{CJK})\begin{CJK}{UTF8}{mj}设\end{CJK} $f(\lambda), m(\lambda)$ \begin{CJK}{UTF8}{mj}分别为数域\end{CJK} $\mathbb{F}$ \begin{CJK}{UTF8}{mj}上矩阵\end{CJK} $A$ \begin{CJK}{UTF8}{mj}的特征多项式和最小多项式\end{CJK}, \begin{CJK}{UTF8}{mj}且有标准分解式\end{CJK}
$$
f(\lambda)=p_{1}(\lambda)^{r_{1}} p_{2}(\lambda)^{r_{2}} \cdots p_{t}(\lambda)^{r_{t}}, m(\lambda)=p_{1}(\lambda)^{s_{1}} p_{2}(\lambda)^{s_{2}} \cdots p_{t}(\lambda)^{s_{t}}
$$
\begin{CJK}{UTF8}{mj}其中\end{CJK} $r_{i} \geqslant 0, s_{i} \geqslant 0$, \begin{CJK}{UTF8}{mj}而\end{CJK} $p_{i}(\lambda)$ \begin{CJK}{UTF8}{mj}是数域\end{CJK} $\mathbb{F}$ \begin{CJK}{UTF8}{mj}上的不可约多项式\end{CJK}, $i=1,2, \cdots, t$.

\begin{CJK}{UTF8}{mj}求证\end{CJK}: \begin{CJK}{UTF8}{mj}矩阵\end{CJK} $p_{i}(A)^{s_{i}}$ \begin{CJK}{UTF8}{mj}的零度即它的核空间的维数等于\end{CJK} $r_{i} \operatorname{deg} p_{i}(\lambda)$.

\section{3. 武汉大学 2012 年研究生入学考试试题高等代数}
\begin{CJK}{UTF8}{mj}李扬\end{CJK}

\begin{CJK}{UTF8}{mj}微信公众号\end{CJK}: sxkyliyang

\begin{CJK}{UTF8}{mj}一\end{CJK}、 (15 \begin{CJK}{UTF8}{mj}分\end{CJK}) \begin{CJK}{UTF8}{mj}设\end{CJK} $A, B$ \begin{CJK}{UTF8}{mj}为同阶方阵\end{CJK}, \begin{CJK}{UTF8}{mj}证明\end{CJK}:
$$
r(A B-I) \leqslant r(A-I)+r(B-I),
$$
\begin{CJK}{UTF8}{mj}这里\end{CJK} $I$ \begin{CJK}{UTF8}{mj}为单位矩阵\end{CJK}.

\begin{CJK}{UTF8}{mj}二\end{CJK}、 (16 \begin{CJK}{UTF8}{mj}分\end{CJK}) \begin{CJK}{UTF8}{mj}设\end{CJK} $A \in \mathbb{R}^{m \times n}, B \in \mathbb{R}^{n \times m}$, \begin{CJK}{UTF8}{mj}试证\end{CJK}:
$$
\lambda^{n}\left|\lambda I_{m}-A B\right|=\lambda^{m}\left|\lambda I_{n}-B A\right|
$$
\begin{CJK}{UTF8}{mj}三\end{CJK}、 $(15$ \begin{CJK}{UTF8}{mj}分\end{CJK}) \begin{CJK}{UTF8}{mj}设\end{CJK} $A$ \begin{CJK}{UTF8}{mj}与\end{CJK} $B$ \begin{CJK}{UTF8}{mj}均为实正交矩阵\end{CJK}, \begin{CJK}{UTF8}{mj}并且\end{CJK} $|A|+|B|=0$, \begin{CJK}{UTF8}{mj}证明\end{CJK} $A+B$ \begin{CJK}{UTF8}{mj}不可逆\end{CJK}.

\begin{CJK}{UTF8}{mj}四\end{CJK}、( 15 \begin{CJK}{UTF8}{mj}分\end{CJK}) \begin{CJK}{UTF8}{mj}设\end{CJK} $A$ \begin{CJK}{UTF8}{mj}为\end{CJK} $n$ \begin{CJK}{UTF8}{mj}阶幂零矩阵\end{CJK} (\begin{CJK}{UTF8}{mj}有正整数\end{CJK} $k$, \begin{CJK}{UTF8}{mj}使得\end{CJK} $A^{k}=0$ ).

\begin{enumerate}
  \item \begin{CJK}{UTF8}{mj}求\end{CJK} $A$ \begin{CJK}{UTF8}{mj}的所有特征值\end{CJK};

  \item \begin{CJK}{UTF8}{mj}设\end{CJK} $r(A)=r$, \begin{CJK}{UTF8}{mj}则\end{CJK} $A^{r+1}=0$;

\end{enumerate}
3 . \begin{CJK}{UTF8}{mj}求\end{CJK} $\operatorname{det}(I+A)$.

\begin{CJK}{UTF8}{mj}五\end{CJK}、 (15 \begin{CJK}{UTF8}{mj}分\end{CJK})\begin{CJK}{UTF8}{mj}设\end{CJK} $A$ \begin{CJK}{UTF8}{mj}为\end{CJK} $m \times n$ \begin{CJK}{UTF8}{mj}实对称\end{CJK}, \begin{CJK}{UTF8}{mj}试证\end{CJK}:
$$
r\left(A^{T} A\right)=r(A) .
$$
\begin{CJK}{UTF8}{mj}只\end{CJK}、 $(15$ \begin{CJK}{UTF8}{mj}分\end{CJK})\begin{CJK}{UTF8}{mj}求方程组\end{CJK}
$$
\left\{\begin{array}{l}
2 x_{1}-x_{2}+3 x_{3}+4 x_{4}=5 \\
4 x_{1}-2 x_{2}+5 x_{3}+6 x_{4}=7 \\
6 x_{1}-3 x_{2}+7 x_{3}+8 x_{4}=9 \\
\lambda x_{1}-4 x_{2}+9 x_{3}+10 x_{4}=11
\end{array}\right.
$$
\begin{CJK}{UTF8}{mj}依赖于参数\end{CJK} $\lambda$ \begin{CJK}{UTF8}{mj}的通解\end{CJK}.

\begin{CJK}{UTF8}{mj}七\end{CJK}、 ( 14 \begin{CJK}{UTF8}{mj}分\end{CJK}) \begin{CJK}{UTF8}{mj}证明\end{CJK}: \begin{CJK}{UTF8}{mj}对任意\end{CJK} $n$ \begin{CJK}{UTF8}{mj}阶复方阵\end{CJK} $A$ \begin{CJK}{UTF8}{mj}都存在可逆矩阵\end{CJK} $P$, \begin{CJK}{UTF8}{mj}使得\end{CJK} $P^{-1} A P=G S$, \begin{CJK}{UTF8}{mj}其中\end{CJK} $G, S$ \begin{CJK}{UTF8}{mj}都是对称方阵\end{CJK}, \begin{CJK}{UTF8}{mj}而且\end{CJK} $G$ \begin{CJK}{UTF8}{mj}可逆\end{CJK}.

\begin{CJK}{UTF8}{mj}八\end{CJK}、 $\left(14\right.$ \begin{CJK}{UTF8}{mj}分\end{CJK}) \begin{CJK}{UTF8}{mj}在\end{CJK} $Q(x, y, z)=\lambda\left(x^{2}+y^{2}+z^{2}\right)+2 x y+2 x z-2 y z$ \begin{CJK}{UTF8}{mj}中\end{CJK}, \begin{CJK}{UTF8}{mj}问\end{CJK}:

\begin{enumerate}
  \item $\lambda$ \begin{CJK}{UTF8}{mj}取什么值时\end{CJK}, $Q$ \begin{CJK}{UTF8}{mj}为正定的\end{CJK}?

  \item $\lambda$ \begin{CJK}{UTF8}{mj}取什么值时\end{CJK}, $Q$ \begin{CJK}{UTF8}{mj}为负定的\end{CJK}?

  \item \begin{CJK}{UTF8}{mj}当\end{CJK} $\lambda=2$ \begin{CJK}{UTF8}{mj}和\end{CJK} $\lambda=-1$ \begin{CJK}{UTF8}{mj}时\end{CJK}, $Q$ \begin{CJK}{UTF8}{mj}为什么类型\end{CJK}?

\end{enumerate}
\begin{CJK}{UTF8}{mj}九\end{CJK}、 $(15$ \begin{CJK}{UTF8}{mj}分\end{CJK}) \begin{CJK}{UTF8}{mj}设\end{CJK} $A$ \begin{CJK}{UTF8}{mj}为实对称正定方阵\end{CJK}, \begin{CJK}{UTF8}{mj}则\end{CJK}
$$
\operatorname{det}(A) \leqslant a_{11} a_{22} \cdots a_{n n},
$$
\begin{CJK}{UTF8}{mj}等号当且仅当\end{CJK} $A$ \begin{CJK}{UTF8}{mj}为对角阵时成立\end{CJK}.

\begin{CJK}{UTF8}{mj}十\end{CJK}、 (15 \begin{CJK}{UTF8}{mj}分\end{CJK})\begin{CJK}{UTF8}{mj}设\end{CJK}
$$
A=\left(\begin{array}{ccc}
0 & 2011 & 11 \\
0 & 0 & 11 \\
0 & 0 & 0
\end{array}\right)
$$
\begin{CJK}{UTF8}{mj}证明\end{CJK} $X^{2}=B$ \begin{CJK}{UTF8}{mj}无解\end{CJK}, \begin{CJK}{UTF8}{mj}这里\end{CJK} $X$ \begin{CJK}{UTF8}{mj}为\end{CJK} 3 \begin{CJK}{UTF8}{mj}阶末知复方阵\end{CJK}.

\section{4. 武汉大学 2013 年研究生入学考试试题高等代数}
\begin{CJK}{UTF8}{mj}李扬\end{CJK}

\begin{CJK}{UTF8}{mj}微信公众号\end{CJK}: sxkyliyang

\begin{CJK}{UTF8}{mj}一\end{CJK}、 (15 \begin{CJK}{UTF8}{mj}分\end{CJK}) \begin{CJK}{UTF8}{mj}设\end{CJK}
$$
A=\left(\begin{array}{lll}
1 & 2 & 2 \\
2 & 5 & 4 \\
2 & 4 & 5
\end{array}\right), B=\left(\begin{array}{ll}
1 & 1 \\
0 & 1
\end{array}\right), C=\left(\begin{array}{ccccc}
1 & 1 & & & \\
& 1 & 1 & & \\
& & 1 & 1 & \\
& & & 1 & 1 \\
& & & & 1
\end{array}\right)
$$
\begin{CJK}{UTF8}{mj}求矩阵\end{CJK} $X$ \begin{CJK}{UTF8}{mj}满足\end{CJK}
$$
X\left(\begin{array}{cc}
O & B \\
A & O
\end{array}\right)=C
$$
\begin{CJK}{UTF8}{mj}二\end{CJK}、 ( 15 \begin{CJK}{UTF8}{mj}分\end{CJK}) \begin{CJK}{UTF8}{mj}求\end{CJK} $n$ \begin{CJK}{UTF8}{mj}阶行列式\end{CJK} $D_{n}$ \begin{CJK}{UTF8}{mj}的所有元素的代数余子式之和\end{CJK}, \begin{CJK}{UTF8}{mj}其中\end{CJK}
$$
D_{n}=\left|\begin{array}{ccccc}
33 & 1 & 1 & \cdots & 1 \\
0 & 61 & 1 & \cdots & 1 \\
0 & 0 & 1 & \cdots & \\
\vdots & \vdots & \vdots & & \vdots \\
0 & 0 & 0 & \cdots & 1
\end{array}\right|
$$
\begin{CJK}{UTF8}{mj}三\end{CJK}、 (15 \begin{CJK}{UTF8}{mj}分\end{CJK}) \begin{CJK}{UTF8}{mj}设\end{CJK} $\alpha_{1}, \alpha_{2}, \alpha_{3}$ \begin{CJK}{UTF8}{mj}是线性方程组\end{CJK}
$$
\left\{\begin{array}{l}
x-3 y+z=2 \\
2 x+y+t z=-1 \\
7 x-2 z=-1
\end{array}\right.
$$
\begin{CJK}{UTF8}{mj}的\end{CJK} 3 \begin{CJK}{UTF8}{mj}个互异的解向量\end{CJK}.

\begin{enumerate}
  \item \begin{CJK}{UTF8}{mj}试求参数\end{CJK} $t$ \begin{CJK}{UTF8}{mj}的值\end{CJK};

  \item \begin{CJK}{UTF8}{mj}证明\end{CJK}: $\alpha_{1}-\alpha_{2}, \alpha_{1}-\alpha_{3}$ \begin{CJK}{UTF8}{mj}线性相关\end{CJK}.

\end{enumerate}
\begin{CJK}{UTF8}{mj}四\end{CJK}、( 15 \begin{CJK}{UTF8}{mj}分\end{CJK} $)$ \begin{CJK}{UTF8}{mj}设\end{CJK} $M_{n}(\mathbb{K})$ \begin{CJK}{UTF8}{mj}表示数域\end{CJK} $\mathbb{K}$ \begin{CJK}{UTF8}{mj}上\end{CJK} $n$ \begin{CJK}{UTF8}{mj}阶方阵全体构成的线性空间\end{CJK}, $A \in M_{n}(\mathbb{K})$, \begin{CJK}{UTF8}{mj}用\end{CJK} $C(A)$ \begin{CJK}{UTF8}{mj}表示\end{CJK} $M_{n}(\mathbb{K})$ \begin{CJK}{UTF8}{mj}中所有\end{CJK} \begin{CJK}{UTF8}{mj}与\end{CJK} $A$ \begin{CJK}{UTF8}{mj}可交换的矩阵构成的集合\end{CJK}.

\begin{enumerate}
  \item \begin{CJK}{UTF8}{mj}证明\end{CJK}: $C(A)$ \begin{CJK}{UTF8}{mj}是\end{CJK} $M_{n}(\mathbb{K})$ \begin{CJK}{UTF8}{mj}的一个线性子空间\end{CJK};

  \item \begin{CJK}{UTF8}{mj}证明\end{CJK}: \begin{CJK}{UTF8}{mj}如果\end{CJK} $A$ \begin{CJK}{UTF8}{mj}有\end{CJK} $n$ \begin{CJK}{UTF8}{mj}个不同的特征值\end{CJK}, \begin{CJK}{UTF8}{mj}那么\end{CJK} $\operatorname{dim} C(A)=n$.

\end{enumerate}
\begin{CJK}{UTF8}{mj}五\end{CJK}、 (15 \begin{CJK}{UTF8}{mj}分\end{CJK}) \begin{CJK}{UTF8}{mj}设\end{CJK} $a_{1}, a_{2}, \cdots, a_{s}$ \begin{CJK}{UTF8}{mj}是互不相等的实数\end{CJK}, \begin{CJK}{UTF8}{mj}矩阵\end{CJK}
$$
B=\left(\begin{array}{cccc}
1 & 1 & \cdots & 1 \\
a_{1} & a_{2} & \cdots & a_{s} \\
a_{1}^{2} & a_{2}^{2} & & a_{s}^{2} \\
a_{1}^{n-1} & a_{2}^{n-1} & \cdots & a_{s}^{n-1}
\end{array}\right)
$$
\begin{CJK}{UTF8}{mj}试讨论矩阵\end{CJK} $A=B^{T} B$ \begin{CJK}{UTF8}{mj}的正定性\end{CJK}.

\begin{CJK}{UTF8}{mj}六\end{CJK}、(15 \begin{CJK}{UTF8}{mj}分\end{CJK}) \begin{CJK}{UTF8}{mj}设\end{CJK} $\sigma_{1}, \sigma_{2}$ \begin{CJK}{UTF8}{mj}都是\end{CJK} $n$ \begin{CJK}{UTF8}{mj}维线性空间\end{CJK} $V$ \begin{CJK}{UTF8}{mj}的线性变换\end{CJK}, \begin{CJK}{UTF8}{mj}证明\end{CJK}: $\operatorname{ker}\left(\sigma_{1}\right) \subset \operatorname{ker}\left(\sigma_{2}\right)$ \begin{CJK}{UTF8}{mj}的充分必要条件是存在\end{CJK} $V$ \begin{CJK}{UTF8}{mj}的\end{CJK} \begin{CJK}{UTF8}{mj}一个线性变换\end{CJK} $\sigma$, \begin{CJK}{UTF8}{mj}使得\end{CJK} $\sigma_{2}=\sigma \sigma_{1}$. \begin{CJK}{UTF8}{mj}七\end{CJK}、(15 \begin{CJK}{UTF8}{mj}分\end{CJK}) \begin{CJK}{UTF8}{mj}设\end{CJK} $A$ \begin{CJK}{UTF8}{mj}是\end{CJK} $n$ \begin{CJK}{UTF8}{mj}阶正定矩阵\end{CJK}, $B$ \begin{CJK}{UTF8}{mj}是\end{CJK} $n$ \begin{CJK}{UTF8}{mj}阶实对称矩阵\end{CJK}, \begin{CJK}{UTF8}{mj}证明\end{CJK}:\begin{CJK}{UTF8}{mj}必存在\end{CJK} $n$ \begin{CJK}{UTF8}{mj}阶可逆实矩阵\end{CJK} $G$, \begin{CJK}{UTF8}{mj}使得\end{CJK}
$$
G^{T} G=E, G^{T} B G=\operatorname{diag}\left(\mu_{1}, \mu_{2}, \cdots, \mu_{n}\right)
$$
\begin{CJK}{UTF8}{mj}其中\end{CJK} $E$ \begin{CJK}{UTF8}{mj}是\end{CJK} $n$ \begin{CJK}{UTF8}{mj}阶单位矩阵\end{CJK}, $\mu_{1}, \mu_{2}, \cdots, \mu_{n}$ \begin{CJK}{UTF8}{mj}是\end{CJK} $|\lambda A-B|=0$ \begin{CJK}{UTF8}{mj}的\end{CJK} $n$ \begin{CJK}{UTF8}{mj}个实根\end{CJK}.

\begin{CJK}{UTF8}{mj}八\end{CJK}、 $\left(15\right.$ \begin{CJK}{UTF8}{mj}分\end{CJK}) \begin{CJK}{UTF8}{mj}设\end{CJK} $A \in M_{2}(\mathbb{C})$ \begin{CJK}{UTF8}{mj}不是数量矩阵\end{CJK}, \begin{CJK}{UTF8}{mj}令\end{CJK} $S=\left\{B \in M_{2}(\mathbb{C}) \mid A B=B A\right\}$. \begin{CJK}{UTF8}{mj}证明\end{CJK}: \begin{CJK}{UTF8}{mj}如果\end{CJK} $X, Y \in S$, \begin{CJK}{UTF8}{mj}那么\end{CJK} $X Y=Y X$.

\begin{CJK}{UTF8}{mj}九\end{CJK}、 ( 15 \begin{CJK}{UTF8}{mj}分\end{CJK}) \begin{CJK}{UTF8}{mj}设\end{CJK} $\lambda_{1}, \lambda_{2}, \cdots, \lambda_{n}$ \begin{CJK}{UTF8}{mj}是\end{CJK} $n$ \begin{CJK}{UTF8}{mj}阶实对称矩阵\end{CJK} $A$ \begin{CJK}{UTF8}{mj}的全部特征值\end{CJK}, \begin{CJK}{UTF8}{mj}但\end{CJK} $-\lambda_{i}(i=1, \cdots, n)$ \begin{CJK}{UTF8}{mj}不是\end{CJK} $A$ \begin{CJK}{UTF8}{mj}的特征值\end{CJK}, \begin{CJK}{UTF8}{mj}定义\end{CJK} $\mathbb{R}^{n \times n}$ \begin{CJK}{UTF8}{mj}的线性变换\end{CJK}
$$
\sigma(X)=A^{T} X+X A, \forall X \in \mathbb{R}^{n \times n}
$$
\begin{CJK}{UTF8}{mj}证明\end{CJK}:

\begin{enumerate}
  \item $\sigma$ \begin{CJK}{UTF8}{mj}是可逆线性变换\end{CJK};

  \item \begin{CJK}{UTF8}{mj}对任意实对称矩阵\end{CJK} $C$, \begin{CJK}{UTF8}{mj}必存在唯一的实对称矩阵\end{CJK} $B$, \begin{CJK}{UTF8}{mj}使得\end{CJK} $A^{T} B+B A=C$.

\end{enumerate}
\begin{CJK}{UTF8}{mj}十\end{CJK}、 ( 15 \begin{CJK}{UTF8}{mj}分\end{CJK})\begin{CJK}{UTF8}{mj}设\end{CJK} $W$ \begin{CJK}{UTF8}{mj}为\end{CJK} $n$ \begin{CJK}{UTF8}{mj}维欧氏空间\end{CJK} $V$ \begin{CJK}{UTF8}{mj}的子空间\end{CJK}, $\alpha \in V$. \begin{CJK}{UTF8}{mj}定义\end{CJK} $\alpha$ \begin{CJK}{UTF8}{mj}到\end{CJK} $W$ \begin{CJK}{UTF8}{mj}的距离\end{CJK}
$$
d(\alpha, W)=\left(\alpha-\alpha^{\prime}\right)
$$
\begin{CJK}{UTF8}{mj}其中\end{CJK} $\alpha^{\prime}$ \begin{CJK}{UTF8}{mj}为\end{CJK} $\alpha$ \begin{CJK}{UTF8}{mj}在子空间\end{CJK} $W$ \begin{CJK}{UTF8}{mj}上的正交投影\end{CJK}. \begin{CJK}{UTF8}{mj}证明\end{CJK}: \begin{CJK}{UTF8}{mj}若\end{CJK} $\alpha_{1}, \alpha_{2}, \cdots, \alpha_{m}$ \begin{CJK}{UTF8}{mj}为\end{CJK} $W$ \begin{CJK}{UTF8}{mj}的一个基\end{CJK}, \begin{CJK}{UTF8}{mj}则\end{CJK}
$$
d(\alpha, W)=\sqrt{\frac{\left|G\left(\alpha_{1}, \alpha_{2}, \cdots, \alpha_{m}, \alpha\right)\right|}{\left|G\left(\alpha_{1}, \alpha_{2}, \cdots, \alpha_{m}\right)\right|}}
$$
\begin{CJK}{UTF8}{mj}其中\end{CJK} $\left|G\left(\alpha_{1}, \alpha_{2}, \cdots, \alpha_{m}\right)\right|$ \begin{CJK}{UTF8}{mj}为向量组\end{CJK} $\alpha_{1}, \alpha_{2}, \cdots, \alpha_{m}$ \begin{CJK}{UTF8}{mj}的\end{CJK} Gram \begin{CJK}{UTF8}{mj}矩阵\end{CJK}.

\section{5. 武汉大学 2014 年研究生入学考试试题高等代数}
\begin{CJK}{UTF8}{mj}李扬\end{CJK}

\begin{CJK}{UTF8}{mj}微信公众号\end{CJK}: sxkyliyang

\begin{CJK}{UTF8}{mj}一\end{CJK}、 (15 \begin{CJK}{UTF8}{mj}分\end{CJK}) \begin{CJK}{UTF8}{mj}由\end{CJK}
$$
A=\left(\begin{array}{cccc}
1 & 2 & 0 & 0 \\
1 & 3 & 0 & 0 \\
0 & 0 & 0 & 2 \\
0 & 0 & -1 & 0
\end{array}\right),
$$
\begin{CJK}{UTF8}{mj}且\end{CJK} $\left[\left(\frac{1}{2} A\right)^{*}\right]^{-1} B A=6 A B+12 E$, \begin{CJK}{UTF8}{mj}求\end{CJK} $B$.

\begin{CJK}{UTF8}{mj}二\end{CJK}、 (15 \begin{CJK}{UTF8}{mj}分\end{CJK}) \begin{CJK}{UTF8}{mj}计算\end{CJK}
$$
D=\left|\begin{array}{ccccc}
s_{0} & s_{1} & \cdots & s_{n-1} & 1 \\
s_{1} & s_{2} & \cdots & s_{n} & x \\
\vdots & \vdots & & \vdots & \vdots \\
s_{n} & s_{n+1} & \cdots & s_{2 n-1} & x^{n}
\end{array}\right|
$$
\begin{CJK}{UTF8}{mj}其中\end{CJK} $s_{k}=x_{1}^{k}+s_{2}^{k}+\cdots+x_{n}^{k}$.

\begin{CJK}{UTF8}{mj}三\end{CJK}、 $(15$ \begin{CJK}{UTF8}{mj}分\end{CJK} $)$ \begin{CJK}{UTF8}{mj}有\end{CJK} $\alpha_{1}, \alpha_{2}, \cdots, \alpha_{s}, \alpha_{s+1}$, \begin{CJK}{UTF8}{mj}且\end{CJK} $\beta_{i}=\alpha_{i}+t_{i} \alpha_{s+1}, i=1, \cdots, s$, \begin{CJK}{UTF8}{mj}证明如果\end{CJK} $\beta_{1}, \beta_{2}, \cdots, \beta_{s}$ \begin{CJK}{UTF8}{mj}线性无关\end{CJK}, \begin{CJK}{UTF8}{mj}则\end{CJK} $\alpha_{1}, \alpha_{2}, \cdots, \alpha_{s}, \alpha_{s+1}$ \begin{CJK}{UTF8}{mj}必定线性无关\end{CJK}.

\begin{CJK}{UTF8}{mj}四\end{CJK}、(15 \begin{CJK}{UTF8}{mj}分\end{CJK}) \begin{CJK}{UTF8}{mj}线性空间\end{CJK} $V$ \begin{CJK}{UTF8}{mj}定义的第\end{CJK} $(3),(4)$ \begin{CJK}{UTF8}{mj}条公理\end{CJK}, \begin{CJK}{UTF8}{mj}即\end{CJK}

(3) \begin{CJK}{UTF8}{mj}任意\end{CJK} $\alpha \in V$, \begin{CJK}{UTF8}{mj}存在\end{CJK} $0 \in V$, \begin{CJK}{UTF8}{mj}使\end{CJK} $\alpha+0=0+\alpha=\alpha$;

(4) \begin{CJK}{UTF8}{mj}任意\end{CJK} $\alpha \in V$, \begin{CJK}{UTF8}{mj}存在\end{CJK} $\beta \in V$, \begin{CJK}{UTF8}{mj}使\end{CJK} $\alpha+\beta=\beta+\alpha=0$.

\begin{CJK}{UTF8}{mj}证明它们等价的条件为\end{CJK}: $\forall \alpha, \beta \in V, \exists x \in V$, \begin{CJK}{UTF8}{mj}使得\end{CJK} $\alpha+x=\beta$.

\begin{CJK}{UTF8}{mj}五\end{CJK}、(15 \begin{CJK}{UTF8}{mj}分\end{CJK}) \begin{CJK}{UTF8}{mj}设\end{CJK} $s l_{n}(F)$ \begin{CJK}{UTF8}{mj}是\end{CJK} $M(F)$ \begin{CJK}{UTF8}{mj}上\end{CJK} $A, B$ \begin{CJK}{UTF8}{mj}矩阵满足\end{CJK} $A B-B A$ \begin{CJK}{UTF8}{mj}生成的子空间\end{CJK}, \begin{CJK}{UTF8}{mj}证明\end{CJK}
$$
\operatorname{dim}\left(s l_{n}(F)\right)=n^{2}-1
$$
\begin{CJK}{UTF8}{mj}六\end{CJK}、(15 \begin{CJK}{UTF8}{mj}分\end{CJK}) \begin{CJK}{UTF8}{mj}设数域\end{CJK} $K$ \begin{CJK}{UTF8}{mj}上的\end{CJK} $n$ \begin{CJK}{UTF8}{mj}维线性\end{CJK} $V$ \begin{CJK}{UTF8}{mj}到\end{CJK} $m$ \begin{CJK}{UTF8}{mj}维线性空间\end{CJK} $V^{\prime}$ \begin{CJK}{UTF8}{mj}上的所有线性映射组成空间\end{CJK} $H o m_{k}\left(V, V^{\prime}\right)$, \begin{CJK}{UTF8}{mj}证明\end{CJK}:

\begin{enumerate}
  \item $\operatorname{Hom}_{k}\left(V, V^{\prime}\right)$ \begin{CJK}{UTF8}{mj}是线性空间\end{CJK};

  \item $\operatorname{Hom}_{k}\left(V, V^{\prime}\right)$ \begin{CJK}{UTF8}{mj}的维数为\end{CJK} $m n$.

\end{enumerate}
\begin{CJK}{UTF8}{mj}七\end{CJK}、 (15 \begin{CJK}{UTF8}{mj}分\end{CJK}) \begin{CJK}{UTF8}{mj}已知\end{CJK}
$$
A=\left(\begin{array}{cccccc}
0 & & & & -c_{0} \\
1 & 0 & & & -c_{1} \\
& 1 & 0 & & \vdots \\
& & 1 & \ddots & -c_{n-3} \\
& & & \ddots & 0 & -c_{n-2} \\
& & & & -c_{n-1}
\end{array}\right)
$$

\begin{enumerate}
  \item \begin{CJK}{UTF8}{mj}求\end{CJK} $F$ \begin{CJK}{UTF8}{mj}的特征多项式\end{CJK} $f(x)$ \begin{CJK}{UTF8}{mj}与最小多项式\end{CJK} $m(x)$;

  \item \begin{CJK}{UTF8}{mj}求所有与\end{CJK} $F$ \begin{CJK}{UTF8}{mj}可交换的矩阵\end{CJK}. \begin{CJK}{UTF8}{mj}八\end{CJK}、(15 \begin{CJK}{UTF8}{mj}分\end{CJK}) \begin{CJK}{UTF8}{mj}设\end{CJK} $\varphi$ \begin{CJK}{UTF8}{mj}是复数域上的线性变换\end{CJK}, $\varepsilon$ \begin{CJK}{UTF8}{mj}为恒等变换\end{CJK}, $\lambda_{0}$ \begin{CJK}{UTF8}{mj}为\end{CJK} $\varphi$ \begin{CJK}{UTF8}{mj}的一个特征值\end{CJK}, $\lambda_{0}$ \begin{CJK}{UTF8}{mj}在\end{CJK} $\varphi$ \begin{CJK}{UTF8}{mj}的最小多项式中的重\end{CJK} \begin{CJK}{UTF8}{mj}数\end{CJK} $m_{0}=\min _{k}\left\{k \in N^{+} \mid \operatorname{ker}\left(\lambda_{0} \varepsilon-\varphi\right)^{k}=\operatorname{ker}\left(\lambda_{0} \varepsilon-\varphi\right)^{k+1}\right\}$.

\end{enumerate}
\begin{CJK}{UTF8}{mj}九\end{CJK}、 ( 15 \begin{CJK}{UTF8}{mj}分\end{CJK}) \begin{CJK}{UTF8}{mj}设\end{CJK} $f(\alpha, \beta)$ \begin{CJK}{UTF8}{mj}为\end{CJK} $V$ \begin{CJK}{UTF8}{mj}上的非退化双线性函数\end{CJK}, \begin{CJK}{UTF8}{mj}对\end{CJK} $\forall g(x) \in V^{*}$, \begin{CJK}{UTF8}{mj}存在唯一的\end{CJK} $\alpha \in V$, \begin{CJK}{UTF8}{mj}使得\end{CJK} $f(\alpha, \beta)=$ $g(\beta), \forall \beta \in V .$

\begin{CJK}{UTF8}{mj}十\end{CJK}、 ( 15 \begin{CJK}{UTF8}{mj}分\end{CJK})\begin{CJK}{UTF8}{mj}设\end{CJK} $\varphi$ \begin{CJK}{UTF8}{mj}是欧氏空间\end{CJK} $V$ \begin{CJK}{UTF8}{mj}上的正交变换\end{CJK}, \begin{CJK}{UTF8}{mj}且\end{CJK} $\varphi^{m}=\varepsilon, m>1$, \begin{CJK}{UTF8}{mj}记\end{CJK} $W=\{x \in V \mid \varphi(x)=x\}, W_{\varphi}^{\perp}$ \begin{CJK}{UTF8}{mj}为其正交补\end{CJK}, \begin{CJK}{UTF8}{mj}对任意的\end{CJK} $\alpha \in V$, \begin{CJK}{UTF8}{mj}若有\end{CJK} $\alpha=\beta+\gamma$, \begin{CJK}{UTF8}{mj}其中\end{CJK} $\beta \in W_{\varphi}, \gamma \in W_{\varphi}^{\perp}$, \begin{CJK}{UTF8}{mj}证明\end{CJK}:
$$
\beta=\frac{1}{m} \sum_{i=1}^{m} \varphi^{i-1}(\alpha)
$$

\section{6. 武汉大学 2015 年研究生入学考试试题高等代数}
\begin{CJK}{UTF8}{mj}李扬\end{CJK}

\begin{CJK}{UTF8}{mj}微信公众号\end{CJK}: sxkyliyang

\begin{CJK}{UTF8}{mj}一\end{CJK}、1. \begin{CJK}{UTF8}{mj}证明\end{CJK} $\left[\begin{array}{cc}A & A \\ C-B & C\end{array}\right]$ \begin{CJK}{UTF8}{mj}可逆的充要条件是\end{CJK} $A B$ \begin{CJK}{UTF8}{mj}可逆\end{CJK};

\begin{enumerate}
  \setcounter{enumi}{2}
  \item \begin{CJK}{UTF8}{mj}若\end{CJK} $\left[\begin{array}{cc}A & A \\ C-B & C\end{array}\right]$ \begin{CJK}{UTF8}{mj}可逆\end{CJK}, \begin{CJK}{UTF8}{mj}求出\end{CJK} $\left[\begin{array}{cc}A & A \\ C-B & C\end{array}\right]$ \begin{CJK}{UTF8}{mj}的逆\end{CJK}.
\end{enumerate}
\begin{CJK}{UTF8}{mj}二\end{CJK}、 $b \neq 0, r(A)=r(A, b)=r, A x=b$ \begin{CJK}{UTF8}{mj}的所有解集合为\end{CJK} $S$, \begin{CJK}{UTF8}{mj}证明\end{CJK}:

\begin{enumerate}
  \item $S$ \begin{CJK}{UTF8}{mj}中包含\end{CJK} $n-r+1$ \begin{CJK}{UTF8}{mj}个线性无关的向量\end{CJK} $\eta_{1}, \eta_{2}, \cdots, \eta_{n-r+1}$;

  \item $\xi$ \begin{CJK}{UTF8}{mj}是\end{CJK} $S$ \begin{CJK}{UTF8}{mj}中元素的充要条件是存在\end{CJK} $k_{i},(i=1,2, \cdots, n-r+1), \sum_{i=1}^{n-r+1} k_{i}=1$, \begin{CJK}{UTF8}{mj}使得\end{CJK} $\xi=\sum_{i=1}^{n-r+1} k_{i} \eta_{i}$.

\end{enumerate}
\begin{CJK}{UTF8}{mj}三\end{CJK}、\begin{CJK}{UTF8}{mj}已知\end{CJK} $A$ \begin{CJK}{UTF8}{mj}为实正交矩阵\end{CJK}, $\operatorname{det}(A)=1$, \begin{CJK}{UTF8}{mj}证明存在正交矩阵\end{CJK} $P$, \begin{CJK}{UTF8}{mj}使得\end{CJK}
$$
P^{\prime} A P=\left(\begin{array}{ccc}
1 & 0 & 0 \\
0 & \cos \theta & -\sin \theta \\
0 & \sin \theta & \cos \theta
\end{array}\right)
$$
\begin{CJK}{UTF8}{mj}其中\end{CJK} $\cos \theta=\frac{a_{11}+a_{22}+a_{33}-1}{2}$.

\begin{CJK}{UTF8}{mj}四\end{CJK}、\begin{CJK}{UTF8}{mj}以下有关矩阵秩的命题在数域\end{CJK} $F$ \begin{CJK}{UTF8}{mj}上判断正误\end{CJK}, \begin{CJK}{UTF8}{mj}如正确请说明理由\end{CJK}, \begin{CJK}{UTF8}{mj}如不正确请举例说明\end{CJK}.

\begin{enumerate}
  \item \begin{CJK}{UTF8}{mj}若\end{CJK} $r(A)=r(B)$, \begin{CJK}{UTF8}{mj}则\end{CJK} $r\left(A^{*}\right)=r\left(B^{*}\right)$;

  \item \begin{CJK}{UTF8}{mj}若\end{CJK} $r(A B)=r(B)$, \begin{CJK}{UTF8}{mj}则\end{CJK} $r(A B C)=r(B C)$;

  \item $r(A)=r\left(A A^{\prime}\right)$;

  \item \begin{CJK}{UTF8}{mj}若一个对称矩阵的秩为\end{CJK} $r$, \begin{CJK}{UTF8}{mj}则有一个非零的\end{CJK} $r$ \begin{CJK}{UTF8}{mj}阶主子式\end{CJK}.

\end{enumerate}
\begin{CJK}{UTF8}{mj}王\end{CJK}、\begin{CJK}{UTF8}{mj}已知矩阵\end{CJK}
$$
P^{\prime} A P=\left(\begin{array}{lll}
1 & 1 & 1 \\
1 & 1 & 0 \\
1 & 0 & 1
\end{array}\right) \text {, }
$$
\begin{CJK}{UTF8}{mj}求正交矩阵\end{CJK} $Q$ \begin{CJK}{UTF8}{mj}和对角线元素为负的上三角矩阵\end{CJK} $R$, \begin{CJK}{UTF8}{mj}使\end{CJK} $A=Q R$.

\begin{CJK}{UTF8}{mj}六\end{CJK}、 $A$ \begin{CJK}{UTF8}{mj}是\end{CJK} $n$ \begin{CJK}{UTF8}{mj}阶实对称矩阵\end{CJK}, \begin{CJK}{UTF8}{mj}其正负惯性指数分别是\end{CJK} $p, q, f(x)=X^{\prime} A X$, \begin{CJK}{UTF8}{mj}记\end{CJK} $N_{f}=\left\{x \mid f(x)=0, x \in R^{n}\right\}$, \begin{CJK}{UTF8}{mj}证明\end{CJK}:

\begin{enumerate}
  \item \begin{CJK}{UTF8}{mj}包含于\end{CJK} $N_{f}$ \begin{CJK}{UTF8}{mj}的线性空间维数至多是\end{CJK} $n-\max (p, q)$;

  \item \begin{CJK}{UTF8}{mj}若\end{CJK} $w$ \begin{CJK}{UTF8}{mj}是\end{CJK} $R^{n}$ \begin{CJK}{UTF8}{mj}的一个线性子空间\end{CJK}, \begin{CJK}{UTF8}{mj}将二次型限定\end{CJK} $w$ \begin{CJK}{UTF8}{mj}在中\end{CJK}, \begin{CJK}{UTF8}{mj}得到的正负惯性指数分别是\end{CJK} $p_{1}, q_{1}$, \begin{CJK}{UTF8}{mj}则有\end{CJK} $p_{1} \leqslant p, q_{1} \leqslant q .$

\end{enumerate}
\begin{CJK}{UTF8}{mj}七\end{CJK}、 1. \begin{CJK}{UTF8}{mj}已知\end{CJK} $A, B$ \begin{CJK}{UTF8}{mj}都是实正定矩阵\end{CJK}, \begin{CJK}{UTF8}{mj}则存在可逆矩阵\end{CJK} $P$, \begin{CJK}{UTF8}{mj}使得\end{CJK} $P^{\prime} A P, P^{\prime} B P$ \begin{CJK}{UTF8}{mj}同时为对角形\end{CJK};

\begin{enumerate}
  \setcounter{enumi}{2}
  \item \begin{CJK}{UTF8}{mj}当\end{CJK} $A, B$ \begin{CJK}{UTF8}{mj}是半正定矩阵时\end{CJK}, \begin{CJK}{UTF8}{mj}上述结论是否成立\end{CJK}, \begin{CJK}{UTF8}{mj}若成立给出证明\end{CJK}, \begin{CJK}{UTF8}{mj}若不成立请说明理由\end{CJK}.
\end{enumerate}
\begin{CJK}{UTF8}{mj}八\end{CJK}、\begin{CJK}{UTF8}{mj}已知\end{CJK} $\alpha_{1}, \alpha_{2}, \alpha_{3}, \alpha_{4}, \alpha_{5}, \alpha_{6}$ \begin{CJK}{UTF8}{mj}是线性空间\end{CJK} $V$ \begin{CJK}{UTF8}{mj}中的一组基\end{CJK}, $\varphi$ \begin{CJK}{UTF8}{mj}是线性空间\end{CJK} $V$ \begin{CJK}{UTF8}{mj}上的线性变换\end{CJK}, \begin{CJK}{UTF8}{mj}且\end{CJK}
$$
\varphi\left(\alpha_{1}\right)=\alpha_{1}, \varphi\left(\alpha_{2}\right)=\alpha_{1}+\alpha_{2}, \varphi\left(\alpha_{3}\right)=\alpha_{2}+\alpha_{3}, \varphi\left(\alpha_{4}\right)=\alpha_{3}+2 \alpha_{4}, \varphi\left(\alpha_{5}\right)=2 \alpha_{5}, \varphi\left(\xi_{6}\right)=\alpha_{5}+3 \alpha_{6}
$$

\begin{enumerate}
  \item \begin{CJK}{UTF8}{mj}求出所有二维的\end{CJK} $\varphi$ \begin{CJK}{UTF8}{mj}的不变子空间\end{CJK}, \begin{CJK}{UTF8}{mj}并说明理由\end{CJK};

  \item \begin{CJK}{UTF8}{mj}证明\end{CJK} $\varphi$ \begin{CJK}{UTF8}{mj}不是循环变换\end{CJK}, \begin{CJK}{UTF8}{mj}即\end{CJK} $\forall \alpha \in V, \alpha, \varphi(\alpha), \varphi^{2}(\alpha), \cdots, \varphi^{5}(\alpha)$ \begin{CJK}{UTF8}{mj}不构成一组基\end{CJK}.

\end{enumerate}
\section{7. 武汉大学 2017 年研究生入学考试试题高等代数}
\begin{CJK}{UTF8}{mj}李扬\end{CJK}

\begin{CJK}{UTF8}{mj}微信公众号\end{CJK}: sxkyliyang

\begin{CJK}{UTF8}{mj}一\end{CJK}、(25 \begin{CJK}{UTF8}{mj}分\end{CJK}) \begin{CJK}{UTF8}{mj}向量组\end{CJK} $(I) : \alpha_{1}, \alpha_{2}, \cdots, \alpha_{r}$ \begin{CJK}{UTF8}{mj}线性无关\end{CJK}, \begin{CJK}{UTF8}{mj}且可以由向量组\end{CJK} $(I I): \beta_{1}, \beta_{2}, \cdots, \beta_{s}$ \begin{CJK}{UTF8}{mj}线性表出\end{CJK}. \begin{CJK}{UTF8}{mj}证明\end{CJK}:

\begin{enumerate}
  \item $r \leqslant s$;

  \item \begin{CJK}{UTF8}{mj}必要时对向量组\end{CJK} $(I I)$ \begin{CJK}{UTF8}{mj}重新编号\end{CJK}, \begin{CJK}{UTF8}{mj}再用向量组\end{CJK} $(I)$ \begin{CJK}{UTF8}{mj}替换向量组\end{CJK} $(I I)$ \begin{CJK}{UTF8}{mj}的前\end{CJK} $r$ \begin{CJK}{UTF8}{mj}个向量\end{CJK}, \begin{CJK}{UTF8}{mj}得到的向量组\end{CJK} $(I I I): \alpha_{1}, \alpha_{2}, \cdots, \alpha_{r}, \beta_{r+1}, \cdots, \beta_{s}$ \begin{CJK}{UTF8}{mj}与向量组\end{CJK} $(I I)$ \begin{CJK}{UTF8}{mj}等价\end{CJK}.

\end{enumerate}
\begin{CJK}{UTF8}{mj}二\end{CJK}、 ( 25 \begin{CJK}{UTF8}{mj}分\end{CJK}) \begin{CJK}{UTF8}{mj}证明维数公式\end{CJK}:
$$
\operatorname{dim} W_{1}+\operatorname{dim} W_{2}=\operatorname{dim}\left(W_{1}+W_{2}\right)+\operatorname{dim}\left(W_{1} \cap W_{2}\right)
$$
\begin{CJK}{UTF8}{mj}三\end{CJK}、 $(25$ \begin{CJK}{UTF8}{mj}分\end{CJK} $)$ \begin{CJK}{UTF8}{mj}设\end{CJK} $\alpha_{1}, \alpha_{2}, \cdots, \alpha_{s}\left(\alpha_{1} \neq 0\right)$ \begin{CJK}{UTF8}{mj}线性相关\end{CJK}, \begin{CJK}{UTF8}{mj}证明\end{CJK}: \begin{CJK}{UTF8}{mj}存在正整数\end{CJK} $i: 1<i \leqslant s$, \begin{CJK}{UTF8}{mj}使向量\end{CJK} $\alpha_{i}$ \begin{CJK}{UTF8}{mj}可以由向量组\end{CJK} $\alpha_{1}, \alpha_{2}, \cdots, \alpha_{i-1}$ \begin{CJK}{UTF8}{mj}线性表出\end{CJK}, \begin{CJK}{UTF8}{mj}且表示法唯一\end{CJK}.

\begin{CJK}{UTF8}{mj}四\end{CJK}、(25 \begin{CJK}{UTF8}{mj}分\end{CJK}) \begin{CJK}{UTF8}{mj}设\end{CJK} $V$ \begin{CJK}{UTF8}{mj}是数域\end{CJK} $F$ \begin{CJK}{UTF8}{mj}上的\end{CJK} $n$ \begin{CJK}{UTF8}{mj}维线性空间\end{CJK}, $\sigma$ \begin{CJK}{UTF8}{mj}是\end{CJK} $V$ \begin{CJK}{UTF8}{mj}上的线性变换\end{CJK}. \begin{CJK}{UTF8}{mj}设\end{CJK} $\alpha_{1}, \alpha_{2}, \cdots, \alpha_{n}$ \begin{CJK}{UTF8}{mj}是\end{CJK} $V$ \begin{CJK}{UTF8}{mj}的一个基\end{CJK}, \begin{CJK}{UTF8}{mj}线性\end{CJK} \begin{CJK}{UTF8}{mj}变换\end{CJK} $\sigma$ \begin{CJK}{UTF8}{mj}在这个基下的矩阵为\end{CJK} $A$, \begin{CJK}{UTF8}{mj}记\end{CJK} $E n d V$ \begin{CJK}{UTF8}{mj}是\end{CJK} $V$ \begin{CJK}{UTF8}{mj}上全体线性变换构成的线性空间\end{CJK}, \begin{CJK}{UTF8}{mj}作\end{CJK} $E n d V$ \begin{CJK}{UTF8}{mj}到\end{CJK} $M_{n}(F)$ \begin{CJK}{UTF8}{mj}的映射\end{CJK} $f: f(\sigma)=A$. \begin{CJK}{UTF8}{mj}证明\end{CJK}: \begin{CJK}{UTF8}{mj}映射\end{CJK} $f$ \begin{CJK}{UTF8}{mj}是\end{CJK} $\operatorname{End} V$ \begin{CJK}{UTF8}{mj}到\end{CJK} $M_{n}(F)$ \begin{CJK}{UTF8}{mj}的一个同构映射\end{CJK}.

\begin{CJK}{UTF8}{mj}五\end{CJK}、( 25 \begin{CJK}{UTF8}{mj}分\end{CJK}) \begin{CJK}{UTF8}{mj}设\end{CJK} $A$ \begin{CJK}{UTF8}{mj}是\end{CJK} $m \times n$ \begin{CJK}{UTF8}{mj}矩阵\end{CJK}, $r(A)=m$, \begin{CJK}{UTF8}{mj}齐次线性方程组\end{CJK} $A x=0$ \begin{CJK}{UTF8}{mj}的一个基础解系为\end{CJK} $\beta_{1}, \beta_{2}, \cdots, \beta_{n-m}$, \begin{CJK}{UTF8}{mj}其\end{CJK} \begin{CJK}{UTF8}{mj}中\end{CJK} $\beta_{i}=\left(b_{1 i}, b_{2 i}, \cdots, b_{n i}\right)^{\prime},(i=1,2, \cdots, n-m)$. \begin{CJK}{UTF8}{mj}求齐次线性方程组\end{CJK} $\sum_{j=1}^{n} b_{j i} y_{j}=0, i=1,2, \cdots, n-m$ \begin{CJK}{UTF8}{mj}的一个\end{CJK} \begin{CJK}{UTF8}{mj}基础解系\end{CJK}.

\begin{CJK}{UTF8}{mj}六\end{CJK}、 ( 25 \begin{CJK}{UTF8}{mj}分\end{CJK}) \begin{CJK}{UTF8}{mj}设\end{CJK} $A$ \begin{CJK}{UTF8}{mj}是\end{CJK} $m \times n$ \begin{CJK}{UTF8}{mj}矩阵\end{CJK}, $B$ \begin{CJK}{UTF8}{mj}是\end{CJK} $n \times p$ \begin{CJK}{UTF8}{mj}矩阵\end{CJK}.

\begin{enumerate}
  \item \begin{CJK}{UTF8}{mj}证明\end{CJK}: $r(A) \leqslant r(A B)$;

  \item \begin{CJK}{UTF8}{mj}证明\end{CJK}: $r(A B) \geqslant r(A)+r(B)-n$;

  \item \begin{CJK}{UTF8}{mj}记\end{CJK} $C$ \begin{CJK}{UTF8}{mj}为\end{CJK} $A$ \begin{CJK}{UTF8}{mj}的一个\end{CJK} $s \times t$ \begin{CJK}{UTF8}{mj}子矩阵\end{CJK}, \begin{CJK}{UTF8}{mj}证明\end{CJK} $r(C) \geqslant r(A)+s+t-m-n$.

\end{enumerate}
\section{8. 武汉大学 2018 年研究生入学考试试题高等代数 
 李扬 
 微信公众号: sxkyliyang}
\begin{CJK}{UTF8}{mj}一\end{CJK}、\begin{CJK}{UTF8}{mj}设\end{CJK} $\alpha, \beta$ \begin{CJK}{UTF8}{mj}是\end{CJK} $n$ \begin{CJK}{UTF8}{mj}维列向量\end{CJK}, \begin{CJK}{UTF8}{mj}且\end{CJK} $\alpha^{T} \beta=3, B=\alpha \beta^{T}, A=E-B$, \begin{CJK}{UTF8}{mj}其中\end{CJK} $E$ \begin{CJK}{UTF8}{mj}是\end{CJK} $n$ \begin{CJK}{UTF8}{mj}阶单位矩阵\end{CJK}.

\begin{enumerate}
  \item \begin{CJK}{UTF8}{mj}证明\end{CJK}: $B^{k}=3^{k-1} B(k \geqslant 0$ \begin{CJK}{UTF8}{mj}为正整数\end{CJK});

  \item \begin{CJK}{UTF8}{mj}证明\end{CJK}: $A+2 E$ \begin{CJK}{UTF8}{mj}和\end{CJK} $A-E$ \begin{CJK}{UTF8}{mj}中至少有一个不可逆\end{CJK};

  \item \begin{CJK}{UTF8}{mj}证明\end{CJK}: $A$ \begin{CJK}{UTF8}{mj}和\end{CJK} $A+E$ \begin{CJK}{UTF8}{mj}都可逆\end{CJK}.

\end{enumerate}
\begin{CJK}{UTF8}{mj}二\end{CJK}、\begin{CJK}{UTF8}{mj}设\end{CJK} $W_{1}, W_{2}$ \begin{CJK}{UTF8}{mj}是数域\end{CJK} $F$ \begin{CJK}{UTF8}{mj}上线性空间\end{CJK} $V$ \begin{CJK}{UTF8}{mj}的子空间\end{CJK}. \begin{CJK}{UTF8}{mj}证明\end{CJK}:
$$
\operatorname{dim} W_{1}+\operatorname{dim} W_{2}=\operatorname{dim}\left(W_{1}+W_{2}\right)-\operatorname{dim}\left(W_{1} \cap W_{2}\right)
$$
\begin{CJK}{UTF8}{mj}三\end{CJK}、\begin{CJK}{UTF8}{mj}设\end{CJK} $V$ \begin{CJK}{UTF8}{mj}是数域\end{CJK} $F$ \begin{CJK}{UTF8}{mj}上的\end{CJK} $n$ \begin{CJK}{UTF8}{mj}维线性空间\end{CJK}, $\alpha_{1}, \alpha_{2}, \cdots, \alpha_{n}$ \begin{CJK}{UTF8}{mj}是\end{CJK} $V$ \begin{CJK}{UTF8}{mj}的一组基\end{CJK}, $\beta_{1}, \beta_{2}, \cdots, \beta_{n}$ \begin{CJK}{UTF8}{mj}是\end{CJK} $V$ \begin{CJK}{UTF8}{mj}任意\end{CJK} $n$ \begin{CJK}{UTF8}{mj}个向量\end{CJK}. \begin{CJK}{UTF8}{mj}证\end{CJK} \begin{CJK}{UTF8}{mj}明\end{CJK}: \begin{CJK}{UTF8}{mj}存在唯一线性变换\end{CJK} $\sigma$, \begin{CJK}{UTF8}{mj}使得\end{CJK} $\sigma\left(\alpha_{i}\right)=\beta_{i}, i=1,2, \cdots, n$.

\begin{CJK}{UTF8}{mj}四\end{CJK}、\begin{CJK}{UTF8}{mj}设\end{CJK} $n$ \begin{CJK}{UTF8}{mj}阶矩阵\end{CJK} $A$ \begin{CJK}{UTF8}{mj}秩为\end{CJK} $r(0<r<n)$, \begin{CJK}{UTF8}{mj}且\end{CJK} $A^{2}=A$, \begin{CJK}{UTF8}{mj}证明\end{CJK}: $A$ \begin{CJK}{UTF8}{mj}与对角矩阵相似\end{CJK}, \begin{CJK}{UTF8}{mj}并求出该对角矩阵\end{CJK}.

\begin{CJK}{UTF8}{mj}五\end{CJK}、\begin{CJK}{UTF8}{mj}判定二次型\end{CJK}
$$
f\left(x_{1}, x_{2}, \cdots, x_{n}\right)=n \sum_{i=1}^{n} x_{i}^{2}-\left(\sum_{i=1}^{n} x_{i}\right)^{2}
$$
\begin{CJK}{UTF8}{mj}的类型\end{CJK}, \begin{CJK}{UTF8}{mj}并说明理由\end{CJK}.

\begin{CJK}{UTF8}{mj}六\end{CJK}、\begin{CJK}{UTF8}{mj}设\end{CJK} $A, B, C$ \begin{CJK}{UTF8}{mj}是\end{CJK} $n$ \begin{CJK}{UTF8}{mj}阶矩阵\end{CJK}, \begin{CJK}{UTF8}{mj}证明\end{CJK}:

\begin{enumerate}
  \item \begin{CJK}{UTF8}{mj}如果\end{CJK} $A^{2}=E$, \begin{CJK}{UTF8}{mj}则\end{CJK} $r(A+E)+r(A-E)=n$;

  \item $r\left(\begin{array}{ll}A & C \\ O & B\end{array}\right) \geqslant r(A)+r(B)$, \begin{CJK}{UTF8}{mj}并给出等号成立的条件\end{CJK};

\end{enumerate}
\begin{CJK}{UTF8}{mj}七\end{CJK}、\begin{CJK}{UTF8}{mj}设\end{CJK} $\sigma$ \begin{CJK}{UTF8}{mj}是数域\end{CJK} $F$ \begin{CJK}{UTF8}{mj}上向量空间\end{CJK} $V$ \begin{CJK}{UTF8}{mj}一个线性变换\end{CJK}, \begin{CJK}{UTF8}{mj}且满足\end{CJK} $\sigma^{2}=\sigma$, \begin{CJK}{UTF8}{mj}证明\end{CJK}:

\begin{enumerate}
  \item $\operatorname{ker} \sigma=\{\xi-\sigma(\xi) \mid \xi \in V\}$;

  \item $V=\operatorname{ker} \sigma \oplus \operatorname{Im} \sigma$;

  \item \begin{CJK}{UTF8}{mj}如果\end{CJK} $\tau$ \begin{CJK}{UTF8}{mj}是\end{CJK} $V$ \begin{CJK}{UTF8}{mj}上的另一个线性变换\end{CJK}, \begin{CJK}{UTF8}{mj}则\end{CJK} $\operatorname{ker} \sigma$ \begin{CJK}{UTF8}{mj}和\end{CJK} $\operatorname{Im} \sigma$ \begin{CJK}{UTF8}{mj}都在\end{CJK} $\tau$ \begin{CJK}{UTF8}{mj}之下不变的充分必要条件是\end{CJK} $\tau \sigma=\sigma \tau$.

\end{enumerate}
\section{1. 西北大学 2009 年研究生入学考试试题数学分析}
\begin{CJK}{UTF8}{mj}李扬\end{CJK}

\begin{CJK}{UTF8}{mj}微信公众号\end{CJK}: sxkyliyang

\begin{CJK}{UTF8}{mj}一\end{CJK}、\begin{CJK}{UTF8}{mj}单项选择题\end{CJK} (\begin{CJK}{UTF8}{mj}本题共\end{CJK} 30 \begin{CJK}{UTF8}{mj}分\end{CJK}, \begin{CJK}{UTF8}{mj}每小题\end{CJK} 6 \begin{CJK}{UTF8}{mj}分\end{CJK})

\begin{enumerate}
  \item \begin{CJK}{UTF8}{mj}若\end{CJK} $a$ \begin{CJK}{UTF8}{mj}是数列\end{CJK} $\left\{x_{n}\right\}_{n=1}^{+\infty}$ \begin{CJK}{UTF8}{mj}的最大聚点\end{CJK}, \begin{CJK}{UTF8}{mj}则\end{CJK} ( )\\
A. $\forall x_{n} \in\left\{x_{n}\right\}$, \begin{CJK}{UTF8}{mj}有\end{CJK} $x_{n} \leqslant a$.\\
B. $\forall \varepsilon>0, \exists N, \forall n>N$, \begin{CJK}{UTF8}{mj}有\end{CJK} $x_{n}<a+\varepsilon$.\\
C. $\exists N, \forall n>N$, \begin{CJK}{UTF8}{mj}有\end{CJK} $x_{n}<a$.\\
D. $\exists N, \forall n>N$, \begin{CJK}{UTF8}{mj}有\end{CJK} $x_{n} \leqslant a$.

  \item \begin{CJK}{UTF8}{mj}下列结论正确的是\end{CJK}()\\
A. \begin{CJK}{UTF8}{mj}若\end{CJK} $x=\varphi(t), y=\psi(t)$, \begin{CJK}{UTF8}{mj}则\end{CJK} $y$ \begin{CJK}{UTF8}{mj}必是\end{CJK} $x$ \begin{CJK}{UTF8}{mj}的函数\end{CJK}.\\
B. \begin{CJK}{UTF8}{mj}若\end{CJK} $f(x)$ \begin{CJK}{UTF8}{mj}在\end{CJK} $(a, b)$ \begin{CJK}{UTF8}{mj}内连续\end{CJK}, \begin{CJK}{UTF8}{mj}则\end{CJK} $f(x)$ \begin{CJK}{UTF8}{mj}在\end{CJK} $(a, b)$ \begin{CJK}{UTF8}{mj}上有界\end{CJK}.\\
C. \begin{CJK}{UTF8}{mj}若\end{CJK} $f(x)$ \begin{CJK}{UTF8}{mj}在\end{CJK} $[a+\varepsilon, b-\varepsilon]$ \begin{CJK}{UTF8}{mj}上连续\end{CJK}, \begin{CJK}{UTF8}{mj}则\end{CJK} $f(x)$ \begin{CJK}{UTF8}{mj}在\end{CJK} $(a, b)$ \begin{CJK}{UTF8}{mj}上一致连续\end{CJK}.\\
D. \begin{CJK}{UTF8}{mj}如果\end{CJK} $\left\{x_{n}\right\}$ \begin{CJK}{UTF8}{mj}是有界数列\end{CJK}, \begin{CJK}{UTF8}{mj}则\end{CJK} $\varlimsup_{n \rightarrow \infty} \leqslant \sup \left\{x_{n}\right\}$.

  \item \begin{CJK}{UTF8}{mj}设\end{CJK} $|f(x)|$ \begin{CJK}{UTF8}{mj}在\end{CJK} $[a, b]$ \begin{CJK}{UTF8}{mj}上可积\end{CJK}, \begin{CJK}{UTF8}{mj}则\end{CJK} $f(x)$ \begin{CJK}{UTF8}{mj}在\end{CJK} $[a, b]$ \begin{CJK}{UTF8}{mj}上\end{CJK}( )\\
A. \begin{CJK}{UTF8}{mj}可积\end{CJK}.\\
B. \begin{CJK}{UTF8}{mj}不可积\end{CJK}.\\
C. \begin{CJK}{UTF8}{mj}不一定可积\end{CJK}.\\
D. \begin{CJK}{UTF8}{mj}只要\end{CJK} $|f(x)|$ \begin{CJK}{UTF8}{mj}连续时\end{CJK}, $f(x)$ \begin{CJK}{UTF8}{mj}就可积\end{CJK}.

  \item \begin{CJK}{UTF8}{mj}级数\end{CJK} $\sum_{n=1}^{\infty}(-1)^{n-1} \frac{x^{n}}{n}$ \begin{CJK}{UTF8}{mj}在\end{CJK} ( )\\
A. $[0,1)$ \begin{CJK}{UTF8}{mj}上一致收敛\end{CJK}.\\
B. $[-1,1]$ \begin{CJK}{UTF8}{mj}上一致收敛\end{CJK}.\\
C. $[1,+\infty)$ \begin{CJK}{UTF8}{mj}上一致收敛\end{CJK}.\\
D. $(-1,1)$ \begin{CJK}{UTF8}{mj}内内闭一致收敛\end{CJK}.

  \item \begin{CJK}{UTF8}{mj}如果\end{CJK} $f(x, y)$ \begin{CJK}{UTF8}{mj}在点\end{CJK} $\left(x_{0}, y_{0}\right)$ \begin{CJK}{UTF8}{mj}沿任意方向的方向导数存在\end{CJK}, \begin{CJK}{UTF8}{mj}则\end{CJK} ( )\\
A. $f(x, y)$ \begin{CJK}{UTF8}{mj}在\end{CJK} $\left(x_{0}, y_{0}\right)$ \begin{CJK}{UTF8}{mj}连续\end{CJK}.\\
B. $f(x, y)$ \begin{CJK}{UTF8}{mj}在\end{CJK} $\left(x_{0}, y_{0}\right)$ \begin{CJK}{UTF8}{mj}全可微\end{CJK}.\\
C. $f_{x}\left(x_{0}, y_{0}\right), f_{y}\left(x_{0}, y_{0}\right)$ \begin{CJK}{UTF8}{mj}都存在\end{CJK}.\\
D. $f_{x}(x, y), f_{y}(x, y)$ \begin{CJK}{UTF8}{mj}在\end{CJK} $\left(x_{0}, y_{0}\right)$ \begin{CJK}{UTF8}{mj}连续\end{CJK}.

\end{enumerate}
\begin{CJK}{UTF8}{mj}二\end{CJK}、\begin{CJK}{UTF8}{mj}解答题\end{CJK} (\begin{CJK}{UTF8}{mj}本题共\end{CJK} 60 \begin{CJK}{UTF8}{mj}分\end{CJK}, \begin{CJK}{UTF8}{mj}每小题\end{CJK} 10 \begin{CJK}{UTF8}{mj}分\end{CJK})

\begin{enumerate}
  \item \begin{CJK}{UTF8}{mj}设\end{CJK} $a>0, x_{0}>0$,
\end{enumerate}
$$
x_{n}=\frac{1}{2}\left(x_{n-1}+\frac{a}{x_{n-1}}\right)(n=1,2,3, \cdots) .
$$
\begin{CJK}{UTF8}{mj}求\end{CJK} $\lim _{n \rightarrow \infty} x_{n}$. 2 . \begin{CJK}{UTF8}{mj}设\end{CJK} $\lim _{x \rightarrow 0} f(x)=0$, \begin{CJK}{UTF8}{mj}且\end{CJK} $f(x)-f\left(\frac{x}{2}\right)=o(x)(x \rightarrow 0)$, \begin{CJK}{UTF8}{mj}求\end{CJK}
$$
\lim _{x \rightarrow 0} \frac{f(x)}{x}
$$

\begin{enumerate}
  \setcounter{enumi}{3}
  \item \begin{CJK}{UTF8}{mj}讨论积分\end{CJK}
\end{enumerate}
$$
\int_{0}^{1} x^{p-1}(1-x)^{q-1} \mathrm{~d} x
$$
\begin{CJK}{UTF8}{mj}的敛散性\end{CJK}.

\begin{enumerate}
  \setcounter{enumi}{4}
  \item \begin{CJK}{UTF8}{mj}设动点\end{CJK} $(x, y)$ \begin{CJK}{UTF8}{mj}在圆周\end{CJK} $x^{2}+y^{2}=1$ \begin{CJK}{UTF8}{mj}上\end{CJK}, \begin{CJK}{UTF8}{mj}求函数\end{CJK}
\end{enumerate}
$$
z=x y
$$
\begin{CJK}{UTF8}{mj}的最大值和最小值\end{CJK}.

\begin{enumerate}
  \setcounter{enumi}{5}
  \item \begin{CJK}{UTF8}{mj}计算二重积分\end{CJK}
\end{enumerate}
$$
I=\iint_{D} \frac{\mathrm{d} x \mathrm{~d} y}{x y\left(\ln ^{2} x+\ln ^{2} y\right)}
$$
\begin{CJK}{UTF8}{mj}其中\end{CJK} $D$ \begin{CJK}{UTF8}{mj}是\end{CJK} $x^{2}+y^{2}=1$ \begin{CJK}{UTF8}{mj}与\end{CJK} $x+y=1$ \begin{CJK}{UTF8}{mj}所围平面区域位于第一象限的部分\end{CJK}.

\begin{enumerate}
  \setcounter{enumi}{6}
  \item \begin{CJK}{UTF8}{mj}计算曲面积分\end{CJK}
\end{enumerate}
$$
I=\iint_{S} x^{2} \mathrm{~d} y \mathrm{~d} z+y^{2} \mathrm{~d} z \mathrm{~d} x+z^{2} \mathrm{~d} x \mathrm{~d} y
$$
\begin{CJK}{UTF8}{mj}其中\end{CJK} $S$ \begin{CJK}{UTF8}{mj}是曲面\end{CJK} $(x-a)^{2}+(y-b)^{2}+(z-c)^{2}=R^{2}$ \begin{CJK}{UTF8}{mj}的外侧\end{CJK}.

\begin{CJK}{UTF8}{mj}三\end{CJK}、\begin{CJK}{UTF8}{mj}证明题\end{CJK} (\begin{CJK}{UTF8}{mj}本题共\end{CJK} 60 \begin{CJK}{UTF8}{mj}分\end{CJK}, \begin{CJK}{UTF8}{mj}每小题\end{CJK} 15 \begin{CJK}{UTF8}{mj}分\end{CJK})

\begin{enumerate}
  \item \begin{CJK}{UTF8}{mj}对任意自然数\end{CJK} $n$ \begin{CJK}{UTF8}{mj}及实数\end{CJK} $\alpha>1$, \begin{CJK}{UTF8}{mj}设\end{CJK}
\end{enumerate}
$$
x_{n}=1+\frac{1}{2^{\alpha}}+\frac{1}{3^{\alpha}}+\cdots+\frac{1}{n^{\alpha}},
$$
\begin{CJK}{UTF8}{mj}则数列\end{CJK} $\left\{x_{n}\right\}$ \begin{CJK}{UTF8}{mj}收敛\end{CJK}.

\begin{enumerate}
  \setcounter{enumi}{2}
  \item \begin{CJK}{UTF8}{mj}设函数\end{CJK} $f(x), g(x)$ \begin{CJK}{UTF8}{mj}在\end{CJK} $[a, b]$ \begin{CJK}{UTF8}{mj}上连续\end{CJK}, \begin{CJK}{UTF8}{mj}且存在\end{CJK} $x_{n} \in[a, b]$ \begin{CJK}{UTF8}{mj}使\end{CJK} $f\left(x_{n+1}\right)=g\left(x_{n}\right)(n=1,2,3, \cdots)$, \begin{CJK}{UTF8}{mj}证明\end{CJK}: \begin{CJK}{UTF8}{mj}必存在\end{CJK} $x_{0} \in[a, b]$ \begin{CJK}{UTF8}{mj}使\end{CJK}
\end{enumerate}
$$
f\left(x_{0}\right)=g\left(x_{0}\right)
$$

\begin{enumerate}
  \setcounter{enumi}{3}
  \item \begin{CJK}{UTF8}{mj}若对任意自然数\end{CJK} $m$, \begin{CJK}{UTF8}{mj}当\end{CJK} $x \geqslant m$ \begin{CJK}{UTF8}{mj}时\end{CJK}, $f(x)$ \begin{CJK}{UTF8}{mj}是一非负单增函数\end{CJK}, \begin{CJK}{UTF8}{mj}则对任意\end{CJK} $\xi \geqslant m$ \begin{CJK}{UTF8}{mj}有\end{CJK}
\end{enumerate}
$$
\left|\sum_{k=m}^{[\xi]} f(k)-\int_{m}^{\xi} f(x) \mathrm{d} x\right| \leqslant f(\xi)
$$

\begin{enumerate}
  \setcounter{enumi}{4}
  \item \begin{CJK}{UTF8}{mj}设\end{CJK} $f_{1}(x)$ \begin{CJK}{UTF8}{mj}在\end{CJK} $[a, b]$ \begin{CJK}{UTF8}{mj}上\end{CJK} Riemann \begin{CJK}{UTF8}{mj}可积\end{CJK}, \begin{CJK}{UTF8}{mj}且\end{CJK}
\end{enumerate}
$$
f_{n+1}(x)=\int_{a}^{x} f_{n}(t) \mathrm{d} t, n=1,2,3, \cdots
$$
\begin{CJK}{UTF8}{mj}则函数列\end{CJK} $\left\{f_{n}(x)\right\}$ \begin{CJK}{UTF8}{mj}在\end{CJK} $[a, b]$ \begin{CJK}{UTF8}{mj}上一致收敛于\end{CJK} 0 .

\section{2. 西北大学 2010 年研究生入学考试试题数学分析 
 李扬 
 微信公众号: sxkyliyang}
\begin{enumerate}
  \item (15 \begin{CJK}{UTF8}{mj}分\end{CJK}) \begin{CJK}{UTF8}{mj}证明若\end{CJK} $f(x)$ \begin{CJK}{UTF8}{mj}在\end{CJK} $[a, b]$ \begin{CJK}{UTF8}{mj}上连续\end{CJK}, \begin{CJK}{UTF8}{mj}则\end{CJK} $f(x)$ \begin{CJK}{UTF8}{mj}在\end{CJK} $[a, b]$ \begin{CJK}{UTF8}{mj}上必有最大值和最小值\end{CJK}.

  \item (18 \begin{CJK}{UTF8}{mj}分\end{CJK}) \begin{CJK}{UTF8}{mj}讨论函数\end{CJK}

\end{enumerate}
$$
z=f(x, y)= \begin{cases}\left(x^{2}+y^{2}\right) \sin \frac{1}{\sqrt{x^{2}+y^{2}}}, & x^{2}+y^{2} \neq 0 \\ 0, & x^{2}+y^{2}=0\end{cases}
$$
\begin{CJK}{UTF8}{mj}在坐标原点处\end{CJK}:\\
(1) \begin{CJK}{UTF8}{mj}是否连续\end{CJK}?\\
(2) \begin{CJK}{UTF8}{mj}是否存在偏导数\end{CJK}?\\
(3) \begin{CJK}{UTF8}{mj}是否可微\end{CJK}?

\begin{enumerate}
  \setcounter{enumi}{3}
  \item ( 15 \begin{CJK}{UTF8}{mj}分\end{CJK}) \begin{CJK}{UTF8}{mj}设级数\end{CJK} $\sum_{n=1}^{\infty} a_{n}$ \begin{CJK}{UTF8}{mj}收敛\end{CJK}, $a_{n}>0$, \begin{CJK}{UTF8}{mj}且数列\end{CJK} $\left\{a_{n}\right\}$ \begin{CJK}{UTF8}{mj}单调递减\end{CJK}, \begin{CJK}{UTF8}{mj}试证\end{CJK}
\end{enumerate}
$$
\lim _{n \rightarrow+\infty} n a_{n}=0 .
$$

\begin{enumerate}
  \setcounter{enumi}{4}
  \item (12 \begin{CJK}{UTF8}{mj}分\end{CJK}) \begin{CJK}{UTF8}{mj}确定\end{CJK}
\end{enumerate}
$$
f(x, y)=4 x+x y^{2}+y^{2}
$$
\begin{CJK}{UTF8}{mj}在圆形区域\end{CJK} $x^{2}+y^{2} \leq 1$ \begin{CJK}{UTF8}{mj}上的最大值和最小值\end{CJK}.

\begin{enumerate}
  \setcounter{enumi}{5}
  \item ( 15 \begin{CJK}{UTF8}{mj}分\end{CJK}) \begin{CJK}{UTF8}{mj}设\end{CJK} $f(x)>0$ \begin{CJK}{UTF8}{mj}且在\end{CJK} $[0,1]$ \begin{CJK}{UTF8}{mj}上连续\end{CJK}, \begin{CJK}{UTF8}{mj}研究函数\end{CJK}
\end{enumerate}
$$
g(y)=\int_{0}^{1} \frac{y f(x)}{x^{2}+y^{2}} \mathrm{~d} x
$$
\begin{CJK}{UTF8}{mj}的连续性\end{CJK}.

\begin{enumerate}
  \setcounter{enumi}{6}
  \item ( 15 \begin{CJK}{UTF8}{mj}分\end{CJK}) \begin{CJK}{UTF8}{mj}设\end{CJK} $f(x)$ \begin{CJK}{UTF8}{mj}在\end{CJK} $[0,1]$ \begin{CJK}{UTF8}{mj}上可微\end{CJK}, \begin{CJK}{UTF8}{mj}且当\end{CJK} $x \in(0,1)$ \begin{CJK}{UTF8}{mj}时\end{CJK}, $0<f^{\prime}(x)<1, f(0)=0$. \begin{CJK}{UTF8}{mj}试证\end{CJK}:
\end{enumerate}
$$
\left(\int_{0}^{1} f(x) \mathrm{d} x\right)^{2}>\int_{0}^{1} f^{3}(x) \mathrm{d} x
$$

\begin{enumerate}
  \setcounter{enumi}{7}
  \item (15 \begin{CJK}{UTF8}{mj}分\end{CJK}) \begin{CJK}{UTF8}{mj}计算曲面积分\end{CJK}
\end{enumerate}
$$
I=\iint_{\Sigma}\left(x^{3}+a z^{2}\right) \mathrm{d} y \mathrm{~d} z+\left(y^{3}+a x^{2}\right) \mathrm{d} z \mathrm{~d} x+\left(z^{3}+a y^{2}\right) \mathrm{d} x \mathrm{~d} y
$$
\begin{CJK}{UTF8}{mj}其中\end{CJK} $\Sigma$ \begin{CJK}{UTF8}{mj}为上半球面\end{CJK} $z=\sqrt{a^{2}-x^{2}-y^{2}}$ \begin{CJK}{UTF8}{mj}的上侧\end{CJK}.

\begin{enumerate}
  \setcounter{enumi}{8}
  \item (15 \begin{CJK}{UTF8}{mj}分\end{CJK}) \begin{CJK}{UTF8}{mj}设\end{CJK} $f(x)$ \begin{CJK}{UTF8}{mj}在\end{CJK} $[a,+\infty)$ \begin{CJK}{UTF8}{mj}上一致连续\end{CJK}, $\varphi(x)$ \begin{CJK}{UTF8}{mj}在\end{CJK} $[a,+\infty)$ \begin{CJK}{UTF8}{mj}上连续\end{CJK},
\end{enumerate}
$$
\lim _{x \rightarrow+\infty}[f(x)-\varphi(x)]=0,
$$
\begin{CJK}{UTF8}{mj}证明\end{CJK}: $\varphi(x)$ \begin{CJK}{UTF8}{mj}在\end{CJK} $[a,+\infty)$ \begin{CJK}{UTF8}{mj}上一致连续\end{CJK}.

\begin{enumerate}
  \setcounter{enumi}{9}
  \item ( 15 \begin{CJK}{UTF8}{mj}分\end{CJK}) \begin{CJK}{UTF8}{mj}证明\end{CJK}: \begin{CJK}{UTF8}{mj}若函数\end{CJK} $f(x)$ \begin{CJK}{UTF8}{mj}在\end{CJK} $(0,+\infty)$ \begin{CJK}{UTF8}{mj}内可微\end{CJK}, \begin{CJK}{UTF8}{mj}且\end{CJK} $\lim _{x \rightarrow+\infty} f^{\prime}(x)=0$, \begin{CJK}{UTF8}{mj}则\end{CJK}
\end{enumerate}
$$
\lim _{x \rightarrow+\infty} \frac{f(x)}{x}=0 .
$$

\begin{enumerate}
  \setcounter{enumi}{10}
  \item (15 \begin{CJK}{UTF8}{mj}分\end{CJK}) \begin{CJK}{UTF8}{mj}证明\end{CJK}:
\end{enumerate}
$$
\left\{\begin{aligned}
\mathrm{e}^{x u} \cos y v &=\frac{x}{\sqrt{2}} \\
\mathrm{e}^{x u} \sin y v &=\frac{y}{\sqrt{2}}
\end{aligned}\right.
$$
\begin{CJK}{UTF8}{mj}在点\end{CJK} $p_{0}=\left(x_{0}, y_{0}, u_{0}, v_{0}\right)=\left(1,1,0, \frac{\pi}{4}\right)$ \begin{CJK}{UTF8}{mj}的某领域里确定了唯一的隐函数\end{CJK} $u=u(x, y), v=v(x, y)$. \begin{CJK}{UTF8}{mj}并求\end{CJK} $\mathrm{d}^{2} u$ \begin{CJK}{UTF8}{mj}在点\end{CJK} $p_{0}$ \begin{CJK}{UTF8}{mj}处的值\end{CJK}.

\section{3. 西北大学 2011 年研究生入学考试试题数学分析}
\begin{CJK}{UTF8}{mj}李扬\end{CJK}

\begin{CJK}{UTF8}{mj}微信公众号\end{CJK}: sxkyliyang

\begin{CJK}{UTF8}{mj}一\end{CJK}、\begin{CJK}{UTF8}{mj}计算题\end{CJK} (\begin{CJK}{UTF8}{mj}本题共\end{CJK} 50 \begin{CJK}{UTF8}{mj}分\end{CJK}, \begin{CJK}{UTF8}{mj}每小题\end{CJK} 10 \begin{CJK}{UTF8}{mj}分\end{CJK})

\begin{enumerate}
  \item (1) \begin{CJK}{UTF8}{mj}求极限\end{CJK} $\lim _{x \rightarrow 0} x\left[\frac{1}{x}\right]$, \begin{CJK}{UTF8}{mj}其中\end{CJK} $[x]$ \begin{CJK}{UTF8}{mj}是\end{CJK} $x$ \begin{CJK}{UTF8}{mj}的最大整数部分\end{CJK}.
\end{enumerate}
(2) \begin{CJK}{UTF8}{mj}求积分\end{CJK}
$$
\int_{0}^{+\infty} \frac{\ln x}{1+x^{2}} \mathrm{~d} x
$$

\begin{enumerate}
  \setcounter{enumi}{2}
  \item \begin{CJK}{UTF8}{mj}求极限\end{CJK}
\end{enumerate}
$$
\lim _{n \rightarrow 0}\left(\frac{1}{n}+\frac{1}{n+1}+\cdots+\frac{1}{2 n}\right)
$$

\begin{enumerate}
  \setcounter{enumi}{3}
  \item \begin{CJK}{UTF8}{mj}求极限\end{CJK}
\end{enumerate}
$$
\lim _{x \rightarrow 0^{+}}\left(\frac{1}{x^{5}} \int_{0}^{x} \mathrm{e}^{-t^{2}} \mathrm{~d} t-\frac{1}{x^{4}}+\frac{1}{3 x^{2}}\right)
$$

\begin{enumerate}
  \setcounter{enumi}{4}
  \item \begin{CJK}{UTF8}{mj}计算\end{CJK}
\end{enumerate}
$$
\int_{l} x^{2} \mathrm{~d} s
$$
\begin{CJK}{UTF8}{mj}其中\end{CJK} $l$ \begin{CJK}{UTF8}{mj}是球面\end{CJK} $x^{2}+y^{2}+z^{2}=1$ \begin{CJK}{UTF8}{mj}与平面\end{CJK} $x+y+z=0$ \begin{CJK}{UTF8}{mj}的交线\end{CJK}.

\begin{enumerate}
  \setcounter{enumi}{5}
  \item \begin{CJK}{UTF8}{mj}计算曲面积分\end{CJK}
\end{enumerate}
$$
I=\iint_{S} x^{3} \mathrm{~d} y \mathrm{~d} z+y^{3} \mathrm{~d} z \mathrm{~d} x+z^{3} \mathrm{~d} x \mathrm{~d} y
$$
\begin{CJK}{UTF8}{mj}其中\end{CJK} $S$ \begin{CJK}{UTF8}{mj}为球面\end{CJK} $x^{2}+y^{2}+z^{2}=1$ \begin{CJK}{UTF8}{mj}的外侧\end{CJK}.

\section{一、解答题}
\begin{enumerate}
  \item ( 10 \begin{CJK}{UTF8}{mj}分\end{CJK}) \begin{CJK}{UTF8}{mj}证明函数\end{CJK}
\end{enumerate}
$$
f(x)=\sum_{n=1}^{\infty} \frac{\sin n x}{n^{3}}
$$
\begin{CJK}{UTF8}{mj}在\end{CJK} $(-\infty,+\infty)$ \begin{CJK}{UTF8}{mj}内连续\end{CJK}.

\begin{enumerate}
  \setcounter{enumi}{2}
  \item (15 \begin{CJK}{UTF8}{mj}分\end{CJK}) \begin{CJK}{UTF8}{mj}设\end{CJK} $f^{\prime}(x)$ \begin{CJK}{UTF8}{mj}在\end{CJK} $[0, a]$ \begin{CJK}{UTF8}{mj}上连续\end{CJK}, \begin{CJK}{UTF8}{mj}且\end{CJK} $f(0)=0$, \begin{CJK}{UTF8}{mj}证明\end{CJK}:
\end{enumerate}
$$
\left|\int_{0}^{a} f(x) \mathrm{d} x\right| \leq \frac{M a^{2}}{2},
$$
\begin{CJK}{UTF8}{mj}其中\end{CJK} $M=\max _{0 \leq x \leq a}|f(x)|$.

\begin{enumerate}
  \setcounter{enumi}{3}
  \item ( 15 \begin{CJK}{UTF8}{mj}分\end{CJK}) \begin{CJK}{UTF8}{mj}设函数\end{CJK} $f(x)$ \begin{CJK}{UTF8}{mj}在\end{CJK} $[a, b]$ \begin{CJK}{UTF8}{mj}上单调增加\end{CJK}, \begin{CJK}{UTF8}{mj}且有\end{CJK} $f(a) \geq a, f(b) \leq b$, \begin{CJK}{UTF8}{mj}证明存在\end{CJK} $x_{0} \in[a, b]$, \begin{CJK}{UTF8}{mj}使\end{CJK}
\end{enumerate}
$$
f\left(x_{0}\right)=x_{0} .
$$

\begin{enumerate}
  \setcounter{enumi}{4}
  \item (15 \begin{CJK}{UTF8}{mj}分\end{CJK}) \begin{CJK}{UTF8}{mj}讨论函数\end{CJK}
\end{enumerate}
$$
f(x, y)= \begin{cases}\frac{x y}{\sqrt{x^{2}+y^{2}}}, & (x, y) \neq(0,0) \\ 0, & (x, y)=(0,0)\end{cases}
$$
\begin{CJK}{UTF8}{mj}在\end{CJK} $(0,0)$ \begin{CJK}{UTF8}{mj}点的连续性和可微性\end{CJK}. 5. ( 15 \begin{CJK}{UTF8}{mj}分\end{CJK}) \begin{CJK}{UTF8}{mj}若函数\end{CJK} $f(x)$ \begin{CJK}{UTF8}{mj}在\end{CJK} $[0,+\infty)$ \begin{CJK}{UTF8}{mj}上一致连续\end{CJK}, \begin{CJK}{UTF8}{mj}且无穷积分\end{CJK} $\int_{0}^{+\infty} f(x) \mathrm{d} x$ \begin{CJK}{UTF8}{mj}收敛\end{CJK}, \begin{CJK}{UTF8}{mj}则\end{CJK}
$$
\lim _{x \rightarrow+\infty} f(x)=0 .
$$

\begin{enumerate}
  \setcounter{enumi}{6}
  \item ( 15 \begin{CJK}{UTF8}{mj}分\end{CJK}) \begin{CJK}{UTF8}{mj}设\end{CJK} $f(x, y)$ \begin{CJK}{UTF8}{mj}在点\end{CJK} $(0,0)$ \begin{CJK}{UTF8}{mj}的某个领域中连续\end{CJK},
\end{enumerate}
$$
F(t)=\iint_{x^{2}+y^{2} \leq t^{2}} f(x, y) \mathrm{d} x \mathrm{~d} y
$$
\begin{CJK}{UTF8}{mj}求\end{CJK} $\lim _{t \rightarrow 0^{+}} \frac{F^{\prime}(t)}{t}$.

\begin{enumerate}
  \setcounter{enumi}{7}
  \item ( 15 \begin{CJK}{UTF8}{mj}分\end{CJK}) \begin{CJK}{UTF8}{mj}设\end{CJK}
\end{enumerate}
$$
F(x)=\mathrm{e}^{\frac{x^{2}}{2}} \int_{x}^{+\infty} \mathrm{e}^{-\frac{t^{2}}{2}} \mathrm{~d} t, x \in[0,+\infty)
$$
\begin{CJK}{UTF8}{mj}试证\end{CJK}:

(1) $\lim _{x \rightarrow+\infty} F(x)=0$;

(2) $F(x)$ \begin{CJK}{UTF8}{mj}在\end{CJK} $(0,+\infty)$ \begin{CJK}{UTF8}{mj}内单调递减\end{CJK}.

\section{4. 西北大学 2012 年研究生入学考试试题数学分析}
\begin{CJK}{UTF8}{mj}李扬\end{CJK}

\begin{CJK}{UTF8}{mj}微信公众号\end{CJK}: sxkyliyang

\begin{enumerate}
  \item (10 \begin{CJK}{UTF8}{mj}分\end{CJK}) \begin{CJK}{UTF8}{mj}求\end{CJK}
\end{enumerate}
$$
\lim _{x \rightarrow 0} \frac{\int_{0}^{x} \mathrm{e}^{\frac{t^{2}}{2}} \cos t \mathrm{~d} t-x}{\left(\mathrm{e}^{x}-1\right)^{2}\left(1-\cos ^{2} x\right) \arctan x} .
$$

\begin{enumerate}
  \setcounter{enumi}{2}
  \item (10 \begin{CJK}{UTF8}{mj}分\end{CJK}) \begin{CJK}{UTF8}{mj}求极限\end{CJK}
\end{enumerate}
$$
\lim _{n \rightarrow \infty} \int_{0}^{1} x^{n} \sqrt{x+3} \mathrm{~d} x
$$

\begin{enumerate}
  \setcounter{enumi}{3}
  \item (12 \begin{CJK}{UTF8}{mj}分\end{CJK}) \begin{CJK}{UTF8}{mj}已知\end{CJK}
\end{enumerate}
$$
\lim _{x \rightarrow+\infty}\left(\frac{x+c}{x-c}\right)^{x}=\int_{-\infty}^{c} t \mathrm{e}^{t} \mathrm{~d} t
$$
\begin{CJK}{UTF8}{mj}求常数\end{CJK} $c$ \begin{CJK}{UTF8}{mj}的值\end{CJK}.

\begin{enumerate}
  \setcounter{enumi}{4}
  \item (12 \begin{CJK}{UTF8}{mj}分\end{CJK}) \begin{CJK}{UTF8}{mj}求曲面积分\end{CJK}
\end{enumerate}
$$
\iint_{S}(x+y) z^{2} \mathrm{~d} y \mathrm{~d} z+\left(x^{2} y-2 z\right) \mathrm{d} z \mathrm{~d} x+(2 x+z) y^{2} \mathrm{~d} x \mathrm{~d} y
$$
\begin{CJK}{UTF8}{mj}其中\end{CJK} $S$ \begin{CJK}{UTF8}{mj}是半球面\end{CJK} $z=\sqrt{1-x^{2}-y^{2}}\left(x^{2}+y^{2} \leq 1\right)$ \begin{CJK}{UTF8}{mj}的上侧\end{CJK}.

\begin{enumerate}
  \setcounter{enumi}{5}
  \item (10 \begin{CJK}{UTF8}{mj}分\end{CJK}) \begin{CJK}{UTF8}{mj}计算积分\end{CJK}:
\end{enumerate}
$$
I=\int_{\Gamma} \frac{x \mathrm{~d} y-y \mathrm{~d} x}{x^{2}+y^{2}}
$$
\begin{CJK}{UTF8}{mj}其中\end{CJK} $\Gamma$ \begin{CJK}{UTF8}{mj}为包含原点的一条分段光滑闭曲线\end{CJK}, \begin{CJK}{UTF8}{mj}取正方向\end{CJK}.

\begin{enumerate}
  \setcounter{enumi}{6}
  \item (10 \begin{CJK}{UTF8}{mj}分\end{CJK}) \begin{CJK}{UTF8}{mj}证明函数\end{CJK}
\end{enumerate}
$$
f(x)=\sum_{n=1}^{\infty} \frac{x^{n}}{n^{2} \ln (1+n)}
$$
\begin{CJK}{UTF8}{mj}在\end{CJK} $[-1,1]$ \begin{CJK}{UTF8}{mj}上连续\end{CJK}.

\begin{enumerate}
  \setcounter{enumi}{7}
  \item (15 \begin{CJK}{UTF8}{mj}分\end{CJK}) \begin{CJK}{UTF8}{mj}叙述并证明闭区间上连续函数的零点存在定理\end{CJK}.

  \item (15 \begin{CJK}{UTF8}{mj}分\end{CJK}) \begin{CJK}{UTF8}{mj}证明积分\end{CJK}

\end{enumerate}
$$
\int_{0}^{+\infty} \frac{\cos \left(x^{2}\right)}{x^{p}} \mathrm{~d} x
$$
\begin{CJK}{UTF8}{mj}在\end{CJK} $|p| \leq p_{0}<1$ \begin{CJK}{UTF8}{mj}上一致收敛\end{CJK}.

\begin{enumerate}
  \setcounter{enumi}{9}
  \item (12 \begin{CJK}{UTF8}{mj}分\end{CJK}) \begin{CJK}{UTF8}{mj}设\end{CJK} $a>0$, \begin{CJK}{UTF8}{mj}求曲线\end{CJK}
\end{enumerate}
$$
\left\{\begin{array}{l}
x^{2}+y^{2}=2 a z \\
x^{2}+y^{2}+x y=a^{2}
\end{array}\right.
$$
\begin{CJK}{UTF8}{mj}上的点到\end{CJK} $x y$ \begin{CJK}{UTF8}{mj}平面的最大和最小距离\end{CJK}.

\begin{enumerate}
  \setcounter{enumi}{10}
  \item ( 14 \begin{CJK}{UTF8}{mj}分\end{CJK}) \begin{CJK}{UTF8}{mj}证明\end{CJK}: \begin{CJK}{UTF8}{mj}若函数\end{CJK} $f(x)$ \begin{CJK}{UTF8}{mj}在\end{CJK} $(a,+\infty)$ \begin{CJK}{UTF8}{mj}可导\end{CJK}, \begin{CJK}{UTF8}{mj}且\end{CJK} $\forall x \in(a,+\infty)$, \begin{CJK}{UTF8}{mj}有\end{CJK} $\left|f^{\prime}(x)\right| \leq M$, \begin{CJK}{UTF8}{mj}其中\end{CJK} $M$ \begin{CJK}{UTF8}{mj}是常数\end{CJK}, \begin{CJK}{UTF8}{mj}则\end{CJK} $f(x)$ \begin{CJK}{UTF8}{mj}在\end{CJK} $(a,+\infty)$ \begin{CJK}{UTF8}{mj}一致连续\end{CJK}.

  \item ( 15 \begin{CJK}{UTF8}{mj}分\end{CJK}) \begin{CJK}{UTF8}{mj}设\end{CJK} $\varphi(x)$ \begin{CJK}{UTF8}{mj}在\end{CJK} $[0, \infty)$ \begin{CJK}{UTF8}{mj}上有连续导数\end{CJK}, \begin{CJK}{UTF8}{mj}并且\end{CJK} $\varphi(0)=1$, \begin{CJK}{UTF8}{mj}令\end{CJK}

\end{enumerate}
$$
f(r)=\iiint_{x^{2}+y^{2}+z^{2} \leq r^{2}} \varphi\left(x^{2}+y^{2}+z^{2}\right) \mathrm{d} x \mathrm{~d} y \mathrm{~d} z,(r \geq 0),
$$
\begin{CJK}{UTF8}{mj}证明\end{CJK}: $f(r)$ \begin{CJK}{UTF8}{mj}在\end{CJK} $r=0$ \begin{CJK}{UTF8}{mj}处三次可微\end{CJK}, \begin{CJK}{UTF8}{mj}并求\end{CJK} $f_{+}^{\prime \prime \prime}(0)$ (\begin{CJK}{UTF8}{mj}右导数\end{CJK}).

\begin{enumerate}
  \setcounter{enumi}{12}
  \item (15 \begin{CJK}{UTF8}{mj}分\end{CJK}) \begin{CJK}{UTF8}{mj}设函数列\end{CJK} $\left\{f_{n}(x)\right\}$ \begin{CJK}{UTF8}{mj}与\end{CJK} $\left\{g_{n}(x)\right\}$ \begin{CJK}{UTF8}{mj}在区间\end{CJK} $I$ \begin{CJK}{UTF8}{mj}上分别一致收敛于\end{CJK} $f(x)$ \begin{CJK}{UTF8}{mj}与\end{CJK} $g(x)$, \begin{CJK}{UTF8}{mj}且假定\end{CJK} $f(x), g(x)$ \begin{CJK}{UTF8}{mj}都在\end{CJK} $I$ \begin{CJK}{UTF8}{mj}上\end{CJK} \begin{CJK}{UTF8}{mj}有界\end{CJK}, \begin{CJK}{UTF8}{mj}试证明\end{CJK}: $\left\{f_{n}(x) g_{n}(x)\right\}$ \begin{CJK}{UTF8}{mj}在\end{CJK} $I$ \begin{CJK}{UTF8}{mj}上一致收敛于\end{CJK} $f(x) g(x)$.
\end{enumerate}
\section{5. 西北大学 2013 年研究生入学考试试题数学分析}
\begin{CJK}{UTF8}{mj}李扬\end{CJK}

\begin{CJK}{UTF8}{mj}微信公众号\end{CJK}: sxkyliyang

\begin{CJK}{UTF8}{mj}一\end{CJK}、\begin{CJK}{UTF8}{mj}计算题\end{CJK} (\begin{CJK}{UTF8}{mj}本题共\end{CJK} 52 \begin{CJK}{UTF8}{mj}分\end{CJK}, \begin{CJK}{UTF8}{mj}前五题每题\end{CJK} 8 \begin{CJK}{UTF8}{mj}分\end{CJK}, \begin{CJK}{UTF8}{mj}第六题\end{CJK} 12 \begin{CJK}{UTF8}{mj}分\end{CJK})

\begin{enumerate}
  \item \begin{CJK}{UTF8}{mj}求极限\end{CJK}
\end{enumerate}
$$
\lim _{n \rightarrow \infty}\left(\frac{n}{n^{2}+1^{2}}+\frac{n}{n^{2}+2^{2}}+\cdots+\frac{n}{n^{2}+n^{2}}\right)
$$

\begin{enumerate}
  \setcounter{enumi}{2}
  \item \begin{CJK}{UTF8}{mj}求极限\end{CJK}
\end{enumerate}
$$
\lim _{x \rightarrow 1}(2-x)^{\tan \frac{\pi x}{2}} .
$$

\begin{enumerate}
  \setcounter{enumi}{3}
  \item \begin{CJK}{UTF8}{mj}已知\end{CJK} $f(x)$ \begin{CJK}{UTF8}{mj}在\end{CJK} $x=a$ \begin{CJK}{UTF8}{mj}处可导\end{CJK}, \begin{CJK}{UTF8}{mj}且\end{CJK} $f(a)>0$, \begin{CJK}{UTF8}{mj}求\end{CJK}
\end{enumerate}
$$
\lim _{n \rightarrow \infty}\left[\frac{f\left(a+\frac{1}{n}\right)}{f(a)}\right]^{n}
$$

\begin{enumerate}
  \setcounter{enumi}{4}
  \item \begin{CJK}{UTF8}{mj}设\end{CJK} $f(x, y)=\int_{0}^{x y} \mathrm{e}^{-t^{2}} \mathrm{~d} t$, \begin{CJK}{UTF8}{mj}求\end{CJK}
\end{enumerate}
$$
\frac{x}{y} \frac{\partial^{2} f}{\partial x^{2}}-2 \frac{\partial^{2} f}{\partial x \partial y}+\frac{y}{x} \frac{\partial^{2} f}{\partial y^{2}}
$$

\begin{enumerate}
  \setcounter{enumi}{5}
  \item \begin{CJK}{UTF8}{mj}计算\end{CJK}
\end{enumerate}
$$
I=\int_{(0,0)}^{(a, b)} \mathrm{e}^{x} \cos y \mathrm{~d} x-\mathrm{e}^{x} \sin y \mathrm{~d} y
$$

\begin{enumerate}
  \setcounter{enumi}{6}
  \item \begin{CJK}{UTF8}{mj}求曲面积分\end{CJK}
\end{enumerate}
$$
\iint_{S}\left(y^{2}-x\right) \mathrm{d} y \mathrm{~d} z+\left(z^{2}-y\right) \mathrm{d} z \mathrm{~d} x+\left(x^{2}-z\right) \mathrm{d} x \mathrm{~d} y
$$
\begin{CJK}{UTF8}{mj}其中\end{CJK} $S$ \begin{CJK}{UTF8}{mj}是曲面\end{CJK} $z=2-x^{2}-y^{2}(1 \leq z \leq 2)$ \begin{CJK}{UTF8}{mj}的上侧\end{CJK}.

\section{一、解答题}
\begin{enumerate}
  \item (15 \begin{CJK}{UTF8}{mj}分\end{CJK}) \begin{CJK}{UTF8}{mj}叙述并证明闭区间上连续函数的有界性定理\end{CJK}.

  \item ( 15 \begin{CJK}{UTF8}{mj}分\end{CJK}) \begin{CJK}{UTF8}{mj}设\end{CJK}

\end{enumerate}
$$
f(x, y)=\sin x+\cos y+\cos (x-y),
$$
\begin{CJK}{UTF8}{mj}求\end{CJK} $f(x, y)$ \begin{CJK}{UTF8}{mj}在区域\end{CJK} $0 \leq x \leq \frac{\pi}{2}, 0 \leq y \leq \frac{\pi}{2}$ \begin{CJK}{UTF8}{mj}内的最大值和最小值\end{CJK}.

\begin{enumerate}
  \setcounter{enumi}{3}
  \item ( 15 \begin{CJK}{UTF8}{mj}分\end{CJK}) \begin{CJK}{UTF8}{mj}判断级数\end{CJK}
\end{enumerate}
$$
\sum_{n=2}^{+\infty} \frac{\ln \ln n}{\ln n} \sin n
$$
\begin{CJK}{UTF8}{mj}的绝对收敛性和条件收敛性\end{CJK}.

\begin{enumerate}
  \setcounter{enumi}{4}
  \item (15 \begin{CJK}{UTF8}{mj}分\end{CJK}) \begin{CJK}{UTF8}{mj}求极限\end{CJK}
\end{enumerate}
$$
\lim _{n \rightarrow \infty} \int_{0}^{1} \frac{\mathrm{d} x}{1+\left(1+\frac{x}{n}\right)^{n}}
$$

\begin{enumerate}
  \setcounter{enumi}{5}
  \item ( 14 \begin{CJK}{UTF8}{mj}分\end{CJK}) \begin{CJK}{UTF8}{mj}设\end{CJK} $f(x)$ \begin{CJK}{UTF8}{mj}在\end{CJK} $[0,1]$ \begin{CJK}{UTF8}{mj}上可导\end{CJK}, \begin{CJK}{UTF8}{mj}当\end{CJK} $0 \leq x \leq 1$ \begin{CJK}{UTF8}{mj}时\end{CJK}, $0 \leq f(x) \leq 1$, \begin{CJK}{UTF8}{mj}且对于区间\end{CJK} $(0,1)$ \begin{CJK}{UTF8}{mj}内所有\end{CJK} $x$ \begin{CJK}{UTF8}{mj}有\end{CJK} $f^{\prime}(x) \neq 1$, \begin{CJK}{UTF8}{mj}证\end{CJK} \begin{CJK}{UTF8}{mj}明在\end{CJK} $[0,1]$ \begin{CJK}{UTF8}{mj}上有且仅有一个\end{CJK} $x_{0}$, \begin{CJK}{UTF8}{mj}使\end{CJK}
\end{enumerate}
$$
f\left(x_{0}\right)=x_{0}
$$

\begin{enumerate}
  \setcounter{enumi}{6}
  \item ( 24 \begin{CJK}{UTF8}{mj}分\end{CJK}) \begin{CJK}{UTF8}{mj}设\end{CJK} $f(u)$ \begin{CJK}{UTF8}{mj}具有连续的导函数且\end{CJK}
\end{enumerate}
$$
\begin{gathered}
\lim _{u \rightarrow+\infty} f^{\prime}(u)=A>0,(R>0) \\
D=\left\{(x, y): x^{2}+y^{2} \leq R^{2}, x, y \geq 0\right\}
\end{gathered}
$$
(1) \begin{CJK}{UTF8}{mj}证明\end{CJK} $\lim _{u \rightarrow+\infty} f(u)=+\infty$;

(2) \begin{CJK}{UTF8}{mj}求\end{CJK} $I_{R}=\iint_{D} f^{\prime}\left(x^{2}+y^{2}\right) \mathrm{d} x \mathrm{~d} y$;

(3) \begin{CJK}{UTF8}{mj}求\end{CJK} $\lim _{R \rightarrow+\infty} \frac{I_{R}}{R^{2}}$.

\section{6. 西北大学 2014 年研究生入学考试试题数学分析 
 李扬 
 微信公众号: sxkyliyang}
\begin{enumerate}
  \item \begin{CJK}{UTF8}{mj}求极限\end{CJK}:
\end{enumerate}
( 8 \begin{CJK}{UTF8}{mj}分\end{CJK}) (1)
$$
\lim _{x \rightarrow 0} \frac{\sqrt[3]{1+3 x^{4}}-\sqrt{1-2 x}}{\sqrt[3]{1+x}-\sqrt{1+x}}
$$
(8 \begin{CJK}{UTF8}{mj}分\end{CJK}) (2) \begin{CJK}{UTF8}{mj}设函数\end{CJK} $f(x)$ \begin{CJK}{UTF8}{mj}在\end{CJK} $x=0$ \begin{CJK}{UTF8}{mj}的某邻域内有定义\end{CJK}, $\lim _{x \rightarrow 0} \frac{f(x)}{x^{2}}=1$, \begin{CJK}{UTF8}{mj}求\end{CJK}
$$
\lim _{x \rightarrow 0}\left(1+\frac{f(x)}{x}\right)^{\frac{1}{x}}
$$

\begin{enumerate}
  \setcounter{enumi}{2}
  \item (8 \begin{CJK}{UTF8}{mj}分\end{CJK}) \begin{CJK}{UTF8}{mj}设\end{CJK}:
\end{enumerate}
$$
F(x, y)=\int_{\frac{x}{y}}^{x y}(x-y z) f(z) \mathrm{d} z
$$
\begin{CJK}{UTF8}{mj}其中\end{CJK} $f(z)$ \begin{CJK}{UTF8}{mj}为可微函数\end{CJK}, \begin{CJK}{UTF8}{mj}求\end{CJK} $F_{x y}^{\prime \prime}(x, y)$.

\begin{enumerate}
  \setcounter{enumi}{3}
  \item (12 \begin{CJK}{UTF8}{mj}分\end{CJK}) \begin{CJK}{UTF8}{mj}设\end{CJK}
\end{enumerate}
$$
S(x)=\int_{0}^{x}|\cos t| \mathrm{d} t
$$
(1) \begin{CJK}{UTF8}{mj}当\end{CJK} $n$ \begin{CJK}{UTF8}{mj}为正整数且\end{CJK} $n \pi<x<(n+1) \pi$ \begin{CJK}{UTF8}{mj}时\end{CJK}, \begin{CJK}{UTF8}{mj}证明\end{CJK}:
$$
2 n \leq S(x) \leq 2(n+1)
$$
(2) \begin{CJK}{UTF8}{mj}求\end{CJK}
$$
\lim _{x \rightarrow+\infty} \frac{S(x)}{x}
$$

\begin{enumerate}
  \setcounter{enumi}{4}
  \item (12 \begin{CJK}{UTF8}{mj}分\end{CJK}) \begin{CJK}{UTF8}{mj}估计积分\end{CJK}
\end{enumerate}
$$
I_{R}=\oint_{x^{2}+y^{2}=R^{2}} \frac{y \mathrm{~d} x-x \mathrm{~d} y}{\left(x^{2}+x y+y^{2}\right)}
$$
\begin{CJK}{UTF8}{mj}的值并求极限\end{CJK} $\lim _{R \rightarrow \infty} I_{R}$.

\begin{enumerate}
  \setcounter{enumi}{5}
  \item (12 \begin{CJK}{UTF8}{mj}分\end{CJK}) \begin{CJK}{UTF8}{mj}计算曲面积分\end{CJK}:
\end{enumerate}
$$
\iint_{\Sigma}\left(x^{3} \cos \alpha+y^{3} \cos \beta+z^{3} \cos \gamma\right) \mathrm{d} S .
$$
\begin{CJK}{UTF8}{mj}其中\end{CJK} $\Sigma$ \begin{CJK}{UTF8}{mj}是雉面\end{CJK} $z^{2}=x^{2}+y^{2},-1 \leq z \leq 0, \cos \alpha, \cos \beta, \cos \gamma$ \begin{CJK}{UTF8}{mj}是曲面\end{CJK} $\Sigma$ \begin{CJK}{UTF8}{mj}上点\end{CJK} $(x, y, z)$ \begin{CJK}{UTF8}{mj}处外法线的方向余弦\end{CJK}.

\begin{enumerate}
  \setcounter{enumi}{6}
  \item (18 \begin{CJK}{UTF8}{mj}分\end{CJK}) \begin{CJK}{UTF8}{mj}叙述并证明康托尔定理\end{CJK}.

  \item ( 12 \begin{CJK}{UTF8}{mj}分\end{CJK}) \begin{CJK}{UTF8}{mj}设\end{CJK} $f(x)$ \begin{CJK}{UTF8}{mj}在\end{CJK} $[0,1]$ \begin{CJK}{UTF8}{mj}上连续\end{CJK}, \begin{CJK}{UTF8}{mj}在\end{CJK} $(0,1)$ \begin{CJK}{UTF8}{mj}内可导\end{CJK}, \begin{CJK}{UTF8}{mj}且满足\end{CJK} $f(1)=2 \int_{0}^{\frac{1}{2}} \mathrm{e}^{1-x^{2}} f(x) \mathrm{d} x$, \begin{CJK}{UTF8}{mj}求证\end{CJK}: \begin{CJK}{UTF8}{mj}存在\end{CJK} $\xi \in(0,1)$, \begin{CJK}{UTF8}{mj}使\end{CJK} \begin{CJK}{UTF8}{mj}得\end{CJK}

\end{enumerate}
$$
f^{\prime}(\xi)=2 \xi f(\xi)
$$

\begin{enumerate}
  \setcounter{enumi}{8}
  \item (18 \begin{CJK}{UTF8}{mj}分\end{CJK}) \begin{CJK}{UTF8}{mj}已知\end{CJK} $\sum_{n=1}^{\infty} \frac{1}{n^{2}}=\frac{\pi^{2}}{6}$, \begin{CJK}{UTF8}{mj}求\end{CJK}
\end{enumerate}
$$
\int_{0}^{+\infty} \frac{x}{\mathrm{e}^{x}+1} \mathrm{~d} x
$$

\begin{enumerate}
  \setcounter{enumi}{9}
  \item ( 12 \begin{CJK}{UTF8}{mj}分\end{CJK}) \begin{CJK}{UTF8}{mj}设二元函数\end{CJK}
\end{enumerate}
$$
f(x, y)=\left\{\begin{array}{lr}
\left(a \sqrt{|x|}+x^{2}+y^{2}+b\right) \frac{\sin x y^{2}}{x^{2}+y^{4}}, & x^{2}+y^{2} \neq 0 \\
0, & x^{2}+y^{2}=0
\end{array}\right.
$$
\begin{CJK}{UTF8}{mj}在点\end{CJK} $(0,0)$ \begin{CJK}{UTF8}{mj}可微\end{CJK}, \begin{CJK}{UTF8}{mj}求常数\end{CJK} $a, b$ \begin{CJK}{UTF8}{mj}的值\end{CJK}.

\begin{enumerate}
  \setcounter{enumi}{10}
  \item (18 \begin{CJK}{UTF8}{mj}分\end{CJK}) \begin{CJK}{UTF8}{mj}设\end{CJK} $f(x)$ \begin{CJK}{UTF8}{mj}在闭区间\end{CJK} $[a, b]$ \begin{CJK}{UTF8}{mj}上连续\end{CJK}, \begin{CJK}{UTF8}{mj}在\end{CJK} $(a, b)$ \begin{CJK}{UTF8}{mj}内可导\end{CJK}, \begin{CJK}{UTF8}{mj}且\end{CJK} $f^{\prime}(x)>0$. \begin{CJK}{UTF8}{mj}若极限\end{CJK} $\lim _{x \rightarrow a^{+}} \frac{f(2 x-a)}{x-a}$ \begin{CJK}{UTF8}{mj}存在\end{CJK}, \begin{CJK}{UTF8}{mj}证明\end{CJK}:
\end{enumerate}
(1) \begin{CJK}{UTF8}{mj}在\end{CJK} $(a, b)$ \begin{CJK}{UTF8}{mj}内\end{CJK} $f(x)>0$;

(2) \begin{CJK}{UTF8}{mj}在\end{CJK} $(a, b)$ \begin{CJK}{UTF8}{mj}内存在点\end{CJK} $\xi$, \begin{CJK}{UTF8}{mj}满足\end{CJK}
$$
\frac{b^{2}-a^{2}}{\int_{a}^{b} f(x) \mathrm{d} x}=\frac{2 \xi}{f(\xi)}
$$
(3) \begin{CJK}{UTF8}{mj}在\end{CJK} $(a, b)$ \begin{CJK}{UTF8}{mj}内存在与\end{CJK} (2) \begin{CJK}{UTF8}{mj}相异的点\end{CJK} $\eta$, \begin{CJK}{UTF8}{mj}满足\end{CJK}
$$
f^{\prime}(\eta)\left(b^{2}-a^{2}\right)=\frac{2 \xi}{\xi-a} \int_{a}^{b} f(x) \mathrm{d} x
$$

\begin{enumerate}
  \setcounter{enumi}{11}
  \item (12 \begin{CJK}{UTF8}{mj}分\end{CJK}) \begin{CJK}{UTF8}{mj}求函数\end{CJK}
\end{enumerate}
$$
z=x^{2}+y^{2}
$$
\begin{CJK}{UTF8}{mj}在闭区域\end{CJK} $(x-\sqrt{2})^{2}+(y-\sqrt{2})^{2} \leq 9$ \begin{CJK}{UTF8}{mj}上的最值\end{CJK}.


\end{document}